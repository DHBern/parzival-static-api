\documentclass[8pt,a4paper,notitlepage]{article}
\usepackage{fullpage}
\usepackage{ulem}
\usepackage{xltxtra}
\usepackage{datetime}
\renewcommand{\dateseparator}{.}
\dmyyyydate
\usepackage{fancyhdr}
\usepackage{ifthen}
\pagestyle{fancy}
\fancyhf{}
\renewcommand{\headrulewidth}{0pt}
\fancyfoot[L]{\ifthenelse{\value{page}=1}{\today, \currenttime{} Uhr}{}}
\begin{document}
\begin{table}[ht]
\begin{minipage}[t]{0.5\linewidth}
\small
\begin{center}*D
\end{center}
\begin{tabular}{rl}
\textbf{301} & \begin{large}D\end{large}es \textbf{erwi\textit{r}be} ich iu die hulde,\\ 
 & daz der künec læt die schulde,\\ 
 & welt ir nâch mîme râte leben,\\ 
 & \textbf{geselleschaft} \textbf{mir} vür in geben."\\ 
5 & Des \textbf{künec} Gahmuretes kint,\\ 
 & dröuwen unt vlêhen was im ein wint.\\ 
 & der tavelrunde hœhster prîs,\\ 
 & Gawan, was \textbf{solher} nœte \textbf{al} wîs.\\ 
 & er het \textbf{si} unsanfte erkant,\\ 
10 & dô er mit dem mezzer durch die hant\\ 
 & stach. des twang in minnen kraft\\ 
 & unt wert wîplîch geselleschaft.\\ 
 & In schiet von tôde ein künegîn,\\ 
 & dô der küene Læhelin\\ 
15 & mit einer tjoste rîche\\ 
 & in twanc \textbf{sô} volleclîche.\\ 
 & diu \textbf{senfte}, süeze, wol gevar\\ 
 & ze pfande sazt ir houbet dar,\\ 
 & \textbf{diu künegîn} \textbf{Inguse} \textbf{von} Bahtarliez,\\ 
20 & alsus diu getriwe hiez.\\ 
 & Dô \textbf{dâhte} mîn hêr Gawan:\\ 
 & "waz, ob diu minne disen man\\ 
 & twinget, als si mich \textbf{dô} \textbf{twanc},\\ 
 & unt sîn getriulîch gedanc\\ 
25 & der minne muoz ir siges jehen?"\\ 
 & Er marcte des Wâleises sehen,\\ 
 & war \textbf{stüenden im} diu ougen sîn.\\ 
 & eine failen \textbf{tuoches} von Surin,\\ 
 & gefurriert mit gelwem zindâl,\\ 
30 & die swang er \textbf{über} di\textit{u} bluotes mâl.\\ 
\end{tabular}
\scriptsize
\line(1,0){75} \newline
D \newline
\line(1,0){75} \newline
\textbf{1} \textit{Initiale} D  \textbf{5} \textit{Majuskel} D  \textbf{13} \textit{Majuskel} D  \textbf{21} \textit{Majuskel} D  \textbf{26} \textit{Majuskel} D  \newline
\line(1,0){75} \newline
\textbf{1} erwirbe] erwibe D \textbf{5} Gahmuretes] Gahmvretes D \textbf{19} diu] de D  $\cdot$ Inguse] Ingvͦse D  $\cdot$ Bahtarliez] Bahtarlîez D \textbf{26} Wâleises] Waleis D \textbf{30} diu] die D \newline
\end{minipage}
\hspace{0.5cm}
\begin{minipage}[t]{0.5\linewidth}
\small
\begin{center}*m
\end{center}
\begin{tabular}{rl}
 & des \textbf{erwürbe} ich iu die hulde,\\ 
 & daz der künic lât die schulde,\\ 
 & wellet ir nâch mînem râte leben,\\ 
 & \textbf{selleschaft} \textbf{mir} vür in geben."\\ 
5 & des \textbf{küniges} Gahmuretes kint,\\ 
 & dröuwen und vlêhen was ime ein wint.\\ 
 & \begin{large}D\end{large}er tavelrunde hœhester prîs,\\ 
 & G\textit{a}wa\textit{n}, wa\textit{s} \textbf{solicher} nœte wîs.\\ 
 & er hete \textbf{si} unsanfte erkant,\\ 
10 & dô er mit dem mezzer durch die hant\\ 
 & stach. des twanc in minn\textit{e} kraft\\ 
 & und wert wîplîch geselleschaft.\\ 
 & in schiet von tôde ein künigîn,\\ 
 & dô der küene Lehelin\\ 
15 & mit einer juste rîche\\ 
 & in twanc \textbf{sô} volleclîche.\\ 
 & diu \textbf{senfte}, \textit{süez}e, wol gevar\\ 
 & ze pfande s\textit{a}zzete \textit{ir houbet d}ar,\\ 
 & \textbf{diu künigîn} \textbf{Inguse} \textbf{von} Bahkarliez,\\ 
20 & alsus diu getriuwe hiez.\\ 
 & dô \textbf{dâhte} mîn hêr Gawa\textit{n}:\\ 
 & "waz, ob diu minne disen man\\ 
 & twing\textit{e}t, alsô si mich \textbf{betwanc},\\ 
 & und sîn getriuwelîch gedanc\\ 
25 & der \textit{minne} muoz ir siges jehen?"\\ 
 & er marhte des Wâleises sehen,\\ 
 & war \textbf{ime stüenden} di\textit{u} ougen sîn.\\ 
 & ein failen \textbf{tuoches} von Sur\textit{i}n,\\ 
 & gefurieret mit ge\textit{l}w\textit{em} zindâl,\\ 
30 & die swanc er \textbf{ûf} diu bluotes mâl.\\ 
\end{tabular}
\scriptsize
\line(1,0){75} \newline
m n o \newline
\line(1,0){75} \newline
\textbf{7} \textit{Initiale} m  \newline
\line(1,0){75} \newline
\textbf{4} selleschaft] Geselleschafft n o \textbf{5} Gahmuretes] gamúretes n gamarutes o \textbf{7} tavelrunde] tafelrunder n (o) \textbf{8} Gawan] Gewas m  $\cdot$ was] wan m \textbf{11} des] das o  $\cdot$ minne] minnes m \textbf{12} wîplîch] wiplicher o \textbf{13} in] Jne o \textbf{17} süeze] minne m \textbf{18} sazzete ir houbet dar] senczete offenbar m setzet ir houbet n seczet er heubet dar o \textbf{19} Inguse] jnguse m n von jnguse o  $\cdot$ von] \textit{om.} n  $\cdot$ Bahkarliez] kakorliesz n kakarlies o \textbf{20} alsus] Also o  $\cdot$ hiez] hiesse o \textbf{21} dâhte] gedochte n (o)  $\cdot$ hêr] herre her n  $\cdot$ Gawan] gawa: m gewan o \textbf{22} ob] \textit{om.} n \textbf{23} twinget] Twingent m \textbf{25} minne] \textit{om.} m  $\cdot$ ir] er n o \textbf{26} des] der n das o  $\cdot$ Wâleises] waleis m waleẏsen n waleiszez o \textbf{27} war] Wenne n Warummb o  $\cdot$ ime] \textit{om.} o  $\cdot$ diu] di m \textbf{28} failen] falen o  $\cdot$ Surin] suͯren m forin n o \textbf{29} gefurieret] Gesnẏeret n  $\cdot$ gelwem] gewin m gelgem o \textbf{30} er] er mit n es o \newline
\end{minipage}
\end{table}
\newpage
\begin{table}[ht]
\begin{minipage}[t]{0.5\linewidth}
\small
\begin{center}*G
\end{center}
\begin{tabular}{rl}
 & des \textbf{erwirbe} ich iu die hulde,\\ 
 & daz der künic lât die schulde,\\ 
 & welt ir nâch mînem râte leben\\ 
 & \textbf{unde} \textbf{geselleschaft} \textbf{her} vür in geben."\\ 
5 & des \textbf{künic} Gahmuretes kint,\\ 
 & drôn unde vlêgen was im ein wint.\\ 
 & der tavelrunder hœhster brîs,\\ 
 & Gawan, was \textbf{dirre} nœte \textbf{al}wîs.\\ 
 & er het \textbf{s\textit{i}} \textit{u}nsanfte erkant,\\ 
10 & dô er mit dem mezzer durch die hant\\ 
 & stach. des twangin minnen kraft\\ 
 & unt wert wîplîch geselleschaft.\\ 
 & in schiet von tôde ein künigîn,\\ 
 & dô der küene Lehelin\\ 
15 & mit einer tjoste rîche\\ 
 & in twanc \textbf{sus} volliclîche.\\ 
 & diu \textbf{senfte}, süeze, wolgevar\\ 
 & ze pfande sazte ir houbet dar,\\ 
 & \textit{\textbf{royn}} \textit{\textbf{Ingwiz}} \textit{\textbf{de} Paitterliez,}\\ 
20 & \textit{alsus diu getriwe hiez.}\\ 
 & dô \textbf{sprach} mîn hêr Gawan:\\ 
 & "waz, op diu minne disen man\\ 
 & twinget, als si mich \textbf{dô} \textbf{twanc},\\ 
 & unde sîn getriulîch gedanc\\ 
25 & der minne muoz ir siges jehen?"\\ 
 & er marcte des Wâleises sehen,\\ 
 & war \textbf{stuonden} diu ougen sîn.\\ 
 & eine væle von Surin,\\ 
 & \begin{large}G\end{large}efurrie\textit{r}t \textit{mit} gelwem zendâl,\\ 
30 & die swang er \textbf{über} diu b\textit{l}uotes mâl.\\ 
\end{tabular}
\scriptsize
\line(1,0){75} \newline
G I O L M Q R Z \newline
\line(1,0){75} \newline
\textbf{5} \textit{Initiale} Z  \textbf{13} \textit{Initiale} I O Q  \textbf{21} \textit{Initiale} L  \textbf{29} \textit{Initiale} G  \newline
\line(1,0){75} \newline
\textbf{1} erwirbe] erwarb R \textbf{3} mînem] mynne M \textbf{4} her] mir Z  $\cdot$ in] úch R \textbf{5} künic] chvniges O (L) (Q) (R)  $\cdot$ Gahmuretes] gahmurtes G Gamvretes O (Z) Gahmuͯretes L gamuretis M gamúres Q \textbf{6} drôn] dro I (L) (R)  $\cdot$ vlêgen] [sla]: slagen L bitten R \textbf{7} hœhster] hoher L hoste M \textbf{8} Gawan] Gawin M  $\cdot$ dirre] disze Q \textbf{9} er] Vnde her M  $\cdot$ het si] hetse oͮch G hetz O (L) (M) (Q) (R) Z \textbf{10} er] \textit{om.} I  $\cdot$ durch] \textit{om.} O \textbf{11} minnen] minne I O (L) meine Q die minne R \textbf{12} wert] úwer R \textbf{13} in] ÷n O  $\cdot$ von] vom L \textbf{14} dô] Da M Z  $\cdot$ küene] werde I kvnig L  $\cdot$ Lehelin] Læhelin O lehlin R \textbf{16} in] \textit{om.} M  $\cdot$ twanc sus] twan so Z \textbf{17} senfte süeze] suͯsse senfftte R  $\cdot$ wolgevar] vol gevar Q \textbf{18} ir haupt sazte si zephande dar I  $\cdot$ ir] er daz R \textbf{19} \textit{Die Verse 301.19-20 fehlen} G   $\cdot$ Die kunigin Jngvse von bahtarliez Z  $\cdot$ Royn] Roy M (R)  $\cdot$ Ingwiz de Paitterliez] ingv̂ze phaterliez O Jnguͯze de paterlies L en gruzte de phaffterliez M in guze depatherlisz Q in Guzadepahterlies R \textbf{21} dô] Da M R Z  $\cdot$ sprach] dachte Q (R) (Z)  $\cdot$ Gawan] gewan R \textbf{22} disen] suszen Q \textbf{23} twinget] gezwinget R  $\cdot$ si] \textit{om.} I M  $\cdot$ dô] da M (Z) \textbf{24} getriulîch] trewlich Q Z \textbf{25} ir] ich O er L \textbf{26} marcte] mahrtes L  $\cdot$ Wâleises] waleis G R Z waleishen I \textbf{27} diu] im die Z \textbf{28} eine] ein I (Q) (R) Eins O (Z)  $\cdot$ væle] væilen tvͦches O (L) (M) (Z) vele tuches Q welen tuͦchs R  $\cdot$ Surin] einen sigelatin I vrin M súin Q \textbf{29} Gefurriert] Gefvrriet G Gefvrrit O Geviermeret M  $\cdot$ mit] von G  $\cdot$ gelwem] einem gruͤnen I gelwen L Q \textbf{30} swang er] schwang R  $\cdot$ diu] di O des R  $\cdot$ bluotes] [boͮt*]: boͮts G \newline
\end{minipage}
\hspace{0.5cm}
\begin{minipage}[t]{0.5\linewidth}
\small
\begin{center}*T
\end{center}
\begin{tabular}{rl}
 & des \textbf{erwirb}ich iu die hulde,\\ 
 & daz der künec lât die schulde,\\ 
 & welt ir nâch mînem râte leben\\ 
 & \textbf{unde} \textbf{geselleschaft} \textbf{her} vür in geben."\\ 
5 & Des \textbf{küneges} Gahmuretes kint,\\ 
 & dröun unde vlêhen was im ein wint.\\ 
 & Der tavelrunder hœheste prîs,\\ 
 & Gawan, was \textbf{dirre} nôt \textbf{al}wîs.\\ 
 & er het \textbf{es} unsanfte erkant,\\ 
10 & dô er mit dem mezzer durch die hant\\ 
 & stach. des twanc in minnen kraft\\ 
 & unde wert wîplîch geselleschaft.\\ 
 & in schiet von tôde ein künegîn,\\ 
 & dô der küene Lehelin\\ 
15 & mit einer tjoste rîche\\ 
 & in twanc \textbf{sô} volleclîche.\\ 
 & diu \textbf{junge}, süeze, wol gevar\\ 
 & ze pfande sazte ir houbet dar,\\ 
 & \textbf{Royn} \textbf{Inguzze} \textbf{de} Pahtelierz,\\ 
20 & alsus diu getriwe hiez.\\ 
 & \begin{large}D\end{large}ô \textbf{sprach} mîn hêr Gawan:\\ 
 & "waz, ob diu minne disen man\\ 
 & twinget, als si mich \textbf{dô} \textbf{betwanc},\\ 
 & unde sîn getriuwelîch gedanc\\ 
25 & der minne muoz ir siges jehen?"\\ 
 & er marhte des Wâleises sehen,\\ 
 & war \textbf{stuonden} di\textit{u} ougen sîn.\\ 
 & ein vêle, \textbf{tuoch} von Surin,\\ 
 & gefurrieret mit gelwem zindâl,\\ 
30 & di\textit{e} swang er \textbf{über} diu bluotes mâl.\\ 
\end{tabular}
\scriptsize
\line(1,0){75} \newline
T U V W \newline
\line(1,0){75} \newline
\textbf{5} \textit{Initiale} W   $\cdot$ \textit{Majuskel} T  \textbf{7} \textit{Majuskel} T  \textbf{19} \textit{Majuskel} T  \textbf{21} \textit{Initiale} T U  \newline
\line(1,0){75} \newline
\textbf{1} erwirbich] erwerbe ich U erwúrbe ich V (W)  $\cdot$ die] \textit{om.} W \textbf{4} vür in] mir W \textbf{5} Gahmuretes] gamvretes V (W) \textbf{6} was im] ist U \textbf{7} hœheste] [hohe*]: hohester V hoͤchster W \textbf{8} alwîs] al zuͦ wis U \textbf{9} het es] hete U \textbf{14} Lehelin] lehalin W \textbf{16} in] \textit{om.} W  $\cdot$ sô] sos U sus W \textbf{17} junge] [*]: senfte V \textbf{19} Royn] Koyn U [*n]: Die kv́negin V  $\cdot$ Inguzze] in gvzze T in gozze U ingozze V in guze W  $\cdot$ Pahtelierz] pachte liez U pahteliez V patherließ W \textbf{21} sprach] [*]: gedohte V \textbf{22} disen] dise U \textbf{23} betwanc] twanck W \textbf{24} getriuwelîch] getruwelichen U treúlicher W \textbf{25} ir] er W \textbf{26} Wâleises] walleises V \textbf{27} war] [*]: War im V  $\cdot$ diu] die T \textbf{28} vêle tuoch] veletvͦch V pfellel tuͦch W  $\cdot$ Surin] serin U sydin V surein W \textbf{29} gefurrieret] Geformieret U \textbf{30} die swang er] div [s*]: swanger T Daz schwang er W  $\cdot$ diu] des V (W) \newline
\end{minipage}
\end{table}
\end{document}
