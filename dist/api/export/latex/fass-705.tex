\documentclass[8pt,a4paper,notitlepage]{article}
\usepackage{fullpage}
\usepackage{ulem}
\usepackage{xltxtra}
\usepackage{datetime}
\renewcommand{\dateseparator}{.}
\dmyyyydate
\usepackage{fancyhdr}
\usepackage{ifthen}
\pagestyle{fancy}
\fancyhf{}
\renewcommand{\headrulewidth}{0pt}
\fancyfoot[L]{\ifthenelse{\value{page}=1}{\today, \currenttime{} Uhr}{}}
\begin{document}
\begin{table}[ht]
\begin{minipage}[t]{0.5\linewidth}
\small
\begin{center}*D
\end{center}
\begin{tabular}{rl}
\textbf{705} & \textit{\begin{large}G\end{large}}awane ein bischof messe sanc.\\ 
 & von storîe wart dâ grôz gedranc;\\ 
 & ritter unde vrouwen\\ 
 & man mohte zorse schouwen\\ 
5 & an Artuses ringe,\\ 
 & ê daz man dâ gesinge.\\ 
 & der künec Artus selbe stuont,\\ 
 & dâ die pfaffen daz ambet tuont.\\ 
 & dô \textbf{der benediz} wart getân,\\ 
10 & dô wâpende sich hêr Gawan.\\ 
 & man sach \textbf{ê} tragen \textbf{den} stolzen\\ 
 & sîn îserîne kolzen\\ 
 & an wol geschicten beinen.\\ 
 & d\textit{ô} begunden vrouwen weinen.\\ 
15 & Daz \textbf{her} zogete ûz über al,\\ 
 & dâ si \textbf{mit} swerten hôrten schal\\ 
 & unt viwer ûz \textbf{helmen} \textbf{swingen}\\ 
 & unt slege mit kreften \textbf{bringen}.\\ 
 & Der künec Gramoflanz pflac site,\\ 
20 & im versmâhte sêre, daz er strite\\ 
 & mit einem man. dô dûhte in nû,\\ 
 & daz hie sehse griffen strîtes zuo.\\ 
 & ez was doch Parzival al ein,\\ 
 & der gein im werlîche schein.\\ 
25 & Er het in underwîset\\ 
 & einer zuht, die man \textbf{noch} prîset:\\ 
 & er \textbf{en}genam sît nimmer mêre\\ 
 & mit rede an sich die êre,\\ 
 & daz er zwein mannen \textbf{büte} strît,\\ 
30 & wan einer \textbf{s}im ze vil \textbf{dâ} gît.\\ 
\end{tabular}
\scriptsize
\line(1,0){75} \newline
D \newline
\line(1,0){75} \newline
\textbf{1} \textit{Initiale} D  \textbf{15} \textit{Majuskel} D  \textbf{19} \textit{Majuskel} D  \textbf{25} \textit{Majuskel} D  \newline
\line(1,0){75} \newline
\textbf{1} Gawane] Sawane D \textbf{5} Artuses] Artvss D \textbf{14} dô] da D \textbf{23} Parzival] Parcifal D \newline
\end{minipage}
\hspace{0.5cm}
\begin{minipage}[t]{0.5\linewidth}
\small
\begin{center}*m
\end{center}
\begin{tabular}{rl}
 & \begin{large}G\end{large}awan ein bisch\textit{of} messe sanc.\\ 
 & vo\textit{n} \dag stor\dag  wart d\textit{â} grôz gedranc;\\ 
 & \textbf{beidiu} ritter und vrouwen\\ 
 & man mohte zuo rosse schouwen\\ 
5 & an Artuses ringe,\\ 
 & ê daz man d\textit{â} gesinge.\\ 
 & der künic Artus selbe stuont,\\ 
 & d\textit{â} die pfaffen daz ambaht tuont.\\ 
 & dô \textbf{der benedi\textit{z}} wart getân,\\ 
10 & dô wâpent sich hêr Gawan.\\ 
 & man sach \textbf{ê} tragen \textbf{de\textit{n}} stolzen\\ 
 & sîn îserîne kolzen\\ 
 & an wol geschicten beinen.\\ 
 & dô begunden vrowen weinen.\\ 
15 & daz \textbf{volc} zog\textit{t}e ûz über al,\\ 
 & dô si \textbf{von} swerten hôrten schal\\ 
 & und viur ûz \textbf{helmen} \textbf{swingen}\\ 
 & und slege mit kreften \textbf{bringen}.\\ 
 & der künic Gramolanz pflac site,\\ 
20 & im versmâhte sêre, daz er strite\\ 
 & mit eine\textit{m} man. dô dûht in nû,\\ 
 & daz hie sehse griffen strîtes zuo.\\ 
 & ez was doch Parcifal alein,\\ 
 & der gegen im werlîch schein.\\ 
25 & er het in underwîset\\ 
 & einer zuht, die man \textbf{noch} prîset:\\ 
 & er genam sît niemer mêre\\ 
 & mit rede an sich die êre,\\ 
 & daz er zwein mannen \textbf{büte} strît,\\ 
30 & wan einer im zuo vil \textbf{dâ} gît.\\ 
\end{tabular}
\scriptsize
\line(1,0){75} \newline
m n o Fr69 \newline
\line(1,0){75} \newline
\textbf{1} \textit{Initiale} m   $\cdot$ \textit{Capitulumzeichen} n  \newline
\line(1,0){75} \newline
\textbf{1} bischof messe] bischmesse m \textbf{2} von] Vo m  $\cdot$ dâ] do m n o \textbf{4} mohte] moͯchte n \textbf{5} Artuses] artuͯses o \textbf{6} dâ] do m n o \textbf{7} Artus] artuͯs o \textbf{8} dâ] Do m n o \textbf{9} benediz] benedig m n (o) ::: Fr69 \textbf{10} wâpent] wapende Fr69 \textbf{11} den] des m n \textbf{12} îserîne] ẏserem o \textbf{14} dô] daz Fr69  $\cdot$ begunden] begunde o \textbf{15} zogte] zoge m ::: Fr69 \textbf{19} Gramolanz] gramolantz m n gramolancz o \textbf{21} einem] einen m  $\cdot$ dûht] d::ch o \textbf{29} büte] bitte n \textbf{30} dâ] do n  $\cdot$ gît] giht o \newline
\end{minipage}
\end{table}
\newpage
\begin{table}[ht]
\begin{minipage}[t]{0.5\linewidth}
\small
\begin{center}*G
\end{center}
\begin{tabular}{rl}
 & \begin{large}G\end{large}awane ein bischof messe sanc.\\ 
 & von storîe wart d\textit{â} grôz gedranc;\\ 
 & rîter unde vrouwen\\ 
 & man mohte ze orse schouwen\\ 
5 & an Artuses ringe,\\ 
 & ê daz man dâ gesinge.\\ 
 & der künic Artus selbe stuont,\\ 
 & dâ die pfaffen daz ambet tuont.\\ 
 & dô \textbf{der benediz} wart getân,\\ 
10 & dô wâpent sich hêr Gawan.\\ 
 & man sach \textbf{dar} tragen \textbf{dem} stolzen\\ 
 & sîne îserîne kolzen\\ 
 & an wol geschicten beinen.\\ 
 & dô begunden vrouwen weinen.\\ 
15 & daz \textbf{her} zogte ûz überal,\\ 
 & dô si \textbf{mit} swerten hôrten schal\\ 
 & unde viur ûz \textbf{helme} \textbf{springen}\\ 
 & unde slege mit kreften \textbf{dringen}.\\ 
 & der künic Gramoflanz pflac site,\\ 
20 & im versmâhte sêre, daz er strite\\ 
 & mit einem man. dô dûhte in nû,\\ 
 & daz hie sehse griffen strîtes zuo.\\ 
 & ez was doch Parcival al ein,\\ 
 & der gein im werlîchen schein.\\ 
25 & er het in underwîset\\ 
 & einer zuht, die man brîset:\\ 
 & er genam sît nimmer mêre\\ 
 & mit  an sich die êre,\\ 
 & daz er zwein mannen \textbf{hiet} strît,\\ 
30 & wan einer im\textbf{s} ze vil gît.\\ 
\end{tabular}
\scriptsize
\line(1,0){75} \newline
G I L M Z Fr18 \newline
\line(1,0){75} \newline
\textbf{1} \textit{Initiale} G L Z  \textbf{9} \textit{Initiale} I  \newline
\line(1,0){75} \newline
\textbf{1} Gawane] Gawan L  $\cdot$ bischof] pissholf I \textbf{2} storîe] stvrie G L (M) Z Fr18  $\cdot$ wart] was I  $\cdot$ dâ] do G \textit{om.} L \textbf{4} man mohte] moht man I (L) man moht da Fr18 \textbf{5} Artuses] artus G Z \textbf{6} daz] \textit{om.} I \textbf{7} selbe] da selbis M \textbf{9} dô] Da M Z  $\cdot$ benediz] segen L (M) \textbf{10} dô] Da M Z  $\cdot$ wâpent] wapinde M \textbf{11} dem] den I \textbf{12} îserîne] isenine I ysenie L \textbf{13} streich er an sin wol gestrichen bein I \textbf{14} dô] Da L M Z Fr18 \textbf{15} zogte] zogt L Z Fr18 zcog M \textbf{16} dô] Da L M Z Fr18 \textbf{17} helme] helmen L (M) Z Fr18 \textbf{18} dringen] ringen L bringen M \textbf{19} Gramoflanz] gramorflanz M gramoflantz Z (Fr18) \textbf{20} versmâhte] versmahet I  $\cdot$ strite] e strite Z \textbf{21} dô] da M Z \textbf{22} hie] \textit{om.} Fr18  $\cdot$ strîtes] \textit{om.} I L \textbf{23} ez] strites ez I  $\cdot$ doch] ouch Z  $\cdot$ Parcival] parcifal G Z Fr18 Parzifal I (L) (M)  $\cdot$ al] \textit{om.} L \textbf{24} gein im werlîchen] werliche gein im L \textbf{27} er] Ern Z Fr18 \textbf{28} mit an sich] wider an sich I Mit rede an sich L M Fr18 An sich mit rede Z \textbf{30} ims] isz yme M  $\cdot$ gît] da git Z \newline
\end{minipage}
\hspace{0.5cm}
\begin{minipage}[t]{0.5\linewidth}
\small
\begin{center}*T
\end{center}
\begin{tabular}{rl}
 & \begin{large}G\end{large}awane ein bischof messe sanc.\\ 
 & von storîen wart d\textit{â} grôz gedranc;\\ 
 & rîtære und vrouwen\\ 
 & man mohte zuo orse schouwen\\ 
5 & an Artuses ringe,\\ 
 & ê daz man d\textit{â} gesinge.\\ 
 & der künec Artus sel\textit{b}e stuont,\\ 
 & d\textit{â} die pfaffen daz ambet tuont.\\ 
 & dô \textbf{diu benedictio} wart getân,\\ 
10 & dô wâpente sich hêr Gawan.\\ 
 & man sach \textbf{d\textit{â}} tragen \textbf{den} stolzen\\ 
 & sîne îserînen kolzen\\ 
 & an wol geschicketen beinen.\\ 
 & dô begunden vrouwen weinen.\\ 
15 & daz \textbf{her} zoget \textit{ûz} überal,\\ 
 & dô si \textbf{mit} swerten hôrten schal\\ 
 & und viur ûz \textbf{helmen} \textbf{springen}\\ 
 & und slege mit kreften \textbf{dringen}.\\ 
 & der künec Gramoflanz pflac site,\\ 
20 & im versmâhete sêre, daz er strite\\ 
 & mit eime man. dô dûhtin nuo,\\ 
 & daz hie sehse griffen \textit{strîtes} zuo.\\ 
 & ez was doch Parcifal alein,\\ 
 & der gein im werlîche schein.\\ 
25 & er hete in underwîset\\ 
 & eine\textit{r} zuht, die man prîset:\\ 
 & er \textbf{en}genam sît niemer mêre\\ 
 & mit rede an sich die êre,\\ 
 & daz er zwein mannen \textbf{hete} strît,\\ 
30 & wan einer im \textbf{es} zuo vil gît.\\ 
\end{tabular}
\scriptsize
\line(1,0){75} \newline
U V W Q R \newline
\line(1,0){75} \newline
\textbf{1} \textit{Initiale} U V R  \textbf{7} \textit{Initiale} W  \newline
\line(1,0){75} \newline
\textbf{1} Gawane] Gwenen R  $\cdot$ messe] do messe R \textbf{2} storîen] der storien V Storie R  $\cdot$ wart] was Q  $\cdot$ dâ] do U W Q R \textit{om.} V \textbf{3} [*]: Beide rittere vnde vrowen V \textbf{4} mohte] moͤhte V  $\cdot$ orse] orsen V \textbf{5} Artuses] kúnig artus W artus Q (R) \textbf{6} dâ] do U V R do gar W \textbf{7} selbe] selle U selber W Q \textbf{8} dâ] Do U W Q Alda V R  $\cdot$ pfaffen] priester W \textbf{9} diu] der V (W) (Q) (R)  $\cdot$ benedictio] segen V R benedig W benditze Q \textbf{10} wâpente] wappet Q (R)  $\cdot$ hêr] mein herr W  $\cdot$ Gawan] gewan Q \textbf{11} dâ] do U V W Q \textbf{12} îserînen] yserine V (W) (Q) [yserime]: yserẏne  R \textbf{13} geschicketen] geschicktem Q geschnitten R \textbf{14} begunden vrouwen] begund manig frawe W \textbf{15} daz] Do Q  $\cdot$ zoget ûz] zoget U ausz zogte Q \textbf{16} mit] [*]: von V \textbf{17} ûz] auß der W \textbf{18} dringen] [*]: dringen V \textbf{19} Gramoflanz] [gramaflantz]: gramaflanz V gramoflantz W Q Gramoflancz R \textbf{20} im] [J*]: Jm V Jn R \textbf{21} man] \textit{om.} R \textbf{22} strîtes] \textit{om.} U \textbf{23} ez] Er W  $\cdot$ Parcifal] Parzifal U parzefal V herr partzifal W partzifal Q Parczifal R \textbf{25} er] Fr W  $\cdot$ hete] hat R \textbf{26} einer] Eine U  $\cdot$ prîset] [*]: noch priset V hoch preiset W \textbf{27} engenam] genam Q R \textbf{29} er] er zvͦ V es het Q  $\cdot$ hete] gebe W \textbf{30} im es] imme V im sein W Im do R  $\cdot$ gît] do git V (W) (Q) \newline
\end{minipage}
\end{table}
\end{document}
