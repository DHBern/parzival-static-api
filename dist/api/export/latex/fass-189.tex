\documentclass[8pt,a4paper,notitlepage]{article}
\usepackage{fullpage}
\usepackage{ulem}
\usepackage{xltxtra}
\usepackage{datetime}
\renewcommand{\dateseparator}{.}
\dmyyyydate
\usepackage{fancyhdr}
\usepackage{ifthen}
\pagestyle{fancy}
\fancyhf{}
\renewcommand{\headrulewidth}{0pt}
\fancyfoot[L]{\ifthenelse{\value{page}=1}{\today, \currenttime{} Uhr}{}}
\begin{document}
\begin{table}[ht]
\begin{minipage}[t]{0.5\linewidth}
\small
\begin{center}*D
\end{center}
\begin{tabular}{rl}
\textbf{189} & \textbf{dâr nâch er güetlîche} an mich sach,\\ 
 & sît \textbf{uns} ze sitzen \textbf{hie} geschach.\\ 
 & \begin{large}E\end{large}r hât sich zuht gein mir enbart.\\ 
 & mîn rede ist al ze vil gespart.\\ 
5 & hie\textbf{ne} sol niht mêr geswigen sîn."\\ 
 & zir gaste sprach diu künegîn:\\ 
 & "Hêrre, ein wirtîn reden muoz.\\ 
 & ein kus erwarp mir iwern gruoz.\\ 
 & \textbf{ouch but ir} dienst dâ her în.\\ 
10 & \textbf{sus} seit ein juncvrouwe mîn.\\ 
 & \textbf{des} habent \textbf{uns} geste niht gewent.\\ 
 & des hât mîn herze sich \textbf{gesent}.\\ 
 & Hêrre, ich vrâge iuch mære,\\ 
 & wannen iwer reise wære."\\ 
15 & "vrouwe, ich reit bî disem tage\\ 
 & von einem man, \textbf{den ich} \textbf{in} klage\\ 
 & liez mit triwen âne \textbf{schanz}.\\ 
 & der vürste \textbf{heizet} Gurnamanz;\\ 
 & \textbf{von Graharz ist er} genant.\\ 
20 & \textbf{dannen} \textbf{reit ich} \textbf{hiute} in \textbf{ditze} lant."\\ 
 & \textbf{Alsus} sprach diu \textbf{werde} magt:\\ 
 & "het\textbf{z anders iemen mir} gesagt,\\ 
 & der volge \textbf{würde im niht} verjehen,\\ 
 & deiz \textbf{eines tages wære} geschehen;\\ 
25 & wan swelch mîn bote \textbf{ie} baldeste reit,\\ 
 & die reise er zwêne tage \textbf{vermeit}.\\ 
 & \textbf{sîn swester} was diu \textbf{muoter mîn},\\ 
 & iwers wirtes. \textbf{sîner} tohter schîn\\ 
 & sich \textbf{ouch} \textbf{vor} jâmer \textbf{krenken} mac.\\ 
30 & wir haben manegen \textbf{sûren} tac\\ 
\end{tabular}
\scriptsize
\line(1,0){75} \newline
D \newline
\line(1,0){75} \newline
\textbf{3} \textit{Initiale} D  \textbf{7} \textit{Majuskel} D  \textbf{13} \textit{Majuskel} D  \textbf{21} \textit{Majuskel} D  \newline
\line(1,0){75} \newline
\newline
\end{minipage}
\hspace{0.5cm}
\begin{minipage}[t]{0.5\linewidth}
\small
\begin{center}*m
\end{center}
\begin{tabular}{rl}
 & \textbf{vil güetlîch er} an mich sach,\\ 
 & sît \textbf{uns} ze sitzene \textbf{hie} geschach.\\ 
 & er hât sich zuht gegen mir enbart.\\ 
 & mîn rede ist al ze vil gespart.\\ 
5 & hie \textbf{en}sol niht mêr geswigen sîn."\\ 
 & zuo ir gaste sprach diu künigîn:\\ 
 & "hêrre, ein wirtîn reden muoz.\\ 
 & \textit{e}in kus erwarp mir iuweren gruoz.\\ 
 & \textbf{ouch butet ir} dienest dâ her în.\\ 
10 & \textbf{sus} seite ein juncvrouwe mîn.\\ 
 & \textbf{des} habent \textbf{uns} geste niht gewenet.\\ 
 & des hât mîn herze sich ge\textit{s}enet.\\ 
 & hêrre, ich vrâge iuch mære,\\ 
 & wannen iuwer reise wære."\\ 
15 & "\textit{\begin{large}V\end{large}}rouwe, ich reit bî disem tage\\ 
 & von einem man, \textbf{den ich} \textbf{in} klage\\ 
 & liez mit triuwen âne \textbf{schranz}.\\ 
 & der vürste \textbf{heizet} Gurnemanz;\\ 
 & \textbf{von Graharz ist er} genant.\\ 
20 & \textbf{dannen} \textbf{reit ich} in \textbf{daz} lant."\\ 
 & \textbf{dô} sprach diu \textbf{zühterîche} maget:\\ 
 & "het \textbf{ez anders iemen mir} gesaget,\\ 
 & der volge \textbf{ime würde niht} \textit{ver}jehen,\\ 
 & daz ez \textbf{wære eines tages} geschehen;\\ 
25 & wand welich mîn bote \textbf{ie} baldest reit,\\ 
 & die reise er zwêne tage \textbf{vermeit}.\\ 
 & \textbf{sîn swester} was diu \textbf{muoter mîn}.\\ 
 & iuweres wirtes tohter schîn\\ 
 & sich \textbf{vor} jâmer \textbf{krenken} mac.\\ 
30 & wir haben manigen \textbf{sûren} tac\\ 
\end{tabular}
\scriptsize
\line(1,0){75} \newline
m n o Fr69 \newline
\line(1,0){75} \newline
\textbf{15} \textit{Initiale} m   $\cdot$ \textit{Capitulumzeichen} n  \newline
\line(1,0){75} \newline
\textbf{1} sach] \textit{om.} Fr69 \textbf{2} geschach] beschach n o \textbf{4} rede] rode o  $\cdot$ al ze] also o \textbf{5} ensol] sol n o \textbf{7} ein wirtîn] enwirtin o \textbf{8} ein] Er ein m  $\cdot$ iuweren] iren m n (o) \textbf{9} butet] bútet n  $\cdot$ dâ] \textit{om.} n do o \textbf{11} des] Das o \textbf{12} gesenet] gelenet m n o \textbf{13} iuch] ich n \textbf{14} wannen] Wennen n (o) \textbf{15} Vrouwe] Arouwe m \textbf{16} in] nuͯ n \textbf{17} liez] Liesse n \textbf{18} Gurnemanz] gurnemancz m o gúrnemantz n \textbf{19} Graharz] graharcz m o grahartz n \textbf{20} dannen] Dennen n  $\cdot$ daz] dis Fr69 \textbf{23} ime würde] wurde jme n (o) (Fr69)  $\cdot$ verjehen] iehen m \textbf{24} geschehen] beschehen n o \textbf{25} welich] swelch Fr69  $\cdot$ baldest] ballest o \textbf{26} zwêne] zwein Fr69  $\cdot$ tage] tagen Fr69  $\cdot$ vermeit] wermeit o \textbf{28} iuweres] Jres m Jrs n o  $\cdot$ tohter] siner dochter n (o) \textbf{29} sich] Sich ouch n (o) \newline
\end{minipage}
\end{table}
\newpage
\begin{table}[ht]
\begin{minipage}[t]{0.5\linewidth}
\small
\begin{center}*G
\end{center}
\begin{tabular}{rl}
 & \textbf{dâr nâch er guotlîch} an mich sach,\\ 
 & sît \textbf{uns} ze sitzene \textbf{hie} geschach.\\ 
 & er hât sich zuht gein mir enbart.\\ 
 & mîn rede ist alze vil gespart.\\ 
5 & hie sol niht mê geswigen sîn."\\ 
 & ze ir gaste sprach diu künigîn:\\ 
 & "hêrre, ein wirtîn reden muoz.\\ 
 & ein kus erwarp mir iweren gruoz.\\ 
 & \textbf{ouch enbut ir} dienst dâ her în,\\ 
10 & \textbf{als} sagte ein juncvrouwe mîn.\\ 
 & \textbf{sône} hânt \textbf{mich} geste niht gewent.\\ 
 & des hât mîn herze sich \textbf{versent}.\\ 
 & hêrre, ich vrâge iuch mære,\\ 
 & wannen iwer reise wære."\\ 
15 & "vrouwe, ich reit bî disem tage\\ 
 & \begin{large}V\end{large}on einem man, \textbf{der mich} \textbf{mit} klage\\ 
 & liez mit triwen âne \textbf{schranz}.\\ 
 & der vürste \textbf{heizet} Gurnomanz;\\ 
 & \textbf{von Graharz i\textit{st} er} genant.\\ 
20 & \textbf{dannen} \textbf{r\textit{eit} ich} \textbf{hiut} \textit{i}n \textbf{diz} lant."\\ 
 & \textbf{alsus} sprach diu \textbf{werde} maget:\\ 
 & "\textit{h}et\textbf{z anders iemen mir} gesaget,\\ 
 & der volge \textbf{im nimer würde} vergehen,\\ 
 & daz ez \textbf{eines tages m\textit{ö}ht} geschehen;\\ 
25 & \textit{wan} \textit{s}welch mîn bot \textbf{al} baldest reit,\\ 
 & \textit{d}ie reise er zwêne tage \textbf{meit}.\\ 
 & \textbf{\textit{m}în muoter} was diu \textbf{swester sîn},\\ 
 & \textit{i}wers wirtes. \textbf{sîner} tohter schîn\\ 
 & sich \textbf{ouch} \textbf{von} jâmer \textbf{krenken} mac.\\ 
30 & wir haben manigen \textbf{swæren} \textit{t}ac\\ 
\end{tabular}
\scriptsize
\line(1,0){75} \newline
G I O L M Q R Z Fr47 \newline
\line(1,0){75} \newline
\textbf{5} \textit{Initiale} I  \textbf{13} \textit{Initiale} M  \textbf{16} \textit{Initiale} G  \textbf{19} \textit{Initiale} Z  \textbf{21} \textit{Initiale} I  \textbf{25} \textit{Initiale} O  \newline
\line(1,0){75} \newline
\textbf{1} er guotlîch] er goͮlich G guͯtlich er R  $\cdot$ mich] si O \textbf{3} Er hat sich gein mir zuͯht enbart L  $\cdot$ [Er sich z]: Er sich zucht hat gen mir enbart Q  $\cdot$ sich] sin I (Fr47)  $\cdot$ gein] geim O  $\cdot$ enbart] gewart I [*bart]: erbart O \textbf{4} mîn] Mit M Meiner Fr47 \textbf{5} sol] ensol R \textbf{8} kus] kuͤsh I (M)  $\cdot$ iweren] úwer R (Fr47) \textbf{9} enbut] but I (L) bot M enpewt Q (R) enbot Fr47  $\cdot$ ir] dir Q  $\cdot$ dâ her] do her Q har R  $\cdot$ în] hin M \textbf{10} als] Svs Z  $\cdot$ sagte] seit mir I seit O (M) (Q) (R) (Z) (Fr47) \textbf{11} sône hânt] So hant O L (Q) Sin hant M Wir hand R Des enhabnt Z Si habent Fr47  $\cdot$ mich] vns O M (Q) R Z (Fr47) vnz dez L \textbf{12} da von hat sich min herze [versent]: vershent I  $\cdot$ sich] [mich]: sich O \textbf{13} vrâge iuch] vragete uͯch gerne L \textbf{14} reise] Reisen R \textbf{15} disem tage] disen tagen R \textbf{16} der mich] den ich L Z  $\cdot$ mit] in L \textbf{18} Gurnomanz] Gurnamanz I kvrnemanz O Gvrnomantz L (Q) gurnemancz M Gurnamancz R gvrnemantz Z \textbf{19} von] Vnd ist von L  $\cdot$ Graharz] grahars Q Graharcz R  $\cdot$ ist er] ir er G \textit{om.} L \textbf{20} dannen] Von dem L  $\cdot$ reit] rite G  $\cdot$ in diz] :n diz G in daz I (M) \textbf{21} alsus] Do L  $\cdot$ werde] mynnecliche L \textbf{22} hetz] :etz G Hette mirs sust R  $\cdot$ iemen mir] nimant mir M mir ymant Q yeman R \textbf{23} im nimer würde] wurde im niht O (Q) (R) (Z) wirde im niht L yn wurde ym nicht M \textbf{24} ez] \textit{om.} O Q  $\cdot$ möht] moht G (I) O (L) (M) (Q) Z  $\cdot$ geschehen] beschechen R \textbf{25} wan swelch] :welch G wan swenn I ÷an swelich O Wan welch L (M) (Q) (Z) Wan welhe R  $\cdot$ al] ie I O (L) M (Q) (R) Z  $\cdot$ baldest] badest R \textbf{26} die] :ie G \textbf{27} mîn] :in G \textbf{28} iwers] :wers G  $\cdot$ sîner] \textit{om.} L \textbf{30} swæren] svͦren O (L) (M) (R) susszen Q  $\cdot$ tac] :ach G \newline
\end{minipage}
\hspace{0.5cm}
\begin{minipage}[t]{0.5\linewidth}
\small
\begin{center}*T
\end{center}
\begin{tabular}{rl}
 & \textbf{dâr nâch er guotlîche} an mich sach,\\ 
 & sît \textbf{im} ze sitzenne geschach.\\ 
 & er hât sich zuht gegen mir enbart.\\ 
 & mîn rede ist alze vil gespart.\\ 
5 & hie sol niht mê geswigen sîn."\\ 
 & zir gaste sprach diu künegîn:\\ 
 & "hêrre, ein wirtîn reden muoz.\\ 
 & ein kus erwarp mir iuwern gruoz.\\ 
 & \textbf{\textit{i}r en\textit{b}utet ouch} dienst dâ her în.\\ 
10 & \textbf{daz} sagete ein juncvrouwe mîn.\\ 
 & \textbf{ouch} hânt\textbf{s} \textbf{uns} geste niht gewent.\\ 
 & des hât mîn herze sich \textbf{gesent}.\\ 
 & Hêrre, ich vrâgiuch mære,\\ 
 & wannen iuwer reise wære."\\ 
15 & "\begin{large}V\end{large}rouwe, ich reit bî disem tage\\ 
 & von einem man, \textbf{der mich} \textbf{in} klage\\ 
 & liez mit triuwen âne \textbf{schranz}.\\ 
 & der vürste \textbf{hiez} Gurnemanz\\ 
 & \textbf{unde ist von Greharz} genant.\\ 
20 & \textbf{von dem} \textbf{ich reit} \textbf{hiute} in \textbf{diz} lant."\\ 
 & \textbf{Dô} sprach diu \textbf{minneclîchiu} maget:\\ 
 & "hete \textbf{ieman anders mirz} gesaget,\\ 
 & der volge \textbf{w\textit{ü}rd im niht} verjehen,\\ 
 & daz \textbf{eines tages wære} geschehen;\\ 
25 & wan swelch mîn bote \textbf{ie} baldeste reit,\\ 
 & die reise er zwêne tage \textbf{meit}.\\ 
 & \textbf{sîn swester} was diu \textbf{muoter mîn}.\\ 
 & iuwers wirtes tohter schîn\\ 
 & sich \textbf{ouch} \textbf{von} jâmere \textbf{senen} mac.\\ 
30 & wir hân \textbf{vil} manegen \textbf{sûren} tac\\ 
\end{tabular}
\scriptsize
\line(1,0){75} \newline
T U V W \newline
\line(1,0){75} \newline
\textbf{7} \textit{Initiale} W  \textbf{13} \textit{Majuskel} T  \textbf{15} \textit{Initiale} T U  \textbf{21} \textit{Majuskel} T  \newline
\line(1,0){75} \newline
\textbf{2} sitzenne geschach] sitzenne hie gesach U (V) streitene hie geschach W \textbf{3} sich] sine V \textbf{4} alze] [al*]: alze V also W \textbf{6} zir] Zuͦ dem U \textbf{8} ein] Einen V  $\cdot$ iuwern] v́wer V \textbf{9} Eúwern dienst entbitten ir do herin W  $\cdot$ ir enbutet] er en bivtet T Jr enbuͦte U \textbf{11} ouch hânts] [*]: Dez habent V So hand W \textbf{12} gesent] versenet W \textbf{13} vrâgiuch] vrâgiv T vrage U \textbf{16} der mich] [de*]: den ich V \textbf{18} vürste] wir W  $\cdot$ hiez] heizet U (V) (W)  $\cdot$ Gurnemanz] Gvrnnemanz T gurnamantz W \textbf{19} Greharz] grahars W \textbf{20} ich reit] rait ich W \textbf{21} minneclîchiu maget] [*]: minnencliche maget V \textbf{22} hete] Het es W \textbf{23} der] Die U  $\cdot$ volge] volgo V  $\cdot$ würd im] [im]: wurd im T im wurde W \textbf{24} daz] Das daz W  $\cdot$ eines] sines U [*]: ez einez V  $\cdot$ wære] muͤge W  $\cdot$ geschehen] [ges*hen]: gesehen U \textbf{25} swelch] welch W  $\cdot$ baldeste] deste balde U \textit{om.} W  $\cdot$ reit] gerait W \textbf{26} meit] [*]: vermeit V \textbf{29} senen] sinen U \textbf{30} sûren] suͤssen W \newline
\end{minipage}
\end{table}
\end{document}
