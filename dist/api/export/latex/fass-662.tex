\documentclass[8pt,a4paper,notitlepage]{article}
\usepackage{fullpage}
\usepackage{ulem}
\usepackage{xltxtra}
\usepackage{datetime}
\renewcommand{\dateseparator}{.}
\dmyyyydate
\usepackage{fancyhdr}
\usepackage{ifthen}
\pagestyle{fancy}
\fancyhf{}
\renewcommand{\headrulewidth}{0pt}
\fancyfoot[L]{\ifthenelse{\value{page}=1}{\today, \currenttime{} Uhr}{}}
\begin{document}
\begin{table}[ht]
\begin{minipage}[t]{0.5\linewidth}
\small
\begin{center}*D
\end{center}
\begin{tabular}{rl}
\textbf{662} & stuont gein ein ander âne wanc,\\ 
 & daz si nie valsch underswanc.\\ 
 & \textit{\begin{large}A\end{large}}rnive wart des weinens innen.\\ 
 & si sprach: "ir sult beginnen\\ 
5 & vreude mit vreuden schalle;\\ 
 & hêrre, \textbf{daz} trœst uns alle.\\ 
 & gein der riwe sult ir sîn ze wer.\\ 
 & hie kumt der herzoginne her;\\ 
 & daz trœstet iuch vürbaz schiere."\\ 
10 & Herberge, baniere\\ 
 & sach Arnive und Gawan\\ 
 & manege vüeren ûf den plân,\\ 
 & bî den allen niht wan \textbf{einen} schilt.\\ 
 & des wâpen wâren \textbf{sus} gezilt,\\ 
15 & daz in Arnive \textbf{niht} erkande:\\ 
 & Isajesen si nande\\ 
 & \textbf{den} marschalc Utepandragun.\\ 
 & den vuort ein ander Bertun,\\ 
 & mit den schœnen schenkeln Maurin,\\ 
20 & der marschalc der künegîn.\\ 
 & Arnive wesse wênec des:\\ 
 & Utepandragun und Isajes\\ 
 & wâren bêde erstorben.\\ 
 & Maurin het erworben\\ 
25 & sînes vater ambet, daz \textbf{was} reht.\\ 
 & gein dem urvar ûf \textbf{den} anger sleht\\ 
 & reit diu grôze mahinante.\\ 
 & der vrouw\textit{en} sarjante\\ 
 & herberge nâmen,\\ 
30 & die vrouwen wol gezâmen,\\ 
\end{tabular}
\scriptsize
\line(1,0){75} \newline
D \newline
\line(1,0){75} \newline
\textbf{3} \textit{Initiale} D  \textbf{10} \textit{Majuskel} D  \newline
\line(1,0){75} \newline
\textbf{3} Arnive] ÷Rnîve D \textbf{16} Isajesen] Jsaiesen D \textbf{17} Utepandragun] Vͦtepandragvn D \textbf{19} Maurin] Mavrin D \textbf{21} Arnive] Arnîve D \textbf{22} Utepandragun] Vͦtepandragvn D  $\cdot$ Isajes] Jsâies D \textbf{24} Maurin] Mavrin D \textbf{28} vrouwen] froͮw D \newline
\end{minipage}
\hspace{0.5cm}
\begin{minipage}[t]{0.5\linewidth}
\small
\begin{center}*m
\end{center}
\begin{tabular}{rl}
 & stuont gegen ein ander âne wanc,\\ 
 & daz si nie valsch underswanc.\\ 
 & Ar\textit{niv}e wart des weinens innen.\\ 
 & si sprach: "ir solt beginnen\\ 
5 & vröude mit vröuden schalle;\\ 
 & hêrre, \textbf{daz} trœst\textit{e}t uns alle.\\ 
 & gegen der riuwe solt ir sîn zuo wer.\\ 
 & hie kumt der herzogîn her;\\ 
 & daz trœstet iuch vürbaz schier."\\ 
10 & herberge, banier\\ 
 & sach Ar\textit{niv}e und Gawan\\ 
 & manige vüeren ûf den plân,\\ 
 & bî den allen niht wan \textbf{einen} schilt.\\ 
 & des wâpen wâren \textbf{sô} gezilt,\\ 
15 & daz in Ar\textit{niv}e erkande.\\ 
 & Isaies si \textbf{in} nande,\\ 
 & \textbf{den} marschalc Utrap\textit{a}ndragun.\\ 
 & den vuorte ein ander Britu\textit{n},\\ 
 & mit den schœnen schenk\textit{e}ln Maurin,\\ 
20 & der marschalc \textbf{was} der künigîn.\\ 
 & Ar\textit{niv}e wiste wênic des:\\ 
 & Ut\textit{ra}pan\textit{dr}agun und Isai\textit{e}s\\ 
 & wâren beide erstorben.\\ 
 & Maurin het erworben\\ 
25 & sînes vater ambaht, daz reht.\\ 
 & gegen dem urvar ûf \textbf{dem} anger sleht\\ 
 & reit diu grôze mahinante.\\ 
 & der vrowen sarjante\\ 
 & herberge \textbf{dô} nâmen,\\ 
30 & die vrouwen wol gezâmen,\\ 
\end{tabular}
\scriptsize
\line(1,0){75} \newline
m n o \newline
\line(1,0){75} \newline
\newline
\line(1,0){75} \newline
\textbf{3} Arnive] Arune m Arniwe n  $\cdot$ des weinens] das veinens o \textbf{5} vröuden] frouide o \textbf{6} trœstet] troͯstent m (o) \textbf{11} Arnive] aruͯne m arniwe n \textbf{13} allen] ellen o \textbf{14} gezilt] gezelt o \textbf{15} Arnive] arune m arniwe n \textbf{16} Isaies] Jsaies m n o \textbf{17} Utrapandragun] vtrappendragun m uter pendragun n vter pentraguͯn o \textbf{18} Britun] brittum m brituͯm o \textbf{19} schenkeln] schenckeiln m schenckel o  $\cdot$ Maurin] muͯren o \textbf{21} Arnive] Arune m Arniwe n \textbf{22} Utrapandragun] Vterpantagun m Vterpendragun n Vter pandraguͯn o  $\cdot$ Isaies] jsaias m ysayes n \textbf{25} daz] sins o \textbf{26} ûf] gegen o  $\cdot$ dem] den n \textbf{27} mahinante] machmante o \textbf{28} Die frowen sariente o \textbf{29} dô] da o \newline
\end{minipage}
\end{table}
\newpage
\begin{table}[ht]
\begin{minipage}[t]{0.5\linewidth}
\small
\begin{center}*G
\end{center}
\begin{tabular}{rl}
 & stuont gein ein ander âne wanc,\\ 
 & daz si nie valsch underswanc.\\ 
 & \begin{large}A\end{large}rnive wart des weine\textit{n}s innen.\\ 
 & si sprach: "ir sult beginnen\\ 
5 & vröuden mit vröuden schalle;\\ 
 & hêrre, \textbf{sô} trœst \textbf{ir} uns alle.\\ 
 & gein der riwe sult ir sîn ze wer.\\ 
 & hie kumt der herzoginne her;\\ 
 & daz trœst iuch vürbaz schiere."\\ 
10 & herberge, \textbf{manige} baniere\\ 
 & sach Arnive unde Gawan\\ 
 & manige vüeren ûf den plân,\\ 
 & bî den allen niht wan \textbf{ein} schilt.\\ 
 & des wâpen wâren \textbf{sus} gezilt,\\ 
15 & daz in Arnive erkande.\\ 
 & Ysagesen si \textbf{in} nande\\ 
 & \textbf{des}, marschalc Utpandragun.\\ 
 & den vuorte ein ander Britun,\\ 
 & mit den schœnen schenkelen Maurin,\\ 
20 & der marschalc der küningîn.\\ 
 & Arnive wesse wênic des:\\ 
 & Utpandragun unde Ysages\\ 
 & wâren bêde erstorben.\\ 
 & Maurin het erworben\\ 
25 & sînes vater ambet, daz \textbf{was} reht.\\ 
 & gein dem urvar ûf \textbf{den} anger sleht\\ 
 & reit diu grôze mahinande.\\ 
 & der vrouwen sarjande\\ 
 & herberge nâmen,\\ 
30 & die vrouwen wol gezâmen,\\ 
\end{tabular}
\scriptsize
\line(1,0){75} \newline
G I L M Z Fr45 \newline
\line(1,0){75} \newline
\textbf{3} \textit{Initiale} G L Z  \textbf{11} \textit{Initiale} I  \newline
\line(1,0){75} \newline
\textbf{1} ein] an I \textit{om.} L \textbf{2} valsch] dehain valsh I \textbf{3} Arnive] Arniue I  $\cdot$ weinens] weines G \textbf{4} ir] herre ir Z \textbf{5} freude mit freuden shalle I  $\cdot$ Frovden mit schalle L  $\cdot$ Vrouden mit vroude schalle M \textbf{7} sult ir sîn] sýt L \textbf{10} herbergen mit manger paniere I  $\cdot$ herberge] \textit{om.} L Her get M  $\cdot$ manige] \textit{om.} Z \textbf{11} Arnive] arniue I (Fr45) \textbf{12} manige] mangen I \textbf{14} wâren sus] wart alsus I \textbf{15} Arnive] arniue I arnẏue Fr45 \textbf{16} Ysagesen] ẏsagesen G ysagensen I Je sa L Jsagesen M isagesen Fr45  $\cdot$ in] \textit{om.} Z \textbf{17} des] den I (L) Der M  $\cdot$ Utpandragun] vpandragun I adir pandragrun M uterpandragun Fr45 \textbf{18} den] de Fr45  $\cdot$ vuorte] vurt I (L) (Z) (Fr45)  $\cdot$ Britun] prituͦn I Brittvͯn L brittun Fr45 \textbf{19} den] dem Fr45  $\cdot$ schœnen] schone L  $\cdot$ schenkelen] senkeln I schenkelin Fr45  $\cdot$ Maurin] Mâvrin G Mavrin L mauryn M \textbf{21} Arnive] Arniue I Fr45 \textbf{22} Utpandragun] vpadragun I Vdir padragun M Vnpandragun Z  $\cdot$ Ysages] ẏsages G isages M \textbf{24} Maurin] Mavrin G L Marin M  $\cdot$ het] hat M \textbf{26} den] dē L M \textbf{30} vrouwen] vrouwe M \newline
\end{minipage}
\hspace{0.5cm}
\begin{minipage}[t]{0.5\linewidth}
\small
\begin{center}*T
\end{center}
\begin{tabular}{rl}
 & stuont gên ein ander âne wanc,\\ 
 & daz si nie valsch underswanc.\\ 
 & Arnyve wart des weine\textit{n}s innen.\\ 
 & si sprach: "ir solt beginnen\\ 
5 & vreude mit vreude schalle;\\ 
 & hêrre, \textbf{sô} trœst \textbf{ir} uns alle.\\ 
 & gên der riuwe sult ir sîn zuo wer.\\ 
 & hie kumt der herzogîn her;\\ 
 & daz trœste\textit{t} iuch vürbaz schiere."\\ 
10 & herberge, \textbf{manige} baniere\\ 
 & sach Arnyve und Gawan\\ 
 & manege vüeren ûf den plân,\\ 
 & bî den allen niht wan \textbf{ein} schilt.\\ 
 & des wâpen wâren \textbf{alle} gezilt,\\ 
15 & daz in Arnyve erkante.\\ 
 & Ysaiesen si \textbf{in} nante,\\ 
 & \textbf{den} marschalc Utpandragun.\\ 
 & den vuort ein ander Britun,\\ 
 & mit den schœnen schenkeln Maurin,\\ 
20 & der marschalc der künigîn.\\ 
 & Arnyve weste wênic des:\\ 
 & Utpandragun und Ysaies\\ 
 & wâren bêde erstorben.\\ 
 & Maurin het erworben\\ 
25 & sînes vater ambet, daz \textbf{was} reht.\\ 
 & gên dem urvar ûf \textbf{den} anger sleht\\ 
 & reit diu grôze mahinande.\\ 
 & der vrouwen sarjande\\ 
 & herberge nâmen,\\ 
30 & die vrouwen wol gezâmen,\\ 
\end{tabular}
\scriptsize
\line(1,0){75} \newline
Q R W V \newline
\line(1,0){75} \newline
\textbf{3} \textit{Initiale} R W V  \newline
\line(1,0){75} \newline
\textbf{2} si] sich R  $\cdot$ valsch] valschait W \textbf{3} Arnyve] Arniue Q Arnẏue R ARnyue W Arniue V  $\cdot$ weinens] weines Q weinen R \textbf{5} Froͯde mit froͯden schalle R (W) (V) \textbf{7} riuwe] treúwe W \textbf{8} herzogîn] herczoginen R [herzoͤg*]: herzoͤginne V \textbf{9} trœstet] troster Q \textbf{10} manige baniere] mangem baniere R [mani*]: manige baniere V \textbf{11} Arnyve] arniue Q V Arnyue R (W) \textbf{12} vüeren] fᵫre R \textbf{14} wâren] sahe man W  $\cdot$ alle] sust R (W) (V) \textbf{15} Arnyve] arniue Q V Arnyue R (W) \textbf{16} Ysaiesen] [ysasen]: ysaiesen Q Ysagesen R [Y*]: Ysaygesen V  $\cdot$ in] den W \textbf{17} Utpandragun] vtpandragún Q vterpandragun W [*]: vterpandragvn V \textbf{18} Britun] brittuͯn Q brittvn V \textbf{19} den] \textit{om.} W  $\cdot$ schenkeln] schenkel R  $\cdot$ Maurin] maúrin Q \textbf{20} der] [*]: waz der V \textbf{21} Arnyve] Arniue Q V Arnyue R W  $\cdot$ des] das R \textbf{22} Utpandragun] Vtpandragún Q Vterpandragun W [Vterpandragv*]: Vterpandragvn V  $\cdot$ Ysaies] ysagas R [*]: ysaies V \textbf{24} Maurin] Mawrin Q  $\cdot$ het] hat R \textbf{25} sînes] Sin R  $\cdot$ daz was] vnd waz R das ist W \textbf{26} urvar] úberfar W  $\cdot$ den] dem R V dē W \textbf{29} nâmen] [*]: do nomen V \newline
\end{minipage}
\end{table}
\end{document}
