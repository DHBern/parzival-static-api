\documentclass[8pt,a4paper,notitlepage]{article}
\usepackage{fullpage}
\usepackage{ulem}
\usepackage{xltxtra}
\usepackage{datetime}
\renewcommand{\dateseparator}{.}
\dmyyyydate
\usepackage{fancyhdr}
\usepackage{ifthen}
\pagestyle{fancy}
\fancyhf{}
\renewcommand{\headrulewidth}{0pt}
\fancyfoot[L]{\ifthenelse{\value{page}=1}{\today, \currenttime{} Uhr}{}}
\begin{document}
\begin{table}[ht]
\begin{minipage}[t]{0.5\linewidth}
\small
\begin{center}*D
\end{center}
\begin{tabular}{rl}
\textbf{251} & \textbf{\textit{\begin{large}I\end{large}}ch wæne}, \textbf{hêrre}, diu ist iu \textbf{niht bekant}.\\ 
 & Munsalvæsche ist si genant.\\ 
 & der \textbf{bürge} \textbf{wirtes} \textbf{royâm},\\ 
 & \textbf{Terre} de Salvæsche \textbf{ist} sîn nam.\\ 
5 & \textbf{ez} brâhte der alte Titurel\\ 
 & an sînen sun. \textbf{der künec} Frimutel,\\ 
 & \textbf{sus} hiez der werde wîgant.\\ 
 & manegen prîs erwarp sîn hant.\\ 
 & \textbf{der} lac \textbf{von} einer tjoste tôt,\\ 
10 & als im \textbf{diu} \textbf{minne} \textbf{dar} gebôt.\\ 
 & der selbe liez vier werdiu kint.\\ 
 & bî rîcheit driu \textbf{in} jâmer sint.\\ 
 & der vierde \textbf{hât} armuot,\\ 
 & durch got vür sünde er daz tuot.\\ 
15 & der \textbf{selbe heizet} Trevrizent.\\ 
 & Anfortas, sîn bruoder, lent.\\ 
 & der mac gerîten noch gegên\\ 
 & noch geligen noch gestên.\\ 
 & der ist \textbf{ûf} Munsalvæsche wirt.\\ 
20 & ungenâde in niht verbirt."\\ 
 & \textbf{Si sprach}: "hêrre, wæret ir \textbf{komen} dar\\ 
 & zuo der jæmerlîchen schar,\\ 
 & sô wære dem wirte worden rât\\ 
 & vil kumber\textit{s}, den er lange hât."\\ 
25 & Der Waleis zuo der meide sprach:\\ 
 & "\textbf{grœzlîch} wunder ich dâ sach\\ 
 & unt manege vrouwen wolgetân."\\ 
 & bî der stimme erkante si den man.\\ 
 & \textbf{\begin{large}D\end{large}ô sprach si}: "\textbf{dû bist} Parzival!\\ 
30 & nû sage \textbf{êt}, sæhe dû den Grâl\\ 
\end{tabular}
\scriptsize
\line(1,0){75} \newline
D \newline
\line(1,0){75} \newline
\textbf{1} \textit{Initiale} D  \textbf{21} \textit{Majuskel} D  \textbf{25} \textit{Majuskel} D  \textbf{29} \textit{Initiale} D  \newline
\line(1,0){75} \newline
\textbf{1} Ich] ÷ch \textit{nachträglich korrigiert zu:} Jch D \textbf{2} Munsalvæsche] Mvnsalvæsce D \textbf{4} Salvæsche] Salvæsce D \textbf{5} Titurel] Tytvrel D \textbf{8} erwarp] [*warp]: erwarp D \textbf{19} Munsalvæsche] Mvnsalvæsce D \textbf{24} kumbers] chvmberz D \newline
\end{minipage}
\hspace{0.5cm}
\begin{minipage}[t]{0.5\linewidth}
\small
\begin{center}*m
\end{center}
\begin{tabular}{rl}
 & \textbf{ich wæne}, diu ist iu \textbf{niht bekant}.\\ 
 & M\textit{u}n\textit{t}salvasche ist si genant.\\ 
 & der \textbf{bürge} \textbf{wirtes} \textbf{roiâm},\\ 
 & \textbf{Terre} d\textit{e} Salvasche \textbf{ist} sîn nam.\\ 
5 & \textbf{ez} brâhte der alte Titurel\\ 
 & an sînen sun. \textbf{rois} Frimutel,\\ 
 & \textbf{sus} hiez der werde wîgant.\\ 
 & manigen prîs erwarp sîn hant.\\ 
 & \textbf{er} lac \textbf{von} einer juste tôt,\\ 
10 & als ime \textbf{diu} \textbf{minne} \textbf{dar} gebôt.\\ 
 & der selb\textit{e} liez vier werd\textit{iu} k\textit{in}t.\\ 
 & bî rîcheit driu \textbf{in} jâmer sint.\\ 
 & der vierde, \textbf{der} \textbf{hât} armuot,\\ 
 & durch got vür sünde er daz tuot.\\ 
15 & der \textbf{sel\textit{b}e heizet} Trevrizent.\\ 
 & Anfortas, sîn bruoder, lent.\\ 
 & der mac gerîten noch gegên\\ 
 & noch geligen noch gestên.\\ 
 & der ist \textbf{vo\textit{n}} \textit{Mu}n\textit{t}salvasche wirt.\\ 
20 & ungnâde in niht verbirt."\\ 
 & \textbf{si sprach}: "hêrre, wæret ir \textbf{komen} dar\\ 
 & zuo der jâmerlîchen schar,\\ 
 & sô wære dem wirte worden rât\\ 
 & vil kumbers, den er lange hât."\\ 
25 & \begin{large}D\end{large}er Waleis zuo der megde sprach:\\ 
 & "\textbf{grôz} wunder ich dâ sach\\ 
 & und manic vrouwen w\textit{o}l getân."\\ 
 & bî der stimme erkante si den man.\\ 
 & \textbf{dô sprach si}: "\textbf{bistû} Parcifal?\\ 
30 & nû \textit{sage} \textbf{eht}, sæhe dû den Grâl\\ 
\end{tabular}
\scriptsize
\line(1,0){75} \newline
m n o Fr69 \newline
\line(1,0){75} \newline
\textbf{25} \textit{Initiale} m   $\cdot$ \textit{Capitulumzeichen} n  \newline
\line(1,0){75} \newline
\textbf{2} Muntsalvasche] Min saluasche m Monsaluasch n Mont saluasc o  $\cdot$ si] [zuͦ]: suͯ o \textbf{3} roiâm] friam n (o) \textbf{4} Terre de Salvasche] Terre do saluasce m Terre de saluasch n Terre de saluasc o Terra de salvasche Fr69 \textbf{5} Titurel] túttrel n tituͯrel o \textbf{6} Frimutel] froͧwel n froͯwel o \textbf{10} dar] [das]: dar m \textbf{11} selbe] selben m  $\cdot$ werdiu kint] werdekeit m \textbf{12} driu] dú Fr69 \textbf{13} der hât] hette n \textbf{15} selbe] selle m  $\cdot$ Trevrizent] treurizent m trenfirent n treuriczens o \textbf{16} Anfortas] An fortes n An fortas o \textbf{17} gerîten] nit riten n o  $\cdot$ gegên] gan n o \textbf{18} noch geligen] Geligen n o \textbf{19} von Muntsalvasche] von salua muin saluasche m vff muntsaluasch n uͯff mẏntsaluasc o \textbf{26} dâ] do n o \textbf{27} vrouwen] froͧwe n (o)  $\cdot$ wol] wel m \textbf{29} bistû] bist n \textbf{30} sage] \textit{om.} m \newline
\end{minipage}
\end{table}
\newpage
\begin{table}[ht]
\begin{minipage}[t]{0.5\linewidth}
\small
\begin{center}*G
\end{center}
\begin{tabular}{rl}
 & \textbf{ich wæne}, \textit{\textbf{hêrre}}, \textit{diu i}st iu \textbf{unbekant}.\\ 
 & Muntsalvatsche ist si genant.\\ 
 & der \textbf{burgære} \textbf{wirt} \textit{ist} \textbf{roiâ\textit{m}},\\ 
 & \textbf{der} de Salvatsche \textbf{was} sîn nam.\\ 
5 & \textbf{daz} brâht der alte Titurel\\ 
 & an sînen sun. \textbf{rois} Frimutel\\ 
 & hiez der werde wîgant.\\ 
 & \textbf{vil} manigen brîs erwarp sîn hant.\\ 
 & \textbf{der} lac \textbf{an} einer tjoste tôt,\\ 
10 & als im \textbf{ein} \textbf{künigîn} gebôt.\\ 
 & der selbe lie vier werdiu kint.\\ 
 & bî rîcheit drî \textbf{mit} jâmer sint.\\ 
 & der vierde, \textit{\textbf{der}} \textbf{\textit{hâ}t} armuot,\\ 
 & durch got vür sünde er daz tuot.\\ 
15 & der \textbf{ist geheizen} Trevrizzent.\\ 
 & Anfortas, sîn bruoder, lent.\\ 
 & der\textbf{ne} mac gerîten noch gegên\\ 
 & noch geligen noch gestên.\\ 
 & \begin{large}D\end{large}er ist \textbf{ûf} Muntsalvatsche wirt.\\ 
20 & ungenâde in niht verbirt."\\ 
 & "hêrre, w\textit{æ}rt ir \textbf{komen} dar\\ 
 & zuo der jæmerlîchen schar,\\ 
 & sô wære dem wirte worden rât\\ 
 & vil kumbers, den er lange hât."\\ 
25 & der Waleis zer meide sprach:\\ 
 & "\textbf{grôziu} wunder ich dâ sach\\ 
 & unde manige vrouwen wolgetân."\\ 
 & bî der stimme erkande si den man.\\ 
 & \textbf{si sprach}: "\textbf{\textit{dû} bis\textit{t}} Parzival!\\ 
30 & nû sage, sæhe dû den Grâl\\ 
\end{tabular}
\scriptsize
\line(1,0){75} \newline
G I O L M Q R Z Fr21 Fr36 Fr40 Fr51 \newline
\line(1,0){75} \newline
\textbf{1} \textit{Initiale} I L  \textbf{9} \textit{Initiale} O Fr21 Fr40   $\cdot$ \textit{Capitulumzeichen} L  \textbf{11} \textit{Initiale} Fr36  \textbf{13} \textit{Initiale} Z  \textbf{19} \textit{Initiale} G  \textbf{21} \textit{Initiale} I  \textbf{29} \textit{Initiale} Fr51  \newline
\line(1,0){75} \newline
\textbf{1} MErre die ist uͯch niht bekant L  $\cdot$ hêrre diu ist] sist G diu ist I  $\cdot$ iu] \textit{om.} R  $\cdot$ unbekant] vnerchant O (Q) (R) (Z) (Fr21) \textbf{2} Muntsalvatsche] mvntschalfatsch G muntschalusch I Munsalvatsche M Muntsalwasche Q Munsaluashe R Montsalvatsche Z Mvnsalvasche Fr21 mvnsalvashe Fr40 Mvͦntsaluasche Fr51 \textbf{3} burgære wirt] bvrge wirt O (L) (M) (Q) (Z) Fr21 (Fr40) wirte burg R borges wert Fr51  $\cdot$ ist] was G  $\cdot$ roiâm] roian G Q Fr40 boiam O Roiant L sie roÿam M \textbf{4} Terre de salvatsche ist sin lant L  $\cdot$ der de Salvatsche] der deschalvatsche G der shaluasche I Der desalvatsche O M Der den salwasche Q Derdesaluashe R Der tschalvasche Fr21 der desalvashe Fr40 Saluasche Fr51  $\cdot$ was] [ist]: was Q  $\cdot$ nam] man Q an Fr40 \textbf{5} Titurel] tytvrel O Týttuͯrel L titural M tunrel Q tyturel R Fr40 ti::: Fr36 tẏturel Fr51 \textbf{6} rois] \textit{om.} I  $\cdot$ Frimutel] frimuntel I (O) (Fr36) Frýmuͯtel L frimuͯtal M Fæimvtel Fr21 fẏrmitel Fr51 \textbf{7} hiez] Suͯs hiez L (Z)  $\cdot$ werde] selbe I \textbf{8} erwarp] ir warf Fr51 \textbf{9} der] ÷er O Her Fr51  $\cdot$ an] von L  $\cdot$ einer] ein Q \textbf{10} ein künigîn] die mýnne L (Q) (R) (Z) (Fr40)  $\cdot$ gebôt] dar gebot O L M Z Fr21 Fr36 Fr40 do gebot Q das gebot R \textbf{11} der] Die Fr51  $\cdot$ selbe] \textit{om.} I  $\cdot$ lie] lie hie I \textit{om.} Fr51  $\cdot$ werdiu] werde R \textbf{12} rîcheit] reichet Q  $\cdot$ drî mit] die in L \textbf{13} der hât] lidet G hat M \textbf{14} got] \textit{om.} Q  $\cdot$ sünde] sine sunde I  $\cdot$ daz] ez Z \textbf{15} der] Her Fr51  $\cdot$ Trevrizzent] trevrezent G treurezent I treurezzent O Trefriszent L trefrezcent M trischrizzent Q trefrissent R treverzzent Z trefresent Fr21 trefrezent Fr36 trefrizzent Fr40 treuerezent Fr51 \textbf{16} Anfortas] Anfrotas I An fortas M Q A:::tas Fr21 Amfortas Fr51 \textbf{17} derne] der I (R) (Fr21) Hern Fr51  $\cdot$ gerîten] ligen Fr51  $\cdot$ gegên] geen Q gan Fr51 \textbf{18} noch geligen] Noch lign Fr21 Nieweder liggen Fr51  $\cdot$ gestên] sten L (Fr51) \textbf{19} Der] Her Fr51  $\cdot$ Muntsalvatsche] mvntsalvatsch G muntshalvasche I Munsalfatsche M múnszalbalsche Q Munsalanach R montsalvatsche Z Mvnsalvatsche Fr21 mvnsalvashe Fr40 mvͦntsaluasche Fr51 \textbf{20} in] im Q  $\cdot$ verbirt] gebirt Q \textbf{21} hêrre] Si sprach herre O (M) (Q) (R) (Z) Fr36 Fr40 Sý sprach L (Fr21) (Fr51)  $\cdot$ wært] wart G O Fr36 war Fr51 \textbf{22} jæmerlîchen] iemmerliche Q iamerberen Fr51 \textbf{23} wirte] werde Fr51 \textbf{24} den er] der in L \textbf{25} Waleis] walois I waleise L waleẏs Fr51 \textbf{26} grôziu] Groz I Fr21 Fr51 Grozer O Groszlichen L Grose R  $\cdot$ wunder] kvmber L  $\cdot$ dâ] do Q Fr36  $\cdot$ sach] gesach O \textbf{27} unde manige vrouwen] vnd manc fro I (L) (Q) (Fr36) Fr40 Vnde mannigen vrouwen M Von megden frowen R \textbf{28} erkande] bekant Fr40 (Fr51)  $\cdot$ den man] in san I \textbf{29} dû bist] bistuz G dv bist ez O (M) (Q) (R) (Z) Fr21 Fr36 Fr40 (Fr51)  $\cdot$ Parzival] parzifal I L M Fr40 Parcifal O (Z) (Fr21) (Fr36) partzifal Q parczifal R parzẏual Fr51 \textbf{30} nû] \textit{om.} I L  $\cdot$ sage] sagan I (Q) (Fr40) Saga L \newline
\end{minipage}
\hspace{0.5cm}
\begin{minipage}[t]{0.5\linewidth}
\small
\begin{center}*T
\end{center}
\begin{tabular}{rl}
 & \textbf{si sprach}: "\textbf{hêrre}, diust iu \textbf{niht bekant}.\\ 
 & Munsalvasche ist si genant.\\ 
 & der \textbf{bürge} \textbf{wirtes} \textbf{roiâm},\\ 
 & \textbf{Terre} de Salvasche \textbf{ist} sîn nam.\\ 
5 & \textbf{ez} brâhte der alte Tyturel\\ 
 & an sînen sun. \textbf{rois} Frimutel,\\ 
 & \textbf{sus} hiez der werde wîgant.\\ 
 & \textbf{vil} manegen prîs erwarp sîn hant.\\ 
 & \textbf{der} lac \textbf{an} einer tjost tôt,\\ 
10 & als im \textbf{diu} \textbf{minne} \textbf{dar} gebôt.\\ 
 & der selbe lie vier werdiu kint.\\ 
 & bî rîcheit driu \textbf{in} jâmere sint.\\ 
 & der vierde \textbf{lîdet} armuot,\\ 
 & durch got vür sünde er daz tuot.\\ 
15 & der \textbf{ist geheizen} Trefrizent.\\ 
 & Anfortas, sîn bruoder, lent.\\ 
 & der mac gerîten noch gegân\\ 
 & noch geligen noch gestân.\\ 
 & der ist \textbf{ûf} Munsalvasche wirt.\\ 
20 & ungnâde in \textbf{dâ} niht verbirt."\\ 
 & \textbf{Si sprach}: "hêrre, wæret ir da\textit{r}\\ 
 & \textbf{komen}, zer jâmerlîchen schar,\\ 
 & sô wære dem wirte worden rât\\ 
 & vil kumbers, den er \textbf{nû} lange hât."\\ 
25 & \begin{large}D\end{large}er Waleis zuo der megde sprach:\\ 
 & "\textbf{grôz} wunder ich dâ sach\\ 
 & unde manege vrouwen wol getân."\\ 
 & Bî der stimme erkante si den man.\\ 
 & \textbf{dô sprach si}: "\textbf{dû bist} Parcifal!\\ 
30 & nû sage, sæhe dû den Grâl\\ 
\end{tabular}
\scriptsize
\line(1,0){75} \newline
T U V W Fr26 \newline
\line(1,0){75} \newline
\textbf{1} \textit{Überschrift:} Hie was her partzifal in den gral geritten vnd kam wider zuͦ sigunen vnd dem toten ritter W   $\cdot$ \textit{Platz für Illustration ausgespart} W   $\cdot$ \textit{Initiale} W  \textbf{21} \textit{Majuskel} T  \textbf{25} \textit{Initiale} T U V W Fr26  \textbf{28} \textit{Majuskel} T  \newline
\line(1,0){75} \newline
\textbf{1} si sprach hêrre] [*]: Jch wene V SY sprach W \textbf{2} Munsalvasche] Mvnsalvasce T Muͦnsalvasche U Muntsaluaths W \textbf{3} wirtes] [wúrte*]: wúrt ist V wirt W \textbf{4} Salvasche] Salvasce T salvatsche U saluast W \textbf{5} Tyturel] Tytuͦrel U Tẏturel V titurel W \textbf{6} sînen sun] seinem W  $\cdot$ rois] [*]: kv́nig V  $\cdot$ Frimutel] frimuͦtel U frimvntel V frimidel W \textbf{10} dar] gar W \textbf{14} vür] vnd fúr W \textbf{15} Trefrizent] treuerzent W \textbf{16} Anfortas] Anfortes W \textbf{17} mac] enmag W \textbf{18} noch geligen] Geligen W \textbf{19} ûf] zuͦ W  $\cdot$ Munsalvasche] Mvnsalvasce T Muͦntsalvasche U mvntsalvasche V muntsaluaths W \textbf{20} in dâ niht] nit do U in do niht V sein nit W  $\cdot$ verbirt] verwirt W \textbf{21} wæret] welt W  $\cdot$ dar] dar c T \textbf{23} dem] auch dem W \textbf{24} nû] \textit{om.} U W \textbf{25} Waleis] walleis V \textbf{26} grôz] Grozlich U (V)  $\cdot$ dâ] do U V W \textbf{27} vrouwen] vrôuwe T (U) (W) \textbf{28} stimme erkante] :::stande Fr26 \textbf{29} Parcifal] Parzifal T (V) partzifal W \textbf{30} sage] sage an W \newline
\end{minipage}
\end{table}
\end{document}
