\documentclass[8pt,a4paper,notitlepage]{article}
\usepackage{fullpage}
\usepackage{ulem}
\usepackage{xltxtra}
\usepackage{datetime}
\renewcommand{\dateseparator}{.}
\dmyyyydate
\usepackage{fancyhdr}
\usepackage{ifthen}
\pagestyle{fancy}
\fancyhf{}
\renewcommand{\headrulewidth}{0pt}
\fancyfoot[L]{\ifthenelse{\value{page}=1}{\today, \currenttime{} Uhr}{}}
\begin{document}
\begin{table}[ht]
\begin{minipage}[t]{0.5\linewidth}
\small
\begin{center}*D
\end{center}
\begin{tabular}{rl}
\textbf{331} & \begin{large}S\end{large}i\textbf{ne} mugen niht langer \textbf{hie} gestên,\\ 
 & ez muoz \textbf{nû} an \textbf{ein} scheiden gên.\\ 
 & dô sprach der Waleise\\ 
 & zArtuse dem Berteneise\\ 
5 & \textbf{unt} ze\textbf{n} rittern unt \textbf{zen} vrouwen,\\ 
 & er wolde ir urloup schouwen\\ 
 & unt mit ir hulden \textbf{vernemen}.\\ 
 & \textbf{des} \textbf{moht} \textbf{êt} niemen dâ gezemen,\\ 
 & daz er sô trûrec von in reit.\\ 
10 & ich wæne, \textbf{daz was} in allen leit.\\ 
 & Artus lobt im an die hant,\\ 
 & kœme iemer in sölhe nôt sîn lant,\\ 
 & als ez von Clamide gewan,\\ 
 & des \textbf{lasters} wolder pflihte hân.\\ 
15 & im wære ouch leit, daz Læhelin\\ 
 & im næme zwô \textbf{rîche} krône sîn.\\ 
 & vil dienstes im dâ maneger bôt.\\ 
 & \textbf{den helt} treip von in trûrens nôt.\\ 
 & Vrou Cunneware, diu clâre magt,\\ 
20 & nam den helt unverzagt\\ 
 & \textbf{mit} ir \textbf{hant} unt vuort in dan.\\ 
 & dô kusten mîn hêr Gawan.\\ 
 & \textbf{Dô} sprach \textbf{der manlîche}\\ 
 & ze dem \textbf{helde} ellens rîche:\\ 
25 & "ich weiz wol, \textbf{vriwent}, daz dîn vart\\ 
 & gein strîtes reise ist ungespart.\\ 
 & dâ geb dir got gelücke zuo\\ 
 & unt helfe ouch mir, daz ich getuo\\ 
 & dir noch \textbf{den} dienst, \textbf{als ich kan gern}.\\ 
30 & des \textbf{müeze} mich \textbf{sîn} kraft \textbf{gewern}."\\ 
\end{tabular}
\scriptsize
\line(1,0){75} \newline
D \newline
\line(1,0){75} \newline
\textbf{1} \textit{Initiale} D  \textbf{19} \textit{Majuskel} D  \textbf{23} \textit{Majuskel} D  \newline
\line(1,0){75} \newline
\textbf{13} Clamide] Chlamidê D \textbf{17} dienstes] diens D \newline
\end{minipage}
\hspace{0.5cm}
\begin{minipage}[t]{0.5\linewidth}
\small
\begin{center}*m
\end{center}
\begin{tabular}{rl}
 & \begin{large}S\end{large}i\textbf{ne} megen niht langer \textbf{hie} gestên,\\ 
 & ez muoz \textbf{nû} an scheiden gên.\\ 
 & dô sprach der Waleise\\ 
 & \dag~\dag\ Artuse dem Br\textit{i}tuneise\\ 
5 & \textbf{und} ze ritteren und vrouwen,\\ 
 & er wolte ir urloup schouwen\\ 
 & und mit ir \textbf{allen} hulden \textbf{nemen}.\\ 
 & \textbf{des} \textbf{en}\textbf{moht} \textbf{eht} niemen d\textit{â} gezemen,\\ 
 & daz er sô trûric von in reit.\\ 
10 & ich wæne, \textbf{daz was} in allen leit.\\ 
 & Artus lobete \textit{i}m an die hant,\\ 
 & k\textit{æm}e iemer in soliche nôt sîn lant,\\ 
 & als ez von Clamide gewan,\\ 
 & des \textbf{lasters} wolte er pflihte hân.\\ 
15 & ime wære ouch leit, daz Lehelin\\ 
 & im næme zwô krône sîn.\\ 
 & vil dienstes ime d\textit{â} maniger bôt.\\ 
 & \textbf{den helt} treip von in trûrens nôt.\\ 
 & vrouwe Cu\textit{nne}w\textit{a}re, diu clâre maget,\\ 
20 & nam den helt unverzaget\\ 
 & \textbf{an} ir \textbf{hant} und vuorte in dan.\\ 
 & dô kuste in mî\textit{n} hêrre Gawan\\ 
 & \textbf{und} sprach \textbf{gezogenlîche}\\ 
 & ze dem \textbf{helde} ellens rîche:\\ 
25 & "ich weiz wol, \textbf{vriunt}, daz dîn vart\\ 
 & gegen strîtes reise ist ungespart.\\ 
 & dâ gebe dir got glücke zuo\\ 
 & und helfe ouch mir, daz ich getuo\\ 
 & dir noch \textbf{den} dienest \textbf{mîn}, \textbf{des ich ger},\\ 
30 & des mich \textbf{des gotes} kraft \textbf{gewer}."\\ 
\end{tabular}
\scriptsize
\line(1,0){75} \newline
m n o \newline
\line(1,0){75} \newline
\textbf{1} \textit{Initiale} m   $\cdot$ \textit{Capitulumzeichen} n  \newline
\line(1,0){75} \newline
\textbf{1} Sine megen] Sú moͯchten n Sie mochten o \textbf{2} muoz] muͯs m (o) \textbf{3} Waleise] waleis o \textbf{4} Artuse] Artuͯse o  $\cdot$ Brituneise] bruttuneisse m britoneise n britaneise o \textbf{5} und ze] Zuͯ n (o)  $\cdot$ vrouwen] zuͯ froͧwen n (o) \textbf{6} ir] iren n [*ren]: jren o \textbf{7} allen hulden] alle hulde n o \textbf{8} des enmoht] Das moͯchte n Das mocht o  $\cdot$ dâ] do m n o \textbf{9} sô] sie o  $\cdot$ in] ir n o  $\cdot$ reit] [scheide]: reit o \textbf{11} im an] man m \textbf{12} kæme] Kune m \textbf{13} \textit{Versfolge 331.14-13} n  \textbf{14} pflihte] pflichtig o \textbf{17} dâ] do m n o \textbf{18} in] jme n \textbf{19} Cunneware] cumuwere m coneware n Conne waren o \textbf{21} vuorte] fuͯrt n (o) \textbf{22} kuste] kust n o  $\cdot$ mîn] mine m  $\cdot$ Gawan] her gawan n \textbf{24} ellens] allens o \textbf{28} getuo] duͯ n (o) \textbf{29} dir] Dar n o  $\cdot$ des] den o \textbf{30} Das mich die gottes crafft gewer n (o) \newline
\end{minipage}
\end{table}
\newpage
\begin{table}[ht]
\begin{minipage}[t]{0.5\linewidth}
\small
\begin{center}*G
\end{center}
\begin{tabular}{rl}
 & \textit{Si}\textbf{ne} m\textit{ugen} niht langer \textbf{sô} gestân,\\ 
 & ez muoz an \textbf{ein} scheiden gân.\\ 
 & dô sprach \textbf{aber} der Waleis\\ 
 & \begin{large}Z\end{large}e Artus dem Britaneis,\\ 
5 & ze rîteren unde \textbf{ze} vrouwen,\\ 
 & er wolt ir urloup schouwen\\ 
 & unde mit ir hulden \textbf{vernemen}.\\ 
 & \textbf{des} \textbf{dorfte} niemen dâ gezemen,\\ 
 & daz er sô trûric von in reit.\\ 
10 & ich wæne, \textbf{ez was} in allen leit.\\ 
 & Artus lobte im an die hant,\\ 
 & kœme immer in solhe nôt sîn lant,\\ 
 & als ez von Clamide gewan,\\ 
 & des \textbf{kumbers} wolter pflihte hân.\\ 
15 & im wære ouch leit, daz Lehelin\\ 
 & im næme zwô \textbf{rîche} krône sîn.\\ 
 & vil dienstes im dâ maniger bôt.\\ 
 & \textbf{den helt} treip von in trûrens nôt.\\ 
 & vrou Kuneware, diu clâre maget,\\ 
20 & nam den helt unverzaget\\ 
 & \textbf{mit} ir unde vuort in dan.\\ 
 & dô kust in mîn hêr Gawan\\ 
 & \textbf{unde} sprach \textbf{manlîche}\\ 
 & ze dem \textbf{helde} ellens rîche:\\ 
25 & "\textbf{vriunt}, ich weiz wol, daz dîn vart\\ 
 & gein strîtes reise ist ungespart.\\ 
 & dâ gebe dir got gelücke zuo\\ 
 & unde helfe ouch mir, daz ich getuo\\ 
 & dir noch \textbf{den} dienst, \textbf{als ich kan geren}.\\ 
30 & des \textbf{müeze} mich \textbf{sîn} kraft \textbf{geweren}."\\ 
\end{tabular}
\scriptsize
\line(1,0){75} \newline
G I O L M Q R Z Fr21 Fr27 Fr39 \newline
\line(1,0){75} \newline
\textbf{1} \textit{Initiale} L Q Fr21 Fr39  \textbf{2} \textit{Capitulumzeichen} R  \textbf{3} \textit{Überschrift:} Aventiwer wie Parzifal trovrich rait daz bvͦch nicht me vom in sait von chvnich artus schiet er dan nu gent Gawans auentiwer an I   $\cdot$ \textit{Initiale} I  \textbf{4} \textit{Initiale} G  \textbf{13} \textit{Initiale} O  \textbf{15} \textit{Initiale} I  \textbf{23} \textit{Initiale} Z  \newline
\line(1,0){75} \newline
\textbf{1} \textit{Versfolge 331.2-1} G   $\cdot$ Sine mugen] ezne mac G Si mvgen O  $\cdot$ niht langer] nu nih lenger I nv lenger niht O min nit lenger R  $\cdot$ sô] sust I (M) o\textit{m. } O R hie Z \textbf{2} an] nu an I (L) (M) (Q) Z (Fr27) (Fr39) et an O \textbf{3} dô] Da M  $\cdot$ aber] \textit{om.} O L M Q R Z Fr21 Fr27 Fr39  $\cdot$ Waleis] walois I waleys O waleýs L \textbf{4} Ze] Zv dem O  $\cdot$ Artus] Artuse I (O) (M) (Fr21) Fr39 Artuͯse L  $\cdot$ dem] \textit{om.} O  $\cdot$ Britaneis] pritonis I Britaneys O Brittaneýs L britteneis Q Britonis R brituneis Z brittaneis Fr39 \textbf{5} ze] Vnd zv Z \textbf{7} hulden] hulde Q R \textbf{8} des] desn I (L) (M) (R) (Fr21) (Fr27) (Fr39) Dor zu Q  $\cdot$ dorfte] enmoht Z  $\cdot$ niemen dâ] do nymant Q nyen da R niemant Z nieman do Fr39 \textbf{9} sô] \textit{om.} O \textbf{10} was] wer I (Z)  $\cdot$ in] \textit{om.} R \textbf{11} lobte] lopt I (O) Q (Z) Fr21 glopt L (R) (Fr39)  $\cdot$ an] in L  $\cdot$ die] sin I \textbf{12} lant] hand R \textbf{13} als] ÷ls O  $\cdot$ ez] her M erz Z  $\cdot$ Clamide] klamide I Glamide O Clamyde L \textbf{14} kumbers] lasters Z  $\cdot$ pflihte] teil R \textbf{15} Lehelin] lechelin R lehlin Fr27 \textbf{16} zwô] zu R  $\cdot$ krône] cronen R \textbf{17} dienstes] dienst I Fr39  $\cdot$ dâ] do O L Q R Fr39 \textbf{18} treip] reip I  $\cdot$ in] ym M \textbf{19} Kuneware] kunware I (M) Gvnware O Cvneware L (Fr39) kúnware Q Cuͦnware R kunneware Z Kvnw::: Fr21 :::unwar Fr27  $\cdot$ clâre] werde R \textbf{21} ir] ir hant O (Z) ir henden L (M) (Q) R Fr39 ir hend::: Fr21 :::den Fr27  $\cdot$ vuort] nam Q \textbf{22} dô] Da M Z  $\cdot$ in] \textit{om.} R \textbf{23} unde sprach] Do sprach der O L Q Z (Fr21) Fr39 Da Sprach dy M De sprach der R ::: der Fr27 \textbf{24} ze] Zede O  $\cdot$ helde ellens] helde eren Q heldenllen R  $\cdot$ rîche] richen I \textbf{25} wol] \textit{om.} M  $\cdot$ daz] \textit{om.} O \textbf{26} ungespart] gespart O \textbf{27} Dir gebe got glucke dor zu Q \textbf{28} ouch] \textit{om.} M \textbf{29} kan] \textit{om.} M  $\cdot$ geren] gerne I getan R \textbf{30} müeze] mvͦz O (Q)  $\cdot$ mich] ich O  $\cdot$ kraft] kras O  $\cdot$ geweren] erlon R \newline
\end{minipage}
\hspace{0.5cm}
\begin{minipage}[t]{0.5\linewidth}
\small
\begin{center}*T
\end{center}
\begin{tabular}{rl}
 & \begin{large}S\end{large}i mugen niht langer \textbf{sus} gestân,\\ 
 & ez muoz \textbf{nû} an \textbf{ein} scheiden gân.\\ 
 & Dô sprach der Waleis\\ 
 & Ze Artuse dem Brituneis\\ 
5 & \textbf{unde} ze rîtern unde \textbf{ze} vrouwen,\\ 
 & er wolt ir urloup schouwen\\ 
 & unde mit ir hulden \textbf{vernemen}.\\ 
 & \textbf{ez} \textbf{en}\textbf{dorfte} niemen dâ gezemen,\\ 
 & daz er sô trûric von in reit.\\ 
10 & ich wæne, \textbf{ez wære} in allen leit.\\ 
 & Artus lobetim an die hant,\\ 
 & kæme iemer in solhe nôt sîn lant,\\ 
 & als ez von Clamide gewan,\\ 
 & des \textbf{kumbers} wolt er pflihte hân.\\ 
15 & im wære ouch leit, daz Lehelin\\ 
 & im næme zwô \textbf{rîche} krônen sîn.\\ 
 & Vil dienstes im dâ maneger bôt.\\ 
 & \textbf{in} treip von in \textbf{dâ} trûrens nôt.\\ 
 & Vrou Cunneware, diu clâre maget,\\ 
20 & nam den helt unverzaget\\ 
 & \textbf{an} ir \textbf{hant} unde vuortin dan.\\ 
 & Dô kustin mîn hêr Gawan.\\ 
 & \textbf{dô} sprach \textbf{der manlîche}\\ 
 & Ze dem \textbf{degene} ellens rîche:\\ 
25 & "\textbf{helt}, ich weiz wol, daz dîn vart\\ 
 & gegen strîtes reise ist un\textit{ge}spart.\\ 
 & dâ gebe dir got glücke zuo\\ 
 & unde helf ouch mir, daz ich getuo\\ 
 & dir noch dienst, \textbf{als ich kan gern}.\\ 
30 & des \textbf{müeze} mich \textbf{sîn} kraft \textbf{gewern}."\\ 
\end{tabular}
\scriptsize
\line(1,0){75} \newline
T U V W \newline
\line(1,0){75} \newline
\textbf{1} \textit{Initiale} T U W  \textbf{3} \textit{Initiale} V   $\cdot$ \textit{Majuskel} T  \textbf{4} \textit{Majuskel} T  \textbf{17} \textit{Majuskel} T  \textbf{19} \textit{Majuskel} T  \textbf{22} \textit{Majuskel} T  \textbf{24} \textit{Majuskel} T  \newline
\line(1,0){75} \newline
\textbf{1} Si] [S*]: Sv́nen V  $\cdot$ langer] lang W  $\cdot$ sus] [*stan]: so V hie W \textbf{2} ez] Fs W \textbf{3} Dô] So W  $\cdot$ Waleis] stolze walleis V \textbf{4} Artuse] artus W  $\cdot$ Brituneis] Brituͦneis U \textbf{5} unde ze rîtern] Zuͦ ritern U (V) (W)  $\cdot$ ze vrouwen] zuͦ den frawen W \textbf{7} ir hulden] [*]: ir aller hulden V irm vrlaub W  $\cdot$ vernemen] nemen V \textbf{8} ez endorfte] [*]: Dez enmoͤhte ech V Es endorffte in W  $\cdot$ dâ] do V W \textbf{9} sô] [*]: so T \textbf{11} Artus] [*]: Artvs V  $\cdot$ lobetim] gelobete im W  $\cdot$ die] der W \textbf{13} Als es do vor was dick getan W \textbf{15} Lehelin] lehalein W \textbf{16} næme] [*]: neme oͮch V  $\cdot$ krônen] [*rone]: crone U \textbf{17} dâ] do V W \textbf{18} in treip] Den helt treip V (W)  $\cdot$ dâ] do U \textit{om.} V W \textbf{19} Cunneware] kuͦmeware U kvnneware V kunnewar W \textbf{21} an] Mit W \textbf{23} [*]: Vnde sprach [*]: gezoͤgenliche V  $\cdot$ dô] Vnd W  $\cdot$ der] so W \textbf{25} dîn] die W \textbf{26} ungespart] vnspart T \textbf{27} dâ] Do U V W \textbf{29} dir noch dienst als] [D*]: Dir noch den dienst dez V Den dienst dir als W  $\cdot$ ich] \textit{om.} U  $\cdot$ kan] kuͦnde U \textit{om.} V willen W \textbf{30} müeze] muͦz U  $\cdot$ mich sîn kraft] sein krafft mich W \newline
\end{minipage}
\end{table}
\end{document}
