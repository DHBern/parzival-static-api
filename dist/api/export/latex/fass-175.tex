\documentclass[8pt,a4paper,notitlepage]{article}
\usepackage{fullpage}
\usepackage{ulem}
\usepackage{xltxtra}
\usepackage{datetime}
\renewcommand{\dateseparator}{.}
\dmyyyydate
\usepackage{fancyhdr}
\usepackage{ifthen}
\pagestyle{fancy}
\fancyhf{}
\renewcommand{\headrulewidth}{0pt}
\fancyfoot[L]{\ifthenelse{\value{page}=1}{\today, \currenttime{} Uhr}{}}
\begin{document}
\begin{table}[ht]
\begin{minipage}[t]{0.5\linewidth}
\small
\begin{center}*D
\end{center}
\begin{tabular}{rl}
\textbf{175} & dô muosen kleiniu stückelîn\\ 
 & al dâ von trunzûnen sîn.\\ 
 & \textbf{sus} stach er ir vünfe nider.\\ 
 & der wirt in nam unt vuorten wider.\\ 
5 & \textbf{al} dâ behielt er \textbf{schimpfes} prîs.\\ 
 & \textbf{er} wart ouch sît an strîte wîs.\\ 
 & \begin{large}D\end{large}ie sîn rîten \textbf{gesâhen},\\ 
 & \textbf{al} die wîsen \textbf{im} des jâhen,\\ 
 & dâ \textbf{vüere} kunst unt ellen bî.\\ 
10 & "\textbf{nû wirt mîn hêrre} jâmers vrî.\\ 
 & \textit{s}ich mac \textbf{nû jungen wol} sîn leben.\\ 
 & er sol im ze wîbe geben\\ 
 & sîne tohter, unser vrouwen.\\ 
 & ob wir in bî witzen schouwen,\\ 
15 & sô \textbf{lischet} im \textbf{sîn} jâmers nôt.\\ 
 & vür sîner drîer sune tôt\\ 
 & ist im ein gelt ze hûs geriten.\\ 
 & nû hât in sælde niht vermiten."\\ 
 & Sus kom der \textbf{vürste} \textbf{sâbents} în.\\ 
20 & der tisch gedecket muose sîn.\\ 
 & \textbf{sîne tohter bat er} komen\\ 
 & \textbf{ze tische, alsus hân ichz} vernomen.\\ 
 & Dô er die magt komen sach,\\ 
 & nû hœret, wie der wirt sprach\\ 
25 & zuo der schœnen Liazen:\\ 
 & "\textbf{dû solt} \textbf{in} küssen lâzen,\\ 
 & disen ritter. biut im êre,\\ 
 & er vert mit sælden lêre.\\ 
 & ouch \textbf{solt} an iuch gedinget sîn,\\ 
30 & daz ir der meide ir vingerlîn\\ 
\end{tabular}
\scriptsize
\line(1,0){75} \newline
D \newline
\line(1,0){75} \newline
\textbf{7} \textit{Initiale} D  \textbf{19} \textit{Majuskel} D  \textbf{23} \textit{Majuskel} D  \newline
\line(1,0){75} \newline
\textbf{11} sich] ich D \newline
\end{minipage}
\hspace{0.5cm}
\begin{minipage}[t]{0.5\linewidth}
\small
\begin{center}*m
\end{center}
\begin{tabular}{rl}
 & dô muosen kleiniu stückelîn\\ 
 & al dâ von trunzûnen sîn.\\ 
 & \textbf{sus} stach er ir vünfe \textbf{dar} nider.\\ 
 & der wirt in nam und vuort in wi\textit{der}.\\ 
5 & \textbf{al}dâ behielt er \textbf{schimpfes} prîs.\\ 
 & \textbf{er} wart ouch sît an strîte wîs.\\ 
 & die sîn rîten \textbf{gesâhen},\\ 
 & \textbf{alle} die wîsen des jâhen,\\ 
 & dâr \textbf{vüere} kunst und ellen bî.\\ 
10 & "\textbf{nû wirt mîn hêrre} jâmers vrî.\\ 
 & sich mac \textbf{nû jungen wol} sîn leben.\\ 
 & er sol ime ze wîbe geben\\ 
 & sîne tohter, unser vrouwen.\\ 
 & ob wir in bî witzen schouwen,\\ 
15 & sô \textbf{leschet} ime \textbf{sîn} jâmers nôt.\\ 
 & vür sîner drîer sune tôt\\ 
 & ist im ein gelt ze hûse geriten.\\ 
 & nû hât in sælde niht vermiten."\\ 
 & \begin{large}S\end{large}us kam der \textbf{wirt} \textbf{des âbendes} în.\\ 
20 & der tisch gedecket muos sîn.\\ 
 & \textbf{sîne tohter bat er} kom\textit{e}n\\ 
 & \textbf{ze tisch, als ich ez hân} vernomen.\\ 
 & dô er die maget komen sach,\\ 
 & nû hœret, wie der wirt sprach\\ 
25 & zuo der schœnen Liazen:\\ 
 & "\textbf{dû solt} \textbf{dich} küssen lâzen\\ 
 & dise\textit{n} ritter. biut ime êre,\\ 
 & er ver\textit{t} mit sæl\textit{d}en lêre.\\ 
 & ouch \textbf{solte} an iuch gedinget sîn,\\ 
30 & daz ir der megde ir vingerlîn\\ 
\end{tabular}
\scriptsize
\line(1,0){75} \newline
m n o Fr69 \newline
\line(1,0){75} \newline
\textbf{19} \textit{Initiale} m  \newline
\line(1,0){75} \newline
\textbf{3} \textit{Versfolge 175.4-3} n  \textbf{4} wider] wirt m  $\cdot$ in] \textit{om.} Fr69 \textbf{9} dâr vüere] Do fúr n Do vor o  $\cdot$ ellen] allen n \textbf{12} geben] [gegen]: geben o \textbf{14} bî] \textit{om.} n \textbf{15} sô leschet] Sú loͯschet n Sie lahest o \textbf{16} drîer] drie o \textbf{19} Sus kam des obendes der wurt in n \textbf{20} muos] der muͯs o \textbf{21} komen] koman m \textbf{22} ez] \textit{om.} n o \textbf{23} komen] bat komen n \textbf{25} Liazen] liossen n loissen o \textbf{27} disen] Dissem m  $\cdot$ êre] \sout{dar} ere o \textbf{28} vert mit sælden] ver mit selben m  $\cdot$ lêre] mere n \textbf{29} ouch] Vnd n o \textbf{30} ir vingerlîn] ein fingerlin o \newline
\end{minipage}
\end{table}
\newpage
\begin{table}[ht]
\begin{minipage}[t]{0.5\linewidth}
\small
\begin{center}*G
\end{center}
\begin{tabular}{rl}
 & dâ muosen kleiniu stückelîn\\ 
 & al dâ von trunzûnen sîn.\\ 
 & \textbf{alsus} stach \textit{er} ir vünfe nider.\\ 
 & der wirt in nam unde vuorte in wider.\\ 
5 & \textbf{seht}, dâ behielt er \textbf{schimpfes} brîs\\ 
 & \textbf{unde} wart ouch sît an strîte wîs.\\ 
 & die sîn rîten \textbf{dâ} \textbf{gesâhen},\\ 
 & die wîsen \textbf{im} des jâhen,\\ 
 & dâ \textbf{vüere} kunst unde ellen bî.\\ 
10 & "\textbf{mîn hêrre wirt nû} jâmers vrî.\\ 
 & sich mac \textbf{wol ju\textit{n}gen nû} sîn leben.\\ 
 & er sol im ze wîbe geben\\ 
 & sîne tohter, unser vrouwen.\\ 
 & obe wirn bî witzen schouwen,\\ 
15 & sô \textbf{erlischet} im \textbf{sîn} \textit{j}âmers nôt.\\ 
 & vür sîner drîer sune tôt\\ 
 & ist im ein gelt ze hûs geriten.\\ 
 & nû hât in sælde niht vermiten."\\ 
 & sus kom der \textbf{vürste} \textbf{wider} în.\\ 
20 & der tisch \textit{ge}deckt muose sîn.\\ 
 & \textbf{der wirt hiez ze tische} komen\\ 
 & \textbf{sîne tohter, sus hân ich} vernomen.\\ 
 & dô er die maget komen sach,\\ 
 & nû hœret, wie der wirt sprach\\ 
25 & ze der schœnen Liazen:\\ 
 & "\textbf{nû soltû} küssen lâzen\\ 
 & disen rîter \textbf{unde} biut im êre.\\ 
 & er vert mit sælden lêre.\\ 
 & ouch \textbf{solt} an iuch gedinget sîn,\\ 
30 & daz ir der meide ir vingerlîn\\ 
\end{tabular}
\scriptsize
\line(1,0){75} \newline
G I O L M Q R Z Fr47 \newline
\line(1,0){75} \newline
\textbf{1} \textit{Initiale} Q  \textbf{7} \textit{Überschrift:} Hie hat parcifal sin erstez rennen getan mit dem sper Z   $\cdot$ \textit{Initiale} O R Z  \textbf{15} \textit{Initiale} L M Fr47  \textbf{19} \textit{Initiale} I  \newline
\line(1,0){75} \newline
\textbf{1} dâ] do I (O) (L) (Q) (R) (Fr47)  $\cdot$ muosen] muste Fr47  $\cdot$ kleiniu] kleine R \textbf{2} al] \textit{om.} Fr47  $\cdot$ von] vom R  $\cdot$ trunzûnen] drvmzeln O trurenzulen M den trinnlern Fr47 \textbf{3} alsus] Also O Q Fr47  $\cdot$ stach er] stach G R stache er L  $\cdot$ ir] \textit{om.} Z \textbf{4} \textit{Vers 175.4 fehlt} R  \textbf{5} seht] Al Z  $\cdot$ dâ] do Q R \textbf{6} wîs] was Z \textbf{7} die] ÷ie O Dy do Q  $\cdot$ dâ] \textit{om.} O L Q  $\cdot$ gesâhen] sahen O (M) Q \textbf{8} die] Alle die L  $\cdot$ des] das M R \textbf{9} kunst] crafft M  $\cdot$ unde] mit I  $\cdot$ ellen] ere Q \textbf{10} mîn hêrre wirt nû] Nv wirt min herre O (L) (M) (Q) (R) Z (Fr47) \textbf{11} sich] Jch Q  $\cdot$ wol jungen nû] wol ivgen nu G nu iungen wol I wol Jungen R  $\cdot$ sîn leben] [min]: sin leben O sin lib vnd leben R \textbf{12} im] ymms Q \textbf{15} sô] ÷O Fr47  $\cdot$ erlischet] lischet L erloͯscht R  $\cdot$ sîn] sins I (M) R  $\cdot$ jâmers] amers G \textbf{17} ist im] im ist I (Q) Jm Z Jst Fr47  $\cdot$ gelt] helt L \textbf{19} sus] Ausz Q So Fr47  $\cdot$ wider] des abendes Z \textbf{20} gedeckt muose] verdecht moͮse G muͦst gedeket R \textbf{21} Der wirt zcu tische hiez komen M (Fr47) \textbf{22} sus hân ich] han ich I als hab ich Q han ich sus R so han ich Fr47 \textbf{23} dô] Da M Z \textbf{25} schœnen] schone Q  $\cdot$ Liazen] liazzen I lýazzen L liaszin M lizazen Q lyasen R lyazzen Z liassen Fr47 \textbf{26} nû soltû] du solt I Dv solt niht Z Dv solt dich Fr47 \textbf{27} disen] Disem Z Fr47  $\cdot$ unde] \textit{om.} M  $\cdot$ biut] er bivte O (Q) \textbf{29} solt] sol O Fr47 \textbf{30} ir vingerlîn] [ein]: ir vingerlin G \newline
\end{minipage}
\hspace{0.5cm}
\begin{minipage}[t]{0.5\linewidth}
\small
\begin{center}*T
\end{center}
\begin{tabular}{rl}
 & dâ muosen kleiniu stückelîn\\ 
 & al dâ von trunzûnen sîn.\\ 
 & \textbf{Al dâ} stach er ir vünfe nider.\\ 
 & der wirt in nam unde vuortin wider.\\ 
5 & \textbf{seht}, dâ behielt er \textbf{schim\textit{p}fens} prîs\\ 
 & \textbf{unde} wart ouch sît an strîte wîs.\\ 
 & \begin{large}D\end{large}ie sîn rîten \textbf{dâ} \textbf{sâhen},\\ 
 & die wîsen \textbf{im} des jâhen,\\ 
 & dâ \textbf{wære} kunst unde ellen bî.\\ 
10 & "\textbf{Nû wirt mîn hêrre} jâmers vrî.\\ 
 & sich mac \textbf{wol jungen nû} sî\textit{n} leben.\\ 
 & er sol im ze wîbe geben\\ 
 & sîne tohte\textit{r}, \textit{u}nser vrouwen.\\ 
 & obe wirn bî witzen schouwen,\\ 
15 & sô \textbf{erleschet} im \textbf{sînes} jâmers nôt.\\ 
 & vür sîner drîer sune tôt\\ 
 & ist im ein gelt ze hûse geriten.\\ 
 & nû \textbf{en}hât in sælde niht vermiten."\\ 
 & Sus kom der \textbf{vürste} \textbf{wider} în.\\ 
20 & der tisch gedecket muose sîn.\\ 
 & \textbf{Der wirt ze tische hiez} komen\\ 
 & \textbf{sîne tohter, sus hân ich} vernomen.\\ 
 & dô er die maget komen sach,\\ 
 & nû hœret, wie der wirt sprach\\ 
25 & zuo der schœnen Lyazen:\\ 
 & "\textbf{nû solt dû} küssen lâzen\\ 
 & disen rîter \textbf{dich} \textbf{unde} biut im êre.\\ 
 & er vert mit sælde lêre.\\ 
 & ouch \textbf{sol} an iuch gedinget sîn,\\ 
30 & daz ir der megede ir vingerlîn\\ 
\end{tabular}
\scriptsize
\line(1,0){75} \newline
T U V W \newline
\line(1,0){75} \newline
\textbf{3} \textit{Majuskel} T  \textbf{7} \textit{Initiale} T U V W  \textbf{10} \textit{Majuskel} T  \textbf{19} \textit{Majuskel} T  \textbf{21} \textit{Majuskel} T  \newline
\line(1,0){75} \newline
\textbf{1} dâ] Do W  $\cdot$ muosen] mvesen T \textbf{2} von trunzûnen] vonden truͦwen U von den trunzen W \textbf{3} Al dâ] [*]: Svz V Do W  $\cdot$ nider] [*]: der nider V \textbf{4} \textit{nach 175.4: Einschub 175.4\textasciicircum1-4\textasciicircum2\textasciicircum6 (entspr. 'Conte du Graal', V. 1511-1534)} V  \textbf{5} seht dâ] Sit do U Sehent svz V Sehent do W  $\cdot$ schimpfens] schvmfphens T schimplichen U [schin*]: schinphes V schimpfes W \textbf{6} sît] sein W \textbf{7} rîten] riter U  $\cdot$ dâ] do U V W  $\cdot$ sâhen] saben W \textbf{9} kunst] kund W  $\cdot$ ellen] elle V \textbf{11} sîn] si T \textbf{13} tohter unser] tohter vnser vnser T \textbf{15} erleschet] erlischet U W  $\cdot$ sînes] sin V \textbf{16} vür] Von W  $\cdot$ drîer sune] súne dreyer W \textbf{18} enhât] het V hat W \textbf{20} muose] mvese T \textbf{21} ze tische hiez] hieß zuͦ tische W \textbf{25} Lyazen] lŷazen T lyasen V W \textbf{26} solt dû küssen] sol die kusche U solt du dich kússen W \textbf{27} dich] \textit{om.} W \textbf{28} sælde] selden V W \textbf{29} iuch] îv T \textbf{30} vingerlîn] virgerlein W \newline
\end{minipage}
\end{table}
\end{document}
