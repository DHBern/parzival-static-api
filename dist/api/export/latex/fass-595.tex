\documentclass[8pt,a4paper,notitlepage]{article}
\usepackage{fullpage}
\usepackage{ulem}
\usepackage{xltxtra}
\usepackage{datetime}
\renewcommand{\dateseparator}{.}
\dmyyyydate
\usepackage{fancyhdr}
\usepackage{ifthen}
\pagestyle{fancy}
\fancyhf{}
\renewcommand{\headrulewidth}{0pt}
\fancyfoot[L]{\ifthenelse{\value{page}=1}{\today, \currenttime{} Uhr}{}}
\begin{document}
\begin{table}[ht]
\begin{minipage}[t]{0.5\linewidth}
\small
\begin{center}*D
\end{center}
\begin{tabular}{rl}
\textbf{595} & \begin{large}G\end{large}awan sus mit kumber ranc,\\ 
 & ir \textbf{mugt} wol hœren, waz in twanc:\\ 
 & vür schande het er \textbf{an sich} genomen\\ 
 & des werden Turkoten komen.\\ 
5 & in twungen \textbf{ouch} wunden sêre\\ 
 & unt diu minne michels mêre\\ 
 & unt \textbf{der} \textbf{vier} \textbf{vrouwen} riwe,\\ 
 & wande\textbf{r} \textbf{sach an in} triwe.\\ 
 & Er bat si weinen verbern;\\ 
10 & \textbf{sîn munt dar zuo} begunde gern\\ 
 & \textbf{harnasch, ors} unde swert.\\ 
 & die vrouwen clâr und wert\\ 
 & vuorten Gawanen wider.\\ 
 & er bat si \textbf{vor} im gên dar nider,\\ 
15 & dâ die andern vrouwen wâren,\\ 
 & die süezen unt die clâren.\\ 
 & Gawan ûf \textbf{sînes} strîtes vart\\ 
 & balde aldâ gewâpent wart\\ 
 & \textbf{mit} weinenden liehten ougen.\\ 
20 & si tâtenz alsô tougen,\\ 
 & daz niemen vriesch diu mære\\ 
 & \textbf{niwan} der \textbf{kamerære};\\ 
 & der hiez sîn ors erstrîchen.\\ 
 & Gawan begunde slîchen,\\ 
25 & aldâ Gringuljete stuont.\\ 
 & doch was er \textbf{sô} sêre wunt,\\ 
 & \textbf{den schilt er} kûme \textbf{dar} \textbf{getruoc};\\ 
 & \textbf{der} was dürkel \textbf{ouch} genuoc.\\ 
 & \begin{large}Û\end{large}f\textbf{ez} ors saz hêr Gawan.\\ 
30 & dô kêrt er von der burc her dan\\ 
\end{tabular}
\scriptsize
\line(1,0){75} \newline
D Z Fr7 \newline
\line(1,0){75} \newline
\textbf{1} \textit{Überschrift:} Hie zevht her gawan vz als wunder Vnd wil aber striten mit einem ritter der qvam mit der hertzoginne Z   $\cdot$ \textit{Initiale} D Z  \textbf{9} \textit{Majuskel} D  \textbf{29} \textit{Initiale} D  \newline
\line(1,0){75} \newline
\textbf{4} des] der Z  $\cdot$ Turkoten] Tvrkoiten Z \textbf{6} diu] \textit{om.} Z \textbf{9} verbern] gar verbern Z \textbf{10} Da zv sin munt begunde gern Z \textbf{11} harnasch ors] Orss harnasch Z \textbf{13} wider] nider Z \textbf{14} vor] von Z  $\cdot$ dar nider] wider Z \textbf{15} \textit{Versfolge 596.16-15} Z  \textbf{17} Gawan] Gaw:: Fr7 \textbf{19} mit] Bi Z Fr7 \textbf{21} diu] ir Z \textbf{22} w::: Fr7  $\cdot$ niwan] Wan Z  $\cdot$ kamerære] kramere Z \textbf{24} Gawan] Ga::: Fr7 \textbf{25} Gringuljete] Gringvliet D (Z) \textbf{29} Ûfez] Vf sin Z  $\cdot$ hêr] min her Z \textbf{30} dô kêrt] Da kerte Z \newline
\end{minipage}
\hspace{0.5cm}
\begin{minipage}[t]{0.5\linewidth}
\small
\begin{center}*m
\end{center}
\begin{tabular}{rl}
 & Gawan sus \textit{m}it kumber ranc,\\ 
 & ir \textbf{m\textit{ö}ht} wol hœren, waz in twanc:\\ 
 & vür schande het er \textbf{sich an} genomen\\ 
 & des werden Turkoi\textit{t}en komen.\\ 
5 & in twungen \textbf{ouch} wunde\textit{n} sêre\\ 
 & und diu minne michels mêre\\ 
 & und \textbf{der} \textbf{vier} \textbf{vrouwen} riuwe,\\ 
 & wan \textbf{er} \textbf{sach an in} triuwe.\\ 
 & er bat si weinen verbern;\\ 
10 & \textbf{sîn munt dar zuo} begunde gern\\ 
 & \textbf{harnasch, ros} und swert.\\ 
 & die vrowen clâr und wert\\ 
 & vuorten Gawanen wider.\\ 
 & er bat si \textbf{vor} im gên dar nider,\\ 
15 & d\textit{â} die andern vrouwen wâren,\\ 
 & die süezen und die clâren.\\ 
 & Gawan ûf \textbf{sîn} strîtes vart\\ 
 & balde aldâ gewâpent wart\\ 
 & \textbf{bî} weinenden liehten ougen.\\ 
20 & si tâtenz alsô tougen,\\ 
 & daz niemen vr\textit{ie}sch diu mære\\ 
 & \textbf{niht wan} der \textbf{kamerære};\\ 
 & der hiez sîn ros erstrîchen.\\ 
 & Gawan begunde slîchen,\\ 
25 & aldâ Gringulet stuont.\\ 
 & doch was er \textbf{sô} sêre wunt,\\ 
 & \textbf{daz er den schilt} kûme \textbf{getruoc};\\ 
 & \textbf{der} was dürkel \textbf{al}genuoc.\\ 
 & ûf \textbf{daz} ros saz hêr Gawan.\\ 
30 & dô kêrt er von der burc her dan\\ 
\end{tabular}
\scriptsize
\line(1,0){75} \newline
m n o \newline
\line(1,0){75} \newline
\newline
\line(1,0){75} \newline
\textbf{1} mit] nit m \textbf{2} möht] moht m (o) moͯgent n \textbf{4} des] Das o  $\cdot$ Turkoiten] turcoiken m n tortoiken o \textbf{5} wunden] wunder m (o) \textbf{9} weinen] veinen o \textbf{12} clâr] clore n \textbf{15} dâ] Do m n o \textbf{18} balde] balda o \textbf{20} Die tatancz also schowen o \textbf{21} vriesch] freisch m \textbf{22} kamerære] kromere n (o) \textbf{27} getruoc] truͦg n \textbf{28} algenuoc] ouch genuͦg n (o) \newline
\end{minipage}
\end{table}
\newpage
\begin{table}[ht]
\begin{minipage}[t]{0.5\linewidth}
\small
\begin{center}*G
\end{center}
\begin{tabular}{rl}
 & Gawan sus mit kumber ranc,\\ 
 & ir \textbf{muget} wol hœren, waz in dwanc:\\ 
 & \multicolumn{1}{l}{ - - - }\\ 
 & \multicolumn{1}{l}{ - - - }\\ 
5 & in dwungen wunden sêre\\ 
 & unde diu minne michels mêre\\ 
 & unde \textbf{juncvrouwen} riuwe,\\ 
 & wan \textit{\textbf{diu}} \textbf{erschein\textit{de} i\textit{m}} triuwe.\\ 
 & er bat si weinen \textbf{gar} verbern;\\ 
10 & \textbf{dar zuo sîn munt} begunde gern\\ 
 & \textbf{orses, harnasch} unde swert.\\ 
 & die vrouwen clâ\textit{r} unde wert\\ 
 & vuorten Gawanen wider.\\ 
 & er bat si \textbf{von} im gên dar nider,\\ 
15 & dâ die andern vrouwen wâren,\\ 
 & die süezen unde die clâren.\\ 
 & Gawan ûf \textbf{sînes} strîtes vart\\ 
 & balde aldâ gewâpent wart\\ 
 & \textbf{bî} weinenden liehten ougen.\\ 
20 & si tâtenz alsô tougen,\\ 
 & daz niemen vri\textit{e}sch di\textit{u} mære\\ 
 & \textbf{wan} der \textbf{kamerære};\\ 
 & der hiez sîn ors erstrîchen.\\ 
 & Gawan begunde slîchen,\\ 
25 & al dâ Gringuliet stuont.\\ 
 & doch was er \textbf{sô} sêre wunt,\\ 
 & \textbf{den schilt er} kûme \textbf{dar} \textbf{truoc};\\ 
 & \textbf{der} was dürkel \textbf{ouch} genuoc.\\ 
 & ûf \textbf{sîn} ors saz hêr Gawan.\\ 
30 & dô kêrt er von der burc her dan\\ 
\end{tabular}
\scriptsize
\line(1,0){75} \newline
G I L M Z \newline
\line(1,0){75} \newline
\textbf{1} \textit{Überschrift} Hie zevht her gawan vz als wunder Vnd wil aber striten mit einem ritter der qvam mit der hertzoginne Z   $\cdot$ \textit{Initiale} L Z  \textbf{9} \textit{Initiale} I  \textbf{29} \textit{Initiale} I  \newline
\line(1,0){75} \newline
\textbf{3} \textit{Die Verse 595.3-4 fehlen} G I L M   $\cdot$ Fvr schande het er an sich genomen Z \textbf{4} Der werden Tvrkoiten komen Z \textbf{5} wunden] die wuͯnden L (M) ouch wunden Z \textbf{6} unde] \textit{om.} I  $\cdot$ diu] \textit{om.} Z  $\cdot$ minne] minne \sout{min} minne I \textbf{7} juncvrouwen] der jvngfrowen L (M) der vier frowen Z \textbf{8} diu erscheinde im] er erschein in G er sach an in L (M) Z \textbf{11} harnasch] harnaish I (M)  $\cdot$ orses] Rosz L (M) \textbf{12} clâr] clare G \textbf{13} Gawanen] Gawan I M  $\cdot$ wider] nider Z \textbf{14} von im gên] gen vor im L vor yme gen M  $\cdot$ dar nider] wider Z \textbf{15} \textit{Verfolge 595.16-15} Z   $\cdot$ die andern] ander I \textbf{21} vriesch] vriensch G  $\cdot$ diu] die G (L) (M) ir Z \textbf{22} kamerære] kramere L (M) Z \textbf{23} erstrîchen] strichen L (M) \textbf{25} Gringuliet] gringulier G Gringuͯliet L gringulet M \textbf{26} doch] do I  $\cdot$ sô] also M \textbf{27} den schilt er] Daz er den schilt L  $\cdot$ dar] \textit{om.} L M  $\cdot$ truoc] getruͤc I (M) (Z) \textbf{28} der] Er L M  $\cdot$ dürkel ouch] ouch dvrkel L \textbf{29} sîn ors] \textit{om.} I  $\cdot$ hêr] min her Z \textbf{30} dô] Da M Z  $\cdot$ kêrt] karte M (Z)  $\cdot$ her] \textit{om.} L hyn M \newline
\end{minipage}
\hspace{0.5cm}
\begin{minipage}[t]{0.5\linewidth}
\small
\begin{center}*T
\end{center}
\begin{tabular}{rl}
 & \begin{large}G\end{large}awan sus mit kumber ranc,\\ 
 & ir \textbf{mogt} wol hœren, waz in twanc:\\ 
 & \multicolumn{1}{l}{ - - - }\\ 
 & \multicolumn{1}{l}{ - - - }\\ 
5 & in twungen wunden sêre\\ 
 & und diu minne michels mêre\\ 
 & und \textbf{der} \textbf{juncvrouwen} riw\textit{e},\\ 
 & wan \textbf{er} \textbf{sach an ir} triwe.\\ 
 & er bat si weinen \textbf{gar} verbern;\\ 
10 & \textbf{dâ zuo sîn munt} begunde gern\\ 
 & \textbf{ros, harnasch} und swert.\\ 
 & die vrouwen clâren und wert\\ 
 & vuorten Gawanen wider.\\ 
 & er bat si \textbf{vor} im gên dar nider,\\ 
15 & d\textit{â} die andern vrouwen wâren,\\ 
 & die süezen und die clâren.\\ 
 & Gawan ûf \textbf{sînes} strîte\textit{s} vart\\ 
 & balde aldâ gewâpent wart\\ 
 & \textbf{bî} weinenden liehten ougen.\\ 
20 & si tâtenz alsô tougen,\\ 
 & daz nieman vriesch diu mære\\ 
 & \textbf{wan} der \textbf{krâmære};\\ 
 & der hiez sîn ros erstrîchen.\\ 
 & Gawan begunde slîchen,\\ 
25 & aldâ Krynguliet stuont.\\ 
 & doch was er \textbf{alsô} sêre wunt,\\ 
 & \textbf{den schilt er} kûme \textbf{getruoc};\\ 
 & \textbf{er} was dürkel \textbf{ouch} genuoc.\\ 
 & ûf \textbf{sîn} ros saz hêr Gawan.\\ 
30 & dô kêrt er von der burc her dan\\ 
\end{tabular}
\scriptsize
\line(1,0){75} \newline
Q R W V U \newline
\line(1,0){75} \newline
\textbf{1} \textit{Initiale} Q R W V  \textbf{29} \textit{Initiale} W  \newline
\line(1,0){75} \newline
\textbf{1} \textit{Die Verse 553.1-599.30 fehlen} U   $\cdot$ sus] als Q \textbf{2} wol] nun W \textbf{3} \textit{Die Verse 595.3-4 fehlen} Q R W   $\cdot$ \textit{Die Verse 595.3-4 sind am Rand nachgetragen und später radiert:} F::: s::: / Dez ::: V  \textbf{5} twungen] tungen R \textbf{7} der] [*]: der vier V  $\cdot$ riwe] rewenn Q \textbf{8} ir] in groß W in V \textbf{12} clâren] clar R (W) V \textbf{13} Gawanen] Gawinen R  $\cdot$ wider] [*er]: wider V \textbf{14} si] fv́ V  $\cdot$ im] in R [*]: im V \textbf{15} dâ] Do Q W Die R \textbf{17} Gawan] Gawin R  $\cdot$ sînes] sin R  $\cdot$ strîtes] streiten Q \textbf{18} balde] Schier R Gar balde W  $\cdot$ aldâ] er do W \textbf{19} weinenden liehten] weinenden lichten Q liechten weinenden W \textbf{21} vriesch] freisch R \textbf{22} krâmære] [kamrere]: kramere R \textbf{23} ros] roß ee W \textbf{24} Gawan] Gawin R \textbf{25} Krynguliet] [kinguliet]: kringuliet Q krungulet R kringuliet W cringulet V \textbf{26} doch] Do R  $\cdot$ was] wus W \textbf{27} Daz er den schilt vil kvme trvͦg V \textbf{28} dürkel] dunkel R V \textbf{29} saz hêr] [*]: saz V \textbf{30} her] hin R V \newline
\end{minipage}
\end{table}
\end{document}
