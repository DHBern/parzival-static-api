\documentclass[8pt,a4paper,notitlepage]{article}
\usepackage{fullpage}
\usepackage{ulem}
\usepackage{xltxtra}
\usepackage{datetime}
\renewcommand{\dateseparator}{.}
\dmyyyydate
\usepackage{fancyhdr}
\usepackage{ifthen}
\pagestyle{fancy}
\fancyhf{}
\renewcommand{\headrulewidth}{0pt}
\fancyfoot[L]{\ifthenelse{\value{page}=1}{\today, \currenttime{} Uhr}{}}
\begin{document}
\begin{table}[ht]
\begin{minipage}[t]{0.5\linewidth}
\small
\begin{center}*D
\end{center}
\begin{tabular}{rl}
\textbf{824} & \begin{large}W\end{large}elt ir nû hœren vürbaz?\\ 
 & sît über \textbf{lant} ein vrouwe saz,\\ 
 & vor aller \textbf{valscheit} bewart.\\ 
 & rîcheit und \textbf{hôher art}\\ 
5 & \textbf{ûf si beidiu} geerbet wâren.\\ 
 & si kunde \textbf{alsô} gebâren,\\ 
 & daz si mit rehter kiusche \textbf{warp}.\\ 
 & al \textbf{menneschlîch} gir an ir verdarp.\\ 
 & Werder liute warb umbe si genuoc,\\ 
10 & der etslîcher krône truoc,\\ 
 & unt manec vürste, ir genôz.\\ 
 & ir diemuot was sô grôz,\\ 
 & daz si sich dran niht wande.\\ 
 & vil grâven \textbf{von} ir lande\\ 
15 & begunden\textbf{z} \textbf{an} si hazzen,\\ 
 & wes si sich wolde lazzen,\\ 
 & daz si einen man niht næme,\\ 
 & der in ze hêrren \textbf{zæme}.\\ 
 & Si hete sich gar an got verlân,\\ 
20 & swaz \textbf{zornes wart gein ir} getân,\\ 
 & unschulde maneger \textbf{an si} \textbf{rach}.\\ 
 & einen hof si ir landes hêrren sprach.\\ 
 & manec bote ûz \textbf{verrem lande} vuor\\ 
 & hin zir. die man si gar verswuor,\\ 
25 & wan \textbf{den} si got bewîste;\\ 
 & des minne si gerne prîste.\\ 
 & Si was vürstîn in Brabant.\\ 
 & von Munsalvæsche wart gesant\\ 
 & der, den \textbf{der} swane brâhte\\ 
30 & unt des ir got gedâhte.\\ 
\end{tabular}
\scriptsize
\line(1,0){75} \newline
D \newline
\line(1,0){75} \newline
\textbf{1} \textit{Initiale} D  \textbf{9} \textit{Majuskel} D  \textbf{19} \textit{Majuskel} D  \textbf{27} \textit{Majuskel} D  \newline
\line(1,0){75} \newline
\textbf{28} Munsalvæsche] Mvnsalvæsce D \newline
\end{minipage}
\hspace{0.5cm}
\begin{minipage}[t]{0.5\linewidth}
\small
\begin{center}*m
\end{center}
\begin{tabular}{rl}
 & welt ir nû hœren vürbaz?\\ 
 & sît über \textbf{lanc} ein vrowe saz,\\ 
 & vor aller \textbf{valscheit} bewart.\\ 
 & rîcheit und \textbf{hôchvart}\\ 
5 & \textbf{ûf si} geerbet wâren.\\ 
 & si kunde \textbf{alsô} \textit{g}e\textit{b}âren,\\ 
 & daz si mit rehter kiusche \textbf{\textit{w}arp}.\\ 
 & al \textbf{menschlîch} gir an ir verdarp.\\ 
 & werder liute warp umb si genuoc,\\ 
10 & der etlîcher krône truoc,\\ 
 & und manic vürste, ir genôz.\\ 
 & ir diemuot was sô grôz,\\ 
 & daz si sich dar an niht wande.\\ 
 & vil grâven \textbf{von} ir lande\\ 
15 & begunden\textbf{z} \textbf{an} si hazzen,\\ 
 & wes si sich wolte lazzen,\\ 
 & daz si einen man niht næme,\\ 
 & der in zuo hêrren \textbf{wol} \textbf{gezæme}.\\ 
 & si het sich gar an got verlân,\\ 
20 & waz \textbf{zornes wart gegen ir} getân,\\ 
 & unschulde maniger \textbf{an s\textit{i}} \textbf{rach}.\\ 
 & einen hof si ir \textit{lan}de\textit{s} hêrren sprach.\\ 
 & manic bot ûz \textbf{ve\textit{r}rem lande} vuor\\ 
 & hin zuo ir. d\textit{ie} man si gar verswuor,\\ 
25 & wan \textbf{den} si got bewîste;\\ 
 & des minne si gerne prîste.\\ 
 & si was vürstîn in Brabant.\\ 
 & von Muntsalvasche wart gesant\\ 
 & der, de\textit{n} \textbf{\textit{d}er} swane brâhte\\ 
30 & und des ir got gedâhte.\\ 
\end{tabular}
\scriptsize
\line(1,0){75} \newline
m n V V' W \newline
\line(1,0){75} \newline
\textbf{1} \textit{Initiale} V W  \newline
\line(1,0){75} \newline
\textbf{1} Er beginc wunders so vil V' \textbf{2} Daz ich nit alles sagen wil \textit{(Fortsetzung in 824.23)} V' \textbf{3} \textit{Die Verse 824.3-22 fehlen} V'  \textbf{4} \textit{Die Verse 824.4-7 fehlen} W   $\cdot$ hôchvart] hoher art V \textbf{6} gebâren] bewarn m \textbf{7} warp] erstarp m [*]: erwarb V \textbf{8} gir] begir n \textbf{17} einen] ein m n V \textbf{18} in] ir W \textbf{19} verlân] gelan n erlan W \textbf{20} waz] Swaz V \textbf{21} si] sich m n \textbf{22} einen] Ein V  $\cdot$ ir landes hêrren] ir vor den herren m n vor den herren W \textbf{23} \textit{statt 824.23-826.2:} Wie er zu der herzoginnen gein brabant quam (vgl. 825.15: herzogîn) / Vnd die zu einer amyen nam (Fortsetzung von 824.2; weiterer Text in 826.23) V'   $\cdot$ verrem] femrem m \textbf{24} hin] \textit{om.} W  $\cdot$ die man si] do man suͯ m (n) do sy manne W \textbf{27} vürstîn] fúrsten n  $\cdot$ Brabant] brobant n probrant V probant W \textbf{28} Muntsalvasche] muntsaluasce m n munsalfasce V montsaluatsch W \textbf{29} den der] den het der m \newline
\end{minipage}
\end{table}
\newpage
\begin{table}[ht]
\begin{minipage}[t]{0.5\linewidth}
\small
\begin{center}*G
\end{center}
\begin{tabular}{rl}
 & \begin{large}W\end{large}elt ir nû hœren vürbaz?\\ 
 & sît über \textbf{lanc} ein vrouwe saz,\\ 
 & vo\textit{r} aller \textbf{untât} bewart.\\ 
 & rîcheit unde \textbf{hôher art}\\ 
5 & \textbf{bêde ûf si} geerbet wâren.\\ 
 & si kunde \textbf{alsô} gebâren,\\ 
 & daz si mit rehter kiusche \textbf{erwarp}.\\ 
 & al \textbf{werltlîch} gir an ir verdarp.\\ 
 & werder liute warb umbe si genuoc,\\ 
10 & der etslîcher krône truoc,\\ 
 & unde manic vürste, ir genôz.\\ 
 & ir diemuot was sô grôz,\\ 
 & daz si sich dran niht wande.\\ 
 & vil grâven \textbf{in} ir lande\\ 
15 & begunden si hazzen,\\ 
 & wes si sich wolde lazzen,\\ 
 & daz si einen man niht næme,\\ 
 & der in ze hêrren \textbf{zæme}.\\ 
 & si het sich gar an got verlân,\\ 
20 & swaz \textbf{gein ir zornes wart} getân,\\ 
 & unschulde maniger \textbf{hin ze ir} \textbf{rach}.\\ 
 & einen hof sir landes hêrren sprach.\\ 
 & manic bot ûz \textbf{verren landen} vuor\\ 
 & hin zir. die man si gar verswuor,\\ 
25 & wan \textit{\textbf{des}} \textit{si} got bewîste;\\ 
 & des minne si gerne brîste.\\ 
 & si was vürstîn in Brabant.\\ 
 & von Muntsalfatsche wart gesant\\ 
 & der, den \textbf{der} swane brâhte\\ 
30 & unde des ir got gedâhte.\\ 
\end{tabular}
\scriptsize
\line(1,0){75} \newline
G I L Z \newline
\line(1,0){75} \newline
\textbf{1} \textit{Initiale} G I L Z  \textbf{19} \textit{Initiale} I  \newline
\line(1,0){75} \newline
\textbf{3} vor] von G \textbf{4} hôher] solher Z \textbf{5} bêde] beidiu I  $\cdot$ wâren] [wart]: waren Z \textbf{7} erwarp] warp Z \textbf{8} al] Daz L \textbf{15} si] si alle I an sie Z \textbf{16} sich] \textit{om.} L \textbf{19} gar] \textit{om.} I \textbf{20} swaz] Waz L Z \textbf{21} unschulde] vnshuldic I \textbf{22} sir] vor ir I sý zuͯ ir L  $\cdot$ landes hêrren] lantherren L  $\cdot$ sprach] si sprach I \textbf{23} verren landen] verrem lande I \textbf{24} hin] hinz I \textbf{25} des si] si des G  $\cdot$ bewîste] [bewiste]: beweiste Z \textbf{26} brîste] leiste Z \textbf{28} Muntsalfatsche] mvntsalvatsche G (L) muntshaluasche I montsalvatsche Z  $\cdot$ gesant] ir gesant Z \textbf{29} der den] Den Z  $\cdot$ swane] shonen I \newline
\end{minipage}
\hspace{0.5cm}
\begin{minipage}[t]{0.5\linewidth}
\small
\begin{center}*T
\end{center}
\begin{tabular}{rl}
 & welt ir nû hœren vürbaz?\\ 
 & sît über \textbf{lanc} ein vrouwe saz,\\ 
 & vor aller \textbf{untât} bewart.\\ 
 & rîcheit und \textbf{hôhiu art}\\ 
5 & \textbf{beide ûf si} geerbet wâren.\\ 
 & si kunde \textbf{sô} gebâren,\\ 
 & daz si mit rehter kiusche \textbf{warp}.\\ 
 & alliu \textbf{wertlîche} gir an ir verdarp.\\ 
 & werder liute warp umb si genuoc,\\ 
10 & der etslîcher krône truoc,\\ 
 & und manec vürste, ir genôz.\\ 
 & ir diemuot was sô grôz,\\ 
 & daz si sich dran niht wande.\\ 
 & vil grâven \textbf{in} i\textit{r} lande\\ 
15 & begunden \textbf{an} si hazzen,\\ 
 & wes si sich wolt\textit{e} lazzen,\\ 
 & daz si einen man niht næme,\\ 
 & der in zuo hêrren \textbf{zæme}.\\ 
 & si hete sich gar an got verlân,\\ 
20 & waz \textbf{gein ir zornes wart} getân,\\ 
 & unschulde maneger \textbf{hin zuo ir} \textbf{sprach}.\\ 
 & einen hof si ir landes hêrren sprach.\\ 
 & manic bote ûz \textbf{verrem lande} vuor\\ 
 & hin zuo ir. die man si gar verswuor,\\ 
25 & wan \textbf{des} si got bewîste;\\ 
 & des minne si gerne prîste.\\ 
 & si was vürstinne in Brabant.\\ 
 & von Munsalvasche wart gesant\\ 
 & d\textit{e}r den swane brâhte\\ 
30 & und des ir got gedâhte.\\ 
\end{tabular}
\scriptsize
\line(1,0){75} \newline
U Q R \newline
\line(1,0){75} \newline
\textbf{1} \textit{Initiale} R  \textbf{22} \textit{Initiale} R  \newline
\line(1,0){75} \newline
\textbf{1} \textit{Die Verse 821.21-826.30 fehlen} Q  \textbf{2} über] úwer R \textbf{4} hôhiu] hocher R \textbf{7} warp] erwarb R \textbf{8} All weltlich [dins]: ding an Jr erwarb R \textbf{14} ir] irn U \textbf{16} wolte] wolten U \textbf{19} gar] gancz R \textbf{21} sprach] zoch R \textbf{23} verrem lande] ferren landen R \textbf{24} si] so R \textbf{26} si] \textit{om.} R \textbf{27} vürstinne] ein stime R  $\cdot$ Brabant] farbant R \textbf{28} Munsalvasche] Muntsalvatsche U Munsaualesche R \textbf{29} der] Dar U Die der R  $\cdot$ brâhte] brechte R \textbf{30} gedâhte] gedechtte R \newline
\end{minipage}
\end{table}
\end{document}
