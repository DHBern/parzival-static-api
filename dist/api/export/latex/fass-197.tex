\documentclass[8pt,a4paper,notitlepage]{article}
\usepackage{fullpage}
\usepackage{ulem}
\usepackage{xltxtra}
\usepackage{datetime}
\renewcommand{\dateseparator}{.}
\dmyyyydate
\usepackage{fancyhdr}
\usepackage{ifthen}
\pagestyle{fancy}
\fancyhf{}
\renewcommand{\headrulewidth}{0pt}
\fancyfoot[L]{\ifthenelse{\value{page}=1}{\today, \currenttime{} Uhr}{}}
\begin{document}
\begin{table}[ht]
\begin{minipage}[t]{0.5\linewidth}
\small
\begin{center}*D
\end{center}
\begin{tabular}{rl}
\textbf{197} & fillu roy Gahmuret.\\ 
 & der het der burgære gebet.\\ 
 & Diz was sîn êrste swertes strît.\\ 
 & er nam den poinder wol sô wît,\\ 
5 & daz von sîner tjoste hurt\\ 
 & bêden orsen wart engurt.\\ 
 & darmgürtel \textbf{in} brâsten umbe daz.\\ 
 & iewede\textit{r} ors ûf hehsen saz.\\ 
 & \textbf{die ê des ûf in sâzen},\\ 
10 & \textbf{ir swert} si niht vergâzen.\\ 
 & in den scheiden si \textbf{die} vunden.\\ 
 & Kingrun truoc wunden\\ 
 & durch den arm unt \textbf{in} die brust.\\ 
 & \textbf{disiu} tjost in lêrte vlust\\ 
15 & an sölhem prîse, des er pflac\\ 
 & unz an \textbf{sîn} hôchvart \textbf{swindens} tac.\\ 
 & Sölch ellen was ûf in gezalt:\\ 
 & sehs ritter \textbf{solt er} hân \textbf{gewalt},\\ 
 & die gein im \textbf{kœmen} ûf ein velt.\\ 
20 & Parzival im brâhte gelt\\ 
 & mit sîner ellenthaften hant,\\ 
 & \textbf{daz Kingrun scheneschalt}\\ 
 & \textbf{wânde vremder} mære,\\ 
 & wie ein pfeterære\\ 
25 & mit würfen \textbf{an} in seigete.\\ 
 & ander strît in \textbf{neigete}.\\ 
 & Ein swert im durch den helm erklanc.\\ 
 & Parzival in nider swanc.\\ 
 & \textbf{e\textit{r}} sazt im an die brust ein knie.\\ 
30 & \textbf{er bôt}, daz \textbf{wart} geboten nie\\ 
\end{tabular}
\scriptsize
\line(1,0){75} \newline
D Fr15 \newline
\line(1,0){75} \newline
\textbf{3} \textit{Majuskel} D  \textbf{17} \textit{Initiale} Fr15   $\cdot$ \textit{Majuskel} D  \textbf{27} \textit{Majuskel} D  \newline
\line(1,0){75} \newline
\textbf{1} Gahmuret] Gahmvret D :::ret Fr15 \textbf{4} \textit{Die Verse 197.4-5 sind auf drei Zeilen aufgeteilt, von denen nur noch Teile des Versinneren zu lesen sind:} ::: wol so ::: / ::: tioste ::: Fr15  \textbf{7} darmgürtel in] taremgvrteln D \textbf{8} ieweder] iewederr D  $\cdot$ ûf] vfen Fr15 \textbf{9} des ûf in] \textit{om.} Fr15 \textbf{10} swert] :::rte Fr15 \textbf{12} Kingrun] :::n Fr15  $\cdot$ truoc] der trvͦch Fr15 \textbf{15} sölhem] sulhen Fr15  $\cdot$ pflac] ie pflach Fr15 \textbf{16} unz] Biz Fr15 \textbf{18} gewalt] [gev*]: gewalt D \textbf{19} kœmen] quamen Fr15 \textbf{20} Parzival] Partsival Fr15 \textbf{22} Kingrun] kyngrun Fr15  $\cdot$ scheneschalt] der wigant Fr15 \textbf{25} an] vf Fr15 \textbf{28} Parzival] Partsival Fr15 \textbf{29} er] ez D \newline
\end{minipage}
\hspace{0.5cm}
\begin{minipage}[t]{0.5\linewidth}
\small
\begin{center}*m
\end{center}
\begin{tabular}{rl}
 & fili rois Gahmuret.\\ 
 & der hete der burgære gebet.\\ 
 & diz was sîn êrste swertes strît.\\ 
 & er nam de\textit{n} poinder wol sô wît,\\ 
5 & daz von sîner juste h\textit{u}rt\\ 
 & beiden rossen wart engurt.\\ 
 & darmgürtel brâsten umb daz.\\ 
 & ietwede\textit{r or}s ûf \textbf{die} hehsen saz.\\ 
 & \textbf{die ê des ûf in sâzen},\\ 
10 & \textbf{ir swert} si niht vergâzen.\\ 
 & in den scheiden si \textbf{die} vunden.\\ 
 & \textit{K}ingr\textit{u}n truoc wunden\\ 
 & durch den arm und \textbf{in} die brust.\\ 
 & \textbf{diu} just in lêrte vlust\\ 
15 & an solichem prîse, des er pflac\\ 
 & unz an \textbf{des} hôchverte \textbf{endes} tac.\\ 
 & solich ellen was ûf in gezalt:\\ 
 & sehs ritter \textbf{solt er} hân \textbf{gevalt},\\ 
 & die gegen ime \textbf{kômen} ûf ein velt.\\ 
20 & Parcifal im brâhte gelt\\ 
 & mit sîner ellenthaften hant,\\ 
 & \textbf{daz Kingr\textit{u}n schinschant}\\ 
 & \textbf{wând vrömder} mære,\\ 
 & wie ein pfeterære\\ 
25 & mi\textit{t} \textit{w}ürfen \textbf{an} in seigete.\\ 
 & \textbf{ein} ander strît in \textbf{veigete}.\\ 
 & ein swert im durch den helm erklanc.\\ 
 & Parcifal in nider swanc.\\ 
 & \textbf{er} sast ime an die brust ein knie.\\ 
30 & \textbf{dô bôt er}, daz geboten nie\\ 
\end{tabular}
\scriptsize
\line(1,0){75} \newline
m n o Fr69 \newline
\line(1,0){75} \newline
\newline
\line(1,0){75} \newline
\textbf{1} Gahmuret] gahnuret n gamuret o gachmuret Fr69 \textbf{3} sîn êrste] sins ersten o \textbf{4} den] dem m \textbf{5} hurt] hert m hort o \textbf{7} darmgürtel] Dar jngurtel o Die tarngúrtel Fr69 \textbf{8} ietweder ors] Jettweders m (n) (o)  $\cdot$ hehsen] hoͯhest n hoͯheste o hese Fr69 \textbf{9} ê des] edasz o \textbf{11} den] dem n \textbf{12} Kingrun] Bringrin m Kingruͦn n Konigrim o \textbf{13} den] die n  $\cdot$ und] \textit{om.} o \textbf{14} diu] Disv́ Fr69 \textbf{15} solichem] so solichem o solken Fr69 \textbf{16} des] sin n o siner Fr69 \textbf{19} kômen] kemen n o \textbf{22} Kingrun] kingrin m konigrim o  $\cdot$ schinschant] scuͯnscant m (n) (o) \textbf{24} pfeterære] pfedelere m pfedolere n pfedolere were o \textbf{25} mit würfen] Mit j wuͯrffen m  $\cdot$ seigete] serte n \textbf{26} in] ẏm do o  $\cdot$ veigete] ferte n \textbf{30} bôt er] gebot jme n bat ẏm o  $\cdot$ geboten nie] gebot nuͯ o \newline
\end{minipage}
\end{table}
\newpage
\begin{table}[ht]
\begin{minipage}[t]{0.5\linewidth}
\small
\begin{center}*G
\end{center}
\begin{tabular}{rl}
 & filirois Gahmuret.\\ 
 & der het der burgære gebet.\\ 
 & diz was sîn êrster swertes strît.\\ 
 & er nam den ponder wol sô wît,\\ 
5 & daz von sîner tjoste hurt\\ 
 & beiden orsen wart engurt.\\ 
 & darmgürtel brâsten umbe daz.\\ 
 & ietweder ors ûf hah\textit{s}en saz.\\ 
 & \textbf{die ê des ûf in sâzen},\\ 
10 & \textbf{der swerte} si niht vergâzen.\\ 
 & in den scheiden si \textbf{si} vunden.\\ 
 & Kingrun truoc wunden\\ 
 & durch den arm unde \textbf{durch} die brust.\\ 
 & \textbf{disiu} tjost in lêrte vlust\\ 
15 & an solhem prîse, des er pflac\\ 
 & unze an \textbf{sînen} hôchvart \textbf{swindes} tac.\\ 
 & solch ellen was ûf in gezalt:\\ 
 & sehs rîter \textbf{er solt} hân \textbf{gevalt},\\ 
 & die gein im \textbf{kômen} ûf ein velt.\\ 
20 & Parzival im brâhte gelt\\ 
 & \begin{large}M\end{large}it sîner ellenthaften hant,\\ 
 & \textbf{daz Kingrun seneschalt}\\ 
 & \textbf{wânde vrömder} mære,\\ 
 & wie ein pfeterære\\ 
25 & mit würfen \textbf{ûf} in seigte.\\ 
 & ander strît in \textbf{neigete}.\\ 
 & ein swert im dur den helm erklanc.\\ 
 & Parzival in nider swanc.\\ 
 & \textbf{er} sazte im an die brust ein knie.\\ 
30 & \textbf{er bôt}, daz \textbf{wart} geboten nie\\ 
\end{tabular}
\scriptsize
\line(1,0){75} \newline
G I O L M Q R Z \newline
\line(1,0){75} \newline
\textbf{3} \textit{Initiale} I  \textbf{11} \textit{Initiale} M  \textbf{21} \textit{Initiale} G Z  \textbf{27} \textit{Initiale} I  \newline
\line(1,0){75} \newline
\textbf{1} Gahmuret] Gahmvret G Gamvret O Gahmuͯret L gamuret M Z gamúret Q \textbf{2} burgære] burge R  $\cdot$ gebet] bet I \textbf{3} diz] ÷az I Daz O (M) (Q)  $\cdot$ êrster] erste I M  $\cdot$ swertes] swerster O swert L schwette R \textbf{4} den ponder] die poinder I (O) (L) (Q) die pan R  $\cdot$ wol sô] alse M \textbf{5} von] \textit{om.} O \textbf{6} engurt] engvͦrte O [gehuͯrt]: engehuͯrt L \textbf{7} darmgürtel] die darnGurtel I Die armgurtule M  $\cdot$ brâsten] brachen O (Q) (R) Z \textbf{8} ietweder] daz ietdweders I Jwerdez O J wedirs M Jetweders Z  $\cdot$ ûf] ander I uff den M vf die Z  $\cdot$ hahsen] haschen G (I) hasin M hatschen Q  $\cdot$ saz] Gesaz I \textbf{9} ê des] des Q e Z \textbf{11} si si] si O \textbf{12} Kingrun] Kyngrvn O L M (R) Kingrún Q \textbf{13} unde durch] vnd in I (M) vnde O (Q) (R) in L Z \textbf{14} disiu tjost] Div tyost O (L) Diser strit R  $\cdot$ in lêrte] in lert I lerte in O lert in R \textbf{15} an] Al Z  $\cdot$ solhem] sulem M  $\cdot$ des] als Q  $\cdot$ pflac] pfak Q \textbf{16} unze] bisz Q  $\cdot$ sînen] sins I siner L sine R  $\cdot$ hôchvart] hochuartes I  $\cdot$ swindes tac] swindens tach O (L) (Z) endistac M schwinenden tag R \textbf{17} ellen] [endin]: ellin M eren Q  $\cdot$ gezalt] bezalt I \textbf{18} er solt] solde er I (O) (L) (M) (Q) (R) (Z)  $\cdot$ gevalt] erualt I gewalt Z \textbf{20} Parzival] Parzifal I M Parcifal O L Z Partzifal Q Barczifal R  $\cdot$ im brâhte] brachte im Q bracht wider R \textbf{21} Mit sines ellens gewalt L  $\cdot$ ellenthaften] ellentwafter I ellenthafter O (R) eren haffter Q \textbf{22} \textit{nach 197.22:} Mit siner ellenthaften hant / Sin pris vil gar gein ým verswant L   $\cdot$ Kingrun] chingrun I kyngrvn O (M) kýngrvn L kingrún Q kúngrunt R  $\cdot$ seneschalt] sinetschaltz lant Z \textbf{23} wânde] Vnd wande L Wander M \textbf{24} pfeterære] pfedelere L phidelere M pferterre R \textbf{25} würfen] wurffe Q  $\cdot$ ûf in] gein im I an in O Q (R) Z vff eyn M  $\cdot$ seigte] sagete I (R) \textbf{26} neigete] neitte R \textbf{27} ein] Sein Q (R)  $\cdot$ im] \textit{om.} O in Q \textbf{28} Parzival] pazival G Parzifal I Parcifal O L Z Parzival M Partzifal Q Parczifal R \textbf{29} sazte] satz O Z \textbf{30} nie] hie L [ye]: nye Q \newline
\end{minipage}
\hspace{0.5cm}
\begin{minipage}[t]{0.5\linewidth}
\small
\begin{center}*T
\end{center}
\begin{tabular}{rl}
 & Filliroys Gahmuret.\\ 
 & der hete der burgære gebet.\\ 
 & diz was sîn êrster swertes strît.\\ 
 & er nam den poynder wol sô wît,\\ 
5 & daz von sîner tjoste hurt\\ 
 & beiden orsen wart engurt.\\ 
 & \textbf{die} darmgürtele brâsten umbe daz.\\ 
 & ietweder ors ûf \textbf{den} hehse\textit{n} saz.\\ 
 & \textbf{beide si doch gesâzen},\\ 
10 & \textbf{der swerte} si niht vergâzen.\\ 
 & in den scheiden si \textbf{die} vunden.\\ 
 & Kyngrun truoc wunden\\ 
 & durch den arm unde \textbf{in} die brust.\\ 
 & \textbf{dis\textit{iu}} tjost in lêrte vlust\\ 
15 & an sölhem prîse, des er pflac\\ 
 & unz an \textbf{sîner} hôchvart \textbf{swindens} tac.\\ 
 & solch ellen was ûf in gezalt:\\ 
 & sehs rîter \textbf{solter} hân \textbf{gevalt},\\ 
 & die gegen im \textbf{kâmen} ûf ein velt.\\ 
20 & Parcifal im brâhte gelt\\ 
 & mit sîner ellenthaften hant.\\ 
 & \textbf{sîn prîs gegen im vil gar verswant}.\\ 
 & \textbf{In dûhte vremede} mære,\\ 
 & wie ein pfeterære\\ 
25 & mit würfen \textbf{ûf} in seigete.\\ 
 & ander strît in \textbf{neigete}.\\ 
 & ein swert im durch den helm erklanc.\\ 
 & Parcifal in nider swanc\\ 
 & \textbf{unde} sazti\textit{m} an die brust ein knie.\\ 
30 & \textbf{er bôt}, daz \textbf{wart} \textbf{von im} geboten nie\\ 
\end{tabular}
\scriptsize
\line(1,0){75} \newline
T U V W \newline
\line(1,0){75} \newline
\textbf{1} \textit{Majuskel} T  \textbf{23} \textit{Majuskel} T  \newline
\line(1,0){75} \newline
\textbf{1} Filliroys] SVn dez kv́niges V Eilli roys W  $\cdot$ Gahmuret] Gahmuͦret U gamvret V gamuret W \textbf{3} diz] Daz V  $\cdot$ êrster swertes] erste V \textbf{5} tjoste] tiosten T \textbf{6} beiden orsen wart] Baide roß wurden W \textbf{7} brâsten] brachen U V W \textbf{8} ietweder] Ietweders W  $\cdot$ hehsen] hehsenen T heseden U hêsen V  $\cdot$ saz] gesaz V \textbf{9} doch] auff in W \textbf{11} Die sy in den schaiden funden W  $\cdot$ si die] die sv́ V \textbf{12} Kyngrun] Kyngruͦn U  $\cdot$ truoc] der trvͦg V \textbf{13} den] die W  $\cdot$ unde] \textit{om.} U W \textbf{14} disiu] dise T \textbf{16} unz] Bit U  $\cdot$ swindens] [*]: endez V swindes W \textbf{17} gezalt] bezalt W \textbf{18} gevalt] [gewalt]: gevalt T \textbf{19} kâmen] kemen V \textbf{20} Parcifal] Parzifal V Partzifal W \textbf{21} ellenthaften] ellenthafter U V \textbf{22} [S*sw*]: Daz kẏngrun schineschant V \textbf{26} ander] [*]: Ein ander V \textbf{27} erklanc] klang W \textbf{28} Parcifal] Parzifal V Partzifal W \textbf{29} saztim] saztin T  $\cdot$ die] sein W \textbf{30} von im] \textit{om.} W \newline
\end{minipage}
\end{table}
\end{document}
