\documentclass[8pt,a4paper,notitlepage]{article}
\usepackage{fullpage}
\usepackage{ulem}
\usepackage{xltxtra}
\usepackage{datetime}
\renewcommand{\dateseparator}{.}
\dmyyyydate
\usepackage{fancyhdr}
\usepackage{ifthen}
\pagestyle{fancy}
\fancyhf{}
\renewcommand{\headrulewidth}{0pt}
\fancyfoot[L]{\ifthenelse{\value{page}=1}{\today, \currenttime{} Uhr}{}}
\begin{document}
\begin{table}[ht]
\begin{minipage}[t]{0.5\linewidth}
\small
\begin{center}*D
\end{center}
\begin{tabular}{rl}
\textbf{502} & \begin{large}D\end{large}urch rât \textbf{si hânt} den betterisen.\\ 
 & in sîner jugent vurt und wisen\\ 
 & reit er \textbf{vil} durch tjostieren.\\ 
 & wil dû dîn leben zieren\\ 
5 & unt \textbf{rehte werdeclîche} varn,\\ 
 & sô \textbf{muostû} haz gein wîben sparn.\\ 
 & wîp unde pfaffen sint \textbf{erkant},\\ 
 & die tragent unwerlîche hant.\\ 
 & sô reichet über pfaffen gotes segen.\\ 
10 & der sol \textbf{dîn dienst} mit triwen pflegen\\ 
 & dar umbe, \textbf{ob} wirt dîn ende guot,\\ 
 & dû muost ze\textbf{n} pfaffen haben muot.\\ 
 & swaz dîn ouge ûf erden siht,\\ 
 & daz gelîchet sich dem priester niht.\\ 
15 & sîn munt die marter sprichet,\\ 
 & diu unser vlust zerbrichet.\\ 
 & \textbf{ouch} grîfet sîn gewîhtiu hant\\ 
 & an daz hœhste pfant,\\ 
 & daz ie vür schult \textbf{gesetzet} wart.\\ 
20 & swelch priester \textbf{sich hât} \textbf{sô} bewart,\\ 
 & daz er dem kiusche kan gegeben,\\ 
 & wie \textbf{m\textit{ö}hte} \textbf{der} \textbf{heileclîcher} leben?"\\ 
 & diz was ir \textbf{zweier} scheidens tac.\\ 
 & Trevrizent sich des bewac,\\ 
25 & Er sprach: "gip mir dîne \textbf{sünde} her\\ 
 & - vor gote \textbf{ich bin} dîn wandels wer -\\ 
 & und leist, als ich dir hân gesagt.\\ 
 & belîp des willen unverzagt."\\ 
 & von ein ander \textbf{schieden} sie.\\ 
30 & \textbf{ob ir welt, sô prüevet} wie.\\ 
\end{tabular}
\scriptsize
\line(1,0){75} \newline
D Fr11 \newline
\line(1,0){75} \newline
\textbf{1} \textit{Initiale} D Fr11  \textbf{25} \textit{Majuskel} D  \newline
\line(1,0){75} \newline
\textbf{2} in] vnd in Fr11 \textbf{5} rehte] vnrecht Fr11 \textbf{8} die] divͯ Fr11  $\cdot$ unwerlîche] vnwerlichivͯ Fr11 \textbf{9} gotes] der gotes Fr11 \textbf{10} dîn] dem Fr11 \textbf{12} zen] ze Fr11 \textbf{13} erden] erde Fr11 \textbf{15} die] divͯ Fr11 \textbf{22} möhte] mohte D \newline
\end{minipage}
\hspace{0.5cm}
\begin{minipage}[t]{0.5\linewidth}
\small
\begin{center}*m
\end{center}
\begin{tabular}{rl}
 & durch rât \textbf{si hânt} den betterisen.\\ 
 & in sîner jugent vurt und wisen\\ 
 & reit er \textbf{vil} durch justieren.\\ 
 & wiltû dîn leben zieren\\ 
5 & und \textbf{reht wirdeclîchen} varn,\\ 
 & sô \textbf{muostû} haz gegen wî\textit{b}e\textit{n} sp\textit{ar}n.\\ 
 & wîp und pfaffen sint \textbf{erkant},\\ 
 & die tragent unwerlîche hant.\\ 
 & sô reichet über \textit{pfaffen} gotes segen.\\ 
10 & der sol \textbf{dînes dienstes} mit triuwen pflegen,\\ 
 & dar umb wirt dîn ende guot,\\ 
 & dû muost zuo\textbf{m} pfaffen haben muot.\\ 
 & waz dîn ouge ûf erden siht,\\ 
 & daz glîchet sich dem priester niht.\\ 
15 & sîn munt die martel sprichet,\\ 
 & diu unser vlu\textit{s}t zerbrichet.\\ 
 & \textbf{ouch} grîfet sîn gewîhtiu hant\\ 
 & an daz hœhste pfant,\\ 
 & daz i\textit{e} vür schult \textbf{gesetzet} wart.\\ 
20 & welich priester \textbf{sich het} \textbf{sô} bewart,\\ 
 & daz er dem kiusche ka\textit{n} gegeben,\\ 
 & wie \textbf{m\textit{ö}ht} \textbf{er} \textbf{hêrlîcher} leben?"\\ 
 & diz was ir \textbf{zweier} scheidens tac.\\ 
 & Trevrizent sich des bewac,\\ 
25 & er sprach: "gip mi\textit{r} dîn \textbf{sünde} her\\ 
 & - vor got \textbf{ich bin} dîn wandels wer -\\ 
 & und leiste, als ich dir hân gesaget.\\ 
 & blîp des willen unverzaget."\\ 
 & von ein ander \textbf{scheiden} sie.\\ 
30 & \textbf{ob ir nû wellet, sô brüefet} wie.\\ 
\end{tabular}
\scriptsize
\line(1,0){75} \newline
m n o \newline
\line(1,0){75} \newline
\newline
\line(1,0){75} \newline
\textbf{2} \textit{Versdoppelung} o  \textbf{3} vil] >vil< o \textbf{5} varn] [warn]: varn o \textbf{6} wîben sparn] wise spran m \textbf{7} pfaffen] paffen o \textbf{9} pfaffen] \textit{om.} m \textbf{11} wirt] so wurt n \textbf{12} zuom] zuͯ den n \textbf{13} ouge] augen o  $\cdot$ siht] [schiet]: sicht o \textbf{16} unser vlust] vnser fluht m (o) vnsern verlust n \textbf{19} ie] ẏr m \textbf{21} kan] kam m \textbf{22} möht] moht m (o)  $\cdot$ hêrlîcher] heileclicher n heidelicher o \textbf{23} scheidens] [scheẏden]: scheẏdens o \textbf{24} Trevrizent] Treurizent m Trerizent n Tre vrizent o \textbf{25} mir dîn] mirs [sin]: din m \textbf{29} scheiden] schieden n \textbf{30} wellet] wellen n  $\cdot$ brüefet] prieffen o \newline
\end{minipage}
\end{table}
\newpage
\begin{table}[ht]
\begin{minipage}[t]{0.5\linewidth}
\small
\begin{center}*G
\end{center}
\begin{tabular}{rl}
 & \begin{large}D\end{large}urch rât \textbf{si hânt} den betterisen.\\ 
 & in sîner jugent vurt unde wisen\\ 
 & reit er \textbf{vil} durch tjostieren.\\ 
 & wil dû dîn leben zieren\\ 
5 & unde \textbf{rehte werdeclîchen} varn,\\ 
 & sô \textbf{muostû} haz gein wîben sparn.\\ 
 & wîp unde pfaffen sint \textbf{erkant},\\ 
 & die tragent unwerlîche hant.\\ 
 & sô reichet über pfaffen gotes segen.\\ 
10 & der sol \textbf{dîn dienst} mit triuwen pflegen\\ 
 & dar umbe, \textbf{obe} wirt dî\textit{n} ende guot,\\ 
 & dû muost ze pfaffen haben muot.\\ 
 & swaz dîn ouge ûf erden siht,\\ 
 & daz gelîchet sich dem priester niht.\\ 
15 & sîn munt die marter sprichet,\\ 
 & diu unser vlust zerbrichet.\\ 
 & \textbf{ouch} \textit{g}rîfet sîn gewîhtiu hant\\ 
 & an daz hœheste pfant,\\ 
 & daz ie vür schult \textit{\textbf{gesetzet}} \textit{wart}.\\ 
20 & swelch priester \textbf{hât sich} \textbf{sô} bewart,\\ 
 & daz er dem kiusche kan gegeben,\\ 
 & wie \textbf{m\textit{ö}ht} \textbf{der} \textbf{heiliger} leben?"\\ 
 & diz was ir \textbf{beider} scheidens tac.\\ 
 & Trevrizzent sich des bewac,\\ 
25 & er sprac\textit{h}: "\textit{g}ib mir dîn \textbf{sünde} her\\ 
 & - vor got \textbf{ich bin} dîn wandels wer -\\ 
 & unde leist, als ich dir hân gesaget.\\ 
 & belîp des willen unverzaget."\\ 
 & von ein ander \textbf{schieden} sie.\\ 
30 & \textbf{ob ir welt, sô prüevet} wie.\\ 
\end{tabular}
\scriptsize
\line(1,0){75} \newline
G I L M Z Fr57 \newline
\line(1,0){75} \newline
\textbf{1} \textit{Initiale} G I L Z  \textbf{15} \textit{Initiale} I  \textbf{29} \textit{Überschrift:} Hie lazen sich aventiwer von Parzifal vnd heuent sich von Gawan I   $\cdot$ \textit{Initiale} I  \newline
\line(1,0){75} \newline
\textbf{1} si hânt] habent si I \textbf{2} vurt] fruͤt I fvͤrt Z \textbf{3} reit] So reit L \textbf{8} die] Si L \textbf{10} dîn] di I \textbf{11} obe] \textit{om.} L  $\cdot$ wirt dîn ende] wirt dine ende G din dinst wirt M \textbf{13} swaz] Waz L (M)  $\cdot$ erden] der erden I erde L  $\cdot$ siht] [giht]: siht Z \textbf{17} grîfet] gerifet G  $\cdot$ gewîhtiu] gewelltigiv Fr57 \textbf{18} hœheste] aller hochste M \textbf{19} ie] ir L  $\cdot$ gesetzet wart] wart gesetzet G \textbf{20} swelch] Welch L (M)  $\cdot$ hât sich sô] [sih hat]: hat sih so G sich hat so L M Z sich so hat Fr57 \textbf{21} er] \textit{om.} M  $\cdot$ kiusche] kuschen M \textbf{22} möht] moht G I (L) (M) (Z) (Fr57)  $\cdot$ heiliger] helleclicher L heiliclichir M (Z) \textbf{23} beider] zweier L (M)  $\cdot$ scheidens] scheiden Z \textbf{24} Trevrizzent] Trevrizent G Treuerescent I Trevriszent L Tiffezens M :refrizent Fr57  $\cdot$ sich des] dez sich L \textbf{25} sprach gib] sprach nv gib G  $\cdot$ dîn] die L \textbf{26} vor] Von L  $\cdot$ dîn] dins I (Fr57) o\textit{m. } L  $\cdot$ wandels] waldels Z \textbf{27} \textit{Versfolge 502.28-27} Fr57  \textbf{30} sô] \textit{om.} I \newline
\end{minipage}
\hspace{0.5cm}
\begin{minipage}[t]{0.5\linewidth}
\small
\begin{center}*T
\end{center}
\begin{tabular}{rl}
 & \begin{large}D\end{large}urch rât \textbf{hânt si} den betterisen.\\ 
 & in sîner jugent vurt unde wisen\\ 
 & reit er durch tjostieren.\\ 
 & wiltû dîn leben zieren\\ 
5 & unde \textbf{nâch rehter werdecheite} varn,\\ 
 & sô \textbf{soltû} haz gegen wîben sparn.\\ 
 & wîp unde pfaffen sint \textbf{bekant},\\ 
 & die tragent unwerlîche hant.\\ 
 & sô reichet über \textbf{die} pfaffen \textbf{der} gotes segen.\\ 
10 & der sol \textbf{dîn dienst} mit triuwen pflegen,\\ 
 & dar umbe wirt dîn ende guot,\\ 
 & dû muost ze pfaffen haben muot.\\ 
 & \textbf{wan} swaz dîn ouge ûf erden siht,\\ 
 & daz glîchet sich dem priester niht.\\ 
15 & sîn munt die marter sprichet,\\ 
 & diu unser vlust zerbrichet.\\ 
 & \textbf{sô} grîfet sîn gewîht\textit{iu} hant\\ 
 & an daz \textbf{aller} hœheste pfant,\\ 
 & daz ie vür schult \textbf{versetzet} wart.\\ 
20 & swelch priester \textbf{sich hât} bewart,\\ 
 & daz er dem kiusche kan gegeben,\\ 
 & wie \textbf{mac} \textbf{der} \textbf{heileclîc\textit{h}er} leben?"\\ 
 & Diz was ir \textbf{zweier} scheidens tac.\\ 
 & Trefrizent sich des bewac,\\ 
25 & er sprach: "gip mir dîne \textbf{schulde} her\\ 
 & - vor gote \textbf{bin ich} dîn wandels wer -\\ 
 & unde leiste, als ich dir hân gesaget.\\ 
 & blîp des willen unverzaget."\\ 
 & von ein ander \textbf{schieden} sie,\\ 
30 & \textbf{ir habt wol gehœret} wie.\\ 
\end{tabular}
\scriptsize
\line(1,0){75} \newline
T U V W O Q R Fr39 \newline
\line(1,0){75} \newline
\textbf{1} \textit{Initiale} T V O  \textbf{23} \textit{Majuskel} T  \newline
\line(1,0){75} \newline
\textbf{1} \textit{Die Verse 453.1-502.30 fehlen} U   $\cdot$ Durch] ÷vrch O  $\cdot$ hânt] hanbt O han Q hat R  $\cdot$ betterisen] bette reisen Q \textbf{2} wisen] weisen Q \textbf{5} nâch rehter werdecheite] recht wirdiglichen W (O) (Q) (R) rehte werdeclihe Fr39 \textbf{6} haz] baz O \textbf{8} unwerlîche] ::werlichiv Fr39 \textbf{9} die] \textit{om.} V R Fr39  $\cdot$ der] \textit{om.} W R Fr39 \textbf{10} dîn] \textit{om.} O \textbf{11} dîn] [die]: din O \textbf{12} ze] zuͦn W (Q) zdem O \textbf{13} swaz] was W Q R  $\cdot$ ouge] aűgen Q  $\cdot$ ûf] vff der R \textbf{14} priester] prieste Fr39 \textbf{15} die] diu Fr39 \textbf{16} unser] vnsern W  $\cdot$ vlust] flvht O \textbf{17} gewîhtiu] gewihte T gewaltig R \textbf{19} vür schult versetzet] fv́r schult gesetzet V (W) (R) verschult gesetzet Q fûr schilt gesetzet Fr39 \textbf{20} swelch] Welch W (Q) (R)  $\cdot$ hât] hat so V W so hat O sust hat R \textbf{21} dem kiusche] der kv́sche V (O) (R) den keuschen Q \textbf{22} heileclîcher] heileclicler T heylicher Q heilkicher R heilecliche Fr39 \textbf{23} zweier] beider V weier Q  $\cdot$ scheidens tac] scheidetag V schadens tac Q \textbf{24} Trefrizent] Trefizent V Trefrissent W Trefizzent Q Trefrizzent Fr39 \textbf{25} schulde] [*h*]: sv́nde V súnde W (O) (Q) R \textbf{26} bin ich] ich bin O \textbf{28} des] [das]: des Q \textbf{30} [J* h*]: Ob ir nv wellent so prvͤfen wie V \newline
\end{minipage}
\end{table}
\end{document}
