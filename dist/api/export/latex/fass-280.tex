\documentclass[8pt,a4paper,notitlepage]{article}
\usepackage{fullpage}
\usepackage{ulem}
\usepackage{xltxtra}
\usepackage{datetime}
\renewcommand{\dateseparator}{.}
\dmyyyydate
\usepackage{fancyhdr}
\usepackage{ifthen}
\pagestyle{fancy}
\fancyhf{}
\renewcommand{\headrulewidth}{0pt}
\fancyfoot[L]{\ifthenelse{\value{page}=1}{\today, \currenttime{} Uhr}{}}
\begin{document}
\begin{table}[ht]
\begin{minipage}[t]{0.5\linewidth}
\small
\begin{center}*D
\end{center}
\begin{tabular}{rl}
\textbf{280} & \begin{large}W\end{large}elt ir nû hœren, wie Artus\\ 
 & \textbf{von} Karidol ûz sîme hûs\\ 
 & \textbf{unt ouch von sîme lande} schiet,\\ 
 & als im diu massenîe riet?\\ 
5 & sus reit er mit den werden\\ 
 & sînes landes unt \textbf{anderer} erden,\\ 
 & \textbf{diz} mære giht, \textbf{den ahten} tac,\\ 
 & \textbf{sô} daz er suochens pflac\\ 
 & den, \textbf{der sich 'der} rîter rôt'\\ 
10 & \textbf{nante unt} im solh êre bôt,\\ 
 & daz er in schiet von kumber grôz,\\ 
 & dô er den künec Ithern schôz,\\ 
 & unt Clamiden unt Kingrunen\\ 
 & \textbf{ouch} sande gein \textbf{den} Bertunen\\ 
15 & in sînen hof besunder.\\ 
 & über die tavelrunder\\ 
 & wolt er in durch gesellecheit\\ 
 & laden; durch daz er nâch im reit,\\ 
 & \textbf{alsô bescheidenlîche},\\ 
20 & beide arme und rîche,\\ 
 & die schildes ambet ane want,\\ 
 & \textbf{lobten} Artuses hant,\\ 
 & \multicolumn{1}{l}{ - - - }\\ 
 & \multicolumn{1}{l}{ - - - }\\ 
 & swâ si sæhen rîterschaft,\\ 
 & daz si durch \textbf{ir gelübde} kraft\\ 
25 & decheine tjost \textbf{en}tæten,\\ 
 & ez enwære, ob si in bæten,\\ 
 & daz er si lieze strîten.\\ 
 & er \textbf{jach}: "wir müezen rîten\\ 
 & in \textbf{manec} lant, \textbf{daz} rîters tât\\ 
30 & uns wol ze gegenstrîte \textbf{hât}.\\ 
\end{tabular}
\scriptsize
\line(1,0){75} \newline
D \newline
\line(1,0){75} \newline
\textbf{1} \textit{Großinitiale} D  \newline
\line(1,0){75} \newline
\textbf{12} Ithern] Jthern D \textbf{22} Artuses] Artvss D \newline
\end{minipage}
\hspace{0.5cm}
\begin{minipage}[t]{0.5\linewidth}
\small
\begin{center}*m
\end{center}
\begin{tabular}{rl}
 & wellet ir nû hœren, wie Artus\\ 
 & \textbf{\textit{\begin{large}V\end{large}}on} Ka\textit{r}idol ûz sînem hûs\\ 
 & \textbf{und ouch von sînem lande} schiet,\\ 
 & als ime diu \textit{m}assenîe riet?\\ 
5 & sus reit er mit den werden\\ 
 & \dag ins\dag  landes und \textbf{anderre} erden,\\ 
 & \textbf{disiu} mær giht, \textbf{den ahten} tac,\\ 
 & \textbf{sô} daz er suochenes pflac\\ 
 & den, \textbf{der sich 'der} ritter rôt'\\ 
10 & \textbf{nante und} ime soliche êre bôt,\\ 
 & daz er in schiet von kumbe\textit{r} grôz,\\ 
 & dô er den künic Ithern schôz,\\ 
 & und Clamide und Kingrunen\\ 
 & \textbf{ouch} sante gegen \textbf{dem} Britunen\\ 
15 & in sînen hof besunder.\\ 
 & über die tavelrunder\\ 
 & wolte er in durch gesellecheit\\ 
 & laden; durch daz er nâch ime reit.\\ 
 & \textbf{dô lobetenz alle gelîche},\\ 
20 & beide arme und rîche,\\ 
 & die schiltes ambet ane want,\\ 
 & \textbf{des küniges} Artuses hant,\\ 
 & \multicolumn{1}{l}{ - - - }\\ 
 & \multicolumn{1}{l}{ - - - }\\ 
 & wâ si sæhen ritterschaft,\\ 
 & daz si durch \textbf{ir gelübde} kraft\\ 
25 & dekeine just \textbf{en}tæten,\\ 
 & ez enwære, ob si in bæten,\\ 
 & daz er si lieze strîten.\\ 
 & er \textbf{sprach}: "wir müezen rîten\\ 
 & in \textbf{manic} lant, \textbf{daz} ritters tât\\ 
30 & uns wol ze gegenstrîte \textbf{hât}.\\ 
\end{tabular}
\scriptsize
\line(1,0){75} \newline
m n o \newline
\line(1,0){75} \newline
\textbf{1} \textit{Illustration mit Überschrift:} Wie parcifal Segramors vnd keien nider stach vnd mit gawane do fuͯr tv̂se fur m  Also parcifal [sogramusz]: sogramursz vnd keẏen nẏder stach vnd mit gawane do vor tuͯſe fuͦr n  Also parcifal segra mirsz vnd keyen nider stach vnd nit gewanne [*]: do ver tuͯse fur o   $\cdot$ \textit{Initiale} n o  \textbf{2} \textit{Initiale} m  \newline
\line(1,0){75} \newline
\textbf{2} Von] VVon m  $\cdot$ Karidol] kaidol m caridol n karadal o \textbf{4} massenîe] teassenie m \textbf{6} landes] landen o \textbf{8} suochenes] suches o \textbf{9} den] Denne n \textbf{11} kumber] kvmbers m \textbf{13} Clamide] klamide m  $\cdot$ Kingrunen] kingunen n koingrunen o \textbf{14} Britunen] brittunen m \textbf{15} An sẏnem hofe búsúnen der o \textbf{16} über] Vnd vber o \textbf{17} durch] vmb n ẏm o  $\cdot$ gesellecheit] geselicklicheit o \textbf{18} durch] \textit{om.} o \textbf{19} alle gelîche] [alleglich]: allglich o \textbf{22} Artuses] artusses n artuͯsez o \textbf{23} sæhen] sehent n o \textbf{25} dekeine] Do keine n \textbf{29} daz] des m n (o)  $\cdot$ tât] tot \textit{nachträglich korrigiert zu:} rot m dot o \newline
\end{minipage}
\end{table}
\newpage
\begin{table}[ht]
\begin{minipage}[t]{0.5\linewidth}
\small
\begin{center}*G
\end{center}
\begin{tabular}{rl}
 & welt ir nû hœren, wie Artus\\ 
 & \textbf{ze} Karidol ûz sînem hûs\\ 
 & \textbf{mit rîtæren unde mit vrouwen} schiet,\\ 
 & als im diu messenîe riet?\\ 
5 & sus reit er mit den werden\\ 
 & sînes landes unde \textbf{ander} erden,\\ 
 & \textbf{daz} mære giht, \textbf{naht unde} tac,\\ 
 & \textbf{alsô} daz er suochen\textit{e}s pflac,\\ 
 & den \textbf{ma\textit{n d}en} rîter rôt\\ 
10 & \textbf{nande unde} im solhe êre bôt,\\ 
 & daz er in schiet von kumber grôz,\\ 
 & dô er den künic Itheren schôz,\\ 
 & unde Clamide unde Kingrun\\ 
 & \textbf{ouch} sande gein \textbf{den} Britun\\ 
15 & in sînen hof besunder.\\ 
 & über die tavelrunder\\ 
 & \begin{large}W\end{large}olt er in durch gesellicheit\\ 
 & laden; durch daz er nâch im reit,\\ 
 & \textbf{alsô bescheidenlîche},\\ 
20 & beide arme unde rîche,\\ 
 & die schiltes ambet ane want,\\ 
 & \textbf{die lobten} Artuses hant,\\ 
 & \multicolumn{1}{l}{ - - - }\\ 
 & \multicolumn{1}{l}{ - - - }\\ 
 & swâ si sæhen rîterschaft,\\ 
 & daz si durch \textbf{ir gelübedes} kraft\\ 
25 & deheine tjost \textbf{dâ} tæten,\\ 
 & ez enwære, op si\textbf{s} in bæten,\\ 
 & daz er si lieze strîten.\\ 
 & er \textbf{sprach}: "wir müezen rîten\\ 
 & in \textbf{manec} lant, \textbf{daz} rîters tât\\ 
30 & uns wol ze gegenstrîte \textbf{stât}.\\ 
\end{tabular}
\scriptsize
\line(1,0){75} \newline
G I O L M Q R Z Fr30 \newline
\line(1,0){75} \newline
\textbf{1} \textit{Überschrift:} Wie Parzifal dru plvͦtstrophen vant / vͦf sne da von er sinnen wart gephant I  Wy partzifal gesuchet wart von dem konick artus dem werden mannen Q  Wie kvnic artus von sinem hvs vnd von sinem lande schiet Z   $\cdot$ \textit{Initiale} I O L M Q R Z Fr30  \textbf{5} \textit{Initiale} I  \textbf{17} \textit{Initiale} G  \textbf{27} \textit{Initiale} I  \newline
\line(1,0){75} \newline
\textbf{1} welt] ÷elt O Solt Q  $\cdot$ nû] \textit{om.} R  $\cdot$ wie] wie kunc I \textit{om.} Q \textbf{2} ze Karidol] ze charidol G Zecaridol O Zuͯ karidol L (M) (Fr30) Czu kandol Q Zu kardolvs R Von karidol Z  $\cdot$ ûz] in I \textit{om.} R \textbf{3} Vnd mit im diu werde diet I  $\cdot$ vnde von sinem lande schiet O (L) (M) (Q) (R)  $\cdot$ Vnd ouch von sinem lande schiet Z (Fr30) \textbf{4} im] in R \textbf{5} den] dem Q \textbf{6} ander erden] vf der erden G andrú erde R \textbf{7} daz] Ditze O (L) (M) (Q) (R) (Z) Fr30  $\cdot$ giht] iet M giht so Z  $\cdot$ naht unde] den ahtoden L (R) (Z) den haten Q \textbf{8} alsô] \textit{om.} Z so Fr30  $\cdot$ er] si O  $\cdot$ suochenes] soͮchenens G suchen Q \textbf{9} den man] den man da \sout{nande} G Den den man L (M) (Q) Den den man nempt R Den der sich Z Den man den er Fr30  $\cdot$ den rîter] ritter R \textbf{10} nande] >nande< G  $\cdot$ im] \textit{om.} I in L  $\cdot$ solhe êre] soͯlicherre R  $\cdot$ bôt] erbot I \textbf{12} \textit{Versdoppelung 280.11 nach 280.12:} I   $\cdot$ dô] Da M Z des tages do Fr30  $\cdot$ den künic] den chuͤnen I \textit{om.} Fr30  $\cdot$ Itheren] Jtern I Jthern O (M) jhtern L (R) ithern Q ychern Z Jetheren Fr30 \textbf{13} Clamide] Glamiden O Clamiden L (M) R (Z) clamyde Fr30  $\cdot$ Kingrun] chingrun G kingrunen I (Z) kyngrvne O kyngrvnen L M (Q) (R) kyngrvͦn Fr30 \textbf{14} sande] sante er L  $\cdot$ den Britun] pritunen I britvne O Brittvnen L brytunen M britungen Q britanyen R den britunen Z dem britvͦn Fr30 \textbf{15} sînen hof] sinem hofe I sinē hoff M (Q) \textbf{16} \textit{Vers 280.16 fehlt} Q   $\cdot$ die] der I \textbf{17} in] \textit{om.} M  $\cdot$ gesellicheit] [geschelleschaft]: geschellescheit O \textbf{19} \textit{Versdoppelung:} Also bescheidenliche / Also bescheidenliche \sout{be*} Z  \textbf{20} beide] beidiu I (Fr30) \textbf{22} die lobten] Lobtes Z  $\cdot$ Artuses] Artuͯses L artus M Q R \textbf{23} swâ] Wa L M (Q) R  $\cdot$ sæhen] sahen I L (M) (Q) (R) \textbf{24} gelübedes] geluͯbede L \textbf{25} tjost dâ] tiost O (L) Q R Z Fr30 not M  $\cdot$ tæten] taten I L entæten O (M) (Q) (R) \textbf{26} enwære] were R  $\cdot$ sis in] si ins I (Fr30) si O sie in L (M) (Q) (R) (Z) \textbf{28} müezen] suln I \textbf{29} daz] da L do Q  $\cdot$ tât] tag R \textbf{30} gegenstrîte] [geben]: gegen streite Q gegenstrite strite Z  $\cdot$ stât] hat O L M (Q) R Z [hant]: hatt Q \newline
\end{minipage}
\hspace{0.5cm}
\begin{minipage}[t]{0.5\linewidth}
\small
\begin{center}*T
\end{center}
\begin{tabular}{rl}
 & \textit{\begin{large}W\end{large}}elt \textit{ir} nû hœren, wie Artus\\ 
 & \textbf{ze} Karidol ûz sînem hûs\\ 
 & \textbf{unde ouch von sînem lande} schiet,\\ 
 & als im di\textit{u} massenîe riet?\\ 
5 & Sus reit er mit den werden\\ 
 & sînes landes unde \textbf{ûf der} erden,\\ 
 & \textbf{diz} mære giht, \textbf{naht unde} tac,\\ 
 & \textbf{alsô} daz er suochens pflac,\\ 
 & den \textbf{man dâ nande den} rîter rôt,\\ 
10 & \textbf{der} im solch êre bôt,\\ 
 & daz er in schiet von kumber grôz,\\ 
 & dô er den künec Itheren schôz,\\ 
 & unde Clamiden unde Kyngrun\\ 
 & sante gegen \textbf{den} Britun\\ 
15 & in sînen hof besunder.\\ 
 & über die tavelrunder\\ 
 & woltern durch gesellecheit\\ 
 & laden; durch daz er nâch im reit,\\ 
 & \textbf{alsô bescheidenlîche},\\ 
20 & beide arm unde rîche,\\ 
 & die schiltes ambet ane want,\\ 
 & \textbf{die lobeten} Artuses hant,\\ 
 & daz siz tæten durch sînen willen,\\ 
 & daz sirn muot begunden stillen,\\ 
 & swâ si sæhen rîterschaft,\\ 
 & daz si durch \textbf{gelübdes} kraft\\ 
25 & deheine tjost tæten,\\ 
 & ez enwære, ob sin bæten,\\ 
 & daz er si lieze strîten.\\ 
 & er \textbf{sprach}: "wir müezen rîten\\ 
 & in \textbf{manegiu} lant, \textbf{dâ} rîters tât\\ 
30 & uns wol ze gegenstrîte \textbf{hât}.\\ 
\end{tabular}
\scriptsize
\line(1,0){75} \newline
T U V W \newline
\line(1,0){75} \newline
\textbf{1} \textit{Initiale} T U W  \textbf{5} \textit{Majuskel} T  \newline
\line(1,0){75} \newline
\textbf{1} Welt ir nû] Helt nv T Went ir [*]: nv V \textbf{2} ze] [*]: Von V  $\cdot$ Karidol] karydol U \textbf{4} diu] die T \textbf{5} reit] riet U \textbf{6} ûf der] [*]: anderre V \textbf{7} diz] Dise U [D*]: Diz V  $\cdot$ giht] git U  $\cdot$ naht unde tac] [*]: den ahten [t*]: tag V \textbf{9} dâ] do U V  $\cdot$ nande] \textit{om.} W \textbf{10} der] Vnde V Nande vnd W \textbf{12} Itheren] Jthêren T Jthern U ytern V ythern W \textbf{13} Clamiden] [Clami*]: Clamide V klamiden W  $\cdot$ Kyngrun] kyngruͦn U kyngrunen V kingrun W \textbf{14} gegen den Britun] gegn den Britv̂n T gein Brituͦn U gegen [*]: den brittunen V her gen britun W \textbf{17} woltern durch] Wolte er die W \textbf{18} laden] lande U \textbf{19} Do lobetenz alle geliche V \textbf{21} ambet] amht W \textbf{22} \textit{Versfolge 288.23-22} W  \textbf{22} \textit{Die Verse 280.22¹-22² fehlen} W  \textbf{23} swâ] Wa U (W)  $\cdot$ sæhen] sahen W \textbf{24} gelübdes] ir gelúbde W \textbf{27} \textit{Die Verse 280.27-28 fehlen} U  \textbf{29} \textit{Versdoppelung (erster Vers getilgt)} U   $\cdot$ dâ] [*]: da T do V W  $\cdot$ rîters tât] riterschaft U \newline
\end{minipage}
\end{table}
\end{document}
