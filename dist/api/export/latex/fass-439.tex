\documentclass[8pt,a4paper,notitlepage]{article}
\usepackage{fullpage}
\usepackage{ulem}
\usepackage{xltxtra}
\usepackage{datetime}
\renewcommand{\dateseparator}{.}
\dmyyyydate
\usepackage{fancyhdr}
\usepackage{ifthen}
\pagestyle{fancy}
\fancyhf{}
\renewcommand{\headrulewidth}{0pt}
\fancyfoot[L]{\ifthenelse{\value{page}=1}{\today, \currenttime{} Uhr}{}}
\begin{document}
\begin{table}[ht]
\begin{minipage}[t]{0.5\linewidth}
\small
\begin{center}*D
\end{center}
\begin{tabular}{rl}
\textbf{439} & Cundrie la surziere\\ 
 & mir dannen bringet schiere\\ 
 & alle samztage naht\\ 
 & mîne spîse - des \textbf{hât} si sich bedâht -,\\ 
5 & \textbf{die} ich \textbf{ganze} wochen \textbf{haben} sol."\\ 
 & si sprach: "wære mir anders wol,\\ 
 & ich sorgete wênec umbe die nar.\\ 
 & der bin ich \textbf{bereitet} \textbf{gar}."\\ 
 & Dô wânde Parzival, si lüge\\ 
10 & unt daz si in anders \textbf{gerne} trüge.\\ 
 & er sprach in schimpfe zir dar în:\\ 
 & "durch wen tragt ir \textbf{daz} vingerlîn?\\ 
 & ich hôrt ie sagen mære,\\ 
 & klôsenærinne unt klôsenære,\\ 
15 & \textbf{die} solten mîden amûrschaft."\\ 
 & si sprach: "het iwer rede kraft,\\ 
 & ir \textbf{wolt} mich velschen gerne.\\ 
 & swenne ich nû valsch gelerne,\\ 
 & sô hebt mirn ûf, sît ir dâ bî.\\ 
20 & ruochtes got, ich bin \textbf{vor} valsche vrî.\\ 
 & ich \textbf{enkan} decheinen widersaz."\\ 
 & si sprach: "disen mahelschaz\\ 
 & trag ich \textbf{durch einen} lieben man,\\ 
 & des minne ich nie an \textbf{mich} gewan\\ 
25 & mit menneschlîcher tæte.\\ 
 & \textbf{magetuomlîches} herzen ræte\\ 
 & mir gein im râtent minne."\\ 
 & si sprach: "den hân ich hie inne,\\ 
 & des kleinôt ich sider truoc,\\ 
30 & sît \textbf{Oriluse\textit{s} tjost} in \textbf{sluoc}.\\ 
\end{tabular}
\scriptsize
\line(1,0){75} \newline
D Fr31 \newline
\line(1,0){75} \newline
\textbf{1} \textit{Initiale} Fr31  \textbf{9} \textit{Majuskel} D  \newline
\line(1,0){75} \newline
\textbf{1} Gvndrie lasvrzile Fr31 \textbf{2} dannen bringet schiere] bringet dannan::: Fr31 \textbf{9} Parzival] Parcifal D \textbf{30} Oriluses] Orilvs D \newline
\end{minipage}
\hspace{0.5cm}
\begin{minipage}[t]{0.5\linewidth}
\small
\begin{center}*m
\end{center}
\begin{tabular}{rl}
 & Condrie la surziere\\ 
 & mir dannen bringet schiere\\ 
 & al\textit{l}e sameztage  naht\\ 
 & mîne spîse - des  si sich bedâht -,\\ 
5 & \textbf{der} ich \textbf{ganze} woche \textbf{leben} sol."\\ 
 & si sprach: "wære mir anders wol,\\ 
 & ich s\textit{or}get\textit{e} \textit{w}ênic umb die nar.\\ 
 & der bin ich \textbf{wol} \textbf{bereitet} \textbf{dâr}."\\ 
 & \begin{large}D\end{large}ô w\textit{â}nde Parcifal, si lüge\\ 
10 & und daz sin anders \textbf{gerne} trüge.\\ 
 & er sprach in schimpfe zuo ir dar în:\\ 
 & "durch wen traget ir \textbf{daz} vingerlîn?\\ 
 & ich hôrte ie sagen mære,\\ 
 & klôsenærinne und klôsenære,\\ 
15 & \textbf{die} solten mîden amûrschaft."\\ 
 & si sprach: "hett iuwer rede kraft,\\ 
 & ir \textbf{woltet} mich v\textit{e}lschen gerne.\\ 
 & wenne ich nû valsch gelerne,\\ 
 & sô hebet mir in ûf, sît ir d\textit{â} bî.\\ 
20 & ruoche\textit{t}s got, ich bin \textbf{vor} valsche vrî.\\ 
 & ich \textbf{enkan} d\textit{e}keinen widersaz."\\ 
 & si sprach: "disen mahelschaz\\ 
 & trage ich \textbf{durch einen} lieben man,\\ 
 & des minne ich nie an \textbf{mînen ougen} gewan\\ 
25 & mit \textit{men}sc\textit{h}l\textit{îch}er tæte.\\ 
 & \textbf{magetuomlîches} herzen ræte\\ 
 & mir gegen ime râtent \textit{minn}e."\\ 
 & si sprach: "den hân ich hinne,\\ 
 & des kleinôte ich sider truoc,\\ 
30 & sît \textbf{Orilus} in \textbf{sluoc}.\\ 
\end{tabular}
\scriptsize
\line(1,0){75} \newline
m n o \newline
\line(1,0){75} \newline
\textbf{9} \textit{Initiale} m n  \newline
\line(1,0){75} \newline
\textbf{1} Condrie] Kuͯndrie o  $\cdot$ la surziere] lasintzier n [geforcziet]: geforczier o \textbf{2} dannen bringet] bringet denne n \textbf{3} alle] Ale m  $\cdot$ naht] zuͯ nacht n o \textbf{4} mîne] Min n (o)  $\cdot$ si] hette sú n (o) \textbf{5} ganze] gantz n (o) \textbf{7} sorgete] sagette yme m \textbf{8} bereitet] bereiten o  $\cdot$ dâr] gar n o \textbf{9} wânde] wunde m  $\cdot$ si] ich n \textbf{12} traget] trag o \textbf{13} ie] e o \textbf{15} die] Do o \textbf{16} hett iuwer] hettewe:r m  $\cdot$ rede] \textit{om.} n \textbf{17} woltet] wolten n o  $\cdot$ velschen] valschen m \textbf{18} wenne] Wan o \textbf{19} hebet] habent o  $\cdot$ dâ] do m n o \textbf{20} ruochets] Ruchencz m  $\cdot$ got] gat o \textbf{21} enkan] kan n o  $\cdot$ dekeinen] den keinen m do keinen n \textbf{22} mahelschaz] mechel schacz m michelen schatz n michel schacz o \textbf{24} des] Dasz o  $\cdot$ mînen ougen] mich n sie o \textbf{25} menschlîcher] einem schiltter m  $\cdot$ tæte] [dete]: bete n \textbf{26} magetuomlîches] Magtumlich: o  $\cdot$ ræte] dete n o \textbf{27} râtent] rotet n (o)  $\cdot$ minne] ẏme m \textbf{28} den] dem o  $\cdot$ hân] habe n (o)  $\cdot$ hinne] hinnen o \textbf{29} des] Das n o \textbf{30} Orilus] orilus juste n vrilus just o \newline
\end{minipage}
\end{table}
\newpage
\begin{table}[ht]
\begin{minipage}[t]{0.5\linewidth}
\small
\begin{center}*G
\end{center}
\begin{tabular}{rl}
 & Gundrie la surz\textit{ier}e\\ 
 & mir dannen bringet schiere\\ 
 & alle sam\textit{ez}ta\textit{g}e na\textit{ht}\\ 
 & mîn spîse - des \textbf{het} si sich bedâht -,\\ 
5 & \textbf{die} ich \textbf{ganze} wochen \textbf{hab\textit{en}} sol."\\ 
 & si sprach: "wær mir anders wol,\\ 
 & ich sorgete wênic umbe die nar.\\ 
 & der bin ich \textbf{wol} \textbf{gereit} \textbf{gar}."\\ 
 & dô wânt Parcival, si lüge\\ 
10 & unt daz sin anders trüge.\\ 
 & er sprach in schimpfe zir dar în:\\ 
 & "durch wen traget ir \textbf{daz} vingerlîn?\\ 
 & ich hôrte ie sagen mære,\\ 
 & klôsenærinne und klôsenær,\\ 
15 & \textbf{die} solden mîden amûrschaft."\\ 
 & si sprach: "het iuwer rede kraft,\\ 
 & ir \textbf{wolt} mich velschen gerne.\\ 
 & swenne ich nû valsch gelerne,\\ 
 & sô hebet mirn ûf, sît ir dâ bî.\\ 
20 & ruochet es got, ich bin \textbf{von} valsche vrî.\\ 
 & ich \textbf{enkan} deheine\textit{n} widersaz."\\ 
 & si sprach: "disen maheln schaz\\ 
 & trage ich \textbf{durch einen} lieben man,\\ 
 & des minne ich nie an \textbf{mich} gewan\\ 
25 & mit mennischlîcher tæte.\\ 
 & \textbf{magetuomes} herzen ræte\\ 
 & mir gegen im râtent minne."\\ 
 & si sprach: "den hân ich hinne,\\ 
 & des kleinôde ich sider truoc,\\ 
30 & sît \textbf{Orilluse\textit{s} tjost} in \textbf{ersluoc}.\\ 
\end{tabular}
\scriptsize
\line(1,0){75} \newline
G I L M Z Fr25 \newline
\line(1,0){75} \newline
\textbf{1} \textit{Initiale} I L Z Fr25  \textbf{11} \textit{Initiale} I  \newline
\line(1,0){75} \newline
\textbf{1} Gundrie la surziere] Gvndrîe la surz:::e G Gundrie lasurziere I Kvndrie Lazvrziere L (Z) Kundrie la surziere M G:::drie la svrziere Fr25 \textbf{2} dannen bringet] brenget danne M \textbf{3} sameztage naht] samzetaginne na:: G samztagen naht I (L) Z Fr25 samet tage nacht M \textbf{4} mîn] mine I M  $\cdot$ het] hat I L M Z Fr25 \textbf{5} die] Des L  $\cdot$ ganze] die gantzen L (M) (Fr25)  $\cdot$ haben] hab:: G \textbf{6} mir] \textit{om.} Fr25 \textbf{7} sorgete] [han]: sorgite G  $\cdot$ wênic] [iwench]: wench Fr25 \textbf{8} bin] bi Fr25  $\cdot$ gereit] beraten I bereitet M Z Fr25 [beitet]: bereitet L  $\cdot$ gar] \textit{om.} M \textbf{9} dô] Da M Z  $\cdot$ Parcival] parziual G parzifal I L M Fr25 parcifal Z  $\cdot$ si lüge] daz si luͤge I \textbf{10} anders] \textit{om.} M  $\cdot$ trüge] gerne grug I gerne troge M (Z) (Fr25) grime truͯge L \textbf{11} sprach] sprache G \textbf{12} vingerlîn] [uingel]: uingerlin G \textbf{14} und] oder Fr25 \textbf{15} die] \textit{om.} L Fr25  $\cdot$ mîden] miner I \textbf{16} het] \textit{om.} Fr25 hat M \textbf{18} swenne] swenn I (Z) Wenne L So wanne M \textbf{19} mirn] mirz I  $\cdot$ sît] vnd sit M  $\cdot$ ir] \textit{om.} L M \textbf{20} ruochet] Wil M  $\cdot$ es] sin I  $\cdot$ von valsche] valsches I vor valsche L (M) (Z) (Fr25) \textbf{21} enkan deheinen] en chan deheiner G han dehainn I enkan dehein L en kan ichein M kan keinen Z chan dehein Fr25  $\cdot$ widersaz] [wid*]: widir sazt M hindersatz Z \textbf{22} maheln] mahel I Z (G) (Fr25) gemahel L  $\cdot$ schaz] [scz]: schatz L \textbf{24} gewan] genam I \textbf{25} mennischlîcher] mynneclicher L  $\cdot$ tæte] gerete M \textbf{26} magetuomes] Magtumlichess M (Z)  $\cdot$ herzen] \textit{om.} M  $\cdot$ ræte] state L gerete M \textbf{29} kleinôde] [clenot]: cleinot I  $\cdot$ sider] sit I  $\cdot$ truoc] [ruͤc]: truͤc I \textbf{30} Orilluses] orlûs G Orilus I (L) (Fr25) orilusz M  $\cdot$ tjost] mit der Tiost I zcu der tioste M (Fr25)  $\cdot$ ersluoc] sluͤc I (L) (M) (Z) (Fr25) \newline
\end{minipage}
\hspace{0.5cm}
\begin{minipage}[t]{0.5\linewidth}
\small
\begin{center}*T
\end{center}
\begin{tabular}{rl}
 & Kundrie Lazursiere\\ 
 & mir dannen bringet schiere\\ 
 & alle samztage naht\\ 
 & mîne spîse - des \textbf{hât} si sich bedâht -,\\ 
5 & \textbf{die} ich \textbf{die} wochen \textbf{haben} sol."\\ 
 & si sprach: "wære mir anders wol,\\ 
 & ich sorgete wênic umbe die nar.\\ 
 & der bin ich \textbf{wol} \textbf{bereitet} \textbf{gar}."\\ 
 & Dô wânde Parcifal, si lüge\\ 
10 & und daz sin anders \textbf{gerne} trüge.\\ 
 & er sprach in schimpf zuo ir dar în:\\ 
 & "durch wen traget ir \textbf{diz} vingerlîn?\\ 
 & ich hôrte ie sagen mære,\\ 
 & \textbf{daz} klôsenærinne und klôsenere\\ 
15 & solten mîden amûrschaft."\\ 
 & Si sprach: "hât iuwer rede kraft,\\ 
 & ir \textbf{wolt} mich velschen gerne.\\ 
 & swennich nû valsch gelerne,\\ 
 & sô hebt mirn ûf, sît ir dar bî.\\ 
20 & ruochet es \textit{got}, ich bin \textbf{vor} valsche vrî.\\ 
 & ich \textbf{kan} deheinen widersaz."\\ 
 & si sprach: "disen mahelschaz\\ 
 & tragich \textbf{einem} lieben man,\\ 
 & des minne ich nie an \textbf{mich} gewan\\ 
25 & mit menschlîcher tæte.\\ 
 & \textbf{magetuomlîches} herzen ræte\\ 
 & mir gegen im râtent minne."\\ 
 & si sprach: "den hân ich hinne,\\ 
 & des kleinôde ich sider truoc,\\ 
30 & sît \textbf{Orilus zer tjost} in \textbf{sluoc}.\\ 
\end{tabular}
\scriptsize
\line(1,0){75} \newline
T U V W O Q R \newline
\line(1,0){75} \newline
\textbf{1} \textit{Initiale} O   $\cdot$ \textit{Capitulumzeichen} R  \textbf{9} \textit{Majuskel} T  \textbf{16} \textit{Majuskel} T  \newline
\line(1,0){75} \newline
\textbf{1} Kundrie Lazursiere] Kuͦndrie Lazuͦrziere U Kvndrie [lasur*]: lasursvrziere V ÷vndrie la svrziere O Kundrye lasuͯrziere R \textbf{2} dannen bringet schiere] dannan [brin*]: bringet schiere V bringet danne schiere O \textbf{3} naht] ze naht V (R) \textbf{4} mîne] Min U (W) (Q) R  $\cdot$ hât si sich bedâht] sv́ sich [*]: hat bedaht V \textbf{5} die] [D*]: Der V  $\cdot$ wochen] ganzen wochen O  $\cdot$ haben] leben U (V) \textbf{7} die] ir Q \textit{om.} R \textbf{9} Parcifal] parzifal T V partzifal W Q Barcifal O parczifal R \textbf{10} sin] sy R  $\cdot$ trüge] betrúge W \textbf{11} ir] in Q \textbf{12} durch] [*]: Durch V  $\cdot$ diz] daz U V (W) O (Q) (R) \textbf{14} daz] \textit{om.} U V W O Q R  $\cdot$ klôsenærinne] Closerin R  $\cdot$ und] oder O \textbf{15} mîden] liden R  $\cdot$ amûrschaft] frv́ntschaft V buͯlschafftt R \textbf{16} hât] \textit{om.} O \textbf{17} wolt] woltent V \textbf{18} swennich] Wan ich U (Q) Swenne ich V Wenn ich W R  $\cdot$ nû] [vv]: nv O \textbf{19} mirn] mir iz U [*]: mirn V  $\cdot$ dar bî] do bi V (W) (Q) \textbf{20} got] \textit{om.} T  $\cdot$ bin] \textit{om.} R  $\cdot$ vor valsche] [von]: vor [valsch]: falsch Q \textbf{21} ich] [Jc*]: Jch Q  $\cdot$ kan deheinen] kan dekeinen U enkan keinen W (Q) \textbf{22} mahelschaz] michel schatz W \textbf{23} tragich] dragen ich U  $\cdot$ einem] duͦrch einen U (V) (W) (O) (Q) (R) \textbf{24} gewan] genan R \textbf{25} menschlîcher] [m*]: menlicher U \textbf{26} herzen] herze U \textbf{27} gegen] [gege*]: gegen V  $\cdot$ râtent] ratet W \textbf{29} kleinôde] cleit U [cleinoͤter*]: cleinoͤter V kleinenden R \textbf{30} sît] Sein Q  $\cdot$ zer] \textit{om.} V  $\cdot$ in sluoc] erschluͦg W \newline
\end{minipage}
\end{table}
\end{document}
