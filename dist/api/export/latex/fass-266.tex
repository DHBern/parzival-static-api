\documentclass[8pt,a4paper,notitlepage]{article}
\usepackage{fullpage}
\usepackage{ulem}
\usepackage{xltxtra}
\usepackage{datetime}
\renewcommand{\dateseparator}{.}
\dmyyyydate
\usepackage{fancyhdr}
\usepackage{ifthen}
\pagestyle{fancy}
\fancyhf{}
\renewcommand{\headrulewidth}{0pt}
\fancyfoot[L]{\ifthenelse{\value{page}=1}{\today, \currenttime{} Uhr}{}}
\begin{document}
\begin{table}[ht]
\begin{minipage}[t]{0.5\linewidth}
\small
\begin{center}*D
\end{center}
\begin{tabular}{rl}
\textbf{266} & betwungen, \textbf{swes} \textbf{man} an \textbf{im} warp,\\ 
 & er tet, \textbf{sô} der ungerne starp.\\ 
 & \textbf{\begin{large}E\end{large}r} sprach ze Parzivale sân:\\ 
 & "owê, \textbf{junc}, starker man,\\ 
5 & wâ gediente ich ie dise nôt,\\ 
 & daz ich \textbf{vor} dir sol ligen tôt?"\\ 
 & "\textbf{Jâ lâze ich} dich vil gerne leben",\\ 
 & sprach Parzival, "ob dû wilt geben\\ 
 & dirre vrouwen dîne hulde."\\ 
10 & "ich \textbf{en}tuon\textbf{s} niht. ir schulde\\ 
 & ist gein mir \textbf{ze} grœzlîch.\\ 
 & si was werdecheite rîch.\\ 
 & die hât si gar verkrenket\\ 
 & unt mich in nôt gesenket.\\ 
15 & ich leiste anders, swes dû gerst,\\ 
 & ob dû mich des lebens werst.\\ 
 & daz het ich etswenne von gote;\\ 
 & nû ist dîn hant des worden bote,\\ 
 & daz ichs danke \textbf{dîme} prîse",\\ 
20 & \textbf{sus} sprach der vürste wîse,\\ 
 & "Mîn leben kouf ich schône.\\ 
 & in zwein landen krône\\ 
 & treit gewaldeclîche\\ 
 & mîn bruoder, der ist rîche.\\ 
25 & der nim \textbf{dir}, \textbf{swederz} dû wellest,\\ 
 & daz dû mich tôt niht vellest.\\ 
 & ich bin im liep, er lœset mich,\\ 
 & als ich gedinge wider dich.\\ 
 & \begin{large}D\end{large}ar zuo \textbf{nim} \textbf{ich} mîn herzentuom\\ 
30 & von dir. dîn prîslîcher ruom\\ 
\end{tabular}
\scriptsize
\line(1,0){75} \newline
D \newline
\line(1,0){75} \newline
\textbf{3} \textit{Initiale} D  \textbf{7} \textit{Majuskel} D  \textbf{21} \textit{Majuskel} D  \textbf{29} \textit{Initiale} D  \newline
\line(1,0){75} \newline
\newline
\end{minipage}
\hspace{0.5cm}
\begin{minipage}[t]{0.5\linewidth}
\small
\begin{center}*m
\end{center}
\begin{tabular}{rl}
 & betwungen, \textbf{waz} \textbf{man} an \textbf{in} warp,\\ 
 & er tet, \textbf{als} der ungerne starp,\\ 
 & \textbf{und} sprach ze Parcifalen sân:\\ 
 & "owê, \textbf{küene}, starker man,\\ 
5 & wâ gedient ich ie dise nôt,\\ 
 & daz \textit{ich} \textbf{vor} dir sol ligen tôt?"\\ 
 & "\textbf{jâ lâz ich} dich vil gerne leben",\\ 
 & sprach Parcifal, "ob dû wilt geben\\ 
 & dirre vrouwen dîne hulde."\\ 
10 & "ich \textbf{en}tuon \textbf{es} niht. ir schulde\\ 
 & ist gegen mir \textbf{ze} grœzlîch.\\ 
 & si was werdecheite rîch.\\ 
 & die hât si gar verkre\textit{n}ket\\ 
 & und mich in nôt gesenket.\\ 
15 & i\textit{ch} leiste anders, wes d\textit{û} gerst,\\ 
 & ob dû mich de\textit{s} lebenes werst.\\ 
 & d\textit{az} hete ich etwenne von gote;\\ 
 & nû ist dîn hant des w\textit{o}rden bote,\\ 
 & daz ich es danke \textbf{dînem} prîse",\\ 
20 & \textbf{sus} sprach der vürste wîse,\\ 
 & "mîn leben kouf ich schône.\\ 
 & in zwein landen krône\\ 
 & treit gewalteclîche\\ 
 & mîn bruoder, der \textit{i}s\textit{t} rîche.\\ 
25 & der nim \textbf{dô}, \textbf{wederez} dû wellest,\\ 
 & daz dû mich tôt niht vellest.\\ 
 & ich bin ime liep, er lœset mich,\\ 
 & als ich gedinge wider dich.\\ 
 & dar zuo \textbf{nim} \textbf{ich} mîn herzentuom\\ 
30 & von di\textit{r}. \textit{d}în prî\textit{s}lîcher ruom\\ 
\end{tabular}
\scriptsize
\line(1,0){75} \newline
m n o \newline
\line(1,0){75} \newline
\newline
\line(1,0){75} \newline
\textbf{1} betwungen] Betwung n  $\cdot$ waz] wes n \textbf{2} als] alsam n \textbf{3} Parcifalen] parcifal n o \textbf{4} owê] Ower n \textbf{6} ich] \textit{om.} m \textbf{7} jâ] Jo n \textbf{10} entuon] tuͯnde n duͯn o  $\cdot$ niht] mit n \textbf{11} ze] so n o \textbf{13} hât] hette n  $\cdot$ verkrenket] verkrecket m \textbf{15} ich] Jst m [J*]: Jch n o  $\cdot$ wes] was o  $\cdot$ dû] do m \textbf{16} des] de m  $\cdot$ lebenes] leben n lebens o \textbf{17} daz] Do m  $\cdot$ etwenne] \textit{om.} n \textbf{18} worden] werden m o \textbf{22} in] An n  $\cdot$ zwein] zweẏ o \textbf{24} ist] was m \textbf{25} dô wederez] dir welhe n o \textbf{26} dû] [din]: duͯ o  $\cdot$ tôt] do n \textbf{27} er lœset] erloͯsset m \textbf{29} ich] ouch n (o)  $\cdot$ herzentuom] hertzogetuͯm n herczotuͯm o \textbf{30} Von dir der din pristlicher ruͯm m \newline
\end{minipage}
\end{table}
\newpage
\begin{table}[ht]
\begin{minipage}[t]{0.5\linewidth}
\small
\begin{center}*G
\end{center}
\begin{tabular}{rl}
 & betwungen, \textbf{swes} \textbf{man} an \textbf{in} warp,\\ 
 & er tet, \textbf{als} der ungerne starp.\\ 
 & \textbf{er} sprach ze Parzivale sân:\\ 
 & "owê, \textbf{küene}, starker man,\\ 
5 & wâ gedient ich ie dise nôt,\\ 
 & daz ich \textbf{von} dir sol ligen tôt?"\\ 
 & "\textbf{ich lâze} dich vil gerne leben",\\ 
 & sprach Parzival, "obe dû wil geben\\ 
 & dirre vrouwen dîne hulde."\\ 
10 & "ich tuon \textbf{sîn} niht. ir schulde\\ 
 & \begin{large}I\end{large}st gein mir \textbf{\textit{z}e} grœzlîch.\\ 
 & si was werdecheite rîch.\\ 
 & die hât si gar verkrenket\\ 
 & unde mich in nôt gesenket.\\ 
15 & ich leist anders, swes dû gerst,\\ 
 & op dû mich des lebens werst.\\ 
 & daz het ich etswenne von gote;\\ 
 & nû ist dîn hant des worden bote,\\ 
 & daz ich es danke \textbf{dînem} prîse."\\ 
20 & \textbf{dô} sprach der vürste wîse:\\ 
 & "mîn leben koufe ich schône.\\ 
 & in zwein landen krône\\ 
 & treit gewalticlîche\\ 
 & mîn bruoder, der ist rîche.\\ 
25 & der nim \textbf{dir}, \textbf{swederz} dû wellest,\\ 
 & daz dû mich tôt niht vellest.\\ 
 & ich bin im liep, er lœset mich,\\ 
 & als ich gedinge wider dich.\\ 
 & dar zuo \textbf{nim} mîn herzentuom\\ 
30 & von \textit{d}ir. dîn brîslîcher ruom\\ 
\end{tabular}
\scriptsize
\line(1,0){75} \newline
G I O L M Q R Z Fr21 \newline
\line(1,0){75} \newline
\textbf{11} \textit{Initiale} G  \textbf{15} \textit{Initiale} I  \textbf{27} \textit{Initiale} O L Q Z Fr21  \newline
\line(1,0){75} \newline
\textbf{1} betwungen] Betungen R  $\cdot$ swes] swaz I waz L wes Q R \textbf{2} tet] reit R  $\cdot$ der] er L \textbf{3} ze Parzivale] zebarzivale G zuͤ parzifaln I ze parcifal O zuͯ parzifale L (M) zu partzifale Q zu parczifalen R zv parcifaln Z zeparcifale Fr21  $\cdot$ sân] sone M \textbf{5} \textit{Versfolge 266.6-5} M   $\cdot$ Wa mite vor diente ich dise not M  $\cdot$ wâ] Wan O  $\cdot$ ie] \textit{om.} I O Q R Fr21 \textbf{6} \textit{Vers 266.6 fehlt} Q   $\cdot$ ich] ich hie I  $\cdot$ von dir sol] vor dir sol O R Z sol vor dir L sal vor dy M vor sol Fr21 \textbf{8} Parzival] parzifal I M parcifal O (L) Z Fr21 partzifal Q parczifal R \textbf{10} ich tuon sîn] des entuͤn ich I Jch en thun sin M Jch en tun es Q \textbf{11} ze grœzlîch] alzegrozlich G so groͯsklich R \textbf{12} werdecheite] an wirdekeit so R \textbf{13} gar verkrenket] gar vertrenket R gekrenket Fr21 \textbf{14} gesenket] versencket Q \textbf{15} swes] waz L wes M Q R Z  $\cdot$ dû] gu R \textbf{16} werst] gewerst O [gerst]: werst R \textbf{17} het] hat R  $\cdot$ etswenne] entwe R \textbf{18} dîn] dem Q  $\cdot$ des worden] des vorden Q min R  $\cdot$ bote] rot M \textbf{19} daz] Des R  $\cdot$ es] \textit{om.} L Q R  $\cdot$ danke] dancken Q \textbf{20} dô] so I (O) (R) Da M Sust Z \textbf{21} ich] \textit{om.} Z \textbf{22} landen] lande Q \textbf{23} treit] treit min hant I \textbf{24} der ist rîche] vnd ich R \textbf{25} nim] myn M  $\cdot$ dir] \textit{om.} I  $\cdot$ swederz] weders >so< L weders Q (R) so widir M \textbf{26} tôt niht] nih tot I (R) ze tode niht O toten iht Z \textbf{27} ich] ÷ch O \textbf{29} nim] nim ich O (M) Q Z Fr21 nim och R \textbf{30} von] Wan L  $\cdot$ dir] mir G  $\cdot$ dîn brîslîcher] din prischlicher L den preiszlichen Q \newline
\end{minipage}
\hspace{0.5cm}
\begin{minipage}[t]{0.5\linewidth}
\small
\begin{center}*T
\end{center}
\begin{tabular}{rl}
 & betwungen, \textbf{swes} \textbf{er} an \textbf{in} warp,\\ 
 & er tet, \textbf{als} der ungerne starp.\\ 
 & \textbf{er} sprach ze Parcifale sân:\\ 
 & "ouwê, \textbf{küene}, starker man,\\ 
5 & wâ gediend ich ie dise nôt,\\ 
 & daz ich \textbf{von} dir sol ligen tôt?"\\ 
 & "\textbf{Ich lâze} dich vil gerne leben",\\ 
 & sprach Parcifal, "ob dû wilt geben\\ 
 & dirre vrouwen dîne hulde."\\ 
10 & "I\textbf{ne} tuon \textbf{ez} niht. ir schulde\\ 
 & ist gegen mir \textbf{al}ze grœzlîch.\\ 
 & si was werdecheite rîch.\\ 
 & die hât si gar verkrenket\\ 
 & unde mich in nôt gesenket.\\ 
15 & ich leiste anders, swes dû gerst,\\ 
 & ob dû mich des lebens werst.\\ 
 & daz hetich etswenne von gote;\\ 
 & nû ist dîn hant des worden bote,\\ 
 & daz ichs danke prîse",\\ 
20 & \textbf{sus} sprach der vürste wîse,\\ 
 & "mîn leben koufich schône.\\ 
 & in zwein landen krône\\ 
 & treit gewalteclîche\\ 
 & mîn bruoder, der ist rîche.\\ 
25 & der nim \textbf{dir}, \textbf{sweder} dû wellest,\\ 
 & daz dû mich tôt niht vellest.\\ 
 & ich bin im liep, er lœset mich,\\ 
 & alsich gedinge wider dich.\\ 
 & dar zuo \textbf{enpfahe} \textbf{ich} mîn herzogentuom\\ 
30 & von dir. dîn prîslîcher ruom\\ 
\end{tabular}
\scriptsize
\line(1,0){75} \newline
T U V W \newline
\line(1,0){75} \newline
\textbf{7} \textit{Majuskel} T  \textbf{10} \textit{Majuskel} T  \newline
\line(1,0){75} \newline
\textbf{1} swes] wes U was W \textbf{3} er] [*]: Vnd V  $\cdot$ Parcifale] parzefale T Parcifal U parzifale V partzifali W \textbf{5} gediend] gediete U verdienet W  $\cdot$ ich ie] ich U W [*]: ich ie V \textbf{6} von] vor U V W \textbf{7} lâze dich vil] lazen dich wil U \textbf{8} Parcifal] parzifal T V partzifal W \textbf{10} Ine] Ich W  $\cdot$ ir] von ir U \textbf{11} alze grœzlîch] alzegroͤlich V als groͤßlich W \textbf{15} swes] swez T V wes U W \textbf{16} lebens] lebendes V \textbf{18} worden] werden V \textbf{19} danke] [*]: danke dinem V dancke deinem W \textbf{20} sus] Also W \textbf{21} koufich] kuͦnftic U [koͮf*]: koͮfe ich V \textbf{22} in] Mit U [*]: Jn V  $\cdot$ landen] landes U \textbf{24} der] [*]: der V \textbf{25} dir] du W  $\cdot$ sweder] weder iz U sweders V welchs W \textbf{26} tôt] toten W \textbf{29} dar] Das W  $\cdot$ enpfahe ich] nim ich U V neim W \newline
\end{minipage}
\end{table}
\end{document}
