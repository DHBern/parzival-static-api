\documentclass[8pt,a4paper,notitlepage]{article}
\usepackage{fullpage}
\usepackage{ulem}
\usepackage{xltxtra}
\usepackage{datetime}
\renewcommand{\dateseparator}{.}
\dmyyyydate
\usepackage{fancyhdr}
\usepackage{ifthen}
\pagestyle{fancy}
\fancyhf{}
\renewcommand{\headrulewidth}{0pt}
\fancyfoot[L]{\ifthenelse{\value{page}=1}{\today, \currenttime{} Uhr}{}}
\begin{document}
\begin{table}[ht]
\begin{minipage}[t]{0.5\linewidth}
\small
\begin{center}*D
\end{center}
\begin{tabular}{rl}
\textbf{12} & daz ir in deste werder hât,\\ 
 & swie \textbf{joch} mir mîn dinc ergât."\\ 
 & \textit{\begin{large}A\end{large}}ls uns diu âventiure saget,\\ 
 & \textbf{dô} het der helt unverzaget\\ 
5 & enpfangen durch liebe kraft\\ 
 & unt durch wîplîch geselleschaft\\ 
 & \textbf{kleinôtes} tûsent marke wert.\\ 
 & swâ noch ein jude pfandes gert,\\ 
 & er m\textit{ö}htez dar vür enpfâhen.\\ 
10 & ez \textbf{en}dorft \textbf{im} niht versmâhen.\\ 
 & daz sande im \textbf{ein} sîn vriu\textit{n}dîn.\\ 
 & an sînem dienste lac gewin,\\ 
 & der \textbf{wîbe} minne und ir gruoz.\\ 
 & \textbf{doch} wart im selten kumbers buoz.\\ 
15 & urloup nam der wîgant.\\ 
 & muoter, bruoder \textbf{noch des} lant\\ 
 & sîn ouge nimmer mêr erkôs.\\ 
 & dâr an \textbf{doch} maneger vil verlôs.\\ 
 & \textbf{der sich hete an im} erkant,\\ 
20 & \textbf{ê} \textbf{daz} er \textbf{wære dan} gewant,\\ 
 & mit deheiner slahte günste zil,\\ 
 & \textbf{den} wart von im gedanket vil.\\ 
 & es dûhte in \textbf{mêre} denne genuoc.\\ 
 & durch sîne zuht er \textbf{nie} gewuoc,\\ 
25 & daz si\textbf{z} \textbf{tæten} umbe reht.\\ 
 & sîn muot was ebener denne sleht.\\ 
 & swer selbe sagt, wie werd er sî,\\ 
 & dâ ist \textbf{lîhte} ein ungeloube bî.\\ 
 & \textbf{ez} solten die umbesæzen jehen\\ 
30 & und ouch die heten gesehen\\ 
\end{tabular}
\scriptsize
\line(1,0){75} \newline
D \newline
\line(1,0){75} \newline
\textbf{3} \textit{Initiale} D  \newline
\line(1,0){75} \newline
\textbf{3} Als] ÷ls \textit{nachträglich korrigiert zu:} Als D \textbf{4} het] [helt]: het D \textbf{9} möhtez] mohtez D \textbf{11} vriundîn] frivdin D \newline
\end{minipage}
\hspace{0.5cm}
\begin{minipage}[t]{0.5\linewidth}
\small
\begin{center}*m
\end{center}
\begin{tabular}{rl}
 & daz ir in deste werde\textit{r} hât,\\ 
 & wie \textbf{halt} mir mîn dinc ergât."\\ 
 & als uns diu âventiure saget,\\ 
 & \textbf{sô} het der helt unverzaget\\ 
5 & enpfangen durch liebe kraft\\ 
 & und durch wîplîche geselleschaft\\ 
 & \textbf{kleinœte\textit{r}} tûsent marke wert.\\ 
 & wâ noch ein \dag rede\dag  pfandes gert,\\ 
 & er m\textit{ö}hte ez dâ vür enpfâhen.\\ 
10 & ez dorfte \textbf{in} niht versmâhen.\\ 
 & daz sante ime sîn vriundîn.\\ 
 & an sînem dienste lac gewin,\\ 
 & der \textbf{wîbe} minne und ir gruoz.\\ 
 & \textbf{dô} wart im selten kumbers buoz.\\ 
15 & urloup nam der wîgant.\\ 
 & muoter, bruoder \textbf{und ouch daz} lant\\ 
 & sîn ouge nimer mê erkôs.\\ 
 & dâr an \textbf{ouch} meniger vil verlôs.\\ 
 & \textbf{der sich hete an ime} erkant,\\ 
20 & \textbf{wie} \textbf{daz} er \textbf{wære den} gewant,\\ 
 & mit keine\textit{r} slahte günste zil,\\ 
 & \textbf{den} wart von ime gedanket vil.\\ 
 & es dûhte in \textbf{minner} denne genuoc.\\ 
 & durch sîne zuht \dag in\dag  \textbf{nie} gewuoc,\\ 
25 & daz \textit{si} \textbf{tæten} umb reht.\\ 
 & sîn muot was ebene\textit{r} \textit{d}e\textit{nne} sleht.\\ 
 & wer selbe saget, wie wert er sî,\\ 
 & dâ ist \textbf{lîhte} ein unglou\textit{b}e bî.\\ 
 & \textbf{ez} solten die umbesæzen jehen\\ 
30 & und ouch die heten gesehen\\ 
\end{tabular}
\scriptsize
\line(1,0){75} \newline
m n o \newline
\line(1,0){75} \newline
\textbf{3} \textit{Capitulumzeichen} n  \newline
\line(1,0){75} \newline
\textbf{1} werder] werde \textit{nachträglich korrigiert zu:} werder m \textbf{2} halt mir] \textit{nachträglich korrigiert zu:} mir doch m \textbf{4} het] hat n \textbf{6} geselleschaft] geseschafft o \textbf{7} kleinœter] Cleinoteres m \textbf{9} möhte] mochte m \textbf{10} dorfte] bedurffte n \textbf{14} dô] Doch n o \textbf{16} ouch] \textit{om.} n o \textbf{17} erkôs] verkosz n \textbf{18} verlôs] verkosz n \textbf{20} daz er] es n das ẏme o  $\cdot$ den] sin o \textbf{21} keiner] keine m  $\cdot$ günste] genúste o \textbf{25} si] \textit{om.} m  $\cdot$ umb] yme o \textbf{26} muot] munt o  $\cdot$ ebener] ebenere m eber n [*]: obener o  $\cdot$ denne] vnde m \textbf{27} selbe] selbes n selber o \textbf{28} ungloube] vngloule m  $\cdot$ bî] sy n \textbf{30} und] >vnd< o  $\cdot$ die] die \textit{nachträglich korrigiert zu:} die es m \textit{om.} n o \newline
\end{minipage}
\end{table}
\newpage
\begin{table}[ht]
\begin{minipage}[t]{0.5\linewidth}
\small
\begin{center}*G
\end{center}
\begin{tabular}{rl}
 & daz ir in deste werder hât,\\ 
 & swie \textbf{halt} mir mîn dinc ergât."\\ 
 & als uns diu âventiure saget,\\ 
 & \textbf{dô} het der helt unverzaget\\ 
5 & enpfangen durch liebe kraft\\ 
 & unt durch wîplîch geselleschaft\\ 
 & \textbf{kleinôdes} tûsent marke wert.\\ 
 & swâ noch ein jude pfandes gert,\\ 
 & er m\textit{ö}htez dar vür enpfâhen.\\ 
10 & ez dorfte \textbf{im} niht versmâhen.\\ 
 & daz sande im \textbf{ein} sîn vriundîn.\\ 
 & an sînem dienste lac gewin,\\ 
 & \begin{large}D\end{large}er \textbf{wîbe} minne und ir gruoz.\\ 
 & \textbf{des} wart im selten kumbers buoz.\\ 
15 & urloup nam der wîgant.\\ 
 & muoter, bruoder \textbf{noch des} lant\\ 
 & sîn ouge nimer mêr erkôs.\\ 
 & dâr an \textbf{doch} maniger vil verlôs.\\ 
 & \textbf{der sich hete an im} erkant,\\ 
20 & \textbf{ê} \textbf{daz} er \textbf{wære dane} gewant,\\ 
 & mit deheiner slahte gunstes zil,\\ 
 & \textbf{dem} wart von im gedankt vil.\\ 
 & es dûhte in \textbf{mê} dane genuoc.\\ 
 & durch sîne zuht er \textbf{niht} gewuoc,\\ 
25 & daz si\textbf{z} \textbf{tæten} umbe reht.\\ 
 & sîn muot was ebener dane sleht.\\ 
 & swer selbe saget, wie wert er sî,\\ 
 & dâ ist \textbf{lîhte} ein ungeloube bî.\\ 
 & \textbf{es} solten die umbesæzen jehen\\ 
30 & unde ouch die heten gesehen\\ 
\end{tabular}
\scriptsize
\line(1,0){75} \newline
G O L M Q R W Z Fr29 Fr32 \newline
\line(1,0){75} \newline
\textbf{1} \textit{Initiale} O M  \textbf{3} \textit{Initiale} L Q R W Z Fr29 Fr32  \textbf{13} \textit{Initiale} G   $\cdot$ \textit{Versal} Fr32  \textbf{27} \textit{Initiale} Fr32  \newline
\line(1,0){75} \newline
\textbf{1} daz] ÷az O \textbf{2} swie] Wie L Q R W  $\cdot$ halt] had M doch R \textbf{3} uns] \textit{om.} M  $\cdot$ âventiure] alwetewr͑ Q \textbf{4} dô] Da M W Z  $\cdot$ het] hat R \textbf{5} liebe] groͤsser liebe W \textbf{7} kleinôdes] Cleinode Z \textbf{8} swâ] Wa L (Q) (W) R \textbf{9} möhtez] mohtez G (O) (L) (M) (Z) (Fr29) (Fr32) mocht es noch Q \textbf{10} ez dorfte] Jzn dorft O (M) (R) (Fr29) (Fr32) Vnd dorfft Q Vnd endúrffte W  $\cdot$ im] in L R Fr32  $\cdot$ niht] \textit{om.} Fr29 \textbf{11} daz] Dar Fr29  $\cdot$ ein] \textit{om.} Q \textbf{14} Deme doch selten wart kummers buͦß W  $\cdot$ des] Doch O L Q R Z Fr29 (Fr32) Da M  $\cdot$ im] in L  $\cdot$ kumbers] chvmber O \textbf{16} \textit{Die Verse 12.16-20 fehlen} R   $\cdot$ noch des] noch daz O L Z Fr32 vnd W \textbf{17} ouge] ougen M (Q)  $\cdot$ erkôs] verkoß W \textbf{18} vil] \textit{om.} M \textbf{19} der] Fe er W  $\cdot$ hete] hat L \textbf{20} Der von dannen wer gewant W  $\cdot$ wære dane] dannen wer Q (Fr32) \textbf{21} \textit{Versfolge 12.23-24-21-22} Z   $\cdot$ gunstes] gundes O gvnste L (M) (Q) (R) (W) (Z) Fr32 \textbf{22} dem] Den L Z Fr29  $\cdot$ im] [im]: mir Z \textbf{23} es] Sîn O Ez in L Das Q (Fr32)  $\cdot$ dûhte] duch R  $\cdot$ in] in ie Z  $\cdot$ dane] wanne M  $\cdot$ genuoc] zu vil Q guͦt R \textbf{24} niht] nie O L (M) (Q) (R) W Z (Fr29) Fr32  $\cdot$ gewuoc] gewuͦt R \textbf{25} siz] es R  $\cdot$ umbe] \textit{om.} Q \textbf{26} sîn muot] Er W  $\cdot$ ebener dane] ebener wanne M aller dinge Z \textbf{27} swer] Wer L Q R W  $\cdot$ selbe] selben M selber W (Z) \textbf{28} ein] \textit{om.} Z  $\cdot$ ungeloube] vnglawben Q \textbf{29} es] Ez Fr29 (Fr32)  $\cdot$ solten] svln L  $\cdot$ die] de G  $\cdot$ umbesæzen] vnuorsesszin M vm messen \textit{nachträglich korrigiert zu:} vm vessen Q \textbf{30} die] die ez L \newline
\end{minipage}
\hspace{0.5cm}
\begin{minipage}[t]{0.5\linewidth}
\small
\begin{center}*T
\end{center}
\begin{tabular}{rl}
 & daz ir in deste werder hât,\\ 
 & swie \textbf{halt} mir mîn dinc ergât."\\ 
 & \begin{large}A\end{large}ls uns diu âventiure saget,\\ 
 & \textbf{sô} hete der helt unverzaget\\ 
5 & enpfangen durch lieb\textit{e} kraft\\ 
 & und durch wîplîche geselleschaft\\ 
 & \textbf{kleinôde} tûsent marke wert.\\ 
 & swâ noch ein jude pfandes gert,\\ 
 & er m\textit{ö}htez dar vür enpfâhen.\\ 
10 & ez \textbf{en}dorft\textbf{in} niht versmâhen.\\ 
 & daz santim \textbf{ein} sîn vriundîn.\\ 
 & an sînem dienste lac gewin,\\ 
 & der minne und \textbf{ouch} ir gruoz.\\ 
 & \textbf{doch} wart im selten kumbers buoz.\\ 
15 & urloup nam der wîgant.\\ 
 & muoter, bruoder \textbf{noch des} lant\\ 
 & sîn ouge niemer mêre erkôs.\\ 
 & dâr an \textbf{doch} maneger vil verlôs.\\ 
 & \textbf{an dem er sich hete} erkant,\\ 
20 & \textbf{ê} er \textbf{von dannen wære} gewant,\\ 
 & mit dekeiner slahte gunstes zil,\\ 
 & \textbf{dem} wart von im gedanket vil.\\ 
 & e\textit{s} dûhtin \textbf{mêre} danne genuoc.\\ 
 & durch sîne zuht er \textbf{nie} gewuoc,\\ 
25 & daz si\textbf{z} \textbf{tâten} umbe reht.\\ 
 & sîn muot was ebener danne sleht.\\ 
 & Swer selbe saget, wie werd er sî,\\ 
 & dâ ist ein ungeloube bî.\\ 
 & \textbf{ez} solten die umbesæzen jehen\\ 
30 & und ouch die heten gesehen\\ 
\end{tabular}
\scriptsize
\line(1,0){75} \newline
T U V \newline
\line(1,0){75} \newline
\textbf{3} \textit{Initiale} T U V  \textbf{27} \textit{Initiale} U V   $\cdot$ \textit{Majuskel} T  \newline
\line(1,0){75} \newline
\textbf{1} ir in] er iu U \textbf{2} swie halt] Wie hart U swie so V \textbf{4} sô] Do U (V) \textbf{5} liebe] liebiv T grosser liebe V \textbf{7} kleinôde] Cleynodes U  $\cdot$ wert] writ U \textbf{8} swâ] Wa U \textbf{9} möhtez] mohtez T (U) \textbf{13} minne] wibe minne V  $\cdot$ ouch ir] der U [*r]: ir V \textbf{16} des] [d*]: des V \textbf{17} ouge] oͮgen V \textbf{18} vil] \sout{sit} vil T \textbf{19} Der sich hete an im erkant U (V) \textbf{23} es] ez T \textbf{25} siz] [*]: sv́z V \textbf{26} ebener] ebern U \textbf{27} Wer selbe saget wie er wirt si U \textbf{28} ist] ist lichte U (V) \textbf{29} umbesæzen] vmb sitzen U \textbf{30} heten] hetten es V \newline
\end{minipage}
\end{table}
\end{document}
