\documentclass[8pt,a4paper,notitlepage]{article}
\usepackage{fullpage}
\usepackage{ulem}
\usepackage{xltxtra}
\usepackage{datetime}
\renewcommand{\dateseparator}{.}
\dmyyyydate
\usepackage{fancyhdr}
\usepackage{ifthen}
\pagestyle{fancy}
\fancyhf{}
\renewcommand{\headrulewidth}{0pt}
\fancyfoot[L]{\ifthenelse{\value{page}=1}{\today, \currenttime{} Uhr}{}}
\begin{document}
\begin{table}[ht]
\begin{minipage}[t]{0.5\linewidth}
\small
\begin{center}*D
\end{center}
\begin{tabular}{rl}
\textbf{804} & \begin{large}M\end{large}anec \textbf{juncvrouwe} unt \textbf{ir} ander diet\\ 
 & sich von der küneginne schiet,\\ 
 & sô daz si tâten klage schîn.\\ 
 & dô \textbf{nâmen} Loherangrin\\ 
5 & unt \textbf{si}, muoter wolgetân,\\ 
 & die templeise unt riten \textbf{dan}\\ 
 & gein Munsalvæsche balde.\\ 
 & "\textbf{zeiner zît ûf} disem walde",\\ 
 & sprach Parzival, "dâ sach ich stên\\ 
10 & eine klôsen, dâ durch \textbf{balde} gên\\ 
 & einen snellen brunnen clâr.\\ 
 & ob ir si wizzet, sô wîset mich dar."\\ 
 & von sînen gesellen wart im gesagt,\\ 
 & si wisten eine: "dâ wont ein magt\\ 
15 & al klagende ûf vriwendes sarche.\\ 
 & \textbf{diu} ist rehter güete ein arche.\\ 
 & unser \textbf{reise} gêt \textbf{ir} nâhe bî.\\ 
 & man vind\textit{e}t si selten jâmers vrî."\\ 
 & Der künec sprach: "wir sulen si sehen."\\ 
20 & des wart \textbf{i\textit{m}} volge \textbf{an in} verjehen.\\ 
 & si riten \textbf{vür sich} drâte\\ 
 & unt vunden des âbents spâte\\ 
 & Sigunen an ir venje tôt.\\ 
 & \textbf{dâ} \textbf{sach} diu künegîn \textbf{jâmers} nôt.\\ 
25 & Si brâchen zuo \textbf{z}ir dar în.\\ 
 & Parzival durch die nifteln sîn\\ 
 & \textbf{bat} ûf wegen \textbf{den} sarches stein.\\ 
 & \textbf{Schianatulander} schein\\ 
 & unervûlt schône balsemvar.\\ 
30 & man leite si nâhe zuo \textbf{z}im dar,\\ 
\end{tabular}
\scriptsize
\line(1,0){75} \newline
D \newline
\line(1,0){75} \newline
\textbf{1} \textit{Initiale} D  \textbf{19} \textit{Majuskel} D  \textbf{25} \textit{Majuskel} D  \newline
\line(1,0){75} \newline
\textbf{7} Munsalvæsche] Mvnsalvæsce D \textbf{9} Parzival] Parcifal D \textbf{18} vindet] vindent D \textbf{20} im] in D \textbf{26} Parzival] Parcifal D \textbf{28} Schianatulander] Scianatvlander D \newline
\end{minipage}
\hspace{0.5cm}
\begin{minipage}[t]{0.5\linewidth}
\small
\begin{center}*m
\end{center}
\begin{tabular}{rl}
 & manigiu \textbf{juncvrowe} und ander diet\\ 
 & sich von der künigîn schiet,\\ 
 & sô daz si tâten klage schîn.\\ 
 & dô \textbf{nâmen} Lohelangrin\\ 
5 & und \textbf{sîn} muoter wol getân\\ 
 & die templeis und riten \textbf{sân}\\ 
 & gegen Muntsalva\textit{sche} balde.\\ 
 & "\textbf{in} disem \textbf{selben} walde",\\ 
 & sprach Parcifal, "d\textit{â} sach ich stên\\ 
10 & \textit{e}in klôsen \textit{\textbf{und}} d\textit{â} durch gên\\ 
 & einen snellen brunnen clâr.\\ 
 & \textit{obe ir si wizzet, sô wîset mich dar."}\\ 
 & von sînen gesellen wart i\textit{m} gesaget,\\ 
 & si wusten ein: "d\textit{â} wonte ein maget\\ 
15 & al klagende ûf vriundes sarc.\\ 
 & \textbf{diu} ist rehter güete ein arc.\\ 
 & unser \textbf{reise} gât \textbf{ir} nâhe bî.\\ 
 & man vint si selten jâmers vrî."\\ 
 & der künic sprach: "wir sullen si sehen."\\ 
20 & des wart \textbf{im} volge ver\textit{j}ehen.\\ 
 & si riten \textbf{vür sich} drâte\\ 
 & und vunden\textbf{s} des âbendes spâte,\\ 
 & Sigunen, an ir venjen tôt.\\ 
 & \textbf{d\textit{â}} \textbf{sach} diu künigîn \textbf{jâmers} nôt.\\ 
25 & si brâch\textit{en} zuo ir dar în.\\ 
 & Parcifal durch die niftel sîn\\ 
 & \textbf{bat} ûf wege\textit{n} \textbf{des} sarkes stein.\\ 
 & \textbf{Schi\textit{a}natulander} schein\\ 
 & unervûlet schône balsamvar.\\ 
30 & man leit si nâhe zuo im dar,\\ 
\end{tabular}
\scriptsize
\line(1,0){75} \newline
m n o V V' W \newline
\line(1,0){75} \newline
\textbf{19} \textit{Initiale} W  \newline
\line(1,0){75} \newline
\textbf{1} juncvrowe] jungfrowen o \textbf{2} künigîn] konig o \textbf{3} Grosze clage sie taten schin V'  $\cdot$ daz] \textit{om.} o  $\cdot$ klage] clagen o \textbf{4} nâmen] [nomen*]: noment sv́ V namen sie V'  $\cdot$ Lohelangrin] lohelangrim o lohelagrin V' lohelangrein W \textbf{6} die] Den p o Vnd die V'  $\cdot$ sân] [*]: dan V dan V' W \textbf{7} Muntsalvasche] muͯntsalua m muntsaluasce n o munsalfasche V mvnschalfasche V' montsoluatsch W \textbf{8} disem] deme V' (W)  $\cdot$ walde] [wuͯnde]: walde o \textbf{9} Herr partzifal sach stan W  $\cdot$ Parcifal] pazifal V parzifal V'  $\cdot$ dâ] do m n o V V' \textbf{10} ein klôsen und] Vnd ein closen m n o  $\cdot$ dâ] do m n o V W \textbf{12} \textit{Vers 804.12 fehlt} m  \textbf{13} im] in m n o \textbf{14} dâ] do m n o V V' W  $\cdot$ wonte] wonet V' (W) \textbf{15} al] Alle n Alzu V'  $\cdot$ vriundes] [*]: ir frúdez V ir frundes V' \textbf{16} ein] eyn e o \textbf{20} volge] volge an in V volgen an in V'  $\cdot$ verjehen] verehen m \textbf{21} drâte] traten o \textbf{22} vundens] fundent W  $\cdot$ des] \textit{om.} V V' \textbf{23} Sigunen] Sigunin o \textbf{24} dâ] Do m n o V V' W \textbf{25} brâchen] brach m  $\cdot$ zuo ir] zvͦzir V  $\cdot$ dar] hin V' \textbf{26} Parcifal] Parzefal V Parzifal V' Herr partzifal W  $\cdot$ niftel] nifteln V \textbf{27} wegen] wegens m  $\cdot$ des sarkes] den starcken W \textbf{28} [*]: Er smarket wol er schein V'  $\cdot$ Schianatulander] Schionatulander m n Schionantulander o Zschinatulander V Tschionatulanders W  $\cdot$ schein] schin o \textbf{29} unervûlet] Vnferfuͯlet o [*fulet]: Vnerfulet V  $\cdot$ schône] wol V'  $\cdot$ balsamvar] balsemen var V gepelsamet var W \textbf{30} leit] leite V V' W  $\cdot$ nâhe] naher V' \newline
\end{minipage}
\end{table}
\newpage
\begin{table}[ht]
\begin{minipage}[t]{0.5\linewidth}
\small
\begin{center}*G
\end{center}
\begin{tabular}{rl}
 & \begin{large}M\end{large}anic \textbf{juncvrouwe} unde \textbf{ir} ander diet\\ 
 & sich von der küneginne schiet,\\ 
 & sô daz si tâten klage schîn.\\ 
 & dô \textbf{nam} Loherangrin\\ 
5 & unde \textbf{sîn} muoter wolgetân\\ 
 & die templeis unde riten \textbf{dan}\\ 
 & gên Muntsalfatsche balde.\\ 
 & "\textbf{zeiner zît ûf} disem walde",\\ 
 & sprach Parcival, "dâ sach ich stên\\ 
10 & eine klôsen, dâ durch \textbf{balde} gên\\ 
 & einen snellen brunnen clâr.\\ 
 & obe ir si wizzet, sô wîset mich dar."\\ 
 & von sîne\textit{n} gesellen wart im gesaget,\\ 
 & si wessen ein: "dâ wont ein maget\\ 
15 & al klagende ûf \textbf{ir} vriundes sarch.\\ 
 & \textbf{ir herze} ist rehter güete ein arch.\\ 
 & unser \textbf{strâze} gêt \textbf{dâ} nâhen bî.\\ 
 & man vindet si selten jâmers vrî."\\ 
 & der künic sprach: "wir sulen si sehen."\\ 
20 & des wart \textbf{ein} volge \textbf{an in} verjehen.\\ 
 & si riten \textbf{des endes} drâte\\ 
 & unde vunden des âbendes spâte\\ 
 & Sigunen an ir venje tôt.\\ 
 & \textbf{des} \textbf{kom} diu küniginne \textbf{in} nôt.\\ 
25 & si brâchen zuo \textit{i}r dar în.\\ 
 & Parcival durch die niftelen sîn\\ 
 & \textbf{hiez} ûf wegen \textbf{des} sarches stein.\\ 
 & \textbf{dâr ûz der tôte rîter} schein\\ 
 & unervûlt schône balsemvar.\\ 
30 & man leit si nâhen zuo \textit{i}m dar,\\ 
\end{tabular}
\scriptsize
\line(1,0){75} \newline
G I L Z Fr48 \newline
\line(1,0){75} \newline
\textbf{1} \textit{Initiale} G I Z Fr48  \textbf{19} \textit{Initiale} I  \newline
\line(1,0){75} \newline
\textbf{1} ir] \textit{om.} L Z \textbf{4} nam] namen si I namen L Fr48  $\cdot$ Loherangrin] Leharagrin I joherangrin L lohagrin Z Lohrangrin Fr48 \textbf{7} Muntsalfatsche] mvntsalvatsche G L muntshaluasce I montsalvatsche Z Munsaluatsch Fr48 \textbf{9} Parcival] parzival G parzifal I L parcifal Z partzifal Fr48  $\cdot$ dâ] do L Fr48 \textbf{11} clâr] chlare G \textbf{12} ir si] irsz L  $\cdot$ sô] \textit{om.} L \textbf{13} sînen] sinem G  $\cdot$ im] in Z \textbf{14} dâ] do L Fr48 \textbf{15} al] Alle Fr48 \textbf{16} ir herze] Die L Z (Fr48) \textbf{17} dâ nâhen] nahe da L \textbf{20} in] im I \textbf{23} Sigunen] :::unen Fr48 \textbf{24} in] \textit{om.} L \textbf{25} si brâchen] ::: brach Fr48  $\cdot$ ir] zir G \textbf{26} Parcival] parzival G parzifal I (L) Parcifal Z :::rtzifal Fr48  $\cdot$ niftelen] niftel L \textbf{27} des sarches] den sarg L \textbf{28} dâr ûz der tôte rîter] Shýonatv delander L Tschionatulander Z :::chianatulander Fr48 \textbf{29} unervûlt] Vnverfvlet Z \textbf{30} im] zim G \newline
\end{minipage}
\hspace{0.5cm}
\begin{minipage}[t]{0.5\linewidth}
\small
\begin{center}*T
\end{center}
\begin{tabular}{rl}
 & manegiu \textbf{junge vrouwe} und \textbf{ir} ander diet\\ 
 & sich von der küneginne schiet,\\ 
 & sô daz si tâten klage schîn.\\ 
 & dô \textbf{nâmen} Lohrangrin\\ 
5 & und \textbf{sîne} muoter wol getân\\ 
 & die templeise und riten \textbf{dan}\\ 
 & gein Munsalvasche balde.\\ 
 & "\textbf{zuo einer zît ûf} disem walde",\\ 
 & sprach Parcifal, "d\textit{â} sach ich stên\\ 
10 & eine klôse, d\textit{â} durch \textbf{balde} gên\\ 
 & einen snellen brunnen clâr.\\ 
 & ob ir si wizzet, sô wîset mich dar."\\ 
 & von sînen gesellen wart im gesaget,\\ 
 & si wisten eine: "dâ wont einiu maget\\ 
15 & al klagende ûf \textbf{ir} vriundes sarke.\\ 
 & \textbf{diu} ist \textbf{ouch} rehter güete ein arke.\\ 
 & unser \textbf{strâze} gêt \textbf{ir} nâhe bî.\\ 
 & man vindet si selten jâmers vrî."\\ 
 & \begin{large}D\end{large}er künec sprach: "wir soln si sehen."\\ 
20 & des wart \textbf{ein} volge \textbf{an in} verjehen.\\ 
 & si riten \textbf{des endes} drâte\\ 
 & und vunden des âbendes spâte\\ 
 & Sygunen an ir venje tôt.\\ 
 & \textbf{des} \textbf{kam} diu küneginne \textbf{in} nôt.\\ 
25 & si brâchen zuo ir dar în.\\ 
 & Parcifal durch die niftel sîn\\ 
 & \textbf{hiez} ûf wegen \textbf{des} sarkes stein.\\ 
 & \textbf{Schinohtudelander} schein\\ 
 & une\textit{r}vûlt schône balsemvar.\\ 
30 & man leite si \textbf{im} nâhe zuo im dar,\\ 
\end{tabular}
\scriptsize
\line(1,0){75} \newline
U Q R \newline
\line(1,0){75} \newline
\textbf{1} \textit{Initiale} R  \textbf{19} \textit{Initiale} U  \newline
\line(1,0){75} \newline
\textbf{1} manegiu junge vrouwe] Manick iunckfrawe Q (R)  $\cdot$ ir] \textit{om.} R \textbf{2} der küneginne] dem kung R \textbf{3} klage] clagen Q \textbf{4} Lohrangrin] lorangrin U lohrangrein Q \textbf{6} templeise] templeiser Q \textbf{7} Munsalvasche] muͦntsalvatsche U muntsaluasche Q [Munschaual*]: Munshauasche R \textbf{9} Parcifal] Parzifal U partzifal Q parczifal R  $\cdot$ dâ] do U Q R \textbf{10} klôse] closen R  $\cdot$ dâ] do U Q \textbf{11} snellen] schlechtten R \textbf{13} von] Vo Q  $\cdot$ im] in Q \textbf{14} wisten] wosten Q  $\cdot$ dâ] do Q  $\cdot$ wont] wuͦnt U \textbf{15} ir] irm U irs Q R \textbf{16} ouch] \textit{om.} Q R \textbf{17} nâhe] [naher]: nahez Q \textbf{23} Sygunen] Syguͦnen U Siegúne Q Sygunnen R \textbf{25} brâchen] [sprachen]: brachen Q \textbf{26} Parcifal] Parzifal U Partzifal Q Parczifal R \textbf{28} Schinohtudelander] Syonatulander U Sciotulanders Q Schionatulander R \textbf{29} unervûlt] Vnevult U Vnerfullet Q  $\cdot$ balsemvar] balseme var U \textbf{30} leite] legt R  $\cdot$ si im] sie Q \textit{om.} R \newline
\end{minipage}
\end{table}
\end{document}
