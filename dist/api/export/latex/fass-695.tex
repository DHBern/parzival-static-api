\documentclass[8pt,a4paper,notitlepage]{article}
\usepackage{fullpage}
\usepackage{ulem}
\usepackage{xltxtra}
\usepackage{datetime}
\renewcommand{\dateseparator}{.}
\dmyyyydate
\usepackage{fancyhdr}
\usepackage{ifthen}
\pagestyle{fancy}
\fancyhf{}
\renewcommand{\headrulewidth}{0pt}
\fancyfoot[L]{\ifthenelse{\value{page}=1}{\today, \currenttime{} Uhr}{}}
\begin{document}
\begin{table}[ht]
\begin{minipage}[t]{0.5\linewidth}
\small
\begin{center}*D
\end{center}
\begin{tabular}{rl}
\textbf{695} & \textit{\begin{large}I\end{large}}ch sage \textbf{iu} \textbf{mære}, ob ich kan:\\ 
 & \textbf{dô} \textbf{sprach} von disem einem man\\ 
 & in bêden hern die wîsen,\\ 
 & daz si begunden prîsen\\ 
5 & sîne rîterlîche tât:\\ 
 & "der dâ den prîs genomen hât,\\ 
 & welt irs jehen, deist Parzival."\\ 
 & der was \textbf{ouch} sô lieht gemâl,\\ 
 & ez \textbf{en}wart nie rîter baz getân,\\ 
10 & des jâhen wîb unde man,\\ 
 & dô in Gawan brâhte,\\ 
 & der des \textbf{hin} zim gedâhte,\\ 
 & daz er in hiez kleiden.\\ 
 & dô truoc man dar in beiden\\ 
15 & von tiwerer koste \textbf{glîch} gewant.\\ 
 & über \textbf{al diz} mære \textbf{wart} \textbf{erkant},\\ 
 & daz Parzival dâ wære komen,\\ 
 & von dem sô dicke was vernomen,\\ 
 & daz er hôhen prîs bejagete;\\ 
20 & vür wâr daz \textbf{maneger} sagete.\\ 
 & Gawan sprach: "wiltû schouwen\\ 
 & dînes künnes vier vrouwen\\ 
 & unt ander vrouwen wol gevar,\\ 
 & sô gên ich gern mit dir dar."\\ 
25 & Dô sprach Gahmuretes kint:\\ 
 & "ob hie werde vrouwen sint,\\ 
 & den soltû mich \textbf{unmæren} niht.\\ 
 & \textbf{ein} ieslîch \textbf{vrouwe} mich ungern siht,\\ 
 & diu bî dem Plimizœl gehôrt\\ 
30 & \textbf{hât} von mir \textbf{valschlîchiu} wort.\\ 
\end{tabular}
\scriptsize
\line(1,0){75} \newline
D \newline
\line(1,0){75} \newline
\textbf{1} \textit{Initiale} D  \textbf{25} \textit{Majuskel} D  \newline
\line(1,0){75} \newline
\textbf{1} Ich] ÷ch D \textbf{7} Parzival] Parcifal D \textbf{17} Parzival] Parcifal D \textbf{25} Gahmuretes] Gahmvrets D \textbf{29} Plimizœl] Plimizoͤl D \newline
\end{minipage}
\hspace{0.5cm}
\begin{minipage}[t]{0.5\linewidth}
\small
\begin{center}*m
\end{center}
\begin{tabular}{rl}
 & ich sage \textbf{mêre}, ob ich kan:\\ 
 & \textbf{dô} \textbf{sprach} von disem einen man\\ 
 & in beiden heren die wîsen,\\ 
 & daz si begunden prîsen\\ 
5 & sîne ritterlîche tât:\\ 
 & "der d\textit{â} den prîs genomen hât,\\ 
 & wolt irs jehen, daz ist Parcifal."\\ 
 & der was \textbf{ouch} sô lieht gemâl,\\ 
 & ez wart nie ritter baz getân,\\ 
10 & de\textit{s} jâhen wîp und man,\\ 
 & dô in Gawan brâhte,\\ 
 & der des zuo im gedâhte,\\ 
 & daz \textit{er} in hiez kleiden.\\ 
 & dô truoc man dar in beiden\\ 
15 & von tiurer koste \textbf{rîch} gewant,\\ 
 & über \textbf{alliu dis\textit{iu}} mære \textbf{erkant}.\\ 
 & daz Parcifal d\textit{â} wær komen,\\ 
 & von dem sô dicke was vernomen,\\ 
 & daz er hôhen prîs bejagte;\\ 
20 & vür wâr daz \textbf{maniger} sagte.\\ 
 & Gawan sprach: "wiltû schouwen\\ 
 & dînes künnes vier vrouwen\\ 
 & und ander vrowen wolgevar,\\ 
 & sô gân ich gerne mit dir dar."\\ 
25 & dô sprach Gahmuretes kint:\\ 
 & "ob hie werde vrowen sint,\\ 
 & den soltû mich \textbf{unmæren} niht.\\ 
 & \textbf{ein} ieglîch \textbf{vrowe} \textit{mich} ungern siht,\\ 
 & diu bî de\textit{m} Plimizol gehôrt\\ 
30 & \textbf{het} von mir \textbf{valschlîchiu} wort.\\ 
\end{tabular}
\scriptsize
\line(1,0){75} \newline
m n o \newline
\line(1,0){75} \newline
\newline
\line(1,0){75} \newline
\textbf{1} sage] sage úch n (o) \textbf{6} dâ] do m n o \textbf{10} des] De m \textbf{11} Gawan] g:wan o \textbf{13} er] \textit{om.} m \textbf{16} disiu] dis m (o) \textbf{17} dâ] do m n o \textbf{22} künnes vier] konnens vie o \textbf{24} dar] [dan]: dar n \textbf{25} Gahmuretes] gahmurettes m gamúretes n gahúmeretes o  $\cdot$ kint] [hant]: kint n \textbf{28} mich] \textit{om.} m \textbf{29} bî] bin o  $\cdot$ dem] den m  $\cdot$ Plimizol] blúmzol n \textbf{30} het] hette n \newline
\end{minipage}
\end{table}
\newpage
\begin{table}[ht]
\begin{minipage}[t]{0.5\linewidth}
\small
\begin{center}*G
\end{center}
\begin{tabular}{rl}
 & \begin{large}I\end{large}ch sage \textbf{iu} \textbf{mêre}, obe ich kan:\\ 
 & \textbf{nû} \textbf{sprach} von disem einem man\\ 
 & in beiden heren die wîsen,\\ 
 & daz si begunden brîsen\\ 
5 & sîne rîterlîche tât:\\ 
 & "der dâ den brîs genomen hât,\\ 
 & welt irs jehen, daz ist Parcival."\\ 
 & der was \textbf{êt} sô lieht gemâl,\\ 
 & ez\textbf{ne} wart nie rîter baz getân,\\ 
10 & des jâhen wîp unde man,\\ 
 & dô in Gawan brâhte,\\ 
 & der des \textbf{hin} ze im gedâhte,\\ 
 & daz er in hieze kleiden.\\ 
 & dô truoc man dar in beiden\\ 
15 & von tiwer koste \textbf{glîch} gewant.\\ 
 & über \textbf{al ditze} mære \textbf{wart} \textbf{erkant},\\ 
 & daz Parcival dâ wære komen,\\ 
 & von dem sô dicke was vernomen,\\ 
 & daz er hôhen brîs bejagte;\\ 
20 & vür wâr daz \textbf{maniger} sagte.\\ 
 & Gawan sprach: "wil dû schouwen\\ 
 & dînes künnes vier vrouwen\\ 
 & unde ander vrouwen wol gevar,\\ 
 & sô gên ich gerne mit dir dar."\\ 
25 & dô sprach Gahmuretes kint:\\ 
 & "op hie werde vrouwen sint,\\ 
 & den soltû mich \textbf{unmæren} niht.\\ 
 & etslîch \textbf{vrouwe} mich ungerne siht,\\ 
 & diu bî dem Blimzol gehôrt\\ 
30 & \textbf{hât} von mir \textbf{valschlîchiu} wort.\\ 
\end{tabular}
\scriptsize
\line(1,0){75} \newline
G I L M Z Fr20 \newline
\line(1,0){75} \newline
\textbf{1} \textit{Initiale} G L Z Fr20  \textbf{11} \textit{Initiale} I  \newline
\line(1,0){75} \newline
\textbf{1} Ich] ÷ch Fr20 \textbf{2} nû] Do Z  $\cdot$ sprach] sprachen I L  $\cdot$ einem] einen I Z (Fr20) eȳ M \textbf{3} in] Jnden M \textbf{4} si] si in I \textbf{5} tât] getat I \textbf{7} irs] ir sin I  $\cdot$ jehen] sehin M  $\cdot$ ist] was I  $\cdot$ Parcival] parcifal G Z Parzifal I (L) (M) \textbf{8} êt] \textit{om.} Z  $\cdot$ lieht] licht L M \textbf{9} rîter] man L \textbf{10} des] Das M  $\cdot$ jâhen] iach da I sprachen M \textbf{11} dô] Da M Z \textbf{14} dô] Da M Z  $\cdot$ dar] \textit{om.} L \textbf{15} glîch] rich I geliche L in glich Z \textbf{16} al] allez I alle M  $\cdot$ ditze] daz L (M)  $\cdot$ erkant] bikant M \textbf{17} Parcival] parcifal G Z Parzifal I (L) parzzifal M  $\cdot$ dâ] danne M \textbf{18} sô] \textit{om.} I  $\cdot$ was] wer I \textbf{25} dô] Da M  $\cdot$ Gahmuretes] Gahmv̂retes G Gahmvretes L gamuͯretis M gamuretes Z \textbf{26} hie werde] die werden L \textbf{27} mich] \textit{om.} M \textbf{28} etslîch] Ein ieslich Z  $\cdot$ vrouwe] \textit{om.} L vrouwen M \textbf{29} dem] den M  $\cdot$ Blimzol] [*]: blimizol I plimizol L M Z \textbf{30} valschlîchiu] valschev I (L) (M) \newline
\end{minipage}
\hspace{0.5cm}
\begin{minipage}[t]{0.5\linewidth}
\small
\begin{center}*T
\end{center}
\begin{tabular}{rl}
 & \begin{large}I\end{large}ch sage \textbf{mêre}, ob ich kan:\\ 
 & \textbf{nû} \textbf{sprâchen} von diseme einen man\\ 
 & in beiden hern die wîsen,\\ 
 & daz si begunden prîsen\\ 
5 & sîne rîterlîche tât:\\ 
 & "der d\textit{â} den prîs genomen hât,\\ 
 & wolt ir e\textit{s} jehen, daz ist Parcifal."\\ 
 & der was \textbf{doch} sô lieht gemâl,\\ 
 & ez \textbf{en}wart nie rîter baz getân,\\ 
10 & des j\textit{â}hen wîp und man,\\ 
 & dô in Gawan brâhte,\\ 
 & der des \textbf{hin} zuo i\textit{m} gedâhte,\\ 
 & daz er in hiez kleiden.\\ 
 & dô truoc man dar in beiden\\ 
15 & von tiurer koste \textbf{glîch} gewant.\\ 
 & über \textbf{al daz} mær \textbf{wart} \textbf{bekant},\\ 
 & daz Parcifal d\textit{â} wære komen,\\ 
 & von dem sô dicke was vernomen,\\ 
 & daz er hôhen prîs bejagete;\\ 
20 & vür wâr daz \textbf{man} sagete.\\ 
 & Gawan sprach: "wilt dû schouwen\\ 
 & dînes künnes viere vrouwen\\ 
 & und ander vrouwen wol gevar,\\ 
 & sô gên ich gerne mit dir dar."\\ 
25 & dô sprach Gahmuretes kint:\\ 
 & "ob hie werde vrouwen sint,\\ 
 & den soltû mich \textbf{unêren} niht.\\ 
 & \textbf{ein} ieclîchiu mich ungerne siht,\\ 
 & diu bî dem Plymizol \textbf{hât} gehôrt\\ 
30 & von mir \textbf{valschiu} wort.\\ 
\end{tabular}
\scriptsize
\line(1,0){75} \newline
U V W Q R \newline
\line(1,0){75} \newline
\textbf{1} \textit{Großinitiale} U   $\cdot$ \textit{Initiale} Q  \textbf{21} \textit{Initiale} W  \newline
\line(1,0){75} \newline
\textbf{1} mêre] [*]: v́ch mere V eúch mere W (Q) (R) \textbf{2} nû] Do R  $\cdot$ sprâchen] sprach Q R  $\cdot$ diseme einen] dem eynigen W \textbf{3} wîsen] wissen Q \textbf{5} tât] tag R \textbf{6} dâ] do U V W Q \textbf{7} es] iz U  $\cdot$ daz] \textit{om.} W  $\cdot$ Parcifal] Parzifal U her Parzefal V herr partzifal W partzifal Q parczifal R \textbf{8} doch] eht V (W) (R) auch Q  $\cdot$ lieht] licht Q \textbf{9} ez] Czu Q  $\cdot$ enwart] wart Q (R) \textbf{10} des] Das W  $\cdot$ jâhen] iehen U  $\cdot$ wîp und] [*]: imme wip vnd V frawen vnd auch W \textbf{11} Gawan] herr gawan W Gawin R \textbf{12} im] in U \textbf{14} dar in] in dar Q \textbf{15} glîch] [*]: gelich V \textbf{16} mær] her W  $\cdot$ bekant] erkant W \textbf{17} Parcifal] Parzifal U Parzefal V partzifal W Q parczifal R  $\cdot$ dâ] do U V W Q \textbf{20} man] maniger V (W) (Q) (R) \textbf{21} Gawan] Gawin R \textbf{22} viere] vier V W Q R \textbf{25} Gahmuretes] Gamuͦhretes U Gamerettes V gamuretes W gamúretes Q Gahamurtes R \textbf{26} hie werde] die werden Q \textbf{27} soltû] [*]: solt du V solt Q  $\cdot$ unêren] vnmeren V (Q) R \textbf{28} mich ungerne] [*]: frowe mich vngerne V frawe mich vngerne W (Q) frow mich vngert R \textbf{29} Die [*]: bi deme plimenzol hat gehort V  $\cdot$ Plymizol] plimizol Q plimiczol R  $\cdot$ hât] \textit{om.} W Q R \textbf{30} von] Hat von W (Q) R  $\cdot$ valschiu] valschliche W Q falschlichú R \newline
\end{minipage}
\end{table}
\end{document}
