\documentclass[8pt,a4paper,notitlepage]{article}
\usepackage{fullpage}
\usepackage{ulem}
\usepackage{xltxtra}
\usepackage{datetime}
\renewcommand{\dateseparator}{.}
\dmyyyydate
\usepackage{fancyhdr}
\usepackage{ifthen}
\pagestyle{fancy}
\fancyhf{}
\renewcommand{\headrulewidth}{0pt}
\fancyfoot[L]{\ifthenelse{\value{page}=1}{\today, \currenttime{} Uhr}{}}
\begin{document}
\begin{table}[ht]
\begin{minipage}[t]{0.5\linewidth}
\small
\begin{center}*D
\end{center}
\begin{tabular}{rl}
\textbf{562} & \begin{large}G\end{large}edenket, hêrre, ob ir sît wert,\\ 
 & disen schilt unt iwer swert\\ 
 & lâzet \textbf{ninder} von iu komen.\\ 
 & sô ir wænet, daz ende habe g\textit{e}nomen\\ 
5 & iwer kumber grôzlîch,\\ 
 & alrêst \textbf{strîte ist er gelîch}."\\ 
 & Dô Gawan ûf sîn ors gesaz,\\ 
 & diu magt wart an vreuden laz.\\ 
 & alle, die dâ wâren, klageten;\\ 
10 & wênic si des \textbf{verdageten}.\\ 
 & Er sprach zem wirte: "gan mirs got,\\ 
 & iwer getriulîch urbot,\\ 
 & daz ir mîn sus pflâget,\\ 
 & \textbf{geltes} mich niht betrâget."\\ 
15 & Urloup er zer meide nam,\\ 
 & die grôzes jâmers wol gezam.\\ 
 & er reit hin, si klageten hie.\\ 
 & ob ir nû gerne \textbf{hœret}, wie\\ 
 & Gawane dâ geschæhe,\\ 
20 & deste gerner ichs \textbf{iu} \textbf{verjæhe}.\\ 
 & Ich sage, als ichz hân vernomen:\\ 
 & dô er was vür die porten komen,\\ 
 & er vant den krâmære\\ 
 & \textbf{unt} des krâm niht lære.\\ 
25 & dâ lac inne veile,\\ 
 & daz ichs wære der geile,\\ 
 & het ich alsô rîche habe.\\ 
 & Gawan vor im erbeizet abe.\\ 
 & sô rîchen market er nie gesach,\\ 
30 & als im ze sehen \textbf{al} dâ geschach.\\ 
\end{tabular}
\scriptsize
\line(1,0){75} \newline
D \newline
\line(1,0){75} \newline
\textbf{1} \textit{Initiale} D  \textbf{7} \textit{Majuskel} D  \textbf{11} \textit{Majuskel} D  \textbf{15} \textit{Majuskel} D  \textbf{21} \textit{Majuskel} D  \newline
\line(1,0){75} \newline
\textbf{4} genomen] gonomn D \newline
\end{minipage}
\hspace{0.5cm}
\begin{minipage}[t]{0.5\linewidth}
\small
\begin{center}*m
\end{center}
\begin{tabular}{rl}
 & gedenket, hêrre, ob ir\textbf{s} sît wert,\\ 
 & dise\textit{n} schilt und iuwer swert\\ 
 & lâzet \textbf{ninder} von iu komen.\\ 
 & sô ir wænet, daz ende ha\textit{b} genomen\\ 
5 & iuwer kumber grôzlîch,\\ 
 & \textbf{dan} allerêrst \textbf{sô hebe\textit{t} er sich}."\\ 
 & \begin{large}D\end{large}ô Gawan ûf sîn ros gesaz,\\ 
 & diu maget wart a\textit{n} vröuden laz.\\ 
 & alle, die d\textit{â} wâren, klagten;\\ 
10 & wênic si des \textbf{gedagten}.\\ 
 & er sprach zem \textit{wirt}: "gan mirs got,\\ 
 & iuwer getriuwelîch urbot,\\ 
 & daz ir mîn \textit{s}us pflâget,\\ 
 & \textbf{soldes} mich niht betrâget."\\ 
15 & urloup er zuor megde nam,\\ 
 & die grôzes jâmers wol gezam.\\ 
 & er reit hin, si klagten hie.\\ 
 & ob ir \textit{nû} gerne \textbf{hôrtet}, wie\\ 
 & Gawan d\textit{â} geschæhe,\\ 
20 & deste gerner ichs \textbf{iu} \textbf{verjæhe}.\\ 
 & ich sage \textbf{iu}, als ichz \textit{hân} vernomen:\\ 
 & dô er was vür die porte komen,\\ 
 & er vant den k\textit{r}â\textit{m}ære\\ 
 & \textbf{und} des krâme niht lære.\\ 
25 & dâ lac in \textit{vei}le,\\ 
 & daz ichs \textit{wære} der geile,\\ 
 & het ich alsô rîche hab.\\ 
 & Gawan vor im erbeizte ab.\\ 
 & sô rîchen market er nie gesach,\\ 
30 & als im zuo sehen d\textit{â} geschach.\\ 
\end{tabular}
\scriptsize
\line(1,0){75} \newline
m n o \newline
\line(1,0){75} \newline
\textbf{7} \textit{Illustration mit Überschrift:} Also gawan vff sasz vnd hin weg reit von der jungfrouwen in den strit n   $\cdot$ \textit{Überschrift:} Also gawan vff sas vnd enweg reit von der schonen jvngfrouwen in den stritt m   $\cdot$ \textit{Großinitiale} n   $\cdot$ \textit{Initiale} m  \newline
\line(1,0){75} \newline
\textbf{2} disen] Diser m n o \textbf{3} ninder] nẏergent n nider o  $\cdot$ iu] \textit{om.} n \textbf{4} hab] han m n [hab]: heb o \textbf{6} hebet] hebe m \textbf{7} \textit{Die Verse 562.7-564.18 fehlen} o  \textbf{8} an] al m \textbf{9} dâ] do m n \textbf{11} wirt] \textit{om.} m \textbf{13} sus] flus m \textbf{14} soldes] Geltes n  $\cdot$ betrâget] betrogent n \textbf{16} jâmers] jamer m (n) \textbf{18} nû] mir m \textbf{19} dâ] do m n \textbf{21} hân] \textit{om.} m \textbf{23} krâmære] kamerere m \textbf{25} veile] craele m \textbf{26} wære] \textit{om.} m \textbf{28} erbeizte] erbeisset n \textbf{30} dâ] das m do n \newline
\end{minipage}
\end{table}
\newpage
\begin{table}[ht]
\begin{minipage}[t]{0.5\linewidth}
\small
\begin{center}*G
\end{center}
\begin{tabular}{rl}
 & \begin{large}G\end{large}edenket, hêrre, ob ir sît wert,\\ 
 & disen schilt unde iuwer swert\\ 
 & lâzet \textbf{niener} von iu komen.\\ 
 & sô ir wænet, daz ende habe genomen\\ 
5 & iuwer kumber grôzlîch,\\ 
 & alrêrst \textbf{danne} \textbf{strîte ist er gelîch}."\\ 
 & dô Gawan ûf sîn ors gesaz,\\ 
 & diu maget wart an vröuden laz.\\ 
 & alle, die dâ wâren, klageten;\\ 
10 & wênic si des \textbf{verdageten}.\\ 
 & er sprach ze dem wirte: "gan mirs got,\\ 
 & iuwer getriuwelîch urbot,\\ 
 & daz ir mîn sus pflâget,\\ 
 & \textbf{geltes} mich niht betrâget."\\ 
15 & urloup er ze der meide nam,\\ 
 & die grôzes jâmers wol gezam.\\ 
 & er reit hin, si klageten hie.\\ 
 & ob ir nû gerne \textbf{hœret}, wie\\ 
 & Gawane dâ geschæhe,\\ 
20 & deste gerner ich es \textbf{iu} \textbf{jæhe}.\\ 
 & ich sag, als ichz hân vernomen:\\ 
 & dô er was vür die porten komen,\\ 
 & er vant de\textit{n} krâmære\\ 
 & \textbf{unde} des krâme niht lære.\\ 
25 & dâ lac inne veile,\\ 
 & daz ich es wære der geile,\\ 
 & het ich alsô rîche habe.\\ 
 & Gawan vor im erbeizte abe.\\ 
 & sô rîchen market er nie gesac\textit{h},\\ 
30 & als im ze sehenne \textbf{al} dâ geschach.\\ 
\end{tabular}
\scriptsize
\line(1,0){75} \newline
G I L M Z \newline
\line(1,0){75} \newline
\textbf{1} \textit{Initiale} G L  \textbf{7} \textit{Initiale} Z  \textbf{11} \textit{Initiale} I  \newline
\line(1,0){75} \newline
\textbf{1} Gedenket] Bedenkit M  $\cdot$ ob] daz I \textbf{3} niener] Nummer M \textbf{4} ir] \textit{om.} I  $\cdot$ daz] daz ez Z \textbf{5} \textit{Versdoppelung 562.5 nach 562.6} G  \textbf{6} alrest strit ist er dann gelich I  $\cdot$ Alrerst danne ist er strite gelich L  $\cdot$ Danne alrest ist er stritlich M  $\cdot$ Alrerst ist er danne strite gelich Z \textbf{7} dô] Da M \textbf{9} wâren] warn in I \textbf{10} verdageten] verzagten L \textbf{11} gan] vnd Gan I  $\cdot$ mirs] mir sin I \textbf{12} urbot] irbot M \textbf{13} mîn sus] suͯsz mýn L \textbf{14} mich] ir mich M \textbf{16} wol] da Z \textbf{20} gerner] liebir M  $\cdot$ ich es iu] ich ev des I ich uch M (Z)  $\cdot$ jæhe] [gihe]: gahe G veriehe I (L) (Z) vorgebe M \textbf{22} dô] Da M Z  $\cdot$ er] er nu I \textbf{23} den] dem G Z der M  $\cdot$ krâmære] Gramere I \textbf{25} dâ] Al M \textbf{28} Gawan erbaizte vor im ab I (Z) \textbf{29} gesach] gesac G \textbf{30} al] \textit{om.} Z \newline
\end{minipage}
\hspace{0.5cm}
\begin{minipage}[t]{0.5\linewidth}
\small
\begin{center}*T
\end{center}
\begin{tabular}{rl}
 & gedenket, hêrre, ob ir sît wert,\\ 
 & disen schilt unde iuwer swert\\ 
 & lâzet \textbf{niemer} von iu komen.\\ 
 & sô ir wænet, da\textit{z ende} habe genomen\\ 
5 & iuwer kumber grôzlîch,\\ 
 & alrêst \textbf{dem strîte ist ez gelîch}."\\ 
 & \textit{\begin{large}D\end{large}}ô Gawan ûf sîn ors gesaz,\\ 
 & diu maget wart an vröuden laz.\\ 
 & alle, die dâ wâren, klageten;\\ 
10 & wênic si des \textbf{gedageten}.\\ 
 & er sprach zem wirte: "gan mirs got,\\ 
 & iuwer getriuwelîch urbot,\\ 
 & daz ir mîn sus pflâget,\\ 
 & \textbf{geltes} mich niht betrâget."\\ 
15 & urloup er zer megede nam,\\ 
 & di\textit{e} grôzes jâmers wol gezam.\\ 
 & er reit hin, si klageten hie.\\ 
 & ob ir nû gerne \textbf{hœret}, wie\\ 
 & Gawane dâ geschæhe,\\ 
20 & deste gerner ichs \textbf{verjæhe}.\\ 
 & Ich sage\textbf{z} \textbf{iu}, als ichz hân vernomen:\\ 
 & dô er was vür die porten komen,\\ 
 & er vant den krâmære,\\ 
 & des krâm \textbf{was} niht lære.\\ 
25 & dâ lac inne veile,\\ 
 & daz ichs wære der geile,\\ 
 & het ich alsô rîche habe.\\ 
 & Gawan vor im erbeizet abe.\\ 
 & sô rîchen market er nie gesach,\\ 
30 & als im ze sehenne \textbf{al}dâ geschach.\\ 
\end{tabular}
\scriptsize
\line(1,0){75} \newline
T U V W Q R Fr25 Fr39 Fr40 \newline
\line(1,0){75} \newline
\textbf{1} \textit{Initiale} Fr25 Fr39 Fr40   $\cdot$ \textit{Capitulumzeichen} R  \textbf{7} \textit{Überschrift:} Hie reit her gawan auff die burg castel marfeile vnd erstrait do die auenteúre W  · Initiale T V W  \textbf{21} \textit{Initiale} R   $\cdot$ \textit{Majuskel} T  \newline
\line(1,0){75} \newline
\textbf{1} \textit{Die Verse 553.1-599.30 fehlen} U   $\cdot$ gedenket] Dewcht euch Q Denket Fr40  $\cdot$ ob ir sît] sind ir R ob sit Fr39  $\cdot$ wert] wer Fr25 \textbf{2} unde] noch R \textbf{3} niemer] niender W (Q) (R) (Fr25) Fr39 (Fr40) \textbf{4} daz ende habe] da habe T ez habe ende V \textbf{6} Danne alrest so [he*]: hebet er sich V  $\cdot$ ez] \textit{om.} W \textbf{7} Dô] ÷o T  $\cdot$ Gawan] Gawin R \textbf{8} wart] was Q Fr25 \textbf{9} dâ] do W Fr39 \textit{om.} Q  $\cdot$ wâren] frawen Q \textbf{10} gedageten] verdageten V (W) (Q) (R) (Fr25) (Fr39) \textbf{11} Gan] gam Fr25 \textbf{12} urbot] vrlop Q \textbf{13} sus pflâget] als pfleget Q \textbf{16} die] div T \textbf{18} gerne hœret] gerne hortent V [hor]: hoeret gerne Fr25 \textbf{19} Gawane] Gawan W Q Fr25 Fr39  $\cdot$ dâ] do V W Q R Fr39 \textbf{20} gerner] gerne R  $\cdot$ ichs] ich v́ch dez V ich eúchs W (R) ichs euch Q ivchs iv Fr25 (Fr39)  $\cdot$ verjæhe] iehe W Q (Fr25) Fr39 \textbf{21} Ich sagez] [J*]: Jch sage V Ich sag W (Q) (R) (Fr25) (Fr39)  $\cdot$ ichz] ich V W Fr39 \textbf{22} dô] Das W \textbf{24} des] Vnde dez V (R) (Fr25) (Fr39) Vnd auch des W Vnd den Q  $\cdot$ was] \textit{om.} V W Q R Fr25 Fr39 \textbf{26} ichs] ich W Q ich des Fr25 \textbf{27} ich] \textit{om.} R \textbf{28} Gawin erbeiczet vor Jm abe R \textbf{29} \textit{Versfolge 562.30-29} W   $\cdot$ market] marschalc Fr25 \textbf{30} aldâ] do V \newline
\end{minipage}
\end{table}
\end{document}
