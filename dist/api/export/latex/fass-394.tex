\documentclass[8pt,a4paper,notitlepage]{article}
\usepackage{fullpage}
\usepackage{ulem}
\usepackage{xltxtra}
\usepackage{datetime}
\renewcommand{\dateseparator}{.}
\dmyyyydate
\usepackage{fancyhdr}
\usepackage{ifthen}
\pagestyle{fancy}
\fancyhf{}
\renewcommand{\headrulewidth}{0pt}
\fancyfoot[L]{\ifthenelse{\value{page}=1}{\today, \currenttime{} Uhr}{}}
\begin{document}
\begin{table}[ht]
\begin{minipage}[t]{0.5\linewidth}
\small
\begin{center}*D
\end{center}
\begin{tabular}{rl}
\textbf{394} & \begin{large}G\end{large}awan bî Scherulese enbôt\\ 
 & sîner vrouwen Obilot,\\ 
 & daz er si gerne wolde sehen\\ 
 & unt ouch mit wârheite jehen\\ 
5 & sînes lîbes undertân\\ 
 & unt \textbf{er wolt} \textbf{ouch} ir urloup hân.\\ 
 & "\textbf{unt} sagt, ich lâze ir den künec hie.\\ 
 & bittet si sich bedenken, wie\\ 
 & \textbf{daz} sin alsô behalte,\\ 
10 & daz prîs ir vuore walte."\\ 
 & \textbf{Dise rede hôrte} Meljanz.\\ 
 & \textbf{der} sprach: "Obilot wirt kranz\\ 
 & aller wîplîchen güete.\\ 
 & \textbf{daz} senftet mir mîn gemüete,\\ 
15 & \textbf{ob} ich ir sicherheit \textbf{muoz} geben,\\ 
 & \textbf{daz ich} ir vrides hie sol leben."\\ 
 & "Ir sult si dâ vür hân erkant,\\ 
 & iuch \textbf{en}vienc hie niemen wan ir hant",\\ 
 & \textbf{sus} sprach der werde Gawan.\\ 
20 & "mînen prîs sol si al eine hân."\\ 
 & Scherules kom vür geriten.\\ 
 & \textbf{nû}\textbf{ne} was ze hove niht vermiten,\\ 
 & dâne wære magt, man \textbf{unt} wîp,\\ 
 & in solher wæte \textbf{ieslîches} lîp,\\ 
25 & daz man kranker, armer wât\\ 
 & des tages \textbf{dâ hete} lîhten rât.\\ 
 & Mit Melyanze ze hove reit\\ 
 & al die dort ûze \textbf{ir} sicherheit\\ 
 & ze pfande heten lâzen.\\ 
30 & dort elliu vieriu sâzen:\\ 
\end{tabular}
\scriptsize
\line(1,0){75} \newline
D \newline
\line(1,0){75} \newline
\textbf{1} \textit{Initiale} D  \textbf{11} \textit{Majuskel} D  \textbf{17} \textit{Majuskel} D  \textbf{27} \textit{Majuskel} D  \newline
\line(1,0){75} \newline
\textbf{1} Scherulese] Scervlese D \textbf{11} Meljanz] Melianz D \textbf{21} Scherules] Scervres D \newline
\end{minipage}
\hspace{0.5cm}
\begin{minipage}[t]{0.5\linewidth}
\small
\begin{center}*m
\end{center}
\begin{tabular}{rl}
 & Gawan bî Scherules enbôt\\ 
 & sîner vrouwen Obilot,\\ 
 & daz er si gerne wolte sehen\\ 
 & und ouch mit wârheit jehen\\ 
5 & \textbf{ir} sînes lîbes undertân\\ 
 & und \textbf{er wolte} \textbf{ouch} ir urloup hân.\\ 
 & "\textbf{und} sagt, ich lâze ir den künic hie.\\ 
 & bittet si sich bedenken, wie\\ 
 & \textbf{dâ} si in alsô behalte,\\ 
10 & daz prîs ir vuore walte."\\ 
 & \textbf{dise rede hôrt} Mel\textit{i}anz.\\ 
 & \textbf{er} sprach: "Obilot wirt kranz\\ 
 & aller wîplîcher güete.\\ 
 & \textbf{daz} senftet mir mîn gemüete,\\ 
15 & \textbf{ob} ich ir sicherheit \textbf{muoz} geben,\\ 
 & \textbf{daz ich} ir vrides hie sol leben."\\ 
 & "ir sult si dâ vür hân erkant,\\ 
 & \dag ich\dag  \textbf{en}vienc hie niemen wanne ir hant",\\ 
 & sprach der werde Gawan.\\ 
20 & "mînen prîs sol si aleine hân."\\ 
 & Scherules kam vür geriten.\\ 
 & \textbf{nû} \textbf{en}was ze hove niht vermiten,\\ 
 & dâ enwære magt, man \textbf{und} wîp,\\ 
 & in solher wæte \textbf{ieglîches} \textit{l}îp,\\ 
25 & daz man \textbf{d\textit{â}} kranker, armer wât\\ 
 & des tages \textbf{hete dâ} lîhten rât.\\ 
 & mit Mel\textit{i}anz ze hove reit\\ 
 & al die dort ûze \textbf{ir} sicherheit\\ 
 & ze pfande heten lâzen.\\ 
30 & dor\textit{t} al vieriu sâzen:\\ 
\end{tabular}
\scriptsize
\line(1,0){75} \newline
m n o \newline
\line(1,0){75} \newline
\newline
\line(1,0){75} \newline
\textbf{1} Scherules] scerules m n Sterules o \textbf{2} Obilot] ibilot n ybelot o \textbf{3} \textit{Verse 394.3-4 kontrahiert zu:} Das er sie gerne wolte jehen o  \textbf{6} ir] iren n (o) \textbf{7} sagt] sagete n \textbf{9} dâ si] Das n Das si o \textbf{11} Melianz] melancz m meliantz n meliancz o \textbf{12} Obilot] ibilot n ẏbelot o  $\cdot$ wirt] wart o \textbf{13} wîplîcher] wiplichen n (o) \textbf{14} senftet] senfftert n o  $\cdot$ mir mîn gemüete] min muͯte n \textbf{16} vrides] friden n friden ir o \textbf{17} vür] vor vor o \textbf{18} envienc] ving n (o)  $\cdot$ wanne] denne n da o \textbf{19} sprach] Susz sprach n (o) \textbf{21} Scherules] Scerules m Sterules n Steruͯles o \textbf{22} enwas] was n o \textbf{23} dâ enwære] Do were n o \textbf{24} ieglîches] ẏegelicher n iglichen o  $\cdot$ lîp] wip m \textbf{25} dâ] do m n o  $\cdot$ kranker armer] [rich]: krancker armuͦt o \textbf{26} hete dâ] do hette n (o) \textbf{27} Melianz] meleanz m meliantz n meliancz o \textbf{28} al die] Alhie n o  $\cdot$ sicherheit] sicherherheit o \textbf{29} lâzen] gelossen n (o) \textbf{30} dort] Dor m \newline
\end{minipage}
\end{table}
\newpage
\begin{table}[ht]
\begin{minipage}[t]{0.5\linewidth}
\small
\begin{center}*G
\end{center}
\begin{tabular}{rl}
 & Gawan bî Tscherules enbôt\\ 
 & sîner vrouwen Obilot,\\ 
 & daz er si gerne wolt sehen\\ 
 & unde ouch mit wârheite jehen\\ 
5 & sînes lîbes undertân\\ 
 & "un\textit{d} \textbf{\textit{ich} welle} ir urloup hân.\\ 
 & saget, ich lâze ir den künic hie.\\ 
 & bit si sich bedenken, wie\\ 
 & si in alsô behalte,\\ 
10 & daz prîs ir vuore walte."\\ 
 & \textbf{des antwurte} Melianz;\\ 
 & \textbf{er} \textit{sprach}: "Obilote wirdet kranz\\ 
 & aller wîplîchen güete.\\ 
 & \textbf{daz} senftet mir mîn gemüete,\\ 
15 & \textbf{daz} ich ir sicherheit \textbf{sol} geben\\ 
 & \textbf{unde ouch} ir vrides hie sol leben."\\ 
 & "ir sult si dâ vür hân erkant,\\ 
 & iuch vienc hie niemen wan ir hant",\\ 
 & \textbf{sô} sprach der werde Gawan.\\ 
20 & "mînen brîs sol si al eine hân."\\ 
 & Tscherules kom vür geriten.\\ 
 & \textbf{nû}\textbf{ne} was ze hove niht vermiten,\\ 
 & dâne wære maget, man \textbf{unde} wîp,\\ 
 & in solher wæte \textbf{ieslîch} lîp,\\ 
25 & daz man \textbf{dâ} kranker, armer wât\\ 
 & des tages \textbf{hete} lîhten rât.\\ 
 & mit Melianze ze hove reit\\ 
 & alle die dort ûze \textbf{ir} sicherheit\\ 
 & ze pfande heten lâzen.\\ 
30 & dort elliu vieriu sâzen:\\ 
\end{tabular}
\scriptsize
\line(1,0){75} \newline
G I O L M Q R Z Fr28 \newline
\line(1,0){75} \newline
\textbf{1} \textit{Initiale} O L Z  \textbf{3} \textit{Initiale} I   $\cdot$ \textit{Capitulumzeichen} R  \textbf{17} \textit{Initiale} I  \newline
\line(1,0){75} \newline
\textbf{1} \textit{Die Verse 370.13-412.12 fehlen} Q   $\cdot$ Gawan] ÷Awan O  $\cdot$ Tscherules] Schurles I tschervles O Tsheruͯles L scerules M scherules R \textbf{2} vrouwen] frowelin Fr28  $\cdot$ Obilot] obẏlot Fr28 \textbf{5} undertân] [vndertone*]: vndertonen R \textbf{6} und ich welle] vnder welle G Vnde ich wolde M Vnd er wolte R vnd ich wil Fr28  $\cdot$ ir] ovch ir O (L) (M) (R) (Z) (Fr28) \textbf{7} saget] sagt ir I (L) \textbf{8} bit] Biz L  $\cdot$ sich bedenken] gedenchen I sich bedenche Fr28 \textbf{9} si] daz si I (O) (M) (R) (Z) (Fr28)  $\cdot$ behalte] halte L \textbf{10} prîs] prises M Fr28  $\cdot$ walte] behalte Z \textbf{11} Melianz] Melyanz O Meliancz R melianze Z \textbf{12} sprach] \textit{om.} G  $\cdot$ Obilote] obilot I (L) M (Z) Obylot O oblet R obẏlot Fr28  $\cdot$ wirdet] wirt ein Fr28 \textbf{13} aller] Alle Z  $\cdot$ wîplîchen] wipplicher L (R) (Fr28) \textbf{14} senftet] senfftte sy R \textbf{15} ir] \textit{om.} M R  $\cdot$ sol] mvͦz O (L) (M) (R) (Z) (Fr28)  $\cdot$ geben] Jechen R \textbf{16} sol] mvͦz O (L) (R) \textbf{18} vienc] envie L (M) (R) (Z) (Fr28)  $\cdot$ hie] da O  $\cdot$ niemen] mymant M  $\cdot$ wan] dan Z Fr28 \textbf{19} sô] Do L R Da M \textbf{20} mînen] \textit{om.} I  $\cdot$ al] \textit{om.} I M Fr28 \textbf{21} Tscherules] Schurles I Tschervles O Tshervles L Scerules M Scherules R \textbf{22} nûne] Nv O \textbf{23} dâne] Da O R  $\cdot$ maget man] man maget I (Fr28)  $\cdot$ unde] noch L \textbf{24} in] vnde O  $\cdot$ solher wæte] schoner varwe I  $\cdot$ ieslîch] ýesliches L \textbf{25} dâ] der I  $\cdot$ kranker armer] cranchen armen I armer kranker R \textbf{26} des] Der Z  $\cdot$ hete] het da O da het Z  $\cdot$ lîhten] liehten R \textbf{27} Melianze] melianz I (L) Melyanze O Meliancze R meliantze Z \textbf{28} dort] \textit{om.} M dar Fr28 \textbf{29} heten] hat R  $\cdot$ lâzen] gilazin M (R) (Fr28) \newline
\end{minipage}
\hspace{0.5cm}
\begin{minipage}[t]{0.5\linewidth}
\small
\begin{center}*T
\end{center}
\begin{tabular}{rl}
 & \begin{large}G\end{large}awan bî Tscherules enbôt\\ 
 & sîner vrouwen Obylot,\\ 
 & daz er si gerne wolte sehen\\ 
 & unde ouch mit wârheite jehen\\ 
5 & sînes lîbes undertân\\ 
 & "unde \textbf{ich welle} \textbf{ouch} ir urloup hân.\\ 
 & saget, ich lâze ir den künec hie.\\ 
 & bit si sich bedenken, wie\\ 
 & \textbf{daz} sin alsô behalte,\\ 
10 & daz prîs ir vuore walte."\\ 
 & \textbf{Des antwurtim} Melyanz;\\ 
 & \textbf{er} sprach: "Obylot wirt \textbf{ein} kranz\\ 
 & aller wîplîchen güete.\\ 
 & \textbf{di\textit{u}} senftet mir mîn gemüete,\\ 
15 & \textbf{ob} ich ir sicherheit \textbf{muoz} geben\\ 
 & \textbf{unde ouch} ir vrides hie sol leben."\\ 
 & "Ir sult si dâ vür hân erkant,\\ 
 & iuch vienc hie nieman \textit{wa}n ir hant",\\ 
 & \textbf{sô} sprach der werde Gawan.\\ 
20 & "mînen prîs sol si aleine hân."\\ 
 & \begin{large}T\end{large}scherules kom vür geriten.\\ 
 & \textbf{dô} was ze hove niht vermiten,\\ 
 & dâne wære maget, man \textbf{noch} wîp,\\ 
 & in solher wæte \textbf{iegeslîches} lîp,\\ 
25 & daz man \textbf{dâ} kranker, armer wât\\ 
 & des tages \textbf{hete} lîhten rât.\\ 
 & Mit Melyanze ze hove reit\\ 
 & alle die dort ûze sicherheit\\ 
 & ze pfande heten gelâzen.\\ 
30 & dort alliu vieriu sâzen:\\ 
\end{tabular}
\scriptsize
\line(1,0){75} \newline
T V W \newline
\line(1,0){75} \newline
\textbf{1} \textit{Initiale} T W  \textbf{11} \textit{Majuskel} T  \textbf{17} \textit{Majuskel} T  \textbf{21} \textit{Initiale} T  \textbf{27} \textit{Majuskel} T  \newline
\line(1,0){75} \newline
\textbf{1} Gawan] GAwaan W  $\cdot$ Tscherules] Tscervles T scherules V W \textbf{2} Obylot] Obylôt T obilot V \textbf{5} sînes lîbes] Jr sines libes V Seines lobes W \textbf{6} ich welle] er wolte W  $\cdot$ ir] irn V \textbf{7} saget] Sagent ir V \textbf{9} sin] sein W \textbf{11} Des antwurtim] Dise rede horte V Des antwurte W  $\cdot$ Melyanz] melianz V meliantz W \textbf{12} Obylot] obilot V  $\cdot$ ein] \textit{om.} W \textbf{13} wîplîchen] wiplicher V (W) \textbf{14} diu] die T Daz V (W)  $\cdot$ senftet] senftert V  $\cdot$ gemüete] genuͤte W \textbf{15} ob] Das W \textbf{16} sol] muͦß W \textbf{18} iuch] iv T Eúcb W  $\cdot$ vienc] envieng V  $\cdot$ wan] [vor]: von T dann W \textbf{19} sô] Sus V \textbf{21} Tscherules] Scervles T Scherules V Tscherules W \textbf{22} dô was] Nv enwas V Nun was W \textbf{23} dâne wære] Do enwere V W  $\cdot$ noch] vnde V oder W \textbf{25} dâ] do V W \textbf{26} hete] do hette V hetten W \textbf{27} Melyanze] melianze V melianz W \textbf{28} ûze] vsse ir V aus mit W \textbf{29} gelâzen] lassen W \newline
\end{minipage}
\end{table}
\end{document}
