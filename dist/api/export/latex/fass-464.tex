\documentclass[8pt,a4paper,notitlepage]{article}
\usepackage{fullpage}
\usepackage{ulem}
\usepackage{xltxtra}
\usepackage{datetime}
\renewcommand{\dateseparator}{.}
\dmyyyydate
\usepackage{fancyhdr}
\usepackage{ifthen}
\pagestyle{fancy}
\fancyhf{}
\renewcommand{\headrulewidth}{0pt}
\fancyfoot[L]{\ifthenelse{\value{page}=1}{\today, \currenttime{} Uhr}{}}
\begin{document}
\begin{table}[ht]
\begin{minipage}[t]{0.5\linewidth}
\small
\begin{center}*D
\end{center}
\begin{tabular}{rl}
\textbf{464} & \textit{\begin{large}P\end{large}}arzival \textbf{hin} zim dô sprach:\\ 
 & "hêrre, ich wæne, daz \textbf{ie} geschach.\\ 
 & von wem was der man \textbf{erborn},\\ 
 & von dem sîn ane hât verlorn\\ 
5 & den magetuom, als ir mir sagt?\\ 
 & daz moht ir gerne hân verdagt."\\ 
 & Der wirt sprach aber \textbf{wider} zim:\\ 
 & "von dem zwîvel ich iuch nim.\\ 
 & sag\textbf{ich} niht \textbf{wâr} die wârheit,\\ 
10 & sô lât iu sîn mîn \textbf{triegen} leit.\\ 
 & diu erde Adames muoter was.\\ 
 & von erden vruht Adam genas.\\ 
 & dannoch was diu erde \textbf{ein} magt.\\ 
 & noch hân ich iu niht gesagt,\\ 
15 & wer ir den magetuom benam.\\ 
 & Kaines vater was Adam.\\ 
 & \textbf{er} sluoc Abeln umbe krankez guot.\\ 
 & dô ûf die reinen erden daz bluot\\ 
 & viel, ir magetuom was vervaren.\\ 
20 & den nam ir Adames baren.\\ 
 & dô huop sich êrst \textbf{der} menschen nît.\\ 
 & alsô wert er immer sît.\\ 
 & in der werlt \textbf{doch} \textbf{niht sô reines} ist\\ 
 & sô diu magt âne valschen list.\\ 
25 & nû prüevet, wie reine die meide sint.\\ 
 & got \textbf{was selbe} der meide kint.\\ 
 & von meiden sint zwei mensche komen.\\ 
 & Got selbe antlütze hât genomen\\ 
 & nâch der êrsten meide vruht.\\ 
30 & daz was sîner hôhen art ein zuht.\\ 
\end{tabular}
\scriptsize
\line(1,0){75} \newline
D \newline
\line(1,0){75} \newline
\textbf{1} \textit{Initiale} D  \textbf{7} \textit{Majuskel} D  \textbf{28} \textit{Majuskel} D  \newline
\line(1,0){75} \newline
\textbf{1} Parzival] ÷arcifal D \newline
\end{minipage}
\hspace{0.5cm}
\begin{minipage}[t]{0.5\linewidth}
\small
\begin{center}*m
\end{center}
\begin{tabular}{rl}
 & Parcifal \textbf{hin} zuo im dô sprach:\\ 
 & "hêrre, ich wæne, daz \textbf{nie} geschach.\\ 
 & von wem was der man \textbf{erborn},\\ 
 & von dem sîn ane het verlorn\\ 
5 & den magetuom, als \dag er\dag  mir \dag sagete\dag ?\\ 
 & daz mohtet ir gerne hân \dag verdagete\dag ."\\ 
 & der wirt sprach aber zuo im:\\ 
 & "von dem zwîvel ich iuch nim.\\ 
 & sage  \textbf{iu} niht die wârheit,\\ 
10 & sô \textit{lât} iu sîn mîn \textbf{kriegen} leit.\\ 
 & diu \textit{er}de Adames muoter was.\\ 
 & von erden vruht Adam genas.\\ 
 & dannoch was diu erde maget.\\ 
 & noch hân ich iu niht gesaget,\\ 
15 & wer ir den magetuom benam.\\ 
 & Kayn\textit{e}s vater was Adam.\\ 
 & \dag den\dag  sluoc Abeln umb krankez guot.\\ 
 & dô ûf die reinen erden daz bluot\\ 
 & viel, ir magetuom was vervarn.\\ 
20 & den nam ir Adames barn.\\ 
 & dô huop sich êrst \textbf{der} menschen nît.\\ 
 & alsô wert er iemer sît.\\ 
 & in der werlt \textbf{doch} \textbf{niht sô reines} ist\\ 
 & sô diu maget âne valschen list.\\ 
25 & nû brüefet, wie reine die megde sint.\\ 
 & got \textbf{was selbe} der megde kint.\\ 
 & von megden sint zwei menschen komen.\\ 
 & got selbe antlitze het genomen\\ 
 & nâch der êrsten megde vruht.\\ 
30 & daz was sîner hôhe\textit{n} art ein zuht.\\ 
\end{tabular}
\scriptsize
\line(1,0){75} \newline
m n o \newline
\line(1,0){75} \newline
\textbf{1} \textit{Überschrift:} Also parcifal gar frintlichen mit sime wurt rette vnd ẏme der wurt antwurt n   $\cdot$ \textit{Initiale} n  \newline
\line(1,0){75} \newline
\textbf{1} \textit{Die Verse 462.25-464.23 fehlen} o   $\cdot$ dô] \textit{om.} n \textbf{3} erborn] erkorn n \textbf{4} het] hette n \textbf{5} \textit{Versfolge 464.6-5} n  \textbf{6} mohtet] mohtten m moͯchten n  $\cdot$ gerne] gerner n \textbf{7} aber] aber wider n \textbf{10} lât] \textit{om.} m \textbf{11} erde] rede m  $\cdot$ Adames] adams m o \textbf{15} ir] [ich]: ir n \textbf{16} Kaynes] Kaynis m Kaẏnis n \textbf{17} Abeln] abelen n \textbf{18} erden] erde n \textbf{20} Adames] adams m o \textbf{22} er] also er n \textbf{26} got] Go o \textbf{27} megden sint zwei] megde sint zwein o \textbf{28} antlitze] anczlit o \textbf{30} hôhen] hoher m \newline
\end{minipage}
\end{table}
\newpage
\begin{table}[ht]
\begin{minipage}[t]{0.5\linewidth}
\small
\begin{center}*G
\end{center}
\begin{tabular}{rl}
 & \begin{large}P\end{large}arzival \textbf{hin} ze im dô sprach:\\ 
 & "hêrre, ich wæne, daz \textbf{ie} geschach.\\ 
 & von wem was der man \textbf{geborn},\\ 
 & von dem sîn an hât verlorn\\ 
5 & de\textit{n} magetuom, als ir mir saget?\\ 
 & daz moht ir gerne hân verdaget."\\ 
 & der wirt sprach aber \textbf{wider} zim:\\ 
 & "von dem zwîvel ich iuch nim.\\ 
 & sage \textbf{ich} niht \textbf{wâr} die wârheit,\\ 
10 & sô lât iu sîn mîn \textbf{triegen} leit.\\ 
 & diu erde Adames muoter was.\\ 
 & von erden vruht Adam genas.\\ 
 & dannoch was diu erde \textbf{ein} maget.\\ 
 & noch hân ich iu niht \textbf{gar} gesaget,\\ 
15 & wer ir de\textit{n} magetuom benam.\\ 
 & Caines vater was Adam.\\ 
 & \textbf{der} sluoc Abel umbe krankez guot.\\ 
 & dô ûf die reinen erde daz bluot\\ 
 & viel, ir magetuom was vervarn.\\ 
20 & den nam ir Adames barn.\\ 
 & dô huop sich êrst \textbf{der} menschen nît.\\ 
 & alsô wert er immer sît.\\ 
 & in der werlt \textbf{noch} \textbf{niht sô reines} ist\\ 
 & sô diu maget ân valschen list.\\ 
25 & nû prüevet, wie reine die meide sint.\\ 
 & got \textbf{was selbe} der meide kint.\\ 
 & von meiden sint zwei mensch komen.\\ 
 & got selbe antlütze hât genomen\\ 
 & nâch der êrsten meide vruht.\\ 
30 & daz was sîner hôhen art ein zuht.\\ 
\end{tabular}
\scriptsize
\line(1,0){75} \newline
G I O L M Z Fr18 Fr22 Fr61 \newline
\line(1,0){75} \newline
\textbf{1} \textit{Initiale} G I O L Z Fr61  \textbf{11} \textit{Initiale} L  \textbf{15} \textit{Initiale} I  \newline
\line(1,0){75} \newline
\textbf{1} Parzival] Parzifal I L M ÷Arcifal O Parcifal Z Fr61  $\cdot$ hin] \textit{om.} L  $\cdot$ dô] da M \textbf{2} ich] \textit{om.} O  $\cdot$ daz] diz I (Fr61)  $\cdot$ ie] [êe]: îe Fr61 \textbf{3} was] ward Fr61 \textbf{4} dem] den M  $\cdot$ hât] ward Fr61 \textbf{5} den] Dem G  $\cdot$ saget] habt gesagt I \textbf{6} gerne] gerner I (Fr61) \textbf{7} sprach aber wider] do aber sprach I sprach aber Fr61 \textbf{9} ich] ich ev I (O) (L) (M)  $\cdot$ wâr] \textit{om.} I O L M wan Fr61 \textbf{10} iu] \textit{om.} L \textbf{11} diu erde] EErde L  $\cdot$ Adames] adamesz L \textbf{12} erden vruht] erde I erde fruͯht L \textbf{14} noch] Nv L  $\cdot$ hân] \textit{om.} I enhan Fr61  $\cdot$ niht gar] niht O L Z (Fr61) o\textit{m. } M  $\cdot$ gesaget] gesagte Fr61 \textbf{15} ir] in L  $\cdot$ den] dem G \textbf{16} Caines] Cains G kains I (Z) Cayn O Kayn L Kain M Kaẏen Fr61 \textbf{17} Abel] abelen I (Fr61) Abeln O (L) (M) (Z)  $\cdot$ umbe] duͯrch L  $\cdot$ krankez] armes M \textbf{18} Do er vf die erde daz reine bluͯt L  $\cdot$ dô] Da M Z  $\cdot$ erde daz] erden I erdin das M \textbf{20} den] Deme M  $\cdot$ ir] \textit{om.} Fr61  $\cdot$ Adames] adams I :::es Fr18 \textbf{21} dô] Da M Z  $\cdot$ der] des I (O) (L) o\textit{m. } Fr61  $\cdot$ menschen] mensche M  $\cdot$ nît] not Z \textbf{22} wert er] vert ir M  $\cdot$ immer] immer mere I  $\cdot$ sît] sit der tot Z \textbf{23} noch] doch O L Z Fr18 Fr22 \textit{om.} M Fr61  $\cdot$ sô] \textit{om.} I  $\cdot$ reines] rainers I \textbf{24} valschen] valsche M \textbf{25} prüevet] pruͯfe L \textbf{26} selbe der] selbin der M einer Fr61 \textbf{27} mensch] [mennich]: mennisch G menshen I (Z) (Fr61) \textbf{28} selbe] selbern M \textbf{29} êrsten] erde I \textbf{30} hôhen] hachen Fr61 \newline
\end{minipage}
\hspace{0.5cm}
\begin{minipage}[t]{0.5\linewidth}
\small
\begin{center}*T
\end{center}
\begin{tabular}{rl}
 & Parcifal zi\textit{m} dô sprach:\\ 
 & "hêrre, ich wæne, daz \textbf{ie} geschach.\\ 
 & von wem was der man \textbf{geborn},\\ 
 & von dem sîn ane hât verlorn\\ 
5 & den magetuom, alsir mir saget?\\ 
 & daz mohtir gerne hân verdaget."\\ 
 & Der wirt sprach aber \textbf{wider} zim:\\ 
 & "von dem zwîvele ich iuch nim.\\ 
 & sag\textbf{ich} \textbf{iu} niht die wârheit,\\ 
10 & sô lât iu sîn mîn \textbf{triegen} leit.\\ 
 & Diu erde Adames muoter was.\\ 
 & von erden vruht Adam genas.\\ 
 & dannoch was diu erde \textbf{ein} maget.\\ 
 & noch hân ich iu niht gesaget,\\ 
15 & wer ir den magetuom benam.\\ 
 & Caynes vater was Adam.\\ 
 & \textbf{der} sluoc Abeln umbe krankez guot.\\ 
 & dô ûf die reinen erde daz bluot\\ 
 & viel, ir magetuom was vervarn.\\ 
20 & den nam ir Adames barn.\\ 
 & dô huop sich êrst \textbf{des} menschen nît.\\ 
 & alse wert er iemer sît.\\ 
 & in der werlte \textbf{doch} \textbf{sô reines niht} ist\\ 
 & sô diu maget âne valschen list.\\ 
25 & nû prüevet, wie reine die megde sint.\\ 
 & got \textbf{selbe was} der megde kint.\\ 
 & von megden sint zwei menschen komen.\\ 
 & got selbe antlitze hât genomen\\ 
 & nâch der êrsten megde vruht.\\ 
30 & daz was sîner hôhen art ein zuht.\\ 
\end{tabular}
\scriptsize
\line(1,0){75} \newline
T U V W Q R Fr42 \newline
\line(1,0){75} \newline
\textbf{1} \textit{Initiale} W Q   $\cdot$ \textit{Capitulumzeichen} R  \textbf{7} \textit{Majuskel} T  \textbf{11} \textit{Majuskel} T  \newline
\line(1,0){75} \newline
\textbf{1} \textit{Die Verse 453.1-502.30 fehlen} U   $\cdot$ Parcifal] Parzifal T PArtzifal W (Q) Parczifal R  $\cdot$ zim] zinn T hin zum Q hin zum wirt R \textbf{2} daz ie] [*]: daz nie V des ie W das es nie R \textbf{5} mir] mirs V \textbf{6} mohtir] moht er V  $\cdot$ gerne] gerner V  $\cdot$ verdaget] vertagt W \textbf{8} iuch] îv T nun R \textbf{9} sagich] Vch Sag ich R \textbf{10} sîn mîn] mit W  $\cdot$ triegen] trugen Q \textbf{13} was diu erde] die erde waz V \textbf{14} iu] \textit{om.} W \textbf{15} den] die W \textbf{16} Caynes] Kaẏns V Kayns Q R \textbf{17} sluoc] erschluͦg R \textbf{18} dô] [Daz]: Da V  $\cdot$ erde] [erd*]: erden V erden W (R) \textbf{20} den] Der Q \textbf{21} êrst des] der Q \textbf{23} sô reines niht ist] so [rein*]: reines niht enist V nit so reines ist W (Q) nit Reiners ist R \textbf{25} prüevet] pruͤff W \textbf{26} selbe was] was selber W Q was selb R \textbf{27} megden] magde Q  $\cdot$ zwei menschen komen] zwen mensch geborn R \textbf{28} selbe] [selb*]: selbe V selber W  $\cdot$ antlitze] sin menscheit R \textbf{29} êrsten megde] meide ersten R \textbf{30} hôhen art ein] art ein hoche R \newline
\end{minipage}
\end{table}
\end{document}
