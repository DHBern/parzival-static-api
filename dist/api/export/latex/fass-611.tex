\documentclass[8pt,a4paper,notitlepage]{article}
\usepackage{fullpage}
\usepackage{ulem}
\usepackage{xltxtra}
\usepackage{datetime}
\renewcommand{\dateseparator}{.}
\dmyyyydate
\usepackage{fancyhdr}
\usepackage{ifthen}
\pagestyle{fancy}
\fancyhf{}
\renewcommand{\headrulewidth}{0pt}
\fancyfoot[L]{\ifthenelse{\value{page}=1}{\today, \currenttime{} Uhr}{}}
\begin{document}
\begin{table}[ht]
\begin{minipage}[t]{0.5\linewidth}
\small
\begin{center}*D
\end{center}
\begin{tabular}{rl}
\textbf{611} & \textit{\begin{large}S\end{large}}i gâben fîanze,\\ 
 & daz si ze Joflanze\\ 
 & mit \textbf{rîtern} und \textbf{mit} vrouwen her\\ 
 & kœmen durch ir zweier \textbf{wer}.\\ 
5 & \textbf{alsô} \textbf{was} benant \textbf{daz} teidinc:\\ 
 & si \textbf{zwêne} al eine \textbf{ûf} einen rinc.\\ 
 & \textbf{Sus} schiet mîn hêr Gawan\\ 
 & dannen von dem werden man.\\ 
 & mit vreude er leischierte,\\ 
10 & der kranz in \textbf{zimierte}.\\ 
 & er \textbf{wolte} daz ors niht ûf \textbf{enthaben},\\ 
 & mit sporn \textbf{treib erz} an den graben.\\ 
 & Gringuljete nam bezîte\\ 
 & \textbf{sînen} sprunc \textbf{sô} wîte,\\ 
15 & daz Gawan vallen gar vermeit.\\ 
 & Zuo \textbf{z}im diu herzoginne reit,\\ 
 & al dâ der helt erbeizet was\\ 
 & von dem orse ûf \textbf{ein} gras\\ 
 & \textbf{unt} er \textbf{dem orse gurte}.\\ 
20 & ze sîner antwurte\\ 
 & erbeizte snellîche\\ 
 & diu herzoginne rîche.\\ 
 & \textbf{gein} \textbf{sînen vüezen} si sich bôt.\\ 
 & \textbf{dô sprach si}: "hêrre, solher nôt,\\ 
25 & als ich hân an iuch \textbf{gegert},\\ 
 & der wart \textbf{nie mîn wirde} wert.\\ 
 & vür wâr mir iwer arbeit\\ 
 & vüeget \textbf{solhiu} herzeleit,\\ 
 & \textbf{di\textit{u}} enpfâhen sol getriwez wîp\\ 
30 & umb ir lieben vriwendes lîp."\\ 
\end{tabular}
\scriptsize
\line(1,0){75} \newline
D Z \newline
\line(1,0){75} \newline
\textbf{1} \textit{Initiale} D Z  \textbf{7} \textit{Majuskel} D  \textbf{16} \textit{Majuskel} D  \newline
\line(1,0){75} \newline
\textbf{1} Si] ÷i D \textbf{2} Joflanze] Tschofflantze Z \textbf{5} alsô] Svs Z \textbf{6} zwêne] bede Z \textbf{9} Mit frevden er lesierte Z \textbf{11} er wolte] Ern moht Z  $\cdot$ enthaben] haben Z \textbf{12} treib erz] erz treip Z  $\cdot$ graben] [gaben]: graben Z \textbf{13} Gringuljete] Gringvliet D [Ki]: Kringvliet Z \textbf{14} sô] wol so Z \textbf{16} zim] im Z \textbf{18} ein] daz Z \textbf{19} unt] Vntz Z  $\cdot$ gurte] gegvrte Z \textbf{25} an] \textit{om.} Z \textbf{28} solhiu] soͤlich Z \textbf{29} diu] di D Die Z \newline
\end{minipage}
\hspace{0.5cm}
\begin{minipage}[t]{0.5\linewidth}
\small
\begin{center}*m
\end{center}
\begin{tabular}{rl}
 & si gâben fîanze,\\ 
 & daz si  \textit{Jo}flanze\\ 
 & mit \textbf{ritter} und vrouwen her\\ 
 & k\textit{œ}men \textit{durch} ir zweier \textbf{wer}.\\ 
5 & \textbf{aldâ} \textbf{was} benant \textit{\textbf{daz}} tegedinc:\\ 
 & si \textbf{zwên} alein \textbf{in} einen rinc.\\ 
 & \textbf{\begin{large}S\end{large}us} schiet mîn hêr Gawan\\ 
 & danne von dem werden man.\\ 
 & mit vröuden er leischierte,\\ 
10 & der kranz i\textit{n} \textbf{zimierte}.\\ 
 & er \textbf{wolt} daz ros niht ûf \textbf{enthaben},\\ 
 & mit sporn \textbf{treip erz} an den graben.\\ 
 & Gringulet nam bî zîte\\ 
 & \textbf{einen} sprunc \textbf{alsô} wîte,\\ 
15 & daz Gawan vallen g\textit{a}r vermeit.\\ 
 & zuo im diu herzoginne reit,\\ 
 & aldâ der h\textit{e}lt erbeize\textit{t w}as\\ 
 & von dem ros ûf \textbf{daz} gras\\ 
 & \textbf{und} er \textbf{dem rosse gurte}.\\ 
20 & zuo sîner antwurte\\ 
 & erbeizte snelleclîche\\ 
 & diu herzogîn rîche.\\ 
 & \textbf{geg\textit{e}n} \textbf{sînen vüezen} si sich bôt.\\ 
 & \textbf{si sprach}: "hêrre, solher nôt,\\ 
25 & als ich hab an iuch \textbf{begert},\\ 
 & der war\textit{t} \textbf{nie mîn wirde} wert.\\ 
 & vür wâr mir \textit{iuwer} arbeit\\ 
 & vüeget \textbf{solich} herzeleit,\\ 
 & \textbf{die} enpfâhen sol getriuwez wîp\\ 
30 & umb ir lieben vriundes lîp."\\ 
\end{tabular}
\scriptsize
\line(1,0){75} \newline
m n o \newline
\line(1,0){75} \newline
\textbf{7} \textit{Illustration mit Überschrift:} Also her gawan vff sime rosse sprengete vnd die kv́nigin mit ime rette n (o)   $\cdot$ \textit{Initiale} m n o  \newline
\line(1,0){75} \newline
\textbf{2} Joflanze] kouflancz m koufflantz n kauff lancz o \textbf{4} kœmen] Komen m n (o)  $\cdot$ durch] \textit{om.} m \textbf{5} aldâ] Also n o  $\cdot$ daz] \textit{om.} m \textbf{7} hêr] herre her n \textbf{9} leischierte] lastierte o \textbf{10} in] ẏm m \textbf{15} gar] ger m \textbf{17} helt] hilt m o  $\cdot$ erbeizet was] erbeisset wart vnd was m \textbf{21} erbeizte] Er erbeisete n \textbf{23} gegen] Gegegen m \textbf{24} hêrre] \textit{om.} o \textbf{26} wart] war m  $\cdot$ nie mîn wirde] min wúrde nẏe n \textbf{27} iuwer] \textit{om.} m \textbf{30} ir] irn m (n) o \newline
\end{minipage}
\end{table}
\newpage
\begin{table}[ht]
\begin{minipage}[t]{0.5\linewidth}
\small
\begin{center}*G
\end{center}
\begin{tabular}{rl}
 & \begin{large}S\end{large}i gâben fîanze,\\ 
 & daz si ze Tschofflanze\\ 
 & mit \textbf{rîtern} und \textbf{mit} vrouwen her\\ 
 & k\textit{œ}men durch ir zweier \textbf{wer}.\\ 
5 & \textbf{sus} \textit{\textbf{was}} benant \textbf{daz} teidinc:\\ 
 & si \textbf{bêde} al ein \textbf{ûf} einen rinc.\\ 
 & \textbf{sus} schiet mîn hêr Gawan\\ 
 & dannen von dem werden man.\\ 
 & mit vröuden er leisierte,\\ 
10 & der kranz in \textbf{condwierte}.\\ 
 & er\textbf{n} \textbf{moht}z ors niht ûf \textbf{enthaben},\\ 
 & mit sporn \textbf{erz treip} an den graben.\\ 
 & Gringuliet nam bezît\\ 
 & \textbf{sînen} sprunc \textbf{wol} \textbf{alse} wît,\\ 
15 & daz Gawan vallen gar vermeit.\\ 
 & zuo ime diu herzoginne reit,\\ 
 & al dâ der helt erbeizet was\\ 
 & von dem rosse ûf \textbf{daz} gras.\\ 
 & \textbf{unze} er \textbf{daz gegurte},\\ 
20 & zuo sîner antwurte\\ 
 & erbeizte snellîche\\ 
 & diu herzoginne rîche.\\ 
 & \textbf{gein} \textbf{sînem vuoze} si sich bôt.\\ 
 & \textbf{dô sprach si}: "hêrre, solcher nôt,\\ 
25 & als ich hân an iuch \textbf{ge\textit{k}e\textit{r}t},\\ 
 & de\textit{r} wart \textbf{mîn wirde nie} wert.\\ 
 & vür wâr mir iuwer arbeit\\ 
 & vüeget \textbf{solich} herzeleit,\\ 
 & \textbf{die} enpfâhen sol getriuwez wîp\\ 
30 & u\textit{mb}e ir l\textit{i}eben vriundes lîp."\\ 
\end{tabular}
\scriptsize
\line(1,0){75} \newline
G I L M Z Fr34 Fr51 \newline
\line(1,0){75} \newline
\textbf{1} \textit{Initiale} G L Z Fr51  \textbf{3} \textit{Initiale} I  \textbf{23} \textit{Initiale} I  \newline
\line(1,0){75} \newline
\textbf{2} ze Tschofflanze] zetschofanze G zeseffanze I Tschoflanze L zcu schofflancze M zv Tschofflantze Z zefloritschantze Fr34 so schoulance Fr51 \textbf{3} rîtern] ritter M Froͯwen Fr34  $\cdot$ und mit] vnd L  $\cdot$ vrouwen] Riͯtern Fr34 \textbf{4} kœmen] Chomin G (I) (L) (M) (Fr51)  $\cdot$ zweier] zawair I \textbf{5} was] \textit{om.} G  $\cdot$ teidinc] gedinch Fr51 \textbf{6} si] daz si I  $\cdot$ al ein] chomen I  $\cdot$ ûf] an Fr51  $\cdot$ einen] den I (L) (M) \textbf{7} mîn] \textit{om.} Fr51 \textbf{8} dannen] \textit{om.} Fr51  $\cdot$ dem] den selben Fr51 \textbf{9} er] vnd I \textbf{10} in] im L  $\cdot$ condwierte] zimierte Z \textbf{11} ern] er M  $\cdot$ ûf] \textit{om.} I  $\cdot$ enthaben] gehaben M (Fr51) haben Z \textbf{12} erz treip] dreb hers Fr51  $\cdot$ an den] vnz anden I \textbf{13} Gringuliet] Gringvliet L [Ki]: Kringuliet Z \textbf{14} wol alse] so L M wol so Z also Fr51 \textbf{15} gar] \textit{om.} Fr51 \textbf{17} Dar her nider gestan was Fr51 \textbf{19} unze] Da M Wann Fr51  $\cdot$ daz] dem rosze L (Z) das ros M (Fr51) \textbf{21} erbeizte] Stunt se Fr51 \textbf{23} ::: zo sinen vozen bot Fr51  $\cdot$ sînem vuoze] sinen vuzen I (L) (Z)  $\cdot$ si sich] [her]: sy sich M \textbf{24} dô sprach si] si sprach I (Fr51) Da sprach sy M \textbf{25} an] \textit{om.} Z  $\cdot$ gekert] geleit G gegert L (M) Z \textbf{26} der] des G  $\cdot$ mîn wirde nie] nie mýn wirde L (M) (Z) \textbf{28} Fvͤget solhez hertzen leit Fr34 \textbf{29} die] Das Fr51  $\cdot$ getriuwez] [getruwer]: getruwerz M getruwe Fr51 \textbf{30} umbe] Vnde G  $\cdot$ ir] ires Fr51  $\cdot$ lieben] lebin G libens Fr34 \newline
\end{minipage}
\hspace{0.5cm}
\begin{minipage}[t]{0.5\linewidth}
\small
\begin{center}*T
\end{center}
\begin{tabular}{rl}
 & si gâben fîanze,\\ 
 & daz si zuo Tschoflanze\\ 
 & mit \textbf{rîtern} und \textbf{mit} vrouwen her\\ 
 & k\textit{œ}men durch ir zweier \textbf{ger}.\\ 
5 & \textbf{sus} \textbf{wart} be\textit{n}an\textit{t} \textbf{ein} \textit{teged}inc\\ 
 & - si \textbf{beide} aleine \textbf{in} einen rinc -\\ 
 & \textbf{und} schiet mîn hêr Gawan\\ 
 & dannen von dem werden \textit{man}.\\ 
 & mit vreuden er le\textit{i}sierte,\\ 
10 & der kranz in \textbf{zimierte}.\\ 
 & er \textbf{en}\textbf{wolte} daz ors niht ûf \textbf{erhaben},\\ 
 & mit sporn \textbf{er ez treip} \textit{\textbf{unz}} an den graben.\\ 
 & Krynguliet nam bezît\\ 
 & \textbf{sînen} sprunc \textbf{wol} \textbf{sô} wît,\\ 
15 & daz Gawan \textit{vallen} gar vermeit.\\ 
 & zuo im diu herzoginne reit,\\ 
 & al dâ der helt erbeizet was\\ 
 & \textit{von dem orse ûf \textbf{daz} gras}.\\ 
 & \textbf{unz} er \textbf{dem orse gegurte},\\ 
20 & zuo sîner antwurte\\ 
 & erbeizte snellîche\\ 
 & diu herzoginne rîche.\\ 
 & \textbf{zuo} \textbf{sînen vüezen} si sich bôt.\\ 
 & \textbf{dô sprach si}: "hêrre, solicher nôt,\\ 
25 & als ich hân ane \textit{iuch} \textbf{gegert},\\ 
 & \textit{der\textbf{n} wart \textbf{nie mîn wirde} wert}.\\ 
 & vür wâr mir iuwer arbeit\\ 
 & vüeget \textbf{solich} herzeleit,\\ 
 & \textbf{die} enpfâhen sol getriuwe\textit{z} wîp\\ 
30 & umb ir l\textit{i}eben v\textit{riund}es lîp."\\ 
\end{tabular}
\scriptsize
\line(1,0){75} \newline
U V W Q R \newline
\line(1,0){75} \newline
\textbf{1} \textit{Initiale} Q   $\cdot$ \textit{Capitulumzeichen} R  \textbf{19} \textit{Überschrift:} Hie kam Gawin widerumb v́ber den graben vnd das wasser vnd die herczogin kam zu Im vnd begert gnad vnd der minne her Gawins R  \newline
\line(1,0){75} \newline
\textbf{1} fîanze] fianszen Q \textbf{2} Tschoflanze] Tscoflanze U Schoflanze V (W) schofflanze Q schoflancze R \textbf{4} kœmen] Comen U (W) (Q) (R) [K*men]: Kemen  V  $\cdot$ ger] wer Q R \textbf{5} sus] Als Q  $\cdot$ benant] begangen U  $\cdot$ ein tegedinc] ein rinc U [*]: daz tegeding V des tages geding W das teidinc Q daz teing dink R \textbf{6} beide] beidú R  $\cdot$ in] auff W (Q) (R) \textbf{7} und] [*]: Svz V Suß W (R) Do Q \textbf{8} man] \textit{om.} U \textbf{9} leisierte] lesierte U lasierte V laiserte R \textbf{10} in] Jm R \textbf{11} er enwolte] Er wolt W R  $\cdot$ erhaben] gehaben W haben Q (R) \textbf{12} er ez treip unz] er iz dreip mit U treib ers bys R \textbf{13} Krynguliet] Kringulet V Kringuliet W Q Krᵫngult R \textbf{14} sînen] Den W \textbf{15} Gawan] Gawin R  $\cdot$ vallen] \textit{om.} U  $\cdot$ gar] \textit{om.} W \textbf{16} zuo] Zvͦz V \textbf{17} helt] hertzog Q \textbf{18} \textit{Vers 611.18 fehlt} U  \textbf{19} \textit{Versfolge 611.20-19} V   $\cdot$ unz] mit U  $\cdot$ dem orse] [*]: dem orse V  $\cdot$ gegurte] gurte R \textbf{21} erbeizte] Er erbaißte W \textbf{23} vüezen] fuͦsze R \textbf{24} si] der R \textbf{25} iuch] \textit{om.} U \textbf{26} \textit{Vers 611.26 fehlt} U   $\cdot$ dern wart] Des enward W Der wart Q (R)  $\cdot$ mîn] miner Q \textbf{27} mir] \textit{om.} W \textbf{28} vüeget] Fuͯgt sich R \textbf{29} die] wie W  $\cdot$ sol] so R  $\cdot$ getriuwez] getruͦwen U \textbf{30} ir] irs W iren Q  $\cdot$ lieben vriundes] leben von des U [*]: lieben frv́ndes V \newline
\end{minipage}
\end{table}
\end{document}
