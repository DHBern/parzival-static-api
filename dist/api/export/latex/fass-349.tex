\documentclass[8pt,a4paper,notitlepage]{article}
\usepackage{fullpage}
\usepackage{ulem}
\usepackage{xltxtra}
\usepackage{datetime}
\renewcommand{\dateseparator}{.}
\dmyyyydate
\usepackage{fancyhdr}
\usepackage{ifthen}
\pagestyle{fancy}
\fancyhf{}
\renewcommand{\headrulewidth}{0pt}
\fancyfoot[L]{\ifthenelse{\value{page}=1}{\today, \currenttime{} Uhr}{}}
\begin{document}
\begin{table}[ht]
\begin{minipage}[t]{0.5\linewidth}
\small
\begin{center}*D
\end{center}
\begin{tabular}{rl}
\textbf{349} & \begin{large}S\end{large}us hât der zorn sich vür genomen,\\ 
 & \textbf{daz} bêde künege wellent komen\\ 
 & vür Bearosche, dâ man muoz\\ 
 & \textbf{gedienen mit arbeit} \textbf{wîbe} gruoz.\\ 
5 & \textbf{vil sper muoz man} dâ brechen,\\ 
 & bêdiu hurten unde stechen.\\ 
 & Bearosche ist sô ze wer,\\ 
 & ob wir heten zweinzec her,\\ 
 & ieslîchez grœzer denne wir hân,\\ 
10 & wir m\textit{üe}sen\textbf{s} unzervüert lân.\\ 
 & Mîn reise ist daz hinder her verholn.\\ 
 & disen schilt hân ich dan verstoln\\ 
 & ûz von andern kinden,\\ 
 & ob mîn hêrre m\textit{ö}hte vinden\\ 
15 & eine tjost durch sînen êrsten schilt\\ 
 & mit \textbf{hurten} poynder dar gezilt."\\ 
 & Der knappe hinder sich dô sach.\\ 
 & sîn hêrre vuor im balde nâch.\\ 
 & \textbf{diu} ors unt zwelf wîziu sper\\ 
20 & gâheten mit im \textbf{balde} her.\\ 
 & ich wæne, sîn gir des \textbf{iemen} trüge,\\ 
 & er wolde gern ze vorvlüge\\ 
 & die êrsten tjost dâ hân bejagt.\\ 
 & \textbf{sus} \textbf{hât} mir diu âventiure \textbf{gesagt}.\\ 
25 & \textbf{Der knappe sprach} ze Gawan:\\ 
 & "hêrre, lât mich iwern urloup hân."\\ 
 & \textbf{der} kêrte sîme hêrren zuo.\\ 
 & waz welt ir, daz Gawan nû tuo,\\ 
 & er\textbf{n} besehe, waz disiu mære sîn?\\ 
30 & doch lêrt in zwîvel strengen pîn.\\ 
\end{tabular}
\scriptsize
\line(1,0){75} \newline
D \newline
\line(1,0){75} \newline
\textbf{1} \textit{Initiale} D  \textbf{11} \textit{Majuskel} D  \textbf{17} \textit{Majuskel} D  \textbf{25} \textit{Majuskel} D  \newline
\line(1,0){75} \newline
\textbf{3} Bearosche] Bearosce D \textbf{7} Bearosche] Bearosce D \textbf{10} müesens] mvͦsens D \textbf{14} möhte] mohte D \newline
\end{minipage}
\hspace{0.5cm}
\begin{minipage}[t]{0.5\linewidth}
\small
\begin{center}*m
\end{center}
\begin{tabular}{rl}
 & sus hât der zorn sich vür genomen,\\ 
 & \textbf{daz} beide künige wellent komen\\ 
 & vür Bearosche, d\textit{â} man muoz\\ 
 & \textbf{gedienen mit arbeit} \textbf{wîbes} gruoz.\\ 
5 & \textbf{vil sper muoz man} d\textit{â} brechen,\\ 
 & beidiu h\textit{u}rten und stechen.\\ 
 & Bearosche ist sô ze wer,\\ 
 & ob wir heten zweinzic her,\\ 
 & ieglîche\textit{z} grœzer danne wir hân,\\ 
10 & wir müesen  unzervüeret lân.\\ 
 & mîn reise ist daz hinder her verholn.\\ 
 & disen schilt hân ich dan verstoln\\ 
 & ûz von \textbf{den} andern kinden,\\ 
 & ob mîn hêrre möhte \textit{v}inden\\ 
15 & eine just durch sînen êrsten schilt\\ 
 & mit \textbf{hurtes} poinder dar gezilt."\\ 
 & \begin{large}D\end{large}er knappe hinder sich dô sach.\\ 
 & sîn hêrre vuor im balde nâch.\\ 
 & \textbf{dr\textit{iu}} ros und zwelf wîziu sper\\ 
20 & gâhete\textit{n} mit im \textbf{balde} her.\\ 
 & ich wæne, sîn gir des \textbf{iemen} trüge,\\ 
 & er \textbf{en}wolde gerne ze vorvlüge\\ 
 & die êrsten just d\textit{â} hân be\textit{j}aget.\\ 
 & \textbf{sus} \textbf{hât} mir diu âventiure \textbf{gesaget}.\\ 
25 & \textbf{dô sprach \textit{der} knappe} ze Gawan:\\ 
 & "hêrre, lât mich iuwer urloup hân."\\ 
 & \textbf{der} kêrte sînem hêrren zuo.\\ 
 & waz welt ir, \textit{d}az Gawan nû tuo,\\ 
 & er besehe, waz disiu mære sîn?\\ 
30 & doch lêrt in zwîvel strengen pîn.\\ 
\end{tabular}
\scriptsize
\line(1,0){75} \newline
m n o \newline
\line(1,0){75} \newline
\textbf{17} \textit{Illustration mit Überschrift:} Also das her dem knappen noch ilte mit spern vnd rossen vnd er vrlop bat von gawan n (o)   $\cdot$ \textit{Initiale} m n o  \newline
\line(1,0){75} \newline
\textbf{1} Sich hette der zorn susz fúr genomen n  $\cdot$ vür genomen] vernomen o \textbf{3} Bearosche] bearosce m bearosc n beabesc o  $\cdot$ dâ] do m n do man o \textbf{4} wîbes] wibe n o \textbf{5} muoz] muͯs m  $\cdot$ dâ] do m n o \textbf{6} hurten] hirten m herte o \textbf{7} Bearosche] Bearosce m Bearosc n o \textbf{8} wir] ir o \textbf{9} ieglîchez grœzer] Yeglicher grosser m (n) Jglicher grossen o \textbf{10} müesen] muͯssen m (n) (o)  $\cdot$ unzervüeret] vnzerfurent o  $\cdot$ lân] lant n \textbf{11} reise] heise o \textbf{14} möhte] kunde n mochte o  $\cdot$ vinden] winden m \textbf{15} eine] Ein n (o) \textbf{16} hurtes] hortes o  $\cdot$ dar] gar n o  $\cdot$ gezilt] gezelt o \textbf{19} driu] Druẏ m \textbf{20} gâheten] Gahette m Gehabten o \textbf{21} des] das o \textbf{22} enwolde] wolte n o \textbf{23} êrsten] erste n o  $\cdot$ dâ] do m n o  $\cdot$ bejaget] betaget m gegaget o \textbf{25} der] \textit{om.} m \textbf{26} iuwer] uweren n (o) \textbf{27} der kêrte] Do kerte er n o  $\cdot$ sînem] sẏnen o  $\cdot$ hêrren] her n \textbf{28} daz] was m  $\cdot$ nû] \textit{om.} n o \textbf{30} lêrt] lerte n o  $\cdot$ strengen] strenge n o \newline
\end{minipage}
\end{table}
\newpage
\begin{table}[ht]
\begin{minipage}[t]{0.5\linewidth}
\small
\begin{center}*G
\end{center}
\begin{tabular}{rl}
 & sus hât der zorn sich vür genomen:\\ 
 & bêde künige wellent komen\\ 
 & vür Bearotsche, dâ man muoz\\ 
 & \textbf{mit arbeit dienen} \textbf{wîbe} gruoz.\\ 
5 & \textbf{man muoz vil sper} dâ brechen,\\ 
 & beidiu hurten unde stechen.\\ 
 & Bearotsche ist sô ze wer,\\ 
 & obe wir heten zweinzic her,\\ 
 & iegeslîchez gr\textit{œ}zer danne wir hân,\\ 
10 & wir m\textit{üe}sen \textbf{si} unzervüeret lân.\\ 
 & mîn reise ist daz hinder her verholen.\\ 
 & disen schilt hân ich dan verstolen\\ 
 & ûz von andern kinden,\\ 
 & op mîn hêrre m\textit{ö}hte vinden\\ 
15 & \begin{large}E\end{large}ine tjost durch sînen êrsten schilt\\ 
 & mit \textbf{hurtes} ponder dar gezilt."\\ 
 & der knappe hinder sich dô sach.\\ 
 & sîn hêrre vuor im balde nâch.\\ 
 & \textbf{driu} ors unt zwelf wîziu sper\\ 
20 & gâhten \textbf{vaste} mit im her.\\ 
 & ich wæne, sîn gir des \textbf{niemen} trüge,\\ 
 & er wolte gerne ze vorvlüge\\ 
 & die êrsten tjost dâ hân bejaget,\\ 
 & \textbf{sô} mir diu âventiure \textbf{saget}.\\ 
25 & \textbf{der knappe sprach} ze Gawan:\\ 
 & "\textit{hêrre}, lât mich iwer urloup hân."\\ 
 & \textbf{er} kêrte sînem hêrren zuo.\\ 
 & waz welt ir, daz Gawan nû tuo,\\ 
 & er \textbf{en}besehe, waz disiu mære sîn?\\ 
30 & doch lêrte in zwîvel strengen pîn.\\ 
\end{tabular}
\scriptsize
\line(1,0){75} \newline
G I O L M Q R Z Fr39 Fr40 \newline
\line(1,0){75} \newline
\textbf{1} \textit{Initiale} I O L Z Fr39 Fr40   $\cdot$ \textit{Capitulumzeichen} R  \textbf{15} \textit{Initiale} G  \textbf{17} \textit{Initiale} I  \textbf{25} \textit{Initiale} M  \newline
\line(1,0){75} \newline
\textbf{1} sus] ÷vs O Als Q  $\cdot$ sich] sus I fich O \textit{om.} M  $\cdot$ vür genomen] vernomen Q \textbf{2} bêde] Daz beide L Das beidiv Fr39 \textbf{3} Bearotsche] bearotsce I Bearotsh L Fr39 Bearotsch M Bearosche R bearotshe Fr40  $\cdot$ dâ] do Q R Fr39 \textbf{4} mit arbeit dienen] Gedienen mit arbeit O (M) Z  $\cdot$ wîbe] wibes I M \textbf{5} man] Vnd L Fr39  $\cdot$ dâ] \textit{om.} I R do Q Fr39  $\cdot$ brechen] zerbrechen R \textbf{6} beidiu] beide Fr40 \textbf{7} Bearotsche] Bearotsch O M Bearotsh L Fr39 Beroshe Q Bearosche R Bearoshe Fr40  $\cdot$ sô] \textit{om.} L \textbf{9} grœzer] groͮzer G \textbf{10} müesen si] moͮsense G (O) (Z) (Fr39) (Fr40) Muszens M musten Q muͯssent R \textbf{11} mîn] Miner O  $\cdot$ daz] des I (Q) (R) (Z)  $\cdot$ hinder her] hindren hers I hindir M  $\cdot$ verholen] verlorn O \textbf{12} dan] da Z  $\cdot$ verstolen] verzolen Q \textbf{14} möhte] mohte G I O L (M) (Q) (Z) Fr39 Fr40 \textbf{15} Eine] Ein O L Q R Z Fr39 (Fr40) \textbf{16} mit] Min M  $\cdot$ hurtes] hurtens Z \textbf{17} knappe] kappe Q  $\cdot$ dô] da Z \textbf{18} sîn] [der]: sin I  $\cdot$ vuor] fvͦrt O \textbf{19} driu] Dy M Trew Q  $\cdot$ zwelf] [xx]: xij Q zwei Fr40  $\cdot$ wîziu] wise R \textbf{20} gâhten] Die gahten L (Fr39) Gachtte R  $\cdot$ vaste] vasten L \textbf{21} sîn gir] sy gar M  $\cdot$ des] \textit{om.} Q Fr40  $\cdot$ niemen] iemen O (L) (M) (Q) (R) (Z) (Fr39) (Fr40)  $\cdot$ trüge] trige L \textbf{22} er] ern Fr40  $\cdot$ vorvlüge] vorsluge Q \textbf{23} dâ] \textit{om.} L do Q \textbf{24} sô] Als O (L) (M) Q R Z (Fr40) \textbf{26} hêrre] \textit{om.} G  $\cdot$ iwer] \textit{om.} I \textbf{27} er] Der O M Q R Z (Fr40)  $\cdot$ kêrte] chert I \textbf{28} daz Gawan nû] nv daz Gawan L \textbf{29} er enbesehe] er besehe I (O) (R) Ern besehn Q \textbf{30} lêrte] lert I L Q Fr39 Fr40  $\cdot$ zwîvel strengen] strengen zwiuels I zweyfel strenge Q zwiffel menge R zweifel solhen Z \newline
\end{minipage}
\hspace{0.5cm}
\begin{minipage}[t]{0.5\linewidth}
\small
\begin{center}*T
\end{center}
\begin{tabular}{rl}
 & \begin{large}S\end{large}us hât der zorn sich vür genomen,\\ 
 & \textbf{daz} beide künege wellent komen\\ 
 & vür Bearosch, dâ man muoz\\ 
 & \textbf{gedienen mit arbeit} \textbf{wîbe} gruoz\\ 
5 & \textbf{unde muoz vil sper} dâ brechen,\\ 
 & beid\textit{iu} hurten unde stechen.\\ 
 & Bearosch ist sô ze wer,\\ 
 & obe wir heten zwênzic her,\\ 
 & etslîchez grœzer danne wir hân,\\ 
10 & wir müesen\textbf{z} unzervüeret lân.\\ 
 & mîn reise ist daz hinder her verholn.\\ 
 & disen schilt hân ich dan verstoln\\ 
 & ûz von \textbf{den} andern kinden,\\ 
 & ob mîn hêrre m\textit{ö}hte vinden\\ 
15 & eine tjost durch sînen êrsten schilt\\ 
 & mit \textbf{hurtes} poynder dar gezilt."\\ 
 & Der knappe hinder sich dô sach.\\ 
 & sîn hêrre vuor im balde nâch.\\ 
 & \textbf{driu} ors unde zwelf wîziu sper\\ 
20 & gâheten \textbf{vaste} mit im her.\\ 
 & ich wæne, sîn gir des \textbf{iemen} trüge,\\ 
 & er wolte gerne ze vorvlüge\\ 
 & die êrsten tjost dâ hân bejaget,\\ 
 & \textbf{als} mir di\textit{u} âventiure \textbf{saget}.\\ 
25 & \textbf{\begin{large}D\end{large}er knappe sprach} ze Gawan:\\ 
 & "hêrre, lât mich iuwern urloup hân."\\ 
 & \textbf{der} kêrte sînem hêrren zuo.\\ 
 & Waz welt ir, daz Gawan nû tuo,\\ 
 & er\textbf{n} besehe, waz dis\textit{iu} mære sîn?\\ 
30 & doch lêret in zwîvel strengen pîn.\\ 
\end{tabular}
\scriptsize
\line(1,0){75} \newline
T V W \newline
\line(1,0){75} \newline
\textbf{1} \textit{Initiale} T W  \textbf{17} \textit{Majuskel} T  \textbf{25} \textit{Initiale} T V  \textbf{28} \textit{Majuskel} T  \newline
\line(1,0){75} \newline
\textbf{1} sich vür] fúr sich W \textbf{2} wellent] woͤllen W \textbf{3} Bearosch] Bearotsch T bearotsche V betrosch W  $\cdot$ dâ] do V W \textbf{4} mit arbeit] werder W \textbf{5} Das sper zerbrechen W  $\cdot$ dâ] do V \textbf{6} beidiu] beide T \textbf{7} Bearosch] Bearotsch T Bearotsche V Betrosch W \textbf{10} müesenz unzervüeret] mvͤssen vnser fuͦre V muͤssen sy vnzerfuͤret W \textbf{11} verholn] [verlo*]: verholn T \textbf{13} den] \textit{om.} W \textbf{14} möhte] mohte T \textbf{15} eine] Einen V Ein W \textbf{16} poynder] [poynder]: poyndier T \textbf{22} gerne] geren W \textbf{23} dâ] do V W \textbf{24} diu] die T \textbf{26} lât] nun lond W  $\cdot$ iuwern] \textit{om.} V \textbf{27} der] Er W \textbf{29} disiu] dise T dirre V \textbf{30} doch] Do W \newline
\end{minipage}
\end{table}
\end{document}
