\documentclass[8pt,a4paper,notitlepage]{article}
\usepackage{fullpage}
\usepackage{ulem}
\usepackage{xltxtra}
\usepackage{datetime}
\renewcommand{\dateseparator}{.}
\dmyyyydate
\usepackage{fancyhdr}
\usepackage{ifthen}
\pagestyle{fancy}
\fancyhf{}
\renewcommand{\headrulewidth}{0pt}
\fancyfoot[L]{\ifthenelse{\value{page}=1}{\today, \currenttime{} Uhr}{}}
\begin{document}
\begin{table}[ht]
\begin{minipage}[t]{0.5\linewidth}
\small
\begin{center}*D
\end{center}
\begin{tabular}{rl}
\textbf{675} & \textbf{\begin{large}I\end{large}n sîne} herberge reit\\ 
 & maneger, dem von herzen leit\\ 
 & was sîn langez \textbf{ûz} wesen.\\ 
 & nû was ouch Keie genesen\\ 
5 & bî dem Plimizœl der tjoste;\\ 
 & der \textbf{prüevete} Gawans koste.\\ 
 & er sprach: "mînes hêrren swâger Lot,\\ 
 & von dem was uns dehein nôt\\ 
 & ebenhiuzen \textbf{noch} \textbf{sunderringes}."\\ 
10 & dô \textbf{dâhte}r \textbf{noch} des dinges,\\ 
 & \textbf{wand} in Gawan dort niht rach,\\ 
 & dâ im sîn zeswer arm \textbf{zerbrach}.\\ 
 & "got mit den liuten wunder tuot:\\ 
 & wer gap Gawane \textbf{die} vrouwen luot?"\\ 
15 & \textbf{Sus} sprach \textit{Keie} in sîme schimpf;\\ 
 & daz was gein \textbf{vriwende} ein swach gelimpf.\\ 
 & Der getriwe ist vriwendes êren vrô;\\ 
 & der ungetriwe 'wâfenô'\\ 
 & rüefet, swenne \textbf{ein liep} \textbf{geschiht}\\ 
20 & \textbf{sînem vriwende} unt er daz siht.\\ 
 & Gawan pflac sælde unt êre.\\ 
 & gert iemen vürbaz mêre,\\ 
 & war wil der mit gedanken?\\ 
 & sô \textbf{sint} die muotes kranken\\ 
25 & \textbf{gîtes} und hazzes vol,\\ 
 & sô tuot dem ellenthaften wol,\\ 
 & \textbf{swâ} sînes vriwentes \textbf{prîs} gestêt,\\ 
 & daz schande vlühtec von im gêt.\\ 
 & \begin{large}G\end{large}awan âne valschen haz\\ 
30 & \textbf{manlîcher triwen nie} vergaz.\\ 
\end{tabular}
\scriptsize
\line(1,0){75} \newline
D Fr8 \newline
\line(1,0){75} \newline
\textbf{1} \textit{Initiale} D  \textbf{15} \textit{Majuskel} D  \textbf{17} \textit{Majuskel} D  \textbf{29} \textit{Initiale} D  \newline
\line(1,0){75} \newline
\textbf{1} reit] do reit Fr8 \textbf{4} Keie] Keye D keẏe Fr8 \textbf{5} Plimizœl] Plimizoͤl D Plẏmizole Fr8 \textbf{15} Keie] Gawan D \newline
\end{minipage}
\hspace{0.5cm}
\begin{minipage}[t]{0.5\linewidth}
\small
\begin{center}*m
\end{center}
\begin{tabular}{rl}
 & \textbf{\begin{large}I\end{large}n Gawans} herberge reit\\ 
 & manige\textit{r}, dem von herzen leit\\ 
 & was sîn langez \textbf{ûz} wesen.\\ 
 & nû was ouch Keie genesen\\ 
5 & bî dem Plimizol der juste;\\ 
 & der \textbf{prîset} Gawanes kuste.\\ 
 & er sprach: "mînes hêrren swâger Lot,\\ 
 & von de\textit{m} was uns dekein nôt\\ 
 & \dag ebenhitz\dag  \textbf{noch} \textbf{sunders ring\textit{e}s}."\\ 
10 & dô \textbf{gedâhte} er \textbf{noch} des dinges,\\ 
 & \textbf{daz} in Gawan dort niht rach,\\ 
 & dô im sîn zesewer arm \textbf{zerbrach}.\\ 
 & "got mit den liuten wunder tuot:\\ 
 & wer gap Gawan \textbf{die} vrowen luot?"\\ 
15 & \textbf{sus} sprach Keie in sînem schimpf;\\ 
 & daz \textit{was} gegen \textbf{vriunde} ein swach gelimpf.\\ 
 & der getriuwe ist vriundes êren vrô;\\ 
 & der ungetriuwe 'wâfenô'\\ 
 & rüef\textit{t}, wan \textbf{eim liep} \textbf{beschiht},\\ 
20 & \textbf{sînem vr\textit{iu}nde}, und er daz siht.\\ 
 & Gawan pflac sælde und êre.\\ 
 & gert ieman vürbaz mêre,\\ 
 & war wil der mit gedanken?\\ 
 & sô \textbf{sint} die muotes kranken\\ 
25 & \textbf{gîtes} und hazzes vol,\\ 
 & sô tuot dem ellenthaften wol,\\ 
 & \textbf{wâ} sînes vriundes \textbf{prîs} gestât,\\ 
 & daz schande vlühtic von im gât.\\ 
 & Gawan âne valsche\textit{n h}az\\ 
30 & \textbf{manlîcher triuwen nie} vergaz.\\ 
\end{tabular}
\scriptsize
\line(1,0){75} \newline
m n o Fr69 \newline
\line(1,0){75} \newline
\textbf{1} \textit{Initiale} m   $\cdot$ \textit{Capitulumzeichen} n  \newline
\line(1,0){75} \newline
\textbf{2} maniger] Mangen m Manigem n (o) \textbf{4} Keie] keẏe n o \textbf{5} Plimizol] blimzol n \textbf{8} dem] den m  $\cdot$ dekein] dekeins o \textbf{9} sunders] sunder n (o)  $\cdot$ ringes] ringens m n ringencz o \textbf{10} dinges] dingens n o \textbf{11} niht] mit o \textbf{12} dô] da Fr69  $\cdot$ zerbrach] brach Fr69 \textbf{14} luot] lot o \textbf{15} Keie] keẏe n \textbf{16} was] \textit{om.} m \textbf{19} rüeft] Rufftte m (o)  $\cdot$ eim] eȳ o  $\cdot$ beschiht] geschicht o \textbf{20} sînem] Sine: o  $\cdot$ vriunde] frande m :rende Fr69  $\cdot$ und] \textit{om.} n  $\cdot$ siht] gicht n sich: o \textbf{23} der] er o  $\cdot$ mit] mit mir n \textbf{26} dem] den n \textbf{29} valschen haz] falschen list vnd has m \textbf{30} nie] nit o \newline
\end{minipage}
\end{table}
\newpage
\begin{table}[ht]
\begin{minipage}[t]{0.5\linewidth}
\small
\begin{center}*G
\end{center}
\begin{tabular}{rl}
 & \textbf{\begin{large}A\end{large}n sîne} herberge reit\\ 
 & maniger, dem von herzen leit\\ 
 & was sîn langez \textbf{ûz} wesen.\\ 
 & nû was ouch Kay \textbf{wol} genesen\\ 
5 & bî dem Blimzol der tjost;\\ 
 & der \textbf{brüevet} Gawans kost.\\ 
 & er sprach: "mînes hêrren swâger Lot,\\ 
 & von dem was uns dehein nôt\\ 
 & ebenhiuze \textbf{unde} \textbf{sunderringes}."\\ 
10 & dô \textbf{dâhte}r \textbf{noch} des dinges,\\ 
 & \textbf{wan} in Gawan dort niht rach,\\ 
 & dô im sîn zeswer arm \textbf{zerbrach}.\\ 
 & "got mit den liuten wunder tuot:\\ 
 & \textit{wer} gab Gawan \textbf{die} vrouwen luot?",\\ 
15 & sprach Kay in sîne\textit{m} schimpf;\\ 
 & daz was gein \textbf{vriunde} ein swacher gelimpf.\\ 
 & der getriwe ist vriundes êren vrô;\\ 
 & der ungetriwe 'wâfenô'\\ 
 & ruofet, swenne \textbf{im lieb} \textbf{geschiht},\\ 
20 & \textbf{sîne\textit{m} vriunde}, unde er da\textit{z} siht.\\ 
 & Gawan pflac sælde unde êre.\\ 
 & gert iemen vürbaz mêre,\\ 
 & war wil der mit gedanken?\\ 
 & sô \textbf{sint} die muotes kranken\\ 
25 & \textbf{nîdes} unde hazzes vol,\\ 
 & sô tuot dem ellenthaften wol,\\ 
 & \textbf{swaz} sînes vriundes \textbf{strît} gestêt,\\ 
 & daz schande vlü\textit{h}tic von im gêt.\\ 
 & \textbf{sît} Gawan âne val\textit{s}chen haz\\ 
30 & \textbf{stæte mit triwen nie} vergaz,\\ 
\end{tabular}
\scriptsize
\line(1,0){75} \newline
G I L M Z Fr61 \newline
\line(1,0){75} \newline
\textbf{1} \textit{Initiale} G L M Z  \textbf{7} \textit{Initiale} I  \textbf{21} \textit{Initiale} I  \newline
\line(1,0){75} \newline
\textbf{1} An] Jn L (M) (Z) Fr61 \textbf{2} von] von im I \textit{om.} M \textbf{3} ûz] vsze M vzzen Z \textbf{4} Kay] kei G kayn I keý L keye M key Z kaẏ Fr61  $\cdot$ wol] \textit{om.} L M Z Fr61 \textbf{5} Blimzol] plimizol I M Z plýmizol L Plimzol Fr61  $\cdot$ der] ze der Fr61 \textbf{6} brüevet] pruͯfte L (M)  $\cdot$ Gawans] Gawanes Fr61 \textbf{7} er sprach] \textit{om.} Fr61  $\cdot$ Lot] lôt G \textbf{9} sunderringes] svnders ringes L (Fr61) sunder [tinges]: twinges  Z \textbf{10} dô] Da M doch Fr61  $\cdot$ dâhter] gedachte er L (M) (Z) (Fr61) \textbf{11} wan] Daz L (M) Z  $\cdot$ in] [*]: in G \textbf{12} dô] Daz L (M) Da Z Fr61  $\cdot$ zerbrach] brach L Fr61 \textbf{13} [dor]: Got wunder mit den luten tuͤt I \textbf{14} wer] \textit{om.} G  $\cdot$ Gawan] Gawane L M Fr61  $\cdot$ luot] flvt L \textbf{15} Kay] kei G keý L keye M key Z kaẏ Fr61  $\cdot$ sînem] sinen G \textbf{16} vriunde] friunden I frovden L  $\cdot$ ein] \textit{om.} L  $\cdot$ swacher gelimpf] vngelimpfe Fr61 \textbf{17} der] \textit{om.} M  $\cdot$ êren] ere Fr61 \textbf{18} dem getriwen ist also I  $\cdot$ ungetriwe] vngeriwe L \textbf{19} ruofet] der ruͦffet I  $\cdot$ swenne] wenne L (M)  $\cdot$ im] \textit{om.} I sin Z \textbf{20} iemen lieb vngern er sicht Fr61  $\cdot$ sînem] Sine G M Z  $\cdot$ daz] da G M \textbf{21} sælde unde êre] selden [mer]: vnde ere I \textbf{25} nîdes] Guͯtes L (M) (Fr61) Gites Z \textbf{26} dem] den I (L) (Fr61)  $\cdot$ ellenthaften] ellinthaffte M \textbf{27} swaz] swa I (Z) Wa L Was M  $\cdot$ vriundes] [stri*]: vriundes I  $\cdot$ strît] pris Z  $\cdot$ gestêt] geste I \textbf{28} vlühtic] floͮtch G  $\cdot$ von] vor L Fr61  $\cdot$ gêt] ge I \textbf{29} valschen] falchen G \textbf{30} Manlicher truͯwe (trewen Z ) nie (\textit{om.} M ) vergaz L (M) (Z) (Fr61) \newline
\end{minipage}
\hspace{0.5cm}
\begin{minipage}[t]{0.5\linewidth}
\small
\begin{center}*T
\end{center}
\begin{tabular}{rl}
 & \textbf{in sîne} herberge reit\\ 
 & manege\textit{r}, dem von herzen leit\\ 
 & was sîn langez \textbf{ûzen} wesen.\\ 
 & nû was ouch Keye genesen\\ 
5 & bî dem Plymizol der tjost;\\ 
 & der \textbf{pruofte} Gawanes kost.\\ 
 & er sprach: "mînes hêrren swâger Lot,\\ 
 & von dem was uns kein nôt\\ 
 & ebenh\textit{iu}ze \textbf{und} \textbf{sunderringes}."\\ 
10 & dô \textbf{gedâht} er \textbf{doch} des dinges,\\ 
 & \textbf{daz} in Gawan dort niht rach,\\ 
 & dô im sîn zeswer arm \textbf{brach}.\\ 
 & "got mit den liuten wunder tuot:\\ 
 & wer gap Gawane \textbf{der} vrouwen luot?",\\ 
15 & sprach Key in sînem schimpf;\\ 
 & daz was gên \textbf{vreude} ein \textit{swacher} gelimpf.\\ 
 & der getriuwe ist vriu\textit{n}des êren vrô;\\ 
 & der ungetriuwe 'wâfenô'\\ 
 & rüefet, wan \textbf{ein liebe} \textbf{geschiht}\\ 
20 & \textbf{sînen vriunden} und er daz siht.\\ 
 & Gawan pflac sælde und êre.\\ 
 & gert ieman vürbaz mêre,\\ 
 & war wil der mit gedanken?\\ 
 & sô \textbf{sîn} die muotes kranken\\ 
25 & \textbf{nîdes} und hazzes vol,\\ 
 & sô tuot dem e\textit{ll}enthaften wol,\\ 
 & \textbf{waz} sînes vriundes \textbf{strît} gestêt,\\ 
 & daz schande vlühtic von im gêt.\\ 
 & Gawan âne valschen haz\\ 
30 & \textbf{menlîcher triuwe niht} vergaz.\\ 
\end{tabular}
\scriptsize
\line(1,0){75} \newline
Q R V W \newline
\line(1,0){75} \newline
\textbf{1} \textit{Initiale} W  \textbf{21} \textit{Initiale} W  \newline
\line(1,0){75} \newline
\textbf{1} [*]: Jn Gawans herberge reit V  $\cdot$ reit] er rait W \textbf{2} [*]: Maniger dem von herzen leit V  $\cdot$ maneger] Manchem Q \textbf{3} ûzen] vsse R (W) [*]: vs  V \textbf{4} [*]: Nv waz keye genesen V  $\cdot$ Keye] key R W \textbf{5} [*]: Bi dem plimizols tiost V  $\cdot$ Plymizol] plimizol Q R \textbf{6} Gawanes] Gawins R herr gawans W \textbf{9} ebenhiuze] Eben herze Q \textbf{10} doch] noch R W V \textbf{11} Gawan dort] gewan doͯrt R dort herr gawan W  $\cdot$ niht rach] [nih*]: niht enrach V \textbf{12} sîn] der W  $\cdot$ brach] zerbrach R W V \textbf{14} Gawane] gawinen R  $\cdot$ der] die R den W [d*]: der V  $\cdot$ luot] flvͦt V \textbf{16} Daz waz [Gege*]: Gegen froͤide ein swach gelinpf V  $\cdot$ vreude] froͯden R frúndin W  $\cdot$ swacher] \textit{om.} Q gar schwacher R \textbf{17} vriundes] frewdes Q treúwer W \textbf{18} wâfenô] Vnfro R waffen io W [wafeno]: wafen V \textbf{19} wan] [swem]: swen V  $\cdot$ ein] eim W \textbf{20} sînen vriunden] Seinem frúnd W (V) \textbf{21} Gawan] Gawin R HErr gawan W \textbf{23} gedanken] gedenken R \textbf{24} sô] Sv́ V  $\cdot$ die] des R  $\cdot$ kranken] krenken R \textbf{26} ellenthaften] erenthafften Q \textbf{27} waz] Wann W [Swaz]: Swa V  $\cdot$ strît gestêt] [streites stet]: streitge stet Q [*]: pris so stet V \textbf{28} schande] schanden R \textbf{29} Gawan] Gawin R Herr gawan W \textbf{30} triuwe] trv́wen V  $\cdot$ niht] nie R W [m*]: nie V \newline
\end{minipage}
\end{table}
\end{document}
