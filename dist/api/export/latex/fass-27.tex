\documentclass[8pt,a4paper,notitlepage]{article}
\usepackage{fullpage}
\usepackage{ulem}
\usepackage{xltxtra}
\usepackage{datetime}
\renewcommand{\dateseparator}{.}
\dmyyyydate
\usepackage{fancyhdr}
\usepackage{ifthen}
\pagestyle{fancy}
\fancyhf{}
\renewcommand{\headrulewidth}{0pt}
\fancyfoot[L]{\ifthenelse{\value{page}=1}{\today, \currenttime{} Uhr}{}}
\begin{document}
\begin{table}[ht]
\begin{minipage}[t]{0.5\linewidth}
\small
\begin{center}*D
\end{center}
\begin{tabular}{rl}
\textbf{27} & verrâtens ich doch \textbf{wênic} kan,\\ 
 & swie mich \textbf{des} zîhen sîne man.\\ 
 & er was mir lieber danne in.\\ 
 & âne \textbf{geziuge} ich des niht bin,\\ 
5 & mit den ichz sol bewæren \textbf{noch}.\\ 
 & die \textbf{rehten} wârheit wizzen doch\\ 
 & mîne gote unt \textbf{ouch} die sîne.\\ 
 & er gap mir manege pîne.\\ 
 & Nû hât mîn schamendiu wîpheit\\ 
10 & sîn lôn \textbf{erlenget} - mîn leit.\\ 
 & dem helde erwarp mîn magetuom\\ 
 & an rîterschefte manegen ruom.\\ 
 & \textbf{dô versuocht ich} \textbf{in}, ob er kunde sîn\\ 
 & ein vriunt. daz wart vil balde schîn.\\ 
15 & er gap durch mich sîn harnas\\ 
 & enwec. daz als ein palas\\ 
 & dort stêt, daz ist ein hôch gezelt;\\ 
 & daz brâhten Schotten ûf \textbf{ditze} velt.\\ 
 & dô \textbf{daz} der helt âne wart,\\ 
20 & sîn \textbf{lîp dô wênic wart gespart}.\\ 
 & des lebens in dâ nâch verdrôz,\\ 
 & manege âventiure suochter blôz.\\ 
 & dô ditz \textbf{alsô} was,\\ 
 & ein vürste, Prothizilas\\ 
25 & \textbf{er hiez}, mîn messenîe,\\ 
 & vor zageheit der vrîe,\\ 
 & \textbf{ûz} durch âventiure reit,\\ 
 & \textbf{dâ} grôz schade in niht vermeit.\\ 
 & zem fôrest in Azagouc\\ 
30 & ein tjost \textbf{im sterben} niht erlouc,\\ 
\end{tabular}
\scriptsize
\line(1,0){75} \newline
D \newline
\line(1,0){75} \newline
\textbf{9} \textit{Majuskel} D  \newline
\line(1,0){75} \newline
\textbf{18} Schotten] scotten D \textbf{29} Azagouc] Azagoͮch D \newline
\end{minipage}
\hspace{0.5cm}
\begin{minipage}[t]{0.5\linewidth}
\small
\begin{center}*m
\end{center}
\begin{tabular}{rl}
 & verrâten\textit{s i}ch doch \textbf{wênic} kan,\\ 
 & wie mich \textbf{des} z\textit{î}hen sîne man.\\ 
 & er was mir lieber \textbf{vil} danne in.\\ 
 & âne \textbf{ziuge} ich des niht bin,\\ 
5 & mit den ichz sol bewæren \textbf{noch}.\\ 
 & die \textbf{rehten} wârheit wizzen doch\\ 
 & mîne gote und \textbf{ouch} die sîne.\\ 
 & er gap mir manige pîne.\\ 
 & nû hât mîn schamendi\textit{u} wîpheit\\ 
10 & sîn lôn \textbf{erlenget}, \textbf{mir} mîn leit.\\ 
 & dem helde erwarp mîn magetuom\\ 
 & an ritte\textit{r}schaft manigen ruom.\\ 
 & \textbf{dô versuochte ich}, ob er konde sîn\\ 
 & ein vriunt. daz wart vil balde schîn.\\ 
15 & er gap durch mich sîn harnas\\ 
 & enwec. daz als ein palas\\ 
 & dort stât, daz ist ein hôch gezelt;\\ 
 & daz brâhten Schotten ûf \textbf{daz} velt.\\ 
 & dô \textbf{des} der helt âne wart,\\ 
20 & sîn \textbf{lîp dô wênic wart gespart}.\\ 
 & des leben\textit{s} in dâ nâch verdrôz,\\ 
 & manige âventiure suochte er blôz.\\ 
 & dô diz \textbf{alsô} \textbf{komen} was,\\ 
 & ein vürste, \textbf{hiez} Pro\textit{t}izilas,\\ 
25 & \textbf{\textit{\begin{large}D\end{large}}er was} mîn massenîe,\\ 
 & vor zageheit der vrîe,\\ 
 & \textbf{ûz} durch âventiure reit,\\ 
 & \textbf{d\textit{â}} grôz schade in niht vermeit.\\ 
 & zem fôrest in \dag das sagowe\dag \\ 
30 & ein just \textbf{im sterben} niht \dag erloe\dag ,\\ 
\end{tabular}
\scriptsize
\line(1,0){75} \newline
m n o \newline
\line(1,0){75} \newline
\textbf{23} \textit{Capitulumzeichen} n  \textbf{25} \textit{Initiale} m  \newline
\line(1,0){75} \newline
\textbf{1} \textit{Die Verse 26.4-29.1 fehlen} o   $\cdot$ verrâtens ich] Verratten sich m \textbf{2} zîhen] ziehen m zihent n \textbf{5} ichz] ich das n  $\cdot$ noch] doch n \textbf{6} rehten] rechte n  $\cdot$ doch] noch n \textbf{9} schamendiu] schamen die m schame die n \textbf{12} ritterschaft] ritteschaft m \textbf{13} dô] [S]: Do m  $\cdot$ konde] [kinde]: koͯnde m \textbf{15} sîn] [min]: sin m \textbf{18} daz] disz n \textbf{21} lebens in] leben: \textit{nachträglich korrigiert zu:} lebensin m  $\cdot$ dâ] do n  $\cdot$ nâch] \textit{om.} n \textbf{22} suochte] suͦchet n \textbf{24} Protizilas] prochisilas m prothisalas n \textbf{25} Der] DDer m \textbf{28} dâ] Do m Der n  $\cdot$ grôz] grosse n \textbf{30} sterben] das sterben n  $\cdot$ erloe] erlouwe n \newline
\end{minipage}
\end{table}
\newpage
\begin{table}[ht]
\begin{minipage}[t]{0.5\linewidth}
\small
\begin{center}*G
\end{center}
\begin{tabular}{rl}
 & verrâtens ich doch \textbf{lützel} kan,\\ 
 & swie mich \textbf{es} zîhen sîne man.\\ 
 & er was mir lieber danne in.\\ 
 & âne \textbf{geziuc} ich des niht bin,\\ 
5 & mit den ich ez sol bewæren \textbf{doch}.\\ 
 & die \textbf{reht} wârheit wizzen doch\\ 
 & mîne gote und die sîne.\\ 
 & er ga\textit{p} mir manige pîne.\\ 
 & nû hât mîn schemendiu wîpheit\\ 
10 & sîn lôn \textbf{erlenget} \textbf{und} mîn leit.\\ 
 & dem helde erwarp mîn magtuom\\ 
 & an rîterschaft \textbf{vil} manigen ruom.\\ 
 & \textbf{ich versuocht} \textbf{in}, ob er kunde sîn\\ 
 & ein vriunt. daz wart vil balde schîn.\\ 
15 & er gap durch mich sîn harnas\\ 
 & enwec. daz als ein palas\\ 
 & dort stêt, daz ist ein hôch gezelt;\\ 
 & d\textit{a}z brâhten Schotten û\textit{f} \textbf{diz} velt.\\ 
 & dô \textbf{daz} der helt âne wart,\\ 
20 & sîn \textbf{manheit was vil ungespart}.\\ 
 & des lebens in dar nâch verdrôz,\\ 
 & manege âventiure suochter blôz.\\ 
 & dô diz \textbf{alsô} was,\\ 
 & ein vürste, Prozitalas\\ 
25 & \textbf{hiez}, mîn messenîe,\\ 
 & vor zageheit der vrîe,\\ 
 & \textbf{ûz} durch âventiure reit.\\ 
 & \textbf{ein} grôzer schade in niht vermeit.\\ 
 & zem fôreis in Azagouc\\ 
30 & ein tjost \textbf{in sterbens} niht erlouc,\\ 
\end{tabular}
\scriptsize
\line(1,0){75} \newline
G O L M Q R W Z Fr29 Fr32 Fr71 \newline
\line(1,0){75} \newline
\textbf{1} \textit{Initiale} O M Fr29  \textbf{23} \textit{Initiale} L Q R W Z Fr32  \textbf{29} \textit{Initiale} Fr71  \newline
\line(1,0){75} \newline
\textbf{1} verrâtens] ÷erratens O Verates Q \textbf{2} swie] Wie L (M) Q R W  $\cdot$ es] sin O L (M) (Q) R Fr29 Fr32 ye W  $\cdot$ zîhen] zichent R \textbf{4} Ein getzewge des ich nicht bin Q  $\cdot$ ich des niht bin] [i]: des [mút]: nút enbin R ich des bin Fr32  $\cdot$ geziuc] gesig W \textbf{5} den] dem Fr29  $\cdot$ ez] \textit{om.} M W  $\cdot$ sol bewæren] sols beweren R wol bewere Z  $\cdot$ doch] noch O L M Q R W (Z) Fr29 ouch Fr32 \textbf{6} reht] rechten L (M) (Q) (R) (Z) (Fr29) (Fr32)  $\cdot$ wizzen] wiszent L (R) wisset Q \textit{om.} Fr32 \textbf{7} mîne gote] Mein got Q Mine goten R [Mit]: Min gote Z  $\cdot$ die] auch dy Q (R) (Z) \textbf{8} gap] gam G gar Q  $\cdot$ mir manige] mit manchein Q \textbf{9} mîn] mit Q  $\cdot$ schemendiu] schamende O Fr29 (Fr32) \textbf{10} erlenget] gelenget O  $\cdot$ und] daz ist W \textbf{12} vil] \textit{om.} O L M Q R W Z Fr29 Fr32 \textbf{13} ich] E ich L Do Q R (Fr32)  $\cdot$ in] \textit{om.} L W ich Q R ich in Fr32 \textbf{16} enwec] Ey weck Q  $\cdot$ als] was Q  $\cdot$ ein] ein glasz Q \textbf{17} hôch] hohe M grosz Q (Fr32)  $\cdot$ gezelt] palasz geczelt \textit{nachträglich korrigiert zu:} geczelt Q \textbf{18} daz] d:z G  $\cdot$ Schotten] schoten G O schottin M dy schotten Q  $\cdot$ ûf diz] v: diz G avf daz O (Q) (R) (W) (Fr32) \textbf{19} \textit{Die Verse 27.19-20 fehlen} R  \textbf{20} Sin lip do (da M doch Z ) wenich (lútzel W ) wart gespart O (L) (M) (Q) (W) (Z) (Fr29) (Fr32)  $\cdot$ ungespart] [vgespart]: vngespart G \textbf{21} in dar nâch] dar nach in L in dannoch Q (Fr71) \textbf{22} suochter] er svͦhte O sucht er Q (R) (Z) (Fr71)  $\cdot$ blôz] sit bloß W \textbf{23} dô] Da Z  $\cdot$ alsô] alsvs O (L) (M) (Z) (Fr32) alles Q alles [sus]: us R alsus alles W  $\cdot$ was] wasz ergangen Q \textbf{24} Prozitalas] [portialas]: portizalas O Portizalas L brotisallas M partifalsz Q portizalas R prothizalas W protizalas Z Fr32 \textbf{25} hiez] Der hiez Z  $\cdot$ mîn] mýner L (Q) nime W \textbf{26} vor zageheit] Des zcageheit M Vor zaget Q \textbf{28} ein] Do Q Da R Z (Fr32)  $\cdot$ schade] schaden Q (R) \textbf{29} zem fôreis] Zu den fursten Q In dem voieiz Fr71  $\cdot$ in] \textit{om.} Z ze Fr71  $\cdot$ Azagouc] azagoͮch G azagavch O Azagovch L azagoick Q azagog W mazagovc Z azagôvc Fr32 achsagovch Fr71 \textbf{30} ein] en Fr32 in Fr71  $\cdot$ in] ym M (R) (Z) (Fr32)  $\cdot$ sterbens] sterben O L (M) R W Z Fr32 strebet Q  $\cdot$ niht] \textit{om.} Fr32  $\cdot$ erlouc] erflock Q enlog W (Fr71) \newline
\end{minipage}
\hspace{0.5cm}
\begin{minipage}[t]{0.5\linewidth}
\small
\begin{center}*T
\end{center}
\begin{tabular}{rl}
 & verrâtens ich doch \textbf{lützel} kan,\\ 
 & swie mich\textbf{s} zîhen sîne man.\\ 
 & er was mir lieber danne in.\\ 
 & âne \textbf{geziuge} ich des niht bin,\\ 
5 & mit den ich\textit{z} sol bewæren \textbf{noch}.\\ 
 & die \textbf{mîne} wârheit wizzen doch\\ 
 & mîne gote und die sîne.\\ 
 & er gap mir manege pîne.\\ 
 & \begin{large}N\end{large}û hât mîn schamendiu wîpheit\\ 
10 & sînen lôn \textbf{und} mîn leit.\\ 
 & dem helde erwarp mîn magetuom\\ 
 & an rîterschefte manegen ruom.\\ 
 & \textbf{ich ver\textit{s}uochte}, ob er kunde sîn\\ 
 & ein vriunt. daz wart vil balde schîn.\\ 
15 & er gap durch mich sîn harnas\\ 
 & enwec. daz als ein palas\\ 
 & dort stêt, daz ist ein hôch gezelt;\\ 
 & daz brâhten Schotten ûf \textbf{diz} velt.\\ 
 & dô \textbf{des} der helt âne wart,\\ 
20 & sîn \textbf{lîp dô wênic wart gespart}.\\ 
 & des lebens in dar nâch verdrôz,\\ 
 & manege âventiure suochter blôz.\\ 
 & \textbf{\begin{large}D\end{large}ar nâch}, dô diz \textbf{alsus} was,\\ 
 & Ein vürste, \textbf{hiez} Protizalas\\ 
25 & \textbf{in mîner} massenîe,\\ 
 & vor zageheit der vrîe,\\ 
 & \textbf{ouch} durch âventiure reit.\\ 
 & \textbf{ein} grôzer schade in niht vermeit.\\ 
 & zem fôreht in Azagouc\\ 
30 & ein tjost \textbf{im sterben} niht erlouc,\\ 
\end{tabular}
\scriptsize
\line(1,0){75} \newline
T U V \newline
\line(1,0){75} \newline
\textbf{9} \textit{Initiale} T  \textbf{23} \textit{Initiale} T  \textbf{24} \textit{Majuskel} T  \newline
\line(1,0){75} \newline
\textbf{2} swie michs] Wie mich sin U swie mich V \textbf{3} danne in] dan er in V \textbf{4} des niht] niht des V \textbf{5} ichz sol] ichs sol T ich iz U (V)  $\cdot$ noch] wil V \textbf{6} mîne] miner V  $\cdot$ doch] vil V \textbf{10} und] [*]: erlenget vnd V \textbf{13} versuochte] verscvͦhte T \textbf{18} brâhten] brathe U  $\cdot$ Schotten] schoten T  $\cdot$ diz] daz U \textbf{22} suochter blôz] suͦht [*]: er sit blos V \textbf{24} Protizalas] Pothisalas U [P*othi*]: Prothissalas V \textbf{29} zem] Zuͦ eim U  $\cdot$ Azagouc] Azagôvc T azogove U [az*eloͮg]: azagoͮg V \textbf{30} erlouc] enloͮg V \newline
\end{minipage}
\end{table}
\end{document}
