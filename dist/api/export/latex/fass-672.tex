\documentclass[8pt,a4paper,notitlepage]{article}
\usepackage{fullpage}
\usepackage{ulem}
\usepackage{xltxtra}
\usepackage{datetime}
\renewcommand{\dateseparator}{.}
\dmyyyydate
\usepackage{fancyhdr}
\usepackage{ifthen}
\pagestyle{fancy}
\fancyhf{}
\renewcommand{\headrulewidth}{0pt}
\fancyfoot[L]{\ifthenelse{\value{page}=1}{\today, \currenttime{} Uhr}{}}
\begin{document}
\begin{table}[ht]
\begin{minipage}[t]{0.5\linewidth}
\small
\begin{center}*D
\end{center}
\begin{tabular}{rl}
\textbf{672} & \begin{large}A\end{large}rtus erbeizte unt gienc dar în.\\ 
 & er saz zuo dem neven sîn;\\ 
 & den bestuont er sus mit mæren,\\ 
 & wer die \textbf{vünf} vrouwen wæren.\\ 
5 & Dô huop mîn hêr Gawan\\ 
 & an der eldesten \textbf{zem êrsten} an.\\ 
 & sus sprach er zuo dem Bertun:\\ 
 & "erkant ir Utepandragun,\\ 
 & sô ist diz Arnive, sîn wîp;\\ 
10 & von den zwein kom \textbf{iwer} lîp.\\ 
 & sô ist diz diu muoter mîn,\\ 
 & von Norwæge diu künegîn.\\ 
 & \textbf{dise} zwô mîne swester sint.\\ 
 & nû seht, wie vlætigiu kint!"\\ 
15 & Ein ander küssen dâ \textbf{geschach}.\\ 
 & vreude unt jâmer sach\\ 
 & al die \textbf{daz} sehen wolten;\\ 
 & von der liebe si daz dolten.\\ 
 & \textbf{beidiu} lachen und weinen\\ 
20 & \textbf{kunde ir munt} \textbf{vil} wol \textbf{bescheinen};\\ 
 & von grôzer liebe daz geschach.\\ 
 & Artus ze Gawane sprach:\\ 
 & "neve, ich bin \textbf{des} \textbf{mæres noch} vrî,\\ 
 & wer diu clâre \textbf{vürste} vrouwe sî."\\ 
25 & dô sprach Gawan, der kurtoys:\\ 
 & "ez ist diu \textbf{herzoginne} von Logroys,\\ 
 & in der \textbf{gnâden} bin ich hie.\\ 
 & mir ist gesagt, ir habt gesuochet sie.\\ 
 & swaz ir des habt genozzen,\\ 
30 & daz zeiget unverdrozzen.\\ 
\end{tabular}
\scriptsize
\line(1,0){75} \newline
D Fr8 \newline
\line(1,0){75} \newline
\textbf{1} \textit{Initiale} D Fr8  \textbf{5} \textit{Majuskel} D  \textbf{15} \textit{Majuskel} D  \newline
\line(1,0){75} \newline
\textbf{1} Artus] ARthus Fr8  $\cdot$ dar] hin Fr8 \textbf{2} er] Vnde Fr8 \textbf{6} zem] \textit{om.} Fr8 \textbf{7} Bertun] [b*]: brẏttun Fr8 \textbf{8} Utepandragun] Vͦtepandragvn D Vterpandragun Fr8 \textbf{9} sô] [Sv]: So Fr8  $\cdot$ Arnive] Arnẏue Fr8 \textbf{12} Norwæge] Norwege Fr8 \textbf{16} \textit{Die Verse 672.16-20 fehlen} Fr8  \textbf{21} \textit{nach 672.21 am Rand nachgetragen:} Vz den ovgen ein grozer bach Fr8   $\cdot$ von] [Vz]: Von Fr8  $\cdot$ geschach] sich brach Fr8 \textbf{22} Artus] Arthus Fr8 \textbf{23} noch] \textit{om.} Fr8 \textbf{24} clâre vürste] vumfte Fr8 \textbf{26} diu] de D  $\cdot$ Logroys] Logroẏs D Fr8 \textbf{28} habt gesuochet] suͦchtet Fr8 \newline
\end{minipage}
\hspace{0.5cm}
\begin{minipage}[t]{0.5\linewidth}
\small
\begin{center}*m
\end{center}
\begin{tabular}{rl}
 & \begin{large}A\end{large}rtus erbeizte und gienc dar în.\\ 
 & er saz zuo dem neven sîn;\\ 
 & den bestuont er sus mit mæren,\\ 
 & wer die \textbf{vünf} vrowen wæren.\\ 
5 & dô huop mîn hêr Gawan\\ 
 & an de\textit{r} altesten \textbf{zem êrsten} an.\\ 
 & sus sprach er zuo dem Britu\textit{n}:\\ 
 & "erkan\textit{t} ir Utrapandragun,\\ 
 & sô ist diz Ar\textit{niv}e, sîn wîp;\\ 
10 & von den zwein kam \textbf{iuwer} lîp.\\ 
 & sô ist diz diu muoter mîn,\\ 
 & von N\textit{o}rwæge diu künigîn.\\ 
 & \textbf{die} zwô mîn swester sint.\\ 
 & nû seht, wie vlætigiu kint!"\\ 
15 & ein ander küssen dâ \textbf{beschach}.\\ 
 & vröude und jâmer \textbf{dâ} sach\\ 
 & alle, die \textbf{ez} seh\textit{en w}olten;\\ 
 & von der liebe si daz dolten.\\ 
 & \textbf{beidiu} lache\textit{n} und weinen\\ 
20 & \textbf{kunde ir munt} wol \textbf{besch\textit{e}inen};\\ 
 & von grôzer liebe daz geschach.\\ 
 & Artus zuo Gawane sprach:\\ 
 & "neve, ich \textit{bin} \textbf{noch mæres} vrî,\\ 
 & wer diu clâ\textit{r}e \textbf{\textit{vü}nfte} vrowe sî."\\ 
25 & dô sprach Gawan, der \textit{k}ur\textit{t}ois:\\ 
 & "ez ist diu \textbf{herzogîn} von Logrois,\\ 
 & in der \textbf{gnâden} bin ich hie.\\ 
 & mir ist gesaget, ir habt gesuochet \textit{s}ie.\\ 
 & waz ir des habt genozzen,\\ 
30 & daz zöuget unverdrozzen.\\ 
\end{tabular}
\scriptsize
\line(1,0){75} \newline
m n o \newline
\line(1,0){75} \newline
\textbf{1} \textit{Initiale} m n  \newline
\line(1,0){75} \newline
\textbf{5} hêr] herre her n \textbf{6} der] den m \textbf{7} Britun] brittuͯm m britẏm o \textbf{8} erkant] Erkante m (o) Erkanten n  $\cdot$ Utrapandragun] utrapandraguͯn m vterpendragun n uter pandragẏm o \textbf{9} Arnive] arune m arniwe n arnife o \textbf{12} Norwæge] nurwege m norwege n o \textbf{13} die] Dise n o \textbf{15} dâ] do m n o  $\cdot$ beschach] geschach n o \textbf{16} dâ] do n o \textbf{17} Alle die es sehehen soltten vnd wolten m \textbf{18} daz] des o \textbf{19} lachen] lache m lach o \textbf{20} bescheinen] beschinen m \textbf{21} grôzer] grosse o \textbf{23} bin] \textit{om.} m \textbf{24} clâre] clorore m  $\cdot$ vünfte] senftte m \textbf{25} kurtois] turkois m n turkeis o \textbf{28} sie] hie m \textbf{29} habt] hab: o \textbf{30} zöuget] zeigent n \newline
\end{minipage}
\end{table}
\newpage
\begin{table}[ht]
\begin{minipage}[t]{0.5\linewidth}
\small
\begin{center}*G
\end{center}
\begin{tabular}{rl}
 & \begin{large}A\end{large}rtus erbeizte unde gie dar în.\\ 
 & er saz zuo dem neven sîn;\\ 
 & den bestuont er sus mit mæren,\\ 
 & wer die \textbf{vier} vrouwen wæren.\\ 
5 & dô huop mîn hêr Gawan\\ 
 & an der eltesten \textbf{alrêrst} an.\\ 
 & sus sprach er zuo dem Britun:\\ 
 & "erkandet ir Utpandragun,\\ 
 & sô ist ditze Arnive, sîn wîp;\\ 
10 & von den zwein kom \textbf{iu der} lîp.\\ 
 & sô ist ditze diu muoter mîn,\\ 
 & von Norwæge diu künigîn.\\ 
 & \textbf{dise} zwô mîne swester sint.\\ 
 & nû seht, wie vlætigiu kint!"\\ 
15 & ein ander küssen dâ \textbf{geschach}.\\ 
 & vröude unde jâmer sach\\ 
 & alle, die \textbf{daz} sehen wolden;\\ 
 & von der liebe si daz dolden.\\ 
 & lachen unde weinen\\ 
20 & \textbf{si kunden} wol \textbf{erscheinen};\\ 
 & von grôzer liebe daz geschach.\\ 
 & Artus ze Gawane sprach:\\ 
 & "neve, ich bin \textbf{des} \textbf{mæres noch} vr\textit{î},\\ 
 & wer diu clâre vrouwe sî."\\ 
25 & dô sprach Gawan, der kurtois:\\ 
 & "ez ist diu \textbf{herzogîn} von Logrois,\\ 
 & in der \textbf{gnâden} bin ich hie.\\ 
 & mir ist geseit, ir habet gesuochet sie.\\ 
 & swaz ir des habet genozzen,\\ 
30 & daz zeiget unverdrozzen.\\ 
\end{tabular}
\scriptsize
\line(1,0){75} \newline
G I L M Z Fr61 \newline
\line(1,0){75} \newline
\textbf{1} \textit{Initiale} G L Z Fr61  \textbf{11} \textit{Initiale} I  \newline
\line(1,0){75} \newline
\textbf{1} Artus] ARtuͯs L Artaus Fr61  $\cdot$ erbeizte] [erbai*]: erbaizt I \textbf{3} den bestuont] dem bestuͤnde I \textbf{4} wæren] waren L \textbf{5} dô] Da M Z \textbf{6} alrêrst] \textit{om.} L Z Fr61 \textbf{7} Britun] prituͦn I Brittvͯn L Bruͯnen M Britavn Fr61 \textbf{8} Utpandragun] vtreprandagruͦn I vterpandragrunen M vtepandragavn Fr61 \textbf{9} ditze] \textit{om.} I daz L (M) Fr61  $\cdot$ Arnive] Arniue I \textbf{10} iu der] ewer Z \textbf{11} ditze] daz L Fr61 \textbf{12} Norwæge] [r*]: norwegen I Norwege L (M) Z Norweg Fr61 \textbf{13} dise] Die Fr61  $\cdot$ swester] [swer]: swestern L \textbf{14} nû] \textit{om.} Fr61  $\cdot$ vlætigiu] flehtige L \textbf{15} küssen dâ] da zechusshen I \textbf{16} sach] da gesach Z \textbf{17} daz] ez L (Fr61)  $\cdot$ wolden] wollin M \textbf{18} der] \textit{om.} Fr61  $\cdot$ dolden] solden I \textbf{20} kunden] begunden Fr61 \textbf{22} Artus] Artaus Fr61  $\cdot$ Gawane] Gawan I gawanen Z \textbf{23} noch] \textit{om.} Fr61  $\cdot$ vrî] frô G \textbf{24} vrouwe] fvnfte frowe Z  $\cdot$ sî] sin M \textbf{25} dô] Da M  $\cdot$ Gawan] Gawen Fr61 \textbf{26} ist] [sitzet]: ist I  $\cdot$ herzogîn] chuniginne Fr61  $\cdot$ Logrois] logroẏs G Logroys I Fr61 Logroýs L \textbf{27} gnâden] gnade L M (Fr61)  $\cdot$ hie] hie da bei Fr61 \textbf{28} gesuochet] geshuchet I  $\cdot$ sie] [h]: sie Z sei Fr61 \textbf{29} swaz] Waz L (M)  $\cdot$ des] daz L \newline
\end{minipage}
\hspace{0.5cm}
\begin{minipage}[t]{0.5\linewidth}
\small
\begin{center}*T
\end{center}
\begin{tabular}{rl}
 & Artus erbeizet und gienc dar în.\\ 
 & er saz zuo dem neven sîn;\\ 
 & den bestuont er sus mit mæren,\\ 
 & wer die \textbf{vier} vrouwen wæren.\\ 
5 & dô huop mîn hêr Gawan\\ 
 & an der eltesten \textbf{alrêst} an.\\ 
 & sus sprach er zuo dem Britun:\\ 
 & "erkant ir Utpandragun,\\ 
 & sô ist diz Arnyve, sîn wîp;\\ 
10 & von den zweien kom \textbf{iu der} lîp.\\ 
 & sô ist diz diu muoter mîn,\\ 
 & von Norwæge diu künigîn.\\ 
 & \textbf{dise} zw\textit{ô} mîn swester sint.\\ 
 & nû seht, wie vlætigiu kint!"\\ 
15 & ein ander küssen dô \textbf{geschach}.\\ 
 & vreude und jâmer sach\\ 
 & alle, die \textbf{daz} sehen wolten;\\ 
 & von der liebe si daz dolten.\\ 
 & \textbf{beidiu} lachen und weinen\\ 
20 & \textbf{kund ir munt} wol \textbf{bescheinen};\\ 
 & von grôzer liebe daz geschach.\\ 
 & Artus zuo Gawane sprach:\\ 
 & "neve, ich bin \textbf{des} \textbf{mæres noch} vrî,\\ 
 & wer diu clâre vrouwe sî."\\ 
25 & dô sprach Gawan, der \textit{k}ur\textit{t}o\textit{i}s:\\ 
 & "ez ist diu \textbf{künigîn} von Logrois,\\ 
 & in der \textbf{gnâde} bin ich hie.\\ 
 & mir ist gesagt, ir habt gesuochet sie.\\ 
 & waz ir des habt genozzen,\\ 
30 & daz zeiget unverdrozzen.\\ 
\end{tabular}
\scriptsize
\line(1,0){75} \newline
Q R W V \newline
\line(1,0){75} \newline
\textbf{1} \textit{Initiale} Q R W  \textbf{25} \textit{Initiale} V  \newline
\line(1,0){75} \newline
\textbf{1} erbeizet] erbeiczte R (W) (V) \textbf{2} er saz] Vnd satz sich W \textbf{3} sus] als Q \textbf{4} die] dise W  $\cdot$ vier] [*]: vúnf V \textbf{5} Gawan] gawann Q \textbf{6} eltesten] aller eltesten R  $\cdot$ alrêst] huͯb er R zuͦm ersten W [*]: zem ersten V \textbf{7} sus] Als Q  $\cdot$ Britun] brituͯn Q [*]: prittvn V \textbf{8} erkant ir] Habt ir erkant W [J*t]: Erkantent ir V  $\cdot$ Utpandragun] vtpandraguͯn Q vterpandragun W (V) \textbf{9} Arnyve] arniue Q V Arnyue R (W) \textbf{11} diz] auch diß W \textbf{12} Norwæge] norwege Q (R) W V  $\cdot$ diu] die werde W \textbf{13} zwô] zwe Q  $\cdot$ swester] schwestren R \textbf{16} sach] men [*]: do sach V \textbf{18} von] Vor W  $\cdot$ liebe] liebu R \textbf{20} bescheinen] erscheinen R \textbf{21} liebe] liebú R \textbf{22} Gawane] gawinen R herr gawane W \textbf{23} Neve ich bin [*]: noch meres vri V \textbf{24} clâre] clare [v*]: fv́nfte V \textbf{25} Gawan] Gawin R herr gawan W  $\cdot$ kurtois] turkoitis Q \textbf{26} künigîn] herczogin R (W) (V)  $\cdot$ Logrois] logoris R logroys V \textbf{27} gnâde] gnaden R  $\cdot$ bin ich] ich bin V \textbf{28} gesuochet sie] gesuͦchet ye R sy gesuͦchet ie W \textbf{29} habt] haben R \textbf{30} zeiget] zoͯgt R  $\cdot$ unverdrozzen] vns [*]: vnverdrossen V \newline
\end{minipage}
\end{table}
\end{document}
