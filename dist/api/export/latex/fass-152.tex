\documentclass[8pt,a4paper,notitlepage]{article}
\usepackage{fullpage}
\usepackage{ulem}
\usepackage{xltxtra}
\usepackage{datetime}
\renewcommand{\dateseparator}{.}
\dmyyyydate
\usepackage{fancyhdr}
\usepackage{ifthen}
\pagestyle{fancy}
\fancyhf{}
\renewcommand{\headrulewidth}{0pt}
\fancyfoot[L]{\ifthenelse{\value{page}=1}{\today, \currenttime{} Uhr}{}}
\begin{document}
\begin{table}[ht]
\begin{minipage}[t]{0.5\linewidth}
\small
\begin{center}*D
\end{center}
\begin{tabular}{rl}
\textbf{152} & Dô sprach der unwîse:\\ 
 & "iwerm werdem prîse\\ 
 & ist gegeben ein smæhiu letze.\\ 
 & ich bin sîn \textbf{vengec netze},\\ 
5 & \textbf{ich} sol \textbf{in} wider in iuch smîden,\\ 
 & daz ir\textbf{z} \textbf{enpfindet} ûf den liden.\\ 
 & Ez ist dem künege Artus\\ 
 & ûf sînen hof unt in sîn hûs\\ 
 & sô manec werder man geriten,\\ 
10 & durch \textbf{den} ir lachen hât vermiten,\\ 
 & unt lachet nû durch einen man,\\ 
 & der niht mit ritters vuore kan."\\ 
 & \textbf{In zorne} wunders vil geschiht.\\ 
 & sînes slages wære im erteilet niht\\ 
15 & vorem rîche ûf dise magt,\\ 
 & diu vil von vriwenden \textbf{wart} geklagt.\\ 
 & ob si halt schilt solde tragen!\\ 
 & diu \textbf{unvuoge} \textbf{ist dâ} geslagen,\\ 
 & wan si was von arde ein \textbf{vürstîn}.\\ 
20 & Orilus und Læhelin,\\ 
 & ir bruoder, hetenz die gesehen,\\ 
 & der slege minre \textbf{wære} geschehen.\\ 
 & \begin{large}D\end{large}er verswigene Anthanor,\\ 
 & der durch swîgen \textbf{dûht} ein tôr,\\ 
25 & sîn rede unt ir lachen\\ 
 & was gezilt mit \textbf{einen} sachen:\\ 
 & er\textbf{n} wolde nimmer wort gesagen,\\ 
 & si\textbf{ne} lachete, diu dâ \textbf{wa\textit{r}t} geslagen.\\ 
 & dô ir lachen \textbf{wart} getân,\\ 
30 & sîn munt sprach ze Keien sân:\\ 
\end{tabular}
\scriptsize
\line(1,0){75} \newline
D \newline
\line(1,0){75} \newline
\textbf{1} \textit{Majuskel} D  \textbf{7} \textit{Majuskel} D  \textbf{13} \textit{Majuskel} D  \textbf{23} \textit{Initiale} D  \newline
\line(1,0){75} \newline
\textbf{28} wart] wat D \textbf{30} Keien] keyen D \newline
\end{minipage}
\hspace{0.5cm}
\begin{minipage}[t]{0.5\linewidth}
\small
\begin{center}*m
\end{center}
\begin{tabular}{rl}
 & dô sprach der unwîse:\\ 
 & "iuwerm werden prîse\\ 
 & ist geben ein smæhiu letze.\\ 
 & ich bin sîn \textbf{vengic netze},\\ 
5 & \textbf{ich} sol \textbf{in} wider in iuch smîden,\\ 
 & daz ir\textbf{s} \textbf{enpfindet} ûf den liden.\\ 
 & ez ist dem künige Artus\\ 
 & ûf sînen hof und in sîn hûs\\ 
 & sô manic wert man geriten,\\ 
10 & durch \textbf{die} ir lachen hât vermiten,\\ 
 & und lachet nû durch einen man,\\ 
 & der niht mit ritters vuore kan."\\ 
 & \textbf{in zorne} wunders \textit{v}il geschiht.\\ 
 & sînes slages wære ime erteilet niht\\ 
15 & vorme rîche ûf dise maget,\\ 
 & diu vil von vriunden \textbf{wirt} geklaget.\\ 
 & ob si halt schilt solte tragen!\\ 
 & diu \textbf{unvuoge} \textbf{ist dâ} geslagen,\\ 
 & wanne si was von arde ein \textbf{vürstîn}.\\ 
20 & Orilus und Lehelin,\\ 
 & ir bruoder, hetenz die gesehen,\\ 
 & der slege minner \textbf{wære} geschehen.\\ 
 & \begin{large}D\end{large}er verswigene Anth\textit{a}n\textit{o}r,\\ 
 & der durch swîgen \textbf{dûhte} ein tôr,\\ 
25 & sîn rede und ir lachen\\ 
 & was gezilt mit \textbf{einen} sachen:\\ 
 & er \textbf{en}wolte niemer wort gesagen.\\ 
 & \multicolumn{1}{l}{ - - - }\\ 
 & dô ir lachen \textbf{wart} getân,\\ 
30 & sîn munt sprach ze Keien sân:\\ 
\end{tabular}
\scriptsize
\line(1,0){75} \newline
m n o \newline
\line(1,0){75} \newline
\textbf{23} \textit{Initiale} m o   $\cdot$ \textit{Capitulumzeichen} n  \newline
\line(1,0){75} \newline
\textbf{2} iuwerm] Vwern n o \textbf{5} in wider] wider n  $\cdot$ in iuch] vch o \textbf{6} enpfindet] enpfinden n o \textbf{10} ir] [l]: jre n  $\cdot$ hât] was n \textbf{13} vil] wil m \textbf{16} vil von] vor den n (o) \textbf{18} unvuoge] vngefuͦge o  $\cdot$ dâ] do n \textbf{20} Orilus] Orilius n \textbf{21} Hettens ir bruͯder gesehen n (o) \textbf{22} minner] nymmer n (o)  $\cdot$ geschehen] beschehen n \textbf{23} Anthanor] anthener m anthenor n o \textbf{25} ir] sin n \textbf{26} gezilt mit einen] gezelt zuͯ einer n (o) \textbf{27} enwolte] wolte n (o) \textbf{28} \textit{Vers 152.28 fehlt} m n o  \textbf{30} Keien] keẏen n \newline
\end{minipage}
\end{table}
\newpage
\begin{table}[ht]
\begin{minipage}[t]{0.5\linewidth}
\small
\begin{center}*G
\end{center}
\begin{tabular}{rl}
 & dô sprach der unwîse:\\ 
 & "iwere\textit{m} werden brîse\\ 
 & ist gegeben ein smæhiu letze.\\ 
 & ich bin sîn \textbf{vencnetze}\\ 
5 & \textbf{unde} sol \textbf{in} \textbf{hin} wider in iuch smîden,\\ 
 & daz ir\textbf{s} \textbf{enpfindet} ûf den liden.\\ 
 & ez ist dem künige Artus\\ 
 & ûf sînen hof und in sîn hûs\\ 
 & sô manic wert man geriten,\\ 
10 & durch \textbf{den} ir lachen habet vermiten,\\ 
 & unde lachet nû durch einen man,\\ 
 & der niht mit rîters vuore kan."\\ 
 & \textbf{in zorne} wunders vil geschiht.\\ 
 & sînes slages wære im erteilt niht\\ 
15 & vor dem rîche ûf dise maget,\\ 
 & diu vil von vriunden \textbf{wirt} geklaget.\\ 
 & op si halt schilt solte tragen!\\ 
 & diu \textbf{ungevuoge} \textbf{ist hie} geslagen,\\ 
 & wan si was von arde ein \textbf{vürstîn}.\\ 
20 & Orilus und Lehelin,\\ 
 & ir bruoder, hetenz die gesehen,\\ 
 & der slege minner \textbf{wære} geschehen.\\ 
 & der verswigen Antanor,\\ 
 & der durch swîgen \textbf{dûht} ein tôr,\\ 
25 & sîn rede und ir lachen\\ 
 & was gezilt mit \textbf{zwein} sachen:\\ 
 & er wolte nimer wort gesagen,\\ 
 & \begin{large}S\end{large}i\textbf{ne} lachte, diu dâ \textbf{wart} geslagen.\\ 
 & dô ir lachen \textbf{wart} getân,\\ 
30 & sîn munt sprach ze Kayn sân:\\ 
\end{tabular}
\scriptsize
\line(1,0){75} \newline
G I O L M Q R Z Fr65 \newline
\line(1,0){75} \newline
\textbf{7} \textit{Initiale} M  \textbf{13} \textit{Initiale} I Q  \textbf{23} \textit{Initiale} O L R Z  \textbf{28} \textit{Initiale} G  \newline
\line(1,0){75} \newline
\textbf{1} unwîse] vngeweyse Q \textbf{2} iwerem werden] iweren werden G (M) ewerm werdem I Ewrē werdem Q \textbf{3} ist] Wirt O Q  $\cdot$ smæhiu letze] soͯliche leczin R \textbf{4} sîn vencnetze] sin vinchen neze O weninc Nucze M sint vinc netze Q sin vengech netze Z \textbf{5} unde] Jch L  $\cdot$ sol in] sol L Q  $\cdot$ hin] \textit{om.} I  $\cdot$ in iuch] în O ouch M  $\cdot$ smîden] sniden M \textbf{6} irs] ir sin I ir des L ir Q es R  $\cdot$ enpfindet] entsebit M  $\cdot$ den] dē L dem R \textbf{8} ûf] in I  $\cdot$ sînen] sinē M (Q) sinem R  $\cdot$ in] \textit{om.} M Q \textbf{9} sô] \textit{om.} O L Q R  $\cdot$ wert] \textit{om.} R \textbf{10} den] die L (Q) R \textbf{12} rîters] riter O  $\cdot$ vuore] furen Q fuͦge R \textbf{13} wunders] dinges Z \textbf{14} slages] slahens I  $\cdot$ im] in L  $\cdot$ erteilt] erlauptet I \textbf{16} diu wart von friunden vil Gechagt I  $\cdot$ diu] Do Q  $\cdot$ von] \textit{om.} Z  $\cdot$ vriunden] vrouden M  $\cdot$ wirt] wart O \textbf{17} op] O R \textbf{18} ungevuoge] jvngfrauͯwe L \textbf{20} Orilus] [orlus]: orilus G Orillus L  $\cdot$ Lehelin] Lechelin R \textbf{22} geschehen] [gewesen]: geshehen I \textbf{23} der] ÷er O  $\cdot$ verswigen] vorswiget Q  $\cdot$ Antanor] [Anator]: Antanor O antonor R \textbf{24} durch swîgen] vorswigene M \textbf{25} sîn] Jn Q  $\cdot$ ir] sin R \textbf{26} gezilt] giczelt M (R)  $\cdot$ mit] von O \textbf{27} er] ern I (M) (R) (Z)  $\cdot$ nimer] nie I \textbf{28} Sine] Sinen O Sy R  $\cdot$ lachte] lachet I lachte ê O (L) (Q)  $\cdot$ dâ] \textit{om.} O do Q \textbf{29} dô] Da M Z  $\cdot$ wart] was O (L) M R do wart Q \textbf{30} sîn munt] Do L  $\cdot$ ze] er zuͯ L  $\cdot$ Kayn] [kan]: kain G chain I kay O Q keyn M key R Z \newline
\end{minipage}
\hspace{0.5cm}
\begin{minipage}[t]{0.5\linewidth}
\small
\begin{center}*T (U)
\end{center}
\begin{tabular}{rl}
 & dô sprach der unwîse:\\ 
 & "iuwerm werden prîse\\ 
 & ist gegeben ein smæhiu letze.\\ 
 & ich bin sîn \textbf{vincnetze}\\ 
5 & \textbf{und} sol \textbf{ez} \textbf{hin} wider in iuch smîden,\\ 
 & daz ir\textbf{s} \textbf{gevüelet} ûf den liden.\\ 
 & ez ist dem künege Artus\\ 
 & ûf sînen hof und in sîn hûs\\ 
 & sô manec werder man geriten,\\ 
10 & durch \textbf{den} ir lachen hât vermiten,\\ 
 & und lachet nû durch einen man,\\ 
 & der niht mit rîters vuore kan."\\ 
 & \textbf{\begin{large}D\end{large}urch zorn} wunders vil geschiht.\\ 
 & sînes slages \textit{wære} im erteilet niht\\ 
15 & vorme rîch ûf dise maget,\\ 
 & diu vil von vriunden \textbf{wirt} geklaget.\\ 
 & ob si h\textit{a}lt schilt solte tragen!\\ 
 & diu \textbf{ungevuoge} \textbf{hie ist} geslagen,\\ 
 & wan si was von art ein \textbf{künegîn}.\\ 
20 & Orilus und Lehelin,\\ 
 & ir bruoder, hetenz die gesehen,\\ 
 & der slege minre \textbf{wæren} geschehen.\\ 
 & d\textit{er} versw\textit{i}gen Antenor,\\ 
 & der durch swîgen \textbf{was} ein tôr,\\ 
25 & sîn rede und ir lachen\\ 
 & was gezilt mit \textbf{zwein} sachen:\\ 
 & er \textbf{en}wolte niem\textit{er} wort gesagen,\\ 
 & si lachete, diu d\textit{â} \textbf{was} geslagen.\\ 
 & dô ir lachen \textbf{was} getân,\\ 
30 & sîn munt sprach zuo Key sân:\\ 
\end{tabular}
\scriptsize
\line(1,0){75} \newline
U V W T \newline
\line(1,0){75} \newline
\textbf{1} \textit{Majuskel} T  \textbf{7} \textit{Initiale} T  \textbf{13} \textit{Initiale} U V   $\cdot$ \textit{Majuskel} T  \textbf{19} \textit{Majuskel} T  \textbf{21} \textit{Majuskel} T  \textbf{23} \textit{Initiale} W T  \textbf{29} \textit{Majuskel} T  \newline
\line(1,0){75} \newline
\textbf{4} vincnetze] vancnetze T \textbf{5} sol ez] [*]: sol V soln T \textbf{6} gevüelet] v́nphindent V (W) (T)  $\cdot$ den liden] dem ligen W \textbf{7} Artus] Artv̂s T \textbf{8} in] auff W \textbf{10} den] die W \textbf{11} nû] \textit{om.} W \textbf{13} durch] Jn T  $\cdot$ wunders] wunder V \textbf{14} sînes slages] Sines [sla*]: slahenz V Seiner slege W sines slahens T  $\cdot$ wære] \textit{om.} U wern W  $\cdot$ im] ir W \textbf{15} \textit{nach 152.15:} Das wissent sicherleiche W   $\cdot$ Zuͦ hofe noch vor dem reiche W \textbf{16} \textit{nach 152.16:} Die keúsche suͤsse raine maget W   $\cdot$ diu vil von vriunden wirt] [d*]: die wol von frowen wart V Sy wirt von frúnden vil W \textbf{17} Solte sy von rechte schilte tragen W  $\cdot$ halt] helt U eht V \textit{om.} T \textbf{18} Die hie so sere ist geschlag en W · sine wêre so sere niht geslagen T  $\cdot$ hie ist] ist hie V \textbf{19} wan si was] Doch was si T  $\cdot$ künegîn] vurstin T \textbf{20} Lehelin] lehalin W \textbf{22} minre wæren] minre were V wêre niht T \textbf{23} Do versweig in Antschanor U  $\cdot$ Antenor] anthenor W \textbf{26} gezilt] gezelt U \textbf{27} niemer] niemans U \textbf{28} si lachete] Sv́ enlachete V (W) (T)  $\cdot$ dâ] do U W \textbf{29} was] do was W \textbf{30} Key] keyn V keyen W \newline
\end{minipage}
\end{table}
\end{document}
