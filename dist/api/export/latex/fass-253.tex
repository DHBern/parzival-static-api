\documentclass[8pt,a4paper,notitlepage]{article}
\usepackage{fullpage}
\usepackage{ulem}
\usepackage{xltxtra}
\usepackage{datetime}
\renewcommand{\dateseparator}{.}
\dmyyyydate
\usepackage{fancyhdr}
\usepackage{ifthen}
\pagestyle{fancy}
\fancyhf{}
\renewcommand{\headrulewidth}{0pt}
\fancyfoot[L]{\ifthenelse{\value{page}=1}{\today, \currenttime{} Uhr}{}}
\begin{document}
\begin{table}[ht]
\begin{minipage}[t]{0.5\linewidth}
\small
\begin{center}*D
\end{center}
\begin{tabular}{rl}
\textbf{253} & \begin{large}D\end{large}es ist dîn houbet blôz \textbf{getân}.\\ 
 & \textbf{z}em fôrest \textbf{in} Brizljan\\ 
 & sach ich dich dô \textbf{vil} minneclîch,\\ 
 & swie dû wærest jâmers rîch.\\ 
5 & dû hâst verlorn varwe \textbf{unt} kraft.\\ 
 & dîner herten geselleschaft\\ 
 & verdrüzze mich, solt ich \textbf{die} haben.\\ 
 & wir sulen \textbf{disen} tôten man begraben."\\ 
 & Dô nazzeten \textbf{diu} ougen \textbf{ir die} wât.\\ 
10 & ouch was vroun Luneten rât\\ 
 & ninder dâ bî ir gewesen.\\ 
 & diu riet ir vrouwen: "lât genesen\\ 
 & disen man, der den iweren sluoc.\\ 
 & er mac \textbf{ergetzen iuch} genuoc."\\ 
15 & Sigune gerte ergetzens niht\\ 
 & als wîp, die man bî wanke siht,\\ 
 & \textbf{manege}, der \textbf{ich wil} gedagen.\\ 
 & hœret mêr Sigunen triwe sagen.\\ 
 & \textbf{diu} sprach: "sol mich iht gevröun,\\ 
20 & daz \textbf{tuot} ein dinc, ob \textbf{in} sîn töun\\ 
 & læzet, \textbf{den} vil \textbf{trûrigen} man.\\ 
 & schiede dû helflîche dan,\\ 
 & sô ist dîn lîp wol prîses wert.\\ 
 & dû vüerest ouch umbe dich sîn swert.\\ 
25 & \textbf{bekennestû} des swertes segen,\\ 
 & dû maht ân angest strîtes pflegen.\\ 
 & \begin{large}S\end{large}în ecke ligent im rehte.\\ 
 & von edelem geslehte\\ 
 & worht \textbf{ez} Trebuchetes hant.\\ 
30 & ein brunne stêt bî Karnant.\\ 
\end{tabular}
\scriptsize
\line(1,0){75} \newline
D \newline
\line(1,0){75} \newline
\textbf{1} \textit{Initiale} D  \textbf{9} \textit{Majuskel} D  \textbf{27} \textit{Initiale} D  \newline
\line(1,0){75} \newline
\textbf{2} Brizljan] Prizlian D \textbf{10} Luneten] Lunetten D \textbf{29} Trebuchetes] Trebvchets D \newline
\end{minipage}
\hspace{0.5cm}
\begin{minipage}[t]{0.5\linewidth}
\small
\begin{center}*m
\end{center}
\begin{tabular}{rl}
 & des ist dîn houbet blôz \textbf{getân}.\\ 
 & \textbf{z}em fôrest \textbf{in} Pricilan\\ 
 & sach ich dich dâ minneclîch,\\ 
 & wie dû wærest jâmers rîch.\\ 
5 & dû hâst verlorn varwe \textbf{und} kraft.\\ 
 & dîner herten geselleschaft\\ 
 & verdrüzze mich, solt ich \textbf{die} haben.\\ 
 & wir sulen \textbf{disen} tôten man begraben."\\ 
 & \begin{large}D\end{large}ô nazzeten \textbf{diu} ougen \textbf{ir die} wât.\\ 
10 & ouch was vrouwen L\textit{u}neten rât\\ 
 & \textit{ni}ender d\textit{â} bî ir gewesen.\\ 
 & diu riet ir vrouwen: "lât genesen\\ 
 & disen man, der den iuweren sluoc.\\ 
 & er mac \textbf{es} \textbf{ergetzen iuch} genuoc."\\ 
15 & Sigune gerte ergetzens niht\\ 
 & als wîp, die man bî wanke siht,\\ 
 & \textbf{manige}, der \textbf{ich wil} gedagen.\\ 
 & hœret mê Sigunen triuwe sagen.\\ 
 & \textbf{diu} sprach: "sol mich iht gevrouwen,\\ 
20 & daz \textbf{tuot} ein dinc, ob \textbf{in} sîn touwen\\ 
 & lâzet, \textbf{den} vil \textbf{trûrigen} man.\\ 
 & schiedû helflîch\textit{e} dan,\\ 
 & sô ist dîn \textit{l}î\textit{p} wol prîses wert.\\ 
 & dû vüerest ouch umb \textit{d}ich sîn swert.\\ 
25 & \textbf{bekennest dû} des swertes segen,\\ 
 & dû maht âne angest strîtes pflegen.\\ 
 & sî\textit{n} ecke ligent ime rehte.\\ 
 & von edelem geslehte\\ 
 & worhte\textbf{z} Trebuchetes hant.\\ 
30 & ein brunne stât bî Karnant.\\ 
\end{tabular}
\scriptsize
\line(1,0){75} \newline
m n o Fr69 \newline
\line(1,0){75} \newline
\textbf{9} \textit{Initiale} m Fr69   $\cdot$ \textit{Capitulumzeichen} n  \newline
\line(1,0){75} \newline
\textbf{2} fôrest] forast o  $\cdot$ Pricilan] bricilan m n britane o \textbf{3} dâ] do vil n so vil o \textbf{5} dû] [Da]: Dv m \textbf{6} herten] herren n (o) \textbf{8} disen] den Fr69 \textbf{9} nazzeten] nacztet o \textbf{10} vrouwen] frouwe m (n) (o)  $\cdot$ Luneten] lonneten m lẏmeten o \textbf{11} niender] Mender m Niergent n  $\cdot$ dâ] do m n o \textbf{12} diu] Sie o  $\cdot$ vrouwen] frouwe n (o) \textbf{13} iuweren] iren m n eren o \textbf{14} es] \textit{om.} Fr69 \textbf{15} Sigune] Syguͯne o  $\cdot$ gerte] gert o \textbf{16} wanke] wanckes o \textbf{17} ich wil gedagen] uch wil betagen o \textbf{18} Sigunen] [sg]: [sig*nni]: sigunen n \textbf{19} diu] Sú n \textit{om.} o \textbf{20} ob] das n  $\cdot$ touwen] tougen n tavmm o \textbf{21} trûrigen] truwen n \textbf{22} \textit{Versdoppelung (mit Anteil aus Vers 253.21):} Schiede duͯ helffeklichen mann / Schiede duͯ helffekliche dann o   $\cdot$ helflîche] helflichv m \textbf{23} lîp] pris m  $\cdot$ prîses] gepriset o \textbf{24} dich] mich m \textbf{25} segen] sehen o \textbf{26} angest] [gast]: angast o \textbf{27} sîn] Sine m  $\cdot$ ligent] liget o \textbf{29} Trebuchetes] trebuchtes n trebucetez o \textbf{30} Karnant] karnancz o \newline
\end{minipage}
\end{table}
\newpage
\begin{table}[ht]
\begin{minipage}[t]{0.5\linewidth}
\small
\begin{center}*G
\end{center}
\begin{tabular}{rl}
 & \begin{large}D\end{large}es ist dîn houbt blôz \textbf{gestân}.\\ 
 & \textbf{in} dem fôreis \textbf{ze} Brizilan\\ 
 & sach ich dich dô \textbf{vil} minniclîch,\\ 
 & swie dû wærst jâmers rîch.\\ 
5 & dû hâst verlorn varwe \textbf{unde} kraft.\\ 
 & dîner herten geselleschaft\\ 
 & verdrüzze mich, solt ich \textbf{si} haben.\\ 
 & wir sulen \textbf{den} tôten man begraben."\\ 
 & dô nazte\textit{n} \textbf{\textit{d}iu} ougen \textbf{ir} wât.\\ 
10 & ouch was vrôn Luneten rât\\ 
 & ninder dâ bî ir gewesen.\\ 
 & diu riet \textit{ir} vrouwe\textit{n}: "lât genesen\\ 
 & disen man, der \textbf{iu} den iwern sluoc.\\ 
 & er mag \textbf{ergetzen iuch} genuoc."\\ 
15 & Si\textit{gu}ne gerte ergetzens niht\\ 
 & als wîp, die man bî wanke siht,\\ 
 & \textbf{maniger}, der \textbf{ich wil} gedagen.\\ 
 & hœret mêr \textbf{von} Sigunen \textit{triwen} sagen.\\ 
 & \textbf{si} sprach: "sol mich iht gevröuwen,\\ 
20 & daz \textbf{ist} ein dinc, op sîn töuwen\\ 
 & lât \textbf{de\textit{n}} vil \textbf{\textit{t}r\textit{ûr}i\textit{g}e\textit{n}} man.\\ 
 & schiede dû helfeclîche dan,\\ 
 & sô ist dîn lîp wol brîses wert.\\ 
 & dû vüerst ouch umbe dich sîn swert.\\ 
25 & \textbf{hâstû gelernt} des swertes segen,\\ 
 & dû maht âne angest strîtes pflegen.\\ 
 & sîn ecke ligent im rehte.\\ 
 & von edelem geslehte\\ 
 & worhte\textbf{z} Trebuchetes hant.\\ 
30 & ein brunne stêt bî Karnant.\\ 
\end{tabular}
\scriptsize
\line(1,0){75} \newline
G I O L M Q R Z Fr21 Fr40 Fr51 \newline
\line(1,0){75} \newline
\textbf{1} \textit{Initiale} G  \textbf{9} \textit{Initiale} O L Q R Fr21 Fr40  \textbf{15} \textit{Initiale} Z  \textbf{19} \textit{Initiale} I  \newline
\line(1,0){75} \newline
\textbf{1} dîn] \textit{om.} Fr51  $\cdot$ gestân] bistan M gethon Q (R) (Z) (Fr40) \textbf{2} [z*st]: zu dem forst in brezzilian Fr40  $\cdot$ in dem] Zem O (M) Q R (Z) (Fr51)  $\cdot$ ze Brizilan] zebrizilan G ze brizilian I in Breziliam O in Brecilian L M in brezzilian Q Z in Breczilion R in breizilian Fr21 in britzylian Fr51 \textbf{3} dô] da M \textit{om.} Q R Fr51 doch Z Fr40 \textbf{4} swie] Wie L (Q) Z Fr51 Schwig R  $\cdot$ wærst] wirst M  $\cdot$ jâmers rîch] yemerliche Q \textbf{5} verlorn] verlon R \textbf{6} dîner herten] Dines hertzen L Deiner herte Q \textbf{7} verdrüzze] Wer druffe Q  $\cdot$ mich] mir Fr51  $\cdot$ solt] sol Fr40 \textbf{8} tôten] totte R  $\cdot$ begraben] begrab Z \textbf{9} dô] ÷o O So M Da Z  $\cdot$ nazten diu] natzten ir div G natzten O (M) (R) Fr21 (Fr40) nastet Q natzent die Fr51  $\cdot$ wât] di wat O (L) (M) (Q) (R) (Z) (Fr21) Fr40 \textbf{10} was] ne hatte Fr51  $\cdot$ vrôn] frov L (M) (R)  $\cdot$ Luneten] lunetzen Q lun enten R lunetten Fr51 \textbf{11} ninder] Nirgen M Nicht Fr51  $\cdot$ dâ] do Q  $\cdot$ ir] \textit{om.} M R ye Q \textbf{12} riet] reyt Fr51  $\cdot$ ir vrouwen] froͮwe G ir frowe Z \textbf{13} iu den] \textit{om.} O den L M Q R Z Fr40 Fr51  $\cdot$ iwern] viben Fr51 \textbf{14} da von ir herze iamer truͤc I  $\cdot$ ergetzen] ergesszin M vor gelden Fr51 \textbf{15} er mach ergezzen evch sin nih I  $\cdot$ Sigune] sine G Sýgvne L Sygvne M Sie guͯne Q Sygunde R Sẏgune Fr51  $\cdot$ gerte] gert O Q R Z  $\cdot$ ergetzens] geldes Fr51 \textbf{16} als] Jez Z  $\cdot$ die man] die ma Fr21 deman Fr51  $\cdot$ bî] in O  $\cdot$ wanke] wandils M wanken R Z Fr40 \textbf{17} maniger] Mange L (Z) (Fr40)  $\cdot$ wil] vil R wole Fr51  $\cdot$ gedagen] bedagen Q dagen Fr51 \textbf{18} hœret] Vnd L  $\cdot$ mêr] mir M  $\cdot$ Sigunen] sygvne O sygvnen L (R) sẏngvne Fr21  $\cdot$ triwen] \textit{om.} G trúwe R (Z) \textbf{19} mich] mir Fr51  $\cdot$ iht] ich R  $\cdot$ gevröuwen] vrowen Fr51 \textbf{20} ist] tut Q Fr40  $\cdot$ ein dinc] \textit{om.} Fr51  $\cdot$ op sîn] daz in sin O ob mich sin L uff yn sin M ob in sein Q (Z) Fr40 (Fr51) sol in sin R daz ich sin Fr21  $\cdot$ töuwen] deung R entovn Fr21 \textbf{21} lât] Laszen L Laz Fr51  $\cdot$ den vil trûrigen] der vil getriwe G den vil getruͯwen L den lib vil trurrigen R \textbf{22} schiede dû] Scheide du R (Fr21) Vorestu Fr51  $\cdot$ helfeclîche] heflichen I hvfslichen O hoffelichen M (Fr21) \textbf{24} vüerst] hast Fr51  $\cdot$ ouch] doch R  $\cdot$ dich] [din]: dich G \textbf{25} gelernt] geleret Fr40 Fr51 \textbf{26} \textit{Versfolge 253.27-28-29-30-26} Q   $\cdot$ dû maht] so maht du I So maht O  $\cdot$ strîtes] wol strites Z swertes Fr40 \textbf{27} sîn ecke] Sine eckin M  $\cdot$ ligent im] im ligent I licht im Fr51 \textbf{28} edelem] edelen Fr51 \textbf{29} worhtez] Machtiz Fr51  $\cdot$ Trebuchetes] Trebuchundes I Trebvrchetes O Trebuͯchetes L tribuchetis M Trebuketes Q Trebucketes R trebukeres Fr40 trebucetes Fr51 \textbf{30} Karnant] charnant I karrant O \newline
\end{minipage}
\hspace{0.5cm}
\begin{minipage}[t]{0.5\linewidth}
\small
\begin{center}*T
\end{center}
\begin{tabular}{rl}
 & des ist dîn houbet blôz \textbf{getân}.\\ 
 & \textbf{z}em fôreht \textbf{in} Prezilian\\ 
 & sach ich dich dô \textbf{vil} minneclîch,\\ 
 & swie dû wære\textit{st} jâmers rîch.\\ 
5 & dû hâst verlorn varwe kraft.\\ 
 & dîner herten geselleschaft\\ 
 & verdrüzze mich, solt ich \textbf{si} haben.\\ 
 & wir suln \textbf{disen} tôten man begraben."\\ 
 & Dô nazten \textbf{ir} ougen \textbf{die} wât.\\ 
10 & ouch \textbf{en}was vroun Luneten rât\\ 
 & niender dâ bî ir gewesen.\\ 
 & diu riet ir vrouwen: "lât genesen\\ 
 & disen man, der den iuwern sluoc.\\ 
 & er mac \textbf{iuch sîn ergetzen} genuoc."\\ 
15 & Sygune gerte ergetzens niht\\ 
 & al\textit{s} wîp, die man bî wanke siht,\\ 
 & \textbf{genuoge}, der \textbf{wil ich} gedagen.\\ 
 & hœret mê \textbf{von} Sygunen triuwe sagen.\\ 
 & \textbf{Si} sprach: "\textbf{unde} sol mich iht gevröun,\\ 
20 & daz \textbf{ist} ein dinc, ob \textbf{in} sîn töun\\ 
 & lât, \textbf{den} vil \textbf{getriuwen} man.\\ 
 & schiedû helfeclîche dan,\\ 
 & sô ist dîn lîp wol prîses wert.\\ 
 & dû vüerest ouch umb dich sîn swert.\\ 
25 & \textbf{hâstû gelernet} des swertes segen,\\ 
 & dû maht âne angest strîtes pflegen.\\ 
 & Sîn ecke ligent im rehte.\\ 
 & von edelme geslehte\\ 
 & worhte\textbf{s} Trebuketes hant.\\ 
30 & ein brunne stêt bî Garnant.\\ 
\end{tabular}
\scriptsize
\line(1,0){75} \newline
T U V W Fr26 \newline
\line(1,0){75} \newline
\textbf{9} \textit{Majuskel} T  \textbf{15} \textit{Majuskel} T  \textbf{19} \textit{Majuskel} T  \textbf{27} \textit{Initiale} W Fr26   $\cdot$ \textit{Majuskel} T  \newline
\line(1,0){75} \newline
\textbf{2} in] zuͦ W  $\cdot$ Prezilian] Precilian U (V) priz::: Fr26 \textbf{3} dô] \textit{om.} W \textbf{4} swie] Wie U W  $\cdot$ wærest] wêrez T \textbf{5} kraft] vnd kraft U (V) (W) \textbf{6} geselleschaft] geseschaft U \textbf{7} verdrüzze] verdruͦzzen U  $\cdot$ si haben] sie sehe U [*haben]: die haben V \textbf{9} ir ougen die wât] ir ovgen div wât T die auͦgen ir wat U die oͮgen ir [*]: die wat V die augen ir die wat W \textbf{10} enwas] waz V  $\cdot$ vroun] vrauͦwe U (W)  $\cdot$ Luneten] [Lut*]: Luneten U lunten W \textbf{11} niender] Niergent V  $\cdot$ dâ bî ir] do bei W \textbf{12} diu] die T \textbf{13} man] wann W  $\cdot$ den] \textit{om.} W \textbf{14} iuch sîn] îv sin T sein eúch W \textbf{15} Sygune] Sigvne T Syguͦne U Sẏgvne V  $\cdot$ gerte ergetzens] vergessens gerte W \textbf{16} als] al T \textbf{17} genuoge der] Gnuͦgen die U [*]: Manige der V  $\cdot$ gedagen] betagen W \textbf{18} von] \textit{om.} W  $\cdot$ Sygunen] Syguͦnen U sigunen W (Fr26) \textbf{20} ist] tuͦt V  $\cdot$ in] im W \textbf{21} getriuwen] [*]: trurigen V traurigen W (Fr26) \textbf{22} schiedû] scheide du U \textbf{24} vüerest] gortes U fuͤret W  $\cdot$ ouch] doch W Fr26  $\cdot$ sîn swert] ein schwet W \textbf{26} maht] muͦst U \textbf{27} Sîn] Sine U Fr26  $\cdot$ rehte] schlechte W \textbf{28} Ich sage dirs mit rechte W  $\cdot$ geslehte] gesiehte U \textbf{29} worhtes] Worhte ez V Es worchte W  $\cdot$ Trebuketes] trebuchidis W \textbf{30} Garnant] karvant U [*]: karnant V granant W \newline
\end{minipage}
\end{table}
\end{document}
