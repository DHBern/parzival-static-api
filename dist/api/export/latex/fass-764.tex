\documentclass[8pt,a4paper,notitlepage]{article}
\usepackage{fullpage}
\usepackage{ulem}
\usepackage{xltxtra}
\usepackage{datetime}
\renewcommand{\dateseparator}{.}
\dmyyyydate
\usepackage{fancyhdr}
\usepackage{ifthen}
\pagestyle{fancy}
\fancyhf{}
\renewcommand{\headrulewidth}{0pt}
\fancyfoot[L]{\ifthenelse{\value{page}=1}{\today, \currenttime{} Uhr}{}}
\begin{document}
\begin{table}[ht]
\begin{minipage}[t]{0.5\linewidth}
\small
\begin{center}*D
\end{center}
\begin{tabular}{rl}
\textbf{764} & \begin{large}D\end{large}er wirt \textbf{unt} Jofreit;\\ 
 & etswenne \textbf{ich ouch} den prîs erstreit,\\ 
 & daz man mîn drüber gerte,\\ 
 & des ich si dô gewerte."\\ 
5 & Si nâmen \textbf{diu tischlachen} dan\\ 
 & vor \textbf{al} den vrouwen unt \textbf{vor} den man.\\ 
 & des was zît. dô man geaz,\\ 
 & Gawan, der wirt, niht langer saz.\\ 
 & Die herzogîn unt \textbf{ouch} sîn anen\\ 
10 & begunde\textit{r} bitten und manen,\\ 
 & daz si Sangiven \textbf{ê}\\ 
 & unt die süezen Cundrie\\ 
 & næmen unt giengen dar,\\ 
 & \textbf{al}dâ der \textbf{heiden} \textbf{buntgevar}\\ 
15 & saz, unt daz si pflægen sîn.\\ 
 & Feirefiz Anschevin\\ 
 & sach \textbf{dise} vrouwen gein im gên.\\ 
 & gein den begunder ûf \textbf{dô} stên.\\ 
 & als tet sîn bruoder Parzival.\\ 
20 & diu herzogîn lieht gemâl\\ 
 & nam Feirefizen \textbf{mit} \textbf{ir} hant.\\ 
 & swaz si vrouwen unt rîter \textbf{stên} \textbf{dâ} vant,\\ 
 & die \textbf{bâten} si sitzen alle.\\ 
 & dô reit dar zuo mit schalle\\ 
25 & Artus mit den sînen.\\ 
 & man hôrt dâ pusînen,\\ 
 & tambûren, \textbf{floitieren}, stîven.\\ 
 & der sun Arniven\\ 
 & reit dar zuo mit krache.\\ 
30 & dirre vrœlîchen sache\\ 
\end{tabular}
\scriptsize
\line(1,0){75} \newline
D Fr12 \newline
\line(1,0){75} \newline
\textbf{1} \textit{Initiale} D  \textbf{5} \textit{Majuskel} D  \textbf{9} \textit{Majuskel} D  \newline
\line(1,0){75} \newline
\textbf{10} begunder] begvnden D \textbf{11} Sangiven] Sangîven D \textbf{12} Cundrie] Cvndrîe D \textbf{16} Anschevin] Anscivin D \textbf{19} Parzival] Parcifal D \textbf{28} der] den Fr12  $\cdot$ Arniven] Arnîven D \newline
\end{minipage}
\hspace{0.5cm}
\begin{minipage}[t]{0.5\linewidth}
\small
\begin{center}*m
\end{center}
\begin{tabular}{rl}
 & der wirt \textbf{und} Jofr\textit{e}it;\\ 
 & etw\textit{a}n \textbf{ich ouch} den prîs erstr\textit{e}it,\\ 
 & daz man mîn dar über gert,\\ 
 & des ich si dô gewert."\\ 
5 & \begin{large}S\end{large}i n\textit{â}men \textbf{diu tischlachen} dan\\ 
 & vo\textit{r} den vrowen und d\textit{en} man.\\ 
 & des was zît. dô man geaz,\\ 
 & Gawan, der wirt, niht lenger saz.\\ 
 & die herzogîn und sîn anen\\ 
10 & begunde er bitten und manen,\\ 
 & daz si San\textit{g}i\textit{v}en \textbf{ê}\\ 
 & und die süezen Condrie\\ 
 & næmen und giengen dar,\\ 
 & d\textit{â} der \textbf{heiden} \textbf{buntgevar}\\ 
15 & saz, und daz si pfl\textit{æ}gen sîn.\\ 
 & Ferefiz A\textit{n}schevin\\ 
 & sach \textbf{dise} vrowen gegen im gân.\\ 
 & gegen den begunde er ûf stân.\\ 
 & als tet sîn bruoder Pa\textit{r}cifal.\\ 
20 & diu herzogîn liehtgemâl\\ 
 & nam Ferefizen \textbf{bî} \textbf{der} hant.\\ 
 & waz si vrowen und ritter \textbf{stên} vant,\\ 
 & die \textbf{bat} si sitzen alle.\\ 
 & dô reit dar zuo mit schalle\\ 
25 & Artus mit den sînen.\\ 
 & man hôrte d\textit{â} busînen,\\ 
 & \textit{t}ambûren, \textbf{flöuten}, s\textit{t}îven.\\ 
 & der \textbf{werde} sun Ar\textit{niv}en\\ 
 & reit dar zuo mit krach.\\ 
30 & diser vr\textit{œ}lîchen sach\\ 
\end{tabular}
\scriptsize
\line(1,0){75} \newline
m n o V V' W \newline
\line(1,0){75} \newline
\textbf{5} \textit{Illustration mit Überschrift:} Also gawan vnd die schoͯnen frowen gingent fuͯr den heiden pontgefar mit einem grossen geschele m  Also gawan vnd die schoͯnen frowen gossent vnd gingent fúr den heiden bungefar mit eime grossen geschelle n   $\cdot$ \textit{Großinitiale} n   $\cdot$ \textit{Initiale} m V  \textbf{25} \textit{Initiale} W  \newline
\line(1,0){75} \newline
\textbf{1} \textit{Die Verse 762.22-764.27 fehlen} o   $\cdot$ \textit{Die Verse 762.3-764.22 fehlen} V'   $\cdot$ Jofreit] jofrit m Joffreit V iofrit W \textbf{2} etwan] Etwatn m  $\cdot$ erstreit] erstrit m W \textbf{3} mîn dar über gert] [*]: min drúber gerte V \textbf{4} gewert] gewerte V \textbf{5} nâmen] nemen m n \textbf{6} vor] Von m Vor alle W  $\cdot$ den man] die man m man n W \textbf{7} dô] als W \textbf{8} niht] do nit n \textbf{9} sîn] auch sein W \textbf{11} Sangiven] sangwinen m n Sagiuen V seyuen W \textbf{12} süezen] suͯsse n  $\cdot$ Condrie] kvndrie V (W) \textbf{13} næmen] Genemen n \textbf{14} dâ] Do m n V W  $\cdot$ buntgevar] wunt gefar W \textbf{15} pflægen] pflagen m W pflugen n \textbf{16} Ferefiz] Ferefis m Ferrefis n Ferevis V Ferafiß W  $\cdot$ Anschevin] auscevin m n anschefin V antscheuein W \textbf{17} dise] die V W \textbf{18} ûf] uf do V \textbf{19} Parcifal] paricifal m parzefal V herr partzifal W \textbf{21} Ferefizen] fere fizen m ferrefizen n ferefisen V ferafissen W  $\cdot$ bî] mit n V \textbf{22} waz] Swaz V  $\cdot$ stên] stende V \textbf{23} \textit{statt 764.23-25:} Der kvnig bereitte sich zu hant / Vnd reit do mit schalle / Do er sy bie ein ander uant mit alle V'  \textbf{24} dar zuo] sy hin zuͦ W \textbf{25} Artus] aRtuse W \textbf{26} \textit{statt 764.26-28:} Man horte bosunen vnd pfiffen / [Den]: Der werde svn arnifen V'   $\cdot$ dâ] do m n V W \textbf{27} tambûren] Campuren m  $\cdot$ flöuten] floͤytieren W  $\cdot$ stîven] siwen m friwen n stiren W \textbf{28} werde] werden W  $\cdot$ Arniven] arunen m arniwen n arnifen V arnyuen W \textbf{29} krach] rache V' \textbf{30} vrœlîchen] froͯwelichen m werlichen V' \newline
\end{minipage}
\end{table}
\newpage
\begin{table}[ht]
\begin{minipage}[t]{0.5\linewidth}
\small
\begin{center}*G
\end{center}
\begin{tabular}{rl}
 & \begin{large}D\end{large}er wirt \textbf{sprach ze} Jofreit:\\ 
 & "eteswenne \textbf{ich ouch} den brîs erstreit,\\ 
 & daz man mîn drüber gerte,\\ 
 & des ich si dô gewerte."\\ 
5 & si nâmen \textbf{die tische} dan\\ 
 & vor \textit{\textbf{al}} de\textit{n} vrouwen unde \textbf{vor} de\textit{n} man.\\ 
 & des was zît. dô man geaz,\\ 
 & Gawan, der wirt, niht langer saz.\\ 
 & die herzoginne unde \textbf{ouch} sîn anen\\ 
10 & begunder biten unde manen,\\ 
 & daz si Sagive\\ 
 & unde die süezen Gundrie\\ 
 & næmen unde giengen dar,\\ 
 & \textbf{al}dâ der \textbf{blancgevar}\\ 
15 & saz, unde daz si pflægen sîn.\\ 
 & Feirafiz Antschevin\\ 
 & sach \textbf{dise} vrouwen gein im gên.\\ 
 & gein den begunder ûf stên.\\ 
 & als tet sîn bruoder Parzival.\\ 
20 & diu herzoginne lieht gemâl\\ 
 & nam Feirafizen \textbf{mit} \textbf{der} hant.\\ 
 & swaz si vrouwen unde rîter vant\\ 
 & \textbf{stên}, die \textbf{bat} si sitzen alle.\\ 
 & dô reit dar zuo mit schalle\\ 
25 & Artus mit den sînen.\\ 
 & man hôrte dâ busînen,\\ 
 & tambûren, \textbf{floitiern}, stîven.\\ 
 & der sun Arniven\\ 
 & reit dar zuo mit krache.\\ 
30 & dirre vrœlîchen sache\\ 
\end{tabular}
\scriptsize
\line(1,0){75} \newline
G I L M Z Fr45 \newline
\line(1,0){75} \newline
\textbf{1} \textit{Initiale} G L  \textbf{5} \textit{Initiale} Z  \textbf{9} \textit{Initiale} I  \newline
\line(1,0){75} \newline
\textbf{1} sprach ze] vnde der M vnd Z Fr45  $\cdot$ Jofreit] Iofreit G Jofreẏd Fr45 \textbf{2} den] \textit{om.} Z \textbf{3} daz] Da Z \textbf{4} si dô] so I sy da M (Z) \textbf{5} tische] týslachen L (M) (Z) (Fr45)  $\cdot$ dan] hin dan I \textbf{6} al den] der G den L  $\cdot$ vor den] vor dem G \textbf{7} dô] da M \textbf{8} Gawan] Gowan Fr45  $\cdot$ niht] do nih I \textbf{9} die] Diu I  $\cdot$ ouch] \textit{om.} L M  $\cdot$ anen] man L ane Fr45 \textbf{10} begunder] begunden I Bigonden ir M \textbf{11} Sagive] die chlaren saifen I zuͯ Sagiven L saiven nemen e M Seiven e Z Sagiuen namen e Fr45 \textbf{12} unde] vnde auch I (M) (Z) (Fr45)  $\cdot$ Gundrie] kvndrie G (Z) Fr45 Gundrien I kvndrien L \textbf{13} næmen] Namen L (M) o\textit{m. } Fr45  $\cdot$ dar] zvchticlichen dar Fr45 \textbf{14} aldâ] Da L (Fr45)  $\cdot$ der] der haiden I (L) (M) (Z) (Fr45)  $\cdot$ blancgevar] wech gevar M (Fr45) bvnt gevar Z \textbf{15} saz] Sasze L  $\cdot$ pflægen] pflagen L \textbf{16} Feirafiz] Ferefiz L Feirefisz M Feirefiz Z feyrafẏs Fr45  $\cdot$ Antschevin] Anschevin G antsheuin I Anshevin L (Z) ansevin M anschewin Fr45 \textbf{17} dise] die L (M) Fr45 \textbf{18} ûf] da uff M vf da Z \textbf{19} tet] der L  $\cdot$ Parzival] parcifal G Z parzifal I L M \textbf{20} gemâl] gimal M \textbf{21} nam] vnd I  $\cdot$ Feirafizen] ferefiz L feirefisz M feirefiz Z feẏrafiz Fr45  $\cdot$ mit der] an die I mit ir M \textbf{22} swaz] Waz L (M)  $\cdot$ si] \textit{om.} I  $\cdot$ unde] \textit{om.} L \textbf{23} stên] Sten da M (Z) Da sten Fr45  $\cdot$ si] da I \textbf{24} dô] Da M Z \textbf{26} busînen] uil businen I \textbf{27} \textit{Versfolge 764.28-27} Fr45   $\cdot$ tambûren] taburen Fr45  $\cdot$ floitiern] floẏten Fr45 \textbf{28} der kuniginne sun arniuen I  $\cdot$ Der broder Saguien Fr45 \textbf{30} dirre] der Fr45 \newline
\end{minipage}
\hspace{0.5cm}
\begin{minipage}[t]{0.5\linewidth}
\small
\begin{center}*T
\end{center}
\begin{tabular}{rl}
 & \begin{large}D\end{large}er wirt \textbf{und} Jofreit;\\ 
 & etswan \textbf{ouch ich} den prîs erstreit,\\ 
 & daz man mîn dar über gerte,\\ 
 & des ich si dô gewerte."\\ 
5 & si nâmen \textbf{tischlachen} dan\\ 
 & vor \textbf{al} den vrouwen und \textbf{vor} \textbf{al} den man.\\ 
 & des was zît. dô man geaz,\\ 
 & Gawan, der wirt, niht langer saz.\\ 
 & die herzogîn und \textbf{ouch} sîn anen\\ 
10 & begunder b\textit{i}ten und manen,\\ 
 & daz si Seyven \textbf{næmen} \textbf{ê}\\ 
 & und die süezen Kundrie\\ 
 & næmen und giengen dar,\\ 
 & \textbf{al} dâr der \textbf{heiden} \textbf{buntgevar}\\ 
15 & saz, und daz si pflægen sîn.\\ 
 & Ferefis \textbf{von} Anschevin\\ 
 & sach \textbf{die} vrouwen gein im gên.\\ 
 & gein den begunder ûf \textbf{dô} stên.\\ 
 & als te\textit{t} sîn bruoder Parcifal.\\ 
20 & diu herzoginne lieht gemâl\\ 
 & nam Ferefis \textbf{mit} \textbf{der} hant.\\ 
 & waz si vrouwen und rîter vant\\ 
 & \textbf{dâ} \textbf{stân}, die \textbf{bat} si sitzen alle.\\ 
 & dô reit dar zuo mit schalle\\ 
25 & Artus mit den sînen.\\ 
 & man hôrte dâ busînen,\\ 
 & tambûren, \textbf{floitieren}, stîven.\\ 
 & der sun Arnyven\\ 
 & reit dar zuo mit krache.\\ 
30 & dirre vrœlîchen sache\\ 
\end{tabular}
\scriptsize
\line(1,0){75} \newline
U Q R \newline
\line(1,0){75} \newline
\textbf{1} \textit{Initiale} U Q  \newline
\line(1,0){75} \newline
\textbf{3} mîn] \textit{om.} Q \textbf{5} tischlachen] die tislachen Q (R) \textbf{6} al den vrouwen] allen frow:: R  $\cdot$ und vor al den man] vnd man Q ::: man R \textbf{7} des] Das R \textbf{8} Gawan] Gawin R \textbf{9} und] \textit{om.} R \textbf{10} biten] bieten U \textbf{11} Seyven] Seynen U seynnen Q Senyuen R  $\cdot$ næmen] neme R \textbf{12} Kundrie] kuͦndrie U kondrie R \textbf{13} \textit{Die Verse 764.13-774.30 fehlen} R  \textbf{15} pflægen] pflagen Q \textbf{16} Ferefis von Anschevin U feirefisz anschevin Q \textbf{18} dô] \textit{om.} Q \textbf{19} tet] der U  $\cdot$ Parcifal] partzifal Q \textbf{20} lieht] licht Q \textbf{21} Ferefis] feirefisz Q  $\cdot$ der] ir Q \textbf{23} dâ] Do Q \textbf{25} sînen] sinnen Q \textbf{26} dâ] do Q \textbf{28} Arnyven] [arni]: arnyven Q \textbf{30} vrœlîchen] frawliche Q \newline
\end{minipage}
\end{table}
\end{document}
