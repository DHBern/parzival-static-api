\documentclass[8pt,a4paper,notitlepage]{article}
\usepackage{fullpage}
\usepackage{ulem}
\usepackage{xltxtra}
\usepackage{datetime}
\renewcommand{\dateseparator}{.}
\dmyyyydate
\usepackage{fancyhdr}
\usepackage{ifthen}
\pagestyle{fancy}
\fancyhf{}
\renewcommand{\headrulewidth}{0pt}
\fancyfoot[L]{\ifthenelse{\value{page}=1}{\today, \currenttime{} Uhr}{}}
\begin{document}
\begin{table}[ht]
\begin{minipage}[t]{0.5\linewidth}
\small
\begin{center}*D
\end{center}
\begin{tabular}{rl}
\textbf{410} & kunde ir lîp \textbf{vil} wol gereizen.\\ 
 & ir engesâhet nie âmeizen,\\ 
 & diu bezzers gelenkes pflac,\\ 
 & denne si was, dâ der gürtel lac:\\ 
5 & \begin{large}D\end{large}az gab ir gesellen\\ 
 & Gawane manlîch ellen.\\ 
 & si \textbf{tûwerte} mit im in der nôt.\\ 
 & sîn benantez gîsel was der tôt\\ 
 & unt anders dechein gedinge.\\ 
10 & Gawanen wac vil ringe\\ 
 & vîende haz, \textbf{swenn} er die magt erkôs.\\ 
 & dâ von ir vil den lîp verlôs.\\ 
 & Dô kom der künec Vergulaht.\\ 
 & der sach die strîteclîchen maht\\ 
15 & gegen Gawane kriegen.\\ 
 & ich en\textbf{wolt} iuch \textbf{denne} triegen,\\ 
 & sône mag ich in niht beschœnen,\\ 
 & ern welle sich selben hœnen\\ 
 & an sînem werden gaste.\\ 
20 & der stuont ze wer al vaste.\\ 
 & Dô tet der wirt selbe schîn,\\ 
 & daz mich riwet Gandin,\\ 
 & der künec von Anschouwe,\\ 
 & daz ein sô werdiu vrouwe,\\ 
25 & sîn tohter, ie den sun gebar,\\ 
 & der mit ungetriwer schar\\ 
 & sîn volc bat sêre strîten.\\ 
 & Gawan muose bîten,\\ 
 & unze der künec gewâpent wart.\\ 
30 & \textbf{er} huop sich selbe an strîtes vart.\\ 
\end{tabular}
\scriptsize
\line(1,0){75} \newline
D \newline
\line(1,0){75} \newline
\textbf{5} \textit{Initiale} D  \textbf{13} \textit{Majuskel} D  \textbf{21} \textit{Initiale} D  \newline
\line(1,0){75} \newline
\textbf{13} Vergulaht] Vergvlaht D \textbf{23} Anschouwe] Anscoͮwe D \newline
\end{minipage}
\hspace{0.5cm}
\begin{minipage}[t]{0.5\linewidth}
\small
\begin{center}*m
\end{center}
\begin{tabular}{rl}
 & kunde ir lîp wol gereizen.\\ 
 & ir engesâhet nie âmeizen,\\ 
 & diu bezzers gelenkes pflac,\\ 
 & danne si was, d\textit{â} der gürtel lac:\\ 
5 & daz gap ir gesellen\\ 
 & Gawane manlîch ellen.\\ 
 & si \textbf{dûrte} mit ime in der nôt.\\ 
 & sîn benantez gîsel was der tôt\\ 
 & und anders enkein gedinge.\\ 
10 & Gawanen wac vil ringe\\ 
 & vîende haz, \textbf{wa\textit{n}} er die magt erkôs.\\ 
 & dâ von ir vil den lîp verlôs.\\ 
 & \begin{large}D\end{large}ô kom der künic Vergulaht.\\ 
 & der sach die strîteclîchen maht\\ 
15 & gege\textit{n} Gawane k\textit{r}iegen.\\ 
 & ich en\textbf{wolt} iuch \textbf{gerne} triegen,\\ 
 & sô enmac ich in \textit{n}iht besch\textit{œn}en,\\ 
 & er enwelle sich selben h\textit{œn}en\\ 
 & an sînem werden gaste.\\ 
20 & der stuont ze wer alvaste.\\ 
 & dô tet der wirt selbe schîn,\\ 
 & daz mich riuwet Gandin,\\ 
 & der künic von Anschouwe,\\ 
 & daz ein sô werdiu vrouwe,\\ 
25 & sîn tohter, ie den sun gebar,\\ 
 & der mit ungetriuwer schar\\ 
 & sîn volc bat sêre strîten.\\ 
 & Gawan muose bîten,\\ 
 & unz der künic gewâpent wart.\\ 
30 & \textbf{er} huop sich selbe an strîtes vart.\\ 
\end{tabular}
\scriptsize
\line(1,0){75} \newline
m n o \newline
\line(1,0){75} \newline
\textbf{13} \textit{Initiale} m n  \newline
\line(1,0){75} \newline
\textbf{1} kunde] Kuͯnde o \textbf{2} engesâhet] gesohen n gesohent o \textbf{3} gelenkes] glúckes n \textbf{4} dâ] do m n o \textbf{5} ir] iren o \textbf{6} Gawane] Gawanen n o  $\cdot$ manlîch] manig n o \textbf{8} der] sin n \textbf{9} enkein] do kein n \textbf{10} Gawanen] Gawane n Gawan o \textbf{11} vîende] Vigendes n (o)  $\cdot$ wan] was m \textbf{13} Vergulaht] vergulacht n verguͯlaht o \textbf{14} sach] [sc*]: schach o  $\cdot$ strîteclîchen] strittelich n dirte o  $\cdot$ maht] maget o \textbf{15} gegen] Gege m  $\cdot$ Gawane] gawanen o  $\cdot$ kriegen] kiegen m \textbf{16} gerne] denne n (o) \textbf{17} enmac] mag n o  $\cdot$ niht] iht m  $\cdot$ beschœnen] beschouwen m (o) \textbf{18} enwelle] welle n o  $\cdot$ sich] sich denne n  $\cdot$ selben] selb n selbe o  $\cdot$ hœnen] houwen m (o) \textbf{19} An sinen werden gast o \textbf{22} Gandin] gaudin n \textbf{23} Anschouwe] anscouwe m n ascowe o \textbf{28} muose] muͯsse m muͯste n \textbf{30} strîtes] die n o \newline
\end{minipage}
\end{table}
\newpage
\begin{table}[ht]
\begin{minipage}[t]{0.5\linewidth}
\small
\begin{center}*G
\end{center}
\begin{tabular}{rl}
 & kunde ir lîp wol gereizen.\\ 
 & irn gesâhet nie âmeizen,\\ 
 & diu bezzers gelenkes pflac,\\ 
 & dane si was, dâ der gürtel lac:\\ 
5 & daz gap ir gesellen\\ 
 & Gawane manlîch ellen.\\ 
 & si \textbf{trûrte} mit im in der nôt.\\ 
 & sîn benantez gîsel was der tôt\\ 
 & unde anders dehein gedinge.\\ 
10 & Gawanen wac vil ringe\\ 
 & vîende haz, \textbf{dô} er die maget erkôs.\\ 
 & dâ von ir vil den lîp verlôs.\\ 
 & dô kom der künic Vergulaht.\\ 
 & der sach die strîticlîchen maht\\ 
15 & gein Gawane kriegen.\\ 
 & ichne \textbf{welle} iuch \textbf{dane} triegen,\\ 
 & sône mag ich in niht beschœnen,\\ 
 & er enwelle sich selben hœnen\\ 
 & an sînem werden gaste.\\ 
20 & der stuont ze wer alvaste.\\ 
 & dô tet der wirt selbe schîn,\\ 
 & daz mich riwet Gandin,\\ 
 & der künic von Anschouwe,\\ 
 & daz ein sô werdiu vrouwe,\\ 
25 & sîn tohter, ie den sun gebar,\\ 
 & der mit ungetriwer schar\\ 
 & sîn volc bat sêre strîten.\\ 
 & Gawan muose bîten,\\ 
 & unze der künic gewâpent wart.\\ 
30 & \textbf{der} huop sich selbe an strîtes vart.\\ 
\end{tabular}
\scriptsize
\line(1,0){75} \newline
G I O L M Q R Z \newline
\line(1,0){75} \newline
\textbf{3} \textit{Initiale} O L Z  \textbf{5} \textit{Initiale} I  \textbf{13} \textit{Initiale} M  \textbf{19} \textit{Initiale} I  \newline
\line(1,0){75} \newline
\textbf{1} \textit{Die Verse 370.13-412.12 fehlen} Q   $\cdot$ wol] vil wol I O M Z  $\cdot$ gereizen] geczieren R \textbf{2} irn] ir I (O) (R)  $\cdot$ gesâhet] gesachen R  $\cdot$ âmeizen] amasieren R \textbf{3} diu] ÷iv O  $\cdot$ bezzers] bezzer I \textbf{4} dâ] da ir I do O \textbf{6} Gawane] Gawan I O Gawin R Gawanen Z  $\cdot$ manlîch] manich L menliche M \textbf{7} trûrte] rûrte I tvrte O (M) (Z)  $\cdot$ in] \textit{om.} L \textbf{8} gîsel was] was gisel isz M \textbf{9} dehein] keyne M deheins R \textbf{10} Gawanen] Gawan I O L M Z Gawine R  $\cdot$ wac vil] was gar R \textbf{11} vîende] Der viende R  $\cdot$ dô] swenn I (O) wenne L (M) (R) Z \textbf{12} dâ] daz I  $\cdot$ vil] vele M  $\cdot$ den] der L \textbf{13} dô] Da Z  $\cdot$ Vergulaht] virgulaht I vergvlaht L Z vergulacht M (R) \textbf{14} der] Vnd Z  $\cdot$ strîticlîchen] stritliche I \textbf{15} Gawane] Gawan I O (M) (Z) Gawaine R \textbf{16} ichne] ich I (O) (R)  $\cdot$ welle] wolt I (O) (L) (M) (R) (Z)  $\cdot$ iuch] in R  $\cdot$ triegen] betrigen R \textbf{17} sône] So O R Z  $\cdot$ ich in] ich O L in Z \textbf{18} enwelle] wil I welle O R (Z)  $\cdot$ selben] selber R selbe Z \textbf{19} werden] werdem I \textbf{20} alvaste] also faste R \textbf{21} dô] Da M Z Das R  $\cdot$ selbe] selbin M selber R \textbf{22} Gandin] Candin I \textbf{23} Anschouwe] [anschoe]: anschove I Anschawe O Anschowe L (M) (R) anshowe Z \textbf{24} sô werdiu] sin werde R \textbf{26} Den rúwtt die vngetrúwe schar R \textbf{27} bat] bar R  $\cdot$ sêre] \textit{om.} I \textbf{28} Gawan] Gawain R  $\cdot$ muose] muͤste I do muͯste L  $\cdot$ bîten] beten M \textbf{29} unze] Vsz M \textbf{30} der] Er O L M R Z  $\cdot$ selbe] balde I selber R selb Z  $\cdot$ an] an des R \newline
\end{minipage}
\hspace{0.5cm}
\begin{minipage}[t]{0.5\linewidth}
\small
\begin{center}*T
\end{center}
\begin{tabular}{rl}
 & kunde ir lîp wol gereizen.\\ 
 & irn gesâhet nie âmeizen,\\ 
 & di\textit{u} bezzers gelenkes pflac,\\ 
 & danne si was, dâ der gürtel lac:\\ 
5 & daz gap ir gesellen\\ 
 & Gawane manlîch ellen.\\ 
 & si \textbf{tûrte} mit im in der nôt.\\ 
 & sîn benante\textit{z} gîsel was der tôt\\ 
 & unde anders dehein gedinge.\\ 
10 & Gawan wac vil ringe\\ 
 & vîende haz, \textbf{swenne}r die maget erkôs.\\ 
 & dâ von ir vil den lîp verlôs.\\ 
 & \begin{large}D\end{large}ô kom der künec Vergulaht.\\ 
 & der sach die strîteclîche maht\\ 
15 & gegen Gawane kriegen.\\ 
 & ine \textbf{wolt}iuch \textbf{danne} triegen,\\ 
 & sône mag ich in niht beschœnen,\\ 
 & ern welle sich selben hœnen\\ 
 & an sînem werden gaste.\\ 
20 & der stuont ze wer alvaste.\\ 
 & dâ tet der wirt selbe schîn,\\ 
 & daz mich riuwet Gandin,\\ 
 & der künec von Anschouwe,\\ 
 & daz ein sô werd\textit{iu} vrouwe,\\ 
25 & sîn tohter, ie den sun gebar,\\ 
 & der mit ungetriuwer schar\\ 
 & sîn volc bat sêre strîten.\\ 
 & Gawan muose bîten,\\ 
 & unze \textbf{daz} der künec gewâpent wart.\\ 
30 & \textbf{er} huop sich selbe an strîtes vart.\\ 
\end{tabular}
\scriptsize
\line(1,0){75} \newline
T U V W \newline
\line(1,0){75} \newline
\textbf{13} \textit{Initiale} T U W  \textbf{17} \textit{Initiale} V  \newline
\line(1,0){75} \newline
\textbf{1} wol] gar wol W \textbf{2} irn] Ir W \textbf{3} diu] die T \textbf{4} danne] Daz U  $\cdot$ dâ] do U V W \textbf{6} Gawane] Gawan W \textbf{7} tûrte] traurte W  $\cdot$ der] die U \textbf{8} benantez] benantes T \textbf{11} swenner] wan er U V (W) \textbf{13} Vergulaht] Vergvlaht T vergulacht U W virgulaht V \textbf{14} strîteclîche] stritlichen V (W) \textbf{16} ine] Ich W  $\cdot$ woltiuch] woltiv T \textbf{17} sône] So W  $\cdot$ in] ir U \textbf{18} welle] wolde U  $\cdot$ selben] selbe U selber V W \textbf{20} alvaste] al zuͦ vaste U \textbf{21} dâ] Do U W  $\cdot$ selbe] selber W \textbf{22} Gandin] Gaudin U gaudein W \textbf{23} Anschouwe] aschowe U anschowe V antschawe W \textbf{24} werdiu] werde T \textbf{27} sêre] so sere V \textbf{28} muose] mvese T \textbf{29} unze] Mit U  $\cdot$ gewâpent] verwapent W \textbf{30} selbe] selber V W  $\cdot$ strîtes] die W \newline
\end{minipage}
\end{table}
\end{document}
