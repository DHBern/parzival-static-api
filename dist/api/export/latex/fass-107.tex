\documentclass[8pt,a4paper,notitlepage]{article}
\usepackage{fullpage}
\usepackage{ulem}
\usepackage{xltxtra}
\usepackage{datetime}
\renewcommand{\dateseparator}{.}
\dmyyyydate
\usepackage{fancyhdr}
\usepackage{ifthen}
\pagestyle{fancy}
\fancyhf{}
\renewcommand{\headrulewidth}{0pt}
\fancyfoot[L]{\ifthenelse{\value{page}=1}{\today, \currenttime{} Uhr}{}}
\begin{document}
\begin{table}[ht]
\begin{minipage}[t]{0.5\linewidth}
\small
\begin{center}*D
\end{center}
\begin{tabular}{rl}
\textbf{107} & mit golde \textbf{wart} gehêret,\\ 
 & grôz rîcheit dran gekêret\\ 
 & \textbf{mit} edelem gesteine,\\ 
 & dâ inne lît der reine.\\ 
5 & gebalsemt wart sîn junger rê.\\ 
 & \textbf{vor jâmer wart vil liute} wê.\\ 
 & ein tiwer rubîn ist der stein\\ 
 & ob sîme grabe, dâ durch er schein.\\ 
 & uns wart gevolget hie mite:\\ 
10 & ein kriuze nâch der \textbf{marter} site,\\ 
 & als uns \textbf{Kristes tôt} \textbf{lôste},\\ 
 & \textbf{liez} man \textbf{stôzen im} ze trôste,\\ 
 & ze scherm der sêle überz grap.\\ 
 & der bâruc die koste gap.\\ 
15 & \textbf{ez} was ein tiwer smarât.\\ 
 & wir tâtenz âne der heiden rât.\\ 
 & ir orden kan \textbf{niht} kriuzes pflegen,\\ 
 & als Kristes tôt \textbf{uns} liez den segen.\\ 
 & ez beten\textit{t} heiden sunder spot\\ 
20 & an in als \textbf{an} ir werden got,\\ 
 & niht durch des kriuzes êre\\ 
 & noch durch des toufes lêre,\\ 
 & \textbf{der} zem urteillîchem ende\\ 
 & uns lœsen \textbf{sol} gebende.\\ 
25 & diu manlîch triwe sîn\\ 
 & gît im ze himel liehten schîn\\ 
 & unt \textbf{ouch} sîn riwic bîhte.\\ 
 & der valsch was \textbf{an} im sîhte.\\ 
 & \textit{\begin{large}I\end{large}}n \textbf{sînen} helm, \textbf{den} adamas,\\ 
30 & \textbf{ein} epitafjum ergraben was,\\ 
\end{tabular}
\scriptsize
\line(1,0){75} \newline
D Fr33 \newline
\line(1,0){75} \newline
\textbf{19} \textit{Initiale} Fr33  \textbf{29} \textit{Initiale} D  \newline
\line(1,0){75} \newline
\textbf{11} Kristes] christes D \textbf{18} Kristes] christes D \textbf{19} betent] betten D \textbf{22} durch] \textit{om.} Fr33 \textbf{23} der zem urteillîchem] dem zer vrteilichen Fr33 \textbf{28} an] \textit{om.} Fr33 \textbf{29} In] ÷n D \textbf{30} ein] \textit{om.} Fr33  $\cdot$ ergraben] ergraben \textit{nachträglich korrigiert zu:} gegraben D \newline
\end{minipage}
\hspace{0.5cm}
\begin{minipage}[t]{0.5\linewidth}
\small
\begin{center}*m
\end{center}
\begin{tabular}{rl}
 & mit golt \textbf{wol} gehêret,\\ 
 & grôz rîcheit dran gekêret\\ 
 & \textbf{mit} edelem gesteine,\\ 
 & dâ \textit{i}nne lît der reine.\\ 
5 & gebalsemt wart sîn junger rê.\\ 
 & \textbf{vor jâmer wart vil liuten} wê.\\ 
 & ein tiure rubîn ist der stein\\ 
 & ob sînem grabe, dâ durch er schein.\\ 
 & uns wart gevolget hie mite:\\ 
10 & ein kriuze nâch der \textbf{meister} site,\\ 
 & alsô uns \textbf{Kristus tôt} \textbf{erlôste},\\ 
 & \textbf{liez} man \textbf{stôzen ime} z\textit{e} trôste,\\ 
 & ze scherme der sêle über daz grap.\\ 
 & der bâruc die koste gap.\\ 
15 & \textbf{ez} was ein tiure smarât.\\ 
 & wir tâtenz ân der heiden rât.\\ 
 & ir orden kan \textbf{iht} kriuzes pflegen,\\ 
 & als Kristus tôt \textbf{ime} liez den segen.\\ 
 & ez betent heiden sunder spot\\ 
20 & an in als ir werden got,\\ 
 & niht durch des kriuzes êre\\ 
 & noch durch des toufe\textit{s} \textit{l}êre,\\ 
 & \textbf{der} zem urteillîchem ende\\ 
 & uns l\textit{œ}sen \textbf{sol} gebende.\\ 
25 & diu manlîche triuwe sîn\\ 
 & gît im ze himele liehten schîn\\ 
 & und \textbf{ouch} sîn riuwic bîhte.\\ 
 & der valsch was \textbf{an} im sîhte.\\ 
 & in \textbf{sînem} helm, \textbf{den} adamas,\\ 
30 & \textbf{ein} \textit{e}pitafjum ergraben was,\\ 
\end{tabular}
\scriptsize
\line(1,0){75} \newline
m n o \newline
\line(1,0){75} \newline
\newline
\line(1,0){75} \newline
\textbf{4} dâ inne] Danne m \textbf{6} vor] Von n Won o \textbf{7} rubîn] rúbin n o  $\cdot$ der] ein n \textbf{10} meister] martel n (o) \textbf{11} Kristus] cristus m n cristuͯs o  $\cdot$ erlôste] erlúst o \textbf{12} liez] Liesse n  $\cdot$ ze trôste] [zestoste]: zestroste m \textbf{13} sêle] selen o  $\cdot$ über] v́be n \textbf{14} die koste] den kosten n o \textbf{16} der] den n des o \textbf{17} iht] nicht n (o) \textbf{18} Kristus] cristus m n cristuͯs o \textbf{19} betent] bieten o \textbf{20} als ir] also an iren n vnd an jrem o \textbf{22} toufes lêre] touffes ere vnd lere m \textbf{23} urteillîchem] vrteilichen n (o) \textbf{24} lœsen] lassen m \textbf{27} bîhte] gebiht o \textbf{29} in] An o  $\cdot$ den] ein n  $\cdot$ adamas] adamast m \textbf{30} epitafjum] appitaffium m appitasum n apitagsum o \newline
\end{minipage}
\end{table}
\newpage
\begin{table}[ht]
\begin{minipage}[t]{0.5\linewidth}
\small
\begin{center}*G
\end{center}
\begin{tabular}{rl}
 & mit golde \textbf{wart} \textbf{sîn grap} gehêrt,\\ 
 & grôz rîcheit dran gekêrt\\ 
 & \textbf{von} edelem gesteine,\\ 
 & dâr inne lît der reine.\\ 
5 & gebalsemt wart sîn junger rê.\\ 
 & \textbf{sîn tôt tet Sarrazinen} wê.\\ 
 & ein tiure rubîn ist der stein\\ 
 & obe sînem grabe, dâ durch er schein.\\ 
 & uns wart gevolget hie mit:\\ 
10 & ein kriuze nâch der \textbf{marter} sit,\\ 
 & als uns \textbf{Krist des tôdes} \textbf{\textit{er}lôste},\\ 
 & \textbf{lie} man \textbf{stôzen im} ze trôste,\\ 
 & ze scherme der sêle überz grap.\\ 
 & der bâruc die koste gap.\\ 
15 & \textbf{ez} was ein tiure smarât.\\ 
 & wir tâtenz âne der heiden rât.\\ 
 & ir orden kan \textbf{niht} kriuzes pflegen,\\ 
 & als Kristes tôt \textbf{uns} lie den segen.\\ 
 & ez betent heiden sunder spot\\ 
20 & an in als \textbf{an} ir werden got,\\ 
 & niht durch des kriuzes êre\\ 
 & noch durch des toufes lêre,\\ 
 & \textbf{der} zem urteillîchen ende\\ 
 & uns lœsen \textbf{sol} \textbf{der} gebende.\\ 
25 & diu manlîche triwe sîn\\ 
 & gît ime ze himele liehten schîn\\ 
 & unde \textbf{ouch} sîn riwic bîhte.\\ 
 & der valsch was \textbf{an} im sîhte.\\ 
 & in \textbf{sînen} helm, \textbf{den} adamas,\\ 
30 & epitafjum ergraben was,\\ 
\end{tabular}
\scriptsize
\line(1,0){75} \newline
G I O L M Q R Z Fr21 \newline
\line(1,0){75} \newline
\textbf{17} \textit{Initiale} I  \textbf{29} \textit{Initiale} L Q R Z  \newline
\line(1,0){75} \newline
\textbf{1} wart] er wart Q  $\cdot$ sîn grap] si sarc I er O \textit{om.} L M Q R Z Fr21  $\cdot$ gehêrt] ge erit M \textbf{4} dâr] Do er Q (R) \textbf{5} sîn junger] der iunge I (Fr21) \textbf{6} Von iamer wart vil levten we O (M) (R) (Fr21) Von iamer wart vil luͦte we L Vor iamer wart (ovch Z ) vil leuten wee Q (Z)  $\cdot$ Sarrazinen] sarazinen G den sarrazinen I \textbf{7} rubîn] rubein Q \textbf{8} obe] Vff M \textbf{9} hie] da L \textbf{11} als] al I  $\cdot$ Krist] christ G crist I M christes O Fr21 cristus L R cristes Q Z  $\cdot$ des tôdes] \textit{om.} I tot O (L) Q Z Fr21 von dem tod R  $\cdot$ erlôste] loste G erlost R Fr21 \textbf{13} ze scherme der sêle] der sel zesherme I Zcu beschermene der sele M Ze schirme der selbe R \textbf{14} bâruc] brauc Q \textbf{15} was] wart L  $\cdot$ smarât] smarade I smarac Q schmarat R \textbf{17} niht] \textit{om.} Z \textbf{18} Kristes] christes G O cristes I (L) M Q Z Fr21 cristus R  $\cdot$ tôt] tuͤt I  $\cdot$ uns] vnd R \textbf{20} an in] an an I Jn an O  $\cdot$ an ir] ir O \textbf{21} des] \textit{om.} R \textbf{22} des] \textit{om.} L R  $\cdot$ toufes] taufel I toifers M tewfels Q \textbf{23} urteillîchen] vrtailichem I (O) (L) (Q) (Z) ordentlicheme M  $\cdot$ ende] tage ende L \textbf{24} uns] Vsz L  $\cdot$ der] \textit{om.} I Z  $\cdot$ gebende] [ende]: bende M \textbf{25} manlîche triwe] mancliche truwe M trewlich manheit Z \textbf{26} gît] Get M  $\cdot$ ime] in I vns O \textit{om.} M  $\cdot$ ze] zcu ome M  $\cdot$ liehten] lichten L (M) (Q) \textbf{27} riwic] Rúwe R rewe Z \textbf{28} der] Kein Q  $\cdot$ an] \textit{om.} Z  $\cdot$ sîhte] lihte O nichte Q \textbf{29} in] an I (L) (R)  $\cdot$ sînen] sinem I O L R Z  $\cdot$ den] der I (O) dem Q R \textit{om.} Z \textbf{30} epitafjum] ein epitafium I (L) (Q) (R) [Einen]: Ein epitafivum Z  $\cdot$ ergraben] er grab M \newline
\end{minipage}
\hspace{0.5cm}
\begin{minipage}[t]{0.5\linewidth}
\small
\begin{center}*T (U)
\end{center}
\begin{tabular}{rl}
 & mit golde \textbf{wart} ge\textit{h}êret,\\ 
 & grôz rîcheit dran gekêret\\ 
 & \textbf{von} edelem gesteine,\\ 
 & dâr inne lît der reine.\\ 
5 & gebalsemet wart sîn junger rê.\\ 
 & \textbf{vor jâmere wart vil liuten} wê.\\ 
 & ein tiure rubîn ist der stein\\ 
 & ob sîme grabe, dâ durch er schein.\\ 
 & uns wart gevolget hie mite:\\ 
10 & ein kriuze nâch der \textbf{martere} site,\\ 
 & als uns \textbf{Krist\textit{e}s jost} \textbf{erlôste},\\ 
 & \textbf{hiez} man \textbf{im stôzen} zuo trôste,\\ 
 & zuo schirme der sêle überz grap.\\ 
 & der bâruc die kost gap.\\ 
15 & \textbf{daz} was ein tiure \textit{s}marât.\\ 
 & wir tâtenz âne der heiden rât.\\ 
 & ir orden kan \textbf{niht} kriuzes pflegen,\\ 
 & als Krist\textit{e}s tôt \textbf{uns} li\textit{e} den segen.\\ 
 & ez betent heiden sunder spot\\ 
20 & an i\textit{n} als \textbf{an} i\textit{r} werden got,\\ 
 & niht durch des kriuzes êre\\ 
 & noch durch des toufes lêre,\\ 
 & \textbf{sô er} zuo dem urteillîchem ende\\ 
 & uns lœsen \textbf{wolte} \textbf{von} gebende.\\ 
25 & diu manlîc\textit{h}e triuwe sîn\\ 
 & gît im zuo himele liehten schîn\\ 
 & und sîn riuwige bîhte.\\ 
 & der valsc\textit{h} was im sîhte.\\ 
 & \begin{large}I\end{large}n \textbf{sînen} helm, \textbf{dem} adamas,\\ 
30 & \textbf{ein} epitafjum ergraben was,\\ 
\end{tabular}
\scriptsize
\line(1,0){75} \newline
U V W T \newline
\line(1,0){75} \newline
\textbf{29} \textit{Initiale} U V W T  \newline
\line(1,0){75} \newline
\textbf{1} gehêret] gekeret U er geheret W \textbf{6} vor] von V (W) \textbf{7} der] sein W \textbf{8} sîme] e sinem T \textbf{11} da mit vns criste erlôste T  $\cdot$ Kristes] cristus U cristes V christus W  $\cdot$ jost] dot V (W) \textbf{12} hiez] lie T \textbf{13} sêle] selen V W \textbf{14} kost] selben koste W \textbf{15} daz] ez T  $\cdot$ smarât] marat U \textbf{18} Kristes] cristus U cristes V T christus W  $\cdot$ uns lie den] [*]: vns liden U hie lie den V lies vns den W \textbf{19} betent] betten V \textbf{20} in] ir U in reht T  $\cdot$ ir] in U ein W \textbf{21} durch] \textit{om.} W  $\cdot$ kriuzes] \textit{om.} T \textbf{23} sô er] Doch er W der T  $\cdot$ urteillîchem] vrteillichen V vrtailischem W \textbf{24} wolte von gebende] [*]: sol der gebende V solt von gebende W sol der bende T \textbf{25} manlîche] man lichte U manlichen T \textbf{26} im] \textit{om.} W  $\cdot$ liehten] lichten U \textbf{27} Vnd seiner treúwen pflichte W  $\cdot$ und] vnde ouch T \textbf{28} valsch] valsche U  $\cdot$ im sîhte] an [*]: imme sihte V im nichte W an im sihte T \textbf{29} sînen] seinem W (T)  $\cdot$ dem] [*]: den V den W \newline
\end{minipage}
\end{table}
\end{document}
