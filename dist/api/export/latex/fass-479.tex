\documentclass[8pt,a4paper,notitlepage]{article}
\usepackage{fullpage}
\usepackage{ulem}
\usepackage{xltxtra}
\usepackage{datetime}
\renewcommand{\dateseparator}{.}
\dmyyyydate
\usepackage{fancyhdr}
\usepackage{ifthen}
\pagestyle{fancy}
\fancyhf{}
\renewcommand{\headrulewidth}{0pt}
\fancyfoot[L]{\ifthenelse{\value{page}=1}{\today, \currenttime{} Uhr}{}}
\begin{document}
\begin{table}[ht]
\begin{minipage}[t]{0.5\linewidth}
\small
\begin{center}*D
\end{center}
\begin{tabular}{rl}
\textbf{479} & \begin{large}D\end{large}er ruoft ist zer diemuot\\ 
 & iedoch niht volleclîchen guot.\\ 
 & eines tages der künec al eine reit\\ 
 & - daz was gar den sînen leit -\\ 
5 & ûz durch âventiure,\\ 
 & durch vreude an minnen stiure;\\ 
 & des twanc in der minnen ger.\\ 
 & mit einem geluptem sper\\ 
 & wart er ze \textbf{tjostieren} wunt,\\ 
10 & sô daz \textbf{er} nimmer mêr gesunt\\ 
 & wart, der süeze œheim dîn,\\ 
 & durch die heidruose sîn.\\ 
 & \textbf{ez} was ein heiden, der dâ streit\\ 
 & unt der die selben tjoste reit,\\ 
15 & geborn von Ethnise,\\ 
 & dâ ûzem pardîse\\ 
 & rinnet diu Tigris.\\ 
 & der selbe heiden was gewis,\\ 
 & sîn ellen solde den Grâl \textbf{behaben}.\\ 
20 & \textbf{in}me sper \textbf{was sîn nam} ergraben.\\ 
 & er suochte die verren rîterschaft;\\ 
 & niht wan durch des Grâles kraft\\ 
 & streich er wazzer unde lant.\\ 
 & von sîme strîte uns vreude swant.\\ 
25 & dînes œheimes strît man prîsen\\ 
 & muoz. des spers îsen\\ 
 & vuorter in sîme lîbe dan.\\ 
 & dô der junge, werde man\\ 
 & kom heim zuo den sînen,\\ 
30 & dâ sach man jâmer schînen.\\ 
\end{tabular}
\scriptsize
\line(1,0){75} \newline
D Fr31 \newline
\line(1,0){75} \newline
\textbf{1} \textit{Initiale} D Fr31  \newline
\line(1,0){75} \newline
\textbf{5} ûz] Daz Fr31 \textbf{8} geluptem] gilvpten Fr31 \textbf{10} er] \textit{om.} Fr31 \textbf{15} Ethnise] ehtuyse Fr31 \textbf{17} diu Tigris] der tygris Fr31 \textbf{19} Grâl] pris Fr31 \textbf{21} verren] verre Fr31 \textbf{30} dâ] do Fr31 \newline
\end{minipage}
\hspace{0.5cm}
\begin{minipage}[t]{0.5\linewidth}
\small
\begin{center}*m
\end{center}
\begin{tabular}{rl}
 & der ruof ist zuor diemuot\\ 
 & iedoch niht volleclîchen guot.\\ 
 & eines tages der künic alein reit\\ 
 & - daz was gar den sînen leit -\\ 
5 & ûz durch âventiure,\\ 
 & durch vröide an minnen stiure;\\ 
 & des twanc in der minne ger.\\ 
 & mit einem gelupten sper\\ 
 & wart er ze \textbf{justieren} wunt,\\ 
10 & sô daz \textbf{er} nimer mêr gesunt\\ 
 & wart, der süeze œheim dîn,\\ 
 & durch die heidruose sîn.\\ 
 & \textbf{ez} was ein heiden, der d\textit{â} streit\\ 
 & und der die selben juste reit,\\ 
15 & geborn von E\textit{th}nise,\\ 
 & d\textit{â} û\textit{z} dem paradîse\\ 
 & r\textit{inn}et diu Ti\textit{g}ris.\\ 
 & der selbe heiden \textit{was} \textit{ge}wis,\\ 
 & sîn ellen solte den Grâl \dag behalten\dag .\\ 
20 & \textbf{in} dem sper \textbf{was sîn name} ergraben.\\ 
 & er suoht die verren ritterschaft;\\ 
 & niht wan \textit{durch} des Grâles kraft\\ 
 & streich er wazzer und lant.\\ 
 & von sînem strît uns vröude swant.\\ 
25 & dînes œheimes strît man prîsen\\ 
 & muoz. des spers îsen\\ 
 & vuort er in sînem lîbe dan.\\ 
 & dô der junge, werde man\\ 
 & kam heim zuo den sînen,\\ 
30 & dô sach man jâmer schînen.\\ 
\end{tabular}
\scriptsize
\line(1,0){75} \newline
m n o \newline
\line(1,0){75} \newline
\newline
\line(1,0){75} \newline
\textbf{2} niht] mit o  $\cdot$ volleclîchen] [*]: solleclichen o \textbf{5} ûz] Vnd o \textbf{6} An freiden an mẏnne stúre o \textbf{7} minne] mẏnnen n (o) \textbf{8} gelupten] gelippenten n \textbf{13} ez] Er o  $\cdot$ dâ] do m n o \textbf{14} selben] selbe m n o \textbf{15} Ethnise] ehtnise m echtnise n o \textbf{16} dâ] Do m n o  $\cdot$ ûz] vff m \textbf{17} rinnet] Rumet m n  $\cdot$ Tigris] tigeris m tigrisz o \textbf{18} was gewis] wis m \textbf{19} Grâl behalten] grole behabten n \textbf{22} durch] \textit{om.} m \textbf{27} in] \textit{om.} n \newline
\end{minipage}
\end{table}
\newpage
\begin{table}[ht]
\begin{minipage}[t]{0.5\linewidth}
\small
\begin{center}*G
\end{center}
\begin{tabular}{rl}
 & \begin{large}D\end{large}er ruoft ist zer diemuot\\ 
 & iedoch niht volleclîchen guot.\\ 
 & eines tages der künic \textit{al ein reit}\\ 
 & - daz was gar \textit{den sînen leit} -\\ 
5 & ûz d\textit{u}r\textit{ch} âventiure,\\ 
 & durch vröude an minnen stiure;\\ 
 & des twanc in der minnen ger.\\ 
 & mit einem geluppetem sper\\ 
 & wart er ze \textbf{tjostierne} wunt,\\ 
10 & sô daz \textbf{er} nimmer mêr gesunt\\ 
 & wart, der süeze œheim dîn,\\ 
 & durch die heidruose sîn.\\ 
 & \textbf{e\textit{z}} was ein heiden, der dâ streit\\ 
 & unt der die selben tjoste reit,\\ 
15 & geborn von Ethnise,\\ 
 & dâ ûzzem paradîse\\ 
 & rinnet diu Tygris.\\ 
 & der selbe heiden was gewis,\\ 
 & sîn ellen solde den Grâl \textbf{behaben}.\\ 
20 & \textbf{in} dem sper \textbf{was sîn name} ergraben.\\ 
 & er suochte die verren rîterschaft;\\ 
 & niht wan durch des Grâles kraft\\ 
 & streich er wazzer unde lant.\\ 
 & von sînem strîte uns vröude swant.\\ 
25 & dînes œheimes strît man brîsen\\ 
 & muoz. des spers îsen\\ 
 & vuort er in sînem lîbe dan.\\ 
 & dô der junge, werde man\\ 
 & kom heim zuo den sînen,\\ 
30 & dô sach man jâmer schînen.\\ 
\end{tabular}
\scriptsize
\line(1,0){75} \newline
G I O L M Z \newline
\line(1,0){75} \newline
\textbf{1} \textit{Initiale} G I O L Z  \textbf{13} \textit{Initiale} I  \newline
\line(1,0){75} \newline
\textbf{1} Der ruoft] Der raup I ÷er rvͦf O Der ruͯf L Den ruft M  $\cdot$ zer] zcu den M zv Z \textbf{3} al ein reit] reit al ein G eine reit I \textbf{4} Daz was gar leit den sin G \textbf{5} durch] der G \textbf{6} vröude] frouden M  $\cdot$ an] \textit{om.} I \textbf{7} minnen] minne O (L) Z  $\cdot$ ger] gêr G \textbf{8} geluppetem] gelvpten O (L) Z geluptē M \textbf{9} er ze tjostierne] er zuͤ der Tioste I (O) er zuͯ tiost L zcu der tjostern M \textbf{10} nimmer] nie I  $\cdot$ mêr] \textit{om.} M \textbf{11} dîn] myn M \textbf{13} ez] Er G  $\cdot$ heiden] heide I (M) \textbf{15} geborn] Goborn M  $\cdot$ Ethnise] etnise O Ethuͯise L othnise M \textbf{16} dâ] Do O \textbf{17} diu] der L  $\cdot$ Tygris] tygrîs G tigris I O L M \textbf{18} heiden] heide M \textbf{19} behaben] beiagen L \textbf{20} in] an I  $\cdot$ was sîn name] sin nam waz L  $\cdot$ ergraben] begraben O L \textbf{21} suochte] svͦht O  $\cdot$ verren] verre L \textbf{23} streich] Erstreich Z \textbf{24} sînem] sinen L  $\cdot$ swant] verswant Z \textbf{25} œheimes] ohemen M \textbf{28} dô] Da M Z \textbf{29} kom heim] Heim kom L \textbf{30} dô] Da O M Z \newline
\end{minipage}
\hspace{0.5cm}
\begin{minipage}[t]{0.5\linewidth}
\small
\begin{center}*T
\end{center}
\begin{tabular}{rl}
 & der ruoft ist zer diemuot\\ 
 & iedoch niht volleclîche guot.\\ 
 & \hspace*{-.7em}\big| daz was gar den sînen leit.\\ 
 & \hspace*{-.7em}\big| \textit{\begin{large}E\end{large}}ines tages der künec aleine reit\\ 
5 & ûz durch âventiure,\\ 
 & durch vröude an minne stiure;\\ 
 & des twanc in der minne ger.\\ 
 & mit einem gelupeten sper\\ 
 & wart er ze\textbf{r tjostiure} wunt,\\ 
10 & sô daz niemer mêr gesunt\\ 
 & wart der süeze œheim dîn\\ 
 & durch die heidruose sîn.\\ 
 & \textbf{er} was ein heiden, der dâ streit\\ 
 & unde der die selben tjost reit,\\ 
15 & geborn von Etnise,\\ 
 & Dâ ûz dem paradîse\\ 
 & rinnet di\textit{u} Tygris.\\ 
 & der selbe heiden was gewis,\\ 
 & sîn ellen solte den Grâl \textbf{bejagen}.\\ 
20 & \textbf{an}me sper \textbf{sîn name was} ergraben.\\ 
 & er suochte die verren rîterschaft;\\ 
 & niht wan durch des Grâles kraft\\ 
 & streich er wazzer unde lant.\\ 
 & von sînem strîte uns vröude swant.\\ 
25 & dînes œheimes strît man prîsen\\ 
 & muoz. des spers îsen\\ 
 & vuorter in sînem lîbe dan.\\ 
 & dô der junge, werde man\\ 
 & kom heim zuo den sînen,\\ 
30 & dô sach man jâmer schînen.\\ 
\end{tabular}
\scriptsize
\line(1,0){75} \newline
T U V W Q R \newline
\line(1,0){75} \newline
\textbf{1} \textit{Capitulumzeichen} R  \textbf{3} \textit{Initiale} T W R  \textbf{16} \textit{Majuskel} T  \newline
\line(1,0){75} \newline
\textbf{1} \textit{Die Verse 453.1-502.30 fehlen} U   $\cdot$ ruoft] rvͦf V (W) (R)  $\cdot$ zer] zu Q \textbf{2} iedoch] Doch R  $\cdot$ volleclîche] willenklichen R \textbf{4} \textit{Versfolge 479.3-4} W Q R  \textbf{3} Eines] Sines T  $\cdot$ tages] males W \textbf{6} minne] minnen V meinen Q \textbf{7} minne] minnen V (W) \textbf{8} gelupeten] gelúpfften W \textbf{9} zer tjostiure] ze tiostieren V (W) (Q) ze stritte do R \textbf{10} daz] daz er V (W) (Q) (R) \textbf{11} süeze] suͤssen W  $\cdot$ dîn] min R \textbf{12} heidruose] frv́ndinne V \textbf{13} er] [E*]: Ez V Es W Q R  $\cdot$ dâ] do V W Q \textbf{14} selben] selbe Q \textit{om.} R \textbf{15} Etnise] etnysi V ethnise Q Ehnyse R \textbf{16} Dâ] Do V W Q R \textbf{17} diu] die T  $\cdot$ Tygris] tigrys T \textbf{19} ellen] eren Q  $\cdot$ bejagen] behaben Q [behaltten]: behabten R \textbf{20} anme] One R  $\cdot$ sîn name was] was sein nam Q  $\cdot$ ergraben] begraben W \textbf{22} wan] [was]: wan R  $\cdot$ des] \textit{om.} W \textbf{25} strît] pris R  $\cdot$ prîsen] prise R \textbf{28} werde] werder Q \newline
\end{minipage}
\end{table}
\end{document}
