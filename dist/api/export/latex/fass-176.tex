\documentclass[8pt,a4paper,notitlepage]{article}
\usepackage{fullpage}
\usepackage{ulem}
\usepackage{xltxtra}
\usepackage{datetime}
\renewcommand{\dateseparator}{.}
\dmyyyydate
\usepackage{fancyhdr}
\usepackage{ifthen}
\pagestyle{fancy}
\fancyhf{}
\renewcommand{\headrulewidth}{0pt}
\fancyfoot[L]{\ifthenelse{\value{page}=1}{\today, \currenttime{} Uhr}{}}
\begin{document}
\begin{table}[ht]
\begin{minipage}[t]{0.5\linewidth}
\small
\begin{center}*D
\end{center}
\begin{tabular}{rl}
\textbf{176} & \textbf{liezet}, ob siz m\textit{ö}hte hân.\\ 
 & nû\textbf{ne} hât sis niht noch vürspan.\\ 
 & wer gæbe ir sölhen volleist\\ 
 & \textbf{sô} der vrouwen in dem fôreist?\\ 
5 & diu het \textbf{etswen}, von dem si enpfie\textit{n}c\\ 
 & daz iu zenpfâhene sît ergienc.\\ 
 & ir muget Liazen niht genemen."\\ 
 & der gast begunde sich \textbf{des} schemen.\\ 
 & \begin{large}I\end{large}edoch kuster si an den munt;\\ 
10 & dem was wol viwers varwe kunt.\\ 
 & Liazen lîp was minneclîch,\\ 
 & dâr \textbf{zuo} der wâren kiusche rîch.\\ 
 & der tisch was nider \textbf{und} lanc.\\ 
 & der wirt \textbf{mit niemen sich dâ} dranc;\\ 
15 & er saz al eine an den ort.\\ 
 & sînen gast \textbf{hiez} er sitzen dort\\ 
 & zwischen im unt sîme kinde.\\ 
 & ir blanken hende linde\\ 
 & muosen \textbf{snîden}, sô der wirt gebôt,\\ 
20 & \textbf{den man dâ hiez der} ritter rôt,\\ 
 & \textbf{swaz} \textbf{der} ezzen wolde.\\ 
 & niemen si wenden solde,\\ 
 & si\textbf{ne} gebârten heinlîche.\\ 
 & diu magt \textbf{mit zühten} rîche\\ 
25 & leiste ir vater willen gar.\\ 
 & \textbf{si unt der} gast \textbf{wâren} wol gevar.\\ 
 & Dar nâch schiere gie diu magt wider.\\ 
 & \textbf{sus} pflac man des heldes sider\\ 
 & unz an den vierzehenden tac.\\ 
30 & bî sîme herzen kumber lac\\ 
\end{tabular}
\scriptsize
\line(1,0){75} \newline
D \newline
\line(1,0){75} \newline
\textbf{9} \textit{Initiale} D  \textbf{27} \textit{Majuskel} D  \newline
\line(1,0){75} \newline
\textbf{1} möhte] mohte D \textbf{5} enpfienc] enpfiech D \newline
\end{minipage}
\hspace{0.5cm}
\begin{minipage}[t]{0.5\linewidth}
\small
\begin{center}*m
\end{center}
\begin{tabular}{rl}
 & \textbf{liezet}, ob si ez möhte hân.\\ 
 & nû \textbf{en}hât si es niht noch vürs\textit{p}a\textit{n}.\\ 
 & wer gæbe ir solichen volleist\\ 
 & \textbf{sô} der vrouwen in dem fôreist?\\ 
5 & diu het \textbf{\textit{e}tewan}, von dem si enpfienc\\ 
 & daz iu zenpfâ\textit{h}en sît ergienc.\\ 
 & ir muget Liazen niht genemen."\\ 
 & der gast begunde sich \textbf{des} schemen.\\ 
 & iedoch \textbf{sô} kuste er \textit{si} an den munt;\\ 
10 & dem was wol viures varwe kunt.\\ 
 & Liaze\textit{n} lîp was minneclîch,\\ 
 & dâr \textbf{zuo} der wâren kiusche rîch.\\ 
 & der tisch was nider \textbf{und} lanc.\\ 
 & der wirt \textbf{\textit{mit} niema\textit{n} sich d\textit{â}} dranc;\\ 
15 & er saz aleine an den ort.\\ 
 & sînen gast \textbf{hiez} er sitzen dort\\ 
 & z\textit{w}ischen \textit{ime} und sînem kinde.\\ 
 & ir blanken hende linde\\ 
 & muosen, sô der wirt gebôt,\\ 
20 & \textbf{vinden dem} ritter \dag ort\dag ,\\ 
 & \textbf{waz} \textbf{der} ezzen wolte.\\ 
 & niemen si wenden solte,\\ 
 & si gebârten heimlîche.\\ 
 & diu maget \textbf{an zühten} rîche\\ 
25 & leiste ir vater willen gar.\\ 
 & \textbf{sô tet der} gast wol ge\textit{v}ar.\\ 
 & dâ nâch schiere gienc diu maget wider.\\ 
 & \textbf{sus} pflac man des heldes sider\\ 
 & unz an den vierzehenden tac.\\ 
30 & bî sînem herzen kumber lac\\ 
\end{tabular}
\scriptsize
\line(1,0){75} \newline
m n o Fr69 \newline
\line(1,0){75} \newline
\newline
\line(1,0){75} \newline
\textbf{1} liezet] Liessen n o \textbf{2} enhât] hat n o  $\cdot$ vürspan] fursprang m verstan o \textbf{5} het etewan] hettewan m \textbf{6} zenpfâhen] zemphanen m zú empahen o \textbf{7} Liazen] hasen n laisen o \textbf{8} des] das o \textbf{9} sô] \textit{om.} n  $\cdot$ si] in m \textbf{10} was] ist n o \textbf{11} Liazen] Liazem m Liasen n o \textbf{12} kiusche] kúschen n (o) \textbf{14} mit nieman] niemam m  $\cdot$ dâ] do m n o \textbf{17} zwischen ime] Zwwischen m  $\cdot$ sînem] sine n \textbf{19} muosen] Muͯsten n o \textbf{20} Snyden do dem ritter brot n (o) \textbf{22} wenden] erwenden n o \textbf{25} leiste] Leist n \textbf{26} der] >der< m  $\cdot$ gevar] gewar m \textbf{27} dâ nâch] Die noch o \textbf{29} vierzehenden] vieczehenden o \newline
\end{minipage}
\end{table}
\newpage
\begin{table}[ht]
\begin{minipage}[t]{0.5\linewidth}
\small
\begin{center}*G
\end{center}
\begin{tabular}{rl}
 & \textbf{liezet}, op siz m\textit{ö}ht hân.\\ 
 & nû hât sis niht noch vürspan.\\ 
 & wer gæbe ir solhen volleist\\ 
 & \textbf{s\textit{am}} der vrouwen in dem fôreist?\\ 
5 & \begin{large}D\end{large}iu het \textbf{etwen}, von dem si enpfienc\\ 
 & daz iu zenpfâhene sît ergienc.\\ 
 & ir muget Liazen niht genemen."\\ 
 & der gast begunde sich \textbf{des} schemen.\\ 
 & iedoch kuster si an den munt;\\ 
10 & dem was wol viures varwe kunt.\\ 
 & Liazen lîp was minniclîch,\\ 
 & dâ \textbf{bî} der wâren kiusche rîch.\\ 
 & der tisch was nidere \textbf{unde} lanc.\\ 
 & der wirt \textbf{sich dâ mit niemen} dranc;\\ 
15 & er saz al eine an den ort.\\ 
 & sînen gast \textbf{liez} er sitzen dort\\ 
 & zwischen im unde sînem kinde.\\ 
 & ir blanken hende linde\\ 
 & \begin{large}M\end{large}uosen \textbf{snîden}, sô der \textit{wirt} gebôt,\\ 
20 & \textbf{den man dâ hiez den} rîter rôt,\\ 
 & \textbf{al daz} \textbf{er} ezzen wolde.\\ 
 & niemen si wenden solde,\\ 
 & si\textbf{ne} gebârten heinlîche.\\ 
 & diu maget \textbf{zühte} rîche\\ 
25 & leist ir vater willen gar.\\ 
 & \textbf{si unde der} gast \textbf{wâren} wolgevar.\\ 
 & dar nâch schiere gienc diu maget wider.\\ 
 & \textbf{alsus} pflac man des heldes sider\\ 
 & unze an den vierzehenden tac.\\ 
30 & bî sînem herzen kumber lac\\ 
\end{tabular}
\scriptsize
\line(1,0){75} \newline
G I O L M Q R Z Fr47 \newline
\line(1,0){75} \newline
\textbf{5} \textit{Initiale} G  \textbf{9} \textit{Initiale} I O R Z   $\cdot$ \textit{Capitulumzeichen} L  \textbf{15} \textit{Initiale} Fr47  \textbf{19} \textit{Initiale} G  \textbf{27} \textit{Initiale} I  \newline
\line(1,0){75} \newline
\textbf{1} liezet] lazzet I (Fr47) Wiset Q  $\cdot$ möht] moht G (I) (L) (M) (Q) (Z) mvhte O \textbf{2} hât] enhat L (M)  $\cdot$ sis] susz M sie Z (Fr47)  $\cdot$ niht noch] niht noch ir I sin niht noch Z noch niht Fr47 \textbf{3} gæbe] Gab I  $\cdot$ ir] úch R \textit{om.} Fr47  $\cdot$ solhen] solhe I (Q) (R) \textbf{4} sam] so G  $\cdot$ der vrouwen] die frowe Fr47 \textbf{5} het] hat M  $\cdot$ dem] den M dē Fr47  $\cdot$ si] sis M (R) \textbf{6} iu] ir Fr47  $\cdot$ zenpfâhene] zcyn phande M entpfhoen Q \textbf{7} ir] irn I (M) (Z)  $\cdot$ Liazen] liazzen I Lýazen L lazin M Liassen Q laszen R lyazzen Z ::: Fr47 \textbf{8} der] Des Q Z \textbf{9} iedoch] [÷iedoch]: ÷edoch O Doch Q Iedoch so R  $\cdot$ kuster si] kúst ers Q (Z) \textbf{10} dem was wol viures] der was fiͤursh I \textbf{11} Liazen] Lýazen L liazin M Liassen Q Fr47 lyasen R Liazzen Z  $\cdot$ minniclîch] wunneclich I \textbf{14} sich dâ mit niemen] mit niemen sich da O (L) (M) (Z) mit niman sich do Q (Fr47) sich mit niemen da R \textbf{15} er] ÷R Fr47  $\cdot$ den] dem I (O) (L) (M) (R) (Fr47) dasz Q \textbf{16} sînen] Den O  $\cdot$ liez] hiesz Q \textbf{17} sînem] siner Fr47 \textbf{18} blanken] wisen R \textbf{19} sô] da M wen R alz Fr47  $\cdot$ wirt] \textit{om.} G \textbf{20} den man] Dem den man Q (R)  $\cdot$ dâ] do O \textit{om.} Q R Fr47  $\cdot$ rôt] tot M \textbf{21} al daz] Als das Q R Da Fr47  $\cdot$ wolde] soltte R \textbf{22} \textit{Vers 176.22 fehlt} R  \textbf{23} sine] Sie O L Q Z (Fr47)  $\cdot$ heinlîche] hinlich R \textbf{24} maget] magten Fr47  $\cdot$ zühte] zuhten I (R) mit zvhten O (L) (M) Z (Fr47)  $\cdot$ rîche] richen L \textbf{25} leist] Leiste L M R Z  $\cdot$ ir] irem M irs Q R Z (Fr47)  $\cdot$ gar] wol gar O \textbf{26} gast] \textit{om.} L  $\cdot$ wâren wolgevar] wol waren gevar M \textbf{27} schiere gienc] gie I (Fr47) gie schiere L \textbf{28} alsus] So Fr47  $\cdot$ heldes] gastes L R  $\cdot$ sider] wider R \textbf{29} unze] Bys Q \textbf{30} sînem herzen] seinem hertz Q sinen herren R \newline
\end{minipage}
\hspace{0.5cm}
\begin{minipage}[t]{0.5\linewidth}
\small
\begin{center}*T
\end{center}
\begin{tabular}{rl}
 & \textbf{lâzet}, ob siz m\textit{ö}hte hân.\\ 
 & nû\textbf{ne} hât sis niht noch vürspan.\\ 
 & wer gæbir solhen volleist\\ 
 & \textbf{sam} der vrouwen in dem fôreist?\\ 
5 & diu  \textbf{etswen}, von dem si\textbf{s} enpfienc,\\ 
 & daz iu zenpfâhenne sît ergienc.\\ 
 & ir muget Lyazen niht genemen."\\ 
 & Der gast begunde sich \textbf{dô} schemen.\\ 
 & iedoch kuster si an den munt;\\ 
10 & dem was wol viures varwe kunt.\\ 
 & Lyazen lîp was minneclîch,\\ 
 & dâ \textbf{bî} der wâren kiusche rîch.\\ 
 & Der tisch was nidere, \textbf{niht ze} lanc.\\ 
 & der wirt \textbf{mit niemen sich dâ} dranc;\\ 
15 & er saz aleine an den ort.\\ 
 & sînen gast \textbf{liez} er sitzen dort\\ 
 & zwischen im unde sînem kinde.\\ 
 & ir blanken hende linde\\ 
 & muosen \textbf{snîden}, sô der wirt gebôt\\ 
20 & \textbf{dem}, \textbf{den man dâ hiez der} rîter rôt,\\ 
 & \textbf{Al daz} \textbf{er} ezzen wolte.\\ 
 & nieman si\textbf{s} wenden solte,\\ 
 & si gebârten heimlîche.\\ 
 & di\textit{u} maget \textbf{zühte} rîche\\ 
25 & leiste ir vater willen gar.\\ 
 & \textbf{si unde ir} gast \textbf{wâren} wolgevar.\\ 
 & dar nâch schiere gie diu maget wider.\\ 
 & \textbf{alsus} pflac man des heldes sider\\ 
 & unz an den vierzehenden tac.\\ 
30 & bî sînem herzen kumber lac\\ 
\end{tabular}
\scriptsize
\line(1,0){75} \newline
T U V W \newline
\line(1,0){75} \newline
\textbf{8} \textit{Majuskel} T  \textbf{13} \textit{Initiale} W   $\cdot$ \textit{Majuskel} T  \textbf{21} \textit{Majuskel} T  \newline
\line(1,0){75} \newline
\textbf{1} lâzet] Liessent W  $\cdot$ möhte] mohte T (U) \textbf{2} nûne] Nuͦ U (V) (W)  $\cdot$ vürspan] den fúrspan W \textbf{4} sam] Als U \textbf{5} etswen von dem sis] [*]: hette etteswen von demme sv́ V het etwenne von dem sis W \textbf{7} Lyazen] lŷazen T \textbf{9} kuster] kust er W \textbf{11} Lyazen] Lyzen W \textbf{12} Dar zuͦ was er aller seldenreich W  $\cdot$ der] dir U \textbf{13} niht] vnd nit W \textbf{14} dâ] do U V \textit{om.} W \textbf{15} an den] an daz U an ein V in das W \textbf{17} zwischen] Inzwischen W \textbf{19} muosen] mvesen T  $\cdot$ sô] waz es W \textbf{20} dem den] Dem V Den W  $\cdot$ dâ] do U V W  $\cdot$ hiez] iach V  $\cdot$ der] den U W \textbf{21} daz] do daz U \textbf{22} sis] sy W \textbf{23} si] [S*]: Sv́ V \textbf{24} diu] die T  $\cdot$ zühte] zv́hten V mit zúchten W \textbf{25} leiste ir vater] leiste [i*]: irz vatter V Laistet irs vatters W \textbf{26} si] [S*]: Sv́ V  $\cdot$ ir] [*]: der V der W \textbf{27} wider] wi W \textbf{29} unz] Bit U \newline
\end{minipage}
\end{table}
\end{document}
