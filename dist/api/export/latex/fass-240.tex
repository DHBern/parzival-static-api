\documentclass[8pt,a4paper,notitlepage]{article}
\usepackage{fullpage}
\usepackage{ulem}
\usepackage{xltxtra}
\usepackage{datetime}
\renewcommand{\dateseparator}{.}
\dmyyyydate
\usepackage{fancyhdr}
\usepackage{ifthen}
\pagestyle{fancy}
\fancyhf{}
\renewcommand{\headrulewidth}{0pt}
\fancyfoot[L]{\ifthenelse{\value{page}=1}{\today, \currenttime{} Uhr}{}}
\begin{document}
\begin{table}[ht]
\begin{minipage}[t]{0.5\linewidth}
\small
\begin{center}*D
\end{center}
\begin{tabular}{rl}
\textbf{240} & \textbf{\begin{large}S\end{large}wenne ir geprüevet} sînen art,\\ 
 & ir sît gein strîte dâr mite bewart."\\ 
 & Owê, daz er niht vrâgete dô!\\ 
 & des bin ich vür in noch unvrô,\\ 
5 & wand erz enpfienc \textbf{in sîne} hant,\\ 
 & dô was er vrâgens \textbf{mit} \textbf{ermant}.\\ 
 & ouch riwet mich sîn süezer wirt,\\ 
 & \textbf{den} \textbf{ungenande} niht verbirt,\\ 
 & \textbf{des} im von \textbf{vrâgen} \textbf{nû} wære rât.\\ 
10 & genuoc man dâ gegeben hât:\\ 
 & Dies pflâgen, die grifenz an;\\ 
 & \textbf{si} truogenz gerüste \textbf{wider} dan.\\ 
 & \textbf{vier} karrâschen man \textbf{ê} luot.\\ 
 & ieslîch vrouwe ir dienest tuot,\\ 
15 & ê die jungesten, nû die êrsten.\\ 
 & dô schuofen si aber die hêrsten\\ 
 & wider zuo dem Grâle.\\ 
 & dem wirte unt Parzivale\\ 
 & mit zühten neic diu künegîn\\ 
20 & unt al diu juncvröuwelîn.\\ 
 & si brâhten wider în zer tür,\\ 
 & daz si mit \textbf{zuht} ê truogen vür.\\ 
 & Parzival in blicte nâch.\\ 
 & an eime spanbette er \textbf{ersach}\\ 
25 & in einer kemenâten,\\ 
 & ê si nâch in zuo \textbf{getâten},\\ 
 & den aller schœnsten alten man,\\ 
 & \textbf{des er} kunde ie gewan.\\ 
 & ich \textbf{mag ez} wol \textbf{sprechen} âne guft:\\ 
30 & er was noch \textbf{wîzer} dan \textbf{der} tuft.\\ 
\end{tabular}
\scriptsize
\line(1,0){75} \newline
D \newline
\line(1,0){75} \newline
\textbf{1} \textit{Initiale} D  \textbf{3} \textit{Majuskel} D  \textbf{11} \textit{Majuskel} D  \newline
\line(1,0){75} \newline
\newline
\end{minipage}
\hspace{0.5cm}
\begin{minipage}[t]{0.5\linewidth}
\small
\begin{center}*m
\end{center}
\begin{tabular}{rl}
 & \textbf{wenne ir gebrüefet} sînen art,\\ 
 & ir sît gegen strîte dâ mite bewart."\\ 
 & \begin{large}O\end{large}wê, daz er niht vrâgete dô!\\ 
 & des bin \textit{ich} vür in n\textit{o}ch unvrô,\\ 
5 & wanne \textbf{dô} erz enpfienc \textbf{in sîn} hant,\\ 
 & dô was e\textit{r} vrâgens \textbf{mite} \textbf{er\textit{m}ant}.\\ 
 & ouch riuwet mich sîn süezer wirt,\\ 
 & \textbf{den} \textbf{ungnâde} niht verbirt,\\ 
 & \textbf{des} ime von \textbf{vrâge} \textbf{nû} wære rât.\\ 
10 & genuoc man dâ gegeben hât:\\ 
 & die es pflâgen, die grifenz an;\\ 
 & \textbf{si} truogenz gerüste \textbf{wider} dan.\\ 
 & \textbf{die} karrâschen man \textbf{d\textit{ô}} luot.\\ 
 & ieglîchiu vrouwe ir dienest tuot,\\ 
15 & ê die jungesten, nû die êrsten.\\ 
 & dô schuofens aber die hêrste\textit{n}\\ 
 & wider zuo dem Grâle.\\ 
 & dem wirt und Parcifale\\ 
 & mit zühten neic diu künigîn\\ 
20 & und alliu diu juncvröuwelîn.\\ 
 & si brâh\textit{t}en wider în zer tür,\\ 
 & daz si mit \textbf{in} ê truogen vür.\\ 
 & Parcifal in blicte nâch.\\ 
 & an einem spanbette er \textbf{sach}\\ 
25 & in einer kemenâten,\\ 
 & ê si nâch in zuo \textbf{getâten},\\ 
 & den aller schœnesten alten man.\\ 
 & \textbf{der \textit{ir}} kunde ie gewan,\\ 
 & ich \textbf{mac ez} wol \textbf{sprechen} âne guft:\\ 
30 & er was noch \textbf{grâwer} danne tuft.\\ 
\end{tabular}
\scriptsize
\line(1,0){75} \newline
m n o Fr69 \newline
\line(1,0){75} \newline
\textbf{3} \textit{Initiale} m Fr69   $\cdot$ \textit{Capitulumzeichen} n  \newline
\line(1,0){75} \newline
\textbf{1} wenne] Swe::: Fr69  $\cdot$ gebrüefet] gepruͯfen n \textbf{2} ir sît gegen strîte] Gegen strit sit ir n o  $\cdot$ dâ mite] mit im Fr69  $\cdot$ bewart] gewart o \textbf{4} ich] \textit{om.} m  $\cdot$ noch] nach m \textbf{6} er] es m  $\cdot$ mite ermant] [vnge]: mite erwant m nit ermant o \textbf{9} des] Das o  $\cdot$ von] waz Fr69  $\cdot$ nû] niemer o \textbf{10} dâ] do n o \textbf{11} es] [s]: es m \textbf{13} die] Vier n o  $\cdot$ dô] die m \textbf{16} hêrsten] herstens m \textbf{19} zühten] zuhsten o \textbf{21} brâhten] brachen m \textbf{22} si] \textit{om.} o  $\cdot$ truogen] brachte Fr69 \textbf{23} blicte] blicket n o \textbf{26} si] siz Fr69 \textbf{27} den] Der o  $\cdot$ alten] eyn alter o \textit{om.} Fr69 \textbf{28} ir kunde] enkunde m \textbf{30} noch] ouch n [*]: auch o  $\cdot$ grâwer] graher o  $\cdot$ tuft] der túfft n (o) \newline
\end{minipage}
\end{table}
\newpage
\begin{table}[ht]
\begin{minipage}[t]{0.5\linewidth}
\small
\begin{center}*G
\end{center}
\begin{tabular}{rl}
 & \textbf{swenne ir geprüevet} sînen art,\\ 
 & ir sît gein strîte dâr mite bewart."\\ 
 & owê, daz er niht vrâgte dô!\\ 
 & des bin ich vür in noch unvrô,\\ 
5 & wan \textbf{dô} erz enpfie \textbf{in sîne} hant,\\ 
 & dô was er vrâgens \textbf{dâr mite} \textbf{gemant}.\\ 
 & ouch riuwet mich sîn süezer wirt.\\ 
 & \textbf{ungenâde} \textbf{in} niht verbirt,\\ 
 & \textbf{des} im von \textbf{vrâge} wære rât.\\ 
10 & genuoc man dâ gegeben hât:\\ 
 & dies pflâgen, die grifenz an\\ 
 & \textbf{unt} truogenz gerüste \textbf{wider} dan.\\ 
 & \textbf{vier} k\textit{a}rrâschen man \textbf{dô} luot.\\ 
 & ieslîch vrouwe ir dienst tuot,\\ 
15 & ê die jungesten, nû die êrsten.\\ 
 & dô schuofen si aber die hêrsten\\ 
 & wider zuo dem Grâle.\\ 
 & dem wirte unde Parzivale\\ 
 & mit zühten neic diu künigîn\\ 
20 & unde al diu juncvröuwelîn.\\ 
 & si brâhten wider în zer tür,\\ 
 & daz si mit \textbf{zuht} ê truogen vür.\\ 
 & Parzival i\textit{n} blicte nâch.\\ 
 & an einem spanbette er \textbf{sach}\\ 
25 & in einer kemenâten,\\ 
 & ê si nâch in zuo \textbf{tâten},\\ 
 & den aller schœnsten alten man,\\ 
 & \textbf{des er} kunde ie gewan.\\ 
 & ich \textbf{muoz} wol \textbf{sprechen} âne guft:\\ 
30 & er was noch \textbf{grâwer} danne \textbf{ein} tuft.\\ 
\end{tabular}
\scriptsize
\line(1,0){75} \newline
G I O L M Q R Z \newline
\line(1,0){75} \newline
\textbf{3} \textit{Initiale} I L Z  \textbf{9} \textit{Initiale} R  \textbf{13} \textit{Initiale} I  \textbf{23} \textit{Capitulumzeichen} L  \newline
\line(1,0){75} \newline
\textbf{1} Gepruͯfet ir rechte sin art L  $\cdot$ swenne] [swem]: swen I Swenn et O Wann Q (R)  $\cdot$ geprüevet] gebrvnet O gefruhet Q  $\cdot$ sînen] sin I L (M) Z \textbf{2} gein] an I O Q R in L \textbf{3} owê] Awe O  $\cdot$ vrâgte] fragt O \textbf{4} vür in noch] fvr in O noch fur in Q (R) \textbf{5} dô] daz O da M Z  $\cdot$ sîne] die O L (Q) \textbf{6} dô] Da O M Z  $\cdot$ dâr mite] \textit{om.} M  $\cdot$ gemant] ermant R \textbf{7} ouch] noch I  $\cdot$ sîn] \textit{om.} O mein Q  $\cdot$ süezer] reiner L \textbf{8} ungenâde in] Den vngenade L (Q) (Z) Der vngnade R \textbf{9} des] Der L Das Q  $\cdot$ im] nv L  $\cdot$ vrâge] fragen I O fragen nű Q (Z) \textbf{10} dâ] do Q  $\cdot$ gegeben] geben R \textbf{11} dies] die sin I (O) (M) (Q) (R) (Z)  $\cdot$ die] si O [sie]: do M \textit{om.} R \textbf{12} truogenz] truͤgen daz I (L)  $\cdot$ wider] \textit{om.} L \textbf{13} vier] Swie er I Vie R  $\cdot$ karrâschen] chræschen G tscharroten O koͯrb R  $\cdot$ dô] da I L M Z \textbf{14} vrouwe] varwe O  $\cdot$ dienst] dinste L Q (R) \textbf{15} ê] Die ê O \textbf{16} dô] Da M Z  $\cdot$ schuofen si] liezen si O (L) (M) schuͦffencz R schuffenz Z  $\cdot$ aber] \textit{om.} O  $\cdot$ hêrsten] herschen R \textbf{18} dem] den I  $\cdot$ Parzivale] [parzifal]: Parzifal I Parcifale O (L) (Z) [p*]: parzifale  M partzifale Q parczifale R \textbf{20} al] ander L \textbf{21} wider în] wider O (M) in wider Q \textbf{22} mit zuht ê] mit zuhten ê I (R) mit zvhten O (Q) E mit zuͯchten L  $\cdot$ truogen] brahten Z \textbf{23} Parzival] [parzifal]: Parzifal I Barcifal O Parcifal L Z Parzifal M Partzifal Q Parczifal R  $\cdot$ in] im G \textbf{24} an einem] ein I  $\cdot$ er] er do I \textbf{26} ê si nâch in] Er si en nach M \textbf{28} kunde ie] ie chunde I \textbf{29} muoz] mag ez L (M) (Q) (Z) mag das R \textbf{30} noch] \textit{om.} O \newline
\end{minipage}
\hspace{0.5cm}
\begin{minipage}[t]{0.5\linewidth}
\small
\begin{center}*T
\end{center}
\begin{tabular}{rl}
 & \textbf{geprüevet ir} \textbf{rehte} sînen art,\\ 
 & ir sît gegen strîte dâr mit bewart."\\ 
 & Ouwê, daz er niht vrâgete dô!\\ 
 & des bin ich vür in noch unvrô,\\ 
5 & wan \textbf{dô} erz enpfienc \textbf{von sîner} hant,\\ 
 & dô was er vrâgens \textbf{gemant}.\\ 
 & ouch riuwet mich sîn süezer wirt,\\ 
 & \textbf{den} \textbf{ungnâde} niht verbirt,\\ 
 & \textbf{der} im von \textbf{vrâgene} wære rât.\\ 
10 & Genuoc man dâ gegeben hât:\\ 
 & die\textit{s} pflâgen, die grifen\textit{z} an\\ 
 & \textbf{unde} truogen daz gerüste dan.\\ 
 & \textbf{Vier} karratschen man \textbf{dâ} luot.\\ 
 & ieglîch vrouwe ir dienst tuot,\\ 
15 & ê die jungesten, nû die êrsten.\\ 
 & dô schuofens aber die hêrsten\\ 
 & wider zuome Grâle.\\ 
 & dem wirte unde Parcifale\\ 
 & mit zühten neic diu künegîn\\ 
20 & unde al diu juncvröuwelîn.\\ 
 & si brâhten wider în zer tür,\\ 
 & daz si mit \textbf{zühten} ê truogen vür.\\ 
 & \begin{large}P\end{large}arcifal in blicte nâch.\\ 
 & an einem spanbette er \textbf{sach}\\ 
25 & in einer kemenâten,\\ 
 & ê si nâch in zuo \textbf{getâten},\\ 
 & den aller schœnesten alten man,\\ 
 & \textbf{des er} kunde ie gewan.\\ 
 & ich \textbf{mag ez} wol \textbf{sagen} âne guft:\\ 
30 & er was noch \textbf{grâwer} dan \textbf{der} tuft.\\ 
\end{tabular}
\scriptsize
\line(1,0){75} \newline
T U V W \newline
\line(1,0){75} \newline
\textbf{3} \textit{Initiale} U V W   $\cdot$ \textit{Majuskel} T  \textbf{10} \textit{Majuskel} T  \textbf{13} \textit{Majuskel} T  \textbf{23} \textit{Initiale} T  \newline
\line(1,0){75} \newline
\textbf{1} [*]: swen ir gepruͤvent sin art V \textbf{4} vür in noch unvrô] noch fúr in nit fro W \textbf{5} von sîner] [*]: in sine V in sein W \textbf{6} gemant] do mide ermant U mitte v́rmant V do mit erwand W \textbf{7} sîn] mein W  $\cdot$ süezer] [*]: suͤzzer V \textbf{8} ungnâde] gnade W \textbf{9} der] [De*]: Dez V  $\cdot$ von vrâgene] fragens W  $\cdot$ wære] [*]: nv were V wer worden W \textbf{10} dâ] do U V W \textbf{11} dies] diez T  $\cdot$ grifenz] grîfens T griffen U \textbf{12} dan] [*]: wider dan V \textbf{13} karratschen] karren W  $\cdot$ dâ] e U do V W \textbf{14} ir] Jrn U \textbf{15} nû] vnd U [*]: nv V \textbf{18} Parcifale] Parzifale T (V) Parfale U partzifale W \textbf{20} diu] din U \textbf{23} Parcifal] Parzifal T (U) (V) Partzifal W \textbf{24} sach] do sach W \textbf{26} getâten] [getân]: getâten T taten W \textbf{28} des er] Den er zuͦ W \textbf{29} sagen] [*]: sprechen V sprechen W \newline
\end{minipage}
\end{table}
\end{document}
