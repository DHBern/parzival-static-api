\documentclass[8pt,a4paper,notitlepage]{article}
\usepackage{fullpage}
\usepackage{ulem}
\usepackage{xltxtra}
\usepackage{datetime}
\renewcommand{\dateseparator}{.}
\dmyyyydate
\usepackage{fancyhdr}
\usepackage{ifthen}
\pagestyle{fancy}
\fancyhf{}
\renewcommand{\headrulewidth}{0pt}
\fancyfoot[L]{\ifthenelse{\value{page}=1}{\today, \currenttime{} Uhr}{}}
\begin{document}
\begin{table}[ht]
\begin{minipage}[t]{0.5\linewidth}
\small
\begin{center}*D
\end{center}
\begin{tabular}{rl}
\textbf{476} & \textit{\begin{large}M\end{large}}issewende was sîn riwe,\\ 
 & er balsem ob der triwe.\\ 
 & al werltlîchiu schande in vlôch;\\ 
 & werdecheit sich in sîn herze zôch.\\ 
5 & dich solden hazzen werdiu wîp\\ 
 & durch sînen minneclîchen lîp.\\ 
 & sîn dienst was gein \textbf{in} sô ganz,\\ 
 & \textbf{ez} machte wîbes ougen glanz,\\ 
 & die in gesâhen, von sîner süeze.\\ 
10 & got daz erbarmen müeze,\\ 
 & daz dû ie gevrumtest sölhe nôt.\\ 
 & mîn swester lac \textbf{ouch} nâch dir tôt,\\ 
 & Herzeloyde, dîn muoter."\\ 
 & "neinâ, hêrre guoter!\\ 
15 & waz sagt ir \textbf{nû}?", sprach Parzival.\\ 
 & "wære ich denne hêrre übern Grâl,\\ 
 & der m\textit{ö}hte mich ergetzen niht\\ 
 & \textbf{des mæres} \textbf{mir} iwer munt vergiht.\\ 
 & bin ich iwer swester kint,\\ 
20 & sô tuot, als die mit triwen sint,\\ 
 & unt sagt mir sunder \textbf{wankes} vâr:\\ 
 & sint disiu mære beidiu wâr?"\\ 
 & Dô sprach aber der guote man:\\ 
 & "ich \textbf{en}bin\textbf{z} niht, der \textbf{dâ} \textbf{triegen} kan.\\ 
25 & dîner muoter daz ir triwe \textbf{erwarp}:\\ 
 & dô dû von ir schiede, zehant si starp.\\ 
 & dû wære daz tier, daz si \textbf{dâ} souc,\\ 
 & unt der trache, der von ir \textbf{dâ} vlouc.\\ 
 & ez widervuor \textbf{in slâfe ir} gar,\\ 
30 & ê daz diu süeze dich gebar.\\ 
\end{tabular}
\scriptsize
\line(1,0){75} \newline
D Fr31 \newline
\line(1,0){75} \newline
\textbf{1} \textit{Initiale} D  \textbf{23} \textit{Majuskel} D  \newline
\line(1,0){75} \newline
\textbf{1} Missewende] ÷issewende \textit{nachträglich korrigiert zu:} Missewende D \textbf{13} Herzeloyde] Herceloyde D \textbf{15} Parzival] Parcifal D :::l Fr31 \textbf{17} möhte] mohte D \newline
\end{minipage}
\hspace{0.5cm}
\begin{minipage}[t]{0.5\linewidth}
\small
\begin{center}*m
\end{center}
\begin{tabular}{rl}
 & missewende was sîn riuwe,\\ 
 & er balsam ob \textit{d}er triuwe.\\ 
 & al werltlîchiu schande in vlôch;\\ 
 & wirdicheit sich in sîn herze zôch.\\ 
5 & dich solten hazzen werdiu wîp\\ 
 & durch sînen minneclîchen lîp.\\ 
 & sîn dienst was gegen \textbf{im} sô ganz,\\ 
 & \textbf{er} m\textit{a}chte wîbes ougen glanz,\\ 
 & die in gesâhen, von sîner süeze.\\ 
10 & got daz erbarmen müeze,\\ 
 & daz dû ie gevromtest soliche nôt.\\ 
 & mîn swester lac nâch dir tôt,\\ 
 & Herczeloide, dîn muoter."\\ 
 & "neinâ, hêrre guoter!\\ 
15 & waz saget ir \textbf{mir}?", sprach Parcifal.\\ 
 & "wær ich den hêrre über den Grâl,\\ 
 & der möhte mich ergetzen niht\\ 
 & \textbf{de\textit{s} mæres} \textbf{mir} iuwer munt vergiht.\\ 
 & bin ich iuwer sweste\textit{r} \textit{k}int,\\ 
20 & \textit{s}ô tuot, als die mit triuwen sint,\\ 
 & und sagt mir sunder \textbf{wankes} vâr:\\ 
 & sint disiu mære beidiu wâr?"\\ 
 & dô sprach aber der guote man:\\ 
 & "ich bin\textbf{z} niht, der \textbf{kriegen} kan.\\ 
25 & dîner muoter daz ir triuw\textit{e} \textbf{\textit{w}arp}:\\ 
 & dô dû von \textit{ir} schiede, zehant si starp.\\ 
 & dû wær daz tier, daz si \textbf{d\textit{â}} s\textit{o}uc,\\ 
 & und der trach, der von ir vl\textit{o}uc.\\ 
 & ez widervuor \textbf{ir in slâfe} gar,\\ 
30 & ê daz diu süeze \textit{dich} gebar.\\ 
\end{tabular}
\scriptsize
\line(1,0){75} \newline
m n o \newline
\line(1,0){75} \newline
\newline
\line(1,0){75} \newline
\textbf{2} der] er m \textbf{4} sich] >sich< o \textbf{5} solten] solte o \textbf{6} minneclîchen] mẏnnecliche o \textbf{7} im] in n o \textbf{8} machte] mohte m \textbf{10} müeze] musse e o \textbf{13} Herczeloide] Hertzeloide n Liebe herczeleide o \textbf{15} mir] nuͯ n ẏm o \textbf{16} den] \textit{om.} n \textbf{17} möhte] mochte n (o) \textbf{18} des] Der m \textbf{19} swester kint] [swer]: swester suͯn vnd kint m \textbf{20} sô] Do m  $\cdot$ die] \textit{om.} o \textbf{24} kriegen] kriege o \textbf{25} dîner] [Da]: Diner o  $\cdot$ triuwe warp] truwe was vnd warp m truwe erwarp n o \textbf{26} ir] \textit{om.} m  $\cdot$ schiede] scheide o  $\cdot$ zehant] \sout{so} zú hant o \textbf{27} dâ] do m n o  $\cdot$ souc] sluͯg m \textbf{28} vlouc] fluͯg m floch o \textbf{29} widervuor] wider fuͦre n fúr wider fúr o \textbf{30} dich] wol m \newline
\end{minipage}
\end{table}
\newpage
\begin{table}[ht]
\begin{minipage}[t]{0.5\linewidth}
\small
\begin{center}*G
\end{center}
\begin{tabular}{rl}
 & \begin{large}M\end{large}issewende was sîn riuwe,\\ 
 & er balse\textit{m} ob der triuwe.\\ 
 & al werltlîchiu schande in vlôch;\\ 
 & werdecheit sich in sîn herze zôch.\\ 
5 & dich solden hazzen werdiu wîp\\ 
 & durch sînen minneclîchen lîp.\\ 
 & sîn dienst was gegen \textbf{i\textit{n}} sô ganz,\\ 
 & \textbf{ez} machet wîbes ougen glanz,\\ 
 & die in gesâhen, von sîner süeze.\\ 
10 & got daz erbarmen müeze,\\ 
 & daz d\textit{û} ie gevrumtest solhe nôt.\\ 
 & mîn swester lac \textbf{ouch} nâch dir tôt,\\ 
 & Herzeloide, dîn muoter."\\ 
 & "neinâ, hêrre guoter!\\ 
15 & waz saget ir \textbf{nû}?", sprach Parzival.\\ 
 & "wære ich danne hêrre über den Grâl,\\ 
 & der m\textit{ö}hte mich erge\textit{t}zen niht\\ 
 & \textbf{des mæres}, \textbf{des} iuwer munt vergiht.\\ 
 & bin ich iuwer swester kint,\\ 
20 & sô tuot, als die mit triuwen sint,\\ 
 & und saget mir sunde\textit{r} \textbf{wankes} vâr:\\ 
 & sint disiu mære beidiu wâr?"\\ 
 & dô sprach aber der guote man:\\ 
 & "ich \textbf{en}bin\textbf{z} niht, der \textbf{dâ} \textbf{triegen} kan.\\ 
25 & dîner muoter daz ir triuwe \textbf{warp}:\\ 
 & dô dû von ir schiet, zehant si starp.\\ 
 & dû wær daz tier, da\textit{z} si \textbf{dâ} souc,\\ 
 & unt der trache, der von ir \textbf{dâ} vlouc.\\ 
 & ez widervuor \textbf{ir in ir slâfe} gar,\\ 
30 & ê daz diu süeze dich ge\textit{b}ar.\\ 
\end{tabular}
\scriptsize
\line(1,0){75} \newline
G I O L M Z Fr18 Fr49 \newline
\line(1,0){75} \newline
\textbf{1} \textit{Initiale} G I O L Z Fr18  \textbf{19} \textit{Initiale} I  \newline
\line(1,0){75} \newline
\textbf{1} Missewende] ÷issewende O  $\cdot$ was] \textit{om.} O Fr18 \textbf{2} balsem] balsent G  $\cdot$ ob] uff M \textbf{3} werltlîchiu] werlt I \textbf{4} \textit{Versdoppelung 476.4 nach 476.28} O  \textbf{5} werdiu] ellev I \textbf{7} in] im G \textbf{8} ez] er I Jch O  $\cdot$ machet] mahte I (Z) \textbf{9} in] \textit{om.} I \textbf{11} dû] de G \textbf{12} nâch] vor L von Fr18 \textbf{13} Herzeloide] Herzeloyde G herzenlaude I Herzelavde O Hertzeleuͯde L Herczeloude M Hertzelovde Z Herzelovde Fr18 \textbf{15} Parzival] parziual G Parzifal I (L) (M) Fr49 Barcifal O parcifal Z (Fr18) \textbf{17} der] Dern Fr18  $\cdot$ möhte] mohte G (O) (L) (M) (Fr18) (Fr49)  $\cdot$ ergetzen] ergezzen G I (M) (Fr49) \textbf{18} des mæres] Der mere L  $\cdot$ des iuwer] des mir ewer I (O) (M) (Fr18) der uwer L mir ewer Z (Fr49)  $\cdot$ vergiht] giht I (Fr49) \textbf{19} \textit{nach 476.19:} \sout{So bin ich iwer swester sint} O   $\cdot$ bin] [Durch]: Bin G  $\cdot$ ich] \textit{om.} O \textbf{21} sunder] svnders G  $\cdot$ wankes] valsches O Fr18 valschen L \textbf{22} Sint dise maͤr beide war Fr49 \textbf{23} dô] Da M \textbf{24} enbinz] binz I (Fr49) bin O en bin L  $\cdot$ dâ] \textit{om.} O M Fr18 \textbf{25} dîner] Dine L  $\cdot$ ir] [mit]: ir O  $\cdot$ triuwe] triwen O  $\cdot$ warp] erwarp I L M Z Fr18 (Fr49) \textbf{26} dô] Da M (Fr49) \textbf{27} wær] [wart]: waert Fr49  $\cdot$ daz si] da si G  $\cdot$ dâ] \textit{om.} L Fr49  $\cdot$ souc] sluͦc I (Fr49) \textbf{28} von ir dâ] si da O von ir L Fr18  $\cdot$ vlouc] [vloch]: vlach I sovch O \textbf{29} ir in ir slâfe] ir in shlafe I (O) (Fr18) (Fr49) in slaffe ir L (M) (Z) \textbf{30} süeze] suͤzzev I  $\cdot$ gebar] gevar G \newline
\end{minipage}
\hspace{0.5cm}
\begin{minipage}[t]{0.5\linewidth}
\small
\begin{center}*T
\end{center}
\begin{tabular}{rl}
 & missewende was sîn riuwe,\\ 
 & er balsem\textit{e} ob der triuwe.\\ 
 & all\textit{iu} werltlîche schande in vlôch;\\ 
 & werdecheit sich in sîn herze zôch.\\ 
5 & dich solten hazzen werdiu wîp\\ 
 & durch sînen minneclîchen lîp.\\ 
 & sîn dienst was gegen \textbf{im} sô ganz,\\ 
 & \textbf{er} machte wîbes ougen glanz,\\ 
 & die in gesâhen, von sîner süeze.\\ 
10 & got daz erbarmen müeze,\\ 
 & daz dû ie gevrumtest sölhe nôt.\\ 
 & mîn swester lac \textbf{ouch} nâch dir tôt,\\ 
 & Herzeloyde, dîn muoter."\\ 
 & "Neinâ, hêrre guoter!\\ 
15 & waz saget ir \textbf{nû}?", sprach Parcifal.\\ 
 & "wærich danne hêrre übern Grâl,\\ 
 & der m\textit{ö}hte mich erge\textit{t}zen niht\\ 
 & \textbf{der mære}, \textbf{der} \textbf{mir} iuwer munt vergiht.\\ 
 & bin ich iuwerre swester kint,\\ 
20 & sô tuot, als die mit triuwen sint,\\ 
 & unde saget mir sunder \textbf{valschen} vâr:\\ 
 & sint disiu mære beidiu wâr?"\\ 
 & \begin{large}D\end{large}ô sprach aber der guote man:\\ 
 & "ich bin niht, der \textbf{triegen} kan.\\ 
25 & dîner muoter daz ir triuwe \textbf{warp}:\\ 
 & dô dû von ir schiede, zehant si starp.\\ 
 & dû wære daz tier, daz si souc,\\ 
 & unde der trache, der von ir vlouc.\\ 
 & ez widervuor \textbf{in slâfe ir} gar,\\ 
30 & ê daz diu süeze dich gebar.\\ 
\end{tabular}
\scriptsize
\line(1,0){75} \newline
T U V W Q R \newline
\line(1,0){75} \newline
\textbf{1} \textit{Initiale} Q  \textbf{14} \textit{Majuskel} T  \textbf{23} \textit{Initiale} T V W  \newline
\line(1,0){75} \newline
\textbf{1} \textit{Die Verse 453.1-502.30 fehlen} U  \textbf{2} balseme] balsemt T \textbf{3} alliu werltlîche] alle weltliche T R  $\cdot$ in] ich Q \textbf{4} sich] \textit{om.} V \textbf{5} werdiu] werde R \textbf{6} minneclîchen] minnentlichen Q \textbf{7} im] in V W \textbf{9} in gesâhen] Jm geschachen R \textbf{12} dir] [ir]: dir Q \textbf{13} Herzeloyde] Herzelaude V Hertzeloyde W Hertzeloude Q Herczelaude R \textbf{14} guoter] so guͦter W \textbf{15} nû] \textit{om.} R  $\cdot$ Parcifal] parzifal V partzifal W Q parczifal R \textbf{17} möhte] mohte T V (Q)  $\cdot$ ergetzen] ergezzen T \textbf{18} der mære] Des mers W (Q) (R)  $\cdot$ der mir] dez mir V (W) (Q) (R)  $\cdot$ vergiht] gicht W R \textbf{21} valschen vâr] vor V \textbf{22} mære beidiu] beide mere gar R \textbf{23} aber] \textit{om.} W \textbf{24} bin] enbins W R bins Q  $\cdot$ niht der] der niht V \textbf{25} warp] [*]: warp T erwarp V (W) Q ward R \textbf{26} zehant] \textit{om.} R \textbf{27} wære] wert R  $\cdot$ souc] do sauc Q \textbf{28} der] dtr W  $\cdot$ vlouc] do flock Q \textbf{29} in slâfe ir] ir im schlaffe R \textbf{30} diu] sy W  $\cdot$ gebar] [bewar]: gebar Q \newline
\end{minipage}
\end{table}
\end{document}
