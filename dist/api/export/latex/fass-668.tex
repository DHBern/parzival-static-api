\documentclass[8pt,a4paper,notitlepage]{article}
\usepackage{fullpage}
\usepackage{ulem}
\usepackage{xltxtra}
\usepackage{datetime}
\renewcommand{\dateseparator}{.}
\dmyyyydate
\usepackage{fancyhdr}
\usepackage{ifthen}
\pagestyle{fancy}
\fancyhf{}
\renewcommand{\headrulewidth}{0pt}
\fancyfoot[L]{\ifthenelse{\value{page}=1}{\today, \currenttime{} Uhr}{}}
\begin{document}
\begin{table}[ht]
\begin{minipage}[t]{0.5\linewidth}
\small
\begin{center}*D
\end{center}
\begin{tabular}{rl}
\textbf{668} & \textit{\begin{large}S\end{large}}eytiez unt snecken,\\ 
 & mit rotte \textbf{der} quecken\\ 
 & \textbf{beidiu} zorse unt \textbf{ouch} ze vuoz,\\ 
 & mit dem marschalc über muoz,\\ 
5 & Sarjande, garzûne,\\ 
 & hin nâch dem Bertune\\ 
 & si kêrten \textbf{her} unt dâ\\ 
 & \textbf{mit} Gawans marschalke ûf \textbf{die} slâ.\\ 
 & Si vuorten ouch, des sît gewis,\\ 
10 & ein gezelt, daz Iblis\\ 
 & \textbf{Clinschore} durch minne sande,\\ 
 & dâ \textbf{von} man \textbf{êrste} erkande\\ 
 & ir \textbf{zweier} tougen über lût;\\ 
 & si wâren beidiu ein ander trût.\\ 
15 & dem gezelte was \textbf{koste} \textbf{niht vermiten}:\\ 
 & mit schære nie bezzer\textbf{z} wart gesniten\\ 
 & wan einez, daz Isenhartes was.\\ 
 & bî Artuse sunder ûf ein gras\\ 
 & wart \textbf{daz} gezelt ûf geslagen.\\ 
20 & manec \textbf{zelt}, hôrt ich sagen,\\ 
 & sluoc man drumbe \textbf{an} wîten rinc;\\ 
 & daz \textbf{dûhten} \textbf{rîlîchiu} \textbf{dinc}.\\ 
 & \textbf{Vor} Artuse wart vernomen,\\ 
 & Gawans marschalc \textbf{wære} komen,\\ 
25 & der herbergete ûf \textbf{den} plân,\\ 
 & unt \textbf{daz} der werde Gawan\\ 
 & solt \textbf{ouch} komen bî dem tage.\\ 
 & \textbf{daz} wart ein gemeiniu sage\\ 
 & \textbf{von} al der messenîe.\\ 
30 & Gawan, der valsches vrîe,\\ 
\end{tabular}
\scriptsize
\line(1,0){75} \newline
D Fr10 \newline
\line(1,0){75} \newline
\textbf{1} \textit{Initiale} D  \textbf{5} \textit{Majuskel} D  \textbf{9} \textit{Majuskel} D  \textbf{23} \textit{Majuskel} D  \newline
\line(1,0){75} \newline
\textbf{1} Seytiez] ÷eytîez D \textbf{3} beidiu] Baiden Fr10  $\cdot$ ouch] \textit{om.} Fr10 \textbf{5} Sarjande] Sariand vnd Fr10 \textbf{6} Bertune] britune Fr10 \textbf{8} marschalke] marschach Fr10 \textbf{9} vuorten] faͦrtn Fr10 \textbf{10} Iblis] Jblis D Fr10 \textbf{11} Clinschore] Clinscore D Chlinsor Fr10 \textbf{17} Isenhartes] Jsenharts D ::: Fr10 \textbf{18} Artuse] Artusen Fr10  $\cdot$ sunder] stuͦnder Fr10 \textbf{20} zelt] gezelt Fr10 \textbf{22} dûhten] dauhte Fr10 \textbf{23} Vor Artuse] Von Artus Fr10 \textbf{24} marschalc] marschach Fr10 \textbf{25} herbergete] hertzog Fr10 \textbf{28} gemeiniu] gemaine Fr10 \textbf{30} valsches] valsch Fr10 \newline
\end{minipage}
\hspace{0.5cm}
\begin{minipage}[t]{0.5\linewidth}
\small
\begin{center}*m
\end{center}
\begin{tabular}{rl}
 & seitieze und snecken,\\ 
 & mit rotte \textbf{der} quecken\\ 
 & \textbf{beidiu} ze rosse und zuo vuoz,\\ 
 & mit dem marschalc über muo\textit{z},\\ 
5 & sarjande, garzûne,\\ 
 & hin nâch \textit{dem} Britu\textit{n}e\\ 
 & si kêrten \textbf{her} und dâ\\ 
 & \textbf{mit} Gawans marschalc ûf \textbf{die} slâ.\\ 
 & si vuorten ouch, des sît gewis,\\ 
10 & ein gezelt, daz Iblis\\ 
 & \textbf{Clinsor} durch minne sante,\\ 
 & dâ \textbf{von} man \textbf{êrste} erkante\\ 
 & ir \textbf{zweier} tougen über lût;\\ 
 & si wâren beidiu ein ander trût.\\ 
15 & dem gezelte was \textbf{koste} \textbf{niht vermiten}:\\ 
 & mit schære nie bezzer\textbf{z} wart gesniten\\ 
 & wan einez, daz Isenhartes was.\\ 
 & bî Artuse sunder ûf ein gras\\ 
 & wart \textbf{daz} gezelt ûf geslagen.\\ 
20 & manic \textbf{gezel\textit{t}}, hôrte ich sagen,\\ 
 & sluoc man dar umbe, \textbf{einen} wîten rinc;\\ 
 & daz \textbf{dûhte in} \textbf{rîlîch} \textbf{gedinc}.\\ 
 & \textbf{vor} Artuse wart vernomen,\\ 
 & Gawans marschalc \textbf{wær} komen,\\ 
25 & der herberget ûf \textbf{dem} plân,\\ 
 & und \textbf{daz} der werde Gawan\\ 
 & solte \textbf{ouch} komen bî dem tage.\\ 
 & \textbf{daz} wart ein gemeiniu sage\\ 
 & \textbf{von} alle\textit{r} der massenîe.\\ 
30 & Gawan, der valsches vrîe,\\ 
\end{tabular}
\scriptsize
\line(1,0){75} \newline
m n o Fr69 \newline
\line(1,0){75} \newline
\textbf{23} \textit{Initiale} Fr69  \newline
\line(1,0){75} \newline
\textbf{1} seitieze] Setiese m \textbf{4} muoz] muͯsse m n o \textbf{5} Sariande vnd gartzune n \textbf{6} dem] \textit{om.} m  $\cdot$ Britune] brittume m britúne o \textbf{10} Iblis] jblis m (o) ẏblis n \textbf{11} Clinsor] Clinsore n o Fr69 \textbf{12} von] vom Fr69 \textbf{17} wan] [Wem]: Wen o  $\cdot$ Isenhartes] jsenharttes m \textbf{20} gezelt] gezeltte m (o) gezelt oͮch Fr69 \textbf{21} einen wîten] anwite Fr69 \textbf{22} dûhte] dúchte o \textbf{23} vor] Wenne vor n  $\cdot$ Artuse] artuͯse o \textbf{24} Gawans] [*]: Gawans m Gawanes n o \textbf{25} herberget] herbergte Fr69  $\cdot$ dem] den Fr69 \textbf{29} aller] alle m o \newline
\end{minipage}
\end{table}
\newpage
\begin{table}[ht]
\begin{minipage}[t]{0.5\linewidth}
\small
\begin{center}*G
\end{center}
\begin{tabular}{rl}
 & \multicolumn{1}{l}{ - - - }\\ 
 & \multicolumn{1}{l}{ - - - }\\ 
 & \textbf{\textit{\begin{large}S\end{large}}i, di\textit{e}} ze orse unde ze vuo\textit{z}\\ 
 & mit dem marschalke über muo\textit{z},\\ 
5 & sarjande, garzûne,\\ 
 & hin nâch dem Britune\\ 
 & si kêrten \textbf{hin} unde dâ,\\ 
 & \textbf{hin nâch} Gawans marschalke ûf \textbf{ir} slâ.\\ 
 & si vuorten ouch, des sît gewis,\\ 
10 & ein gezelt, daz Ibilis\\ 
 & \textbf{Gawan} durch minne sande,\\ 
 & dâ \textbf{bî} man \textbf{êrste} erkande\\ 
 & ir \textbf{vil} tougen über lût;\\ 
 & si wâren bêde ein ander trût.\\ 
15 & dem gezelte was \textbf{koste} \textbf{niht vermiten}:\\ 
 & mit schære nie bezzer wart gesniten\\ 
 & wan einez, daz Ysenhartes was.\\ 
 & bî Artus sunder ûf ein gras\\ 
 & wart \textbf{ditze} gezelt ûf geslagen.\\ 
20 & manic \textbf{gezelt}, hôrte ich sagen,\\ 
 & sluoc man drumbe \textbf{an} wîten rinc;\\ 
 & daz \textbf{\textit{dûht}en} \textbf{rîchlîchiu} \textbf{dinc}.\\ 
 & \textbf{vor} Artus wart vernomen,\\ 
 & \textbf{daz} Gawans marschalc \textbf{solde} komen,\\ 
25 & der herberget ûf \textbf{den} plân,\\ 
 & unde \textbf{daz} der werde Gawan\\ 
 & solde komen bî dem tage.\\ 
 & \textbf{dô} wart ein gemeiniu sage\\ 
 & al der messenîe.\\ 
30 & Gawan, der valsches vrîe,\\ 
\end{tabular}
\scriptsize
\line(1,0){75} \newline
G I L M Z Fr61 \newline
\line(1,0){75} \newline
\textbf{1} \textit{Initiale} L  \textbf{3} \textit{Initiale} G  \textbf{7} \textit{Initiale} I  \textbf{23} \textit{Initiale} I  \newline
\line(1,0){75} \newline
\textbf{1} \textit{Die Verse 668.1-2 fehlen} G I   $\cdot$ Senes (Seicziez M Seitizzen Z ) vnd snechen L (M) (Z) \textbf{2} Mit rote (rotten Z ) die quechen L (M) (Z) \textbf{3} Si die] Didiv G Beide L M Z  $\cdot$ vuoz] fvͦzzen G \textbf{4} muoz] mvͦzen G mvzze Z \textbf{6} dem] \textit{om.} M  $\cdot$ Britune] pritune I brituͯne M \textbf{7} hin] hie Z \textbf{8} hin nâch] nach I Mit L M Z  $\cdot$ Gawans] Gawanz L gewans M  $\cdot$ ir] die I \textbf{10} Ibilis] yblis I jblis L (Z) iblis M \textbf{11} Gawan] Gawane L M Clingezor Z \textbf{13} vil tougen] tovgen vil L zweier tovgen Z \textbf{15} niht vermiten] vnvermiten L (M) \textbf{16} schære] shern I  $\cdot$ bezzer] bezzerz I (L) (Z)  $\cdot$ wart gesniten] war versniten L \textbf{17} einez] ienc I [eine*]: einez Z  $\cdot$ daz] [w*s]: das M  $\cdot$ Ysenhartes] Isenhartes G ysenhartsz L Jsenarthes M \textbf{18} Artus] Artuse L \textbf{19} ditze gezelt] das [gelt]: gezelt L \textbf{21} an] ein I \textbf{22} dûhten] waren G \textbf{23} vor] Von M  $\cdot$ Artus] Artuse L \textbf{24} daz] Dans M  $\cdot$ Gawans] Gawansz L  $\cdot$ solde] wer Z \textbf{25} der herberget] Zcu der herberge M  $\cdot$ den] dem L \textbf{26} der] \textit{om.} M \textbf{28} dô] daz I (Z) Da M \textbf{29} al] Von al Z \textbf{30} der] des Z  $\cdot$ valsches] valsche I \newline
\end{minipage}
\hspace{0.5cm}
\begin{minipage}[t]{0.5\linewidth}
\small
\begin{center}*T
\end{center}
\begin{tabular}{rl}
 & seitiez und snecken,\\ 
 & mit rotte \textbf{die} quecken\\ 
 & \textbf{beidiu} zuo ros und zuo vuoz\\ 
 & mit dem marschalc über muoz,\\ 
5 & sarjande, garzûne,\\ 
 & hin nâch dem Britune\\ 
 & si kêrten \textbf{her} und dâ\\ 
 & \textbf{mit} Gawans marschalke ûf \textbf{ir} slâ.\\ 
 & si vuorten ouch, des sît gewis,\\ 
10 & ein gezelt, daz Ibilis\\ 
 & \textbf{Clynsor} durch minne sante,\\ 
 & dâ \textbf{bî} man\textbf{z} erkante,\\ 
 & ir \textbf{zweier} tougen über lût;\\ 
 & si wâren beidiu ein ander trût.\\ 
15 & dem gezelte was \textbf{rîcheit} \textbf{unvermiten}:\\ 
 & mit schære nie bezzer wart gesniten\\ 
 & wan einez, daz Isenhartes was.\\ 
 & bî Artus sunder ûf ein gras\\ 
 & wart \textbf{diz} gezelt ûf geslagen.\\ 
20 & manic \textbf{gezelt}, hôrt ich sagen,\\ 
 & sluoc man drumb \textbf{an} wîten rinc;\\ 
 & daz \textbf{dûhten} \textbf{\textit{r}îch\textit{e}lîc\textit{h}iu} \textbf{dinc}.\\ 
 & \textbf{von} Artuse wart vernomen,\\ 
 & \textbf{daz} Gawans marschalc \textbf{wær} komen,\\ 
25 & der herberget ûf \textbf{dem} plân,\\ 
 & und der werde Gawan\\ 
 & solt \textbf{ouch} komen bî dem tage.\\ 
 & \textbf{daz} wart ein gemeiniu sage\\ 
 & \textbf{von} alder massenîe.\\ 
30 & Gawan, der valsches vrîe,\\ 
\end{tabular}
\scriptsize
\line(1,0){75} \newline
Q R W V \newline
\line(1,0){75} \newline
\newline
\line(1,0){75} \newline
\textbf{2} rotte] Rotten R (V) \textbf{5} Sariande vnd [gar]: Garzune R \textbf{6} Britune] brittúnne Q brittune V \textbf{7} dâ] dan R \textbf{8} ûf ir slâ] vnd den schlan R vf [*]: ir sla V \textbf{10} Ibilis] yblis Q W V Jbilis R \textbf{11} Clynsor] Clinszhor Q Clinshor R Klinshor W Clinsor V \textbf{12} manz] man erst R W men [*]: erst V \textbf{15} rîcheit] kost R (W) (V) \textbf{16} bezzer] bas R bessers W V \textbf{17} Isenhartes] eysenhartes Q Jsenharttes R ysenhartes W ysinhartes V \textbf{18} ein] dem R das W \textbf{19} diz] das R (V)  $\cdot$ gezelt] zelt R \textbf{20} ich] ich auch W \textbf{21} man] men me V  $\cdot$ an] einen vil R \textbf{22} dûhten] duchtte sy R daucht in W [duht*]: duhten V  $\cdot$ rîchelîchiu] leichtlichte Q Riliche R \textbf{23} von] [Von*]: Vor V  $\cdot$ Artuse] Artus R \textbf{24} Gawans] Gawins R  $\cdot$ komen] Vff den plan komen R \textbf{25} herberget] herbergte W (V)  $\cdot$ dem] [*]: dem V \textbf{26} der] daz der R (W) (V)  $\cdot$ Gawan] gawann Q \textbf{27} Solte komen [*]: oͮch bi dem tage V \textbf{28} wart] was R  $\cdot$ gemeiniu] gemeine R \textbf{29} alder] aller W [*]: al der V \textbf{30} Gawan] Gawann Q Gawin R  $\cdot$ der] [de*]: der Q des R \newline
\end{minipage}
\end{table}
\end{document}
