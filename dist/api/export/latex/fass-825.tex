\documentclass[8pt,a4paper,notitlepage]{article}
\usepackage{fullpage}
\usepackage{ulem}
\usepackage{xltxtra}
\usepackage{datetime}
\renewcommand{\dateseparator}{.}
\dmyyyydate
\usepackage{fancyhdr}
\usepackage{ifthen}
\pagestyle{fancy}
\fancyhf{}
\renewcommand{\headrulewidth}{0pt}
\fancyfoot[L]{\ifthenelse{\value{page}=1}{\today, \currenttime{} Uhr}{}}
\begin{document}
\begin{table}[ht]
\begin{minipage}[t]{0.5\linewidth}
\small
\begin{center}*D
\end{center}
\begin{tabular}{rl}
\textbf{825} & \begin{large}Z\end{large}\textbf{Antwerp} \textbf{wart er} \textbf{ûz} gezogen.\\ 
 & si was an im vil unbetrogen.\\ 
 & er kunde wol gebâren.\\ 
 & man \textbf{muose} in vür den clâren\\ 
5 & unt vür den manlîchen\\ 
 & haben in \textbf{al den} rîchen,\\ 
 & swâ man sîn künde ie gewan.\\ 
 & höfsch, mit zühten \textbf{wîs ein} man,\\ 
 & \textbf{mit triwen} milte \textbf{ân} \textbf{ander stôz}\\ 
10 & was \textbf{sîn} \textbf{lîp}, missewende blôz.\\ 
 & Des landes vrouwe in \textbf{schône} enpfienc.\\ 
 & nû hœret, wie sîn rede ergienc.\\ 
 & rîch unt arme \textbf{ez} hôrten,\\ 
 & \textbf{die dâ stuonden} \textbf{an} \textbf{allen orten}.\\ 
15 & \textbf{Dô} sprach \textbf{er}: "vrouwe herzogîn,\\ 
 & \textbf{sol ich} \textbf{hie} landes hêrre sîn,\\ 
 & dâr umbe lâz ich als vil.\\ 
 & nû \textbf{hœret}, wes ich \textbf{iuch} \textbf{bitten} wil:\\ 
 & gevrâget nimmer, wer ich sî,\\ 
20 & sô mag ich iu belîben bî.\\ 
 & bin ich \textbf{z}iuwerer \textbf{vrâge} \textbf{erkorn},\\ 
 & sô habt ir minne an mir verlorn.\\ 
 & \textbf{ob} ir niht \textbf{sît} gewarnet des,\\ 
 & sô \textbf{warnet} mich got, \textbf{er} weiz wol wes."\\ 
25 & Si sazte wîbes sicherheit,\\ 
 & diu sît durch liebe wenken leit,\\ 
 & si wolde ze sîme gebote stên\\ 
 & unt nimmer übergên,\\ 
 & swaz er si leisten hieze,\\ 
30 & ob si got bî sinne lieze.\\ 
\end{tabular}
\scriptsize
\line(1,0){75} \newline
D \newline
\line(1,0){75} \newline
\textbf{1} \textit{Initiale} D  \textbf{11} \textit{Majuskel} D  \textbf{15} \textit{Majuskel} D  \textbf{25} \textit{Majuskel} D  \newline
\line(1,0){75} \newline
\newline
\end{minipage}
\hspace{0.5cm}
\begin{minipage}[t]{0.5\linewidth}
\small
\begin{center}*m
\end{center}
\begin{tabular}{rl}
 & zuo \textbf{Antwer\textit{p}} \textbf{war\textit{t} e\textit{r}} \textbf{ûz} gezogen.\\ 
 & si was an im vil unbetrogen.\\ 
 & er kunde wol gebâren.\\ 
 & man \textbf{muoste} in vür den clâren\\ 
5 & und vür den manlîchen\\ 
 & haben in \textbf{allen} rîchen,\\ 
 & wâ man sîn künde ie gewan.\\ 
 & h\textit{ü}bsch, mit zühten \textbf{wîs ein} man,\\ 
 & \textbf{mit triuwen} milte \textbf{âne} \textbf{âderstôz}\\ 
10 & was \textbf{sîn} \textbf{lîp}, missewende blôz.\\ 
 & des landes vrowe in \textbf{schôn} enpfienc.\\ 
 & nû hœret, wie sîn rede ergienc.\\ 
 & rîch und arm \textbf{ez} hôrten,\\ 
 & \textbf{die d\textit{â} stuonden} \textbf{an} \textbf{allen orten}.\\ 
15 & \textbf{er} sprach: "\textbf{mîn} vrowe herzogîn,\\ 
 & \textbf{sol ich} \textbf{hie} landes \textit{h}êrre sîn,\\ 
 & dâr umb lâz ich als vil.\\ 
 & nû \textbf{merket}, wes ich \textbf{iuch} \textbf{bitten} wil:\\ 
 & gevrâget niemer, wer ich sî,\\ 
20 & sô mac ich iu blî\textit{b}en bî.\\ 
 & bin ich \textbf{ane} iuwer \textbf{vrâge} \textbf{erkorn},\\ 
 & sô habt ir minne an mir verlorn.\\ 
 & \textbf{ob} ir niht \textbf{sît} gewarnet des,\\ 
 & sô \textbf{warnet} mich got, \textbf{er} weiz wol wes."\\ 
25 & si s\textit{a}zte wîbes sicherheit,\\ 
 & diu sît durch liebe wenken leit,\\ 
 & si wolte zuo sînem gebote stân\\ 
 & und nimer übergân,\\ 
 & waz er si leisten hiez,\\ 
30 & ob si got bî sinnen liez.\\ 
\end{tabular}
\scriptsize
\line(1,0){75} \newline
m n V V' W \newline
\line(1,0){75} \newline
\textbf{11} \textit{Initiale} W  \newline
\line(1,0){75} \newline
\textbf{1} \textit{statt 824.23-826.2:} Wie er zu der herzoginnen gein brabant quam (vgl. 825.15: herzogîn) / Vnd die zu einer amyen nam (Fortsetzung von 824.2; weiterer Text in 826.23) V'   $\cdot$ Antwerp] antwerg m [antwer*]: antwerg n antwerb W  $\cdot$ wart er] woren m begund er W  $\cdot$ gezogen] zogen W \textbf{2} vil] vil gar W \textbf{3} gebâren] goboren V \textbf{4} muoste] muͤste V \textbf{7} wâ] Swa V \textbf{8} hübsch] Hbsch m \textbf{9} âderstôz] vnderstoß W \textbf{10} missewende] missewenden V \textbf{12} ergienc] ging W [*]: ergieng V \textbf{13} ez hôrten] erhorten W \textbf{14} dâ] do m n V W \textbf{16} hêrre] fere m \textbf{17} als] eúch wol so W \textbf{18} merket] mercken n  $\cdot$ wes] was m n wie V \textbf{20} blîben] bligen m  $\cdot$ bî] hie W \textbf{21} ane] zvͦ V \textbf{24} er] ich W \textbf{25} sazte] santztte m \textbf{29} waz] Swaz V  $\cdot$ hiez] hiesze V \textbf{30} sinnen liez] sinne liesze V \newline
\end{minipage}
\end{table}
\newpage
\begin{table}[ht]
\begin{minipage}[t]{0.5\linewidth}
\small
\begin{center}*G
\end{center}
\begin{tabular}{rl}
 & \begin{large}Z\end{large}e \textbf{Antwerp} \textbf{er wart} \textbf{ûz} gezogen.\\ 
 & si was an im vil unbetrogen.\\ 
 & er kunde wol gebâren.\\ 
 & man \textbf{muose} in vür den clâren\\ 
5 & unde vür den manlîchen\\ 
 & haben in \textbf{allen} rîchen,\\ 
 & swâ man sîn künde ie gewan.\\ 
 & hövesch, mit zühten, \textbf{ein wîse} man,\\ 
 & \textbf{getriu}, milte, \textbf{ân} \textbf{âderstôz}\\ 
10 & was \textbf{sîn} \textbf{lîp}, missewende blôz.\\ 
 & des landes vrouwe in \textbf{wol} enpfienc.\\ 
 & nû hœrt, wie sîn rede ergienc,\\ 
 & \textbf{daz si} rîche unde arme hôrten,\\ 
 & \textbf{die dâ stuonden} \textbf{in} \textbf{allen orten}.\\ 
15 & \textbf{dô} sprach \textbf{er}: "vrouwe herzogîn,\\ 
 & \textbf{ich sol} \textbf{hie} landes hêrre sîn,\\ 
 & dâr umbe lâze ich als vil.\\ 
 & nû \textbf{hœrt}, wes ich \textbf{biten} wil:\\ 
 & gevrâget nimer, wer ich sî,\\ 
20 & sô mag ich iu belîben bî.\\ 
 & bin ich \textbf{zuo} iwer \textbf{vrâge} \textbf{erborn},\\ 
 & sô habet ir minne an mir verlorn.\\ 
 & \textbf{sît} ir niht \textbf{vor} gewarnet des,\\ 
 & sô \textbf{warnet} mich got, \textbf{er} weiz wol wes."\\ 
25 & si sazete wîbes sicherheit,\\ 
 & diu sît durch liebe wenken leit,\\ 
 & si wolde ze sînem gebot stên\\ 
 & unde nimmer übergên,\\ 
 & swaz er si leisten hieze,\\ 
30 & ob si got bî sinne lieze.\\ 
\end{tabular}
\scriptsize
\line(1,0){75} \newline
G I L Z \newline
\line(1,0){75} \newline
\textbf{1} \textit{Initiale} G L Z  \textbf{3} \textit{Initiale} I  \textbf{25} \textit{Initiale} I  \newline
\line(1,0){75} \newline
\textbf{1} Ze Antwerp] Vze Antwerp G Zantwep L Disse antwerp Z  $\cdot$ er wart] wart er I \textbf{2} was] wart I  $\cdot$ an] \textit{om.} L \textbf{3} wol] auch wol I \textbf{7} swâ] Wa L  $\cdot$ man sîn] si I \textbf{8} ein wîse] wiser L wis ein Z \textbf{9} âderstôz] vnderstoz I vndestoz L \textbf{10} lîp] wip Z  $\cdot$ missewende] an missewende I \textbf{13} daz si] daze G daz ez I Daz L \textbf{14} in] an Z \textbf{15} er] \textit{om.} L \textbf{16} ich sol] Sol ich L Z  $\cdot$ hêrre] herren I \textbf{23} niht] mich I \textbf{24} warnet] warn I  $\cdot$ er] ich I L \textbf{26} liebe] lieben L  $\cdot$ wenken] wachen L wenket Z \textbf{28} nimmer] nimmer nih I \textbf{29} swaz] Waz L  $\cdot$ leisten] lazen I \textbf{30} sinne] sinnen I (L) Z \newline
\end{minipage}
\hspace{0.5cm}
\begin{minipage}[t]{0.5\linewidth}
\small
\begin{center}*T
\end{center}
\begin{tabular}{rl}
 & \begin{large}Z\end{large}uo \textbf{âventiure} \textbf{er wart} gezogen.\\ 
 & si was an im vil unbetrogen.\\ 
 & er kunde wol gebâren.\\ 
 & man \textbf{muoz} in vür den clâren\\ 
5 & und vür den manlîchen\\ 
 & haben in \textbf{allen} rîchen,\\ 
 & wâ man sîn künde ie gewan.\\ 
 & hövesch, mit zühten, \textbf{ein wîse} man,\\ 
 & \textbf{getriuwe}, milte, \textbf{an} \textbf{lîbe grôz}\\ 
10 & was \textbf{er} \textbf{und} missewende blôz.\\ 
 & des landes vrouwe in \textbf{wol} enpfienc.\\ 
 & nû hœret, wie sîn rede ergienc,\\ 
 & \textbf{daz ez} rîche und arme hôrten\\ 
 & \textbf{alsus} \textbf{mit} \textbf{schœnen worten}.\\ 
15 & \textbf{dô} sprach \textbf{er}: "vrouwe herzogîn,\\ 
 & \textbf{sol ich} \textbf{diz} landes hêrre sîn,\\ 
 & dâr umb lâz ich als vil.\\ 
 & nû \textbf{hœret}, wes ich \textbf{bieten} wil:\\ 
 & gevrâget \textit{ni}mer, wer ich sî,\\ 
20 & sô mac ich iu blîben bî.\\ 
 & bin ich \textbf{zuo} iuwer \textbf{vrâgen} \textbf{erkorn},\\ 
 & sô hât ir minne an mir verlorn.\\ 
 & \textbf{sît} ir niht \textbf{vor} gewarnet des,\\ 
 & sô \textbf{wert} mich got, \textbf{ich} weiz wol wes."\\ 
25 & si sazte wîbes sicherheit,\\ 
 & diu sît durch liebe wenken leit,\\ 
 & si wolte zuo sîme gebote stên\\ 
 & und \textbf{daz} niemer übergên,\\ 
 & waz er si leisten hieze,\\ 
30 & ob si got bî sinnen lieze.\\ 
\end{tabular}
\scriptsize
\line(1,0){75} \newline
U Q R \newline
\line(1,0){75} \newline
\textbf{1} \textit{Initiale} U  \newline
\line(1,0){75} \newline
\textbf{1} \textit{Die Verse 821.21-826.30 fehlen} Q   $\cdot$ âventiure] anttwerb R  $\cdot$ gezogen] vsz geczogen R \textbf{4} muoz] muͦst R \textbf{6} in allen] alzemal vnd R \textbf{8} Hofflich mit zuchtte wise ein man R \textbf{9} an lîbe grôz] on ader stos R \textbf{10} er und] sin lib R \textbf{12} ergienc] angieng R \textbf{13} ez] sy R \textbf{14} Die da stuͯnden an allen orten R \textbf{16} diz] hie R \textbf{18} bieten] bitten R \textbf{19} nimer] vmer U \textbf{21} vrâgen erkorn] frage erborn R \textbf{23} sît] [Si*]: Sint U \textbf{24} wert] warne R \textbf{25} wîbes] guͯtte R \textbf{26} sît] sy R  $\cdot$ wenken] wenket R \textbf{28} daz] \textit{om.} R \newline
\end{minipage}
\end{table}
\end{document}
