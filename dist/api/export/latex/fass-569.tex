\documentclass[8pt,a4paper,notitlepage]{article}
\usepackage{fullpage}
\usepackage{ulem}
\usepackage{xltxtra}
\usepackage{datetime}
\renewcommand{\dateseparator}{.}
\dmyyyydate
\usepackage{fancyhdr}
\usepackage{ifthen}
\pagestyle{fancy}
\fancyhf{}
\renewcommand{\headrulewidth}{0pt}
\fancyfoot[L]{\ifthenelse{\value{page}=1}{\today, \currenttime{} Uhr}{}}
\begin{document}
\begin{table}[ht]
\begin{minipage}[t]{0.5\linewidth}
\small
\begin{center}*D
\end{center}
\begin{tabular}{rl}
\textbf{569} & \begin{large}D\end{large}ie steine wâren ouch verbolt.\\ 
 & er hete selten ê gedolt\\ 
 & sô swinde würfe ûf in gevlogen.\\ 
 & Nû was zem schuzze ûf gezogen\\ 
5 & vünf hundert armbrust oder mêr.\\ 
 & die heten al gelîchen kêr\\ 
 & reht ûf daz bette, \textbf{al} dâ er lac.\\ 
 & swer \textbf{ie solher nœte} \textbf{pflac},\\ 
 & der \textbf{mag} erkennen pfîle.\\ 
10 & daz werte kurze wîle,\\ 
 & unze daz si wâren versnurret gar.\\ 
 & swer wil gemaches nemen war,\\ 
 & der\textbf{n} kum an solch \textbf{bette} niht:\\ 
 & gemaches im dâ niemen giht.\\ 
15 & es m\textit{ö}hte jugent werden grâ,\\ 
 & des gemaches, \textbf{alsô dâ}\\ 
 & Gawan an dem bette vant.\\ 
 & dannoch sîn herze unt \textbf{ouch} sîn hant\\ 
 & der zagheit \textbf{lâgen} eine.\\ 
20 & die pfîle unt \textbf{ouch} die steine\\ 
 & heten in niht gar vermiten.\\ 
 & zerquaschiert unt \textbf{ouch} versniten\\ 
 & was er durch die ringe.\\ 
 & dô het er gedinge,\\ 
25 & sînes kumbers wære ein ende.\\ 
 & dannoch \textbf{mit} sîner hende\\ 
 & muoser prîs erstrîten.\\ 
 & An de\textit{n} selben zîten\\ 
 & tet sich \textbf{gein im ûf} ein \textbf{tür}.\\ 
30 & ein \textbf{starker} \textbf{gebûwer} gienc \textbf{dar vür};\\ 
\end{tabular}
\scriptsize
\line(1,0){75} \newline
D \newline
\line(1,0){75} \newline
\textbf{1} \textit{Initiale} D  \textbf{4} \textit{Majuskel} D  \textbf{28} \textit{Majuskel} D  \newline
\line(1,0){75} \newline
\textbf{15} möhte] mohte D \textbf{28} den] dem D \newline
\end{minipage}
\hspace{0.5cm}
\begin{minipage}[t]{0.5\linewidth}
\small
\begin{center}*m
\end{center}
\begin{tabular}{rl}
 & die stein wâren ouch verbolt.\\ 
 & er het selten ê gedolt\\ 
 & sô swinde würf ûf in gevlogen.\\ 
 & nû was zem \dag schriff\dag  ûf gezogen\\ 
5 & vünf hundert armbrust oder mêr.\\ 
 & die heten alle glîch\textit{en} \textit{k}êr\\ 
 & reht ûf daz bette, \textbf{al}dâ er lac.\\ 
 & wer \textbf{solher nœte ie} \textbf{gepflac},\\ 
 & der \textbf{mac} erkennen pf\textit{î}le.\\ 
10 & daz werte kurzwîle,\\ 
 & unz daz si wâren versnurret gar.\\ 
 & wer wil gemaches nemen war,\\ 
 & der kom an solich \textbf{gebette} niht:\\ 
 & gemaches im d\textit{â} niemen giht.\\ 
15 & es m\textit{ö}hte jugent werden grâ,\\ 
 & des gemaches, \textbf{aldâ}\\ 
 & Gawan an dem bett\textit{e} \textit{v}ant.\\ 
 & dannoch sîn herz und sîn hant\\ 
 & der zageheit \textbf{lac} \textbf{al}ein.\\ 
20 & die pfîl und \textbf{ouch} die stein\\ 
 & heten in niht gar vermiten.\\ 
 & zerquaschieret und versniten\\ 
 & was er durch die ringe.\\ 
 & dô het er gedinge,\\ 
25 & \dag sîn kumber\dag  wær ein ende.\\ 
 & dannoch \textbf{mit} sîner hende\\ 
 & muost er prîs erstrîten.\\ 
 & an den selben zîten\\ 
 & tet sich \textbf{ûf \textit{gegen} im} ein \textbf{tor}.\\ 
30 & ein \textbf{starker} \textbf{bûr} gienc \textbf{dar vor};\\ 
\end{tabular}
\scriptsize
\line(1,0){75} \newline
m n o \newline
\line(1,0){75} \newline
\newline
\line(1,0){75} \newline
\textbf{3} swinde] suͯnde o \textbf{4} zem schriff] z:m schriss o \textbf{6} glîchen kêr] glich vnd er m \textbf{9} pfîle] [pflie]: pflle m \textbf{13} gebette] bette n o \textbf{14} dâ] do m n o  $\cdot$ niemen] niemans n \textbf{15} möhte] mohtte m \textbf{17} bette vant] bette lag vnd fant m \textbf{29} gegen] \textit{om.} m \newline
\end{minipage}
\end{table}
\newpage
\begin{table}[ht]
\begin{minipage}[t]{0.5\linewidth}
\small
\begin{center}*G
\end{center}
\begin{tabular}{rl}
 & \begin{large}D\end{large}ie steine wâren ouch verbolt.\\ 
 & er hete selten ê gedolt\\ 
 & sô swinde würfe ûf in gevlogen.\\ 
 & nû was zem schuzze ûf gezogen\\ 
5 & vünf hundert armbrust ode mêr.\\ 
 & die heten al gelîchen kêr\\ 
 & rehte ûf daz bette, \textit{\textbf{al}} dâ er lac.\\ 
 & swer \textbf{ie solher nœte} \textbf{gepflac},\\ 
 & der \textbf{m\textit{ö}ht} erkennen pfîle.\\ 
10 & daz werte kurze wîle,\\ 
 & unze daz si wâren versnurret gar.\\ 
 & swer wil gemaches nemen war,\\ 
 & der\textbf{ne} kome an solch \textbf{bette} niht:\\ 
 & gemaches im dâ niemen giht.\\ 
15 & es m\textit{ö}hte jugent werden grâ,\\ 
 & des gemaches, \textbf{alsô dâ}\\ 
 & Gawan an dem bette vant.\\ 
 & dan\textit{no}ch sîn herze unde \textbf{ouch} sîn hant\\ 
 & der zageheit \textbf{lac} \textbf{al} eine.\\ 
20 & die pfîle unde \textbf{ouch} die steine\\ 
 & heten in niht gar vermiten.\\ 
 & zerquatschiuret unde \textbf{ouch} versniten\\ 
 & was er durch die ringe.\\ 
 & dô hete er gedinge,\\ 
25 & sînes kumbers w\textit{æ}r\textit{e} ein ende.\\ 
 & dannoch \textbf{von} sîner hende\\ 
 & muoser prîs erstrîten.\\ 
 & an den selben zîten\\ 
 & tet sich \textbf{gein im ûf} ein \textbf{tür}.\\ 
30 & ein \textbf{grôzer} \textbf{gebûre} gienc \textbf{dar vür};\\ 
\end{tabular}
\scriptsize
\line(1,0){75} \newline
G I L M Z \newline
\line(1,0){75} \newline
\textbf{1} \textit{Initiale} G I L Z  \textbf{15} \textit{Initiale} I  \newline
\line(1,0){75} \newline
\textbf{4} schuzze] schuͯtze L slozze Z \textbf{6} diu heten alliu geliche chêr I  $\cdot$ gelîchen] geliche Z \textbf{7} al dâ] dar da G da L \textbf{8} swer] Wer L M  $\cdot$ gepflac] phlac I \textbf{9} möht] moht G I mach L (M) (Z) \textbf{10} daz werte] Die werten Z \textbf{11} daz] \textit{om.} L  $\cdot$ versnurret] virsnvͦrt G [versnurret]: versnurren Z \textbf{12} swer] Wer L M  $\cdot$ wil] \textit{om.} L \textbf{13} kome] koͤme Z  $\cdot$ solch] solche Z \textbf{15} es möhte] Es mohte G Ez mohte I (L) (M) Z \textbf{18} dannoch] danch G  $\cdot$ ouch] \textit{om.} M \textbf{19} lac] lach er L (M) \textbf{20} ouch] \textit{om.} I  $\cdot$ steine] zeine L \textbf{22} versniten] versinten L \textbf{24} dô] Da M \textbf{25} sînes kumbers] sin chunber I  $\cdot$ wære] [ware]: wart G het I \textbf{26} von] mit L M Z \textbf{27} muoser] muͤst er I \textbf{30} grôzer] starcker L (M) (Z)  $\cdot$ dar] her L Z \newline
\end{minipage}
\hspace{0.5cm}
\begin{minipage}[t]{0.5\linewidth}
\small
\begin{center}*T
\end{center}
\begin{tabular}{rl}
 & \textit{\begin{large}D\end{large}}ie steine wâren ouch verbolt.\\ 
 & er h\textit{ete} selten ê gedolt\\ 
 & sô swinde würfe ûf in gevlogen.\\ 
 & Nû was zem schuzze ûf gezogen\\ 
5 & vünf hundert armbrust oder mêr.\\ 
 & die heten alle glîchen kêr\\ 
 & rehte ûf daz bette, dâ er lac.\\ 
 & swer \textbf{ie sölher nôt} \textbf{gepflac},\\ 
 & der \textbf{mac} erkennen pfîle.\\ 
10 & daz werte kurze wîle,\\ 
 & unze daz si wâren versnurret gar.\\ 
 & swer wil gemaches nemen war,\\ 
 & der kom a\textit{n} solich \textbf{bette} niht:\\ 
 & gemaches im dâ niemen giht.\\ 
15 & es m\textit{ö}hte jugent werden grâ,\\ 
 & des gemaches \textbf{als dâ},\\ 
 & \textbf{als} Gawan an dem bette vant.\\ 
 & dannoch - sîn herze unde sîn hant -\\ 
 & der zageheite \textbf{lac} \textbf{er} \textbf{al}eine.\\ 
20 & Die pfîle unde die steine\\ 
 & heten in niht gar vermiten.\\ 
 & zerquaschieret unde versniten\\ 
 & was er durch die ringe.\\ 
 & dô het er gedinge,\\ 
25 & sînes kumbers wære ein ende.\\ 
 & dannoch \textbf{mit} sîner hende\\ 
 & muoser \textit{prîs} erstrîten.\\ 
 & An den selben zîten\\ 
 & tet sich \textbf{gein im ûf} ein \textbf{tür}.\\ 
30 & ein \textbf{starker} \textbf{gebûr} gie \textbf{her vür};\\ 
\end{tabular}
\scriptsize
\line(1,0){75} \newline
T U V W Q R Fr39 \newline
\line(1,0){75} \newline
\textbf{1} \textit{Initiale} T   $\cdot$ \textit{Capitulumzeichen} R  \textbf{4} \textit{Majuskel} T  \textbf{17} \textit{Initiale} W  \textbf{20} \textit{Majuskel} T  \textbf{28} \textit{Majuskel} T  \newline
\line(1,0){75} \newline
\textbf{1} \textit{Die Verse 553.1-599.30 fehlen} U   $\cdot$ Die] +ie T \textbf{2} er] Der Q  $\cdot$ hete] h::: T  $\cdot$ selten] salten Q \textbf{3} sô] Wo Q  $\cdot$ gevlogen] pflegen R \textbf{4} zem] zu Q  $\cdot$ schuzze ûf] [sch*f]: schv́ze vf V \textbf{5} armbrust] armbroff W armbrot R \textbf{6} glîchen] gleiche Q (R) \textbf{7} rehte] \textit{om.} V  $\cdot$ dâ er] do er V W do er do Q \textbf{8} [*]: Swer ie solher not gepflag V  $\cdot$ swer] Wer W Q R  $\cdot$ gepflac] pflac Q (R) \textbf{9} mac] [*]: mag V \textbf{10} daz werte] [*]: Daz werte V  $\cdot$ kurze wîle] kútzeweile W \textbf{11} unze] Vnd R \textbf{12} Wer wil gelukes wil nemen gwar R  $\cdot$ swer] Wer W Q \textbf{13} kom] enkome V (Q) (Fr39)  $\cdot$ an solich bette] alsolich bette T ab solche botte Q \textbf{14} dâ] do V W Q Fr39 \textbf{15} \textit{Versfolge 596.16-15} V   $\cdot$ möhte] mohte T (Q) (R) Fr39 \textbf{16} des] Solchs W  $\cdot$ als dâ] alda W also do Q alles da R \textbf{17} als] \textit{om.} Q R Fr39  $\cdot$ Gawan] Gawin R  $\cdot$ an] in W  $\cdot$ vant] vank R \textbf{20} unde] vnde oͮch V (W) (Q) (R) vnd o::: Fr39 \textbf{21} niht gar] ouch gar nicht R \textbf{22} zerquaschieret] [*vatschieret]: Geqvatschieret V  $\cdot$ unde] vnde oͮch V (W) (Q)  $\cdot$ versniten] zu sniten Q \textbf{25} Sein kumber hett ein ende W \textbf{27} muoser] mveser T [M*er]: Mvͤster V  $\cdot$ prîs] \textit{om.} T \textbf{29} ein] die W \textbf{30} gebûr] bauwer W (R)  $\cdot$ her] dar V W (Fr39) \newline
\end{minipage}
\end{table}
\end{document}
