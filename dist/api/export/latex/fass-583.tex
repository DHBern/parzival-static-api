\documentclass[8pt,a4paper,notitlepage]{article}
\usepackage{fullpage}
\usepackage{ulem}
\usepackage{xltxtra}
\usepackage{datetime}
\renewcommand{\dateseparator}{.}
\dmyyyydate
\usepackage{fancyhdr}
\usepackage{ifthen}
\pagestyle{fancy}
\fancyhf{}
\renewcommand{\headrulewidth}{0pt}
\fancyfoot[L]{\ifthenelse{\value{page}=1}{\today, \currenttime{} Uhr}{}}
\begin{document}
\begin{table}[ht]
\begin{minipage}[t]{0.5\linewidth}
\small
\begin{center}*D
\end{center}
\begin{tabular}{rl}
\textbf{583} & \begin{large}S\end{large}wer im nû ruowe næme,\\ 
 & ob ruowens \textbf{in} gezæme,\\ 
 & ich wæne, \textbf{der} hetes sünde.\\ 
 & nâch der âventiure \textbf{urkünde}\\ 
5 & het er sich \textbf{gearbeitet},\\ 
 & gehœhet unt gebreitet\\ 
 & sînen prîs mit grôzer nôt.\\ 
 & swaz der werde Lanzilot\\ 
 & ûf der swertbrücke erleit\\ 
10 & \textbf{unt} sît mit \textbf{Meljacanze} streit,\\ 
 & daz was gein dirre nôt ein niht,\\ 
 & unt des man Garele giht,\\ 
 & dem \textbf{stolzem} künege rîche,\\ 
 & der alsô rîterlîche\\ 
15 & den lewen von dem palas\\ 
 & warf, der \textbf{dâ} ze Nantes was.\\ 
 & Garel \textbf{ouch}z mezzer holte,\\ 
 & dâ \textbf{von} er kumber dolte\\ 
 & in der \textbf{marmelînen sûl}.\\ 
20 & trüege dise pfîle ein mûl,\\ 
 & \textbf{er} wære \textbf{ze vil geladen} \textbf{dâr} mite,\\ 
 & die Gawan \textbf{durch} \textbf{ellens} site\\ 
 & gein sîme verhe snurren liez,\\ 
 & als in sîn manlîch \textbf{herze} hiez.\\ 
25 & \textbf{Ligweiz} Prelljus, der vurt,\\ 
 & und Erek, \textbf{der} Schoydelakurt\\ 
 & erstreit ab Mabonagrin,\\ 
 & der \textbf{newederz} gap sô hôhen pîn,\\ 
 & noch \textbf{dô} der stolze Iwan\\ 
30 & sînen \textbf{guz} niht wolde lân\\ 
\end{tabular}
\scriptsize
\line(1,0){75} \newline
D Z Fr7 \newline
\line(1,0){75} \newline
\textbf{1} \textit{Großinitiale} D   $\cdot$ \textit{Initiale} Z Fr7  \textbf{25} \textit{Majuskel} D  \newline
\line(1,0){75} \newline
\textbf{1} nû] \textit{om.} Fr7  $\cdot$ næme] mere Fr7 \textbf{3} ich] jc Fr7 \textbf{4} der] \textit{om.} Fr7  $\cdot$ urkünde] vrkúnde Hiete Fr7 \textbf{10} Meljacanze] Meliacanze D meliahkantze Z \textbf{13} stolzem] stoltzen Z \textbf{25} Prelljus] prellivs D prellius Z  $\cdot$ vurt] furte Z \textbf{26} vnd Ereck der scoy delakvrt D  $\cdot$ Vnd Erech de tschoi delakurte Z \textbf{27} Mabonagrin] Mabonagrîn D \textbf{28} newederz] entwederz Z \textbf{29} dô] da Z  $\cdot$ Iwan] Jwan D \newline
\end{minipage}
\hspace{0.5cm}
\begin{minipage}[t]{0.5\linewidth}
\small
\begin{center}*m
\end{center}
\begin{tabular}{rl}
 & \begin{large}W\end{large}er im nû ruowe næme,\\ 
 & ob ruowens \textbf{im} gezæme,\\ 
 & ich wæne, \textbf{der} hete es sünde.\\ 
 & nâch der âventiur \textbf{urkünde}\\ 
5 & het er sich \textbf{ê} \textbf{erarbeitet},\\ 
 & gehœhet und gebreitet\\ 
 & sînen prîs mit grôzer nôt.\\ 
 & waz der werde Lanzelot\\ 
 & ûf der swertbrücke erleit\\ 
10 & \textbf{und} sît mit \textbf{Melianz} streit,\\ 
 & daz was gegen diser nôt ein niht,\\ 
 & und des man Gar\textit{e}lle giht,\\ 
 & dem \textbf{stolzen} künic rîch,\\ 
 & der alsô ritterlîch\\ 
15 & den lewen von dem palas\\ 
 & warf, der zuo Nantes was.\\ 
 & Garel \textbf{ouch} daz mezzer h\textit{o}lt,\\ 
 & dâ \textbf{mit} er kumber dolt\\ 
 & in der \textbf{marmelînen sûl}.\\ 
20 & trüege dise pfîle ein mûl,\\ 
 & \textbf{er} wær \textbf{geladen wol} \textbf{dâ} mit\textit{e},\\ 
 & die Gawan \textbf{mit} \textbf{ellens} sit\textit{e}\\ 
 & gegen sînem verh\textit{e} snurren liez,\\ 
 & als in sîn manlîch \textbf{herz} hiez.\\ 
25 & \textbf{Ligweis} Prellius, de\textit{r} \textit{v}urt,\\ 
 & und Erec, \textbf{der} Sch\textit{o}idel\textit{ac}urt\\ 
 & erstreit \textit{a}b Mobonagrin,\\ 
 & der \textbf{ietweder} gap sô hôhe pîn,\\ 
 & noch \textbf{dô} der stolz Iwan\\ 
30 & sînen \textbf{guz} niht wolte \textit{l}ân\\ 
\end{tabular}
\scriptsize
\line(1,0){75} \newline
m n o \newline
\line(1,0){75} \newline
\textbf{1} \textit{Initiale} m n  \newline
\line(1,0){75} \newline
\textbf{4} âventiur] offentúren n \textbf{5} erarbeitet] gearbeitet n o \textbf{8} Lanzelot] lancze lot m lantzelot n lanczelot o \textbf{10} Melianz] meliancz m o meliantz n \textbf{12} Garelle] gar alle m gar elle n o \textbf{16} zuo] do zuͯ n (o) \textbf{17} holt] halt m \textbf{19} marmelînen] marmerin n morinerin o \textbf{21} dâ mite] do mitten m \textbf{22} mit] durch n (o)  $\cdot$ site] sitten m \textbf{23} sînem] sinen o  $\cdot$ verhe] verhen m \textbf{24} \textit{nach 583.24:} Als eyn sin manlich hies o  \textbf{25} der vurt] der sneit vnd furt m \textbf{26} Erec] ereg m n o  $\cdot$ Schoidelacurt] scheidel fuͯrt m scheidelin kúrt n scheidekurt o \textbf{27} ab] ob m  $\cdot$ Mobonagrin] mobogranẏn o \textbf{28} hôhe] hohen o \textbf{29} stolz] stoltze n  $\cdot$ Iwan] jwan m ywan o \textbf{30} sînen] Einen n  $\cdot$ lân] han m \newline
\end{minipage}
\end{table}
\newpage
\begin{table}[ht]
\begin{minipage}[t]{0.5\linewidth}
\small
\begin{center}*G
\end{center}
\begin{tabular}{rl}
 & \begin{large}S\end{large}wer im nû ruowe næme,\\ 
 & ob ruowens \textbf{in} gezæme,\\ 
 & ich wæne, \textbf{er} hetes sünde.\\ 
 & nâch der âventiure \textbf{urkünde}\\ 
5 & het er sich \textbf{gearbeitet},\\ 
 & gehœhet unde ge\textit{b}reitet\\ 
 & sînen brîs mit grôzer nôt.\\ 
 & swaz der werde Lanzelot\\ 
 & ûf der swertbrücke erleit\\ 
10 & \textbf{unde} sît mit \textbf{Melianzen} streit,\\ 
 & daz was gein dirre nôt ein niht,\\ 
 & unde des man Garel giht,\\ 
 & dem \textbf{werden} künege rîche,\\ 
 & der alsô rîterlîche\\ 
15 & de\textit{n} lewen von dem pa\textit{la}s\\ 
 & \textit{warf}, der \textbf{dâ} ze Nantis was.\\ 
 & Garel daz mezzer holte,\\ 
 & dâ \textbf{von} er kumber dolte\\ 
 & in der \textbf{marmelsûl}.\\ 
20 & trüege dise pfîle ein mûl,\\ 
 & \textbf{der} wære \textbf{ze vil geladen} \textbf{dâr} mite,\\ 
 & die Gawan \textbf{durch} \textbf{ellens} site\\ 
 & gein sînem verhe snurren liez,\\ 
 & als in sîn manlîch \textbf{ellen} hiez.\\ 
25 & \textbf{Lishoys} Prillius, der vurt,\\ 
 & unde Erek \textbf{de} Tschoidelakurt\\ 
 & erstreit abe Mobonagrin,\\ 
 & der \textbf{dewederz} gap sô hôhen pîn,\\ 
 & noch der stolze Ywan\\ 
30 & sînen \textbf{gruoz} niht wolde lân\\ 
\end{tabular}
\scriptsize
\line(1,0){75} \newline
G I L M Fr19 \newline
\line(1,0){75} \newline
\textbf{1} \textit{Initiale} G I L M Fr19  \textbf{25} \textit{Initale} I  \newline
\line(1,0){75} \newline
\textbf{1} Swer] Wer L \textbf{2} ob in nu ruwens zeme I  $\cdot$ ob] Ob er L  $\cdot$ in] yme M \textbf{3} hetes] het sin I \textbf{5} het] Hatte M \textbf{6} gebreitet] gereitet G \textbf{8} swaz] Waz L M  $\cdot$ Lanzelot] Lanzilot L Fr19 lanzcolot M \textbf{9} swertbrücke] brucken were M brvke swære Fr19 \textbf{10} unde] Oder L (M) (Fr19)  $\cdot$ Melianzen] miliahkanze G valerine L Melianze M Fr19  $\cdot$ streit] gestreit L M Fr19 \textbf{12} des man] man I ouch des man L  $\cdot$ Garel] Charel G karelle I Karl L karle M karln Fr19  $\cdot$ giht] spricht M \textbf{13} rîche] richen I \textbf{14} der] \textit{om.} M \textbf{15} den] dem G  $\cdot$ palas] pas G \textbf{16} warf] \textit{om.} G  $\cdot$ dâ] \textit{om.} I  $\cdot$ ze Nantis] zenantes G cenantis Fr19 \textbf{17} Garel] Karel G I Karle L Karl M Fr19  $\cdot$ daz] des M \textbf{18} er] ir M \textbf{19} marmelsûl] marmelinen svl L (Fr19) Marmeln sul M \textbf{21} dâr] \textit{om.} L M \textbf{23} snurren] surren M \textbf{24} in] \textit{om.} M \textbf{25} Lishoys] Ligis G Ligus I Lýgoýs L Ligoÿs M Lygois Fr19  $\cdot$ der vurt] [de*]: de fuͯrt L defurt M (Fr19) \textbf{26} Vnde erech dedschoydelachuvrt G  $\cdot$ vnd ereche deshoidelachurt I  $\cdot$ Vnd Erech de schoy delakuͯrt L  $\cdot$ Vnde Erek desoy delacurt M  $\cdot$ Vnd erec de shoẏ delakvrt Fr19 \textbf{27} Mobonagrin] mohonagrin G Mobonagrim I Mabonagrin L Nubonagrin M Mvbonagrin Fr19 \textbf{28} pîn] pris M \textbf{29} der] da der M do der Fr19  $\cdot$ Ywan] ẏwan Fr19 \textbf{30} gruoz] gvz L (M) Fr19 \newline
\end{minipage}
\hspace{0.5cm}
\begin{minipage}[t]{0.5\linewidth}
\small
\begin{center}*T
\end{center}
\begin{tabular}{rl}
 & Wer im nû ruo næme,\\ 
 & ob \textit{ruo}wens \textbf{in} gezæme,\\ 
 & ich wæn, \textbf{er} het es sünde.\\ 
 & nâch der âventiur \textbf{künde}\\ 
5 & het er sich \textbf{gearbeitet},\\ 
 & gehœhet und gebreitet\\ 
 & sînen prîs mit grôzer nôt.\\ 
 & waz der werde Lanzelot\\ 
 & ûf der swertbrücke erleit\\ 
10 & \textbf{oder} sît mit \textbf{Melyanz} streit,\\ 
 & daz was gên diser nôt \textit{ein} niht,\\ 
 & und des man Garel giht,\\ 
 & dem \textbf{werden} künige rîche,\\ 
 & der alsô ritterlîche\\ 
15 & den lewen von dem palas\\ 
 & warf, der \textbf{d\textit{â}} ze Nantis was.\\ 
 & Garel daz mezzer holte,\\ 
 & dâ \textbf{von} er kumber dolte\\ 
 & in der \textbf{marmelînen sûl}.\\ 
20 & trüege dise pfîle ein mûl,\\ 
 & \textbf{der} wære \textbf{zuo vil geladen} mite,\\ 
 & die Gawan \textbf{durch} \textbf{êren} site\\ 
 & gên sînem verhe snurren liez,\\ 
 & als in sîn menlîch \textbf{êre} hiez.\\ 
25 & \textbf{Lygweiz} Prillus, der vurt,\\ 
 & und Erec, \textbf{der} Schoydelakurt\\ 
 & erstreit ab Mobonagrin,\\ 
 & der \textbf{dewederz} gap sô hôhen pîn,\\ 
 & noch der stolze Ywan\\ 
30 & sînen \textbf{gruoz} niht wolte lân\\ 
\end{tabular}
\scriptsize
\line(1,0){75} \newline
Q R W V U \newline
\line(1,0){75} \newline
\textbf{1} \textit{Überschrift:} Hie hat gawan erliten schusse vnd wurffe vnd den leben vber striten Q   $\cdot$ \textit{Großinitiale} R   $\cdot$ \textit{Initiale} Q W V  \newline
\line(1,0){75} \newline
\textbf{1} \textit{Die Verse 553.1-599.30 fehlen} U   $\cdot$ wer] Swer V  $\cdot$ im] \textit{om.} W \textbf{2} ruowens] er wens Q  $\cdot$ in] im V \textbf{3} Das wen ich erhet es súnden R  $\cdot$ Ich meine er hette es súnde W \textbf{4} künde] vrkunden R vrkúnde W (V) \textbf{6} gebreitet] gebeitet R \textbf{8} waz] swaz V  $\cdot$ Lanzelot] lanzilot Q (R) lantzillot W \textbf{10} mit] \textit{om.} R  $\cdot$ Melyanz] meliahkanzen Q [maleakanze]: maljakanze R melyanze W meliahganze V \textbf{11} ein] \textit{om.} Q \textbf{12} des] das R  $\cdot$ Garel] [karel]: kareln Q karel R W [*arel]: garel V \textbf{16} warf] Auff W  $\cdot$ dâ] do Q W V  $\cdot$ Nantis] natis R nantes V \textbf{17} Garel] Karel Q R W [Ga*el]: Garel V  $\cdot$ holte] [h*lte]: oͮch holte V \textbf{20} trüege] Truͯg R (W) \textbf{21} Der wer geladen zuͦuil do mitte W (V) \textbf{22} die] De R  $\cdot$ Gawan] Gawin R [gawa* *n]: gawan V  $\cdot$ êren site] ellen sitte R ellends sitte W [*]: ellent site V \textbf{23} verhe] werche R [*]: verhe V \textbf{24} êre] ellend R W (V) \textbf{25} Lygweiz Prillus] ligweis prellius Q W Ligweis prillius R Ligeweis prelius V  $\cdot$ der vurt] [*]: der wurt V \textbf{26} Vnd erek der shoydelakurt Q  $\cdot$ Vnd Ereck der schoidelakuͦrt R  $\cdot$ Vnd ereck der schoydelakurt W  $\cdot$ Vnde erek de [s*]: schoẏdelekvrt V \textbf{27} Mobonagrin] Mabonagrin R molionagrin W [*]: mabonagrin V \textbf{28} hôhen] hoche R \textbf{29} der] do der R V dro der W  $\cdot$ Ywan] Jwan R \textbf{30} gruoz] gvz er V  $\cdot$ lân] han R \newline
\end{minipage}
\end{table}
\end{document}
