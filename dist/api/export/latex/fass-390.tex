\documentclass[8pt,a4paper,notitlepage]{article}
\usepackage{fullpage}
\usepackage{ulem}
\usepackage{xltxtra}
\usepackage{datetime}
\renewcommand{\dateseparator}{.}
\dmyyyydate
\usepackage{fancyhdr}
\usepackage{ifthen}
\pagestyle{fancy}
\fancyhf{}
\renewcommand{\headrulewidth}{0pt}
\fancyfoot[L]{\ifthenelse{\value{page}=1}{\today, \currenttime{} Uhr}{}}
\begin{document}
\begin{table}[ht]
\begin{minipage}[t]{0.5\linewidth}
\small
\begin{center}*D
\end{center}
\begin{tabular}{rl}
\textbf{390} & \begin{large}M\end{large}it urloube tet er dankêre.\\ 
 & vünfzehen ors oder mêre\\ 
 & liez er \textbf{in} âne wunden.\\ 
 & die knappen danken kunden.\\ 
5 & \textbf{si} bâten in belîben vil.\\ 
 & vürbaz \textbf{gestôzen was} sîn zil.\\ 
 & Dô kêrte der gehiure,\\ 
 & dâ \textbf{grôz} gemach was tiure.\\ 
 & er\textbf{n} suochte niht wan strîten.\\ 
10 & ich wæne, bî sînen zîten\\ 
 & ie dechein man sô vil gestreit.\\ 
 & daz ûzer her al zogende reit\\ 
 & \textbf{ze} herbergen durch gemach.\\ 
 & dort inne der vürste Lyppaut sprach\\ 
15 & unt vrâgte, wie ez dâ wære komen,\\ 
 & \textbf{wande}r hete vernomen,\\ 
 & Melyanz \textbf{wære} gevangen.\\ 
 & daz was im liebe ergangen.\\ 
 & ez kom im sît ze trôste.\\ 
20 & Gawan den ermel lôste\\ 
 & âne zerren vonme schilte\\ 
 & - sînen prîs er hœher zilte -,\\ 
 & den gab er Clauditten.\\ 
 & an \textbf{dem orte} unt \textbf{ouch dâ mitten}\\ 
25 & was er durchstochen unt durchslagen.\\ 
 & \textbf{er hiez in} Obilote tragen.\\ 
 & dô wart der meide vreude grôz.\\ 
 & ir arm was blanc unt blôz.\\ 
 & dar über he\textit{f}te si in dô sân.\\ 
30 & si sprach: "wer hât mir \textbf{dâ} getân?",\\ 
\end{tabular}
\scriptsize
\line(1,0){75} \newline
D \newline
\line(1,0){75} \newline
\textbf{1} \textit{Initiale} D  \textbf{7} \textit{Majuskel} D  \newline
\line(1,0){75} \newline
\textbf{14} Lyppaut] Lyppaot D \textbf{23} Clauditten] Chlavditten D \textbf{29} hefte] hete D \newline
\end{minipage}
\hspace{0.5cm}
\begin{minipage}[t]{0.5\linewidth}
\small
\begin{center}*m
\end{center}
\begin{tabular}{rl}
 & mit urloube tet er dankêre.\\ 
 & vünfzehen ros oder mêre\\ 
 & liez er \textbf{in} âne wunden.\\ 
 & die knappen danken kunden.\\ 
5 & \textbf{si} bâten in \textit{b}elîben vil.\\ 
 & vürbaz \textbf{was gestôzen} sîn zil.\\ 
 & dô kêrte der gehiure,\\ 
 & d\textit{â} \textbf{grôz} gemach was tiure.\\ 
 & er \textbf{en}suohete n\textit{i}wan strîten.\\ 
10 & ich w\textit{æ}ne, bî sînen zîten\\ 
 & ie kein man sô vil gestreit.\\ 
 & daz ûzer her alzogende reit.\\ 
 & \multicolumn{1}{l}{ - - - }\\ 
 & dort inne der vürste Lippo\textit{u}t sprach\\ 
15 & und vrâgete, wie ez d\textit{â} wære komen,\\ 
 & \textbf{wand} er hate vernomen,\\ 
 & Mel\textit{i}anz \textbf{wære} gevangen.\\ 
 & daz was ime liebe ergangen.\\ 
 & ez kam ime \textbf{ouch} sît ze trôste.\\ 
20 & Gawan den ermel lôste\\ 
 & âne zerren von dem schilte\\ 
 & - sînen prîs er hœher zilte -,\\ 
 & den gap er Clauditt\textit{e}n.\\ 
 & an \textbf{dem orte} und \textbf{enmitten}\\ 
25 & was er durchstochen und durchslagen.\\ 
 & \textbf{er hiez in} Obil\textit{o}t\textit{e} tragen.\\ 
 & dô wart der megde vröude grôz.\\ 
 & ir arm was blanc und blôz.\\ 
 & dar über hefte si in dô sân.\\ 
30 & si sprach: "wer hât mir getân?",\\ 
\end{tabular}
\scriptsize
\line(1,0){75} \newline
m n o \newline
\line(1,0){75} \newline
\newline
\line(1,0){75} \newline
\textbf{3} liez] Liesse n \textbf{5} bâten] boten n  $\cdot$ belîben] meliben m \textbf{8} dâ] Do m n \textbf{9} ensuohete] suͯchte n o  $\cdot$ niwan] nwen m nit wenne n \textbf{10} wæne] wane m  $\cdot$ sînen] sÿne o \textbf{13} \textit{Vers 390.13 fehlt} m o   $\cdot$ \textit{Versfolge 390.14-13} n   $\cdot$ Dort do er sú an sach n \textbf{14} Lippout] lippoat m lippaot n lipaot o \textbf{15} dâ] do m n o \textbf{16} hate] hette n o \textbf{17} Melianz] Meleancz m Meliantz n Meliancz o  $\cdot$ wære] herre o \textbf{18} liebe] lieber o \textbf{21} zerren] zerre m \textbf{23} Clauditten] claudittan m clauditen n o \textbf{24} enmitten] aldo mitten n (o) \textbf{26} Obilote] [bo]: obilito m oblito n abiloten o \textbf{29} in] \textit{om.} n o \textbf{30} getân] do getan n o \newline
\end{minipage}
\end{table}
\newpage
\begin{table}[ht]
\begin{minipage}[t]{0.5\linewidth}
\small
\begin{center}*G
\end{center}
\begin{tabular}{rl}
 & mit urloube tet er dankêre.\\ 
 & vünfzehen ors oder mêre\\ 
 & liez er \textbf{in} âne wunden.\\ 
 & die knappen danken kunden\\ 
5 & \textbf{\textit{\begin{large}U\end{large}}\textit{nde}} bâten in belîben vil.\\ 
 & vürbaz \textbf{gestôzen was} sîn zil.\\ 
 & dô kêrte der gehiure,\\ 
 & dâ \textbf{guot} gemach was tiure.\\ 
 & er suohte niht wan strîten.\\ 
10 & ich wæne, bî sînen zîten\\ 
 & ie dehein man sô vil gestreit.\\ 
 & daz ûzer her al zogende reit\\ 
 & \textbf{gein} herb\textit{erg}en durch gemach.\\ 
 & dort inne \textit{der vürste} Libau\textit{t} \textit{s}prach\\ 
15 & unde vrâgte, wiez dâ wære komen.\\ 
 & \textbf{ich wæne}r hete vernomen,\\ 
 & Melianz \textbf{wære} gevangen.\\ 
 & daz was im liebe ergangen.\\ 
 & ez kom im sît ze trôste.\\ 
20 & Gawan den ermel lôste\\ 
 & âne zerren vome schilte\\ 
 & - sînen brîs er hœher zilte -,\\ 
 & den gap er Clauditen.\\ 
 & an \textbf{dem orte} unde \textbf{an dem mitten}\\ 
25 & was er durchstochen unde durchslagen.\\ 
 & \textbf{den bat er} Obilote tragen.\\ 
 & dô wart der magede vröude grôz.\\ 
 & ir arm was blanc unde blôz.\\ 
 & dar über hafte sin dô sân.\\ 
30 & si sprach: "wer hât mir \textbf{dâ} getân?",\\ 
\end{tabular}
\scriptsize
\line(1,0){75} \newline
G I O L M Q R Z Fr28 \newline
\line(1,0){75} \newline
\textbf{1} \textit{Überschrift:} Hie hat der strit ein ende den gawan gevohten hat vnd der rot ritter vnd scheident sich die her vnd reit ieslicher siner auentevre nach Z   $\cdot$ \textit{Initiale} O L Z  \textbf{5} \textit{Initiale} G  \textbf{7} \textit{Initiale} I   $\cdot$ \textit{Capitulumzeichen} R  \textbf{27} \textit{Initiale} I  \newline
\line(1,0){75} \newline
\textbf{1} \textit{Die Verse 370.13-412.12 fehlen} Q   $\cdot$ mit] ÷it O  $\cdot$ tet] tuͤt I \textbf{2} vünfzehen] niunzehen I \textbf{3} wunden] wuͯden M \textbf{4} danken] yme danken M \textbf{5} Vnd bliben batten sy in vil R  $\cdot$ Unde] Si G \textbf{7} dô] Da O M Z  $\cdot$ kêrte] chert I (O) \textbf{9} er] ern I (M) (Z) Fr28  $\cdot$ suohte] svͦht O (R)  $\cdot$ niht] nist L \textbf{10} zîten] gezcyten M \textbf{11} \textit{nach 390.11:} zuht vnd maht im wonte mit I   $\cdot$ ie] Daz Z  $\cdot$ sô] sol L \textbf{12} \textit{nach 390.12:} so daz ez striten gar vermeit I   $\cdot$ al zogende] alzoges I al zogen R \textbf{13} herbergen] herben G herberg R  $\cdot$ durch] guͦt Fr28 \textbf{14} der vürste] \textit{om.} G der herczog R  $\cdot$ Libaut sprach] libaut do sprach G Lybavt sprach O Z libavt sprach L Lybant sprach R lẏbavt sprach Fr28 \textbf{15} vrâgte] fragt O  $\cdot$ wiez dâ] wie daz L (M) (Fr28) \textbf{16} hete] heten O \textbf{17} Melianz] Melyanz O \textbf{19} ez] wande ez I (L) (M) (R) (Fr28) Wan er O \textbf{20} ermel] ermeln Z \textbf{21} zerren] zerten Z \textbf{23} Clauditen] clauditten I (L) M (Z) Clavditten O chlauditten Fr28 \textbf{24} an dem] Jndem O  $\cdot$ unde] im I  $\cdot$ an dem mitten] da en mitten I den mittem O enmitten L R ander mitten M ouch da mitten Z dar mitten Fr28 \textbf{26} Obilote] obiloten I (L) (M) Obylot O Obliten R Obylot Z obẏloten Fr28 \textbf{27} dô] Da O M \textbf{29} über] \textit{om.} I  $\cdot$ sin] sy den R  $\cdot$ dô] da O M Z \textit{om.} R  $\cdot$ sân] an L \textbf{30} dâ] diz I daz Fr28 \newline
\end{minipage}
\hspace{0.5cm}
\begin{minipage}[t]{0.5\linewidth}
\small
\begin{center}*T
\end{center}
\begin{tabular}{rl}
 & \begin{large}M\end{large}it urloube tet er dankêre.\\ 
 & vünfzehen ors oder mê\textit{r}e\\ 
 & liez er \textbf{dâ} âne wunden.\\ 
 & die knappen danken kunden\\ 
5 & \textbf{unde} bâten in blîben vil.\\ 
 & vürbaz \textbf{gestôzen was} sîn zil.\\ 
 & Dô kêrte der gehiure,\\ 
 & dâ \textbf{guot} gemach was tiure.\\ 
 & er\textbf{n} \textit{s}uochte niht wan strîten.\\ 
10 & ich wæne, bî sînen zîten\\ 
 & ie dekein man sô vil gestreit.\\ 
 & Daz ûzer her alzogende reit\\ 
 & \textbf{gegen} herbergen durch gemach.\\ 
 & Dort inne der vürste Lybaut sprach\\ 
15 & unde vrâgete, wiez dâ wære komen.\\ 
 & \textbf{ich wæne}, er hete vernomen,\\ 
 & Melyanz \textbf{wart} gevangen.\\ 
 & daz was im liebe ergangen,\\ 
 & \textbf{wand} ez kom im sît ze trôste.\\ 
20 & Gawan den ermel lôste\\ 
 & âne zerren vonme schilte\\ 
 & - sînen prîs er hœher zilte -,\\ 
 & den gab er Clauditen.\\ 
 & an \textbf{den orten} unde \textbf{enmitten}\\ 
25 & was er durchstochen unde durchslagen.\\ 
 & \textbf{den bat er} Obylote tragen.\\ 
 & \begin{large}D\end{large}ô wart der megde \textit{vröude} grôz.\\ 
 & ir arm was blanc unde blôz.\\ 
 & dar über hafte sin dô sân.\\ 
30 & si sprach: "wer hât mir \textbf{hie} getân?",\\ 
\end{tabular}
\scriptsize
\line(1,0){75} \newline
T V W \newline
\line(1,0){75} \newline
\textbf{1} \textit{Initiale} T W  \textbf{7} \textit{Majuskel} T  \textbf{12} \textit{Majuskel} T  \textbf{14} \textit{Majuskel} T  \textbf{27} \textit{Initiale} T  \newline
\line(1,0){75} \newline
\textbf{2} mêre] mêrere T \textbf{3} dâ] in V W \textbf{8} dâ] Do V W \textbf{9} ern] Er V W  $\cdot$ suochte] scvͦchte T \textbf{11} vil] wol W \textbf{13} gegen] [*]: Ze V \textbf{14} Lybaut] lẏbaut V \textbf{15} dâ] do W \textbf{16} ich wæne] [*]: Wande V \textbf{17} Melyanz] Melẏanz V Melianz W  $\cdot$ wart] were V (W) \textbf{18} liebe] leyde W \textbf{19} wand ez] Was er W  $\cdot$ im] \textit{om.} W \textbf{22} hœher zilte] hoch erzilt W \textbf{23} Clauditen] clauditten V klanditten W \textbf{24} den orten] dem orte V (W) \textbf{25} was] Do waz V  $\cdot$ durchslagen] zerschlagen W \textbf{26} den] Der W  $\cdot$ Obylote] obẏloten V abilot W \textbf{27} vröude] \textit{om.} T \textbf{30} hie] da V das W \newline
\end{minipage}
\end{table}
\end{document}
