\documentclass[8pt,a4paper,notitlepage]{article}
\usepackage{fullpage}
\usepackage{ulem}
\usepackage{xltxtra}
\usepackage{datetime}
\renewcommand{\dateseparator}{.}
\dmyyyydate
\usepackage{fancyhdr}
\usepackage{ifthen}
\pagestyle{fancy}
\fancyhf{}
\renewcommand{\headrulewidth}{0pt}
\fancyfoot[L]{\ifthenelse{\value{page}=1}{\today, \currenttime{} Uhr}{}}
\begin{document}
\begin{table}[ht]
\begin{minipage}[t]{0.5\linewidth}
\small
\begin{center}*D
\end{center}
\begin{tabular}{rl}
\textbf{58} & \textbf{was worden} dâ ze Zazamanc.\\ 
 & sîn hant \textbf{die gunst} erranc.\\ 
 & dennoch swebt er ûf dem sê.\\ 
 & die snellen winde \textbf{im tâten} wê.\\ 
5 & \textbf{einen sîdînen segel sach er} rôten.\\ 
 & den \textbf{truoc ein kocke} unt \textbf{ouch} die boten,\\ 
 & die von Schotten Vridebrant\\ 
 & vroun Belakanen hete gesant.\\ 
 & er bat si, daz si ûf in verkür,\\ 
10 & \textbf{swer} den mâg durch si verlür,\\ 
 & daz si von im gesuochet was.\\ 
 & dô vuorten si den adamas\\ 
 & \textbf{unt} ein swert, \textbf{einen} halsperc und zwô hosen.\\ 
 & hie muget ir \textbf{grôz wunder} losen,\\ 
15 & daz im der kocke widervuor,\\ 
 & als \textbf{im} diu âventiure swuor.\\ 
 & si gâbenz im. dô lobt ouch er:\\ 
 & sîn munt der botschefte ein wer\\ 
 & \textbf{würde}, \textbf{swenn}er kœme zir.\\ 
20 & si schieden sich. man sagete mir,\\ 
 & daz mer \textbf{in truoc} in \textbf{eine} habe.\\ 
 & ze Sibilje kêrt er \textbf{drabe}.\\ 
 & mit golde galt der küene man\\ 
 & sînem marnære sân\\ 
25 & harte wol sîn arbeit.\\ 
 & si schieden sich. daz was dem leit.\\ 
 & \textbf{\begin{Large}D\end{Large}â} ze Spane ime lande\\ 
 & \textbf{er den künec} \textbf{erkande}.\\ 
 & daz was sîn neve Kaylet.\\ 
30 & \textbf{nâch} dem kêrt er ze Dolet.\\ 
\end{tabular}
\scriptsize
\line(1,0){75} \newline
D \newline
\line(1,0){75} \newline
\textbf{27} \textit{Großinitiale} D  \newline
\line(1,0){75} \newline
\textbf{1} Zazamanc] Zazamanch D \textbf{7} Schotten] Scotten D \textbf{22} Sibilje] Sibilie D \newline
\end{minipage}
\hspace{0.5cm}
\begin{minipage}[t]{0.5\linewidth}
\small
\begin{center}*m
\end{center}
\begin{tabular}{rl}
 & \textbf{was worden} dâ ze Zazamanc.\\ 
 & sîn hant \textbf{d\textit{â}} \textbf{\textit{den} si\textit{c}} erranc.\\ 
 & dannoch swebete er ûf dem sê.\\ 
 & die snellen winde \textbf{tâten im} wê.\\ 
5 & \textbf{einen sîdînen segel sach er} rôten.\\ 
 & den \textbf{truoc ein kocke} und \textbf{ouch} die boten,\\ 
 & die von Schotten \textbf{lant} Fridebrant\\ 
 & vrowe Belakanen hete gesant.\\ 
 & er bat si, daz si ûf in verkür,\\ 
10 & \textbf{wie er} den mâc durch si verlür,\\ 
 & daz si von ime gesuochet was.\\ 
 & dô vuorten si den adam\textit{a}s,\\ 
 & \textit{e}in swert, \textbf{einen} halsper\textit{c} und zwô hosen.\\ 
 & hie muget ir \textbf{grôz wunder} losen,\\ 
15 & daz ime der kock widervuor,\\ 
 & als \textbf{mir} diu âventiure swuor.\\ 
 & si gâbenz ime. dô lobete ouch er:\\ 
 & sîn munt der botschaft ein wer\\ 
 & \textbf{würde}, \textbf{wenne} er k\textit{œ}me zuo ir.\\ 
20 & si schieden sich. man sagete mir,\\ 
 & daz mer \textbf{ir trüege}, in \textbf{einer} habe\\ 
 & ze Sibilie kêrt er \textbf{drabe}.\\ 
 & mit golde galt der küene man\\ 
 & sînem marnære sân\\ 
25 & harte wol sîn arbeit.\\ 
 & si schi\textit{e}den sich. daz was dem leit.\\ 
 & \textbf{\begin{large}D\end{large}\textit{â}} ze Spane in dem lande\\ 
 & \textbf{er den künic} \textbf{erkande}.\\ 
 & daz was \textit{sîn} neve Kailet.\\ 
30 & \textbf{mit} dem kêrter ze Dolet.\\ 
\end{tabular}
\scriptsize
\line(1,0){75} \newline
m n o \newline
\line(1,0){75} \newline
\textbf{27} \textit{Illustration mit Überschrift:} Von eime turneÿe ze kanvoleis wie da Gahmuret die kunigin herczeloiden erwarb m  Also ein turneẏ beschach zuͦ kanfaleisz (kanfeleis o  ) vnd wie gahmiret (gahmuͯret o  ) die koͯnnigin hertzeloẏden (herczeleiden o  ) erwarp n (o)   $\cdot$ \textit{Initiale} m n o  \newline
\line(1,0){75} \newline
\textbf{1} dâ] do n o  $\cdot$ ze Zazamanc] zezazamang m zuͦ zazamang n zuͦ [zamang]: zazamang o \textbf{2} dâ] do m n o  $\cdot$ den sic] sigehafft m \textbf{3} swebete] swebet o \textbf{4} \textit{Vers 58.4 fehlt} o   $\cdot$ tâten im] jme dotent n \textbf{5} sîdînen] sidin o \textbf{6} kocke] bock n bog o  $\cdot$ ouch] \textit{om.} n o  $\cdot$ boten] [rotten]: botten o \textbf{7} Schotten] schotte o  $\cdot$ Fridebrant] vridebrant m \textbf{8} Belakanen] belakannen m o belekanen n  $\cdot$ hete] hat n \textbf{12} adamas] adams \textit{nachträglich korrigiert zu:} adamas m \textbf{13} ein] Eein m  $\cdot$ halsperc] halspere m \textbf{14} grôz] \textit{om.} n groffe o \textbf{15} kock widervuor] [koch]: kock wider [far]: fuͯr m \textbf{17} lobete] gelobet n \textbf{18} munt] muͦt o \textbf{19} würde wenne] Wenne wurde n  $\cdot$ kœme] kome m \textbf{22} Sibilie] [k*]: sibilie n sẏbilie o  $\cdot$ kêrt] kerte n o \textbf{24} sînem] Sinen n o \textbf{26} schieden] schienden \textit{nachträglich korrigiert zu:} schieden m \textbf{27} Dâ] DO m n o  $\cdot$ Spane] spane \textit{nachträglich korrigiert zu:} spanye m spanie n o \textbf{29} sîn] \textit{om.} m  $\cdot$ Kailet] kaẏlet n kaͯlet o \newline
\end{minipage}
\end{table}
\newpage
\begin{table}[ht]
\begin{minipage}[t]{0.5\linewidth}
\small
\begin{center}*G
\end{center}
\begin{tabular}{rl}
 & \textbf{wart} dâ ze Zazamanc.\\ 
 & sîn hant \textbf{dâ} \textbf{sigenunft} erranc.\\ 
 & dannoch swebter ûf dem sê.\\ 
 & die snellen winde \textbf{im tâten} wê.\\ 
5 & \textbf{dô sach er einen segel} rôten.\\ 
 & den \textbf{truogen kocken} und die boten,\\ 
 & die von Schotten Fridebrant\\ 
 & vroun Belacanen hete gesant.\\ 
 & er bat si, daz si ûf in verkür,\\ 
10 & \textbf{swier} den mâc durch si verlür,\\ 
 & daz si von im gesuochet was.\\ 
 & dô vuorten si den adamas,\\ 
 & ein swert, \textbf{den} halsberc und zwô hosen.\\ 
 & hie muge\textit{t} ir \textbf{grôzes wunders} losen,\\ 
15 & daz im der kocke widervuor,\\ 
 & als \textbf{mir} diu âventiure swuor.\\ 
 & si gâbenz im. dô lobt ouch er:\\ 
 & sîn m\textit{unt} der botschefte eine wer\\ 
 & \textbf{wære}, \textbf{swenn}er k\textit{œ}me zir.\\ 
20 & si schieden sich. man sagte mir,\\ 
 & daz mer \textbf{warf in} in \textbf{eine} habe.\\ 
 & ze Sibilie kêrter \textbf{abe}.\\ 
 & mit golde galt der küene man\\ 
 & sînem marnære sân\\ 
25 & harte wol sîn arbeit.\\ 
 & si schieden sich. daz was dem leit.\\ 
 & ze Spange in dem lande\\ 
 & \textbf{den künic er} \textbf{erkande}.\\ 
 & daz was sîn neve Kailet.\\ 
30 & \textbf{\textit{\begin{large}N\end{large}}âch} dem kêrter ze Dolet.\\ 
\end{tabular}
\scriptsize
\line(1,0){75} \newline
G I O L M Q R Z Fr21 Fr37 Fr44 \newline
\line(1,0){75} \newline
\textbf{1} \textit{Initiale} O  \textbf{27} \textit{Initiale} O L M Q Z Fr21 Fr37 Fr44  \textbf{30} \textit{Initiale} G  \newline
\line(1,0){75} \newline
\textbf{1} wart] ÷as O Waz worden L (Z) Was M (Q) (R) (Fr21) (Fr37) (Fr44)  $\cdot$ dâ] von den O Q (R) (Fr37) (Fr44) von den da M da von den Fr21  $\cdot$ ze] \textit{om.} O uon Fr37 (Fr44)  $\cdot$ Zazamanc] zazamanch G O L sazamanc M zazamant Q Zasamanc R Zachzamanch Fr37 \textbf{2} dâ] \textit{om.} I do Q ze Fr44  $\cdot$ erranc] errant Q ranc Fr44 \textbf{3} swebter] swebt er I O (L) (Q) (Z) (Fr21) Fr37 schwebe er R  $\cdot$ sê] mere R \textbf{4} im tâten] tattent Jm R (Z) \textbf{5} Einen sydinen (siden M [ Q R ] Fr21 [ Fr37 ] Fr44 ) segel sach (sach er L [ M Q R Z Fr21 Fr37 Fr44 ]) roten (raten Q ) O (L) (M) (Q) (R) (Z) (Fr21) (Fr37) (Fr44) \textbf{6} der chok truͤgt die gabe vnd auch die boten I  $\cdot$ den] Das M  $\cdot$ truogen kocken] trvͦg ein choche O (L) (M) (Q) (Z) (Fr21) (Fr37) (Fr44) truͦg ein kocken R  $\cdot$ und] vnde ouch M (Q) (Z) (Fr21) (Fr37)  $\cdot$ die boten] die bot: Fr37 boten Fr44 \textbf{7} Schotten] schoten G shotten I Schottin M sotten Fr21  $\cdot$ Fridebrant] vridbrant I fridebant Q fridebrand R :ridebrant Fr37 \textbf{8} vroun] Vrou L (R) Virn Fr44  $\cdot$ Belacanen] bellicanen I Belecanen L belakanen M (Q) delacanen Z Belachanen Fr44 \textbf{9} \textit{Die Verse 58.9-63.24 fehlen (Blattverlust)} R   $\cdot$ bat si] bat O L Q bats Fr21  $\cdot$ verkür] er kur Q \textbf{10} swier den] Swie O Wie er den L Q Wy her danne M \textbf{12} dô] Da O Z Fr44 du Fr37  $\cdot$ si] \textit{om.} O \textbf{13} ein] \textit{om.} I  $\cdot$ den] \textit{om.} I L einen O (M) (Q) Z (Fr37) (Fr44) an Fr21  $\cdot$ halsberc] halperc I hailspant M halspere Z  $\cdot$ und] \textit{om.} O L M Q Z Fr21 Fr37 Fr44  $\cdot$ zwô hosen] zirn eyszhosen \textit{nachträglich korrigiert zu:} zwen eyszhoszen Q \textbf{14} hie] Jr M  $\cdot$ muget ir] mvge ir G muget Fr21  $\cdot$ grôzes] grozeiv O grosz M Q (Z) (Fr21) (Fr37) (Fr44)  $\cdot$ wunders] [wundes]: wunders G wunder I O (Q) Z (Fr21) (Fr37) (Fr44) wuder M \textbf{15} im der] deme M \textbf{16} mir] im I \textbf{17} gâbenz im] gaben ýmsz L (Q) (Fr21) (Fr37)  $\cdot$ dô] da M Z  $\cdot$ lobt] lop I Q Fr44 lopte L  $\cdot$ ouch] vnd Fr44 \textbf{18} munt] mer G  $\cdot$ der] eyn M  $\cdot$ eine] \textit{om.} Q Fr44 \textbf{19} Abir so er widdir kome zcu ir M  $\cdot$ swenner] so er wider O L (Fr21) sie er wider Q er wider Z so er Fr37 er so wider Fr44  $\cdot$ kœme] chome G I (O) (M) (Q) (Fr21) (Fr37) komen Z (Fr44)  $\cdot$ zir] under zv ir Fr37 \textbf{20} sagte] seit I (O) L Fr21 \textbf{21} mer] \textit{om.} L  $\cdot$ warf in] wurf in I in trvͦge O (L) (M) (Q) (Z) (Fr21) in truͦch Fr37 truge in Fr44  $\cdot$ in eine] eine L \textbf{22} Sibilie] sibiligen I sibille Q sẏbilie Fr21 sybille Fr37 [s]: Sibilien Fr44  $\cdot$ kêrter] chert er O (L) (Fr21) (Fr37) (Fr44) kerrer M do kert er Q da kert er Z  $\cdot$ abe] drabe O M Z Fr21 \textbf{23} galt] \textit{om.} Fr37 \textbf{24} sînem marnære] sinen marnern I (O)  $\cdot$ sân] sam L \textbf{25} harte] Vil harte O L M Q Z (Fr21) (Fr37) Fr44  $\cdot$ wol] uol Fr37  $\cdot$ sîn] ir I \textbf{26} sich] \textit{om.} O  $\cdot$ was] \textit{om.} O Fr21  $\cdot$ dem] in I ým L \textbf{27} ze] ÷e O Jn M  $\cdot$ Spange] spangen I spanie O Fr21 (Fr44) yspanie L Jspanie M \textbf{28} er] er wol I  $\cdot$ erkande] bekande M Q der kande Z \textbf{29} Kailet] Gahilet I kaylet O M Fr21 Fr37 Fr44 kaẏlet L kaylayt Q Gailet Z \textbf{30} Nâch] Unach G  $\cdot$ kêrter] kert er I Q Z (Fr37)  $\cdot$ Dolet] Tolet L doleth Q \newline
\end{minipage}
\hspace{0.5cm}
\begin{minipage}[t]{0.5\linewidth}
\small
\begin{center}*T (U)
\end{center}
\begin{tabular}{rl}
 & \textbf{was worden} dâ zuo Zazamanc.\\ 
 & sîn hant \textbf{dâ} \textbf{sige\textit{nun}ft} erranc.\\ 
 & dannoch \textit{s}w\textit{e}bet er ûf dem sê.\\ 
 & die snellen winde \textbf{im tâten} wê.\\ 
5 & \textbf{einen sîdînen segel sach er} rôten.\\ 
 & den \textbf{truoc ein kocke} und \textbf{ouch} die boten,\\ 
 & die von Schotten Fridebrant\\ 
 & vrouwe Belacane hete gesant.\\ 
 & er bat si, daz si ûf in \textit{v}erkür,\\ 
10 & \textbf{wie er} den mâc durch si verlür,\\ 
 & daz si von im gesuochet was.\\ 
 & dô vuorten si den adamas,\\ 
 & ein swert, \textbf{einen} halsperc und zwô hosen.\\ 
 & hie muget ir \textbf{grôz wunder} l\textit{o}sen,\\ 
15 & daz im der kocke widervuor,\\ 
 & als \textbf{mir} diu âventiure swuor.\\ 
 & si gâbenz im. dô lobete ouch er:\\ 
 & sîn munt der botschefte ein wer\\ 
 & \textbf{wære}, \textbf{sô} er kæme \textbf{wider} zuo ir.\\ 
20 & si schieden sich. man saget mir,\\ 
 & daz mer \textbf{trüege \textit{in}} in \textbf{eine} habe.\\ 
 & zuo Sybilie \textbf{dô} kêrte er \textbf{abe}.\\ 
 & mit golde galt der küene man\\ 
 & sîme marnære sân\\ 
25 & harte wol sîn arbeit.\\ 
 & si schieden sich. daz \textit{w}as dem leit.\\ 
 & \textit{\begin{large}Z\end{large}}e Spanie in dem lande\\ 
 & \textbf{den künec er} \textbf{bekande}.\\ 
 & daz was sîn neve Kaylet.\\ 
30 & \textbf{nâch} dem kêrter zuo Dolet.\\ 
\end{tabular}
\scriptsize
\line(1,0){75} \newline
U V W T \newline
\line(1,0){75} \newline
\textbf{3} \textit{Majuskel} T  \textbf{5} \textit{Majuskel} T  \textbf{12} \textit{Majuskel} T  \textbf{22} \textit{Majuskel} T  \textbf{27} \textit{Großinitiale} U T   $\cdot$ \textit{Initiale} V W  \newline
\line(1,0){75} \newline
\textbf{1} dâ] do V W  $\cdot$ Zazamanc] zazamang V W \textbf{2} dâ] do V W  $\cdot$ sigenunft] [sigeh*]: sigeshaft U \textbf{3} swebet] wibet U  $\cdot$ dem] den W \textbf{6} den truoc] Den truͦgen W die trvͦc T  $\cdot$ ein kocke] koͤche W  $\cdot$ ouch] \textit{om.} T  $\cdot$ die] \textit{om.} W \textbf{7} Schotten] schoten U  $\cdot$ Fridebrant] fridebrand W \textbf{8} vrouwe] vrôun T  $\cdot$ Belacane] Belekanen V pelakanen W Belacanen T  $\cdot$ hete] warn W \textbf{9} bat si] bat V W  $\cdot$ verkür] erkuͦr U verkúß W verkôs T \textbf{10} wie] swie V T  $\cdot$ verlür] verlúß W [vercôs]: verlôs T \textbf{11} si tet als si gebeten was T \textbf{13} ein swert einen halsperc] [E*]: Ein halsperg ein swert V Ein schwert ein halspere W \textbf{14} losen] lesen U \textbf{15} der kocke] die selde W \textbf{16} mir] im W \textbf{17} gâbenz] gâbens T  $\cdot$ lobete] gelobet V lobet T \textbf{18} sîn] Das sein W  $\cdot$ ein] \textit{om.} W T \textbf{19} kæme wider] wider côme T \textbf{20} schieden] sceiden T  $\cdot$ sich] \textit{om.} W  $\cdot$ saget] sagete V T \textbf{21} trüege in] truͦge U truͦg in V \textbf{22} Sybilie] [Sẏbil*]: Sẏbilie V sibillye W  $\cdot$ dô] \textit{om.} W T  $\cdot$ kêrte] kert W  $\cdot$ abe] drabe W \textbf{23} küene] werde T \textbf{24} Seinen lieben marneren sân W  $\cdot$ den marnern vnd sciet von dan T \textbf{25} vnd lonet in ir arbeit T \textbf{26} was] ewas U \textit{om.} W  $\cdot$ dem] [i*]: dem V im T \textbf{27} Ze] DE U  $\cdot$ Spanie] Spangen U (V) hyspania W hyspanie T \textbf{28} bekande] wol kande W \textbf{29} Kaylet] kylet U Kaẏlet V gaylet W \textbf{30} kêrter] kert er V W  $\cdot$ Dolet] Tolet V \newline
\end{minipage}
\end{table}
\end{document}
