\documentclass[8pt,a4paper,notitlepage]{article}
\usepackage{fullpage}
\usepackage{ulem}
\usepackage{xltxtra}
\usepackage{datetime}
\renewcommand{\dateseparator}{.}
\dmyyyydate
\usepackage{fancyhdr}
\usepackage{ifthen}
\pagestyle{fancy}
\fancyhf{}
\renewcommand{\headrulewidth}{0pt}
\fancyfoot[L]{\ifthenelse{\value{page}=1}{\today, \currenttime{} Uhr}{}}
\begin{document}
\begin{table}[ht]
\begin{minipage}[t]{0.5\linewidth}
\small
\begin{center}*D
\end{center}
\begin{tabular}{rl}
\textbf{7} & der stæten hilfe an \textbf{mich} versehen,\\ 
 & denne ich sô gâhes welle jehen.\\ 
 & er sol mîn ingesinde sîn.\\ 
 & \textbf{deiswâr}, ich tuon \textbf{iu} allen schîn,\\ 
5 & daz uns beide ein muoter truoc.\\ 
 & er hât wênic und ich genuoc.\\ 
 & daz sol im teilen sô mîn hant,\\ 
 & \textbf{daz} \textbf{des} mîn sælde \textbf{niht} sî pfant\\ 
 & vor dem, der gibt und nimt.\\ 
10 & ûf reht in \textbf{bêder} \textbf{der} gezimt."\\ 
 & dô die vürsten rîche\\ 
 & vernâmen al gelîche,\\ 
 & daz ir hêrre triwen pflac,\\ 
 & \textbf{diz} was in ein lieber tac.\\ 
15 & \textbf{ieslîcher} im \textbf{sunder} neic.\\ 
 & Gahmuret niht langer sweic\\ 
 & \textbf{der volge}, als im sîn herze jach.\\ 
 & zem künege er guotlîche sprach:\\ 
 & "hêrre unde bruoder mîn,\\ 
20 & wolt ich \textbf{ingesinde} sîn\\ 
 & iwer oder decheines man,\\ 
 & sô het ich \textbf{mîn gemach} getân.\\ 
 & \textbf{nû prüevet dâr nâch} mînen prîs\\ 
 & - ir sît getriwe unde wîs -\\ 
25 & \textbf{und} râtet, als ez geziehe nuo.\\ 
 & dâ grîfet helflîche zuo.\\ 
 & niht wan harnasch ich hân.\\ 
 & het ich dâr inne mêr getân,\\ 
 & daz \textbf{virrec} lop mir bræhte,\\ 
30 & etswâ man mîn gedæhte."\\ 
\end{tabular}
\scriptsize
\line(1,0){75} \newline
D \newline
\line(1,0){75} \newline
\textbf{16} \textit{Versal} D  \newline
\line(1,0){75} \newline
\textbf{16} Gahmuret] Gahmvret D \newline
\end{minipage}
\hspace{0.5cm}
\begin{minipage}[t]{0.5\linewidth}
\small
\begin{center}*m
\end{center}
\begin{tabular}{rl}
 & der stæten helfe an \textbf{mich} versehen,\\ 
 & denn \textit{i}ch sô gâhes welle jehen.\\ 
 & er sol mîn ingesinde sîn.\\ 
 & \textbf{zwâr} ich tuon \textbf{ouch} allen schîn,\\ 
5 & d\textit{az} uns beide ein muoter truoc.\\ 
 & er het wênic und \textit{ich} genuoc.\\ 
 & daz sol im teilen sô mîn hant,\\ 
 & \textbf{des} mîn sælde \textbf{niht} sî pfant\\ 
 & vor dem, der gibt und nimt.\\ 
10 & ûf reht in \textbf{beiden} \textbf{der} gezimt."\\ 
 & \begin{large}D\end{large}ô die vürsten rîche\\ 
 & vern\textit{â}men al gelîche,\\ 
 & daz ir hêrre \textit{t}riuwen pfla\textit{c},\\ 
 & \textbf{daz} was in ein lieber ta\textit{c}.\\ 
15 & \textbf{bîlicher} im \textbf{sunder} neic.\\ 
 & Ga\textit{h}mu\textit{re}t niht langer sweic.\\ 
 & \textbf{er volgete}, als im sîn herze jach.\\ 
 & zuo dem künige er guo\textit{te}lîchen sprach:\\ 
 & "hêrre und bruoder mîn,\\ 
20 & wolt ich \textbf{ein gesinde} sîn\\ 
 & \textit{\begin{large}I\end{large}u}wer oder keines man,\\ 
 & sô het ich \textbf{mir ungemach} getân.\\ 
 & \textbf{nû prüefet dâr nâch} mînen brîs\\ 
 & - ir sît getriuwe und wîs -\\ 
25 & \textbf{und} râtet, als ez geziehe \textit{nuo}.\\ 
 & d\textit{â} grîfet helflîchen zuo.\\ 
 & niht wan harnasch ich hân.\\ 
 & het ich dâr inne mê getân,\\ 
 & daz \textbf{villîht} lop mir bræhte,\\ 
30 & etwâ \textit{man} mîn gedæhte."\\ 
\end{tabular}
\scriptsize
\line(1,0){75} \newline
m n o W \newline
\line(1,0){75} \newline
\textbf{11} \textit{Initiale} m n o W  \textbf{21} \textit{Initiale} m  \newline
\line(1,0){75} \newline
\textbf{1} versehen] fuͯrseren o \textbf{2} Das wil ich im mit warhait iehen W  $\cdot$ denn ich] Dennoch m n o  $\cdot$ gâhes] ga es m o es n \textbf{4} ouch] úch n (o) (W) \textbf{5} daz] Do m n W \textbf{6} het] hette n W  $\cdot$ ich] \textit{om.} m \textbf{8} des] Das o W  $\cdot$ sælde] selde\textit{nachträglich korrigiert zu:} sele m sele W \textbf{9} vor] Do vor n \textbf{10} der] der \textit{nachträglich korrigiert zu:} ders m das n o W \textbf{12} vernâmen] Vernemend \textit{nachträglich korrigiert zu:} Vernomend m  $\cdot$ al gelîche] alle glich n \textbf{13} triuwen] ruwen m ruwe n (o)  $\cdot$ pflac] pflage m \textbf{14} was] >was< m  $\cdot$ in] im W  $\cdot$ tac] tage m \textbf{15} bîlicher] Billicher \textit{nachträglich korrigiert zu:} Yzllicher m Wellicher n (o) (W)  $\cdot$ im] im do W \textbf{16} Gahmuret] Gahamuert m Gamiret n Gamuret o W \textbf{17} volgete] volget n (o) W \textbf{18} guotelîchen] gutleklichen m \textbf{20} gesinde] ingesinde W \textbf{21} Iuwer] ÷Wer m  $\cdot$ keines] do keins n \textbf{22} ich mir ungemach] ich >mir< [*ngemach]: vngemach m min gemach n (o) (W) \textbf{23} nû] So n o W  $\cdot$ dâr nâch] dennoch W \textbf{25} geziehe nuo] geziehe wol \textit{nachträglich korrigiert zu:} geziemhet nuͯ m geziehe wo n o gezyme nuͦ W \textbf{26} dâ] Do m n o So W \textbf{27} ich] icb W \textbf{28} dâr inne mê] me dar inne W \textbf{29} villîht] billich W \textbf{30} etwâ] Etwann W  $\cdot$ man] \textit{om.} m \newline
\end{minipage}
\end{table}
\newpage
\begin{table}[ht]
\begin{minipage}[t]{0.5\linewidth}
\small
\begin{center}*G
\end{center}
\begin{tabular}{rl}
 & der stæten helfe an \textbf{mich} versehen,\\ 
 & dane ich sô gâhes welle jehen.\\ 
 & \textit{e}r sol mîn ingesinde sîn.\\ 
 & \textbf{\textit{de}swâr}, ich tuon \textbf{iu} allen schîn,\\ 
5 & daz uns bêde ein muoter truoc.\\ 
 & er hât wênic und ich genuoc.\\ 
 & daz sol im teilen sô mîn hant,\\ 
 & \textbf{daz} \textbf{es} mîn sælde \textbf{iht} sî pfant\\ 
 & vor dem, der gît und nimet.\\ 
10 & ûf reht in \textbf{beider} gezimet."\\ 
 & dô die vürsten rîche\\ 
 & vernâmen algelîche,\\ 
 & daz ir hêrre triwen pflac,\\ 
 & \textbf{daz} was in ein \textbf{vil} lieber tac.\\ 
15 & \textbf{\begin{large}I\end{large}egelîcher} im \textbf{sunder} neic.\\ 
 & Gahmuret niht langer sweic\\ 
 & \textbf{der volge}, als im sîn herze jac\textit{h}.\\ 
 & zem künige er guotlîchen sprach:\\ 
 & "hêrre und bruoder mîn,\\ 
20 & wolt ich \textbf{ingesinde} sîn\\ 
 & iwer oder deheines man,\\ 
 & sô hete ich \textbf{mîn gemach} getân.\\ 
 & \textbf{dâr nâch prüevet} mînen brîs.\\ 
 & ir sît getriw und wîs.\\ 
25 & \textbf{nû} râtet, als ez geziehe nû.\\ 
 & dâ grîfet helfeclîche zuo.\\ 
 & niht wan harnasch ich hân.\\ 
 & het ich dâr inne mê getân,\\ 
 & daz \textbf{virrec} lop mir br\textit{æ}hte,\\ 
30 & etswâ man mîn ged\textit{æ}hte."\\ 
\end{tabular}
\scriptsize
\line(1,0){75} \newline
G O L M Q Z Fr29 Fr32 \newline
\line(1,0){75} \newline
\textbf{1} \textit{Initiale} O  \textbf{11} \textit{Initiale} Fr32  \textbf{15} \textit{Initiale} G  \textbf{23} \textit{Versal} Fr32  \newline
\line(1,0){75} \newline
\textbf{1} der stæten] Steter Q  $\cdot$ mich] mir L (M) Q Fr32  $\cdot$ versehen] gescheen Q \textbf{2} dane] Er danne M Do Q  $\cdot$ gâhes welle] snel wold M  $\cdot$ jehen] veryehen L (Q) (Z) \textbf{3} er] :r G  $\cdot$ sol] solde M  $\cdot$ ingesinde] gesinden \textit{nachträglich korrigiert zu:} gesinde Q \textbf{4} deswâr] :::s war G Czwar Q (Z)  $\cdot$ tuon iu allen] ev allen tvn Z \textbf{6} und ich] ich hon Q \textbf{7} daz] So Q  $\cdot$ sô] \textit{om.} Q \textbf{8} es] des O L M Q Z Fr29 Fr32  $\cdot$ sælde] sel O (L) (M) (Q) Z  $\cdot$ iht sî] sie eyn M \textbf{9} der] der da Z \textbf{10} in beider] inbeidin M (Fr32)  $\cdot$ gezimet] des gezimt O (Q) (Fr32) der gezimet L M (Z) (Fr29) \textbf{11} dô] Da Z  $\cdot$ rîche] algeliche Fr32 \textbf{12} algelîche] alle geliche O (L) (M) (Q) Z (Fr29) von dem riche Fr32 \textbf{13} hêrre] herren Fr32  $\cdot$ triwen] triwe O \textbf{14} daz] Ez Fr29  $\cdot$ vil] \textit{om.} M Z \textbf{15} Iegelîcher] Jslicher O (Q) (Fr29)  $\cdot$ sunder] besvnder O L (M) (Q) (Z) Fr29  $\cdot$ neic] [nech]: neich G newgt Q \textbf{16} Gahmuret] Gamvret O (Fr32) Gahmuͯret L Gachmuͯret M Gamúert Q Gamuret Z Gahmvͦret Fr29  $\cdot$ langer] lange M \textit{om.} Z \textbf{17} jach] iac: G \textbf{18} er] er do Q \textbf{20} ingesinde] ir gesinde M dein gesinde Q \textbf{21} Wer ich ein in heynusz man Q  $\cdot$ iwer] Vwers L  $\cdot$ deheines] icheines M keines Z \textbf{22} mîn] [mir]: min O \textbf{25} nû] Vnd Q (Fr32)  $\cdot$ ez] ir M  $\cdot$ geziehe] gezeme L gezihen Q gezieht Fr29 \textbf{26} dâ] Do Q \textbf{28} mê] iht me O \textbf{29} virrec] verrer L virde M \textit{om.} Q werdic Z  $\cdot$ mir] man mir Q \textit{om.} Z  $\cdot$ bræhte] brahte G \textbf{30} etswâ] Wa L Ezwan Q  $\cdot$ man mîn] mein man Q man Z  $\cdot$ gedæhte] gedahte G \newline
\end{minipage}
\hspace{0.5cm}
\begin{minipage}[t]{0.5\linewidth}
\small
\begin{center}*T
\end{center}
\begin{tabular}{rl}
 & der stæten helfe an \textbf{mir} versehen,\\ 
 & dann ich sô gâhes welle jehen.\\ 
 & er sol mîn ingesinde sîn.\\ 
 & \textbf{deiswâr}, ich tuon \textbf{iu} allen schîn,\\ 
5 & daz uns beide ein muoter truoc.\\ 
 & er hât wênic und ich genuoc.\\ 
 & daz sol im teilen sô mîn hant,\\ 
 & \textbf{daz} \textbf{des} mîn sælde \textbf{iht} sî pfant\\ 
 & vor dem, der gît und nimt.\\ 
10 & ûf reht in \textbf{beiden} \textbf{daz} gezimt."\\ 
 & Dô die vürsten rîche\\ 
 & vernâmen algelîche,\\ 
 & daz ir hêrre triuwen pflac,\\ 
 & \textbf{daz} was in ein \textbf{vil} lieber tac.\\ 
15 & \textbf{ir} \textbf{ieglîcher} im \textbf{besunder} neic.\\ 
 & Gahmuret niht langer sweic\\ 
 & \textbf{der volge}, als im sîn herze jach.\\ 
 & zem künege er güetlîchen sprach:\\ 
 & "Hêrre und bruoder mîn,\\ 
20 & wolt ich \textbf{ingesinde} sîn\\ 
 & iuwer oder deheines man,\\ 
 & sô het ich \textbf{mîn gemach} getân.\\ 
 & \textbf{dâr nâch prüevet} mînen prîs.\\ 
 & ir sît getriuwe und wîs.\\ 
25 & \textbf{nû} râtet, als ez geziehe nuo.\\ 
 & dâ grîfet helfeclîche zuo.\\ 
 & niht wan harnasch ich hân.\\ 
 & het ich dâr inne mê getân,\\ 
 & daz \textbf{virric} lob mir bræhte,\\ 
30 & etswâ man mîn gedæhte."\\ 
\end{tabular}
\scriptsize
\line(1,0){75} \newline
T U V \newline
\line(1,0){75} \newline
\textbf{11} \textit{Majuskel} T  \textbf{19} \textit{Majuskel} T  \newline
\line(1,0){75} \newline
\textbf{1} stæten] stede U \textbf{2} dann ich] Dannoch U \textbf{4} deiswâr] Des was U Des swar V  $\cdot$ iu] in U \textbf{10} in beiden] [*]: im beider V  $\cdot$ daz] der U \textbf{12} algelîche] alle geliche U \textbf{15} ir] \textit{om.} U V \textbf{16} Gahmuret] Gahmvret T Gahmuͦret U Gamuret V \textbf{18} zem] Zuͦ eim U \textbf{21} iuwer] Wer U  $\cdot$ deheines] dekeines U denkeines V \textbf{27} harnasch] hernach U \textbf{29} daz virric] Dar verit U Das [wirdi*]: wirdig V \newline
\end{minipage}
\end{table}
\end{document}
