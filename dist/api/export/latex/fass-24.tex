\documentclass[8pt,a4paper,notitlepage]{article}
\usepackage{fullpage}
\usepackage{ulem}
\usepackage{xltxtra}
\usepackage{datetime}
\renewcommand{\dateseparator}{.}
\dmyyyydate
\usepackage{fancyhdr}
\usepackage{ifthen}
\pagestyle{fancy}
\fancyhf{}
\renewcommand{\headrulewidth}{0pt}
\fancyfoot[L]{\ifthenelse{\value{page}=1}{\today, \currenttime{} Uhr}{}}
\begin{document}
\begin{table}[ht]
\begin{minipage}[t]{0.5\linewidth}
\small
\begin{center}*D
\end{center}
\begin{tabular}{rl}
\textbf{24} & \textbf{si} \textbf{nam} in \textbf{selbe} \textbf{mit} der hant.\\ 
 & gein den vîenden \textbf{an} die want\\ 
 & sâzen si in diu venster wît\\ 
 & ûf \textbf{einen} kulter, \textbf{gesteppet} samît.\\ 
5 & dâr under ein weichez bette lac.\\ 
 & ist iht \textbf{liehters} denne der tac,\\ 
 & dem gelîchet niht diu künegîn.\\ 
 & si hete wîplîchen \textbf{sin}\\ 
 & unt was \textbf{aber} anders rîterlîch,\\ 
10 & der touwigen rôsen ungelîch.\\ 
 & nâch swarzer varwe was ir schîn,\\ 
 & ir krône ein liehter rubîn.\\ 
 & ir houbet man dar durch sach.\\ 
 & diu \textbf{wirtîn} zir gaste sprach,\\ 
15 & daz ir \textbf{liep wære} sîn komen.\\ 
 & "hêrre, ich hân \textbf{von iu} vernomen\\ 
 & vil \textbf{rîterlîcher} werdecheit.\\ 
 & durch iwer zuht \textbf{lât} \textbf{iu} niht \textbf{wesen} leit,\\ 
 & \textbf{ob} ich \textbf{iu} mînen kumber klage,\\ 
20 & den ich nâhe \textbf{in} mînem herzen trage."\\ 
 & "\begin{large}M\end{large}în helfe i\textit{u}ch, \textbf{vrouwe}, niht irret.\\ 
 & swaz iu war oder wirret,\\ 
 & \textbf{swâ} daz wenden sol mîn hant,\\ 
 & diu sî ze dienste dar \textbf{benant}.\\ 
25 & ich bin niht wan einec man.\\ 
 & swer iu tuot oder hât getân,\\ 
 & dâ \textbf{biut ich gein} mînen schilt.\\ 
 & Die vîende wênec des bevilt."\\ 
 & mit zühten sprach ein vürste sân:\\ 
30 & "heten wir einen houbetman,\\ 
\end{tabular}
\scriptsize
\line(1,0){75} \newline
D Fr9 Fr14 \newline
\line(1,0){75} \newline
\textbf{21} \textit{Initiale} D Fr9 Fr14  \textbf{28} \textit{Majuskel} D  \newline
\line(1,0){75} \newline
\textbf{1} selbe mit] selben bi Fr9 \textbf{3} in] an Fr9 \textbf{5} [weichet]: weichez Fr9 \textbf{6} liehters] lechteres Fr9 \textbf{7} gelîchet] gelichete Fr9 (Fr14) \textbf{9} aber] ouch Fr9 \textbf{12} rubîn] robin Fr9 \textbf{13} sach] wol sach Fr14 \textbf{14} zir] zvͦ dem Fr9 \textbf{15} sîn] ir Fr14 \textbf{18} wesen] \textit{om.} Fr9 >sin< Fr14 \textbf{20} nâhe in] so nahe Fr9 \textbf{21} iuch] iwch D \textbf{23} sol] kan Fr9 \textbf{24} benant] gewant Fr9 \textbf{25} einec] eẏn Fr9 \newline
\end{minipage}
\hspace{0.5cm}
\begin{minipage}[t]{0.5\linewidth}
\small
\begin{center}*m
\end{center}
\begin{tabular}{rl}
 & \textbf{si} \textbf{nam} in \textbf{selbes} \textbf{mit} der hant.\\ 
 & gegen den vîenden \textbf{an} die want\\ 
 & sâzen si in diu venster wît\\ 
 & ûf \textbf{einem} kulter, \textbf{gesteppet} samît.\\ 
5 & dâr under ein weichez bette lac.\\ 
 & ist iht \textbf{li\textit{e}hters} den der tac,\\ 
 & dem glîchete niht diu künigîn.\\ 
 & si het \textbf{doch} wîplîchen \textbf{sin}\\ 
 & und was \textbf{ouch} anders ritterlîch,\\ 
10 & der tou\textit{wigen rôsen} ungelîch.\\ 
 & nâch swarzer varwe w\textit{as} ir schîn,\\ 
 & ir krône ein liehter rubîn.\\ 
 & ir houbet man dar durch \textbf{wol} sach.\\ 
 & diu \textbf{vürst\textit{î}n} zuo i\textit{r} gaste sprach,\\ 
15 & daz ir \textbf{liep wære} \textit{sîn} komen.\\ 
 & "hêrre, ich hân vernomen\\ 
 & vil \textbf{ritterlîche} wirdicheit.\\ 
 & durch iuwer zuht \textbf{lât} \textbf{ir} niht leit.\\ 
 & \textbf{v\textit{o}rabe} ich mînen kumber klage,\\ 
20 & den ich nâhe \textbf{an} mînem herzen trage."\\ 
 & "mîn helfe iuch, \textbf{vrowe}, niht irret.\\ 
 & waz iu war oder \textit{w}irret,\\ 
 & \textbf{zwâr}, daz wenden sol mîn hant,\\ 
 & diu sî zuo dienste dar \textbf{benant}.\\ 
25 & ich bin niht wenne \textbf{ein} einic man.\\ 
 & wer i\textit{u} tuot oder hât getân,\\ 
 & dâ \textbf{bi\textit{u}te ich gegen} mînen schilt.\\ 
 & die vîende wênic des bevilt."\\ 
 & \begin{large}M\end{large}it zühten sprach ein vürste sân:\\ 
30 & "heten wir einen houbetman,\\ 
\end{tabular}
\scriptsize
\line(1,0){75} \newline
m n o \newline
\line(1,0){75} \newline
\textbf{29} \textit{Initiale} m n  \newline
\line(1,0){75} \newline
\textbf{3} in] an n \textbf{4} einem] einen n  $\cdot$ samît] mit samit n o \textbf{6} liehters] lichters m \textbf{7} glîchete] glichet n o \textbf{8} sin] [pin]: schin o \textbf{10} touwigen rôsen] tou rosen wigand m tonwigen rosen o \textbf{11} was ir] wir m \textbf{12} rubîn] ruͦbin n ruͯbin o \textbf{13} wol] >wol< o \textbf{14} vürstîn] fursten \textit{nachträglich korrigiert zu:} furſtÿn m fúrsten n (o)  $\cdot$ ir] irem m n [*]: irm o \textbf{15} sîn] \textit{om.} m \textbf{16} vernomen] von úch genomen n von vch vernummen o \textbf{19} vorabe] Var abe m  $\cdot$ ich] ich ie o  $\cdot$ mînen] minen \textit{nachträglich korrigiert zu:} uch minnen m \textbf{22} wirret] iret \textit{nachträglich korrigiert zu:} wiret m \textbf{25} wenne] dan o \textbf{26} iu] ir m n o  $\cdot$ hât] het o \textbf{27} biute] bitte m \textbf{28} des] das m n o \newline
\end{minipage}
\end{table}
\newpage
\begin{table}[ht]
\begin{minipage}[t]{0.5\linewidth}
\small
\begin{center}*G
\end{center}
\begin{tabular}{rl}
 & \textbf{unde} \textbf{nam} in \textbf{bî} der hant.\\ 
 & gein den vînden \textbf{an} die want\\ 
 & sâzen si in diu venster wît\\ 
 & ûf \textbf{einen} gulter \textbf{von} samît.\\ 
5 & dâr under ein weichez bette lac.\\ 
 & ist iht \textbf{liehter} danne der tac,\\ 
 & dem gelîchte niht diu künigîn.\\ 
 & si hete \textbf{aber} wîplîchen \textbf{schîn}\\ 
 & unde was anders rîterlîch,\\ 
10 & der touwegen rôsen ungelîch.\\ 
 & nâch swarzer varwe was ir schîn,\\ 
 & ir krône ein liehter rubîn.\\ 
 & ir houbet man dâ durch \textbf{wol} sach.\\ 
 & diu \textbf{wirtîn} zir gaste sprach,\\ 
15 & daz ir \textbf{wære liep} sîn komen.\\ 
 & "hêrre, ich hân \textbf{von iu} vernomen\\ 
 & vil \textbf{rîterlîcher} werdicheit.\\ 
 & durch iwer zuht \textbf{sî} \textbf{iu} niht leit,\\ 
 & \textbf{obe} ich \textbf{iu} mînen kumber klage,\\ 
20 & den ich nâhen mîne\textit{m} herzen trage."\\ 
 & "\begin{large}M\end{large}în helfe iuch \textbf{des} niht irret.\\ 
 & swaz iu war oder wirret,\\ 
 & \textbf{swâ} daz wenden sol mîn hant,\\ 
 & diu sî ze dienste dar \textbf{bewant}.\\ 
25 & ich bin niht wan \textbf{ein} einic man.\\ 
 & swer iu tuot oder hât getân,\\ 
 & dâ \textbf{engegene biut ich} mînen schilt.\\ 
 & die vînde wênic des bevilt."\\ 
 & mit zühten sprach ein vürste sân:\\ 
30 & "heten wir einen houbetman,\\ 
\end{tabular}
\scriptsize
\line(1,0){75} \newline
G O L M Q R W Z Fr29 Fr32 Fr71 \newline
\line(1,0){75} \newline
\textbf{1} \textit{Initiale} O M  \textbf{5} \textit{Initiale} Fr71  \textbf{7} \textit{Versal} Fr32  \textbf{21} \textit{Initiale} G L Q R W Z Fr32 Fr71  \textbf{29} \textit{Initiale} M  \newline
\line(1,0){75} \newline
\textbf{1} unde] Sy W (Z)  $\cdot$ nam] fienge O vie L (M) (Q) (R) (Fr32) Fr71  $\cdot$ in] in selbe O L (Fr32) on selben M in selber Q (R) (W) (Z)  $\cdot$ bî] mit W \textbf{2} den] dem Q \textbf{3} si] \textit{om.} Q  $\cdot$ in] an W  $\cdot$ diu] ein L W \textbf{4} einen] eine Fr32  $\cdot$ von] wasz Q (R) \textbf{5} weichez] wehez O wahs L weisses W (Z) \textbf{6} iht] \textit{om.} Q  $\cdot$ liehter] liehters O Z Fr32 Fr71 lýchtes L lichters M Q  $\cdot$ danne] wanne M \textbf{7} gelîchte] gelichet O L (M) (Q) R (W) (Z) Fr29 Fr32 (Fr71) \textbf{8} hete] hat R  $\cdot$ aber] \textit{om.} Z  $\cdot$ wîplîchen] wiplich R  $\cdot$ schîn] sin Q R Z Fr32 Fr71 \textbf{9} anders] avch anders O (L) (M) (Q) (R) (W) (Fr32) (Fr71) aber anders Z \textbf{10} der] Dem L (R)  $\cdot$ rôsen] \textit{om.} R \textbf{11} swarzer] rabens L rappen W swertzer Z \textbf{12} liehter] lîhter O (L) (M) (Q)  $\cdot$ rubîn] ruͦbin Fr32 robin L rubein W \textbf{13} dâ durch wol] ir dar durch wol O wol daduͯrch L (R) do durch Q \textbf{14} wirtîn] fúrstin W  $\cdot$ zir gaste] zuͯ irem gaste L (R) zcu orn gestin M \textbf{15} wære liep] lieb wer O (L) (M) (Q) (R) W (Z) (Fr32) \textbf{16} hêrre ich hân] Jch han herre M herr ih han wol Fr71 \textbf{17} rîterlîcher] ritterlichiv O (L) (M) \textbf{18} sî iu niht] lat euch seyn Q lat eúch nit W lat ev niht sin Z \textbf{19} obe] Sein das W  $\cdot$ mînen kumber] nymmer kummer \textit{nachträglich korrigiert zu:} mynen kummer Q mein angst W \textbf{20} den] Die W  $\cdot$ nâhen] in L nahen in Q (R) (Z) (Fr32) o\textit{m. } W  $\cdot$ mînem] mine \textit{nachträglich korrigiert zu:} minen G mynē L an dem Fr71  $\cdot$ herzen] hetze Q  $\cdot$ trage] [tage]: trage O \textbf{21} iuch des] euch frawe Q (Z) (Fr32) frowe úch R \textbf{22} Was úch von úwerm land wirret W  $\cdot$ swaz] Waz L (M) (Q) (R)  $\cdot$ war] werre O M R (Fr32) were Q (Z) (Fr71) \textbf{23} swâ daz wenden sol] Wa daz wenden sol L (M) (R) Das sol wenden Q Was do wenden sol W \textbf{24} diu] Sy W  $\cdot$ dar] eúch W  $\cdot$ bewant] benant O L (M) Q W Z Fr29 Fr32 \textbf{25} \textit{Versfolge 24.26-25} Q   $\cdot$ bin] on ben M (R) (Fr32)  $\cdot$ ein einic] ænich O [eẏm]: eẏn enich L eyn M \textbf{26} swer] Wer L M Q R W \textbf{27} dâ] Dan Fr71  $\cdot$ engegene biut ich] bivt ich gegn O (L) (Q) (Z) (Fr29) (Fr32) bite ich gegin M bút ich engegen R gegen beút ich W gege::: Fr71 \textbf{28} des] das R \textbf{29} zühten] zcuchte M (Q)  $\cdot$ ein] der W Z  $\cdot$ vürste sân] fursten [*]: * M \newline
\end{minipage}
\hspace{0.5cm}
\begin{minipage}[t]{0.5\linewidth}
\small
\begin{center}*T
\end{center}
\begin{tabular}{rl}
 & \textbf{und} \textbf{vienc} in \textbf{selbe} \textbf{bî} der hant.\\ 
 & gegen den vîenden \textbf{in} die want\\ 
 & sâzen sin diu venster wît\\ 
 & ûf \textbf{einen} kulter \textbf{von} samît.\\ 
5 & dâr under ein weichez bette lac.\\ 
 & ist iht \textbf{liehters} danne der tac,\\ 
 & dem glîchet niht diu künegîn.\\ 
 & si hete \textbf{aber} wîplîchen \textbf{sin}\\ 
 & und was \textbf{ouch} anders rîterlîch,\\ 
10 & der touwigen rôsen unglîch.\\ 
 & nâch swarzer varwe was ir schîn,\\ 
 & ir krône ein liehter rubîn.\\ 
 & ir houbt man dar durch \textbf{wol} sach.\\ 
 & Diu \textbf{küneginne} zir gaste sprach,\\ 
15 & daz ir \textbf{liep wære} sîn komen.\\ 
 & "hêrre, ich hân \textbf{von iu} vernomen\\ 
 & vil \textbf{rîterlîche} werdecheit.\\ 
 & durch iuwer zuht \textbf{sî} \textbf{iu} niht leit,\\ 
 & \textbf{ob} ich \textbf{iu} mînen kumber klage,\\ 
20 & den ich nâhen \textbf{an} mînem herzen trage."\\ 
 & \begin{large}M\end{large}în helfe iuch \textbf{des} niht irret.\\ 
 & swaz iu war oder wirret,\\ 
 & \textbf{swâ} daz wenden sol mîn hant,\\ 
 & di\textit{u s}î ze dienste dar \textbf{benant}.\\ 
25 & i\textbf{ne} bin niht wan \textbf{ein} einic man.\\ 
 & swer iu \textbf{iht} tuot oder hât getân,\\ 
 & dâ \textbf{biut ich gegen} mînen schilt.\\ 
 & die vîende wênic des bevilt."\\ 
 & Mit zühten sprach ein vürste sân:\\ 
30 & "hete wir einen houbetman,\\ 
\end{tabular}
\scriptsize
\line(1,0){75} \newline
T U V \newline
\line(1,0){75} \newline
\textbf{14} \textit{Majuskel} T  \textbf{21} \textit{Initiale} T  \textbf{29} \textit{Initiale} U V   $\cdot$ \textit{Majuskel} T  \newline
\line(1,0){75} \newline
\textbf{1} vienc] [*]: nam V  $\cdot$ selbe] selber V \textbf{3} sin diu] [*]: sv́ an ein V \textbf{4} einen] eine U V  $\cdot$ kulter] kuter V \textbf{6} der] den U \textbf{7} glîchet] gelichete U V \textbf{8} sin] schin U V \textbf{12} rubîn] ruͦbin U \textbf{13} man dar durch] dar dorch man U (V) \textbf{14} zir] zuͦ dem U zuͦ irm V \textbf{17} rîterlîche] ritterlicher V \textbf{18} zuht] zuͦch U \textbf{19} ich] \textit{om.} U \textbf{21} iuch] îv T \textbf{22} swaz] Waz U  $\cdot$ war oder wirret] [*]: vnde uwerme lande wirret V \textbf{23} swâ] Wol U Wo V \textbf{24} diu] div en T  $\cdot$ dar benant] dar bewant U [*]: v́ch dar benant V \textbf{25} Ine] Jch U \textbf{26} swer] Wer U \textbf{27} mînen] mime U \textbf{29} ein] der V \newline
\end{minipage}
\end{table}
\end{document}
