\documentclass[8pt,a4paper,notitlepage]{article}
\usepackage{fullpage}
\usepackage{ulem}
\usepackage{xltxtra}
\usepackage{datetime}
\renewcommand{\dateseparator}{.}
\dmyyyydate
\usepackage{fancyhdr}
\usepackage{ifthen}
\pagestyle{fancy}
\fancyhf{}
\renewcommand{\headrulewidth}{0pt}
\fancyfoot[L]{\ifthenelse{\value{page}=1}{\today, \currenttime{} Uhr}{}}
\begin{document}
\begin{table}[ht]
\begin{minipage}[t]{0.5\linewidth}
\small
\begin{center}*D
\end{center}
\begin{tabular}{rl}
\textbf{348} & \textbf{\begin{large}M\end{large}an kunde dâ} niht gâhen,\\ 
 & sô daz \textbf{Lyppaut wolde} vâhen\\ 
 & \textbf{sînen hêrren}, wander was sîn wirt,\\ 
 & \textbf{als} noch \textbf{getriwer} man verbirt.\\ 
5 & der künec ân urloup \textbf{dâ} dannen schiet,\\ 
 & als im sîn kranker sin geriet.\\ 
 & \textbf{Sîne} knappen, vürsten kindelîn,\\ 
 & al weinende \textbf{tâten} \textbf{klagen} schîn,\\ 
 & die mit dem künege dâ wâren gewesen.\\ 
10 & vor den mac Lyppaut wol genesen,\\ 
 & wander \textbf{si mit triwe hât} erzogen,\\ 
 & \textbf{gein} werder vuore niht betrogen,\\ 
 & ez ensî denne mîn \textit{hêrre} al ein,\\ 
 & an dem \textbf{doch} des vürsten triwe \textbf{erschein}.\\ 
15 & mîn hêrre ist ein Franzeys,\\ 
 & \textbf{der burcgrâve} \textbf{von} Beaveys.\\ 
 & \textbf{er} heizet Lisavander.\\ 
 & \textbf{die} eine unt \textbf{die} ander\\ 
 & muosen dem vürsten widersagen,\\ 
20 & dô si schildes ambet \textbf{muosen} tragen.\\ 
 & \textbf{Bî}me künege ritter worden sint\\ 
 & \textbf{vil vürsten} hiute unt ander kint.\\ 
 & des vorderen hers pfligt ein man,\\ 
 & der wol mit scharpfen strîten kan:\\ 
25 & \textbf{der künec} Poydiconjunz von Gors,\\ 
 & \textbf{der} vüeret manec wol gewâpent ors.\\ 
 & Meljanz ist sînes bruoder sun.\\ 
 & \textbf{si} kunnen bêde hôchvart tuon,\\ 
 & der junge unt \textbf{ouch} der alde,\\ 
30 & daz es unvuoge walde.\\ 
\end{tabular}
\scriptsize
\line(1,0){75} \newline
D \newline
\line(1,0){75} \newline
\textbf{1} \textit{Initiale} D  \textbf{7} \textit{Majuskel} D  \textbf{21} \textit{Majuskel} D  \newline
\line(1,0){75} \newline
\textbf{2} Lyppaut] Lyppaot D \textbf{10} Lyppaut] Lyppaot D \textbf{13} hêrre] \textit{om.} D \textbf{17} Lisavander] Lisavandr D \textbf{25} Poydiconjunz] Poydiconivnz D \textbf{27} Meljanz] Melianz D \newline
\end{minipage}
\hspace{0.5cm}
\begin{minipage}[t]{0.5\linewidth}
\small
\begin{center}*m
\end{center}
\begin{tabular}{rl}
 & \textbf{\textit{L}ipp\textit{o}ut wolte} niht gâhen,\\ 
 & sô daz \textbf{er sînen hêrren} \textit{v}âhen\\ 
 & \textbf{wol\textit{t}}, \textit{w}and e\textit{r w}as sîn wirt,\\ 
 & \textbf{daz} n\textit{o}ch \textbf{getriuwe\textit{r}} man verbirt.\\ 
5 & der künic âne urloup dannen schiet,\\ 
 & als ime sîn kranker sin geriet.\\ 
 & \textbf{sîne} knappen, vürsten kindelîn,\\ 
 & alweinende \textbf{tâten} \textbf{klagen} schîn,\\ 
 & die mit dem künige dâ wâren gewesen.\\ 
10 & vor den mac Lipp\textit{ou}t wol genesen,\\ 
 & wand er \textbf{si mit triuwen hât} erzogen,\\ 
 & \textbf{gegen} werder vuore niht betrogen,\\ 
 & ez ensî denne mîn hêrre alein,\\ 
 & an dem \textbf{doch} des vürsten triuwe \textbf{sch\textit{e}in}.\\ 
15 & mîn hêrre, \textbf{der} ist ein Franzo\textit{i}s,\\ 
 & \textbf{der burcgrâve} \textbf{de} Beavois.\\ 
 & \textbf{er} heizet Lisavander.\\ 
 & \textbf{die} eine und \textbf{die} ander\\ 
 & muosen dem vürsten widersagen,\\ 
20 & dô si schiltes ambet \textbf{muosen} tragen.\\ 
 & \textbf{bî} dem künige ritter worden sint\\ 
 & \textbf{vil vürsten} hiute und ander kint.\\ 
 & des vorderen hers pfliget ein man,\\ 
 & der wol mit scha\textit{r}pfen strîten kan:\\ 
25 & \textbf{der künic} Poidiconiunz von Gros,\\ 
 & \textbf{der} vüeret manic wol gewâpent ros.\\ 
 & Melianz ist sînes bruoder sun.\\ 
 & \textbf{si} k\textit{ünn}e\textit{n} beide hôchvart tuon,\\ 
 & der junge und \textbf{ouch} der alte,\\ 
30 & daz \textit{es} unvuoge walte.\\ 
\end{tabular}
\scriptsize
\line(1,0){75} \newline
m n o \newline
\line(1,0){75} \newline
\newline
\line(1,0){75} \newline
\textbf{1} Lippout] Kippavt m Lippaot n Lopolt o \textbf{2} sînen] sẏnem o  $\cdot$ vâhen] nohen m n (o) \textbf{3} Wolte er wand er wa was sin wirt m \textbf{4} noch getriuwer] nach getruwen m \textbf{7} knappen] knappe n \textbf{8} alweinende] Alle weinen n o  $\cdot$ tâten] \textit{om.} o  $\cdot$ klagen] clage n o \textbf{9} dâ] do n o  $\cdot$ wâren] weren n wore o \textbf{10} Lippout] lippoat m lippaot n lipolt o  $\cdot$ wol] \textit{om.} n o \textbf{11} hât] hette n \textbf{13} ensî] sy n (o) \textbf{14} triuwe] truwen o  $\cdot$ schein] schin m \textbf{15} Franzois] [franc*]: franczos m \textbf{16} Beavois] beatois n beators o \textbf{17} Lisavander] lisauander m [li*]: lisanander n lisanander o \textbf{19} muosen] [Musten]: Mussen o \textbf{22} vürsten] fuͯrste o \textbf{23} \textit{Versfolge 348.24-23} o  \textbf{24} der] den o  $\cdot$ scharpfen] schampfen m scharffer o \textbf{25} Poidiconiunz] poidicomus n podicamus o  $\cdot$ von] eyn o  $\cdot$ Gros] gors n grosz o \textbf{26} vüeret] fuͦrte n (o)  $\cdot$ gewâpent] gewoppen n \textbf{27} Melianz] Meliancz m Meliantz n Melancz o \textbf{28} künnen] kuͯmment m kondent o  $\cdot$ beide] beider o \textbf{30} es] \textit{om.} m  $\cdot$ unvuoge] vngefuge o \newline
\end{minipage}
\end{table}
\newpage
\begin{table}[ht]
\begin{minipage}[t]{0.5\linewidth}
\small
\begin{center}*G
\end{center}
\begin{tabular}{rl}
 & \textbf{man kunde dâ} niht gâhen,\\ 
 & sô daz \textbf{Libaut wolte} vâhen\\ 
 & \textbf{\begin{large}S\end{large}înen hêrren}, wan er was sîn wirt,\\ 
 & \textbf{als} noch \textbf{getriwer} man verbirt.\\ 
5 & der künic âne urloup dannen schiet,\\ 
 & als im sîn kranker sin geriet.\\ 
 & knappen, vürsten, \textbf{sîniu} kindelîn\\ 
 & al weinende \textbf{tæten} \textbf{klagens} schîn,\\ 
 & die mit dem künige dâ wâren gewesen.\\ 
10 & vor den mac Libaut wol genesen,\\ 
 & wan er \textbf{si hât mit triwe} erzogen,\\ 
 & \textbf{an} werder vuore niht betrogen,\\ 
 & ez ensî dane mîn hêrre al ein,\\ 
 & an dem \textbf{doch} des vürsten triwe \textbf{schein}.\\ 
15 & mîn hêrre ist ein Franzoys,\\ 
 & \textbf{li tschatelurre} \textbf{de} Beavoys.\\ 
 & \textbf{der} heizet Lisavander.\\ 
 & \textbf{die} ein\textit{e} unde \textbf{ouch} \textbf{die} ander\\ 
 & muosen dem vürsten widersagen,\\ 
20 & dô si schiltes ambet \textbf{solten} tragen.\\ 
 & \textbf{mit} dem künige rîter worden sint\\ 
 & \textbf{manic vürste} hiute unde anderiu kint.\\ 
 & des vorderen hers pfliget ein man,\\ 
 & der wol mit scharfen strîten kan:\\ 
25 & \textbf{der künic} Poydeconiunz von Gors\\ 
 & vüeret manic wol gewâpent ors.\\ 
 & Melianz ist sînes bruoder sun.\\ 
 & \textbf{die} kunnen bêde hôchvart tuon,\\ 
 & der junge unde der alde,\\ 
30 & daz es ungevüege walde.\\ 
\end{tabular}
\scriptsize
\line(1,0){75} \newline
G I O L M Q R Z Fr39 \newline
\line(1,0){75} \newline
\textbf{1} \textit{Initiale} I O L Q R Z  \textbf{3} \textit{Initiale} G  \textbf{15} \textit{Initiale} I  \newline
\line(1,0){75} \newline
\textbf{1} man] ÷an O  $\cdot$ dâ] do Q R  $\cdot$ gâhen] iagen Q \textbf{2} Libaut] Lybavt O lybauͯt L libavt M lylaut Q lybant R lybait Z  $\cdot$ vâhen] [nahen]: vahen I \textbf{3} Sinen herren was es sin wort vnd wirt R \textbf{4} noch getriwer man] man noch getrewe Q  $\cdot$ verbirt] gebirt O R \textbf{7} Sine (Sne M ) chnappen fvͤrsten chindelin (sine kindelin L ) O (L) (M) (Q) (R) (Z)  $\cdot$ knappen vürsten] fursten chnappen I \textbf{8} al] \textit{om.} I Alle O L (Q) Z  $\cdot$ weinende tæten] taten wainunde I  $\cdot$ klagens] clage L R (Z) clagin M (Q) \textbf{9} dâ] \textit{om.} Q Z  $\cdot$ wâren] seint Q (R)  $\cdot$ gewesen] wesin M \textbf{10} den] dem L R Z dē M  $\cdot$ mac] mohte I  $\cdot$ Libaut] Libavt O (M) Lybavt L Lybant R lybait Z  $\cdot$ wol] niht O \textbf{11} er si hât] er hat si O (Q) erz hat L  $\cdot$ triwe] triwen I (L) (M) (Q) (Z)  $\cdot$ erzogen] erlogen M gezogen Q \textbf{12} vuore] frevde O \textbf{13} ensî dane] si dann (Q) (R) I si O \textbf{14} schein] erscheine L \textbf{15} Franzoys] frozoys I franzeis O Frantzeis L (Z) franczeis M franzies Q franczois R \textbf{16} li tschatelurre] Lihtschahtelvrre O Lýtshate [lv*]: lvrre L (M) Lytschachteliuͯt Q Der burcgrefe Z  $\cdot$ de Beavoys] de beaueis G I der beaveys O de beaveis L (M) debeareis Q der Beawis R von beaveis Z \textbf{17} Lisavander] lisauander I Lysavander L (Z) lisafander M lyzauͯander Q Lysauander R \textbf{18} eine] einen G \textit{om.} I  $\cdot$ ouch] \textit{om.} O Q R \textbf{19} muosen] muͤsten I (Fr39)  $\cdot$ vürsten] wirte I \textbf{20} dô] Da M Z \textbf{22} hiute] \textit{om.} I L M Q R Fr39 \textbf{23} \textit{Versfolge 348.24-23} O   $\cdot$ des] Der Z \textbf{24} scharfen strîten] sharphem strite I \textbf{25} Poydeconiunz] poyde comunz I Poydekomvͦnz O poy de Conivnz L Fr39 poide kvnivnz M poydekonivnz Q poidekomvrcz R poidekonivnz Z  $\cdot$ Gors] kors I goͤrs O (Z) gorsz L gorz M gros R \textbf{26} vüeret] Vurte M  $\cdot$ wol] \textit{om.} I  $\cdot$ gewâpent] [gepa]: gewapnes R \textbf{27} Melianz] Meliantz L Z Melyas Q Meliancz R  $\cdot$ sînes] sein Q \textbf{28} die] Si O (L) (M) (Q) (R) (Z) (Fr39)  $\cdot$ kunnen] chvnden O (Q) (R)  $\cdot$ bêde] beidú R (Fr39) \textbf{29} unde] vnd auch I (O) (L) (M) Q (Z) (Fr39) \textbf{30} es] er Q  $\cdot$ ungevüege] vnfuͯge L (M) (Q) (Z) (Fr39) \newline
\end{minipage}
\hspace{0.5cm}
\begin{minipage}[t]{0.5\linewidth}
\small
\begin{center}*T
\end{center}
\begin{tabular}{rl}
 & \textbf{Sine kunden dâ} niht gâhen,\\ 
 & sô daz \textbf{Lybaut wolte} \textit{v}âhen\\ 
 & \textbf{sînen hêrren}, \textit{w}ander was sîn wirt,\\ 
 & \textbf{alse} noch \textbf{untriuwen} man verbirt.\\ 
5 & der künec âne urloup dannen \textit{schie}t,\\ 
 & \textit{als im sîn kranker sin geriet.}\\ 
 & knappen, vürsten, \textbf{sîn\textit{iu}} kindelîn\\ 
 & alweinde \textbf{tâten} \textbf{klage} schîn,\\ 
 & die mit dem künege dâ wâren gewesen.\\ 
10 & vor den mac Lybaut wol genesen,\\ 
 & wander \textbf{hât si mit triuwen} erzogen\\ 
 & \textbf{unde} \textbf{an} werder vuore niht betrogen,\\ 
 & ez ensî danne mîn hêrre al ein,\\ 
 & an dem \textbf{ouch} des vürsten triuwe \textbf{schein}.\\ 
15 & mîn hêrre ist ein Franzoys,\\ 
 & \textbf{Lethschach de lurre} \textbf{de} Beavoys.\\ 
 & \textbf{der} heizet Lysavander.\\ 
 & \textbf{der} eine unde \textbf{ouch} \textbf{der} ander\\ 
 & muosen dem vürsten widersagen,\\ 
20 & dâ si schiltes ambet \textbf{suln} tragen.\\ 
 & \textbf{mit} dem künege rîter worden sint\\ 
 & \textbf{manec vürste} hiute unde anderiu kint.\\ 
 & Des vorderen hers pfliget ein man,\\ 
 & der wol mit scharpfen strîten kan:\\ 
25 & Poydekuniunz von Gors,\\ 
 & \textbf{der} v\textit{ü}eret manec wol gewâpent ors.\\ 
 & Melyanz ist sînes bruoder suon.\\ 
 & \textbf{Si} künne\textit{n} beide hôchvart tuon,\\ 
 & der junge unde \textbf{ouch} der alte,\\ 
30 & daz es unvuoge walte.\\ 
\end{tabular}
\scriptsize
\line(1,0){75} \newline
T V W \newline
\line(1,0){75} \newline
\textbf{1} \textit{Initiale} W   $\cdot$ \textit{Majuskel} T  \textbf{23} \textit{Majuskel} T  \textbf{28} \textit{Majuskel} T  \newline
\line(1,0){75} \newline
\textbf{1} Sine kunden] Lybaut wolte V SY kunden W  $\cdot$ dâ] \textit{om.} V do W \textbf{2} So das er [*ahen]: sinen herren uahen V  $\cdot$ Lybaut] Lybavt T lybot W  $\cdot$ vâhen] nahen T W \textbf{3} sînen hêrren] Wolte V Seinem herren W  $\cdot$ wander] vander T \textbf{4} alse] Das V  $\cdot$ untriuwen] getruwer V getreúwe W  $\cdot$ verbirt] verwirt W \textbf{5} schiet] reit T \textbf{6} \textit{Vers 348.6 fehlt (Zeile ausgespart)} T  \textbf{7} sîniu] sine T \textbf{9} dâ] do V \textit{om.} W \textbf{10} Lybaut] lybavt T lybot W \textbf{11} hât si] sy hat W  $\cdot$ erzogen] wol erzogen V \textbf{12} An bernder frucht nit betrogen W \textbf{14} ouch] doch V \textbf{15} Franzoys] Franzôys T franzois V \textbf{16} [Lets*]: Lethschach de Lvrre de Beavoys T  $\cdot$ Der burggrove de beavois V  $\cdot$ Lesach de lurre de bauoys W \textbf{17} Lysavander] Lŷsavander T [*avander]: lẏsavander V lizanander W \textbf{19} muosen] mvesen T Mvͤssen V (W)  $\cdot$ dem] den W \textbf{20} dâ] Do V W  $\cdot$ suln] [*]: soltent V muͤssen W \textbf{22} hiute unde anderiu] [*]: vnde andere V vnd andre W \textbf{24} strîten] streit W \textbf{25} Poydekuniunz] poydekvnivns T [poydekv*]: poydekvmvns V Poyde guniunz W \textbf{26} der vüeret] der vneret T Fuͤrent W \textbf{27} Melyanz] Melianz V Melians W \textbf{28} künnen] kvnne T \textbf{29} unde] vn W \newline
\end{minipage}
\end{table}
\end{document}
