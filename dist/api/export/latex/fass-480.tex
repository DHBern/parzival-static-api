\documentclass[8pt,a4paper,notitlepage]{article}
\usepackage{fullpage}
\usepackage{ulem}
\usepackage{xltxtra}
\usepackage{datetime}
\renewcommand{\dateseparator}{.}
\dmyyyydate
\usepackage{fancyhdr}
\usepackage{ifthen}
\pagestyle{fancy}
\fancyhf{}
\renewcommand{\headrulewidth}{0pt}
\fancyfoot[L]{\ifthenelse{\value{page}=1}{\today, \currenttime{} Uhr}{}}
\begin{document}
\begin{table}[ht]
\begin{minipage}[t]{0.5\linewidth}
\small
\begin{center}*D
\end{center}
\begin{tabular}{rl}
\textbf{480} & \begin{large}D\end{large}en heiden het er dort erslagen;\\ 
 & den sul \textbf{ouch wir} ze mâze klagen.\\ 
 & dô uns der künec kom \textbf{sô} bleich\\ 
 & unt im sîn kraft gar gesweich,\\ 
5 & in die wunden greif eines arzetes hant,\\ 
 & unz er \textbf{des spers îsen} vant\\ 
 & - der trunzûn was rœrîn -\\ 
 & ein teil in der wunden sîn;\\ 
 & diu gewan der ar\textit{zâ}t \textbf{beidiu} wider.\\ 
10 & mîne venje viel ich nider.\\ 
 & dâ \textbf{lobt} ich der gotes kraft,\\ 
 & daz ich deheine rîterschaft\\ 
 & getæte nimer mêre,\\ 
 & daz got durch sîn êre\\ 
15 & \textbf{mînem} bruoder hülfe \textbf{von} der nôt.\\ 
 & ich \textbf{verswuor} \textbf{ouch} vleisch, wîn unt brôt\\ 
 & unt dar nâch \textbf{aldaz} trüege bluot,\\ 
 & daz ichs nimmer \textbf{mêr} \textbf{gewünne} muot.\\ 
 & daz was \textbf{der diet} ander klage,\\ 
20 & lieber neve, als ich dir sage,\\ 
 & daz ich schiet von dem swerte mîn.\\ 
 & si sprâchen: 'wer sol schirmære sîn\\ 
 & über des Grâles tougen?'\\ 
 & dô wei\textit{n}den liehtiu ougen.\\ 
25 & si truogen den künec sunder twâl\\ 
 & durch \textbf{die} gotes helfe vür den Grâl.\\ 
 & dô der künec den Grâl \textbf{gesach},\\ 
 & daz was sîn ander ungemach,\\ 
 & daz er niht sterben mohte,\\ 
30 & wand im sterben dô niht tohte,\\ 
\end{tabular}
\scriptsize
\line(1,0){75} \newline
D Fr31 \newline
\line(1,0){75} \newline
\textbf{1} \textit{Initiale} D Fr31  \newline
\line(1,0){75} \newline
\textbf{1} Den] Dden Fr31 \textbf{4} gar gesweich] so gar entwaich Fr31 \textbf{5} in die] inde D  $\cdot$ greif] girif Fr31 \textbf{7} der] Daz Fr31 \textbf{9} arzât] Arlt D \textbf{11} dâ] Do Fr31 \textbf{12} ich] \textit{om.} Fr31 \textbf{16} wîn] \textit{om.} Fr31 \textbf{18} ichs] ich Fr31  $\cdot$ muot] gvͦt Fr31 \textbf{21} mîn] \textit{om.} Fr31 \textbf{24} weinden] weiden D \textbf{30} wand] Wain Fr31 \newline
\end{minipage}
\hspace{0.5cm}
\begin{minipage}[t]{0.5\linewidth}
\small
\begin{center}*m
\end{center}
\begin{tabular}{rl}
 & den \dag helt\dag  er dort erslagen;\\ 
 & den sollen \textbf{wir} ze mâze klagen.\\ 
 & \begin{large}D\end{large}ô uns der künic kam \textbf{zuo} bleich\\ 
 & und im sîn kraft \textbf{sô} gar gesweich,\\ 
5 & in die wunden greif eines arzetes hant,\\ 
 & unz er \textbf{des spers îsen} vant\\ 
 & - der trunzûn was rœrîn -\\ 
 & ein teil in der wunden sîn;\\ 
 & diu g\textit{e}wan der arzet wider.\\ 
10 & mîn venje viel ich nider.\\ 
 & dô \textbf{lobet} ich der gotes kraft,\\ 
 & daz ich kein ritterschaft\\ 
 & getæte nimer mêre,\\ 
 & daz got durch sîn êre\\ 
15 & \textbf{mînen} bruoder hülfe \textbf{ûz}er nôt.\\ 
 & ich \textbf{verswüere} \textbf{ouch} vleisch, wîn und brôt\\ 
 & und dar nâch \textbf{al daz} trüege bluot,\\ 
 & daz ichs nimer \textbf{gewinne} muot.\\ 
 & daz was \textbf{des volkes} ander klage,\\ 
20 & lieber neve, \textit{a}l\textit{sô}, \textit{i}ch dir sage,\\ 
 & daz ich schiet von dem swerte mîn.\\ 
 & si sprâchen: 'wer sol schirmer sîn\\ 
 & über des Grâles to\textit{u}gen?'\\ 
 & dô weineten liehtiu ougen.\\ 
25 & si truogen den künic sunder twâl\\ 
 & durch gotes helfe vür den Grâl.\\ 
 & dô der künic den Grâl \textbf{ersach},\\ 
 & daz was sîn ander ungemach,\\ 
 & daz er niht sterben m\textit{o}hte,\\ 
30 & wan i\textit{m} sterben dô niht tohte,\\ 
\end{tabular}
\scriptsize
\line(1,0){75} \newline
m n o \newline
\line(1,0){75} \newline
\textbf{3} \textit{Initiale} m  \newline
\line(1,0){75} \newline
\textbf{2} mâze] [*]: mossen n massen o \textbf{3} zuo] so o \textbf{4} gesweich] versweich o \textbf{5} arzetes] artzas n (o) \textbf{6} des] dasz o \textbf{7} trunzûn] tronczin o \textbf{9} gewan] gawan m  $\cdot$ wider] beide wider n (o) \textbf{10} nider] wol nyder n \textbf{15} mînen] [Mẏner]: Mẏnen o  $\cdot$ hülfe] halff n \textbf{16} verswüere] verswúr o \textbf{20} alsô ich] los waz ich m \textbf{21} ich schiet] ichs [nẏmer]: schiet o \textbf{22} sprâchen] sprach o \textbf{23} tougen] toigen m tagen o \textbf{27} Grâl] grole n \textbf{29} mohte] moͯhte m (n) \textbf{30} im] in m [in]: im o  $\cdot$ tohte] doͯchte n \newline
\end{minipage}
\end{table}
\newpage
\begin{table}[ht]
\begin{minipage}[t]{0.5\linewidth}
\small
\begin{center}*G
\end{center}
\begin{tabular}{rl}
 & \begin{large}D\end{large}en heiden het er dort erslagen;\\ 
 & den suln \textbf{wir ouch} ze mâzen klagen.\\ 
 & dô uns der künic kom \textbf{sô} bleich\\ 
 & unde im sîn kraft \textbf{sô} gar gesweich,\\ 
5 & in die wunden greif eines arzâtes hant,\\ 
 & unze er \textbf{des spers îsen} vant\\ 
 & - der trunzûn was rœrîn -\\ 
 & ein teil in der wunden sîn;\\ 
 & diu gewan der arzet \textbf{beidiu} wider.\\ 
10 & mîne venje viel ich nider.\\ 
 & dâ \textbf{lobet} ich der gotes kraft,\\ 
 & daz ich dehein rîterschaft\\ 
 & getæte nimmer mêre,\\ 
 & daz got durch sîn êre\\ 
15 & \textbf{mînem} bruoder hülfe \textbf{von} der nôt.\\ 
 & ich \textbf{verswuor} \textbf{ouch} vleisch, wîn unde brôt\\ 
 & unt dar nâch \textbf{al daz} trüege bluot,\\ 
 & daz ichs nimmer \textbf{mêr} \textbf{gewünne} muot.\\ 
 & daz was \textbf{der diet} ander klage,\\ 
20 & lieber neve, als ich dir sage,\\ 
 & daz ich schiet von dem swerte mîn.\\ 
 & si sprâchen: 'wer sol schirmære sîn\\ 
 & über des Grâles tougen?'\\ 
 & \textit{dô} \textit{weinten liehtiu ougen}.\\ 
25 & si truogen de\textit{n} künic sunder twâl\\ 
 & durch \textbf{die} gotes helfe vür den Grâl.\\ 
 & dô der künic den Grâl \textbf{gesach},\\ 
 & daz was sîn ander ungemach,\\ 
 & daz er niht sterben mohte,\\ 
30 & wan im sterben dô niht tohte,\\ 
\end{tabular}
\scriptsize
\line(1,0){75} \newline
G I O L M Z \newline
\line(1,0){75} \newline
\textbf{1} \textit{Initiale} G I O L Z  \textbf{17} \textit{Initiale} I  \textbf{27} \textit{Initiale} M  \newline
\line(1,0){75} \newline
\textbf{1} Den] ÷en O \textbf{2} wir ouch] wir I ovch wir O (L) (M) (Z)  $\cdot$ ze mâzen] zemaze O (L) (M) \textbf{3} dô] Da M Z  $\cdot$ künic] \textit{om.} I O  $\cdot$ sô] also I \textbf{4} sô gar] \textit{om.} O gar L M \textbf{5} greif] \textit{om.} L \textbf{6} des] das M \textbf{7} was] der was Z \textbf{9} diu] die I  $\cdot$ gewan] wan Z  $\cdot$ beidiu] beide I \textbf{11} dâ] do I Daz Z \textbf{14} durch] [der]: dvrch G \textbf{15} von] vz O L \textbf{16} ouch] \textit{om.} O L M \textbf{17} al daz] swaz O \textbf{18} ichs] ich I ich sin O Z  $\cdot$ mêr] \textit{om.} O Z  $\cdot$ gewünne] \textit{om.} I \textbf{19} daz was] Gewunne daz I  $\cdot$ diet] thuet M  $\cdot$ ander] ir ander I dy andir M \textbf{22} sprâchen] sprach M \textbf{23} über des Grâles] des grales vber I \textbf{24} \textit{Vers 480.24 fehlt} G   $\cdot$ dô] Da M Z  $\cdot$ liehtiu] lýchten L (M) \textbf{25} den] der G  $\cdot$ sunder] sundern M \textbf{26} die] \textit{om.} I O \textbf{27} dô] Da Z  $\cdot$ gesach] ersach L \textbf{30} sterben dô] do sterben I \newline
\end{minipage}
\hspace{0.5cm}
\begin{minipage}[t]{0.5\linewidth}
\small
\begin{center}*T
\end{center}
\begin{tabular}{rl}
 & Den heiden heter dort erslagen;\\ 
 & den suln \textbf{wir} \textbf{hie} ze mâzen klagen.\\ 
 & \begin{large}D\end{large}ô uns der künec kom \textbf{sô} bleich\\ 
 & unde im sîn kraft gar gesweich,\\ 
5 & in die wunden greif eines arzâtes hant,\\ 
 & unzer \textbf{daz sperîsen} vant\\ 
 & - der trunzûn was rœrîn -\\ 
 & ein teil in der wunden sîn;\\ 
 & di\textit{u} gewan der arzât \textbf{beidiu} wider.\\ 
10 & Mîne venje viel ich nider.\\ 
 & dô \textbf{gelobt}ich der gotes kraft,\\ 
 & daz ich deheine rîterschaft\\ 
 & getæte niemer mêre,\\ 
 & daz got durch sîn êre\\ 
15 & \textbf{mînem} bruoder hülfe \textbf{ûz}er nôt.\\ 
 & ich \textbf{verswuor} vleisch, wîn unde brôt\\ 
 & unde dar nâch \textbf{swaz} trüege bluot,\\ 
 & daz ich\textit{s} niemer \textbf{gewinne} muot.\\ 
 & daz was \textbf{der diet} ander klage,\\ 
20 & lieber neve, alsich dir sage,\\ 
 & daz ich schiet von dem swerte mîn.\\ 
 & si sprâchen: 'wer sol schirmer sîn\\ 
 & über des Grâles tougen?'\\ 
 & dâ weinden liehtiu ougen.\\ 
25 & \begin{large}S\end{large}i truogen den künec sunder twâl\\ 
 & durch \textbf{die} gotes helfe vür den Grâl.\\ 
 & dô der künec den Grâl \textbf{gesach},\\ 
 & daz was sîn ander ungemach,\\ 
 & daz er niht sterben mohte,\\ 
30 & wandim sterben dô niht tohte,\\ 
\end{tabular}
\scriptsize
\line(1,0){75} \newline
T U V W Q R \newline
\line(1,0){75} \newline
\textbf{1} \textit{Initiale} W Q   $\cdot$ \textit{Capitulumzeichen} R   $\cdot$ \textit{Majuskel} T  \textbf{3} \textit{Initiale} T V  \textbf{10} \textit{Majuskel} T  \textbf{25} \textit{Initiale} T  \textbf{27} \textit{Initiale} W  \newline
\line(1,0){75} \newline
\textbf{1} \textit{Die Verse 453.1-502.30 fehlen} U   $\cdot$ heiden] [*]: heiden V  $\cdot$ heter] hat er R \textbf{2} den] Die Q  $\cdot$ wir hie] wir [*]: oͮch V wir W auch wir Q (R)  $\cdot$ mâzen] masse W (Q) mere R \textbf{3} der künec kom] kam der kung R \textbf{4} gar] so gar V noch W nochen Q \textit{om.} R  $\cdot$ gesweich] entweich R \textbf{6} daz sperîsen] dez spers ysin V (W) (Q) (R) \textbf{7} rœrîn] roͯrrig R \textbf{9} diu] die T \textbf{10} Mîne] Vf mine V An mine R \textbf{11} gelobtich] [*]: lobete ich V lobt ich W Q (R) \textbf{13} getæte] [*]: Getete V Stete R \textbf{14} durch sîn] [*]: durch sin V sin durch R durch seiner W \textbf{15} hülfe] [*]: húlfe V halff R  $\cdot$ ûzer] [*]: vz der V \textbf{16} ich] Vnd R \textbf{17} unde] \textit{om.} W  $\cdot$ swaz] [*]: als daz V alles das W Q alles daz da R \textbf{18} ichs] ichz T ich W Q R  $\cdot$ niemer] niemer [g*]: mer V nymmer mer W (Q) (R)  $\cdot$ gewinne] [*]: gewúnne V gewúne Q  $\cdot$ muot] [*]: mvͦt V guͦt W \textbf{19} der diet ander] [*]: dez volkes ander V \textbf{24} dâ] Do W Q R  $\cdot$ weinden] weinte Q \textbf{25} truogen] trungen Q \textbf{26} durch die] Mit W Durch R  $\cdot$ vür] \textit{om.} Q \textbf{29} sterben] sterbe Q \textbf{30} dô] \textit{om.} R  $\cdot$ tohte] endochte R \newline
\end{minipage}
\end{table}
\end{document}
