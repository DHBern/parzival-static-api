\documentclass[8pt,a4paper,notitlepage]{article}
\usepackage{fullpage}
\usepackage{ulem}
\usepackage{xltxtra}
\usepackage{datetime}
\renewcommand{\dateseparator}{.}
\dmyyyydate
\usepackage{fancyhdr}
\usepackage{ifthen}
\pagestyle{fancy}
\fancyhf{}
\renewcommand{\headrulewidth}{0pt}
\fancyfoot[L]{\ifthenelse{\value{page}=1}{\today, \currenttime{} Uhr}{}}
\begin{document}
\begin{table}[ht]
\begin{minipage}[t]{0.5\linewidth}
\small
\begin{center}*D
\end{center}
\begin{tabular}{rl}
\textbf{642} & \begin{large}B\end{large}ene und Arnive dô\\ 
 & schuofen, daz ez \textbf{stuont} alsô,\\ 
 & dâ von der wirt gemach erleit.\\ 
 & diu herzogîn daz niht vermeit,\\ 
5 & dâ\textbf{ne} wære ir helfe nâhe bî.\\ 
 & \textbf{Gawanen} \textbf{vuorten} dise drî\\ 
 & mit \textbf{in} dan durch sîn gemach.\\ 
 & In einer kemenâten er \textbf{ersach}\\ 
 & \textbf{zwei} bette sunder ligen.\\ 
10 & nû wirt \textbf{iu} gar von mir verswigen,\\ 
 & wie diu gehêrt wæren;\\ 
 & ez næhet andern mæren.\\ 
 & Arnive zer herzoginne sprach:\\ 
 & "nû sult ir schaffen guot gemach\\ 
15 & disem rîter, den ir brâhtet her.\\ 
 & \textbf{ob} der \textbf{helfe an iu} ger,\\ 
 & iwerer helfe habt ir êre.\\ 
 & i\textbf{ne} sage iu \textbf{nû} niht mêre,\\ 
 & Wan daz sîne wunden\\ 
20 & mit kunst \textbf{sô sint} \textbf{gebunden},\\ 
 & er m\textit{ö}hte nû wol wâpen tragen;\\ 
 & doch sult ir sînen kumber klagen,\\ 
 & ob ir \textbf{im} senftet, daz ist guot.\\ 
 & \textbf{lêret} \textbf{ir} in hôhen muot,\\ 
25 & des muge wir \textbf{alle} geniezen;\\ 
 & nû lâts iuch niht verdriezen."\\ 
 & Diu künegîn Arnive gienc,\\ 
 & dô si ze hove urloup enpfienc;\\ 
 & Bene ein lieht \textbf{vor ir truoc} dan.\\ 
30 & die tür beslôz hêr Gawan.\\ 
\end{tabular}
\scriptsize
\line(1,0){75} \newline
D Z Fr1 \newline
\line(1,0){75} \newline
\textbf{1} \textit{Initiale} D Z Fr1  \textbf{8} \textit{Majuskel} D  \textbf{19} \textit{Majuskel} D  \textbf{27} \textit{Majuskel} D  \newline
\line(1,0){75} \newline
\textbf{1} Arnive] Arnîve D Fr1 \textbf{2} daz] da Z \textbf{5} dâne wære] Da enweren Z \textbf{6} Gawanen] Gawan Z \textbf{8} ersach] sach Z \textbf{12} næhet] nahet nu Z \textbf{13} Arnive] Arnîve D Fr1 \textbf{16} iu] evch Z (Fr1) \textbf{21} möhte] mohte D (Z) \textbf{24} in] in nu Z \textbf{26} lâts] lat Z \textbf{27} Arnive] Arnîve D Fr1 \textbf{28} dô] Da Z \textbf{30} hêr] min her Z \newline
\end{minipage}
\hspace{0.5cm}
\begin{minipage}[t]{0.5\linewidth}
\small
\begin{center}*m
\end{center}
\begin{tabular}{rl}
 & Bene und Ar\textit{niv}e dô\\ 
 & schuofen, daz ez \textbf{stüende} alsô,\\ 
 & dâ von der wirt gemach erleit.\\ 
 & diu herzogîn daz niht vermeit,\\ 
5 & d\textit{â} wær ir he\textit{l}fe nâhe bî.\\ 
 & \textbf{Gawanen} \textbf{vuorten} dise drî\\ 
 & mit \textbf{in} dan durch sîn gemach.\\ 
 & in einer kemenâten er \textbf{sach}\\ 
 & \textbf{zwei} bette sunder ligen.\\ 
10 & nû wirt \textbf{ouch} gar von mir verswigen,\\ 
 & wie diu gehêret wæren;\\ 
 & ez nâhet andern mæren.\\ 
 & Ar\textit{niv}e zer herzogîn sprach:\\ 
 & "nû solt ir schaffen guot gemach\\ 
15 & dise\textit{m} ritter, den ir brâhtet her.\\ 
 & \textbf{dô} der \textbf{an iu helfe} ger,\\ 
 & iuwer helfe habt ir êre.\\ 
 & ich sage iu \textbf{nû} niht mêre,\\ 
 & wan daz sîn wunden\\ 
20 & mit kunst \textbf{sint sô} \textbf{gebunden},\\ 
 & er m\textit{ö}hte nû wol wâpen tragen;\\ 
 & doch solt ir sînen kumber klagen,\\ 
 & ob ir \textbf{im} \textbf{in} senftet, daz ist guot.\\ 
 & \textbf{gelêret} \textbf{ir} in hôhen muot,\\ 
25 & des mugen wir \textbf{als} geniezen;\\ 
 & nû lâts iuch niht verdriezen."\\ 
 & diu künigîn Ar\textit{niv}e \textbf{ouch} gienc,\\ 
 & dô si zuo hof urloup enpfienc;\\ 
 & Bene ein lieht \textbf{truoc vür in} dan.\\ 
30 & die tür beslôz hêr Gawan.\\ 
\end{tabular}
\scriptsize
\line(1,0){75} \newline
m n o \newline
\line(1,0){75} \newline
\newline
\line(1,0){75} \newline
\textbf{1} Arnive] arune m arniwe n arnwe o \textbf{3} gemach] gemaches o \textbf{5} dâ] Do m n Der o  $\cdot$ helfe] heffe m \textbf{6} Gawan fuͯrte dise dri o \textbf{7} in dan] zúcht o \textbf{8} einer kemenâten] eẏne camenate o \textbf{12} Es nohent ander meren o \textbf{13} Arnive] Arune m Arniwe n Arnwe o \textbf{15} disem] Disen m  $\cdot$ den] der o  $\cdot$ brâhtet] brochten n \textbf{21} möhte] mohtte m (o) \textbf{23} im in] in jme n (o) \textbf{24} ir in] in ir n in in o \textbf{25} als] alle n o \textbf{27} Arnive] arurvne m arniwe n arnwe o  $\cdot$ ouch] \textit{om.} n \textbf{29} truoc vür in] vor ir truͯg n (o) \newline
\end{minipage}
\end{table}
\newpage
\begin{table}[ht]
\begin{minipage}[t]{0.5\linewidth}
\small
\begin{center}*G
\end{center}
\begin{tabular}{rl}
 & Bene unde Arnive dô\\ 
 & schuofen, daz ez \textbf{stuont} alsô,\\ 
 & dâ von der wirt gemach erleit.\\ 
 & diu herzogîn daz niht vermeit,\\ 
5 & dâ\textbf{ne} wære ir helfe nâhen bî.\\ 
 & \textbf{Gawan} \textbf{vuorte} dise drî\\ 
 & mit \textbf{i\textit{m}} dan durch sîn gemach.\\ 
 & in einer kemenâten er \textbf{sach}\\ 
 & \textbf{vier} bette sunder ligen.\\ 
10 & nû wirt \textbf{iuch} gar von mir verswigen,\\ 
 & wie diu gehêrt wæren;\\ 
 & ez nâhet \textbf{nû} andern mæren.\\ 
 & Arnive zer herzoginne sprach:\\ 
 & "nû sult ir schaffen guoten gemach\\ 
15 & disem rîter, den ir brâhtet her.\\ 
 & \textbf{op} der \textbf{helfe an iu} ger,\\ 
 & iwer helfe habet ir êre.\\ 
 & ich\textbf{n} sage iu \textbf{nû} niht mêre,\\ 
 & wan daz sîne wunden\\ 
20 & mit kunst \textbf{sô sint} \textbf{gebunden},\\ 
 & er m\textit{ö}ht nû wol wâpen tragen;\\ 
 & doch sult ir sînen kumber klagen,\\ 
 & ob ir \textbf{in} senftet, daz ist guot.\\ 
 & \textbf{lêrt} \textbf{ir} in \textbf{nû} hôhen muot,\\ 
25 & des muge wir \textbf{alle} geniezen;\\ 
 & nû lâts iuch niht verdriezen."\\ 
 & diu künegîn Arnive gienc,\\ 
 & dô si ze hove urloup enpfienc;\\ 
 & Bene ein lieh\textit{t} \textbf{vor in truoc} dan.\\ 
30 & die tür beslôz \textbf{mîn} hêr Gawan.\\ 
\end{tabular}
\scriptsize
\line(1,0){75} \newline
G I L M Z Fr18 \newline
\line(1,0){75} \newline
\textbf{1} \textit{Initiale} L Z Fr18  \textbf{21} \textit{Initiale} I  \newline
\line(1,0){75} \newline
\textbf{1} Arnive] arniue I ARnyue Fr18  $\cdot$ dô] da M \textbf{2} daz] da Z \textbf{5} dâne wære] dane [w*]: weir I Da enweren Z \textbf{6} vuorte] fuͯrten L (Z) \textbf{7} im] in G L (M) Z  $\cdot$ sîn] \textit{om.} L \textbf{8} sach] gesach L \textbf{9} vier] Zwei Z \textbf{10} wirt] wart Fr18  $\cdot$ iuch] och L (M) (Fr18) \textbf{11} wæren] waren L (Fr18) \textbf{12} nâhet] nahent I \textbf{13} Arnive] Arniue I ARnẏne Fr18 \textbf{14} schaffen] [slaffen]: shaffen I  $\cdot$ guoten] guͤt I (L) (M) \textbf{15} disem] Dem L \textbf{16} iu] ewer I \textbf{18} nû] \textit{om.} L \textbf{20} sô] \textit{om.} I \textbf{21} möht] moht G (I) (L) (M) (Fr18) \textbf{23} in] im I \textbf{24} in] \textit{om.} Fr18 \textbf{25} des] Das M  $\cdot$ geniezen] wol geniezen Fr18 \textbf{26} lâts iuch] lat evch sin I lat evch Z ::: ivch Fr18 \textbf{27} Arnive] Arniue I ARnẏue Fr18 \textbf{28} dô] Da M Z \textbf{29} lieht] lieh G  $\cdot$ vor in truoc] vor in >troͮch< G truͤc vor in I (Fr18) trvͯch vor ir L truc vor [yn]: yr M vor ir trvc Z \textbf{30} hêr Gawan] ergawan M \newline
\end{minipage}
\hspace{0.5cm}
\begin{minipage}[t]{0.5\linewidth}
\small
\begin{center}*T
\end{center}
\begin{tabular}{rl}
 & Bene und Arnyve dô\\ 
 & schuofen, daz ez \textbf{stuont} alsô,\\ 
 & dâ von der wirt gemach erleit.\\ 
 & diu herzogîn daz niht vermeit,\\ 
5 & d\textit{â} \textbf{en}wære ir helfe nâhe bî.\\ 
 & \textbf{Gawan} \textbf{vuort\textit{e}n} dise drî\\ 
 & mit \textbf{in} dan durch sîn gemach.\\ 
 & in einer kemenâten er \textbf{sach}\\ 
 & \textbf{zwei} bette sunder ligen.\\ 
10 & nû wirt \textbf{iu} gar von mir verswigen,\\ 
 & wie diu gehêret wæren;\\ 
 & ez nâhet \textbf{nû} andern mæren.\\ 
 & \begin{large}A\end{large}rnyve zuo der herzoginne sprach:\\ 
 & "nû solt ir schaffen guot gemach\\ 
15 & disem rîter, den ir brâhtet her.\\ 
 & \textbf{ob} der \textbf{helfe an iu} ger,\\ 
 & iuwer helfe habet ir êre.\\ 
 & ich \textbf{en}sage iu niht mêre,\\ 
 & wan daz sîne wunden\\ 
20 & mit kunst \textbf{dô sint} \textbf{verbunden},\\ 
 & er m\textit{ö}hte nû wol wâpen tragen;\\ 
 & doch solt \textit{i}r sînen kumber klagen,\\ 
 & ob ir \textbf{in} senftet, daz ist guot.\\ 
 & \textbf{lêret} in \textbf{nû} h\textit{ôh}en muot,\\ 
25 & des mogen wir \textbf{alle} geniezen;\\ 
 & nû lât e\textit{s} iuch niht verdriezen."\\ 
 & diu küneginne Arnyve gienc,\\ 
 & dô si zuo hove urloup entvienc;\\ 
 & Bene ein lieht \textbf{truoc vor ir} dan.\\ 
30 & die türe beslôz hêr Gawan.\\ 
\end{tabular}
\scriptsize
\line(1,0){75} \newline
U V W Q R \newline
\line(1,0){75} \newline
\textbf{1} \textit{Initiale} W Q  \textbf{13} \textit{Initiale} U V R  \newline
\line(1,0){75} \newline
\textbf{1} Arnyve] arniue V Q arnyue W R \textbf{4} daz] die R \textbf{5} dâ] Do U W  $\cdot$ enwære] werre R \textbf{6} Gawan] Gawanen V W (Q) R  $\cdot$ vuorten] vuͦrt in U \textbf{7} dan] do Q \textbf{8} Jn einer [kemenat*]: kemenaten er sach V  $\cdot$ er] er do Q \textbf{10} iu] \textit{om.} W  $\cdot$ mir] mir doch R \textbf{12} nû] \textit{om.} W  $\cdot$ andern mæren] andrú mere R \textbf{13} Arnyve] Arniue V Q Arnyue W R  $\cdot$ herzoginne] herczoginen R \textbf{14} schaffen] [*affen]: schaffen V slaffen Q \textbf{15} disem] [Di*]: Disem U Disen W (Q)  $\cdot$ brâhtet] hant broht V bachttent R \textbf{18} ensage] sag Q R  $\cdot$ iu] eúch nun W (Q) \textbf{20} dô sint] so sint V (W) Q sind so R  $\cdot$ verbunden] gebunden Q R \textbf{21} möhte] mochte U (Q) [mohte]: moͤhte  V moͯch R \textbf{22} ir] er U \textbf{23} in] [*]: im V im den R  $\cdot$ daz] \textit{om.} R \textbf{24} in] irn V (W) (Q)  $\cdot$ hôhen] horten U \textbf{25} wir] wil Q \textbf{26} lât es] latiz U (R) [*]: lant ez  V last W (Q) \textbf{27} Arnyve] [a*]: arniue V arnyue W R arniue Q  $\cdot$ gienc] die gienc R \textbf{30} beslôz] beschloß mein W (R) schlosz im Q \newline
\end{minipage}
\end{table}
\end{document}
