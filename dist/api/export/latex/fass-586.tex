\documentclass[8pt,a4paper,notitlepage]{article}
\usepackage{fullpage}
\usepackage{ulem}
\usepackage{xltxtra}
\usepackage{datetime}
\renewcommand{\dateseparator}{.}
\dmyyyydate
\usepackage{fancyhdr}
\usepackage{ifthen}
\pagestyle{fancy}
\fancyhf{}
\renewcommand{\headrulewidth}{0pt}
\fancyfoot[L]{\ifthenelse{\value{page}=1}{\today, \currenttime{} Uhr}{}}
\begin{document}
\begin{table}[ht]
\begin{minipage}[t]{0.5\linewidth}
\small
\begin{center}*D
\end{center}
\begin{tabular}{rl}
\textbf{586} & den iwer kraft dâ zuo \textbf{betwanc},\\ 
 & daz der junge, süeze ranc\\ 
 & nâch werder âmîen,\\ 
 & von Kanadic Florien.\\ 
5 & sînes vater lant von kinde er vlôch;\\ 
 & diu selbe küneginne in zôch;\\ 
 & ze Bertane \textbf{er was} \textbf{ein} gast.\\ 
 & Florie \textbf{in luot} mit \textbf{minnen} last,\\ 
 & \textbf{daz} si in \textbf{verjagte} vürz lant.\\ 
10 & in ir dienste man in vant\\ 
 & tôt, als ir wol hât vernomen.\\ 
 & Gawans künne ist dicke komen\\ 
 & \textbf{durch} minne in \textbf{herzebæriu} sêr.\\ 
 & Ich nenne iu sîner mâge mêr,\\ 
15 & den ouch von minne ist worden wê:\\ 
 & \textbf{wes} \textbf{twanc} der bluotvarwe snê\\ 
 & Parzivals \textbf{getriwen} lîp?\\ 
 & \textbf{daz schuof diu} küneginne, sîn wîp.\\ 
 & Galoesen und Gahmureten,\\ 
20 & die habt ir bêde übertreten,\\ 
 & daz ir si gâbet an den rê.\\ 
 & diu \textbf{junge}, werde Itonje\\ 
 & \textbf{truoc} nâch \textbf{dem künege} Gramoflanz\\ 
 & mit \textbf{triwen stæte} minne ganz;\\ 
25 & daz was Gawans \textbf{swester clâr}.\\ 
 & vrou Minne, ir teilet \textbf{ouch} \textbf{iwern} vâr\\ 
 & \textbf{Surdamur} \textbf{durch} Alexandern.\\ 
 & die eine unt die andern,\\ 
 & \begin{large}S\end{large}waz Gawan künnes ie gewan,\\ 
30 & vrou Minne, die wolt ir \textbf{niht} erlân,\\ 
\end{tabular}
\scriptsize
\line(1,0){75} \newline
D Z \newline
\line(1,0){75} \newline
\textbf{7} \textit{Initiale} Z  \textbf{14} \textit{Majuskel} D  \textbf{29} \textit{Initiale} D  \newline
\line(1,0){75} \newline
\textbf{3} âmîen] ameyen Z \textbf{4} Kanadic] kanedich D  $\cdot$ Florien] Florŷen D floreyen Z \textbf{5} \textit{Versfolge 586.6-5} Z  \textbf{7} Bertane er was] britanie was er Z \textbf{8} Florie] Floreie Z \textbf{13} durch] von Z \textbf{17} Parzivals] Parcifals Z \textbf{19} Galoesen] Galôesen D  $\cdot$ Gahmureten] Gahmvreten D Gamureten Z \textbf{22} Itonje] Jtonîe D iconie Z \textbf{23} Leit ouch nach rois Gramoflans Z \textbf{26} ouch] \textit{om.} Z \textbf{27} Surdamur] Svrdamor Z \textbf{30} wolt] woldet Z \newline
\end{minipage}
\hspace{0.5cm}
\begin{minipage}[t]{0.5\linewidth}
\small
\begin{center}*m
\end{center}
\begin{tabular}{rl}
 & den iuwer kraft dâ zuo \textbf{betwanc},\\ 
 & daz der junge, süeze ranc\\ 
 & nâch werder âm\textit{î}en,\\ 
 & von K\textit{a}n\textit{ed}ic Flo\textit{r}ien.\\ 
5 & sîn\textit{es} vater lant von kinde  vlôch;\\ 
 & diu selbe künigîn in zôch;\\ 
 & zuo Britane \textbf{er wær} \textbf{ein} gast.\\ 
 & Florie \textbf{in luot} mit \textbf{minne} last,\\ 
 & \textbf{dô} si in \textbf{verjagte} vür daz lant.\\ 
10 & in ir dienst man i\textit{n} vant\\ 
 & tôt, als ir wol hât vernomen.\\ 
 & Gawans k\textit{ü}n\textit{n}e ist dicke komen\\ 
 & \textbf{durch} minne in \textbf{herzebæriu} sêr.\\ 
 & ich nenne iu sîner mâge mêr,\\ 
15 & den ouch von minne ist worden wê:\\ 
 & \textbf{des} \textbf{twanc} der bluotvarwe snê\\ 
 & Parcifals \textbf{getriuwen} lîp;\\ 
 & \textbf{daz schuof diu} künigîn, sîn wîp.\\ 
 & Galo\textit{e}sen und Gahmureten,\\ 
20 & die habt ir beide \textit{üb}ertreten,\\ 
 & daz ir si gâbet an den rê.\\ 
 & diu \textbf{junge}, werde Ithonie\\ 
 & \textbf{truoc} nâch \textbf{rois} Gram\textit{o}lanz\\ 
 & mit \textbf{stæten triuwen} minne ganz;\\ 
25 & daz was Gawans \textbf{clâriu swester zwâr}.\\ 
 & vrouwe Minne, \dag und teilt\textit{e\dag } \textbf{\textit{o}uch} vâr\\ 
 & \textbf{Surdamur} \textbf{durch} Alexandren.\\ 
 & die ein und die andren,\\ 
 & waz Gawan kü\textit{nn}es ie gewan,\\ 
30 & vrouwe Minne,  wolt \dag in\dag  \textbf{niht} erlân,\\ 
\end{tabular}
\scriptsize
\line(1,0){75} \newline
m n o \newline
\line(1,0){75} \newline
\newline
\line(1,0){75} \newline
\textbf{1} iuwer] vwern o \textbf{3} âmîen] amren m (o) \textbf{4} Von kunnig flozien m  $\cdot$ Von kanadig flozien n  $\cdot$ Von kanadig florzien o \textbf{5} sînes] Sin m \textbf{7} Britane] brittane m  $\cdot$ wær] was n o \textbf{8} Florie] Flurie o  $\cdot$ Minne] mẏnnen n (o) \textbf{10} ir] jren o  $\cdot$ in] ym m \textbf{12} künne] kunige m \textbf{13} herzebæriu sêr] hertzeberende ser n [hercze sere]: herczebere sere o \textbf{19} Galoesen] Galoͯsen m  $\cdot$ Gahmureten] gamuretten m gamireten n gamureten o \textbf{20} übertreten] wider tretten m \textbf{22} diu] Der o  $\cdot$ Ithonie] jtonie m itonie n o \textbf{23} Gramolanz] gramulancz m gramonlantz n gramanlancz o \textbf{24} stæten] stete n \textbf{25} Gawans] gawanes n o  $\cdot$ clâriu] \textit{om.} n o  $\cdot$ zwâr] [c*]: clor n clar o \textbf{26} teilte ouch] teilte vnd teilte ouch m teilte ouch vwern n teẏlte vwer o \textbf{27} Surdamur] Súrdamúr o  $\cdot$ Alexandren] allexandern n alexandern o \textbf{28} andren] ander o \textbf{29} Gawan] gawanes n  $\cdot$ künnes] komes m kúnne n konnen o  $\cdot$ ie] \textit{om.} n \newline
\end{minipage}
\end{table}
\newpage
\begin{table}[ht]
\begin{minipage}[t]{0.5\linewidth}
\small
\begin{center}*G
\end{center}
\begin{tabular}{rl}
 & den iuwer kraft dar zuo \textbf{dwanc},\\ 
 & daz der junge, süeze r\textit{a}n\textit{c}\\ 
 & nâch werder âmîen,\\ 
 & von Kanadic Florien.\\ 
 & \hspace*{-.7em}\big| diu selbe küneginne in zôch;\\ 
5 & \hspace*{-.7em}\big| sînes vater lant von kin\textit{d}e er vlôch;\\ 
 & ze Britanie \textbf{was er} gast.\\ 
 & Florie \textbf{luot in} mit last,\\ 
 & \textbf{daz} si in \textbf{jagete} \textit{vür} daz lant.\\ 
10 & in ir dienst man in vant\\ 
 & tôt, als ir wol habet vernomen.\\ 
 & Gawans künne ist dicke komen\\ 
 & \textbf{von} minne in \textbf{herzebæri\textit{u}} sêr.\\ 
 & ich nenne iu sîner mâg\textit{e} mêr,\\ 
15 & den ouch von minne ist worden wê:\\ 
 & \textbf{wie} \textbf{bedwanc} der bluotvarwe snê\\ 
 & \textbf{des werden} Parcivals lîp?\\ 
 & \textbf{durch die} künegîn, sîn wîp.\\ 
 & Galoes unde Gahmureten,\\ 
20 & die habet ir bêde übertreten,\\ 
 & daz ir s\textit{i g}â\textit{bet} an den \textit{r}ê.\\ 
 & diu werde Itonie\\ 
 & \textbf{leit ouch} nâch \textbf{roys} Gramoflanz\\ 
 & mit \textbf{triuwen stæte} minne ganz;\\ 
25 & \begin{large}D\end{large}az was Gawans \textbf{swester clâr}.\\ 
 & vrô Minne, ir teilt \textbf{iuwer} vâr\\ 
 & \textbf{Sardomorde} \textbf{\textit{n}âch} Alexander.\\ 
 & die einen unde die ander,\\ 
 & swaz Gawan künnes ie gewan,\\ 
30 & vrô Minne, di\textit{e} welt ir \textbf{niht} erlân,\\ 
\end{tabular}
\scriptsize
\line(1,0){75} \newline
G I L M Fr19 Fr23 \newline
\line(1,0){75} \newline
\textbf{6} \textit{Initiale} L Fr19  \textbf{15} \textit{Initiale} I  \textbf{25} \textit{Initiale} G  \newline
\line(1,0){75} \newline
\textbf{1} dar] das M \textbf{2} ranc] reine G \textbf{4} Kanadic] kanadich G L ganadic M (Fr19) \textbf{5} kinde] chine G \textbf{7} ze Britanie] zebritannie G ze pritanie I Zuͯ Brittanie L Zcu britanie M  $\cdot$ er] ir M \textbf{8} Florie] Florîe G florien I Florine M Florine Fr19  $\cdot$ luot in] zuht in I in lvt L (Fr19) lute M  $\cdot$ last] mýnne last L (Fr19) libe last M \textbf{9} vür] in G \textbf{12} Gawans] Gawanes L \textbf{13} minne] minnen I libe M  $\cdot$ in herzebæriu] Inherzebæric G in hertzecliche L \textbf{14} iu] \textit{om.} M  $\cdot$ mâge] magin G \textbf{15} minne] libe M \textbf{16} Wie bedwanc] Wy twanc M \textbf{17} Parcivals] parziuals G parzifals I parzifalz L parzifals M parcifals Fr19 Fr23 \textbf{19} Gahmureten] Gamvrehten G Gahmuͯreten L Gamuͯreten M \textbf{20} bêde] \textit{om.} Fr23  $\cdot$ übertreten] getreten M Fr19 Fr23 \textbf{21} si gâbet] sighafte G  $\cdot$ rê] ie G \textbf{22} Itonie] tronie G Jtanie I \textbf{23} leit] lit Fr23  $\cdot$ Gramoflanz] Gramoflantz L gramorflanz M oramflanz Fr23 \textbf{24} minne] libe M  $\cdot$ ganz] glanz Fr23 \textbf{25} Daz] Dy M  $\cdot$ Gawans] Gawansz L \textbf{26} Minne] libe M  $\cdot$ iuwer vâr] ewerev iar I uwern var L \textbf{27} Sardomorde von vnde nah alexander G  $\cdot$ sarde morde noch alexander I  $\cdot$ Sordamvr nach Alexander L  $\cdot$ Sardomorde nach allexandir M  $\cdot$ Sardomor de nach alexander Fr19  $\cdot$ Sardoniot nah alexander Fr23 \textbf{28} einen] eine L (M) Fr19  $\cdot$ die ander] diu ander I \textbf{29} swaz] Waz L (M)  $\cdot$ Gawan] gawans M  $\cdot$ künnes] chunbers I kunne M \textbf{30} Minne] libe M  $\cdot$ die] diene G  $\cdot$ erlân] lan Fr23 \newline
\end{minipage}
\hspace{0.5cm}
\begin{minipage}[t]{0.5\linewidth}
\small
\begin{center}*T
\end{center}
\begin{tabular}{rl}
 & den iuwer kraft dar zuo \textbf{twanc},\\ 
 & daz der junge, süeze ranc\\ 
 & nâch werde\textit{r} âmîen,\\ 
 & von Kanadic Florien.\\ 
 & \hspace*{-.7em}\big| Diu selbe küniginne in zôch;\\ 
5 & \hspace*{-.7em}\big| sînes vater lant von kinde er vlôch;\\ 
 & zuo Britanie \textbf{was er} gast.\\ 
 & Florie \textbf{in luot} mit \textbf{minnen} last,\\ 
 & \textbf{daz} si in \textbf{jagte} vür daz lant.\\ 
10 & in ir dienst man in vant\\ 
 & t\textit{ô}t, als ir wol habt vernomen.\\ 
 & Gawans künne ist dicke komen\\ 
 & \textbf{von} minne in \textbf{herzebære} sêr.\\ 
 & ich nenne iu sîner mâge mêr,\\ 
15 & den ouch von minne ist worden wê:\\ 
 & \textbf{wie} \textbf{betwanc} der bluotvarwe snê\\ 
 & \textbf{des werden} Parcifals lîp?\\ 
 & \textbf{durch die} künigîn, sîn wîp.\\ 
 & Galoes und Gahmureten,\\ 
20 & die habt ir bêde übertreten,\\ 
 & daz ir si gâbet \textit{a}n den rê.\\ 
 & diu \textit{wer}de Itonie\\ 
 & \textbf{leit ouch} nâch \textbf{künic} Gramoflanz\\ 
 & mit \textbf{triuwen stæte} minne ganz;\\ 
25 & diu was Gawans \textbf{swester clâr}.\\ 
 & vrou Minne, ir teilt \textbf{iuwern} vâr\\ 
 & \textbf{Syrdamur} \textbf{durch} Alexander.\\ 
 & die eine und \textbf{ouch} die ander,\\ 
 & waz Gawan\textbf{s} künnes ie gewan,\\ 
30 & vrou Minne, die wolt ir \textbf{nie} erlân,\\ 
\end{tabular}
\scriptsize
\line(1,0){75} \newline
Q R W V U \newline
\line(1,0){75} \newline
\textbf{6} \textit{Initiale} Q   $\cdot$ \textit{Capitulumzeichen} R  \newline
\line(1,0){75} \newline
\textbf{1} \textit{Die Verse 553.1-599.30 fehlen} U   $\cdot$ dar zuo twanc] [*]: do zuͦ twang V \textbf{3} werder] werden Q \textbf{4} kanadick florien Q W  $\cdot$ kandick florizien R  $\cdot$ [*]: kanadig florien V \textbf{7} Britanie] britange Q Britanie er R britania W brittanie V \textbf{9} Daz [der *ge d* *ang*]: sv́ in veriagete fúrzlant V \textbf{11} tôt] Tet Q \textbf{12} Gawans] Gawins R \textbf{13} von] [*o*]: durch V  $\cdot$ herzebære] hetzebernde W \textbf{14} ich nenne] [J*ner]: Jch nenne V  $\cdot$ mâge] magen R \textbf{17} Parcifals] partzifals Q W parczifales R parzifales V \textbf{19} Galoes] [Galo*]: Galoes V  $\cdot$ Gahmureten] gamfreten Q Gamuretten R (V) gamureten W \textbf{20} bêde] beidu R \textbf{21} gâbet] gebt R  $\cdot$ an] in Q \textbf{22} werde] frewde Q [*]: iunge werde V  $\cdot$ Itonie] ytonie Q W Jtonie R [*]: ytonie V \textbf{23} [L*]:  Trvͦg noch roys gramolans V  $\cdot$ künic] rois R (W)  $\cdot$ Gramoflanz] gramoflansz Q Gamoflancz R gramoflantz W \textbf{24} triuwen] treúmen W \textbf{25} diu] Das R W (V)  $\cdot$ Gawans] Gawins R Gawanes V  $\cdot$ clâr] dar V \textbf{27} Syrdamur] Syrdamûr Q Surdamur R (W) (V)  $\cdot$ Alexander] allexander Q W Alexande R \textbf{28} Dú eine vnd och dú ander R \textbf{29} waz] Swaz V  $\cdot$ Gawans künnes] Gawin kumers R \textbf{30} nie] nit R (W) (V) \newline
\end{minipage}
\end{table}
\end{document}
