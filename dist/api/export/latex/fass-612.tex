\documentclass[8pt,a4paper,notitlepage]{article}
\usepackage{fullpage}
\usepackage{ulem}
\usepackage{xltxtra}
\usepackage{datetime}
\renewcommand{\dateseparator}{.}
\dmyyyydate
\usepackage{fancyhdr}
\usepackage{ifthen}
\pagestyle{fancy}
\fancyhf{}
\renewcommand{\headrulewidth}{0pt}
\fancyfoot[L]{\ifthenelse{\value{page}=1}{\today, \currenttime{} Uhr}{}}
\begin{document}
\begin{table}[ht]
\begin{minipage}[t]{0.5\linewidth}
\small
\begin{center}*D
\end{center}
\begin{tabular}{rl}
\textbf{612} & \textit{\begin{large}D\end{large}}ô sprach er: "vrouwe, ist \textbf{daz} wâr,\\ 
 & daz ir mich grüezet âne vâr,\\ 
 & sô \textbf{næhert} ir dem prîse.\\ 
 & ich bin doch wol sô wîse:\\ 
5 & ob der schilt sîn reht sol hân,\\ 
 & \textbf{an dem} hât ir missetân.\\ 
 & Des schildes ambet \textbf{ist} sô hôch,\\ 
 & daz \textbf{der von spotte ie sich gezôch},\\ 
 & swer rîterschaft ze rehte pflac.\\ 
10 & vrouwe, ob ich sô sprechen mac,\\ 
 & swer mich dâr bî hât gesehen,\\ 
 & der muoz mir rîterschefte jehen.\\ 
 & etswenne ir\textbf{s} anders jâhet,\\ 
 & sît ir mich êrst sâhet.\\ 
15 & Daz lâz ich sîn. nemt hin den kranz.\\ 
 & ir sult durch iwer varwe glanz\\ 
 & neheime rîter mêre\\ 
 & \textbf{erbieten} solh unêre.\\ 
 & \textbf{solt} iwer spot wesen mîn,\\ 
20 & ich wolt ê âne minne sîn."\\ 
 & Diu clâre unt diu rîche\\ 
 & sprach weinende herzenlîche:\\ 
 & "hêrre, \textbf{als} ich \textbf{iu} nôt \textbf{gesage},\\ 
 & waz ich \textbf{der} \textbf{ime} herzen trage,\\ 
25 & sô gebt ir jâmers mir gewin.\\ 
 & gein swem sich krenket mîn sin,\\ 
 & der solz durch zuht verkiesen.\\ 
 & i\textbf{ne} mac \textbf{niemêre} verliesen\\ 
 & vreuden, denne ich hân verlorn\\ 
30 & an Cidegaste, dem ûzerkorn.\\ 
\end{tabular}
\scriptsize
\line(1,0){75} \newline
D Z \newline
\line(1,0){75} \newline
\textbf{1} \textit{Initiale} D Z  \textbf{7} \textit{Majuskel} D  \textbf{15} \textit{Majuskel} D  \textbf{21} \textit{Majuskel} D  \newline
\line(1,0){75} \newline
\textbf{1} Dô] ÷o D \textbf{3} næhert] nahet Z \textbf{7} ist] was ie Z \textbf{8} Daz er spot sich da von zoch Z \textbf{13} irs] ir Z \textbf{14} êrst] von erste Z \textbf{16} sult] en svlt Z \textbf{24} der ime] in minem Z \textbf{25} jâmers] \textit{om.} Z \textbf{29} vreuden] Frevde Z \textbf{30} Cidegaste] Citegaste Z \newline
\end{minipage}
\hspace{0.5cm}
\begin{minipage}[t]{0.5\linewidth}
\small
\begin{center}*m
\end{center}
\begin{tabular}{rl}
 & dô sprach er: "vrouwe, ist \textbf{ez} wâr,\\ 
 & daz ir mich grüezet âne vâr,\\ 
 & sô \textbf{nâhet} ir dem prîse.\\ 
 & ich bin doch wol sô wîse:\\ 
5 & ob der schilt sîn reht sol hân,\\ 
 & \textbf{an dem} h\textit{abe}t ir missetân.\\ 
 & des schiltes ambet \textbf{ist} sô hôch,\\ 
 & daz \textbf{er von spotte ie sich gezôch},\\ 
 & wer ritterschaft zuo rehte pflac.\\ 
10 & vrowe, ob ich sô sprechen mac,\\ 
 & wer mich dâ bî het geseh\textit{e}n,\\ 
 & der muoz mir ritterschaft jehen.\\ 
 & etw\textit{enne} ir \textit{anders} jâhet,\\ 
 & sît ir mich \textit{ê}rs\textit{te} s\textit{â}h\textit{e}t.\\ 
15 & daz lâz ich sîn. nemt hin den kranz.\\ 
 & ir solt durch iuwer varwe glanz\\ 
 & dekeinem ritter mêre\\ 
 & \textbf{erbieten} solich unêr\textit{e}.\\ 
 & \textbf{solt} iuwer spot wesen mîn,\\ 
20 & ich wolt ê âne minne sîn."\\ 
 & \begin{large}D\end{large}iu clâre und diu rîche\\ 
 & sprach weinende herzelîche:\\ 
 & "hêrre, \textbf{waz} ich \textbf{iu} nôt \textbf{gesage},\\ 
 & waz ich \textbf{der} \textbf{in mînem} herzen trage,\\ 
25 & sô gebt ir jâmers \textit{mir} gewin.\\ 
 & gege\textit{n} \textit{w}em sich krenket mîn sin,\\ 
 & der sol ez durch zuht verkiesen.\\ 
 & ich mac \textbf{niemêr} verliesen\\ 
 & vröude, d\textit{a}n ich hân verlorn\\ 
30 & an Zidegast, dem ûzerkorn.\\ 
\end{tabular}
\scriptsize
\line(1,0){75} \newline
m n o \newline
\line(1,0){75} \newline
\textbf{21} \textit{Initiale} m   $\cdot$ \textit{Capitulumzeichen} n  \newline
\line(1,0){75} \newline
\textbf{1} ez] das n o \textbf{6} habet] houpt m \textbf{11} gesehen] gesehehen m \textbf{13} Ettwas ir iohent m \textbf{14} êrste sâhet] anders senhent m \textbf{17} dekeinem] Do keinen n Dekeinē o \textbf{18} unêre] vnerere m \textbf{20} wolt ê] woltte m \textbf{25} gebt] hebt o  $\cdot$ mir] \textit{om.} m \textbf{26} gegen wem] gegen mir wem m \textbf{29} dan] dar in m o dar jnne n \textbf{30} Zidegast] zitegast n \newline
\end{minipage}
\end{table}
\newpage
\begin{table}[ht]
\begin{minipage}[t]{0.5\linewidth}
\small
\begin{center}*G
\end{center}
\begin{tabular}{rl}
 & \begin{large}D\end{large}ô sprach er: "vrouwe, ist \textbf{daz} wâr,\\ 
 & daz ir mich grüezet âne vâr,\\ 
 & s\textit{ô} \textbf{\textit{n}âhet} ir dem prîse.\\ 
 & ich bin doch wol sô wîse:\\ 
5 & ob der schilt sîn reht sol hân,\\ 
 & \textbf{anders} habet ir missetân.\\ 
 & des schiltes ambet \textbf{was ie} sô hôch,\\ 
 & daz \textbf{der spot sich dâ von zôch},\\ 
 & swer rîterschaft ze rehte \textbf{ie} pflac.\\ 
10 & vrouwe, ob ich sô sprechen mac,\\ 
 & swer mich dâr bî hât gesehen,\\ 
 & der muoz mir rîterschefte jehen.\\ 
 & etswenne ir\textbf{s} anders jâhet,\\ 
 & sît ir mich \textbf{von} êrste sâhet.\\ 
15 & daz lâze ich sîn. nemet hin den kranz.\\ 
 & ir\textbf{n} sult durch iuwer varwe glanz\\ 
 & deheinem rîter mêre\\ 
 & \textbf{erbi\textit{e}ten} solhe unêre.\\ 
 & \textbf{sult} iuwer spot wesen mîn,\\ 
20 & ich wolde ê âne minne sîn."\\ 
 & diu clâre unt diu rîche\\ 
 & sprach weinende herzenlîche:\\ 
 & "hêrre, \textbf{als} ich nôt \textbf{geklage},\\ 
 & waz ich \textbf{in mînem} herzen trage,\\ 
25 & sô gebet ir jâmers mir gewin.\\ 
 & gein swem sich krenket mîn sin,\\ 
 & der sol ez durch zuht verkiesen.\\ 
 & ich \textbf{en}mac \textbf{niht mêre} verliesen\\ 
 & vröude, dan ich hân verlorn\\ 
30 & an Zidegast, dem ûzerkorn,\\ 
\end{tabular}
\scriptsize
\line(1,0){75} \newline
G I L M Z Fr34 Fr51 \newline
\line(1,0){75} \newline
\textbf{1} \textit{Initiale} G L Z Fr34 Fr51  \textbf{11} \textit{Initiale} I  \textbf{29} \textit{Initiale} I  \newline
\line(1,0){75} \newline
\textbf{1} Dô sprach er] Da sprach her M ÷O sprach er Fr34 Her sprach Fr51  $\cdot$ wâr] was Fr51 \textbf{3} \textit{Versfolge 612.4-3} M   $\cdot$ sô nâhet ir] Sahet ir G so nach ich Fr51  $\cdot$ dem] den Fr34 \textbf{5} der] \textit{om.} I  $\cdot$ sîn] \textit{om.} L  $\cdot$ hân] [syn]: han M \textbf{6} anders] An dem Z \textbf{7} ambet] amp I  $\cdot$ was ie] ist wol Fr34 \textbf{8} der spot sich] sich der spot L \textbf{9} swer] Wer L M Fr51  $\cdot$ ze rehte] \textit{om.} Fr34  $\cdot$ ie] \textit{om.} Z  $\cdot$ pflac] gepflach Fr34 \textbf{10} ich sô] ichz I  $\cdot$ sprechen] :::t Fr34 \textbf{11} swer] Wer L M Fr51  $\cdot$ mich] mir Fr51 \textbf{12} der] frowe der I \textbf{13} irs] ir I L M Fr51  $\cdot$ jâhet] iahn Fr51 \textbf{14} sâhet] sahn Fr51 \textbf{15} hin] \textit{om.} Fr51 \textbf{16} irn] Jr L  $\cdot$ sult] suln Fr51 \textbf{17} deheinem] Deheinen L (Fr51) \textbf{18} erbieten] Erbiten G bieten I Beiten Fr51  $\cdot$ unêre] ere L \textbf{20} ê] \textit{om.} I \textbf{23} nôt geklage] min not chlage I uͯch sage L v not gesage M (Z) (Fr51) \textbf{24} ich] ich ir I  $\cdot$ in mînem] an minen Fr51 \textbf{25} gebet ir] gede Fr51  $\cdot$ jâmers mir] mir iamers I (Fr51) mir Z \textbf{26} swem] wem L (M) Fr51  $\cdot$ sich] mich I  $\cdot$ mîn] \textit{om.} Fr51 \textbf{27} ez] ich Fr51  $\cdot$ zuht] zuhte G got I vch Fr51 \textbf{28} ich] Jchn Fr51  $\cdot$ enmac] mach I \textbf{29} vröude] Mere L \textbf{30} Zidegast] cydegast G Zitegastem I Citegaste L Z zcitegaste M cidegaste Fr34 sidegaste Fr51  $\cdot$ dem ûzerkorn] der waz erkorn L \newline
\end{minipage}
\hspace{0.5cm}
\begin{minipage}[t]{0.5\linewidth}
\small
\begin{center}*T
\end{center}
\begin{tabular}{rl}
 & \begin{large}D\end{large}ô sprach er: "vrouwe, ist \textbf{daz} wâr,\\ 
 & daz ir mich grüezet âne vâr,\\ 
 & sô \textbf{nâhe\textit{t}} ir dem prîse.\\ 
 & ich bin doch wol sô wîse:\\ 
5 & ob der schilt sîn reht sol hân,\\ 
 & \textbf{anders} hât ir missetân.\\ 
 & des schiltes ambet \textbf{was ie} sô hôch,\\ 
 & daz \textbf{sich der spot dâ von zôch},\\ 
 & wer rîterschaft zuo reht \textbf{ie} pflac.\\ 
10 & vrouwe, ob ich sô sprechen mac,\\ 
 & wer mich dâ bî hât gesehen,\\ 
 & der muoz mir rîterschefte jehen.\\ 
 & etswan ir anders jâhet,\\ 
 & sît ir mich êrst sâhet.\\ 
15 & daz lâz ich sîn. nemt hin den kranz.\\ 
 & ir solt durch iuwer varwe glanz\\ 
 & dekeine\textit{m} rîter mêre\\ 
 & \textbf{gebieten} solich unêre.\\ 
 & \textbf{sol} iuwer spot wesen mîn,\\ 
20 & ich wolte ê âne minne sîn."\\ 
 & diu clâre und diu rîche\\ 
 & sprach weinde herzenlîche:\\ 
 & "hêrre, \textbf{als} ich \textbf{iu} nôt \textbf{gesage},\\ 
 & waz ich \textbf{in mîme} herzen trage,\\ 
25 & sô gebet ir jâmers mir gewin.\\ 
 & gein wem sich krenket mîn sin,\\ 
 & der sol ez durch zuht verkiesen.\\ 
 & ich \textbf{en}mac \textbf{niht mê} verliesen\\ 
 & vreuden, dan ich hân verlorn\\ 
30 & an Cydegasten, dem ûzerkorn.\\ 
\end{tabular}
\scriptsize
\line(1,0){75} \newline
U V W Q R \newline
\line(1,0){75} \newline
\textbf{1} \textit{Initiale} U V W Q R  \textbf{23} \textit{Initiale} R  \newline
\line(1,0){75} \newline
\textbf{1} ist daz] das ist W \textbf{2} grüezet] so gruͯssen R  $\cdot$ vâr] geuar W \textbf{3} nâhet] naher U \textbf{6} anders] An dem V \textbf{8} sich] \textit{om.} V W Q  $\cdot$ dâ von] sich dannan V sich do von W (Q) \textbf{9} wer] Swer V \textbf{11} wer] Swer V \textbf{14} sît] Do W  $\cdot$ êrst] zuͦm ersten W \textbf{16} varwe] frawe Q frowen R \textbf{17} dekeinem] Dekeine U Deheinen Q \textbf{18} gebieten] Erbieten W (Q) Enbietten R \textbf{19} sol] Solt V W Q Soͯlt R \textbf{20} âne] an alle W \textbf{21} und] dar zuͦ W \textbf{23} iu] \textit{om.} R  $\cdot$ gesage] han gesagt R \textbf{24} Waz [*]: ich der in mime herzen trage V  $\cdot$ herzen] hertze Q  $\cdot$ trage] tragt R \textbf{26} wem] swem V  $\cdot$ sich krenket] krenket sich V ich krencket Q \textbf{27} sol ez] soltz V (Q) (R)  $\cdot$ zuht] got Q \textbf{28} ich enmac] [*]: Jch en mag V Jch mag Q R \textbf{30} Cydegasten] gydegaste V cytegast W Cydegaste Q kydasten R \newline
\end{minipage}
\end{table}
\end{document}
