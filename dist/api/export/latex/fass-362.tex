\documentclass[8pt,a4paper,notitlepage]{article}
\usepackage{fullpage}
\usepackage{ulem}
\usepackage{xltxtra}
\usepackage{datetime}
\renewcommand{\dateseparator}{.}
\dmyyyydate
\usepackage{fancyhdr}
\usepackage{ifthen}
\pagestyle{fancy}
\fancyhf{}
\renewcommand{\headrulewidth}{0pt}
\fancyfoot[L]{\ifthenelse{\value{page}=1}{\today, \currenttime{} Uhr}{}}
\begin{document}
\begin{table}[ht]
\begin{minipage}[t]{0.5\linewidth}
\small
\begin{center}*D
\end{center}
\begin{tabular}{rl}
\textbf{362} & \begin{large}I\end{large}ch sol \textbf{nû} selbe marschalc sîn.\\ 
 & liute unt guot, swaz heizet mîn,\\ 
 & daz kêr ich iu gein dienstes siten.\\ 
 & nie gast zuo wirte kom geriten,\\ 
5 & der im wære \textbf{als} undertân."\\ 
 & "\textbf{hêrre, iwer gnâde}", sprach Gawan,\\ 
 & "daz hân ich \textbf{ungedient} noch.\\ 
 & \textbf{ich} sol iu gerne volgen doch."\\ 
 & Scherules, der lobes gehêrte,\\ 
10 & sprach, als in \textbf{sîn} triwe lêrte:\\ 
 & "sît ez sich hât \textbf{an} mich gezogt,\\ 
 & ich bin vor vlüste \textbf{nû} iwer vogt,\\ 
 & ez ennem iu denne daz ûzer her.\\ 
 & dâ bin ich mit iu an der wer."\\ 
15 & \textbf{mit lachendem munde er} sprach\\ 
 & \textbf{hin} z\textbf{al den knappen}, die er dâ sach:\\ 
 & "ladet ûf \textbf{iwer} harnasch über al,\\ 
 & wir sulen hin nider inz tal."\\ 
 & Gawan vuor \textbf{mit sîme} wirt.\\ 
20 & Obie \textbf{nû daz} niht verbirt,\\ 
 & ein spilwîp si sande,\\ 
 & \textbf{die} ir vater wol \textbf{erkande},\\ 
 & \textbf{unt enbôt im} \textbf{solhiu} mære,\\ 
 & dâ vüere ein \textbf{valschære}.\\ 
25 & "des habe \textbf{ist} rîch unde guot.\\ 
 & bitte in durch \textbf{rehten rîters} muot,\\ 
 & sît er vil soldiere hât,\\ 
 & ûf ors, ûf silber unt ûf wât,\\ 
 & daz diz sî ir êrster gelt.\\ 
30 & \textbf{ez} vrümt wol siben ûfez velt."\\ 
\end{tabular}
\scriptsize
\line(1,0){75} \newline
D Fr4 \newline
\line(1,0){75} \newline
\textbf{1} \textit{Initiale} D  \newline
\line(1,0){75} \newline
\textbf{1} nû] ime Fr4 \textbf{3} dienstes] diens D \textbf{9} Scherules] Scervles D Tserules Fr4 \textbf{12} nû] \textit{om.} Fr4 \textbf{16} hin] \textit{om.} Fr4  $\cdot$ dâ] \textit{om.} Fr4 \textbf{19} sîme] sinen Fr4 \textbf{20} Obie] Obye D obẏe Fr4 \textbf{23} solhiu] sulhe Fr4 \textbf{28} ûf silber unt] vnde uf silber Fr4 \textbf{29} êrster] erste Fr4 \newline
\end{minipage}
\hspace{0.5cm}
\begin{minipage}[t]{0.5\linewidth}
\small
\begin{center}*m
\end{center}
\begin{tabular}{rl}
 & ich sol \textbf{nû} selber marschalc sîn.\\ 
 & liute und guot, waz heizet mîn,\\ 
 & daz kêr ich iu gegen dienstes siten.\\ 
 & \textit{ni}e gast ze wirte kam geriten,\\ 
5 & der ime wære \textbf{als} undertân."\\ 
 & "\textbf{hêrre, iuwer gnâde}", sprach Gawan,\\ 
 & "daz hân ich \textbf{ungedienet} noch.\\ 
 & \textbf{ich} sol iu gerne volgen doch."\\ 
 & Scher\textit{u}l\textit{e}s, der lobes ge\textit{hê}rte,\\ 
10 & sprach, als in \textbf{sîn} \textit{triuwe} lêrte:\\ 
 & "sît ez sich hât \textbf{an} mich gezoget,\\ 
 & ich bin vor vlü\textit{s}te iuwer voget,\\ 
 & ez e\textit{n}nem iu danne daz ûzer her.\\ 
 & dâ bin ich mit iu an der wer."\\ 
15 & \textbf{mit lach\textit{e}ndem munde er} sprach\\ 
 & \textbf{hin} zuo \textbf{de\textit{n} knappen}, die er dâ sach:\\ 
 & "ladet ûf \textbf{iuwer} harnasch über al,\\ 
 & wir sullen hin nider in daz tal."\\ 
 & Gawan vuor \textbf{mit sînem} wirt.\\ 
20 & Obie \textbf{nû daz} niht verbirt,\\ 
 & ein spilwîp si sante,\\ 
 & \textbf{die} ir vater wol \textbf{erkante},\\ 
 & \textbf{und enbôt ime} \textbf{bî ir} mære,\\ 
 & dâ vüer ein \textbf{valschære},\\ 
25 & des habe \textbf{wær} rîch und guot.\\ 
 & "bitte in durch \textbf{ritterlîchen} muot,\\ 
 & sît er vil soldiere hât,\\ 
 & ûf ros, ûf silber und ûf wât,\\ 
 & daz diz sî ir êrstez gelt.\\ 
30 & \textbf{e\textit{z}} vrümt wol sibene ûf daz velt."\\ 
\end{tabular}
\scriptsize
\line(1,0){75} \newline
m n o \newline
\line(1,0){75} \newline
\newline
\line(1,0){75} \newline
\textbf{1} selber] [selbe]: selber m \textbf{4} nie] Me m \textbf{7} ungedienet] vngienet o \textbf{9} Scherules] Scerelus m Sterules n Stenles o  $\cdot$ gehêrte] gerte m \textbf{10} triuwe] \textit{om.} m \textbf{11} sît ez sich hât] Sich es het sich n Sit es het o  $\cdot$ gezoget] gezeigt n gezahet o \textbf{12} vlüste] fluhtte m verflust n \textbf{13} ez ennem iu] Es einem uͯch m Vnd keine ouch n (o)  $\cdot$ danne] >dan< o \textbf{14} dâ] Do n o \textbf{15} lachendem] lachandem m \textbf{16} den] der m  $\cdot$ dâ] do n o \textbf{18} nider] wider o \textbf{20} Obie] Obẏe n \textbf{22} ir] iren n (o) \textbf{23} ir] der o \textbf{24} dâ] Do n o \textbf{30} ez] Er m \newline
\end{minipage}
\end{table}
\newpage
\begin{table}[ht]
\begin{minipage}[t]{0.5\linewidth}
\small
\begin{center}*G
\end{center}
\begin{tabular}{rl}
 & ich sol \textbf{iu} selbe marschalc sîn.\\ 
 & liute unde guot, swaz heizet mîn,\\ 
 & daz kêre ich iu gein dienstes siten.\\ 
 & nie gast zuo wirte kom geriten,\\ 
5 & der im \textbf{sô gar} wære undertân."\\ 
 & "\textbf{genâde}", sprach \textbf{hêr} Gawan,\\ 
 & "daz hân ich \textbf{unverdient} noch\\ 
 & \textbf{unde} sol iu gerne volgen doch."\\ 
 & Tscherules, der lobes gehêrte,\\ 
10 & sprach, als in triwe lêrte:\\ 
 & "\begin{large}S\end{large}ît ez sich hât \textbf{ûf} mich gezoget,\\ 
 & ich bin vor vlust \textbf{nû} iwer voget,\\ 
 & ez ennem iu dane daz ûzer her.\\ 
 & dâ bin ich \textit{m}i\textit{t} iu an der wer."\\ 
15 & \textbf{sîn munt dô lachende} sprach\\ 
 & z\textbf{en knappen allen}, die er dâ sach:\\ 
 & "ladet ûf \textbf{daz} harnasch über al,\\ 
 & wir sulen hin nider inz tal."\\ 
 & Gawan vuor \textbf{mit sîne\textit{m}} wirt.\\ 
20 & Obie \textbf{daz nû} niht verbirt,\\ 
 & ein spilwîp si sande,\\ 
 & \textbf{die} ir vater wol \textbf{bekande}.\\ 
 & \textbf{dem enbôt \textit{si}} \textbf{solhiu} mære,\\ 
 & dâ vüere ein \textbf{va\textit{l}schære}.\\ 
25 & "des habe \textbf{ist} rîche unde guot.\\ 
 & bit in durch \textbf{rehten rîters} muot,\\ 
 & sît er vil soldiere hât,\\ 
 & ûf ors, ûf silber unde ûf wât,\\ 
 & daz diz sî ir êrste gelt.\\ 
30 & \textbf{ez} vrümet wol sibene ûf daz velt."\\ 
\end{tabular}
\scriptsize
\line(1,0){75} \newline
G I O L M Q R Z Fr38 \newline
\line(1,0){75} \newline
\textbf{1} \textit{Initiale} I O L Z   $\cdot$ \textit{Capitulumzeichen} R  \textbf{11} \textit{Initiale} G  \textbf{15} \textit{Initiale} I  \textbf{19} \textit{Initiale} M  \newline
\line(1,0){75} \newline
\textbf{1} ich] ÷ch O  $\cdot$ iu selbe] ewer selbe I selbe iwer O uch selben M nu selb Z \textbf{2} liute unde guot] guͤt vnd lute vnd I Luͯte guͯt vnd L Lu͑te vnd gutes Q Lútte vnd guͦt vnd R  $\cdot$ swaz] waz L (M) (Q) (R)  $\cdot$ heizet] ist I \textbf{3} gein] zv Z  $\cdot$ dienstes] dienste Z (Fr38) \textbf{4} zuo wirte] so wirdig R \textbf{5} der] Deme M  $\cdot$ sô gar wære] were so gar L \textbf{6} genâde sprach hêr] Jwer gnade herre sprach O L (M) (Q) (R) (Z) (Fr38)  $\cdot$ Gawan] her Gawan R \textbf{7} unverdient] vngedienet L \textbf{8} unde] Jch Z  $\cdot$ iu gerne] gerne iv O úch R \textbf{9} Tscherules] Scrules I Tschervles O Fr38 Tshervles L Scherulus M Schirulus R  $\cdot$ gehêrte] gehere R \textbf{10} triwe] sin ellen I \textbf{11} sich] \textit{om.} R  $\cdot$ ûf] an O L M Q R Z Fr38  $\cdot$ gezoget] gezogen Q (R) (Fr38) \textbf{12} nû] noch O (Z) o\textit{m. } M R \textbf{13} ennem] nêm O (R) (Z)  $\cdot$ dane] \textit{om.} I  $\cdot$ ûzer] vber I \textbf{14} dâ] Do Q  $\cdot$ mit] bi G \textbf{15} dô] da M Z  $\cdot$ lachende] lachen R \textbf{16} zen knappen allen] zuͤ den cnappen I Ze allen chnappen O (M) (Z) Zuͯ al den knappen L (Q) (R) (Fr38)  $\cdot$ dâ] do O (Q) \textbf{17} ûf] \textit{om.} L  $\cdot$ daz] den R \textbf{18} nider] wider L Q  $\cdot$ inz] vff das Q \textbf{19} sînem] sinen G sinnē Q (Fr38) \textbf{20} Obie] Obye O R Z Fr38  $\cdot$ daz nû] nun das Q \textbf{22} die] diu I  $\cdot$ bekande] er chande O (L) (Fr38) \textbf{23} si] er G  $\cdot$ solhiu] solich R \textbf{24} dâ] Do Q  $\cdot$ valschære] vaschare G \textbf{26} rehten] rehtes I (M) (R)  $\cdot$ rîters] ritter Q \textbf{27} soldiere] soldierer I soldener Q solner R \textbf{28} Vf silber vf ros vnd vf wat L  $\cdot$ ors] ors vnd Fr38 \textbf{29} ir êrste] erstez I ir erster O ir erstes Q sin erste R erste Z \newline
\end{minipage}
\hspace{0.5cm}
\begin{minipage}[t]{0.5\linewidth}
\small
\begin{center}*T
\end{center}
\begin{tabular}{rl}
 & ich sol selbe \textbf{iuwer} marschalc sîn.\\ 
 & liute unde guot, swaz heizet mîn,\\ 
 & daz kêre ich iu gegen dienstes siten.\\ 
 & nie gast ze wirte kom geriten,\\ 
5 & der im \textbf{sô gar} wære undertân."\\ 
 & "\textbf{Iuwer gnâde, hêrre}", sprach Gawan,\\ 
 & "daz hân ich \textbf{ungedienet} noch\\ 
 & \textbf{unde} sol iu gerne volgen doch."\\ 
 & Tscherules, der lobes gehêrte,\\ 
10 & sprach, als in \textbf{sîn} triuwe lêrte:\\ 
 & "sît ez sich hât \textbf{an} mich gezoget,\\ 
 & ich bin vor verlüste iuwer voget,\\ 
 & ez ennem iu danne daz ûzer her.\\ 
 & dâ bin ich mit iu an der wer."\\ 
15 & \textbf{Sîn munt dô lachende} sprach\\ 
 & zuo \textbf{den knappen}, die er dâ sach:\\ 
 & "ladet ûf \textbf{daz} harnasch über al,\\ 
 & wir suln hin nider in daz tal."\\ 
 & \begin{large}G\end{large}awan vuor \textbf{unde sîn} wirt.\\ 
20 & Obye \textbf{nû daz} niht verbirt,\\ 
 & ein spilwîp si sante,\\ 
 & \textbf{diu} ir vater wol \textbf{bekante}.\\ 
 & \textbf{dem enbôt si} \textbf{sölhe} mære,\\ 
 & dâ vüere ein \textbf{triegære}.\\ 
25 & "des habe \textbf{ist} rîche unde guot.\\ 
 & bit in durch \textbf{rehten rîters} muot,\\ 
 & sît er vil soldiere hât,\\ 
 & ûf ors, ûf silber unde ûf wât,\\ 
 & daz diz sî ir êrste gelt.\\ 
30 & \textbf{daz} vr\textit{üm}t wol sibene ûf daz velt."\\ 
\end{tabular}
\scriptsize
\line(1,0){75} \newline
T V W \newline
\line(1,0){75} \newline
\textbf{6} \textit{Majuskel} T  \textbf{15} \textit{Majuskel} T  \textbf{19} \textit{Initiale} T  \newline
\line(1,0){75} \newline
\textbf{1} selbe iuwer] eúch selben W \textbf{2} swaz] vnd was W \textbf{3} dienstes] dienst W \textbf{4} kom] \textit{om.} W \textbf{6} hêrre sprach] sprach her W \textbf{8} unde] Jch V  $\cdot$ doch] ioch W \textbf{9} Tscherules] Tscerules T Schervles V De scherules W  $\cdot$ gehêrte] geherten W \textbf{11} hât] \textit{om.} W \textbf{14} dâ] Do V W \textbf{15} lachende] lachte vnd W \textbf{16} den] allen W  $\cdot$ dâ] do V W \textbf{17} daz] den W \textbf{19} unde sîn] mit sinem V (W) \textbf{20} Obye] Obẏe V Obyen W  $\cdot$ nû daz] dez nv V auch do W \textbf{21} ein] Iesa ein W  $\cdot$ si] sy dar W \textbf{23} Vnde enbot im bi ir mere V \textbf{24} dâ vüere] Do fvͤre V Do var W \textbf{25} ist] were V sy W \textbf{26} rehten rîters] ritterlichen V rechtes ritters W \textbf{29} êrste] [er*e*]: erstez V \textbf{30} vrümt] vrivnt T \newline
\end{minipage}
\end{table}
\end{document}
