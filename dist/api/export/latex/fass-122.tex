\documentclass[8pt,a4paper,notitlepage]{article}
\usepackage{fullpage}
\usepackage{ulem}
\usepackage{xltxtra}
\usepackage{datetime}
\renewcommand{\dateseparator}{.}
\dmyyyydate
\usepackage{fancyhdr}
\usepackage{ifthen}
\pagestyle{fancy}
\fancyhf{}
\renewcommand{\headrulewidth}{0pt}
\fancyfoot[L]{\ifthenelse{\value{page}=1}{\today, \currenttime{} Uhr}{}}
\begin{document}
\begin{table}[ht]
\begin{minipage}[t]{0.5\linewidth}
\small
\begin{center}*D
\end{center}
\begin{tabular}{rl}
\textbf{122} & er \textbf{en}hete sô \textbf{liehtes} niht erkant.\\ 
 & ûfem touwe der wâpenroc erwant.\\ 
 & mit guldînen schellen kleine\\ 
 & vor \textbf{iewederm} beine\\ 
5 & wâren die stegreife erklenget\\ 
 & unt ze rehter mâze erlenget.\\ 
 & sîn \textbf{zeswer} arm von schellen klanc,\\ 
 & swar er \textbf{den} bôt oder swanc.\\ 
 & der was durch swertslege sô hel.\\ 
10 & der helt was \textbf{gein} prîse snel.\\ 
 & sus vuor der vürste rîche\\ 
 & gezimieret wünneclîche.\\ 
 & \begin{large}A\end{large}ller manne schœne ein \textbf{bluomen} kranz,\\ 
 & den vrâgete Karnahkarnanz:\\ 
15 & "Junchêrre, sâhet ir vür iuch varn\\ 
 & zwêne ritter, die sich niht bewarn\\ 
 & kunnen an ritterlîcher zunft?\\ 
 & si ringent mit der nôtnunft\\ 
 & unt sint an werdecheit verzagt.\\ 
20 & si \textbf{vüerent roubes} eine magt."\\ 
 & Der knappe wânde, swaz er sprach,\\ 
 & \textbf{ez} wære got, als im verjach\\ 
 & vrou Herzeloyde, diu künegîn,\\ 
 & dô si im underschiet den liehten schîn.\\ 
25 & dô rief er lûte sunder spot:\\ 
 & "\textbf{nû} hilf mir, hilferîcher got!"\\ 
 & vil dicke viel an sîn gebet\\ 
 & fillii roy Gahmuret.\\ 
 & der vürste sprach: "ich bin niht got,\\ 
30 & ich leiste aber gerne sîn gebot.\\ 
\end{tabular}
\scriptsize
\line(1,0){75} \newline
D \newline
\line(1,0){75} \newline
\textbf{13} \textit{Initiale} D  \textbf{15} \textit{Majuskel} D  \textbf{21} \textit{Majuskel} D  \newline
\line(1,0){75} \newline
\textbf{24} liehten] lihten D \textbf{28} Gahmuret] Gahmvret D \newline
\end{minipage}
\hspace{0.5cm}
\begin{minipage}[t]{0.5\linewidth}
\small
\begin{center}*m
\end{center}
\begin{tabular}{rl}
 & er \textbf{en}hete \textbf{ê} sô \textbf{liehte} niht erkant.\\ 
 & ûfem touwe der wâpenroc erwant.\\ 
 & mit guldînen schellen kleine\\ 
 & vor \textbf{ietwederm} bei\textit{n}e\\ 
5 & wâren die stegereife erklenget\\ 
 & und ze rehter mâze e\textit{r}lenget.\\ 
 & sîn \textbf{zeswer} arm von schellen klanc,\\ 
 & war er \textbf{den} bôt oder swanc.\\ 
 & der was durch swertslege sô hel.\\ 
10 & der helt was \textbf{gegen} prîse snel.\\ 
 & sus vuor der vürst\textit{e} rîch\\ 
 & gezimieret wünneclîch.\\ 
 & \begin{large}A\end{large}ller manne schœne ein \textbf{bluomen} kranz,\\ 
 & den vrâgete Karnachkarnanz:\\ 
15 & "junchêrr\textit{e}, s\textit{â}het ir vür iuch varn\\ 
 & zwêne ritter, die sich niht bewarn\\ 
 & k\textit{önn}en an ritterlîcher zunft?\\ 
 & si ri\textit{n}gent mit der nôtnunft\\ 
 & und sint an wirdicheit verzaget.\\ 
20 & si \textbf{vüeren\textit{t} roubes} \textit{ei}ne maget."\\ 
 & der knappe wânde, waz er sprach,\\ 
 & \textbf{er} wære g\textit{o}t, als ime verjach\\ 
 & vrouwe Herczeloide, diu künigîn,\\ 
 & dô si im undersch\textit{ie}t den liehten schîn.\\ 
25 & dô r\textit{i}ef er lûte sunder sp\textit{o}t:\\ 
 & "\textbf{nû} h\textit{i}lf mir, helferîcher got!"\\ 
 & vil dicke viel \textit{an} sîn gebet\\ 
 & fili rois Gahmuret.\\ 
 & der vürste sprach: "i\textbf{ne} bin niht got,\\ 
30 & ich leiste aber gerne sîn gebot.\\ 
\end{tabular}
\scriptsize
\line(1,0){75} \newline
m n o \newline
\line(1,0){75} \newline
\textbf{13} \textit{Initiale} m n o  \newline
\line(1,0){75} \newline
\textbf{1} enhete] hette n (o) \textbf{2} ûfem] Vff ein n o  $\cdot$ touwe] tauͯge o \textbf{4} ietwederm] ietwiedern o  $\cdot$ beine] beide m \textbf{6} erlenget] erclenget m \textbf{7} zeswer] zwoswor o \textbf{8} bôt] bat o \textbf{9} swertslege] swert slegen n swert slegel o \textbf{11} sus] Suͯs sus o  $\cdot$ vürste] fursten m \textbf{12} gezimieret] Gezẏmmert o \textbf{14} Karnachkarnanz] [karnach ka*]: karnach karnancz m karnoch karnantz n karnach karnancz o \textbf{15} junchêrre] Jungherren m  $\cdot$ sâhet] sehent m (o) \textbf{17} können] Kommen m (o) \textbf{18} ringent] rigent m \textbf{20} vüerent] furen m fuͦrent o  $\cdot$ eine] one m \textbf{22} got] gat m guͦt o \textbf{23} Herczeloide] hertzeloide n herczeleid o \textbf{24} underschiet] vnder scheid m \textbf{25} rief] reff m ruͦff o  $\cdot$ spot] spat m \textbf{26} hilf] helf m \textbf{27} an] \textit{om.} m \textbf{28} Gahmuret] gamiret n gamúret o \textbf{29} ine] ich n o \newline
\end{minipage}
\end{table}
\newpage
\begin{table}[ht]
\begin{minipage}[t]{0.5\linewidth}
\small
\begin{center}*G
\end{center}
\begin{tabular}{rl}
 & er hete sô \textbf{liehtes} niht erkant.\\ 
 & ûf dem touwe der wâpenroc erwant.\\ 
 & mit guldînen schellen kleine\\ 
 & vor \textbf{ietwederm} beine\\ 
5 & wâren die stegreife erklenget\\ 
 & unt ze rehter mâze erlenget.\\ 
 & sîn \textbf{zeswer} arm von schellen klanc,\\ 
 & swar er \textbf{in} bôt oder swanc.\\ 
 & der was durch swertslege sô hel.\\ 
10 & der helt was \textbf{gein} brîse snel.\\ 
 & sus vuor der vürste rîche\\ 
 & gezimiert wünniclîche.\\ 
 & \begin{large}A\end{large}ller manne schœne ein \textbf{bluomen} kranz,\\ 
 & den vrâgte Karnakarnanz:\\ 
15 & "ju\textit{n}chêrre, sâhet ir vür iuch varen\\ 
 & zwêne rîter, die sich niht bewaren\\ 
 & kunnen an rîterlicher zunft?\\ 
 & si ringent mit der nôtnunft\\ 
 & unde sint an werdicheit verzaget.\\ 
20 & si \textbf{vüerent roubes} eine maget."\\ 
 & der k\textit{n}appe wânde, swaz er sprach,\\ 
 & \textbf{er} wære \textbf{ein} got, als im verjach\\ 
 & vrô Herzeloide, diu künigîn,\\ 
 & dô sim underschiet den liehten schîn.\\ 
25 & dô rief er lûte sunder spot:\\ 
 & "\textbf{nû} hilf mir, helferîcher got!"\\ 
 & vil dicke viel an sîn gebet\\ 
 & filiroys Gahmuret.\\ 
 & der vürste sprach: "ich bin niht got,\\ 
30 & ich leiste aber gerne sîn gebot.\\ 
\end{tabular}
\scriptsize
\line(1,0){75} \newline
G I O L M Q R Z Fr36 \newline
\line(1,0){75} \newline
\textbf{1} \textit{Initiale} O  \textbf{7} \textit{Initiale} I  \textbf{13} \textit{Initiale} G I L R Z  \textbf{29} \textit{Initiale} I  \newline
\line(1,0){75} \newline
\textbf{1} er] Ern I (L) (M) Q (R) Z ÷r O  $\cdot$ sô liehtes] so lichtes O L M Q sin liecht R \textbf{2} dem] ein Q  $\cdot$ der] den O sin L dy M  $\cdot$ erwant] er vant O [want]: erwant L \textbf{3} guldînen] gulin Q \textbf{4} ietwederm] icwedirn M (Q) \textbf{5} wâren] Waz L  $\cdot$ die] im die I sin L \textbf{7} zeswer] \textit{om.} Q Rechtter R  $\cdot$ klanc] blanck Q \textbf{8} swar] War L M (Q) R  $\cdot$ in] [der]: den L den Z \textbf{9} swertslege] swertes slege I O Q Z [swert*]: swertslegin M \textbf{10} helt] helin Q \textbf{11} sus] Ausz Q  $\cdot$ vürste] mvͦtes O \textbf{13} Aller] ÷ller I Alle R  $\cdot$ schœne] \textit{om.} L  $\cdot$ bluomen] bluͤm I (L) \textbf{14} den] Der O  $\cdot$ vrâgte] vragt I (Z)  $\cdot$ Karnakarnanz] [karnachantanc]: karnachgantanc I karnah karnaz L karnacarnanz M karnahkarnatz Q karnahkarnancz R karnahkarnantz Z \textbf{15} junchêrre] ivcherre G  $\cdot$ sâhet] sagt Q  $\cdot$ ir] ich M \textbf{17} zunft] zuht I (R) \textbf{18} ringent] ringet Q  $\cdot$ nôtnunft] notdurfft Q \textbf{19} sint] sein Q \textbf{20} roubes] ravbet O \textbf{21} knappe] chappe G  $\cdot$ swaz] waz L (M) (Q) (R) Z \textbf{22} er] Ez O (Q)  $\cdot$ ein] \textit{om.} L \textbf{23} Herzeloide] herzenlaude I herzenlavde O Hertzelauͯde L herczeloide M herzeloúde Q herczelaude R herzelovde Z :::de Fr36 \textbf{24} dô] Da M Z  $\cdot$ sim] si ein M  $\cdot$ liehten] lichten L M (Q) \textbf{25} dô] Da M Z  $\cdot$ er] \textit{om.} I O \textbf{26} nû] \textit{om.} I Fr36  $\cdot$ hilf] helf L (Q)  $\cdot$ helferîcher] helflicher I \textbf{27} viel] viel er I O (L) (M) (Q) (Z) \textbf{28} Gahmuret] Gahmvret G Fr36 Gamvret O Gahmuͯret L gamuͯret M gamuret Q Z \textbf{29} vürste] \textit{om.} I  $\cdot$ ich bin] ich en bin M (Fr36) \newline
\end{minipage}
\hspace{0.5cm}
\begin{minipage}[t]{0.5\linewidth}
\small
\begin{center}*T (U)
\end{center}
\begin{tabular}{rl}
 & er\textbf{n} hete sô \textbf{liehtes} niht erkant.\\ 
 & ûf dem touwe der wâpenroc erwant.\\ 
 & mit guldînen schellen kleine\\ 
 & vor \textbf{ieclîchem} beine\\ 
5 & wâren die stegreife erklenget\\ 
 & und zuo rehter mâz erlenget.\\ 
 & sîn \textbf{rehter} arm von schellen klanc,\\ 
 & war er \textbf{in} bôt oder swanc.\\ 
 & der was durch swertes slege sô hel.\\ 
10 & der helt was \textbf{von} prîse snel.\\ 
 & sus vuor der vürste rîche\\ 
 & gezimieret wünneclîche.\\ 
 & aller manne schœne ein \textbf{bluomender} kranz,\\ 
 & den vrâgete Garnagarnanz:\\ 
15 & "junchêrre, sâhet ir vor iuch varn\\ 
 & zwêne ritter, die sich niht bewarn\\ 
 & kunnen an ritterlîcher zunft?\\ 
 & si ringent mit der nôtnunft\\ 
 & und sint an wirdecheit verzaget.\\ 
20 & si \textbf{vuoren rouben} eine maget."\\ 
 & der knappe wânte, waz er sprach,\\ 
 & \textbf{er} wære \textbf{ein} got, als im verjach\\ 
 & vrou Herzeloyde, diu künegîn,\\ 
 & dô si im underschiet den liehten schîn.\\ 
25 & dô rief er lûte sunder spot:\\ 
 & "hilf mir, helferîcher got!"\\ 
 & vil dicke viel an sîn gebet\\ 
 & filli roys Gahmuret.\\ 
 & \begin{large}D\end{large}er vürste sprach: "ich \textbf{en}bin niht got,\\ 
30 & ich leiste aber gerne sîn gebot.\\ 
\end{tabular}
\scriptsize
\line(1,0){75} \newline
U V W T \newline
\line(1,0){75} \newline
\textbf{7} \textit{Majuskel} T  \textbf{13} \textit{Initiale} W T  \textbf{21} \textit{Majuskel} T  \textbf{25} \textit{Majuskel} T  \textbf{29} \textit{Initiale} U V   $\cdot$ \textit{Majuskel} T  \newline
\line(1,0){75} \newline
\textbf{1} erkant] bekant V \textbf{2} der] [*]: sin V \textbf{4} ieclîchem] iewederme V (W) (T)  $\cdot$ beine] beide T \textbf{5} wâren die stegreife] [*]: waz sin stegereif V \textbf{6} zuo] zer T  $\cdot$ erlenget] geenget W \textbf{7} rehter] zesewer V (W) (T) \textbf{8} war] Swar V (T) \textbf{9} swertes slege] swert sloge T \textbf{10} von] [*]: gegen V gegn T \textbf{13} bluomender] blvͦmen V (W) (T) \textbf{14} den vrâgete] Der fraget W  $\cdot$ Garnagarnanz] garnagarnantz V gurnagarnantz W \textbf{17} zunft] [*]: kvnst V kunst W \textbf{18} ringent] ringen W  $\cdot$ der] ritterlicher W \textbf{19} verzaget] veriaget W \textbf{20} vuoren rouben] fvͤrent roͮbes V fuͦrent raubens W vuerent [*]: rovbes T \textbf{21} waz] swaz T \textbf{23} Herzeloyde] herzeleide U herzelaude V hertzeloyde W \textbf{26} hilf] nv hilf T  $\cdot$ helferîcher] herre richer V \textbf{27} viel] viel er V W \textbf{28} filli roys] Dez kv́niges svn V  $\cdot$ Gahmuret] Gahmuͦret U gamvret V gamuret W Gahmvret T \textbf{29} vürste] riter T  $\cdot$ enbin] bin V W \newline
\end{minipage}
\end{table}
\end{document}
