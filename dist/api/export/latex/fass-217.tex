\documentclass[8pt,a4paper,notitlepage]{article}
\usepackage{fullpage}
\usepackage{ulem}
\usepackage{xltxtra}
\usepackage{datetime}
\renewcommand{\dateseparator}{.}
\dmyyyydate
\usepackage{fancyhdr}
\usepackage{ifthen}
\pagestyle{fancy}
\fancyhf{}
\renewcommand{\headrulewidth}{0pt}
\fancyfoot[L]{\ifthenelse{\value{page}=1}{\today, \currenttime{} Uhr}{}}
\begin{document}
\begin{table}[ht]
\begin{minipage}[t]{0.5\linewidth}
\small
\begin{center}*D
\end{center}
\begin{tabular}{rl}
\textbf{217} & etslîcher hin zir spræche,\\ 
 & daz in ir minne \textbf{stæche}\\ 
 & unt \textbf{im die} \textbf{vreude} blante.\\ 
 & ob si die nôt \textbf{erwante},\\ 
5 & daz \textbf{dient} er vor und nâch.\\ 
 & mir wære \textbf{ê} mit ir dannen gâch.\\ 
 & Ich hân gereit umbe mîn dinc.\\ 
 & nû hœret, wie Artuses rinc\\ 
 & \textbf{sunder} was erkenneclîch\\ 
10 & vor ûz mit maneger joie rîch.\\ 
 & diu messenîe vor im \textbf{az}:\\ 
 & manec werder man gein valsche laz\\ 
 & unt manec \textbf{juncvrouwe} stolz,\\ 
 & daz niht \textbf{wan} tjoste was \textbf{ir} bolz.\\ 
15 & ir vriwent si gein \textbf{dem vîent} schôz.\\ 
 & \textbf{lêrt in strît dâ kumber} grôz,\\ 
 & \textbf{sus} stuont lîhte ir gemüete,\\ 
 & daz \textbf{siz} galt mit güete.\\ 
 & Clamide, der jungelinc,\\ 
20 & reit mitten in den rinc.\\ 
 & verdecket ors, gewâpent lîp\\ 
 & \textbf{sach} an im Artuses wîp,\\ 
 & \textbf{sînen} helm, \textbf{sînen} schilt verhouwen.\\ 
 & daz sâhen gar die vrouwen.\\ 
25 & \textbf{\begin{large}S\end{large}us} was er ze hove komen.\\ 
 & ir habt ê wol vernomen,\\ 
 & \textbf{daz} er des wart betwungen.\\ 
 & er erbeizte. vil gedrungen\\ 
 & wart sîn lîp, ê er \textbf{sitzen} vant\\ 
30 & vroun Cunnewaren de Lalant.\\ 
\end{tabular}
\scriptsize
\line(1,0){75} \newline
D \newline
\line(1,0){75} \newline
\textbf{7} \textit{Majuskel} D  \textbf{25} \textit{Initiale} D  \newline
\line(1,0){75} \newline
\textbf{8} Artuses] Artvss D \textbf{19} Clamide] Chlamide D \textbf{22} Artuses] Artvss D \newline
\end{minipage}
\hspace{0.5cm}
\begin{minipage}[t]{0.5\linewidth}
\small
\begin{center}*m
\end{center}
\begin{tabular}{rl}
 & etslîcher hin zuo ir spræche,\\ 
 & daz in ir minne \textbf{erst\textit{æ}che}\\ 
 & und \textbf{in die} \textbf{vröude} blante.\\ 
 & ob si die nôt \textbf{erwante},\\ 
5 & daz \textbf{dienet} er vor und nâch.\\ 
 & mir wær \textbf{ê} mit ir dannen gâch.\\ 
 & \begin{large}I\end{large}ch hân gereit umb mîn dinc.\\ 
 & nû hœret, wie Artuses rinc\\ 
 & \textbf{besunder} was erkenneclîch\\ 
10 & vor ûz mit maniger joie rîch.\\ 
 & diu massenîe vor ime \textbf{was}:\\ 
 & manic wert man gegen valsche laz\\ 
 & und manige \textbf{juncvrouwe} stolz,\\ 
 & daz niht \textbf{wanne} juste was \textbf{ir} \textit{b}olz.\\ 
15 & ir vriunt si gegen \textbf{dem vîende} schôz.\\ 
 & \textbf{lêrte in strît dâ kumber} grôz,\\ 
 & \textbf{sô} stuont lîhte ir gemüete,\\ 
 & daz \textbf{si ez} \textbf{ime} galt mit güete.\\ 
 & Clamide, der jungelinc,\\ 
20 & reit mitten in den rinc.\\ 
 & verdecket ros, gewâpent lîp\\ 
 & \textbf{sach} an ime Artuses wîp,\\ 
 & \textbf{sînen} helm, \textbf{sînen} schilt verhouwen.\\ 
 & daz sâhen gar die vrouwen.\\ 
25 & \textbf{sus} w\textit{a}s er ze hove komen.\\ 
 & ir habet ê wol vernomen,\\ 
 & \textbf{daz} er des wart betwungen.\\ 
 & er erbeizete. vil gedrungen\\ 
 & wart sîn lîp, ê er \textbf{sitzen} vant\\ 
30 & vrouwen Cu\textit{nn}ewaren d\textit{e} Lalant.\\ 
\end{tabular}
\scriptsize
\line(1,0){75} \newline
m n o Fr69 \newline
\line(1,0){75} \newline
\textbf{7} \textit{Initiale} m   $\cdot$ \textit{Capitulumzeichen} n  \newline
\line(1,0){75} \newline
\textbf{2} erstæche] erstache m \textbf{6} wær] was n \textbf{8} Artuses] Artus m n o \textbf{9} besunder was erkenneclîch] Erkentlich was besunder n (o) \textbf{10} maniger joie rîch] richem ponder n (o) \textbf{11} was] lasz n o \textbf{14} daz] Da o  $\cdot$ bolz] holcz m \textbf{16} lêrte] [Leit]: Lert Fr69  $\cdot$ dâ] do n o \textbf{17} sô] Do n o \textbf{21} gewâpent] verwoppent n \textbf{25} was] [wart]: wars m \textbf{27} des] das o \textbf{30} vrouwen] Frouwe m n (o)  $\cdot$ Cunnewaren] cumewaren m connewaren n (o) :::waren Fr69  $\cdot$ de] da m \newline
\end{minipage}
\end{table}
\newpage
\begin{table}[ht]
\begin{minipage}[t]{0.5\linewidth}
\small
\begin{center}*G
\end{center}
\begin{tabular}{rl}
 & etslîcher hin ze ir spræche,\\ 
 & \textit{daz} in ir minne \textbf{stæche}\\ 
 & unde \textbf{im die} \textbf{sinne} blande.\\ 
 & op si die nôt \textbf{erkande},\\ 
5 & daz \textbf{diende}r vor unde nâch.\\ 
 & mir wære \textbf{êt} mit ir dannen gâch.\\ 
 & ich hân gereit umbe mîn dinc.\\ 
 & nû hœrt, wie Artuses rinc\\ 
 & \textbf{sunder} was erkenneclîch\\ 
10 & vor ûz mit maniger schoye rîch.\\ 
 & diu messenîe vor im \textbf{az}:\\ 
 & manic werder man gein valsche laz\\ 
 & unde manic \textbf{werdiu vrouwe} stolz,\\ 
 & daz niht \textbf{ein} tjost was bolz.\\ 
15 & ir vriunt si gein \textbf{vînde} schôz.\\ 
 & \textbf{wart sîn arbeit dâ} grôz,\\ 
 & \textbf{sus} stuont lîhte ir gemüete,\\ 
 & daz \textbf{si daz} galt mit güete.\\ 
 & \begin{large}C\end{large}lamide, der jungelinc,\\ 
20 & reit enmitten in den rinc.\\ 
 & verdaht ors, gewâpent lîp\\ 
 & \textbf{kôs} an im Artuses wîp,\\ 
 & \textbf{sînen} helm, \textbf{sînen} schilt verhouw\textit{en}.\\ 
 & daz sâhen gar die vrouwen.\\ 
25 & \textbf{alsus} was er ze hove komen.\\ 
 & ir habet ê wol vernomen,\\ 
 & \textbf{daz} er des wart betwungen.\\ 
 & er erbeizte. vil gedrungen\\ 
 & wart sîn lîp, ê er \textbf{sitzen} vant\\ 
30 & vrôn Kunewaren de Lalant.\\ 
\end{tabular}
\scriptsize
\line(1,0){75} \newline
G I O L M Q R Z Fr21 Fr40 \newline
\line(1,0){75} \newline
\textbf{7} \textit{Initiale} L Q R Fr21 Fr40  \textbf{11} \textit{Initiale} I  \textbf{19} \textit{Initiale} G M Z  \newline
\line(1,0){75} \newline
\textbf{1} spræche] spreche I Q R Z sprachen M \textbf{2} daz] vnde G Da O  $\cdot$ stæche] steche I O Q Z \textbf{3} im] in R  $\cdot$ die sinne] die sinen I freyde Q die frevde Z (Fr40)  $\cdot$ blande] enblante Q (Fr40) \textbf{4} Vnd ob sy im das wante R  $\cdot$ die nôt] im die not Q (Fr40)  $\cdot$ erkande] bechande I erbande O erwande L M (Q) Z Fr21 Fr40 \textbf{5} daz] Doch O  $\cdot$ diender vor] dient er vor I O R Z Fr21 (Fr40) vor dinte er vort M dint vor Q \textbf{6} mir] Mit Q  $\cdot$ êt mit ir] ê mit ir I (Q) (Z) mit ir O (M) Fr21 Fr40 mit ir von L von R \textbf{7} gereit] geeret O  $\cdot$ mîn] ir O \textbf{8} Artuses] artus Q R Z Fr40  $\cdot$ rinc] [dinch]: rinch O \textbf{9} erkenneclîch] irkentlich M (Q) \textbf{10} ûz] vnsz Q (R)  $\cdot$ schoye rîch] tho rich O tschie [n]: rich M froͯlichen dink R svͦr rich Fr21 \textbf{11} messenîe] messye R  $\cdot$ az] vͣz O saz L was R \textbf{12} gein] \textit{om.} Q gen Im R  $\cdot$ valsche laz] gevalschelasz Q \textbf{13} werdiu vrouwe] frowe I (Q) (Fr40) ivnchfrowe O (M) (R) (Z) (Fr21) juͯngrowe L \textbf{14} ein] wan O L (M) (Q) (R) Z Fr21 Fr40  $\cdot$ bolz] ir bolz O (L) (Fr40) ir holcz M R (Fr21) \textbf{15} vriunt si] vriunde sin I  $\cdot$ vînde] ir veinde O vianden L den vienden M (Q) (Z) (Fr40) den wienden R dem viende Fr21  $\cdot$ schôz] stosz M \textbf{16} wart sîn arbeit] des wart sin arbait I Lert in streit Q (R) (Fr40) Leit in strit Z  $\cdot$ dâ] \textit{om.} I do O L Q R  $\cdot$ grôz] kumer grosz Q (Z) (Fr40) kumers gros R \textbf{17} sus] So M Auch Q  $\cdot$ lîhte] leicht Q (Fr40)  $\cdot$ ir] sin L \textbf{18} si daz] si I \textbf{19} Clamide] klamide I Glamide O \textbf{20} enmitten] mitten M  $\cdot$ den] de I \textbf{22} Artuses] artus G Q R Z (Fr40) \textbf{23} sînen helm sînen] Den helm den L  $\cdot$ verhouwen] verhoͮw::: G \textbf{24} daz] \textit{om.} L Dar M Dis R (Z)  $\cdot$ sâhen] Sahet L suchent R \textbf{25} alsus] Also Q  $\cdot$ er] ez ir I  $\cdot$ hove] hore Q \textbf{26} \textit{Vers 217.26 fehlt} L   $\cdot$ ir habet] Er hat M \textbf{27} wart] E was R \textbf{28} erbeizte] erbeizt O \textbf{29} ê] ê daz I (M) \textbf{30} vrôn] fro I (L) (M) (Q) (R)  $\cdot$ Kunewaren] kunware I kvnwaren O (M) Cvnewaren L konwaren Q Cuͦnwartten R kvnnewaren Z gvnwaren Fr21 kvneware::: Fr40  $\cdot$ de] von I R der O  $\cdot$ Lalant] labant R \newline
\end{minipage}
\hspace{0.5cm}
\begin{minipage}[t]{0.5\linewidth}
\small
\begin{center}*T
\end{center}
\begin{tabular}{rl}
 & etslîcher hin zir spræche,\\ 
 & daz in ir minne \textbf{stæche}\\ 
 & unde \textbf{in diu} \textbf{vröude} blante.\\ 
 & ob si di\textit{e} nôt \textbf{erwante},\\ 
5 & daz \textbf{gediende}r vor unde nâch.\\ 
 & mir wære mit ir dannân gâch.\\ 
 & Ich hân geredet umbe mîn dinc.\\ 
 & Nû hœret \textbf{ouch}, wie Artuses rinc\\ 
 & \textbf{sunder} was erkennelîch\\ 
10 & vor ûz mit maneger schoye rîch.\\ 
 & Diu massenîe vor im \textbf{az}:\\ 
 & manec werder man gegen valsche laz\\ 
 & unde manec \textbf{juncvrouwe} stolz,\\ 
 & daz niht \textbf{wan} tjost was \textbf{ir} bolz.\\ 
15 & ir vriunt si gegen \textbf{den vîenden} schôz.\\ 
 & \textbf{wart sîn arbeit dô iht} grôz,\\ 
 & \textbf{sô} stuont lîhte ir gemüete,\\ 
 & daz \textbf{si daz} galt mit güete.\\ 
 & \begin{large}C\end{large}lamide, der jungelinc,\\ 
20 & reit enmitten in den rinc.\\ 
 & verdecket ors, gewâpent lîp\\ 
 & \textbf{sach} an im Artuses wîp,\\ 
 & \textbf{den} helm, \textbf{den} schilt verhouwen.\\ 
 & daz sâhen gar die vrouwen.\\ 
25 & \textbf{Alsus} was er ze hove komen.\\ 
 & ir habet ê wol vernomen,\\ 
 & \textbf{wie} er des wart betwungen.\\ 
 & er erbeizete. vil gedrungen\\ 
 & wart sîn lîp, ê er \textbf{sitzende} vant\\ 
30 & vroun Cunnewaren de Lalant.\\ 
\end{tabular}
\scriptsize
\line(1,0){75} \newline
T U V W \newline
\line(1,0){75} \newline
\textbf{7} \textit{Initiale} U W   $\cdot$ \textit{Majuskel} T  \textbf{8} \textit{Majuskel} T  \textbf{11} \textit{Majuskel} T  \textbf{19} \textit{Initiale} T U V  \textbf{25} \textit{Majuskel} T  \newline
\line(1,0){75} \newline
\textbf{1} zir] zuͦ dir U \textbf{3} \textit{Die Verse 217.3-6 fehlen} W   $\cdot$ in] [i*]: im V  $\cdot$ vröude] [*]: sinne V \textbf{4} si] ich U  $\cdot$ die] div T \textbf{6} wære] were [*]: e V \textbf{7} geredet] geeret W \textbf{8} ouch] \textit{om.} W \textbf{9} sunder] [Svnder]: Bi svnder V \textbf{10} ûz] vns W  $\cdot$ schoye] schone U \textbf{14} bolz] bloz U holtz V \textbf{16} [*]: Lerte in strit do kvmber groz V  $\cdot$ dô] da U \textbf{17} lîhte] leichte W \textbf{18} si daz] [sv́*]: sv́z im V \textbf{20} enmitten] miten U \textbf{21} gewâpent] verwapent W \textbf{23} den helm] [S*]: Sinen helm V  $\cdot$ den schilt] [*]: sinen schilt V dan schilt W \textbf{29} sitzende] sitzen W \textbf{30} vroun] Vrou U (W)  $\cdot$ Cunnewaren] Cuͦnnewaren U kunnewarn W  $\cdot$ Lalant] laland W \newline
\end{minipage}
\end{table}
\end{document}
