\documentclass[8pt,a4paper,notitlepage]{article}
\usepackage{fullpage}
\usepackage{ulem}
\usepackage{xltxtra}
\usepackage{datetime}
\renewcommand{\dateseparator}{.}
\dmyyyydate
\usepackage{fancyhdr}
\usepackage{ifthen}
\pagestyle{fancy}
\fancyhf{}
\renewcommand{\headrulewidth}{0pt}
\fancyfoot[L]{\ifthenelse{\value{page}=1}{\today, \currenttime{} Uhr}{}}
\begin{document}
\begin{table}[ht]
\begin{minipage}[t]{0.5\linewidth}
\small
\begin{center}*D
\end{center}
\begin{tabular}{rl}
\textbf{667} & \begin{large}N\end{large}û lât Artusen stille ligen.\\ 
 & Gawans \textbf{grüezen} wart verswigen\\ 
 & \textbf{in}; den tac unsanfte erz meit.\\ 
 & des morgens vruo mit \textbf{krache} reit\\ 
5 & gein Joflanze Artuses her.\\ 
 & sîne nâchhuote schuof er ze wer;\\ 
 & dô \textbf{die} niht \textbf{strîtes vunden} dâ,\\ 
 & si kêrten nâch \textbf{im} ûf \textbf{die} slâ.\\ 
 & Dô nam mîn hêr Gawan\\ 
10 & sîn ambetliute sunder dan;\\ 
 & niht langer er wolde bîten.\\ 
 & \textbf{er hiez den marschalc} rîten\\ 
 & \textbf{ze} Joflanze ûf den plân.\\ 
 & "sunder \textbf{leger} wil ich hân.\\ 
15 & dû sihst daz grôze her \textbf{wol} ligen.\\ 
 & \textbf{ez} ist \textbf{êt} nû alsô gedigen,\\ 
 & ir hêrren \textbf{muoz} ich \textbf{iu} nennen,\\ 
 & daz ir \textbf{den muget} erkennen:\\ 
 & Ez ist mîn œheim Artus,\\ 
20 & in des hove unt in des hûs\\ 
 & ich von kinde bin \textbf{erzogen}.\\ 
 & nû schaffet mir vür unbetrogen\\ 
 & mîne reise \textbf{alsô} mit koste dar,\\ 
 & daz mans vür rîcheit neme war,\\ 
25 & Unt lât hie ûffe unvernomen,\\ 
 & daz Artuses her durch mich sî komen."\\ 
 & si leisten, swaz er in gebôt.\\ 
 & des wart Plippalinot\\ 
 & dar nâch unmüezic schiere.\\ 
30 & Kocken, ussiere,\\ 
\end{tabular}
\scriptsize
\line(1,0){75} \newline
D Fr10 \newline
\line(1,0){75} \newline
\textbf{1} \textit{Initiale} D  \textbf{9} \textit{Majuskel} D  \textbf{19} \textit{Majuskel} D  \textbf{25} \textit{Majuskel} D  \textbf{27} \textit{Initiale} Fr10  \textbf{30} \textit{Majuskel} D  \newline
\line(1,0){75} \newline
\textbf{5} Artuses] Artvs D \textbf{23} alsô] \textit{om.} Fr10 \textbf{26} Artuses] Artvs D (Fr10)  $\cdot$ sî] ist Fr10 \textbf{29} schiere] sere Fr10 \newline
\end{minipage}
\hspace{0.5cm}
\begin{minipage}[t]{0.5\linewidth}
\small
\begin{center}*m
\end{center}
\begin{tabular}{rl}
 & \begin{large}N\end{large}û lât Artusen stille ligen.\\ 
 & Gawans \textbf{gruoz} \textbf{in} wart verswigen;\\ 
 & den \textbf{ganzen} tac; unsanft erz meit.\\ 
 & des morgens vruo mit \textbf{vorhte} reit\\ 
5 & gegen \textit{J}o\textit{f}lanze Artuses her.\\ 
 & sîn nâchhuote schuof er zuo wer;\\ 
 & dô \textbf{die} niht \textbf{strîtes vunden} dâ,\\ 
 & si kârten nâch \textbf{im} ûf \textbf{die} slâ.\\ 
 & dô nam mîn hêr Gawan\\ 
10 & sîn ambetliute sunder dan;\\ 
 & niht langer er wolt bîten.\\ 
 & \textbf{er hiez den marsch\textit{al}c} rîten\\ 
 & \textbf{zuo} Joflanze ûf den plân.\\ 
 & \textbf{er sprach}: "\textbf{mîn} sunder \textbf{leger} wil ich hân.\\ 
15 & dû sihest daz grôze her \textbf{d\textit{â}} ligen.\\ 
 & \textbf{ez} ist \textbf{eht} nû alsô gedigen,\\ 
 & ir hêrren \textbf{muoz} ich \textbf{iu} nennen,\\ 
 & daz ir \textbf{den müget} erkennen:\\ 
 & ez ist mîn œheim Artus,\\ 
20 & in des hof und in des hûs\\ 
 & ich von kinde bin \textbf{erzogen}.\\ 
 & nû schaffet mir vür unbetrogen\\ 
 & mîn reise \textbf{alsô} mit koste dar,\\ 
 & daz mans vür rîcheit neme war,\\ 
25 & und lât hie ûf unvernomen,\\ 
 & daz Artuses \textit{h}er durch mich sî komen."\\ 
 & si leiste\textit{n}, waz er in gebôt.\\ 
 & des wart Pl\textit{i}ppalinot\\ 
 & dar nâch unmüezic schiere.\\ 
30 & kocken, ussiere,\\ 
\end{tabular}
\scriptsize
\line(1,0){75} \newline
m n o \newline
\line(1,0){75} \newline
\textbf{1} \textit{Initiale} m   $\cdot$ \textit{Capitulumzeichen} n  \newline
\line(1,0){75} \newline
\textbf{1} stille] stillen o \textbf{3} den] Der o \textbf{5} Joflanze] koulantz m koufflantz n kauͯfflancz o  $\cdot$ Artuses] artuͯses o \textbf{7} \textit{Die Verse 667.7-8 fehlen} o  \textbf{9} hêr] herre her n \textbf{12} hiez den] >hies den< o  $\cdot$ marschalc] marschlag m \textbf{13} Joflanze] jofflantz m n Jofflancz o \textbf{14} ich] \textit{om.} o \textbf{15} dâ] do m n o \textbf{19} Artus] artús n \textbf{20} des hûs] das huͯs o \textbf{21} von] \textit{om.} o \textbf{23} reise] reise reise o \textbf{26} Artuses] artus m n artuͯs o  $\cdot$ her] der m \textbf{27} leisten] leistet m n o \textbf{28} Plippalinot] plimppalinot m plipp:lmot o \newline
\end{minipage}
\end{table}
\newpage
\begin{table}[ht]
\begin{minipage}[t]{0.5\linewidth}
\small
\begin{center}*G
\end{center}
\begin{tabular}{rl}
 & \begin{large}N\end{large}û lât Artusen stille ligen.\\ 
 & Gawans \textbf{grüezen} wart verswigen\\ 
 & \textbf{al} den tac; unsanfte erz meit.\\ 
 & des morgens vruo mit \textbf{krache} reit\\ 
5 & gein Tschofflanze Artuses her.\\ 
 & sîn nâchhuote schuof er ze wer;.\\ 
 & dô \textbf{dise} niht \textbf{strîtes vunden} dâ,\\ 
 & si kêrten nâch \textbf{in} ûf \textbf{ir} slâ.\\ 
 & dô nam mîn hêr Gawan\\ 
10 & sîn ambetliute sunder dan.\\ 
 & niht langer er wolde bîten;\\ 
 & \textbf{sînen marschalc hiez er} rîten\\ 
 & \textbf{gein} Tschofflanze ûf den plân.\\ 
 & "sunder \textbf{lenger wîle} wil ich hân.\\ 
15 & dû sihest daz grôze her \textbf{dâ} ligen.\\ 
 & \textbf{daz} ist \textbf{êt} nû alsô gedigen,\\ 
 & ir hêrren \textbf{wil} ich nennen,\\ 
 & daz ir \textbf{ruochet in} \textit{er}kennen:\\ 
 & ez ist mîn œheim Artus,\\ 
20 & in des hove unde in des hûs\\ 
 & ich von kinde bin \textbf{erzogen}.\\ 
 & nû schaffet mir vür unbetrogen\\ 
 & mîn reise \textbf{sô} mit koste dar,\\ 
 & daz mans vür rîcheit neme war,\\ 
25 & unde lât hie ûffe unvernomen,\\ 
 & daz Artuses her durch mich sî komen."\\ 
 & si leisten, swaz er in gebôt.\\ 
 & des wart Pliplalinot\\ 
 & dar nâch unm\textit{üezec} schiere.\\ 
30 & kocken \textbf{unde} \textit{urs}siere,\\ 
\end{tabular}
\scriptsize
\line(1,0){75} \newline
G I L M Z \newline
\line(1,0){75} \newline
\textbf{1} \textit{Initiale} G I L M Z  \textbf{15} \textit{Initiale} I  \newline
\line(1,0){75} \newline
\textbf{1} Artusen] Artv̂sen G Artus I Artuͯsen L \textbf{2} Gawans] Gawanz L  $\cdot$ grüezen] zorn I \textbf{4} krache] craft er I \textbf{5} Tschofflanze] shaffanze I Tschoflanze L schofflanze M Tschofflanz Z  $\cdot$ Artuses] Artvs G \textbf{6} sîn] Sy en M  $\cdot$ er] \textit{om.} I \textbf{7} dô] So L Da M Z  $\cdot$ dise] si I die Z \textbf{8} in] \textit{om.} L  $\cdot$ ir] die I \textbf{9} dô] Da M Z  $\cdot$ hêr Gawan] ergawan M \textbf{11} er wolde] wolt er I L \textbf{13} Tschofflanze] shoffanz I tschoflanze L schofflancze M Tschofflantz Z \textbf{14} sunder lenger] Svnder leger L (Z)  $\cdot$ wîle wil] D wil \textit{(Freiraum vor} wil \textit{fälschlicherweise mit} D\textit{-Initiale von 667.15 aufgefüllt)} I wil L Z \textbf{15} dû] ÷u I  $\cdot$ dâ] dort I \textbf{17} ich] ich ev Z \textbf{18} ir ruochet in] ir in ruchet I (L) irn geruchet M ir geruchet in Z  $\cdot$ erkennen] bechennen G \textbf{19} Artus] Artuͯs L \textbf{22} schaffet] shaffe I  $\cdot$ unbetrogen] vngelogen I \textbf{24} vür] von M \textbf{25} hie] die Z \textbf{26} Artuses] Artvs G (I) (L) (M) (Z) \textbf{27} swaz] waz L (M)  $\cdot$ in] \textit{om.} I \textbf{28} Pliplalinot] plipalinot G I L M Z \textbf{29} unmüezec] vnmoͮzlich G \textbf{30} urssiere] visîere G vrshiere I \newline
\end{minipage}
\hspace{0.5cm}
\begin{minipage}[t]{0.5\linewidth}
\small
\begin{center}*T
\end{center}
\begin{tabular}{rl}
 & nû lât Artusen stille ligen.\\ 
 & Gawans \textbf{grüeze\textit{n}} w\textit{a}rt verswigen\\ 
 & \textbf{al} den tac; unsanft erz meit.\\ 
 & des morgens vruo mit \textbf{krache} reit\\ 
5 & gên Tschoflanze Artuses her.\\ 
 & sîn nâchhuote schuof er zuo wer;\\ 
 & dô \textbf{dise} niht \textbf{vunden strîtes} d\textit{â},\\ 
 & si kêrten nâch \textbf{im} ûf \textbf{die} slâ.\\ 
 & dô nam mîn hêr Gawan\\ 
10 & sîn ambetliute sunder dan;\\ 
 & niht lenger er wolte bîten.\\ 
 & \textbf{sînen marschalc hiez er} rîten\\ 
 & \textbf{gên} Tschoflanze ûf den plân.\\ 
 & "sunder \textbf{leger} wil ich hân.\\ 
15 & dû sihest daz grôz her \textbf{d\textit{â}} ligen.\\ 
 & \textbf{d\textit{az}} ist \textbf{ouch} nû alsô gedigen,\\ 
 & ir hêrren \textbf{wil} ich nennen,\\ 
 & daz ir \textbf{den mugt} erkennen:\\ 
 & ez ist mîn œheim Artus,\\ 
20 & in des hove und in des hûs\\ 
 & ich von kinde bin \textbf{gezogen}.\\ 
 & nû schaffet mir vür unbetrogen\\ 
 & mîn reise \textbf{sô} mit koste dar,\\ 
 & daz mans vür rîcheit neme war,\\ 
25 & und lât hie ûf unvernomen,\\ 
 & daz Artuses her durch mich sî komen."\\ 
 & si leisten, waz er in gebôt.\\ 
 & des wart Plipalinot\\ 
 & dâ nâch unmüezic schiere.\\ 
30 & kocken \textbf{und} ussiere,\\ 
\end{tabular}
\scriptsize
\line(1,0){75} \newline
Q R W V \newline
\line(1,0){75} \newline
\textbf{1} \textit{Initiale} R  \textbf{27} \textit{Initiale} W V  \newline
\line(1,0){75} \newline
\textbf{1} Artusen] artuse W [artvs*]: artvsen V \textbf{2} grüezen] grusse Q  $\cdot$ wart] wirt Q W \textbf{4} morgens] morgen R  $\cdot$ krache] krafftte R \textbf{5} Tschoflanze] schofflantze Q Schoflancze R tschoflantze W schoflanze V  $\cdot$ Artuses] artus Q R \textbf{7} vunden strîtes] strittes funden R (W) strit fvnden V  $\cdot$ dâ] do Q [d*]: da V \textbf{8} die] dem R \textbf{9} Gawan] [*]: Gawan V \textbf{11} bîten] bitten R \textbf{13} gên] Ze V  $\cdot$ Tschoflanze] schofflantze Q schafelancze R tschoflantze W Scoflanze V  $\cdot$ ûf den plân] dan W \textbf{14} Sunder legeren auff den plan W  $\cdot$ [S*]: Er sprach min svnder herberge wil ich han V \textbf{15} dâ] do Q W [*]: do V \textbf{16} daz] Do Q  $\cdot$ ist] icht R  $\cdot$ ouch] echt R W \textit{om.} V \textbf{17} [J*]: Jrn herren [*]: mvͦz ich v́ch nennen V \textbf{20} des] [dem]: desz Q \textbf{21} gezogen] erczogen R (W) (V) \textbf{22} schaffet] schaffe V \textbf{24} vür rîcheit] von nicht R \textbf{25} ûf] \textit{om.} R \textbf{27} leisten waz] leisten das R (W) [leiste*]: leistetent daz  V \textbf{28} wart] wirt W  $\cdot$ Plipalinot] plipanot R plypalinot W \textbf{30} und] \textit{om.} R \newline
\end{minipage}
\end{table}
\end{document}
