\documentclass[8pt,a4paper,notitlepage]{article}
\usepackage{fullpage}
\usepackage{ulem}
\usepackage{xltxtra}
\usepackage{datetime}
\renewcommand{\dateseparator}{.}
\dmyyyydate
\usepackage{fancyhdr}
\usepackage{ifthen}
\pagestyle{fancy}
\fancyhf{}
\renewcommand{\headrulewidth}{0pt}
\fancyfoot[L]{\ifthenelse{\value{page}=1}{\today, \currenttime{} Uhr}{}}
\begin{document}
\begin{table}[ht]
\begin{minipage}[t]{0.5\linewidth}
\small
\begin{center}*D
\end{center}
\begin{tabular}{rl}
\textbf{641} & \begin{large}D\end{large}ar nâch \textbf{schiere ein ende nam} der tanz.\\ 
 & juncvrouwen mit varwen glanz\\ 
 & sâzen dort und hie,\\ 
 & die rîter sâzen zwischen sie.\\ 
5 & des vreude sich an sorgen rach,\\ 
 & swer dâ nâch werder minne sprach,\\ 
 & ob er vant süeziu gegenwort.\\ 
 & von dem wirte wart gehôrt,\\ 
 & man solt\textbf{z} trinken vür \textbf{in} tragen.\\ 
10 & daz mohten \textbf{werber} klagen.\\ 
 & der wirt warp mit den gesten,\\ 
 & in kunde ouch minne lesten.\\ 
 & ir sitzen dûht in gar ze lanc,\\ 
 & sîn herze ouch werdiu minne twanc.\\ 
15 & Daz trink\textit{en} gab in urloup.\\ 
 & manegen \textbf{kerzînen} schoup\\ 
 & \textbf{truogen knappen} vor den rîtern dan.\\ 
 & dô bevalch mîn hêr Gawan\\ 
 & dise zwêne geste in allen;\\ 
20 & daz muose \textbf{in} wol gevallen.\\ 
 & Lischoys unt Florant\\ 
 & vuoren slâfen \textbf{al} zehant.\\ 
 & diu herzoginne \textbf{was} sô bedâht,\\ 
 & \textbf{diu} \textbf{sprach}, si gunde in guoter naht.\\ 
25 & Dô vuor ouch al der vrouwen schar,\\ 
 & dâ si gemaches nâmen war.\\ 
 & ir nîgens si begunden\\ 
 & mit \textbf{zuht}, die si \textbf{wol} kunden.\\ 
 & \textbf{Sangive} und Itonje\\ 
30 & vuoren dan, als tet \textbf{ouch} Cundrie.\\ 
\end{tabular}
\scriptsize
\line(1,0){75} \newline
D Z Fr1 \newline
\line(1,0){75} \newline
\textbf{1} \textit{Initiale} D Z Fr1  \textbf{15} \textit{Majuskel} D  \textbf{25} \textit{Majuskel} D   $\cdot$ \textit{Versal} Fr1  \newline
\line(1,0){75} \newline
\textbf{1} ein ende nam] nam ende Z \textbf{5} vreude] frewte Z \textbf{6} swer] Wer Z \textbf{10} daz] Da Z  $\cdot$ klagen] claren Z \textbf{12} in] Jch Z \textbf{15} trinken] trinch D \textbf{16} von chleinen chercen manegen scoͮp Fr1 \textbf{17} dan] dar Z \textbf{18} dô] Da Z \textbf{21} Lischoys] Liscoys D Lishois Z Lẏscois Fr1 \textbf{22} al] sa Z \textbf{23} sô] \textit{om.} Z \textbf{24} diu] Sie Z  $\cdot$ guoter] wol gvͦter Fr1 \textbf{25} Dô] Da Z \textbf{28} zuht] zvhten Z \textbf{29} Sangive] Sangîve D Fr1 Seyve Z  $\cdot$ Itonje] Jtonie D Fr1 Jconie Z \textbf{30} Cundrie] Cvndrîe D kundrie Z \newline
\end{minipage}
\hspace{0.5cm}
\begin{minipage}[t]{0.5\linewidth}
\small
\begin{center}*m
\end{center}
\begin{tabular}{rl}
 & dar nâch \textbf{nam ein ende} der tanz.\\ 
 & juncvrowen mit varwe glanz,\\ 
 & \textbf{die} sâzen dort und hie,\\ 
 & die ritter sâzen zwischen sie.\\ 
5 & des vröude sich an sorgen \textit{r}ach,\\ 
 & wer dar nâch werder minne sprach,\\ 
 & ob er \dag nante\dag  süeziu gegenwort.\\ 
 & von dem wirt wart gehôrt,\\ 
 & man solt \textbf{daz} trinken vür \textbf{in} tragen.\\ 
10 & daz mohten \textbf{werbære} klagen.\\ 
 & der wirt \dag wart\dag  mit den gesten,\\ 
 & in kunde ouch minne lesten.\\ 
 & ir sitzen dûhte in gar zuo lanc,\\ 
 & sîn herze ouch werdiu minne twanc.\\ 
15 & daz trinken gap in urloup.\\ 
 & manigen \textbf{kerzînen} schoup\\ 
 & \textbf{truoc man} vor den rittern dan.\\ 
 & dô bevalch mîn hêr Gawan\\ 
 & dise zwêne geste in allen;\\ 
20 & daz muoste \textbf{in} wol gevallen.\\ 
 & Lischois und Florant\\ 
 & vuoren slâfen \textbf{al} zehant.\\ 
 & diu herzogîn \textbf{wart} sô bedâht,\\ 
 & \textbf{si} \textbf{sprach}, si gonde i\textit{n} guoter naht.\\ 
25 & dô vuor ouch al der vrowen schar,\\ 
 & d\textit{â} si gemaches nâmen war.\\ 
 & ir nîgens si begunden\\ 
 & mit \textbf{zuht}, die si \textbf{wol} kunden.\\ 
 & \textbf{San\textit{g}i\textit{ve}} und Ithonie\\ 
30 & vuoren dan, als tet \textbf{ouch} Condrie.\\ 
\end{tabular}
\scriptsize
\line(1,0){75} \newline
m n o \newline
\line(1,0){75} \newline
\newline
\line(1,0){75} \newline
\textbf{2} varwe] frouͯide o \textbf{5} vröude] froͯden m (n) (o)  $\cdot$ rach] vach m [pflag]: rach n \textbf{6} dar] do n o \textbf{8} wart] gar wart n \textbf{10} mohten] moͯchten n \textbf{13} ir] Jn n \textbf{17} rittern] ritter o \textbf{18} bevalch] befalsch o  $\cdot$ hêr] herre her n \textbf{21} Lischois] Liscois m n o \textbf{22} al] alle o \textbf{24} gonde] guͯnde o  $\cdot$ in] ir m n o \textbf{26} dâ] Do m n o \textbf{29} Sangive] Sangwin m n o  $\cdot$ Ithonie] jthonie m jthonien n Jtonie o \textbf{30} als] auch o  $\cdot$ Condrie] condrien n Cúndrie o \newline
\end{minipage}
\end{table}
\newpage
\begin{table}[ht]
\begin{minipage}[t]{0.5\linewidth}
\small
\begin{center}*G
\end{center}
\begin{tabular}{rl}
 & \begin{large}D\end{large}ar nâch \textbf{schier nam ende} der tanz.\\ 
 & juncvrouwen mit varwe glanz\\ 
 & sâzen dort unde hie,\\ 
 & die rîter sâzen zwischen sie.\\ 
5 & des vröude sich an sorgen rac\textit{h},\\ 
 & swer dâ nâch werder minne sprach,\\ 
 & obe er vant süeziu gegenwort.\\ 
 & von dem wirt wart gehôrt,\\ 
 & man solde trinken vür \textbf{in} tragen.\\ 
10 & daz mohten \textbf{minnære} \textbf{wol} klagen.\\ 
 & der wirt warp mit den gesten,\\ 
 & in kunde ouch minne lesten.\\ 
 & ir sitzen dûhte in gar ze lanc,\\ 
 & sîn herze ouch werd\textit{iu} minne twanc.\\ 
15 & daz trinken gab in urloup.\\ 
 & manic \textbf{kerzînen} schoup\\ 
 & \textbf{truogen knappen} vor den rîtern dan.\\ 
 & dô bevalch mîn hêrre Gawan\\ 
 & dise zwêne geste in allen;\\ 
20 & daz muos \textbf{in} wol gevallen.\\ 
 & Lishois unde Florant\\ 
 & vuoren slâfen \textbf{sân} zehant.\\ 
 & diu herzogîn \textbf{was} sô bedâht,\\ 
 & \textbf{si} \textbf{jach}, si gunde in guoter naht.\\ 
25 & d\textit{ô} vuor ouch al der vrouwen schar,\\ 
 & dâ si gemaches nâmen war.\\ 
 & ir nîgens si begunden\\ 
 & mit \textbf{zühten}, die si kunden.\\ 
 & \textbf{Sagive} unde Itonie\\ 
30 & vuoren dan, als tet Gundrie.\\ 
\end{tabular}
\scriptsize
\line(1,0){75} \newline
G I L M Z Fr18 Fr48 \newline
\line(1,0){75} \newline
\textbf{1} \textit{Initiale} G I Z Fr18 Fr48  \textbf{15} \textit{Initiale} I  \textbf{23} \textit{Initiale} M  \textbf{27} \textit{Initiale} I  \newline
\line(1,0){75} \newline
\textbf{1} Dar nâch] Do I  $\cdot$ schier nam ende] nam shier ende I ennde nam M schier ende kam Fr18 \textbf{2} varwe] frawen Fr48 \textbf{5} \textit{Versfolge 641.6-5} L   $\cdot$ vröude] frewte Z  $\cdot$ sorgen] sorge M  $\cdot$ rach] rac G \textbf{6} swer] Wer L M \textbf{8} gehôrt] daz Gehort I \textbf{9} trinken] das trinken Z  $\cdot$ in] \textit{om.} I \textbf{10} Da mohten werbere claren Z \textbf{12} in] [Io]: In G Jch M Z Fr18 \textbf{14} werdiu] werde G Fr18 wer die L \textbf{15} in] yn ouch M \textbf{16} kerzînen] cherzenin G (M) kerzen I grosze kertzen L  $\cdot$ schoup] hovp L \textbf{17} dan] dar Z da: Fr18 \textbf{18} dô] Da M Z  $\cdot$ hêrre Gawan] ergawan M \textbf{19} geste] ritter M \textbf{20} muos] muͤs I \textbf{21} Lishois] Liscoys I Lýtschoýs L Lisois M Lẏshoẏs Fr18  $\cdot$ Florant] floriant G I \textbf{22} sân] \textit{om.} I L \textbf{23} sô] \textit{om.} Z \textbf{24} jach] sprach L (M) Z Fr18  $\cdot$ in] in allen wol I \textbf{25} dô] da G (M) (Z)  $\cdot$ al] aller I \textbf{26} dâ] Daz L \textbf{27} ir] Jn L [In]: Ir Fr18  $\cdot$ nîgens] in gense Fr18 \textbf{28} si] \textit{om.} I sie wol Z \textbf{29} Sagive] saive I (M) Seyve Z Saẏue Fr18  $\cdot$ Itonie] ytonie G Jtonie I (L) Jthonie M Jconie Z ytony: Fr18 \textbf{30} dan als tet] tet danne ouch alle M dan als tet ouch Z dan sam tet ovch Fr18  $\cdot$ Gundrie] kundrien M kundrie Z (Fr18) \newline
\end{minipage}
\hspace{0.5cm}
\begin{minipage}[t]{0.5\linewidth}
\small
\begin{center}*T
\end{center}
\begin{tabular}{rl}
 & \begin{large}D\end{large}ar nâch \textbf{schiere ende nam} der tanz.\\ 
 & juncvrouwen mit varwe glanz\\ 
 & sâzen dort und hie,\\ 
 & die rîter sâzen zwischen sie.\\ 
5 & des vreude sich an sorgen rach,\\ 
 & wer dâ nâch werder minne sprach,\\ 
 & ob er vant süeziu gegenwort.\\ 
 & von dem wirte wart gehôrt,\\ 
 & man solte trinken vür \textbf{si} tragen.\\ 
10 & daz mohte\textit{n} \textbf{die} \textbf{werbære} \textbf{wol} klagen.\\ 
 & der wirt warp mit den gesten,\\ 
 & in kunde ouch minne lesten.\\ 
 & ir sitzen dûht in gar zuo lanc,\\ 
 & sîn herze ouch werdiu minne twanc.\\ 
15 & daz trinken gap in urloup.\\ 
 & manegen \textbf{kerzen} schoup\\ 
 & \textbf{truogen knappen} vor den rîtern dan.\\ 
 & dô bevalch mîn hêr Gawan\\ 
20 & \hspace*{-.7em}\big| - daz muose wol gevallen -\\ 
 & \hspace*{-.7em}\big| dise zwêne geste in allen.\\ 
 & Lyschoys und Florant\\ 
 & vuoren slâfen \textbf{al} zehant.\\ 
 & \begin{large}D\end{large}iu herzogîn \textbf{was} sô bedâht,\\ 
 & \textbf{si} \textbf{sprach}, si gunde in guoter naht.\\ 
25 & dô vuor ouch al der vrouwen schar,\\ 
 & d\textit{â} si gemaches nâmen war.\\ 
 & ir nîgens si begunden\\ 
 & mit \textbf{zühten}, die si \textbf{wol} kunden.\\ 
 & \textbf{Seyve} und Itonje\\ 
30 & vuoren dan, als tet \textbf{ouch} Kuondrie.\\ 
\end{tabular}
\scriptsize
\line(1,0){75} \newline
U V W Q R \newline
\line(1,0){75} \newline
\textbf{1} \textit{Initiale} U V W   $\cdot$ \textit{Capitulumzeichen} R  \textbf{23} \textit{Initiale} U  \newline
\line(1,0){75} \newline
\textbf{1} schiere ende nam] schiere rvmde men V nam ein ende W  $\cdot$ der] den V \textbf{2} varwe] [var*]: varwen V wauren R \textbf{3} sâzen] Die sazen V (W) Lassen Q \textbf{5} des] Die W  $\cdot$ vreude] frewden Q \textbf{6} wer] Swer V  $\cdot$ dâ nâch] do nach U Q R [d*]: do noch V darnach W \textbf{7} er] der R  $\cdot$ süeziu] suͯsze R \textbf{9} trinken] das trincken W  $\cdot$ si] in W Q R \textbf{10} daz] Des R  $\cdot$ mohten] mochte U [mohten*]: moͤhten V  $\cdot$ die] \textit{om.} Q R  $\cdot$ wol] \textit{om.} W \textbf{11} warp] \textit{om.} W \textbf{12} in] Jr Q \textbf{14} ouch werdiu] in auch werder W nach werder R  $\cdot$ twanc] tzanck Q trang R \textbf{16} kerzen] kerzinen V (W) (Q) \textbf{20} \textit{Versfolge 641.19-20} W Q R   $\cdot$ muose] mvͤsten V muͦst in W (Q) (R) \textbf{21} Lyschoys] Lyschois U R Lischoys V Lyshois W Lishois Q \textbf{22} vuoren] Fvͦrten V  $\cdot$ al] so W Q do R \textbf{23} sô] also W \textbf{24} guoter] guͦte W \textbf{25} al der] aldar der V aller W \textbf{26} dâ] Do U V W Q R \textbf{28} zühten] zvht V  $\cdot$ die si] [*]: die sv́ V sy das W \textbf{29} Seyve] Seyue V W Q R  $\cdot$ Itonje] Jtonie U ytonẏe V ytonie W Q Nytonie R \textbf{30} Kuondrie] kvndrie V (W) (Q) kondrie R \newline
\end{minipage}
\end{table}
\end{document}
