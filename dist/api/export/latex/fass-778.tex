\documentclass[8pt,a4paper,notitlepage]{article}
\usepackage{fullpage}
\usepackage{ulem}
\usepackage{xltxtra}
\usepackage{datetime}
\renewcommand{\dateseparator}{.}
\dmyyyydate
\usepackage{fancyhdr}
\usepackage{ifthen}
\pagestyle{fancy}
\fancyhf{}
\renewcommand{\headrulewidth}{0pt}
\fancyfoot[L]{\ifthenelse{\value{page}=1}{\today, \currenttime{} Uhr}{}}
\begin{document}
\begin{table}[ht]
\begin{minipage}[t]{0.5\linewidth}
\small
\begin{center}*D
\end{center}
\begin{tabular}{rl}
\textbf{778} & \textit{\begin{large}I\end{large}}eslîch vrouwe hete prîs,\\ 
 & diu dâ saz bî ir amîs.\\ 
 & \textbf{maneger} durch gerndes herzen rât\\ 
 & gedient \textbf{was} mit \textbf{hôher} tât.\\ 
5 & Feirefiz unt Parzival\\ 
 & mit prüeven heten süeze \textit{w}al,\\ 
 & jene \textbf{vrouwen} unt dise.\\ 
 & man \textbf{gesach} ûf acker \textbf{noch} \textbf{ûf} wise\\ 
 & \textbf{liehter} vel noch \textbf{rœter} munt\\ 
10 & sô manegen nie ze keiner stunt,\\ 
 & alsô man an dem ringe vant.\\ 
 & des wart dem heiden vreude erkant.\\ 
 & Wol \textbf{dem künfteclîchem tage},\\ 
 & geêrt \textbf{sî} ir süezen \textbf{mære sage},\\ 
15 & als von ir \textbf{munde} \textbf{wart} vernomen!\\ 
 & man sach eine \textbf{juncvrouwen} komen.\\ 
 & \textbf{ir kleider} \textbf{wâren} tiwer \textbf{unt} wol gesniten,\\ 
 & kostebære nâch Franzoyser siten,\\ 
 & Ir kappe, ein rîcher samît,\\ 
20 & noch swerzer denn \textbf{ein} \textbf{gênît}.\\ 
 & arâbesch golt gap drûffe schîn,\\ 
 & wol geworht manec turteltiubelîn\\ 
 & nâch dem insigel des Grâles.\\ 
 & si wart des selben mâles\\ 
25 & \textbf{beschouwet vil} durch wunders ger.\\ 
 & nû lât si heistieren her.\\ 
 & Ir gebende was hôch und \textbf{blanc}.\\ 
 & mit manegem dickem umbevanc\\ 
 & was ir antlütze \textbf{verdecket}\\ 
30 & unt niht ze sehen enblecket.\\ 
\end{tabular}
\scriptsize
\line(1,0){75} \newline
D \newline
\line(1,0){75} \newline
\textbf{1} \textit{Initiale} D  \textbf{13} \textit{Majuskel} D  \textbf{19} \textit{Majuskel} D  \textbf{27} \textit{Majuskel} D  \newline
\line(1,0){75} \newline
\textbf{1} Ieslîch] ÷eslich D \textbf{5} Parzival] Parcifal D \textbf{6} wal] mal D \textbf{21} arâbesch] Arabesc D \newline
\end{minipage}
\hspace{0.5cm}
\begin{minipage}[t]{0.5\linewidth}
\small
\begin{center}*m
\end{center}
\begin{tabular}{rl}
 & ieglîch vrowe het prîs,\\ 
 & diu d\textit{â} saz bî ir amîs.\\ 
 & \textbf{maniger} durch gerndes herzen rât\\ 
 & gedienet \textbf{was} mit \textbf{hôher} tât.\\ 
5 & Ferefiz und Parcifal\\ 
 & mit brüeven heten süeze \textit{w}al,\\ 
 & jene \textbf{vrowen} und dise.\\ 
 & man \textbf{gesach} ûf \textit{a}cker \textbf{noch} \textit{w}ise\\ 
 & \textbf{liehter} vel noch \textbf{rœter} munt\\ 
10 & sô manigen nie zuo keiner stunt,\\ 
 & als man an dem ringe vant.\\ 
 & des wart dem heiden vröude erkant.\\ 
 & \begin{large}W\end{large}ol \textbf{dem künfteclîchen tage},\\ 
 & geêret \textbf{sî} ir süezen \textbf{mære sage},\\ 
15 & als von ir \textbf{minne} \textbf{wirt} vernomen!\\ 
 & man sach ein \textbf{juncvrowe} komen.\\ 
 & \textbf{ir kleider} \textbf{wâren} tiur \textbf{und} wol gesniten,\\ 
 & kostbær nâch Franzoser siten,\\ 
 & ir kappe, ein rîcher samît,\\ 
20 & noch swarzer dan \textbf{ein} \textbf{gênît}.\\ 
 & arâbesch golt \textit{gap} dâr ûf schîn,\\ 
 & wol geworht manic turteltiube\textit{l}în\\ 
 & nâch dem ingesigel des Grâles.\\ 
 & si wart des selben mâles\\ 
25 & \textbf{geschouwet vil} durch wunders ger.\\ 
 & nû lât si heis\textit{t}ieren her.\\ 
 & ir gebende was hôch und \textbf{lanc}.\\ 
 & mit manigem dicken umbevanc\\ 
 & was ir antlütz \textbf{verdecket}\\ 
30 & und niht zuo sehen enblecket.\\ 
\end{tabular}
\scriptsize
\line(1,0){75} \newline
m n o V V' W \newline
\line(1,0){75} \newline
\textbf{13} \textit{Überschrift:} Hie kvmmet kvndrie zvͦ kv́nig artus hof nach parzefale daz er herre werde zvͦme Grole V  Hie komet kvndrie nach parzifal vnd seite ime daz er herre solte werdin zv dem gral V'   $\cdot$ \textit{Initiale} m V V'   $\cdot$ \textit{Capitulumzeichen} n  \textbf{16} \textit{Initiale} W  \newline
\line(1,0){75} \newline
\textbf{1} ieglîch] [E*ich]: Ettelich V [*]: Ettelich V' \textbf{2} dâ] do m n o V V' W \textbf{3} gerndes herzen rât] gerendes hertz rat n hertze zernde not W \textbf{4} was] hat n [*]: wart V wart V'  $\cdot$ hôher] [hohem]: hoher V'  $\cdot$ tât] art V' \textbf{5} Ferefiz] Ferefis m o Ferrefis n Ferevis V V' Ferafis W  $\cdot$ Parcifal] parzefal V parzifal V' partzifal W \textbf{6} süeze] suͯssen n [svͤszen]: svͤsze V riche V' sy die W  $\cdot$ wal] mal m \textbf{7} vrowen] frowe o V (V') \textbf{8} gesach] sach V' W  $\cdot$ acker] ancker m  $\cdot$ wise] vise m vff wise n (o) (V) (V') (W) \textbf{10} manigen] manger o  $\cdot$ keiner] komer o \textbf{12} des] Das o  $\cdot$ heiden] heide o  $\cdot$ erkant] bekant W \textbf{13} künfteclîchen] kunsteclichen o kv́nfteclichem V (V')  $\cdot$ tage] [man]: tage o \textbf{15} \textit{Versfolge 778.16-15} W   $\cdot$ ir minne] irem munde V (V') (W) \textbf{16} juncvrowe] jvngfrouwen V' \textbf{18} kostbær] Kostlich W  $\cdot$ Franzoser] frantzosser m frantzoischer n franczoiser o franzoẏser V franzoisine V' frantzoyser W \textbf{19} rîcher] [rich ein]: riche V' \textbf{20} noch swarzer] Nach swarze V'  $\cdot$ gênît] gemit o \textbf{21} Arâbesch] Arabasch m o [Arabes*]: Arabes V [Ab*]: Arabesch V' Arabisch W  $\cdot$ gap] \textit{om.} m \textbf{22} geworht] gewircket n gefehet W  $\cdot$ turteltiubelîn] turteltubebin m \textbf{23} ingesigel] jngesigels o insigel W \textbf{24} selben] selbes W \textbf{25} geschouwet] Beschouwet n (o) (V) (V') (W) \textbf{26} si heistieren] sẏ heischieren m (n) (W) bescherenn o sv́ hestieren V sie hastieren V' \textbf{27} lanc] blang n o V V' \textbf{28} manigem] mangen o  $\cdot$ dicken] dickem n V \textbf{29} Was werdecket ir anczlicz o  $\cdot$ verdecket] bedecket V' \textbf{30} zuo sehen] zesehende V (V')  $\cdot$ enblecket] enblicken o \newline
\end{minipage}
\end{table}
\newpage
\begin{table}[ht]
\begin{minipage}[t]{0.5\linewidth}
\small
\begin{center}*G
\end{center}
\begin{tabular}{rl}
 & \begin{large}E\end{large}tslîch vrouwe het brîs,\\ 
 & diu dâ saz bî ir amîs,\\ 
 & \textbf{manigiu} durch gerndes herzen rât.\\ 
 & gedient \textbf{wart} mit \textbf{hôher} tât.\\ 
5 & Feirafiz unde Parcival\\ 
 & mit prüeven heten süeze wal,\\ 
 & jene \textbf{vrouwen} unde dise.\\ 
 & man \textbf{sach} ûf acker \textbf{unde} \textbf{ûf} wise\\ 
 & \textbf{lieht} vel noch \textbf{rôten} munt\\ 
10 & sô manigen nie ze deheiner stunt,\\ 
 & als man an dem ringe vant.\\ 
 & des wart dem heiden vröude erkant.\\ 
 & wol \textbf{der künfteclîchen sage},\\ 
 & geêrt \textbf{sît} ir süezen \textbf{sumer tage},\\ 
15 & als von ir \textbf{munde} \textbf{wart} vernomen!\\ 
 & man sach eine \textbf{juncvrouwen} komen\\ 
 & \textbf{in kleidern} tiure, wol gesniten,\\ 
 & kostbære nâch Franzoiser siten,\\ 
 & ir kappe, ein rîcher samît,\\ 
20 & noch swarzer danne \textbf{timît}.\\ 
 & arâbesch golt gap drûfe schîn,\\ 
 & wol geworht manic türteltiubelîn\\ 
 & nâch dem insigel des Grâles.\\ 
 & si wart des selben mâles\\ 
25 & \textbf{vil geschouwet} durch wunders ger.\\ 
 & nû lât si heistieren her.\\ 
 & ir gebende was hôch unde \textbf{blanc}.\\ 
 & mit manigem dicken umbe\textit{v}anc\\ 
 & was ir antlütze \textbf{bedeckt}\\ 
30 & unde niht ze sehene enbleckt.\\ 
\end{tabular}
\scriptsize
\line(1,0){75} \newline
G I L M Z \newline
\line(1,0){75} \newline
\textbf{1} \textit{Initiale} G L Z  \textbf{11} \textit{Initiale} I  \textbf{23} \textit{Initiale} I  \newline
\line(1,0){75} \newline
\textbf{1} brîs] des pris Z \textbf{2} dâ] \textit{om.} Z  $\cdot$ bî] mit I \textbf{3} durch gerndes herzen rât] mit herzen gernder art I \textbf{4} tât] art M Z \textbf{5} Feirafiz] Feirefiz G Z Ferefiz L Feirefisz M  $\cdot$ Parcival] parcifal G Z [parzifal]: Parzifal I parzifal L M \textbf{6} heten liehtev mal I  $\cdot$ wal] [*al]: wal L mal Z \textbf{7} vrouwen] froͮ I \textbf{8} sach] gesach L Z  $\cdot$ unde] noch L M Z \textbf{9} lieht] [lehtez]: liehtez I Lýchter L Liechter M (Z)  $\cdot$ rôten] roter L M Z \textbf{10} manigen] mange L (M) \textbf{13} der künfteclîchen sage] der kvmpflichen sage L deme kunfftlicheme tage M (Z) \textbf{14} sît] sin I Z sý L (M)  $\cdot$ ir] die I  $\cdot$ süezen sumer tage] svze svmer tage L susze sage M svͤzzen mere sage Z \textbf{16} komen] kome M \textbf{17} wol] vnd wol L (M) (Z) \textbf{18} kostbære] kostechliche I  $\cdot$ Franzoiser] franzoẏser G fronzoyser I franzoýser L frantzoiser Z \textbf{19} \textit{Vers 778.19 fehlt} M   $\cdot$ rîcher] vil richer I swartzer Z \textbf{20} timît] ein Timit I geniten M ein Jenit Z \textbf{21} arâbesch] Arabensch G Arabischez I Arabisch M Arabich Z \textbf{22} wol geworht manec] Wolgeworchte L \textbf{23} insigel] jngesigil M \textbf{24} des] das M \textbf{25} geschouwet] bischowet M (Z)  $\cdot$ wunders] wvnder L \textbf{26} heistieren] haisiern I \textbf{28} manigem] mangen L  $\cdot$ dicken] dichem I \textit{om.} M  $\cdot$ umbevanc] vmbehanch G \newline
\end{minipage}
\hspace{0.5cm}
\begin{minipage}[t]{0.5\linewidth}
\small
\begin{center}*T
\end{center}
\begin{tabular}{rl}
 & etslîchiu vrouwe hete prîs,\\ 
 & diu dâ saz bî ir amîs.\\ 
 & \textbf{maneger} durch gerndes herzen rât\\ 
 & gedienet \textbf{hât} mit \textbf{werder} tât.\\ 
5 & Ferefis und Parcifal\\ 
 & mit prüeven heten süeze \textit{w}al,\\ 
 & jene \textbf{vrouwe} und dise.\\ 
 & man \textbf{sach} ûf acker \textbf{noch} \textbf{ûf} wise\\ 
 & \textbf{liehter} vel noch \textbf{rœter} munt\\ 
10 & sô manegen nie zuo keiner stunt,\\ 
 & als man an dem ringe vant.\\ 
 & des wart dem heiden vreude erkant.\\ 
 & wol \textbf{dem künfte\textit{clîch}en tage},\\ 
 & geêret \textbf{sî} ir süezen \textbf{mære sage},\\ 
15 & als von i\textit{r} \textbf{munde} \textbf{wart} vernomen!\\ 
 & man sach eine \textbf{junge vrouwen} komen\\ 
 & \textbf{in kleidern} tiure \textbf{und} wol gesniten,\\ 
 & kostbære nâch Franzoyser siten,\\ 
 & ir kappe, ein rîcher samît,\\ 
20 & noch swarzer dan \textbf{ein} \textbf{gênît}.\\ 
 & arâbesch golt gap dâr ûf schîn,\\ 
 & wol gewirket manec turteltiubelîn\\ 
 & nâch dem ingesigel des Grâles.\\ 
 & si wart des selben mâles\\ 
25 & \textbf{vil beschouwet} durch wunders ger.\\ 
 & nû lât si heistieren her.\\ 
 & ir gebende was hôch und \textbf{lanc}.\\ 
 & mit manegem dicken umbevanc\\ 
 & was ir antlitz\textit{e} \textbf{bedecket}\\ 
30 & und niht zuo sehene enblecket.\\ 
\end{tabular}
\scriptsize
\line(1,0){75} \newline
U Q R \newline
\line(1,0){75} \newline
\textbf{1} \textit{Initiale} Q  \newline
\line(1,0){75} \newline
\textbf{1} hete] hett do Q \textbf{2} dâ] do Q \textbf{3} maneger] Jr menger R \textbf{4} hât] wart Q (R)  $\cdot$ werder] hoher Q (R)  $\cdot$ tât] art Q \textbf{5} Ferefis] Feirefisz Q Feirefis R  $\cdot$ Parcifal] partzifal Q parczifal R \textbf{6} süeze] suͯszen R  $\cdot$ wal] mal U \textbf{7} vrouwe] frawen Q (R) \textbf{8} sach] gesach Q R \textbf{9} liehter] Lichter Q \textbf{10} manegen] menge R \textbf{12} erkant] vil bekant R \textbf{13} dem künfteclîchen] dem kunftenhenden U den kunstlichen R \textbf{14} süezen] susze Q \textbf{15} ir] irn U \textbf{16} junge vrouwen] iuͦnge vreuͦwe U iunckfrawe Q Junckfrowen R \textbf{18} Franzoyser] franzioser Q franczoiser R \textbf{20} gênît] schmit R \textbf{22} gewirket] gewolcht Q  $\cdot$ turteltiubelîn] turtulblin Q \textbf{23} dem] \textit{om.} R \textbf{25} beschouwet] geschwoet R \textbf{27} ir] Jch Q  $\cdot$ lanc] glanck Q blank R \textbf{28} manegem] manchen Q  $\cdot$ dicken] dicke R \textbf{29} antlitze] antlitzet U \textbf{30} sehene] sechent R  $\cdot$ enblecket] erblecket Q \newline
\end{minipage}
\end{table}
\end{document}
