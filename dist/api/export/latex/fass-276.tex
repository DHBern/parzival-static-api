\documentclass[8pt,a4paper,notitlepage]{article}
\usepackage{fullpage}
\usepackage{ulem}
\usepackage{xltxtra}
\usepackage{datetime}
\renewcommand{\dateseparator}{.}
\dmyyyydate
\usepackage{fancyhdr}
\usepackage{ifthen}
\pagestyle{fancy}
\fancyhf{}
\renewcommand{\headrulewidth}{0pt}
\fancyfoot[L]{\ifthenelse{\value{page}=1}{\today, \currenttime{} Uhr}{}}
\begin{document}
\begin{table}[ht]
\begin{minipage}[t]{0.5\linewidth}
\small
\begin{center}*D
\end{center}
\begin{tabular}{rl}
\textbf{276} & der vürste kniete \textbf{vor der} magt.\\ 
 & er sprach: "dû hâst al wâr gesagt,\\ 
 & ich bin\textbf{z}, dîn bruoder Orilus.\\ 
 & der rôte rîter twanc mich sus,\\ 
5 & daz ich dir \textbf{sicherheit muoz} geben.\\ 
 & dâ mit \textbf{erkoufte} ich dô mîn leben.\\ 
 & \textbf{die} enpfâch, sô wirt hie gar getân,\\ 
 & als ich gein im gelobt hân."\\ 
 & \textbf{Doch} enpfienc si triwe in wîze hant\\ 
10 & von \textbf{im}, der truoc den serpant,\\ 
 & unt liezen ledec. dô daz geschach,\\ 
 & dô stuont er ûf unde sprach:\\ 
 & "Ich sol unt muoz \textbf{durch triwe} klagen:\\ 
 & \textbf{ouwê}, wer hât dich geslagen?\\ 
15 & dîne slege tuont mir \textbf{nimmer} wol,\\ 
 & wirt\textbf{s} zît, daz ich \textbf{die} rechen sol,\\ 
 & ich \textbf{geinre} den, \textbf{swer}z ruochet sehen,\\ 
 & daz mir grôz leit ist dran geschehen.\\ 
 & Ouch hilfet mirz klagen der küenste man,\\ 
20 & den muoter ie zer werlt gewan.\\ 
 & \textbf{der} nennet sich '\textbf{der} ritter rôt'.\\ 
 & hêr künec, vrou künegîn, er enbôt\\ 
 & iu beiden samt dienest sîn,\\ 
 & dar zuo benamen der swe\textit{s}ter mîn.\\ 
25 & er \textbf{bitet} \textbf{sîn dienst iuch} letzen\\ 
 & \textbf{unt} \textbf{dise} magt \textbf{ir} slege ergetzen.\\ 
 & \begin{large}O\end{large}uch het ich\textbf{s} dô genozzen\\ 
 & gein dem helde unverdrozzen,\\ 
 & wesser, wie si mich bestêt\\ 
30 & unt mir ir leit ze herzen gêt."\\ 
\end{tabular}
\scriptsize
\line(1,0){75} \newline
D \newline
\line(1,0){75} \newline
\textbf{9} \textit{Majuskel} D  \textbf{13} \textit{Majuskel} D  \textbf{19} \textit{Majuskel} D  \textbf{27} \textit{Initiale} D  \newline
\line(1,0){75} \newline
\textbf{24} swester] sweter D \newline
\end{minipage}
\hspace{0.5cm}
\begin{minipage}[t]{0.5\linewidth}
\small
\begin{center}*m
\end{center}
\begin{tabular}{rl}
 & \begin{large}D\end{large}er vürste k\textit{ni}ete \textbf{vür die} maget.\\ 
 & er sprach: "dû hâst al wâr gesaget,\\ 
 & ich bin\textbf{z}, dîn bruoder Orilus.\\ 
 & der rôte ritter twanc mich sus,\\ 
5 & daz ich dir \textbf{muose sicherheit} ge\textit{b}en.\\ 
 & dâ mit \textbf{erkouft} ich dô mîn leben.\\ 
 & \textbf{die} enpfâch, sô wirt hie gar getân,\\ 
 & als ich gegen ime gelobet hân."\\ 
 & \textbf{dô} enpfienc si triuwe in wîze hant\\ 
10 & von \textbf{ime}, der truoc den ser\textit{p}ant,\\ 
 & und liez in ledic. dô daz geschach,\\ 
 & dô stuont er ûf und sprach:\\ 
 & "ich sol und muoz \textbf{von schuld\textit{en}} \textit{k}lagen:\\ 
 & \textbf{swester}, wer hât dich geslagen?\\ 
15 & dîne slege tuont mir \textbf{niemer} wol,\\ 
 & wirt \textbf{es} zît, daz ich \textbf{die} rechen sol,\\ 
 & ich \textbf{gei\textit{n}re} den, \textbf{swer}z ruochet sehen,\\ 
 & daz mir grôz leit ist dran geschehen.\\ 
 & ouch hilfet mirz klagen der küeneste man,\\ 
20 & den muoter ie zuo der werlt gewan.\\ 
 & \textbf{der} nemmt sich '\textbf{der} ritter rôt'.\\ 
 & hêr künic, vrouwe künigîn, er enbôt\\ 
 & iu beiden sament \textbf{den} dienes\textit{t s}în,\\ 
 & dar zuo benamen der swester mîn.\\ 
25 & er \textbf{bitte} \textbf{sînen dienst iuch} letzen\\ 
 & \textbf{und} \textbf{dise} maget \textbf{ir} slege ergetzen.\\ 
 & ouch hete ich\textbf{s} dô genozzen\\ 
 & gegen dem helde unverdrozzen,\\ 
 & \dag weiz\dag  er, wie si mich bestât\\ 
30 & und mir ir leit zuo herzen gât."\\ 
\end{tabular}
\scriptsize
\line(1,0){75} \newline
m n o \newline
\line(1,0){75} \newline
\textbf{1} \textit{Initiale} m   $\cdot$ \textit{Capitulumzeichen} n  \newline
\line(1,0){75} \newline
\textbf{1} kniete] kunette m \textbf{2} wâr] fur o \textbf{3} binz] bin n o  $\cdot$ Orilus] vrilus o \textbf{5} muose sicherheit] muͯsse sicherheit m sicherheit muͦsz n sicherheit muͯs o  $\cdot$ geben] gegen m \textbf{7} sô] die n  $\cdot$ hie] dir hie n dir o  $\cdot$ getân] gargetan o \textbf{8} ime] dir o \textbf{9} si] die o \textbf{10} den] der n  $\cdot$ serpant] serprant m \textbf{13} schulden klagen] schuld sag clagen m \textbf{17} geinre] gingre m gúre n  $\cdot$ swerz] swertz n (o) \textbf{19} küeneste] kuͯne o \textbf{23} dienest sîn] dienest min vnd sin m \textbf{24} benamen] by nammer n \textbf{25} bitte] bittet n o  $\cdot$ sînen] sin n o \textbf{26} slege] slahe o \textbf{29} weiz er] Wieisser o  $\cdot$ bestât] verstat o \newline
\end{minipage}
\end{table}
\newpage
\begin{table}[ht]
\begin{minipage}[t]{0.5\linewidth}
\small
\begin{center}*G
\end{center}
\begin{tabular}{rl}
 & der vürste kniete \textbf{vür die} maget.\\ 
 & er sprach: "dû hâst al wâr gesaget,\\ 
 & ich bin\textbf{z}, dîn bruoder Orillus.\\ 
 & der rôte rîter twanc mich sus,\\ 
5 & daz ich dir \textbf{sicherheit muoz} geben.\\ 
 & dâ mit \textbf{koufte} ich dô mîn leben.\\ 
 & \textbf{die} enpfâch, sô wirt hie gar getân,\\ 
 & als ich gein im gelobet hân."\\ 
 & \textbf{dô} enpfie si triuwe in wîze hant\\ 
10 & von \textbf{im}, der truoc den serpant,\\ 
 & unt lie in ledic. dô daz geschach,\\ 
 & dô stuont er ûf unde sprach:\\ 
 & "ich sol unde muoz \textbf{mit triuwen} klagen:\\ 
 & \textbf{owê}, wer hât dich geslagen?\\ 
15 & dîne slege tuont mir \textbf{niender} wol,\\ 
 & wirt \textbf{es} zît, daz ich \textbf{dich} rechen sol,\\ 
 & ich \textbf{innere} den, \textbf{swer}z ruochet sehen,\\ 
 & daz mir grôz leit ist dran geschehen.\\ 
 & ouch hilfet mirz klagen der küenste man,\\ 
20 & den muoter ie zer werlt gewan.\\ 
 & \textbf{\textit{d}er} nennet sich \textbf{den} rîter rôt.\\ 
 & hêr künic \textbf{unde} vrou künigîn, er enbôt\\ 
 & iu beidentsament \textbf{den} dienst sîn,\\ 
 & dar zuo benamen der swester mîn.\\ 
25 & er \textbf{bit} \textbf{iuch sîn dienst} letzen\\ 
 & \textit{\textbf{und}} \textbf{die} maget \textbf{ir} slege ergetzen.\\ 
 & ouch het ich \textbf{es} dô genozzen\\ 
 & \begin{large}G\end{large}ein dem helde unverdrozzen,\\ 
 & wesser, wie si mich bestêt\\ 
30 & unde mir ir leit ze herzen gêt."\\ 
\end{tabular}
\scriptsize
\line(1,0){75} \newline
G I O L M Q R Z Fr36 \newline
\line(1,0){75} \newline
\textbf{1} \textit{Initiale} O L Q  \textbf{11} \textit{Initiale} I  \textbf{22} \textit{Initiale} Fr36  \textbf{25} \textit{Initiale} Z  \textbf{28} \textit{Initiale} G  \textbf{29} \textit{Initiale} M  \newline
\line(1,0){75} \newline
\textbf{1} der] ÷er O  $\cdot$ kniete] sprach kniet Q [s]: knúwte R kniet Z Fr36  $\cdot$ vür die] vor der Q R al da fvr die Z fuͤr div Fr36 \textbf{2} hâst] het R \textbf{3} binz] bin O L  $\cdot$ Orillus] Orilus I (O) (M) (Q) R (Z) (Fr36) \textbf{4} rôte rîter] Ritter Rot R  $\cdot$ twanc] betwanc I (O) M (Q) (R) Z \textbf{5} dir] \textit{om.} R  $\cdot$ sicherheit muoz] sicherheit muͤz I muͯsz Fiantze L sicherheit muͯse R (Fr36) \textbf{6} koufte] :auff Fr36  $\cdot$ dô] \textit{om.} O L M Q Fr36 da Z  $\cdot$ leben] lieben R \textbf{7} enpfâch] :::fahe Fr36  $\cdot$ gar] \textit{om.} I M Fr36 \textbf{8} gein] \textit{om.} I  $\cdot$ gelobet] gebetten R \textbf{9} dô] Da M Z  $\cdot$ enpfie] nam L  $\cdot$ in] vnde O (Fr36) \textbf{10} im] dem O  $\cdot$ der] der da I ouch M  $\cdot$ den] der Z \textbf{11} unt] Si I  $\cdot$ dô] da M Z \textbf{12} dô] Da M Z \textbf{13} muoz] muͤz I \textbf{15} tuont] die tuͤnt I  $\cdot$ niender] niht O L (M) (Q) (R) nimmer Z  $\cdot$ wol] wl R \textbf{16} es] des I (M) (Z) sin O \textit{om.} L  $\cdot$ rechen] reche M \textbf{17} innere den] geunere in I ginres des den O gynrre dich L geynner den M (Z) geinres den Q dinge des R :::inres den Fr36  $\cdot$ swerz] swenne irz I swer O wer ez L (R) ders M Fr36 der Q  $\cdot$ ruochet] geruchet M \textbf{18} mir grôz leit ist dran] leide dran mir ist O (Q) mir leide ist dran L (M) mir leide daran ist R lait mir dar an ist Fr36 \textbf{19} ouch] Da M  $\cdot$ mirz] mir I O  $\cdot$ klagen] rechen R \textbf{20} ie zer werlt] îe O (M) (Fr36) ye zu [kunste]: kint  Q ye zu kind R vf werlde ie Z \textbf{21} \textit{Versfolge 276.22-21} Fr36   $\cdot$ der] er G  $\cdot$ den] der L M Q R Z \textbf{22} Dem kung der kᵫnginnen er enbot R  $\cdot$ unde] \textit{om.} O L Fr36  $\cdot$ vrou] \textit{om.} I M  $\cdot$ enbôt] iv enbot O (Fr36) yn bot M \textbf{23} beidentsament] beiden L  $\cdot$ sîn] [min]: sin G \textbf{25} er] ir I Vnd L (Q)  $\cdot$ sîn] sinen L (Z) (Fr36) \textbf{26} und] \textit{om.} G  $\cdot$ ir slege] \textit{om.} I der slege O  $\cdot$ ergetzen] ertzeget Q \textbf{27} ouch] ir slege auch I  $\cdot$ het ich es] hiet si O (L) (Q) (R) (Fr36) hatte sy M  $\cdot$ dô] da M Z doch Q \textit{om.} Fr36 \textbf{28} helde] helden R  $\cdot$ unverdrozzen] verdroszen L \textbf{29} wesser] west ir I Wesser er M \textbf{30} unde] vnd daz I  $\cdot$ ze herzen] so nahen Q \newline
\end{minipage}
\hspace{0.5cm}
\begin{minipage}[t]{0.5\linewidth}
\small
\begin{center}*T
\end{center}
\begin{tabular}{rl}
 & \begin{large}D\end{large}er vürste kniete \textbf{vür die} maget.\\ 
 & er sprach: "dû hâst al wâr gesaget,\\ 
 & ich bin dîn bruoder Orilus.\\ 
 & der rôte rîter twanc mich sus,\\ 
5 & daz ich dir \textbf{sicherheit muoz} geben.\\ 
 & dâ mit \textbf{kouft}ich dô mîn leben.\\ 
 & \textbf{daz} enpfâch, sô wirt hie gar getân,\\ 
 & als ich gegen im gelobt hân."\\ 
 & \textbf{Dô} enpfienc si triuwe in wîze hant\\ 
10 & von \textbf{dem}, der truoc den serpant,\\ 
 & unde liez in ledic. dô daz geschach,\\ 
 & Dô stuont er ûf unde sprach:\\ 
 & "Ich sol unde muoz \textbf{durch triuwe} klagen:\\ 
 & \textbf{ouwê}, wer hât dich geslagen?\\ 
15 & dîne slege tuont mir \textbf{niemer} wol,\\ 
 & wirt zît, daz ich \textit{\textbf{dich}} rechen sol,\\ 
 & ich \textbf{geunêre} den, \textbf{der}z ruochet sehen,\\ 
 & daz mir grôz leit ist dran geschehen.\\ 
 & ouch hilfet mirz klagen der küeneste man,\\ 
20 & den muoter ie zer werlt gewan.\\ 
 & \textbf{er} nennet sich '\textbf{der} rîter rôt'.\\ 
 & hêr künec, vrou künegîn, er enbôt\\ 
 & iu beiden samt \textbf{den} dienst sîn,\\ 
 & dar zuo benamen der swester mîn.\\ 
25 & er \textbf{bittet} \textbf{iuch sînen dienst} letzen,\\ 
 & \textbf{die} magt \textbf{der} slege ergetzen.\\ 
 & ouch hetich\textbf{z} dô genozzen\\ 
 & gegen dem helde unverdrozzen,\\ 
 & wesser, wie si mich bestêt\\ 
30 & unde mir ir leit ze herzen gêt."\\ 
\end{tabular}
\scriptsize
\line(1,0){75} \newline
T U V W \newline
\line(1,0){75} \newline
\textbf{1} \textit{Initiale} T U V W  \textbf{9} \textit{Majuskel} T  \textbf{12} \textit{Majuskel} T  \textbf{13} \textit{Majuskel} T  \newline
\line(1,0){75} \newline
\textbf{2} al] al zuͦ U \textbf{5} geben] duͦn U \textbf{6} kouftich] [k*]: kaufte ich U  $\cdot$ mîn] das W \textbf{7} daz] [D*]: Die V Die W \textbf{8} gegen] \textit{om.} W \textbf{10} dem der] im der do W  $\cdot$ den] der U \textbf{14} ouwê] [*]: Swester V \textbf{16} wirt] [*]: Wurt ez V Wirt mir W  $\cdot$ ich dich] ich T [*]: ich die V \textbf{17} Jch genere den ders geruͦchen sehen U  $\cdot$ [*]: Jch gedinge des swers ruͦchet sehen V  $\cdot$ Ich bring ins innen der es geruͦcht sehen W \textbf{18} ist dran] dar an ist W  $\cdot$ geschehen] beschehen V \textbf{19} mirz] mir iz U \textbf{20} den] Die U  $\cdot$ ie] \textit{om.} W  $\cdot$ gewan] ie gewan W \textbf{21} er] [*]: Der V \textbf{23} iu beiden samt] Jn beidesamt U \textbf{24} zuo] \textit{om.} W \textbf{25} er] Der W  $\cdot$ bittet] buͦtet U bitten W  $\cdot$ iuch] îv T \textit{om.} W  $\cdot$ letzen] eúch letzen W \textbf{26} die] [D*]: Dise V Der W  $\cdot$ der] [*]: ir V \textbf{27} hetichz] hete ich iz U het ich es V W \textbf{30} unde] Wann W  $\cdot$ leit] ser V \newline
\end{minipage}
\end{table}
\end{document}
