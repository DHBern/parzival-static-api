\documentclass[8pt,a4paper,notitlepage]{article}
\usepackage{fullpage}
\usepackage{ulem}
\usepackage{xltxtra}
\usepackage{datetime}
\renewcommand{\dateseparator}{.}
\dmyyyydate
\usepackage{fancyhdr}
\usepackage{ifthen}
\pagestyle{fancy}
\fancyhf{}
\renewcommand{\headrulewidth}{0pt}
\fancyfoot[L]{\ifthenelse{\value{page}=1}{\today, \currenttime{} Uhr}{}}
\begin{document}
\begin{table}[ht]
\begin{minipage}[t]{0.5\linewidth}
\small
\begin{center}*D
\end{center}
\begin{tabular}{rl}
\textbf{785} & \textit{\begin{large}E\end{large}}r vrâgete den künec Gramoflanz,\\ 
 & ob diu \textbf{liebe} wære ganz\\ 
 & zwischen im unt der nifteln sîn,\\ 
 & daz er daz tæte an im \textbf{nû} schîn:\\ 
5 & "helfet ir unt \textbf{mîn neve} Gawan,\\ 
 & swaz wir \textbf{hie} künege \textbf{unt} vürsten hân,\\ 
 & \textbf{barûne} unt \textbf{arme} \textbf{rîter} gar,\\ 
 & daz \textbf{der} decheiner hinnen var,\\ 
 & ê si mîn kleinœde ersehen.\\ 
10 & mir wære ein laster \textbf{hie} geschehen,\\ 
 & schiede ich vor gâbe hinnen vrî.\\ 
 & swaz hie varendes volkes sî,\\ 
 & die warten alle \textbf{gâbe} an mich.\\ 
 & Artus, \textbf{nû wil ich} bitten dich,\\ 
15 & \textbf{deiz} den hôhen niht versmâhe,\\ 
 & \textbf{des} gewerbes \textbf{gein in gâhe}\\ 
 & unt wis des lasters vür \textbf{si} pfant\\ 
 & - si erkanten nie sô rîche hant -\\ 
 & unt gib mir boten \textbf{in} mîne habe,\\ 
20 & dâ der prêsente sol komen abe."\\ 
 & \textbf{si lobten} dem heiden,\\ 
 & si\textbf{ne} wolten sich niht scheiden\\ 
 & von dem velde \textbf{in} vier tagen.\\ 
 & der heiden wart vrô, \textbf{sus} hôrt ich sagen.\\ 
25 & Artus im wîse boten gap,\\ 
 & die er solde senden \textbf{an}z hap.\\ 
 & \textbf{Feirefiz}, Gahmuretes kint,\\ 
 & \textbf{nam} tincten unt permint.\\ 
 & \textbf{sîn schrift wârzeichens} niht verdarp.\\ 
30 & ich wæne, ie brief sô vil erwarp.\\ 
\end{tabular}
\scriptsize
\line(1,0){75} \newline
D \newline
\line(1,0){75} \newline
\textbf{1} \textit{Initiale} D  \newline
\line(1,0){75} \newline
\textbf{1} Er] ÷r D \textbf{27} Gahmuretes] Gahmvrets D \newline
\end{minipage}
\hspace{0.5cm}
\begin{minipage}[t]{0.5\linewidth}
\small
\begin{center}*m
\end{center}
\begin{tabular}{rl}
 & er vrâgte den künic Gramolanz,\\ 
 & ob diu \textbf{liebe} wær \textbf{sô} ganz\\ 
 & zwischen im und der nifteln sîn,\\ 
 & daz er daz tæte an im schîn:\\ 
5 & "helfet ir \textbf{mir} und \textbf{mîn neve} Gawan,\\ 
 & waz wir \textbf{hie} künige \textbf{und} vürsten hân,\\ 
 & \textbf{barûne} und \textbf{armer} \textbf{ritter} g\textit{a}r,\\ 
 & daz \textbf{der} dekeiner hinnen var,\\ 
 & \textit{ê si mîn kleinœte ersehen}.\\ 
10 & mir wær ein laster \textbf{hie} geschehen,\\ 
 & schied ich vor gâbe hinnen vrî.\\ 
 & waz hie v\textit{ar}ndes volkes sî,\\ 
 & die warten alle \textbf{gâbe} an mich.\\ 
 & \textbf{neve} Artus, \textbf{ich wil} bitten dich,\\ 
15 & \textbf{daz ez} de\textit{n h}ôhen niht versmâhe,\\ 
 & \textbf{des} gewerbes \textbf{gegen in gâhe}\\ 
 & und wis des lasters vür \textbf{mich} pfant\\ 
 & - si erkanten nie sô rîche hant -\\ 
 & und gip mir boten \textbf{in} mîn habe,\\ 
20 & d\textit{â} der prêsent sol komen abe."\\ 
 & \textbf{d\textit{ô} lobten si} dem heiden,\\ 
 & si wolte\textit{n} sich niht scheiden\\ 
 & von dem velde \textbf{in} vier tagen.\\ 
 & der heiden wart vrô, \textbf{sus} h\textit{ô}r\textit{t} ich sagen.\\ 
25 & Artus im wîse boten gap,\\ 
 & die er solte senden \textbf{an} daz hap.\\ 
 & \textbf{Ferefiz}, Gahmuretes kint,\\ 
 & \textbf{nam} tinten und birmint.\\ 
 & \textbf{sîn geschrift wortzeichens} niht verdarp.\\ 
30 & ich wæne, ie brief sô vil erwarp.\\ 
\end{tabular}
\scriptsize
\line(1,0){75} \newline
m n o V V' W Fr6 \newline
\line(1,0){75} \newline
\textbf{1} \textit{Initiale} W  \textbf{21} \textit{Initiale} V V' Fr6  \newline
\line(1,0){75} \newline
\textbf{1} vrâgte] fraget V'  $\cdot$ Gramolanz] gramolantz m n gramolancz o Gramaflanz V [gr*aflantz]: granaflancz V' gramoflantz W Gramovlanz Fr6 \textbf{2} liebe] liebú V  $\cdot$ sô] \textit{om.} V V' Fr6 \textbf{3} nifteln] nifttel m (n) (o) (W) \textbf{4} im] imme nv V (V') (Fr6) \textbf{5} helfet] Helffert o  $\cdot$ ir mir] mir V V' mir ir Fr6  $\cdot$ Gawan] gawann o \textbf{6} waz] Swaz V (Fr6)  $\cdot$ wir] ir W \textbf{7} armer] arme o Fr6 [a*]: ander V ander V'  $\cdot$ ritter] ritte o  $\cdot$ gar] ger m o \textbf{8} dekeiner] do keiner n \textbf{9} \textit{Vers 785.9 fehlt} m n o   $\cdot$ \textit{Versfolge 785.10-9} W  \textbf{10} \textit{nach 785.10:} Das mans fuͯr schand mohte jehen m   $\cdot$ geschehen] beschehen n o W \textbf{11} vor] fúr o von V'  $\cdot$ gâbe] goben V (V') \textbf{12} waz] [War]: Was n Swaz V (Fr6)  $\cdot$ varndes] frandes m \textbf{13} die] Sú n  $\cdot$ warten] wartent n V W  $\cdot$ an] her an V' \textbf{14} neve] \textit{om.} V'  $\cdot$ Artus] artuͯs o arrus W \textbf{15} den hôhen] den heiden hohen m \textbf{16} in] in im W \textbf{20} dâ] Do m n o V V' W  $\cdot$ sol] solt o \textbf{21} dô] Da m o  $\cdot$ lobten] gelobeten n [b]: gelovbten V'  $\cdot$ dem] den o W \textbf{22} si] sine Fr6  $\cdot$ wolten] woltte m \textbf{23} velde] welde o \textbf{24} sus] \textit{om.} V'  $\cdot$ hôrt] her m hoͯre n (o) \textbf{25} wîse] wisen n \textbf{26} an daz] in den W \textbf{27} Ferefiz] Ferefis m o Ferrefis n Ferevis V V' Ferafis W  $\cdot$ Gahmuretes] gahmurettes m gamiretes n gamuͯretez o Gamurettes V kamereten V' gamuretes W Gahmvretes Fr6 \textbf{28} tinten und birmint] birment vnd thint W \textbf{29} geschrift] schrift V V' (W) Fr6  $\cdot$ wortzeichens] wurczeichens o warzeichens V' (W) (Fr6) \textbf{30} vil] wol V'  $\cdot$ erwarp] [verdarp]: erwarp o erwarf V' \newline
\end{minipage}
\end{table}
\newpage
\begin{table}[ht]
\begin{minipage}[t]{0.5\linewidth}
\small
\begin{center}*G
\end{center}
\begin{tabular}{rl}
 & \begin{large}E\end{large}r vrâgte den künic Gramoflanz,\\ 
 & op diu \textbf{suone} wære ganz\\ 
 & zwischen im unde der nifteln sîn,\\ 
 & daz er daz tæte an im schîn:\\ 
5 & "helft ir unde Gawan,\\ 
 & swaz wir künige \textbf{oder} vürsten hân,\\ 
 & \textbf{Britun} unde \textbf{ander} \textbf{vürsten} gar,\\ 
 & daz \textbf{der} deheiner hinnen var,\\ 
 & ê si mîn kleinœde ersehen.\\ 
10 & mir wære ein laster \textbf{dran} geschehen,\\ 
 & schiede ich vor gâbe hinnen vrî.\\ 
 & swaz hie varndes volkes sî,\\ 
 & die warten alle \textbf{gâbe} an mich.\\ 
 & Artus, \textbf{nû wil ich} biten dich,\\ 
15 & \textbf{daz} den hôhen niht versmâh\textit{e}\\ 
 & \textbf{mînes} gewerbes \textbf{gâbe}\\ 
 & unde wis des lasters vür \textbf{si} pfant\\ 
 & - si\textbf{ne} erkanten nie sô rîch\textit{e} \textit{h}ant -\\ 
 & unde gip mir boten \textbf{in} mîne habe,\\ 
20 & dâ der prêsent sol komen abe."\\ 
 & \textbf{dô enbuten si} dem heiden,\\ 
 & si\textbf{ne} wolden sich niht scheiden\\ 
 & von dem velde \textbf{inner} vier tagen.\\ 
 & der heiden wart vrô, hôrt ich sagen.\\ 
25 & Artus im wîse boten gap,\\ 
 & die er solde senden \textbf{in} daz hap.\\ 
 & \textbf{dô nam} Gahmuretes kint\\ 
 & tinten unde bermint.\\ 
 & \textbf{sîner schrift wârzeichen} niht verdarp.\\ 
30 & ich wæne, ie brief sô vil erwarp.\\ 
\end{tabular}
\scriptsize
\line(1,0){75} \newline
G I L M Z \newline
\line(1,0){75} \newline
\textbf{1} \textit{Initiale} G L Z  \textbf{9} \textit{Initiale} I  \newline
\line(1,0){75} \newline
\textbf{1} vrâgte] fragt Z  $\cdot$ Gramoflanz] gramoflantz Z \textbf{2} suone] [svne]: svnne M liebe Z \textbf{3} nifteln] niftel L Z \textbf{5} helft ir] Vnde helfit mir M \textbf{6} swaz] Waz L (M)  $\cdot$ wir] wer M \textbf{7} Baron vnd die andern ritter gar Z  $\cdot$ Britun] Brýtvne L (M) \textbf{8} der] \textit{om.} I \textbf{9} \textit{Die Verse 785.9-10 fehlen} L   $\cdot$ ersehen] sehen I gesen M \textbf{11} Jm en sẏ E myn gabe bý L  $\cdot$ vor] von I M \textbf{12} swaz] Waz L (M) \textbf{13} warten] [waren]: warten G wartent Z \textbf{15} versmâhe] versmahen G \textbf{16} mînes] min I (M) Des Z  $\cdot$ gâbe] gahe M gein im iahe Z \textbf{18} sine] Sy L (M) (Z)  $\cdot$ rîche hant] rihiv lant G \textbf{20} der prêsent] die presente L \textbf{21} dô] Da M Z  $\cdot$ enbuten] lobten Z \textbf{22} sine wolden] sin enwolten I  $\cdot$ sich] \textit{om.} L \textbf{23} inner] in L (M) Z \textbf{24} der heiden wart] Des wart er L Der heide wart M  $\cdot$ hôrt] hoͤr I \textbf{25} im] im im I \textbf{26} in] an L (M) Z \textbf{27} dô] Da M Z  $\cdot$ Gahmuretes] Gahmuͯretes L Gamuretes M gamureten Z \textbf{29} schrift] schrif Z  $\cdot$ wârzeichen] wortzeichen L \textbf{30} ie] nie I \newline
\end{minipage}
\hspace{0.5cm}
\begin{minipage}[t]{0.5\linewidth}
\small
\begin{center}*T
\end{center}
\begin{tabular}{rl}
 & er vrâgete den künec Gramoflanz,\\ 
 & ob diu \textbf{liebe} wære ganz\\ 
 & zwischen im und der nifteln sîn,\\ 
 & daz er daz tæte an im schîn:\\ 
5 & "helfet ir und Gawan,\\ 
 & waz wir künege \textbf{oder} vürsten hân,\\ 
 & \textbf{barûne} und \textbf{armer} \textbf{rîter} gar,\\ 
 & daz \textbf{ir} dekeine\textit{r} h\textit{inne}n \textbf{en}var,\\ 
 & ê si mîn kleinœde ersehen.\\ 
10 & mir wære ein laster \textbf{hie} geschehen,\\ 
 & schiede ich vor gâbe hinnen vrî.\\ 
 & waz hie varndes volkes sî,\\ 
 & die warten alle \textbf{gâben} an mich.\\ 
 & Artus, \textbf{nû wil ich} b\textit{i}ten dich,\\ 
15 & \textbf{daz ez} den hôhen niht versmâhe,\\ 
 & \textbf{des} gewerbes \textbf{gein i\textit{n} gâhe}\\ 
 & \textit{und bis des lasters vür \textbf{si} pfant}\\ 
 & \textit{- si erkanten nie sô rîche hant -}\\ 
 & \textit{und gib mir boten \textbf{an} mîne habe,}\\ 
20 & d\textit{â} der prîsant sol komen abe."\\ 
 & \textbf{d\textit{ô en}b\textit{u}ten si} dem heiden,\\ 
 & si \textbf{en}wolten sich niht scheiden\\ 
 & von dem velde \textbf{in} vier tagen.\\ 
 & der heiden wart vrô, hôrt ich sagen.\\ 
25 & Artus im wîse boten gap,\\ 
 & die er solte senden \textbf{in} daz hap.\\ 
 & \textbf{Ferefis}, Gahmuretes kint,\\ 
 & \textbf{nam} tinten und permint.\\ 
 & \textbf{sîner schrifte wârzeichen} niht verdarp.\\ 
30 & ich wæne, ie brief sô vil erwarp.\\ 
\end{tabular}
\scriptsize
\line(1,0){75} \newline
U Q R \newline
\line(1,0){75} \newline
\textbf{1} \textit{Initiale} R  \newline
\line(1,0){75} \newline
\textbf{1} \textit{Die Verse 784.9-789.19 fehlen} Q   $\cdot$ vrâgete] fragt R  $\cdot$ Gramoflanz] Gramoflancz R \textbf{3} nifteln] nyftel R \textbf{5} und] \textit{om.} R \textbf{6} wir] \textit{om.} R  $\cdot$ vürsten] fúrste R \textbf{7} barûne und armer] Barum vnd ander R \textbf{8} dekeiner] dekeine U  $\cdot$ hinnen] her in U  $\cdot$ envar] far R \textbf{9} ersehen] gesechen R \textbf{11} vor gâbe] von habe R \textbf{13} gâben] gabe R \textbf{14} biten] bieten U \textbf{16} in] im U \textbf{17} \textit{Die Verse 785.17-19 fehlen} U  \textbf{20} dâ] Do U \textbf{21} dô enbuten] Dem boten U \textbf{22} enwolten] woͯltent R \textbf{23} in] inrent R \textbf{26} daz] die R \textbf{27} Ferefis] Feirefis R  $\cdot$ Gahmuretes] Gahmuͦretes U Gahmurtes R \textbf{28} tinten] tinte U túmpten R \textbf{29} sîner schrifte wârzeichen] Siner geschrifftt verczeichnent R \newline
\end{minipage}
\end{table}
\end{document}
