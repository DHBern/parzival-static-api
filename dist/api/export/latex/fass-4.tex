\documentclass[8pt,a4paper,notitlepage]{article}
\usepackage{fullpage}
\usepackage{ulem}
\usepackage{xltxtra}
\usepackage{datetime}
\renewcommand{\dateseparator}{.}
\dmyyyydate
\usepackage{fancyhdr}
\usepackage{ifthen}
\pagestyle{fancy}
\fancyhf{}
\renewcommand{\headrulewidth}{0pt}
\fancyfoot[L]{\ifthenelse{\value{page}=1}{\today, \currenttime{} Uhr}{}}
\begin{document}
\begin{table}[ht]
\begin{minipage}[t]{0.5\linewidth}
\small
\begin{center}*D
\end{center}
\begin{tabular}{rl}
\textbf{4} & vröud und angest vert dâ bî.\\ 
 & nû lât mîn eines wesen drî,\\ 
 & Der ieslîcher sunder pflege,\\ 
 & daz mîner künste w\textit{ide}rwege.\\ 
5 & dar zuo \textbf{gehôrte} wilder vunt,\\ 
 & ob si \textbf{iu} gerne tæten kunt,\\ 
 & daz ich iu eine kunden wil.\\ 
 & si het\textit{en} arbeite vil.\\ 
 & Ein mære \textbf{ich iu wil} niwen,\\ 
10 & daz seit von \textbf{grôzen} triwen,\\ 
 & wîplîchez wîbes reht\\ 
 & unt mannes manheit alsô sleht,\\ 
 & diu sich gein herte nie gebouc.\\ 
 & sîn herze \textbf{in} dâr an \textbf{niht} betrouc,\\ 
15 & er stahel, swâ er ze strîte kam.\\ 
 & sîn hant dâ sigelîchen nam\\ 
 & \textbf{vil} manegen lobelîchen prîs;\\ 
 & \textbf{er} küene, \textbf{træclîche} wîs:\\ 
 & den helt ich alsus grüeze,\\ 
20 & \textbf{er} wîbes ougen süeze\\ 
 & und dâ bî wîbes herzen suht,\\ 
 & vor missewende ein wâriu vluht.\\ 
 & den ich hie \textbf{zuo hân} erkorn,\\ 
 & \textbf{er} ist mæreshalp noch ungeborn,\\ 
25 & dem man dirre âventiure giht\\ 
 & unt wunders vil, des dran geschiht.\\ 
 & \textit{\begin{large}S\end{large}}i pflege\textit{nt}\textbf{s} noch, als man\textbf{s} dô pflac,\\ 
 & swâ lît unt walsch gerihte lac.\\ 
 & \textbf{des} pfliget \textbf{ouch} tiuscher erde ein ort.\\ 
30 & daz ha\textit{b}t ir \textbf{âne mich} gehôrt.\\ 
\end{tabular}
\scriptsize
\line(1,0){75} \newline
D \newline
\line(1,0){75} \newline
\textbf{3} \textit{Versal} D  \textbf{9} \textit{Versal} D  \textbf{27} \textit{Initiale} D  \newline
\line(1,0){75} \newline
\textbf{1} vröud] frovd \textit{nachträglich korrigiert zu:} froͤvd D \textbf{4} widerwege] w:::rwege \textit{nachträglich korrigiert zu:} widerwege D \textbf{8} heten] hett: \textit{nachträglich korrigiert zu:} hettú D \textbf{27} Si] ÷ie D  $\cdot$ pflegents] pflegetns D \textbf{28} walsch] welich D \textbf{30} habt] ha:t D \newline
\end{minipage}
\hspace{0.5cm}
\begin{minipage}[t]{0.5\linewidth}
\small
\begin{center}*m
\end{center}
\begin{tabular}{rl}
 & \multicolumn{1}{l}{ - - - }\\ 
 & \multicolumn{1}{l}{ - - - }\\ 
 & \multicolumn{1}{l}{ - - - }\\ 
 & \multicolumn{1}{l}{ - - - }\\ 
5 & \multicolumn{1}{l}{ - - - }\\ 
 & \multicolumn{1}{l}{ - - - }\\ 
 & \multicolumn{1}{l}{ - - - }\\ 
 & \multicolumn{1}{l}{ - - - }\\ 
 & \begin{large}E\end{large}in mære \textbf{ich hie wil} niuwen,\\ 
10 & daz seit von \textbf{grôzen} triuwen,\\ 
 & wîplîche\textit{z} wîbes reht\\ 
 & und mannes manheit alsô sl\textit{e}ht,\\ 
 & diu si\textit{ch} gegen herte nie gebouc.\\ 
 & sîn herz \textbf{in} dâr an \textbf{nie} betrouc,\\ 
15 & er stahel, wâ er zuo strîte kam.\\ 
 & sîn hant d\textit{â} sigelîchen nam\\ 
 & m\textit{a}nigen loblîchen prîs;\\ 
 & \textbf{der} küene, \textbf{træclîch\textit{e}} wîs:\\ 
 & den helt ich alsus grüeze,\\ 
20 & \textbf{der} wîbes ougen süeze\\ 
 & und dâ bî wîbes herzen suht,\\ 
 & vor missewende ei\textit{n} \dag varwe\dag  vluht.\\ 
 & den ich hi\textit{e} \textbf{hân zuo} erk\textit{or}n,\\ 
 & \textbf{er} ist \dag megenshalp\dag  noch ungeborn,\\ 
25 & de\textit{m} man dirre âventiure giht\\ 
 & und wunders vil, d\textit{e}s dâr an g\textit{e}s\textit{ch}iht.\\ 
 & \begin{large}S\end{large}i p\textit{fl}egent \textbf{es} noch, als man dô pflac,\\ 
 & wâ l\textit{î}t und welsch gerihte lac.\\ 
 & \textbf{daz} p\textit{f}l\textit{i}g\textit{e}t \textbf{ouch} tiutscher erde ein ort.\\ 
30 & daz habet ir \textbf{an mir} gehôrt.\\ 
\end{tabular}
\scriptsize
\line(1,0){75} \newline
m n o W \newline
\line(1,0){75} \newline
\textbf{9} \textit{Initiale} m W   $\cdot$ \textit{Capitulumzeichen} n  \textbf{27} \textit{Initiale} m   $\cdot$ \textit{Capitulumzeichen} n  \newline
\line(1,0){75} \newline
\textbf{1} \textit{Die Verse 3.25-4.8 fehlen} m n o W  \textbf{9} Ein] Eine n \textbf{11} wîplîchez] Wiplicher m n o (W) \textbf{12} sleht] slescht m \textbf{13} sich] sÿ m sú sich n (o)  $\cdot$ herte] herte \textit{nachträglich korrigiert zu:} hertz m \textbf{14} an] \textit{om.} o \textbf{15} er stahel] Er stahel \textit{nachträglich korrigiert zu:} stach m Gestahel n  $\cdot$ wâ er] wo der n  $\cdot$ kam] kan W \textbf{16} dâ] do m n o W \textbf{17} \textit{Versdoppelung:} Manigen lobelichen mann / Maniger lobelich pris o   $\cdot$ manigen] Minnichen \textit{nachträglich korrigiert zu:} Minnclichen m  $\cdot$ loblîchen] lobelich o \textbf{18} træclîche] trachlicher m trachelichen n o trachenliche W \textbf{19} alsus] alsus ir o \textbf{21} suht] sicht W \textbf{22} missewende ein] miswende eÿ m misse >wende eyn< o  $\cdot$ vluht] schlicht W \textbf{23} hie] [han]: hÿre m  $\cdot$ zuo] \textit{om.} n o W  $\cdot$ erkorn] erkenen m \textbf{24} megenshalp] megens half \textit{nachträglich korrigiert zu:} medenn half m morgens halp n o (W) \textbf{25} dem] Den m \textbf{26} des] das m \textit{om.} n o W  $\cdot$ an geschiht] angslicht m an beschicht W \textbf{27} pflegent] pliegend m pflegen n [plegen]: pflegen o  $\cdot$ als] [es]: als m  $\cdot$ dô] \textit{om.} n o es W \textbf{28} lît] lut \textit{nachträglich korrigiert zu:} dutsch m lúte n (o) (W)  $\cdot$ welsch] welsche n wellich W \textbf{29} pfliget] pliegend m  $\cdot$ ouch] \textit{om.} W  $\cdot$ erde ein] eren n erd o (W)  $\cdot$ ort] art n W aet o \textbf{30} gehôrt] gehart n W \newline
\end{minipage}
\end{table}
\newpage
\begin{table}[ht]
\begin{minipage}[t]{0.5\linewidth}
\small
\begin{center}*G
\end{center}
\begin{tabular}{rl}
 & vröude und angest vert dâ bî.\\ 
 & nû lât mîn eines wesen drî,\\ 
 & der ieslîcher sunder pflege,\\ 
 & daz mîner künste widerwege.\\ 
5 & dar zuo \textbf{gehôrte} wilder vunt,\\ 
 & op si \textbf{iu} gerne tæten kunt,\\ 
 & daz ich iu eine künden wil.\\ 
 & si heten arbeite \textit{v}il.\\ 
 & \begin{large}E\end{large}in mære \textbf{wil ich iu} niwen,\\ 
10 & daz seit von \textbf{ganzen} triwen,\\ 
 & wîplîchez wîbes reht\\ 
 & und mannes manheit als sleht,\\ 
 & diu sich gein herte nie gebouc.\\ 
 & sîn herze \textbf{in} dâr an \textbf{niht} be\textit{trouc},\\ 
15 & er stahel, swâ er ze strîte kam.\\ 
 & sîn hant dâ sigelîchen nam\\ 
 & \textbf{vil} manigen lobelîchen brîs;\\ 
 & \textbf{er} küene, \textbf{træclîchen} wîs:\\ 
 & den helt ich alsus grüeze,\\ 
20 & \textbf{er} wîbes ougen süeze\\ 
 & unt dâ bî wîbes herzen suht,\\ 
 & vor missewende ein wâriu vluht.\\ 
 & den ich hie \textbf{zuo hân} erkoren,\\ 
 & \textbf{der} ist mæreshalp noch ungeboren,\\ 
25 & dem man dirre âventiure giht\\ 
 & unde wunders vil, des dran geschiht.\\ 
 & si pfl\textit{e}gent noch, als man dô pflac,\\ 
 & swâ lît und walhisch gerihte lac.\\ 
 & \textbf{des} pfliget \textbf{noch} tiuscher erde ein ort.\\ 
30 & daz habet ir \textbf{âne mich} gehôrt.\\ 
\end{tabular}
\scriptsize
\line(1,0){75} \newline
G O L M Q Z Fr32 Fr58 \newline
\line(1,0){75} \newline
\textbf{1} \textit{Initiale} O  \textbf{9} \textit{Initiale} G  \textbf{27} \textit{Überschrift:} Hy fur gaműert vber mer Q   $\cdot$ \textit{Initiale} L Q Z  \newline
\line(1,0){75} \newline
\textbf{1} vröude] Not Q  $\cdot$ vert] ouch M \textbf{2} mîn] im Q  $\cdot$ drî] [brey]: drey Q \textbf{3} pflege] lege M \textbf{4} daz mîner] So wan her M Das minner Q \textbf{5} gehôrte] gehoret O (M) (Z) horat L \textbf{6} iu] \textit{om.} L  $\cdot$ tæten] thede M \textbf{7} iu] \textit{om.} Z \textbf{8} vil] :il G \textbf{9} niwen] Muwen M \textbf{10} ganzen] grozzen O (L) (M) (Q) Z \textbf{11} reht] [muͦt]: reht O \textbf{12} mannes manheit] manheit mannes O \textbf{13} sich gein herte] sie geherte Q \textbf{14} sîn] Myn M  $\cdot$ in] \textit{om.} Q  $\cdot$ betrouc] be::: G troch Q \textbf{15} stahel] streit L schel Q  $\cdot$ swâ] wo L da M siro Q  $\cdot$ strîte] in Q \textbf{16} dâ] \sout{hat} O  $\cdot$ sigelîchen] sicherlichen M (Q) \textbf{17} vil manigen] An mangem L \textbf{18} \textit{Vers 4.18 fehlt} L   $\cdot$ küene] kunde M \textbf{22} \textit{Vers 4.22 fehlt} Q  \textbf{23} hie] da O  $\cdot$ zuo hân] han zu Q \textbf{24} der] Er L \textbf{25} man] manne M  $\cdot$ dirre] diesse M (Q) \textbf{26} vil des dran] dar an vil L vil dasz dann Q \textbf{27} pflegent] phl:gent G pflegent [sin*]: sin Z pflegentz Fr32  $\cdot$ dô] \textit{om.} Q da Z Fr58 \textbf{28} swâ] Wa L So Q  $\cdot$ walhisch] walhesch G welhchs O walscher L valsch M Z welches Q welsc Fr32 walsch Fr58 \textbf{29} noch] \textit{om.} O M auch Q (Z) (Fr32)  $\cdot$ tiuscher] tuscher G tevscher O tvtscher L dutscher M der Q devtscher Z tivtscher Fr32 dutscher Fr58 \textbf{30} habet ir] ir [aabt]: habt Q  $\cdot$ âne mich] nach Z (Fr58)  $\cdot$ gehôrt] wol gehort O vngehort Z Fr58 \newline
\end{minipage}
\hspace{0.5cm}
\begin{minipage}[t]{0.5\linewidth}
\small
\begin{center}*T
\end{center}
\begin{tabular}{rl}
 & vröude und angest vert dâ bî.\\ 
 & nû lât mîn eines wesen drî,\\ 
 & der ieglîcher sunder pflege,\\ 
 & daz mîner künste widerwege.\\ 
5 & dar zuo \textbf{hôret} wilder vunt,\\ 
 & ob si gerne tæten kunt,\\ 
 & daz ich iu eine künden wil.\\ 
 & si heten arbeite vil.\\ 
 & Ein mære \textbf{wil ich iu} niuwen,\\ 
10 & daz saget von \textbf{grôzen} triuwen,\\ 
 & wîplîche\textit{z} wî\textit{be}s reht\\ 
 & und mannes manheit alsô sleht,\\ 
 & di\textit{u} sich geg\textit{e}n \textbf{der} herte nie gebouc.\\ 
 & sîn herze \textbf{niht} dâr an betrouc,\\ 
15 & er stahel, swâ er ze strîte kam.\\ 
 & sîn hant dâ sigelîchen nam\\ 
 & \textbf{vil} manegen lobelîchen prîs;\\ 
 & \textbf{er} küene, \textbf{stæte, milte}, wîs:\\ 
 & den helt ich alsus grüeze,\\ 
20 & \textbf{er} wîbes ougen süeze\\ 
 & und dâ bî wîbes herzen suht,\\ 
 & vor missewende ein wâr\textit{iu} vluht.\\ 
 & Den ich hie \textbf{zuo hân} erkorn,\\ 
 & \textbf{der}st mæreshalp noch ungeborn,\\ 
25 & dem man dirre âventiure giht\\ 
 & und wunders vil, des dran geschiht.\\ 
 & \begin{large}S\end{large}i pflegent\textit{\textbf{s}} noch, als man dô pflac,\\ 
 & swâ lît und welsch gerihte lac.\\ 
 & \textbf{des} pfliget \textbf{noch} tiuscher erde ein ort.\\ 
30 & daz habt ir \textbf{âne mich} gehôrt.\\ 
\end{tabular}
\scriptsize
\line(1,0){75} \newline
T U V Fr32 \newline
\line(1,0){75} \newline
\textbf{5} \textit{Versal} Fr32  \textbf{9} \textit{Majuskel} T  \textbf{23} \textit{Majuskel} T  \textbf{27} \textit{Initiale} T U V Fr32  \newline
\line(1,0){75} \newline
\textbf{1} vert] virt U \textbf{5} hôret] horte V \textbf{6} gerne] úch gerne V nv gerne Fr32  $\cdot$ tæten] tete Fr32 \textbf{8} heten] hede U hettent alle V \textbf{9} Ein] Eine V \textbf{10} saget] seit vns Fr32 \textbf{11} wîplîchez wîbes] wipliches wi::s T Wibes wipliches V wipliche: wibes Fr32 \textbf{13} diu] die T  $\cdot$ der] \textit{om.} U V Fr32 \textbf{14} niht dâr an] in nit dar ane U in dar an nie V iv niht dar an Fr32 \textbf{15} er stahel] Er waz stahel V ein stahel Fr32  $\cdot$ swâ er] swa ez Fr32 \textbf{16} dâ] vil V \textbf{18} er was kv̂ene vnd starch alle wîs Fr32 \textbf{19} alsus] also Fr32 \textbf{20} er] Er waz V \textbf{21} dâ] \textit{om.} U \textbf{22} wâriu vluht] ware vluht T *r*ht \textit{nachträglich korrigiert zu:} varba schlicht U wore fruht V (Fr32) \textbf{23} hie zuo] darzvͦ V \textbf{27} pflegents] pflegentz T  $\cdot$ man] mans V \textbf{28} swâ] Wa U  $\cdot$ welsch] welsc T welchs U \textbf{29} noch tiuscher erde] tútzsch erde noch V \textbf{30} habt ir] hat U \newline
\end{minipage}
\end{table}
\end{document}
