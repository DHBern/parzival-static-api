\documentclass[8pt,a4paper,notitlepage]{article}
\usepackage{fullpage}
\usepackage{ulem}
\usepackage{xltxtra}
\usepackage{datetime}
\renewcommand{\dateseparator}{.}
\dmyyyydate
\usepackage{fancyhdr}
\usepackage{ifthen}
\pagestyle{fancy}
\fancyhf{}
\renewcommand{\headrulewidth}{0pt}
\fancyfoot[L]{\ifthenelse{\value{page}=1}{\today, \currenttime{} Uhr}{}}
\begin{document}
\begin{table}[ht]
\begin{minipage}[t]{0.5\linewidth}
\small
\begin{center}*D
\end{center}
\begin{tabular}{rl}
\textbf{238} & \begin{large}S\end{large}ô \textbf{dâ} gedienet wære.\\ 
 & nû hœrt \textbf{ein ander} mære:\\ 
 & hundert knappen man gebôt,\\ 
 & die nâmen in wîze tweheln brôt\\ 
5 & mit zühten vor dem Grâle.\\ 
 & \textbf{die} giengen al ze mâle\\ 
 & \textbf{unt} teilten vür die taveln sich.\\ 
 & man sagte mir - \textbf{diz} sag ouch ich\\ 
 & ûf iwer ieslîches eit -,\\ 
10 & daz vorem Grâle \textbf{wære} bereit\\ 
 & - sol ich des iemen triegen,\\ 
 & sô müezet ir mit mir liegen -,\\ 
 & swâ nâch \textbf{jener} bôt die hant,\\ 
 & daz er al bereite vant\\ 
15 & spîse warm, spîse kalt,\\ 
 & spîse niwe \textbf{unt} \textbf{dar zuo} alt,\\ 
 & daz zam unt daz wilde.\\ 
 & \textbf{es} en\textbf{würde} nie dechein bilde,\\ 
 & beginnet maneger sprechen.\\ 
20 & der wil sich \textbf{übel} rechen.\\ 
 & Wan der Grâl was der sælden vruht,\\ 
 & der \textbf{werlde} süeze \textbf{al}sölh genuht,\\ 
 & er wac vil nâch gelîche,\\ 
 & \textbf{als} man saget von himelrîche.\\ 
25 & In kleiniu goltvaz man nam,\\ 
 & als ieslîcher spîse zam,\\ 
 & salsen, pfeffer, agraz.\\ 
 & dâ het der kiusche unt der vrâz\\ 
 & alle gelîche genuoc.\\ 
30 & mit \textbf{grôzer} \textbf{zuht} man\textbf{z} vür si truoc.\\ 
\end{tabular}
\scriptsize
\line(1,0){75} \newline
D Fr3 \newline
\line(1,0){75} \newline
\textbf{1} \textit{Initiale} D  \textbf{21} \textit{Majuskel} D  \textbf{25} \textit{Majuskel} D  \newline
\line(1,0){75} \newline
\textbf{4} wîze] wîzen Fr3 \textbf{5} dem] de Fr3 \textbf{8} diz] daz Fr3 \textbf{9} iwer ieslîches] v w îegeliches Fr3 \textbf{11} des iemen] d:::emanne Fr3 \newline
\end{minipage}
\hspace{0.5cm}
\begin{minipage}[t]{0.5\linewidth}
\small
\begin{center}*m
\end{center}
\begin{tabular}{rl}
 & sô \textbf{dâ} gedienet wære.\\ 
 & nû hœret \textbf{ein ander} mære:\\ 
 & hundert knappen man gebôt,\\ 
 & die nâmen in wîze twehelen brôt\\ 
5 & mit zühten vorme Grâle.\\ 
 & \textbf{die} giengen alle ze mâle.\\ 
 & \textbf{si} teilten vür die tavelen sich.\\ 
 & man sagete mir - \textbf{nû} sage ouch ich\\ 
 & ûf iuwer ieglîches eit -,\\ 
10 & daz vorme Grâ\textit{l}e \textbf{wære} bereit\\ 
 & - sol ich des ieman triegen,\\ 
 & sô müezet ir mit mir liegen -,\\ 
 & wâ nâch \textbf{jener} bôt die hant,\\ 
 & daz er albereit vant\\ 
15 & spîse warm, spîse kalt,\\ 
 & \textit{spîse niuwe, \textbf{dar zuo} alt},\\ 
 & daz zame und daz wilde.\\ 
 & "\textbf{es} en\textbf{wart} nie kein bilde",\\ 
 & beginnet maniger sprechen.\\ 
20 & der wil sich \textbf{unschœne} rechen:\\ 
 & wanne der Grâl was der sæl\textit{d}en vruht,\\ 
 & der \textbf{werde}, süeze, \textbf{ein} solich genuht!\\ 
 & er wac vil nâch glîche,\\ 
 & \textbf{als} man saget von himelrîche.\\ 
25 & in kleiniu goltvaz man nam,\\ 
 & als ieglîcher spîse zam,\\ 
 & salsen, pfeffer, agraz.\\ 
 & d\textit{â} hete der kiusche und der vrâz\\ 
 & alle \textbf{wol} gelîche genuoc.\\ 
30 & mit \textbf{grôzer} \textbf{zuht} man\textbf{z} vür si truoc.\\ 
\end{tabular}
\scriptsize
\line(1,0){75} \newline
m n o Fr69 \newline
\line(1,0){75} \newline
\newline
\line(1,0){75} \newline
\textbf{1} dâ] do n \textbf{5} Grâle] grabe n \textbf{6} alle ze] alzuͦ o \textbf{7} tavelen] tafel o \textbf{9} iuwer] uwer uwer n  $\cdot$ eit] [leit]: eit n eit \sout{daz vor dem gr} Fr69 \textbf{10} Grâle] grabe m \textbf{13} jener] ẏemer n (o)  $\cdot$ bôt] bat o \textbf{14} albereit] alle bereit o \textbf{15} spîse kalt] vnd kalt n o \textbf{16} \textit{Vers 238.16 fehlt} m   $\cdot$ dar zuo] oder Fr69 \textbf{18} enwart] wart n o \textbf{21} sælden] selben m \textbf{22} genuht] gemuͯht o \textbf{24} saget] sagete n \textbf{25} in kleiniu] Jn clein n o Enkeinv́ Fr69 \textbf{27} salsen] Salsan m Salssan n Solsan o \textbf{28} dâ] Do m n o \textbf{30} si] \textit{om.} n o \newline
\end{minipage}
\end{table}
\newpage
\begin{table}[ht]
\begin{minipage}[t]{0.5\linewidth}
\small
\begin{center}*G
\end{center}
\begin{tabular}{rl}
 & sô gedienet wære.\\ 
 & nû hœret \textbf{anderiu} mære:\\ 
 & \textit{h}undert knappen man gebôt,\\ 
 & die nâmen in wîze twehelen brôt\\ 
5 & mit zühten vor dem Grâle.\\ 
 & \textbf{si} giengen alzemâle\\ 
 & \textbf{unde} teilten vür die tavelen sich.\\ 
 & man seite mir - \textbf{daz} sage ouch ich\\ 
 & ûf iuwer ieslîches eit -,\\ 
10 & daz vor dem Grâle \textbf{was} bereit\\ 
 & - sol ich des iemen triegen,\\ 
 & sô müezet ir mit mir liegen -,\\ 
 & \textbf{wan} swâ nâch \textbf{jener} bôt die hant,\\ 
 & daz er\textbf{z} alberei\textit{t v}ant:\\ 
15 & spîse warm, spîse kalt,\\ 
 & spîse niuwe, \textbf{dar zuo} alt,\\ 
 & daz zam unde daz wilde.\\ 
 & \textbf{es}ne \textbf{würde} nie dehein bilde,\\ 
 & beginnet maniger sprechen.\\ 
20 & der wil sich \textbf{übel} rechen.\\ 
 & wan der Grâl was der sælden vruht,\\ 
 & der \textbf{werlde} süeze \textbf{ein} solch genuht,\\ 
 & er wac vil nâch gelîche,\\ 
 & \textbf{als} man saget von himelrîche.\\ 
25 & in kleiniu goltvaz man nam,\\ 
 & als ieslîcher spîse zam,\\ 
 & salsen, pfeffer, agraz.\\ 
 & dâ het der kiusche unde der vrâz\\ 
 & alle gelîche genuoc.\\ 
30 & mit \textbf{zühten} man\textbf{z} vür si truoc.\\ 
\end{tabular}
\scriptsize
\line(1,0){75} \newline
G I O L M Q R Z Fr40 Fr51 \newline
\line(1,0){75} \newline
\textbf{1} \textit{Initiale} Q Fr40   $\cdot$ \textit{Capitulumzeichen} Z  \textbf{3} \textit{Initiale} L R Z  \textbf{9} \textit{Initiale} I  \textbf{21} \textit{Initiale} I  \textbf{25} \textit{Initiale} O  \newline
\line(1,0){75} \newline
\textbf{1} sô] so daz I (R) So da O Z Fr40 [*o]: So da L Do da M Q Also daz Fr51 \textbf{2} anderiu] ein ander I ander Fr40 \textbf{3} hundert] wol hundert G  $\cdot$ man] do Q \textbf{4} wîze] wise R witzen Fr51  $\cdot$ twehelen] tuͦch R  $\cdot$ brôt] bot M \textbf{5} dem] den Fr51 \textbf{8} seite] seit I O M (Q) R  $\cdot$ mir] ez mir O  $\cdot$ daz] disz M (Z) \textbf{9} iuwer] \textit{om.} O M \textbf{10} daz] Waz L \textbf{11} sol] sold I  $\cdot$ iemen] niemen O (M) \textbf{12} müezet] muͦs R \textbf{13} swâ] war L wo M (Q) (R)  $\cdot$ nâch] noch M  $\cdot$ jener] einer O yegelicher L ymer R  $\cdot$ bôt] rot Q  $\cdot$ die] sin R \textbf{14} erz] er daz I Fr40 \textit{om.} O er L M Q R Z  $\cdot$ albereit vant] albereit da vant G alberaitet vant I \textbf{15} warm] warme O  $\cdot$ spîse] vnd L \textbf{16} dar zuo] spise I vnde da zvͦ O (L) (Q) (R) (Z) (Fr40) vnde spise M \textbf{18} esne] Es Q  $\cdot$ würde] ward R  $\cdot$ nie dehein] [in]: nie dehaim I \textbf{19} beginnet] Begýnnet nv L \textbf{20} übel] ubir M \textbf{21} der sælden] ein I der selben R \textbf{22} werlde] wer ie R  $\cdot$ süeze] selde O  $\cdot$ solch] selich I (Q) (Fr40)  $\cdot$ genuht] genvͯch L frucht R \textbf{23} er wac] Es was R \textbf{24} als] so I (L)  $\cdot$ von] vmb daz I von deme M (Q) \textbf{25} in] ÷n O  $\cdot$ kleiniu] Glein I kleine R  $\cdot$ man] \textit{om.} Z \textbf{26} zam] sam Q gezam R \textbf{27} salsen] Salze L  $\cdot$ agraz] vnd agrasz Q \textbf{28} dâ] Do Q  $\cdot$ het] hatt R  $\cdot$ der kiusche] der kusse M er der kúnsche R  $\cdot$ der] \textit{om.} O \textbf{30} zühten] grozzer zvht O (L) (Q) (R) Z  $\cdot$ manz] man O (L) (Q) Z man da M  $\cdot$ si] \textit{om.} M \newline
\end{minipage}
\hspace{0.5cm}
\begin{minipage}[t]{0.5\linewidth}
\small
\begin{center}*T
\end{center}
\begin{tabular}{rl}
 & sô \textbf{dâ} gedienet wære.\\ 
 & Nû hœret \textbf{ein ander} mære:\\ 
 & hundert knappen man gebôt,\\ 
 & die nâmen in wîze tweheln brôt\\ 
5 & mit zühten vor dem Grâle.\\ 
 & \textbf{die} giengen al zemâle\\ 
 & \textbf{unde} teileten vür die taveln sich.\\ 
 & man sagete mir - \textbf{daz} sage ouch ich\\ 
 & ûf iuwer ieglîches eit -,\\ 
10 & daz vor dem Grâle \textbf{wære} bereit\\ 
 & - sol ich des iemannen triegen,\\ 
 & sô müezet ir mit mir liegen -,\\ 
 & swâ nâch \textbf{ieglîcher} bôt die hant,\\ 
 & daz er al bereit vant\\ 
15 & spîse warm, spîse kalt,\\ 
 & spîse niuwe, \textbf{spîse} alt,\\ 
 & daz zame unde daz wilde.\\ 
 & \textbf{ez} en\textbf{würde} nie kein bilde,\\ 
 & beginnet maneger sprechen.\\ 
20 & der wil sich \textbf{mit übele} rechen.\\ 
 & wan der Grâl was der sælden vruht,\\ 
 & der \textbf{werlte} süeze \textbf{ein} solch genuht,\\ 
 & er wac vil nâch gelîche,\\ 
 & \textbf{sô} man saget von himelrîche.\\ 
25 & in klein\textit{iu} goltvaz man nam,\\ 
 & als ieglîcher spîse zam,\\ 
 & salsen, pfeffer, agraz.\\ 
 & dâ hete der kiusche unde der vrâz\\ 
 & alglîche genuoc.\\ 
30 & mit \textbf{grôzer} \textbf{zuht} man vür si truoc\\ 
\end{tabular}
\scriptsize
\line(1,0){75} \newline
T U V W \newline
\line(1,0){75} \newline
\textbf{2} \textit{Majuskel} T  \textbf{3} \textit{Initiale} W  \newline
\line(1,0){75} \newline
\textbf{1} dâ] do U V W \textbf{4} tweheln] twehele W \textbf{5} vor] von U [vo*]: vor V \textbf{7} teileten] [teilen]: teileten T \textbf{8} sagete] sagt W  $\cdot$ daz] [*]: nv V \textbf{9} iuwer ieglîches eit] [*]: uwer ieclichez eit V \textbf{10} wære] [*]: was V  $\cdot$ bereit] breit U \textbf{11} des] es W \textbf{13} swâ] Wa U (W)  $\cdot$ ieglîcher] iechcher U  $\cdot$ die] sein W \textbf{14} er] ers V  $\cdot$ al bereit] albereit do V do beraitet W \textbf{15} spîse kalt] speisen kalt W \textbf{16} Spise [nv́w*]: nv́we vnde darzvͦ alt V  $\cdot$ Speise neúwe vnd dar zuͦ alt W \textbf{20} mit übele] [*]: v́bel V úbel W \textbf{21} Grâl was der] \textit{om.} W \textbf{22} solch] selich U (V) \textbf{23} Er wac] Es ist W \textbf{24} sô] [*]: Als V  $\cdot$ saget] sagete V \textbf{25} in kleiniu] in cleine T (W) Ein cleine U (V)  $\cdot$ goltvaz] gol U  $\cdot$ nam] do nam W \textbf{28} dâ] Do U V W \textbf{29} [A*]: Alle wol geliche genuͦg V  $\cdot$ alglîche] Alle gleich W \textbf{30} man] [*]: manz V  $\cdot$ si] \textit{om.} U W \newline
\end{minipage}
\end{table}
\end{document}
