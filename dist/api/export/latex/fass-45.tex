\documentclass[8pt,a4paper,notitlepage]{article}
\usepackage{fullpage}
\usepackage{ulem}
\usepackage{xltxtra}
\usepackage{datetime}
\renewcommand{\dateseparator}{.}
\dmyyyydate
\usepackage{fancyhdr}
\usepackage{ifthen}
\pagestyle{fancy}
\fancyhf{}
\renewcommand{\headrulewidth}{0pt}
\fancyfoot[L]{\ifthenelse{\value{page}=1}{\today, \currenttime{} Uhr}{}}
\begin{document}
\begin{table}[ht]
\begin{minipage}[t]{0.5\linewidth}
\small
\begin{center}*D
\end{center}
\begin{tabular}{rl}
\textbf{45} & si brâhten opfers vil ir goten,\\ 
 & \textbf{die} von der stat. \textbf{waz wart} geboten\\ 
 & \begin{large}D\end{large}em küenem Razalige,\\ 
 & dô er schiet von dem wîge,\\ 
5 & daz leister durch triwe.\\ 
 & doch \textbf{wart} sîn jâmer niwe\\ 
 & nâch sîme hêrren Isenhart.\\ 
 & der burcgrâve \textbf{des} innen wart,\\ 
 & daz er \textbf{kom}. dô wart ein schal.\\ 
10 & dar kômen die vürsten überal\\ 
 & \textbf{ûz der küneginne lande} von Zazamanc.\\ 
 & \textbf{die} sageten im \textbf{des} prîses danc,\\ 
 & den \textbf{er het} aldâ bezalt.\\ 
 & ze rehter tjost het er \textbf{gevalt}\\ 
15 & vier unt zweinzec rîter nider\\ 
 & unt zôch ir ors \textbf{al meistic} wider.\\ 
 & Dâ wâren \textbf{gevangen} \textbf{vürsten} drî.\\ 
 & den reit \textbf{manec} rîter bî\\ 
 & ze hove ûf den palas.\\ 
20 & \textbf{entslâfen} unt enbizzen was,\\ 
 & unt wünneclîche gefeit\\ 
 & mit kleidern wol bereit\\ 
 & was des \textbf{hœhsten} \textbf{wirtes} lîp.\\ 
 & diu ê \textbf{hiez} maget, diu \textbf{was} nû wîp;\\ 
25 & \textbf{diu in her ûz vuorte} an ir hant.\\ 
 & \textbf{si} sprach: "\textbf{mîn lîp} unt \textbf{ouch} mîn lant\\ 
 & ist disem rîter undertân,\\ 
 & ob \textbf{ez im} \textbf{vîende} wellent lân."\\ 
 & Dâ wart gevolget Gahmurete\\ 
30 & einer hovelîchen bete:\\ 
\end{tabular}
\scriptsize
\line(1,0){75} \newline
D Fr14 \newline
\line(1,0){75} \newline
\textbf{3} \textit{Initiale} D Fr14  \textbf{17} \textit{Majuskel} D  \textbf{29} \textit{Majuskel} D  \newline
\line(1,0){75} \newline
\textbf{2} wart] was Fr14 \textbf{3} Razalige] Razalîge D :::zalige Fr14 \textbf{5} leister] leist er Fr14 \textbf{7} Isenhart] Jsenhart D :::rt Fr14 \textbf{9} dô] da Fr14 \textbf{11} Zazamanc] Zazamanch D zaza:::ch Fr14 \textbf{29} Gahmurete] gahmvrete D Gahmvret Fr14 \newline
\end{minipage}
\hspace{0.5cm}
\begin{minipage}[t]{0.5\linewidth}
\small
\begin{center}*m
\end{center}
\begin{tabular}{rl}
 & si brâhten opfers vil ir goten,\\ 
 & \textbf{die} von der stat. \textbf{waz wart} geboten\\ 
 & dem küenen Razalige,\\ 
 & dô er schiet von dem wî\textit{g}e,\\ 
5 & daz leister durch triuwe.\\ 
 & doch \textbf{wart} sîn jâmer niuwe\\ 
 & nâch sînem hêr\textit{r}en Ysenhart.\\ 
 & der burcgrâve \textbf{des} innen wart,\\ 
 & daz er \textbf{kam}. dô wart ein schal.\\ 
10 & dar kôme\textit{n} die vürsten überal\\ 
 & \textbf{ûz dem lande} von Zazamanc.\\ 
 & \textbf{die} sageten ime \textbf{des} prîses danc,\\ 
 & den \textbf{er hette} aldâ bezalt.\\ 
 & ze rehter just hette er \textbf{gevalt}\\ 
15 & \textbf{ir} vier und zweinzic ritter nider\\ 
 & und zôch ir ros \textbf{aleinic} wider.\\ 
 & d\textit{â} wâren \textbf{gevangen} \textbf{vürsten} drî.\\ 
 & den reit \textbf{manic} ritter bî\\ 
 & ze hove ûf den palas.\\ 
20 & \textbf{entslâfen} und enbizzen was,\\ 
 & und wünneclîchen \textit{ge}feitet\\ 
 & mit kleidern wol bereitet\\ 
 & was \textit{des} \textbf{hœhesten} \textbf{wirtes} lîp.\\ 
 & diu ê \textbf{was} maget, diu \textbf{was} nû wîp;\\ 
25 & \textbf{diu \textit{in} her ûz vuorte} an ir hant\\ 
 & \textbf{und} \textit{sprach}: "\textbf{mîn lîp} und \textbf{ouch} mîn lant\\ 
 & ist disem ritter undertân,\\ 
 & ob \textbf{uns} \textbf{die} \textbf{vîende} wellent lân."\\ 
 & \begin{large}D\end{large}â wart gevolget Gahmurete\\ 
30 & einer hovenlîchen bete.\\ 
\end{tabular}
\scriptsize
\line(1,0){75} \newline
m n o \newline
\line(1,0){75} \newline
\textbf{29} \textit{Initiale} m n o  \newline
\line(1,0){75} \newline
\textbf{2} waz wart] jme was n vmmb was o \textbf{3} küenen] [keinen]: kuͯnen n  $\cdot$ Razalige] razalige \textit{nachträglich korrigiert zu:} razalibe m racalie o \textbf{4} schiet] [schieb]: schied m  $\cdot$ wîge] wibe m \textbf{5} leister] leist er n o \textbf{7} hêrren] herreren \textit{nachträglich korrigiert zu:} herren m  $\cdot$ Ysenhart] ÿsenhart m jsenhart n isenhart o \textbf{10} kômen] komend m (n) \textbf{11} von] \textit{om.} n  $\cdot$ Zazamanc] [zazamant]: zazamang m zazamang n o \textbf{12} sageten] [sagen]: sagettend m \textbf{13} hette aldâ] do hette do n \textbf{14} gevalt] gewalt o \textbf{15} ir] Wol n o \textbf{16} aleinic] einig n \textbf{17} dâ] Do m n o  $\cdot$ gevangen] gefaren n \textbf{18} den] Dem n \textbf{19} den] dem n o \textbf{21} gefeitet] enfeitet m \textbf{23} des] \textit{om.} m  $\cdot$ lîp] [wip]: lip m \textbf{24} nû] im o \textbf{25} in] \textit{om.} m >in< o  $\cdot$ ir] der n \textbf{26} sprach] \textit{om.} m  $\cdot$ ouch] \textit{om.} n o \textbf{29} Dâ] DO n o  $\cdot$ Gahmurete] Gahmurette m gamiret n gamuͯret o \textbf{30} einer] Ein m \newline
\end{minipage}
\end{table}
\newpage
\begin{table}[ht]
\begin{minipage}[t]{0.5\linewidth}
\small
\begin{center}*G
\end{center}
\begin{tabular}{rl}
 & si brâhten opfers vil ir goten,\\ 
 & \textbf{als ez} von der stat \textbf{was} geboten\\ 
 & dem küenen Razalige,\\ 
 & dô er schiet von dem wîge.\\ 
5 & daz leister durch triwe.\\ 
 & doch \textbf{was} sîn jâmer niwe\\ 
 & nâch sînem hêrren Ysenhart.\\ 
 & \textbf{dô} der burcgrâve innen wart,\\ 
 & daz er \textbf{was komen}, dô wart ein schal.\\ 
10 & dar kômen die vürsten überal\\ 
 & \textbf{der künigîn} von Zazamanc\\ 
 & \textbf{unde} seiten im \textbf{des} brîses danc,\\ 
 & den \textbf{er het} al dâ bezalt.\\ 
 & ze rehter tjost heter \textbf{gevalt}\\ 
15 & \textbf{in} vier unt zweinzec rîter nider\\ 
 & unt zôch ir ors \textbf{almeistec} wider.\\ 
 & dâ wâren \textbf{gevangener} \textbf{vürsten} drî.\\ 
 & den reit \textbf{ouch mêr} rîter bî\\ 
 & ze hove ûf den palas.\\ 
20 & \textbf{e\textit{ntsl}â\textit{f}e\textit{n}} und enbizzen was,\\ 
 & unt wünneclîchen gefeit\\ 
 & mit kleideren \textbf{harte} wol bereit\\ 
 & was des \textbf{obersten} \textbf{wirtes} lîp.\\ 
 & diu ê \textbf{was} maget, diu \textbf{was} nû wîp;\\ 
25 & \textbf{diu in her ûz vuorte} an ir hant.\\ 
 & \textbf{si} sprach: "\textbf{\textit{mîn} liute} und \textit{mîn} lant\\ 
 & \textit{i}s\textit{t} disme rîter undertân,\\ 
 & \begin{large}O\end{large}p \textit{\textbf{ims}} \textbf{die} \textbf{vînde} wellent lân."\\ 
 & dô wart gevolget Gahmuret\\ 
30 & einer höveschlîchen bet:\\ 
\end{tabular}
\scriptsize
\line(1,0){75} \newline
G I O L M Q R Z Fr21 \newline
\line(1,0){75} \newline
\textbf{1} \textit{Initiale} O L M  \textbf{11} \textit{Initiale} R  \textbf{13} \textit{Initiale} I  \textbf{17} \textit{Initiale} M  \textbf{28} \textit{Initiale} G  \textbf{29} \textit{Initiale} I L Q Fr21  \newline
\line(1,0){75} \newline
\textbf{1} \textit{Die Verse 44.7-51.12 fehlen} Z   $\cdot$ si] ÷i O  $\cdot$ opfers] opffer R  $\cdot$ vil ir] vele den M wil ir Q \textbf{2} als ez] Den O (L) M Fr21 Dy Q Das R  $\cdot$ von] vor L  $\cdot$ was] das was Q \textbf{3} küenen] kunch I  $\cdot$ Razalige] kasalic M Kazaligîe Fr21 \textbf{4} dô er] der I Da ir M Do Fr21  $\cdot$ dem] \textit{om.} O \textbf{5} durch] wan er truͤc I dvrch sin O mit R \textbf{6} doch] Da M \textbf{7} nâch sînem] Vmb sinen R  $\cdot$ Ysenhart] ẏsenhart G Jsenhart L R Fr21 Jsenarte M eysenhart Q \textbf{8} innen] des inne O (L) (M) (Q) (Fr21) das Jnen R \textbf{9} was komen] chom O (L) (M) (Q) (R) o\textit{m. } Fr21  $\cdot$ dô] da I L M \textit{om.} Fr21  $\cdot$ wart] war R \textbf{11} künigîn] kvnig da L kunge R  $\cdot$ Zazamanc] zazamanch G O (L) sasamangk M zazamat Q \textbf{12} seiten] seit I \textbf{13} er] \textit{om.} Fr21  $\cdot$ het al dâ] da hatte M hat alda R \textbf{14} gevalt] ervalt O Fr21 gewalt Q \textbf{15} in] \textit{om.} I O L M Q Fr21 \textbf{16} zôch] fuͯrte L  $\cdot$ almeistec] al maiste O (L) (Fr21) al messick Q aller meiste R \textbf{17} dâ] Do Q  $\cdot$ gevangener] geuangen I (O) (L) (M) (Q) (R) (Fr21) \textbf{18} den] Da L  $\cdot$ ouch mêr] manich O (L) (M) (Q) (R) (Fr21) \textbf{19} Vnd erbeitzeten vor den pallas L \textbf{20} entslâfen] erwachet G Vrschlafe L  $\cdot$ enbizzen] entbeysset Q \textbf{21} wünneclîchen] [wudelichen]: wundelichen Q wunderlich R \textbf{22} harte] \textit{om.} O L M Q R Fr21  $\cdot$ bereit] becleit L \textbf{23} obersten] hosten O L (M) (R) (Fr21) hochstes Q  $\cdot$ wirtes] [wirten]: wirtes Fr21 \textbf{24} ê] for M \textit{om.} Q  $\cdot$ was nû] ist nu I (L) (Q) (R) (Fr21) osz Nu syn M \textbf{25} her ûz] irusz M  $\cdot$ ir] der I Q \textbf{26} mîn liute] lvte G min lip O (L) (M) (R) Fr21 nún mein leyp Q  $\cdot$ und] \textit{om.} I  $\cdot$ mîn lant] lant G \textbf{27} ist] si G \textbf{28} ims] mirz G iz im O (Q) (R) (Fr21) osz Nu M  $\cdot$ die vînde] veinde O (Fr21) vienden M  $\cdot$ lân] lont Q \textbf{29} dô] Da M R Fr21  $\cdot$ Gahmuret] hahmuret I Gamvret O Gahmuͯret L gammuraten M gamuert Q Gahmoret Fr21 \textbf{30} höveschlîchen] huͯffelichen L (Q) (R) (Fr21) \newline
\end{minipage}
\hspace{0.5cm}
\begin{minipage}[t]{0.5\linewidth}
\small
\begin{center}*T (U)
\end{center}
\begin{tabular}{rl}
 & \begin{large}S\end{large}i brâhten opfers vil ir goten,\\ 
 & \textbf{die} von der stat. \textbf{waz was} geboten\\ 
 & dem küenen Razalige,\\ 
 & dô er schiet von de\textit{m} w\textit{î}ge,\\ 
5 & daz leiste er durch triuwe.\\ 
 & doch \textbf{was} sîn jâmer niuwe\\ 
 & nâch sîme hêrren Isenhart.\\ 
 & \textbf{dô} der burcgrâve \textbf{des} innen wart,\\ 
 & daz er \textbf{kam}, dô wart ein schal.\\ 
10 & dar kômen die vürsten überal\\ 
 & \textbf{der künegîn} von Zazamanc\\ 
 & \textbf{und} sageten im \textbf{sînes} prîses danc,\\ 
 & den \textbf{het er} aldâ bezalt.\\ 
 & ze rehte\textit{r} jost het er \textbf{ervalt}\\ 
15 & viere und zwenzic ritter nider\\ 
 & und zôch ir ors \textbf{alm\textit{ei}stic} wider.\\ 
 & d\textit{â} wâren \textbf{gevange\textit{ne}r} \textbf{künige} drî.\\ 
 & den reit \textbf{vil manic} ritter bî\\ 
 & zuo hove ûf den palas.\\ 
20 & \textbf{geslâfen} und enbizzen was,\\ 
 & und wünneclîche gefeitet\\ 
 & mit kleidern wol bereitet\\ 
 & was des \textbf{hœ\textit{hest}en} \textbf{wîbes} lîp.\\ 
 & diu ê \textbf{hiez} maget, diu \textbf{ist} nû wîp;\\ 
25 & \textbf{si vuorte in vür} an ir hant\\ 
 & \textbf{und} sprach: "\textbf{mîn lîp} und mîn lant\\ 
 & ist disem ritter undertân,\\ 
 & ob \textbf{ez nû} \textbf{die} \textbf{næhesten} wellent lân."\\ 
 & \begin{large}D\end{large}ô wart gevolget Gahmurete\\ 
30 & einer höveschlîchen bete:\\ 
\end{tabular}
\scriptsize
\line(1,0){75} \newline
U V W T \newline
\line(1,0){75} \newline
\textbf{1} \textit{Initiale} U V W T  \textbf{8} \textit{Majuskel} T  \textbf{17} \textit{Majuskel} T  \textbf{29} \textit{Initiale} U V W T  \newline
\line(1,0){75} \newline
\textbf{2} die] [*]: Das V  $\cdot$ waz was] [*]: waz V was W nv was T \textbf{3} Razalige] razzalige W \textbf{4} dem wîge] den wege U \textbf{5} leiste er] leistet er V laster W  $\cdot$ durch] mit T \textbf{6} doch] noch V \textbf{7} Isenhart] ysenharte U Jsinhart V ysenhart W Jsenhart T \textbf{8} der burcgrâve des] des der bvrcgrave T \textbf{9} dô] das W da T \textbf{11} Zazamanc] [*]: zazamang V \textbf{12} sînes] des T \textbf{13} het er aldâ] hette er eine aldo W er hete da T \textbf{14} rehter] rechte [er]: * U  $\cdot$ het] hat W T  $\cdot$ ervalt] geualt V (T) gewalt W \textbf{15} viere] [J*]: Jr vier V Zwene W in vier T \textbf{16} almeistic] al mastic U \textbf{17} dâ] Do U W [*]: Hie V  $\cdot$ gevangener] gevanger U geuangen V  $\cdot$ künige] [*]: fv́rsten V fúrsten W (T) \textbf{18} vil] \textit{om.} T \textbf{20} geslâfen] [E*slafen]: Entslafen V Beschloffen W \textbf{23} hœhesten wîbes] hoffen wibes U [*]: hoͤhesten wúrtes V hoͤchten weibes W niͮwen wirtes T \textbf{24} ist nû] waz ein V hieß nun W (T) \textbf{25} si vuorte in vür] [S* fvͦrte]: Die in her vs fvͦrte V div in vuͦrte her v̂z T \textbf{26} und] Sy W (T) \textbf{28} ez nû die næhesten] [*]: ez im viende V ez im die viende T ims die veinde W \textbf{29} Dô] Da T  $\cdot$ Gahmurete] Gahmuͦrete U Gamurette V gamuret W \textbf{30} einer] siner T  $\cdot$ höveschlîchen] houelichen V [hovelicher]: hovelichen T \newline
\end{minipage}
\end{table}
\end{document}
