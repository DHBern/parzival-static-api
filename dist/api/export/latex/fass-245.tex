\documentclass[8pt,a4paper,notitlepage]{article}
\usepackage{fullpage}
\usepackage{ulem}
\usepackage{xltxtra}
\usepackage{datetime}
\renewcommand{\dateseparator}{.}
\dmyyyydate
\usepackage{fancyhdr}
\usepackage{ifthen}
\pagestyle{fancy}
\fancyhf{}
\renewcommand{\headrulewidth}{0pt}
\fancyfoot[L]{\ifthenelse{\value{page}=1}{\today, \currenttime{} Uhr}{}}
\begin{document}
\begin{table}[ht]
\begin{minipage}[t]{0.5\linewidth}
\small
\begin{center}*D
\end{center}
\begin{tabular}{rl}
\textbf{245} & \begin{large}P\end{large}arzival niht eine lac:\\ 
 & geselleclîche unz an den tac\\ 
 & was bî im strengiu arbeit.\\ 
 & ir boten künftigiu leit\\ 
5 & sanden im in slâfe dar,\\ 
 & sô daz der junge wol gevar\\ 
 & sîner muoter troum gar widerwac,\\ 
 & des si nâch Gahmurete pflac.\\ 
 & Sus \textbf{wart} \textbf{gesteppet} im sîn troum\\ 
10 & mit swertslegen umbe den soum,\\ 
 & dâr vor mi\textit{t} maneger tjoste rîch.\\ 
 & von rabbîne hurteclîch\\ 
 & er leit in slâfe \textbf{etslîch} nôt.\\ 
 & m\textit{ö}hter drîzec stunt sîn tôt,\\ 
15 & daz het er wachende ê gedolt.\\ 
 & sus teilte im ungemach den solt.\\ 
 & Von disen strengen sachen\\ 
 & muoser durch nôt erwachen.\\ 
 & im swizten \textbf{âdern} unt bein.\\ 
20 & der tag ouch durch diu venster \textbf{schein}.\\ 
 & dô sprach er: "wê, wâ sint diu kint,\\ 
 & daz si hie vor mir niht sint?\\ 
 & wer sol mir bieten mîn gewant?"\\ 
 & sus warte \textbf{ir} der wîgant,\\ 
25 & unz er \textbf{an der stunt} entslief.\\ 
 & niemen dâ redete noch \textbf{en}rief.\\ 
 & si wâren gar verborgen.\\ 
 & \textbf{umbe den} mitten morgen,\\ 
 & dô erwachete aber der junge man.\\ 
30 & ûf rihte sich der küene sân.\\ 
\end{tabular}
\scriptsize
\line(1,0){75} \newline
D \newline
\line(1,0){75} \newline
\textbf{1} \textit{Initiale} D  \textbf{9} \textit{Majuskel} D  \textbf{17} \textit{Majuskel} D  \newline
\line(1,0){75} \newline
\textbf{8} Gahmurete] Gahmvrete D \textbf{11} mit] mir D \textbf{14} möhter] mohter D \newline
\end{minipage}
\hspace{0.5cm}
\begin{minipage}[t]{0.5\linewidth}
\small
\begin{center}*m
\end{center}
\begin{tabular}{rl}
 & Parcifal niht eine lac:\\ 
 & geselleclîch unz an den tac\\ 
 & was bî ime strengiu arbeit.\\ 
 & ir boten künftigiu leit\\ 
5 & santen ime in slâfe dar,\\ 
 & sô daz der junge wol gevar\\ 
 & sîner muoter troum gar widerwac,\\ 
 & des si nâch Gahmurete pflac.\\ 
 & sus \textbf{was} \textbf{ges\textit{t}ep\textit{p}et} i\textit{m} sîn troum\\ 
10 & mit swertslege\textit{n} umb den soum,\\ 
 & dâr vor mit maniger juste rîch.\\ 
 & von rabîne h\textit{u}rteclîch\\ 
 & er leit in slâfe \textbf{etslîche} nôt.\\ 
 & m\textit{ö}hter drîzic stunt sîn tôt,\\ 
15 & daz het er wachende ê gedolt.\\ 
 & sus teilte ime ungemach den solt.\\ 
 & von disen strengen sachen\\ 
 & muos er durc\textit{h} \textit{n}ôt erwachen.\\ 
 & im swizeten \textbf{âderen} und bein.\\ 
20 & der tac ouch durch diu venste\textit{r} \textbf{ersch\textit{e}in}.\\ 
 & dô sprach er: "wê, wâ sint diu \textit{k}int,\\ 
 & daz si hie vor mir niht sint?\\ 
 & wer sol mir bieten mîn gewant?"\\ 
 & sus w\textit{a}rtete \textbf{ir} der wîgant,\\ 
25 & unz \textbf{daz} er \textbf{an der stunt} entslief.\\ 
 & nieman dâ redet\textit{e} noch \textbf{en}rief.\\ 
 & si wâren gar verborgen.\\ 
 & \textbf{umb den} mitten morgen,\\ 
 & dô erwachete aber der junge man.\\ 
30 & ûf rih\textit{t}e sich der küene sân.\\ 
\end{tabular}
\scriptsize
\line(1,0){75} \newline
m n o Fr69 \newline
\line(1,0){75} \newline
\newline
\line(1,0){75} \newline
\textbf{1} niht eine] mit eẏme o \textbf{3} strengiu] stenge o \textbf{4} künftigiu] kúnfftig n \textbf{7} troum] truͦg o  $\cdot$ widerwac] vnderwag Fr69 \textbf{8} des] Das o  $\cdot$ Gahmurete] gahmvrete m gahmúreten n gamuͯrete o gachmurette Fr69 \textbf{9} was] wart n o Fr69  $\cdot$ gesteppet im sîn] geschepfet in sin m gestoppfet in dem n gestopfet in sẏme o \textbf{10} swertslegen] swert slegem m \textbf{11} dâr vor] Der wirt o \textbf{12} hurteclîch] hertteclich m \textbf{13} etslîche] ettelicher o \textbf{14} möhter] Mohter m (o) \textbf{15} Er hett erwachete e gedult o \textbf{16} teilte] deilet n (o) \textbf{17} strengen] trengen o \textbf{18} muos] Muͯste n  $\cdot$ durch nôt] durch not not m doch not o \textbf{19} swizeten] swiczent o \textbf{20} ouch] \textit{om.} n durch o  $\cdot$ venster erschein] vensteren erschin m venster in schein n venster schein o \textbf{21} kint] sint m \textbf{22} sint] ensint Fr69 \textbf{24} wartete] wertete m wartet n o \textbf{26} dâ] \textit{om.} n o  $\cdot$ redete noch] redetette nach m enret oder n redet oder o  $\cdot$ enrief] rieff n o \textbf{30} rihte] riche m  $\cdot$ sich] sich aber n \newline
\end{minipage}
\end{table}
\newpage
\begin{table}[ht]
\begin{minipage}[t]{0.5\linewidth}
\small
\begin{center}*G
\end{center}
\begin{tabular}{rl}
 & Parzival niht eine lac:\\ 
 & geselliclîche unze an den tac\\ 
 & was bî im \textbf{ein} strengiu arbeit.\\ 
 & ir boten künftigiu leit\\ 
5 & sanden im in slâfe dar,\\ 
 & sô daz der junge wolgevar\\ 
 & sîner muoter troum gar widerwac,\\ 
 & des si nâch Gahmurete pflac.\\ 
 & sus \textbf{wart} \textbf{gestabet} im sîn troum\\ 
10 & mit swertslegen umbe den soum,\\ 
 & dâ vor mit maniger tjoste rîch.\\ 
 & von rabbîne hurticlîch\\ 
 & er leit in slâfe \textbf{solche} nôt,\\ 
 & m\textit{ö}hter drîzic stunt sîn tôt,\\ 
15 & daz heter wachende ê gedolt.\\ 
 & sus teilt im ungemach den solt.\\ 
 & von disen strengen sachen\\ 
 & muoser durch nôt erwachen.\\ 
 & \begin{large}I\end{large}m swizten \textbf{âder} unde bein.\\ 
20 & der tac ouch durch diu venster \textbf{schein}.\\ 
 & dô sprach er: "wê, wâ sint diu kint,\\ 
 & daz si hie vor mir niht sint?\\ 
 & wer sol mir bieten mîn gewant?"\\ 
 & sus wart \textbf{in} der wîgant,\\ 
25 & unzer \textbf{ander\textit{stu}n\textit{t}} entslief.\\ 
 & niemen dâ redete noch rief,\\ 
 & \textbf{wan} si wâren gar verborgen.\\ 
 & \textbf{reht an dem} mitten morgen,\\ 
 & dô erwachte aber der junge man.\\ 
30 & ûf rihte sich der küene sân.\\ 
\end{tabular}
\scriptsize
\line(1,0){75} \newline
G I O L M Q R Z Fr54 \newline
\line(1,0){75} \newline
\textbf{1} \textit{Initiale} L M Z  \textbf{5} \textit{Initiale} I  \textbf{9} \textit{Initiale} R  \textbf{19} \textit{Initiale} G  \textbf{25} \textit{Initiale} I  \newline
\line(1,0){75} \newline
\textbf{1} Parzival] parzifal I (M) Parcifal O Z [Pa]: Parcifal L Partzifal Q Parczifal R  $\cdot$ eine lac] [enlac]: einlac I [ein]: eine lag L einig lag R \textbf{2} \textit{Vers 245.2 fehlt} Q  \textbf{3} was] Bas O  $\cdot$ ein] \textit{om.} R Z  $\cdot$ strengiu] strenge R \textbf{4} ir boten] Jr O Er botten R  $\cdot$ künftigiu] [chunt]: chvnftigev I kunstliche M kᵫnfftige R \textbf{5} slâfe] slaffin M \textbf{6} sô daz] \textit{om.} I  $\cdot$ der] das L \textbf{7} troum] tram I \textbf{8} si] \textit{om.} I  $\cdot$ Gahmurete] Gamvreten O Gahmuͯrete L gamuͯreten M gamúreten Q Gahmuretes R gamureten Z \textbf{9} wart] war I  $\cdot$ gestabet] gestept O (L) (M) (Q) (Z) gelútret R g::: Fr54  $\cdot$ im] \textit{om.} R  $\cdot$ troum] ovm Fr54 \textbf{10} swertslegen] swertes schlegen Q (Z) (Fr54) schwertten schlegen R  $\cdot$ umbe] in O  $\cdot$ soum] son L \textbf{11} \textit{Versfolge 245.12-11} L   $\cdot$ dâ vor] Di er O \textit{om.} L Der vur M \textbf{12} hurticlîch] burtichleiche O \textbf{13} in] im R  $\cdot$ solche] solhiv O etliche Z \textbf{14} möhter] mohter G (I) (O) (L) (M) (Q) Er moͯch R Er moht Z  $\cdot$ stunt] vnt R \textbf{15} ê] \textit{om.} Q \textbf{16} teilt] teyl O Q teilte L \textbf{17} \textit{Versfolge 245.18-17} O   $\cdot$ von] Mit O M \textbf{18} erwachen] wachen I R \textbf{19} swizten] [switze]: switzten G schwiczte R  $\cdot$ âder] arm I (L) adern Z  $\cdot$ bein] die bein Z \textbf{20} diu] \textit{om.} L \textbf{21} dô] Da O L M  $\cdot$ sprach] dahte L  $\cdot$ wê] \textit{om.} M R \textbf{22} \textit{Versfolge 245.21-23-24-25-26-27-28-29-30-22} Q   $\cdot$ hie vor mir niht] vor mir hie niht I sie vor myr nirgen hie M vor mir nicht R niht hie vor mir Z \textbf{24} wart in] wartet in O wartet er L \textbf{25} unzer] Sus er M  $\cdot$ anderstunt] an der wende G an der wæide O andir weit M ander werb R \textbf{26} dâ] do O Q R  $\cdot$ redete] niht redte I redet O L ent redete M reit Z  $\cdot$ rief] enrief O L (M) (Q) (R) Z \textbf{27} wan] \textit{om.} L  $\cdot$ wâren] hoeren Q \textbf{28} reht an dem] Hin vmbe den L Vmb den Q Bis an den R \textbf{29} dô] \textit{om.} I Da M Z  $\cdot$ erwachte] erwacht O (Z)  $\cdot$ aber] \textit{om.} I \textbf{30} sich] he sich M  $\cdot$ küene] degen I chvnich O \newline
\end{minipage}
\hspace{0.5cm}
\begin{minipage}[t]{0.5\linewidth}
\small
\begin{center}*T
\end{center}
\begin{tabular}{rl}
 & \begin{large}P\end{large}arcifal niht eine lac:\\ 
 & geselleclîche unz an den tac\\ 
 & was bî im streng\textit{iu} arbeit.\\ 
 & ir boten künftigiu leit\\ 
5 & santen im in slâfe dar,\\ 
 & sô daz der junge wol gevar\\ 
 & sîner muoter troum gar widerwac,\\ 
 & des si nâch Gahmurete pflac.\\ 
 & sus \textbf{wart} \textbf{gesteppet} im sîn troum\\ 
10 & mit swertslegen umbe den soum,\\ 
 & dâr vor mit maneger tjost rîche.\\ 
 & von rabîne hurteclîche\\ 
 & er leit in slâfe \textbf{sölhe} nôt,\\ 
 & m\textit{ö}hter drîzic stunt sîn tôt,\\ 
15 & daz het er wachende ê gedolt.\\ 
 & sus teiltim ungemach den solt.\\ 
 & Von disen strengen sachen\\ 
 & muoser durch nôt erwachen.\\ 
 & im swizten \textbf{âdern} unde bein.\\ 
20 & der tac ouch durch di\textit{u} venster \textbf{schein}.\\ 
 & Dô sprach er: "wê, wâ sint diu kint,\\ 
 & daz si hie vor mir niht sint?\\ 
 & wer sol mir bieten mîn gewant?"\\ 
 & sus wartet\textbf{ir} der wîgant,\\ 
25 & unz \textbf{daz} er \textbf{ander stunt} entslief.\\ 
 & niemen dâ redete noch \textbf{en}rief.\\ 
 & si wâren gar verborgen.\\ 
 & \textbf{Rehte umb den} mitten morgen,\\ 
 & dô erwachete aber der jung\textit{e} man.\\ 
30 & ûf rihte sich der küene sân.\\ 
\end{tabular}
\scriptsize
\line(1,0){75} \newline
T U V W \newline
\line(1,0){75} \newline
\textbf{1} \textit{Initiale} T U V  \textbf{17} \textit{Initiale} W   $\cdot$ \textit{Majuskel} T  \textbf{21} \textit{Majuskel} T  \textbf{28} \textit{Majuskel} T  \newline
\line(1,0){75} \newline
\textbf{1} Parcifal] Parzifal T (V) Partzifal W  $\cdot$ eine] in U \textbf{2} unz] mit U \textbf{3} strengiu] strenge T \textbf{4} ir] [*]: Jr V \textbf{8} Gahmurete] gahmvrete T Gahmuͦrete U gamurette V W \textbf{11} dâr vor] Ward von im W \textbf{13} in] im W \textbf{14} möhter] mohter T (U)  $\cdot$ drîzic stunt] dreissigualt W \textbf{16} teiltim] leite im U tailt im W \textbf{18} muoser] mveser T \textbf{19} âdern] arm U (V) \textbf{20} ouch] \textit{om.} U V W  $\cdot$ diu] die T  $\cdot$ schein] \textit{om.} U \textbf{21} diu] dine U \textbf{24} wartetir] [warte*]: warte ir U wartet ir V W \textbf{25} unz] Mit U \textbf{26} dâ] \textit{om.} U [*]: do V do W  $\cdot$ redete] Jnredete U \textbf{28} umb] mit an U \textbf{29} junge] ivngen T \textbf{30} rihte] rachte U \newline
\end{minipage}
\end{table}
\end{document}
