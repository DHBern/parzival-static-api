\documentclass[8pt,a4paper,notitlepage]{article}
\usepackage{fullpage}
\usepackage{ulem}
\usepackage{xltxtra}
\usepackage{datetime}
\renewcommand{\dateseparator}{.}
\dmyyyydate
\usepackage{fancyhdr}
\usepackage{ifthen}
\pagestyle{fancy}
\fancyhf{}
\renewcommand{\headrulewidth}{0pt}
\fancyfoot[L]{\ifthenelse{\value{page}=1}{\today, \currenttime{} Uhr}{}}
\begin{document}
\begin{table}[ht]
\begin{minipage}[t]{0.5\linewidth}
\small
\begin{center}*D
\end{center}
\begin{tabular}{rl}
\textbf{38} & \textit{\begin{large}E\end{large}}r reit ûf in unt trat in nider.\\ 
 & \textbf{des} erholte \textit{er} sich dicke wider.\\ 
 & er tet werlîchen willen schîn.\\ 
 & \textbf{doch} steckete in dem arme sîn\\ 
5 & diu Gahmuretes lanze.\\ 
 & \textbf{der} iesch fîanze.\\ 
 & sînen meister \textbf{het} er vunden.\\ 
 & "wer hât mich überwunden?",\\ 
 & \textbf{alsô} sprach der \textbf{küene} man.\\ 
10 & der sigehafte \textbf{jach} \textbf{dô} sân:\\ 
 & "ich bin Gahmuret Anschevin."\\ 
 & er sprach: "mîn sicherheit sî dîn."\\ 
 & Die \textbf{enpfieng} er unt sanden în.\\ 
 & \textbf{des muos er vil geprîset} sîn\\ 
15 & von \textbf{den} vrouwen, die daz sâhen.\\ 
 & dort her begunde gâhen\\ 
 & von Normandie Gaschier,\\ 
 & \textbf{der} ellens rîche degen fier,\\ 
 & der \textbf{starke} tjostiure.\\ 
20 & hie h\textit{iel}t ouch der gehiure\\ 
 & Gahmuret zer \textbf{anderen} tiost bereit.\\ 
 & sîme sper \textbf{was daz îser} breit\\ 
 & unt der schaft veste.\\ 
 & \textbf{al dâ} werten die geste\\ 
25 & ein ander. ungelîch ez wac.\\ 
 & Gaschier dar nider \textbf{lac}\\ 
 & mit orse mitalle\\ 
 & vo\textit{n} der tjost valle\\ 
 & \textbf{unt wart} betwungen sicherheit,\\ 
30 & ez wære im liep oder leit.\\ 
\end{tabular}
\scriptsize
\line(1,0){75} \newline
D \newline
\line(1,0){75} \newline
\textbf{1} \textit{Initiale} D  \textbf{13} \textit{Majuskel} D  \newline
\line(1,0){75} \newline
\textbf{1} Er] ÷R D \textbf{2} er] \textit{om.} D \textbf{5} Gahmuretes] Gahmvrets D \textbf{11} Gahmuret] Gahmvret D  $\cdot$ Anschevin] Anscivin D \textbf{17} Gaschier] Gascier D \textbf{20} hielt] heilet D \textbf{21} Gahmuret] Gahmvret D \textbf{26} Gaschier] Gaschir D \textbf{28} von] vol D \newline
\end{minipage}
\hspace{0.5cm}
\begin{minipage}[t]{0.5\linewidth}
\small
\begin{center}*m
\end{center}
\begin{tabular}{rl}
 & er reit ûf in und tra\textit{t} in nider.\\ 
 & \textbf{de\textit{s}} erholter sich dicke wider.\\ 
 & er tet we\textit{r}lîchen wille\textit{n s}chîn.\\ 
 & \textbf{doch} steckete in dem arme sîn\\ 
5 & diu Gahmuretes lanze.\\ 
 & \textbf{der} iesch \textbf{die} f\textit{î}anze.\\ 
 & sînen meister \textbf{hete}r vunden.\\ 
 & "wer hât mich überwunden?",\\ 
 & \textbf{alsô} sprach der \textbf{küene} man.\\ 
10 & der sigehafte \textbf{jach} \textbf{dô} sân:\\ 
 & "ich bin Gahm\textit{ure}t Anschevin."\\ 
 & er sprach: "mîn sicherheit sî dîn."\\ 
 & die \textbf{enpfien\textit{c}} er und sante in în.\\ 
 & \textbf{des muos er vil geprîset} sîn\\ 
15 & von vrowen, die daz sâhen.\\ 
 & dort her begunde gâhen\\ 
 & von Normand\textit{i}e Gaschier,\\ 
 & \textbf{der} \textit{e}llens rîche, \textbf{der} degen fie\textit{r},\\ 
 & der \textbf{starke} justiure.\\ 
20 & hie hielt ouch der gehiure\\ 
 & Gahmuret zuo der \textbf{anderen} just bereit.\\ 
 & sînem sper, \textbf{daz was daz} \dag îsenbreit\dag \\ 
 & und der schaft veste.\\ 
 & \textbf{aldâ} w\textit{e}rten die geste\\ 
25 & ein ander. ungelîche ez wa\textit{c}.\\ 
 & G\textit{a}schier dar nider \textbf{lac}\\ 
 & mit rosse mitalle\\ 
 & von der juste valle\\ 
 & \textbf{und wart} betwungen sicherheit,\\ 
30 & ez wær im liep oder leit.\\ 
\end{tabular}
\scriptsize
\line(1,0){75} \newline
m n o W \newline
\line(1,0){75} \newline
\textbf{11} \textit{Initiale} W  \newline
\line(1,0){75} \newline
\textbf{1} trat] trapte m \textbf{2} des] [E]: Der \textit{nachträglich korrigiert zu:} Den m  $\cdot$ erholter] erholt er n o \textbf{3} Er det wellichen wilen er schin m \textbf{4} steckete] stecket n o \textbf{5} diu] Des W  $\cdot$ Gahmuretes] gahmurettes m gamiretes n gamuͯretes o gamuret W \textbf{6} fîanze] francze m (n) (o) \textbf{9} alsô] \textit{om.} n o W \textbf{10} sigehafte] gesigehafft o  $\cdot$ jach] sprach n o W \textbf{11} Gahmuret] gahmert m gamúret n gemuret o gamuret W  $\cdot$ Anschevin] ansceuin m auscefin n anscefin o antscheuin W \textbf{12} dîn] dann o \textbf{13} enpfienc] enpfinge m  $\cdot$ în] hin n o W \textbf{16} her] hert n  $\cdot$ begunde] begunden n o W  $\cdot$ gâhen] hohen o \textbf{17} Normandie] normande m mormandies o  $\cdot$ Gaschier] Gascier m (n) (o) gatschier W \textbf{18} ellens rîche der] enllensriche der m ellentrichen n ellentlichen o ellentriche W  $\cdot$ fier] viere m \textbf{19} starke] starcken n o W \textbf{20} ouch] [er]: och m \textbf{21} Gahmuret] Gamúret n Gamuret o W  $\cdot$ just] justie o \textbf{22} sînem] Sin n (o) (W)  $\cdot$ was daz] was sein W \textbf{24} werten] worten m \textbf{25} ein ander] Einader o  $\cdot$ ungelîche] vnglúckes W  $\cdot$ ez] \textit{om.} n o W  $\cdot$ wac] was \textit{nachträglich korrigiert zu:} wag m was n o \textbf{26} Gaschier] garschier m (n) (o) Gatschier W  $\cdot$ lac] sasz n (o) \textbf{27} rosse] juste vnd n o roß vnd W \textbf{28} valle] vallen o \textbf{29} betwungen] betzungen W \newline
\end{minipage}
\end{table}
\newpage
\begin{table}[ht]
\begin{minipage}[t]{0.5\linewidth}
\small
\begin{center}*G
\end{center}
\begin{tabular}{rl}
 & er reit ûf in und trat in nider.\\ 
 & \textbf{des} erholter sich dicke wider.\\ 
 & er tet werlîchen willen schîn.\\ 
 & \textbf{dô} stecket in dem arme sîn\\ 
5 & diu Gahmuretes lanze.\\ 
 & \textbf{er} iesch \textbf{die} fîanze.\\ 
 & sînen meister \textbf{hete}r vunden.\\ 
 & "wer hât mich überwunden?",\\ 
 & sprach der \textbf{sigelôse} man.\\ 
10 & der sigehafte \textbf{sprach} \textbf{dô} sân:\\ 
 & "ich bin Gahmuret Antschevin."\\ 
 & er sprach: "mîn sicherheit sî dîn."\\ 
 & die \textbf{nam} er und sande in în.\\ 
 & \textbf{des muoser vil gebrîset} sîn\\ 
15 & von \textbf{den} vrouwen, die daz sâhen.\\ 
 & dort her begunde gâhen\\ 
 & von Normandie Gatschier,\\ 
 & \textbf{ein} ellens rîcher degen fier,\\ 
 & der \textbf{starke} tjostiure.\\ 
20 & hie hielt och der gehiure\\ 
 & Gahmuret zer tjost bereit.\\ 
 & sînem sper \textbf{was daz îsen} breit\\ 
 & unt der schaft veste.\\ 
 & \textbf{hie} werten die geste\\ 
25 & ein ander. ungelîchez wac.\\ 
 & Gatschier dar nidere \textbf{gelac}\\ 
 & \begin{large}M\end{large}it orse mitalle.\\ 
 & von der tjoste valle\\ 
 & \textbf{wart er} betwungen sicherheit,\\ 
30 & ez wære im liep oder leit.\\ 
\end{tabular}
\scriptsize
\line(1,0){75} \newline
G O L M Q R Z Fr21 \newline
\line(1,0){75} \newline
\textbf{1} \textit{Initiale} M  \textbf{13} \textit{Initiale} Q R Z Fr21  \textbf{27} \textit{Initiale} G  \newline
\line(1,0){75} \newline
\textbf{1} er] Der M  $\cdot$ trat] trett Q (Z) \textbf{2} erholter] er holt er O (L) (Q) (R) (Z) (Fr21)  $\cdot$ sich] sich des Q \textbf{3} werlîchen] vil werlichen Fr21 \textbf{4} dô] Doch Z  $\cdot$ stecket] staht im O stegte M (R) stercke Q staht Fr21 \textbf{5} diu] Des Z  $\cdot$ Gahmuretes] Gahmvrets G Gamvretes O Gahmvretes L gamuretis M gamúretes Q [Ga*]: Gahmuretes R gamuretes Z Gahmoretes Fr21 \textbf{6} er iesch die] Des iesch er O (L) (M) (Q) Fr21 Der ysch Q Der iesch die Z  $\cdot$ fîanze] fyfancze schierheit R freantze Z \textbf{7} sînen] Einen O  $\cdot$ heter] \textit{om.} O hat er R \textbf{9} sprach] So sprach O M Fr21 Sie sprach \textit{nachträglich korrigiert zu:} So* sprach Q Do sprach R Also sprach Z  $\cdot$ sigelôse] kvne Z \textbf{10} dô] da M R \textbf{11} bin] bin osz M (Q) o\textit{m. } R  $\cdot$ Gahmuret] Gahmvret G L Gamvret O gamuret M (Z) gamúret Q Gahmoret Fr21  $\cdot$ Antschevin] anschevin O (R) anshevin L Z (Fr21) an sevyn M ansheűin Q \textbf{13} die] Do R  $\cdot$ nam] enphie O (L) (M) (Q) (R) (Z) (Fr21)  $\cdot$ er] er in R er \sout{er} Z  $\cdot$ sande in în] [sprach]: sante yn M \textbf{16} begunde gâhen] begvnden nahen O beguͯnt zu iahen Q \textbf{17} Normandie] [normadie]: normandie G ormandie L normadie Q  $\cdot$ Gatschier] catschier G gaitschier M \textbf{18} ein ellens rîcher] Ein ellent richer O Ein [elles]: errent reicher Q Der ellensriche Z Ein ellens riche Fr21  $\cdot$ fier] wert Q \textbf{19} starke] starcken L (Q) (Z) strach Fr21 \textbf{20} hie] Die Fr21  $\cdot$ hielt] helt Q \textbf{21} Gahmuret] Gahmvret G Gamvret O Gahmuͯret L Gamuret M Q Z Gahmoret Fr21  $\cdot$ zer] zv andern Z  $\cdot$ tjost] rosz Q \textbf{22} sînem] Sin R \textbf{24} hie] Alda Z  $\cdot$ werten] werin M warten Q \textbf{25} ungelîchez] vnglich M (Q)  $\cdot$ wac] [was]: wac M \textbf{26} Gatschier] chatschier G Gotischier M Gatschir Q  $\cdot$ gelac] lach O (L) (M) (Q) (R) (Z) (Fr21) \textbf{27} mitalle] vnd mit alle R \textbf{28} der tjoste valle] dem rosse gevallen Q \textbf{29} wart er] Vnd wart Z  $\cdot$ sicherheit] der sicherheyt Q \textbf{30} wære] war Fr21  $\cdot$ liep] liebir M \newline
\end{minipage}
\hspace{0.5cm}
\begin{minipage}[t]{0.5\linewidth}
\small
\begin{center}*T (U)
\end{center}
\begin{tabular}{rl}
 & er r\textit{ei}t ûf in und trat in nider.\\ 
 & \textbf{dô} erholter sich dicke wider.\\ 
 & er tet werlîchen willen schîn.\\ 
 & \textbf{doch} stacte in dem arm sîn\\ 
5 & diu Gahmuretes lanze.\\ 
 & \textbf{der} iesch \textbf{die} fîanze.\\ 
 & sînen meister \textbf{hât} e\textit{r v}unden.\\ 
 & \textit{"wer hât mich überwunden?",}\\ 
 & \textbf{als} sprach der \textbf{küene} man.\\ 
10 & der s\textit{i}ge\textit{h}a\textit{f}te \textbf{sprach} \textbf{aber} sân:\\ 
 & "ich bin Gahmuret Anschevin."\\ 
 & er sprach: "mîn sicherheit sî dîn."\\ 
 & die \textbf{entvienc} er und sante \textit{in} în.\\ 
 & \textbf{dâ solt er sîn gevangen} sîn\\ 
15 & von \textbf{den} vrouwen, die daz sâhen.\\ 
 & dort her begunde gâhen\\ 
 & von Normandie Gatschier,\\ 
 & \textbf{ein} ellens rîcher degen fier\\ 
 & der \textbf{starken} \textit{jo}sti\textit{u}re.\\ 
20 & hie h\textit{i}elt ouch der gehiure\\ 
 & Gahmuret zuo der jost bereit.\\ 
 & sîme sper \textbf{was daz îsen} \textit{b}reit\\ 
 & und der schaft veste.\\ 
 & \textbf{hie} w\textit{e}rten \textbf{sich} die geste\\ 
25 & ein ander. ungelîchez \textbf{doch} wac.\\ 
 & Gatschier dar nider \textbf{lac}\\ 
 & mit \textbf{dem} ors betalle.\\ 
 & von der jost valle\\ 
 & \textbf{war\textit{t} er} betwungen sicherheit,\\ 
30 & ez wære im liep oder leit.\\ 
\end{tabular}
\scriptsize
\line(1,0){75} \newline
U V T \newline
\line(1,0){75} \newline
\textbf{10} \textit{Majuskel} T  \textbf{13} \textit{Majuskel} T  \textbf{16} \textit{Majuskel} T  \newline
\line(1,0){75} \newline
\textbf{1} reit] rath U  $\cdot$ trat] trettet V \textbf{2} dô] des V T \textbf{3} tet] tet [*]: im V \textbf{4} doch stacte] do stecket T \textbf{5} diu] \textit{om.} T  $\cdot$ Gahmuretes] Gahmuͦretes U Gamuretes V \textbf{6} der iesch die] do iesch er T \textbf{7} hât er vunden] hate wuͦnden U den hat er funden V \textbf{8} \textit{Vers 38.8 fehlt} U   $\cdot$ wer hât] [*]: Jr hant V \textbf{9} als] svs T  $\cdot$ küene] sigelose T \textbf{10} sigehafte] sie geachte U  $\cdot$ aber] do T \textbf{11} Gahmuret] Gahmuͦret U Gamuret V  $\cdot$ Anschevin] Anscheuin V \textbf{13} sante in în] sante in U sante [i*]: in in V santin hin in T \textbf{14} [d*]: des mvͤste er wol geprvͤfet sin V des mveser vil gepriset sin T \textbf{17} Gatschier] Gatscier T \textbf{19} jostiure] stiere U \textbf{20} hie hielt ouch] Hie helt auch U oͮch hielt hie V hie hielt T \textbf{21} Gahmuret] Gachmuͦret U Gamuret V \textbf{22} breit] bereit U \textbf{24} werten] wortin U  $\cdot$ sich] \textit{om.} T \textbf{25} doch] \textit{om.} T \textbf{26} Gatschier] Gatscier T \textbf{27} dem] \textit{om.} T \textbf{28} der jost] des tẏostes V \textbf{29} wart] War U \newline
\end{minipage}
\end{table}
\end{document}
