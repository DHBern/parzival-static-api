\documentclass[8pt,a4paper,notitlepage]{article}
\usepackage{fullpage}
\usepackage{ulem}
\usepackage{xltxtra}
\usepackage{datetime}
\renewcommand{\dateseparator}{.}
\dmyyyydate
\usepackage{fancyhdr}
\usepackage{ifthen}
\pagestyle{fancy}
\fancyhf{}
\renewcommand{\headrulewidth}{0pt}
\fancyfoot[L]{\ifthenelse{\value{page}=1}{\today, \currenttime{} Uhr}{}}
\begin{document}
\begin{table}[ht]
\begin{minipage}[t]{0.5\linewidth}
\small
\begin{center}*D
\end{center}
\begin{tabular}{rl}
\textbf{610} & \begin{large}E\end{large}in dinc tuot mir an iu wol,\\ 
 & daz ich mit iu strîten sol.\\ 
 & ouch ist iu \textbf{hôher} prîs geschehen,\\ 
 & daz ich iu \textbf{einem} hân verjehen\\ 
5 & gein iu ze kampfe kumende.\\ 
 & \textbf{ist uns} ze prîse vrumende,\\ 
 & ob wir werde vrouwen\\ 
 & den kampf lâzen schouwen;\\ 
 & \textbf{Vünfzehen} hundert bringe ich dar.\\ 
10 & ir habt ouch eine clâre schar\\ 
 & ûf Schastel Marveile.\\ 
 & \textbf{iu bringet} ziwerm teile\\ 
 & iwer œheim Artus\\ 
 & von eime lande, daz alsus,\\ 
15 & Lœver, ist genennet.\\ 
 & habt ir die stat erkennet,\\ 
 & \textbf{Bems} bî der \textbf{Korca}?\\ 
 & \textbf{diu} messenîe ist elliu dâ.\\ 
 & Von hiute über\textbf{n ahten} \textbf{tac}\\ 
20 & mit grôzer \textbf{joie} er komen mac.\\ 
 & \textbf{von hiute ame sehzendem} tage\\ 
 & kum ich \textbf{durch} mîn alte klage\\ 
 & ûf den plân ze Joflanze\\ 
 & nâch gelte disem kranze."\\ 
25 & \textbf{Der künec} \textbf{Gawanen} \textbf{mit im} bat\\ 
 & ze \textbf{Rosche Sabbins} \textbf{in} die stat:\\ 
 & "ir \textbf{en}mugt niht anderer brücken hân."\\ 
 & dô sprach mîn hêr Gawan:\\ 
 & "ich wil \textbf{hin wider} alse her;\\ 
30 & anders leiste ich iwer ger."\\ 
\end{tabular}
\scriptsize
\line(1,0){75} \newline
D Z \newline
\line(1,0){75} \newline
\textbf{1} \textit{Initiale} D Z  \textbf{9} \textit{Majuskel} D  \textbf{19} \textit{Majuskel} D  \textbf{25} \textit{Majuskel} D  \newline
\line(1,0){75} \newline
\textbf{6} ist uns] Vns ist Z \textbf{9} hundert] hundert frowen Z \textbf{11} Schastel] Scastel D tschahtel Z \textbf{15} Lœver] Loͤver D Z \textbf{17} Bems] Zv Gabins Z  $\cdot$ Korca] choͤrcha D kortha Z \textbf{18} ist] was Z \textbf{20} joie] tschoie Z \textbf{21} ame sehzendem] vͤber den sehtzehenden Z \textbf{23} Joflanze] tschofflantze Z \textbf{25} Gawanen] Gawann D Gawan Z \textbf{26} Rosche Sabbins] Rosce Sabbins D Rotteschesabins Z \textbf{27} brücken] brucke Z \newline
\end{minipage}
\hspace{0.5cm}
\begin{minipage}[t]{0.5\linewidth}
\small
\begin{center}*m
\end{center}
\begin{tabular}{rl}
 & ein dinc tuot mir an iu wol,\\ 
 & daz ich mit iu strîten sol.\\ 
 & ouch ist iu \textbf{hôher} prîs geschehen,\\ 
 & daz ich iu \textbf{eine\textit{m}} hân verjehen\\ 
5 & gegen iu zuo kampf komende\\ 
 & \textbf{und ist} zuo prîs vromende,\\ 
 & ob wir werde vrouwen\\ 
 & den kampf lâzen schouwen;\\ 
 & \textbf{vünfzehen} hundert bringe ich dar.\\ 
10 & ir habt ouch ein clâre schar\\ 
 & ûf Scha\textit{h}tel Mar\textit{ve}ile.\\ 
 & \textbf{si bringt iu} zuo iuwerm teile,\\ 
 & \textbf{vrouwen}, iuwe\textit{r} œheim Artus\\ 
 & von einem lande, daz alsus,\\ 
15 & Lo\textit{ve}r, ist genennet.\\ 
 & habt ir die stat erkennet,\\ 
 & \textbf{reinez Be\textit{m}s} bî der \textbf{\textit{Ko}rca}?\\ 
 & \textbf{des} massenîe ist alliu dâ.\\ 
 & von hiute über \textbf{aht} \textbf{tage}\\ 
20 & mit grôzer \textbf{joi} \dag ich ir\dag  komen \dag mage\dag .\\ 
 & \textbf{von hiute über ahzehen} tage\\ 
 & kum ich \textbf{an} mîn alte klage\\ 
 & ûf den plân zuo \textit{J}oflanze\\ 
 & nâch gelte disem kranze."\\ 
25 & \textbf{der künic} \textbf{Gawanen} bat\\ 
 & zuo \textbf{R\textit{o}sche Sabins} \textbf{in} die stat:\\ 
 & "ir moget niht ander brücke hân."\\ 
 & dô sprach mîn hêr Gawan:\\ 
 & "ich wil \textbf{wider hin} als her;\\ 
30 & anders leist ich iuwer ger."\\ 
\end{tabular}
\scriptsize
\line(1,0){75} \newline
m n o \newline
\line(1,0){75} \newline
\newline
\line(1,0){75} \newline
\textbf{4} einem] einen m n einē o \textbf{11} Vff scastel marnaile m  $\cdot$ Vff kastel morueile n  $\cdot$ Vff scastel marneile o \textbf{13} iuwer] uͯwerm \textit{(krit. Text emendiert nach V#'* ͫ)} m uwern n vwen o \textbf{15} Lover] Lonor m Loner n Louer o \textbf{17} reinez Bems] Reines bemes m Reines beines n Bemes baumes o  $\cdot$ Korca] quarcka m quercka n o \textbf{18} des] Das o \textbf{20} ich] \textit{om.} n o  $\cdot$ komen] koment n  $\cdot$ mage] mag o \textbf{21} ahzehen] sechszehen n xvi o \textbf{22} kum] Kome n \textbf{23} zuo] \textit{om.} n  $\cdot$ Joflanze] choflancz m choflantz n kloflantz o \textbf{26} Rosche Sabins] rotsce sabbins m rotste sabbins n o \textbf{27} brücke] brucken n o \textbf{28} hêr] herre her n \newline
\end{minipage}
\end{table}
\newpage
\begin{table}[ht]
\begin{minipage}[t]{0.5\linewidth}
\small
\begin{center}*G
\end{center}
\begin{tabular}{rl}
 & \textit{ein} dinc tuot mir \textbf{hart} an iu wol,\\ 
 & daz ich mit iu strîten sol.\\ 
 & ouch ist iu \textbf{hôher} prîs geschehen,\\ 
 & daz ich iu \textbf{einem} hân verjehen\\ 
5 & gein i\textit{u z}e kampfe komende.\\ 
 & \textbf{uns ist} ze prîse vromende,\\ 
 & ob wir werde vrouwen\\ 
 & den kampf lâzen schouwen;\\ 
 & \textbf{vünf} hundert \textbf{vrouwen} bringe ich dar.\\ 
10 & ir habet ouch eine clâre schar\\ 
 & ûf Tschastel Marveile.\\ 
 & \textbf{iu bringet} ze iuwerme teile\\ 
 & iuwer œheim Artus\\ 
 & von einem lande, daz alsus,\\ 
15 & Lover, ist genennet.\\ 
 & habe\textit{t i}r die stat erkennet,\\ 
 & \textbf{ze Sabins} bî der \textbf{Chronica}?\\ 
 & \textbf{diu} massenîe ist alliu dâ.\\ 
 & von hiut über \textbf{den ahten} \textbf{tac}\\ 
20 & mit grôzer \textbf{schouwe} er komen mac.\\ 
 & \textbf{dar nâch an dem anderm} tage\\ 
 & kum ich \textbf{durch} mîn alte klage\\ 
 & ûf den plân ze Tschofflanze\\ 
 & nâch gelte disem kranze."\\ 
25 & \textbf{Gramoflanz} \textbf{in} \textbf{mit im} bat\\ 
 & ze \textbf{Roisabins} \textbf{durch} die stat:\\ 
 & "ir muget niht ander brücke hân."\\ 
 & dô sprach mîn hêr Gawan:\\ 
 & "ich wil \textbf{hin} als her;\\ 
30 & anders \textit{leist} ich iuwer ger."\\ 
\end{tabular}
\scriptsize
\line(1,0){75} \newline
G I L M Z Fr34 Fr51 \newline
\line(1,0){75} \newline
\textbf{1} \textit{Initiale} L Z  \textbf{9} \textit{Initiale} I  \textbf{25} \textit{Capitulumzeichen} Fr51  \newline
\line(1,0){75} \newline
\textbf{1} Ein] ::: G Min Fr51  $\cdot$ hart] doch I Fr34 \textit{om.} L M Z Fr51 \textbf{3} iu] nv Fr51  $\cdot$ hôher] grozzer I hoe Fr51  $\cdot$ geschehen] g::: Fr51 \textbf{5} gein iv riten zechamphe chomende G \textbf{7} werde] werdir M  $\cdot$ vrouwen] rittere vnd vrowen L (M) (Fr51) \textbf{9} vünf] Fvnfzehen Z \textbf{11} Tschastel] tschatel G shahte I kastel L iscastel M tschahtel Z  $\cdot$ Marveile] marueile I \textbf{12} iu bringet ze] \textit{om.} I \textbf{13} iuwer] [ewern]: ewer I  $\cdot$ Artus] konnick Artus M \textbf{15} Lover] Louer I (M) (Z) Leover L \textbf{16} habet ir] habet et ir G  $\cdot$ die stat erkennet] stat bekennet Fr51 \textbf{17} ze Sabins] zesabins G (Fr34) Zuͯ [*]: Sabins  L Zcu sabins M Zv Gabins Z Zo sabins Fr51  $\cdot$ Chronica] Chorcha L korcha M Fr51 kortha Z ch::: Fr34 \textbf{18} ist] was Z  $\cdot$ alliu dâ] al da I (Fr51) \textbf{19} über] [anden]: ubir M  $\cdot$ ahten] achzcenden M (Fr51) \textbf{20} schouwe] thsoy I (Z) schone L schir Fr51  $\cdot$ mac] da Fr34 \textbf{21} dar nâch] Von huͯte L (M) (Z) (Fr51)  $\cdot$ an dem] anden M vͤber den Z ufden Fr51  $\cdot$ anderm] sechzehenden L (M) (Z) (Fr51) \textbf{22} alte] aldiu M alten Fr51 \textbf{23} ze] \textit{om.} I  $\cdot$ Tschofflanze] tscheffanze G zeffanze I Schoflanze L schoflancze M tschofflantze Z tscheffranze Fr34 schoiflanz Fr51 \textbf{24} Zo gebebe dessen kranz Fr51  $\cdot$ disem] disen I dises Fr34 \textbf{25} Gramoflanz] Der kvnig Gramoflanz L Der konninck gramorflancz M Der kvnic Gawan Z  $\cdot$ in] \textit{om.} Z \textbf{26} ze Roisabins] zerois sabins G ze Roy sabins I Zuͯ Roý sabinsz L Durkoy sabins M Zv Rotteschesabins Z Zeroys sabins Fr34 Ze rois sabins Fr51  $\cdot$ durch] in L M Z \textbf{27} ir muget] irn mugt I (M) (Z)  $\cdot$ ander] Andisszir M an her Fr34  $\cdot$ brücke] brukken I (Fr51) \textbf{28} dô] Da M  $\cdot$ mîn] \textit{om.} Fr51  $\cdot$ hêr Gawan] ergawan M \textbf{29} hin] hin wider Z \textbf{30} leist ich iuwer] tæte ih iͮwer G niht ist min Fr34 \newline
\end{minipage}
\hspace{0.5cm}
\begin{minipage}[t]{0.5\linewidth}
\small
\begin{center}*T
\end{center}
\begin{tabular}{rl}
 & ein dinc tuot mir an iu wol,\\ 
 & daz ich mit iu strîten sol.\\ 
 & ouch ist iu \textbf{grôzer} prîs geschehen,\\ 
 & daz ich iu hân verjehen\\ 
5 & gein iu zuo kampfe komende.\\ 
 & \textbf{uns ist} zuo prîse vr\textit{o}mende,\\ 
 & ob wir werde \textbf{rîter und} vrouwen\\ 
 & den kampf lâzen schouwen;\\ 
 & \textbf{vünf} hundert \textbf{vrouwen} bring ich dar.\\ 
10 & ir hât ouch eine clâre schar\\ 
 & ûf Tschahtel Marveile.\\ 
 & \textbf{iu bringet} zuo iuwerme teile\\ 
 & iuwer œheim Artus\\ 
 & von eime lande, daz alsus,\\ 
15 & Lover, ist genennet.\\ 
 & hât ir die stat erkennet,\\ 
 & \textbf{Bems} bî der \textbf{Korcha}?\\ 
 & \textbf{diu} massenîe ist alliu dâ.\\ 
 & von hiute über \textbf{den ahten} \textbf{tac}\\ 
20 & mit grôzer \textbf{schœne} er komen mac.\\ 
 & \textbf{von hiute über sehzehen} tage\\ 
 & kum ich \textbf{durch} mîn alte klage\\ 
 & ûf den plân zuo Tschoflanze\\ 
 & nâch gelte diseme kranze."\\ 
25 & \textbf{\begin{large}D\end{large}er künec} \textbf{Gawanen} \textbf{mit im} bat\\ 
 & zuo \textbf{Rotschasabyirs} \textbf{in} die stat:\\ 
 & "ir moget niht ander brücken hân."\\ 
 & dô sprach mîn hêr Gawan:\\ 
 & "ich wil \textbf{hin} als her;\\ 
30 & anders leist ich iuwer ger."\\ 
\end{tabular}
\scriptsize
\line(1,0){75} \newline
U V W Q R \newline
\line(1,0){75} \newline
\textbf{25} \textit{Initiale} U V W  \newline
\line(1,0){75} \newline
\textbf{1} iu] eúch gar W \textbf{3} ouch] Aucb W  $\cdot$ geschehen] [*]: geschehen V beschehen W (R) \textbf{4} daz] Des R  $\cdot$ iu] v́ch eine V eúch einem W (R) euch einen Q \textbf{6} vromende] vremende U [*]: vromende V \textbf{7} werde] \textit{om.} W  $\cdot$ und] vnd auch W \textbf{10} ouch] \textit{om.} W \textbf{11} Of Tschatel marvelei U  $\cdot$ Vf schahtelmarveile V  $\cdot$ Auff kastel marfeile W  $\cdot$ Vff schachtel marueile Q  $\cdot$ Vff schahtel Marveile R \textbf{12} iu] Jr R  $\cdot$ zuo] \textit{om.} V  $\cdot$ iuwerme] einem Q \textbf{13} iuwer œheim] [*]: Vrowen uwer V Eúwer oͤhem der kúnig W Ewrem ohem Q \textbf{14} daz] adasz Q \textbf{15} Lover] Luͦver U Loͮver V Loͤuer W Leuer Q \textbf{17} Bems] Benis V Beras W Gins Q  $\cdot$ Korcha] koicha U korka V R \textbf{18} alliu dâ] [*]: alle da V alda R \textbf{19} ahten] achttenden R \textbf{20} schœne] schoye V W (Q) (R) \textbf{21} über sehzehen tage] an dem (den R ) sechtzehenden tag W (R) am sechzehendem tage Q \textbf{22} durch] dar durch W \textbf{23} Tschoflanze] Tscoflanze U schoflanze V schoflantze W tschofflanze Q schoflancze R \textbf{24} gelte diseme] gelten disen R \textbf{25} Gawanen] gawan W Gawin R  $\cdot$ mit im bat] [*]: mit im riten bat V \textbf{26} Rotschasabyirs] [*]: Rotschesabins V roytschesabins W rotschesabins Q Roitsche sabins R  $\cdot$ in] [*]: in V \textbf{27} niht] kein W  $\cdot$ ander] andere W  $\cdot$ brücken] brucke Q (R) \textbf{29} als her] [*]: wider alse her V \newline
\end{minipage}
\end{table}
\end{document}
