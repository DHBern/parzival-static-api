\documentclass[8pt,a4paper,notitlepage]{article}
\usepackage{fullpage}
\usepackage{ulem}
\usepackage{xltxtra}
\usepackage{datetime}
\renewcommand{\dateseparator}{.}
\dmyyyydate
\usepackage{fancyhdr}
\usepackage{ifthen}
\pagestyle{fancy}
\fancyhf{}
\renewcommand{\headrulewidth}{0pt}
\fancyfoot[L]{\ifthenelse{\value{page}=1}{\today, \currenttime{} Uhr}{}}
\begin{document}
\begin{table}[ht]
\begin{minipage}[t]{0.5\linewidth}
\small
\begin{center}*D
\end{center}
\begin{tabular}{rl}
\textbf{776} & \begin{large}E\end{large}z ist \textbf{selten worden} naht,\\ 
 & wan \textbf{deiz} der sunnen ist geslaht,\\ 
 & si\textbf{ne} \textbf{bræhte} ie den tac dâr nâch.\\ 
 & al daz selbe \textbf{ouch} dâ geschach;\\ 
5 & er schein in süeze, lûter, clâr.\\ 
 & dâ streich manec ritter wol sîn hâr,\\ 
 & dâr ûf \textbf{blüemîniu} schapel.\\ 
 & manec ungevelschet \textbf{vrouwen} vel\\ 
 & man dâ bî \textbf{rôten münden} sach,\\ 
10 & ob Kyot die wârheit sprach.\\ 
 & Ritter und vrouwen truogen gewant,\\ 
 & niht gesniten in eime lant;\\ 
 & \textbf{wîbe} gebende, nider, hôch,\\ 
 & als ez nâch ir lantwîse zôch.\\ 
15 & \textbf{dâ} was ein wît gesamentiu diet,\\ 
 & durch daz ir site \textbf{sich} underschiet:\\ 
 & Swelch vrouwe was sunder amîs,\\ 
 & diu getorste niht \textbf{decheinen gewîs}\\ 
 & über tavelrunde komen.\\ 
20 & het si dienst \textbf{ûf} ir lôn genomen\\ 
 & unt gap \textbf{si} lônes sicherheit,\\ 
 & an tavelrunde \textbf{rinc} si reit.\\ 
 & die andern muosenz lâzen;\\ 
 & in ir herbergen si sâzen.\\ 
25 & Dô Artus \textbf{messe hete} vernomen,\\ 
 & man sach Gramoflanzen komen\\ 
 & \textbf{unt} den herzogen von Gowerzin\\ 
 & unt Floranden, den gesellen sîn.\\ 
 & die \textbf{drî} \textbf{gerten} \textbf{sunder}\\ 
30 & pfliht über tavelrunder.\\ 
\end{tabular}
\scriptsize
\line(1,0){75} \newline
D \newline
\line(1,0){75} \newline
\textbf{1} \textit{Initiale} D  \textbf{11} \textit{Majuskel} D  \textbf{17} \textit{Majuskel} D  \textbf{25} \textit{Majuskel} D  \newline
\line(1,0){75} \newline
\textbf{27} Gowerzin] Gowerzîn D \newline
\end{minipage}
\hspace{0.5cm}
\begin{minipage}[t]{0.5\linewidth}
\small
\begin{center}*m
\end{center}
\begin{tabular}{rl}
 & \begin{large}E\end{large}z ist \textbf{selten worden} naht,\\ 
 & wan \textbf{ez} der sunnen ist geslaht,\\ 
 & si \textbf{brâht} i\textit{e} den tac dâr nâch.\\ 
 & al daz selbe \textbf{ouch} d\textit{â} geschach;\\ 
5 & er schein in süeze, lûter, clâr.\\ 
 & dô streich manic ritter wol sîn hâr,\\ 
 & dâr ûf \textbf{ein} \textbf{blüemîniu} schapel.\\ 
 & manic ungevelschet vel\\ 
 & man d\textit{â} bî \textbf{rôtem munde} sach,\\ 
10 & ob Kiot die wârheit sprach.\\ 
 & ritter und vrowen truogen gewant,\\ 
 & niht gesniten \textit{in} einem lant;\\ 
 & \textbf{wîbe} gebende, nider \textbf{und} hôch,\\ 
 & als ez nâch ir lantwîse zôch.\\ 
15 & \textbf{d\textit{â}} was ein wît gesamet diet,\\ 
 & durch daz ir sit \textbf{si} underschiet:\\ 
 & welich vrowe \textbf{d\textit{â}} was sunder a\textit{mî}s,\\ 
 & diu getorste niht \textbf{dekein wîs}\\ 
 & über tavelrunde komen.\\ 
20 & het si dienst \textbf{ûf} ir lôn genomen\\ 
 & und gap \textbf{si} lônes sicherheit,\\ 
 & an tavelrund\textit{e} \textbf{rinc} si reit.\\ 
 & die andern muostenz lâzen;\\ 
 & in ir herberge si sâzen.\\ 
25 & dô Artus \textbf{messe het} vernomen,\\ 
 & man sach Gramolanzen komen,\\ 
 & de\textit{n} herzogen von Gowertzin\\ 
 & und Floranden, den geselle\textit{n s}în.\\ 
 & die \textbf{drî} \textbf{gerten} \textbf{besund\textit{e}r}\\ 
30 & pfliht über tavelrunder.\\ 
\end{tabular}
\scriptsize
\line(1,0){75} \newline
m n o V V' W \newline
\line(1,0){75} \newline
\textbf{1} \textit{Initiale} m n V W  \textbf{25} \textit{Initiale} V  \newline
\line(1,0){75} \newline
\textbf{1} \textit{Die Verse 775.21-776.24 fehlen} V'  \textbf{3} si brâht] Sv́ enbrehte V  $\cdot$ ie] ir m \textbf{4} al] Aldo n Alle o Als W  $\cdot$ dâ] do m n o V W \textbf{5} er] Ein n \textbf{7} ein] \textit{om.} V  $\cdot$ blüemîniu] bluͯm o bluͦmen W \textbf{8} vel] vrowen vel V \textbf{9} dâ] do m n o V W  $\cdot$ rôtem] roten V \textbf{10} Kiot] kyot V W  $\cdot$ die] der o  $\cdot$ sprach] gesprach W \textbf{12} in] \textit{om.} m n o \textbf{13} wîbe] Nider o \textbf{14} \textit{Verse 776.14-16 kontrahiert zu:} Als es noch ir sitt sich vnder schiet o   $\cdot$ zôch] gezoch V \textbf{15} dâ] Do m n V W  $\cdot$ ein gesamet] eine gesamte V \textbf{16} si] sich n o V W \textbf{17} welich] Wellicher n Swelich V  $\cdot$ dâ] do m n o V W  $\cdot$ amîs] anuͯs m \textbf{18} dekein] do keine n \textbf{19} tavelrunde] tauelrunder W \textbf{20} ûf] [*]: uf V  $\cdot$ ir] irn W \textbf{22} tavelrunde] tafelrundes m n o (W) \textbf{23} muostenz] muͯstens n o muͤszens V muͦstent W \textbf{26} Gramolanzen] gramolantzen m n gramolanczes o Gramaflanzen V gramaflantzen V' gramoflantzen W \textbf{27} den] Der m Vnde den V (V')  $\cdot$ Gowertzin] gowerczin o (V) (V') gouertzein W \textbf{28} gesellen sîn] gesellen min sin m \textbf{29} gerten] begerten so n begerten W  $\cdot$ besunder] besunden der m \newline
\end{minipage}
\end{table}
\newpage
\begin{table}[ht]
\begin{minipage}[t]{0.5\linewidth}
\small
\begin{center}*G
\end{center}
\begin{tabular}{rl}
 & ez ist \textbf{selten worden} naht,\\ 
 & wan \textbf{ez} der sunne ist geslaht,\\ 
 & si\textbf{ne} \textbf{bræhte} ie den tac dâr nâch.\\ 
 & al daz selbe \textbf{ouch} dâ geschach;\\ 
5 & er schein in süeze, lûter, clâr.\\ 
 & dâ streich manic rîter wol sîn hâr,\\ 
 & dâr ûf \textbf{bluomen} schapel.\\ 
 & manic ungevelschet \textbf{vrouwen} vel\\ 
 & man dâ bî \textbf{rôtem munde} sach,\\ 
10 & obe Kiot die wârheit sprach.\\ 
 & rîter unde vrouwen truogen gewant,\\ 
 & niht gesniten in einem lant;\\ 
 & \textbf{vrouwen} gebende, nider, hôch,\\ 
 & als ez nâch ir lantwîse zôch.\\ 
15 & \textbf{ez} was ein wît gesament diet,\\ 
 & durch daz ir site \textbf{sich} underschiet:\\ 
 & swelch vrouwe was sunder amîs,\\ 
 & diu getorste niht \textbf{deheine wîs}\\ 
 & über tavelrunder komen.\\ 
20 & het si dienst \textbf{nâch} ir lône genomen\\ 
 & unde gap \textbf{ir} lônes sicherheit,\\ 
 & an tavelrunder si reit.\\ 
 & die andern muosenz lâzen;\\ 
 & in ir herberge si sâzen.\\ 
25 & dô Artus \textbf{messe het} vernomen,\\ 
 & man sach Gramoflanze komen\\ 
 & \textbf{unde} den herzogen von Gowerzin\\ 
 & unde Floranden, den gesellen sîn.\\ 
 & die \textbf{zwêne} \textbf{gerten} \textbf{sunder}\\ 
30 & pfliht über tavelrunder.\\ 
\end{tabular}
\scriptsize
\line(1,0){75} \newline
G I L M Z Fr48 \newline
\line(1,0){75} \newline
\textbf{1} \textit{Initiale} Z Fr48  \textbf{13} \textit{Initiale} I  \textbf{25} \textit{Initiale} I  \newline
\line(1,0){75} \newline
\textbf{1} selten] shelten I \textbf{2} sunne] sunnen I (L) (Z)  $\cdot$ ist] iht Z \textbf{3} sine] sinen I Si Fr48  $\cdot$ bræhte] bracht L (Fr48)  $\cdot$ ie den tac] den tac ie I e den tac M  $\cdot$ nâch] [naht]: nach Z \textbf{4} ouch] \textit{om.} I \textbf{5} er schein] er ershein I  $\cdot$ in] \textit{om.} L  $\cdot$ süeze lûter] shoͤner suͤze luͤter I lutter suszer M \textbf{6} dâ] do I \textbf{7} ûf] usz M  $\cdot$ bluomen] von bluͦmen I blvͤmine Z (Fr48)  $\cdot$ schapel] stapil M \textbf{8} ungevelschet] vngeuelschtez I  $\cdot$ vel] vil M \textbf{9} dâ] do Fr48  $\cdot$ rôtem] roten I Z Fr48 rotē M  $\cdot$ munde] mvͤnden Z (Fr48) \textbf{10} Kiot] kẏot G Fr48 kýot L kyot Z \textbf{12} einem] ein L \textbf{14} Alsisz yme nach irme lantsite zcoch M  $\cdot$ lantwîse] lande wise I \textbf{15} ez] Da M Fr48 Daz Z \textbf{16} ir site sich] sich ir sit I ir site sy M \textbf{17} swelch] Welch L (M)  $\cdot$ sunder] ane L \textbf{18} diu] Dine M \textbf{19} über] Vnder Z  $\cdot$ tavelrunder] Tavelrvnde L \textbf{20} ir] \textit{om.} L \textbf{21} gap] gab sy M (Z) (Fr48) \textbf{22} tavelrunder] Tavelrvnde L tavelrundir [kome]: rinc M tavelrvnder rinc Z (Fr48)  $\cdot$ si] dy M \textbf{24} herberge] herbergen L  $\cdot$ sâzen] azen L \textbf{25} dô] Da M Z \textbf{26} Gramoflanze] Gramoflanzen I (M) Gramoflanz L gramoflantz Z Gramoflantzen Fr48 \textbf{27} von] \textit{om.} Fr48  $\cdot$ Gowerzin] Gowertzin Fr48 \textbf{28} Floranden] florianden I \textbf{30} über] ob der I ubir der M \newline
\end{minipage}
\hspace{0.5cm}
\begin{minipage}[t]{0.5\linewidth}
\small
\begin{center}*T
\end{center}
\begin{tabular}{rl}
 & \textit{\begin{large}E\end{large}}z ist \textbf{worden selten} naht,\\ 
 & wan \textbf{ez} der sunnen ist geslaht,\\ 
 & si \textbf{en}\textbf{bræhte} ie den tac dâr nâch.\\ 
 & al daz selbe \textbf{al} dâ geschach;\\ 
5 & er schein in süeze, lûter, clâr.\\ 
 & dô streich manec rîter wol sîn hâr,\\ 
 & dâr ûf \textbf{ein} \textbf{bluomen} schapel.\\ 
 & manec ungevelschet \textbf{vrouwen} vel\\ 
 & man d\textit{â} bî \textbf{rôtem munde} sach,\\ 
10 & ob Kyot die wârheit sprach.\\ 
 & rîter und vrouwen truogen gewant,\\ 
 & niht gesniten in eime lant;\\ 
 & \textbf{vrouwen} gebende, nider, hôch,\\ 
 & als ez nâch ir lantwîse zôch.\\ 
15 & \textbf{daz} was ein wîter gesamet diet,\\ 
 & durch daz ir site \textbf{sich} underschiet:\\ 
 & welchiu vrouwe was sunder amîs,\\ 
 & diu get\textit{or}ste niht \textbf{dekeine wîs}\\ 
 & über tavelrunde\textit{r} komen.\\ 
20 & hete si dienst \textbf{ûf} ir \textit{lôn} genomen\\ 
 & und gap \textbf{si ir} lônes sicherheit,\\ 
 & an tavelrunder \textbf{rinc} si reit.\\ 
 & die andern muosen ez lâzen;\\ 
 & in ir herberge si sâzen.\\ 
25 & \begin{large}D\end{large}ô Artus \textbf{hete messe} vernomen,\\ 
 & man sach Gramoflanzen komen\\ 
 & \textbf{und} den herzogen \textit{von} Gauwerzin\\ 
 & und Floranden, den gesellen sîn.\\ 
 & die \textbf{drîe} \textbf{riten} \textbf{sunder}\\ 
30 & pflihte über tavelrunder.\\ 
\end{tabular}
\scriptsize
\line(1,0){75} \newline
U Q R \newline
\line(1,0){75} \newline
\textbf{1} \textit{Initiale} U R  \textbf{25} \textit{Initiale} U R  \newline
\line(1,0){75} \newline
\textbf{1} Ez] Vz U  $\cdot$ worden selten] selden worden Q (R) \textbf{2} wan ez] Wandes R  $\cdot$ sunnen] sunne Q \textbf{3} enbræhte] brechte R \textbf{4} al] Als Q  $\cdot$ al dâ] auch do Q ye R \textbf{5} er] Ein Q  $\cdot$ schein in] schen ye R \textbf{7} ein] \textit{om.} Q von R  $\cdot$ schapel] spappel R \textbf{8} ungevelschet] vngefelscher R \textbf{9} dâ] do U Q R  $\cdot$ rôtem] raten Q \textbf{10} Kyot] koyt U \textbf{11} truogen] [truͯrgen]: truͯgen Q \textbf{13} gebende] gebute Q \textbf{14} ez] es R \textbf{15} daz] Do Q Da R  $\cdot$ wîter] weyt Q (R)  $\cdot$ gesamet] gesamnete R \textbf{17} sunder] von sunder R \textbf{18} getorste] getroste U Q :::st R \textbf{19} tavelrunder] davelrunderer U \textbf{20} ûf] vnd uff Q  $\cdot$ lôn] dinst U \textbf{22} si] die Q \textbf{23} muosen ez] musten Q muͦszent R \textbf{24} \textit{nach 776.24:} Die andern musten laszen Q  \textbf{25} hete messe] messe hett Q (R) \textbf{26} Gramoflanzen] gramofflantzen Q Gramoflanczen R \textbf{27} von] \textit{om.} U  $\cdot$ Gauwerzin] Gauͦwerzin U kawerzin Q goverzin R \textbf{28} und] Von Q \textbf{29} riten] gerten Q R \newline
\end{minipage}
\end{table}
\end{document}
