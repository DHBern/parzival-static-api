\documentclass[8pt,a4paper,notitlepage]{article}
\usepackage{fullpage}
\usepackage{ulem}
\usepackage{xltxtra}
\usepackage{datetime}
\renewcommand{\dateseparator}{.}
\dmyyyydate
\usepackage{fancyhdr}
\usepackage{ifthen}
\pagestyle{fancy}
\fancyhf{}
\renewcommand{\headrulewidth}{0pt}
\fancyfoot[L]{\ifthenelse{\value{page}=1}{\today, \currenttime{} Uhr}{}}
\begin{document}
\begin{table}[ht]
\begin{minipage}[t]{0.5\linewidth}
\small
\begin{center}*D
\end{center}
\begin{tabular}{rl}
\textbf{60} & \textbf{der} \textbf{wolde}, als in sîn hêrre bat,\\ 
 & herberge n\textit{e}men \textbf{in} der stat.\\ 
 & dô was im snellîchen gâch.\\ 
 & man zôch im soumære nâch.\\ 
5 & sîn ouge ninder hûs dâ sach,\\ 
 & schilde w\textit{æ}ren sîn ander dach\\ 
 & unt die wende \textbf{gar} behangen\\ 
 & \multicolumn{1}{l}{ - - - }\\ 
 & mit spern \textbf{al} umbevangen.\\ 
 & Diu künegîn von Waleis\\ 
10 & gesprochen hete \textbf{ze} Kanvoleis\\ 
 & einen turnei alsô gezilt,\\ 
 & des manegen zagen noch bevilt,\\ 
 & \textbf{swâ} er \textbf{dem} gelîche werben siht,\\ 
 & von sîner hant es niht geschiht.\\ 
15 & si was ein maget \textbf{unt} niht ein wîp\\ 
 & \textbf{unt} bôt zwei lant unt ir lîp,\\ 
 & \textbf{swer} dâ den prîs bezalte.\\ 
 & diz mære manegen valte\\ 
 & hinderz ors ûf den sâmen.\\ 
20 & die \textbf{solch} gevelle nâmen,\\ 
 & \textbf{ir} schanze wart \textbf{gein} vlust gesaget.\\ 
 & des pflâgen helde unverzaget.\\ 
 & si tâten rîters ellen schîn.\\ 
 & mit hurteclîcher rabbîn\\ 
25 & wart dâ manec ors ersprenget\\ 
 & unt swerte vil erklenget.\\ 
 & \textit{\begin{large}E\end{large}}in schifbrücke \textbf{ûf} \textbf{einem} plân\\ 
 & gieng über \textbf{einen} wazzers trân,\\ 
 & mit einem tor beslozzen.\\ 
30 & \textbf{ein} knappe unverdrozzen\\ 
\end{tabular}
\scriptsize
\line(1,0){75} \newline
D \newline
\line(1,0){75} \newline
\textbf{9} \textit{Majuskel} D  \textbf{27} \textit{Initiale} D  \newline
\line(1,0){75} \newline
\textbf{2} nemen] niͤmen D \textbf{6} wæren] wâren D \textbf{9} Waleis] Valeis D \textbf{27} ein] ÷in D \newline
\end{minipage}
\hspace{0.5cm}
\begin{minipage}[t]{0.5\linewidth}
\small
\begin{center}*m
\end{center}
\begin{tabular}{rl}
 & \textbf{der} \textbf{wolte}, als in sîn hêrre bat,\\ 
 & herberge nemen \textbf{an} der stat.\\ 
 & dô was ime snelleclîchen gâch.\\ 
 & man zôch ime soumære nâch.\\ 
5 & sîn ouge niender hûs d\textit{â} sach,\\ 
 & schilte \textbf{e\textit{n}}wæren sîn ander dach\\ 
 & und die wende \textbf{gar} behangen\\ 
 & \multicolumn{1}{l}{ - - - }\\ 
 & mit spern \textbf{al}umbevangen.\\ 
 & \begin{large}D\end{large}iu künigîn von Waleis\\ 
10 & gesprochen hete \textbf{zuo} Kanv\textit{o}leis\\ 
 & einen turnei alsô gezilt,\\ 
 & des manigen zagen noch bevilt,\\ 
 & \textbf{wâ} er \textbf{dem} glîche werben siht,\\ 
 & von sîner hant es niht geschiht.\\ 
15 & si was ein maget, niht ein wîp\\ 
 & \textbf{und} bôt zwei lant und ir lîp,\\ 
 & \textbf{wer} d\textit{â} den prîs bezalte.\\ 
 & diz mære manigen valte\\ 
 & hinderz ros ûf den sâmen.\\ 
20 & die \textbf{solichiu} gevelle nâmen,\\ 
 & \textbf{ir} schanze wart \textbf{gegen} \dag vluht\dag  gesaget.\\ 
 & des pflâgen helde unverzaget.\\ 
 & si tâten ritter\textit{s} ellen schîn.\\ 
 & mit herteclîcher rabîn\\ 
25 & wart d\textit{â} manic ros ersprenget\\ 
 & und swerte vil erkl\textit{e}ng\textit{e}t.\\ 
 & \begin{large}E\end{large}in schifbrücke \textbf{ûf} \textbf{einen} plân\\ 
 & gienc über \textbf{eines} wazzers trân,\\ 
 & mit einem tor beslozzen.\\ 
30 & \textbf{der} knappe unverdrozzen\\ 
\end{tabular}
\scriptsize
\line(1,0){75} \newline
m n o \newline
\line(1,0){75} \newline
\textbf{9} \textit{Initiale} m   $\cdot$ \textit{Capitulumzeichen} n  \textbf{27} \textit{Initiale} m  \newline
\line(1,0){75} \newline
\textbf{1} wolte] wol o \textbf{2} an] in n o \textbf{5} niender] man der n o  $\cdot$ dâ] do m n >do< o \textbf{6} schilte] Die Schilt o  $\cdot$ enwæren] eniweren m \textbf{8} spern] sporn o \textbf{9} Waleis] valeis o \textbf{10} hete zuo] hat n  $\cdot$ Kanvoleis] kanveleis m kanfoleis n o \textbf{11} einen turnei] Einer torner o \textbf{13} siht] wilt o \textbf{14} von] Won o  $\cdot$ geschiht] beschicht n (o) \textbf{16} bôt] bat o \textbf{17} dâ] do m n o \textbf{18} diz] Dise n Das o \textbf{19} ros] rosse n \textbf{20} solichiu] sollich n \textbf{23} ritters ellen] ritterschellen m ritters allen o \textbf{24} hurteclîcher] hertteklicher m (n) herteklichen o \textbf{25} dâ] do m n o \textbf{26} erklenget] erklangent m \textbf{27} Ein] [Er]: Ein n  $\cdot$ einen] einem n \newline
\end{minipage}
\end{table}
\newpage
\begin{table}[ht]
\begin{minipage}[t]{0.5\linewidth}
\small
\begin{center}*G
\end{center}
\begin{tabular}{rl}
 & \textbf{der} \textbf{solt}, als in sîn hêrre bat,\\ 
 & herberge nemen \textbf{in} der stat.\\ 
 & dô was im snellîchen gâch.\\ 
 & man zôch im soumære nâch.\\ 
5 & sîn ouge ninder hûs dâ sach,\\ 
 & schilde wæren sîn ander dach\\ 
 & unde die wende \textbf{alsam} behangen\\ 
 & \multicolumn{1}{l}{ - - - }\\ 
 & mit speren umbevangen.\\ 
 & diu künigîn von Waleis\\ 
10 & gesprochen hete \textbf{vor} Kanvoleiz\\ 
 & einen turnei alsô gezilt,\\ 
 & \begin{large}D\end{large}es manigen zagen noch bevilt,\\ 
 & \textbf{swâ} er \textbf{dem} gelîche werben siht,\\ 
 & von sîner hant es niht geschiht.\\ 
15 & si was ein maget, niht ein wîp.\\ 
 & \textbf{si} bôt zwei lant und ir lîp,\\ 
 & \textbf{der} dâ den brîs bezalte.\\ 
 & diz mære manigen valte\\ 
 & hinderz ors ûf den sâmen.\\ 
20 & die \textbf{solch} gevelle nâmen,\\ 
 & \textbf{der} schanze wart \textbf{ze} vlust gesaget.\\ 
 & des pflâgen helde unverzaget.\\ 
 & si tâten rîters ellen schîn.\\ 
 & mit hurticlîcher rabîn\\ 
25 & wart dâ manic ors ersprenget\\ 
 & \textit{und} swer\textit{t} vil \textit{er}klenget.\\ 
 & ein schifbrücke \textbf{an} \textbf{einen} plân\\ 
 & gienc über \textbf{einen} wazzers trân,\\ 
 & mit einem tor beslozzen.\\ 
30 & \textbf{der} knappe unverdrozzen\\ 
\end{tabular}
\scriptsize
\line(1,0){75} \newline
G I O L M Q R Z Fr21 Fr37 Fr44 \newline
\line(1,0){75} \newline
\textbf{1} \textit{Initiale} O  \textbf{9} \textit{Initiale} I  \textbf{12} \textit{Initiale} G  \textbf{27} \textit{Initiale} I L M Q Z Fr37 Fr44  \newline
\line(1,0){75} \newline
\textbf{1} \textit{Die Verse 58.9-63.24 fehlen (Blattverlust)} R   $\cdot$ der] er I (M) (Q) (Z) (Fr21) (Fr37) (Fr44) ÷r O  $\cdot$ solt] wolte Fr44  $\cdot$ als] \textit{om.} Fr37 \textbf{2} nemen] vahen Fr21 \textbf{3} dô] Da Z  $\cdot$ im] \textit{om.} I in L  $\cdot$ snellîchen] snellic I (Fr21) snlliclichen L  $\cdot$ gâch] iach M \textbf{4} zôch] treip L \textbf{5} sîn ouge] Sine oigen M  $\cdot$ ninder] neriken M nirgen Fr44  $\cdot$ hûs dâ] huser I \textbf{6} wæren] warn I (L) (M) (Z) (Fr37) (Fr44) was O Fr21 warden Q  $\cdot$ sîn] ir I \textbf{7} die] \textit{om.} L Fr44  $\cdot$ wende] [hende]: bende O  $\cdot$ alsam behangen] al umb hangen I gar behangen Z \textbf{8} umbevangen] gar vmbe vangen O L (M) (Q) (Z) (Fr21) (Fr37) (Fr44) \textbf{9} Waleis] walais I waleiz L waleisz M Waleys Fr44 \textbf{10} gesprochen] Gesprochet Fr21  $\cdot$ hete] hat L M  $\cdot$ vor] ze O (M) (Q) (Z) Fr21 Fr37 (Fr44)  $\cdot$ Kanvoleiz] ganfolais I kanphalisz M kanúoleisz Q kamfoleis Z kanvoleis Fr21 kamvoleis Fr37 kanuoleys Fr44 \textbf{11} alsô] so I O Q Fr21 \textbf{12} manigen zagen noch] noch mangen zagen O (Fr21)  $\cdot$ bevilt] gefilt Q \textbf{13} swâ] Wa L (M) (Q) (Z)  $\cdot$ er] der I  $\cdot$ gelîche] glichen M \textbf{14} geschiht] gescht Fr37 \textbf{15} niht] vnde nih I (M) (Q) \textbf{16} si] Div O (M) (Q) (Fr21) (Fr37) Fr44 Vnd Z  $\cdot$ ir] ei I \textbf{17} der] Swer O Z Fr21 Fr37 Fr44 Wer L M Q  $\cdot$ dâ] do Q \textit{om.} Fr37 Fr44  $\cdot$ bezalte] gezalte Q \textbf{18} manigen valte] manich valte O manchifalte M \textbf{19} den] dem O dē M \textbf{20} \textit{Versfolge 60.21-22-20} Q   $\cdot$ solch] solche Z  $\cdot$ gevelle] geuellen Fr44 \textbf{21} der] Jr Z  $\cdot$ ze vlust] gein flvst Z zefluse Fr21 ze flucht Fr37  $\cdot$ gesaget] \sout{bezalt} [gezalt]: gesagt I \textbf{23} ellen] eren Q \textbf{24} hurticlîcher] horticlichen M \textbf{25} dâ] do O Q Fr44  $\cdot$ ersprenget] gesprenget Z \textbf{26} und swert] mit swerten G  $\cdot$ erklenget] gechlenget G \textbf{27} ein] Sin M  $\cdot$ schifbrücke] schfbruk Fr37  $\cdot$ an] vber I uff Q  $\cdot$ einen] ein O Fr44 eynē M (Q) einem Z \textbf{28} gienc] was gerihtet I Sie L  $\cdot$ einen] eins I (L) (Fr44) \textbf{29} einem] einen I einē Q \newline
\end{minipage}
\hspace{0.5cm}
\begin{minipage}[t]{0.5\linewidth}
\small
\begin{center}*T (U)
\end{center}
\begin{tabular}{rl}
 & \textbf{er} \textbf{solt} \textbf{tuon}, als in sîn hêrre bat,\\ 
 & herberge nemen \textbf{in} der stat.\\ 
 & dô was im snellîchen gâch.\\ 
 & man zôch im soumære nâch.\\ 
5 & sîn ouge ni\textit{n}der \textit{h}ûs dâ sach,\\ 
 & schilte, \textbf{die} w\textit{æ}ren sîn ander dach\\ 
 & und die wende \textbf{alsam} behangen\\ 
 & von banieren manecvach,\\ 
 & mit spern \textbf{gar} umbevangen.\\ 
 & \begin{large}D\end{large}iu künegîn von Waleis\\ 
10 & gesprochen hete \textbf{vor} Kanvoleis\\ 
 & einen turnei alsô gezilt,\\ 
 & des manigen zagen noch bevilt,\\ 
 & \textbf{wenne}r \textbf{die} glîch werben siht,\\ 
 & von sîner hant ez niht geschiht.\\ 
15 & si was ein maget, niht ein wîp.\\ 
 & \textbf{diu} bôt zwei lant und ir lîp,\\ 
 & \textbf{wer} dâ den prîs bezalte.\\ 
 & diz mære manigen valte\\ 
 & hinder\textit{z} ors ûf den sâmen.\\ 
20 & die \textbf{soli\textit{ch}iu} gevelle nâmen,\\ 
 & \textbf{der} \textit{sch}anze wart \textbf{zuo} verluste gesaget.\\ 
 & des pflâgen helde unverzaget.\\ 
 & si tâten rîters elle\textit{n} schîn.\\ 
 & mit hurteclîcher rabîn\\ 
25 & wart dô manec ors ersprenget\\ 
 & und swerte vil er\textit{k}lenget.\\ 
 & \begin{large}E\end{large}in schifbrücke \textbf{an} \textbf{eime} plân\\ 
 & gienc über \textbf{eines} wazzers trân,\\ 
 & mit eime tor beslozzen.\\ 
30 & \textbf{der} knappe unverdrozzen\\ 
\end{tabular}
\scriptsize
\line(1,0){75} \newline
U V W T \newline
\line(1,0){75} \newline
\textbf{3} \textit{Initiale} T  \textbf{9} \textit{Initiale} U   $\cdot$ \textit{Majuskel} T  \textbf{15} \textit{Majuskel} T  \textbf{23} \textit{Majuskel} T  \textbf{27} \textit{Initiale} U W   $\cdot$ \textit{Majuskel} T  \textbf{30} \textit{Majuskel} T  \newline
\line(1,0){75} \newline
\textbf{1} tuon] \textit{om.} V W T \textbf{3} snellîchen] [s*elleklichen]: snelleklichen V \textbf{4} \textit{Vers 60.4 fehlt} T   $\cdot$ soumære] seine saumere W \textbf{5} ouge] oͮgen V  $\cdot$ ninder] nider U niergent V nidert W  $\cdot$ hûs] vͦz U  $\cdot$ dâ] [*]: do V do W \textbf{6} \textit{nach 60.6:} von banieren manec vach T   $\cdot$ die] \textit{om.} V W T  $\cdot$ wæren] waren U (W) \textbf{6} \textit{Vers 60.6#'1 fehlt} U V W  \textbf{8} umbevangen] bevangen T \textbf{10} vor] zuͦ W  $\cdot$ Kanvoleis] kanuoleis W \textbf{12} des manigen zagen] Das vilmangen W \textbf{13} \textit{Die Verse 60.13-14 fehlen} T   $\cdot$ wenner die] [*]: Swa er dem V Wo er den W \textbf{14} ez] er W \textbf{17} wer] [*]: Swer V swer T  $\cdot$ dâ] [*]: do V do W \textbf{19} hinderz] hinder U \textbf{20} solichiu] selege U [*]: solich V sollich W (T) \textbf{21} schanze] ganze U  $\cdot$ zuo verluste] [*]: zer fluht V \textbf{23} ellen] ellenden U \textbf{25} dô manec ors] manig roß do W da manec ors T \textbf{26} swerte vil] manig schwert W  $\cdot$ erklenget] [e*]: erlenget U \textbf{27} Ein] SEin W  $\cdot$ schifbrücke] schiffunge V  $\cdot$ an eime] an einen V zeinem T \textbf{28} eines] ein V  $\cdot$ wazzers trân] wasserzstran V (W) \newline
\end{minipage}
\end{table}
\end{document}
