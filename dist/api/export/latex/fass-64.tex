\documentclass[8pt,a4paper,notitlepage]{article}
\usepackage{fullpage}
\usepackage{ulem}
\usepackage{xltxtra}
\usepackage{datetime}
\renewcommand{\dateseparator}{.}
\dmyyyydate
\usepackage{fancyhdr}
\usepackage{ifthen}
\pagestyle{fancy}
\fancyhf{}
\renewcommand{\headrulewidth}{0pt}
\fancyfoot[L]{\ifthenelse{\value{page}=1}{\today, \currenttime{} Uhr}{}}
\begin{document}
\begin{table}[ht]
\begin{minipage}[t]{0.5\linewidth}
\small
\begin{center}*D
\end{center}
\begin{tabular}{rl}
\textbf{64} & \textbf{si sageten}z in vür \textbf{unbetrogen}.\\ 
 & dô begunden \textbf{si} \textbf{an} die brücke zogen,\\ 
 & ander volc unt \textbf{ouch} \textbf{die} sîne.\\ 
 & von dem liehten schîne,\\ 
5 & der von der künegîn \textbf{erschein},\\ 
 & \textbf{zuckt im} neben sich sîn bein.\\ 
 & ûf \textbf{regte} sich der degen wert\\ 
 & als ein vederspil, daz gert.\\ 
 & diu herberge \textbf{dûht in} guot.\\ 
10 & alsô stuont des heldes muot.\\ 
 & Si dolt ouch wol, diu wirtîn,\\ 
 & von Waleis diu künegîn.\\ 
 & \textbf{Dô} \textbf{vriesch} der künec von Spane,\\ 
 & daz ûf der Leoplane\\ 
15 & \textbf{stüende} ein gezelt, daz Gahmurete\\ 
 & durch des küenen Razaliges bete\\ 
 & beleip vor Patelamunt.\\ 
 & daz tet im ein riter kunt.\\ 
 & dô vuor er springende als ein tier.\\ 
20 & \textbf{er} was der \textbf{vreuden} soldier.\\ 
 & der selbe rîter aber sprach:\\ 
 & "iwer muomen sun ich sach\\ 
 & kumende, als er \textbf{ê} was, fier.\\ 
 & ez sint hundert banier\\ 
25 & zuo \textbf{eime} schilde ûf grüene velt\\ 
 & \textbf{gestôzen} vür sîn hôch gezelt.\\ 
 & die sint \textbf{ouch} alle grüene.\\ 
 & \textbf{ouch} hât der helt küene\\ 
 & \begin{large}D\end{large}rî hermîn anker \textbf{lieht} gemâl\\ 
30 & ûf \textbf{ieslîchen} zindâl."\\ 
\end{tabular}
\scriptsize
\line(1,0){75} \newline
D Fr9 \newline
\line(1,0){75} \newline
\textbf{11} \textit{Majuskel} D  \textbf{13} \textit{Initiale} Fr9   $\cdot$ \textit{Majuskel} D  \textbf{29} \textit{Initiale} D  \newline
\line(1,0){75} \newline
\textbf{5} künegîn erschein] kvninginnen schein Fr9 \textbf{6} zuckt im] Des zvchter Fr9 \textbf{7} regte] richte Fr9 \textbf{11} Si dolt ouch] Ouch dolte in Fr9 \textbf{13} vriesch] vreisch Fr9  $\cdot$ Spane] Spâne D \textbf{14} Leoplane] leoplâne D lewe plane Fr9 \textbf{15} Gahmurete] Gahmvrete D gamvrete Fr9 \textbf{16} Razaliges] Razalîges D \textbf{17} beleip] Bliep Fr9  $\cdot$ Patelamunt] Patêlamvnt D \textbf{24} sint] sin wol Fr9 \textbf{30} ieslîchen] ieslichem Fr9 \newline
\end{minipage}
\hspace{0.5cm}
\begin{minipage}[t]{0.5\linewidth}
\small
\begin{center}*m
\end{center}
\begin{tabular}{rl}
 & \textbf{man sagete} ez in vür \textbf{unbetrogen}.\\ 
 & dô begunde\textit{n} \textbf{si} \textbf{an} die brücken zogen,\\ 
 & ander volc und \textbf{ouch} \textbf{daz} sîne.\\ 
 & von dem liehten schîne,\\ 
5 & der von der künigîn \textbf{schein},\\ 
 & \textbf{dô zuckete er} neben sich sîn bein.\\ 
 & ûf \textbf{rihtete} sich der degen wert\\ 
 & als ein vederspil, daz gert.\\ 
 & diu herberge \textbf{dûhte in} guot.\\ 
10 & als \textbf{dô} stuont des heldes muot.\\ 
 & si dolte ouch wol, diu wirtîn,\\ 
 & von Waleis diu künigîn.\\ 
 & \textbf{\begin{large}D\end{large}ô} \textbf{verhiez} der künic von Spane,\\ 
 & daz ûf der Lewe plane\\ 
15 & \textbf{stüende} ein gezelt, daz Gahmurete\\ 
 & durch des küenen \textit{R}azaliges bete\\ 
 & bleip vor Patelamunt.\\ 
 & daz tet im ein ritter kunt.\\ 
 & dô vuor er springende als ein tier.\\ 
20 & \textbf{er} was der \textbf{vrowen} soldier.\\ 
 & der selbe ritter aber sprach:\\ 
 & "i\textit{uwer} muomen sun ich sach\\ 
 & k\textit{o}mende, als er \textbf{ie} was, fier.\\ 
 & ez sint \textbf{wol} hundert banier\\ 
25 & ze \textbf{einem} schilte ûf grüene velt\\ 
 & \textbf{gestôzen} vür sîn hôch gezelt.\\ 
 & die sint \textbf{ouch} alle grüene.\\ 
 & \textbf{ouch} het der helt küene\\ 
 & drîe hermî\textit{n} anker \textbf{lieht} gemâl\\ 
30 & ûf \textbf{ieglîchem} zindâl."\\ 
\end{tabular}
\scriptsize
\line(1,0){75} \newline
m n o \newline
\line(1,0){75} \newline
\textbf{13} \textit{Initiale} m   $\cdot$ \textit{Capitulumzeichen} n  \newline
\line(1,0){75} \newline
\textbf{1} vür] \textit{om.} n o \textbf{2} begunden] begundencz m  $\cdot$ brücken] brucke n (o) \textbf{3} ouch daz] an der n \textbf{6} zuckete] [zucket]: zuckete m zvckt o \textbf{7} rihtete] richtet n o \textbf{8} daz] disz n \textbf{10} dô] \textit{om.} o \textbf{13} verhiez] verheis o \textbf{14} Lewe plane] loewe plane m louwe plane n lowe plone o \textbf{15} stüende] Stunt n (o)  $\cdot$ Gahmurete] gahmurette m gemúret n gamuret o \textbf{16} Razaliges] bazaliges m n bezalies o \textbf{17} Patelamunt] patlamunt m \textbf{19} springende] springen n spr:ngen o \textbf{22} iuwer] Jre m Jr n o \textbf{23} komende] kamende m Komen n o \textbf{25} ze] Wol zuͯ n  $\cdot$ grüene] gruͯnem n \textbf{26} gestôzen] Gestochet o \textbf{29} hermîn] herminen m (n) hermeneẏn o \newline
\end{minipage}
\end{table}
\newpage
\begin{table}[ht]
\begin{minipage}[t]{0.5\linewidth}
\small
\begin{center}*G
\end{center}
\begin{tabular}{rl}
 & \textbf{si seiten}z in vür \textbf{ungelogen}.\\ 
 & dô begunden \textbf{über} die brücke zogen\\ 
 & ander volc und \textbf{die} sîne.\\ 
 & von dem liehten schîne,\\ 
5 & der von der küniginne \textbf{schein},\\ 
 & \textbf{derzucte im} neben sich sîn bein.\\ 
 & ûf \textbf{rihte} sich der degen wert\\ 
 & \textbf{reht} als ein vederspil, daz gert.\\ 
 & diu herberge \textbf{dûht in} guot.\\ 
10 & alsô stuont des heldes muot.\\ 
 & si dolt ouch wol, diu wirtîn,\\ 
 & von Waleis diu künigîn.\\ 
 & \textbf{nû} \textbf{vriesch} der künic von Spange,\\ 
 & daz ûf der Lewen plange\\ 
15 & \textbf{stüende} ein gezelt, daz Gahmuret\\ 
 & durch des küenen Razaliges bet\\ 
 & beleip vor Patelamunt.\\ 
 & daz tet im ein \textbf{sîn} rîter kunt.\\ 
 & dô vuor er springende als ein tier.\\ 
20 & \textbf{er} was der \textbf{vröuden} soldier.\\ 
 & der selbe rîter aber sprach:\\ 
 & "iwer muomensun ich sach\\ 
 & komen, als er \textbf{ie} was, fier.\\ 
 & ez sint \textbf{wol} hundert banier\\ 
 & \hspace*{-.7em}\big| \textbf{gest\textit{ô}zen} vür sîn hôch gezelt\\ 
25 & \hspace*{-.7em}\big| zuo \textbf{einem} schilte ûf grüene velt.\\ 
 & die sint \textbf{ouch} alle grüene.\\ 
 & \textbf{ez} hât der helt küene\\ 
 & drî härmîn anker \textbf{wol} gemâl\\ 
30 & ûf \textbf{iegelîchen} zendâl."\\ 
\end{tabular}
\scriptsize
\line(1,0){75} \newline
G I O L M Q R Z Fr37 Fr44 \newline
\line(1,0){75} \newline
\textbf{1} \textit{Initiale} O  \textbf{7} \textit{Initiale} I  \textbf{13} \textit{Capitulumzeichen} L  \textbf{27} \textit{Überschrift:} Hie qvam gamuret zv waleis Z   $\cdot$ \textit{Initiale} I M Z Fr37 Fr44  \newline
\line(1,0){75} \newline
\textbf{1} si] ÷i O  $\cdot$ seitenz in] sagtenz im O (M) (Fr44) sagten ez L seiten [inz]: esz Q sagten imz Fr37  $\cdot$ ungelogen] vnerlogen R \textbf{2} dô] Da M Z  $\cdot$ begunden] begvndens O (L) (Q) (Z) (Fr37) begunde sie M beguncz R  $\cdot$ brücke zogen] brucken czyn M \textbf{3} \textit{Vers 64.3 fehlt} O   $\cdot$ \textit{Versfolge 64.4-3} R   $\cdot$ die] ouch daz L ovch die Z daz Fr44  $\cdot$ sîne] synen M (Q) \textbf{4} liehten] vil liehtem I lychten L (M) (Q) (Fr44) liechtem Fr37  $\cdot$ schîne] scheinen Q \textbf{5} von] vor R  $\cdot$ küniginne] kúnginnen R (Fr44)  $\cdot$ schein] erschein O (M) Z Fr37 [schein]: er schein L \textbf{6} zucht er im ein bein vf daz bein I  $\cdot$ derzucte] Dv zvht er O Er zuͯchte L (M) (Fr44) der zucht Fr37  $\cdot$ im neben sich] neben im O (M) enneben sich L ym [lieben]: neben Q im nahen Fr37 neben sich Fr44  $\cdot$ sîn] daz Fr44 \textbf{7} \textit{Versfolge 64.8-7} O   $\cdot$ degen] \textit{om.} I \textbf{8} reht] \textit{om.} L Fr37  $\cdot$ daz] daz da L (M) (R) (Fr37) \textbf{9} dûht in] in duhte I \textbf{10} heldes] [helde]: heldes Q helden R \textbf{11} si] ez I  $\cdot$ dolt] dolte O L (M) Q Fr44  $\cdot$ ouch wol] avch O wol Q \textbf{12} Waleis] walois I waileic M wayleis Fr37 \textbf{13} vriesch] erfrisch M [erfeisches]: erfrische Q vernam R  $\cdot$ der] den M  $\cdot$ von] \textit{om.} Q  $\cdot$ Spange] spanie O M (Fr44) Jspanie L spangen Z spanne Fr37 \textbf{14} der] dem R  $\cdot$ Lewen plange] lewe planie O L (Fr44) lewe plane M [leuprange]: leup lange Q leuplange R lewen plangen Z zeswen planie Fr37 \textbf{15} stüende] Stund Q (Z) (Fr44)  $\cdot$ Gahmuret] Gahmurete I Gamuret O (Z) Gahmuͯret L gamurete M gamûert Q [gal]: gahmuret R Gamuͦret Fr44 \textbf{16} durch] Das M Doch Q  $\cdot$ küenen] chunges Fr37 (Fr44)  $\cdot$ Razaliges] razalies G kazaliges M Fr44 \textbf{17} Patelamunt] pantalamunt I patelamut Q petalamunt R Pathelamunt Fr44 \textbf{18} daz] Dis L  $\cdot$ sîn] \textit{om.} O L M Q R Z Fr37 Fr44 \textbf{19} dô] Da Z Du Fr37  $\cdot$ springende] springen L (Q)  $\cdot$ tier] ber Z \textbf{20} er] Vnde O (L) M (Q) (R) (Z) (Fr37) (Fr44)  $\cdot$ vröuden] [frowen]: frawen I (Fr44)  $\cdot$ soldier] soldierer Z so fier Fr37 \textbf{21} selbe] selber R \textbf{22} iwer] iwere G \textbf{23} komen] Chomende O (L) (Q) (R) (Z)  $\cdot$ er] \textit{om.} M  $\cdot$ ie] E R \textit{om.} Z Fr44 \textbf{24} wol] \textit{om.} O L M Q R Z Fr37 Fr44 \textbf{26} \textit{Versfolge 64.25-26} O L M Q R Z Fr37 Fr44   $\cdot$ gestôzen] gestoͮzen G Gestochen Fr44  $\cdot$ vür] uff M  $\cdot$ hôch] Rich R \textbf{25} einem] einen I einē Q  $\cdot$ grüene] gruͤnem I ein grvͦne O >dasz< grún Q \textbf{27} ouch] \textit{om.} L Fr37 \textbf{28} ez] Avch O (L) (M) (Q) (R) (Z) (Fr37) (Fr44) \textbf{29} \textit{Vers 64.29 fehlt} R   $\cdot$ wol] lieht O (L) Z licht M Q Fr44  $\cdot$ gemâl] ginal M \textbf{30} iegelîchen] iegelichem I (O) (L) (Q) (R) (Z) iezleichen ein Fr37 \newline
\end{minipage}
\hspace{0.5cm}
\begin{minipage}[t]{0.5\linewidth}
\small
\begin{center}*T (U)
\end{center}
\begin{tabular}{rl}
 & \textbf{man sagete}z in vür \textbf{ungelogen}.\\ 
 & dô begunden \textbf{si} \textbf{über} die brücke zogen,\\ 
 & ander volc und \textbf{die} sîne.\\ 
 & von dem liehten schîne,\\ 
5 & der von der küneginne \textbf{schein},\\ 
 & \textbf{zucket im} neben sich sîn bein.\\ 
 & ûf \textbf{rihte} sich der degen wert\\ 
 & \textbf{rehte} als ein vederspil, daz gert.\\ 
 & diu herberge \textbf{in dûhte} \textbf{harte} guot.\\ 
10 & als \textbf{dô} stuont des heldes muot.\\ 
 & si dolte ouch wol, diu wirtîn,\\ 
 & von Waleis diu künigîn.\\ 
 & \textbf{\begin{large}D\end{large}ô} \textbf{ervriesch} der künec von Spanie,\\ 
 & daz ûf der Lewe planie\\ 
15 & \textbf{stuont} ein gezelte, daz Gahmurete\\ 
 & durch des küenen Razaliges bete\\ 
 & bleip vor Patelamunt.\\ 
 & daz tet im ein ritter kunt.\\ 
 & dô vuor er springende als ein tier\\ 
20 & \textbf{und} was der \textbf{vreuden} soldier.\\ 
 & der selbe ritter aber \textbf{dô} sprach:\\ 
 & "iuwer muomen sun ich sach\\ 
 & komen, als er \textbf{ie} was, fier.\\ 
 & ez sint hundert banier\\ 
 & \hspace*{-.7em}\big| \textbf{gestecket} vür sîn hôch gezelt\\ 
25 & \hspace*{-.7em}\big| zuo \textbf{sîme} schilte ûf grüenez velt.\\ 
 & die sint alle grüene.\\ 
 & \textbf{ouch} hât der helt küene\\ 
 & drîe härmîn anker \textbf{lieht} gemâl\\ 
30 & ûf \textbf{ieclîchem} zindâl."\\ 
\end{tabular}
\scriptsize
\line(1,0){75} \newline
U V W T \newline
\line(1,0){75} \newline
\textbf{11} \textit{Majuskel} T  \textbf{13} \textit{Initiale} U T  \textbf{19} \textit{Majuskel} T  \textbf{21} \textit{Majuskel} T  \textbf{22} \textit{Majuskel} T  \textbf{27} \textit{Initiale} W  \newline
\line(1,0){75} \newline
\textbf{1} man sagetez in] Sv́ seitent es in V Man sagt ins W daz sach ich iv T \textbf{2} begunden si] begunden [*]: sv́ V begund W (T) \textbf{5} schein] erschein W  $\cdot$ küneginne] kuͦneginnen U \textbf{6} zucket im] Zuhte er V [*]: zvhtim T \textbf{8} rehte als] Recht alsam W als am T \textbf{9} in dûhte] duhte in V (W) (T)  $\cdot$ harte] rehte V \textit{om.} T \textbf{10} als dô] Alse V (W) \textbf{12} Waleis] walleis V waleiß W \textbf{13} Dô] Nv T  $\cdot$ ervriesch] friesch V (T) erfuͦr W  $\cdot$ Spanie] spane V hyspanie W \textbf{14} Lewe planie] lewen planie U Leoplane V planie W \textbf{15} Gahmurete] Gahmuͦrete U Gamurette V gamuret W \textbf{16} Razaliges] razzaliches W  $\cdot$ bete] gebet W \textbf{17} Patelamunt] Patelamuͦnt U petelamunt V patelamund W \textbf{19} als] alsam W \textbf{20} der vreuden] \textit{om.} W  $\cdot$ soldier] soldenier V \textbf{21} dô] \textit{om.} V T \textbf{22} iuwer] Uwerre V Juwern T  $\cdot$ sach] do sach W \textbf{23} komen] Kvmmende V \textbf{24} hundert] wol hvndert V \textbf{26} \textit{Versfolge 64.25-26} V W T   $\cdot$ gestecket] Gestochen W gestôzen T  $\cdot$ hôch] hochs W \textbf{25} zuo sîme schilte] zeiner slihte T \textbf{27} alle] oͮch alle V (T) \textbf{29} lieht] licht W \newline
\end{minipage}
\end{table}
\end{document}
