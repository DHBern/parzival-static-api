\documentclass[8pt,a4paper,notitlepage]{article}
\usepackage{fullpage}
\usepackage{ulem}
\usepackage{xltxtra}
\usepackage{datetime}
\renewcommand{\dateseparator}{.}
\dmyyyydate
\usepackage{fancyhdr}
\usepackage{ifthen}
\pagestyle{fancy}
\fancyhf{}
\renewcommand{\headrulewidth}{0pt}
\fancyfoot[L]{\ifthenelse{\value{page}=1}{\today, \currenttime{} Uhr}{}}
\begin{document}
\begin{table}[ht]
\begin{minipage}[t]{0.5\linewidth}
\small
\begin{center}*D
\end{center}
\begin{tabular}{rl}
\textbf{41} & gereht ze bêden sîten,\\ 
 & küene, dâ man solde strîten,\\ 
 & \textit{\begin{large}V\end{large}}erhalden unt dræte.\\ 
 & waz \textbf{er} dâr ûfe tæte?\\ 
5 & des muoz \textbf{ich} \textbf{im} \textbf{vür ellen} jehen.\\ 
 & er reit, dâ in môren \textbf{mohten} sehen,\\ 
 & al dâ \textbf{die} lâgen mit ir her\\ 
 & westerhalp \textbf{dor\textit{t}} \textbf{an} dem mer.\\ 
 & ein vürste Razalik dâ hiez.\\ 
10 & \textbf{deheinen tac} daz \textbf{nimmer} liez\\ 
 & der rîcheste von Azagouc\\ 
 & - sîn geslehte \textbf{im des} niht \textbf{louc},\\ 
 & von küneges \textbf{vruhte} was sîn art -,\\ 
 & \textbf{der} \textbf{huop} \textbf{sich} immer dannewart\\ 
15 & durch tjostieren \textbf{gein der} stat.\\ 
 & \textbf{al} dâ tet sîner krefte m\textit{a}t\\ 
 & der \textbf{helt von} Anschouwe.\\ 
 & daz klagete ein swarziu vrouwe,\\ 
 & diu in \textbf{hete dar} gesant,\\ 
20 & daz in dâ iemen überwant.\\ 
 & Ein knappe bôt \textbf{al} sunder bete\\ 
 & sîme hêrren Gahmurete\\ 
 & ein sper, dem was der schaft ein rôr.\\ 
 & dâ mite stach er den môr\\ 
25 & hinder\textbf{z ors} ûf\textbf{en} griez.\\ 
 & niht langer er in ligen liez,\\ 
 & \textbf{dâ twanc in} sicherheit \textbf{sîn hant}.\\ 
 & dô was daz urliuge \textbf{lant}\\ 
 & unt \textbf{im ein grôzer prîs} geschehen.\\ 
30 & Gahmuret begunde sehen\\ 
\end{tabular}
\scriptsize
\line(1,0){75} \newline
D Fr14 \newline
\line(1,0){75} \newline
\textbf{3} \textit{Initiale} D  \textbf{21} \textit{Majuskel} D  \newline
\line(1,0){75} \newline
\textbf{3} verhalden] ÷erhalden D \textbf{8} dort] dor D \textbf{9} Razalik] Rasalik D Razalich Fr14 \textbf{11} Azagouc] Azagoͮch D Azagovch Fr14 \textbf{12} louc] enloͮch Fr14 \textbf{13} küneges] k. Fr14 \textbf{16} mat] maht D \textbf{17} Anschouwe] Anscoͮwe D Anscowe Fr14 \textbf{18} klagete] chlaget Fr14 \textbf{22} Gahmurete] Gahmvrete D Fr14 \textbf{30} Gahmuret] Gahmvret D Fr14 \newline
\end{minipage}
\hspace{0.5cm}
\begin{minipage}[t]{0.5\linewidth}
\small
\begin{center}*m
\end{center}
\begin{tabular}{rl}
 & gereht ze beiden sîten,\\ 
 & küene, d\textit{â} man solte strîten,\\ 
 & verhalten und d\textit{r}æte.\\ 
 & waz \textbf{der} dâr ûf tæte?\\ 
5 & des muoz \textit{\textbf{ich}} \textit{\textbf{nû}} \textbf{\textit{vü}r ellen} jehen.\\ 
 & er reit, d\textit{â} in môre \textbf{mohten} sehen,\\ 
 & al dâ \textbf{die} lâgen mit ir her\\ 
 & westerhalp \textbf{dort} \textbf{an} dem mer.\\ 
 & \textit{\begin{large}E\end{large}i}n vürste Razali\textit{c} d\textit{â} hiez.\\ 
10 & \textbf{d\textit{urc}h \textit{k}ein\textit{e} tât} \textbf{er} daz \textbf{niht} liez,\\ 
 & der rîcheste von Azagouc\\ 
 & - sîn geslehte \textbf{ime des} niht \textbf{louc},\\ 
 & von küniges \textbf{vruhte} was sîn art -,\\ 
 & \textbf{er} \textbf{hüebe} \textbf{sich} iemer dannewart\\ 
15 & durch justieren \dag gegen vür die\dag  stat.\\ 
 & \textbf{al}dâ tet sîner kref\textit{t}e m\textit{a}t\\ 
 & der \textbf{helt von} Anschouwe.\\ 
 & daz klagete ein swarziu vrouwe,\\ 
 & diu in \textbf{hete dar} gesant,\\ 
20 & daz i\textit{n} dâ iemen überwant.\\ 
 & ein knappe bôt \textbf{al} sunder bete\\ 
 & sîne\textit{m} hêrren Gahmurete\\ 
 & ein sper, dem was der schaft ein rôr.\\ 
 & dâ mite stach er den môr\\ 
25 & \dag hinder\dag  \textbf{ros} ûf \textbf{ein} griez.\\ 
 & niht langer er in ligen liez,\\ 
 & \textbf{er twanc in} sicherheit \textbf{zehant}.\\ 
 & d\textit{ô} was daz urliuge \textbf{verrant}\\ 
 & und \textbf{ime ein grôzer brîs} geschehen.\\ 
30 & Gahmuret begunde sehen\\ 
\end{tabular}
\scriptsize
\line(1,0){75} \newline
m n o W \newline
\line(1,0){75} \newline
\textbf{9} \textit{Initiale} m W   $\cdot$ \textit{Capitulumzeichen} n  \newline
\line(1,0){75} \newline
\textbf{1} gereht] Zuͦ recht n (o) (W) \textbf{2} dâ] do m n o wo W  $\cdot$ man] man do W \textbf{3} verhalten] Verhalte o  $\cdot$ dræte] diette m \textbf{4} der] er n o W \textbf{5} ich nû vür] [iuwir]: juͯwir m  $\cdot$ ellen] allen W \textbf{6} dâ] do m n o W  $\cdot$ môre] moͤrin W \textbf{9} Ein] [Din]: Den m Er n  $\cdot$ vürste] [furte]: furste o  $\cdot$ Razalic] Razalit m retzalit n retzalet o rasalick W  $\cdot$ dâ] do m n o W \textbf{10} durch keine tât] Docheinen dot m Durch keme das n \textbf{11} Azagouc] azagouͯck m azagouck n azagauck o \textbf{14} er hüebe sich] Er huͦp sich n o \textbf{15} gegen vür] fúr n (o) \textbf{16} sîner] siner siner o  $\cdot$ krefte] kreffe m  $\cdot$ mat] maht m \textbf{17} von] won o  $\cdot$ Anschouwe] [anschonwe]: anschouwe m aschouwe n anschowe o \textbf{20} in] im m  $\cdot$ dâ] do n o \textbf{21} al sunder] besunder o \textbf{22} sînem] Sinen m  $\cdot$ Gahmurete] Gahmurette m gamirette n gemúret o \textbf{24} mite] mit do n \textbf{25} hinder] Hinder \textit{nachträglich korrigiert zu:} Hindersz m Húnder n (o) \textbf{27} er] [Ein]: Er m \textbf{28} dô] Da m \textbf{30} Gahmuret] Gamiret n Gamuͯret o \newline
\end{minipage}
\end{table}
\newpage
\begin{table}[ht]
\begin{minipage}[t]{0.5\linewidth}
\small
\begin{center}*G
\end{center}
\begin{tabular}{rl}
 & gereht ze beiden sîten,\\ 
 & küene, dâ man solte strîten,\\ 
 & verhalden und dræte.\\ 
 & waz \textbf{er} dâr ûffe tæte?\\ 
5 & des muoz \textbf{man} \textbf{im} \textbf{vür ellen} jehen.\\ 
 & er reit, dâ in môre \textbf{muosen} sehen,\\ 
 & al dâ \textbf{die} lâgen mit ir her\\ 
 & westerthalben \textbf{bî} dem mer.\\ 
 & ein vürste Razalic dâ hiez.\\ 
10 & \textbf{neheinen morgen} \textbf{er} daz liez,\\ 
 & der rîcheste von Azagouc\\ 
 & - sîn geslähte \textbf{in dâr an} niht \textbf{betrouc},\\ 
 & von küniges \textbf{vruhte} was sîn art -,\\ 
 & \textbf{der} \textbf{kêrte} imer dannewart\\ 
15 & durch tjostieren \textbf{vür die} stat.\\ 
 & dâ tet sîner krefte mat\\ 
 & der \textbf{vürste ûz} Anschouwe.\\ 
 & daz klagte ein swarziu vrowe,\\ 
 & diu in \textbf{hete dar} gesant,\\ 
20 & daz in dâ iemen überwant.\\ 
 & ein knappe, \textbf{der} bôt sunder bet\\ 
 & sînem hêrren Gahmuret\\ 
 & ein sper, dem was der schaft ein rôr.\\ 
 & dâ mit stach er den môr\\ 
25 & hinder\textbf{z ors} ûf \textbf{den} griez.\\ 
 & niht langer er in ligen liez,\\ 
 & \textbf{in dwünge} sicherheit \textbf{sîn hant}.\\ 
 & dô was daz urliuge \textbf{gelant}\\ 
 & unde \textbf{im ein grôzer brîs} geschehen.\\ 
30 & Gahmuret begunde sehen\\ 
\end{tabular}
\scriptsize
\line(1,0){75} \newline
G O L M Q R Z Fr21 \newline
\line(1,0){75} \newline
\textbf{1} \textit{Initiale} M  \textbf{9} \textit{Initiale} L Q R Z Fr21  \newline
\line(1,0){75} \newline
\textbf{1} gereht] Rechte M \textbf{2} dâ] do O Q R \textbf{5} des] Das Q R  $\cdot$ man] ich O L M Q R Z (Fr21)  $\cdot$ ellen] [ellen]: erren Q alle R \textbf{6} dâ] das Q R  $\cdot$ Môre] moren O L (M) R Z (Fr21) dy moren Q  $\cdot$ muosen] mohten Z \textbf{7} al dâ] Da M Do Q  $\cdot$ die] si O (L) (M) (Q) Fr21 die da R \textbf{8} westerthalben bî] Wester halb dort an O (L) (M) (Q) (R) (Fr21) Dort westerhalp an Z \textbf{9} ein] Sin Fr21  $\cdot$ Razalic] razalich G (O) (L) balasich M rasalik Q bazalic Z kazalic Fr21  $\cdot$ dâ] do Q das R \textbf{10} neheinen] Dehin Fr21  $\cdot$ morgen] tach O (L) (M) (Q) (R) (Z) (Fr21)  $\cdot$ er daz] er nimmer O (L) (Q) (R) (Fr21) er nicht da M daz nimmer Z  $\cdot$ liez] verliez Fr21 \textbf{11} rîcheste] rich ist O reisszte M  $\cdot$ von] vor M  $\cdot$ Azagouc] azagoͮch G azagavch O Azagovch L azagoyc M azaguck Q \textbf{12} in dâr an] im des L dar an M in dar vnd Q  $\cdot$ betrouc] travch O (M) (Q) louch L \textbf{13} vruhte] geslechte M \textbf{14} kêrte] hvp Z kert Fr21  $\cdot$ imer] sich ie Z \textit{om.} Fr21  $\cdot$ dannewart] an die vart L (R) dar wart M da enwart Z \textbf{15} tjostieren] tiosten ovch Z \textbf{16} dâ tet] Das tet in L Do tet Q Da tet in R Alda tet Z \textbf{17} vürste ûz] helt van O (L) (M) (Q) (R) (Z) (Fr21)  $\cdot$ Anschouwe] anschoͮwe G anschawe O Anshawe L askouwe M anshorne Q anschowe R (Fr21) antschowe Z \textbf{18} klagte] chlagt O (Q) (R) (Z) (Fr21) clage L  $\cdot$ swarziu] schwercze R \textbf{19} hete dar] dar hette Q hat dar R \textbf{20} in] \textit{om.} O Q  $\cdot$ dâ iemen] do nymant Q \textbf{21} der] \textit{om.} M Z  $\cdot$ bôt] bat Q  $\cdot$ sunder] ane L al sunder Z \textbf{22} sînem] Sinen L  $\cdot$ hêrren] herre R  $\cdot$ Gahmuret] Gamvret O Gahmuͯret L gamuraten M gamuert Q gamurete Z Gahmoret Fr21 \textbf{23} der] sin O Fr21 die M  $\cdot$ schaft] schaff R \textbf{24} den] dem O \textbf{25} hinderz] Vnter Q  $\cdot$ den] daz L (R) \textbf{27} in dwünge] Jn entwunge O Ome twunge M Er zwunge Q Da twanc in Z Jn betwnge Fr21  $\cdot$ sîn hant] den man Q \textbf{28} dô] Da M Z  $\cdot$ gelant] genant M lan Q \textbf{30} Gahmuret] Gahmvret G L Gamvret O Gamurat M Gamúret Q Gamureten Z Gahmoret Fr21  $\cdot$ begunde] begvnden Z \newline
\end{minipage}
\hspace{0.5cm}
\begin{minipage}[t]{0.5\linewidth}
\small
\begin{center}*T (U)
\end{center}
\begin{tabular}{rl}
 & gereh\textit{t} ze beiden sîten,\\ 
 & küene, dâ man solte strîten,\\ 
 & verhalten und dræte.\\ 
 & waz \textbf{er} dâr ûf tæte?\\ 
5 & des muoz \textbf{ich} \textbf{im} \textbf{von schulde} jehen.\\ 
 & er reit, d\textit{â} in môre \textbf{muosen} sehen,\\ 
 & al dâ \textbf{si} lâgen mit ir her\\ 
 & westerhalp \textbf{dort} \textbf{an} dem mer.\\ 
 & ein vürste Razalic d\textit{â} hiez,\\ 
10 & \textbf{der dekeinen tac} daz \textbf{niht} liez\\ 
 & der rîcheste von Azagouc\\ 
 & - sîn geslehte \textbf{in dâr an} niht \textbf{betrouc},\\ 
 & von küneges \textbf{zuht} was sîn art -,\\ 
 & \textbf{d\textit{er}} \textbf{reit} iemer dannewart\\ 
15 & durch jostieren \textbf{gein der} stat.\\ 
 & dâ tet sîner krefte mat\\ 
 & der \textbf{helt von} Anschouw\textit{e}.\\ 
 & daz klaget \textit{ein} swarziu vrouw\textit{e},\\ 
 & diu in \textbf{dar hâ\textit{te}} gesant,\\ 
20 & daz in dâ ieman überwant.\\ 
 & ein knappe bôt sunder bete\\ 
 & sîme hêrren Gahmurete\\ 
 & ein sper, d\textit{em} was der schaft ein rôr.\\ 
 & dâ mit stach er den môr\\ 
25 & hinder \textbf{sich} ûf \textbf{den} griez.\\ 
 & niht langer er in ligen liez,\\ 
 & \textbf{in betwa\textit{n}g} sicherheit \textbf{sîn hant}.\\ 
 & dô was daz urliuge \textbf{gelant}\\ 
 & und \textbf{der prîs im aldâ} geschehen.\\ 
30 & Gahmuret begunde sehen\\ 
\end{tabular}
\scriptsize
\line(1,0){75} \newline
U V W T \newline
\line(1,0){75} \newline
\textbf{9} \textit{Initiale} V T  \textbf{15} \textit{Majuskel} T  \textbf{21} \textit{Majuskel} T  \newline
\line(1,0){75} \newline
\textbf{1} gereht] Gerech U \textbf{2} dâ] do V \textbf{4} tæte] stete V \textbf{5} von schulde] von schulden V vur ellen T \textbf{6} dâ] do U V aldaz T  $\cdot$ in môre] men in V  $\cdot$ muosen] muͤste V mvesen T \textbf{7} al dâ] aldaz T \textbf{8} an] [*]: an V bi T \textbf{9} Razalic] Razalig V  $\cdot$ dâ] do U V \textbf{10} Keinen tag der nie verließ W  $\cdot$ daz] da V \textit{om.} T  $\cdot$ liez] enlies V (T) \textbf{11} Azagouc] Aeagovch U azagoug V W Azagôvc T \textbf{12} an] \textit{om.} W  $\cdot$ betrouc] trôvc T \textbf{13} zuht] fruht V (W) (T) \textbf{14} der] Do U  $\cdot$ reit] kerte T  $\cdot$ dannewart] uf die vart V \textbf{15} durch] \textit{om.} T  $\cdot$ gein der] vur die T \textbf{16} dâ] Do W \textbf{17} Anschouwe] Anschowen U Anschowe V antschowe W Anschôuwe T \textbf{18} klaget] clagete V T  $\cdot$ ein swarziu vrouwe] swarze vreuͦwen U ein swarze vroͮwe T \textbf{19} dar hâte] dar hant U dar [*]: hatte V hate dar T \textbf{20} dâ] do V W  $\cdot$ überwant] widerwant W \textbf{21} bôt] der bot W \textbf{22} sîme] Seinen W  $\cdot$ Gahmurete] Gahmuͦrete U Gamurete V gamurette W Gahmvretete T \textbf{23} dem] daz U  $\cdot$ der] sein W \textbf{24} stach] so stach V \textbf{25} hinder sich] hinders ors V (W) (T)  $\cdot$ den] das W \textbf{27} in betwang] Jn betwag U Do twang in V Im betwunge W in twunge T \textbf{28} gelant] lant W \textbf{29} der prîs im aldâ] im der pris alda V im der preiß als W im ein grôz pris T \textbf{30} Gahmuret] Gamuͦret U Gamuret V W  $\cdot$ begunde sehen] der begunde spehen W \newline
\end{minipage}
\end{table}
\end{document}
