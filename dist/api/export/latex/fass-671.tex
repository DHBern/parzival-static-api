\documentclass[8pt,a4paper,notitlepage]{article}
\usepackage{fullpage}
\usepackage{ulem}
\usepackage{xltxtra}
\usepackage{datetime}
\renewcommand{\dateseparator}{.}
\dmyyyydate
\usepackage{fancyhdr}
\usepackage{ifthen}
\pagestyle{fancy}
\fancyhf{}
\renewcommand{\headrulewidth}{0pt}
\fancyfoot[L]{\ifthenelse{\value{page}=1}{\today, \currenttime{} Uhr}{}}
\begin{document}
\begin{table}[ht]
\begin{minipage}[t]{0.5\linewidth}
\small
\begin{center}*D
\end{center}
\begin{tabular}{rl}
\textbf{671} & \begin{large}A\end{large}ls tet diu küneginne, sîn wîp.\\ 
 & \textbf{diu} \textbf{enpfienc} Gawans lîp\\ 
 & unt ander sîne geselleschaft\\ 
 & mit \textbf{getriulîcher} \textbf{liebe} kraft.\\ 
5 & Dâ wart manec kus getân\\ 
 & von maneger vrouwen wolgetân.\\ 
 & Artus sprach zem neven sîn:\\ 
 & "wer sint die gesellen dîn?"\\ 
 & Gawan sprach: "mîne vrouwen,\\ 
10 & sol ich \textbf{si} \textbf{küssen} schouwen,\\ 
 & daz wære unsanfte bewart;\\ 
 & si sint wol bêde von der art."\\ 
 & Der Turkote Florant\\ 
 & wart \textbf{dâ} geküsset \textbf{al} zehant\\ 
15 & unt der herzoge von Gowerzin\\ 
 & v\textit{on} Ginovern, der künegîn.\\ 
 & Si giengen wider inz gezelt.\\ 
 & manegen dûhte, daz daz \textbf{wîte} velt\\ 
 & vollez \textbf{vrouwen} wære.\\ 
20 & dô warp niht sô der swære\\ 
 & Artus spranc ûf ein kastelân.\\ 
 & al \textbf{dise} vrouwen wolgetân\\ 
 & unt al die ritter \textbf{beneben} in,\\ 
 & er reit den rinc alumbe hin.\\ 
25 & mit zühten Artuses munt\\ 
 & si enpfienc an der selben stunt.\\ 
 & daz was Gawans wille,\\ 
 & \textbf{daz} si alle habten stille,\\ 
 & unze daz er \textbf{mit in dannen} rite;\\ 
30 & daz was ein \textbf{höveschlîcher} site.\\ 
\end{tabular}
\scriptsize
\line(1,0){75} \newline
D Fr8 Fr10 \newline
\line(1,0){75} \newline
\textbf{1} \textit{Initiale} D  \textbf{5} \textit{Majuskel} D  \textbf{13} \textit{Majuskel} D  \textbf{17} \textit{Majuskel} D  \newline
\line(1,0){75} \newline
\textbf{7} Artus] Arthus Fr8 \textbf{13} Turkote] turkoẏte Fr8 \textbf{16} von] vnt D  $\cdot$ Ginovern] Gẏnoueren Fr8 \textbf{21} Artus] Arthus Fr8 \textbf{25} Artuses] Artvs D Arthuses Fr8 \textbf{26} selben] \textit{om.} Fr8 \newline
\end{minipage}
\hspace{0.5cm}
\begin{minipage}[t]{0.5\linewidth}
\small
\begin{center}*m
\end{center}
\begin{tabular}{rl}
 & alsô tet diu künigîn, sîn wîp.\\ 
 & \textbf{diu} \textbf{enpfienc} Gawanes lîp\\ 
 & und ander sîn geselleschaft\\ 
 & mit \textbf{getriuw\textit{e}lîcher} \textbf{wirde} kraft.\\ 
5 & d\textit{â} wart manic kus getân\\ 
 & von maniger vrowen wol getân.\\ 
 & Artus sprach zuom neven sîn:\\ 
 & "wer sint die gesellen dîn?"\\ 
 & Gawan sprach: "mîne vrouwen,\\ 
10 & sol ich \textbf{si} \textbf{küssen} schouwen,\\ 
 & daz wær unsanfte bewar\textit{t};\\ 
 & si sint wol beide von der ar\textit{t}."\\ 
 & der \textit{T}ur\textit{k}oite Florant\\ 
 & wart \textbf{d\textit{â}} geküsset \textbf{al}zehant\\ 
15 & und der herzog\textit{e} von Gowertzin\\ 
 & von Ginover\textit{e}n, der künigîn.\\ 
 & si giengen wider in daz gezelt.\\ 
 & manigen dûhte, daz daz velt\\ 
 & volle\textit{z} \textbf{vrowen} wære.\\ 
20 & dô warp niht sô der swære\\ 
 & Artus spranc ûf ein kastelân.\\ 
 & al \textbf{dise} vrowen wol getân\\ 
 & und alle die ritter \textbf{beneben} in,\\ 
 & er reit den rinc al umbe hin.\\ 
25 & mit zühten Artuses munt\\ 
 & si enpfienc an der selben stunt.\\ 
 & daz was Gawanes wille,\\ 
 & \textbf{daz} si alle habte\textit{n s}tille,\\ 
 & unz daz er \textbf{mit in dannen} \textit{r}ite;\\ 
30 & daz was ein \textbf{hovelîch\textit{er}} site.\\ 
\end{tabular}
\scriptsize
\line(1,0){75} \newline
m n o Fr69 \newline
\line(1,0){75} \newline
\newline
\line(1,0){75} \newline
\textbf{4} getriuwelîcher] getruwclicher m  $\cdot$ wirde] liebe n o Fr69 \textbf{5} dâ] Do m n o \textbf{7} Artus] Artuͯs o  $\cdot$ neven sîn] nefe mẏn o \textbf{11} bewart] bewarte m o \textbf{12} art] arte m o \textbf{13} Turkoite] kurtoite m ::: Fr69  $\cdot$ Florant] ::: Fr69 \textbf{14} dâ] do m n o \textbf{15} herzoge] hertzogin m  $\cdot$ Gowertzin] gowerczin o \textbf{16} von] Vnd o  $\cdot$ Ginoveren] ginoverin m ginoferen n ginovern o \textbf{19} vollez] Foller m n o \textbf{22} al dise] Aldisese o \textbf{25} Artuses] arttuses m artúses o \textbf{28} habten stille] habtten schulde stille m heubten stille o \textbf{29} rite] [reit]: reite m \textbf{30} hovelîcher] hoffelich m \newline
\end{minipage}
\end{table}
\newpage
\begin{table}[ht]
\begin{minipage}[t]{0.5\linewidth}
\small
\begin{center}*G
\end{center}
\begin{tabular}{rl}
 & \begin{large}A\end{large}lsô tet diu künigîn, sîn wîp.\\ 
 & \textbf{si} \textbf{enpfiengen} Gawans lîp\\ 
 & unde ander sîne geselleschaft\\ 
 & mit \textbf{zuhtlîcher} \textbf{liebe} kraft.\\ 
5 & dâ wart manic kus getân\\ 
 & von maniger vrouwen wolgetân.\\ 
 & Artus sprach zem neven sîn:\\ 
 & "wer sint die gesellen dîn?"\\ 
 & Gawan sprach: "mîne vrouwen\\ 
10 & sol ich \textbf{küssen} schouwen,\\ 
 & daz wære unsanfte bewart;\\ 
 & si sint wol bêde von der art."\\ 
 & der Turkoite Florant\\ 
 & wart geküsset \textbf{sân} zehant\\ 
15 & unde der herzoge von Gowerzin\\ 
 & von Schinoveren, der künigîn.\\ 
 & si giengen wider in daz gezelt.\\ 
 & manigen dûhte, daz daz velt\\ 
 & vollez \textbf{rîter} wære.\\ 
20 & dô warp niht sô der swære\\ 
 & Artus spranc ûf ein kastelân.\\ 
 & al \textbf{dise} vrouwen wolgetân\\ 
 & unde al die rîter \textbf{neben} in,\\ 
 & er reit den rinc alumbe hin.\\ 
25 & mit zühten Artuses munt\\ 
 & si enpfienc an der selben stunt.\\ 
 & daz was Gawans wille:\\ 
 & si alle habten stille,\\ 
 & unze daz er \textbf{mit in dannen} rite;\\ 
30 & daz was ein \textbf{hovelîcher} site.\\ 
\end{tabular}
\scriptsize
\line(1,0){75} \newline
G I L M Z Fr61 \newline
\line(1,0){75} \newline
\textbf{1} \textit{Initiale} G L Z  \textbf{5} \textit{Initiale} I  \textbf{21} \textit{Initiale} M  \textbf{25} \textit{Initiale} I  \newline
\line(1,0){75} \newline
\textbf{1} sîn] di syn M \textbf{2} Gawans] Gawansz L Gawanes Fr61 \textbf{4} zuhtlîcher] getruwer L (Fr61) getruwelichir M (Z) \textbf{6} wolgetân] lobesan Z \textbf{7} Artus] Artaus Fr61 \textbf{10} ich] ich sie Z (Fr61)  $\cdot$ küssen] kussende I \textbf{11} unsanfte] in sanfte Fr61 \textbf{12} wol] \textit{om.} Fr61 \textbf{13} Turkoite] Turchoyde I Tuͯrkoite L Turkoit Z Turkoyt Fr61  $\cdot$ Florant] florant : G floriant I \textbf{14} sân] \textit{om.} Fr61 \textbf{15} unde] von Fr61  $\cdot$ Gowerzin] Goruerzin I Gowerzein Fr61 \textbf{16} von] \textit{om.} Fr61  $\cdot$ Schinoveren] Cinoveren G Ginofern I Gýnovern L ginovern M Gynover Z Ginover Fr61  $\cdot$ künigîn] herczogin M \textbf{17} si giengen] Gie Fr61  $\cdot$ gezelt] zelt I \textbf{18} daz daz] wie daz I daz daz wite Z \textbf{19} rîter] frowen Z \textbf{20} dô] Da M Z Er Fr61  $\cdot$ warp] enwarp L M (Fr61)  $\cdot$ sô] als Fr61 \textbf{21} Artus] Artuͯs L Artaus Fr61 \textbf{22} dise] die L \textbf{23} neben in] nebenim I en neben in L \textbf{24} alumbe] vmbe Fr61  $\cdot$ hin] [hant]: hin I \textbf{25} Artuses] Artvs G (Z) Artauses Fr61 \textbf{27} Gawans] Gawanes Fr61 \textbf{28} si] Daz sie Z  $\cdot$ alle habten] haptan vil I hapten alle Fr61 \textbf{29} unze] Bisz M  $\cdot$ daz] \textit{om.} Fr61  $\cdot$ in] nit in L in von Fr61 \textbf{30} hovelîcher] hoveslicher L \newline
\end{minipage}
\hspace{0.5cm}
\begin{minipage}[t]{0.5\linewidth}
\small
\begin{center}*T
\end{center}
\begin{tabular}{rl}
 & als tet diu künigîn, sîn wîp.\\ 
 & \textbf{si} \textbf{enpfiengen} Gawans lîp\\ 
 & und ander sîn geselleschaft\\ 
 & mit \textbf{getriulîcher} \textbf{liebe} kraft.\\ 
5 & d\textit{â} wart manic kus getân\\ 
 & von maneger vrouwen wol getân.\\ 
 & Artus sprach zuom neven sîn:\\ 
 & "wer sint die gesellen dîn?"\\ 
 & Gawan sprach: "mîne vrouwen,\\ 
10 & so\textit{l} ich \textbf{si} \textbf{küssende} schouwen,\\ 
 & daz wære unsanfte bewart;\\ 
 & si sint wol bêde von der art."\\ 
 & der Turkoyte Florant\\ 
 & wart geküsset \textbf{sân} zehant\\ 
15 & und der herzoge von Gowerzin\\ 
 & von Gynovern, der künigîn.\\ 
 & si giengen wider in daz gezelt.\\ 
 & manegen dûhte, daz daz velt\\ 
 & vollez \textbf{ritter} wære.\\ 
20 & dô warp niht sô der swære\\ 
 & Artus spranc ûf ei\textit{n} kastelân.\\ 
 & al \textbf{die} vrouwen wol getân\\ 
 & und alle die ritter \textbf{neben} in,\\ 
 & er reit den rinc al umbe hin.\\ 
25 & mit zühten Artuses munt\\ 
 & si enpfienc an der selben stunt.\\ 
 & daz was Gawans wille,\\ 
 & \textbf{daz} si alle habten stille,\\ 
 & unz daz er \textbf{danne mit im} rite;\\ 
30 & daz was ein \textbf{hovelîcher} site.\\ 
\end{tabular}
\scriptsize
\line(1,0){75} \newline
Q R W V \newline
\line(1,0){75} \newline
\textbf{7} \textit{Initiale} R V  \newline
\line(1,0){75} \newline
\textbf{1} künigîn] herczogin R \textbf{2} enpfiengen] empfienge R  $\cdot$ Gawans] Gewins R herr gawans W gawanes V \textbf{3} und] Vnd auch W \textbf{4} liebe] liebu R \textbf{5} dâ] Do Q W V  $\cdot$ wart] ward auch W \textbf{6} wol getân] [*]: vnde man V \textbf{7} sprach] \textit{om.} V \textbf{8} wer] Sprach wer V \textbf{9} Gawan] Gawann Q Gawin R Herr gawan W  $\cdot$ mîne] minne R \textbf{10} sol] So Q  $\cdot$ si] \textit{om.} R  $\cdot$ küssende] kússen R W (V) \textbf{12} wol] \textit{om.} W \textbf{13} Turkoyte] [kurkoite]: Turkoite Q Turkoite R \textbf{14} sân] do R \textbf{15} Gowerzin] kawerzin Q Gowerczin R \textbf{16} Von [Gin*]: Ginovern der kv́nigin V  $\cdot$ Gynovern] gynoren Q Ginowern R tschinouern W  $\cdot$ der] die R \textbf{19} [*]: Alles vol ritter were V  $\cdot$ vollez] Voller W  $\cdot$ wære] do were W \textbf{20} dô] So R  $\cdot$ sô] [*]: so V \textbf{21} spranc] sprach vnd sprang R  $\cdot$ ein] einen Q ein guͦttes R \textbf{22} al die] Alle dise R [Aldi*]: Aldise V \textbf{23} neben] vnd R \textbf{24} al] \textit{om.} R \textbf{25} Artuses] artus Q (R) kúnig artus W \textbf{26} si enpfienc] Enpfieng sv́ V  $\cdot$ selben] seben V \textbf{27} Gawans] Gawins R herr gawans W [gawa*]: gawanez V \textbf{28} alle habten] allesampt habtent W alle [h*]: habeten V  $\cdot$ stille] [*]: stille V \textbf{29} danne mit im] mit Jnnen dannen R (W) (V) \textbf{30} hovelîcher] hoͤueschlicher W \newline
\end{minipage}
\end{table}
\end{document}
