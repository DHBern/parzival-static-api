\documentclass[8pt,a4paper,notitlepage]{article}
\usepackage{fullpage}
\usepackage{ulem}
\usepackage{xltxtra}
\usepackage{datetime}
\renewcommand{\dateseparator}{.}
\dmyyyydate
\usepackage{fancyhdr}
\usepackage{ifthen}
\pagestyle{fancy}
\fancyhf{}
\renewcommand{\headrulewidth}{0pt}
\fancyfoot[L]{\ifthenelse{\value{page}=1}{\today, \currenttime{} Uhr}{}}
\begin{document}
\begin{table}[ht]
\begin{minipage}[t]{0.5\linewidth}
\small
\begin{center}*D
\end{center}
\begin{tabular}{rl}
\textbf{233} & \begin{large}N\end{large}âch \textbf{den} \textbf{kom} ein herzogîn\\ 
 & unt ir gespil. zwei stöllelîn\\ 
 & \textbf{si truogen} von helfenbein.\\ 
 & ir munt nâch viwers rœte schein.\\ 
5 & die nigen alle viere.\\ 
 & zwô sazten schiere\\ 
 & vür den wirt die stollen.\\ 
 & dâ wart gedient mit vollen.\\ 
 & \textbf{die} stuonden ensamt an \textbf{eine} schar\\ 
10 & \textbf{unt} wâren alle wol gevar.\\ 
 & \textbf{den} vieren was gelîch ir wât.\\ 
 & \textbf{nû} seht, wâ sich niht versûmet hât\\ 
 & ander \textbf{vrouwen} vierstunt zwuo.\\ 
 & \textbf{die wâren dâ geschaffet zuo}:\\ 
15 & \textbf{viere} truogen kerzen grôz.\\ 
 & die anderen viere niht verdrôz,\\ 
 & si\textbf{ne} trüegen einen tiuren stein,\\ 
 & dâ tages diu sunne lieht durchschein.\\ 
 & \textbf{dâ vür was sîn name} erkant:\\ 
20 & ez was ein grânât jâchant,\\ 
 & beide lanc unt breit.\\ 
 & durch die \textbf{lîhte} in dünne sneit,\\ 
 & swer in zeime tische maz.\\ 
 & dâ \textbf{obe} der wirt durch rîcheit az.\\ 
25 & \textbf{Si} giengen \textbf{harte} rehte\\ 
 & vür den \textbf{wirt} al ehte.\\ 
 & \textbf{gein} nîgen si ir houbet wegeten.\\ 
 & viere die taveln legeten\\ 
 & \textbf{ûf helfenbein} \textit{w}îz als ein snê,\\ 
30 & stollen, die \textbf{dâ} kômen ê.\\ 
\end{tabular}
\scriptsize
\line(1,0){75} \newline
D \newline
\line(1,0){75} \newline
\textbf{1} \textit{Initiale} D  \textbf{25} \textit{Majuskel} D  \newline
\line(1,0){75} \newline
\textbf{18} diu] de D \textbf{19} sîn] [si]: sin D \textbf{29} wîz] viz D \newline
\end{minipage}
\hspace{0.5cm}
\begin{minipage}[t]{0.5\linewidth}
\small
\begin{center}*m
\end{center}
\begin{tabular}{rl}
 & nâch \textbf{den} \textbf{zwein} \textbf{kam} ein herzogîn\\ 
 & und ir gespil. zwei stölle\textit{l}în\\ 
 & \textbf{si truogen} von helfenbein.\\ 
 & ir munt nâch \textit{viur}es rœte schein.\\ 
5 & die nigen alle viere.\\ 
 & zwô saste\textit{n} schiere\\ 
 & vür \textit{d}en wirt die st\textit{o}llen.\\ 
 & d\textit{â} wart gedienet mit volle\textit{n}.\\ 
 & \textbf{die} stuonden samet an \textbf{einer} schar\\ 
10 & \textbf{und} wâren alle wol gevar.\\ 
 & \textbf{disen} vieren was gelîch ir wât.\\ 
 & seht, wâ sich niht versûmet hât\\ 
 & anderre \textbf{vrouwen} vierstunt zwô.\\ 
 & \textbf{den was geboten alsô},\\ 
15 & \textbf{daz} \textbf{ir viere} truogen kerzen grôz.\\ 
 & die anderen viere niht verdrôz,\\ 
 & si trüegen einen tiuren stein,\\ 
 & dâ tages diu sunne lieht durchschein.\\ 
 & \textbf{dâ vür was sîn name} erkant:\\ 
20 & ez was ein grân\textit{â}t jâchant,\\ 
 & beide lanc und breit.\\ 
 & durch die \textbf{lîhte} in dünne sneit,\\ 
 & wer in ze einem tische maz.\\ 
 & dar \textbf{abe} \textit{der wirt} durch \textbf{eine} rîcheit az.\\ 
25 & \textbf{si} giengen \textbf{harte} rehte\\ 
 & vür den \textbf{wirt} alle ehte.\\ 
 & \textbf{gegen} nîgen si ir houbet wegeten.\\ 
 & viere die \textit{t}a\textit{v}el\textit{e}n legeten\\ 
30 & \hspace*{-.7em}\big| \textbf{ûf} stollen, die \textbf{dâ} kômen ê,\\ 
 & \hspace*{-.7em}\big| \textbf{helfenbeinîn} wîz als ein snê.\\ 
\end{tabular}
\scriptsize
\line(1,0){75} \newline
m n o Fr69 \newline
\line(1,0){75} \newline
\newline
\line(1,0){75} \newline
\textbf{2} stöllelîn] stoͯllein m \textbf{4} viures] wes m \textbf{6} sasten] saster m \textbf{7} den] dien m  $\cdot$ stollen] stillen m stalen o \textbf{8} dâ] Do m n o  $\cdot$ vollen] vollem m \textbf{11} disen vieren] [D*]: Dẏsen wer n Disz o \textbf{13} zwô] zuvo Fr69 \textbf{15} truogen] truͯgent m o triegent n \textbf{17} trüegen] trugen m (n) (o)  $\cdot$ tiuren] tuͯren turen o \textbf{18} dâ] Des n (o) \textbf{20} grânât] granant m  $\cdot$ jâchant] iachent m [jochans]: jochant o \textbf{22} dünne] do in n o \textbf{23} wer] Vor o Swer Fr69  $\cdot$ einem tische] eynen tusch o \textbf{24} abe] ob o Fr69  $\cdot$ der wirt] \textit{om.} m n o  $\cdot$ eine] \textit{om.} Fr69  $\cdot$ az] sas Fr69 \textbf{27} si] \textit{om.} o \textbf{28} viere] Wer n Vie o  $\cdot$ tavelen] kanelin m kanelen n o tauel Fr69 \textbf{30} dâ] do n \textit{om.} o  $\cdot$ kômen] keinen o \textbf{29} helfenbeinîn] Helffen bein n (o) \newline
\end{minipage}
\end{table}
\newpage
\begin{table}[ht]
\begin{minipage}[t]{0.5\linewidth}
\small
\begin{center}*G
\end{center}
\begin{tabular}{rl}
 & nâch \textbf{de\textit{r}} \textbf{gienc} ein herzogîn\\ 
 & unde ir gespil. zwei stöllelîn\\ 
 & \textbf{si truogen} von helfenbein.\\ 
 & ir munt nâch viures rœte schein.\\ 
5 & die nigen alle viere.\\ 
 & \textbf{die} z\textit{w}ô sazten schiere\\ 
 & vür den wirt die stollen.\\ 
 & dâ wart gedient mit vollen.\\ 
 & \textbf{si} stuonden sament an \textbf{einer} schar\\ 
10 & \textbf{unde} wâren alle wolgevar.\\ 
 & \textbf{den} vieren was gelîch ir wât.\\ 
 & seht, wâ sich niht versûmet hât\\ 
 & anderre \textbf{vrouwen} vierstunt zwô.\\ 
 & \textbf{die wâren dâ geschaffet zuo}:\\ 
15 & \textbf{viere} truogen kerzen grôz.\\ 
 & die anderen viere niht verdrôz,\\ 
 & si\textbf{ne} trüegen einen tiuren stein,\\ 
 & dâ tages diu sunne lieht durchschein.\\ 
 & \textbf{dâ vür was sîn name} erkant:\\ 
20 & e\textit{z} was ein grânât jâchant,\\ 
 & beidiu lanc unde breit.\\ 
 & durch die \textbf{lieht} in dünne sneit,\\ 
 & swer in zeinem tische maz.\\ 
 & dar \textbf{abe} der wirt durch rîcheit az.\\ 
25 & \textbf{\begin{large}S\end{large}i} giengen \textbf{alle} rehte\\ 
 & vür den \textbf{wirt} alle ehte.\\ 
 & \textbf{gein} nîgen si ir houbt wegeten.\\ 
 & vier die tavelen legeten\\ 
 & \textbf{ûf \textit{helfenbein}} \textit{wîz als ein snê,}\\ 
30 & \textit{\textbf{die}} \textit{stollen, die kômen ê.}\\ 
\end{tabular}
\scriptsize
\line(1,0){75} \newline
G I O L M Q R Z Fr21 Fr40 Fr51 \newline
\line(1,0){75} \newline
\textbf{1} \textit{Initiale} L Z Fr21  \textbf{9} \textit{Initiale} I O  \textbf{11} \textit{Capitulumzeichen} L  \textbf{25} \textit{Initiale} G M  \newline
\line(1,0){75} \newline
\textbf{1} der] den G dem Q  $\cdot$ ein] div O \textbf{2} unde] \textit{om.} I \textbf{3} si] die I (O) (L) (M) (Q) (R) (Z) (Fr21)  $\cdot$ von] daz waz L \textbf{4} nâch viures rœte] als ein rubin I nach fúrres roͯtin R nach vures varwe Fr51 \textbf{5} nigen] jungen M (Q) \textbf{6} die] Do R  $\cdot$ zwô] zoͮ G (Fr21) zwey Fr51  $\cdot$ sazten] satzen Fr21 Fr51 \textbf{8} wart] wirt I  $\cdot$ mit] \textit{om.} Fr51 \textbf{9} si] ÷i O So Fr51  $\cdot$ sament] ensamp I (M) (Z) (Fr21) alle sampt R \textit{om.} Fr51  $\cdot$ einer] eine O \textbf{11} vieren] frien Q \textbf{12} seht] sheht I  $\cdot$ sich] si I O  $\cdot$ niht] \textit{om.} M \textbf{13} anderre] An dirre O  $\cdot$ vrouwen] \textit{om.} Fr51  $\cdot$ vierstunt] vier stuͦndent R (Fr51) \textbf{14} dâ] do O Q R  $\cdot$ geschaffet] geschaffen Q (Fr51)  $\cdot$ zuo] [zwo]: zoͮ G so L Q R zo Fr51 \textbf{16} die] Den Fr51  $\cdot$ anderen] ander R  $\cdot$ viere] \textit{om.} I des O vieren Fr51 \textbf{17} sine] si I (O) (L) (M) (Q) (R) (Z) (Fr40)  $\cdot$ trüegen] tragen R \textbf{18} dâ] Des M R Fr51  $\cdot$ tages] \textit{om.} I  $\cdot$ diu] der L  $\cdot$ sunne lieht] lieht svnne O svnne lýcht L (M) (Q) (Fr40) svnne da liecht Fr51 \textbf{19} was] \textit{om.} L  $\cdot$ erkant] enkant Fr51 \textbf{20} ez] er G  $\cdot$ grânât] gruͤner I granat ein R  $\cdot$ jâchant] iochant G Z Fr21 Jochant I O R iachant M iechant Q iok:::nt Fr40 yachant Fr51 \textbf{22} Durch die dúnne ein liech schein R  $\cdot$ die] diu I (O)  $\cdot$ dünne] die svnne Fr21  $\cdot$ lieht] lýchte L (M) (Q) (Z) liht Fr40 \textbf{23} swer] Der L Wer Q R Die Fr51  $\cdot$ zeinem] zo eynen Fr51  $\cdot$ maz] mac Fr51 \textbf{24} abe] obe I van Fr51  $\cdot$ der wirt durch rîcheit] der wirt durch rechtte R die riche wert Fr51 \textbf{25} Si] Die Z  $\cdot$ alle] vil Q R Fr40  $\cdot$ rehte] ehte Z ge slachte Fr51 \textbf{26} den] dem Fr21  $\cdot$ alle] die L  $\cdot$ ehte] rehte Z \textbf{27} gein] Duͦrch Fr51  $\cdot$ nîgen] im I (R)  $\cdot$ ir] diu I \textbf{28} tavelen] tavel O (R) Fr21  $\cdot$ legeten] wegeden Fr51 \textbf{29} \textit{Die Verse 233.29-30 fehlen} G   $\cdot$ ûf] von I  $\cdot$ ein] der L \textit{om.} M R \textbf{30} die stollen] Stollen O L (M) Q R Z Fr21 (Fr40) (Fr51)  $\cdot$ kômen] dar chomen O (L) (M) (R) (Z) (Fr21) (Fr40) (Fr51) do komen Q \newline
\end{minipage}
\hspace{0.5cm}
\begin{minipage}[t]{0.5\linewidth}
\small
\begin{center}*T
\end{center}
\begin{tabular}{rl}
 & Nâch \textbf{den} \textbf{zwein} \textbf{gienc} ein herzogîn\\ 
 & unde ir gespil. zwei stöllelîn\\ 
 & \textbf{truogen s\textit{i}} von helfenbein.\\ 
 & ir munt nâch viures rœte schein.\\ 
5 & die nigen alle viere.\\ 
 & \textbf{die} zwô sazten schiere\\ 
 & vür den wirt die stollen.\\ 
 & dâ wart gedient mit vollen.\\ 
 & \textbf{si} stuonden samt an \textbf{eine} schar.\\ 
10 & \textbf{si} wâren alle wol gevar.\\ 
 & \textbf{\begin{large}D\end{large}en} vieren was glîche ir wât.\\ 
 & seht, wâ sich niht versûmet hât\\ 
 & ander \textbf{juncvrouwen} vierstunt zwô.\\ 
 & \textbf{die wâren dâ geschaffet zuo},\\ 
15 & \textbf{daz} \textbf{si} truogen kerzen grôz.\\ 
 & die andern viere niht verdrôz,\\ 
 & si\textbf{ne} trüegen einen tiuren stein,\\ 
 & dâ tages di\textit{u} sunne lieht durchschein.\\ 
 & \textbf{des name was dâ vür} erkant:\\ 
20 & ez was ein grânât jâchant,\\ 
 & beidiu lanc unde breit.\\ 
 & durch die \textbf{lîhte} in dünne sneit,\\ 
 & swer in zeinem tische maz.\\ 
 & dâr \textbf{obe} der wirt durch rîcheit az.\\ 
25 & \textbf{die} giengen \textbf{harte} rehte\\ 
 & vür den \textbf{künec} alle ehte.\\ 
 & \textbf{durch} nîgen sir houbet wegeten.\\ 
 & viere die taveln legeten\\ 
 & \textbf{ûf helfenbein} wîz als ein snê,\\ 
30 & stollen, die \textbf{dâ} kômen ê.\\ 
\end{tabular}
\scriptsize
\line(1,0){75} \newline
T U V W \newline
\line(1,0){75} \newline
\textbf{1} \textit{Majuskel} T  \textbf{11} \textit{Initiale} T U V W  \newline
\line(1,0){75} \newline
\textbf{3} si] sin T sy zway W \textbf{6} die] Der V \textbf{7} die] seine W \textbf{8} dâ] Do U W \textbf{9} si] [*]: Die V  $\cdot$ samt] alle U  $\cdot$ eine] einer V \textbf{12} wâ] waz U  $\cdot$ sich] sy W  $\cdot$ versûmet] versinnet U \textbf{13} ander] [ad]: ander T  $\cdot$ juncvrouwen] vreuͦwen U (V) (W) \textbf{14} [D*]: Den waz gebotten also V  $\cdot$ dâ] do U W  $\cdot$ geschaffet] geschaffen W \textbf{15} Vnd truͦgen kertz also groß W  $\cdot$ daz si] [D*]: Daz ir viere V \textbf{17} sine] Sie U (V) (W)  $\cdot$ trüegen] truͦgen U W \textbf{18} Der tags lúcht vnd als die sunne schain W  $\cdot$ dâ] Des U Do V  $\cdot$ diu] die T \textbf{20} Er was gehaissen ein iachant W  $\cdot$ ez] Er U  $\cdot$ jâchant] Jochant T U [*]: iochant V \textbf{22} lîhte] liechte U [lieht*]: liehte V  $\cdot$ in] ein U er in W \textbf{23} swer] Wer U W \textbf{24} obe] oben W  $\cdot$ durch] mit U V W  $\cdot$ az] saß W \textbf{26} alle ehte] als knechte W \textbf{27} durch] Gegen W  $\cdot$ sir] ir U  $\cdot$ wegeten] naigten W \textbf{28} taveln] tauele W \textbf{29} ein] der V \textbf{30} stollen] Sollen U Stolle W  $\cdot$ dâ] do W \newline
\end{minipage}
\end{table}
\end{document}
