\documentclass[8pt,a4paper,notitlepage]{article}
\usepackage{fullpage}
\usepackage{ulem}
\usepackage{xltxtra}
\usepackage{datetime}
\renewcommand{\dateseparator}{.}
\dmyyyydate
\usepackage{fancyhdr}
\usepackage{ifthen}
\pagestyle{fancy}
\fancyhf{}
\renewcommand{\headrulewidth}{0pt}
\fancyfoot[L]{\ifthenelse{\value{page}=1}{\today, \currenttime{} Uhr}{}}
\begin{document}
\begin{table}[ht]
\begin{minipage}[t]{0.5\linewidth}
\small
\begin{center}*D
\end{center}
\begin{tabular}{rl}
\textbf{343} & \begin{large}G\end{large}awan bôt des \textbf{manegen} eit,\\ 
 & swaz volkes dâ \textbf{vür in} \textbf{gereit},\\ 
 & daz er des niht erkande.\\ 
 & er sprach: "mîn varn hât schande,\\ 
5 & sît ich mit wârheit niht \textbf{darf} \textbf{jehen},\\ 
 & daz ich ir keinen habe gesehen\\ 
 & vor disem tage an keiner stat,\\ 
 & swar man mîn dienst ie gebat."\\ 
 & Der knappe sprach ze Gawan:\\ 
10 & "hêrre, sô hân ich missetân.\\ 
 & ich solde\textbf{z} iu ê hân gesagt.\\ 
 & dô was mîn \textbf{bezzer} sin verzagt.\\ 
 & nû rihtet mîne schulde\\ 
 & nâch iwer selbes hulde.\\ 
15 & \textbf{ich solz iu dar nâch} gerne sagen.\\ 
 & lât mich mîn unvuoge ê klagen."\\ 
 & "Junchêrre, \textbf{nû} \textbf{sagt} mir, wer si sîn,\\ 
 & durch iwern zuhtebæren pîn."\\ 
 & "hêrre, sus heizet der vor iu vert,\\ 
20 & dem \textbf{doch} sîn reise ist unerwert,\\ 
 & \textbf{der künec} Poydiconjunz\\ 
 & unt \textbf{der herzoge} Astor \textbf{von} Lanverunz.\\ 
 & dâ vert ein unbescheiden lîp,\\ 
 & dem minne nie gebôt dehein wîp.\\ 
25 & er treit der unvuoge kranz\\ 
 & unt heizet Meljacanz.\\ 
 & ez wære wîb oder magt,\\ 
 & swaz er dâ minne hât bejagt,\\ 
 & die nam er gar in nœten.\\ 
30 & man solt in drumbe tœten.\\ 
\end{tabular}
\scriptsize
\line(1,0){75} \newline
D \newline
\line(1,0){75} \newline
\textbf{1} \textit{Initiale} D  \textbf{9} \textit{Majuskel} D  \textbf{17} \textit{Majuskel} D  \newline
\line(1,0){75} \newline
\textbf{21} Poydiconjunz] Poydiconivnz D \textbf{26} Meljacanz] Meliacanz D \newline
\end{minipage}
\hspace{0.5cm}
\begin{minipage}[t]{0.5\linewidth}
\small
\begin{center}*m
\end{center}
\begin{tabular}{rl}
 & Gawan bôt des \textbf{sînen} eit,\\ 
 & waz volkes d\textit{â} \textbf{vür in} \textbf{gereit},\\ 
 & daz er des niht erkande.\\ 
 & er sprach: "mîn varn hât schande,\\ 
5 & sît ich mit wârheit niht \textbf{mac} \textbf{jehen},\\ 
 & daz ich ir keinen \textit{h}â\textit{n} gesehen\\ 
 & vor disem tage an keiner stat,\\ 
 & war man mîn dienest ie gebat."\\ 
 & der knappe sprach ze Gawan:\\ 
10 & "hêrre, sô hân ich missetân.\\ 
 & ich solte \textbf{es} iu ê hân gesaget.\\ 
 & d\textit{ô} was mîn \textbf{bezzer} sin verzaget.\\ 
 & nû rih\textit{t}et mîne schulde\\ 
 & nâch iuwer \dag seldes\dag  hulde.\\ 
15 & \textbf{dar nâch sol ich iu} gerne sagen.\\ 
 & lât mich mîn ungevüege ê klagen."\\ 
 & "junchêrre, \textbf{sagt} mir, wer si sî\textit{n},\\ 
 & durch \textit{iuwer} zuht\textit{b}ær\textit{e} pîn."\\ 
 & "\textit{\begin{large}H\end{large}}êrre, sus heizet der vor iu vert,\\ 
20 & dem \textbf{doch} sîn reise ist unerwert,\\ 
 & \textbf{rois} Poidiconiunz\\ 
 & und \textbf{ouch} Astor \textbf{de} Lav\textit{e}runz.\\ 
 & d\textit{â} vert \textbf{ouch} ein unbescheiden lîp,\\ 
 & dem minne nie gebôt kein wîp.\\ 
25 & er treit der ungevüege kranz\\ 
 & und  Meliaganz.\\ 
 & ez wære wîp ode\textit{r m}agt,\\ 
 & waz er dâ minne hât bejagt,\\ 
 & die nam er gar in nœten.\\ 
30 & man solt in drumbe tœten.\\ 
\end{tabular}
\scriptsize
\line(1,0){75} \newline
m n o \newline
\line(1,0){75} \newline
\textbf{19} \textit{Initiale} m  \newline
\line(1,0){75} \newline
\textbf{2} dâ] do m n o \textbf{3} des] das o \textbf{6} keinen] do keinen n  $\cdot$ hân] mag m \textbf{8} gebat] gebot o \textbf{12} dô] Da m  $\cdot$ bezzer] boͯser n hoher o \textbf{13} rihtet] richet m \textbf{16} mîn ungevüege ê] E min vnfuͯge n e myn vngefuge o \textbf{17} sîn] sint m \textbf{18} iuwer zuhtbære] woren zuht theren m vwer zuͯckbere o \textbf{19} Hêrre] Fere m Ferre n  $\cdot$ der] \textit{om.} o  $\cdot$ vor iu] fúr in n \textbf{20} unerwert] vnder wert o \textbf{21} Poidiconiunz] poidiconiuntz n poidiconiuncz o \textbf{22} Astor] aster o  $\cdot$ de Laverunz] de lauarvncz m de kaueruntz n delaueruͯncz o \textbf{23} dâ] Do m n o \textbf{25} \textit{Verse 343.25-26 kontrahiert zu:} Er truͦg meliancz o   $\cdot$ ungevüege] vnfuͯge n \textbf{26} Meliaganz] meliagancz m melagantz n \textbf{27} ez] Er o  $\cdot$ oder magt] oder man vnd magt m \textbf{28} dâ] do n o  $\cdot$ hât] hette n \newline
\end{minipage}
\end{table}
\newpage
\begin{table}[ht]
\begin{minipage}[t]{0.5\linewidth}
\small
\begin{center}*G
\end{center}
\begin{tabular}{rl}
 & Gawan bôt des \textbf{manigen} eit,\\ 
 & swaz volkes dâ \textbf{vür in} \textbf{gereit},\\ 
 & daz er des niht erkande.\\ 
 & er sprach: "mîn varen hât schande,\\ 
5 & \begin{large}S\end{large}ît ich mit wârheit niht \textbf{mac} \textbf{jehen},\\ 
 & daz ich ir deheinen habe gesehen\\ 
 & vor disem tage an deheiner stat,\\ 
 & swar man mîn dienst ie gebat."\\ 
 & der knappe sprach ze Gawan:\\ 
10 & "hêrre, sô hân ich missetân.\\ 
 & ich solt\textbf{z} iu ê hân gesaget.\\ 
 & dô was mîn \textbf{bester} sin verzaget.\\ 
 & nû rihtet mîne schulde\\ 
 & nâch iwer selbes hulde.\\ 
15 & \textbf{ich solz iu dar nâch} gerne sagen.\\ 
 & lât mich mîn ungevüege ê klagen."\\ 
 & "junchêrre, \textbf{saget} mir, wer si sîn,\\ 
 & durch iweren zuhtbæren pîn."\\ 
 & "hêrre, sus heizet der vor iu vert,\\ 
20 & dem \textbf{noch} sîn reise ist unerwert,\\ 
 & \textbf{roys} Poydeconiunz\\ 
 & unde \textbf{d\textit{uc}} Astor \textbf{de} Lanvarunz.\\ 
 & dâ vert ein unbescheiden lîp,\\ 
 & dem minne nie gebôt dehein wîp.\\ 
25 & er treit der ungevüege kranz\\ 
 & unde heizet Meliahganz.\\ 
 & ez wære wîp oder maget,\\ 
 & swaz er dâ minne hât bejaget,\\ 
 & die nam er gar in nœten.\\ 
30 & man solt in drumbe tœten.\\ 
\end{tabular}
\scriptsize
\line(1,0){75} \newline
G I O L M Q R Z Fr22 Fr39 Fr40 \newline
\line(1,0){75} \newline
\textbf{1} \textit{Initiale} I O L M Z Fr39   $\cdot$ \textit{Capitulumzeichen} R  \textbf{5} \textit{Initiale} G  \textbf{17} \textit{Initiale} I   $\cdot$ \textit{Capitulumzeichen} R  \newline
\line(1,0){75} \newline
\textbf{1} Gawan] ÷Awan O \textbf{2} swaz] Waz L (M) (Q) (R)  $\cdot$ dâ] do Q Fr39  $\cdot$ gereit] reit I O L M Q Fr39 \textbf{3} des] das R  $\cdot$ erkande] enkande Q Z \textbf{4} sin varn were shande I  $\cdot$ varen] fart R \textbf{5} ich] \textit{om.} M  $\cdot$ mac] mvgt M \textbf{6} ich] \textit{om.} M R  $\cdot$ deheinen] cheine M \textbf{7} vor] Von R  $\cdot$ tage] tagen R \textbf{8} swar] swa I War L M R Wan Q  $\cdot$ mîn dienst] mich ze dienst I meins dinstes Q \textbf{11} soltz iu] solt euch des Q (R)  $\cdot$ ê] \textit{om.} I \textbf{12} dô] Da M Z \textbf{14} iwer] uwirs M (Z) (Fr22) \textbf{15} solz iu] sol ev I soldz iv O (M) solt úch es R  $\cdot$ dar nâch gerne] gerne dar nach O gerne dennoch Q dar nach R \textbf{16} ungevüege] vnfvͦge O (L) (M) (Q) (Z) (Fr22)  $\cdot$ ê] \textit{om.} R \textbf{17} saget] nu sagt I (L) (M) Z (Fr22) \textbf{18} zuhtbæren] zvhtbwernde O zuchtebere Q  $\cdot$ pîn] shin I \textbf{19} iu] mir M \textbf{20} noch] doch O (M) Q R Z Fr22 doch ist L \textbf{21} Poydeconiunz] poideconivnz G poy dechomunz I poẏdi komvͦnz O Poý de Conivnz L pode konivnz M poydek vm vnsz Q poidekomvncz R poydekonivnz Z Poyde konẏv̂nz Fr22 poydekuniunz Fr40 \textbf{22} duc Astor] de chastor G auch kastur I dvcachalster O dvr Astor L kaster M duc castor Q die Castor R dv kastor Fr22 der castor Fr40  $\cdot$ de Lanvarunz] delunfarunz G de luͦuarunz I der lvnvarvnz O de Lvͯit varvnz L delon varvnsz Q de Lonvaruncz R de lanvervnz Z de Linvarv̂nz Fr22 de lonv:::untz Fr40 \textbf{23} dâ] Do Q  $\cdot$ vert] virt M  $\cdot$ lîp] weip Q \textbf{24} nie] nit R  $\cdot$ gebôt] gebet R  $\cdot$ dehein] ein O Z  $\cdot$ wîp] leip Q \textbf{25} er] Der Q  $\cdot$ der ungevüege] den vngefuͤgen I der vuͯge L den vnfuͦgen R der vnfuge Z (Fr40) \textbf{26} \textit{Vers 343.26 fehlt} Q   $\cdot$ unde] er I  $\cdot$ Meliahganz] meliaganz I Meliahcanz O Meliachkanz L Meliachkancz M Meliahkancz R Meliahkanz Z meliahkantz Fr40 \textbf{27} wîp] ein wip L Fr22 wibe R  $\cdot$ maget] ein magt Fr22 [man]: magt Fr40 \textbf{28} swaz] Waz L (M) (Q) (R)  $\cdot$ dâ] do Q der R  $\cdot$ minne] minnin Fr22 \textbf{29} die] Die y R  $\cdot$ in nœten] enoten G \newline
\end{minipage}
\hspace{0.5cm}
\begin{minipage}[t]{0.5\linewidth}
\small
\begin{center}*T
\end{center}
\begin{tabular}{rl}
 & \begin{large}G\end{large}awan bôt des \textbf{manegen} eit,\\ 
 & swaz volkes dâ \textbf{vor im} \textbf{reit},\\ 
 & daz er des niht erkande.\\ 
 & er sprach: "mîn varn hât schande,\\ 
5 & sît ich mit wârheit niht \textbf{mac} \textbf{gejehen},\\ 
 & daz ich ir keinen habe gesehen\\ 
 & vor disem tage an deheiner stat,\\ 
 & swar man mîn dienst ie gebat."\\ 
 & Der knappe sprach ze Gawan:\\ 
10 & "hêrre, sô hân ich missetân.\\ 
 & ich solte\textbf{s} iu ê hân gesaget.\\ 
 & dô was mîn \textbf{bester} sin verzaget.\\ 
 & nû rihtet mîne schulde\\ 
 & nâch iuwer selbes hulde.\\ 
15 & \textbf{ich sols iu dar nâch} gerne sagen.\\ 
 & lât mich mîn unvuoge ê klagen."\\ 
 & "Junchêrre, \textbf{nû} \textbf{sage} mir, wer si sîn,\\ 
 & durch iuwern zuhtebæren pîn."\\ 
 & "Hêrre, sus heizet der vor iu vert,\\ 
20 & dem \textbf{doch} sîn reise ist unerwert,\\ 
 & \textbf{Roys} Poydekuniuns\\ 
 & unde \textbf{duc} Astor \textbf{de} Lunveruns.\\ 
 & dâ vert ein unbescheiden lîp,\\ 
 & dem minne nie gebôt dehein wîp.\\ 
25 & er treit der unvuoge kranz\\ 
 & unde heizet Melyahganz.\\ 
 & ez wære wîp oder maget,\\ 
 & swaz er dâ minne hât bejaget,\\ 
 & die nam er gar in nœten.\\ 
30 & man solt in drumbe tœten.\\ 
\end{tabular}
\scriptsize
\line(1,0){75} \newline
T V W \newline
\line(1,0){75} \newline
\textbf{1} \textit{Initiale} T W  \textbf{9} \textit{Majuskel} T  \textbf{17} \textit{Majuskel} T  \textbf{19} \textit{Majuskel} T  \textbf{21} \textit{Majuskel} T  \newline
\line(1,0){75} \newline
\textbf{1} manegen] sinen V \textbf{2} swaz] Was W  $\cdot$ dâ] do V W  $\cdot$ im] in V W \textbf{5} gejehen] iehen V W \textbf{6} keinen] keine W \textbf{8} swar] War W \textbf{9} Gawan] gawane W \textbf{10} Herre so bin ich der vertane W \textbf{11} iu] \textit{om.} W \textbf{13} rihtet] reitent W \textbf{15} sols iu] sol v́ch V sol eúchs W \textbf{17} sage] sagent V (W) \textbf{18} iuwern] uwer V (W) \textbf{21} Roys] Kv́nig V  $\cdot$ Poydekuniuns] poydekvmvns V pondier vniuns W \textbf{22} duc] herzoge V \textit{om.} W  $\cdot$ Astor] Castor T [*]: astor V kastor W  $\cdot$ Lunveruns] [*]: luvernvns V lunueruns W \textbf{23} dâ] Do V W  $\cdot$ ein] [*]: och ein V \textbf{25} unvuoge] vnfuͦgen V \textbf{26} Melyahganz] Meliahganz T meliagantz V W \textbf{28} swaz] Was W  $\cdot$ dâ] do V W \textbf{29} in] mit W \newline
\end{minipage}
\end{table}
\end{document}
