\documentclass[8pt,a4paper,notitlepage]{article}
\usepackage{fullpage}
\usepackage{ulem}
\usepackage{xltxtra}
\usepackage{datetime}
\renewcommand{\dateseparator}{.}
\dmyyyydate
\usepackage{fancyhdr}
\usepackage{ifthen}
\pagestyle{fancy}
\fancyhf{}
\renewcommand{\headrulewidth}{0pt}
\fancyfoot[L]{\ifthenelse{\value{page}=1}{\today, \currenttime{} Uhr}{}}
\begin{document}
\begin{table}[ht]
\begin{minipage}[t]{0.5\linewidth}
\small
\begin{center}*D
\end{center}
\begin{tabular}{rl}
\textbf{703} & \textbf{\textit{\begin{large}O\end{large}}uch} rou den künec Gramoflanz,\\ 
 & daz ein ander \textbf{man} vür sînen kranz\\ 
 & des tages hete gevohten.\\ 
 & dâ getorsten noch enmohten\\ 
5 & die sîne daz niht \textbf{gescheiden}.\\ 
 & er begundez sêre leiden,\\ 
 & daz er sich versûmet hæte.\\ 
 & waz der helt dô tæte,\\ 
 & wander \textbf{ê} prîs bejagete?\\ 
10 & \textbf{reht} \textbf{indes}, dô ez tagete,\\ 
 & was sîn ors gewâpent unt sîn \textbf{selbes} lîp.\\ 
 & ob gæben rîchlôsiu wîp\\ 
 & sîner zimierde stiure?\\ 
 & \textbf{ez} was sus als tiure.\\ 
15 & Er zierten lîp durch \textbf{eine} magt;\\ 
 & der was \textbf{er dienstes} unverzagt.\\ 
 & er reit eine ûf die warte.\\ 
 & den \textbf{künec} daz müete harte,\\ 
 & daz der werde Gawan\\ 
20 & niht schiere kom ûf den plân.\\ 
 & Nû het \textbf{ouch sich} vil gar verholn\\ 
 & Parzival her ûz \textbf{verstoln}.\\ 
 & ûz einer baniere er nam\\ 
 & ein starkez sper von Angram;\\ 
25 & er hete ouch al sîn harnasch an.\\ 
 & der helt reit al eine dan\\ 
 & gein den ronen spiegelîn,\\ 
 & al dâ der kampf solde sîn.\\ 
 & \begin{large}E\end{large}r sach den \textbf{künec} halden dort.\\ 
30 & ê \textbf{daz} \textbf{deweder} ie wort\\ 
\end{tabular}
\scriptsize
\line(1,0){75} \newline
D Fr66 \newline
\line(1,0){75} \newline
\textbf{1} \textit{Initiale} D  \textbf{15} \textit{Majuskel} D  \textbf{21} \textit{Majuskel} D  \textbf{29} \textit{Initiale} D  \newline
\line(1,0){75} \newline
\textbf{1} Ouch] ÷vch D \textbf{11} selbes] \textit{om.} Fr66 \textbf{16} dienstes] diens D \textbf{22} Parzival] Parcifal D \newline
\end{minipage}
\hspace{0.5cm}
\begin{minipage}[t]{0.5\linewidth}
\small
\begin{center}*m
\end{center}
\begin{tabular}{rl}
 & \textbf{\begin{large}N\end{large}û} ro\textit{u} den künic Gramolanz,\\ 
 & daz ein ander \textbf{künic} vü\textit{r} sînen kranz\\ 
 & des tages het gevohten.\\ 
 & dô getorsten noch enmohten\\ 
5 & die sîne daz niht \textbf{gescheiden}.\\ 
 & er begunde ez sêre leiden,\\ 
 & daz er sich versûme\textit{t} hæte.\\ 
 & waz der helt dô tæte,\\ 
 & wan er \textbf{ê} prîs bejagete?\\ 
10 & \textbf{reht} \textbf{inne} dô ez tagete,\\ 
 & was sîn ros gewâpent und sîn lîp.\\ 
 & ob gæben rîchlôsiu wîp\\ 
 & sîner zimierde stiure?\\ 
 & \textbf{si} was sus alsô tiure.\\ 
15 & er zierte den lîp durch \textbf{ein} maget;\\ 
 & der was \textbf{er dienstes} unverzaget.\\ 
 & er reit ein ûf die warte.\\ 
 & den \textbf{künic} daz müete harte,\\ 
 & daz der werde Gawan\\ 
20 & niht schiere kam ûf den plân.\\ 
 & nû het \textbf{sich ouch} vil gar verholn\\ 
 & Parcifal her ûz \textbf{gestoln}.\\ 
 & ûz einer banier er nam\\ 
 & ein starke\textit{z s}per von \textit{An}gram;\\ 
25 & er het ouch al sîn harnasch an.\\ 
 & der helt reit aleine dan\\ 
 & gegen den ronen spiegelîn,\\ 
 & aldâ der kampf solte sîn.\\ 
 & er sach den \textbf{künic} halte\textit{n} dort.\\ 
30 & ê \textbf{daz} \dag deweders\dag  ie wort\\ 
\end{tabular}
\scriptsize
\line(1,0){75} \newline
m n o Fr69 \newline
\line(1,0){75} \newline
\textbf{1} \textit{Illustration mit Überschrift:} Also parcifal vnd gramolantz mit ein ander fohten m (n)   $\cdot$ \textit{Initiale} m n o Fr69  \newline
\line(1,0){75} \newline
\textbf{1} rou] rowe m (n) o  $\cdot$ Gramolanz] gramolantz m n gramolancz o Gramoflanz Fr69 \textbf{2} künic] man Fr69  $\cdot$ vür] fuͯrt m (o) fuͯrte n \textbf{4} dô] da Fr69  $\cdot$ getorsten] entorsten o  $\cdot$ enmohten] enmoͯchten n \textbf{5} sîne] sinen Fr69 \textbf{7} versûmet] versúme m \textbf{11} was] Wenne n  $\cdot$ sîn lîp] sinen lip n \textbf{13} zimierde] zimurde o \textbf{16} der] Er n Das o  $\cdot$ er] ir n \textbf{17} ein] \textit{om.} n  $\cdot$ warte] fart n \textbf{18} daz] den o  $\cdot$ harte] das hart n \textbf{19} Gawan] ::: Fr69 \textbf{20} den] dem o \textbf{21} nû] Sú n \textbf{22} gestoln] verstolen n (o) \textbf{24} Ein starcker man sper von gram m \textbf{25} al sîn] allen sinen n \textbf{26} reit] [er]: reit m \textbf{29} halten] haltte m \textbf{30} deweders] do weders n die widers o  $\cdot$ ie] e o \newline
\end{minipage}
\end{table}
\newpage
\begin{table}[ht]
\begin{minipage}[t]{0.5\linewidth}
\small
\begin{center}*G
\end{center}
\begin{tabular}{rl}
 & \textbf{\begin{large}D\end{large}och} rou den künic Gramoflanz,\\ 
 & daz ein ander \textbf{man} vür sînen kranz\\ 
 & des tages hete gevohten.\\ 
 & dô\textbf{ne} getorsten nochne mohten\\ 
5 & die sîne\textit{n} daz niht \textbf{scheiden}.\\ 
 & er begundez sêre leiden,\\ 
 & daz er sich versûmet hæte.\\ 
 & waz der helt dô tæte,\\ 
 & wan er brîs bejagte?\\ 
10 & \textbf{innen des}, dô ez tagte,\\ 
 & wa\textit{s} sîn ors gewâpent unde sîn lîp.\\ 
 & ob g\textit{æ}ben rîchlôs\textit{iu} wîp\\ 
 & sîner zimierde stiure?\\ 
 & \textbf{si} was sus als tiure.\\ 
15 & er zierte den lîp durch \textbf{die} maget,\\ 
 & der was \textbf{der dienst} unverzaget.\\ 
 & er reit eine ûf die warte,\\ 
 & den daz muote harte,\\ 
 & daz der werde Gawan\\ 
20 & niht schiere kom ûf den plân.\\ 
 & nû het \textbf{ouch sich} \textit{vil} gar verholen\\ 
 & Parcival her ûz \textbf{\textit{v}e\textit{r}stolen}.\\ 
 & ûz einer banier er nam\\ 
 & ein starkez sper von Angram;\\ 
25 & er het ouch al sîn harnasch an.\\ 
 & der helt reit al eine dan\\ 
 & gein den ron\textit{en} spiegelîn,\\ 
 & al dâ der kampf solde sîn.\\ 
 & er sach den \textbf{künic} halden dort.\\ 
30 & ê \textbf{ir} \textbf{dewederre} ie wort\\ 
\end{tabular}
\scriptsize
\line(1,0){75} \newline
G I L M Z \newline
\line(1,0){75} \newline
\textbf{1} \textit{Initiale} G L Z  \textbf{11} \textit{Initiale} I  \newline
\line(1,0){75} \newline
\textbf{1} Doch] Dvrch L Ouch M (Z)  $\cdot$ rou] ruwete M  $\cdot$ den] dem L  $\cdot$ Gramoflanz] gramorflanz M gramoflantz Z \textbf{2} vür] [vuͦrte]: fuͦr I \textbf{4} dône getorsten] Den getorsten L Da entorsten M Da getorsten Z \textbf{5} sînen] sine G  $\cdot$ niht] \textit{om.} Z  $\cdot$ scheiden] gescheiden L Z \textbf{6} begundez] begvnde L \textbf{8} dô] da M Z \textbf{10} innen] Bynnen M  $\cdot$ dô] da M Z \textbf{11} was] Wan G \textbf{12} ob] Auch I  $\cdot$ gæben] gaben G (I) M gaben gaben L  $\cdot$ rîchlôsiu] rih losen G \textbf{14} als] al L \textbf{15} zierte] ziert I (L) Z \textbf{16} in der dienst was er vnuerzaget I  $\cdot$ der] sin L Z \textbf{17} eine] eyner M  $\cdot$ warte] vart L \textbf{18} daz muͦt in auch vil harte I  $\cdot$ den] Den kvnic Z  $\cdot$ harte] vil hart L \textbf{20} niht schiere kom] chom nih shier I \textbf{21} vil] \textit{om.} G \textbf{22} Parcival] parcifal G (Z) parzifal I (M) Sich parzifal L  $\cdot$ verstolen] gestolen G \textbf{24} Angram] angaram I \textbf{25} al] alle M \textbf{27} ronen] ronne G rone I roren L \textbf{30} ê] E daz L (M) Z  $\cdot$ ir dewederre] du widir M entwederre Z  $\cdot$ ie] dehain I \newline
\end{minipage}
\hspace{0.5cm}
\begin{minipage}[t]{0.5\linewidth}
\small
\begin{center}*T
\end{center}
\begin{tabular}{rl}
 & \textbf{\begin{large}O\end{large}uch} rou den künec Gramoflanz,\\ 
 & daz ein ander \textbf{man} vür sînen kranz\\ 
 & des tages \textbf{dâ} hete gevohten.\\ 
 & dô \textbf{en}getorsten \textbf{si} noch enmohten\\ 
5 & die sîne daz niht \textbf{gescheiden}.\\ 
 & er begundez sêre leiden,\\ 
 & daz er sich versûmet hæte.\\ 
 & waz der helt dô tæte,\\ 
 & wander prîs bejagete?\\ 
10 & \textbf{innen \textit{des}}, dô ez tagete,\\ 
 & was sîn ors gewâpent und sîn lîp.\\ 
 & o\textit{b} gæben r\textit{î}c\textit{h}lôsiu wîp\\ 
 & sîner zimierde stiure?\\ 
 & \textbf{si} was sus alsô tiure.\\ 
15 & er zierete den lîp durch \textbf{die} maget,\\ 
 & der was \textbf{sîn dienst} unverzaget.\\ 
 & er reit \textbf{al}eine ûf die warte.\\ 
 & den \textbf{künec} daz muote harte,\\ 
 & daz der werde Gawan\\ 
20 & niht schiere kam ûf den plân.\\ 
 & nû hete \textbf{ouch sich} vil gar verholn\\ 
 & Parcifal her ûz \textbf{verstoln}.\\ 
 & ûz einer banier er nam\\ 
 & ein starkez sper von Angram;\\ 
25 & er hete ouch a\textit{l} sîn harnasch an.\\ 
 & der helt reit aleine dan\\ 
 & gein den ronen spiegelîn,\\ 
 & al dâ der kampf solte sîn.\\ 
 & er sach den \textbf{helt} halten dort.\\ 
30 & ê \textbf{ietweder} ie wort\\ 
\end{tabular}
\scriptsize
\line(1,0){75} \newline
U V W Q R \newline
\line(1,0){75} \newline
\textbf{1} \textit{Überschrift:} Hie stritet parzefal mit gramaflanczen do er mit Gawan solte gekenpfet han V   $\cdot$ \textit{Großinitiale} U   $\cdot$ \textit{Initiale} V W Q R  \newline
\line(1,0){75} \newline
\textbf{1} rou] rawe W  $\cdot$ Gramoflanz] [gramafla*]: gramaflanz V gramoflantz W Q gramoflancz R \textbf{3} dâ] do V W Q R \textbf{4} dô engetorsten] Do entorsten W Do getorsten R  $\cdot$ si] \textit{om.} V W Q R  $\cdot$ enmohten] eyn [w*]: mochten Q mochten R \textbf{5} gescheiden] scheiden R \textbf{6} er begundez] Es begunde in Q \textbf{7} versûmet] versunnen R \textbf{8} dô] da R \textbf{9} wander] Wan der U Wand [e*]: er e V Wan ir R \textbf{10} des] \textit{om.} U \textbf{11} sîn ors gewâpent] gewapent sein roß W \textbf{12} ob gæben] Oder geben U [O*]: Obe geben V  $\cdot$ rîchlôsiu] rechelose U Richlose R \textbf{14} sus] vns Q \textbf{15} die] [*]: eine V \textbf{16} der] Er W  $\cdot$ sîn] seinem W  $\cdot$ unverzaget] vnuersaget V (Q) \textbf{17} aleine] eine V W (Q) enig R  $\cdot$ warte] farte R \textbf{18} daz muote] muͯt das R \textbf{19} Gawan] Gawin R \textbf{21} sich] \textit{om.} W  $\cdot$ vil] \textit{om.} Q nun R \textbf{22} Parcifal] Parzifal U Parzefal V Partzifal W Q Parczifal R \textbf{24} Angram] agram W R angran Q \textbf{25} al sîn] alle sin U allen sinen V allen sein W \textbf{27} den] dem R  $\cdot$ ronen] rouen W \textbf{29} helt] kv́nig V (W) (Q) (R) \textbf{30} [*]: E das ir dewederre ie wort V  $\cdot$ ê] Ee das ir Q (R)  $\cdot$ ietweder] deweder W Q (R) \newline
\end{minipage}
\end{table}
\end{document}
