\documentclass[8pt,a4paper,notitlepage]{article}
\usepackage{fullpage}
\usepackage{ulem}
\usepackage{xltxtra}
\usepackage{datetime}
\renewcommand{\dateseparator}{.}
\dmyyyydate
\usepackage{fancyhdr}
\usepackage{ifthen}
\pagestyle{fancy}
\fancyhf{}
\renewcommand{\headrulewidth}{0pt}
\fancyfoot[L]{\ifthenelse{\value{page}=1}{\today, \currenttime{} Uhr}{}}
\begin{document}
\begin{table}[ht]
\begin{minipage}[t]{0.5\linewidth}
\small
\begin{center}*D
\end{center}
\begin{tabular}{rl}
\textbf{680} & \begin{large}S\end{large}i tâten ir poynder rehte;\\ 
 & ûz der tjoste geslehte\\ 
 & wâren si bêde \textbf{samt} \textbf{erborn}.\\ 
 & wênec gew\textit{u}nnen, vil verlorn\\ 
5 & hât, swer behaldet dâ den prîs;\\ 
 & der \textbf{klagt}\textbf{z} doch immer, \textbf{ist} er wîs.\\ 
 & Gein ein ander stuont ir triwe,\\ 
 & der enweder alt \textbf{noch} niwe\\ 
 & dürkel scharten nie enpfienc.\\ 
10 & nû hœret, wie diu tjoste ergienc:\\ 
 & \textbf{hurteclîche} unt doch alsô,\\ 
 & si mohtens bêde sîn unvrô.\\ 
 & Erkantiu sippe unt hôch geselleschaft\\ 
 & was dâ mit \textbf{herzenlîcher} kraft\\ 
15 & durch \textbf{scharpfen strît} zein ander komen.\\ 
 & von \textbf{swem} \textbf{der prîs dâ wirt} \textbf{genomen},\\ 
 & des vreude \textbf{ist drumbe} sorgen pfant.\\ 
 & die tjoste brâhte \textbf{iewederiu} hant,\\ 
 & daz die mâge unt die gesellen\\ 
20 & ein ander muosen vellen\\ 
 & mit \textbf{orse} \textbf{mitalle} nider.\\ 
 & alsus wurben si dô sider.\\ 
 & Ez wart aldâ verzwicket,\\ 
 & mit swerten \textbf{verbicket}.\\ 
25 & schildes schirben unt daz grüene gras\\ 
 & ein glîchiu \textbf{temperîe} was,\\ 
 & sît si begunden strîten.\\ 
 & si muosen scheidens bîten\\ 
 & al ze lange; si begundens vruo.\\ 
30 & dâ\textbf{ne} greif êt niemen scheidens zuo.\\ 
\end{tabular}
\scriptsize
\line(1,0){75} \newline
D Fr10 \newline
\line(1,0){75} \newline
\textbf{1} \textit{Initiale} D  \textbf{7} \textit{Majuskel} D  \textbf{13} \textit{Majuskel} D  \textbf{23} \textit{Majuskel} D  \newline
\line(1,0){75} \newline
\textbf{3} samt] en samt Fr10 \textbf{4} gewunnen] gewinnen D gewunen vnd Fr10 \textbf{6} immer] nimmer Fr10 \textbf{8} der enweder] ::: werd Fr10 \textbf{12} mohtens] mohtn Fr10 \textbf{15} zein] ez ain Fr10 \textbf{18} iewederiu] ietweders Fr10 \textbf{19} unt die] vnd Fr10 \textbf{21} mitalle] vnd mit alle Fr10 \textbf{22} dô] da Fr10 \textbf{24} verbicket] ver blichet Fr10 \textbf{30} dâne] Da Fr10 \newline
\end{minipage}
\hspace{0.5cm}
\begin{minipage}[t]{0.5\linewidth}
\small
\begin{center}*m
\end{center}
\begin{tabular}{rl}
 & si tâten ir ponder rehte;\\ 
 & ûz der juste geslehte\\ 
 & wâren si beide \textbf{er\textit{b}orn}.\\ 
 & wênic gewunnen, vil verlorn\\ 
5 & het, wer behaltet d\textit{â} den prîs;\\ 
 & der \textbf{klagte} \textbf{ez} doch iemer, \textbf{wirt} er wîs.\\ 
 & gegen ein ander stuont ir triuwe,\\ 
 & der enweder alt \textbf{und} niuwe\\ 
 & dürkel scharten nie enpfienc.\\ 
10 & nû hœret, wie diu juste ergienc:\\ 
 & \textbf{hurteclîch} und doch alsô,\\ 
 & si mohtens beide sîn unvrô.\\ 
 & erkantiu sippe und hôhiu geselleschaft\\ 
 & was d\textit{â} mit \textbf{hazzelîcher} kraft\\ 
15 & durch \textbf{scharpfen strît} zuo ein ander komen.\\ 
 & von \textbf{wem} \textbf{der prîs d\textit{â} wirt} \textbf{vernomen},\\ 
 & des vröude \textbf{dâr umb ist} sorgen pfant.\\ 
 & die juste brâht \textbf{ietweders} hant,\\ 
 & daz die mâge und die gesellen\\ 
20 & ein ander muosten vellen\\ 
 & mit \textbf{rosse} \dag und alle\dag  nider.\\ 
 & a\textit{l}sus wurben si dô sider.\\ 
 & ez wart aldâ verzwicket,\\ 
 & mit swerten \textbf{verbicket}.\\ 
25 & schiltes schir\textit{b}en und daz grüene gras\\ 
 & ein glîchiu \textbf{tem\textit{p}erunge} was,\\ 
 & sît si begunden strîten.\\ 
 & si muosten scheidens bîten\\ 
 & al zuo lange; si begunde\textit{n}s vruo.\\ 
30 & dô grei\textit{f} eht niemen scheidens zuo,\\ 
\end{tabular}
\scriptsize
\line(1,0){75} \newline
m n o Fr69 \newline
\line(1,0){75} \newline
\newline
\line(1,0){75} \newline
\textbf{1} tâten] dochten n \textbf{3} erborn] erkorn m samt erborn Fr69 \textbf{5} het] Hette n  $\cdot$ behaltet] behalten n  $\cdot$ dâ] do m n o \textbf{6} der klagte ez] ::: o  $\cdot$ doch] \textit{om.} n  $\cdot$ wirt] ist n o \textbf{8} enweder] ein weder n  $\cdot$ und] noch n o \textbf{11} hurteclîch] Herteclich n \textbf{12} si] Sin o  $\cdot$ mohtens] moͯchtens n \textbf{13} hôhiu] hoͯhen o doch Fr69 \textbf{14} dâ] do m n o  $\cdot$ hazzelîcher] hassecliche Fr69 \textbf{16} der] do n  $\cdot$ dâ] do m n o  $\cdot$ vernomen] genomen n (o) \textbf{17} vröude] froides o  $\cdot$ dâr umb ist] ist dar v́mb n (o) \textbf{22} alsus] Alusus m  $\cdot$ sider] wider n \textbf{23} wart] wirt o \textbf{24} verbicket] verbicken o \textbf{25} schirben] schirmen m \textbf{26} temperunge] tempierunge m (o) \textbf{28} bîten] bitten m \textbf{29} begundens] begundes m \textbf{30} greif] greis m \newline
\end{minipage}
\end{table}
\newpage
\begin{table}[ht]
\begin{minipage}[t]{0.5\linewidth}
\small
\begin{center}*G
\end{center}
\begin{tabular}{rl}
 & \begin{large}S\end{large}i tâten ir poynder rehte;\\ 
 & ûz der tjost geslehte\\ 
 & w\textit{âr}en si bêde \textbf{sament} \textbf{geborn}.\\ 
 & wênic gewunnen \textbf{unde} vil verlorn\\ 
5 & hât, swer behalt dâ den brîs;\\ 
 & der \textbf{klaget} doch immer, \textbf{ist} er wîs.\\ 
 & gein ein ander stuont ir triwe,\\ 
 & der newederiu alt \textbf{noch} niwe\\ 
 & dürkel scharten nie enpfie.\\ 
10 & nû hœret, wie diu tjost ergie:\\ 
 & \textbf{hurticlîch} unde doch alsô,\\ 
 & si mohtens bêde sîn unvrô.\\ 
 & erkantiu sippe unde hôch geselleschaft\\ 
 & was dâ mit \textbf{hazlîcher} kraft\\ 
15 & durch \textbf{scharpf\textit{e} \textit{tjost}} zein ander komen.\\ 
 & von \textbf{swederm} \textbf{der brîs dâ wirt} \textbf{genomen},\\ 
 & des vröude \textbf{ist umbe} sorgen pfant.\\ 
 & die tjost brâhte \textbf{ietweders} hant,\\ 
 & daz die mâge unde die gesellen\\ 
20 & ein ander muosen vellen\\ 
 & mit \textbf{ors} \textbf{mitalle} nider.\\ 
 & alsus wurben si dô sider.\\ 
 & ez wart al dâ verzwicket,\\ 
 & mit swerten \textbf{verblicket}.\\ 
25 & schiltes schirben unde daz grüene gras\\ 
 & ein glîchiu \textbf{temperîe} was,\\ 
 & sît si begunden strîten.\\ 
 & si muosen scheidens bîten\\ 
 & al ze lange; si begundens vruo.\\ 
30 & dô\textbf{ne} greif êt niemen scheidens zuo.\\ 
\end{tabular}
\scriptsize
\line(1,0){75} \newline
G I L M Z Fr18 Fr22 Fr24 Fr52 \newline
\line(1,0){75} \newline
\textbf{1} \textit{Initiale} G L Z Fr24 Fr52  \textbf{11} \textit{Initiale} I  \newline
\line(1,0){75} \newline
\textbf{1} ir] irn Fr52 \textbf{3} wâren] wrden G  $\cdot$ bêde sament] bedeensampt I \textbf{4} unde] \textit{om.} Fr52  $\cdot$ vil] \textit{om.} I M \textbf{5} hât swer] hat sweder I [Hatwer]: Hat wer L Hatte wer M  $\cdot$ behalt dâ] bihielt da M da beheldet Fr52 \textbf{6} klaget] clagtez Z chlagte Fr24  $\cdot$ immer] nymmer M \textbf{7} stuont] \textit{om.} L \textbf{8} der newederiu] der den weder I Der en wer M ir keine Fr52 \textbf{9} scharten] shart I (M) (Z) (Fr18) (Fr24) (Fr52) schrat L  $\cdot$ nie] an Fr52  $\cdot$ enpfie] einpfie L \textbf{12} unvrô] fro Fr18 \textbf{13} erkantiu] Er kandie L  $\cdot$ hôch] doch L ovch Fr22 \textbf{14} hazlîcher] hartzlicher L hercziclicher M \textbf{15} durch scharpfe tjost] dvrh scharphen bris G Duͯrch [scha*]: scharphen strit L Durch scharphen strit M (Z) (Fr18) (Fr24) ::rchscharpfin strit Fr22  $\cdot$ zein ander] zv einer Z \textbf{16} swederm] widirme M dem Z  $\cdot$ dâ] \textit{om.} M  $\cdot$ wirt] wart I L Fr22  $\cdot$ genomen] vernomen I \textbf{17} umbe] drvmme Z  $\cdot$ sorgen] sorge Fr18 \textbf{18} die] diu I \textbf{21} ors] orshen I (Fr18) (Fr24)  $\cdot$ mitalle] betalle L \textbf{22} wurben] [wurden]: wurbn I  $\cdot$ dô] da L M Fr22 \textbf{23} Alda isz wart vor zcwicket M  $\cdot$ al dâ] ald I \textbf{24} verblicket] verbichet L (M) (Fr18) Fr24 \textbf{25} schirben] schirmen M (Fr18) \textbf{29} begundens] begunden I \textbf{30} dône greif] Da en greiff M (Z) Do greif Fr24  $\cdot$ êt] \textit{om.} I M Z  $\cdot$ scheidens] scheiden M \newline
\end{minipage}
\hspace{0.5cm}
\begin{minipage}[t]{0.5\linewidth}
\small
\begin{center}*T
\end{center}
\begin{tabular}{rl}
 & si tâten ir poynder rehte;\\ 
 & ûz der jost geslehte\\ 
 & wâren si beide \textbf{samet} \textbf{geborn}.\\ 
 & wênic gewunnen \textbf{und} vil verlorn\\ 
5 & \textit{hât}, \textit{wer} behalte\textit{t} d\textit{â} den prîs;\\ 
 & der \textbf{klaget} doch immer, \textbf{ist} er wîs.\\ 
 & gein ein ander stuont ir triuwe,\\ 
 & der dewederiu alt \textbf{noch} niuwe\\ 
 & dürkel scharte nie entvienc.\\ 
10 & nû hœret, wie diu jost ergienc:\\ 
 & \textbf{herteclîche} und doch alsô,\\ 
 & si mohtens beide sîn unvrô.\\ 
 & erkantiu sippe und hôch geselleschaft\\ 
 & was d\textit{â} mit \textbf{hazlîcher} kraft\\ 
15 & durch \textbf{scharpfen strît} zuo ein ander komen.\\ 
 & von \textbf{swederm} \textbf{d\textit{â} wirt der prîs} \textbf{genomen},\\ 
 & des vreude \textbf{ist umb} sorge pfant.\\ 
 & die jost brâhte \textbf{ietweders} hant,\\ 
 & daz die mâge und die gesellen\\ 
20 & ein ander muosen vellen\\ 
 & mit \textbf{orsen} \textbf{betalle} nider.\\ 
 & alsus wurben si dô sider.\\ 
 & ez wart al dâ verzwicket,\\ 
 & mit swerten \textbf{verbicket}.\\ 
25 & schildes schirben und daz grüene gras\\ 
 & ein glîchiu \textbf{\textit{t}emperîe} was,\\ 
 & sît si begunden strîten.\\ 
 & si muosen scheiden\textit{s} bîten\\ 
 & al zuo lange; si begundens vruo.\\ 
30 & dâ\textbf{n} greif eht nieman scheidens zuo.\\ 
\end{tabular}
\scriptsize
\line(1,0){75} \newline
U V W Q R \newline
\line(1,0){75} \newline
\newline
\line(1,0){75} \newline
\textbf{1} ir poynder] Jrm strit R \textbf{2} ûz] Vncz R \textbf{3} beide samet] beidenthalb R \textbf{4} und] \textit{om.} W R \textbf{5} hât wer] Wer hat U Hat swer V  $\cdot$ behaltet] behalten U  $\cdot$ dâ] do U V W \textit{om.} R \textbf{7} Gegen [*]: einander stvnt ir trvwe V  $\cdot$ ander] andren R  $\cdot$ triuwe] trúwen R \textbf{8} dewederiu] deweders R  $\cdot$ niuwe] Iungen R \textbf{9} dürkel] Dvnkel V Trukel R  $\cdot$ scharte] [schar*]: schartte V scharten W (Q) (R) \textbf{11} herteclîche] Hv́rteklich V (W) (Q) (R) \textbf{12} mohtens] mochten W \textbf{13} Erkantte [spr]: sper vnd doch geselschafftt R  $\cdot$ geselleschaft] geselleschach U \textbf{14} dâ] do U V W Q \textbf{15} zuo ein ander] zuͦ samen W zusamen was R \textbf{16} Von wederme [*]: der pris wurt do genomen V  $\cdot$ swederm dâ] swederm do U wederem W (R) wem Q  $\cdot$ wirt der prîs] der preis ward W der preysz do wirt Q der pris do ward R  $\cdot$ genomen] vernomen Q \textbf{17} sorge] sorgen R \textbf{18} brâhte] brach R  $\cdot$ ietweders] ietweder W \textbf{20} ander] andren R  $\cdot$ muosen] mvͤsten V \textbf{21} mit] Wit W  $\cdot$ orsen] roße W (R)  $\cdot$ betalle] [*]: mit alle V mit alle W Q vnd mit alle R \textbf{22} alsus] Als Q \textbf{23} al dâ] also R \textbf{24} verbicket] verblicket W vertzwicket Q \textbf{25} und] \textit{om.} W  $\cdot$ daz] das gar R \textbf{26} glîchiu] yeglicher R  $\cdot$ temperîe] getempere U [tem*]: temperie V tempire R \textbf{27} si] sy baide W die Q \textbf{28} muosen] mvͤzent V muͯs R  $\cdot$ scheidens] scheiden U  $\cdot$ bîten] bitten R \textbf{30} dân greif] Do engreif V (W) (Q) Do greiff R  $\cdot$ eht] \textit{om.} W auch Q  $\cdot$ nieman] niemans des R \newline
\end{minipage}
\end{table}
\end{document}
