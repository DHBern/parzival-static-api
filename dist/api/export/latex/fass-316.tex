\documentclass[8pt,a4paper,notitlepage]{article}
\usepackage{fullpage}
\usepackage{ulem}
\usepackage{xltxtra}
\usepackage{datetime}
\renewcommand{\dateseparator}{.}
\dmyyyydate
\usepackage{fancyhdr}
\usepackage{ifthen}
\pagestyle{fancy}
\fancyhf{}
\renewcommand{\headrulewidth}{0pt}
\fancyfoot[L]{\ifthenelse{\value{page}=1}{\today, \currenttime{} Uhr}{}}
\begin{document}
\begin{table}[ht]
\begin{minipage}[t]{0.5\linewidth}
\small
\begin{center}*D
\end{center}
\begin{tabular}{rl}
\textbf{316} & \textbf{\begin{large}E\end{large}r} truog iu vür \textbf{den} jâmers last.\\ 
 & ir vil ungetriwer gast!\\ 
 & sîn nôt iuch solt erbarmet hân.\\ 
 & daz \textbf{iu der} munt noch werde wan,\\ 
5 & ich meine der zungen drinne,\\ 
 & als \textbf{iuz} herze ist \textbf{rehter} sinne.\\ 
 & gein der helle ir sît benant\\ 
 & ze himele \textbf{vor} der hœhesten hant.\\ 
 & alsô sît ir ûf der erden,\\ 
10 & versinnent sich die werden.\\ 
 & Ir heiles ban, ir sælden vluoch,\\ 
 & des ganzen prîses reht unruoch!\\ 
 & ir sît manlîcher êren \textbf{schiech}\\ 
 & unt an der werdecheit \textbf{sô} \textbf{siech},\\ 
15 & \textbf{nehein arzet mag iuch} \textbf{des} ernern.\\ 
 & ich wil ûf iwerem houbte swern,\\ 
 & gît mir iemen des \textbf{den} eit,\\ 
 & daz grœzer valsch nie wart bereit\\ 
 & \textbf{neheinem} alsô schœnem man.\\ 
20 & ir vederangel, ir nâtern zan!\\ 
 & iu gab \textbf{iedoch} der wirt ein swert,\\ 
 & \textbf{des} iwer wirde \textbf{wart nie} wert.\\ 
 & dâ \textbf{erwarb} iu swîgen sünden \textbf{zil}.\\ 
 & ir sît der helle hirten spil.\\ 
25 & Geunêrter lîp, hêr Parzival!\\ 
 & ir sâhet \textbf{ouch} vür iuch tragen den Grâl\\ 
 & \textbf{unt} \textbf{snîdende} silber \textbf{unt} bluotic sper.\\ 
 & ir vreuden letze, ir trûrens wer!\\ 
 & wære ze Munsalvæsche iu \textbf{vrâgen} mit,\\ 
30 & in heidenschaft ze Thabronit\\ 
\end{tabular}
\scriptsize
\line(1,0){75} \newline
D \newline
\line(1,0){75} \newline
\textbf{1} \textit{Initiale} D  \textbf{11} \textit{Majuskel} D  \textbf{25} \textit{Majuskel} D  \newline
\line(1,0){75} \newline
\textbf{29} Munsalvæsche] Mvnsalvæsce D \textbf{30} Thabronit] Thabronît D \newline
\end{minipage}
\hspace{0.5cm}
\begin{minipage}[t]{0.5\linewidth}
\small
\begin{center}*m
\end{center}
\begin{tabular}{rl}
 & \textbf{man} truoc iu vür \textbf{des} jâmers last.\\ 
 & ir vil ungetriuwer gast!\\ 
 & sîn \textit{nôt} iuch solte erbarmet hâ\textit{n}.\\ 
 & daz \textbf{iu de\textit{r}} munt noch werde wa\textit{n},\\ 
5 & ich meine der zungen drinne,\\ 
 & als \textbf{iuwer} herze ist \textbf{rehter} sinne.\\ 
 & gegen der helle ir sît benant\\ 
 & ze himele \textbf{vor} der hœhesten hant.\\ 
 & als sît ir ûf der erden,\\ 
10 & vers\textit{inn}me\textit{n}t sich die werden.\\ 
 & ir heiles ban, ir sælden vluoch,\\ 
 & des ganzen prîses reht unruoch!\\ 
 & ir sît manlîcher êren \textbf{schiech}\\ 
 & und an der werdicheit \textbf{sô} \textbf{siech},\\ 
15 & \textbf{daz iuch \textit{kein} arzet mac} ernern.\\ 
 & ich wil ûf iuwerem houbete swern,\\ 
 & gît mir iemen des \textbf{den} eit,\\ 
 & daz grœzer valsch nie wart bereit\\ 
 & \textbf{dekeinem} alsô schœnen man.\\ 
20 & \dag ietweder angel\dag  ir nâteren zan!\\ 
 & iu gap \textbf{iedoch} der wirt ein swert,\\ 
 & \textbf{dâ}\textbf{s} iuwer werde \textbf{wart nie} wert.\\ 
 & dar \textbf{las} iu swîgen \textit{s}ünden \textbf{zil}.\\ 
 & ir sît der helle hirten spil.\\ 
25 & geunêrter lîp, hêr Parcifal!\\ 
 & ir sâhet vür iuch tragen den Grâl,\\ 
 & \textbf{snîdende} silber \textbf{und} bluti\textit{c} sper.\\ 
 & ir vröuden letze, ir trûrens wer!\\ 
 & wære ze Mun\textit{t}salvasche iu \textbf{vrâgen} mit,\\ 
30 & in \textit{h}eidenschaft ze Tabronit\\ 
\end{tabular}
\scriptsize
\line(1,0){75} \newline
m n o \newline
\line(1,0){75} \newline
\newline
\line(1,0){75} \newline
\textbf{1} man] Mag o  $\cdot$ vür] von o \textbf{3} nôt] \textit{om.} m not not o  $\cdot$ hân] hant m \textbf{4} daz] Des n  $\cdot$ der] des m dar o  $\cdot$ munt] \textit{om.} n o  $\cdot$ wan] want m man n o \textbf{7} gegen] Gegenge o \textbf{10} versinnent] Versumet m Versument n (o) \textbf{13} schiech] siech n \textbf{14} sô siech] soschiech n \textbf{15} kein] \textit{om.} m \textbf{16} ûf] uch o  $\cdot$ iuwerem] irem m vwern o \textbf{17} den] ein n \textbf{19} dekeinem] Do keinem n Dekeinen o  $\cdot$ schœnen] schonem n \textbf{20} ietweder] Jetwederem n Jtwiedern o  $\cdot$ nâteren] nater o \textbf{23} iu] ich n  $\cdot$ sünden] vunden m (n) \textbf{24} der] der der o  $\cdot$ hirten] hirte o \textbf{26} sâhet] sagent o  $\cdot$ vür] ouch bẏ n auch vor o \textbf{27} blutic] blutige m \textbf{28} vröuden] freide n (o)  $\cdot$ letze] les o \textbf{29} ze] nuͯ [m*]: zuͯ n  $\cdot$ Muntsalvasche] munsaluasce m muntsaluasce n (o) \textbf{30} heidenschaft] beiden schaft m (n) beidenschaff o  $\cdot$ Tabronit] trabonit n trobonit o \newline
\end{minipage}
\end{table}
\newpage
\begin{table}[ht]
\begin{minipage}[t]{0.5\linewidth}
\small
\begin{center}*G
\end{center}
\begin{tabular}{rl}
 & \textbf{er} truoc iu vür \textbf{den} jâmers last.\\ 
 & ir vil ungetriuwer gast!\\ 
 & sîn nôt iuch solte erbarmet hân.\\ 
 & daz \textbf{iuwer} munt noch werde wan,\\ 
5 & \begin{large}I\end{large}ch meine der zungen drinne,\\ 
 & als \textbf{iuz} herze ist \textbf{guoter} sinne.\\ 
 & gein der helle ir sît benant\\ 
 & ze himele \textbf{von} der hœhesten hant.\\ 
 & als sît ir ûf der erden,\\ 
10 & ve\textit{r}sinnent sich die werden.\\ 
 & ir heiles ban, ir sælden vluoch,\\ 
 & des ganzen brîses reht unruoch!\\ 
 & ir sît manlîcher êren \textbf{schiech}\\ 
 & unde an der werdicheit \textbf{siech},\\ 
15 & \textbf{dehein arzet mac iuch} \textbf{des} erneren.\\ 
 & ich wil ûf iuwerem houbte sweren,\\ 
 & gît mir iemen des \textbf{den} eit,\\ 
 & daz grœzer valsch nie wart bereit\\ 
 & \textbf{\textit{d}eheinem} als schœnen man.\\ 
20 & ir vederangel, ir nâteren zan!\\ 
 & iu gap \textbf{iedoch} der wirt ein swert,\\ 
 & \textbf{des} iuwer wirde \textbf{wart nie} wert.\\ 
 & dâ \textbf{erwarp} iu swîgen sünden \textbf{zil}.\\ 
 & ir sît der helle hirten spil.\\ 
25 & geunêrt lîp, hêr Parzival!\\ 
 & ir sâhet \textbf{doch} vür iuch tragen den Grâl\\ 
 & \textit{\textbf{und}} \textbf{snîden} silber, bluotic sper.\\ 
 & ir vröuden letze, ir trûrens wer!\\ 
 & wære ze Muntsalvatsche iu \textbf{vrâge\textit{n}} mit!\\ 
30 & in heidenschaft ze Tabrunit\\ 
\end{tabular}
\scriptsize
\line(1,0){75} \newline
G I O L M Q R Z Fr39 Fr40 Fr64 \newline
\line(1,0){75} \newline
\textbf{2} \textit{Initiale} I  \textbf{5} \textit{Initiale} G O  \textbf{17} \textit{Initiale} Fr39  \textbf{21} \textit{Initiale} M  \textbf{23} \textit{Initiale} I  \newline
\line(1,0){75} \newline
\textbf{1} \textit{Versfolge 316.2-1} I   $\cdot$ \textit{Vers 316.1 fehlt} Q   $\cdot$ er] Der M  $\cdot$ jâmers] iamer I \textbf{3} iuch solte] solt evch I (L) (M) (Q) (R) (Fr39)  $\cdot$ erbarmet] erbamet O \textbf{4} iuwer] euch der Q (R) (Fr39)  $\cdot$ wan] lam I man Q \textbf{5} Ich] ÷ch O  $\cdot$ der] die O R \textbf{6} iuz] ewer I (Q)  $\cdot$ herze] herre Q \textbf{7} \textit{Die Verse 316.7-318.8 fehlen} L  \textbf{8} von] vor I O Q R Z Fr39 Fr64 \textbf{10} versinnent] vesinnent G Ver suͦment O (Q) (Z) Vor synnet M  $\cdot$ sich] sis M \textbf{11} heiles ban] hæil span O \textbf{12} ganzen] ganzes O (Fr39) (Fr64) gancz in M  $\cdot$ reht] \textit{om.} Q \textbf{13} êren] werdecheit I  $\cdot$ schiech] siech M spiech Q zu nicht R \textbf{14} an der werdicheit] der ern I ander wirdikeitten R  $\cdot$ siech] so siech O Q Z Fr39 Fr64 so swiech M so sicht R \textbf{15} dehein arzet] Deheiner O  $\cdot$ des] \textit{om.} I R Fr39 \textbf{16} \textit{Versfolge 316.16-15} I   $\cdot$ wil] wil des I  $\cdot$ iuwerem] ewer I \textbf{17} des] \textit{om.} I  $\cdot$ den eit] eneydt Q \textbf{18} daz] daz nie I Sit daz O  $\cdot$ grœzer] grozer G (I) (O) (M) (Q) (Z) Fr39  $\cdot$ valsch] valsseit M  $\cdot$ nie wart] wart I wart nîe O  $\cdot$ bereit] hereit M \textbf{19} deheinem] andeheinem G dehain I  $\cdot$ schœnen] shonem I (M) (Q) (R) \textbf{20} vederangel] feder ir angel Q werder ander angel R  $\cdot$ nâteren] natre I vatir M nater R \textbf{21} iedoch der wirt] der wirt doch I  $\cdot$ swert] sper Q \textbf{23} dâ] Do O Q R  $\cdot$ iu] uwir M ir R \textbf{24} hirten] hirt ein O Q Fr39 wirt eyn M herre ein R \textbf{25} lîp] sy R  $\cdot$ hêr] he M ir Fr40  $\cdot$ Parzival] parzifal I M Fr40 Barcifal O partzifal Q parczifal R parcifal Z Fr39 \textbf{26} ir sâhet] Jr sogt M Jch sagt Q  $\cdot$ doch vür iuch] fvͤr ivch doch O ovch fuͥr vͥch Fr39 \textbf{27} und] \textit{om.} G  $\cdot$ snîden] snidic I snidende O (Q) Z Fr39 (Fr40) senden M schinende R  $\cdot$ bluotic] vnd bluͤtich I (Q) (R) (Z) (Fr39) (Fr40) vnde M \textbf{28} Jr fuͯrent lancze in turners wer R  $\cdot$ ir vröuden] vnde frevden O Jr vroidens M  $\cdot$ ir trûrens] vnd truren Z  $\cdot$ wer] ver Q \textbf{29} wære] wer ev I We R  $\cdot$ ze Muntsalvatsche] zemvntsalvatsche G zemuntschalfasche I zemvntschalvatsche O zcu Muntsalfatsche M zu [múntsa*]: múntsalvasche Q ze munsaluashe R Fr39 zv montsalvatsche Z ze munsalvatse Fr40  $\cdot$ iu] \textit{om.} I  $\cdot$ vrâgen] frage G \textbf{30} ze Tabrunit] ze brunete I ze brvnîte O zcu britvnit M zu tabrunick Q zu Taburnit R zvͦ tabruͦnit Fr39 zetabrunit Fr40 \newline
\end{minipage}
\hspace{0.5cm}
\begin{minipage}[t]{0.5\linewidth}
\small
\begin{center}*T
\end{center}
\begin{tabular}{rl}
 & \textbf{er} truoc iu vür \textbf{den} jâmers last.\\ 
 & ir vil ungetriuwer gast!\\ 
 & sîn nôt iuch solte erbarmet hân.\\ 
 & daz \textbf{iuwer} munt noch werde wan,\\ 
5 & ich meine der zungen drinne,\\ 
 & als \textbf{iu daz} herze ist \textbf{guoter} sinne.\\ 
 & gegen der helle ir sît benant\\ 
 & ze himele \textbf{vor} der hœhesten hant.\\ 
 & als sît ir ûf der erden,\\ 
10 & versinnent sich die werden.\\ 
 & ir heiles ban, ir sælden vluoch,\\ 
 & des ganzen prîses reht unruoch!\\ 
 & ir sît manlîcher êren \textbf{siech}\\ 
 & unde an der werdecheit \textbf{sô} \textbf{kriech},\\ 
15 & \textbf{dehein arzât mac iuch} \textbf{des} ernern.\\ 
 & ich wil ûf iuwerm houbete swern,\\ 
 & gît mir ieman des \textbf{einen} eit,\\ 
 & daz grœzer valsch nie wart bereit\\ 
 & \textbf{deheinen} alsô schœnen man.\\ 
20 & ir vederangel, ir nâtern zan!\\ 
 & iu gab \textbf{doch} der wirt ein swert,\\ 
 & \textbf{des} iuwer wirde \textbf{nie wart} wert.\\ 
 & dô \textbf{erwarp} iu swîgen sünden \textbf{vil}.\\ 
 & ir sît der helle hirten spil.\\ 
25 & Geunêrter lîp, hêr Parcifal!\\ 
 & ir sâht \textbf{ouch} vür iuch tragen den Grâl\\ 
 & \textbf{unde} \textbf{snîdende} silber \textbf{unde} bluotic sper.\\ 
 & ir vroüden letze, ir trûrens wer!\\ 
 & wære ze Munsalvasche iu \textbf{vrâge} mite,\\ 
30 & in heidenschaft ze Tabrunite\\ 
\end{tabular}
\scriptsize
\line(1,0){75} \newline
T U V W \newline
\line(1,0){75} \newline
\textbf{25} \textit{Majuskel} T  \newline
\line(1,0){75} \newline
\textbf{1} er] Man V  $\cdot$ den] dez V \textbf{3} iuch solte] îv solte T solt eúch W \textbf{4} noch werde wan] mir noch werdent an W \textbf{5} der] die W \textbf{6} iu daz] uwer V  $\cdot$ guoter] rehter V an guͦter W \textbf{7} ir sît] seint ir W  $\cdot$ benant] genant U V W \textbf{8} himele] hẏmelriche V  $\cdot$ hœhesten] hoͤstesten V \textbf{11} ir heiles] Jrs heiles U  $\cdot$ ban] han W \textbf{12} reht] gantzer W  $\cdot$ unruoch] vor vch U \textbf{13} siech] schiech V \textbf{14} \textit{Vers 316.14 fehlt} U   $\cdot$ Mir ist úwer schone sam ein kriech W  $\cdot$ kriech] siech V \textbf{15} iuch] îv T  $\cdot$ des ernern] generen W \textbf{17} mir ieman] iemant mir W  $\cdot$ einen] den U V \textbf{19} deheinen] Dekeine U Deheinem V (W)  $\cdot$ schœnen] schoͤnem W \textbf{20} nâtern] nater V (W) \textbf{21} doch] ie doch U (V) (W) \textbf{23} vil] zil U V W \textbf{24} hirten] hirte V  $\cdot$ spil] [*]: ein spil V \textbf{25} Parcifal] parzifal T V partzifal W \textbf{26} ouch] \textit{om.} W  $\cdot$ iuch] îv T \textbf{27} unde snîdende] Schneidendes W  $\cdot$ bluotic] bluͦtends W \textbf{28} ir trûrens] in traurens W \textbf{29} ze Munsalvasche] ze [m*]: mvnsalvasce T zuͦ Muͦntsalvatsche U zemvntsaluasche V zuͦ montsaluatz W  $\cdot$ vrâge] vragen V (W) \textbf{30} Tabrunite] Tabruͦnite U tabrunit V tabronitte W \newline
\end{minipage}
\end{table}
\end{document}
