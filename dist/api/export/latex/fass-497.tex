\documentclass[8pt,a4paper,notitlepage]{article}
\usepackage{fullpage}
\usepackage{ulem}
\usepackage{xltxtra}
\usepackage{datetime}
\renewcommand{\dateseparator}{.}
\dmyyyydate
\usepackage{fancyhdr}
\usepackage{ifthen}
\pagestyle{fancy}
\fancyhf{}
\renewcommand{\headrulewidth}{0pt}
\fancyfoot[L]{\ifthenelse{\value{page}=1}{\today, \currenttime{} Uhr}{}}
\begin{document}
\begin{table}[ht]
\begin{minipage}[t]{0.5\linewidth}
\small
\begin{center}*D
\end{center}
\begin{tabular}{rl}
\textbf{497} & \textbf{\begin{large}D\end{large}az} was ê von im dîn sage;\\ 
 & \textbf{ez} ist immer mînes herzen klage.\\ 
 & mîn bruoder ist guotes rîche;\\ 
 & verholne rîterlîche\\ 
5 & er mich dicke von im sande.\\ 
 & sô ich von Munsalvæsche wande,\\ 
 & \textbf{sîn insigel nam ich} dâ\\ 
 & unt \textbf{vuortez ze} Karchobra,\\ 
 & dâ sich sewet der Plimizœl,\\ 
10 & in dem bistuom ze Barbigœl.\\ 
 & der burcgrâve mich dâ beriet\\ 
 & ûff\textbf{ez} insigel, \textbf{ê} ich \textbf{von im} schiet,\\ 
 & knappen unt anderre koste\\ 
 & gein der wilden tjoste\\ 
15 & unt ûf \textbf{ander} \textbf{rîterlîche} vart;\\ 
 & des wart \textbf{vil} wênec von im gespart.\\ 
 & ich m\textit{uo}se al eine komen dar:\\ 
 & \textbf{an} der widerreise liez ich gar\\ 
 & bî im, swaz ich gesindes pflac.\\ 
20 & \textbf{ich} reit, \textbf{dâ} Munsalvæsche lac.\\ 
 & nû \textbf{hœre}, lieber neve mîn:\\ 
 & dô mich der werde \textbf{vater dîn}\\ 
 & ze Sibilje alrêste \textbf{sach},\\ 
 & balde er mîn ze bruoder jach\\ 
25 & Herzeloyden, sînem wîbe;\\ 
 & \textbf{doch} wart von sîme lîbe\\ 
 & mîn antlütze nie mêr gesehen.\\ 
 & man muose ouch \textbf{mir} vür wâr \textbf{dâ} jehen,\\ 
 & daz nie \textbf{schœner mannes bilde} wart;\\ 
30 & dannoch was ich âne bart.\\ 
\end{tabular}
\scriptsize
\line(1,0){75} \newline
D Fr11 \newline
\line(1,0){75} \newline
\textbf{1} \textit{Initiale} D  \newline
\line(1,0){75} \newline
\textbf{6} von] \textit{om.} Fr11  $\cdot$ Munsalvæsche] Mvnsælvæsche D Munsalvaische Fr11  $\cdot$ wande] vande Fr11 \textbf{8} Karchobra] Karchobrâ D \textbf{9} Plimizœl] Plimizoͤl D plimizol Fr11 \textbf{10} Barbigœl] Barbigoͤl D Barbigol Fr11 \textbf{15} rîterlîche] ritterlichivͯ Fr11 \textbf{17} muose] mvͤse D muͯst Fr11 \textbf{20} Munsalvæsche] Mvnsælvæsce D \textbf{23} Sibilje] Sibilie D \textbf{25} Herzeloyden] Herceloyden D \textbf{28} muose] muͯst Fr11 \textbf{29} schœner] schoners Fr11 \newline
\end{minipage}
\hspace{0.5cm}
\begin{minipage}[t]{0.5\linewidth}
\small
\begin{center}*m
\end{center}
\begin{tabular}{rl}
 & \textbf{daz} was ê von im dîn sage\\ 
 & \textbf{und} ist iemer mînes herzen klage.\\ 
 & mîn bruoder ist guotes rîch;\\ 
 & verholn ritterlîch\\ 
5 & er mich dicke von im sante.\\ 
 & sô ich von Mun\textit{t}salvasche wante,\\ 
 & \textbf{sîn ingesigel nam ich} dâ\\ 
 & und \textbf{vuorte ez zuo} Ca\textit{rko}bra,\\ 
 & d\textit{â} sich sewet der Plimizol,\\ 
10 & in dem bistuom zuo Barbig\textit{o}l.\\ 
 & der burcgrâve mich d\textit{â} beriet\\ 
 & ûf \textbf{daz} ingesigel, \textbf{als} ich \textbf{von im} schiet,\\ 
 & knappen und ander koste\\ 
 & gegen der wilden joste\\ 
15 & und ûf \textbf{ritterlîche} vart;\\ 
 & des wart \textbf{vil} wênic von im gespart.\\ 
 & ich muos alein komen dar:\\ 
 & \textbf{an} der widerreise liez ich gar\\ 
 & bî im, waz ich gesindes pflac,\\ 
20 & \textbf{und} reit, \textbf{d\textit{â}} Mun\textit{t}salvasche lac.\\ 
 & nû \textbf{hœre}, lieber neve mîn:\\ 
 & dô mich der werde \textbf{vater dîn}\\ 
 & zuo Sibilje allerêrst \textbf{sach},\\ 
 & balde er mîn zuo bruoder jach\\ 
25 & Herczeloiden, sînem wîbe;\\ 
 & \textbf{doch} wart \textit{von s}înem lîbe\\ 
 & mîn antlitze nie mê gesehen.\\ 
 & man muos ouch \textbf{mir} vür wâr jehen,\\ 
 & daz nie \textbf{mannes bilde schœner} wart;\\ 
30 & dannoch was ich âne bart.\\ 
\end{tabular}
\scriptsize
\line(1,0){75} \newline
m n o \newline
\line(1,0){75} \newline
\textbf{21} \textit{Capitulumzeichen} n  \newline
\line(1,0){75} \newline
\textbf{1} im] dem o  $\cdot$ dîn] sin n o \textbf{6} Muntsalvasche] munsaluasce m montsaluasce n o \textbf{7} dâ] do n \textbf{8} Carkobra] kachtabra m cathabro n kathabra o \textbf{9} dâ] Do m n o  $\cdot$ Plimizol] plúmzal n \textbf{10} Barbigol] barbigal m n parbigal o \textbf{11} dâ] do m n o  $\cdot$ beriet] bereit o \textbf{12} als] E n (o) \textbf{15} ritterlîche] ander ritterliche n o \textbf{17} muos] muͯsz n \textbf{20} dâ] do m n o  $\cdot$ Muntsalvasche] munsaluasce m montsaluasce n muntsaluasce o \textbf{22} dô] Doch o \textbf{23} Sibilje] sibilie m n sibillige o \textbf{25} Herczeloiden] Hertzeloiden n Herczeleide o \textbf{26} von sînem] sin elich sinem m \textbf{27} antlitze] anczlit o  $\cdot$ mê] [nie]: me m \textbf{28} muos] muͯsz n  $\cdot$ jehen] do jehen n do johen o \newline
\end{minipage}
\end{table}
\newpage
\begin{table}[ht]
\begin{minipage}[t]{0.5\linewidth}
\small
\begin{center}*G
\end{center}
\begin{tabular}{rl}
 & \textbf{\begin{large}E\end{large}z} was ê von im dîn sage;\\ 
 & \textbf{ez} ist immer mînes herzen klage.\\ 
 & mîn bruoder ist guotes rîche;\\ 
 & verholne rîterlîche\\ 
5 & er mich dicke von im sande.\\ 
 & sô ich von Muntsalvatsche wande,\\ 
 & \textbf{sîn insigel nam ich} dâ\\ 
 & unde \textbf{vuort ez ze} Charchopra,\\ 
 & dâ sich s\textit{e}wet der Blimzol,\\ 
10 & in dem bistuom ze Barbigol.\\ 
 & der burcgrâve mich dâ beriet\\ 
 & ûf\textbf{ez} insigel, \textbf{ê} ich \textbf{von im} schiet,\\ 
 & knappen unde ander koste\\ 
 & gein der wilden tjoste\\ 
15 & unde ûf \textbf{rîterlîche} vart;\\ 
 & des wart \textbf{vil} wênic von im gespart.\\ 
 & ich muos al ein komen dar:\\ 
 & \textbf{ûf} der widerr\textit{ei}se liez ich gar\\ 
 & bî im, swaz ich gesindes pflac.\\ 
20 & \textit{\textbf{ich}} \textit{reit}, \textit{\textbf{dâ}} \textit{Muntsalvatsche lac.}\\ 
 & \textit{nû} \textit{\textbf{hœre}}, \textit{lieber neve mîn:}\\ 
 & \textit{dô mich der werde} \textit{\textbf{vater dîn}}\\ 
 & \textit{ze Sibiljen alrêrst} \textit{\textbf{gesach}}\textit{,}\\ 
 & balde er mîn ze bruoder jac\textit{h}\\ 
25 & Herzeloiden, sînem wîbe;\\ 
 & \textbf{doch} wart von sînem lîbe\\ 
 & mîn antlütze nie mê gesehen.\\ 
 & man muose ouch \textbf{mir} vür wâr \textbf{dâ} jehen,\\ 
 & daz nie \textbf{schœner mannes bilde} wart;\\ 
30 & dannoch was ich âne bart.\\ 
\end{tabular}
\scriptsize
\line(1,0){75} \newline
G I L M Z Fr61 \newline
\line(1,0){75} \newline
\textbf{1} \textit{Initiale} G I L Z  \textbf{17} \textit{Initiale} I  \textbf{21} \textit{Initiale} Fr61  \newline
\line(1,0){75} \newline
\textbf{1} Ez] Daz L (M) Z (Fr61)  $\cdot$ ê] \textit{om.} I y M \textbf{2} immer] vrumer M  $\cdot$ mînes herzen] min Z in meins hertzen Fr61 \textbf{6} sô] Do Z  $\cdot$ Muntsalvatsche] mvntsaluatsche G munshaluasce I montsalvatsche Z \textbf{7} insigel] jngesigile M \textbf{8} ze Charchopra] zecharoch bra G ze kokebra I zu Karchobra L (M) (Z) ze Kargogra Fr61 \textbf{9} sich sewet] sih swet G sewet sich Fr61  $\cdot$ Blimzol] blimezol G blimizol I (L) M plimizol Z (Fr61) \textbf{10} ze Barbigol] zebarbigol G \textbf{12} von im] dannan I \textbf{15} ûf] vf ander I (L) (M) (Fr61)  $\cdot$ rîterlîche] retterleichen Fr61 \textbf{16} vil] ich L  $\cdot$ von] gein L \textbf{17} muos] mues G \textbf{18} ûf der] An der L Z (Fr61) Ander M  $\cdot$ widerreise] wider riese G \textbf{19} swaz] waz L (M) \textbf{20} \textit{Die Verse 497.20-23 fehlen} G   $\cdot$ Muntsalvatsche] Muntshalvasce I montsalvatsche Z \textbf{21} nû] Vnd Z  $\cdot$ hœre] horet Fr61  $\cdot$ lieber] libe M  $\cdot$ neve] ohaim Fr61 \textbf{22} dô mich] do mich do mich I Do kom L Da mich M Z  $\cdot$ werde] liebe L (M) \textbf{23} ze] Datz Fr61  $\cdot$ Sibiljen] Sibilien I Sibilie L M Z Fr61  $\cdot$ alrêrst gesach] alres gischach M alrerst sach Z alrerst ersach Fr61 \textbf{24} balde] baz I  $\cdot$ mîn ze] zcu myme M  $\cdot$ jach] iac G sprach M \textbf{25} Herzeloiden] Herzeloyden G herzenlauden I Hertzelovden L Herczelouden M Herzeloude Z Hertzenloẏden Fr61  $\cdot$ sînem] sinen I \textbf{27} nie mê] niemer me L (M) êe nie Fr61 \textbf{28} Man muͯste mir ouch da vor war ýehen L \textbf{29} schœner mannes] mannes schoner L  $\cdot$ bilde] lip I \newline
\end{minipage}
\hspace{0.5cm}
\begin{minipage}[t]{0.5\linewidth}
\small
\begin{center}*T
\end{center}
\begin{tabular}{rl}
 & \textbf{daz} was ê von im dîn sage;\\ 
 & \textbf{ez} ist iemer mînes herzen klage.\\ 
 & Mîn bruoder ist guotes rîche;\\ 
 & verholne rîterlîche\\ 
5 & er mich dicke von im sante.\\ 
 & sô ich von Munsalvasche wante,\\ 
 & \textbf{ich nam sîn ingesigel} \textbf{al}dâ\\ 
 & unde \textbf{vuor gegen} Kachopra,\\ 
 & dâ sich sewet der Plymizol,\\ 
10 & in dem bischtuome ze Barbigol.\\ 
 & der \textit{b}urcgrâve mich dâ beriet\\ 
 & ûf \textbf{sîn} insigel, \textbf{ê} ich \textbf{dannen} schiet,\\ 
 & knappen unde ander koste\\ 
 & gegen der wilden tjoste\\ 
15 & unde ûf \textbf{ander} \textbf{manege} vart;\\ 
 & des wart \textbf{dâ} wênic von im gespart.\\ 
 & ich muose aleine komen dar:\\ 
 & \textbf{an} der widerreise liez ich gar\\ 
 & bî im, swaz ich gesindes pflac,\\ 
20 & \textbf{unde} reit, \textbf{swâ} Munsalvasche lac.\\ 
 & \begin{large}N\end{large}û \textbf{hœret}, lieber neve mîn:\\ 
 & dô mich der werde \textbf{Anschevin}\\ 
 & ze Sybilie alrêst \textbf{gesach},\\ 
 & balder mîn ze bruoder jach\\ 
25 & \textbf{von} Herzeloyden, sînem wîbe;\\ 
 & \textbf{ouch} wart von sînem lîbe\\ 
 & mîn antlitze nie mê gesehen.\\ 
 & man muose ouch vür wâr \textbf{des} jehen,\\ 
 & daz nie \textbf{schœner mannes bilde} wart;\\ 
30 & dannoch was ich âne bart.\\ 
\end{tabular}
\scriptsize
\line(1,0){75} \newline
T U V W O Q R \newline
\line(1,0){75} \newline
\textbf{3} \textit{Initiale} W O Q   $\cdot$ \textit{Majuskel} T  \textbf{21} \textit{Initiale} T V R  \newline
\line(1,0){75} \newline
\textbf{1} \textit{Die Verse 453.1-502.30 fehlen} U   $\cdot$ dîn] div O \textbf{3} Mîn] ÷in O  $\cdot$ ist] ich R \textbf{4} verholne] Vor holnnde Q \textbf{6} sô] Do Q  $\cdot$ Munsalvasche] [mvnt*vasche]: mvntsalvasche V montsaluatschs W muntsaluasche Q [Munsala]: Munsaluasche R mvntsalvatsche O \textbf{7} ingesigel] insigel W O (Q) (R)  $\cdot$ aldâ] da V W O do Q \textbf{8} vuor gegen] [f*]: fvͦrte ez zvͦ V fvͦr ze O (Q)  $\cdot$ Kachopra] [*]: karkobra V karchopia W karchapra O karcopra Q karkopra R \textbf{9} dâ] Do V W Q R  $\cdot$ sewet] [swet]: sewet T  $\cdot$ Plymizol] [brimisol]: brimizol O plinizol R \textbf{10} Barbigol] [B*bigol]: Barbigol V \textbf{11} burcgrâve] vurcgrave T  $\cdot$ dâ] do V W Q R \textbf{12} ûf sîn] [Vf s*]: Vffenz V Vnd sein W Vf O  $\cdot$ insigel] ingesigel V (Q)  $\cdot$ ê] das Q  $\cdot$ dannen] [d*]: von im V von im O Q (R) von in W \textbf{13} knappen] [Cchappe]: Cchappe O \textbf{14} tjoste] troste O \textbf{15} manege] [*]: ritterliche V ritterliche W (O) Q R \textbf{16} dâ wênic von im] [*]: vil wening von im V von im wenig do W vil wenich von im O von im do wenick Q (R) \textbf{17} muose] mvese T mvͤste V muͦß W muͦs R \textbf{18} an] Jn V O \textbf{19} swaz] was W Q R  $\cdot$ gesindes] gesundes R \textbf{20} unde] Jch R  $\cdot$ swâ] [*]: do V wo W Q (R)  $\cdot$ Munsalvasche] [*]: mvntsalvasche V montsaluatsche W muntsaluasche Q munsaluasche R mvntsalvatsche O \textbf{21} hœret] mercke du W hore Q (R) \textbf{22} Anschevin] anscevin T [*i*]: vatter din V antscheuein W Anshevin O (R) ansheuin Q \textbf{23} Sybilie] Sybile T sibilie V W O sibille Q Sibibe R  $\cdot$ gesach] sach W O Q ersach R \textbf{25} von] \textit{om.} W O Q R  $\cdot$ Herzeloyden] herzelauden V Hertzeloyden W Herzelavden O Hertzevlouden Q Herczelaude R  $\cdot$ sînem] sin R \textbf{26} ouch] [*]: Doch V \textbf{27} mîn] Do in Q  $\cdot$ nie mê] nie Q \textbf{28} muose] mvese T mvͤst V (R) muͦsse W (O)  $\cdot$ ouch] oͮch mir V (O) (Q) mir W och mir do R  $\cdot$ des] [d*]: do V do W Q da O \textit{om.} R \textbf{29} schœner] schoͯners R  $\cdot$ wart] da wart O \newline
\end{minipage}
\end{table}
\end{document}
