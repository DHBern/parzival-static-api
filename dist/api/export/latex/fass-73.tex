\documentclass[8pt,a4paper,notitlepage]{article}
\usepackage{fullpage}
\usepackage{ulem}
\usepackage{xltxtra}
\usepackage{datetime}
\renewcommand{\dateseparator}{.}
\dmyyyydate
\usepackage{fancyhdr}
\usepackage{ifthen}
\pagestyle{fancy}
\fancyhf{}
\renewcommand{\headrulewidth}{0pt}
\fancyfoot[L]{\ifthenelse{\value{page}=1}{\today, \currenttime{} Uhr}{}}
\begin{document}
\begin{table}[ht]
\begin{minipage}[t]{0.5\linewidth}
\small
\begin{center}*D
\end{center}
\begin{tabular}{rl}
\textbf{73} & der anker kom doch vor an in.\\ 
 & dô stach in hinderz ors dort hin\\ 
 & der werde künec von Zazamanc\\ 
 & unt viengen. dâ \textbf{was} grôz gedranc,\\ 
5 & hôhe vurche sleht \textbf{getennet},\\ 
 & mit swerten vil \textbf{gekemmet}.\\ 
 & \begin{large}D\end{large}â wart verswendet der walt\\ 
 & unt manec ritter ab gevalt.\\ 
 & si wunden sich, sus \textbf{hœre} ich sagen,\\ 
10 & \textbf{hindenort}, \textbf{dâ} hielden zagen.\\ 
 & der strît was wol sô nâhen,\\ 
 & daz gar die vrouwen sâhen,\\ 
 & wer dâ \textbf{bî} prîse solde sîn.\\ 
 & der minnen gernde Riwalin,\\ 
15 & von des \textbf{sper} snîte ein niwe leis.\\ 
 & daz was der künec von Lohneis.\\ 
 & sîne hurte gâben \textbf{kraches} schal.\\ 
 & Morholt in einen rîter stal.\\ 
 & ûzem satel \textbf{er in vür sich huop}.\\ 
20 & \textbf{daz was ein ungevüeger uop}.\\ 
 & \textbf{der hiez} Killirjacac.\\ 
 & von dem hete der künec Lac\\ 
 & dâ vor enpfangen solhen solt,\\ 
 & \textbf{den} \textbf{er} \textbf{vallende} an der erden holt.\\ 
25 & \textbf{er} het ez dâ vil \textbf{guot} getân.\\ 
 & dô luste disen starken man,\\ 
 & daz er in twünge sunder swert.\\ 
 & alsus \textbf{vieng er} den degen wert.\\ 
 & hinder\textit{z} ors stach Kayletes hant\\ 
30 & den herzogen von Brabant.\\ 
\end{tabular}
\scriptsize
\line(1,0){75} \newline
D Fr33 \newline
\line(1,0){75} \newline
\textbf{7} \textit{Initiale} D  \newline
\line(1,0){75} \newline
\textbf{3} Zazamanc] Zazamanch D \textbf{8} manec] \textit{om.} Fr33 \textbf{9} sus] so Fr33  $\cdot$ hœre] [hort]: hor Fr33 \textbf{10} hindenort] hin anden ort Fr33 \textbf{21} Killirjacac] killiriacach D \textbf{22} Lac] Lach D \textbf{29} hinderz] hindrs D \newline
\end{minipage}
\hspace{0.5cm}
\begin{minipage}[t]{0.5\linewidth}
\small
\begin{center}*m
\end{center}
\begin{tabular}{rl}
 & der anker kam doch vor an in.\\ 
 & dô stach in hinderz ros dort hin\\ 
 & der werde künic von Zazamanc\\ 
 & und vienc in. d\textit{â} \textbf{was} grôz gedranc,\\ 
5 & \begin{large}H\end{large}ôhe vurche sleht \textbf{getennet},\\ 
 & mit swerten vil \dag gekennet\dag .\\ 
 & d\textit{â} wart verswendet der walt\\ 
 & und manic ritter abe gevalt.\\ 
 & si wunden sich, sus \textbf{hôrt} ich sagen,\\ 
10 & \textbf{hin an den \textit{or}t}, \textbf{d\textit{â}} hielten zagen.\\ 
 & der strît wa\textit{s} wol sô nâhen,\\ 
 & daz gar \textit{die} vrouwen sâhen,\\ 
 & wer dâ \textbf{bî} prîse solte sîn.\\ 
 & der minnen gernde Riva\textit{l}in,\\ 
15 & von des \textbf{spern} snîte ein niuwe leis.\\ 
 & daz was der künic von Lohneis.\\ 
 & sîne h\textit{u}rte gâben \textbf{kraches} schal.\\ 
 & Morolt in einen ritter stal.\\ 
 & ûzem satel \textbf{er in vür sich huop}.\\ 
20 & \textbf{daz was ein ungevüeger uop}.\\ 
 & \textbf{daz was der werde} Kiliriacac.\\ 
 & von dem hete der künic Lac\\ 
 & dâ vor enpfangen solichen solt,\\ 
 & \textbf{den} \textbf{der} \textbf{vallende} an der erden holt.\\ 
25 & \textbf{er} hete ez dâ vil \textbf{guot} getân.\\ 
 & dô luste disen starken man,\\ 
 & daz er in twünge sunder swert.\\ 
 & alsus \textbf{viengen si} den degen wert.\\ 
 & hinder daz ros stach Kailetes hant\\ 
30 & den herzogen von Brabant.\\ 
\end{tabular}
\scriptsize
\line(1,0){75} \newline
m n o \newline
\line(1,0){75} \newline
\textbf{5} \textit{Initiale} m   $\cdot$ \textit{Capitulumzeichen} n  \newline
\line(1,0){75} \newline
\textbf{1} doch] dort o \textbf{2} hinderz] nider das o \textbf{3} Zazamanc] [zaramang]: zazamang m zazamang n o \textbf{4} in] \textit{om.} n  $\cdot$ dâ] do m n o  $\cdot$ grôz] grosz grosz n \textit{om.} o \textbf{6} gekennet] erkennet o \textbf{7} dâ] Do m n o \textbf{9} si] [Sich]: Si m \textbf{10} den] dem o  $\cdot$ ort] rot m rat n o  $\cdot$ dâ] do m n o \textbf{11} was] wart m  $\cdot$ sô] zuͦ o \textbf{12} gar die] gar so m die n wol die o \textbf{13} wer] Der n  $\cdot$ dâ] do n o  $\cdot$ prîse] prisen n o \textbf{14} minnen gernde] mynnegernde n mÿnnende gerne o  $\cdot$ Rivalin] [rivlain]: rivalain m riolin n o \textbf{15} des] das o  $\cdot$ spern] sper n o  $\cdot$ snîte] snẏt n (o) \textbf{16} der] ein n  $\cdot$ Lohneis] lonoheis n o \textbf{17} hurte] herte m n o \textbf{21} daz] Vnd n o  $\cdot$ Kiliriacac] Kiliria kack m killiakag n kẏliakag o \textbf{22} Lac] lag m n o \textbf{23} enpfangen] entfollen o \textbf{25} ez] das o  $\cdot$ dâ] do n o \textbf{26} luste] loste n lostent o \textbf{27} twünge] twinge n \textbf{28} viengen si] ving er n twing er o  $\cdot$ degen] tagen o \textbf{29} Kailetes] kailettes m laẏlites n lailetes o \textbf{30} Brabant] brobrant n o \newline
\end{minipage}
\end{table}
\newpage
\begin{table}[ht]
\begin{minipage}[t]{0.5\linewidth}
\small
\begin{center}*G
\end{center}
\begin{tabular}{rl}
 & der anker kom doch vor an in.\\ 
 & dô stach in hinderz ors dort hin\\ 
 & der werde künic von Zazamanc\\ 
 & unde vieng in. dâ \textbf{wart} grôz gedranc,\\ 
5 & hôhe vurche sleht \textbf{getennet},\\ 
 & mit swerten vil \textbf{gek\textit{l}emmet}.\\ 
 & dâ wart verswendet der walt\\ 
 & unde manic rîter abe gevalt.\\ 
 & si wunden sich, sus \textbf{hôrt} ich sagen,\\ 
10 & \textbf{hin an den ort}, \textbf{dort} hielten zagen.\\ 
 & der strît was wol sô nâhen,\\ 
 & daz gar die vrouwen sâhen,\\ 
 & wer dâ \textbf{mit} prîse solte sîn.\\ 
 & der minne gernde Riwalin,\\ 
15 & von des \textbf{speren} snîte ein niwiu leis.\\ 
 & daz was der künic von Lohneis.\\ 
 & sîne hurte gâben \textbf{kraches} schal.\\ 
 & Morolt in einen rîter stal.\\ 
 & ûz dem satel \textbf{er in huop}.\\ 
20 & \textbf{daz was ein ungevüeger uop}.\\ 
 & \textbf{der hiez} Kiliriakac.\\ 
 & von dem het der künic Lac\\ 
 & \begin{large}D\end{large}â vor enpfangen solhen solt,\\ 
 & \textbf{den} \textbf{er} \textbf{vallende} an der erde holt,\\ 
25 & \textbf{und} hetz \textit{\textbf{ouch}} dâ vil \textbf{guot} getân.\\ 
 & dô luste disen starken man,\\ 
 & daz er in twünge sunder swert.\\ 
 & alsus \textbf{vienc er} den degen wert.\\ 
 & hinderz ors stach Kailetes hant\\ 
30 & den herzogen von Brabant.\\ 
\end{tabular}
\scriptsize
\line(1,0){75} \newline
G I O L M Q R Z Fr21 Fr56 \newline
\line(1,0){75} \newline
\textbf{1} \textit{Initiale} O  \textbf{11} \textit{Initiale} I  \textbf{23} \textit{Initiale} G  \textbf{29} \textit{Initiale} L R Z Fr21 Fr56  \newline
\line(1,0){75} \newline
\textbf{1} der] ÷er O  $\cdot$ doch] \textit{om.} Q \textbf{2} dô] Da M Z Dort Q  $\cdot$ in] \textit{om.} O her on M  $\cdot$ dort] \textit{om.} Q \textbf{3} künic] chvͦne O  $\cdot$ Zazamanc] zazamach G zazamanch O L zazamant Q [zasmant]: zasamant R \textbf{4} dâ] do O Q R  $\cdot$ gedranc] gedankt R \textbf{5} getennet] getemmet L (M) \textbf{6} \textit{Versdoppelung 73.6-7 (²O) nach 73.7; Lesarten der vorausgehenden Verse mit ¹O bezeichnet} O   $\cdot$ geklemmet] [gechlenbet]: gechenbet G bechlenget I gechemphet O (M) gehemmet Q gekennet R gekemmet Z (Fr21) \textbf{7} dâ] Do \textsuperscript{2}\hspace{-1.3mm} O Q  $\cdot$ wart] \textit{om.} Fr21  $\cdot$ verswendet] geswendet L  $\cdot$ walt] ward R \textbf{8} unde] \textit{om.} O Q Fr21 \textbf{9} wunden] wugen M vonden Q  $\cdot$ sus] so O Fr21 \textit{om.} Q \textbf{10} den] deme M  $\cdot$ dort] da Q  $\cdot$ zagen] sagin M \textbf{11} wol sô] noh so I [also]: wol so L  $\cdot$ nâhen] nahen nahen L \textbf{14} minne] mynnen M  $\cdot$ Riwalin] ribalin I Rywalin L ryvalin M riualeyn Q kywalin Fr21 \textbf{15} speren] sper M  $\cdot$ snîte] [seiten]: sneyden Q sniten Fr56  $\cdot$ ein] \textit{om.} Fr56  $\cdot$ niwiu leis] [niwaleis]: nivwaleis O \textbf{16} Lohneis] iohaneis I Johneis O lohenaiz L loneis M loheneis Z Lehneis Fr21 lonheis Fr56 \textbf{17} kraches] crache I (M) chranchen O crachens L (Z) (Fr21) (Fr56)  $\cdot$ schal] sal Z \textbf{18} Morolt] morholt I (O) (L) (Z) (Fr21) Fr56 Morholtt R \textbf{19} ûz dem] Zu dem Q Vsserm R  $\cdot$ huop] fvr sich hvͦb O (L) (M) (Q) (R) (Z) (Fr56) vf fvͦr sich Fr21 \textbf{20} was] \textit{om.} R  $\cdot$ uop] [h*]: lvp Z \textbf{21} Kiliriakac] kiliriakach G kaliriacac I (M) kyliriacach O killiriakach L Fr56 kalliriack Q kylyriakat R killiriakac Z Fr21 \textbf{22} het] her O  $\cdot$ Lac] lach G I O Fr56 lat R \textbf{23} enpfangen] enphiengen O  $\cdot$ solhen] selbe en O \textbf{24} er] der L Q R Z Fr56  $\cdot$ an] vf I (R)  $\cdot$ erde] erden M \textbf{25} und] Er Z  $\cdot$ ouch dâ] da G auch do Q  $\cdot$ vil] \textit{om.} I  $\cdot$ guot] guͦtz R \textbf{26} dô] Da M Z  $\cdot$ luste] geluste L lo luste M \textbf{27} in] \textit{om.} Z  $\cdot$ sunder] one R \textbf{29} hinderz] Hinder R ÷inderz Fr56  $\cdot$ stach] satz Fr21  $\cdot$ Kailetes] kayletes O L Q Fr56 kaylites R gailetes Z kayletet Fr21 \textbf{30} Brabant] prabant M Braband R pravant Z \newline
\end{minipage}
\hspace{0.5cm}
\begin{minipage}[t]{0.5\linewidth}
\small
\begin{center}*T (U)
\end{center}
\begin{tabular}{rl}
 & der enker kam doch vor an in.\\ 
 & dô stach \textit{in} hinder\textit{z} ors dort hin\\ 
 & der werde künec von Zazamanc\\ 
 & und vienc in. dâ \textbf{wart} grôz gedranc,\\ 
5 & hôhe vurch sleht \textbf{getemmet},\\ 
 & mit swerten vil \textbf{gekemmet}.\\ 
 & d\textit{â} wart verswendet der walt\\ 
 & und manec ritter abe gevalt.\\ 
 & si wunden sich, sus \textbf{hôrt} ich sagen.\\ 
10 & \textbf{hinden an dem orte} h\textit{i}elte\textit{n} zagen.\\ 
 & \begin{large}D\end{large}er strît was wol sô nâhen,\\ 
 & daz gar die vrouwen sâhen,\\ 
 & wer dâ \textbf{mit} prîse solte sîn.\\ 
 & der minne gernde Riwalin,\\ 
15 & von \textit{des} \textbf{sper\textit{n}} snît ein niuwe l\textit{eis}.\\ 
 & daz was der künec von Lohen\textit{eis}.\\ 
 & sîne hurte gâben \textbf{krachenden} schal.\\ 
 & Morolt in einen ritter stal.\\ 
 & ûz dem satele \textbf{huob er in}\\ 
20 & \textbf{vür sich und vuort in hin}.\\ 
 & \textbf{der hiez} Kylliriakac.\\ 
 & von dem hete der künec Lac\\ 
 & dâ vor entvangen solhen solt,\\ 
 & \textbf{sam} \textbf{der} \textbf{vallen} an der erden holt,\\ 
25 & \textbf{und} hetez \textbf{ouch} dâ \textbf{vor} vil \textbf{wol} getân.\\ 
 & dô luste disen starken man,\\ 
 & daz er in twünge sunder swert.\\ 
 & alsus \textbf{vienc er} den degen wert.\\ 
 & hinder\textit{z} ors stach Kayletes hant\\ 
30 & den herzogen von Brabant.\\ 
\end{tabular}
\scriptsize
\line(1,0){75} \newline
U V W T \newline
\line(1,0){75} \newline
\textbf{9} \textit{Majuskel} T  \textbf{11} \textit{Initiale} U V T  \textbf{14} \textit{Majuskel} T  \textbf{29} \textit{Initiale} W   $\cdot$ \textit{Majuskel} T  \newline
\line(1,0){75} \newline
\textbf{2} dô] er V Vnd W (T)  $\cdot$ in hinderz] er hinder U \textbf{3} Zazamanc] zazamang V W \textbf{4} dâ] do V W \textbf{5} Hohe [*]: fúrche sleht getenget V  $\cdot$ Hohe von schlegen getemmet W  $\cdot$ hohe wege nider getembet T \textbf{6} gekemmet] [*]: geclenget V \textbf{7} dâ] Do U W Des V  $\cdot$ verswendet] verswendet gar V (W) \textbf{8} ritter] dægen T \textbf{9} si wunden sich] Der wunden siech V \textbf{10} hinden] \textit{om.} T  $\cdot$ hielten] helte U hielten die T \textbf{12} sâhen] sagen T \textbf{13} dâ] do V W \textbf{14} [*]: do kam der gernde Ruwalin V  $\cdot$ minne] minnen W \textbf{15} des spern] spers U des [sper*]: spern V  $\cdot$ snît] snitte V (T) neig W  $\cdot$ leis] liez U waleiß W \textbf{16} von] \textit{om.} W  $\cdot$ Loheneis] hoheniez U [*oholeis]: Loholeis V loheueiß W \textbf{17} krachenden] kraches V W crachens T \textbf{18} Morolt] Morholt W \textbf{19} huob er in] er in fúr sich nam W \textbf{20} Das was me dan genuͦg getan W  $\cdot$ hin] mit im hin T \textbf{21} Kylliriakac] Kylliriacac U Kẏlliriakag V kilriatag W \textbf{22} hete] herren W  $\cdot$ Lac] Lag V (W) \textbf{23} vor] vuͦr U vor hetten W \textbf{24} Der den val an der erden holt W  $\cdot$ an der] zer T \textbf{25} dâ vor vil wol] do vil guͦt W da gvͦt T \textbf{26} luste] loste W \textbf{27} \textit{nach 73.27:} An schwert vnd an sper W   $\cdot$ sunder swert] seiner ger W \textbf{28} \textit{nach 73.28:} Wann ers mit fleisse het begert W   $\cdot$ vienc er den] viengin der T \textbf{29} hinderz] hinder U  $\cdot$ Kayletes] kalittes U Kaẏletes V gayletes W \textbf{30} Brabant] braband W \newline
\end{minipage}
\end{table}
\end{document}
