\documentclass[8pt,a4paper,notitlepage]{article}
\usepackage{fullpage}
\usepackage{ulem}
\usepackage{xltxtra}
\usepackage{datetime}
\renewcommand{\dateseparator}{.}
\dmyyyydate
\usepackage{fancyhdr}
\usepackage{ifthen}
\pagestyle{fancy}
\fancyhf{}
\renewcommand{\headrulewidth}{0pt}
\fancyfoot[L]{\ifthenelse{\value{page}=1}{\today, \currenttime{} Uhr}{}}
\begin{document}
\begin{table}[ht]
\begin{minipage}[t]{0.5\linewidth}
\small
\begin{center}*D
\end{center}
\begin{tabular}{rl}
\textbf{162} & \begin{large}A\end{large}lsô sprach der tumbe man:\\ 
 & "mîner muoter volc niht bûwen kan,\\ 
 & \textbf{jâ}\textbf{ne} wehset niht sô lanc ir sât,\\ 
 & swaz si \textbf{ir} in dem walde hât.\\ 
5 & grôz regen si selten dâ verbirt."\\ 
 & Gurnemanz de Graharz hiez der wirt\\ 
 & ûf \textbf{dirre} burc, dar zuo er reit.\\ 
 & dâ vor stuont ein linde breit\\ 
 & ûf einen grüenem anger.\\ 
10 & \textbf{der} was breiter noch langer\\ 
 & niht wan ze rehter mâze.\\ 
 & daz ors unt \textbf{ouch} diu strâze\\ 
 & in truogen, dâ er sitzen vant,\\ 
 & des \textbf{was diu burc} unt \textbf{ouch} daz lant.\\ 
15 & Ein grôziu müede in des \textbf{betwanc},\\ 
 & daz er den schilt unrehte swanc\\ 
 & ze verre hinder oder vür,\\ 
 & \textbf{êt} ninder nâch der \textbf{site} kür,\\ 
 & die man dâ gein prîse maz.\\ 
20 & Gurnamanz, \textbf{der vürste}, al eine saz.\\ 
 & ouch gap der linden tolde\\ 
 & ir schaten, als si solde,\\ 
 & dem houptman der wâren zuht.\\ 
 & des site was \textbf{vor} valsche ein vluht.\\ 
25 & \textbf{der} enpfienc den gast, daz was sîn reht.\\ 
 & bî im was ritter noch kneht.\\ 
 & \textbf{Sus antwurte} im dô Parzival\\ 
 & ûz tumben witzen sunder twâl:\\ 
 & "mich bat mîn muoter nemen rât\\ 
30 & ze dem, der grâwe locke hât.\\ 
\end{tabular}
\scriptsize
\line(1,0){75} \newline
D \newline
\line(1,0){75} \newline
\textbf{1} \textit{Initiale} D  \textbf{15} \textit{Majuskel} D  \textbf{27} \textit{Majuskel} D  \newline
\line(1,0){75} \newline
\newline
\end{minipage}
\hspace{0.5cm}
\begin{minipage}[t]{0.5\linewidth}
\small
\begin{center}*m
\end{center}
\begin{tabular}{rl}
 & alsô sprach der tumbe man:\\ 
 & "mîner muoter volc niht bûwen kan,\\ 
 & \textbf{jâ} \textbf{en}wehset niht sô lanc ir sât,\\ 
 & waz si in dem walde hât.\\ 
5 & grôz regen si selten d\textit{â} verbirt."\\ 
 & Gu\textit{r}nemanz de Graharz hiez der wirt\\ 
 & ûf \textbf{dirre} burc, dâ zuo er reit.\\ 
 & dâ vor stuont ein linde breit\\ 
 & ûf einem grüenen anger.\\ 
10 & \textbf{der} was breiter noch langer\\ 
 & niht wanne ze rehter mâze.\\ 
 & daz ros und \textbf{ouch} diu strâze\\ 
 & in tru\textit{og}en, d\textit{â} er sitzen vant,\\ 
 & des \textbf{diu burc was} und daz lant.\\ 
15 & ein grôziu müede in des \textbf{betwanc},\\ 
 & daz er den schilt unrehte swanc\\ 
 & ze verre hinder oder vür,\\ 
 & \textbf{ouch} niendert nâch der \textbf{mâze} kür,\\ 
 & die man dô gegen prîse maz.\\ 
20 & Gurnemanz, \textbf{der vürste}, aleine saz.\\ 
 & ouch gap der linden tolde\\ 
 & ir schaten, als si solde,\\ 
 & dem houbetman der wâren zuht.\\ 
 & des site was \textbf{vor} valsche ein vluht.\\ 
25 & \textbf{der} enpfienc den gast, daz was sîn reht.\\ 
 & bî ime was \textbf{der} ritter noch \textbf{der} kneht.\\ 
 & \textbf{sus sprach zuo} ime dô Parcifal\\ 
 & \textit{û}z tumben witzen sunder twâl:\\ 
 & "mich bat mîn muoter nemen rât\\ 
30 & zuo dem, der grâwe locke hât.\\ 
\end{tabular}
\scriptsize
\line(1,0){75} \newline
m n o \newline
\line(1,0){75} \newline
\textbf{7} \textit{Illustration mit Überschrift:} Also parcipfal (parcifal o  ) vff sime rosz zuͯ der búrge reit vnd ein schone linde vor der búrge stunt do vnder er abe sasz vnd sin rosz dar an bant n (o)   $\cdot$ \textit{Initiale} n o  \newline
\line(1,0){75} \newline
\textbf{3} niht sô lanc] so lang nit n o \textbf{5} dâ] do m n o \textbf{6} Gurnemanz] Gusnemans m n o  $\cdot$ de] die o  $\cdot$ Graharz] grahars n o \textbf{9} grüenen] gruͯnem n \textbf{13} truogen] truͯwen m truͯngen o  $\cdot$ dâ] do m n o \textbf{14} und] vnd ouch n \textbf{15} müede] miete n  $\cdot$ des] das o  $\cdot$ betwanc] twang n o \textbf{18} niendert] meinde er n o  $\cdot$ mâze] mossen n o \textbf{20} Gurnemanz] Gurnemancz m Gurnemantz n Gurnemans o \textbf{21} tolde] talde o \textbf{24} vor] fúr n \textbf{25} der] Die o  $\cdot$ den] der o  $\cdot$ sîn] \textit{om.} n o \textbf{26} was der] \textit{om.} n was o \textbf{28} ûz] Es m \newline
\end{minipage}
\end{table}
\newpage
\begin{table}[ht]
\begin{minipage}[t]{0.5\linewidth}
\small
\begin{center}*G
\end{center}
\begin{tabular}{rl}
 & alsô sprach der tumbe man:\\ 
 & "mîner muoter volc niht bûwen kan,\\ 
 & \textbf{ez} wehset niht sô \textit{lanc} ir sât,\\ 
 & swaz si\textbf{r} in dem walde hât.\\ 
5 & grôz regen si selten dâ verbirt."\\ 
 & Gurnomanz de Graharz hiez der wirt\\ 
 & ûf \textbf{der} burc, dâ zuo er reit.\\ 
 & dâ vor stuont ein linde breit\\ 
 & \begin{large}Û\end{large}f einem grüenen anger.\\ 
10 & \textbf{\textit{e}r}\textit{\textbf{n}} was breiter noch langer\\ 
 & niwan ze rehter mâze.\\ 
 & daz ors unde diu strâze\\ 
 & in truogen, dâ er sitzen vant,\\ 
 & des \textbf{was diu burc} unt \textbf{ouch} daz lant.\\ 
15 & ein grôziu müede in des \textbf{betwanc},\\ 
 & daz er den schilt unrehte \textit{s}wanc\\ 
 & ze verre hinder oder vür,\\ 
 & \textbf{êt} ninder nâch der \textbf{site} kür,\\ 
 & die man dâ gein prîse maz.\\ 
20 & Gurnomanz, \textbf{der vürste}, al eine saz.\\ 
 & ouch gap der linden tolde\\ 
 & ir schate, als si solde,\\ 
 & dem houbetman der wâren zuht.\\ 
 & des site was \textbf{vor} valsche ein vluht.\\ 
25 & \textbf{er} enpfie den gast, daz was sîn reht.\\ 
 & bî im was rîter noch \textbf{der} kneht.\\ 
 & \textbf{des antwurt} im dô Parzival\\ 
 & ûz tumben witzen sunder twâl:\\ 
 & "mich bat mîn muoter nemen rât\\ 
30 & ze dem, der grâwe locke hât.\\ 
\end{tabular}
\scriptsize
\line(1,0){75} \newline
G I O L M Q R Z Fr17 Fr47 \newline
\line(1,0){75} \newline
\textbf{1} \textit{Initiale} O L M Q R Z Fr17 Fr47  \textbf{7} \textit{Überschrift:} Aventiwer wie Parzifal hinz Gurnamanz bequam der in leit riters amt des er sich seit nicht schamt I   $\cdot$ \textit{Initiale} I  \textbf{9} \textit{Initiale} G  \textbf{29} \textit{Initiale} I  \newline
\line(1,0){75} \newline
\textbf{1} alsô] ÷lso O Fr47  $\cdot$ tumbe] [m*]: iuͯnge L \textbf{2} niht] so wol nih I nit also R in Z \textbf{3} ez] ezn I (O) (M) Fr17 So Q Ja Z  $\cdot$ wehset] washet I (Q) (Fr17) [ew]: enwehset  Z  $\cdot$ lanc] hohe G \textbf{4} swaz] Waz L M (Q) (R) \textbf{5} grôz] Grozze Z  $\cdot$ dâ] do Q \textit{om.} Z \textbf{6} Gurnomanz] Gurneman: I Gvrnemanz O Gvrnomantz L (Q) Gurnemans M Gurnamanz R Fr17 Gvrnemantz Z (Fr47)  $\cdot$ de] der O (M) Q von Fr47  $\cdot$ Graharz] [Gaharz]: Graharz O grahars L M (Q) grahartz Z  $\cdot$ hiez] heyst Q \textbf{7} der] dirre L  $\cdot$ burc] brugg Fr47 \textbf{9} Ûf] ef Fr17  $\cdot$ grüenen] Gruͤnem I (Q) grunē M (Fr47) \textbf{10} ern] der G Er O R (Fr17) Fr47 Es Q \textbf{11} niwan] Ywan Q \textbf{12} unde] vnd ovch Z \textbf{13} dâ] do Q  $\cdot$ sitzen] sitze O siczent R (Fr47) \textbf{14} Dez die brugg waz vnd daz lant Fr47  $\cdot$ des] Daz L Des da Z  $\cdot$ ouch] \textit{om.} I O L M Q R \textbf{15} ein] \textit{om.} L Fr47  $\cdot$ des] daz L \textbf{16} swanc] twanch G \textbf{17} hinder] hin hinder I Q (Fr47) hinden O  $\cdot$ oder] vnd L  $\cdot$ vür] her vur I (Q) \textbf{18} êt ninder] ern tet ninder I Ez nirgen M Doch nit R et [nie*]: nieder Fr17  $\cdot$ site] seten M \textbf{19} dâ] \textit{om.} I do L Q R Fr17 Fr47  $\cdot$ gein] Gein dem I noch M  $\cdot$ maz] [wag]: mas R \textbf{20} Gurnomanz] Gurnemanz I (O) (L) Gurnemans M Gurnomantz Q Gurnamanacz R Gvrnemantz Z (Fr47) Gvrnamanz Fr17  $\cdot$ al] \textit{om.} O M Z  $\cdot$ saz] Gesaz I \textbf{21} gap] \textit{om.} Z \textbf{22} schate] schaten O (L) (Q) (R) Z (Fr47) stat M  $\cdot$ solde] wolde I Fr17 \textbf{24} des] Dy M  $\cdot$ valsche ein] valschem O \textbf{25} er] [en]: ern I \textbf{26} noch der] noch I L M Q vnde R \textbf{27} des] Der L  $\cdot$ im dô] do ým L yme da M (Z)  $\cdot$ Parzival] [parzifal]: Parzifal I parcifal O (L) Z partzifal Q parczifal R Parhcifal Fr47 \textbf{28} ûz] Mit Q \textbf{30} ze] Von Fr47  $\cdot$ locke] locher Q \newline
\end{minipage}
\hspace{0.5cm}
\begin{minipage}[t]{0.5\linewidth}
\small
\begin{center}*T
\end{center}
\begin{tabular}{rl}
 & alsô sprach der tumbe man:\\ 
 & "mîner muoter volc niht bûwen kan,\\ 
 & \textbf{ez} \textbf{en}wehset niht sô lanc ir sât,\\ 
 & swaz si\textbf{r} in dem walde hât.\\ 
5 & grôz regen si selten dâ verbirt."\\ 
 & Gurnemanz de Greharz hiez der wirt\\ 
 & ûf \textbf{der} burc, dâ zuo er reit.\\ 
 & dâ vor stuont ein linde breit\\ 
 & ûf einem grüenen anger.\\ 
10 & \textbf{daz} \textbf{en}was breiter noch langer\\ 
 & niuwan ze rehter mâze.\\ 
 & daz ors unde diu strâze\\ 
 & in truogen, dâ er sitzen vant,\\ 
 & des \textbf{was di\textit{u} burc} unde daz lant.\\ 
15 & ein grôze müede in des \textbf{twanc},\\ 
 & daz er den schilt unrehte swanc\\ 
 & ze verre hinder oder vür,\\ 
 & \textbf{eht} niender nâch der \textbf{site} kür,\\ 
 & die man dô gegen prîse maz.\\ 
20 & Gurnemanz aleine saz.\\ 
 & Ouch gap der linden tolde\\ 
 & ir schate, alse si solde,\\ 
 & dem houbetman der wâren zuht.\\ 
 & des site was \textbf{gegen} valsche ein vluht.\\ 
25 & \textbf{er} enpfienc den gast, daz was sîn reht.\\ 
 & bî im was \textbf{weder} rîter noch kneht.\\ 
 & \textbf{\begin{large}D\end{large}es antwurt}im dô Parcifal\\ 
 & ûz tumben witzen sunder twâl:\\ 
 & "mich bat mîn muoter nemen rât\\ 
30 & zuo dem, der grâwe locke hât.\\ 
\end{tabular}
\scriptsize
\line(1,0){75} \newline
T U V W \newline
\line(1,0){75} \newline
\textbf{21} \textit{Majuskel} T  \textbf{27} \textit{Initiale} T U V  \newline
\line(1,0){75} \newline
\textbf{2} niht] so nit W \textbf{3} enwehset] in was sit U wachset W \textbf{4} swaz] Waz U (W)  $\cdot$ walde] [*]: walde V velde W \textbf{5} selten dâ] selten doch U selten [*]: do V do nit W \textbf{6} Gurnemanz] Gvnnemaz T Guͦrnemans U Gurnemantz W  $\cdot$ de] von V  $\cdot$ Greharz] grahars W \textbf{7} der] [*]: der V  $\cdot$ dâ zuo] [*]: do zvͦ V zuͦ dem W \textbf{10} Nit zuo brait noch vil zuo langer W  $\cdot$ daz] [D*]: Der V \textbf{11} niuwan] Nit dan U  $\cdot$ mâze] messe W \textbf{13} dâ] do U W [*]: do V \textbf{14} was diu burc] was [de]: die bvrc T die burg waz W \textbf{18} niender nâch] nider nach U iender in W \textbf{20} Gurnemanz] Gurnemans U Gurnemantz W  $\cdot$ aleine] der fúrste alleine W \textbf{21} Ouch] [ovch]: Oovch T  $\cdot$ linden] linde U \textbf{22} schate] schetten W  $\cdot$ si] er V W \textbf{24} gegen] [g*]: vor V von W \textbf{26} weder] \textit{om.} W \textbf{27} Des] Svz V  $\cdot$ Parcifal] parzifal V partzifal W \newline
\end{minipage}
\end{table}
\end{document}
