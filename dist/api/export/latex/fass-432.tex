\documentclass[8pt,a4paper,notitlepage]{article}
\usepackage{fullpage}
\usepackage{ulem}
\usepackage{xltxtra}
\usepackage{datetime}
\renewcommand{\dateseparator}{.}
\dmyyyydate
\usepackage{fancyhdr}
\usepackage{ifthen}
\pagestyle{fancy}
\fancyhf{}
\renewcommand{\headrulewidth}{0pt}
\fancyfoot[L]{\ifthenelse{\value{page}=1}{\today, \currenttime{} Uhr}{}}
\begin{document}
\begin{table}[ht]
\begin{minipage}[t]{0.5\linewidth}
\small
\begin{center}*D
\end{center}
\begin{tabular}{rl}
\textbf{432} & \begin{large}Z\end{large}e vlüste oder ze \textbf{gewinne}."\\ 
 & diu edele küneginne\\ 
 & kuste \textbf{den} Gawans munt.\\ 
 & der wart an vreuden ungesunt,\\ 
5 & daz er sô gâhes von ir reit.\\ 
 & ich wæne, ez was in beiden leit.\\ 
 & Sîne knappen heten sich bedâht,\\ 
 & daz sîniu ors wâren brâht\\ 
 & ûf den hof vür den palas,\\ 
10 & \textbf{al} dâ der linden schate was.\\ 
 & ouch wâren dem lantgrâven komen\\ 
 & sîne gesellen, \textbf{sus hân ichz} vernomen.\\ 
 & der reit mit im ûz vür die stat.\\ 
 & Gawan in zühteclîchen bat,\\ 
15 & daz er sich arbeite\\ 
 & unt \textbf{sîn gezoc im} leite\\ 
 & ze Bearosche. "\textbf{daz} ist Scherules,\\ 
 & den sulen si \textbf{selbe} bitten des\\ 
 & geleites ze Dianazdrun.\\ 
20 & dâ \textbf{wont} etslîch Bertun,\\ 
 & der si bringet an den hêrren mîn\\ 
 & oder an \textbf{Ginovern}, die künegîn."\\ 
 & Daz lobt \textbf{im} Kyngrimursel.\\ 
 & urloup nam der degen snel.\\ 
25 & Gringuljete wart gewâpent sân,\\ 
 & daz ors, unt mîn hêr Gawan.\\ 
 & \textbf{er} kuste sîne mâge, diu kindelîn,\\ 
 & unt ouch die werden knappen sîn.\\ 
 & nâch dem Grâle im sicherheit gebôt.\\ 
30 & er reit al eine \textbf{gein} wunders nôt.\\ 
\end{tabular}
\scriptsize
\line(1,0){75} \newline
D \newline
\line(1,0){75} \newline
\textbf{1} \textit{Initiale} D  \textbf{7} \textit{Majuskel} D  \textbf{23} \textit{Majuskel} D  \newline
\line(1,0){75} \newline
\textbf{17} Bearosche] Bearosce D  $\cdot$ Scherules] Scervles D \textbf{25} Gringuljete] Gringvliet D \newline
\end{minipage}
\hspace{0.5cm}
\begin{minipage}[t]{0.5\linewidth}
\small
\begin{center}*m
\end{center}
\begin{tabular}{rl}
 & ze verlust oder ze \textbf{gewinne}."\\ 
 & diu edele küniginne\\ 
 & kuste \textbf{den} Gawanes munt.\\ 
 & der wart an vröuden ungesunt,\\ 
5 & daz er sô gâhes von ir reit.\\ 
 & ich wæne, ez was in bei\textit{den lei}t.\\ 
 & sîne knappen heten sich bedâht,\\ 
 & daz sîniu ros wâren brâht\\ 
 & ûf den hof vür den palas,\\ 
10 & \textbf{al}dâ der linden schate was.\\ 
 & ouch wâren dem lantgrâven komen\\ 
 & sîne gesellen, \textbf{sus hân ichz} vernomen.\\ 
 & der reit mit im ûz vür die stat.\\ 
 & Gawan in zühteclîchen bat,\\ 
15 & daz er sich arbeite\\ 
 & und \textbf{sînen gezoc ime} leite\\ 
 & ze Bear\textit{o}sche. "\textbf{dâ} ist Scherules,\\ 
 & den sullen si \textbf{selbe} bitten des\\ 
 & geleites ze Di\textit{an}azdru\textit{n}.\\ 
20 & d\textit{â} \textbf{wont} etslîch Britu\textit{n},\\ 
 & der si bringet an den hêrren mîn\\ 
 & oder an \textbf{mîne vrouwen}, die künigîn."\\ 
 & daz lobete Kingri\textit{m}ursel.\\ 
 & urloup nam der degen snel.\\ 
25 & Gringulet wart gewâpent sân,\\ 
 & daz ros, und mîn hêrre Gawan.\\ 
 & \textbf{er} kuste sîne mâge, diu kindelîn,\\ 
 & und ouch die werden knappen sîn.\\ 
 & nâch dem Grâle ime sicherheit gebôt.\\ 
30 & er reit aleine \textbf{gegen} wunders nôt.\\ 
\end{tabular}
\scriptsize
\line(1,0){75} \newline
m n o \newline
\line(1,0){75} \newline
\newline
\line(1,0){75} \newline
\textbf{3} kuste] Kost n o  $\cdot$ Gawanes] gawes o \textbf{6} was] were n  $\cdot$ beiden leit] beit m \textbf{8} wâren] worden n o \textbf{10} aldâ] Do n o  $\cdot$ linden] linde o \textbf{11} dem lantgrâven] den groffen n (o) \textbf{12} sus] \textit{om.} n o  $\cdot$ ichz] ich n uch o \textbf{14} in] [im]: in m \textbf{16} gezoc] gezúge n gezug o \textbf{17} Bearosche] bearsce m bearosc n bearost o  $\cdot$ dâ] do n o  $\cdot$ Scherules] scerules m n scernles o \textbf{18} selbe] selbes n \textbf{19} Dianazdrun] [dimz]: dimazdrum m drunas drun n druͯnasz drumm o \textbf{20} dâ] Do m n o  $\cdot$ Britun] brittum m britumm o \textbf{22} an] \textit{om.} n  $\cdot$ vrouwen] frouwe n (o) \textbf{23} lobete] lopt jme n lopt in o  $\cdot$ Kingrimursel] kingrinursel m kingrumúrsel n konigrumursel o \textbf{25} Gringulet] Gingrulet n \textbf{27} kuste] kust n kost o \textbf{28} werden] werde m werder o \newline
\end{minipage}
\end{table}
\newpage
\begin{table}[ht]
\begin{minipage}[t]{0.5\linewidth}
\small
\begin{center}*G
\end{center}
\begin{tabular}{rl}
 & \begin{large}Z\end{large}e vlüste oder ze \textbf{winne}."\\ 
 & diu edele küniginne\\ 
 & kuste \textbf{den} Gawans munt.\\ 
 & der wart an vröuden ungesunt,\\ 
5 & daz er sô gâhes von ir reit.\\ 
 & ich wæne, ez was in beiden leit.\\ 
 & sîne knappen heten sich bedâht,\\ 
 & daz sîniu ors wâren brâht\\ 
 & ûf den hof vür den palas,\\ 
10 & \textbf{al} dâ der linden schate was.\\ 
 & ouch wâren dem lantgrâven komen\\ 
 & sîne gesellen, \textbf{sus hân ich} vernomen.\\ 
 & der reit mit im ûz vür die stat.\\ 
 & Gawan in zühticlîchen bat,\\ 
15 & daz er sich arbeite\\ 
 & unde \textbf{sîn gezoc im} leite\\ 
 & ze Bearotsche. "\textbf{dâ} ist Tscherules,\\ 
 & den sulen si \textbf{bêde} biten des\\ 
 & geleites ze Dianazdrun.\\ 
20 & dâ \textbf{wonet} etslîch Britun,\\ 
 & der si bringet an den hêrren mîn\\ 
 & oder an \textbf{Schinoveren}, die künigîn."\\ 
 & daz lobt \textbf{im} Kingrimursel.\\ 
 & urloup nam der degen snel.\\ 
25 & Gringuliet wart gewâpent sân,\\ 
 & daz ors, unde mîn hêr Gawan.\\ 
 & \textbf{er} kuste sîne mâge, diu kindelîn,\\ 
 & unde ouch die \textit{we}r\textit{d}en knappen sîn.\\ 
 & nâch dem Grâle im sicherheit gebôt.\\ 
30 & er reit al eine \textbf{gein} wunders nôt.\\ 
\end{tabular}
\scriptsize
\line(1,0){75} \newline
G I O L M Q R Z \newline
\line(1,0){75} \newline
\textbf{1} \textit{Initiale} G I O L Z  \textbf{15} \textit{Initiale} I  \newline
\line(1,0){75} \newline
\textbf{1} ze] ÷e O  $\cdot$ oder] olde G vnd I  $\cdot$ winne] gewinne O (L) (M) Q R Z \textbf{3} den] \textit{om.} I L an Q  $\cdot$ Gawans] gewaines R \textbf{4} der] Do R  $\cdot$ wart] was M \textbf{5} gâhes] gahas Q geches R  $\cdot$ von] vor Q  $\cdot$ ir] in R \textbf{6} was] were O L (R) \textbf{8} Sine ros hettencz dar bracht R \textbf{9} den] das M \textbf{10} al] \textit{om.} O L M Q R  $\cdot$ dâ] [Da]: Das M Do Q  $\cdot$ schate] schatten R \textbf{12} sus hân ich] han ich I als ichz han O (L) (Q) (R) alse ich han M sus han ichz Z \textbf{13} mit im ûz] mit im L Q (R) usz mit ym M  $\cdot$ vür die] der O vor der M \textbf{14} Gawan] Gawain R  $\cdot$ zühticlîchen] s: zuchtenklichen R \textbf{15} arbeite] [erwete]: erweite Q \textbf{16} sîn gezoc im] sinen gezoc in I yme sin geczoit M sin gelopt im R \textbf{17} \textit{Vers 432.17 fehlt} M   $\cdot$ Bearotsche] bearrosce I Bearotsch L bearosche Q [barosche]: bearosche R  $\cdot$ dâ] \textit{om.} L do Q R  $\cdot$ Tscherules] Scurles I Tschervles O L schureles R \textbf{18} bêde] selbe M Z von mir Q (R) \textbf{19} ze Dianazdrun] zedianazdrun G zediazaztrun I ze dianazdrv̂n O zcu dir Nazdrun M zu dianazún Q ze dyanazrun R zv dyanazdrvn Z \textbf{20} dâ] Do L Q  $\cdot$ wonet] [want]: waͦnt O  $\cdot$ Britun] pritun I britv̂n O Brittvn L brithun M brittúm Q Briton R \textbf{21} bringet] bringen L \textbf{22} oder an] Odan M  $\cdot$ Schinoveren] tschinoveren G ginofern I Gynover O Gynovern L Z ginover M gynoveren Q Gynouern R  $\cdot$ die] div O \textbf{23} lobt] lop Q  $\cdot$ im] in O  $\cdot$ Kingrimursel] Kyngrimvrsel O kingrimuͯrsel M kyngrim vrsel Q kᵫngrinmursel R \textbf{24} urloup] Arlap M \textbf{25} Gringuliet] kingriguliet I Gringvliet O L Z Girnguliet M Kymguliet Q Ringulett R \textbf{26} Gawan] Gawain R \textbf{27} sîne mâge] sein [*]: magde Q \textbf{28} werden] starchen G  $\cdot$ knappen] [chappen]: chnappen I \textbf{30} gein] nach O L Q durch R \newline
\end{minipage}
\hspace{0.5cm}
\begin{minipage}[t]{0.5\linewidth}
\small
\begin{center}*T
\end{center}
\begin{tabular}{rl}
 & zuo verlüste oder zuo \textbf{gewinne}."\\ 
 & Diu edele küneginne\\ 
 & kuste \textbf{hêrn} Gawans munt.\\ 
 & der wart an vröuden ungesunt,\\ 
5 & daz er sô gâhes von ir reit.\\ 
 & ich wæne, ez \textit{was} in beiden leit.\\ 
 & sîne knappen heten sich bedâht,\\ 
 & daz sîniu \textit{ors} wâren brâht\\ 
 & ûf den hof vür den palas,\\ 
10 & dâ der linden schate was.\\ 
 & \begin{large}O\end{large}uch wâren dem lantgrâven komen\\ 
 & sîne gesellen, \textbf{alsich hân} vernomen.\\ 
 & der reit mit im ûz vür die stat.\\ 
 & Gawan in zühteclîche bat,\\ 
15 & daz er sich arbeite\\ 
 & unde \textbf{im sînen gezoc} leite\\ 
 & ze Bearosche. "\textbf{dâ} ist Tscherules,\\ 
 & den suln si \textbf{von mir} biten des\\ 
 & geleites ze Dyanarsun.\\ 
20 & dâ \textbf{ist} etslîch Britun,\\ 
 & der si bringet an den hêrren mîn\\ 
 & oder an \textbf{Gynovern}, die künegîn."\\ 
 & daz lobet\textbf{im} Kyngrimursel.\\ 
 & urloup nam der degen snel.\\ 
25 & Krynguliet wart gewâpent sân,\\ 
 & daz ors, unde mîn hêr Gawan.\\ 
 & \textbf{der} kuste sîne mâge, di\textit{u} kindelîn,\\ 
 & unde ouch die werden knappen sîn.\\ 
 & nâch dem Grâle im sicherheit gebôt.\\ 
30 & er reit aleine \textbf{nâch} wunders nôt.\\ 
\end{tabular}
\scriptsize
\line(1,0){75} \newline
T U V W \newline
\line(1,0){75} \newline
\textbf{1} \textit{Initiale} V  \textbf{2} \textit{Majuskel} T  \textbf{7} \textit{Initiale} W  \textbf{11} \textit{Initiale} T U  \newline
\line(1,0){75} \newline
\textbf{1} gewinne] winne V \textbf{3} hêrn] den U W do V  $\cdot$ Gawans] gawanes V \textbf{6} was] \textit{om.} T \textbf{7} sîne] DIe W \textbf{8} sîniu] sie nie U  $\cdot$ ors] \textit{om.} T \textbf{10} dâ] Do U V W \textbf{12} alsich] als iz U als ich ez V (W) \textbf{14} zühteclîche] zuͦcltecliche U \textbf{15} arbeite] arbaitet W \textbf{16} im sînen gezoc] sinen gezoc U sin gezog im V seinen gezeúg W  $\cdot$ leite] laitet W \textbf{17} Bearosche] Bearosce T Bearotsche U  $\cdot$ dâ] do V W  $\cdot$ Tscherules] Tscervles T scherulez V \textbf{18} biten] bieten U \textbf{19} Dyanarsun] dyanarsuͦn U dyamazrun W \textbf{20} dâ ist] Do wonte U Do wont V W  $\cdot$ Britun] Brîtvn T Brituͦn U brittun V \textbf{22} an Gynovern] an ginovern V antschinouern W \textbf{23} lobetim] lobet im V (W)  $\cdot$ Kyngrimursel] kyngrimorsel U kingrimursel V \textbf{25} Krynguliet] Kringuliet U Gringulet V Kringulyet W \textbf{26} daz] des U W \textbf{27} der] Er U W  $\cdot$ diu] die T \textbf{29} nâch dem] Wann der W \textbf{30} nâch] durch W \newline
\end{minipage}
\end{table}
\end{document}
