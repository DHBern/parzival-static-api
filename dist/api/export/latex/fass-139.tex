\documentclass[8pt,a4paper,notitlepage]{article}
\usepackage{fullpage}
\usepackage{ulem}
\usepackage{xltxtra}
\usepackage{datetime}
\renewcommand{\dateseparator}{.}
\dmyyyydate
\usepackage{fancyhdr}
\usepackage{ifthen}
\pagestyle{fancy}
\fancyhf{}
\renewcommand{\headrulewidth}{0pt}
\fancyfoot[L]{\ifthenelse{\value{page}=1}{\today, \currenttime{} Uhr}{}}
\begin{document}
\begin{table}[ht]
\begin{minipage}[t]{0.5\linewidth}
\small
\begin{center}*D
\end{center}
\begin{tabular}{rl}
\textbf{139} & \multicolumn{1}{l}{ - - - }\\ 
 & \multicolumn{1}{l}{ - - - }\\ 
 & geschach ez mit eime gabylôt?\\ 
 & mich dunket, vrouwe, er \textbf{lige} tôt.\\ 
5 & \textbf{welt} ir \textbf{mir dâ von iht} \textbf{sagen},\\ 
 & \textbf{wer} iu \textbf{den man habe} erslagen?\\ 
 & ob ich in \textbf{mag} errîten,\\ 
 & ich wil gerne mit im strîten."\\ 
 & \begin{large}D\end{large}ô greif der knappe mære\\ 
10 & \textbf{zuo sîme} kochære.\\ 
 & vil scharpfiu gabylôt er vant.\\ 
 & er vuort \textbf{ouch} dannoch beidiu pfant,\\ 
 & diu er von Jeschuten brach\\ 
 & unt ein tumpheit \textbf{dâ} geschach.\\ 
15 & het er gelernet sînes vater site,\\ 
 & \textbf{die} werdeclîche im \textbf{wonten} mite,\\ 
 & diu buckel wære gehurt baz,\\ 
 & dâ diu herzoginne al eine saz,\\ 
 & diu sît vil kumbers durch in leit.\\ 
20 & mêr danne ein ganzez jâr si meit\\ 
 & gruoz von ir mannes lîbe.\\ 
 & unrehte geschach dem wîbe.\\ 
 & Hœrt \textbf{ouch} von Sigunen sagen.\\ 
 & diu kunde ir leit mit jâmer \textbf{klagen}.\\ 
25 & si sprach zem knappen: "dû hâst tugent.\\ 
 & geêret sî dîn \textbf{süeziu} jugent\\ 
 & unt dîn antlütze minneclîch.\\ 
 & \textbf{deiswâr}, dû wirst noch sælden rîch.\\ 
 & disen ritter meit daz gabylôt.\\ 
30 & er lac \textbf{ze tjustieren} tôt."\\ 
\end{tabular}
\scriptsize
\line(1,0){75} \newline
D \newline
\line(1,0){75} \newline
\textbf{9} \textit{Initiale} D  \textbf{23} \textit{Majuskel} D  \newline
\line(1,0){75} \newline
\textbf{1} \textit{Die Verse 139.1-2 fehlen} D  \newline
\end{minipage}
\hspace{0.5cm}
\begin{minipage}[t]{0.5\linewidth}
\small
\begin{center}*m
\end{center}
\begin{tabular}{rl}
 & der knappe unverdrozzen\\ 
 & sprach: "wer hât in erschozzen?\\ 
 & geschach ez \textit{m}i\textit{t} einem gabilôt?\\ 
 & mich dunket, vrouwe, er \textbf{lige} tôt.\\ 
5 & \textbf{wellet} ir \textbf{mir dâ von iht} \textbf{sagen},\\ 
 & \textbf{wer} iu \textbf{den ritter habe} erslagen?\\ 
 & ob ich in \textbf{müg\textit{e}} \textit{e}rrîten,\\ 
 & ich wil gerne mit ime strîten."\\ 
 & dô greif der knappe mære\\ 
10 & \textbf{an sîne\textit{n}} kochære.\\ 
 & vil scharfiu gabilôt er vant.\\ 
 & er vuorte \textbf{ouch} dannoch beidiu pfant,\\ 
 & diu er von J\textit{e}schuten brach\\ 
 & und \textbf{ime} ein tumpheit \textbf{d\textit{â}} geschach.\\ 
15 & hete er gelernet sînes vaters site,\\ 
 & \textbf{die} werdeclîche ime \textbf{wonten} mite,\\ 
 & diu buckel wære geh\textit{u}rtet baz,\\ 
 & dâ diu herzoginne aleine saz,\\ 
 & diu sît vil kumbers durch in leit.\\ 
20 & mê denne ein ganzez jâr si meit\\ 
 & gruoz von ir mannes lîbe.\\ 
 & unreht geschach dem wîbe.\\ 
 & \textbf{\begin{large}N\end{large}û} hœret von Sigunen sagen.\\ 
 & diu kunde ir leit mit jâmer \textbf{tragen}.\\ 
25 & si sprach zuom knappen: "dû hâst tugent.\\ 
 & geêret sî dîn jugent\\ 
 & und dîn an\textit{t}litze minneclîch.\\ 
 & \textbf{des vür wâr}, dû wirst noch sæl\textit{d}en rîch.\\ 
 & disen ritter meit daz gabilôt.\\ 
30 & er lac \textbf{ze justieren} tôt."\\ 
\end{tabular}
\scriptsize
\line(1,0){75} \newline
m n o \newline
\line(1,0){75} \newline
\textbf{23} \textit{Initiale} m   $\cdot$ \textit{Capitulumzeichen} n  \newline
\line(1,0){75} \newline
\textbf{2} \textit{Verse 139.2-3 kontrahiert zu:} Sprach es mit einem gabilot o  \textbf{3} mit] in m \textbf{5} iht] \textit{om.} n o \textbf{6} iu] \textit{om.} n \textbf{7} müge errîten] muge erslahen vnd er ritten m \textbf{10} sînen] sinem m  $\cdot$ kochære] kocherren o \textbf{11} vant] do fant n \textbf{12} ouch] \textit{om.} n echt o \textbf{13} diu] Do n o  $\cdot$ Jeschuten] iscutten m ẏscuten n iscuten o \textbf{14} tumpheit] tumphei n  $\cdot$ dâ] do m n o \textbf{15} gelernet] geleret n o \textbf{16} wonten] wonet n (o) \textbf{17} gehurtet] gehertet m o geharte n \textbf{18} dâ] Do n o \textbf{23} Sigunen] siguͯnin o \textbf{24} ir] er o  $\cdot$ tragen] clagen n \textbf{26} jugent] suͯsse jugent n (o) \textbf{27} antlitze] antzlicze m antzlit o \textbf{28} des] Wisse n (o)  $\cdot$ sælden] selben m \textbf{30} ze] in n o \newline
\end{minipage}
\end{table}
\newpage
\begin{table}[ht]
\begin{minipage}[t]{0.5\linewidth}
\small
\begin{center}*G
\end{center}
\begin{tabular}{rl}
 & der knappe unverdrozzen\\ 
 & sprach: "wer hât in erschozzen?\\ 
 & geschach ez mit einem gabilôt?\\ 
 & mich dunkt, vrouwe, er \textbf{sî} tôt.\\ 
5 & \textbf{kunnet} ir \textbf{mir iht dâr von} \textbf{gesagen},\\ 
 & \textbf{wer} iu \textbf{den rîter habe} erslagen?\\ 
 & obe ich in \textbf{muge} errîten,\\ 
 & ich wil gerne mit im strîten."\\ 
 & dô greif der knappe mære\\ 
10 & \textbf{ze sînem} kochære.\\ 
 & vil scharpfiu gabilôt er vant.\\ 
 & er vuorte \textbf{ouch} dannoch beidiu pfant,\\ 
 & diu er von Jeschuten brach\\ 
 & unde \textbf{im} ein tumpheit \textbf{dran} geschach.\\ 
15 & het er gelernt sînes vater site,\\ 
 & \textbf{der} werdiclîche im \textbf{volgte} mite,\\ 
 & diu buckel wære gehurt baz,\\ 
 & dô diu herzogîn al eine saz,\\ 
 & diu sît vil kumbers durch in leit.\\ 
20 & mê danne ein ganzez jâr si meit\\ 
 & gruoz von ir mannes lîbe.\\ 
 & unrehte geschach dem wîbe.\\ 
 & \textbf{nû} hœrt \textbf{ouch} von Sigunen sagen.\\ 
 & diu kunde ir leit mit jâmer \textbf{klagen}.\\ 
25 & si sprach zem knappen: "dû hâst tugent.\\ 
 & geêret sî dîn \textbf{süeziu} jugent\\ 
 & unt dîn antlütze minniclîch.\\ 
 & \textbf{dêswâr}, dû wirst noch sælden rîch.\\ 
 & disen rîter meit daz gabilôt.\\ 
30 & er lac \textbf{an einer tjoste} tôt.\\ 
\end{tabular}
\scriptsize
\line(1,0){75} \newline
G I O L M Q R Z \newline
\line(1,0){75} \newline
\textbf{1} \textit{Initiale} I  \textbf{9} \textit{Initiale} O L Q Z  \textbf{23} \textit{Initiale} I R   $\cdot$ \textit{Capitulumzeichen} L  \newline
\line(1,0){75} \newline
\textbf{1} knappe] chnappe sprach I \textbf{2} sprach] \textit{om.} I  $\cdot$ erschozzen] gescozzin M \textbf{3} geschach] Beschach R  $\cdot$ mit] von I  $\cdot$ gabilôt] [gabilon]: gabilot Z \textbf{4} sî] lige I O (Q) R Z liet M  $\cdot$ tôt] [kranc]: tot M \textbf{5} kunnet] Chvndet O (M) (Z)  $\cdot$ mir iht dâr von] iht mir von im O mir von ým iht L (R) myr icht [vor]: von ym M mir nicht von im Q mir iht von im Z  $\cdot$ gesagen] sagen Q \textbf{6} wer] Der O L M Q R Z  $\cdot$ iu den rîter habe] hat den riter I iv den ritter hat O (L) (M) (Q) (R) (Z) \textbf{7} muge] moht I \textbf{8} wil] wolde O (L) (Q) \textbf{9} dô] ÷o O Da M Z \textbf{11} gabilôt er] er gabilot do I pfheyle er Q \textbf{12} vuorte] fvͤrt O (Q) (R) (Z)  $\cdot$ ouch] \textit{om.} I \textbf{13} diu] Die O L M Q R Z  $\cdot$ von] von fron I  $\cdot$ Jeschuten] ieschuten G (O) ieskuten I Jescuͯten L Jeseuten M iescuten Q (Z) Jscuten R \textbf{14} im] ym da M  $\cdot$ tumpheit] [tumh]: tumpheit G  $\cdot$ dran] da O L R \textit{om.} M Z do Q \textbf{15} gelernt] geleret Q \textbf{16} der werdiclîche] der werdecheit I Div werdicheit O Die wirdeclich L (M) (Z)  $\cdot$ volgte] wonte I M R Z wonten L folget Q \textbf{17} diu] [Die]: Disz L  $\cdot$ wære] were were Q  $\cdot$ gehurt] gehúret R \textbf{18} dô] Da M Z Dy Q \textbf{20} ganzez] \textit{om.} I L  $\cdot$ meit] reit I \textbf{21} gruoz] an gruͤz I \textbf{23} Sigunen] sygvnen O Sýgvnen L sigúnen Q \textbf{24} jâmer] leide O \textbf{25} tugent] ivgende L \textbf{26} sî] sin R  $\cdot$ süeziu] suͯsze R  $\cdot$ jugent] tuͯgende L \textbf{28} dêswâr] Entzwar Q Zwar Z \textbf{29} disen] Duser M (Q)  $\cdot$ meit] meint Q  $\cdot$ daz] der Z \textbf{30} an einer tjoste] von einer Tioste I ze tyostiern O (L) (M) (R) (Z) von tiostiren Q \newline
\end{minipage}
\hspace{0.5cm}
\begin{minipage}[t]{0.5\linewidth}
\small
\begin{center}*T (U)
\end{center}
\begin{tabular}{rl}
 & der knappe, unverdrozzen\\ 
 & sprach \textbf{er}: "wer hât in erschozzen?\\ 
 & geschach ez \textbf{niht} mit eime gabilôt?\\ 
 & mich dunket, vrouwe, er \textbf{lige} tôt.\\ 
5 & \textbf{kunnet} ir \textbf{iht mir von im} \textbf{gesagen},\\ 
 & \textbf{der} iu \textbf{hât den rîter} erslagen?\\ 
 & ob ich in \textbf{muge} errîten,\\ 
 & ich wil gerne mit im strîten."\\ 
 & \begin{large}D\end{large}ô greif der knappe mære\\ 
10 & \textbf{zuo sîme} kochære.\\ 
 & vil scharpfiu gabilôt er vant.\\ 
 & er vuorte dannoch beidiu pfant,\\ 
 & dier von Jeschuten brach\\ 
 & und \textbf{im} ein tumpheit \textbf{dâ} geschach.\\ 
15 & heter gelernet sînes vater site,\\ 
 & \textbf{diu} wirdeclîche im \textbf{wonte} mite,\\ 
 & diu buckel wære gehurtet baz,\\ 
 & dâ diu herzogîn aleine saz,\\ 
 & diu sî\textit{t} vil kumbers durch in leit.\\ 
20 & mê dan ein ganzez jâr si meit\\ 
 & gruoz von ir mannes lîbe.\\ 
 & unreht geschach dem wîbe.\\ 
 & \textbf{nû} hœret \textbf{ouch} von Sygunen sagen.\\ 
 & diu kunde ir leit mit jâmer \textbf{klagen}.\\ 
25 & si sprach zuo dem knaben: "dû hâst tugent.\\ 
 & geêret \textit{sî} dîn \textbf{süeze} jugent\\ 
 & und dîn antlitze minneclîch.\\ 
 & \textbf{dêswâr}, dû wirst noch sælden rîch.\\ 
 & disen rîter meit daz gabilôt.\\ 
30 & er lac \textbf{von einer jost} tôt.\\ 
\end{tabular}
\scriptsize
\line(1,0){75} \newline
U V W T \newline
\line(1,0){75} \newline
\textbf{1} \textit{Majuskel} T  \textbf{9} \textit{Initiale} U V W   $\cdot$ \textit{Majuskel} T  \textbf{10} \textit{Majuskel} T  \textbf{15} \textit{Majuskel} T  \textbf{19} \textit{Majuskel} T  \textbf{23} \textit{Initiale} T  \textbf{25} \textit{Majuskel} T  \newline
\line(1,0){75} \newline
\textbf{2} er] \textit{om.} V W T \textbf{3} niht] \textit{om.} V W T  $\cdot$ eime] einer W \textbf{5} iht mir von im] mir von im W mir iht T \textbf{6} wer den riter habe erslagen T  $\cdot$ hât den rîter] den ritter hat V (W) \textbf{8} wil gerne] [*]: wolte V wolte T \textbf{11} vant] do vand W \textbf{13} von] frauw W  $\cdot$ Jeschuten] Jescuͦten U iescuten V iestuten W Jescvten T  $\cdot$ brach] abe brach W \textbf{14} und] von T  $\cdot$ dâ] do V W \textbf{16} wirdeclîche] werdekeit T (W)  $\cdot$ im wonte] wonte im W \textbf{17} diu buckel] De bockele U (V) (W)  $\cdot$ wære] weren V \textbf{18} dâ] Do W (T) \textbf{19} sît] sich U  $\cdot$ vil kumbers] kummers vil W \textbf{20} ganzez] ganzes T \textbf{23} hœret ouch] hoͤren W  $\cdot$ von] von frauw W  $\cdot$ Sygunen] Sigunen U (V) (T) \textbf{24} jâmer klagen] zvhten tragen T \textbf{26} sî] \textit{om.} U \textbf{28} dêswâr] \textit{om.} T  $\cdot$ noch] nun W benamen T \textbf{29} daz] des U der V dis W \textbf{30} von einer] zuͦ einer W (T) \newline
\end{minipage}
\end{table}
\end{document}
