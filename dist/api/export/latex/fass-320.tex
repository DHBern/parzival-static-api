\documentclass[8pt,a4paper,notitlepage]{article}
\usepackage{fullpage}
\usepackage{ulem}
\usepackage{xltxtra}
\usepackage{datetime}
\renewcommand{\dateseparator}{.}
\dmyyyydate
\usepackage{fancyhdr}
\usepackage{ifthen}
\pagestyle{fancy}
\fancyhf{}
\renewcommand{\headrulewidth}{0pt}
\fancyfoot[L]{\ifthenelse{\value{page}=1}{\today, \currenttime{} Uhr}{}}
\begin{document}
\begin{table}[ht]
\begin{minipage}[t]{0.5\linewidth}
\small
\begin{center}*D
\end{center}
\begin{tabular}{rl}
\textbf{320} & \begin{large}S\end{large}în muot \textbf{stuont} hôch, doch jâmers vol.\\ 
 & die bêde schanze ich nennen sol:\\ 
 & hôchvart riet sîn manheit,\\ 
 & jâmer lêrt in herzen leit.\\ 
5 & Er reit \textbf{ûzen} z\textit{uo}me ringe.\\ 
 & ob man in dâ iht dringe?\\ 
 & vil \textbf{knappen} spranc dar nâher sân;\\ 
 & dô enpfiengen si den werden man.\\ 
 & sîn schilt unt er wâren unbekant;\\ 
10 & den helm er niht von im bant.\\ 
 & Der vreuden ellende\\ 
 & \textbf{truoc daz swert} \textbf{in sîner} hende,\\ 
 & \textbf{bedecket} mit der scheiden.\\ 
 & dô vrâgter nâch in beiden:\\ 
15 & "wâ ist Artus unt Gawan?"\\ 
 & junchêrren zeigten im die sân.\\ 
 & Sus gieng er \textbf{durch} den rinc wît.\\ 
 & tiwer was sîn kursît,\\ 
 & mit liehtem pfelle wol gevar.\\ 
20 & vür \textbf{den wirt des} ringes schar\\ 
 & \textbf{stuont er} unt sprach alsus:\\ 
 & "got halde den künec Artus,\\ 
 & dar zuo vrowen unt man.\\ 
 & swaz ich der hie \textbf{gesehen} hân,\\ 
25 & \textbf{den} biut ich dienstlîchen gruoz.\\ 
 & wan einem \textbf{tuot mîn dienst} buoz,\\ 
 & \textbf{im} \textbf{en}wirt dienst nimer schîn.\\ 
 & ich wil bî sîme hazze sîn.\\ 
 & swaz hazzes er geleisten mac,\\ 
30 & mîn haz im biutet hazzes slac.\\ 
\end{tabular}
\scriptsize
\line(1,0){75} \newline
D \newline
\line(1,0){75} \newline
\textbf{1} \textit{Initiale} D  \textbf{5} \textit{Majuskel} D  \textbf{11} \textit{Majuskel} D  \textbf{17} \textit{Majuskel} D  \newline
\line(1,0){75} \newline
\textbf{5} zuome] zoͮme D \newline
\end{minipage}
\hspace{0.5cm}
\begin{minipage}[t]{0.5\linewidth}
\small
\begin{center}*m
\end{center}
\begin{tabular}{rl}
 & sîn muot \textbf{was} hôch, doch jâmers vol.\\ 
 & die beide schanze ich nennen sol:\\ 
 & hôchvart riet sîn manheit,\\ 
 & jâmer lêrte in herzeleit.\\ 
5 & er reit \textbf{ûz} zuo dem ringe.\\ 
 & ob man in dâ iht dringe?\\ 
 & vil \textbf{knappen} spranc dar nâher sân;\\ 
 & dô enpfiengen si den werden man.\\ 
 & sîn schilt und er wâren unbekant;\\ 
10 & den helm er niht von im bant.\\ 
 & der vröuden ellende\\ 
 & \textbf{truoc daz swert} \textbf{in sîner} hende,\\ 
 & \textbf{bedecket} mit der scheiden.\\ 
 & dâ vrâgeter nâch in beiden:\\ 
15 & "wâ ist Artus und G\textit{a}wan?"\\ 
 & junchêrren zeig\textit{et}en \dag in\dag  die sân.\\ 
 & sus gienc er \textbf{durch} den rinc wît.\\ 
 & tiure was sîn kursît,\\ 
 & mit liehtem pfelle wol gevar.\\ 
20 & \dag wirt vür und\dag  vür ringes schar\\ 
 & \textbf{er stuont} und sprach alsus:\\ 
 & "got ha\textit{l}te den künic Artus,\\ 
 & dar zuo vrouwen und man.\\ 
 & waz ich der hie \textbf{gesehen} hân,\\ 
25 & \textbf{den} biut ich dienstlîchen gruoz.\\ 
 & wanne eine\textit{m} \textbf{tuot mîn dienest} buoz,\\ 
 & \textbf{dem} wirt \textbf{mîn} dienest nimer schîn.\\ 
 & ich wil bî sînem hazze sîn.\\ 
 & waz hazzes er geleisten mac,\\ 
30 & mîn haz ime \textit{biutet} hazzes slac.\\ 
\end{tabular}
\scriptsize
\line(1,0){75} \newline
m n o \newline
\line(1,0){75} \newline
\newline
\line(1,0){75} \newline
\textbf{1} doch] \textit{om.} n \textbf{4} lêrte] lert n o \textbf{6} dâ] do n o \textbf{9} wâren] die worent n \textbf{11} ellende] ellendenden o \textbf{14} dâ] Do n o \textbf{15} Gawan] gewan m o \textbf{16} zeigeten] zeigen m  $\cdot$ in] ẏne o  $\cdot$ die] do n o \textbf{22} halte] hatte m het n (o)  $\cdot$ Artus] artuͯs o \textbf{26} einem] einen m  $\cdot$ tuot] dot o \textbf{30} biutet] \textit{om.} m \newline
\end{minipage}
\end{table}
\newpage
\begin{table}[ht]
\begin{minipage}[t]{0.5\linewidth}
\small
\begin{center}*G
\end{center}
\begin{tabular}{rl}
 & sîn muot \textbf{stuont} hôch, doch jâmers vol.\\ 
 & die bêde schanze ich nennen sol:\\ 
 & hôchvart riet sîn manheit,\\ 
 & jâmer lêrte in herzeleit.\\ 
5 & er reit \textbf{ûzen} zuo dem ringe.\\ 
 & op man in dâ iht dringe?\\ 
 & vil \textbf{knappen} spranc dar nâher sân;\\ 
 & dô enpfiengen si den werden man.\\ 
 & sîn schilt unde er wâren unbekant;\\ 
10 & den helm er niht von im bant.\\ 
 & der vröuden ellende\\ 
 & \textbf{truoc daz swert} \textbf{in sîner} hende,\\ 
 & \textbf{verdecket} mit der scheiden.\\ 
 & dô vragter nâch \textit{i}n beiden:\\ 
15 & "wâ ist Artus unde Gawan?"\\ 
 & junchêrren zeigten im die sân.\\ 
 & sus gieng er \textbf{durch} den rinc wît.\\ 
 & tiure was sîn kursît,\\ 
 & mit liehtem pfelle wolgevar.\\ 
20 & vür \textbf{den wirt des} ringes schar\\ 
 & \textbf{stuont er} unde sprach alsus:\\ 
 & "got halt den künic Artus,\\ 
 & dar zuo vrouwen unde man.\\ 
 & swaz ich der hie \textbf{ersehen} hân,\\ 
25 & \textbf{den} biut ich dienstlîchen gruoz.\\ 
 & wan einem \textbf{tuot mîn dienst} buoz,\\ 
 & \textbf{dem} wirt \textbf{mîn} dienst nimmer schîn.\\ 
 & ich wil bî sînem hazze sîn.\\ 
 & swaz hazzes er geleisten mac,\\ 
30 & mîn haz im biut hazzes slac.\\ 
\end{tabular}
\scriptsize
\line(1,0){75} \newline
G I O L M Q R Z Fr22 Fr39 Fr40 \newline
\line(1,0){75} \newline
\textbf{9} \textit{Initiale} L M Fr39  \textbf{17} \textit{Initiale} I Z  \textbf{25} \textit{Initiale} Fr40  \newline
\line(1,0){75} \newline
\textbf{1} doch] vnde doch O \textbf{2} bêde schanze ich] tsantze ich beide L (Fr39) beden schantze ich Q \textbf{3} riet] [*]: reit L reit Q Fr39 \textbf{4} lêrte] lert I O L Q R Z Fr39 Fr40  $\cdot$ in] yn syn M \textbf{5} zuo] \textit{om.} L Fr39  $\cdot$ dem ringe] den Ringen R \textbf{6} op] Ab M  $\cdot$ in dâ iht] den da iht O in iht da L da yn icht M in do icht Q in da ich R in iht do Fr39  $\cdot$ dringe] drunge M dringen R \textbf{7} dar] do Fr39 \textbf{8} dô] Vnd L (Fr39) Da M Z  $\cdot$ si] \textit{om.} L Fr39 \textbf{9} unde er wâren] was vil O wasz do Q was da R was Fr40  $\cdot$ unbekant] vnerkant R \textbf{10} niht von im] do von im nicht Q nit vom hobet R \textbf{11} vröuden] vreden L [verden]: vrerden Q froͯde R \textbf{12} truoc] Truck sie Q  $\cdot$ sîner] der L Fr39 \textbf{13} mit] in I \textbf{14} dô] Da M Z  $\cdot$ vragter] vragt er I (O) (L) (Q) (R) (Z) Fr40  $\cdot$ in] den G \textbf{15} Artus] Artv̂s Fr22 \textbf{16} zeigten im die] zeýchten ým L (M) (Fr39) zeigten in di Q zeigtten im den R zêigints si im Fr22 \textbf{17} durch] vur I in Q \textbf{18} tiure] Truwir M \textbf{20} wirt] Ring zem wirt R \textbf{22} halt] halden R  $\cdot$ Artus] Artv̂s Fr22 \textbf{23} dar zuo] vnd darzuͤ I (O) (L) (M) (Q) (R) (Fr22) (Fr39) (Fr40) \textbf{24} swaz] Waz L (M) (Q) (R)  $\cdot$ ersehen] gesehen I L Z Fr39  $\cdot$ hân] kan M R \textbf{25} biut] bite M (Q) \textbf{26} einem] einen L einē Q Fr39 Fr40 \textbf{27} wirt] inwirt Fr22  $\cdot$ nimmer] minner Q \textbf{29} swaz] Waz L (M) (Q) (R)  $\cdot$ er] ich L  $\cdot$ geleisten] geleiste M \textbf{30} haz] hasses Q \newline
\end{minipage}
\hspace{0.5cm}
\begin{minipage}[t]{0.5\linewidth}
\small
\begin{center}*T
\end{center}
\begin{tabular}{rl}
 & sîn muot \textbf{stuont} hôhe, doch jâmers vol.\\ 
 & die beiden schanze ich nennen sol:\\ 
 & hôchvart riet sîn manheit,\\ 
 & jâmer lêrtin herzeleit.\\ 
5 & er reit \textbf{ûzen} zuo dem ringe.\\ 
 & ob man in dâ iht dringe?\\ 
 & vil \textbf{junchêrren} spranc dar nâher sân;\\ 
 & dô enpfiengen si den werden man.\\ 
 & sîn schilt unde er wâren unbekant;\\ 
10 & den helm er niht von im bant.\\ 
 & der vröuden ellende,\\ 
 & \textbf{daz swert truoc er} \textbf{an der} hende,\\ 
 & \textbf{bedecket} mit der scheiden.\\ 
 & dô vrâgeter nâch in beiden:\\ 
15 & "wâ ist Artus unde Gawan?"\\ 
 & \textbf{Vil} junchêrren zeigten im die sân.\\ 
 & Sus gienc er \textbf{vür} den rinc wît.\\ 
 & tiure was sîn kursît,\\ 
 & mit liehtem pfelle wol gevar.\\ 
20 & vür \textbf{den wirt in des} ringes schar\\ 
 & \textbf{stuont er} unde sprach alsus:\\ 
 & "got halte den künec Artus\\ 
 & \textbf{unde} dar zuo vrouwen unde man.\\ 
 & swaz ich der hie \textbf{gesehen} hân,\\ 
25 & \textbf{dem} biut ich dienstlîchen gruoz.\\ 
 & wan einem \textbf{tuon ich dienstes} buoz,\\ 
 & \textbf{dem} wirt \textbf{mîn} dienst niemer schîn.\\ 
 & ich wil bî sînem hazze sîn.\\ 
 & swaz hazzes er geleisten mac,\\ 
30 & mîn haz im biutet hazzes slac.\\ 
\end{tabular}
\scriptsize
\line(1,0){75} \newline
T U V W \newline
\line(1,0){75} \newline
\textbf{16} \textit{Majuskel} T  \textbf{17} \textit{Majuskel} T  \newline
\line(1,0){75} \newline
\textbf{1} muot] [mvnt]: mvͦt T  $\cdot$ doch] vnd W \textbf{2} beiden] beide V \textbf{6} dâ iht] do it U (V) icht do W \textbf{7} spranc dar] sprach do W \textbf{12} Truͦg das schwert in der hende W  $\cdot$ er] \textit{om.} U  $\cdot$ an der] in siner V \textbf{14} vrâgeter] vraget er U  $\cdot$ nâch] von W \textbf{16} zeigten] zeichen U \textbf{17} vür] [*]: durch V durch W \textbf{20} wirt] [*]: wirt T kv́nig V \textbf{22} künec] kúuig W \textbf{24} swaz] Waz U (W) \textbf{25} dem] Den U V W  $\cdot$ biut] bieden U \textbf{26} einem] einem dem V  $\cdot$ ich dienstes] ich dienst V mein dienst nimmer W \textbf{27} dem] Leides W  $\cdot$ niemer] meiner V \textbf{29} \textit{Die Verse 320.29-321.2 fehlen} W   $\cdot$ swaz] Waz U \textbf{30} biutet] butet im U \newline
\end{minipage}
\end{table}
\end{document}
