\documentclass[8pt,a4paper,notitlepage]{article}
\usepackage{fullpage}
\usepackage{ulem}
\usepackage{xltxtra}
\usepackage{datetime}
\renewcommand{\dateseparator}{.}
\dmyyyydate
\usepackage{fancyhdr}
\usepackage{ifthen}
\pagestyle{fancy}
\fancyhf{}
\renewcommand{\headrulewidth}{0pt}
\fancyfoot[L]{\ifthenelse{\value{page}=1}{\today, \currenttime{} Uhr}{}}
\begin{document}
\begin{table}[ht]
\begin{minipage}[t]{0.5\linewidth}
\small
\begin{center}*D
\end{center}
\begin{tabular}{rl}
\textbf{500} & \begin{large}D\end{large}er wirt ân allez bâgen\\ 
 & begunde in vürbaz vrâgen:\\ 
 & "neve, noch hân ich niht vernomen,\\ 
 & wannen dir \textbf{diz} ors sî komen."\\ 
5 & "hêrre, \textbf{daz} ors ich erstreit,\\ 
 & dô ich von Sigunen reit.\\ 
 & vor einer klôsen ich die \textbf{sprach}.\\ 
 & dar nâch ich \textbf{vlügelingen} stach\\ 
 & einen rîter drabe unt zôch ez dan.\\ 
10 & von Munsalvæsche was der man."\\ 
 & \textbf{Der wirt} sprach: "ist aber \textbf{der} genesen,\\ 
 & des ez \textbf{von} rehte \textbf{sol} wesen?"\\ 
 & "hêrre, ich sach in \textbf{vor} mir gên\\ 
 & unt vant daz ors bî mir stên."\\ 
15 & "wil dûs Grâles volc sus rouben\\ 
 & unt dâ bî des gelouben,\\ 
 & dû gewinnest \textbf{ir} noch minne,\\ 
 & \textbf{sô} zweient sich \textbf{die} sinne."\\ 
 & "hêrre, ich namz in eime strît.\\ 
20 & swer mir dar umbe \textbf{sünde} gît,\\ 
 & der prüeve alrêst, wie \textbf{diu} \textbf{stê}.\\ 
 & mîn ors het ich verlorn ê."\\ 
 & \textbf{dô sprach aber} Parzival:\\ 
 & "wer was \textbf{ein} magt, diu den Grâl\\ 
25 & truoc? ir mandel lêch man mir."\\ 
 & der wirt sprach: "\textbf{neve, was er ir}\\ 
 & - diu selbe ist dîn muome -,\\ 
 & sine lêch dir\textit{\textbf{s}} niht ze ruome.\\ 
 & si wânde, dû soldest dâ hêrre sîn\\ 
30 & des Grâles unt \textbf{ir}, dar zuo mîn.\\ 
\end{tabular}
\scriptsize
\line(1,0){75} \newline
D Fr11 \newline
\line(1,0){75} \newline
\textbf{1} \textit{Initiale} D Fr11  \textbf{11} \textit{Majuskel} D  \textbf{27} \textit{Initiale} Fr11  \newline
\line(1,0){75} \newline
\textbf{6} Sigunen] Sygvnen D \textbf{10} Munsalvæsche] Mvnsælvæsche D \textbf{15} dûs] duͯ daz Fr11  $\cdot$ sus] also Fr11 \textbf{18} sich die] sin dein Fr11 \textbf{19} namz] nams Fr11 \textbf{23} Parzival] Parcifal D Partzival Fr11 \textbf{26} er] et Fr11 \textbf{28} sine] si Fr11  $\cdot$ dirs] dir: \textit{nachträglich korrigiert zu:} dirs D \textbf{29} soldest] soltz Fr11 \textbf{30} dar] vnd dar Fr11 \newline
\end{minipage}
\hspace{0.5cm}
\begin{minipage}[t]{0.5\linewidth}
\small
\begin{center}*m
\end{center}
\begin{tabular}{rl}
 & \begin{large}D\end{large}er wirt âne allez bâgen\\ 
 & begunde in vürbaz vrâgen:\\ 
 & "neve, noch hân ich niht vernomen,\\ 
 & wannen dir \textbf{daz} ros sî komen."\\ 
5 & "hêrre, \textbf{daz} ros ich erstreit,\\ 
 & dô ich von Sigunen reit.\\ 
 & vor einer klûsen ich die \textbf{sprach}.\\ 
 & dar nâch ich \textbf{vlügelîchen} stach\\ 
 & einen ritter dar ab und zôch ez dan.\\ 
10 & von Muntsalvasche was der man."\\ 
 & \textbf{er} sprach: "ist aber \textbf{der man} genesen,\\ 
 & des ez \textbf{von} rehte \textbf{solte} wesen?"\\ 
 & "hêrre, ich sach in \textbf{von} mir gân\\ 
 & und vant daz ros bî mir \textbf{d\textit{â}} stân."\\ 
15 & "wiltû des Grâles volc sus rouben\\ 
 & und dâ bî des glouben,\\ 
 & dû gewinnest \textbf{sîn} noch minne,\\ 
 & \textbf{sô} \textit{z}wei\textit{en}t sich \textbf{die} sinne."\\ 
 & "hêrre, ich nam ez in eine\textit{m} strît.\\ 
20 & wer mir dar umb \textbf{sünde} gît,\\ 
 & der brüefe allerêrst, wie \textbf{ez} \textbf{stê}.\\ 
 & mîn ros het ich verlorn ê."\\ 
 & \textbf{\begin{large}A\end{large}ber sprach} Parcifal:\\ 
 & "wer was \textbf{diu} maget, diu den Grâl\\ 
25 & truoc? ir mantel lêch m\textit{an} mir."\\ 
 & der wirt sprach: "\textbf{neve, was er ir}\\ 
 & - diu selbe ist \textit{d}în muome -,\\ 
 & si enlêch dir\textbf{s} niht zuo ruome.\\ 
 & si wânte, dû soltest d\textit{â} hêrre sîn\\ 
30 & des Grâles und dar zuo mîn.\\ 
\end{tabular}
\scriptsize
\line(1,0){75} \newline
m n o \newline
\line(1,0){75} \newline
\textbf{1} \textit{Illustration mit Überschrift:} Also der wurt parcifal sinen nefen frogete wo jme das cluͯge rosz har keme n (o)   $\cdot$ \textit{Initiale} m n o  \textbf{23} \textit{Initiale} m   $\cdot$ \textit{Capitulumzeichen} n  \newline
\line(1,0){75} \newline
\textbf{3} vernomen] vernemen o \textbf{4} daz] dis n \textbf{6} Sigunen] syguͯnen o \textbf{8} vlügelîchen] fluͯgelingen n (o) \textbf{10} Muntsalvasche] muntsaluasce m montsaluasce n múntsaluasce o \textbf{12} des] Das m n o \textbf{13} mir] [min]: mir m \textbf{14} dâ] do m n o \textbf{17} dû] Vnd n \textbf{18} zweient] ze weinet m  $\cdot$ die] din n o \textbf{19} ez] das n  $\cdot$ einem] einen m \textbf{21} allerêrst] allumb alrest o  $\cdot$ ez stê] dieste n (o) \textbf{24} diu maget] [dir]: die maget o \textbf{25} man mir] mir [an]: mir m \textbf{26} was] [s]: wa:s o \textbf{27} dîn] min m \textbf{28} si] Die o  $\cdot$ ruome] rome o \textbf{29} dâ] do m n o \newline
\end{minipage}
\end{table}
\newpage
\begin{table}[ht]
\begin{minipage}[t]{0.5\linewidth}
\small
\begin{center}*G
\end{center}
\begin{tabular}{rl}
 & \textit{\begin{large}D\end{large}}er wirt ân allez bâgen\\ 
 & begunde in vürbaz vrâgen:\\ 
 & "neve, noch \textbf{en}hân ich niht vernomen,\\ 
 & wannen dir \textbf{ditze} ors sî komen."\\ 
5 & "hêrre, \textbf{ditze} ors ich erstreit,\\ 
 & dô ich von Sigunen reit.\\ 
 & vo\textit{r} einer klûsen ich die \textbf{sach}.\\ 
 & dar nâch ich \textbf{vlügelingen} stach\\ 
 & einen rîter drabe unde zôch ez dan.\\ 
10 & von Muntsalvatsche was der man."\\ 
 & \textbf{der wirt} sprach: "ist aber \textbf{der} genesen,\\ 
 & des ez \textbf{von} rehte \textbf{solde} wesen?"\\ 
 & "hêrre, ich sach in \textbf{von} mir gên\\ 
 & unt vant daz ors bî mir stên."\\ 
15 & "wil dûs Grâles volc sus rouben\\ 
 & unt dâ bî des gelouben,\\ 
 & dû gewinnest \textbf{ir} noch minne,\\ 
 & \textbf{sô} zweient sich \textbf{dîn} sinne."\\ 
 & "hêrre, ich nam ez in einem strît.\\ 
20 & swer mir dar umbe \textbf{sünde} gît,\\ 
 & der prüeve alrêrste, wie \textbf{diu} \textbf{gestê}.\\ 
 & mîn ors het ich verlorn ê."\\ 
 & \textbf{dô sprach aber} Parzival:\\ 
 & "wer was \textbf{ein} maget, diu den Grâl\\ 
25 & truoc? ir mantel lêch man mir."\\ 
 & der wirt sprach: "\textbf{neve, was er ir}\\ 
 & - diu selbe ist dîn muome -,\\ 
 & sine lêch dir\textbf{s} niht ze ruome.\\ 
 & si wânde, dû soldest dâ hêrre sîn\\ 
30 & des Grâles unde \textbf{ir}, dar zuo mîn.\\ 
\end{tabular}
\scriptsize
\line(1,0){75} \newline
G I L M Z Fr61 \newline
\line(1,0){75} \newline
\textbf{1} \textit{Initiale} G I L Z Fr61  \textbf{15} \textit{Initiale} I  \newline
\line(1,0){75} \newline
\textbf{1} Der] Eer G \textbf{2} in] \textit{om.} I \textbf{3} neve] Nueue M  $\cdot$ noch enhân ich] ich han noch I noch han ich L (M) (Fr61) nach ich Z \textbf{4} wannen] von wanne Fr61  $\cdot$ ditze] daz L \textbf{5} hêrre] \textit{om.} I  $\cdot$ ditze] daz L (M) \textbf{6} dô] Da M Z  $\cdot$ Sigunen] Sýgvnen L Sẏgune Fr61 \textbf{7} vor] Von G M  $\cdot$ klûsen] kluse I  $\cdot$ die] da I  $\cdot$ sach] sprach M besprach Z Fr61 \textbf{8} nâch] von I  $\cdot$ vlügelingen] fluͯgelichen L \textbf{10} Muntsalvatsche] mvntsalvatsch G Muntshaluasce I montsalvatsche Z \textbf{11} sprach] do sprach I  $\cdot$ aber der] aber er L Z her abir M \textbf{13} in von] vor L in vor M Z \textbf{14} vant daz ors bî] diz ors vant ich vor I vant ros bie M vant daz orss vor Z \textbf{15} volc] \textit{om.} L \textbf{16} des] \textit{om.} I \textbf{17} ir noch] noch ir L \textbf{18} sô] Da M  $\cdot$ dîn] die I L (M) \textbf{20} swer] Wer L M  $\cdot$ mir] \textit{om.} Z \textbf{21} gestê] ste L M Z \textbf{23} Parzival] parzifal I L M parcifal Z \textbf{25} ir] der I \textbf{26} neve was er] was er I neve er waz et L her was ot M \textbf{28} sine] si I  $\cdot$ dirs] dir sin I dir in L disz M \textbf{30} dar zuo] vn darzuͤ I (L) (M) \newline
\end{minipage}
\hspace{0.5cm}
\begin{minipage}[t]{0.5\linewidth}
\small
\begin{center}*T
\end{center}
\begin{tabular}{rl}
 & \begin{large}D\end{large}er wirt âne allez bâgen\\ 
 & begund in vürbaz vrâgen:\\ 
 & "n\textit{e}ve, noch hân ich niht vernomen,\\ 
 & wannen dir \textbf{daz} ors sî komen."\\ 
5 & "Hêrre, \textbf{diz} ors ich \textbf{hie} erstreit,\\ 
 & dô ich von Sygunen reit.\\ 
 & vor einer klôsen ich die \textbf{gesprach}.\\ 
 & dar nâch ich \textbf{vlügelingen} stach\\ 
 & einen rîter drabe unde zôch ez dan.\\ 
10 & von Munsalvasche was der man."\\ 
 & \textbf{Der wirt} sprach: "ist aber \textbf{er} genesen,\\ 
 & des ez \textbf{ze} rehte \textbf{solte} wesen?"\\ 
 & "Hêrre, ich sach in \textbf{von} mir gên\\ 
 & unde vant daz ors bî mir stên."\\ 
15 & "Wiltû des Grâles volc sus rouben\\ 
 & unde dâ bî des gelouben,\\ 
 & dû gewinnest \textbf{ir} noch minne,\\ 
 & \textbf{dâ} zweient sich \textbf{die} sinne."\\ 
 & "Hêrre, ich nam ez in einem strît.\\ 
20 & swer mir dar umbe \textbf{wandel} gît,\\ 
 & der prüeve aller êrst, wie \textbf{der} \textbf{gestê}.\\ 
 & mîn ors het ich verlorn ê."\\ 
 & \textbf{\begin{large}E\end{large}ines tages vrâget in} Parcifal:\\ 
 & "wer was \textbf{ein} maget, diu den Grâl\\ 
25 & truoc? ir mantel lêch man mir."\\ 
 & Der wirt sprach: "\textbf{daz sag ich dir}:\\ 
 & diu selbe ist dîn muome,\\ 
 & si enlêch dir\textbf{n} niht ze ruome.\\ 
 & si wânde, dû soltest dâ hêrre sîn\\ 
30 & des Grâles unde \textbf{ir} \textbf{unde} dar zuo mîn.\\ 
\end{tabular}
\scriptsize
\line(1,0){75} \newline
T U V W O Q R Fr39 \newline
\line(1,0){75} \newline
\textbf{1} \textit{Initiale} T O Q  \textbf{3} \textit{Initiale} R  \textbf{5} \textit{Majuskel} T  \textbf{11} \textit{Initiale} W Fr39   $\cdot$ \textit{Majuskel} T  \textbf{13} \textit{Majuskel} T  \textbf{15} \textit{Capitulumzeichen} R   $\cdot$ \textit{Majuskel} T  \textbf{19} \textit{Capitulumzeichen} R   $\cdot$ \textit{Majuskel} T  \textbf{23} \textit{Initiale} T V  \textbf{26} \textit{Majuskel} T  \newline
\line(1,0){75} \newline
\textbf{1} \textit{Die Verse 453.1-502.30 fehlen} U   $\cdot$ Der] ÷er O  $\cdot$ bâgen] sagen R \textbf{2} in] \textit{om.} O  $\cdot$ vürbaz vrâgen] furbas Jagen vnd fragen R \textbf{3} neve] nve T  $\cdot$ ich] \textit{om.} V \textbf{4} wannen] Von wannan V (W) (O) (Q) Fr39 Von dir wannen R \textbf{5} diz] das Q  $\cdot$ hie] \textit{om.} W O Q R \textbf{6} Sygunen] sigvnen V (Fr39) Sygvͦnen O sichungen Q \textbf{8} ich vlügelingen stach] flv́gelingen [s*]: ich stach V \textbf{9} drabe] [drahe]: drabe Fr39  $\cdot$ zôch] fvͦrt V \textbf{10} Munsalvasche] Mvnsalvasce T [mvntsal*]: mvntsalvasche V montsaluatsche W mvntsalvatsche O muntsaluasche Q Munsaluasche R munsaluahse Fr39  $\cdot$ der] des W \textbf{11} aber er] er aber R aber Q \textbf{12} des] Das W Q  $\cdot$ ez] er Fr39 \textbf{13} von] [vor]: von V vor Q \textbf{14} vant] [want]: vant Q  $\cdot$ bî] vor V  $\cdot$ mir] im R \textbf{15} Wiltû des] Wilt dvs V (Fr39) Wilt du W (R)  $\cdot$ sus] nun Q \textbf{16} des] das Q \textbf{17} [D* g* *h *e]: Dv engewinnest ir noch minne V  $\cdot$ ir noch] noch ir O \textbf{18} dâ] Do W O Q (Fr39) \textbf{19} ich] ihc O  $\cdot$ in] an V von R \textbf{20} swer] Wer W Q R \textbf{21} aller] alle R  $\cdot$ der] daz V  $\cdot$ gestê] gaste W \textbf{22} het] hat R \textbf{23} Eines tages vrâget in] [A* i*]: Aber sprach do V  $\cdot$ Parcifal] parzifal V partzifal W Q Barcifal O parczifal R \textbf{24} ein] die W du R \textbf{25} mir] [ir]: mir Q \textbf{26} daz sag ich dir] neve er was et ir O \textbf{27} diu] Das W  $\cdot$ ist] was W  $\cdot$ dîn] min R \textbf{28} si enlêch] Sy leihe W Si lech O (R) (Fr39) Seine lech Q  $\cdot$ dirn] dirs W R Fr39 \textbf{29} dû] di O  $\cdot$ dâ] do V W Q Fr39 \textbf{30} unde ir unde] vnd ire vnd Q ir vnd R vnd Fr39 \newline
\end{minipage}
\end{table}
\end{document}
