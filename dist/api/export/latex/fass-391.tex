\documentclass[8pt,a4paper,notitlepage]{article}
\usepackage{fullpage}
\usepackage{ulem}
\usepackage{xltxtra}
\usepackage{datetime}
\renewcommand{\dateseparator}{.}
\dmyyyydate
\usepackage{fancyhdr}
\usepackage{ifthen}
\pagestyle{fancy}
\fancyhf{}
\renewcommand{\headrulewidth}{0pt}
\fancyfoot[L]{\ifthenelse{\value{page}=1}{\today, \currenttime{} Uhr}{}}
\begin{document}
\begin{table}[ht]
\begin{minipage}[t]{0.5\linewidth}
\small
\begin{center}*D
\end{center}
\begin{tabular}{rl}
\textbf{391} & \begin{large}I\end{large}mmer \textbf{swenne} si vür ir swester gienc,\\ 
 & diu disen schimpf mit zorn enpfienc.\\ 
 & den rittern \textbf{dâ was} ruowe nôt,\\ 
 & wande \textbf{in} grôz müede daz gebôt.\\ 
5 & Scherules nam Gawan\\ 
 & unt den grâven Laheduman.\\ 
 & dennoch mêr ritter er dâ vant,\\ 
 & die Gawan mit sîner hant\\ 
 & des tages ûf dem velde vienc,\\ 
10 & dâ manec grôziu hurte ergienc.\\ 
 & dô sazte si \textbf{ritterlîche}\\ 
 & der burcgrâve rîche.\\ 
 & \textbf{er} unt alsîn müediu schar\\ 
 & stuonden vor dem künege gar,\\ 
15 & unz Melyanz enbeiz.\\ 
 & guoter handelunge er sich \textbf{dâ} vleiz.\\ 
 & des dûhte Gawane ze vil.\\ 
 & "ob \textbf{ez} der künec erlouben wil,\\ 
 & hêr wirt, sô sult ir sitzen",\\ 
20 & sprach Gawan mit witzen.\\ 
 & sîn zuht in dar zuo jagete.\\ 
 & der wirt die bete versagete.\\ 
 & er sprach: "mîn hêrre ist des küneges man.\\ 
 & \textbf{disen} dienst het er getân,\\ 
25 & ob den künec des gezæme,\\ 
 & daz er sînen dienst næme.\\ 
 & mîn hêrre \textbf{durch zuht sîn niht ensiht},\\ 
 & wand er\textbf{n} hât sîner hulde niht.\\ 
 & gesamnet die vriwentschaft iemer got,\\ 
30 & \textbf{sô leiste wir} alle sîn gebot."\\ 
\end{tabular}
\scriptsize
\line(1,0){75} \newline
D \newline
\line(1,0){75} \newline
\textbf{1} \textit{Initiale} D  \newline
\line(1,0){75} \newline
\textbf{5} Scherules] Scervles D \newline
\end{minipage}
\hspace{0.5cm}
\begin{minipage}[t]{0.5\linewidth}
\small
\begin{center}*m
\end{center}
\begin{tabular}{rl}
 & iemer \textbf{wenne} si vü\textit{r i}r swester gie\textit{n}c,\\ 
 & diu disen schimpf mit zorne enpfienc.\\ 
 & den rittern \textbf{was d\textit{â}} ruowe nôt,\\ 
 & wand \textbf{ir} grôz müede daz gebôt.\\ 
5 & Scherules nam Gawan\\ 
 & und \textit{den} grâve\textit{n} \textit{La}hed\textit{u}m\textit{an}.\\ 
 & dennoch mêr ritter er d\textit{â} vant,\\ 
 & die Gawan mit sîner hant\\ 
 & des tages ûf dem velde vienc,\\ 
10 & d\textit{â} manic grôz h\textit{u}rte ergienc.\\ 
 & \textit{d}ô sazte \textit{si} \textbf{werdeclîche}\\ 
 & der burcgrâve rîche,\\ 
 & und alliu sîn müediu schar\\ 
 & stuonden vor dem künige gar,\\ 
15 & unz Mel\textit{i}anz enbeiz.\\ 
 & guoter handelunge er sich vleiz.\\ 
 & des dûhte Gawanen ze vil.\\ 
 & "ob \textbf{ez} der künic erlouben wil,\\ 
 & hêr wirt, sô sollet ir sitzen",\\ 
20 & sprach Gawan mit witzen.\\ 
 & sîn zuht in dar zuo jagete.\\ 
 & der wirt die bete versagete.\\ 
 & er sprach: "mîn hêrre ist des küniges man.\\ 
 & dienest hete er getân,\\ 
25 & ob den künic des gezæme,\\ 
 & daz er sînen dienest næme.\\ 
 & mîn hêrre \textbf{durch zuht sîn niht ensiht},\\ 
 & wand er hât sîner hulde niht.\\ 
 & gesamet die vriuntschaft iemer got,\\ 
30 & \textbf{sô leisten wir} alle sîn gebot."\\ 
\end{tabular}
\scriptsize
\line(1,0){75} \newline
m n o \newline
\line(1,0){75} \newline
\newline
\line(1,0){75} \newline
\textbf{1} vür ir swester gienc] fur in ir swester gieig m \textbf{2} schimpf mit zorne] zorn mit schẏmpff o \textbf{3} was] det n  $\cdot$ dâ] do m n o \textbf{5} Scherules] Scerules m Sterules n Scernels o  $\cdot$ Gawan] gawann o \textbf{6} den] \textit{om.} m  $\cdot$ grâven Laheduman] graffen von lehedmum m grofen lahedumann o \textbf{7} dâ] do m n o \textbf{10} dâ] Do m n o  $\cdot$ hurte] herte m o \textbf{11} So saczte werdekliche m \textbf{14} gar] dar o \textbf{15} Melianz] meleancz m meliantz n meliancz o \textbf{17} des] Das o  $\cdot$ Gawanen] gawan o \textbf{19} hêr] Hert o \textbf{22} die] der o \textbf{24} dienest] Dienste o  $\cdot$ hete] hat n  $\cdot$ getân] vil getan n o \textbf{25} den künic] dem konige o \textbf{28} hât] hette n (o)  $\cdot$ hulde] helde o \textbf{29} gesamet] Gesamht o  $\cdot$ got] [giht]: got o \newline
\end{minipage}
\end{table}
\newpage
\begin{table}[ht]
\begin{minipage}[t]{0.5\linewidth}
\small
\begin{center}*G
\end{center}
\begin{tabular}{rl}
 & \textit{immer} \textbf{sô} si vür ir swester gienc,\\ 
 & diu disen schimpf mit zorne enpfienc.\\ 
 & den rîtern, \textbf{den was} ruowe nôt,\\ 
 & wan \textbf{in} grôz müede daz gebôt.\\ 
5 & Tscherules nam Gawan\\ 
 & unt den grâven Lachdoman.\\ 
 & dannoch mêr rîter er dâ vant,\\ 
 & die Gawan mit sîner hant\\ 
 & des tages ûf dem velde vienc,\\ 
10 & dâ manic grôz hurt ergienc.\\ 
 & dô satzte si \textbf{rîterlîche}\\ 
 & der burcgrâve rîche.\\ 
 & \textbf{er} unde al sîn müediu schar\\ 
 & stuonden vor dem künige gar,\\ 
15 & unze Melianz enbeiz.\\ 
 & guoter handelunge er sich vleiz.\\ 
 & des dûhte Gawanen ze vil.\\ 
 & "obe \textbf{iuz} der künic erlouben wil,\\ 
 & hêr wirt, sô sult ir sitzen",\\ 
20 & sprach Gawan mit witzen.\\ 
 & sîn zuht in dar zuo jagte.\\ 
 & der wirt die bete ve\textit{r}sagte.\\ 
 & er sprach: "mîn hêrre ist des küniges man.\\ 
 & \textbf{disen} dienst het er getân,\\ 
25 & obe den künic des gezæme,\\ 
 & daz er sîn dienst næme.\\ 
 & mîn hêrre \textbf{sîn durch zuht niht siht},\\ 
 & wan er hât sîner hulde niht.\\ 
 & gesament die vriuntschaft imer got,\\ 
30 & \textbf{wir leisten} alle sîn gebot."\\ 
\end{tabular}
\scriptsize
\line(1,0){75} \newline
G I O L M Q R Z Fr28 \newline
\line(1,0){75} \newline
\textbf{3} \textit{Initiale} O L Z   $\cdot$ \textit{Capitulumzeichen} R  \textbf{17} \textit{Initiale} I  \newline
\line(1,0){75} \newline
\textbf{1} \textit{Die Verse 370.13-412.12 fehlen} Q   $\cdot$ immer] \textit{om.} G  $\cdot$ sô] wenne M swen Z (Fr28) \textbf{2} disen] disem R  $\cdot$ schimpf] [zorn]: shinph I  $\cdot$ enpfienc] vie I vntfe Fr28 \textbf{3} den rîtern] ÷en rittern O  $\cdot$ den was] was I da was O (L) M (Fr28) den was do R was da Z \textbf{4} groz muͤde in daz Gebot I \textbf{5} Tscherules] Scurles I Tschervles O Tsheruͯles L Scerules M Schurules R \textbf{6} Lachdoman] lahdoman G (O) (R) lohodeman I Lahtoman L Jahdoman Z \textbf{7} mêr rîter] riter mer O \textbf{8} Gawan] Gauwan L \textbf{10} dâ] Do O  $\cdot$ grôz hurt] herte I \textbf{11} dô] Da M Z  $\cdot$ satzte] [saite]: sazte O \textbf{13} unde] \textit{om.} R \textbf{14} dem] den I \textbf{15} unze] vnz daz I (O) (L) (R) (Z) biz Fr28  $\cdot$ Melianz] Melyanz O Meliancz R meliantz Z \textbf{16} vleiz] do vleiz O (R) da fleiz Z \textbf{17} Gawanen] Gawan I L (M) R (Z) gawane Fr28 \textbf{18} obe iuz] ob ez ev I (M) [Obv*]: Ob ichz  L \textbf{20} Gawan] her Gawan R \textbf{21} jagte] Jagt R \textbf{22} versagte] vesagte G versagt R \textbf{23} hêrre] \textit{om.} Z  $\cdot$ des] \textit{om.} O L M  $\cdot$ küniges] \textit{om.} R \textbf{24} disen] dusent Fr28  $\cdot$ het] hat L (M) \textbf{25} den künic] dem chvͦnige Fr28 \textbf{26} sîn] sinen L \textbf{27} sîn durch zuht] durch zcucht syn M (R) (Fr28) \textbf{29} gesament] Gesamlet R \textbf{30} sîn] sine Z \newline
\end{minipage}
\hspace{0.5cm}
\begin{minipage}[t]{0.5\linewidth}
\small
\begin{center}*T
\end{center}
\begin{tabular}{rl}
 & iemer \textbf{swenne} si vür ir swester gienc,\\ 
 & diu disen schimpf mit zorne enpfienc.\\ 
 & Den rîtern, \textbf{den was} ruowe nôt,\\ 
 & wand\textbf{in} grôz müede daz gebôt.\\ 
5 & Tscherules nam Gawan\\ 
 & unde den grâven Lachdoman.\\ 
 & dannoch mêr rîter er dâ vant,\\ 
 & die Gawan mit sîner hant\\ 
 & des tages ûf dem velde vienc,\\ 
10 & dâ manec grôz\textit{iu} hurt ergienc.\\ 
 & Dô sazte si \textbf{rîterlîche}\\ 
 & der burcgrâve rîche.\\ 
 & \textbf{er} unde alsîn müediu schar\\ 
 & stuo\textit{n}den vor dem künege gar,\\ 
15 & unze \textbf{daz} Melyanz enbeiz.\\ 
 & guoter handelunge er sich vleiz.\\ 
 & Des dûhte Gawanen ze vil.\\ 
 & "ob \textbf{ez iu} der künec erlouben wil,\\ 
 & hêr wirt, sô sult ir sitzen",\\ 
20 & sprach Gawan mit witzen.\\ 
 & Sîn zuht in dar zuo jagete.\\ 
 & der wirt die bete versagete.\\ 
 & er sprach: "mîn hêrre ist des küneges man.\\ 
 & \textbf{disen} dienst hât er getân,\\ 
25 & ob den künec des gezæme,\\ 
 & daz er sîn dienst næme.\\ 
 & mîn hêrre \textbf{sîn zuht giht},\\ 
 & wander hât sîner hulde niht.\\ 
 & gesament die vriuntschaft iemer got,\\ 
30 & \textbf{wir leisten} alle sîn gebot."\\ 
\end{tabular}
\scriptsize
\line(1,0){75} \newline
T V W \newline
\line(1,0){75} \newline
\textbf{3} \textit{Initiale} W   $\cdot$ \textit{Majuskel} T  \textbf{11} \textit{Majuskel} T  \textbf{17} \textit{Majuskel} T  \textbf{21} \textit{Majuskel} T  \newline
\line(1,0){75} \newline
\textbf{1} swenne] wan V so W \textbf{3} den was] waz do V was W \textbf{5} Tscherules] Schervles V \textbf{6} Lachdoman] lahedoman V lohdoman W \textbf{7} dâ] do V W \textbf{10} dâ] Do V W  $\cdot$ grôziu] groze T \textbf{14} stuonden] stvͦden T \textbf{15} Melyanz] melianz V W \textbf{17} Des] Das W  $\cdot$ Gawanen] gawan W \textbf{18} ez iu] eúchs W \textbf{23} küneges man] [*]: man V \textbf{24} [*]: Dienest hette er getan V \textbf{25} den künec] dem kv́nige V  $\cdot$ des] daz V \textbf{26} sîn] sinen V (W) \textbf{27} sîn zuht giht] sin durch zvht niht [*]: ensiht V durch zucht nicht sicht W \textbf{28} wander] Wann ern W \newline
\end{minipage}
\end{table}
\end{document}
