\documentclass[8pt,a4paper,notitlepage]{article}
\usepackage{fullpage}
\usepackage{ulem}
\usepackage{xltxtra}
\usepackage{datetime}
\renewcommand{\dateseparator}{.}
\dmyyyydate
\usepackage{fancyhdr}
\usepackage{ifthen}
\pagestyle{fancy}
\fancyhf{}
\renewcommand{\headrulewidth}{0pt}
\fancyfoot[L]{\ifthenelse{\value{page}=1}{\today, \currenttime{} Uhr}{}}
\begin{document}
\begin{table}[ht]
\begin{minipage}[t]{0.5\linewidth}
\small
\begin{center}*D
\end{center}
\begin{tabular}{rl}
\textbf{584} & \begin{large}Û\end{large}f \textbf{der} âventiure stein.\\ 
 & \textbf{solten} \textbf{dise} kumber sîn alein,\\ 
 & Gawans kumber slüege vür,\\ 
 & \textbf{wæge} \textbf{iemen} ungemaches kür.\\ 
5 & Welhen kumber mein ich nû?\\ 
 & ob \textbf{iuch des} d\textit{i}uhte niht ze vruo,\\ 
 & \textbf{ich solte}\textbf{n} iu benennen gar:\\ 
 & Orgeluse \textbf{kom} aldar\\ 
 & in Gawans herzen gedanc,\\ 
10 & der ie was zagheite kranc\\ 
 & unt gein \textbf{dem} wâren \textbf{ellen} starc.\\ 
 & wie kom, daz sich \textbf{dâ} verbarc\\ 
 & sô \textbf{grôz} wîp in \textbf{sô} \textbf{kleiner} stat?\\ 
 & si kom einen engen pfat\\ 
15 & in Gawans herze,\\ 
 & daz aller sîn smerze\\ 
 & von disem kumber gar verswant.\\ 
 & ez was iedoch ein \textbf{kurziu want},\\ 
 & dâ sô lanc wîp inne saz,\\ 
20 & der mit triwen \textbf{nie} vergaz\\ 
 & sîn dienstlîchez wachen.\\ 
 & Niemen \textbf{solte}\textbf{s} lachen,\\ 
 & daz \textbf{sus} werlîchen man\\ 
 & ein wîp enschumpfieren kan.\\ 
25 & \textbf{wohri} woch, waz sol \textbf{diz} sîn?\\ 
 & dâ tuot vrou Minne ir \textbf{zürnen} schîn\\ 
 & an dem, der prîs hât bejagt.\\ 
 & werlîche und unverzagt\\ 
 & hât si\textbf{n} \textbf{iedoch} vunden.\\ 
30 & gein \textbf{dem} siechen wunden\\ 
\end{tabular}
\scriptsize
\line(1,0){75} \newline
D Z \newline
\line(1,0){75} \newline
\textbf{1} \textit{Initiale} D Z  \textbf{5} \textit{Majuskel} D  \textbf{22} \textit{Majuskel} D  \textbf{27} \textit{Überschrift:} Hie hat her gawan den lewen erslagen vnd man hat sin wunden gebunden Vnd er senet sich aber nach siner ivncfrowen waz im nv gesche daz lis fvrbaz Z   $\cdot$ \textit{Großinitiale} Z  \newline
\line(1,0){75} \newline
\textbf{6} diuhte] dvhte D \textbf{20} nie] niht Z \textbf{21} wachen] wache Z \textbf{22} soltes] soldez Z \textbf{23} sus] alsus Z \textbf{25} wohri] Wohra Z  $\cdot$ sol diz] ditz sol Z \textbf{30} dem] den Z \newline
\end{minipage}
\hspace{0.5cm}
\begin{minipage}[t]{0.5\linewidth}
\small
\begin{center}*m
\end{center}
\begin{tabular}{rl}
 & ûf \textbf{der} âventiur stein.\\ 
 & \textbf{solten} \textbf{dise} kumber \textit{sîn} alein,\\ 
 & Gawans kumber slüege vür,\\ 
 & \textbf{wæge} \textbf{iema\textit{n}} \textit{un}gemach\textit{es} \textit{k}ür.\\ 
5 & welichen kumber mein ich nû?\\ 
 & ob \textbf{iuch daz} d\textit{i}uhte niht ze vruo,\\ 
 & \textbf{ich solte} \textbf{in} iu benennen gar:\\ 
 & Urgeluse \textbf{kæme} aldar\\ 
 & in Gawans herzegedanc,\\ 
10 & \dag die\dag  ie was zagheite kranc\\ 
 & und gegen \textbf{dem} wâren \textbf{ellen} starc.\\ 
 & wie kam, daz sich \textbf{d\textit{â}} verbarc\\ 
 & sô \textbf{grôz} wîp in \textbf{kleiner} stat?\\ 
 & si kam \textbf{in} einen engen pfat\\ 
15 & in Gawanes herze,\\ 
 & daz aller sîn smerze\\ 
 & von disem kumber gar verswant.\\ 
 & ez was iedoch ein \textbf{kurziu want},\\ 
 & dâ sô lanc wîp in saz,\\ 
20 & der mit triuwen \textbf{niht} vergaz\\ 
 & sîn dienstlîchez wachen.\\ 
 & nieman \textbf{solte} \textbf{es} lachen,\\ 
 & daz \textbf{alsus} werlîch\textit{en} man\\ 
 & ein wîp entschumpfieren kan.\\ 
25 & \textbf{w\textit{o}chri} woch, waz sol \textbf{diz} sîn?\\ 
 & d\textit{â} tuot vrouwe Minne ir \textbf{zürnen} schîn\\ 
 & an dem, der prîs het bejaget.\\ 
 & werlîch und u\textit{n}verzaget\\ 
 & hât si \textbf{in} \textbf{iedoch} vunden.\\ 
30 & geg\textit{e}n \textbf{dem} siechen wunden\\ 
\end{tabular}
\scriptsize
\line(1,0){75} \newline
m n o \newline
\line(1,0){75} \newline
\newline
\line(1,0){75} \newline
\textbf{1} âventiur stein] offentúre sin stein n afentursten o \textbf{2} sîn] \textit{om.} m \textbf{4} Wege ẏemans gemach fuͯr m \textbf{6} diuhte] duhtte m dúckte o \textbf{7} iu] aúch o \textbf{8} Urgeluse] [Vrlelu*]: Vrgeluse n Vrgeluͯse o  $\cdot$ kæme] kam n o \textbf{9} Gawans] [ga*]: gawans o  $\cdot$ herzegedanc] hercz gedang m (n) (o) \textbf{10} kranc] rang n \textbf{12} daz] das das n  $\cdot$ dâ] do m n o \textbf{13} sô grôz] Sogrose m (n) (o) \textbf{23} werlîchen] werlich m \textbf{24} ein] Sin o \textbf{25} wochri woch] Worchri woch m Wochri n  $\cdot$ diz] daz o \textbf{26} dâ] Do m n So o \textbf{28} unverzaget] vuerzaget m \textbf{29} hât] Hette n \textbf{30} gegen] Gegegen m \newline
\end{minipage}
\end{table}
\newpage
\begin{table}[ht]
\begin{minipage}[t]{0.5\linewidth}
\small
\begin{center}*G
\end{center}
\begin{tabular}{rl}
 & ûf \textbf{der} âventiure stein.\\ 
 & \textbf{suln} \textbf{dise} kumber \textit{sîn} alein,\\ 
 & Gawans kumber slüege vür\\ 
 & \textbf{jeneme} ungemaches kür.\\ 
5 & welhen kumber mein ich nû?\\ 
 & ob \textbf{es iuch} d\textit{i}uhte niht ze vruo,\\ 
 & \textbf{ich wolde} iu \textit{b}enennen gar\\ 
 & Orgeluse: \textbf{diu} \textbf{kom} al dar\\ 
 & in Gawans herzen gedanc,\\ 
10 & der ie was zageheit kranc\\ 
 & unde gein \textbf{dem} wâren \textbf{ellen} st\textit{ar}c.\\ 
 & wie kom, daz sich \textbf{dâ} verbarc\\ 
 & sô \textbf{lanc} wîp in \textbf{sô} \textbf{kurze} stat?\\ 
 & \multicolumn{1}{l}{ - - - }\\ 
15 & \multicolumn{1}{l}{ - - - }\\ 
 & \multicolumn{1}{l}{ - - - }\\ 
 & \multicolumn{1}{l}{ - - - }\\ 
 & ez was iedoch ein \textbf{engez pfat},\\ 
 & dâ sô lanc wîp inne saz,\\ 
20 & der mit triuwen \textbf{niht} vergaz\\ 
 & sî\textit{n} dienstlîche\textit{z} wachen.\\ 
 & niemen \textbf{sol} \textbf{des} lachen,\\ 
 & daz \textbf{alsus} werlîchen man\\ 
 & ein wîp enschumpfieren kan.\\ 
25 & \textbf{woch, wâ} woch, waz sol \textbf{daz} sîn?\\ 
 & dâ tuot vrou Minne ir \textbf{zorne} schîn\\ 
 & an dem, der prîs hât bejaget.\\ 
 & werlîch unde unverzaget\\ 
 & hât si \textbf{den helt} \textbf{sus} vunden.\\ 
30 & gein \textbf{den} siechen wunden\\ 
\end{tabular}
\scriptsize
\line(1,0){75} \newline
G I L M Fr19 \newline
\line(1,0){75} \newline
\textbf{1} \textit{Initiale} L Fr19  \textbf{13} \textit{Initiale} I  \newline
\line(1,0){75} \newline
\textbf{2} sîn] \textit{om.} G sy M \textbf{3} Gawans] Gawansz L \textbf{4} jeneme] ienen I Wider iemen L Wer yman M Wer iemens Fr19 \textbf{6} es iuch] uͯchsz L  $\cdot$ diuhte niht] duhte nihte G (M) (Fr19) niht dunchet I dvncke niht L  $\cdot$ ze] so L \textbf{7} ich wolde] So wolte ich L (M) (Fr19)  $\cdot$ iu benennen] iv nebenennen G nu benennen I uͯch benennen L en uch benennen M in iv benennen Fr19 \textbf{8} Orgeluse] Orglvs G Orguluse I Orgelýse L Orgillus M Orgillvs Fr19 \textbf{11} dem] der M (Fr19)  $\cdot$ starc] strac G \textbf{13} kurze] kuͯrzer L \textbf{14} \textit{Die Verse 584.14-17 fehlen} G I L M Fr19  \textbf{21} sines dienstlichen wachen G  $\cdot$ sin diensshlichez wachen I \textbf{22} sol des] sol ez L (M) (Fr19) \textbf{25} woch wâ] wohra I (L) (Fr19) Wocha M \textbf{26} Minne] libe M  $\cdot$ ir zorne] ir zurnens I ir zvrnen L (Fr19) iren M \textbf{29} sus] \textit{om.} I L  $\cdot$ vunden] wunden Fr19 \textbf{30} den siechen] dem siechem I  $\cdot$ wunden] fvnden Fr19 \newline
\end{minipage}
\hspace{0.5cm}
\begin{minipage}[t]{0.5\linewidth}
\small
\begin{center}*T
\end{center}
\begin{tabular}{rl}
 & \textit{ûf} âventiur\textit{e} steine.\\ 
 & \textbf{solden} \textbf{die} kumber sîn al eine,\\ 
 & Gawans kumber slüege vür,\\ 
 & \textbf{wæge} \textbf{ieman} ungemaches kür.\\ 
5 & welhen kumber mein ich nû?\\ 
 & ob \textbf{ez iuch} diuhte niht zuo vruo,\\ 
 & \textbf{sô wolt ich}\textbf{n} iu benennen gar:\\ 
 & Orgeluse, \textbf{diu} \textbf{kam} al dar\\ 
 & in Gawans herzege\textit{d}anc,\\ 
10 & der ie was zagheite kranc\\ 
 & und gên \textbf{den} wâren \textbf{êren} s\textit{t}arc.\\ 
 & wie kam, daz sich verbarc\\ 
 & sô \textbf{grôz} wîp in \textbf{sô} \textbf{kleiner} stat?\\ 
 & s\textit{i} kam \textbf{in} einen engen pfat\\ 
15 & in Gawanes herze,\\ 
 & daz aller sîn smerze\\ 
 & von disem kumber gar verswant.\\ 
 & ez was iedoch ein \textbf{kurziu want},\\ 
 & dâ sô lanc wîp inne saz,\\ 
20 & der mit triuwen \textbf{niht} vergaz\\ 
 & sîn dienstlîch\textit{e}z wachen.\\ 
 & nieman \textbf{sol} \textbf{es} lachen,\\ 
 & daz \textbf{alsus} werlîchen man\\ 
 & ein wîp enschumpfieren kan.\\ 
25 & \textbf{wolra} woch, waz sol \textbf{di\textit{z}} sîn?\\ 
 & d\textit{â} tuot vrou Minne ir \textbf{zorne} schîn\\ 
 & an dem, der prîs hât bejagt.\\ 
 & werlîch und unverzagt\\ 
 & hât si \textbf{in} \textbf{doch} vunden.\\ 
30 & gên \textbf{den} siechen wunden\\ 
\end{tabular}
\scriptsize
\line(1,0){75} \newline
Q R W V U \newline
\line(1,0){75} \newline
\textbf{1} \textit{Capitulumzeichen} R  \newline
\line(1,0){75} \newline
\textbf{1} \textit{Die Verse 553.1-599.30 fehlen} U   $\cdot$ ûf âventiure] Von awentewres Q Vff der auentᵫre R (W) (V) \textbf{2} die] disu R (V) disen W \textbf{3} Gawans] Gawins R \textbf{4} ungemaches] vngemach R vngeleiche W \textbf{6} diuhte] duchtte R (V) \textbf{7} ichn] ich R V \textbf{8} Orgeluse] Orgelusse Q Orguluse R Orgeluse W [Orgel*]: Orgelvse V  $\cdot$ diu] \textit{om.} W V \textbf{9} Gawans] Gawins R  $\cdot$ herzegedanc] hertze gedranc Q herczen gedank R (W) (V) \textbf{10} der] Die W \textbf{11} gên] \textit{om.} W  $\cdot$ den wâren êren] dem wauren ellend R (W) (V)  $\cdot$ starc] strarck Q \textbf{12} sich] sich [der]: do V \textbf{13} sô kleiner] cleine R \textbf{14} si] So Q  $\cdot$ kam in] kam R W [k*]: kam in V \textbf{15} Gawanes] Gawans R (W) \textbf{16} sîn smerze] siner Schercze R \textbf{18} ez] Er R  $\cdot$ kurziu] kurcze R \textbf{20} mit triuwen niht] nicht mit treúwen W \textbf{21} dienstlîchez] dinstlichens Q  $\cdot$ wachen] lachen R \textbf{22} sol] solt V  $\cdot$ es] des R \textbf{25} wolra] Wochra R W Wora V  $\cdot$ diz] disse Q das R \textbf{26} dâ] Do Q W V So R  $\cdot$ zorne] zúrnen R W (V) \textbf{27} bejagt] beiagte R \textbf{28} unverzagt] vnuerczagte R \textbf{29} hât si in] [Ha* *n]: Hat sv́ in V  $\cdot$ doch] yedoch R (W) (V) \textbf{30} den] dem R [de*]: dem V \newline
\end{minipage}
\end{table}
\end{document}
