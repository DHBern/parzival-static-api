\documentclass[8pt,a4paper,notitlepage]{article}
\usepackage{fullpage}
\usepackage{ulem}
\usepackage{xltxtra}
\usepackage{datetime}
\renewcommand{\dateseparator}{.}
\dmyyyydate
\usepackage{fancyhdr}
\usepackage{ifthen}
\pagestyle{fancy}
\fancyhf{}
\renewcommand{\headrulewidth}{0pt}
\fancyfoot[L]{\ifthenelse{\value{page}=1}{\today, \currenttime{} Uhr}{}}
\begin{document}
\begin{table}[ht]
\begin{minipage}[t]{0.5\linewidth}
\small
\begin{center}*D
\end{center}
\begin{tabular}{rl}
\textbf{86} & \textbf{dem dû} vor Patelamunt\\ 
 & tæte ouch fîanze kunt.\\ 
 & des gert dîn prîs an strîte\\ 
 & der hœhe unt \textbf{ouch} der wîte."\\ 
5 & \textbf{Er sprach}: "mîn vrouwe mac wænen, daz dû tobst,\\ 
 & \textbf{sît} dû mich \textbf{alsô} \textbf{verlobst}.\\ 
 & dû\textbf{ne} maht \textbf{mîn doch} verkoufen niht,\\ 
 & wande etswer wandel an mir siht.\\ 
 & \textbf{dîn munt} ist \textbf{lobes ze vil} vernomen.\\ 
10 & sag \textbf{êt}, wie bistû widerkomen?"\\ 
 & "\begin{large}D\end{large}iu werde diet von Punturteys\\ 
 & \textbf{hât} mich unt disen Schamponeys\\ 
 & \textbf{ledic lâzen} überal.\\ 
 & Morholt, der mînen neven stal,\\ 
15 & von de\textit{m} sol er ledic sîn,\\ 
 & mac mîn hêr Brandelidelin\\ 
 & \textbf{ledic sîn} von dîner hant.\\ 
 & wir sîn noch \textbf{anders beide} pfant,\\ 
 & ich unt mîner \textbf{swester} sun.\\ 
20 & dû solt an uns genâde tuon.\\ 
 & \textbf{ein vesperîe ist hie} erliten.\\ 
 & daz turnieren wirt vermiten\\ 
 & an dirre zît vo\textit{r} Kanvoleiz.\\ 
 & die rehten wârheit ich des weiz,\\ 
25 & \textbf{wan diu ûzer} sitzet hie.\\ 
 & nû \textbf{sprich} \textbf{êt}, wâ \textbf{von} oder wie\\ 
 & \textbf{mohten si uns} vor gehalden?\\ 
 & dû \textbf{muost} vil prîses walden."\\ 
 & Diu künegîn sprach ze Gahmurete\\ 
30 & \textbf{von herzen eine süeze} bete:\\ 
\end{tabular}
\scriptsize
\line(1,0){75} \newline
D \newline
\line(1,0){75} \newline
\textbf{5} \textit{Majuskel} D  \textbf{11} \textit{Initiale} D  \textbf{29} \textit{Majuskel} D  \newline
\line(1,0){75} \newline
\textbf{1} Patelamunt] Pantelamvnt D \textbf{11} Punturteys] Pvntvrteẏs D \textbf{12} Schamponeys] Scamponeẏs D \textbf{15} dem] den D \textbf{23} vor] von D \textbf{29} Gahmurete] Gahmvrete D \newline
\end{minipage}
\hspace{0.5cm}
\begin{minipage}[t]{0.5\linewidth}
\small
\begin{center}*m
\end{center}
\begin{tabular}{rl}
 & \textbf{dem dû} vor P\textit{a}telamu\textit{n}t\\ 
 & tæte ouch fîanze kunt.\\ 
 & de\textit{s} gerte dîn prîs an strîte\\ 
 & der hœhe und \textbf{ouch} der wîte."\\ 
 & \hspace*{-.7em}\big| \textbf{er sprach}: "\textbf{sît} dû mich \textbf{alsô} \textbf{lobest},\\ 
5 & \hspace*{-.7em}\big| mîn vrouwe mac wænen, da\textit{z} dû tobest.\\ 
 & dû maht \textbf{mich doch} verkoufen niht,\\ 
 & wanne etwer wandel an mir \textit{s}iht.\\ 
 & \textbf{dînem munde} ist \textbf{lobes vil} vernomen.\\ 
10 & sage \textbf{eht}, wie bistû widerkomen?"\\ 
 & "\begin{large}D\end{large}iu werde diet von Ponturteis\\ 
 & \textbf{hât} mich und dise\textit{n} Schamponeis\\ 
 & \textbf{ledic lâzen} überal.\\ 
 & Morolt, der mînen neven stal,\\ 
15 & von dem sol er ledic sîn,\\ 
 & mac mîn hêrre Brandelid\textit{el}in\\ 
 & \textbf{ledic sîn} von dîner hant.\\ 
 & wir sî\textit{n} noch \textbf{anders beide} pfant,\\ 
 & ich und mîner \textbf{swester} sun.\\ 
20 & dû solt an uns gnâde tuon.\\ 
 & \textbf{ein vesperîe ist hie} erliten.\\ 
 & daz turnieren wirt vermiten\\ 
 & an dirre zît vor Kanvoleiz.\\ 
 & die rehten wârheit ich des weiz,\\ 
25 & \textbf{wand diu ûzere herte} sitzet hie.\\ 
 & nû \textbf{sprich} \textbf{eht}, wâ oder wie\\ 
 & \textbf{mohtens uns} vor gehalten?\\ 
 & dû \textbf{maht} vil prîse\textit{s} walten."\\ 
 & diu künigîn sprach zuo Gahmurete:\\ 
30 & "\textbf{vernemet, hêrre, mîne} bete:\\ 
\end{tabular}
\scriptsize
\line(1,0){75} \newline
m n o \newline
\line(1,0){75} \newline
\textbf{11} \textit{Initiale} m o   $\cdot$ \textit{Capitulumzeichen} n  \newline
\line(1,0){75} \newline
\textbf{1} Patelamunt] pantelamumt m panthalamunt n pantalamunt o \textbf{2} tæte] Die o \textbf{3} des] Der m Das o \textbf{4} hœhe] hoͯwe o  $\cdot$ der wîte] [die]: der wit o \textbf{5} daz] da m \textbf{7} maht] [mlcht]: mecht o \textbf{8} etwer] ietweder n  $\cdot$ siht] sciht m \textbf{9} vil] zuͦ vil n o \textbf{10} bistû] bist o \textbf{11} Ponturteis] ponnturteis m pantuͯrteis o \textbf{12} disen] disem m  $\cdot$ Schamponeis] scamponeis m n o \textbf{16} Brandelidelin] brandelidin m brandilin n brandeli::: o \textbf{18} sîn] sint m n o \textbf{22} wirt] ist hút n \textbf{23} Kanvoleiz] kanuoleis m kanfoleis n o \textbf{24} rehten] rechte n (o)  $\cdot$ des] das o \textbf{25} herte sitzet] herten siczen o \textbf{27} mohtens] Moͯchtens m n  $\cdot$ uns] vns auch o \textbf{28} maht] mach o  $\cdot$ prîses] prisset m \textbf{29} Gahmurete] gahmurette m gamiret n gamuͯrete o \newline
\end{minipage}
\end{table}
\newpage
\begin{table}[ht]
\begin{minipage}[t]{0.5\linewidth}
\small
\begin{center}*G
\end{center}
\begin{tabular}{rl}
 & \textbf{dem dû} vor Patelamunt\\ 
 & tæte ouch fîanze kunt.\\ 
 & des gert dîn prîs an strîte\\ 
 & der hœhe und \textbf{ouch} der wîte."\\ 
5 & "\begin{large}M\end{large}în vrouwe mac wænen, daz dû tobest,\\ 
 & \textbf{sît} dû mich \textbf{alsus} \textbf{vor ir lobest}.\\ 
 & dû\textbf{ne} maht \textbf{mîn doch} verkoufen niht,\\ 
 & wan etswer wandel an mir siht.\\ 
 & \textbf{dîn munt} ist \textbf{lobes ze vil} vernomen.\\ 
10 & sage \textbf{an}, wie bistû widerkomen?"\\ 
 & "diu werde diet von Ponturteis\\ 
 & \textbf{hânt} mich und disen Tschanponeis\\ 
 & \textbf{lâzen ledic} überal.\\ 
 & Morolt, der mînen neven stal,\\ 
15 & von dem sol er ledic sîn,\\ 
 & mac mîn hêr Brandelidelin\\ 
 & \textbf{werden ledic} von dîner hant.\\ 
 & wir sîn noch \textbf{anders bêde} pfant,\\ 
 & ich und mîner \textbf{swester} sun.\\ 
20 & dû solt an uns genâde tuon.\\ 
 & \textbf{hie ist ein vesperîe} erliten.\\ 
 & daz turnieren wirt vermiten\\ 
 & an dirre zît vor Kanvoleiz.\\ 
 & die rehten wârheit ich des weiz,\\ 
25 & \textbf{gar dû\textit{z}er herte} sitzet hie.\\ 
 & nû \textbf{s\textit{prich}} \textbf{êt}, wâ oder wie\\ 
 & \textbf{si uns mohten} vor gehalten.\\ 
 & dû \textbf{muost} vil brîses walten."\\ 
 & diu künigîn sprach ze Gahmurete\\ 
30 & \textbf{von herzen eine süeze} bete:\\ 
\end{tabular}
\scriptsize
\line(1,0){75} \newline
G I O L M Q R Z \newline
\line(1,0){75} \newline
\textbf{1} \textit{Initiale} O  \textbf{5} \textit{Initiale} G I R Z   $\cdot$ \textit{Capitulumzeichen} L  \textbf{25} \textit{Initiale} I  \textbf{29} \textit{Initiale} L  \newline
\line(1,0){75} \newline
\textbf{1} dem] den I (M) ÷em O  $\cdot$ Patelamunt] patelamút Q [pandele*]: pandelemunt R \textbf{2} ouch] \textit{om.} I  $\cdot$ fîanze] sicherheit R \textbf{3} dîn] den I \textbf{4} hœhe] hoch I  $\cdot$ ouch] \textit{om.} I O L Q R \textbf{5} Mîn] ÷in I DJn R  $\cdot$ mac] \textit{om.} Q  $\cdot$ wænen daz] wenet Z \textbf{6} sît] daz I (L) (M) (Z)  $\cdot$ alsus vor ir] vor ir so I svs also O sus vor ir L (M) (Q) also vor ir R also Z  $\cdot$ lobest] bobst Q \textbf{7} dûne] Dv O (L) (Q) (R)  $\cdot$ mîn] myr M mich R \textbf{8} etswer] ettwar R  $\cdot$ wandel] wande et I  $\cdot$ siht] geschiht L \textbf{9} dîn munt] dinem munde I Von dir L \textbf{10} an] \textit{om.} O L M Q R Z  $\cdot$ wie] etwie O (R) \textbf{11} Ponturteis] ponturtoys I Pvntratais L ponturteis M pűnturteis Q puͦntuͦrteis R purturteis Z \textbf{12} \textit{Vers 86.12 fehlt} Q   $\cdot$ hânt] hat I (L) (R)  $\cdot$ disen] den O L M dise Z  $\cdot$ Tschanponeis] schanponois I tschamponeis O Tschampanais L scanponeis M schampnanis R \textbf{13} lâzen ledic] Gelaszen ledig L (M) Ledig gelazen R \textbf{14} Morolt] morholt I (O) (L) (M) (Z) Mocholt R \textbf{16} mac] Man Z  $\cdot$ mîn] nin I  $\cdot$ Brandelidelin] [brandelidelidelin]: Brandelidelin G brandalidelin I Brandlidelin O (Q) Brantlidelin L blandelidelin Z \textbf{17} dîner] meiner Q \textbf{18} sîn] sind R \textit{om.} Z \textbf{22} daz] Dar Z  $\cdot$ wirt] werde M \textbf{23} dirre] der M  $\cdot$ vor] von M  $\cdot$ Kanvoleiz] kanfolaiz I canvoleiz O kanvolaiz L kanvoleis M R kanúoleis Q kampfoleis Z \textbf{24} des] da R \textbf{25} gar dûzer] gar dvz der G Harduz I Die vszer L Gar ez [v*]: vzer M Wann die vnszer Q Wan v́nsser vnsz R Wan die vzzer Z  $\cdot$ herte] herre I M (R)  $\cdot$ sitzet] div ist O sitzet gar L \textbf{26} nû] \textit{om.} Z  $\cdot$ sprich] sage G sich I Sprach M  $\cdot$ êt wâ oder] er wa von vnde I etwa [od]: von od O her wa von vnde M wo von [a*er]: aber Q echt wa von oder R wa von oder Z \textbf{27} uns] músz Q an an vns Z  $\cdot$ mohten vor] vor mohten O von mochten Q mohten Z  $\cdot$ gehalten] halten O (Q) verhalden Z \textbf{28} vil] michels I \textbf{29} Gahmurete] Gamvret O Gahmuͯret L gamurete M gamúret Q gamuret Z \textbf{30} herzen] hertze Q \newline
\end{minipage}
\hspace{0.5cm}
\begin{minipage}[t]{0.5\linewidth}
\small
\begin{center}*T (U)
\end{center}
\begin{tabular}{rl}
 & \textbf{der dir} vor Patelamunt\\ 
 & tæt ouch fîanze kunt.\\ 
 & des gerte dîn prîs an strîte\\ 
 & der hœhe und der wîte."\\ 
5 & "\begin{large}M\end{large}în vrouwe mac wænen, daz dû tobest,\\ 
 & \textbf{daz} \textit{dû} mich \textbf{alsus} \textbf{vor ir lobest}.\\ 
 & dû maht \textbf{doch mich} verkoufen niht,\\ 
 & wan etswer wan\textit{del} an mir siht.\\ 
 & \textbf{vo\textit{n}} \textbf{dînem munt} ist \textbf{zuo vil lobes} vernomen.\\ 
10 & sage \textbf{an}, wie bistû widerkomen?"\\ 
 & "diu werde d\textit{ie}t von Puntertoys\\ 
 & \textbf{hât} mich und disen Schapenoys\\ 
 & \textbf{verlâzen ledic} überal.\\ 
 & Morolt, der mî\textit{n}en neve\textit{n} stal,\\ 
15 & von dem sol er ledic sîn,\\ 
 & mac mîn hêrre Brandelidelin\\ 
 & \textbf{werden ledic} von dîner hant.\\ 
 & wir sîn noch \textbf{beide dar umb} pfant,\\ 
 & ich und mîner \textbf{muomen} suon.\\ 
20 & dû solt an uns gnâde tuon.\\ 
 & \textbf{hie ist ein vesperîe} erliten.\\ 
 & daz turnieren wirt vermiten\\ 
 & an dirre zît vor Kanvoleiz.\\ 
 & die rehte wârheit ich des weiz,\\ 
25 & \textbf{Hardiz, der hêrre}, sitzet hie.\\ 
 & nû \textbf{spr\textit{e}chet}, wâ oder wie\\ 
 & \textbf{si uns mohten} vor gehalten.\\ 
 & dû \textbf{muost} vil prîses walten."\\ 
 & \begin{large}D\end{large}iu künigîn sprach zuo Gahmurete\\ 
30 & \textbf{von her\textit{z}en eine süeze} bete:\\ 
\end{tabular}
\scriptsize
\line(1,0){75} \newline
U V W T \newline
\line(1,0){75} \newline
\textbf{5} \textit{Initiale} U V W   $\cdot$ \textit{Majuskel} T  \textbf{11} \textit{Majuskel} T  \textbf{19} \textit{Majuskel} T  \textbf{29} \textit{Initiale} U V T  \newline
\line(1,0){75} \newline
\textbf{1} der dir] Den du W dem div T  $\cdot$ Patelamunt] Patelamuͦnt U patelemunt W \textbf{2} ouch] \textit{om.} W \textbf{3} gerte] gert V W T \textbf{4} und] vnd oͮch V (W) \textbf{5} daz] \textit{om.} W \textbf{6} dû] \textit{om.} U  $\cdot$ alsus vor ir] vor ir alsvs T \textbf{7} doch mich] min doch V mich doch W T  $\cdot$ verkoufen] verkanfen W \textbf{8} etswer] man T  $\cdot$ wandel] wan U \textbf{9} Deinem munde ist zuͦuil lobe kund W  $\cdot$ von] Vor U  $\cdot$ zuo vil lobes] vil lobes V lobes vil T \textbf{10} Das han ich vernomen an diser stund W  $\cdot$ an] eht an V eht T \textbf{11} diu] die T  $\cdot$ diet] duͦt U die T  $\cdot$ Puntertoys] Puͦntercos U Puntertoẏs V ponterteis W pvntertôys T \textbf{12} hât] hant V (W)  $\cdot$ disen] \textit{om.} W  $\cdot$ Schapenoys] scampenos U Schamponoẏs V schampaneis W Scapenêys T \textbf{13} verlâzen] gelâzen T \textbf{14} Morolt] Morholt W  $\cdot$ mînen neven] minnen neue U \textbf{15} sol] roße W \textbf{16} mac] \textit{om.} W \textbf{17} dîner] seiner W \textbf{18} dar umb] \textit{om.} W anders T \textbf{19} muomen] swester T \textbf{22} turnieren] der turnei V \textbf{23} Kanvoleiz] kanvoleis U Kanuoleis V (W) \textbf{24} rehte] rehten V (W) T \textbf{25} [*]: Wande die usser herte sittzet hie V  $\cdot$ Hardiz] Hardis W  $\cdot$ der hêrre] de rehte T \textbf{26} sprechet] sprichet U W sprenchent V sprich T  $\cdot$ wâ] wo von W \textbf{27} mohten] moͤhten V (W) \textbf{28} dû] Das W \textbf{29} sprach] tet T  $\cdot$ Gahmurete] Gahmuͦret U Gamurette V gamuret W \textbf{30} herzen] herren U \newline
\end{minipage}
\end{table}
\end{document}
