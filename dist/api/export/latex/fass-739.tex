\documentclass[8pt,a4paper,notitlepage]{article}
\usepackage{fullpage}
\usepackage{ulem}
\usepackage{xltxtra}
\usepackage{datetime}
\renewcommand{\dateseparator}{.}
\dmyyyydate
\usepackage{fancyhdr}
\usepackage{ifthen}
\pagestyle{fancy}
\fancyhf{}
\renewcommand{\headrulewidth}{0pt}
\fancyfoot[L]{\ifthenelse{\value{page}=1}{\today, \currenttime{} Uhr}{}}
\begin{document}
\begin{table}[ht]
\begin{minipage}[t]{0.5\linewidth}
\small
\begin{center}*D
\end{center}
\begin{tabular}{rl}
\textbf{739} & unt gein der tjost geschicket\\ 
 & unt \textbf{d}ors mit \textbf{sporn} gezwicket.\\ 
 & \begin{large}H\end{large}ie wart \textbf{diu} tjost alsô geriten,\\ 
 & bêdiu kollier versniten\\ 
5 & \textbf{von} starken spern, diu sich niht bugen.\\ 
 & die sprîzen von der tjoste vlugen.\\ 
 & ez het der heiden gar vür haz,\\ 
 & daz dirre man vor im gesaz,\\ 
 & wan des nieman vor im gepflac,\\ 
10 & gein dem er strîtes sich bewac.\\ 
 & Ob si \textbf{iht swerte} vuorten,\\ 
 & \textbf{dâ} si zein ander ruorten?\\ 
 & diu wâren \textbf{dâ} scharpf \textbf{unt} \textbf{al} \textbf{bereit}.\\ 
 & ir kunst unt ir manheit\\ 
15 & wart dâ erzeiget schiere.\\ 
 & ecidemôn, dem tiere,\\ 
 & wart etslîch wunde geslagen,\\ 
 & ez mohte der helm dâr under klagen.\\ 
 & diu ors \textbf{vor} müede wurden heiz,\\ 
20 & si \textbf{versuochten} manegen niuwen \textbf{kreiz}.\\ 
 & Si bêde ab orsen sprungen;\\ 
 & alrêst diu swert erklungen.\\ 
 & der heiden tet dem getouftem wê.\\ 
 & des krîe was 'Thasme'\\ 
25 & unt swenn er schrîte Thabronit,\\ 
 & sô trat er \textbf{vürbaz} einen trit.\\ 
 & werlîch was der getoufte\\ 
 & ûf \textbf{manegem} \textbf{drætem} loufte,\\ 
 & den si zein ander tâten.\\ 
30 & ir strît was sô gerâten,\\ 
\end{tabular}
\scriptsize
\line(1,0){75} \newline
D \newline
\line(1,0){75} \newline
\textbf{3} \textit{Initiale} D  \textbf{11} \textit{Majuskel} D  \textbf{21} \textit{Majuskel} D  \newline
\line(1,0){75} \newline
\newline
\end{minipage}
\hspace{0.5cm}
\begin{minipage}[t]{0.5\linewidth}
\small
\begin{center}*m
\end{center}
\begin{tabular}{rl}
 & und gegen der juste geschicket\\ 
 & und \textbf{daz} ros mit \textbf{zorn} gezwicket.\\ 
 & hie wart juste alsô geriten,\\ 
 & bei\textit{diu} kollier versniten\\ 
5 & \textbf{mit} starken spern, diu sich niht bugen.\\ 
 & die \textit{s}prîzen von der juste vlugen.\\ 
 & ez het de\textit{r} heiden gar vür haz,\\ 
 & daz diser man vor im ge\textit{s}a\textit{z},\\ 
 & wan des niemen vor im gepflac,\\ 
10 & gegen dem er strîtes sich bewac.\\ 
 & ob si \textbf{iht swerte} vuorten,\\ 
 & \textbf{dô} si zuo ein ander ruorten?\\ 
 & diu wâren scharpf, \textbf{vast} \textbf{und} \textbf{breit}.\\ 
 & ir kunst und ir manheit\\ 
15 & wart dô erzöuget schier.\\ 
 & ecidemôn, dem tier,\\ 
 & wart etlîch wunde geslagen,\\ 
 & ez mohte der helm dâr under klagen.\\ 
 & diu ros \textbf{vor} müede wurden heiz,\\ 
20 & si \textbf{versuochten} manigen niuwen \textbf{kreiz}.\\ 
 & si beide ab rossen sprungen;\\ 
 & allerêrst diu swert \textit{er}klungen.\\ 
 & der heiden tet dem getouften wê.\\ 
 & des \textit{krî}e was 'Thasme'\\ 
25 & und wan er schrît Tabronit,\\ 
 & sô trat er \textbf{vürbaz} einen trit.\\ 
 & werlîch was der getouft\\ 
 & ûf \textbf{manige\textit{m}} \textbf{\textit{d}ræten} louft,\\ 
 & de\textit{n} si zuo ein ander tâten.\\ 
30 & ir strît was sô gerâten,\\ 
\end{tabular}
\scriptsize
\line(1,0){75} \newline
m n o V V' Fr69 \newline
\line(1,0){75} \newline
\newline
\line(1,0){75} \newline
\textbf{1} \textit{Die Verse 738.11-739.3 fehlen} o   $\cdot$ \textit{Die Verse 738.15-739.2 fehlen} V'  \textbf{2} daz] die V  $\cdot$ zorn] sporn n V \textbf{3} \textit{statt 739.3-6:} Eine starke iust die was gros / Daz man an in beiden sie kos V'   $\cdot$ juste] die juste n (V)  $\cdot$ geriten] gestritten n \textbf{4} beidiu] [B*]: Beẏ m \textbf{6} sprîzen] priessen m  $\cdot$ vlugen] fliegen n \textbf{7} ez het] Er het n Ouch hette ez V'  $\cdot$ der] den m n o V' \textbf{8} gesaz] genas m \textbf{9} des] es V ez V'  $\cdot$ vor] von o \textbf{11} Dar nach sie die swert zucten V' \textbf{12} Vnd vaste zv einander ructen V' \textbf{13} \textit{Die Verse 739.13-18 fehlen} V'   $\cdot$ scharpf vast und breit] so scharpf [vn*ereit]: vnde albereit V \textbf{15} erzöuget] er zeiget n (V) \textbf{18} mohte] móchte n moͤhte V \textbf{20} si] Die o  $\cdot$ niuwen] \textit{om.} V' \textbf{21} Dar nach sie uon den orsen sprungen V'  $\cdot$ rossen] den rossen n (V) \textbf{22} erklungen] klungen m \textbf{24} krîe] tiere m  $\cdot$ was] die waz V V' (Fr69)  $\cdot$ Thasme] thasine n o tasme V' \textbf{25} wan] swenne V  $\cdot$ schrît] schriete V V'  $\cdot$ Tabronit] thabronit n o [tobronit]: tabronit V' \textbf{26} trit] strit o \textbf{27} \textit{Die Verse 739.27-30 fehlen} V'  \textbf{28} manigem dræten] mangen stretten m manigem stretem n mangem streten o [manige*]: manigem dretem V \textbf{29} den] Des m \newline
\end{minipage}
\end{table}
\newpage
\begin{table}[ht]
\begin{minipage}[t]{0.5\linewidth}
\small
\begin{center}*G
\end{center}
\begin{tabular}{rl}
 & \begin{large}U\end{large}nde gein der tjost geschicket\\ 
 & unde \textbf{diu} ors mit \textbf{sporn} gezwicket.\\ 
 & hie wart \textbf{diu} tjost alsô geriten,\\ 
 & beidiu kollier versniten\\ 
5 & \textbf{mit} starken spern, diu sich niht bugen.\\ 
 & die sprîzen von der tjost vlugen.\\ 
 & ez hete der heiden gar vür haz,\\ 
 & daz dirre man vor im gesaz,\\ 
 & wan des niemen vor im gepflac,\\ 
10 & gein dem er strîtes sich bewac.\\ 
 & op si \textbf{diu swert iht} vuorten,\\ 
 & \textbf{daz} si zein ander ruorten?\\ 
 & diu wâren \textbf{dâ} scharpf \textbf{und} \textbf{al} \textbf{bereit}.\\ 
 & ir kunst unde ir manheit\\ 
15 & wart dâ erzeiget schiere.\\ 
 & ecidemôn, dem tiere,\\ 
 & wart etslîch wunde \textbf{dâ} geslagen,\\ 
 & ez moht der helm drunder klagen.\\ 
 & diu ors \textbf{von} müede wurden heiz,\\ 
20 & si \textbf{liezen} manigen niwen \textbf{sweiz}.\\ 
 & si bêde abe orsen sprungen;\\ 
 & alrêrst diu swert erklungen.\\ 
 & der heiden tet dem getouften wê.\\ 
 & des krîe was 'Tasme'\\ 
25 & unde swenne er schrîte Tabrunit,\\ 
 & sô trat er \textbf{vür sich} einen trit.\\ 
 & werlîch was der getoufte,\\ 
 & ûf \textbf{manigen} \textbf{trit er} loufte,\\ 
 & den si zein ander tâten.\\ 
30 & \begin{large}I\end{large}r strît was sô gerâten,\\ 
\end{tabular}
\scriptsize
\line(1,0){75} \newline
G I L M Z Fr24 \newline
\line(1,0){75} \newline
\textbf{1} \textit{Initiale} G L Z  \textbf{3} \textit{Initiale} Fr24  \textbf{11} \textit{Initiale} I  \textbf{27} \textit{Initiale} M  \textbf{30} \textit{Initiale} G Z  \newline
\line(1,0){75} \newline
\textbf{1} geschicket] Gesswichet I \textbf{2} sporn] den sporn I \textbf{3} alsô] mit sporn L \textit{om.} Z \textbf{6} sprîzen] splittern M  $\cdot$ der tjost] den spern I \textbf{7} haz] einen haz I \textbf{9} des] das M  $\cdot$ gepflac] phlac M \textbf{10} gein] \sout{wan der} Gein I Uon Fr24 \textbf{12} daz] do I Da M Fr24  $\cdot$ zein ander] ein ander I \textbf{13} dâ] \textit{om.} L  $\cdot$ al bereit] breit L albreit Z \textbf{14} ir manheit] die manheit Z \textbf{15} erzeiget] [erzeiget]: erzeigen Z  $\cdot$ schiere] schere M \textbf{16} ecidemôn] ezidom I \textbf{18} der] dy M \textbf{19} von] vor L M Fr24 \textbf{20} liezen] svchten L (Fr24) versuchten M (Z)  $\cdot$ sweiz] leiz L creisz M (Z) \textbf{21} abe] von I (Fr24) abe den L von deme M  $\cdot$ orsen] orse I (M) \textbf{23} heiden] heide M  $\cdot$ getouften] Getauftem I kristen L \textbf{24} krîe] schrey M  $\cdot$ Tasme] thasme G M Z (Fr24) Talme L \textbf{25} swenne] wenne L (M) Z  $\cdot$ schrîte] strite I  $\cdot$ Tabrunit] Tanprunit I tabruͯnit M Tabr::: Fr24 \textbf{26} vür sich] vorbasz M (Z)  $\cdot$ einen trit] eyn treit M \textbf{27} der] der der I \textbf{28} manigen trit] mangem trit I (M) manigem dreten Z  $\cdot$ er loufte] vrlaufte I lovfte L (M) Z \textbf{29} den] Die Fr24  $\cdot$ zein ander] zv einer Z \newline
\end{minipage}
\hspace{0.5cm}
\begin{minipage}[t]{0.5\linewidth}
\small
\begin{center}*T
\end{center}
\begin{tabular}{rl}
 & und gein der jost geschicket\\ 
 & und \textbf{diu} ors mit \textbf{sporn} gezwicket.\\ 
 & hie wart \textbf{diu} jost alsô geriten,\\ 
 & beidiu kollier versniten\\ 
5 & \textbf{mit} starken spern, diu sich niht bugen.\\ 
 & die sprîzen von der jost vlugen.\\ 
 & \begin{large}E\end{large}z hete der heiden gar vür haz,\\ 
 & daz dirre man vor im gesaz,\\ 
 & wan des nieman vor i\textit{m} gepflac,\\ 
10 & gein dem er strîtes sich bewac.\\ 
 & o\textit{b} si \textbf{diu swerte iht} vuorten,\\ 
 & \textbf{dô} si zuo ein ander ruorten?\\ 
 & diu wâren \textbf{dô} scharpf \textbf{und} \textbf{alle} \textbf{breit}.\\ 
 & ir kunst und ir manheit\\ 
15 & wart dô erzeiget schiere.\\ 
 & ecidemôn, dem tiere,\\ 
 & wart etslîche wunde \textbf{dô} geslagen,\\ 
 & ez mohte der helm dâr under klagen.\\ 
 & diu ors \textbf{von} müede wurden heiz,\\ 
20 & si \textbf{versuocheten} manegen niuwen \textbf{kreiz}.\\ 
 & si beide ab orsen sprungen;\\ 
 & alrêrst diu swert erklungen.\\ 
 & der heiden tet dem getouften wê.\\ 
 & des krîe was 'Thasme'\\ 
25 & und wan er schrîte Tabrunit,\\ 
 & sô trat er \textbf{vürbaz} einen trit.\\ 
 & werlîch was der getoufte\\ 
 & ûf \textbf{manegem} \textbf{dræten} loufte,\\ 
 & den si zuo ein ander tâten.\\ 
30 & ir strît was sô gerâten,\\ 
\end{tabular}
\scriptsize
\line(1,0){75} \newline
U W Q R \newline
\line(1,0){75} \newline
\textbf{3} \textit{Initiale} R  \textbf{7} \textit{Initiale} U W  \newline
\line(1,0){75} \newline
\textbf{1} gein der jost] [gem*]: gendem strit R \textbf{3} geriten] scharpff geritten R \textbf{5} starken] scharpffen R \textbf{6} sprîzen] spidlen R \textbf{7} Ez] er Q \textbf{8} gesaz] do gesessen was R \textbf{9} im] in U \textit{om.} W  $\cdot$ gepflac] pflag R \textbf{10} strîtes sich] sich strittes R \textbf{11} ob] oder U  $\cdot$ iht] Jch R  $\cdot$ vuorten] ruͦrten W \textbf{12} dô] Die W  $\cdot$ zuo ein ander] ein andren R  $\cdot$ ruorten] fuͦrten W \textbf{13} dô] \textit{om.} W  $\cdot$ alle] \textit{om.} W all Q  $\cdot$ breit] bereit Q R \textbf{16} ecidemôn] Essidemon W Etidomen R \textbf{17} dô] do zu mal R \textbf{18} mohte] moͤchte W  $\cdot$ klagen] wol clagen R \textbf{19} müede] mūde Q muͯden R  $\cdot$ heiz] do heis R \textbf{21} ab orsen] ab den roßen W (R) von rosse Q \textbf{22} alrêrst] Allererst recht W \textbf{24} krîe] krick Q  $\cdot$ Thasme] tasme U W Q \textbf{25} Tabrunit] tanbruͦnit U tabrúnit Q Tabruͦnit R \textbf{26} trat] schreit W tett R  $\cdot$ vürbaz] fur sich Q (R)  $\cdot$ trit] schrit W \textbf{28} manegem] mangen W (R)  $\cdot$ dræten] trate Q trette R \textbf{29} ein ander] einandren R \newline
\end{minipage}
\end{table}
\end{document}
