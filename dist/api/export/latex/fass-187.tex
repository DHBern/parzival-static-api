\documentclass[8pt,a4paper,notitlepage]{article}
\usepackage{fullpage}
\usepackage{ulem}
\usepackage{xltxtra}
\usepackage{datetime}
\renewcommand{\dateseparator}{.}
\dmyyyydate
\usepackage{fancyhdr}
\usepackage{ifthen}
\pagestyle{fancy}
\fancyhf{}
\renewcommand{\headrulewidth}{0pt}
\fancyfoot[L]{\ifthenelse{\value{page}=1}{\today, \currenttime{} Uhr}{}}
\begin{document}
\begin{table}[ht]
\begin{minipage}[t]{0.5\linewidth}
\small
\begin{center}*D
\end{center}
\begin{tabular}{rl}
\textbf{187} & \textbf{die} vrouwen mitten an die stegen.\\ 
 & dâ kuste si den werden degen.\\ 
 & die münde wâren bêde rôt.\\ 
 & diu künegîn ir hant im bôt.\\ 
5 & \textbf{Parzivalen} \textbf{si} vuorte wider,\\ 
 & al dâ si \textbf{sâzen beidiu} nider.\\ 
 & vrouwen unt \textbf{ritter kraft}\\ 
 & heten alle swache kraft,\\ 
 & die dâ stuonden und sâzen.\\ 
10 & \textbf{si} heten vreude \textbf{lâzen},\\ 
 & daz gesinde unt diu wirtîn.\\ 
 & Condwiramurs ir schîn\\ 
 & doch schiet von \textbf{disen} strîten:\\ 
 & Jeschuten, Eniten\\ 
15 & unt Cunnewaren de Lalant\\ 
 & \textbf{unt} swâ man lobes die besten vant,\\ 
 & \textbf{dâ} man \textbf{vrouwen schœne} gewuoc,\\ 
 & ir glastes schîn vast \textbf{undersluoc},\\ 
 & unde bêder Isalden.\\ 
20 & \textbf{jâ} \textbf{muose} prîses walden\\ 
 & Condwiramurs.\\ 
 & diu truoc den rehten bêâcurs.\\ 
 & Der name ist tiuschen 'schœner lîp'.\\ 
 & ez wâren wol nütziu wîp,\\ 
25 & die disiu zwei gebâren,\\ 
 & die dâ bî ein ander wâren.\\ 
 & dô schuof wîp unt man\\ 
 & niht mêre, wan daz si sâhen an\\ 
 & \textbf{diu} zwei bî ein ander.\\ 
30 & \textbf{guote vriwent} dâ vander.\\ 
\end{tabular}
\scriptsize
\line(1,0){75} \newline
D \newline
\line(1,0){75} \newline
\textbf{23} \textit{Majuskel} D  \newline
\line(1,0){75} \newline
\textbf{14} Jeschuten] Jescv̂ten D  $\cdot$ Eniten] Enîten D \textbf{19} Isalden] Jsalden D \textbf{23} tiuschen] tivscen D \newline
\end{minipage}
\hspace{0.5cm}
\begin{minipage}[t]{0.5\linewidth}
\small
\begin{center}*m
\end{center}
\begin{tabular}{rl}
 & \textbf{die} vrouwen mitten an die stegen.\\ 
 & dâ kuste si den werden degen.\\ 
 & die münde wâren beide rôt.\\ 
 & diu künigî\textit{n} ir hant im bôt,\\ 
5 & \textbf{Parcifale}, \textbf{und} vuorte \textbf{in} wider,\\ 
 & aldâ si \textbf{beidiu sâzen} nider.\\ 
 & vrouwen und \textbf{ritterschaft}\\ 
 & hete\textit{n} alle swache kraft,\\ 
 & die dâ stuonden und sâzen.\\ 
10 & \textbf{si} heten vröuden \textbf{lâzen},\\ 
 & daz gesinde und diu wirtîn.\\ 
 & Con\textit{d}wi\textit{e}ramurs ir schîn\\ 
 & doch s\textit{chie}t von \textbf{disem} strîten:\\ 
 & Jeschute\textit{n} \textbf{und} Eniten\\ 
15 & und Cu\textit{nn}ewaren de Lalant\\ 
 & \textbf{und} wâ man lobes d\textit{ie} besten vant,\\ 
 & \textbf{d\textit{â}} man \textbf{vrouwen schœne} gewuoc,\\ 
 & ir glastes schîn vast \textbf{undersluoc},\\ 
 & und beider Isalden.\\ 
20 & \textbf{jâ} \textbf{muose} prîses walden\\ 
 & C\textit{o}n\textit{d}wieramurs.\\ 
 & diu truoc den rehten bêâcurs.\\ 
 & der nam ist tiutschen 'schœner lîp'.\\ 
 & ez wâren wol \textbf{zwei} nütziu wîp,\\ 
25 & die disiu zwei gebâren,\\ 
 & die dâ bî ein ander wâren.\\ 
 & dô \textbf{en}schuof \textbf{d\textit{â}} wîp und man\\ 
 & niht mêre, wan daz si sâhen \textit{a}n\\ 
 & \textbf{diu} zwei \textbf{d\textit{â}} bî ein ander.\\ 
30 & \textbf{guote vröude} d\textit{â} vant er.\\ 
\end{tabular}
\scriptsize
\line(1,0){75} \newline
m n o Fr69 \newline
\line(1,0){75} \newline
\newline
\line(1,0){75} \newline
\textbf{2} dâ] Do n o  $\cdot$ den werden] der werde n \textbf{4} künigîn] kunigim m  $\cdot$ im] nie o \textbf{5} Parcifale] Parcifalle m Parcifal n o  $\cdot$ vuorte] fuͯrste o \textbf{8} heten] Hette m \textbf{9} dâ] do n o \textbf{10} vröuden] freide n o  $\cdot$ lâzen] gelossen n o \textbf{12} Condwieramurs] Conden wir amurs m Conduwierten n Cond wirtin o Condv́wier amurs Fr69  $\cdot$ ir] \textit{om.} Fr69 \textbf{13} schiet] streit m schiedent n  $\cdot$ disem] [*]: disen o disen Fr69 \textbf{14} Jeschuten] Jescute m Jescuten n o :::witen Fr69  $\cdot$ Eniten] enitten m n Fr69 \textbf{15} Cunnewaren] cummewaren m conmieweren n Comme weren o  $\cdot$ de] do o  $\cdot$ Lalant] labant o \textbf{16} wâ] swa Fr69  $\cdot$ die] den m n o  $\cdot$ besten] vesten o \textbf{17} dâ] Do m n o  $\cdot$ gewuoc] geroug n \textbf{19} Isalden] ẏsalden n jsadeln o \textbf{20} muose] múste n  $\cdot$ prîses] prisen o \textbf{21} Condwieramurs] Candiwier amurs m Conduwir amurs n Conmar wir arnes o \textbf{22} rehten] rehte o  $\cdot$ bêâcurs] bẏackurs o \textbf{23} tiutschen] tuͯtzschen m tútsch ist n dische o \textbf{26} dâ] do n \textit{om.} o \textbf{27} enschuof] ensluff o  $\cdot$ dâ] do m \textit{om.} n o \textbf{28} sâhen an] sohenen m \textbf{29} dâ] do m n o  $\cdot$ ein ander] eẏander o \textbf{30} vröude] frúnt n (o)  $\cdot$ dâ] do m die n o \newline
\end{minipage}
\end{table}
\newpage
\begin{table}[ht]
\begin{minipage}[t]{0.5\linewidth}
\small
\begin{center}*G
\end{center}
\begin{tabular}{rl}
 & \textbf{ir} vrouwen mitten an die stegen.\\ 
 & dâ kuste si den werden degen.\\ 
 & die münde wâren bêde rôt.\\ 
 & diu künigîn ir hant im bôt.\\ 
5 & \textbf{Parzivalen} \textbf{si} vuorte wider,\\ 
 & al dâ si \textbf{sâzen beidiu} nider.\\ 
 & vrouwen unde \textbf{rîterschaft},\\ 
 & \textbf{die} heten alle swache kraft,\\ 
 & die dâ stuonden unde sâzen.\\ 
10 & \textbf{die} heten vröude \textbf{lâzen},\\ 
 & daz gesinde unde \textbf{ouch} diu wirtîn.\\ 
 & Condwiramurs ir schîn\\ 
 & doch schiet von \textbf{disen} strîten:\\ 
 & Jeschuten \textbf{unde} Eniten\\ 
15 & unde Kunewaren de Lalant\\ 
 & \textbf{oder} swâ man lobes die besten vant,\\ 
 & \textbf{swâ} man \textbf{vrouwen schœne} gewuoc,\\ 
 & ir glastes schîn vaste \textbf{undersluoc}\\ 
 & unde beider Ysalden.\\ 
20 & \textbf{diu dâ} \textbf{muoz} prîses walden,\\ 
 & \textbf{daz was diu künigîn} Condwiramurs.\\ 
 & diu truoc den rehten bêâcurs.\\ 
 & der name ist tiutschen 'schœner lîp'.\\ 
 & ez wâren wol nutziu wîp,\\ 
25 & di\textit{e} disiu zwei gebâren,\\ 
 & diu dâ \textit{bî} ein ander wâren.\\ 
 & dô\textbf{ne} schuof wîp unde man\\ 
 & niht mê, wan daz si sâhen an\\ 
 & \textbf{diu} zwei bî ein ander.\\ 
30 & \textbf{guoten vriunt} dâ vander.\\ 
\end{tabular}
\scriptsize
\line(1,0){75} \newline
G I O L M Q R Z \newline
\line(1,0){75} \newline
\textbf{3} \textit{Initiale} I  \textbf{5} \textit{Initiale} L  \textbf{11} \textit{Initiale} O R  \textbf{13} \textit{Initiale} M  \textbf{21} \textit{Initiale} Z  \textbf{25} \textit{Initiale} I  \newline
\line(1,0){75} \newline
\textbf{1} mitten] en mitten I (O) (L) (M)  $\cdot$ die] der L den Z \textbf{2} dâ] do I (O) (Q) (R)  $\cdot$ kuste] kusten Q  $\cdot$ den werden] der werde L M R \textbf{3} wâren] warn in I (L)  $\cdot$ bêde] beiden I (L) bedú R \textbf{4} ir hant im] ime ir hant I \textbf{5} Parzivalen] Parzifaln I Parcifal O L Parzival M Partzifal Q Parczifal R Parcifalen Z  $\cdot$ wider] weider Q \textbf{6} al dâ] ada I  $\cdot$ sâzen beidiu] beide saszen L (Z) \textbf{9} dâ] do Q \textbf{10} die] Si O (M) (Q) (R) (Z)  $\cdot$ vröude] alle freude I  $\cdot$ lâzen] gelaszen L (M) R \textbf{11} \sout{÷as ge sindt vnd dy wir} Das gesind vnd dy wirtein Q  $\cdot$ daz] ÷az O  $\cdot$ ouch] \textit{om.} O L M R Z \textbf{12} Condwiramurs] conduwiramurs I Kvndwiramvrs O Z Condwuͯr amvrs L Kondwir amuͯrs M Kund wiraműs Q Kondwiramuͦrs R  $\cdot$ ir schîn] erschein R \textbf{13} doch] Do M  $\cdot$ von] noch Q  $\cdot$ disen] disem I (Q)  $\cdot$ strîten] strite I (Q) \textbf{14} Jeschuten] ieschuten G ieskuten I Jecuͯten L Jescuten M Q Z Jesunten R  $\cdot$ unde] \textit{om.} O L M Q R Z  $\cdot$ Eniten] Enyten R \textbf{15} Kunewaren] kunwarn I Gvnwarn O Cvnewaren L kuntvarin M kunwaren Q Cuͦnwaren R kvnnewaren Z  $\cdot$ de] der O von R  $\cdot$ Lalant] Labant R \textbf{16} oder] \textit{om.} L Vnd Q  $\cdot$ swâ] Wa L (M) (Q) (R) (Z)  $\cdot$ lobes] frowen lobes L  $\cdot$ die] der R \textbf{17} swâ] vnd swa I Wa L (Q) R Z  $\cdot$ gewuoc] gewt Q gnuͦg R \textbf{18} schîn vaste] [sin]: shin I vast R \textbf{19} Ysalden] ẏsalden G isalden I Jsalden L R Z isaldin M \textbf{20} diu] Du R  $\cdot$ dâ] \textit{om.} M do Q R  $\cdot$ muoz] muͤzzen I (M) muͯste L (Q) (R)  $\cdot$ prîses] preÿsen Q \textbf{21} daz was diu künigîn] \textit{om.} I O L M Q R Z  $\cdot$ Condwiramurs] [condwiramus]: condwiramurs G conduwiramurs I Kvndwiramvrs O Condvwir amuͯrs L Kondin wir amuͯrs M kund wiramúrs Q Kondwiramuͦrs R Kvndewiramvrs Z \textbf{22} rehten] schoͤnen O \textbf{23} der name] Daz L  $\cdot$ ist] \textit{om.} I  $\cdot$ tiutschen] tuschen G I devtsche O entivschen L \textit{om.} M deuschen Q tútschen R tevtsch Z \textbf{24} wâren] weren L  $\cdot$ nutziu] nuzzer I \textbf{25} die] [div]: di G Vvie I Dú R \textbf{26} diu] di I  $\cdot$ dâ] do Q  $\cdot$ bî] in G  $\cdot$ ander] andren R \textbf{27} dône] dan I Do O L Da en M (Z) Dene R  $\cdot$ wîp unde] weder wip noch I weip noch Q \textbf{28} niht mê wan] niwan I Niht mer nv O Nicht mer dann Q (R) (Z)  $\cdot$ si] si si I sis Q  $\cdot$ sâhen] sehen O \textbf{29} ein] an I  $\cdot$ ander] ander \sout{waren} Z \textbf{30} guoten] Guͤt I (Z) Gvͦte O (L) (M) (Q) (R)  $\cdot$ dâ] do O Q R \newline
\end{minipage}
\hspace{0.5cm}
\begin{minipage}[t]{0.5\linewidth}
\small
\begin{center}*T
\end{center}
\begin{tabular}{rl}
 & \textbf{ir} vrouwen enmitten an die stegen.\\ 
 & dâ kuste si den werden degen.\\ 
 & Die münde wâren beide rôt.\\ 
 & diu künegîn ir hant im bôt.\\ 
5 & \textbf{Parcifaln} \textbf{si} vuorte wider,\\ 
 & aldâ si \textbf{beidiu sâzen} nider.\\ 
 & vrouwen unde \textbf{rîterschaft}\\ 
 & heten alle swache kraft,\\ 
 & die dâ stuonden unde sâzen.\\ 
10 & \textbf{die} heten vröude \textbf{gelâzen},\\ 
 & daz gesinde unde diu wirtîn.\\ 
 & Cundewiramurs ir schîn\\ 
 & doch schiet von \textbf{disen} strîten:\\ 
 & Jeschuten \textbf{unde} Eniten\\ 
15 & unde Cunnewaren de Lalant\\ 
 & \textbf{oder} swâ man lobes die besten vant,\\ 
 & \textbf{oder} \textbf{swâ} man \textbf{schœner vrouwen} gewuoc,\\ 
 & ir glastes schîn vaste \textbf{widersluoc},\\ 
 & unde beider Ysalden.\\ 
20 & \textbf{jâ} \textbf{muose} prîses walden\\ 
 & Cundewiramurs.\\ 
 & di\textit{u} truoc den rehten bêâcurs.\\ 
 & der name ist tiuschen 'schœner lîp'.\\ 
 & ez wâren wol nütziu wîp,\\ 
25 & diu disiu zwei gebâren,\\ 
 & diu dâ bî ein ander wâren.\\ 
 & \begin{large}D\end{large}ô schuof wîp unde man\\ 
 & niht mê, wan daz si sâhen an\\ 
 & \textbf{dis\textit{iu}} zwei bî ein ander.\\ 
30 & \textbf{guote} \textbf{vriunt} dâ vander.\\ 
\end{tabular}
\scriptsize
\line(1,0){75} \newline
T U V W \newline
\line(1,0){75} \newline
\textbf{3} \textit{Initiale} W   $\cdot$ \textit{Majuskel} T  \textbf{27} \textit{Initiale} T U V  \newline
\line(1,0){75} \newline
\textbf{1} enmitten] einitten U mitten W \textbf{2} dâ] Do U V W \textbf{4} ir hant im] im ir hand W \textbf{5} Parcifaln] Parzifalen V Partzifal W  $\cdot$ si vuorte] sie vuͦrtin U vnde fuͦrte in V \textbf{6} aldâ] Aldo W \textbf{8} Die hetten vil schwache karfft W \textbf{9} dâ] do W \textbf{10} die] Sy W  $\cdot$ heten] hete U \textbf{11} diu] din U \textbf{12} Cundewiramurs] Condewiramvrs T Cvndeviramuͦrs U Gundwiramurs W \textbf{14} Jeschuten] Jescvten T (U) Vroͮwe ieschuten V Iestuten W  $\cdot$ Eniten] eineten U \textbf{15} Cunnewaren] kvnnewaren T V kuͦmewaren U kunnewarn W  $\cdot$ Lalant] laland W \textbf{16} oder] Vnd V  $\cdot$ swâ] wo U W \textbf{17} [D*]: Do man vrowen schoͤne gewuͦg V Vnd wo man frawen schoͤne gewuͦg W  $\cdot$ swâ] wo U \textbf{18} glastes] gastes W  $\cdot$ vaste] vastes U  $\cdot$ widersluoc] [*erslvͦg]: vnderslvͦg V vnderschluͦg W \textbf{20} muose] mvese T \textbf{21} Die schoͤne gundwiramurß W \textbf{22} diu] die T \textbf{23} ist] \textit{om.} U  $\cdot$ tiuschen] tvschen T tuͦschen U in tv́schen V (W) \textbf{24} Ez waren [zwei*]: wol zwei nv́tze wip V \textbf{26} dâ] do W \textbf{28} sâhen] sehen W \textbf{29} disiu] dise T  $\cdot$ bî] do bi V \textbf{30} vriunt] froͤde W  $\cdot$ dâ] do V W \newline
\end{minipage}
\end{table}
\end{document}
