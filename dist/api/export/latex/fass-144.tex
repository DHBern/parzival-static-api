\documentclass[8pt,a4paper,notitlepage]{article}
\usepackage{fullpage}
\usepackage{ulem}
\usepackage{xltxtra}
\usepackage{datetime}
\renewcommand{\dateseparator}{.}
\dmyyyydate
\usepackage{fancyhdr}
\usepackage{ifthen}
\pagestyle{fancy}
\fancyhf{}
\renewcommand{\headrulewidth}{0pt}
\fancyfoot[L]{\ifthenelse{\value{page}=1}{\today, \currenttime{} Uhr}{}}
\begin{document}
\begin{table}[ht]
\begin{minipage}[t]{0.5\linewidth}
\small
\begin{center}*D
\end{center}
\begin{tabular}{rl}
\textbf{144} & \textbf{werdent} durch \textbf{die} mül gezücket\\ 
 & \textbf{unt ir} lop \textbf{gebücket}.\\ 
 & sol ich den \textbf{munt} mit spotte zern,\\ 
 & ich wil mînen vriunt mit spotte \textit{w}ern.\\ 
5 & \textbf{Dô} kom der vischære\\ 
 & unt ouch der knappe mære\\ 
 & einer houptstat sô nâhen,\\ 
 & al dâ si Nantes sâhen.\\ 
 & dô sprach er: "kint, got hüete dîn.\\ 
10 & \textbf{nû} sich, dort soltû rîten în."\\ 
 & Dô sprach der knappe an witzen laz:\\ 
 & "dû solt mich wîsen vürbaz."\\ 
 & "\textbf{wie wol} mîn lîp daz bewart!\\ 
 & diu messenîe ist \textbf{al} sölher art,\\ 
15 & genæhete ir immer vilân,\\ 
 & daz \textbf{wære} \textbf{vil sêre} missetân."\\ 
 & \begin{large}D\end{large}er knappe \textbf{al eine} vürbaz reit\\ 
 & ûf einen plân niht ze breit.\\ 
 & der stuont von bluomen lieht gemâl.\\ 
20 & in zôch dehein Curvenal;\\ 
 & er kunde kurtôsîe niht,\\ 
 & als \textbf{ungevarnen} man geschiht.\\ 
 & sîn zoum, der was bästîn\\ 
 & unt harte kranc sîn pferdelîn.\\ 
25 & daz tet von \textbf{strûchen} manegen val.\\ 
 & ouch was sîn satel überal\\ 
 & \textbf{unbeslagen} mit \textbf{niwen ledern}.\\ 
 & samît, \textbf{härmîner vedern}\\ 
 & man \textbf{dâ vil lützel} an im siht.\\ 
30 & er\textbf{n} \textbf{bedorfte} der mantelsnüere niht.\\ 
\end{tabular}
\scriptsize
\line(1,0){75} \newline
D \newline
\line(1,0){75} \newline
\textbf{5} \textit{Majuskel} D  \textbf{11} \textit{Majuskel} D  \textbf{17} \textit{Initiale} D  \newline
\line(1,0){75} \newline
\textbf{4} wern] vern D \newline
\end{minipage}
\hspace{0.5cm}
\begin{minipage}[t]{0.5\linewidth}
\small
\begin{center}*m
\end{center}
\begin{tabular}{rl}
 & \hspace*{-.7em}\big| \textbf{der} lob \textbf{wirt} \textbf{gebücket}\\ 
 & \hspace*{-.7em}\big| \textbf{und} durch \textbf{die} müle gezücket.\\ 
 & sol ich den \textbf{munt} mit spotte zern,\\ 
 & ich wil mînen vriunt mit spotte wern.\\ 
5 & \textbf{\begin{large}D\end{large}ô} kam der vischære\\ 
 & und ouch der knappe mære\\ 
 & einer houbetstat sô nâhen,\\ 
 & al\textit{d}â si Nantes sâhen.\\ 
 & dô sprach er: "kint, got hüete dîn.\\ 
10 & \textbf{vür} sich, dort soltû rîten în."\\ 
 & dô sprach der knappe an witzen laz:\\ 
 & "d\textit{û} solt mich wîsen vürbaz."\\ 
 & "\textbf{wie wol} mîn lî\textit{p} daz bewart!\\ 
 & diu massenîe ist solicher art,\\ 
15 & genâhete ir iemer vilân,\\ 
 & daz \textbf{wære} \textbf{vil sêre} missetân."\\ 
 & der knappe \textbf{aleine} vürbaz reit\\ 
 & ûf einen plân niht ze breit.\\ 
 & der stuont von bluomen lieht gemâl.\\ 
20 & in zôch \dag noch hin\dag  Kurvenal;\\ 
 & er \textbf{en}kunde kurtôsîe niht,\\ 
 & als \textbf{ungevarnem} man geschiht.\\ 
 & sîn z\textit{o}um, der \textit{w}as bestîn\\ 
 & und harte kranc sîn pferdelîn.\\ 
25 & da\textit{z} \textit{t}et von \textbf{strûche} manigen val.\\ 
 & ouch was sîn satel überal\\ 
 & \textbf{unbeslagen} mit \textbf{niuwem leder}.\\ 
 & samît \textbf{und} \textbf{herm\textit{în} vede\textit{r}}\\ 
 & man \textbf{dâ vil lützel} an ime siht.\\ 
30 & er \textbf{bedarf} der mantelsnüere niht.\\ 
\end{tabular}
\scriptsize
\line(1,0){75} \newline
m n o \newline
\line(1,0){75} \newline
\textbf{5} \textit{Initiale} m n o  \newline
\line(1,0){75} \newline
\textbf{2} gebücket] gelucket n o \textbf{1} gezücket] gezocket o \textbf{3} ich] \textit{om.} o \textbf{12} dû] Do m \textbf{13} lîp] lit m  $\cdot$ daz] des o \textbf{14} solicher] sollich n (o) \textbf{15} genâhete] Genohent n Genohet o \textbf{16} missetân] missetat o \textbf{17} der] Das o \textbf{18} einen] einem o \textbf{20} Kurvenal] kornuwal n kurne val o \textbf{21} kurtôsîe] curtosen o \textbf{22} ungevarnem] vngefaren o \textbf{23} zoum] zum m  $\cdot$ der] des o  $\cdot$ was] vas m \textbf{25} daz tet] Dat ret m \textbf{27} Vmbeslagen mit núwen loder o \textbf{28} hermîn veder] hermur [fe*]: federn m \textbf{29} dâ] do n o  $\cdot$ vil] \textit{om.} n \textbf{30} mantelsnüere] mantel snue o \newline
\end{minipage}
\end{table}
\newpage
\begin{table}[ht]
\begin{minipage}[t]{0.5\linewidth}
\small
\begin{center}*G
\end{center}
\begin{tabular}{rl}
 & \textbf{werdent} durch \textbf{die} müle gezücket\\ 
 & \textbf{unde ir} lop \textbf{gebrücket}.\\ 
 & sol ich den \textbf{lîp} mit spotte zeren,\\ 
 & ich wil mînen vriunt mit spotte weren.\\ 
5 & \textbf{dô} kom der vischære\\ 
 & unde ouch der knappe mære\\ 
 & \textit{ein}er \textit{ho}u\textit{btstat sô} nâhen,\\ 
 & \textit{aldâ} si Nantis sâhen.\\ 
 & dô sprach er: "kint, got hüete dîn.\\ 
10 & \textbf{nû} sich, dort soltû rîten în."\\ 
 & dô sprach der knappe an witzen laz:\\ 
 & "dû solt mich wîsen vürbaz."\\ 
 & "\textbf{hei}, mîn lîp daz \textbf{vil wol} bewart!\\ 
 & diu messenîe ist solher art,\\ 
15 & genâhet ir imer vilân,\\ 
 & daz \textbf{ist} \textbf{sêre} missetân."\\ 
 & der knappe \textbf{al eine} vürbaz reit\\ 
 & ûf einen plân niht ze breit.\\ 
 & der stuont von bluomen lieht gemâl.\\ 
20 & in zôch nehein Curfenal;\\ 
 & er\textbf{ne} kunde kurtoise niht,\\ 
 & als \textbf{ungevarnem} man geschiht.\\ 
 & sîn zoum, der was bästîn\\ 
 & unde hart kranc sîn pferdelîn.\\ 
25 & daz tet von \textbf{strûche} manigen val.\\ 
 & ouch was sîn satel überal\\ 
 & \textbf{umbeslagen} mit \textbf{niwen lederen}.\\ 
 & samît, \textbf{härmîner vederen},\\ 
 & \textbf{der zweier} man \textbf{wênic} an im siht.\\ 
30 & er \textbf{bedorfte} der mandelsnüe\textit{re} niht.\\ 
\end{tabular}
\scriptsize
\line(1,0){75} \newline
G I O L M Q R Z \newline
\line(1,0){75} \newline
\textbf{5} \textit{Capitulumzeichen} L  \textbf{11} \textit{Initiale} Q  \textbf{17} \textit{Überschrift:} Wie parcifal des ersten qvam vf kvnic artus hof geriten Z   $\cdot$ \textit{Initiale} L R Z  \newline
\line(1,0){75} \newline
\textbf{1} werdent] Wordin M Erdent Q  $\cdot$ gezücket] geruket R \textbf{2} gebrücket] gebvchet O (L) (M) (Q) (R) (Z) \textbf{3} lîp] mvnt O L (M) (Q) (R) (Z)  $\cdot$ spotte] zorne Q \textbf{4} wil] \textit{om.} O  $\cdot$ mînen] ymmer Q \textbf{5} dô] Da M Z \textbf{6} ouch] \textit{om.} I O L M Q R \textbf{7} einer houbtstat sô] der burch als G \textbf{8} aldâ] daz G  $\cdot$ si] \textit{om.} Q  $\cdot$ Nantis] nantẏs G nantes I (L) natis R \textbf{9} dô] Da M Z  $\cdot$ er] der fischer R  $\cdot$ kint] \textit{om.} O L Q R \textbf{10} nû sich] \textit{om.} I  $\cdot$ dort] do O  $\cdot$ în] hin M \textbf{11} dô] Da O M Z  $\cdot$ witzen] wicze R \textbf{13} Hei wi wol (es L ) min lip daz (\textit{om.} L ) bewart O (L) (M) (Q) (R) (Z)  $\cdot$ daz vil] \textit{om.} I  $\cdot$ bewart] bewart daz I \textbf{14} ist] ist in Z \textbf{15} genâhet ir] Ginaheten M \textbf{16} ist] wer I (Q)  $\cdot$ sêre] vil sere O L R Z vil schire M \textbf{17} der] Eer R \textbf{19} von] vol Q  $\cdot$ lieht] lýht L (M) (Q) wol Z  $\cdot$ gemâl] gemalt Q \textbf{20} Curfenal] kurvenal I (L) M (Z) cvrvenal O kuruenal Q R \textbf{21} erne] er I (R)  $\cdot$ kurtoise] kuͯrtosýe L \textbf{22} als] als als I  $\cdot$ ungevarnem] vnGeuarn I (R) vngevarndem L  $\cdot$ geschiht] da sicht R \textbf{23} zoum] zorrn Q \textbf{24} unde hart] alt vnd I  $\cdot$ kranc] trank Q \textbf{25} strûche] struchen I (O) (L) (Q) (R) (Z) \textbf{27} umbeslagen] Vmb geschlagen Q  $\cdot$ niwen lederen] nwem ledere I (M) (Q) \textbf{28} härmîner] herminen M  $\cdot$ vederen] vedere I weder Q \textbf{29} Man da vil wenic an im siht Z  $\cdot$ man wênic] wenich man I (R) man lvzel O (M) (Q) luͯtzel man L  $\cdot$ an im] da I an im da O L (M) \textbf{30} er] ern I (L) (M) (Q) (Z)  $\cdot$ bedorfte] dorffte M Q  $\cdot$ der] \textit{om.} R  $\cdot$ mandelsnüere] mandel snoͮner G \newline
\end{minipage}
\hspace{0.5cm}
\begin{minipage}[t]{0.5\linewidth}
\small
\begin{center}*T (U)
\end{center}
\begin{tabular}{rl}
 & \textbf{werdent} durch mül gezücket\\ 
 & \textbf{und ir} lop \textbf{gebücket}.\\ 
 & sol ich den \textbf{munt} mit spotte zern,\\ 
 & ich wil mînen vriunt mit spotte wern.\\ 
5 & \textbf{\begin{large}S\end{large}us} kam der vischære\\ 
 & und ouch der knappe mære\\ 
 & einer houbetstat sô nâhen,\\ 
 & aldâ si Nantes sâhen.\\ 
 & dô sprach er: "kint, got hüete dîn.\\ 
10 & \textbf{nû} sich, dort soltû rîten în."\\ 
 & dô sprach der knappe an witzen laz:\\ 
 & "dû solt mich wîsen vürbaz."\\ 
 & "\textbf{hei, wie wol} mîn lîp daz bewart!\\ 
 & diu massenîe ist solicher art,\\ 
15 & genâhetir immer vilân,\\ 
 & daz \textbf{wære} \textbf{schiere} missetân."\\ 
 & d\textit{er} knappe \textbf{dô} vürbaz reit\\ 
 & ûf einen plân niht zuo breit.\\ 
 & der stuont von bluomen lieht gemâl.\\ 
20 & \textit{i}n zôch dekein Curvenal;\\ 
 & er \textbf{en}kunde kurtois\textit{î}e niht,\\ 
 & als \textbf{ungevarnen} man geschiht.\\ 
 & sîn zoum, der was bästîn\\ 
 & und harte kranc sîn pferdelîn.\\ 
25 & daz tet von \textbf{strûchen} manegen val.\\ 
 & ouch was sîn satel überal\\ 
 & \textbf{umbeslagen} mit \textbf{niuwen ledern}.\\ 
 & samît \textbf{und} \textbf{härmîner vedern},\\ 
 & \textbf{der zweier} man \textbf{wênic} an im siht.\\ 
30 & er \textbf{en}\textbf{bedorfte} der mantelsnüere niht.\\ 
\end{tabular}
\scriptsize
\line(1,0){75} \newline
U V W T \newline
\line(1,0){75} \newline
\textbf{5} \textit{Initiale} U V T  \textbf{9} \textit{Majuskel} T  \textbf{11} \textit{Majuskel} T  \textbf{13} \textit{Majuskel} T  \textbf{17} \textit{Majuskel} T  \textbf{23} \textit{Majuskel} T  \textbf{28} \textit{Majuskel} T  \newline
\line(1,0){75} \newline
\textbf{1} mül] [*]: die Mvl V die mul W (T) \textbf{2} gebücket] hie gebucht W \textbf{3} spotte] \textit{om.} W \textbf{4} ich wil] so wil ich T  $\cdot$ mit spotte] spottes W \textbf{6} ouch] \textit{om.} T \textbf{8} Nantes] nantis W \textbf{9} dô sprach er] [*]: Der vischer sprach V  $\cdot$ kint] \textit{om.} V \textbf{11} der knappe] daz kint T \textbf{13} hei] \textit{om.} T \textbf{15} immer] mein W \textbf{16} wære schiere] were sere V (W) daz ist vil sere T \textbf{17} Daz kint alleine vur sich reît T  $\cdot$ der] Do U \textbf{19} der] Do W  $\cdot$ von] vol T  $\cdot$ lieht] licht U liech V wol T \textbf{20} in] Jch in U  $\cdot$ zôch] enzoch V  $\cdot$ Curvenal] Coruenal V kuruenal W \textbf{21} enkunde] kunde W (T)  $\cdot$ kurtoisîe] kuͦrtoyse U mit kurtoyse W \textbf{22} ungevarnen] vneruarnem V vngeuarnem W  $\cdot$ geschiht] beschiht V \textbf{23} der] \textit{om.} T \textbf{27} umbeslagen] [*]: vnbeslagen T  $\cdot$ niuwen ledern] neúwem ledere W [*]: niͮwen vedern T \textbf{28} und härmîner] harmin W (T)  $\cdot$ vedern] federe W \textbf{30} enbedorfte] bedorfte V (W) T  $\cdot$ der] \textit{om.} V \newline
\end{minipage}
\end{table}
\end{document}
