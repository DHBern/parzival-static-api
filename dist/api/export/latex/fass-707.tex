\documentclass[8pt,a4paper,notitlepage]{article}
\usepackage{fullpage}
\usepackage{ulem}
\usepackage{xltxtra}
\usepackage{datetime}
\renewcommand{\dateseparator}{.}
\dmyyyydate
\usepackage{fancyhdr}
\usepackage{ifthen}
\pagestyle{fancy}
\fancyhf{}
\renewcommand{\headrulewidth}{0pt}
\fancyfoot[L]{\ifthenelse{\value{page}=1}{\today, \currenttime{} Uhr}{}}
\begin{document}
\begin{table}[ht]
\begin{minipage}[t]{0.5\linewidth}
\small
\begin{center}*D
\end{center}
\begin{tabular}{rl}
\textbf{707} & \begin{large}U\end{large}nde \textbf{Bernout} \textbf{de} Riviers\\ 
 & unt \textbf{Affinamus} \textbf{von} Clitiers,\\ 
 & mit blôzen houpten dise drî\\ 
 & \textbf{riten dem strîte} \textbf{nâher} bî.\\ 
5 & Artus und Gawan\\ 
 & riten anderthalben ûf \textbf{den} plân\\ 
 & zuo den kampfmüeden zwein.\\ 
 & die \textbf{vünfe} wurden des enein,\\ 
 & si wolden scheiden \textbf{disen} strît.\\ 
10 & Scheidens dûhte rehtiu zît\\ 
 & Gramoflanzen, der sô sprach,\\ 
 & daz er dem siges jach,\\ 
 & den man \textbf{gein im dâ het} \textbf{ersehen}.\\ 
 & des \textbf{muose} \textbf{ouch} mêre \textbf{liute} jehen.\\ 
15 & Dô sprach des \textbf{künec} Lotes sun:\\ 
 & "hêr künec, ich wil hiute tuon,\\ 
 & als ir mir gestern tâtet,\\ 
 & dô ir mich ruowen bâtet.\\ 
 & nû ruowet hînte, \textbf{des wirt} iu nôt.\\ 
20 & swer iu disen strît gebôt,\\ 
 & der het \textbf{iu swache} kraft erkant\\ 
 & gein mîner werlîchen hant.\\ 
 & Ich bestüende iuch \textbf{nû} wol ein,\\ 
 & \textbf{nû veht aber ir} niwan mit zwein.\\ 
25 & ich wilz morgen wâgen eine;\\ 
 & got ez ze rehte erscheine."\\ 
 & Der künec reit \textbf{dannen zuo} den sîn.\\ 
 & er tet \textbf{ê} fîanze \textbf{schîn},\\ 
 & daz er \textbf{s}morgens gein Gawan\\ 
30 & durch \textbf{strîten} kœme ûf den plân.\\ 
\end{tabular}
\scriptsize
\line(1,0){75} \newline
D Fr66 \newline
\line(1,0){75} \newline
\textbf{1} \textit{Initiale} D  \textbf{10} \textit{Majuskel} D  \textbf{15} \textit{Majuskel} D  \textbf{23} \textit{Majuskel} D  \textbf{27} \textit{Majuskel} D  \newline
\line(1,0){75} \newline
\textbf{1} Bernout] Bernoͮt D  $\cdot$ Riviers] Rivîers D \textbf{2} Affinamus] Affinamv̂s D  $\cdot$ Clitiers] Clitîers D \textbf{15} Lotes] Lots D \newline
\end{minipage}
\hspace{0.5cm}
\begin{minipage}[t]{0.5\linewidth}
\small
\begin{center}*m
\end{center}
\begin{tabular}{rl}
 & und \textbf{Bern\textit{o}u\textit{t}} \textbf{de} \textit{Ri}viers\\ 
 & und \textbf{Of\textit{f}inamus} \textbf{von} Clitiers,\\ 
 & mit blôzen \textit{h}oubten dise drî\\ 
 & \textbf{rite\textit{n de}m strît} \textbf{nâ} bî.\\ 
5 & Artus und Gawan\\ 
 & riten anderhalp ûf \textbf{dem} plân\\ 
 & zuo den kampfmüeden zwein.\\ 
 & die \textbf{vünf} wurden des in ein,\\ 
 & si wolten scheiden \textbf{disen} strît.\\ 
10 & scheidens dûhte rehtiu zît\\ 
 & Gramolanzen, der sô sprach,\\ 
 & daz er dem siges jach,\\ 
 & den man \textbf{dâ hete gegen ime} \textbf{ersehen}.\\ 
 & des \textbf{muos} \textbf{ouch} mêr \textbf{liutes} jehen.\\ 
15 & dô sprach des \textbf{küniges} Lotes sun:\\ 
 & "hêr künic, ich wil \textbf{iu} hiute tuon,\\ 
 & als ir mir gestern tâtet,\\ 
 & dô ir mich ruowen bâtet.\\ 
 & nû ruowet hînt, \textbf{daz tuot} iu nôt.\\ 
20 & wer iu disen strît gebôt,\\ 
 & der het \textbf{iu swache} kraft erkant\\ 
 & gegen mîner werlîchen hant.\\ 
 & ich bestüende iuch \textbf{nû} wol ein,\\ 
 & \textbf{ir veht\textit{et} aber} niht wan mit zwein.\\ 
25 & ich wilz morgen wâgen eine;\\ 
 & got ez zuo reht erscheine."\\ 
 & der künic reit \textbf{dannen mit} den sîn.\\ 
 & er tet \textbf{dô} fîanzen \textbf{schîn},\\ 
 & daz er morgens gegen Gawan\\ 
30 & durch \textbf{v\textit{e}hten} kæme ûf de\textit{n} plân.\\ 
\end{tabular}
\scriptsize
\line(1,0){75} \newline
m n o Fr69 \newline
\line(1,0){75} \newline
\newline
\line(1,0){75} \newline
\textbf{1} Bernout] berneus m bernont n burnunt o  $\cdot$ de] vnd de o  $\cdot$ Riviers] viers m rivirs o \textbf{2} Offinamus] offainamus m affmanius n hoffmans o  $\cdot$ Clitiers] cliaers o \textbf{3} houbten] bobtten m \textbf{4} riten dem] Rittem m Ritte dem o \textbf{6} dem] den n o \textbf{11} Gramolanzen] Gramolantzen m n Gramolanczes o \textbf{12} siges] sigen o \textbf{13} dâ] do n o \textbf{14} liutes] [lútes]: lúte o \textbf{15} Lotes] lotz m n o \textbf{17} tâtet] tatten m (n) (o) \textbf{18} bâtet] batten m (n) (o) ::: Fr69 \textbf{19} hînt daz tuot] hin des wirt Fr69 \textbf{20} wer] Swer Fr69 \textbf{23} bestüende] bestvͦnt Fr69 \textbf{24} vehtet] fehtte m (o) \textbf{27} dannen] dennan n \textbf{28} dô] E n (o)  $\cdot$ fîanzen] wianczen o \textbf{29} morgens] morgen o \textbf{30} vehten] fahtten m  $\cdot$ kæme] kome o  $\cdot$ den] dem m \newline
\end{minipage}
\end{table}
\newpage
\begin{table}[ht]
\begin{minipage}[t]{0.5\linewidth}
\small
\begin{center}*G
\end{center}
\begin{tabular}{rl}
 & \begin{large}U\end{large}nde \textbf{Gernout} \textbf{von} Rivirs\\ 
 & unde \textbf{Affinamus} \textbf{de} Cletirs,\\ 
 & mit blôzen houbten dise drî\\ 
 & \textbf{dem strîte riten} \textbf{nâhe} bî.\\ 
5 & Artus unde Gawan\\ 
 & riten anderhalp ûf \textbf{den} plân\\ 
 & zuo den kampfmüeden zwein.\\ 
 & die wurden des enein,\\ 
 & si wolden scheiden \textbf{den} strît.\\ 
10 & scheidens dûhte rehtiu zît\\ 
 & Gramoflanz, der sô sprach,\\ 
 & daz er dem siges jach,\\ 
 & de\textit{n} man \textbf{gein im hete} \textbf{gesehen}.\\ 
 & des \textbf{muose} mêre \textbf{liute} jehen.\\ 
15 & dô sprach des \textbf{künic} Lotes sun:\\ 
 & "hêr künic, ich wil \textbf{iu} hiute tuon,\\ 
 & als ir mir gester tâtet,\\ 
 & dô ir mich ruowen bâtet.\\ 
 & nû ruowet hînt, \textbf{des wirt} iu nôt.\\ 
20 & swer iu disen strît gebôt,\\ 
 & der hât \textbf{iu swache} kraft erkant\\ 
 & gein mîner werlîchen hant.\\ 
 & ich bestüende iuch \textbf{nû} wol ein,\\ 
 & \textbf{nû veht aber ir} niwan mit zwein.\\ 
25 & ich wilz morgen wâgen eine;\\ 
 & got ez ze rehte erscheine."\\ 
 & der künic reit \textbf{gein} den sînen.\\ 
 & er tet \textbf{ouch} fîanze \textbf{schîn\textit{en}},\\ 
 & daz er\textbf{s} morgens gein Gawan\\ 
30 & durch \textbf{strîten} kœme ûf den plân.\\ 
\end{tabular}
\scriptsize
\line(1,0){75} \newline
G I L M Z Fr18 \newline
\line(1,0){75} \newline
\textbf{1} \textit{Initiale} G L Z Fr18  \textbf{11} \textit{Initiale} I  \textbf{15} \textit{Initiale} M  \newline
\line(1,0){75} \newline
\textbf{1} Gernout] geroubit M bernovt Z  $\cdot$ Rivirs] riuers I riviers L M (Z) (Fr18) \textbf{2} Affinamus] Afamamvz L affmamvs Z afẏ namvs Fr18  $\cdot$ de Cletirs] der Gletiers I decletriers L der clitieres M der Cletiers Z de Cleviers Fr18 \textbf{3} blôzen houbten] blozem haupte I (L)  $\cdot$ dise] da si I \textbf{4} dem strîte riten] Riten dem strite L (M) (Z) (Fr18) \textbf{10} dûhte rehtiu] durc rehte I (M) \textbf{11} Gramoflanz] Gramoflanzen L Gramorflanz M Gramoflantz Z Fr18  $\cdot$ sô] da M \textbf{13} den] Dem G L Dē M  $\cdot$ gein] da gein I \textbf{14} muose] muͦsen I (L) (M)  $\cdot$ mêre] ouch mere L M Z (Fr18)  $\cdot$ jehen] sehen L (M) \textbf{15} dô] Da M  $\cdot$ des] der L (M)  $\cdot$ künic] chunges I  $\cdot$ Lotes] lotis M \textbf{17} tâtet] taten M \textbf{18} dô] Da M Z  $\cdot$ ruowen] rvͦuens Fr18  $\cdot$ bâtet] boten M \textbf{19} hînt] hvte L (M) (Z) (Fr18) \textbf{20} swer] Wer L M Der Z \textbf{21} iu] uwer L  $\cdot$ swache] swachen M \textbf{23} bestüende] bestund Z  $\cdot$ nû] noch M  $\cdot$ ein] al [eine]: ein Z [eine]: ein Fr18 \textbf{24} veht] en vecht M  $\cdot$ niwan] niwer I nevr Z wan Fr18 \textbf{27} gein] dan gein L Z da geyn M (Fr18) \textbf{28} er] Der M  $\cdot$ schînen] schin G \textbf{29} ers morgens] er smorgen I \textbf{30} kœme] kom L (M) \newline
\end{minipage}
\hspace{0.5cm}
\begin{minipage}[t]{0.5\linewidth}
\small
\begin{center}*T
\end{center}
\begin{tabular}{rl}
 & und \textbf{Bernuot} \textbf{von} Riviers\\ 
 & und \textbf{Affinamuor} \textbf{de} Cletiers,\\ 
 & mit blôzen houbeten dise drî\\ 
 & \textbf{riten dem strîte} \textbf{nâhe} bî.\\ 
5 & \begin{large}A\end{large}rtus und Gawan\\ 
 & riten anderhalp ûf \textbf{den} plân\\ 
 & zuo de\textit{n} kampfmüeden zwein.\\ 
 & die wurden des enein,\\ 
 & si wolten scheiden \textbf{den} strît.\\ 
10 & scheidens dûht\textit{e} \textit{r}ehtiu zît\\ 
 & Gramoflanzen, der sô sprach,\\ 
 & daz er dem siges jach,\\ 
 & den man \textbf{gein im hete} \textbf{gesehen}.\\ 
 & des \textbf{muosen} \textbf{ouch} mê \textbf{liute} jehen.\\ 
15 & dô sprach des \textbf{küneges} Lotes suon:\\ 
 & "hêr künec, ich wil \textbf{iu} hiute tuon,\\ 
 & als ir mir gestern tâtet,\\ 
 & dô ir mich ruowen bâtet.\\ 
 & nû ruowet hînt, \textbf{des wirt} iu nôt.\\ 
20 & wer iu disen strît gebôt,\\ 
 & der hât \textbf{zuo swacher kraft iuch} erkant\\ 
 & gein mîner werlîchen hant.\\ 
 & ich bestüende iuch wol ein,\\ 
 & \textbf{nû vehtet ir aber} niht wan mit zwein.\\ 
25 & ich wil ez morne wâgen eine;\\ 
 & got ez zuo reht erscheine."\\ 
 & der künec reit \textbf{gein} den sînen.\\ 
 & er tet \textbf{ouch} fîanze \textbf{schînen},\\ 
 & daz er \textbf{des} morgens gein Gawan\\ 
30 & durch \textbf{strîten} k\textit{œ}m ûf den plân.\\ 
\end{tabular}
\scriptsize
\line(1,0){75} \newline
U V W Q R \newline
\line(1,0){75} \newline
\textbf{3} \textit{Initiale} R  \textbf{5} \textit{Initiale} U V W  \textbf{27} \textit{Initiale} W  \newline
\line(1,0){75} \newline
\textbf{1} Bernuot] Bercovt U Bernvt V bernout W (R) bernaft Q  $\cdot$ Riviers] riuers W kyfris Q \textbf{2} Affinamuor] affinamus V W Q R  $\cdot$ Cletiers] Clementiers U kleiters W klirs Q \textbf{4} dem strîte] [den striten]: dē strite V  $\cdot$ nâhe] nahen W (Q) \textbf{5} KVnig artus vnd herr gawan W \textbf{7} den] dem U Q \textbf{8} Die [vún* wu*]: vúnfe wurdent dez in ein V  $\cdot$ des] der W \textbf{10} dûhte rehtiu] duchte sie rechte U duchtte rechte R \textbf{11} Gramoflanzen] Gramaflanzen V Gramoflantz W Gramoflantzen Q Gramoflancz R  $\cdot$ sô] do V Q also W \textbf{13} hete gesehen] [*]: do hette ersehen V \textbf{14} muosen] muͤste V (Q) (R) \textbf{15} küneges] kúnig W  $\cdot$ Lotes] lottes W \textbf{16} ich wil] ioch wil ich W \textbf{17} ir] \textit{om.} Q  $\cdot$ gestern] gestert R \textbf{18} ruowen] ruͯwe R \textbf{19} ruowet] ruͦgen W  $\cdot$ des] es W R  $\cdot$ wirt] thuͦt W \textbf{20} wer] Swer V \textbf{21} Der het úch zvͦ swacher craft erkant V  $\cdot$ Der hat úch schwache krafft erkant W (Q) (R) \textbf{23} bestüende] bestund Q bestande R  $\cdot$ wol] wol nv V nun wol W (Q) R \textbf{24} vehtet] vechten W  $\cdot$ ir aber] aber ir Q \textbf{27} gein] [*]: dannan mit V mit R \textbf{28} fîanze schînen] [fian*]: fianze e schin V \textbf{29} des] \textit{om.} W  $\cdot$ gein] gegen herr W gege R \textbf{30} strîten] streite Q (R)  $\cdot$ kœm] quam U komen Q \newline
\end{minipage}
\end{table}
\end{document}
