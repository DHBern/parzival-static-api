\documentclass[8pt,a4paper,notitlepage]{article}
\usepackage{fullpage}
\usepackage{ulem}
\usepackage{xltxtra}
\usepackage{datetime}
\renewcommand{\dateseparator}{.}
\dmyyyydate
\usepackage{fancyhdr}
\usepackage{ifthen}
\pagestyle{fancy}
\fancyhf{}
\renewcommand{\headrulewidth}{0pt}
\fancyfoot[L]{\ifthenelse{\value{page}=1}{\today, \currenttime{} Uhr}{}}
\begin{document}
\begin{table}[ht]
\begin{minipage}[t]{0.5\linewidth}
\small
\begin{center}*D
\end{center}
\begin{tabular}{rl}
\textbf{30} & \textbf{ob er wolde} baneken rîten,\\ 
 & "\textbf{unt} \textbf{schouwet}, wâ wir strîten,\\ 
 & wie unser porten \textbf{sîn} behuot."\\ 
 & Gahmuret, der \textbf{degen} guot,\\ 
5 & sprach, er wolde gerne sehen,\\ 
 & wâ rîterschaft dâ wære geschehen.\\ 
 & Her \textbf{ab} mit dem helde reit\\ 
 & manec rîter \textbf{vil} gemeit,\\ 
 & hie der wîse, dort der tumbe.\\ 
10 & si vuorten in al umbe\\ 
 & vür sehzehen porten.\\ 
 & \textbf{si} beschieden \textbf{im} mit worten,\\ 
 & daz der decheiniu wære \textbf{bespart},\\ 
 & sît \textbf{würde} gerochen Isenhart,\\ 
15 & "\textbf{an uns mit zorne} naht \textbf{noch} tac.\\ 
 & unser strît vil nâch \textbf{gelîche} wac.\\ 
 & man \textbf{beslôz} ir decheine sît.\\ 
 & uns \textbf{gît} vor ähte porten strît\\ 
 & des \textbf{getriwen} Isenhartes man.\\ 
20 & die \textbf{habent uns schaden vil} getân.\\ 
 & si \textbf{ringent} mit zorne,\\ 
 & die vürsten wol geborne,\\ 
 & des küneges \textbf{man} von Azagouc.\\ 
 & \textbf{von} ieslîcher porte vlouc\\ 
25 & ob küener schar ein liehter van,\\ 
 & ein durchstochen rîter dran,\\ 
 & als Isenhart den lîp verlôs.\\ 
 & sîn volc diu wâpen dâ nâch kôs.\\ 
 & \begin{large}D\end{large}â gein hân wir einen site,\\ 
30 & dâ stille wir ir jâmer mite.\\ 
\end{tabular}
\scriptsize
\line(1,0){75} \newline
D \newline
\line(1,0){75} \newline
\textbf{7} \textit{Majuskel} D  \textbf{29} \textit{Initiale} D  \newline
\line(1,0){75} \newline
\textbf{4} Gahmuret] Gahmvret D \textbf{14} Isenhart] Jsenhart D \textbf{19} Isenhartes] Jsenhartes D \textbf{23} Azagouc] Azagoͮch D \textbf{27} Isenhart] Jsenhart D \newline
\end{minipage}
\hspace{0.5cm}
\begin{minipage}[t]{0.5\linewidth}
\small
\begin{center}*m
\end{center}
\begin{tabular}{rl}
 & "\textbf{hêrre, welt ir} baneken rîten,\\ 
 & \textbf{und} \textbf{schouwen}, wâ wir strîten,\\ 
 & wie unser porten \textbf{sint} behuot."\\ 
 & Gahmuret, der \textbf{degen} guot,\\ 
5 & sprach, er wolte gerne sehen,\\ 
 & wâ ritterschaft d\textit{â} wære geschehen.\\ 
 & \textit{\begin{large}H\end{large}}er \textbf{abe} mit dem helde reit\\ 
 & manic ritter \textbf{vil} gemeit,\\ 
 & hie der wîse, dort der tumbe.\\ 
10 & si vuorten in alle umbe\\ 
 & vür sehzehen porten.\\ 
 & \textbf{si} besch\textit{ie}den \textbf{in} mit worten,\\ 
 & daz der \dag næhste nû\dag  wære \textbf{gespart},\\ 
 & sît \textbf{würt} gerochen Ysenhart,\\ 
15 & "\textbf{an uns mit zorne} naht \textbf{und} tac.\\ 
 & unser strît vi\textit{l n}âch \textbf{glîche} wa\textit{c}.\\ 
 & man \textbf{beslôz} ir deke\textit{i}ne sît.\\ 
 & uns \textbf{gâben} vor aht porten strît\\ 
 & des \textbf{getriuwen} Ysenhartes man.\\ 
20 & die \textbf{habent uns schaden vil} getân.\\ 
 & si \textbf{müejent} mit zorne,\\ 
 & die vürsten wol geborne,\\ 
 & des küniges \textbf{man} von Azago\textit{uc}.\\ 
 & \textbf{vor} ieclîcher porten vlou\textit{c}\\ 
25 & o\textit{b} küene\textit{r s}char ein liehter vane,\\ 
 & ein durchstochener ritter drane,\\ 
 & als Ysenhart den lîp verlôs.\\ 
 & sîn volc diu wâpen dâ nâch kôs.\\ 
 & dâ gegen hân wir eine\textit{n} site,\\ 
30 & dâ stillen wir ir jâmer mite.\\ 
\end{tabular}
\scriptsize
\line(1,0){75} \newline
m n o W \newline
\line(1,0){75} \newline
\textbf{3} \textit{Initiale} W  \textbf{7} \textit{Initiale} m   $\cdot$ \textit{Capitulumzeichen} n  \textbf{29} \textit{Initiale} W  \newline
\line(1,0){75} \newline
\textbf{1} welt] woͤllen W  $\cdot$ baneken] \textit{om.} W \textbf{2} wir] ir W  $\cdot$ strîten] [schri]: striten n \textbf{3} sint] sein W \textbf{4} Gahmuret] Gamiret n Gamuͯret o Gamuret W \textbf{6} dâ wære] do were m (o) (W) were do n  $\cdot$ geschehen] beschehen n o \textbf{7} Her abe] Serabe \textit{(Initialbuchstabe }h \textit{vorgeschrieben)} m  $\cdot$ helde] helden m \textbf{8} vil] so n o W \textbf{10} alle umbe] alvmbe n (o) (W) \textbf{12} si] Vnd n o W  $\cdot$ beschieden] bescheiden m o \textbf{13} næhste] veste W \textbf{14} sît] Suß W  $\cdot$ Ysenhart] ÿsenhart m ẏsenhart n jsenhart o \textbf{16} vil nâch] vil gerne nach m  $\cdot$ wac] was m \textbf{17} beslôz] beslosz vns vor n  $\cdot$ dekeine] deckenn \textit{nachträglich korrigiert zu:} deckeÿne m do keinen n dekeinen o keines W \textbf{18} gâben] geben n gebent o W  $\cdot$ vor] fúr o \textbf{19} Ysenhartes] ẏsenhartes m isenhartes n jsenhartes o \textbf{20} schaden] schadens W \textbf{21} müejent] mugent m (n) moͤgen W \textbf{23} Azagouc] azagoͯwe m azagons n azaguͯns o \textbf{24} vlouc] floͯwe m flous n glons o \textbf{25} ob] Ok m  $\cdot$ küener schar] kuͯnner man vnd schar m \textbf{27} Ysenhart] ÿsenhart m ẏsenhart n o \textbf{29} einen] ein*r \textit{nachträglich korrigiert zu:} einen m eine n  $\cdot$ site] sitten o \textbf{30} dâ] [Aa]: Da m  $\cdot$ stillen wir ir] fallent wir in n (o) (W) \newline
\end{minipage}
\end{table}
\newpage
\begin{table}[ht]
\begin{minipage}[t]{0.5\linewidth}
\small
\begin{center}*G
\end{center}
\begin{tabular}{rl}
 & \textbf{ob er wolte} paneken rîten:\\ 
 & "\textbf{schouwet}, wâ wir strîten,\\ 
 & wie unser borte \textbf{sîn} behuot."\\ 
 & Gahmuret, der \textbf{helt} guot,\\ 
5 & sprach, er wolte gerne sehen,\\ 
 & wâ rîterschaft dâ wære geschehen.\\ 
 & her \textbf{abe} mit dem helde reit\\ 
 & manic rîter gemeit,\\ 
 & hie der wîse, dort der tumbe.\\ 
10 & si vuorten in alumbe\\ 
 & vür sehzehen borten\\ 
 & \textbf{unde} beschieden \textbf{im} mit worten,\\ 
 & daz der deheiniu wære \textbf{verspart},\\ 
 & "sît \textbf{wart} gerochen Ysenhart\\ 
15 & \textbf{mit zorne an uns} naht \textbf{und} tac.\\ 
 & unser strît vil nâch \textbf{gelîche} wac.\\ 
 & man \textbf{verlô\textit{s}} ir deheine sît.\\ 
 & uns \textbf{gît} vor ahte porten strît\\ 
 & des \textbf{küenen} Ysenhartes man,\\ 
20 & \begin{large}D\end{large}ie \textbf{uns den schaden hânt} getân,\\ 
 & \multicolumn{1}{l}{ - - - }\\ 
 & \multicolumn{1}{l}{ - - - }\\ 
 & des küniges von Azagouc.\\ 
 & \textbf{obe} iegeslîcher porte vlouc\\ 
25 & obe küener schar ein liehter van,\\ 
 & ein durchstochen rîter dran\\ 
 & als Ysenhart, \textbf{der} den lîp verlôs.\\ 
 & sîn volc diu wâpen dar nâch kôs.\\ 
 & dâ engegene haben wir einen site,\\ 
30 & dâ stillen wir ir jâmer mite.\\ 
\end{tabular}
\scriptsize
\line(1,0){75} \newline
G O L M Q R Z Fr29 Fr32 \newline
\line(1,0){75} \newline
\textbf{20} \textit{Initiale} G  \textbf{29} \textit{Initiale} L Q R Z Fr32  \newline
\line(1,0){75} \newline
\textbf{1} wolte paneken] banichen wolde O (L) wolte nun Q kurcz wilen woͯlte R \textbf{2} schouwet] Vnd schawet O (L) (M) (Q) (R) (Z) (Fr32) \textbf{3} borte] porten O R Z Fr32 phorten M (Q)  $\cdot$ sîn] sy M (Q) sind R \textbf{4} Gahmuret] Gahmvret G Gamvret O Gahmuͯret L Gamuret M Z Gamuert Q Gahmuͦret R gamvͦret Fr32 \textbf{5} wolte] woldes Q  $\cdot$ sehen] schawen Q \textbf{6} rîterschaft dâ] da ritterschafft M ritterschafft do Q  $\cdot$ wære] weren Q \textbf{7} her] Er R \textbf{8} gemeit] vil gemæit O (L) (M) (Q) (Z) (Fr29) (Fr32) wol gemeit R \textbf{10} alumbe] alle vmbe M R \textbf{11} borten] purten M pforten Q \textbf{12} beschieden] bescheidin M (Q) (R)  $\cdot$ im] yn M (Q) (R) (Z) \textbf{13} daz] Dar Q  $\cdot$ deheiniu] doheime Q  $\cdot$ wære] \textit{om.} M wurde R Z \textbf{14} sît] Ezn Q  $\cdot$ wart] wurde Q Z  $\cdot$ Ysenhart] ẏsenhart G isenhart O Z eysenhart Q Jsenbart R Jsenhart Fr29 \textbf{15} mit zorne an uns] An vns mit zorn O (L) (M) (Q) (R) (Z) Fr29 \textbf{16} wac] lac Z \textbf{17} verlôs] verloz G besloz O (L) M (Q) (R) Z (Fr29) \textbf{18} gît] gebent Q  $\cdot$ porten] pforten Q \textbf{19} küenen] getriwen O (L) (M) (Q) (R) (Z) (Fr29)  $\cdot$ Ysenhartes] isenhartes G O Jsenhartes L Jsenhartes M eysenhartes Q Jsenharttes R isenharten Z (Fr32) Jsenhar::: Fr29 \textbf{20} uns den schaden hânt] habent vns schaden vil O (M) (Q) (R) (Z) (Fr29) \textbf{21} \textit{Die Verse 30.21-22 fehlen} G   $\cdot$ Si ringent (ringent alle Q ) mit zorne O (L) (M) (Q) (R) (Z) (Fr29) \textbf{22} Die fvrsten wol geborne O (L) (M) (Q) (R) (Z) (Fr29) (Fr32) \textbf{23} küniges] chvniges man O (L) (M) (Q) (R) (Z) (Fr29) (Fr32)  $\cdot$ Azagouc] azagoͮch G azagvͦch O Azagoch L azaguk Q \textbf{24} obe] Vor O L M Q R Z (Fr29)  $\cdot$ porte] porten O L M R Z (Fr29) Fr32 pforten Q \textbf{25} küener] yeder R  $\cdot$ liehter] lýchter L (M) licher Q \textbf{27} Ysenhart] isenhart G Z Jsenhart M eysenhart Q ẏsenhart Fr32  $\cdot$ der] \textit{om.} O L M Q R Z Fr29 Fr32 \textbf{28} diu] sin R \textbf{29} dâ] Do Q  $\cdot$ engegene] gegen O  $\cdot$ site] siten Q \newline
\end{minipage}
\hspace{0.5cm}
\begin{minipage}[t]{0.5\linewidth}
\small
\begin{center}*T
\end{center}
\begin{tabular}{rl}
 & \textbf{ob er wolte} baneken rîten,\\ 
 & "\textbf{und} \textbf{schouwet}, wâ wir strîten,\\ 
 & wie unser porten \textbf{sîn} behuot."\\ 
 & Gahmuret, der \textbf{helt} guot,\\ 
5 & sprach, er wolte gerne sehen,\\ 
 & wâ rîterschaft dâ wære geschehen.\\ 
 & Her \textbf{ûz dô} mit dem helde reit\\ 
 & manec rîter \textbf{vil} gemeit,\\ 
 & hie der wîse, dort der tumbe.\\ 
10 & si vuorten in alumbe\\ 
 & vür sehzehen porten\\ 
 & \textbf{und} beschieden \textbf{im} mit worten,\\ 
 & daz der deheiniu wære \textbf{verspart},\\ 
 & "sît \textbf{wart} gerochen Isenhart\\ 
15 & \textbf{an uns mit vlîze} naht \textbf{und} tac.\\ 
 & unser strît vil nâch \textbf{ringe} wac.\\ 
 & man \textbf{beslôz} ir dehein\textit{e} sît.\\ 
 & uns \textbf{gît} vor ahte porten strît\\ 
 & des \textbf{getriuwen} Isenhartes man.\\ 
20 & die \textbf{hânt uns schaden vil} getân.\\ 
 & si \textbf{ringent} mit zorne,\\ 
 & die vürsten wol geborne,\\ 
 & des küneges \textbf{man} von Azagouc.\\ 
 & \textbf{vor} ieslîcher porten vlouc\\ 
25 & ob küener schar ein liehter vane,\\ 
 & ein durchstochen rîter drane,\\ 
 & als Isenhart den lîp verlôs.\\ 
 & sîn volc diu wâpen dar nâch kôs.\\ 
 & Dâ engegen hân wir einen site,\\ 
30 & dâ stillen wir ir jâmer mite.\\ 
\end{tabular}
\scriptsize
\line(1,0){75} \newline
T U V \newline
\line(1,0){75} \newline
\textbf{7} \textit{Majuskel} T  \textbf{29} \textit{Majuskel} T  \newline
\line(1,0){75} \newline
\textbf{2} schouwet wâ wir] scheuͦwen wo sie mohten U schowen swo [*]: sú moͤhten V \textbf{3} Vnd wie ir (die V ) porten weren behuͦt U (V) \textbf{4} Gahmuret] Gahmvret T Gahmuͦret U Gamuret V \textbf{7} Her ûz] Er vz U Har abe V  $\cdot$ dô] \textit{om.} V \textbf{9} der] [dem]: der V \textbf{12} im] in U \textbf{13} verspart] bespart U \textbf{14} wart] [*]: wurde V  $\cdot$ Isenhart] Jsenhart T U Jsinhart V \textbf{15} vlîze] [*]: zorne V \textbf{16} ringe] gelich U (V) \textbf{17} deheine] deheiniv T \textbf{18} gît] gebot U \textbf{19} getriuwen] [gerriuwen]: getriuwen T  $\cdot$ Isenhartes] Jsenhartes T U Jsinhartes V \textbf{23} Azagouc] Azagôvc T azaguͦc U azagoͮg V \textbf{25} küener] keine U ie der V \textbf{27} Isenhart] Jsenhart U V \textbf{29} Dâ] [D*]: Do V  $\cdot$ engegen] gein U  $\cdot$ site] siden U \newline
\end{minipage}
\end{table}
\end{document}
