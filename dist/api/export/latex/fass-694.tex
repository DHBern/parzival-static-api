\documentclass[8pt,a4paper,notitlepage]{article}
\usepackage{fullpage}
\usepackage{ulem}
\usepackage{xltxtra}
\usepackage{datetime}
\renewcommand{\dateseparator}{.}
\dmyyyydate
\usepackage{fancyhdr}
\usepackage{ifthen}
\pagestyle{fancy}
\fancyhf{}
\renewcommand{\headrulewidth}{0pt}
\fancyfoot[L]{\ifthenelse{\value{page}=1}{\today, \currenttime{} Uhr}{}}
\begin{document}
\begin{table}[ht]
\begin{minipage}[t]{0.5\linewidth}
\small
\begin{center}*D
\end{center}
\begin{tabular}{rl}
\textbf{694} & \begin{large}D\end{large}ô \textbf{dis} zornes vil geschach,\\ 
 & der künec Benen \textbf{sunder} sprach;\\ 
 & \textbf{er} bat \textbf{si}: "vrouwe, \textbf{zürne} niht,\\ 
 & daz der kampf von mir geschiht.\\ 
5 & belîp bî dem hêrren dîn.\\ 
 & sage Itonjen, der \textbf{swester sîn},\\ 
 & ich sî vür wâr ir dienstman\\ 
 & unt \textbf{ich} \textbf{wil} ir dienen, swaz ich kan."\\ 
 & Dô Bene \textbf{daz} \textbf{gehôrte}\\ 
10 & \textbf{mit} wærlîchem worte,\\ 
 & daz \textbf{ir hêrre} ir vrouwen bruoder was,\\ 
 & der dâ solde strîten ûfem gras,\\ 
 & dô zugen jâmers ruoder\\ 
 & \textbf{in} ir \textbf{herzen} wol ein vuoder\\ 
15 & der herzenlîchen riwe,\\ 
 & wande si pflac herzen triwe.\\ 
 & Si sprach: "\textbf{vart} hin, vervluochet man!\\ 
 & ir sît, der triwe nie gewan."\\ 
 & \textbf{Der künec reit dan} \textbf{unt al die} sîn.\\ 
20 & Artuses junchêrrelîn\\ 
 & viengen dors disen zwein.\\ 
 & an den orsen sunder kampf \textbf{ouch} schein.\\ 
 & Gawan und Parzival\\ 
 & unt Bene, diu lieht gemâl,\\ 
25 & riten \textbf{dannen} gein ir her.\\ 
 & Parzival mit mannes wer\\ 
 & het den prîs behalden sô,\\ 
 & si wâren sîner künfte vrô;\\ 
 & die in \textbf{dâ} komen sâhen,\\ 
30 & \textbf{hôhes} prîses si im alle jâhen.\\ 
\end{tabular}
\scriptsize
\line(1,0){75} \newline
D \newline
\line(1,0){75} \newline
\textbf{1} \textit{Initiale} D  \textbf{9} \textit{Majuskel} D  \textbf{17} \textit{Majuskel} D  \textbf{19} \textit{Majuskel} D  \newline
\line(1,0){75} \newline
\textbf{6} Itonjen] Jtonîen D \textbf{20} Artuses] Artvss D \textbf{23} Parzival] Parcifal D \textbf{26} Parzival] Parcifal D \newline
\end{minipage}
\hspace{0.5cm}
\begin{minipage}[t]{0.5\linewidth}
\small
\begin{center}*m
\end{center}
\begin{tabular}{rl}
 & dô \textbf{de\textit{s}} zornes vil geschach,\\ 
 & der künic Benen \textbf{sunder} sprach;\\ 
 & \textbf{er} bat \textbf{si}: "vrowe, \textbf{zürnet} niht,\\ 
 & da\textit{z} der kampf von mir geschiht.\\ 
5 & blîp \textbf{hie} bî dem hêrren dîn.\\ 
 & sage Ithonie, der \textbf{vrowen mîn},\\ 
 & ich sî vür wâr ir dienestman\\ 
 & und \textbf{welle} ir dienen, waz ich kan."\\ 
 & dô Bene \textbf{daz} \textbf{erhôrte}\\ 
10 & \textbf{mit} wærlîchem worte,\\ 
 & daz \textbf{ir hêrre} ir vrowen bruoder was,\\ 
 & der d\textit{â} solte strîten ûf dem gras,\\ 
 & dô zugen jâmers ruoder\\ 
 & \textbf{in} ir \textbf{herze} wol ein vuoder\\ 
15 & der herzelîchen riuwe,\\ 
 & wan si pflac herzen triuwe.\\ 
 & si sprach: "\textbf{var} hin, vervluochet man!\\ 
 & ir sît, der triuwen nie g\textit{e}wan."\\ 
 & \textbf{\begin{large}D\end{large}er künic reit dan} \textbf{und die} sîn.\\ 
20 & Artuses junchêrrelîn\\ 
 & viengen diu ros disen zwein.\\ 
 & an den rossen sunder kampf \textbf{ouch} schein.\\ 
 & Gawan und Parcifal\\ 
 & und Bene, diu lieht \textit{g}emâl,\\ 
25 & riten \textbf{dannen} gegen ir her.\\ 
 & Parcifal mit mannes wer\\ 
 & het den prîs behalten sô,\\ 
 & si wâren sîner künfte vrô;\\ 
 & die in \textbf{d\textit{â}} komen sâhen,\\ 
30 & \textbf{hô\textit{h}es} prîses si  alle jâhen.\\ 
\end{tabular}
\scriptsize
\line(1,0){75} \newline
m n o Fr69 \newline
\line(1,0){75} \newline
\textbf{19} \textit{Initiale} m   $\cdot$ \textit{Capitulumzeichen} n  \newline
\line(1,0){75} \newline
\textbf{1} des] der m \textit{om.} o \textbf{2} Benen] bene o \textbf{4} daz] Dar m \textbf{5} blîp] Bleip m o \textbf{6} Ithonie] Jthonie m itone o \textbf{8} welle] wellen o \textbf{12} dâ] do m n o \textbf{13} \textit{Verse 694.13 und 694.15 kontrahiert zu:} Do zugen jamers ruwe o  \textbf{14} \textit{Vers 694.14 fehlt} o  \textbf{18} gewan] gawan m \textbf{24} lieht gemâl] liehtte mal m \textbf{29} dâ] do m n o \textbf{30} hôhes] Hoses m  $\cdot$ jâhen] jagen o \newline
\end{minipage}
\end{table}
\newpage
\begin{table}[ht]
\begin{minipage}[t]{0.5\linewidth}
\small
\begin{center}*G
\end{center}
\begin{tabular}{rl}
 & \begin{large}D\end{large}ô \textbf{des} zornes vil geschach,\\ 
 & der künic \textbf{ze vroun} Benen sprach;\\ 
 & \textbf{die} bat \textbf{er}: "vrouwe, \textbf{zürne} niht,\\ 
 & daz der kampf von mir geschiht,\\ 
5 & \textbf{unde} belîp \textbf{hie} bî dem hêrren dîn\\ 
 & \textbf{unde} sage Itonien, der \textbf{swester sîn},\\ 
 & ich sî vür wâr ir dienstman\\ 
 & unde \textbf{welle} ir dienen, swaz ich kan."\\ 
 & dô \textbf{vrou} Bene \textbf{dô} \textbf{gehôrte}\\ 
10 & \textbf{von} wærlîchem worte,\\ 
 & daz \textbf{er} ir vrouwen bruoder was,\\ 
 & der dâ solde strîten ûfme gras,\\ 
 & dô zugen \textbf{si}, jâmers ruoder,\\ 
 & \textbf{an} ir \textbf{herzen} wol ein vuoder\\ 
15 & der herzenlîchen riwe,\\ 
 & wan si pflac herzen triwe.\\ 
 & si sprach: "\textbf{vart} hin, vervluochet man!\\ 
 & ir sît, der triwe nie gewan."\\ 
 & \textbf{hin reit der künic} \textbf{gein den} sîn.\\ 
20 & Artuses junchêrrelîn\\ 
 & viengen diu ors disen zwein.\\ 
 & an den ors\textit{en} sunder kampf schein.\\ 
 & Gawan unde Parcival\\ 
 & unde \textbf{vrou} Bene, diu lieht gemâl,\\ 
25 & riten \textbf{wider} gein ir her.\\ 
 & Parcival mit mannes wer\\ 
 & het den brîs behalten sô,\\ 
 & si wâren sîner künfte vrô;\\ 
 & die in \textbf{dâ} komen sâhen,\\ 
30 & \textbf{des} brîses sim alle jâhen.\\ 
\end{tabular}
\scriptsize
\line(1,0){75} \newline
G I L M Z Fr20 \newline
\line(1,0){75} \newline
\textbf{1} \textit{Initiale} G I Z Fr20  \textbf{19} \textit{Initiale} I  \newline
\line(1,0){75} \newline
\textbf{1} Dô] Da M ÷o Fr20 \textbf{2} ze vroun] [*o*]: frov L \textit{om.} Z  $\cdot$ sprach] sunder sprach I (L) Z (Fr20) \textbf{3} zürne] zurnet I (L) (Fr20) \textbf{4} von mir] zcu vil M \textbf{5} unde] \textit{om.} I Z \textbf{6} unde] \textit{om.} Z  $\cdot$ Itonien] Jtone M Jconie Z  $\cdot$ swester] swerster M \textbf{8} welle] ich welle Z  $\cdot$ swaz] swa I waz L (M) Z \textbf{9} dô vrou] Da vrowe M (Z)  $\cdot$ dô gehôrte] gehorte I L da gihorte M (Z) \textbf{10} wærlîchem] werliche: I warlichen L (Fr20) \textbf{11} er] ir herre L (M) Z \textbf{13} dô] Da M Z  $\cdot$ si] \textit{om.} L Z \textbf{14} herzen] herze I (L) (M) \textbf{15} herzenlîchen] herzenlicher I hertzecliche L \textbf{16} herzen] \textit{om.} I \textbf{17} vervluochet] ir verfluchter I verfluchte M \textbf{20} Artuses] Artvs G (Z) (Fr20)  $\cdot$ junchêrrelîn] iuncfrowelin Z \textbf{21} diu] disev I \textbf{22} orsen] ors G  $\cdot$ sunder] auch der I ouch Z  $\cdot$ schein] och schein L (M) (Fr20) sunder Z \textbf{23} Parcival] parcifal G Z Parzifal I (L) (M) (Fr20) \textbf{24} lieht] lieh I lichte L (M) \textbf{25} wider] \textit{om.} M  $\cdot$ her] herre M \textbf{26} Parcival] Parcifal G Z Parzifal I L M (Fr20)  $\cdot$ wer] [here]: were M \textbf{27} het den] Hat [des]: den M \textbf{28} künfte] kunste M chvfte Fr20  $\cdot$ vrô] vnfro L \textbf{29} dâ] \textit{om.} L M  $\cdot$ komen sâhen] kome sagin M \newline
\end{minipage}
\hspace{0.5cm}
\begin{minipage}[t]{0.5\linewidth}
\small
\begin{center}*T
\end{center}
\begin{tabular}{rl}
 & \begin{large}D\end{large}ô \textbf{des} zornes vil geschach,\\ 
 & der künec \textbf{zuo vrô} Benen \textbf{sunder} sprach;\\ 
 & \textbf{die} bat \textbf{er}: "vrouwe, \textbf{zürne} niht,\\ 
 & daz der kampf von mir geschiht,\\ 
5 & \textbf{und} belîp \textbf{hie} bî dem hêrren dîn\\ 
 & \textbf{und} sage Itonien, der \textbf{swester sîn},\\ 
 & ich sî vür wâr ir dienstman\\ 
 & und \textbf{wil} ir dienen, waz ich kan."\\ 
 & dô \textbf{vrou} Bene \textbf{daz} \textbf{gehôrte}\\ 
10 & \textbf{mit} wærlîchem worte,\\ 
 & daz \textbf{ir hêrre} ir vrouwen bruoder was,\\ 
 & der d\textit{â} solte strîten ûf dem gras,\\ 
 & dô zugen \textbf{ir} jâmers ruoder\\ 
 & \textbf{an} ir \textbf{herze} wol ein vuoder\\ 
15 & der herzeclîchen riuwe,\\ 
 & wan si pflac herzen triuwe.\\ 
 & si sprach: "\textbf{vart} hin, vervluochter man!\\ 
 & ir sît, der triuwe nie gewan."\\ 
 & \textbf{der künec reit dan} \textbf{und die} sîn.\\ 
20 & Artuses junchêrrelîn\\ 
 & viengen diu ors disen zwein.\\ 
 & an den orsen sunder kampf \textbf{ouch} schein.\\ 
 & Gawan und Parcifal\\ 
 & und \textbf{vrou} Bene, diu lieht gemâl,\\ 
25 & riten \textbf{dannen} gein ir her.\\ 
 & Parcifal mit mannes wer\\ 
 & hete den prîs behalten sô,\\ 
 & si wâren sîner künfte vrô;\\ 
 & die in komen sâhen,\\ 
30 & \textbf{des} prîses si im alle jâhen.\\ 
\end{tabular}
\scriptsize
\line(1,0){75} \newline
U V W Q R \newline
\line(1,0){75} \newline
\textbf{1} \textit{Initiale} U  \textbf{17} \textit{Initiale} W  \newline
\line(1,0){75} \newline
\textbf{2} zuo vrô] [*]: zvͦ fron V zwr frawen Q  $\cdot$ Benen] ben Q  $\cdot$ sunder] \textit{om.} W \textbf{3} zürne] [zv́rn*]: zv́rnent V zᵫrnen R \textbf{6} Itonien] Jtonien U R ytonien V W Q  $\cdot$ swester sîn] [*]: swerster sin V \textbf{8} wil] welle V (W) (Q) (R)  $\cdot$ waz] swaz V wo W \textbf{9} daz] [*]: daz V do W \textit{om.} R  $\cdot$ gehôrte] [*]: gehorte U \textbf{10} wærlîchem] werlichen R \textbf{12} dâ] do U V W Q \textbf{15} riuwe] rewen Q \textbf{16} \textit{Vers 694.16 fehlt} Q   $\cdot$ herzen] [*]: steter V ganczer R \textbf{17} vervluochter] ir verfluͯchet R \textbf{19} die] al die V (W) (Q) (R) \textbf{20} Artuses] [Artuse*]: Artuses V Kúnig artus W Artus Q R \textbf{22} ouch] \textit{om.} V W Q  $\cdot$ schein] erschein V Q \textbf{23} Gawan] Herr gawan W Gawin R  $\cdot$ Parcifal] Parzifal U parzefal V partzifal W Q parczifal R \textbf{24} lieht] licht Q \textbf{26} Parcifal] parzefal V partzifal W Q parczifal R \textbf{28} sîner] [*]: siner U \textbf{29} komen] [*]: do komen V \textbf{30} [*]: Hohez prises [*]: sv́ imme alle iahen V \newline
\end{minipage}
\end{table}
\end{document}
