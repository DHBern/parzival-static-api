\documentclass[8pt,a4paper,notitlepage]{article}
\usepackage{fullpage}
\usepackage{ulem}
\usepackage{xltxtra}
\usepackage{datetime}
\renewcommand{\dateseparator}{.}
\dmyyyydate
\usepackage{fancyhdr}
\usepackage{ifthen}
\pagestyle{fancy}
\fancyhf{}
\renewcommand{\headrulewidth}{0pt}
\fancyfoot[L]{\ifthenelse{\value{page}=1}{\today, \currenttime{} Uhr}{}}
\begin{document}
\begin{table}[ht]
\begin{minipage}[t]{0.5\linewidth}
\small
\begin{center}*D
\end{center}
\begin{tabular}{rl}
\textbf{190} & mit \textbf{nazzen} ougen \textbf{verklaget},\\ 
 & ich unt Liaze, diu maget.\\ 
 & \begin{large}S\end{large}ît ir iwerem wirte holt,\\ 
 & sô \textbf{nemtz hînte}, als wir\textbf{z} gedolt\\ 
5 & hie lange hân, wîb unt man.\\ 
 & \textbf{ein teil ir dienet im} dâr an.\\ 
 & ich \textbf{wil} iu \textbf{unsern} kumber \textbf{sagen}.\\ 
 & wir müezen strengen zadel tragen."\\ 
 & Dô sprach ir veter Kyot:\\ 
10 & "vrouwe, ich sende iu zwe\textit{l}f brôt,\\ 
 & schultern und hammen drî.\\ 
 & dâ ligent aht kæse bî\\ 
 & unt zwei \textbf{buzzel} mit wîn.\\ 
 & \textbf{iuch sol ouch} der bruoder mîn\\ 
15 & \textbf{hînte stiuren}, des ist nôt."\\ 
 & dô sprach Manpfiliot:\\ 
 & "vrouwe, ich sende iu als vil."\\ 
 & dô saz diu \textbf{magt} an vreuden zil.\\ 
 & ir grôzer danc wart niht vermiten.\\ 
20 & si nâmen urloup und riten\\ 
 & dâ bî z\textbf{ir} weidehûsen\\ 
 & zer wilder albe k\textit{l}ûsen.\\ 
 & Die alten sâzen \textbf{sunder} wer.\\ 
 & \textbf{si} heten \textbf{ouch} vride vome her.\\ 
25 & \textbf{ir bote} wider \textbf{kam} gedrabt.\\ 
 & des wart diu kranke diet \textbf{gelabt}.\\ 
 & \textbf{dô was} der burgære nar\\ 
 & gedigen an \textbf{dise} spîse gar.\\ 
 & ir was \textbf{vor hunger maneger} tôt,\\ 
30 & ê daz in \textbf{dar kœme dizze} brôt.\\ 
\end{tabular}
\scriptsize
\line(1,0){75} \newline
D \newline
\line(1,0){75} \newline
\textbf{3} \textit{Initiale} D  \textbf{9} \textit{Majuskel} D  \textbf{23} \textit{Majuskel} D  \newline
\line(1,0){75} \newline
\textbf{10} zwelf] zwef D \textbf{22} klûsen] chvͦsen D \newline
\end{minipage}
\hspace{0.5cm}
\begin{minipage}[t]{0.5\linewidth}
\small
\begin{center}*m
\end{center}
\begin{tabular}{rl}
 & mi\textit{t} \textbf{nazzen} ougen \textbf{verklaget},\\ 
 & ich und Liaze, diu maget.\\ 
 & sît ir iuwerm wirte holt,\\ 
 & sô \textbf{nemt ez hînt}, als wir \textbf{ez} gedolt\\ 
5 & hie lange hân, wîp und man.\\ 
 & \textbf{ir dienet im ein teil} dâr an.\\ 
 & ich \textbf{wil} iu \textbf{unsern} kumber \textbf{klagen}.\\ 
 & wir müezen strengen zadel tragen."\\ 
 & \begin{large}D\end{large}ô sprach ir vetere Kyot:\\ 
10 & "vrouwe, ich sende iu zwelf brôt,\\ 
 & schultern und hammen drî.\\ 
 & dâ ligent aht kæs bî\\ 
 & und zwei \dag bucke\dag  mit wîn.\\ 
 & \textbf{ouch sol iuch} der bruoder mîn\\ 
15 & \textbf{hînaht s\textit{t}iur\textit{e}n}, des ist nôt."\\ 
 & dô sprach Manfilot:\\ 
 & "vrouwe, ich sende iu \textit{a}ls vil."\\ 
 & dô saz diu \textbf{maget} an vröuden zil.\\ 
 & ir grôzer danc wart niht vermiten.\\ 
20 & si nâmen urloup und riten\\ 
 & dâ bî zuo \textbf{ir} weidehûsen\\ 
 & zer wilden albe klûsen.\\ 
 & die alten sâzen \textbf{sunder} wer.\\ 
 & \textbf{si} heten vride vomme her.\\ 
25 & \textbf{ir bote} wider \textbf{kam} gedrabet.\\ 
 & des wart diu kranke diet \textbf{gel\textit{a}bet},\\ 
 & \textbf{wanne} der burgære nar\\ 
 & \textbf{was} ged\textit{i}gen an \textbf{die} spîse gar.\\ 
 & ir was \textbf{vor hunger maniger} tôt,\\ 
30 & ê daz in \textbf{k\textit{œ}me dar daz} brôt.\\ 
\end{tabular}
\scriptsize
\line(1,0){75} \newline
m n o Fr69 \newline
\line(1,0){75} \newline
\textbf{9} \textit{Initiale} m   $\cdot$ \textit{Capitulumzeichen} n  \newline
\line(1,0){75} \newline
\textbf{1} mit] Min m \textbf{2} Liaze] lyasse n laize o \textbf{4} nemt ez] nemmpt ers o  $\cdot$ wir ez] wir n \textbf{8} strengen] srengen o \textbf{9} Kyot] kiot m kẏot o \textbf{11} \textit{Versfolge 190.10-12-13-11 korrigiert zu 190.10-12-11-13} n  \textbf{12} aht] sehsse n \textbf{13} bucke] búrtzel n (o)  $\cdot$ mit] min n \textbf{15} stiuren] strureren m  $\cdot$ des] das n o \textbf{16} Manfiliot] Manfilot m \textbf{17} als] wals m  $\cdot$ vil] \textit{om.} n \textbf{18} diu] \textit{om.} o \textbf{21} dâ] Do n o \textbf{22} wilden] willen o \textbf{24} vride] frieden o \textbf{26} des] Das o  $\cdot$ gelabet] geloubet m \textbf{28} gedigen] gedingen m \textbf{29} vor] von n o \textbf{30} kœme dar] kome dar m o das dar keme n \newline
\end{minipage}
\end{table}
\newpage
\begin{table}[ht]
\begin{minipage}[t]{0.5\linewidth}
\small
\begin{center}*G
\end{center}
\begin{tabular}{rl}
 & mit \textbf{nazzen} ougen \textbf{überklaget},\\ 
 & ich unde Liaze, diu maget.\\ 
 & sît ir iweren wirte holt,\\ 
 & sô \textbf{lîdet}, als wir\textbf{z} gedolt\\ 
5 & hie lange hân, wîp unde man.\\ 
 & \textbf{ein teil ir dienet im} dâr an.\\ 
 & ich \textbf{muoz} iu \textbf{unseren} kumber \textbf{klagen}.\\ 
 & wir müezen stre\textit{n}gen zadel tragen."\\ 
 & dô sprach \textit{ir veter} \textit{Ki}ot:\\ 
10 & "vrouwe, ich \textit{sende iu} zwelf brôt,\\ 
 & schulteren unde hammen drî.\\ 
 & dâ ligent ahte kæse bî\\ 
 & unt zwei \textbf{buzzel} mit wîne.\\ 
 & \textbf{ouch sol} der bruoder mîne\\ 
15 & \textbf{hînt stiuren}, des ist n\textit{ôt}."\\ 
 & dô sprach Manfiliot:\\ 
 & "vrouwe, ich sendiu als vil."\\ 
 & dô saz diu \textbf{maget} an vröuden zil.\\ 
 & ir grôzer danc wart niht vermiten.\\ 
20 & si nâmen urloup unde riten\\ 
 & dâ bî z\textbf{ir} \textit{wei}dehûsen\\ 
 & zer wilden \textit{alben} \textit{k}lûsen.\\ 
 & die alten \textit{sâzen} \textbf{\textit{sun}der} wer\\ 
 & \textbf{unde} heten v\textit{ride} von dem her.\\ 
25 & \textbf{ir boten} wider \textbf{komen} gedrabet.\\ 
 & des wart diu kranke diet \textbf{gelabet}.\\ 
 & \textbf{dô was} der burgære nar\\ 
 & gedigen an \textbf{dise} spîse gar.\\ 
 & ir was \textbf{vor hunger maniger} tôt,\\ 
30 & ê daz in \textbf{kœme dar daz} brôt.\\ 
\end{tabular}
\scriptsize
\line(1,0){75} \newline
G I O L M Q R Z \newline
\line(1,0){75} \newline
\textbf{7} \textit{Initiale} I  \textbf{9} \textit{Initiale} L  \textbf{19} \textit{Initiale} I Z  \newline
\line(1,0){75} \newline
\textbf{2} Liaze] liaz I (O) M Z Lýaze L lyasse Q \textit{om.} R  $\cdot$ diu] die selbe R \textbf{3} iweren] ewerm I (O) (L) (M) (Q) Z  $\cdot$ holt] iht holt I \textbf{4} lîdet] lidet hint O (L) (Q) lidet furbaz M lident mit vns R nemtz hint Z  $\cdot$ wirz] wir R  $\cdot$ gedolt] gedoln M \textbf{5} lange hân] haben lange O \textbf{6} Jir dient im ein teil daran R  $\cdot$ ir dienet im] ir im dienet I (O) dienet ir ym M im dient ir Z \textbf{7} \textit{Vers 190.7 fehlt} M   $\cdot$ iu unseren] vnsern O vnsz R \textbf{8} strengen] stre:gen G  $\cdot$ zadel] zcagil M gebresten R  $\cdot$ tragen] haben I (R) \textbf{9} dô] Da M Z  $\cdot$ ir veter] ::: G  $\cdot$ Kiot] ::ot G kyot O M Q R Z kýot L \textbf{10} sende iu] \textit{om.} G \textbf{11} schulteren] Schulter Q \textbf{12} ligent] ligent auch I legen M  $\cdot$ ahte] æhte O \textbf{13} buzzel] bvͤnzel Z \textbf{14} ouch sol] evch sol auch I (L) Jvch sol O (R) Euch Q (Z) \textbf{15} hînt] Vurbaz M  $\cdot$ stiuren] stúrre R stevret Z  $\cdot$ des ist] dest ev I das ist R  $\cdot$ nôt] n:: G \textbf{16} dô] Da Z  $\cdot$ Manfiliot] Malfilot I Manphiliot L man filot R (Z) \textbf{17} sendiu] sende I \textbf{18} dô] Da Z  $\cdot$ maget] frawe O  $\cdot$ an] on R  $\cdot$ zil] vil Q \textbf{19} danc] [dan*]: danch G \textbf{20} nâmen urloup] saszent vff R \textbf{21} dâ] Do Q  $\cdot$ zir] zit Z  $\cdot$ weidehûsen] :::de husen G waide huse I \textbf{22} zer] Ze O (L) (M) (Z) Czu ir Q Ze einer R  $\cdot$ wilden] wilder O M Z wilde L  $\cdot$ alben klûsen] ::: :hlusen G alben chluse I albe clusen L M (Z) ab clausen Q (R) \textbf{23} alten] alter Z  $\cdot$ sâzen sunder wer] ::: :::der wer G sazen wer I saszen svnder ane wer L \textbf{24} Sie ouch fride heten von dem here Z  $\cdot$ vride] f::: G friunde I  $\cdot$ her] hier L \textbf{25} wider komen] chomen wider I (L) (Q)  $\cdot$ gedrabet] [von dem here]: gedrabt Z \textbf{26} Des wart dy gelabit M  $\cdot$ des] [Der]: Des L  $\cdot$ kranke] chranchev O \textbf{27} dô] Da M Z \textbf{28} an dise] andie O (L) (Q) (R) (Z)  $\cdot$ gar] dar M \textbf{29} vor] von I O (Q) R \textbf{30} ê daz] Er dar M E R  $\cdot$ in] im O  $\cdot$ kœme dar] chome dar G (M) (Z) chome I cheͦme O kame dar L kúme dar Q da keme R \newline
\end{minipage}
\hspace{0.5cm}
\begin{minipage}[t]{0.5\linewidth}
\small
\begin{center}*T
\end{center}
\begin{tabular}{rl}
 & mit \textbf{weinenden} ougen \textbf{geklaget},\\ 
 & ich unde Lyaze, diu maget.\\ 
 & sît ir iuwerm wirte holt,\\ 
 & sô \textbf{nemet ez hînt}, alse wir\textbf{s} gedolt\\ 
5 & hie lange hân, wîp unde man.\\ 
 & \textbf{ir eine dienet im} dâr an.\\ 
 & ich \textbf{muoz} iu \textbf{mînen} kumber \textbf{klagen}.\\ 
 & wir müezen strengen zadel tragen."\\ 
 & \begin{large}D\end{large}ô sprach ir veter Kyot:\\ 
10 & "vrouwe, ich sendiu zwelf brôt,\\ 
 & schultern unde hammen drî.\\ 
 & dâ ligen\textit{t} ahte kæse bî\\ 
 & unde zwei \textbf{barel} mit wîne.\\ 
 & \textbf{iuch sol} der bruoder mîne\\ 
15 & \textbf{stiuren hînt}, des ist \textbf{iu} nôt."\\ 
 & Dô sprach Manfilot:\\ 
 & "vrouwe, ich sendiu alse vil."\\ 
 & dô saz diu \textbf{vrouwe} an vröuden zil.\\ 
 & ir grôzer danc wart niht vermiten.\\ 
20 & si nâmen urloup unde riten\\ 
 & dâ bî ze weidehûsen\\ 
 & zer wilden a\textit{l}be klûsen.\\ 
 & die alten sâzen \textbf{âne} wer\\ 
 & \textbf{unde} heten vride vonme her.\\ 
25 & \textbf{Die boten} wider \textbf{komen} gedrabet.\\ 
 & des wart di\textit{u} kranke diet \textbf{erlabet}.\\ 
 & \textbf{dô was} der burgære nar\\ 
 & gedigen an \textbf{die} spîse gar.\\ 
 & ir was \textbf{vil maneger hungers} tôt,\\ 
30 & ê daz in \textbf{kæme dar daz} brôt.\\ 
\end{tabular}
\scriptsize
\line(1,0){75} \newline
T U V W \newline
\line(1,0){75} \newline
\textbf{9} \textit{Initiale} T U V W  \textbf{16} \textit{Majuskel} T  \textbf{25} \textit{Majuskel} T  \newline
\line(1,0){75} \newline
\textbf{1} geklaget] úber klaget W \textbf{2} Lyaze] lyaß W \textbf{3} iuwerm] vwern U \textbf{4} ez hînt] iz hin U [*]: ez hint V \textbf{6} Jr [*]: dienent imme ein teil dar an V · Wissent recht ir dienent in dar an W \textbf{7} mînen] vwern U vnseren V (W) \textbf{8} müezen] mvͦzen T (U)  $\cdot$ strengen] langen W  $\cdot$ zadel] bresten V  $\cdot$ tragen] [*]: tragen T \textbf{9} Kyot] kŷot T \textbf{10} sendiu] sende U \textbf{12} ligent] ligen T lagen W \textbf{14} iuch] îv T \textbf{15} hînt] hin U  $\cdot$ des] das W \textbf{16} sprach] sprach aber W  $\cdot$ Manfilot] Manfilôt T Manfiliot U manphilot V W \textbf{17} sendiu] sende U \textbf{18} dô] Des W  $\cdot$ vrouwe] [*]: maget V maget W \textbf{21} dâ bî ze] [D*]: Do bi zvͦ ir V Do bei zuͦ W \textbf{22} zer wilden albe klûsen] zer wilden abe clvsen T (U) (V) Zuͦ wilder albe klausen W \textbf{24} vride] vreide U \textbf{25} Die] [*e]: Jr V \textbf{26} diu] die T  $\cdot$ erlabet] glabet W \textbf{27} burgære] burgenere W \textbf{28} die] [*]: die V der W \textbf{30} kæme] keine U  $\cdot$ daz] dis W \newline
\end{minipage}
\end{table}
\end{document}
