\documentclass[8pt,a4paper,notitlepage]{article}
\usepackage{fullpage}
\usepackage{ulem}
\usepackage{xltxtra}
\usepackage{datetime}
\renewcommand{\dateseparator}{.}
\dmyyyydate
\usepackage{fancyhdr}
\usepackage{ifthen}
\pagestyle{fancy}
\fancyhf{}
\renewcommand{\headrulewidth}{0pt}
\fancyfoot[L]{\ifthenelse{\value{page}=1}{\today, \currenttime{} Uhr}{}}
\begin{document}
\begin{table}[ht]
\begin{minipage}[t]{0.5\linewidth}
\small
\begin{center}*D
\end{center}
\begin{tabular}{rl}
\textbf{491} & \begin{large}E\end{large}r mac gerîten noch \textbf{gegên},\\ 
 & der künec, \textbf{noch} \textbf{geligen} noch \textbf{gestên};\\ 
 & er lent âne sitzen\\ 
 & mit siufzebæren witzen.\\ 
5 & gein des \textit{m}â\textit{n}en wandel ist im wê.\\ 
 & Brumbane ist \textbf{genant} ein sê;\\ 
 & dâ treit man in ûf durch süezen luft,\\ 
 & durch sîner sûren wunden gruft.\\ 
 & daz heizet er sînen weidetac;\\ 
10 & swaz er \textbf{al}dâ gevâhen mac\\ 
 & bî \textbf{sô} smerzlîchem sêre,\\ 
 & er bedarf dâ heime mêre.\\ 
 & dâ von kom \textbf{ûz} \textbf{ein} mære,\\ 
 & \textbf{er} wære ein vischære.\\ 
15 & daz mære muoser lîden.\\ 
 & Salmen, lamprîden\\ 
 & hât er \textbf{doch} lützel veile,\\ 
 & der trûrige, niht \textbf{der geile}."\\ 
 & Parzival sprach al zehant:\\ 
20 & "\textbf{in} dem sê den künec ich vant\\ 
 & geankert ûf dem wâge,\\ 
 & ich wæne durch vische lâge\\ 
 & ode durch ander kurzwîle.\\ 
 & ich hete manege mîle\\ 
25 & des tages dar gestrichen.\\ 
 & Pelrapeire ich was entwichen\\ 
 & rehte umbe den mitten morgen.\\ 
 & des âbents pflac ich sorgen,\\ 
 & wâ diu herberge m\textit{ö}hte sîn;\\ 
30 & der \textbf{beriet} mich der œheim mîn."\\ 
\end{tabular}
\scriptsize
\line(1,0){75} \newline
D \newline
\line(1,0){75} \newline
\textbf{1} \textit{Initiale} D  \textbf{16} \textit{Majuskel} D  \newline
\line(1,0){75} \newline
\textbf{5} mânen] namen D \textbf{6} Brumbane] Brvmbange D \textbf{13} dâ] do D \textbf{19} Parzival] Parcifal D \textbf{29} möhte] mohte D \newline
\end{minipage}
\hspace{0.5cm}
\begin{minipage}[t]{0.5\linewidth}
\small
\begin{center}*m
\end{center}
\begin{tabular}{rl}
 & er ma\textit{c} \dag gerten\dag  noch \textbf{gegân},\\ 
 & der künic, \textbf{ligen} noch \textbf{gestân};\\ 
 & er lent \textbf{gar} âne sitzen\\ 
 & mit siufzebæren witzen.\\ 
5 & gegen des \textit{mânen wandel} ist \textit{i}m \textit{w}ê.\\ 
 & Brunbane ist ein sê;\\ 
 & d\textit{â} treit man i\textit{n} ûf durch süeze luft,\\ 
 & durch sîner sûren wunden gruft.\\ 
 & daz heizet er sînen weide tac;\\ 
10 & waz er \textbf{al}dâ gevâhen mac\\ 
 & bî \textbf{sô} smerzlîchem sêre,\\ 
 & er bedarf dâ heim mêre.\\ 
 & dâ von kam \textbf{ûz} \textbf{ein} mære,\\ 
 & \textbf{er} wær ein vischære.\\ 
15 & daz mær muos er lîden.\\ 
 & salmen \textbf{oder} lamprîden\\ 
 & het er \textbf{doch} lütze\textit{l} \textit{v}eil,\\ 
 & der trû\textit{r}ige, niht \textbf{ze geil}."\\ 
 & Parcifal sprach alzehant:\\ 
20 & "\textbf{in} dem sê den künic ich vant\\ 
 & geankert ûf dem w\textit{â}ge,\\ 
 & ich wæne durch visch l\textit{â}ge\\ 
 & oder durch ander kurzewîle.\\ 
 & ich hete manige \textit{m}île\\ 
25 & des tages dar gestrichen.\\ 
 & Pelraperie ich was entwichen\\ 
 & reht umb den mitten morgen.\\ 
 & des âbendes pflac ich sorgen,\\ 
 & wâ diu herberge möhte sîn;\\ 
30 & der \textbf{r\textit{ie}t} mich der œheim mîn."\\ 
\end{tabular}
\scriptsize
\line(1,0){75} \newline
m n o \newline
\line(1,0){75} \newline
\newline
\line(1,0){75} \newline
\textbf{1} mac] mage m  $\cdot$ gegân] gegen o \textbf{2} ligen] liegen o \textbf{4} siufzebæren] herczeberen o \textbf{5} Gegen des wandel monen ist mýnne m \textbf{6} Brunbane] Brumbane n Brúmbane o  $\cdot$ ist] ist genant n (o) \textbf{7} dâ] Do m n o  $\cdot$ in] ẏm m  $\cdot$ süeze] sússen n o  $\cdot$ luft] lust n \textbf{8} sîner] sinen o \textbf{9} sînen] sine o \textbf{11} smerzlîchem] swerczenlichem o \textbf{15} mær] imer o  $\cdot$ muos] muͦsz n \textbf{16} salmen] Salman o  $\cdot$ lamprîden] lantfrẏden m (n) (o) \textbf{17} lützel veil] luczel heil vnd feil m \textbf{18} trûrige] truige m \textbf{21} wâge] weͯge m \textbf{22} visch lâge] fischlouge m \textbf{24} mîle] wile m \textbf{26} Pelraperie] Pelrapeir n Pelrapier o \textbf{29} möhte] mochte o \textbf{30} riet] reit m \newline
\end{minipage}
\end{table}
\newpage
\begin{table}[ht]
\begin{minipage}[t]{0.5\linewidth}
\small
\begin{center}*G
\end{center}
\begin{tabular}{rl}
 & \begin{large}E\end{large}r mac gerîten noch \textbf{gegên},\\ 
 & der künic, \textbf{geligen} noch \textbf{gestên};\\ 
 & er lent âne sitzen\\ 
 & mit siuftebæren witzen.\\ 
5 & gegen des mânen \textit{wandel} ist im wê.\\ 
 & Brunbanie ist \textbf{genant} ein sê;\\ 
 & dâ treit man in ûf durch süezen luft,\\ 
 & durch sîner sûren wunden gruft.\\ 
 & daz heizet er sînen weide tac;\\ 
10 & swaz er dâ gevâhen mac\\ 
 & bî \textbf{sô} smerzlîchem sêre,\\ 
 & er bedarf dâ heime mêre.\\ 
 & dâ von kom \textbf{ein} mære,\\ 
 & \textbf{ez} wære ein vischære.\\ 
15 & daz mære muos er lîden.\\ 
 & salmen, lamprîden\\ 
 & hât er \textbf{doch} lützel veile,\\ 
 & der trûrige, niht \textbf{der geile}."\\ 
 & Parzival sprach al zehant:\\ 
20 & "\textbf{ûf} dem sê den künic ich vant\\ 
 & geankert ûf dem wâge,\\ 
 & ich wæne durch vische lâge\\ 
 & oder durch ander kurzewîle.\\ 
 & ich het manige mîle\\ 
25 & des tages dar gestrichen.\\ 
 & Pelrapeire ich was entwichen\\ 
 & rehte umbe den mitten morgen.\\ 
 & des âbendes pflac ich sorgen,\\ 
 & wâ diu herberge m\textit{ö}hte sîn;\\ 
30 & der \textbf{beriet} mich der œheim mîn."\\ 
\end{tabular}
\scriptsize
\line(1,0){75} \newline
G I O L M Z Fr49 \newline
\line(1,0){75} \newline
\textbf{1} \textit{Initiale} G I O L Z  \textbf{19} \textit{Initiale} I Fr49  \newline
\line(1,0){75} \newline
\textbf{1} Er] ÷r O  $\cdot$ mac] en mag L (M) \textbf{2} der künic] \textit{om.} L  $\cdot$ geligen] noch ligen O Noch geligen L (Z) ligen M \textbf{3} erlampt ame sitzen I \textbf{4} siuftebæren] susgebaren I (Fr49) \textbf{5} wandel] \textit{om.} G wandelvnge L  $\cdot$ wê] so we M \textbf{6} Brunbanie] Brvngang O Brvͯmbanie L Brunbange M Brumbange Z  $\cdot$ ist genant] haizet I \textbf{7} süezen] susze M \textbf{8} sûren wunden] wunden suͤren I  $\cdot$ gruft] guft I (M) \textbf{10} swaz] swa I Waz L  $\cdot$ dâ] al da O L (M) Z \textbf{11} bî sô] bi dem I Ze sinem O Zuͯ so L Bie syneme M  $\cdot$ smerzlîchem] herzenlichem O (M) \textbf{12} dâ heime] deheines L icheine M \textbf{13} kom] chom vns I Fr49 chom vz O (L) (M) (Z) \textbf{15} daz] Die L  $\cdot$ muos] mues G (I)  $\cdot$ er lîden] leide Fr49 \textbf{16} lamprîden] lant priden G lantpride I (Fr49) \textbf{17} hât] het I (Fr49) \textbf{18} niht] vnd nih I (Fr49) \textbf{19} Parzival] Parciual G Parzifal I M Parcifal O L Z Parcîfal Fr49 \textbf{20} ûf] Jn O L M Z  $\cdot$ ich] [vch]: ich O \textbf{22} wæne] wande I (Fr49)  $\cdot$ vische] vishens I (Fr49) \textbf{23} oder] Olde G  $\cdot$ ander] \textit{om.} O \textbf{25} des] [Da*]: Des G \textbf{26} Pelrapeire] Peilrapeire G Pailrapeir I Peilrapeir O pailrapir Fr49 \textbf{27} den] \textit{om.} L \textbf{28} pflac] pfac Z \textbf{29} diu] der Fr49  $\cdot$ möhte] mohte G I O L (M) Z \textbf{30} der œheim] dy oheme M  $\cdot$ mîn] din L \newline
\end{minipage}
\hspace{0.5cm}
\begin{minipage}[t]{0.5\linewidth}
\small
\begin{center}*T
\end{center}
\begin{tabular}{rl}
 & \begin{large}E\end{large}r \textbf{en}mac gerîten noch \textbf{gestên},\\ 
 & der künec, \textbf{gelenen} noch \textbf{gegên};\\ 
 & er lenet âne sitzen\\ 
 & mit siuftebæren witzen.\\ 
5 & gegen des mânen wandel ist im wê.\\ 
 & Brumbanie ist \textbf{genant} ein sê;\\ 
 & dâ treit man in ûf durch süezen luft,\\ 
 & durch sîner sûren wunden gruft.\\ 
 & daz heizet \textit{er} sînen weide tac;\\ 
10 & swaz er \textbf{al}dâ gevâhen mac\\ 
 & bî \textbf{sînem} smerzlîchen sêre,\\ 
 & er bedarf dâ heime mêre.\\ 
 & dâ von kom \textbf{ûz} mære,\\ 
 & \textbf{er} wære ein vischære.\\ 
15 & daz mære muoser lîden.\\ 
 & salmen, lamprîden\\ 
 & hât er lützel veile,\\ 
 & der trûrige, niht \textbf{der geile}."\\ 
 & Parcifal sprach alzehant:\\ 
20 & "\textbf{in} dem sê den künec ich vant\\ 
 & geenkert ûf dem wâge,\\ 
 & ich wæne durch vische lâge\\ 
 & oder durch ander kurzewîle.\\ 
 & ich hete manege mîle\\ 
25 & des tages dar gestrichen.\\ 
 & Peilrapere ich was entwichen\\ 
 & rehte umbe den mitten morgen.\\ 
 & des âbendes pflac ich sorgen,\\ 
 & wâ diu herberge m\textit{ö}hte sîn;\\ 
30 & der \textbf{beriet} mich der œheim mîn."\\ 
\end{tabular}
\scriptsize
\line(1,0){75} \newline
T U V W Q R Fr40 \newline
\line(1,0){75} \newline
\textbf{1} \textit{Initiale} T V W Fr40  \textbf{19} \textit{Initiale} R  \newline
\line(1,0){75} \newline
\textbf{1} \textit{Die Verse 453.1-502.30 fehlen} U   $\cdot$ enmac] mag V R  $\cdot$ gestên] gegen W R Fr40 gen Q \textbf{2} gelenen noch gegên] geligen noch gegen V geligen noch gesten W noch geligen noch gesteen Q (R) (Fr40) \textbf{3} er] Der Q (Fr40)  $\cdot$ lenet] lempt Q laut R leint Fr40 \textbf{5} mânen] mones R \textbf{6} Brumbanie] Brvnbange T [Brvnban*]: Brvnbanye V Brubangen W brűmbange Q Brumbange R (Fr40)  $\cdot$ sê] sne Fr40 \textbf{7} treit] reit Q  $\cdot$ süezen luft] sussze [*uft]: luft Q \textbf{8} sîner] seinen Q  $\cdot$ sûren] \textit{om.} R  $\cdot$ gruft] gufft W (R) \textbf{9} er] \textit{om.} T R \textbf{10} swaz] Was W Q R  $\cdot$ aldâ] do W \textbf{11} [*]: Bi so smertzlichem sere V  $\cdot$ smerzlîchen] schmechlichen R smerzleichem Fr40 \textbf{12} heime] dehein R \textbf{13} \textit{Versfolge 491.14-13} V   $\cdot$ mære] ein mere V W Q R (Fr40) \textbf{14} er] Es W Q R \textbf{15} daz mære] Da merre R \textbf{16} Salmen [l*]: oder lantfriden V \textbf{17} lützel] doch lv́zel V (W) (Q) (R) (Fr40) \textbf{19} Parcifal] Parzifal V Fr40 Partzifal W Q Parczifal R \textbf{20} in] Vff R  $\cdot$ sê] sehe Q  $\cdot$ den künec ich] ich den konick Q \textbf{21} \textit{Versfolge 491.22-21} V  \textbf{24} mîle] wile vnd mile R \textbf{26} Peilrapere] [P*]: Pelrapere V Pelrapeir W Pelrapeire Q R (Fr40)  $\cdot$ entwichen] entschlichen W \textbf{29} möhte] mohte T V (Q) Fr40 \textbf{30} der] Do Q  $\cdot$ mich] micht W \newline
\end{minipage}
\end{table}
\end{document}
