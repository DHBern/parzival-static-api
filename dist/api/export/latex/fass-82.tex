\documentclass[8pt,a4paper,notitlepage]{article}
\usepackage{fullpage}
\usepackage{ulem}
\usepackage{xltxtra}
\usepackage{datetime}
\renewcommand{\dateseparator}{.}
\dmyyyydate
\usepackage{fancyhdr}
\usepackage{ifthen}
\pagestyle{fancy}
\fancyhf{}
\renewcommand{\headrulewidth}{0pt}
\fancyfoot[L]{\ifthenelse{\value{page}=1}{\today, \currenttime{} Uhr}{}}
\begin{document}
\begin{table}[ht]
\begin{minipage}[t]{0.5\linewidth}
\small
\begin{center}*D
\end{center}
\begin{tabular}{rl}
\textbf{82} & \textbf{wan} si sint mir alle sippe\\ 
 & von \textbf{dem} Adames rippe.\\ 
 & Doch wæne \textbf{êt}, Gahmuretes \textit{t}ât\\ 
 & den hœhsten prîs \textbf{dâ} \textbf{erworben} hât."\\ 
5 & die anderen \textbf{tæten} rîterschaft\\ 
 & mit sô \textbf{bewander} zornes kraft,\\ 
 & daz siz \textbf{w\textit{ie}lken vaste} \textbf{unz} \textbf{an} die naht.\\ 
 & die inren heten die ûzern brâht\\ 
 & mit \textbf{strîte} \textbf{unz an} \textbf{ir} poulûn,\\ 
10 & \textbf{niwan} der künec von \textbf{Ascalun}\\ 
 & unt Morholt von Yrlant,\\ 
 & durch die snüere in wære gerant.\\ 
 & \begin{large}D\end{large}â \textbf{was} gewunnen unt verlorn.\\ 
 & genuoge heten schaden erkorn,\\ 
15 & die anderen prîs unt êre.\\ 
 & nû ist zît, daz man si kêre\\ 
 & von ein ander. niemen \textbf{hie} gesiht,\\ 
 & \textbf{sine werte der pfander} liehtes niht.\\ 
 & wer solde ouch \textbf{vinsterlingen} spiln?\\ 
20 & \textbf{es} \textbf{mac} \textbf{die} müeden \textbf{doch} beviln.\\ 
 & Der vinster man vil gar vergaz,\\ 
 & dô mîn hêr Gahmuret \textbf{dort} \textbf{saz},\\ 
 & als ez wære tac. des was ez niht.\\ 
 & dâ wâren aber ungevüegiu lieht,\\ 
25 & von kleinen kerzen manec schoup\\ 
 & geleit ûf \textbf{ölboume} loup.\\ 
 & manec kulter rîche\\ 
 & gestrecket \textbf{vlîzeclîche},\\ 
 & dâr vür \textbf{manec} teppech breit.\\ 
30 & diu küneginne an die snüere reit\\ 
\end{tabular}
\scriptsize
\line(1,0){75} \newline
D \newline
\line(1,0){75} \newline
\textbf{3} \textit{Majuskel} D  \textbf{13} \textit{Initiale} D  \textbf{21} \textit{Majuskel} D  \newline
\line(1,0){75} \newline
\textbf{3} Gahmuretes] Gahmvretes D  $\cdot$ tât] stat D \textbf{7} wielken] wælchen D \textbf{11} Yrlant] yr lant D \textbf{22} Gahmuret] Gahmvret D \newline
\end{minipage}
\hspace{0.5cm}
\begin{minipage}[t]{0.5\linewidth}
\small
\begin{center}*m
\end{center}
\begin{tabular}{rl}
 & \textbf{wanne} si sint mir alle sippe\\ 
 & von \textbf{dem} Adam\textit{e}s rippe.\\ 
 & doch wæne \textbf{ich}, Gahmuretes tât\\ 
 & den hœhesten prîs \textbf{erworben} hât."\\ 
5 & die anderen \textbf{tâten} ritterschaft\\ 
 & mit sô \textbf{bewanter} zornes kraft,\\ 
 & daz si ez \textbf{wielken vaste} \textit{\textbf{in}} die naht.\\ 
 & die inren heten die ûzern brâht\\ 
 & mit \textbf{strîte} \textbf{unz an} \textbf{ir} pavelûn,\\ 
10 & \textbf{niwan} der künic von \textbf{Ascalun}\\ 
 & und Morolt von Irlant,\\ 
 & durch die s\textit{nü}ere in wær gerant.\\ 
 & \dag das\dag  gewunnen und verlorn.\\ 
 & genuoge heten schaden erkorn,\\ 
15 & die a\textit{n}de\textit{r}en prîs und êre.\\ 
 & nû ist zît, daz man si kêre\\ 
 & von ein ander. niemen \textbf{hie} gesiht,\\ 
 & \textbf{der pfender wert si} liehtes niht.\\ 
 & wer solte ouch \textbf{vinsterlîche} spiln?\\ 
20 & \textbf{es} \textbf{mac} \textbf{die} müeden \textbf{gar} beviln.\\ 
 & \begin{large}D\end{large}er vinster man vil gar vergaz,\\ 
 & dô \dag mic\dag  hêrre Gahmuret \textbf{dort} \textbf{az},\\ 
 & als ez wære tac. des \textbf{en}was ez niht.\\ 
 & dâ \dag wanne\dag  aber ungevüegiu lieht,\\ 
25 & von kleinen \textit{k}erzen manic schoup\\ 
 & geleit ûf \textbf{ölboumes} loup.\\ 
 & manic kulter rîche\\ 
 & gestrecket \textbf{vlîzeclîche},\\ 
 & dâr vür \textbf{manic} teppe\textit{ch} breit.\\ 
30 & diu künegîn an die s\textit{nüe}re reit\\ 
\end{tabular}
\scriptsize
\line(1,0){75} \newline
m n o \newline
\line(1,0){75} \newline
\textbf{21} \textit{Initiale} m o   $\cdot$ \textit{Capitulumzeichen} n  \newline
\line(1,0){75} \newline
\textbf{1} si] wir o  $\cdot$ alle] also n o \textbf{2} dem] [adem]: dem n  $\cdot$ Adames] adamas m n o \textbf{3} Gahmuretes] gamiretes n gamutez o \textbf{4} hœhesten] besten n \textbf{6} zornes] \textit{om.} n \textbf{7} si] \textit{om.} o  $\cdot$ wielken] wielen n o  $\cdot$ vaste in] vastem m \textbf{9} pavelûn] panalẏm o \textbf{10} niwan] Jn wan m o Jn [gewa*]: gewan n  $\cdot$ Ascalun] ascaluͯn m ascalim o \textbf{11} Irlant] ir lant m \textbf{12} snüere] sunere m \textbf{13} gewunnen] gewunnen sú n \textbf{14} \textit{Verse 82.14-15 kontrahiert zu:} Genuͦg hetten pris vnd ere o   $\cdot$ heten] hettent sú n \textbf{15} anderen] adernen m \textbf{19} vinsterlîche] vinsterlingen n o \textbf{22} dô mic hêrre] Do micherre m Domit her n Do mit got o  $\cdot$ Gahmuret] gamiret n gamuͯret o \textbf{23} des] das o  $\cdot$ enwas] was n o \textbf{24} dâ] Do n o  $\cdot$ wanne] fan o \textbf{25} kerzen] herczen m  $\cdot$ schoup] fauͯlt o \textbf{26} ölboumes] alebammes m oleubes o  $\cdot$ loup] lopp m \textbf{29} teppech] tepper m \textbf{30} snüere] sunre m sinre o \newline
\end{minipage}
\end{table}
\newpage
\begin{table}[ht]
\begin{minipage}[t]{0.5\linewidth}
\small
\begin{center}*G
\end{center}
\begin{tabular}{rl}
 & si sint mir alle sippe\\ 
 & von \textbf{dem} Adames rippe.\\ 
 & doch wæne \textbf{ich}, Gahmuretes tât\\ 
 & den hœhesten brîs \textbf{behalten} hât."\\ 
5 & \begin{large}D\end{large}ie anderen \textbf{tâten} rîterschaft\\ 
 & mit sô \textbf{getâner} zornes kraft,\\ 
 & daz siz \textbf{wielken vaste} \textbf{unze} \textbf{\textit{i}n} die naht.\\ 
 & die inneren heten die ûzeren brâht\\ 
 & mit \textbf{zorne} \textbf{under} \textbf{diu} pavelûn,\\ 
10 & \textbf{wan} der künic von \textbf{Arragun}\\ 
 & unde Morolt von Yrlant,\\ 
 & durch die snüere in wære gerant.\\ 
 & dâ \textbf{wart} gewunnen u\textit{nde} verloren.\\ 
 & genuoge heten schaden erkoren,\\ 
15 & die anderen brîs und êre.\\ 
 & nû ist zît, daz man si kêre\\ 
 & von ein ander. niemen \textbf{dâ} gesiht,\\ 
 & \textbf{sine wert der pfander} liehtes niht.\\ 
 & wer solt ouch \textbf{vinsterlingen} spilen?\\ 
20 & \textbf{es} \textbf{moht} \textbf{die} müeden \textbf{doch} bevilen.\\ 
 & der vinster man vil gar vergaz,\\ 
 & dâ mîn hêr Gahmuret \textbf{dâ} \textbf{saz},\\ 
 & als ez wære tac. des \textbf{en}was ez niht.\\ 
 & dâ wâren aber ungevüegiu lieht,\\ 
25 & von kleinen kerzen manic schoup\\ 
 & geleit ûf \textbf{\textit{ö}lboumîn} loup,\\ 
 & \textbf{unde} manic gulter rîche\\ 
 & gestrecket \textbf{vlîziclîche},\\ 
 & dâ vür \textbf{manec} tepec breit.\\ 
30 & diu künigîn an die snüere reit\\ 
\end{tabular}
\scriptsize
\line(1,0){75} \newline
G I O L M Q R Z Fr50 \newline
\line(1,0){75} \newline
\textbf{5} \textit{Überschrift:} Hie hat der strit ein ende Z   $\cdot$ \textit{Initiale} G L Z  \textbf{19} \textit{Initiale} I  \newline
\line(1,0){75} \newline
\textbf{1} si] Wan sie Z  $\cdot$ alle] \textit{om.} I \textbf{2} \textit{Vers 82.2 fehlt} Q   $\cdot$ dem] \textit{om.} I O L R  $\cdot$ Adames] Adams I R adamis M \textbf{3} wæne ich] wennigk M meine ich R wenet Z  $\cdot$ Gahmuretes] Gamvretes O Gahmuͯretes L gamuͯretis M gamuertes Q gamuretes Z  $\cdot$ tât] rat L stat M [lant]: rat Q \textbf{4} behalten] da (do Q ) er worben O (M) (Q) erworben L Z erworben da R \textbf{5} anderen] ander Z \textbf{6} sô getâner] so gewanter L (M) (R) Z also gewonter Q \textbf{7} wielken vaste] wielchen O vaste vilhen Q wieltten R  $\cdot$ unze] bisz Q \textit{om.} Z  $\cdot$ in] an G \textbf{9} zorne] strt O strite L M (Q) (R) Z  $\cdot$ under diu] vnder die I vnz an ir O L M (Q) (R) (Z) \textbf{10} Arragun] aragun G arraguͤn I Aschalvn O (M) ascalvn L (R) Z ascaluͯn Q \textbf{11} Morolt] morholt I (O) (L) (M) Z  $\cdot$ von] vnd Q Z  $\cdot$ Yrlant] ẏrlant G irlant I Z ierlant O jrlant L (R) ir lant M Q \textbf{12} in wære] wart L in ir wer Q \textbf{13} dâ] Do Q  $\cdot$ unde] vil G \textbf{14} genuoge] Grug R  $\cdot$ heten] heten da O \textbf{16} si] \textit{om.} O \textbf{17} ander] andren R  $\cdot$ dâ] do Q hie Z Fr50 \textbf{18} Der phender wert sie liechtes niht L  $\cdot$ wert] enwert R  $\cdot$ pfander] phante M pfandes R  $\cdot$ liehtes] bechtis M lichtes Q Z \textbf{19} wer] Uer I Swer Z  $\cdot$ solt] sol O  $\cdot$ ouch] \textit{om.} L  $\cdot$ vinsterlingen] vinsterlich Q (R) \textbf{20} es] ezn I  $\cdot$ doch] wol I Q ovch Z  $\cdot$ bevilen] beuilchn R \textbf{21} der] Die Q  $\cdot$ vil gar] vil I da gar Z  $\cdot$ vergaz] vergar Z \textbf{22} dâ] Daz L Do M Q R  $\cdot$ hêr Gahmuret] her Gamvret O her Gahmuͯret L ergamuret M herr gamuret Q (Z)  $\cdot$ dâ] \textit{om.} I dort O L M Q (R) Z  $\cdot$ saz] shaz I az L (Q) (R) Z \textbf{23} des enwas ez] das was ez I des was ez O Z dez enwaz L des on weiz M \textbf{24} dâ] Das Q  $\cdot$ aber] sus I  $\cdot$ ungevüegiu] vngefuger L vngefuget M vngefuͦge R \textbf{25} kerzen] kerken L herczin M \textbf{26} ölboumîn] chleboͮmin G olbaumes I (O) olbovme L (M) (Q) (Z) albome R  $\cdot$ loup] lop L [lop]: loup Q \textbf{28} \textit{Vers 82.28 fehlt} R   $\cdot$ gestrecket] Gestrigkit M \textbf{29} vür] vf L  $\cdot$ tepec] bette M  $\cdot$ breit] bereit R \newline
\end{minipage}
\hspace{0.5cm}
\begin{minipage}[t]{0.5\linewidth}
\small
\begin{center}*T (U)
\end{center}
\begin{tabular}{rl}
 & si sint mir alle sippe\\ 
 & von Adames rippe.\\ 
 & doch wæn\textbf{ic\textit{h}}, \textbf{daz} Gahmuretes tât\\ 
 & den hœhesten prîs \textbf{erworben} hât."\\ 
5 & die anderen \textbf{tâten} rîterschaft\\ 
 & mit sô \textbf{gewanter} zornes kraft,\\ 
 & daz si ez \textbf{vaste \textit{wielken}} \textit{\textbf{unz} \textbf{in} die naht}.\\ 
 & die innern heten die ûzern brâht\\ 
 & mit \textbf{strîte} \textbf{unz an} \textbf{ein} pavelûn,\\ 
10 & \textbf{wan} der künec von \textbf{Ascalun}\\ 
 & und Morolt von Irlant,\\ 
 & durch die snüere in wære gerant.\\ 
 & \begin{large}D\end{large}\textit{â} \textbf{wart} gewunnen und verlorn.\\ 
 & genuoge heten schaden erkorn,\\ 
15 & die anderen prîs und êre.\\ 
 & nû ist zît, daz man si kêre\\ 
 & von ein ander. nieman \textbf{dâ} gesiht,\\ 
 & \textbf{sine wert der pfender} liehtes niht.\\ 
 & wer solte ouch \textbf{vinsterlingen} spiln?\\ 
20 & \textbf{daz} \textbf{mohte} \textbf{den} müeden \textbf{wol} beviln.\\ 
 & der vinster man vil gar vergaz,\\ 
 & dô mîn hêrre Gahmuret \textbf{dort} \textbf{az},\\ 
 & als ez wære tac. des \textbf{en}was ez niht.\\ 
 & d\textit{â} wâren aber ungevüegiu lieht,\\ 
25 & von kleinen kerzen manec schoup\\ 
 & geleit ûf \textbf{ölboumes} loup,\\ 
 & \textbf{und} manege kulter rîche\\ 
 & gestrecket \textbf{wunneclîche},\\ 
 & dâ vür teppic breit.\\ 
30 & diu künegîn an die snüere reit\\ 
\end{tabular}
\scriptsize
\line(1,0){75} \newline
U V W T \newline
\line(1,0){75} \newline
\textbf{5} \textit{Initiale} W   $\cdot$ \textit{Majuskel} T  \textbf{13} \textit{Initiale} U   $\cdot$ \textit{Majuskel} T  \textbf{21} \textit{Majuskel} T  \textbf{30} \textit{Majuskel} T  \newline
\line(1,0){75} \newline
\textbf{2} Adames] adamas U Adams V (W) \textbf{3} Doch gamuret wenig getat W  $\cdot$ wænich] wenic U  $\cdot$ daz] \textit{om.} T  $\cdot$ Gahmuretes] Gahmuͦretes U Gamuretes V \textbf{4} hœhesten] pesten T \textbf{7} Daz sie iz vaste U  $\cdot$ si ez] [si]: sv́ V sich W  $\cdot$ vaste wielken] vaste [*lken]: wielken V wilken vast W (T)  $\cdot$ unz in] in V vntz an W \textbf{8} heten die ûzern] die vsser hetten V \textbf{9} an ein] an den W anir T \textbf{10} Ascalun] [*]: aschalun V astalun W \textbf{11} Morolt] morholt W  $\cdot$ Irlant] Jrlant U V T yrland W \textbf{12} in wære] warn ein W in waren T \textbf{13} dâ] Do U V W \textbf{16} zît] \textit{om.} T \textbf{17} dâ] [*]: do V do W  $\cdot$ gesiht] geschicht W \textbf{18} sine wert] [*]: Jn enwirt V Sy wert W  $\cdot$ liehtes] leicht W \textbf{19} ouch] \textit{om.} T \textbf{20} daz] Dis W des T  $\cdot$ mohte] moͤhte V (W)  $\cdot$ den] die V W T \textbf{21} vinster] vinsteriv T  $\cdot$ vil] do vil W \textbf{22} dô] da T  $\cdot$ Gahmuret] Gahmuͦret U Gamuret V (W)  $\cdot$ dort] do W  $\cdot$ az] [saz]: as V \textbf{23} \textit{nach 82.23:} Vil klaine in doch das wag / Das es doch nicht / Also noch vil offte geschicht W   $\cdot$ Ob es vil schoͤne wer tag W  $\cdot$ als ez] scein alsez T \textbf{24} \textit{nach 82.24:} So ich eúch rechte sagen wil W   $\cdot$ Do branne liecht on massen vil W  $\cdot$ dâ] Do U (V)  $\cdot$ aber] \textit{om.} T \textbf{26} ölboumes] oleibovmin T \textbf{27} kulter] kuter V \textbf{28} gestrecket] Gestercket W  $\cdot$ wunneclîche] fleissigleiche W (T) \textbf{29} vür] fúr manig V W (T) \newline
\end{minipage}
\end{table}
\end{document}
