\documentclass[8pt,a4paper,notitlepage]{article}
\usepackage{fullpage}
\usepackage{ulem}
\usepackage{xltxtra}
\usepackage{datetime}
\renewcommand{\dateseparator}{.}
\dmyyyydate
\usepackage{fancyhdr}
\usepackage{ifthen}
\pagestyle{fancy}
\fancyhf{}
\renewcommand{\headrulewidth}{0pt}
\fancyfoot[L]{\ifthenelse{\value{page}=1}{\today, \currenttime{} Uhr}{}}
\begin{document}
\begin{table}[ht]
\begin{minipage}[t]{0.5\linewidth}
\small
\begin{center}*D
\end{center}
\begin{tabular}{rl}
\textbf{263} & \begin{large}P\end{large}rîses si bêde gerten.\\ 
 & die blicke von den swerten\\ 
 & unt viwer, \textbf{daz} \textbf{von} helmen spranc,\\ 
 & unt manec ellenthafter swanc,\\ 
5 & die begunden verre glesten,\\ 
 & \textbf{wan} dâ wâren strîtes die besten\\ 
 & mit hurte an ein ander kumen,\\ 
 & ez gê ze schaden oder ze vrumen\\ 
 & Den küenen helden mæren.\\ 
10 & swie \textbf{willic} dors \textbf{in} wæren,\\ 
 & dâ si bêde ûf sâzen,\\ 
 & der sporn si niht vergâzen\\ 
 & \textbf{noch} ir swerte lieht gemâl.\\ 
 & prîs \textbf{gediende} Parzival,\\ 
15 & daz er sich alsus weren kan\\ 
 & \textbf{wol} hundert trachen unt eines man.\\ 
 & Ein trache wart versêret,\\ 
 & sîne wunden gemêret,\\ 
 & der ûf Oriluses helme lac,\\ 
20 & sô durchliuhtec, daz der tac\\ 
 & volleclîche \textbf{durch in schein},\\ 
 & \textbf{wart drab} geslagen manec \textbf{edel} stein.\\ 
 & \textbf{daz} ergienc zorse \textbf{unt} niht ze vuoz.\\ 
 & vroun Jeschuten wart der gruoz\\ 
25 & mit swertes schimpfe al dâ bejagt,\\ 
 & mit heldes handen unverzagt.\\ 
 & Mit hurte si \textbf{dicke} \textbf{z}ein ander \textbf{vlugen},\\ 
 & daz die ringe vor den knien zerstuben,\\ 
 & swie si \textbf{wæren} îserîn.\\ 
30 & \textbf{ruochet} irs, si tâten \textbf{strîtes} schîn.\\ 
\end{tabular}
\scriptsize
\line(1,0){75} \newline
D \newline
\line(1,0){75} \newline
\textbf{1} \textit{Initiale} D  \textbf{9} \textit{Majuskel} D  \textbf{17} \textit{Majuskel} D  \textbf{27} \textit{Majuskel} D  \newline
\line(1,0){75} \newline
\textbf{19} Oriluses] Ôrilvs D \textbf{23} ergienc] er gienc D \textbf{24} Jeschuten] Jescvten D \newline
\end{minipage}
\hspace{0.5cm}
\begin{minipage}[t]{0.5\linewidth}
\small
\begin{center}*m
\end{center}
\begin{tabular}{rl}
 & prîses si beide gerten.\\ 
 & die blick\textit{e} von den swerten\\ 
 & und viur, \textbf{daz} \textbf{von} helmen spranc,\\ 
 & \textit{und manic ellenthafter swanc},\\ 
5 & die begunden verre glesten,\\ 
 & \textbf{wan} dâ \textit{wâren} strîte\textit{s d}ie besten\\ 
 & mit h\textit{u}rte an ein ander komen,\\ 
 & ez gê ze schaden oder ze vromen\\ 
 & den küenen helden mæren.\\ 
10 & wie \textbf{willic} diu ros wæren,\\ 
 & d\textit{â} si beide ûf sâzen,\\ 
 & der sporn si niht vergâzen\\ 
 & \textbf{noch} ir swerte lieht gemâl.\\ 
 & prîs \textbf{gedienet} \textbf{in} \textbf{hie} Parcifal,\\ 
15 & daz er sich alsus wern kan\\ 
 & \textbf{wol} hundert trachen und eines man.\\ 
 & ein trach\textit{e} \textit{w}art versêret,\\ 
 & sîne wunden gemêret,\\ 
 & d\textit{e}r ûf Oriluses helme lac,\\ 
20 & sô durchliuhtec, daz der tac\\ 
 & volleclîch \textbf{durchsch\textit{e}in},\\ 
 & \textbf{wart drabe} geslagen manic stein.\\ 
 & \textbf{daz} ergienc \dag daz ros\dag , niht ze vuoz.\\ 
 & vrouwen Jeschuten wart der gruoz\\ 
25 & mit swertes schimpfe aldâ bejaget,\\ 
 & mit heldes handen unverzaget.\\ 
 & mit h\textit{u}rte si \textbf{dicke} \textbf{zuo} ein ander \textbf{sluogen},\\ 
 & daz die ringe vor den knien zerstuben,\\ 
 & wie si \textbf{wæren} îserîn.\\ 
30 & \textbf{ruochet} irs, si tâten \textit{\textbf{strîten}} schîn.\\ 
\end{tabular}
\scriptsize
\line(1,0){75} \newline
m n o Fr69 \newline
\line(1,0){75} \newline
\newline
\line(1,0){75} \newline
\textbf{2} blicke] blicken m \textbf{4} \textit{Vers 263.4 fehlt} m  \textbf{5} verre] \textit{om.} Fr69 \textbf{6} Wan da strittes glesten vnd die besten m \textbf{7} hurte] herte m horte o \textbf{11} dâ] Do m n o \textbf{12} Der] Die o \textbf{13} lieht] licht Fr69 \textbf{14} gedienet in] gedienet n o gediende Fr69 \textbf{16} und eines man] vnd ein swan n als eyn swan o \textbf{17} ein] Der Fr69  $\cdot$ trache wart] trache was vnd wart m \textbf{18} sîne] Vnd sin Fr69 \textbf{19} der] Dar m  $\cdot$ Oriluses] orilus m n Fr69 oriluͯs o \textbf{20} daz] das das n so o \textbf{21} durchschein] durch schin m \textbf{22} drabe] dar an n \textbf{23} ergienc] er ging n enging o \textbf{24} vrouwen] Frouwe m n (o)  $\cdot$ Jeschuten] jescutten m jescute n gescúten o \textbf{27} hurte] herte m n o  $\cdot$ dicke] \textit{om.} n o  $\cdot$ sluogen] flugen n (o) \textbf{28} vor den knien] von den kuͯnen n vor der kunen o \textbf{30} ruochet irs] Recht erst n  $\cdot$ strîten] \textit{om.} m \newline
\end{minipage}
\end{table}
\newpage
\begin{table}[ht]
\begin{minipage}[t]{0.5\linewidth}
\small
\begin{center}*G
\end{center}
\begin{tabular}{rl}
 & \begin{large}B\end{large}rîses si bêde gerten.\\ 
 & die blicke von den swerten\\ 
 & unde \textbf{daz} viur \textbf{ûz} helmen spra\textit{n}c,\\ 
 & unde manic ellenthafter swanc,\\ 
5 & die begunden verre glesten,\\ 
 & \textbf{wan} dâ wâren strîtes die besten\\ 
 & mit hurte an ein ander komen,\\ 
 & ez gê ze schaden oder ze vromen\\ 
 & den küenen helden mæren.\\ 
10 & swie \textbf{willic} diu ors \textbf{in} wæren,\\ 
 & dâ si bêde ûffe sâzen,\\ 
 & der sporn si niht vergâzen\\ 
 & \textbf{unt} ir swerte lieht gemâl.\\ 
 & brîs \textbf{begie} \textbf{hie} Parzival,\\ 
15 & daz er sich alsus weren kan\\ 
 & \textbf{wol} hundert trachen unde eines man.\\ 
 & ein trache wart versêret,\\ 
 & sîne wunden gemêret,\\ 
 & der ûffe Orilluses helme lac,\\ 
20 & sô durchliuhtic, daz der tac\\ 
 & volliclîche \textbf{durch in schein},\\ 
 & \textbf{drabe wart} geslagen manic \textbf{edel} stein.\\ 
 & \textbf{ditze} ergie ze orse \textbf{unde} niht ze vuoz.\\ 
 & vroun Jeschuten wart der gruoz\\ 
25 & mit swertes schimpfe al dâ bejaget,\\ 
 & mit heldes henden unverzaget.\\ 
 & mit hurte si \textbf{dicke} \textbf{z}ein \textit{and}er \textbf{vlugen},\\ 
 & daz die ringe vor den knien zerstuben,\\ 
 & swie si \textbf{wâren} îserîn.\\ 
30 & \textbf{ruocht} irs, si tâten \textbf{strîtes} schîn.\\ 
\end{tabular}
\scriptsize
\line(1,0){75} \newline
G I O L M Q R Z Fr21 \newline
\line(1,0){75} \newline
\textbf{1} \textit{Initiale} G L M  \textbf{17} \textit{Initiale} I   $\cdot$ \textit{Capitulumzeichen} L  \textbf{27} \textit{Initiale} O L Z Fr21  \newline
\line(1,0){75} \newline
\textbf{2} den] \textit{om.} Q \textbf{3} unde] \textit{om.} R  $\cdot$ daz viur] daz fiwer daz I fivr daz O (M) (Z) Fr21 fvir L (Q)  $\cdot$ ûz] vz den I von den O L Q R (Fr21) vome M von Z  $\cdot$ spranc] sprach G \textbf{4} manic ellenthafter] mengen ellenthafften R \textbf{6} wan] \textit{om.} L  $\cdot$ dâ] do Q \textbf{7} ein] \textit{om.} Fr21  $\cdot$ ander] andren R \textbf{8} gê] si I keme R \textbf{10} swie] Wie L (M) (Q) R  $\cdot$ diu ors in] in die ros R \textbf{11} dâ] Do Q \textbf{13} ir] der I (L)  $\cdot$ lieht] licht M Q \textbf{14} begie] gedient O (Q) Z Fr21 gediende L (M) (R)  $\cdot$ hie] ye Q  $\cdot$ Parzival] parzifal I M Parcifal O (L) (Z) (Fr21) partzifal Q parczifal R \textbf{15} alsus] sus wol I sus R \textbf{16} hundert] kvnt er L  $\cdot$ eines] einen L \textit{om.} Q \textbf{17} wart versêret] wart wart geseret R \textbf{18} sîne] Vnde sin O (L) (M) (Q) (R) (Fr21) \textbf{19} der ûffe] Daruff R  $\cdot$ Orilluses] orillus G (L) orilus I (O) (M) Q (R) Z (Fr21) \textbf{20} durchliuhtic] dvrich lvtich Fr21  $\cdot$ daz] als O was R \textbf{21} durch in] als der tach dvrch in O da duͯrch L \textbf{22} drabe wart] Wart drab Z  $\cdot$ edel] \textit{om.} I gut M  $\cdot$ stein] gestein Q \textbf{23} ze orse] zeorsen I  $\cdot$ unde] \textit{om.} I O R \textbf{24} vroun] Vrow L (M) (Q) (R)  $\cdot$ Jeschuten] ieschuten G ieskuten I Jescuͯten L iescuten M (Q) (Z) Jescuten R  $\cdot$ wart] war Fr21 \textbf{25} schimpfe] slegen Fr21  $\cdot$ al] \textit{om.} Z  $\cdot$ bejaget] betagt L \textbf{26} mit] Von O L M Q R Z (Fr21)  $\cdot$ heldes] helden Fr21 \textbf{27} mit] ÷it O  $\cdot$ zein ander] zeiner G gegen ein andren R  $\cdot$ vlugen] slugen Q (R) (Fr21) \textbf{28} vor] von I Q R Z  $\cdot$ den knien] dem chnie I den ringen O der kiren Q inen R  $\cdot$ zerstuben] stuben R Z \textbf{29} swie] Owe wie O Wie L (Q) R  $\cdot$ wâren] werin M (R) \textbf{30} ruocht] geruchet I Rvͦhte O  $\cdot$ irs] ir des L ir Z  $\cdot$ tâten] waren vnde taten O \newline
\end{minipage}
\hspace{0.5cm}
\begin{minipage}[t]{0.5\linewidth}
\small
\begin{center}*T
\end{center}
\begin{tabular}{rl}
 & prîses si beide gerten.\\ 
 & die blicke von den swerten\\ 
 & unde viur, \textbf{daz} \textbf{von} helmen spranc,\\ 
 & unde manec ellenthafter swanc,\\ 
5 & die begunden verre glesten.\\ 
 & dâ wâren strîtes die besten\\ 
 & mit hurte an ein ander komen,\\ 
 & ez gê ze schaden oder ze vromen\\ 
 & \begin{large}D\end{large}en küenen helden mæren.\\ 
10 & swie \textbf{gewillic} diu ors \textbf{in} wæren,\\ 
 & dâ si beide ûffe sâzen,\\ 
 & der sporn si niht vergâzen\\ 
 & \textbf{unde} ir swerte lieht gemâl.\\ 
 & prîs \textbf{gediende} Parzifal,\\ 
15 & daz er sich alsus wern kan\\ 
 & hundert trachen unde eines man.\\ 
 & ein trache wart versêret,\\ 
 & sîne wunden gemêret,\\ 
 & der ûf Oriluses helme lac,\\ 
20 & sô durchliuhtic, daz der tac\\ 
 & volleclîche \textbf{durch in schein},\\ 
 & \textbf{dâ wart drabe} geslagen manec \textbf{edel} stein.\\ 
 & \textbf{ez} ergienc zors \textbf{unde} niht ze vuoz.\\ 
 & vroun Jeschuten wart der gruoz\\ 
25 & mit swertes schimpfe aldâ bejaget,\\ 
 & mit heldes handen unverzaget.\\ 
 & mit hurte si \textbf{sich} \textbf{an} ein ander \textbf{schuoben},\\ 
 & daz die ringe vor den knien zerstuben,\\ 
 & swie si \textbf{wæren} îserîn.\\ 
30 & \textbf{geruochet} ir\textit{s}, si tâten \textbf{strîten} schîn.\\ 
\end{tabular}
\scriptsize
\line(1,0){75} \newline
T U V W \newline
\line(1,0){75} \newline
\textbf{9} \textit{Initiale} T U W  \newline
\line(1,0){75} \newline
\textbf{3} daz] \textit{om.} W  $\cdot$ von helmen] von dem helme W \textbf{6} dâ] Do U W [*]: Wan do V \textbf{8} gê] kem W \textbf{9} küenen helden] helden kuͤnen W \textbf{10} swie] Wie U W  $\cdot$ gewillic] willig W  $\cdot$ in] \textit{om.} W \textbf{11} dâ] Do U W [D*]: Da V  $\cdot$ si] \textit{om.} U \textbf{13} unde] [*]: Noch V \textbf{14} gediende] gediete U  $\cdot$ Parzifal] Parcifal U parzefal V partzifal W \textbf{16} hundert] [vvundert]: hundert T [*]: Wol hvndert V Wol hundert W \textbf{18} wunden] wunde do W \textbf{20} durchliuhtic] duͦrch luchtit U  $\cdot$ daz] sam W \textbf{21} in] [*]: in hin V \textbf{22} dâ] do U V W \textbf{23} ez] Do U Daz V (W)  $\cdot$ ergienc] er ginc U (W) \textbf{24} vroun] Vreuwe U (W)  $\cdot$ Jeschuten] Jescvten T (U) (V) iestuten W \textbf{25} aldâ] do W \textbf{27} Mit hv́rte sv́ [*]: dicke zeinander flugen V  $\cdot$ sich] \textit{om.} W \textbf{28} zerstuben] stuben W \textbf{29} swie] Wie U W  $\cdot$ îserîn] eisenein W \textbf{30} geruochet] Ruͦchent W  $\cdot$ irs] irz T  $\cdot$ strîten] strites U V (W) \newline
\end{minipage}
\end{table}
\end{document}
