\documentclass[8pt,a4paper,notitlepage]{article}
\usepackage{fullpage}
\usepackage{ulem}
\usepackage{xltxtra}
\usepackage{datetime}
\renewcommand{\dateseparator}{.}
\dmyyyydate
\usepackage{fancyhdr}
\usepackage{ifthen}
\pagestyle{fancy}
\fancyhf{}
\renewcommand{\headrulewidth}{0pt}
\fancyfoot[L]{\ifthenelse{\value{page}=1}{\today, \currenttime{} Uhr}{}}
\begin{document}
\begin{table}[ht]
\begin{minipage}[t]{0.5\linewidth}
\small
\begin{center}*D
\end{center}
\begin{tabular}{rl}
\textbf{423} & \textit{\begin{large}I\end{large}}n die kemenâten sân\\ 
 & gienc diu \textbf{k\textit{ü}neginne} unt die zwêne man.\\ 
 & \textbf{vor} den andern beleip si lære\\ 
 & - des pflâgen kamerære -,\\ 
5 & \textbf{wan} \textbf{clâriu} juncvröuwelîn,\\ 
 & \textbf{der muose vil dort inne} sîn.\\ 
 & diu künegîn mit zühten pflac\\ 
 & Gawans, der ir ze herzen lac.\\ 
 & dâ was der lantgrâve mite.\\ 
10 & \textbf{der} schiet \textbf{si} ninder von dem site.\\ 
 & doch sorgete vil diu \textbf{werdiu} magt\\ 
 & umbe Gawans lîp, wart mir gesagt.\\ 
 & Sus wâren die zwêne dâ inne\\ 
 & bî der küneginne,\\ 
15 & unz daz der tac liez sînen strît.\\ 
 & diu naht kom, dô was ezzens zît.\\ 
 & Môraz, wîn, lûtertranc\\ 
 & brâhten juncvrouwen, \textbf{dâ} mitten kranc,\\ 
 & unt ander guote spîse:\\ 
20 & fasân, pardrîse,\\ 
 & guote vische \textbf{unt} blankiu wastel.\\ 
 & Gawan unt Kyngrimursel\\ 
 & wâren komen ûz grôzer nôt.\\ 
 & sît ez diu küneginne \textbf{gebôt},\\ 
25 & \textbf{si âzen}, als si solten,\\ 
 & unt ander, die\textbf{s} \textbf{iht} wolten.\\ 
 & Antikonie in selbe sneit.\\ 
 & daz was durch zuht in bêden leit.\\ 
 & \textit{\begin{large}S\end{large}}waz man dâ \textbf{kniender schenken} sach,\\ 
30 & ir \textbf{decheinem} \textbf{diu} hosennestel brach.\\ 
\end{tabular}
\scriptsize
\line(1,0){75} \newline
D Fr1 Fr5 \newline
\line(1,0){75} \newline
\textbf{1} \textit{Initiale} D Fr5  \textbf{7} \textit{Versal} Fr1  \textbf{13} \textit{Majuskel} D  \textbf{17} \textit{Majuskel} D  \textbf{29} \textit{Initiale} D  \newline
\line(1,0){75} \newline
\textbf{1} In] ÷n D  $\cdot$ die] eine Fr1 \textbf{2} küneginne] keneginne D frowe Fr1  $\cdot$ die] \textit{om.} Fr5 \textbf{5} clâriu] clarin Fr5 \textbf{8} Gawans] Gauwans Fr5 \textbf{9} \textit{Versfolge 423.11-12-9-10} Fr1   $\cdot$ mite] allez mite Fr1 \textbf{10} schiet] ensciet Fr1  $\cdot$ dem] der Fr5 \textbf{11} doch] Do Fr5  $\cdot$ werdiu] svͤze Fr1 \textbf{12} Gawans] Gauwans Fr5  $\cdot$ lîp] tot Fr1 \textbf{13} [Da]: Svs waren di zwene >da< inne Fr1 \textbf{16} kom] div quam Fr1 \textbf{18} dâ] in Fr5 \textbf{20} Fasâne vnt parterîse Fr1 \textbf{21} guote] \textit{om.} Fr1 \textbf{22} Gawan] Gauwan Fr5  $\cdot$ Kyngrimursel] kyngrimvrsêl D Kẏngrimvrsel Fr1 \textbf{23} ûz] von Fr1 \textbf{24} sît] do Fr1 \textbf{26} dies] die ez Fr5 \textbf{27} Antikonie] Antikonîe D Div kvneginne Fr1 Anthichonie Fr5  $\cdot$ in] an Fr5  $\cdot$ selbe] selbiv Fr5 \textbf{29} Swaz] ÷waz D  $\cdot$ kniender] kinder Fr5 \textbf{30} diu] der Fr5 \newline
\end{minipage}
\hspace{0.5cm}
\begin{minipage}[t]{0.5\linewidth}
\small
\begin{center}*m
\end{center}
\begin{tabular}{rl}
 & in die kemenâten sân\\ 
 & gie diu \textbf{vrouwe} und die zwêne man.\\ 
 & \textbf{von} den andern bleip si lære\\ 
 & - des pflâgen kamerære -,\\ 
5 & \textbf{wanne} \textbf{clâriu} juncvröuwelîn,\\ 
 & \textbf{der muose vil dâr inne} sîn.\\ 
 & diu künigîn mit zühten pflac\\ 
 & Gawanes, der ir ze herze\textit{n} lac.\\ 
 & dâ was der lantgrâve mite.\\ 
10 & \textbf{de\textit{r}} schiet \textbf{si} niender von dem site.\\ 
 & doch sorgete vil diu \textbf{werde} maget\\ 
 & umb Gawanes lîp, wart mir gesaget.\\ 
 & sus wâren die zwêne dâr inne\\ 
 & bî der küniginne,\\ 
15 & un\textit{z} daz der tac liez sînen strît.\\ 
 & diu naht kam, d\textit{ô} was ezzens zît.\\ 
 & môra\textit{z}, wîn \textbf{und} lûtertranc\\ 
 & brâhten juncvrouwen, \textbf{d\textit{â}} mitten \textit{k}ranc,\\ 
 & und andere guote spîse:\\ 
20 & \textit{f}asâne, pardrîse,\\ 
 & guote vische, blankiu wastel.\\ 
 & Gawan und Kingr\textit{im}ursel\\ 
 & wâren komen ûz grôzer nôt.\\ 
 & sît ez diu künigîn \textbf{bôt},\\ 
25 & \textbf{dô âzen si}, als si solten,\\ 
 & und andere, die \textbf{ez} wolten.\\ 
 & Anticonie in selbe sneit.\\ 
 & daz \textit{was} durch zuht in be\textit{i}den leit.\\ 
 & swaz man dâ \textbf{kniewender schenke\textit{n}} sach,\\ 
30 & ir \textbf{dekeinem} \textbf{diu} hosennestel brach.\\ 
\end{tabular}
\scriptsize
\line(1,0){75} \newline
m n o \newline
\line(1,0){75} \newline
\newline
\line(1,0){75} \newline
\textbf{3} bleip] bleit o \textbf{4} kamerære] kammere n \textbf{6} muose] muͯsten n \textbf{8} Gawanes] Gawans n o  $\cdot$ ze herzen] zehercze m \textbf{10} der] De m \textbf{12} Gawanes] gawans n o \textbf{15} unz] Vnd m \textbf{16} dô] das m  $\cdot$ was] was es n \textbf{17} môraz] Mora m \textbf{18} dâ mitten kranc] do mitten trang m do mit in rang n o \textbf{20} Guͯte vische blancken vasane pardrise m  $\cdot$ pardrîse] paradrisse o \textbf{21} blankiu] blang o \textbf{22} Kingrimursel] kingrnvrsel m kingrumúrsel n konigrimursel o \textbf{23} komen] koment n \textbf{24} bôt] gebot n o \textbf{26} ez] es icht n es niht o \textbf{27} Anticonie] Antitonie n o  $\cdot$ in] an o \textbf{28} daz] Do n  $\cdot$ was] \textit{om.} m  $\cdot$ beiden] beilden m \textbf{29} swaz] Was m n o  $\cdot$ dâ] do m n o  $\cdot$ kniewender] knuwende n (o)  $\cdot$ schenken] schenker m  $\cdot$ sach] [dasz]: sach o \textbf{30} dekeinem] do keinem n dekein o  $\cdot$ diu hosennestel] holsen nestel o \newline
\end{minipage}
\end{table}
\newpage
\begin{table}[ht]
\begin{minipage}[t]{0.5\linewidth}
\small
\begin{center}*G
\end{center}
\begin{tabular}{rl}
 & in die kemenâten sân\\ 
 & gienc diu \textbf{künigîn} unde die zwêne man.\\ 
 & \textbf{vor} den andern beleip si lære;\\ 
 & des pflâgen kamerære.\\ 
5 & \textbf{kleiniu} ju\textit{n}cvröuwelîn\\ 
 & \textbf{vil dort inne muose} sîn.\\ 
 & diu künigîn mit zühten pflac\\ 
 & Gawans, der \textit{ir ze} herzen lac.\\ 
 & dâ was der lantgrâve mite.\\ 
10 & \textbf{der} schiet \textbf{sich} ninder von dem site.\\ 
 & \begin{large}D\end{large}och sorgte vil diu \textbf{werde} maget\\ 
 & umbe Gawans lîp, wart mir gesaget.\\ 
 & sus wâren die zwêne dâ inne\\ 
 & bî der küniginne,\\ 
15 & unze daz der tac lie sînen strît.\\ 
 & diu naht kom, dô was ezzens zît.\\ 
 & môraz, wîn, lûtertranc\\ 
 & brâhten juncvrouwen, \textbf{dâ} enmitten kranc,\\ 
 & unde ander guote spîse:\\ 
20 & fasân, pardrîse,\\ 
 & guote vische \textbf{unde} blankiu wastel.\\ 
 & Gawan unde Kingrimursel\\ 
 & wâren komen ûz grôzer nôt.\\ 
 & sît ez diu künigîn \textbf{gebôt},\\ 
25 & \textbf{si âzen}, als si solden,\\ 
 & unde ander, die\textbf{s} \textbf{iht} wolden.\\ 
 & Antikonie in selbe sneit.\\ 
 & daz was durch zuht in beiden leit.\\ 
 & swaz man dâ \textbf{kinder senken} sach,\\ 
30 & ir \textbf{neheinen} \textbf{diu} hosennestel brach.\\ 
\end{tabular}
\scriptsize
\line(1,0){75} \newline
G I O L M Q R Z \newline
\line(1,0){75} \newline
\textbf{1} \textit{Initiale} I O Z  \textbf{11} \textit{Initiale} G  \textbf{17} \textit{Initiale} I  \newline
\line(1,0){75} \newline
\textbf{1} in] ÷n O \textbf{2} die] \textit{om.} L \textbf{3} vor den] von den I Vor dem Q Vor der R \textbf{4} des] was des I Der L \textbf{5} kleiniu] wan vil klarev I Wan chlariv O (M) (Q) (R) (Z) Wan clarie L  $\cdot$ juncvröuwelîn] ivchfroͮwelin G \textbf{6} die vil dort mugen sin I  $\cdot$ Der ([D*]: Der L ) mvͦse vil dort inne sin O (L) (M) (Q) (R) (Z) \textbf{8} Gawans] Gawanes G Gawins R  $\cdot$ ir ze] in ir G Jr am R  $\cdot$ lac] [pflac]: lac Z \textbf{10} sich] sîe O (L) (M) (Z) o\textit{m. } Q  $\cdot$ ninder] nirgen M nyenan R  $\cdot$ dem] den M \textbf{11} Doch] do I (O)  $\cdot$ sorgte] sorget I (O) L R Z  $\cdot$ werde] shone I \textit{om.} O selbe Q R \textbf{12} umbe] Von Q  $\cdot$ Gawans] Gawan I Gawins R  $\cdot$ lîp wart] ist I \textbf{14} küniginne] shonen chuneginne I \textbf{15} daz] \textit{om.} I L R  $\cdot$ lie] hie L  $\cdot$ strît] [schein]: streit Q \textbf{16} naht] nach Q  $\cdot$ dô] ez I da O M Z vnd R  $\cdot$ ezzens] esses Q \textbf{17} wîn] vnde win O  $\cdot$ lûtertranc] loűter trawc Q \textbf{18} dâ] \textit{om.} M do Q de R  $\cdot$ enmitten] mitten O emitten R  $\cdot$ kranc] tranck Q \textbf{20} pardrîse] parturise I peredise L (R) vnde pardise M \textbf{21} guote] Gutter R  $\cdot$ unde blankiu] blanchiv O (L) (Q) wisse R \textbf{22} Gawan] Gawin R  $\cdot$ Kingrimursel] Kyngrimvrsel O (Q) kᵫngrumursel R \textbf{26} dies iht] die sin nih I die ez O (R) die sin iht Z \textbf{27} Antikonie] Anticonia I Antykonye O Anthiconie M Anthikonie Q Antybonye R  $\cdot$ selbe] selbin M (Q) (Z) selber R \textbf{28} daz] Da Z  $\cdot$ zuht] ir zvht O (L) (M) (Q) ir zuch R  $\cdot$ in] \textit{om.} O \textbf{29} swaz] Was L M Q R  $\cdot$ dâ] \textit{om.} O L do Q  $\cdot$ kinder] knîender O (L) (M) (Q) (R) (Z)  $\cdot$ senken] schenchen O (L) (Q) (R) (Z) \textbf{30} neheinen] deheim I (O) (L) (M) (Q) (Z)  $\cdot$ diu] do di Q \newline
\end{minipage}
\hspace{0.5cm}
\begin{minipage}[t]{0.5\linewidth}
\small
\begin{center}*T
\end{center}
\begin{tabular}{rl}
 & In die kemenâten sân\\ 
 & gie di\textit{u} \textbf{künegîn} unde die zwêne man.\\ 
 & \textbf{vor} den andern bleip si lære\\ 
 & - des pflâgen kamerære -,\\ 
5 & \textbf{wan} \textbf{clâriu} juncvröuwelîn,\\ 
 & \textbf{der muose vil dort inne} sîn.\\ 
 & \begin{large}D\end{large}iu künegîn mit zühten pflac\\ 
 & Gawans, der ir ze herzen lac.\\ 
 & dâ was der lantgrâve mite.\\ 
10 & \textbf{den} schiet \textbf{si} niender von dem site.\\ 
 & doch sorgete vil diu \textbf{selbe} maget\\ 
 & umbe Gawans lîp, wart mir gesaget.\\ 
 & Sus wâren die zwêne dâ inne\\ 
 & bî der küneginne,\\ 
15 & unz daz der tac lie sînen strît.\\ 
 & di\textit{u} naht kom, dô was ezzens zît.\\ 
 & Môraz, wîn, lûtertranc\\ 
 & brâhten juncvrouwen, enmitten kranc,\\ 
 & unde ander guote spîse:\\ 
20 & fasân, pardrîse,\\ 
 & guote vische \textbf{unde} blank\textit{iu} wastel.\\ 
 & Gawan unde Kyngrimursel\\ 
 & wâren komen ûz grôzer nôt.\\ 
 & sît ez diu künegîn \textbf{gebôt},\\ 
25 & \textbf{si âzen}, als si solten,\\ 
 & unde andere, die\textbf{s} \textbf{iht} wolten.\\ 
 & Antickonie in selbe sneit.\\ 
 & daz was durch \textbf{ir} zuht in beiden leit.\\ 
 & Swaz man dâ \textbf{kniender schenkel} sach,\\ 
30 & ir \textbf{deheinem} \textbf{kein} hosennestel brach.\\ 
\end{tabular}
\scriptsize
\line(1,0){75} \newline
T U V W \newline
\line(1,0){75} \newline
\textbf{1} \textit{Majuskel} T  \textbf{7} \textit{Initiale} T U  \textbf{13} \textit{Initiale} W   $\cdot$ \textit{Majuskel} T  \textbf{17} \textit{Majuskel} T  \textbf{29} \textit{Majuskel} T  \newline
\line(1,0){75} \newline
\textbf{2} gie diu künegîn] gie die kvnegin T Die kvnigin gie V  $\cdot$ die] \textit{om.} U \textbf{3} vor] [Vo*]: Von V \textbf{6} muose] mvese T mvͤste V muͦsten W  $\cdot$ dort] dar V \textbf{8} Gawans] Gewans W  $\cdot$ ze herzen] zuͦ herre U [*]: zeherzen V \textbf{10} den schiet] Dern schiet V Der enschiede W \textbf{11} sorgete] sorget W  $\cdot$ selbe] [*]: werde V selbig W \textbf{15} unz] Mit U \textbf{16} diu] dîe T \textbf{17} lûtertranc] [*utertrang]: vnde lutertrang V vnd lautertranck W \textbf{18} enmitten] do in miten U (V) \textbf{21} unde] \textit{om.} U V W  $\cdot$ blankiu] blancke T \textbf{22} Kyngrimursel] kingrimorsel U kẏnkrimursel V kingrimursel W \textbf{24} künegîn] kúnige W \textbf{25} si âzen] [*]: Da asen sv́ V \textbf{26} iht] \textit{om.} V \textbf{27} Antickonie] Antikonie U W Antẏkonie V  $\cdot$ selbe] selber V W \textbf{28} ir] \textit{om.} V  $\cdot$ in] [*]: in V \textbf{29} Waz man kinder schenken sach U  $\cdot$ Swaz] Waz W  $\cdot$ dâ] do V W  $\cdot$ kniender] kúnweder W  $\cdot$ schenkel] schenken V (W) \textbf{30} deheinem kein] dekeime die U (W) dekeinre den V \newline
\end{minipage}
\end{table}
\end{document}
