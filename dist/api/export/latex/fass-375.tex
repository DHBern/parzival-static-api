\documentclass[8pt,a4paper,notitlepage]{article}
\usepackage{fullpage}
\usepackage{ulem}
\usepackage{xltxtra}
\usepackage{datetime}
\renewcommand{\dateseparator}{.}
\dmyyyydate
\usepackage{fancyhdr}
\usepackage{ifthen}
\pagestyle{fancy}
\fancyhf{}
\renewcommand{\headrulewidth}{0pt}
\fancyfoot[L]{\ifthenelse{\value{page}=1}{\today, \currenttime{} Uhr}{}}
\begin{document}
\begin{table}[ht]
\begin{minipage}[t]{0.5\linewidth}
\small
\begin{center}*D
\end{center}
\begin{tabular}{rl}
\textbf{375} & \begin{large}D\end{large}âr ûz die heiden manege wât\\ 
 & würkent, diu vil \textbf{spæhe} hât,\\ 
 & \textbf{mit} rehter art û\textit{z} sîden.\\ 
 & Lyppaut hiez balde snîden\\ 
5 & sîner tohter kleider.\\ 
 & er \textbf{missete} gern \textbf{ir} beider,\\ 
 & der bœsten unt der besten.\\ 
 & \textbf{einen pfelle mit golde} vesten,\\ 
 & \textbf{den} sneit man an daz vreuwelîn.\\ 
10 & ir muose ein arm \textbf{geblœzet} sîn.\\ 
 & dâ was ein ermel von genomen,\\ 
 & der solte Gawane komen.\\ 
 & daz \textbf{was} ir prîsente:\\ 
 & pfelle von Nourjente,\\ 
15 & verre ûz heidenschaft gevuort.\\ 
 & \textbf{der} het ir zeswen arm geruort,\\ 
 & doch \textbf{an den roc niht} genæt.\\ 
 & dâ\textbf{ne} wart nie vadem zuo gedræt.\\ 
 & den \textbf{brâhte} Clauditte dar\\ 
20 & Gawane, dem wol gevar.\\ 
 & dô wart sîn lîp \textbf{gar} sorgen vrî.\\ 
 & sîner schilde wâren drî.\\ 
 & ûf einen sluog ern al zehant.\\ 
 & al sîn trûren gar verswant.\\ 
25 & sînen grôzen danc er niht versweic.\\ 
 & vil dicke \textbf{er} dem wege neic,\\ 
 & den diu juncvrouwe gienc,\\ 
 & diu in sô güetlîche enpfienc\\ 
 & unt \textbf{in} sô minneclîche\\ 
30 & an vröuden machte rîche.\\ 
\end{tabular}
\scriptsize
\line(1,0){75} \newline
D \newline
\line(1,0){75} \newline
\textbf{1} \textit{Initiale} D  \newline
\line(1,0){75} \newline
\textbf{3} ûz] v̂f D \textbf{4} Lyppaut] Lyppaot D \textbf{14} Nourjente] Noͮriente D \textbf{19} Clauditte] Clavditte D \newline
\end{minipage}
\hspace{0.5cm}
\begin{minipage}[t]{0.5\linewidth}
\small
\begin{center}*m
\end{center}
\begin{tabular}{rl}
 & dâr ûz die heiden manige \textit{wâ}t\\ 
 & \dag wirket\dag , diu vil \textbf{spæhe} hât,\\ 
 & \textbf{mit} rehter art ûz sîd\textit{e}n.\\ 
 & Lipp\textit{ou}t hiez balde snîden\\ 
5 & \begin{large}S\end{large}îner tohter kleider.\\ 
 & er \textbf{missete} gerne \textbf{ir} beider,\\ 
 & der bœsten und der besten.\\ 
 & \textbf{einen pfelle mit golde} vesten,\\ 
 & \textbf{des} sneit man an daz vröuwelîn.\\ 
10 & ir muose ein arm \textbf{geblœzet} sîn.\\ 
 & dâ was ein ermel von genomen,\\ 
 & der solte Gawane komen.\\ 
 & daz \textbf{was} ir prêsente:\\ 
 & pfelle von Noriente,\\ 
15 & \textit{v}erre ûz heidenschaft gevuort.\\ 
 & \textbf{der} hete ir zesewen arm geruort,\\ 
 & doch  \textbf{den roc niht} genæt.\\ 
 & dâ \textbf{en}wart nie vadem zuo gedræt.\\ 
 & den \textbf{brâhte} Clauditte dar\\ 
20 & Gawane, dem wol gevar.\\ 
 & dô wart sîn lîp \textbf{vor} sorgen vrîe.\\ 
 & sîner schilte wâren drîe.\\ 
 & ûf einen sluoc er  alzehant.\\ 
 & allez sîn trûren gar verswant.\\ 
25 & sînen grôzen danc er niht versweic,\\ 
 & vil dicke dem wege neic,\\ 
 & den diu juncvrowe gienc,\\ 
 & diu in sô güeteclîch enpfienc\\ 
 & und \textbf{in} sô minneclîche\\ 
30 & an vröuden maht\textit{e} rîche.\\ 
\end{tabular}
\scriptsize
\line(1,0){75} \newline
m n o \newline
\line(1,0){75} \newline
\textbf{5} \textit{Initiale} m n  \newline
\line(1,0){75} \newline
\textbf{1} dâr ûz] Darius n o  $\cdot$ die] der n o  $\cdot$ manige] manigen o  $\cdot$ wât] not m \textbf{3} sîden] sidin m (o) \textbf{4} Lippout] Lippaot m o Lẏppaot n  $\cdot$ snîden] smiden o \textbf{6} er] Jr o \textbf{9} des] Der n Dasz o \textbf{10} muose] muͯsse m muͯste n  $\cdot$ geblœzet] geblaset o \textbf{11} von genomen] fur genomnen o \textbf{12} Gawane] gawanen n gewane o \textbf{14} Noriente] norriente m [origente]: norigente o \textbf{15} verre] Were m \textbf{17} roc] rúg o \textbf{18} dâ] Do o  $\cdot$ zuo] \textit{om.} o \textbf{19} Clauditte] Klauditte m claudite n o \textbf{20} Gawane] Gawan n o \textbf{21} lîp vor] von n o \textbf{24} trûren] truwer o \textbf{30} vröuden] frouide o  $\cdot$ mahte] mahten m \newline
\end{minipage}
\end{table}
\newpage
\begin{table}[ht]
\begin{minipage}[t]{0.5\linewidth}
\small
\begin{center}*G
\end{center}
\begin{tabular}{rl}
 & dâr ûz die heidene manige wât\\ 
 & würke\textit{n}t, diu vil \textbf{spæhe} hât,\\ 
 & \textbf{von} rehter art ûz sîden.\\ 
 & Libaut hiez \textit{balde} snîden\\ 
5 & sîner tohter kleider.\\ 
 & er \textbf{miste} gerne beider,\\ 
 & der bœsesten unde der besten.\\ 
 & \textbf{von golde einen pfelle} vesten\\ 
 & \begin{large}S\end{large}neit man an daz vröuwelîn.\\ 
10 & ir muose ein arm \textbf{geblœzet} sîn.\\ 
 & dâ was ein ermel vone genomen,\\ 
 & der solte Gawane komen.\\ 
 & daz \textbf{was} ir prêsente:\\ 
 & \textit{p}felle von Novriente,\\ 
15 & verre ûz heidenschaft gevuort.\\ 
 & \textbf{er} het ir zeswen arm geruort,\\ 
 & doch \textbf{niht an den roc} genæt.\\ 
 & dâ\textbf{ne} wart nie vadem zuo gedræt.\\ 
 & den \textbf{brâhte} Claudite dar\\ 
20 & Gawane, dem wolgevar.\\ 
 & dô wart sîn lîp \textbf{gar} sorgen vrî.\\ 
 & sîner schilte wâren drî.\\ 
 & ûf einen sluog ern al zehant.\\ 
 & alsîn trûren gar verswant.\\ 
25 & sînen grôzen danc er niht versweic.\\ 
 & vil dicke\textbf{r} dem wege neic,\\ 
 & den diu juncvrouwe gienc,\\ 
 & diu in sô güetlîche enpfienc\\ 
 & unde \textbf{in} sô minniclîche\\ 
30 & an vröiden machte rîche.\\ 
\end{tabular}
\scriptsize
\line(1,0){75} \newline
G I O L M Q R Z Fr38 \newline
\line(1,0){75} \newline
\textbf{3} \textit{Initiale} I  \textbf{9} \textit{Initiale} G   $\cdot$ \textit{Capitulumzeichen} R  \textbf{15} \textit{Initiale} I  \newline
\line(1,0){75} \newline
\textbf{1} \textit{Die Verse 370.13-412.12 fehlen} Q  \textbf{2} würkent] wurchet G (R)  $\cdot$ diu] die I  $\cdot$ hât] stat I R Z \textbf{3} ûz] vf O (M) \textbf{4} Libaut] Lybavt O L Z Fr38 Lybant R  $\cdot$ balde] \textit{om.} G \textbf{6} miste] misset L Z \textbf{7} den bosten vnd den besten I  $\cdot$ bœsesten] hosten M \textbf{9} daz vröuwelîn] dem tohterlin Z \textbf{10} muose] muͤse I muͦsz R  $\cdot$ geblœzet] enblozet I \textbf{11} genomen] komen R \textbf{12} solte] muͦste R  $\cdot$ Gawane] Gawanen O gawan M (R) (Z) (Fr38)  $\cdot$ komen] fromen R \textbf{14} pfelle] ein phelle G  $\cdot$ Novriente] grigente I Nevriente O Z Navriente L avriente M oriente R Noriente Fr38 \textbf{15} verre] Were R \textbf{16} zeswen] rechtten R  $\cdot$ arm] hant O \textbf{17} doch] vnd doch I  $\cdot$ roc] arm M \textbf{18} dâne] Da O (R) Fr38  $\cdot$ zuo] \textit{om.} Z \textbf{19} den] [da]: den G  $\cdot$ Claudite] clauditte I R (Z) Clavdite O Clavditte Fr38  $\cdot$ dar] clare M \textbf{20} Gawane] Gawan I O Z Fr38 \textbf{21} dô] Da M Z  $\cdot$ gar] \textit{om.} I \textbf{23} ern] er R  $\cdot$ al] sa I \textit{om.} R \textbf{24} alsîn] Als sin L R  $\cdot$ trûren] trure R \textbf{27} den] Da R \textbf{29} in] \textit{om.} L  $\cdot$ minniclîche] manliche I \textbf{30} machte] machet I  $\cdot$ rîche] richen R \newline
\end{minipage}
\hspace{0.5cm}
\begin{minipage}[t]{0.5\linewidth}
\small
\begin{center}*T
\end{center}
\begin{tabular}{rl}
 & dâ ûz die heidene manege wât\\ 
 & würkent, di\textit{u} vil \textbf{spæhiu} hât,\\ 
 & \textbf{von} rehter art ûz sîden.\\ 
 & Lybaut hiez balde snîden\\ 
5 & sîner tohter kleider.\\ 
 & er \textbf{mischete} gerne beider,\\ 
 & der bœsten unde der besten.\\ 
 & \textbf{von golde einen pfelle} vesten\\ 
 & sneit man an daz vröuwelîn.\\ 
10 & ir muose ein arm \textbf{enblœzet} sîn.\\ 
 & dâ was ein ermel von genomen,\\ 
 & der solte Gawane komen.\\ 
 & daz \textbf{wære} ir prîsente:\\ 
 & pfelle von Noriente,\\ 
15 & verre ûz heidenschaft gevüeret.\\ 
 & \textbf{er} hete ir zeswen arm gerüeret,\\ 
 & doch \textbf{niht an den roc} genæt.\\ 
 & dâ wart nie vaden zuo gedræt.\\ 
 & den \textbf{gap} Claudite dar\\ 
20 & Gawane, dem wol gevar.\\ 
 & Dô wart sîn lîp \textbf{gar} sorgen vrî.\\ 
 & sîner schilte, \textbf{der} wâren drî.\\ 
 & ûf einen sluoc ern alzehant.\\ 
 & alsîn trûren gar verswant.\\ 
25 & sînen grôzen danc er niht versweic.\\ 
 & vil dicke \textbf{er} dem wege neic,\\ 
 & den diu juncvrouwe gienc,\\ 
 & diu in sô güetliche enpfienc\\ 
 & unde sô minneclîche\\ 
30 & an vröuden mahte rîche.\\ 
\end{tabular}
\scriptsize
\line(1,0){75} \newline
T V W \newline
\line(1,0){75} \newline
\textbf{21} \textit{Majuskel} T  \newline
\line(1,0){75} \newline
\textbf{1} dâ] Das W \textbf{2} diu] die T \textbf{4} Lybaut] Lẏbaut V Libout W  $\cdot$ snîden] zerschneiden W \textbf{6} beider] [*]: ir beider V \textbf{10} muose] mvͤste V  $\cdot$ enblœzet] gebloͤzet V (W) \textbf{13} wære] [w*]: waz V war W \textbf{14} Noriente] Novriente V oriente W \textbf{16} zeswen] rehten V \textbf{19} gap] brachte W  $\cdot$ Claudite] klaudite W \textbf{22} der] \textit{om.} W \textbf{23} ern] er W \textbf{25} versweic] [vermeit]: versweic T \textbf{26} wege] \textit{om.} W \textbf{29} unde] [V*]: Vnde in V Vnd in W \textbf{30} an] Mit W \newline
\end{minipage}
\end{table}
\end{document}
