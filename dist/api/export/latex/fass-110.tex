\documentclass[8pt,a4paper,notitlepage]{article}
\usepackage{fullpage}
\usepackage{ulem}
\usepackage{xltxtra}
\usepackage{datetime}
\renewcommand{\dateseparator}{.}
\dmyyyydate
\usepackage{fancyhdr}
\usepackage{ifthen}
\pagestyle{fancy}
\fancyhf{}
\renewcommand{\headrulewidth}{0pt}
\fancyfoot[L]{\ifthenelse{\value{page}=1}{\today, \currenttime{} Uhr}{}}
\begin{document}
\begin{table}[ht]
\begin{minipage}[t]{0.5\linewidth}
\small
\begin{center}*D
\end{center}
\begin{tabular}{rl}
\textbf{110} & sô lâz er\textbf{z} mir ze vrühte komen.\\ 
 & ich hân doch schaden \textbf{ze} vil genomen\\ 
 & \begin{large}A\end{large}n mînem stolzen, werden man.\\ 
 & wie hât der tôt ze mir getân!\\ 
5 & \textbf{er enpfienc} nie wîbes minnen teil,\\ 
 & ern wære \textbf{al ir} vröuden geil,\\ 
 & in müete wîbes riwe.\\ 
 & \textbf{daz} riet sîn manlîch triwe,\\ 
 & \textbf{wand}er was valsches lære."\\ 
10 & nû hœrt ein ander mære,\\ 
 & Waz diu vrouwe dô begienc.\\ 
 & kint unt \textbf{bûch} si zir gevienc\\ 
 & mit armen und mit henden.\\ 
 & si sprach: "mir sol got senden\\ 
15 & die werden vruht von Gahmurete.\\ 
 & daz ist mînes herzen bete.\\ 
 & got wende mich \textbf{sô} tumber nôt.\\ 
 & \textbf{daz} wære Gahmuretes ander tôt,\\ 
 & ob ich mich selben slüege,\\ 
20 & die wîle \textbf{ich bî mir} trüege,\\ 
 & daz ich von sîner minne enpfienc,\\ 
 & der mannes triwe an mir begienc."\\ 
 & Diu vrouwe enruochte, wer daz sach:\\ 
 & daz hemde \textbf{von der brust si} brach.\\ 
25 & ir \textbf{brüstel} linde unt wîz,\\ 
 & dar an \textbf{kêrte si ir} vlîz.\\ 
 & si dructes an ir rôten munt.\\ 
 & si tet wîplîche vuore kunt.\\ 
 & \textbf{Alsus} sprach diu wîse:\\ 
30 & "dû bist \textbf{kaste} eines kindes spîse.\\ 
\end{tabular}
\scriptsize
\line(1,0){75} \newline
D Fr33 \newline
\line(1,0){75} \newline
\textbf{3} \textit{Initiale} D  \textbf{11} \textit{Majuskel} D  \textbf{23} \textit{Initiale} Fr33   $\cdot$ \textit{Majuskel} D  \textbf{29} \textit{Majuskel} D  \newline
\line(1,0){75} \newline
\textbf{1} erz mir] er mirn Fr33 \textbf{3} stolzen] suezen Fr33 \textbf{6} al ir] aller Fr33 \textbf{11} dô] da Fr33 \textbf{12} zir] zu Fr33 \textbf{15} Gahmurete] Gahmvrete D Gamurete Fr33 \textbf{18} Gahmuretes] Gahmvretes D Gamuretes Fr33 \textbf{23} daz] iz Fr33 \textbf{25} brüstel] bruste Fr33 \textbf{26} an] \textit{om.} Fr33 \textbf{29} Alsus] Also Fr33 \textbf{30} dû] di Fr33 \newline
\end{minipage}
\hspace{0.5cm}
\begin{minipage}[t]{0.5\linewidth}
\small
\begin{center}*m
\end{center}
\begin{tabular}{rl}
 & sô lâz er mir \textbf{in} ze vrühte komen.\\ 
 & ich hân doch schaden \textbf{ze} vil genomen\\ 
 & an mînem stolzen, werden man.\\ 
 & wie hât der tôt ze mir getân!\\ 
5 & \textbf{er \textit{en}pfienc} nie wîbes minnen teil,\\ 
 & er enwær \textbf{aller} vröuden geil,\\ 
 & in muote wîbes riuwe.\\ 
 & \textbf{d\textit{e}n} riet sîn manlîch triuwe,\\ 
 & \textbf{wanne} er was valsches lære."\\ 
10 & nû hœret ein ander mære,\\ 
 & waz diu vrouwe dô begienc.\\ 
 & kint und \textbf{bûch} si zuo ir gevienc\\ 
 & mit armen und mit henden.\\ 
 & si sprach: "mir sol got senden\\ 
15 & die werden vruht von Gahmurete.\\ 
 & daz ist mînes herzen bete.\\ 
 & got wende mich tumber nôt.\\ 
 & \textbf{daz} wære Gahmuretes ander tôt,\\ 
 & ob ich mich selben slüege,\\ 
20 & die wîle \textbf{ich bî mir} trüege,\\ 
 & daz ich von sîner minne enpfienc,\\ 
 & der mannes triuwe an mir begienc."\\ 
 & diu vrowe enruochte, wer daz sach:\\ 
 & daz hemede \textbf{von der brust s\textit{i}} brach.\\ 
25 & ir \textbf{brüstel} linde unde wîz,\\ 
 & dar an \textbf{si kêrte ir} vlîz.\\ 
 & si dructe si an ir rôten munt.\\ 
 & si tet wîplîche vuore kunt.\\ 
 & \textbf{alsu\textit{s}} \textit{s}prach diu wîse:\\ 
30 & "dû bist \textbf{\textit{ko}ste} eines kindes spîse.\\ 
\end{tabular}
\scriptsize
\line(1,0){75} \newline
m n o \newline
\line(1,0){75} \newline
\newline
\line(1,0){75} \newline
\textbf{2} ze] so n \textbf{3} mînem] mẏnen o \textbf{5} er enpfienc] Erpfing m  $\cdot$ minnen] mynne n mẏnen o \textbf{7} muote] muͯget n muͦgent o \textbf{8} den] Dan m o  $\cdot$ riet] reit o \textbf{12} gevienc] ving n enpfing o \textbf{15} werden] werde n o  $\cdot$ Gahmurete] gahmurette m gamirete n [gamuͯ*]: gamuͯrete o \textbf{17} tumber] so tumber n o \textbf{18} Gahmuretes] gahmurettes m gamiretes n gamuͯretes o \textbf{19} selben] selb n selbes o \textbf{24} si] sit m \textbf{25} brüstel] bruͯste o \textbf{26} si kêrte ir] so kerte sú ires hertzen n so kerte sie jren o \textbf{27} dructe] trucke o \textbf{29} alsus sprach] Alsus so sprach m  $\cdot$ diu] der o \textbf{30} bist koste] bistste m \newline
\end{minipage}
\end{table}
\newpage
\begin{table}[ht]
\begin{minipage}[t]{0.5\linewidth}
\small
\begin{center}*G
\end{center}
\begin{tabular}{rl}
 & sô lâzer mir\textbf{n} ze vruht komen.\\ 
 & ich hân doch schaden \textit{v}il genomen\\ 
 & an mînem \textit{stolz}en, werden man.\\ 
 & wie hât der tôt ze mir getân!\\ 
5 & \textbf{der gewan} nie wîbes minnen teil,\\ 
 & er enwære \textbf{al ir} vröuden geil,\\ 
 & in muote wîbes riwe.\\ 
 & \textbf{daz} riet sîn manlîch triwe.\\ 
 & er was \textbf{gar} valsches lære."\\ 
10 & nû hœret ein ander mære,\\ 
 & waz diu vrouwe dô begienc.\\ 
 & kint und \textbf{lîp} si zir gevienc\\ 
 & mit armen und mit henden.\\ 
 & si sprach: "mir sol got senden\\ 
15 & die werden vruht von Gahmuret.\\ 
 & daz ist mînes herzen bet.\\ 
 & got wende mich \textbf{sô} tumber nôt.\\ 
 & \textbf{daz} wære Gahmuretes ander tôt,\\ 
 & obe ich mich selben slüege,\\ 
20 & die wîle \textbf{daz ich} trüege,\\ 
 & daz ich von sîner minne enpfie,\\ 
 & der mannes triwe an mir begie."\\ 
 & diu vrouwe enruoht, wer daz sach:\\ 
 & daz hemde \textbf{si von der brüste} brach.\\ 
25 & ir \textbf{brüste} linde und wîz,\\ 
 & dar an \textbf{kêrte si ir} vlîz.\\ 
 & si druhte si an ir rôten munt.\\ 
 & si tet wîplîche vuore kunt.\\ 
 & \textbf{alsô} sprach diu wîse:\\ 
30 & "dû bist \textbf{kaste} eines kindes spîse.\\ 
\end{tabular}
\scriptsize
\line(1,0){75} \newline
G I O L M Q R Z \newline
\line(1,0){75} \newline
\textbf{1} \textit{Initiale} O  \textbf{3} \textit{Initiale} R Z  \textbf{11} \textit{Initiale} I  \textbf{27} \textit{Initiale} I  \newline
\line(1,0){75} \newline
\textbf{1} \textit{Die Verse 110.1-2 fehlen} R   $\cdot$ sô] ÷o O  $\cdot$ lâzer] laze O (L) (Q)  $\cdot$ mirn] mir I in mir L (M) (Q)  $\cdot$ ze vruht] zefreuden I zu fruchten Q (Z) \textbf{2} vil] zevil G \textbf{3} mînem] mynē M (Q) minen R  $\cdot$ stolzen werden] lieben werden G werden stolzen I werdem stoltzen Q \textbf{5} der] ern I Er Z  $\cdot$ gewan] enphie O L (M) (Q) (R) (Z)  $\cdot$ minnen] minnre I minne O (L) (Q) R \textit{om.} M minne ein Z \textbf{6} er] Jr M  $\cdot$ enwære] wer O Q R  $\cdot$ al] aller I (L) alle R  $\cdot$ vröuden] froude M (R) (Z) \textbf{8} riet] Rett R \textbf{9} er] Es R  $\cdot$ valsches] falscher Q \textbf{11} waz] Vaz I  $\cdot$ dô] da O M Z \textbf{12} lîp] bvͦch Z  $\cdot$ gevienc] viench O (M) (Q) (Z) \textbf{15} werden] werde M R  $\cdot$ Gahmuret] Gamvret O Gahmuͯret L gamúret Q gamuret M Z \textbf{17} mich] mir Z  $\cdot$ sô] \textit{om.} L ausz Q  $\cdot$ tumber] meiner Q \textbf{18} daz] Ez O L M (Q) (R)  $\cdot$ Gahmuretes] gahmurets G Gamvretes O Gahmuͯretes L gamuretis M gamúretes Q Gahmurttes R gamuretes Z \textbf{19} selben] selber L (Q) R selbern M \textbf{20} daz ich] so ich L ich das R ich bi mir Z \textbf{23} enruoht] en rvͦchte O (L) (M) (R) enrucht auch Q (Z) \textbf{24} daz] Das selbe R \textbf{26} kêrte] kerten Q kert R so kert Z \textbf{27} druhte] trug Q  $\cdot$ si an] ez an L (M) (Q) (R) (Z) \textbf{30} dû] diu I Die Q  $\cdot$ kaste] gast I koste M  $\cdot$ eines] vnde I \newline
\end{minipage}
\hspace{0.5cm}
\begin{minipage}[t]{0.5\linewidth}
\small
\begin{center}*T (U)
\end{center}
\begin{tabular}{rl}
 & sô lâz er mir\textit{\textbf{n}} zuo vrühte komen.\\ 
 & ich hân doch schaden vil genomen\\ 
 & an mîme stolzen, werden man.\\ 
 & wie hât der tôt zuo mir getân!\\ 
5 & \textbf{er entvienc} nie wîbes minne teil,\\ 
 & er enwære \textbf{al ir} vreuden geil,\\ 
 & in muote wîbes riuwe.\\ 
 & \textbf{da\textit{z}} \textit{r}iet sîn manlîch triuwe.\\ 
 & er was \textbf{gar} valsche\textit{s} lære."\\ 
10 & nû hœret ein ander mære,\\ 
 & waz diu vrouwe dô begienc.\\ 
 & kint und \textbf{lîp} si zuo ir gevienc\\ 
 & mit armen und mit henden.\\ 
 & si sprach: "mir sol got senden\\ 
15 & die werden vruht von Gahmuret.\\ 
 & daz ist mînes herzen bet.\\ 
 & got wende mich \textbf{ûz} tumber nôt.\\ 
 & \textbf{ez} wære Gahmuretes ander tôt,\\ 
 & ob ich mich selbe slüege,\\ 
20 & die wîle \textbf{daz ich} trüege,\\ 
 & daz ich von sîner minne entvienc,\\ 
 & der mannes triuwe an mir begienc."\\ 
 & diu vrouwe enruochte \textbf{ouch}, \textit{wer} daz sach:\\ 
 & daz hemede \textbf{si von der brüste} brach.\\ 
25 & ir \textbf{brüste} lind\textit{e} und wîz,\\ 
 & dar an \textbf{kêrte si ir} vlîz.\\ 
 & \multicolumn{1}{l}{ - - - }\\ 
 & \multicolumn{1}{l}{ - - - }\\ 
 & \textbf{alsô} sprach diu wîse:\\ 
30 & "dû bist \textbf{\textit{k}ast} eines kindes spîse.\\ 
\end{tabular}
\scriptsize
\line(1,0){75} \newline
U V W T \newline
\line(1,0){75} \newline
\textbf{3} \textit{Initiale} W  \textbf{10} \textit{Majuskel} T  \textbf{23} \textit{Initiale} T  \newline
\line(1,0){75} \newline
\textbf{1} er mirn] er mir U er in mir W ern mirn T \textbf{3} stolzen werden] werden stolzen T \textbf{5} minne] \textit{om.} T \textbf{6} al ir] allir U aller [*]: ir V aller ir W  $\cdot$ geil] [*]: heil V \textbf{7} muote] muͦte starcke W \textbf{8} daz riet] Da sie ir iet U Das riete W \textbf{9} valsches] [valsche*]: valscher U \textbf{15} werden] werde V  $\cdot$ Gahmuret] Gahmuͦret U Gamurete V gamuret W Gahmvrete T \textbf{17} ûz] so V W T \textbf{18} Gahmuretes] Gahmuͦretes U Gamuretes V (W) \textbf{19} selbe] selber V W selben T  $\cdot$ slüege] erslvege T \textbf{20} daz] so W \textbf{21} sîner minne] minnen W \textbf{23} ouch] \textit{om.} W T  $\cdot$ wer daz sach] [*i*]: daz sach U \textbf{24} der brüste] irn brústen V den brvsten \textit{(}brvsten \textit{nachträglich teilweise radiert)} T \textbf{25} \textit{Versteile nachträglich weitgehend radiert} T   $\cdot$ ir brüste] [*]: Die worent V Ire brústelin W  $\cdot$ linde] linden U \textbf{26} kêrte] kert W \textbf{27} \textit{Die Verse 110.27-28 fehlen} U W   $\cdot$ [Si]: Sú truhte si an irn roten munt V  $\cdot$ si drvhtez an ir roten mvnt T \textbf{28} [si]: sú tet wipliche fuͦre kunt V  $\cdot$ si tet [wiplicher]: wipliche vuͦre kvnt T \textbf{30} kast] gast U koste V gar W kvsce T  $\cdot$ eines] ein W \newline
\end{minipage}
\end{table}
\end{document}
