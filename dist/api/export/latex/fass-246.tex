\documentclass[8pt,a4paper,notitlepage]{article}
\usepackage{fullpage}
\usepackage{ulem}
\usepackage{xltxtra}
\usepackage{datetime}
\renewcommand{\dateseparator}{.}
\dmyyyydate
\usepackage{fancyhdr}
\usepackage{ifthen}
\pagestyle{fancy}
\fancyhf{}
\renewcommand{\headrulewidth}{0pt}
\fancyfoot[L]{\ifthenelse{\value{page}=1}{\today, \currenttime{} Uhr}{}}
\begin{document}
\begin{table}[ht]
\begin{minipage}[t]{0.5\linewidth}
\small
\begin{center}*D
\end{center}
\begin{tabular}{rl}
\textbf{246} & \begin{large}Û\end{large}fem teppeche \textbf{sach} der degen wert\\ 
 & ligen sîn harnasch unt zwei swert.\\ 
 & daz eine \textbf{der wirt im} geben hiez,\\ 
 & daz ander was von Gaheviez.\\ 
5 & \textbf{Sus} sprach er z\textbf{im} selben sân:\\ 
 & "owê, \textbf{durch waz} ist diz getân?\\ 
 & deiswâr, ich sol mich wâpen drîn.\\ 
 & ich leit in slâfe \textbf{al}sölhen pîn,\\ 
 & daz mir wachende arbeit\\ 
10 & noch hiute \textbf{wênec} ist bereit.\\ 
 & hât dirre wirt urliuges nôt,\\ 
 & sô leist ich gern sîn gebot\\ 
 & unt ir gebot mit triwen,\\ 
 & diu disen mantel niwen\\ 
15 & mir lêch durch ir güete.\\ 
 & wan stüende ir gemüete,\\ 
 & daz si dienst wolde nemen,\\ 
 & \textbf{des} \textbf{kunde} mich durch si gezemen -\\ 
 & unt doch niht \textbf{durch} ir minne,\\ 
20 & wan mîn wîp, diu küneginne,\\ 
 & ist an ir lîbe alse clâr\\ 
 & oder vürbaz, daz ist wâr."\\ 
 & \textbf{Er} tet, als \textbf{er} tuon sol:\\ 
 & von vuoz ûf wâpende er sich wol\\ 
25 & \textbf{durch} strîtes antwurte.\\ 
 & zwei swert er umbe gurte.\\ 
 & zer tür \textbf{ûz gienc} der werde degen.\\ 
 & \textbf{dâ} was sîn ors an die stegen\\ 
 & geheftet. schilt und sper\\ 
30 & lent dâr bî: daz was sîn ger.\\ 
\end{tabular}
\scriptsize
\line(1,0){75} \newline
D \newline
\line(1,0){75} \newline
\textbf{1} \textit{Initiale} D  \textbf{5} \textit{Majuskel} D  \textbf{23} \textit{Majuskel} D  \newline
\line(1,0){75} \newline
\textbf{20} diu] de D \newline
\end{minipage}
\hspace{0.5cm}
\begin{minipage}[t]{0.5\linewidth}
\small
\begin{center}*m
\end{center}
\begin{tabular}{rl}
 & ûf dem teppich \textbf{sach} der degen wert\\ 
 & ligen sînen harnasch und zwei swert.\\ 
 & daz eine \textbf{der wirt ime} geben hiez,\\ 
 & daz ander was von Gaheviez.\\ 
5 & \textbf{dô} sprach er zuo \textbf{dem} selben sân:\\ 
 & "ouwê, \textbf{durch waz} ist diz getân?\\ 
 & deiswâr, ich sol mich wâpen drîn.\\ 
 & ich leit in slâfe \textbf{al}soliche pîn,\\ 
 & daz mir wachende arbeit\\ 
10 & noch hiute, \textbf{wæ\textit{n} ich}, ist bereit.\\ 
 & hât dirre wirt urliuges nôt,\\ 
 & sô leist ich gerne sîn gebot\\ 
 & und ir gebot mit triuwen,\\ 
 & diu disen mantel niuwen\\ 
15 & mir lêch durch ir güete.\\ 
 & wan stüende ir gemüete,\\ 
 & daz si dienest wolte nemen,\\ 
 & \textbf{des} \textbf{kunde} mich durch si gezemen -\\ 
 & und doch niht \textbf{durch} ir minne,\\ 
20 & want mîn wîp, diu küniginne,\\ 
 & ist an ir lîbe als clâr\\ 
 & oder vürbaz, daz ist wâr."\\ 
 & \textbf{er} tet, als \textbf{er} tuon so\textit{l}:\\ 
 & von vuoze ûf wâpenter sich wo\textit{l}\\ 
25 & \textbf{durch} strîtes antwurte.\\ 
 & zwei swert er umbe gurte.\\ 
 & zer tür \textbf{ûz gienc} der werde \dag man\dag .\\ 
 & \textbf{dô} was sîn ros an die stegen\\ 
 & geheftet. schilt und sper\\ 
30 & l\textit{en}te dâr bî: daz was sîn ger.\\ 
\end{tabular}
\scriptsize
\line(1,0){75} \newline
m n o Fr69 \newline
\line(1,0){75} \newline
\newline
\line(1,0){75} \newline
\textbf{1} sach] sich o \textbf{2} sînen] sin n o  $\cdot$ harnasch] harnersch o \textbf{3} der] dor n \textbf{4} Gaheviez] gahe vies m o gohevies n \textbf{8} alsoliche] also sollich n solken Fr69 \textbf{10} wæn] wem m \textbf{15} lêch] lecht o  $\cdot$ ir] >ir< Fr69 \textbf{16} gemüete] genuͯte o \textbf{17} dienest wolte] dienste wol o \textbf{18} des] Das o \textbf{23} sol] solt m (o) \textbf{24} ûf wâpenter] vff woppent er n er vff wappent o  $\cdot$ wol] wolt m \textbf{26} umbe] vmb sich n \textbf{28} dô] Da Fr69  $\cdot$ stegen] stege san n o \textbf{30} lente] Luͯtte m Leite o Lenten Fr69  $\cdot$ bî] an n o \newline
\end{minipage}
\end{table}
\newpage
\begin{table}[ht]
\begin{minipage}[t]{0.5\linewidth}
\small
\begin{center}*G
\end{center}
\begin{tabular}{rl}
 & ûfem tepeche \textbf{vant} der degen wert\\ 
 & ligen sîn harnasch unde zwei swert.\\ 
 & daz eine \textbf{der wirt im} geben hiez,\\ 
 & daz ander was von Kahaviez.\\ 
5 & \textbf{dô} sprach er z\textbf{im} selben sân:\\ 
 & "\textit{a}wê, \textbf{war zuo} ist diz getân?\\ 
 & dêswâr, ich sol mich wâpen drîn.\\ 
 & ich leit in slâfe solhen pîn,\\ 
 & daz mir wachende arbeit\\ 
10 & noch hiute, \textbf{wæ\textit{n} ich}, ist bereit.\\ 
 & hât dirre wirt urliuges nôt,\\ 
 & sô leiste ich gerne sîn gebot\\ 
 & unde ir gebot mit triuwen,\\ 
 & diu disen mandel niuwen\\ 
15 & mir lêch durch ir güete.\\ 
 & wan stüende ir gemüete,\\ 
 & daz si dienst wolte nemen,\\ 
 & \textbf{des} \textbf{\textit{s}olte} mich durch si gezemen -\\ 
 & unde doch niht \textbf{durch} ir minne,\\ 
20 & wan mîn wîp, diu küniginne,\\ 
 & ist an ir lîbe \textit{\textbf{wol}} als clâr\\ 
 & oder vürbaz, daz ist wâr."\\ 
 & \textbf{er} tet, alse\textbf{r} tuon sol:\\ 
 & von vuoze ûf wâpent er sich wol\\ 
25 & \textbf{gein} strîtes antwurte.\\ 
 & zwei swert er umbe gurte.\\ 
 & zer tür \textbf{gieng ûz} der werde degen.\\ 
 & \textbf{\begin{large}D\end{large}ô} was sîn ors an die stegen\\ 
 & geheftet. schilt unde sper\\ 
30 & lent dâr bî: daz was sîn ger.\\ 
\end{tabular}
\scriptsize
\line(1,0){75} \newline
G I O L M Q R Z \newline
\line(1,0){75} \newline
\textbf{1} \textit{Initiale} O L M Z  \textbf{11} \textit{Initiale} I  \textbf{23} \textit{Initiale} L  \textbf{28} \textit{Initiale} G  \newline
\line(1,0){75} \newline
\textbf{1} ûfem] ÷fem O \textbf{2} ligen] Legen M  $\cdot$ zwei] sin O \textbf{4} Kahaviez] [ka*]: kahaviez G Gahauiez I Gaheviez O (M) (Z) kaheviez L kaheweisz Q kaheweis R \textbf{5} dô] Da O  $\cdot$ zim] hintz im Z  $\cdot$ selben] selber L R \textbf{6} awê] we G Owe L M (Q) R Z  $\cdot$ war zuo] warunb I war vf Z \textbf{7} dêswâr] Entswar Q Zwar Z \textbf{8} in] in dem O (Q)  $\cdot$ solhen] ê solhen I solche L (M) (Q) (R) (Z) \textbf{9} mir] mit Q \textbf{10} noch hiute wæn ich] noch hivte watlich G wen ich noch hiut I Noch hute wennic M \textbf{11} dirre] der O M Q \textbf{13} unde ir gebot] Jr gebot leiste ich L \textbf{14} disen] selben Q \textbf{15} mir] Mich R  $\cdot$ lêch] læch O \textbf{18} des] daz I (R)  $\cdot$ solte] wolte G konde Q  $\cdot$ si] sich Q sie wol Z  $\cdot$ gezemen] gemen O \textbf{19} doch] \textit{om.} O M Z  $\cdot$ durch] vmbe L \textbf{21} an ir] am Q  $\cdot$ wol als] als G wol so I O M Z \textbf{22} oder] als si oder I \textbf{24} ûf wâpent er] er wapinde M  $\cdot$ sich] sie Q \textbf{25} gein] Durch Q R  $\cdot$ antwurte] awentewr awontewrten Q auentúre R \textbf{26} umbe] im Q vmb sich R \textbf{27} gieng ûz] vz gie O (L) (Q) (R) (Z)  $\cdot$ werde] \textit{om.} I \textbf{28} Dô] Da M Z \textbf{30} lent] Leintent R Lagen Z \newline
\end{minipage}
\hspace{0.5cm}
\begin{minipage}[t]{0.5\linewidth}
\small
\begin{center}*T
\end{center}
\begin{tabular}{rl}
 & Ûf dem teppiche \textbf{sach} der degen wert\\ 
 & ligen sîn harnasch unde zwei swert.\\ 
 & daz eine \textbf{im der wirt} geben hiez,\\ 
 & daz ander was von Kaheviez.\\ 
5 & \textbf{\begin{large}D\end{large}ô} sprach er z\textbf{im} selben sân:\\ 
 & "ouwê, \textbf{durch waz} ist diz getân?\\ 
 & deiswâr, ich sol mich wâpen drîn.\\ 
 & ich leit in slâfe sölhen pîn,\\ 
 & daz mir wachende arbeit\\ 
10 & noch hiute, \textbf{wænich}, ist bereit.\\ 
 & Hât dirre wirt urliuges nôt,\\ 
 & sô leist ich gerne sîn gebot\\ 
 & unde ir gebot mit triuwen,\\ 
 & diu disen mantel niuwen\\ 
15 & mir lêch durch ir güete.\\ 
 & wan stüende ir gemüete,\\ 
 & daz si \textbf{mîn} dienst wolte nemen,\\ 
 & \textbf{daz} \textbf{kunde} mich durch si gezemen -\\ 
 & unde doch niht \textbf{umb}ir minne,\\ 
20 & wand mîn wîp, diu küneginne,\\ 
 & ist an ir lîbe als clâr\\ 
 & oder vürbaz, daz ist wâr."\\ 
 & \textbf{\begin{large}D\end{large}er} tet, als \textbf{man} tuon sol:\\ 
 & von vuoze ûf wâpenter sich wol\\ 
25 & \textbf{gegen} strîtes antwurte.\\ 
 & Zwei swert er umbe gurte.\\ 
 & zer tür \textbf{gie ûz} der werde degen.\\ 
 & \textbf{Nû} was sîn ors an die stegen\\ 
 & geheftet. schilt unde spe\textit{r}\\ 
30 & lente dâr bî: daz was sîn ger.\\ 
\end{tabular}
\scriptsize
\line(1,0){75} \newline
T U V W \newline
\line(1,0){75} \newline
\textbf{1} \textit{Majuskel} T  \textbf{5} \textit{Initiale} T U V  \textbf{11} \textit{Majuskel} T  \textbf{23} \textit{Initiale} T U W  \textbf{26} \textit{Majuskel} T  \textbf{28} \textit{Majuskel} T  \newline
\line(1,0){75} \newline
\textbf{2} sîn] seinen W  $\cdot$ zwei] das W \textbf{3} daz eine] Vnd das W \textbf{4} Dis ding in wunders nit erlies W \textbf{5} Do sprach der iunge werde man W  $\cdot$ zim] zu dem U \textbf{6} ouwê] \textit{om.} W \textbf{8} in] im W  $\cdot$ sölhen] soliche U (W) \textbf{10} wænich] wenig U wene W \textbf{15} lêch] leit U \textbf{17} mîn] minen V \textit{om.} W \textbf{19} umbir] durch W \textbf{23} Der] Er U (V) (W) \textbf{24} vuoze] vuͦzen U  $\cdot$ wâpenter] wapent er U V (W) \textbf{26} umbe] [vm]: vmb sich U vmbe sich V (W) \textbf{27} gie ûz] vz gieng V \textbf{28} die] der V \textbf{29} sper] sper lente T \textbf{30} lente] Stuͦnd W  $\cdot$ dâr] do U V W \newline
\end{minipage}
\end{table}
\end{document}
