\documentclass[8pt,a4paper,notitlepage]{article}
\usepackage{fullpage}
\usepackage{ulem}
\usepackage{xltxtra}
\usepackage{datetime}
\renewcommand{\dateseparator}{.}
\dmyyyydate
\usepackage{fancyhdr}
\usepackage{ifthen}
\pagestyle{fancy}
\fancyhf{}
\renewcommand{\headrulewidth}{0pt}
\fancyfoot[L]{\ifthenelse{\value{page}=1}{\today, \currenttime{} Uhr}{}}
\begin{document}
\begin{table}[ht]
\begin{minipage}[t]{0.5\linewidth}
\small
\begin{center}*D
\end{center}
\begin{tabular}{rl}
\textbf{602} & "\textit{\begin{large}H\end{large}}ie wil ich mîne reise sparn.\\ 
 & got waldes, welt ir vürbaz varn,\\ 
 & sône \textbf{sult} irz niht lengen,\\ 
 & \textbf{ellenthaftez} sprengen\\ 
5 & müezet ir \textbf{z}orse alsus\\ 
 & über \textbf{Ligweiz} Prelljus."\\ 
 & \textbf{Si habete al stille} ûf dem plân,\\ 
 & vürbaz reit hêr Gawan.\\ 
 & er \textbf{erhôrte} eines dræten wazzers val,\\ 
10 & daz het durchbrochen wît ein tal,\\ 
 & tief, ungeverteclîche.\\ 
 & Gawan, der ellens rîche,\\ 
 & nam daz ors mit den sporn.\\ 
 & \textbf{ez treip} der degen wol geborn,\\ 
15 & daz e\textit{z} mit zwein vüezen trat\\ 
 & hin über an den andern stat.\\ 
 & der sprunc mit valle muoste sîn.\\ 
 & \textbf{des} weinde iedoch diu herzogîn.\\ 
 & Der wâc was snel unt grôz.\\ 
20 & Gawan sîner kraft genôz,\\ 
 & \textbf{doch} truoc er harnaschlast.\\ 
 & \textbf{dô} was eines boumes ast\\ 
 & gewahsen in \textbf{dem} wazzers trân.\\ 
 & den begreif der starke man,\\ 
25 & wande\textbf{r} \textbf{dennoch gerne} lebte.\\ 
 & sîn sper dâ bî \textbf{im} \textbf{swebte};\\ 
 & daz begreif der wîgant.\\ 
 & er \textbf{steic} hin \textbf{ûf} \textbf{an}z lant.\\ 
 & Gringuljete swam ob unt unde,\\ 
30 & dem er helfen \textbf{dô} begunde.\\ 
\end{tabular}
\scriptsize
\line(1,0){75} \newline
D Z Fr7 \newline
\line(1,0){75} \newline
\textbf{1} \textit{Initiale} D Z  \textbf{7} \textit{Majuskel} D  \textbf{9} \textit{Initiale} Fr7  \textbf{19} \textit{Majuskel} D  \newline
\line(1,0){75} \newline
\textbf{1} Hie] Vie D \textbf{2} waldes] waldez Z \textbf{3} sône sult irz] So enduͤrft irs Z Sone durftirz Fr7 \textbf{4} [*]: ch es sprengen Fr7 \textbf{5} müezet] Mvͦset Fr7  $\cdot$ ir] ir al da Z \textbf{6} vber Ligweiz Prellivs D (Z)  $\cdot$ vber ligweis prellius Fr7 \textbf{7} habete] habt Z  $\cdot$ dem] den Fr7 \textbf{9} erhôrte] hort Z  $\cdot$ eines] [*]: iens Fr7  $\cdot$ val] wal Fr7 \textbf{10} wît ein] ein wites Fr7 \textbf{11} Tief vnd vnfvrticliche Z \textbf{14} ez] Daz Z \textbf{15} ez] er D \textbf{16} den andern] daz ander Z Fr7 \textbf{17} muoste] mvͤste Fr7 \textbf{19} was snel] snel was Z \textbf{20} Gawan] ::wan Fr7 \textbf{21} doch] Da Z  $\cdot$ harnaschlast] harnasches last Z (Fr7) \textbf{22} dô was] Nv was ouch Z \textbf{23} dem] des Z den Fr7  $\cdot$ trân] tram Fr7 \textbf{25} Wan er gerne dannoch lebte Z \textbf{28} anz] daz Z \textbf{29} Gringuljete] Gringvliet D (Z) \textbf{30} dô] da Z \newline
\end{minipage}
\hspace{0.5cm}
\begin{minipage}[t]{0.5\linewidth}
\small
\begin{center}*m
\end{center}
\begin{tabular}{rl}
 & "hie wil ich mîn reise sp\textit{ar}n.\\ 
 & got walt es, welt ir vürbaz varn,\\ 
 & sô en\textbf{dorft} irz niht le\textit{n}gen,\\ 
 & \textbf{ellenthaftez} sprengen\\ 
5 & m\textit{üez}et ir \textbf{zuo} ros alsus\\ 
 & ü\textit{b}er \textbf{Ligweis} Prellius."\\ 
 & \textbf{diu vrouwe bleip} ûf dem plân,\\ 
 & vürbaz reit hêr Gawan.\\ 
 & er \textbf{erhôrte} eines dræten wazzers val,\\ 
10 & daz het durchbroch\textit{en} wît ein tal,\\ 
 & tief, ungevert\textit{ec}lîch.\\ 
 & Gawan, der ellens rîch,\\ 
 & nam daz ros mit den sporn.\\ 
 & \textbf{ez treip} der degen wol geborn,\\ 
15 & daz ez mit zwein vüezen trat\\ 
 & hin über an daz ander stat.\\ 
 & der sprunc mit valle muoste sîn.\\ 
 & \textbf{des} weinde iedoch diu herzogîn.\\ 
 & der wâc was snel und grôz.\\ 
20 & Gawan sîner kraft genôz.\\ 
 & \textbf{dô} truoc er harnaschlast.\\ 
 & \dag daz\dag  was eines boumes ast\\ 
 & gewahsen in \textbf{des} wazzers trân.\\ 
 & den begreif der starke man,\\ 
25 & wan \textbf{er} \textbf{dann\textit{o}ch gerne} lebte.\\ 
 & sîn sper d\textit{â} bî \textbf{im} \textbf{swebte};\\ 
 & daz begreif der wîgant.\\ 
 & er \textbf{streich} hin \textbf{ûz} \textbf{an} daz lant.\\ 
 & Gringulet swam obe und unde,\\ 
30 & dem er helfen begunde.\\ 
\end{tabular}
\scriptsize
\line(1,0){75} \newline
m n o \newline
\line(1,0){75} \newline
\newline
\line(1,0){75} \newline
\textbf{1} sparn] spran m \textbf{2} ir] er o \textbf{3} lengen] legen m \textbf{5} müezet] Must m (n) o \textbf{9} dræten] trettens o \textbf{10} durchbrochen] durch broch m  $\cdot$ tal] teyl o \textbf{11} ungeverteclîch] vngefertlich m \textbf{14} ez] Er o \textbf{18} herzogîn] herczogen o \textbf{23} des wazzers trân] das wasser [tram]: tran o \textbf{25} dannoch] dannech m \textbf{26} dâ] do m n o \textbf{28} streich] steig n  $\cdot$ ûz] vff n (o) \textbf{30} begunde] do begunde n (o) \newline
\end{minipage}
\end{table}
\newpage
\begin{table}[ht]
\begin{minipage}[t]{0.5\linewidth}
\small
\begin{center}*G
\end{center}
\begin{tabular}{rl}
 & "hie wil ich mîn reise sparn.\\ 
 & got walt es, welt ir vürbaz varn,\\ 
 & sône \textbf{durfet} irz niht lengen,\\ 
 & \textbf{ellenthafte} sprengen\\ 
5 & muozet ir\textbf{z} or\textit{s}, \textit{a}lsus\\ 
 & über \textbf{Lishoys} Prillius."\\ 
 & \textbf{\begin{large}S\end{large}i habet al stille} ûf dem plân,\\ 
 & vürbaz reit hêr Gawan.\\ 
 & er \textbf{hôrt} eines dræten wazzers val,\\ 
10 & daz het durchbrochen wît ein tal,\\ 
 & tief \textbf{unde} ungeverti\textit{c}lîch.\\ 
 & Gawan, der ellens rîch,\\ 
 & nam daz ors mit den sporn.\\ 
 & \textbf{dô sprach} der degen wol geborn,\\ 
15 & daz ez \textit{mit} zwein vüezen trat\\ 
 & hin über an daz ander stat.\\ 
 & der sprunc mit \textit{v}alle muose sîn.\\ 
 & \textbf{des} weinde iedoch diu herzogîn.\\ 
 & der wâc was snel unde grôz.\\ 
20 & Gawan sîner krefte genôz,\\ 
 & \textbf{doch} truog er harnasches last.\\ 
 & \textbf{nû} was \textbf{ouch} eines boumes ast\\ 
 & gewahsen in \textbf{des} wazzers trân.\\ 
 & den begreif der stark\textit{e} man,\\ 
25 & wande \textbf{er} \textbf{gerne dannoch} lebete.\\ 
 & sîn sper dâ bî \textbf{geswebete};\\ 
 & daz begreif der wîgant.\\ 
 & er \textbf{steic} \textit{h}in \textbf{ûf} \textbf{an} daz lant.\\ 
 & Gringuliet swam ob unde unde,\\ 
30 & dem er helfen \textbf{dô} begunde.\\ 
\end{tabular}
\scriptsize
\line(1,0){75} \newline
G I L M Z Fr51 \newline
\line(1,0){75} \newline
\textbf{1} \textit{Initiale} L Z Fr51  \textbf{7} \textit{Initiale} G I  \textbf{25} \textit{Initiale} I  \newline
\line(1,0){75} \newline
\textbf{1} mîn] nu min I \textbf{3} durfet irz] durft dirz I dorf is Fr51 \textbf{4} ellenthafte] ellenthaftez Z \textbf{5} muozet] Mustet M Moz Fr51  $\cdot$ irz ors] irz ors toͮn G ir zuͯ rosze L (M) ir al da zv orsse Z ir zo orse don Fr51 \textbf{6} über] Als vber I  $\cdot$ Lishoys Prillius] liscoys prillius I Lygors Prilliuͯs L lygois prillius M Ligweiz prellius Z ligoes prillius Fr51 \textbf{7} habet] hatte M helt Fr51  $\cdot$ dem] ienem L eyme M den Fr51 \textbf{8} hêr Gawan] ergawan M \textbf{9} dræten] dyeszenden L dritten M \textbf{10} wît] \textit{om.} L M \textbf{11} unde] \textit{om.} L  $\cdot$ ungeverticlîch] vngeuertilich G (Fr51) vnfurtechlich I (M) \textbf{14} dô sprach] do spranc I Daz treib L (M) (Z) (Fr51)  $\cdot$ degen] helt Fr51 \textbf{15} ez] daz ors I  $\cdot$ mit] \textit{om.} G \textbf{17} des chom er von dem orse sin I  $\cdot$ valle] alle G \textbf{18} iedoch] \textit{om.} Fr51  $\cdot$ herzogîn] chungin I \textbf{19} wâc] vlut Fr51  $\cdot$ was snel] snel was Z \textbf{21} doch] Da Z \textbf{22} nû] do I \textbf{23} wazzers] wazer Fr51 \textbf{24} starke] starcher G \textbf{25} wande] Wnde I  $\cdot$ gerne dannoch] dannoch gerne M gerne Fr51 \textbf{26} bî] bi im I Z Fr51  $\cdot$ geswebete] [sp]: swebete I swebte L Z Fr51 \textbf{28} er] vnde I  $\cdot$ hin] in G \textit{om.} I  $\cdot$ ûf an daz] vz vffes L uff das M (Z) \textbf{29} Gringuliet] Chingruniel G Gr:::e Fr51 \textbf{30} helfen dô] helffin da M (Z) do helfen Fr51 \newline
\end{minipage}
\hspace{0.5cm}
\begin{minipage}[t]{0.5\linewidth}
\small
\begin{center}*T
\end{center}
\begin{tabular}{rl}
 & "hie wil ich mîne reise sparn.\\ 
 & got walt es, wolt ir vürbaz varn,\\ 
 & sô en\textbf{dorfet} ir ez niht lengen,\\ 
 & \textbf{ellenthafte} sprengen\\ 
5 & müezet ir \textbf{zuo} orse alsus\\ 
 & über \textbf{Ligweiz} Prillus."\\ 
 & \textbf{si hânt alle stille} ûf dem plân;\\ 
 & vürbaz reit hêr Gawan.\\ 
 & er \textbf{hôrte} eines dræten wazzers val,\\ 
10 & daz hete durchbrochen wît ein tal,\\ 
 & tief \textbf{und} ungeverteclîche.\\ 
 & Gawan, der ellens rîche,\\ 
 & nam daz ors mit den sporn.\\ 
 & \textbf{daz treip} der degen wol geborn,\\ 
15 & daz e\textit{z} mit zwein vüezen trat\\ 
 & hin über an daz ander stat.\\ 
 & der sprunc mit valle muose sîn,\\ 
 & \textbf{daz} weinte iedoch diu herzogîn.\\ 
 & der wâc was snel und grôz.\\ 
20 & Gawan sîner kraft genôz,\\ 
 & \textbf{doch} truoc er harnaschlast.\\ 
 & \textbf{nû} was \textbf{ouch} eines boumes ast\\ 
 & gewahsen in \textbf{des} wazzers trân.\\ 
 & den begreif der starke man,\\ 
25 & wan \textbf{der} \textbf{dannoch gerne} lebete.\\ 
 & sîn sper d\textit{â} bî \textbf{im} \textbf{swebete};\\ 
 & daz begreif der wîgant.\\ 
 & er \textbf{steic} hin \textbf{ûz} \textbf{ûf} daz lant.\\ 
 & Krynguliet swam oben und unde,\\ 
30 & dem er helfen \textbf{dô} begunde.\\ 
\end{tabular}
\scriptsize
\line(1,0){75} \newline
U V W Q R \newline
\line(1,0){75} \newline
\textbf{1} \textit{Initiale} W R  \textbf{7} \textit{Überschrift:} Hie bracht her gawan orgelusen ein krantz von dem kúnig gramoflantz boyne vnd versprach sich mit im zuͦ kempffen W   $\cdot$ \textit{Platz für Illustration ausgespart} W   $\cdot$ \textit{Initiale} W  \newline
\line(1,0){75} \newline
\textbf{2} got walt es] \textit{om.} R \textbf{3} endorfet] dorfft Q \textbf{4} ellenthafte] Erenthaffte Q  $\cdot$ sprengen] ersprengen V \textbf{6} Ligweiz Prillus] daz wasser [*]: perilus V ligweis prillius (prellius Q ) W (Q) R \textbf{7} [D*]: Die frowe bleip vffe dem plan V  $\cdot$ hânt] habt W R habte Q  $\cdot$ alle] all W (R) \textbf{9} eines] einen Q  $\cdot$ dræten] streten V starken R \textbf{10} durchbrochen] durch broch R \textbf{12} Gawan der] Gawin des R  $\cdot$ ellens] eren Q \textbf{14} wol] hoch R \textbf{15} ez] er U \textbf{16} Hinv́ber an [d*]: den [ander*]: andern stat V \textbf{17} muose] mvͤste V \textbf{19} wâc] wec Q \textbf{20} Gawan] Gawin R \textbf{21} doch] Do V  $\cdot$ harnaschlast] harnesches last V (W) (Q) \textbf{22} ouch] \textit{om.} R \textbf{24} begreif der starke] begriff der stake R \textbf{25} der] er V W (Q) (R)  $\cdot$ dannoch gerne] [*]: dennoch gerne V \textbf{26} dâ] do U V W \textit{om.} R \textbf{28} er] Da mit R  $\cdot$ hin ûz ûf] [hi*]: hin vz vf V er vff R \textbf{29} Krynguliet] Kringulet V Kringuliet W Q Kringulett R  $\cdot$ oben] \textit{om.} R  $\cdot$ und unde] vnder R \textbf{30} er helfen dô] helffe R  $\cdot$ begunde] begunden Q begunder R \newline
\end{minipage}
\end{table}
\end{document}
