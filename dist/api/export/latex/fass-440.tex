\documentclass[8pt,a4paper,notitlepage]{article}
\usepackage{fullpage}
\usepackage{ulem}
\usepackage{xltxtra}
\usepackage{datetime}
\renewcommand{\dateseparator}{.}
\dmyyyydate
\usepackage{fancyhdr}
\usepackage{ifthen}
\pagestyle{fancy}
\fancyhf{}
\renewcommand{\headrulewidth}{0pt}
\fancyfoot[L]{\ifthenelse{\value{page}=1}{\today, \currenttime{} Uhr}{}}
\begin{document}
\begin{table}[ht]
\begin{minipage}[t]{0.5\linewidth}
\small
\begin{center}*D
\end{center}
\begin{tabular}{rl}
\textbf{440} & \begin{large}M\end{large}îner \textbf{jæmerlîchen zîte} jâr\\ 
 & wil ich im minne geben vür wâr.\\ 
 & der rehten minne ich bin sîn wer,\\ 
 & wander mit schilde unt \textbf{ouch} mit sper\\ 
5 & dâ nâch mit ritters \textbf{handen} warp,\\ 
 & unz er in mîme dienste erstarp.\\ 
 & magetuom ich ledeclîche hân.\\ 
 & er ist \textbf{iedoch vor gote} mîn man.\\ 
 & ob gedanke \textbf{würken sulen} diu werc,\\ 
10 & sô trag ich \textbf{niender} den geberc,\\ 
 & der underswinge mir mîn ê.\\ 
 & mîme leben tet sîn sterben wê.\\ 
 & der rehten ê diz vingerlîn\\ 
 & vür got sol mîn geleite sîn.\\ 
15 & daz ist ob mîner triwe \textbf{ein} slôz,\\ 
 & vonme herzen mîner ougen vlôz.\\ 
 & ich bin hinne selbander:\\ 
 & Schianatulander\\ 
 & ist daz eine, daz ander ich."\\ 
20 & Parzival verstuont dô sich,\\ 
 & daz ez Sigune wære.\\ 
 & ir kumber was im swære.\\ 
 & Den helt dô wênec des verdrôz.\\ 
 & vonme hersniere daz houbet blôz\\ 
25 & er machete, ê daz er gein ir sprach.\\ 
 & diu juncvrouwe an im ersach\\ 
 & durch îsers râm vil liehtez vel.\\ 
 & dô erkande si den degen snel.\\ 
 & si sprach: "ir sîtz, hêr Parzival!\\ 
30 & sagt \textbf{an}, wie stêtz \textbf{iu} umben Grâl?\\ 
\end{tabular}
\scriptsize
\line(1,0){75} \newline
D Fr31 \newline
\line(1,0){75} \newline
\textbf{1} \textit{Initiale} D  \textbf{23} \textit{Majuskel} D  \newline
\line(1,0){75} \newline
\textbf{9} diu] \textit{om.} Fr31 \textbf{10} geberc] berch Fr31 \textbf{12} sîn] din Fr31 \textbf{13} ê] e sol Fr31 \textbf{14} sol] \textit{om.} Fr31 \textbf{18} Schianatulander] Scianatvlander D Fr31 \textbf{20} Parzival] Parcifal D Parzifal Fr31 \textbf{21} Sigune] Sigvͦne D sigvne Fr31 \textbf{29} sîtz] sit diz Fr31  $\cdot$ Parzival] Parcifal D parzifal Fr31 \textbf{30} iu] \textit{om.} Fr31 \newline
\end{minipage}
\hspace{0.5cm}
\begin{minipage}[t]{0.5\linewidth}
\small
\begin{center}*m
\end{center}
\begin{tabular}{rl}
 & mîner \textbf{jâmerzîte} jâr\\ 
 & wil ich im minne geben vür wâr.\\ 
 & der rehten minne ich bin sîn wer,\\ 
 & wand er mit schilte und \textbf{ouch} mit sper\\ 
5 & dâ nâch mit ritters \textbf{handen} warp,\\ 
 & unz er in mînem dienste erstarp.\\ 
 & magetuom ich ledeclîch hân.\\ 
 & er ist \textbf{vor gote iedoch} mîn man.\\ 
 & ob gedanke \textbf{wirken sollen} diu werc,\\ 
10 & sô trage ich \textbf{niender} den geberc,\\ 
 & der underswinge mir mîn ê.\\ 
 & mînem lebene tet sîn sterben wê.\\ 
 & der rehten ê diz vingerlîn\\ 
 & vür got sol mîn geleite sîn.\\ 
15 & daz ist o\textit{b} mîner triuwe slôz,\\ 
 & von dem herzen mîner ougen vlôz.\\ 
 & ich bin hinne selber ander:\\ 
 & Schianatulander\\ 
 & i\textit{st} daz eine, daz ander ich."\\ 
20 & Parcifal verstuont dô sich,\\ 
 & daz e\textit{z} Sigune wære.\\ 
 & ir kumber was ime swære.\\ 
 & den helt dô wênic des verdrôz.\\ 
 & von\textit{m}e hersnier daz houbet blôz\\ 
25 & er machte, ê daz er gegen ir sprach.\\ 
 & diu juncvrouwe an ime ersach\\ 
 & durch î\textit{s}e\textit{r}s râm vil liehtez vel.\\ 
 & dô erkante si den degen snel.\\ 
 & si sprach: "ir sît ez, hêr Parcifal!\\ 
30 & sagt \textbf{an}, wie stât ez umben Grâl?\\ 
\end{tabular}
\scriptsize
\line(1,0){75} \newline
m n o \newline
\line(1,0){75} \newline
\newline
\line(1,0){75} \newline
\textbf{1} Mit jomerlicher zite nar n (o)  $\cdot$ jâr] [war]: iar m \textbf{2} minne geben] geben mynne o \textbf{4} ouch] \textit{om.} n o \textbf{6} in] an n o  $\cdot$ mînem] mẏnen o \textbf{9} gedanke wirken sollen] gedang wircken sol n o \textbf{13} diz] das o \textbf{15} ob] od m  $\cdot$ slôz] ein slosz n (o) \textbf{16} \textit{Versdoppelung} m  \textbf{17} ich bin] Bin ich n o  $\cdot$ selber ander] falbander o \textbf{18} Schianatulander] Scianatulander m Schionatulander n Schinate lander o \textbf{19} ist] Jch m [Jch]: Jst n  $\cdot$ ich] sich o \textbf{21} ez] er m o  $\cdot$ Sigune] sigun o \textbf{22} swære] [were]: werlich swere n \textbf{23} des] das o \textbf{24} vonme] Vonne m  $\cdot$ hersnier] hersnenúr n hersener o  $\cdot$ blôz] slosz n o \textbf{25} Er [mach ma]: macht e [geben]: gegen ir sprach o  $\cdot$ machte] macht n \textbf{27} îsers] yres m \textbf{29} ir sît ez] sint irs n sint o \newline
\end{minipage}
\end{table}
\newpage
\begin{table}[ht]
\begin{minipage}[t]{0.5\linewidth}
\small
\begin{center}*G
\end{center}
\begin{tabular}{rl}
 & \begin{large}M\end{large}îner \textbf{jæmerlîchen zîte} \textit{j}âr\\ 
 & wil ich im minne geben vür wâr.\\ 
 & der rehten minne ich bin sîn wer,\\ 
 & wan er mit schilte unde mit sper\\ 
5 & dâ nâch mit rîters \textbf{hande} \textit{w}arp,\\ 
 & unze er in mîne\textit{m} dienste erstarp.\\ 
 & magetuom ich lediclîchen hân.\\ 
 & er ist \textbf{iedoch vor got} mîn man.\\ 
 & obe gedanc \textbf{\textit{würken} suln} diu werc,\\ 
10 & sô\textbf{ne} trag ich \textbf{niemer} den \textit{ge}berc,\\ 
 & der underswinge mir mîn ê.\\ 
 & mînem leb\textit{en te}t sîn sterb\textit{en} wê.\\ 
 & der rehten ê ditze vingerlîn\\ 
 & vor got sol \textit{mîn} ge\textit{leite sîn}.\\ 
15 & daz ist ob mîner triuwe \textbf{ein} s\textit{lôz},\\ 
 & von dem herzen \textit{mîn}er ougen vlôz.\\ 
 & ich bin hinne selbe ander:\\ 
 & Tschianatulander\\ 
 & ist daz ein, daz ander ich."\\ 
20 & Parzival verstuont dô sich,\\ 
 & daz ez Sigune wære.\\ 
 & ir kumber was im swære.\\ 
 & den helt dô wênic des verdrôz.\\ 
 & vom dem härsenier daz houb\textit{et} blôz\\ 
25 & er machet, ê daz er gein ir sprach.\\ 
 & diu juncvrouwe an im ersach\\ 
 & durch îsers râm vil liehtez vel.\\ 
 & dô erkande si den degen snel.\\ 
 & si sprach: "ir sît ez, hêr Parzival!\\ 
30 & saget \textbf{an}, wie stêt ez \textit{umb den} \textit{Grâl}?\\ 
\end{tabular}
\scriptsize
\line(1,0){75} \newline
G I O L M Z Fr25 \newline
\line(1,0){75} \newline
\textbf{1} \textit{Initiale} G L Z   $\cdot$ \textit{Majuskel} M  \textbf{15} \textit{Initiale} I  \textbf{23} \textit{Initiale} M  \textbf{29} \textit{Initiale} I  \newline
\line(1,0){75} \newline
\textbf{1} jâr] :ar G [zewar]: iar I \textbf{2} im minne] minn im I im O Fr25 ome Ninne M \textbf{3} ich] \textit{om.} M \textbf{4} mit sper] ovch sper O ouch mit sper Z (Fr25) \textbf{5} rîters hand] ritters handen O (L) (M) Z (Fr25)  $\cdot$ warp] erwarp G \textbf{6} unze] Daz O (Fr25) Vnde M  $\cdot$ mînem] minen G minnen I  $\cdot$ erstarp] starp O L M Fr25 \textbf{8} iedoch] doch L ledeclichen M \textbf{9} gedanc] gedanken M  $\cdot$ würken suln] suln G svln werben L suln wnschen M \textbf{10} sône trag] So trage O (Fr25) So han Z  $\cdot$ niemer] ninder I (O) (L) (Z) (Fr25) nirgen M  $\cdot$ geberc] berch G \textbf{11} underswinge] [vnders*]: vnderswinge L  $\cdot$ mir] \textit{om.} L \textbf{12} leben tet] leb:::t G leben dvt L  $\cdot$ sterben] sterb:: G \textbf{14} mîn geleite sîn] ::: ge::: G \textbf{15} ob] uff M  $\cdot$ slôz] s::: G \textbf{16} mîner] immer G  $\cdot$ vlôz] slosz M \textbf{18} Tschianatulander] Tshianatvlander G der getriwe shinadulander I Tschinatvlander O Schinatv de lander L Das eine ist schinatulander M Schionatulander Z Schinatẏlander Fr25 \textbf{19} Das ander das bin jch M  $\cdot$ daz] [ich]: daz ander ich Fr25 \textbf{20} Parzival] Parziual G Parzifal I L M Barcifal O Fr25 Parcifal Z \textbf{21} Sigune] sygvne O Sýgvne L siguͯne M \textbf{23} dô] \textit{om.} M da Z  $\cdot$ verdrôz] erdroz L \textbf{24} härsenier] harnasch M  $\cdot$ houbet] huͦp G ouge L \textbf{25} machet] machte I O L M (Z) Fr25  $\cdot$ ê] \textit{om.} I M \textbf{26} an] do an I  $\cdot$ ersach] sach L \textbf{27} îsers] isen I (O) (Fr25) ýsenz L  $\cdot$ râm] raum I \textbf{28} dô] Da M Z  $\cdot$ erkande] [erchende]: erchande G \textbf{29} ir sît ez] sit irz I ir siczet M (Fr25)  $\cdot$ Parzival] parzifal I L M Barcifal O (Fr25) parcifal Z \textbf{30} stêt ez] stet M stetz ev Z  $\cdot$ umb den Grâl] ::: G \newline
\end{minipage}
\hspace{0.5cm}
\begin{minipage}[t]{0.5\linewidth}
\small
\begin{center}*T
\end{center}
\begin{tabular}{rl}
 & mîner \textbf{magetuomlîcher zîte} jâr\\ 
 & wil ich im minne geben vür wâr.\\ 
 & der rehten minne ich bin sîn wer,\\ 
 & wand er mit schilte unde mit sper\\ 
5 & dâr nâch mit rîters \textbf{handen} warp,\\ 
 & unzer in mînem dienste erstarp.\\ 
 & magetuom ich ledeclîche hân.\\ 
 & er ist \textbf{vor gote iedoch} mîn man.\\ 
 & ob gedenke \textbf{suln wirken} di\textit{u} werc,\\ 
10 & sô \textbf{en}tragich \textbf{nie\textit{n}der} den geberc,\\ 
 & der underswinge mir mîn ê.\\ 
 & mînem lebenne tet \textbf{ouch} sîn sterben wê.\\ 
 & der rehten ê diz vingerlîn\\ 
 & vor gote sol mîn geleite sîn.\\ 
15 & daz ist ob mîner triuwe \textbf{ein} slôz,\\ 
 & vonme herzen mîner ougen vlôz.\\ 
 & ich bin hinne selbe ander:\\ 
 & Schinohtudelander\\ 
 & ist daz eine, daz ander ich."\\ 
20 & Parcifal verstuont dô sich,\\ 
 & daz ez Sygune wære.\\ 
 & ir kumber was im swære.\\ 
 & den helt dô wênic des verdrôz.\\ 
 & vonme hersenier daz houbet blôz\\ 
25 & er machte, ê daz er gegen ir sprach.\\ 
 & Diu juncvrouwe an im ersach\\ 
 & durch îsers râm vil liehtez vel.\\ 
 & dô erkande si den degen snel.\\ 
 & si sprach: "ir sîtz, hêr Parcifal!\\ 
30 & saget \textbf{mir}, wie stêt ez umbe den Grâl?\\ 
\end{tabular}
\scriptsize
\line(1,0){75} \newline
T U V W Q R \newline
\line(1,0){75} \newline
\textbf{1} \textit{Initiale} W  \textbf{20} \textit{Initiale} R   $\cdot$ \textit{Majuskel} T  \textbf{26} \textit{Majuskel} T  \newline
\line(1,0){75} \newline
\textbf{1} [M*]: Miner iemerlichen zite iar V  $\cdot$ magetuomlîcher] iemerlichen W (Q) iemlichen R \textbf{3} rehten] rechter U  $\cdot$ sîn] ein U \textbf{4} er] der U \textbf{5} warp] erwarb R \textbf{6} unzer] Mit er U Vntz bisz er Q  $\cdot$ dienste] dienstes R  $\cdot$ erstarp] starp U (Q) (R) \textbf{8} vor gote iedoch] ie doch vor gote U (V) (W) (Q) (R) \textbf{9} gedenke] gedancken W (Q) (R)  $\cdot$ suln wirken] [wrcken]: wircken sullen Q mugen wurken R  $\cdot$ diu] die T \textbf{10} entragich] trag ich R  $\cdot$ niender] nieder T (U) nyedert R  $\cdot$ den geberc] den berc U das geberk R \textbf{11} underswinge] vnderuinge W vnter swingen Q \textbf{12} lebenne] [lebennen]: lebenne T lieben R  $\cdot$ ouch] \textit{om.} W Q R \textbf{13} der] Die R  $\cdot$ diz] das R \textbf{14} sol] \textit{om.} W so R \textbf{16} Vom meinem hertzen ein flosz Q  $\cdot$ vonme] [*]: Vomme V \textbf{17} selbe ander] selben ander Q \textbf{18} Schinohtudelander] Schinohtvlander T Schinotulander U [Schin*]: Schinatulander V Tschionatulander W Scionotulander Q Schionatulander R \textbf{20} Parcifal] Parzifal V Partzifal W Q Parczifal R \textbf{21} Sygune] syguͦne U sigűne Q Sygunne R \textbf{22} im] in R \textbf{23} dô] \textit{om.} U \textbf{24} vonme] Von W  $\cdot$ hersenier] herschenier T hersnirer Q harnasch macher R \textbf{25} machte ê daz] [maht*]: mahte e daz V macht e das Q \textbf{26} ersach] sach W \textbf{27} îsers] [*]: yserz V helmel R  $\cdot$ liehtez] lichte Q liettes R \textbf{28} den] den den T \textbf{29} sîtz] sint R  $\cdot$ Parcifal] parzifal V partzifal W Q parczifal R \textbf{30} mir] \textit{om.} U an V W Q R \newline
\end{minipage}
\end{table}
\end{document}
