\documentclass[8pt,a4paper,notitlepage]{article}
\usepackage{fullpage}
\usepackage{ulem}
\usepackage{xltxtra}
\usepackage{datetime}
\renewcommand{\dateseparator}{.}
\dmyyyydate
\usepackage{fancyhdr}
\usepackage{ifthen}
\pagestyle{fancy}
\fancyhf{}
\renewcommand{\headrulewidth}{0pt}
\fancyfoot[L]{\ifthenelse{\value{page}=1}{\today, \currenttime{} Uhr}{}}
\begin{document}
\begin{table}[ht]
\begin{minipage}[t]{0.5\linewidth}
\small
\begin{center}*D
\end{center}
\begin{tabular}{rl}
\textbf{247} & \begin{large}Ê\end{large} Parzival, der wîgant,\\ 
 & sich des orses underwant,\\ 
 & manegez er der gadem erlief,\\ 
 & sô daz er nâch den liuten rief.\\ 
5 & \textbf{niemen er hôrte} noch ensach,\\ 
 & \textbf{ungevüege} \textbf{leit} im \textbf{dran} geschach.\\ 
 & \textbf{daz} het im zorn gereizet.\\ 
 & er lief, dâ er was erbeizet\\ 
 & des âbents, dô er komen was.\\ 
10 & dâ was erde und gras\\ 
 & mit treten gerüeret\\ 
 & un\textbf{z} tou \textbf{gar} zervüeret.\\ 
 & Al schrîende lief der junge man\\ 
 & wider ze \textbf{sîme} orse sân.\\ 
15 & mit bâgenden worten\\ 
 & saz er \textbf{drûf}. die porten\\ 
 & vander wît offen stên,\\ 
 & dar durch \textbf{ûz} grôze slâ gên.\\ 
 & Niht langer er dô habete,\\ 
20 & vaste ûf die brücke er drabete.\\ 
 & ein verborgen knappe daz seil\\ 
 & \textbf{zôch}, \textbf{daz} der slagebrücken teil\\ 
 & hetz ors \textbf{vil} nâch gevellet nider.\\ 
 & Parzival, der sach \textbf{sich} wider.\\ 
25 & \textbf{dô} wolt \textbf{er} \textbf{hân gevrâget} baz.\\ 
 & "Ir sult varen, der sunnen haz",\\ 
 & sprach der knappe, "ir sît ein gans.\\ 
 & m\textit{ö}ht ir gerüeret hân den vlans\\ 
 & unt \textbf{het} den wirt gevrâget!\\ 
30 & vil prîses \textbf{iuch hât} betrâget."\\ 
\end{tabular}
\scriptsize
\line(1,0){75} \newline
D \newline
\line(1,0){75} \newline
\textbf{1} \textit{Initiale} D  \textbf{13} \textit{Majuskel} D  \textbf{19} \textit{Majuskel} D  \textbf{26} \textit{Majuskel} D  \newline
\line(1,0){75} \newline
\textbf{28} möht] moht D \newline
\end{minipage}
\hspace{0.5cm}
\begin{minipage}[t]{0.5\linewidth}
\small
\begin{center}*m
\end{center}
\begin{tabular}{rl}
 & \begin{large}Ê\end{large} Parcifal, der wîgant,\\ 
 & sich des rosses underwant,\\ 
 & manigez er \dag dem\dag  gadem erlief,\\ 
 & sô daz er nâc\textit{h} \textit{d}en liuten rief.\\ 
5 & \textbf{niemen er hôrte} noch ensach,\\ 
 & \textbf{ungevüege} \textbf{leit} im \textbf{dran} geschach.\\ 
 & \textbf{daz} hete ime zorn gereizet.\\ 
 & er lief, d\textit{â} er was erbeizet\\ 
 & des âbendes, dô er komen was.\\ 
10 & dâ was erde und gras\\ 
 & mit treten gerüeret\\ 
 & und \textbf{daz} tou \textbf{gar} zervüeret.\\ 
 & al schrîende lief der junge man\\ 
 & wider zuo \textbf{sînem} rosse sân.\\ 
15 & mit bâgenden worten\\ 
 & saz er \textbf{drûf}. die porten\\ 
 & vant er wîte offen stân,\\ 
 & dar durch \textbf{ûz} grôze slâ gân.\\ 
 & niht langer er dô habete,\\ 
20 & vaste ûf die brücke er drabete.\\ 
 & ein verborgen knappe daz seil\\ 
 & \textbf{zôch}, \textbf{daz} der slagebrücke\textit{n} teil\\ 
 & hete daz ros \textbf{vil} nâhe gevellet nider.\\ 
 & Parcifal, der sach wider.\\ 
25 & \textbf{dô} wolt \textbf{er} \textbf{gevrâget hân} baz.\\ 
 & "ir sult varn, der sunnen haz",\\ 
 & sprach der knappe, "ir sît ein gans.\\ 
 & m\textit{ö}ht ir gerüeret haben den vlans,\\ 
 & und \textbf{hete er} den wirt gevrâget!\\ 
30 & vil prîses \textbf{hât iuch} betrâget."\\ 
\end{tabular}
\scriptsize
\line(1,0){75} \newline
m n o Fr69 \newline
\line(1,0){75} \newline
\textbf{1} \textit{Initiale} m n o  \newline
\line(1,0){75} \newline
\textbf{2} rosses] landes n \textbf{3} Maniges er in dem gadem erlieff o  $\cdot$ Mangen gaden er erlief Fr69 \textbf{4} nâch den] nach von den m noch von den n o \textbf{5} er hôrte] erhorte m n enhorte o \textbf{7} ime] \textit{om.} Fr69  $\cdot$ gereizet] gezeizet Fr69 \textbf{8} dâ] do m n o  $\cdot$ was] [war]: was m \textbf{9} komen] [gerangen]: komen m komende n \textbf{10} dâ] Do n o \textbf{11} treten] trettent n \textbf{12} daz] \textit{om.} o  $\cdot$ zervüeret] zurstoret o \textbf{13} lief] lieff lieff o \textbf{16} Die burg in allen orten Fr69  $\cdot$ saz er] Das er das n \textbf{17} \textit{Verse 247.16-17 kontrahiert zu:} Das er wite offen stan (Augenabirrung / gemeinsame Vorlage mit Hs. n?) o   $\cdot$ wîte offen] mit offen porten Fr69 \textbf{18} ûz grôze] mit grossen n uͯsz gossen o  $\cdot$ slâ gân] slagen m slan n slahen o \textbf{21} ein] Er n  $\cdot$ daz seil] [*ch]: daz seil Fr69 \textbf{22} slagebrücken] slage brúgel m slagebrugge ein Fr69 \textbf{23} vil] \textit{om.} Fr69 \textbf{25} gevrâget hân] han gefroget n o \textbf{26} ir] Der o  $\cdot$ varn] han n \textbf{28} möht] Moht m (n) o  $\cdot$ ir] er o  $\cdot$ den] den den n \textbf{29} hete er] hette es n het es o \textbf{30} iuch betrâget] auch [betaget]: betraget o \newline
\end{minipage}
\end{table}
\newpage
\begin{table}[ht]
\begin{minipage}[t]{0.5\linewidth}
\small
\begin{center}*G
\end{center}
\begin{tabular}{rl}
 & ê Parzival, der wîgant,\\ 
 & sich des orses underwant,\\ 
 & manigez er der gademe erlief,\\ 
 & sô daz er nâch den liuten rief.\\ 
5 & \textbf{niemen er hôrte} noch ensach,\\ 
 & \textbf{ungevüege} \textbf{leit} im \textbf{dran} geschach.\\ 
 & \textbf{diz} hete im zorn gereizet.\\ 
 & er lief, dâ er was erbeizet\\ 
 & des âbendes, dô er komen was.\\ 
10 & dâ was erde unde gras\\ 
 & mit tretene gerüeret\\ 
 & unt \textbf{daz} tou \textbf{gar} zervüeret.\\ 
 & al schrîende lief der junge man\\ 
 & wider ze \textbf{sînem} orse sân.\\ 
15 & mit bâgenden worten\\ 
 & saz er \textbf{ûf}. die porten\\ 
 & vant er wîte offen stên,\\ 
 & dâ durch \textbf{ûz} grôze slâ gên.\\ 
 & niht langer er dâ habte,\\ 
20 & vaste ûf die \textit{brück}e er drabte.\\ 
 & ein verborgen knappe daz seil\\ 
 & \textbf{zuckte}, \textbf{daz} der slagebrücke \textbf{ein} teil\\ 
 & het daz ors \textbf{vil} nâch gevellet nider.\\ 
 & Parzival, der sach \textbf{sich} wider.\\ 
25 & \textbf{dô} wolte\textbf{r} \textbf{hân gevrâget} baz.\\ 
 & "ir sult varen, der sunnen haz",\\ 
 & sprach der knappe, "ir sît ein gans.\\ 
 & m\textit{ö}ht ir gerüeret hân den vlans\\ 
 & unde \textbf{het} den wirt gevrâget!\\ 
30 & vil brîses \textbf{iuch hât} betrâget."\\ 
\end{tabular}
\scriptsize
\line(1,0){75} \newline
G I O L M Q R Z Fr36 \newline
\line(1,0){75} \newline
\textbf{1} \textit{Initiale} Q  \textbf{3} \textit{Initiale} I  \textbf{7} \textit{Initiale} Z  \textbf{13} \textit{Initiale} O  \textbf{25} \textit{Initiale} I  \newline
\line(1,0){75} \newline
\textbf{1} ê] >e< G \textit{om.} L M Q  $\cdot$ Parzival] Parzifal I L M Parcifal O (Z) Partzifal Q parczifal R \textbf{2} des] sins R \textbf{3} er] es R \textbf{5} er] \textit{om.} M  $\cdot$ hôrte] enhorte L (M) hortten R  $\cdot$ noch ensach] nach ein sach Q \textbf{6} ungevüege] Vngefvͦgez O \textbf{7} gereizet] geczieret R \textbf{8} dâ] daz O do Q  $\cdot$ was erbeizet] wart gebeysset Q was grierret R \textbf{9} dô] da M Z \textbf{10} dâ] Do Q R  $\cdot$ erde] rede Q \textbf{12} unt] \textit{om.} I \textbf{13} al] ÷l O  $\cdot$ lief] rieff R lief er Z  $\cdot$ junge] kvͤne Fr36 \textbf{14} Zuͯ sinem rosze wider dan L  $\cdot$ sân] er kam R \textbf{16} ûf] druf I (O) (L) (Q) (R) (Z) (Fr36)  $\cdot$ porten] phortin M (Q) \textbf{17} vant] Wande Z \textbf{18} \textit{nach 247.18:} Parcifal der huͯp sich nach / Hin vsz zuͯ der porte waz im gach L   $\cdot$ Grosze schla da duͯrch uͯsz gan L  $\cdot$ ûz] \textit{om.} I ovch O (Fr36)  $\cdot$ slâ gên] slagin M (Q) (Fr36) \textbf{19} langer] lange I  $\cdot$ er dâ] er do I L ern O (M) er sich Q er R Z \textit{om.} Fr36  $\cdot$ habte] enthabte Q (Z) (Fr36) \textbf{20} vaste] Vaste er L (Q) (Fr36)  $\cdot$ ûf die] gein der I  $\cdot$ brücke] [v*]: porte G port I bruckin M (Q) burc Z  $\cdot$ er] \textit{om.} L Q Fr36 \textbf{21} verborgen] verborgener I verborger R \textbf{22} zuckte] Zuktt R \textbf{23} vil] \textit{om.} I L \textbf{24} Parzival] parzifal I (M) Parcifal O L Z Partzifal Q Parczifal R  $\cdot$ sich] hin I O L R \textit{om.} M er Q \textbf{25} dô] Das M Da Z \textbf{26} \textit{Vers 247.26 fehlt} R  \textbf{28} möht] Moht G O (L) (M) (Q) Z \textbf{29} unde] \textit{om.} I \textbf{30} iuch hât] hat úch R \newline
\end{minipage}
\hspace{0.5cm}
\begin{minipage}[t]{0.5\linewidth}
\small
\begin{center}*T
\end{center}
\begin{tabular}{rl}
 & \textit{Ê} Parcifal, der wîgant,\\ 
 & sich des orses underwant,\\ 
 & \textbf{vil} manegez er der gadem erlief,\\ 
 & sô daz er nâch den liuten rief.\\ 
5 & \textbf{ern hôrte niemen} nochn sach,\\ 
 & \textbf{unvuoge} \textbf{leide} im geschach.\\ 
 & \textbf{daz} hetim zorn gereizet.\\ 
 & er lief, dâ er was erbeizet\\ 
 & des âbendes, dô er komen was.\\ 
10 & dâ was erde unde gras\\ 
 & mit treten gerüeret\\ 
 & unde \textbf{der} tou zervüeret.\\ 
 & alschrîende lief der junge man\\ 
 & wider zuo \textbf{dem} orse sân.\\ 
15 & mit bâgenden worten\\ 
 & saz er \textbf{drûf}. die porten\\ 
 & vant er wît offen stân,\\ 
 & dar durch grôze slâ gân.\\ 
 & \multicolumn{1}{l}{ - - - }\\ 
20 & \multicolumn{1}{l}{ - - - }\\ 
 & Ein verborgen knappe daz seil\\ 
 & \textbf{zôch}; der slagebrücken teil\\ 
 & hete daz ors nâch gevellet nider.\\ 
 & Parcifal, der sach wider\\ 
25 & \textbf{unde} wolte \textbf{hân gevrâget} baz.\\ 
 & "Ir sult varn, der sunnen haz",\\ 
 & sprach der knappe, "ir sît ein gans.\\ 
 & m\textit{ö}htir gerüeret hân den vlans\\ 
 & unde \textbf{hetet} den wirt gevrâget!\\ 
30 & vil prîses \textbf{iuch hât} betrâget."\\ 
\end{tabular}
\scriptsize
\line(1,0){75} \newline
T U V W \newline
\line(1,0){75} \newline
\textbf{1} \textit{Majuskel} T  \textbf{21} \textit{Initiale} U V W   $\cdot$ \textit{Majuskel} T  \textbf{24} \textit{Majuskel} T  \textbf{26} \textit{Majuskel} T  \newline
\line(1,0){75} \newline
\textbf{1} Ê Parcifal] Parzifal T [:arzifal]: E parzifal V Ee partzifal W \textbf{3} manegez er der] manig gaden er W \textbf{4} nâch] \textit{om.} V \textbf{5} ern hôrte niemen] Niemant hort er W \textbf{6} unvuoge] Vngevuͦge U [V*]: Vngefuͤge V  $\cdot$ leide] leit V leider W  $\cdot$ geschach] [*]: dran geschach V \textbf{8} dâ] [*]: do V do W \textbf{10} dâ] Do U W [*]: Do V \textbf{11} treten] [to*]: treten T \textbf{12} Vnd daz dor zuͦ vuͦret U  $\cdot$ der] das W  $\cdot$ zervüeret] [*]: gar zerfuͤret V \textbf{13} alschrîende] Als schriende U \textbf{14} dem] [*]: sinem V seinen W \textbf{15} bâgenden] klagenden W \textbf{18} dar] Do W  $\cdot$ durch] [*]: durch uz V  $\cdot$ slâ] sia U \textbf{19} \textit{Die Verse 247.19-20 sind am Rand nachgetragen und später radiert:} Niht langer*do* / v*e vf die* V   $\cdot$ [*]: Niht langer er habte V \textbf{20} [*]: Vaste vf die brucke er trabte V \textbf{21} Ein] Din U \textbf{22} slagebrücken] vallebrucken W \textbf{23} hete] Vnd hette W  $\cdot$ nâch] noch V \textbf{24} Parcifal] Parzifal T V Partzifal W \textbf{28} möhtir] mohtir T (U) \textbf{30} iuch hât] îv hat T hat eúch W \newline
\end{minipage}
\end{table}
\end{document}
