\documentclass[8pt,a4paper,notitlepage]{article}
\usepackage{fullpage}
\usepackage{ulem}
\usepackage{xltxtra}
\usepackage{datetime}
\renewcommand{\dateseparator}{.}
\dmyyyydate
\usepackage{fancyhdr}
\usepackage{ifthen}
\pagestyle{fancy}
\fancyhf{}
\renewcommand{\headrulewidth}{0pt}
\fancyfoot[L]{\ifthenelse{\value{page}=1}{\today, \currenttime{} Uhr}{}}
\begin{document}
\begin{table}[ht]
\begin{minipage}[t]{0.5\linewidth}
\small
\begin{center}*D
\end{center}
\begin{tabular}{rl}
\textbf{513} & \begin{large}D\end{large}ô was mîn hêr Gawan\\ 
 & sô gezimiert ein man,\\ 
 & daz ez si lêrte riwe,\\ 
 & wande si heten triwe,\\ 
5 & die des boumgarten pflâgen.\\ 
 & si stuonden oder lâgen\\ 
 & oder s\textit{â}zen in gezelten,\\ 
 & \textbf{die} vergâzen des vil selten,\\ 
 & si\textbf{ne} klageten sînen kumber grôz.\\ 
10 & \textbf{man unt wîp} des niht verdrôz,\\ 
 & Genuoge sprâchen, den \textbf{es} was leit:\\ 
 & "mîner vrouwen trügeheit\\ 
 & wil disen man verleiten\\ 
 & ze grôzen arbeiten.\\ 
15 & ouwê, daz er ir volgen wil\\ 
 & ûf alsus riwebæriu zil!"\\ 
 & Manec wert man \textbf{dâ} gein im gienc,\\ 
 & der in mit armen umbevienc\\ 
 & durch vriwentlîch enpfâhen.\\ 
20 & dar nâch begunder nâhen\\ 
 & einem ölboume; dâ stuont daz pfert.\\ 
 & \textbf{ouch} was maneger marke wert\\ 
 & der zoum unt \textbf{sîn} gereite.\\ 
 & mit einem barte breite,\\ 
25 & wol gevlohten unde grâ,\\ 
 & stuont dâr bî ein rîter dâ\\ 
 & über eine \textbf{krücken} geleinet.\\ 
 & von dem wart \textbf{ez} \textbf{beweinet},\\ 
 & daz Gawan zuo dem pferde gienc.\\ 
30 & mit süezer rede ern doch enpfienc.\\ 
\end{tabular}
\scriptsize
\line(1,0){75} \newline
D \newline
\line(1,0){75} \newline
\textbf{1} \textit{Initiale} D  \textbf{11} \textit{Majuskel} D  \textbf{17} \textit{Majuskel} D  \newline
\line(1,0){75} \newline
\textbf{7} sâzen] sæzen D \newline
\end{minipage}
\hspace{0.5cm}
\begin{minipage}[t]{0.5\linewidth}
\small
\begin{center}*m
\end{center}
\begin{tabular}{rl}
 & dô was mîn hêr Gawan\\ 
 & sô gezimiert ein man,\\ 
 & daz ez si lêrte riuwe,\\ 
 & want si heten triuwe,\\ 
5 & die des boumgarten pflâgen.\\ 
 & si stuonden oder lâgen\\ 
 & oder sâzen in gezelten,\\ 
 & \textbf{s\textit{i}} vergâzen des vil selten,\\ 
 & si klagten sînen kumber grôz.\\ 
10 & \textbf{man und wîp} des niht verdrôz,\\ 
 & genuoge sprâchen, den \textbf{es} was leit:\\ 
 & "mîner vrouwen trügeheit\\ 
 & wil disen man verleiten\\ 
 & zuo grôzen arbeiten.\\ 
15 & ouwê, daz er ir vo\textit{l}ge\textit{n} wil\\ 
 & ûf alsus \dag reinberie\dag  zil!"\\ 
 & manic wert man gegen im gienc,\\ 
 & der in mit armen umbevienc\\ 
 & durch vriuntlîch enpfâhen.\\ 
20 & dar nâch begunde er nâhen\\ 
 & eine\textit{m} oleiboum; d\textit{â} stuont daz pfert.\\ 
 & \textbf{ez} was maniger marc wert\\ 
 & der z\textit{o}um und \textbf{daz} gereite.\\ 
 & mit einem \textit{b}arte breite,\\ 
25 & wol gevlohten und grâ,\\ 
 & stuont dâ bî ein ritter \textit{d}â\\ 
 & über ein \textbf{brücken} geleinet.\\ 
 & von dem wart \textbf{ez} \textbf{geweinet},\\ 
 & daz Gawan zuo dem pferde gienc.\\ 
30 & mit süezer rede ern doch enpfienc.\\ 
\end{tabular}
\scriptsize
\line(1,0){75} \newline
m n o \newline
\line(1,0){75} \newline
\newline
\line(1,0){75} \newline
\textbf{2} gezimiert] gezimiet o \textbf{5} die] Noch die n  $\cdot$ des] das o \textbf{8} si] So m  $\cdot$ des] das o  $\cdot$ vil] [wil]: vil o \textbf{9} sînen] [siner]: sinen m \textbf{11} genuoge] Gnuͯg m (n) (o) \textbf{12} trügeheit] [trgenheit]: trygenheit n \textbf{15} volgen] fogel m \textbf{16} reinberie] einberie n remberge o \textbf{19} vriuntlîch] fruntliche o \textbf{20} begunde er] begunge n \textbf{21} einem] Einen m o  $\cdot$ dâ] do m n \textit{om.} o \textbf{23} zoum] zuͯme m \textbf{24} barte] portte m \textbf{26} dâ] gra m \textbf{27} geleinet] gelenet n \textbf{28} geweinet] gewenet n gemeinet o \textbf{30} süezer] suͯssern o  $\cdot$ ern] er n o \newline
\end{minipage}
\end{table}
\newpage
\begin{table}[ht]
\begin{minipage}[t]{0.5\linewidth}
\small
\begin{center}*G
\end{center}
\begin{tabular}{rl}
 & \begin{large}D\end{large}ô was mîn hêrre Gawan\\ 
 & sô gezimier\textit{t} ein man,\\ 
 & daz ez si lêrte riuwe,\\ 
 & wan si heten triuwe,\\ 
5 & die des boumgarten pflâgen.\\ 
 & si stuo\textit{n}den ode lâgen\\ 
 & ode s\textit{â}zen in \textbf{den} gezelten,\\ 
 & \textbf{die} vergâzen des vil selten,\\ 
 & si\textbf{ne} klageten sînen kumber grôz.\\ 
10 & \textbf{wîp unde man} des niht verdrôz,\\ 
 & genuoge sprâchen, den \textbf{es} was leit:\\ 
 & "mîner vrouwen trügeheit\\ 
 & wil disen man verleiten\\ 
 & ze grôzen arbeiten.\\ 
15 & owê, daz er ir volgen wil\\ 
 & ûf alsô riuwebæriu zil!"\\ 
 & manic wert man \textbf{dâ} gein im gienc,\\ 
 & der in mit armen umbevienc\\ 
 & durch vriuntlîche\textit{z} enpfâhen.\\ 
20 & dar nâch begunde er nâhen\\ 
 & einem ölboume; dâ stuont daz pfert.\\ 
 & \textbf{ouch} was meniger marc wert\\ 
 & der zoum unde \textbf{sîn} gereite.\\ 
 & mit einem barte breite,\\ 
25 & wol gevlohten unde grâ,\\ 
 & stuont dâr bî ein rîter dâ\\ 
 & über eine \textbf{krücken} geleinet.\\ 
 & von dem wart \textbf{ez} \textbf{beweinet},\\ 
 & daz Gawan zuo dem pfert gienc.\\ 
30 & mit süezer rede ern doch enpfienc.\\ 
\end{tabular}
\scriptsize
\line(1,0){75} \newline
G I L M Z \newline
\line(1,0){75} \newline
\textbf{1} \textit{Initiale} G I L Z  \textbf{23} \textit{Initiale} I  \newline
\line(1,0){75} \newline
\textbf{1} Dô] Da M \textbf{2} sô] So wol L  $\cdot$ gezimiert] gezimierte G  $\cdot$ ein] als ein Z \textbf{3} ez] \textit{om.} I her M \textbf{6} stuonden] stoͮden G \textbf{7} sâzen] sæzzen G [ge]: seszin M  $\cdot$ den] \textit{om.} I L Z \textbf{8} des] es I  $\cdot$ vil] \textit{om.} L \textbf{9} sine] Sy M \textbf{10} wîp unde man] Man vnd wip L (M) (Z) \textbf{11} sprâchen] \textit{om.} M \textbf{15} daz] da M \textbf{16} alsô] alsuͯs L (M) (Z) \textbf{17} gein] engegen L \textbf{19} vriuntlîchez] friͮvntliche G \textbf{20} begunde er] bigunden M \textbf{21} ölboume dâ] boyme M  $\cdot$ daz] ez Z \textbf{22} ouch] Das M  $\cdot$ was] was ez I \textbf{23} sîn] daz I \textbf{24} barte] borten I L (M) \textbf{25} grâ] gar M \textbf{26} dâ] gra I [gar]: dar M \textbf{27} krücken] crippen M brucke Z \textbf{30} doch] da M \newline
\end{minipage}
\hspace{0.5cm}
\begin{minipage}[t]{0.5\linewidth}
\small
\begin{center}*T
\end{center}
\begin{tabular}{rl}
 & Dô was mîn hêr Gawan\\ 
 & sô gezimieret ein man,\\ 
 & daz ez si lêrte riuwe,\\ 
 & wan si heten triuwe,\\ 
5 & die des boumgarten pflâgen.\\ 
 & si stuonden oder lâgen\\ 
 & oder sâzen in gezelten,\\ 
 & \textbf{die} vergâzen des vil selten,\\ 
 & si\textbf{ne} klageten sînen kumber grôz.\\ 
10 & \textbf{man unde wîp} des niht verdrôz,\\ 
 & genuoge sprâchen, den \textbf{ez} was leit:\\ 
 & "mîner vrouwen trügeheit\\ 
 & wil disen man verleiten\\ 
 & ze grôzen arbeiten.\\ 
15 & ouwê, daz er ir volgen wil\\ 
 & ûf alse riuwebæriu zil!"\\ 
 & \begin{large}M\end{large}anec wert man \textbf{dâ} gegen im gienc,\\ 
 & der in mit armen umbevienc\\ 
 & durch vriuntlîch enpfâhen.\\ 
20 & dar nâch begunder \textit{n}âhen\\ 
 & eine\textit{m} oleiboume; dâ stuont daz pfert.\\ 
 & \textbf{ouch} was maneger marke wert\\ 
 & der zoum unde \textbf{sîn} gereite.\\ 
 & mit einem barte breite,\\ 
25 & wol gevlohten unde grâ,\\ 
 & stuont dâr bî ein rîter dâ\\ 
 & über eine \textbf{krücke} geleinet.\\ 
 & von dem wart \textbf{beweinet},\\ 
 & daz Gawan zuo dem pferde gienc.\\ 
30 & mit süezer rede ern doch enpfienc.\\ 
\end{tabular}
\scriptsize
\line(1,0){75} \newline
T U V W O Q R Fr40 \newline
\line(1,0){75} \newline
\textbf{1} \textit{Initiale} O Fr40   $\cdot$ \textit{Majuskel} T  \textbf{17} \textit{Initiale} T U V  \newline
\line(1,0){75} \newline
\textbf{1} Gawan] gawann Q Gawain R \textbf{2} sô] Also W  $\cdot$ gezimieret] [*]: gezimmieret V \textbf{4} \textit{Vers 513.4 fehlt} R  \textbf{7} sâzen] sessen W \textbf{8} des vil] [de*]: dez vil V \textit{om.} O \textbf{9} sine klageten] Sie clageten U (W) (O) (Q) (R) (Fr40) \textbf{11} sprâchen] \textit{om.} O sprach Q \textbf{12} trügeheit] trieghait W \textbf{15} ir] irs V  $\cdot$ volgen] volge W \textbf{16} riuwebæriu] trugbere Q \textbf{17} dâ] do U V W Q R \textbf{18} umbevienc] vmb vin U \textbf{19} vriuntlîch] frivntliche O (Q) \textbf{20} nâhen] gahen T \textbf{21} \textit{Versdoppelung 513.21-28 nach 512.21} O   $\cdot$ einem] einen T (Fr40)  $\cdot$ oleiboume] bovme O  $\cdot$ dâ] do U V W Q \textbf{22} ouch] Daz V \textbf{23} gereite] gerete Q \textbf{24} barte] borten V (W) O (Q) (R) \textbf{26} dâr bî] \textit{om.} O  $\cdot$ dâ] [*t]: da V do W Q \textbf{27} krücke] kurtze W \textbf{28} wart] wart iz U (V) (W) (Q) (R) \textbf{29} Gawan] Gawin R \textbf{30} doch] \textit{om.} O \newline
\end{minipage}
\end{table}
\end{document}
