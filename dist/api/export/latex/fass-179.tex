\documentclass[8pt,a4paper,notitlepage]{article}
\usepackage{fullpage}
\usepackage{ulem}
\usepackage{xltxtra}
\usepackage{datetime}
\renewcommand{\dateseparator}{.}
\dmyyyydate
\usepackage{fancyhdr}
\usepackage{ifthen}
\pagestyle{fancy}
\fancyhf{}
\renewcommand{\headrulewidth}{0pt}
\fancyfoot[L]{\ifthenelse{\value{page}=1}{\today, \currenttime{} Uhr}{}}
\begin{document}
\begin{table}[ht]
\begin{minipage}[t]{0.5\linewidth}
\small
\begin{center}*D
\end{center}
\begin{tabular}{rl}
\textbf{179} & sô daz ich \textbf{wol mac} minne gern,\\ 
 & \textbf{ir sult} mich Liazen \textbf{wern},\\ 
 & iwerer tohter, der schœnen magt.\\ 
 & ir habt mir al ze vil geklagt.\\ 
5 & mag ich \textbf{iu} \textbf{jâmer denne} e\textit{n}tsagen,\\ 
 & des lâze ich iuch \textbf{sô vil} niht tragen."\\ 
 & Urloup nam der junge man\\ 
 & \textbf{von} dem getriwen vürsten sân\\ 
 & unt zalder massenîe.\\ 
10 & des vürsten jâmers drîe\\ 
 & was riwic \textbf{an daz} quater komen,\\ 
 & die vierden vlust het er genomen.\\ 
 & \textbf{\begin{large}D\end{large}annen} schiet \textbf{sus} Parzival.\\ 
 & ritters site unt ritters \textbf{mâl}\\ 
15 & sîn lîp mit zühten vuorte.\\ 
 & ouwê, wan daz in ruorte\\ 
 & manec unsüeziu strenge!\\ 
 & \textbf{im} was diu wîte zenge\\ 
 & unt ouch diu breite gar ze smal.\\ 
20 & elliu grüene in dûhte val.\\ 
 & sîn rôt harnasch \textbf{in dûhte} blanc.\\ 
 & sîn herze diu ougen des betwanc.\\ 
 & sît er tumpheit âne wart,\\ 
 & \textbf{dô}ne wolde in Gahmuretes art\\ 
25 & \textbf{denkens} niht erlâzen\\ 
 & nâch der \textbf{schœnen} Liazen,\\ 
 & der meide sælden rîche,\\ 
 & diu im geselleclîche\\ 
 & sunder minne bôt êre.\\ 
30 & \textbf{swar} sîn ors nû kêre,\\ 
\end{tabular}
\scriptsize
\line(1,0){75} \newline
D \newline
\line(1,0){75} \newline
\textbf{7} \textit{Majuskel} D  \textbf{13} \textit{Großinitiale} D  \newline
\line(1,0){75} \newline
\textbf{5} entsagen] etsagen D \textbf{24} Gahmuretes] Gahmvretes D \newline
\end{minipage}
\hspace{0.5cm}
\begin{minipage}[t]{0.5\linewidth}
\small
\begin{center}*m
\end{center}
\begin{tabular}{rl}
 & sô daz ich \textbf{wol mac} minne gern,\\ 
 & \textbf{sô sullet ir} mich Liazen \textbf{gewern},\\ 
 & iuwerre tohter, der schœne\textit{n} maget.\\ 
 & ir habet mir a\textit{l} ze vil geklaget.\\ 
5 & ma\textit{c} ich \textbf{iu} \textbf{jâmers danne} entsagen,\\ 
 & des lâze ich iuch \textbf{ze vil} niht tragen."\\ 
 & urloup nam der junge man\\ 
 & \textbf{von} dem getriuwen vürsten sân\\ 
 & und zal der massenîe.\\ 
10 & des vürsten jâmers drîe\\ 
 & was riuwic \textbf{an daz} quater komen,\\ 
 & die vierden \dag vluht\dag  het er genomen.\\ 
 & \textbf{\begin{large}D\end{large}\textit{anne}n} schiet \textbf{sus} Parcifal.\\ 
 & ritter\textit{s} site und ritters \textbf{wal}\\ 
15 & sîn lîp mit züht\textit{en} vuor\textit{t}e.\\ 
 & owê, wan daz in ruorte\\ 
 & manic unsüeziu strenge\\ 
 & \textbf{und} was diu wîte ze enge\\ 
 & und ouch diu breite gar ze smal!\\ 
20 & alliu grüene in dûhte val.\\ 
 & sîn rôt harna\textit{s}ch \textbf{in dûhte} blanc.\\ 
 & sîn herze diu ougen des betwanc.\\ 
 & sît er tumpheit âne wart,\\ 
 & \textbf{dô} enwolte in Gahmuretes art\\ 
25 & \dag de keis\dag  niht erlâzen\\ 
 & nâch der \textbf{schœnen} Liazen,\\ 
 & der megde sælden rîche,\\ 
 & diu ime geselleclîche\\ 
 & sunder minne bôt êre.\\ 
30 & \textbf{war} sîn ros nû kêre,\\ 
\end{tabular}
\scriptsize
\line(1,0){75} \newline
m n o Fr69 \newline
\line(1,0){75} \newline
\textbf{13} \textit{Illustration mit Überschrift:} Wie parcifal pelraperre erloͯste vnder Do zelande [*]: herre wart m  Also parcẏfal (parcifal o  ) pelrapier (pelraperie o  ) erlost vnd er do zuͦ lande herre wart n (o)   $\cdot$ \textit{Initiale} m n o  \newline
\line(1,0){75} \newline
\textbf{1} ich wol mac] ich ich >wol mag< o \textbf{2} sullet] sallen o  $\cdot$ ir] ir ir n  $\cdot$ Liazen] liassen n lassen o  $\cdot$ gewern] wern n o \textbf{3} iuwerre] Jre m  $\cdot$ schœnen] schoͯne m \textbf{4} al] alle m \textbf{5} mac] Mage m \textbf{6} ze] so n o \textbf{11} quater] vierde Fr69 \textbf{12} die] Jr n o \textbf{13} Dannen] DO [k]: man m \textbf{14} ritters] Ritter m \textbf{15} zühten] zuhttiger m  $\cdot$ vuorte] vuͯre m \textbf{16} daz in] in das n \textbf{20} val] wal o \textbf{21} harnasch] harnach m harnersch o \textbf{22} diu] sin o sinv́ Fr69 \textbf{24} enwolte] wolt n  $\cdot$ Gahmuretes] gahumetes o \textbf{25} de keis] Do keins n Do koms o \textbf{26} Liazen] liossen n laissen o \textbf{30} war] Wan o \newline
\end{minipage}
\end{table}
\newpage
\begin{table}[ht]
\begin{minipage}[t]{0.5\linewidth}
\small
\begin{center}*G
\end{center}
\begin{tabular}{rl}
 & sô daz ich \textbf{mac wol} minnen geren,\\ 
 & \textbf{\textit{ir} sul\textit{t}} \textit{m}ich Liazen \textbf{weren},\\ 
 & iwer tohter, der schœnen maget.\\ 
 & ir habet mir al ze vil geklaget.\\ 
5 & mag ich \textbf{iu} \textbf{jâmer danne} entsagen,\\ 
 & des lâze ich iuch niht \textbf{langer} tragen."\\ 
 & urloup nam der junge man\\ 
 & \textbf{ze} dem getriwen vürsten sân\\ 
 & unt ze al der messenîe.\\ 
10 & de\textit{s} \textit{v}ürsten \textit{jâmers} drîe\\ 
 & was riwic \textbf{an daz} quater komen,\\ 
 & die vierden vlust het er genomen.\\ 
 & \textbf{dannen} schiet \textbf{dô} Parcival.\\ 
 & rîters site unde rîters \textbf{mâl}\\ 
15 & sîn lîp mit zühten vuorte.\\ 
 & owê, wan daz i\textit{n} ruo\textit{r}te\\ 
 & \textbf{vil} manic unsüeziu strenge!\\ 
 & \textbf{im} was diu wîte zenge\\ 
 & unde ouch diu breite gar ze smal.\\ 
20 & elliu grüene in dûhte val.\\ 
 & sîn rôt harnasch \textbf{dûhte in} blanc.\\ 
 & sîn herze diu ougen des betwanc.\\ 
 & sît er tumpheit âne wart,\\ 
 & \textbf{sô}ne wolt in Gahmuretes art\\ 
25 & \textbf{gedenkens} niht erlâzen\\ 
 & nâch der \textbf{schœnen} Liazen,\\ 
 & \begin{large}D\end{large}er meide sælden rîche,\\ 
 & diu im geselliclîche\\ 
 & sunder minne bôt êre.\\ 
30 & \textbf{swar} sîn ors nû kêre,\\ 
\end{tabular}
\scriptsize
\line(1,0){75} \newline
G I O L M Q R Z Fr40 Fr47 \newline
\line(1,0){75} \newline
\textbf{13} \textit{Überschrift:} Awentewr wy patzifal reyt gen pelrapeir die mit iamer was besatzt Q  Hie schiet parcifal von der schonen lyazzen vnd von irem vater mit manigem leide Z   $\cdot$ \textit{Initiale} I O L M Q R Z Fr40 Fr47  \textbf{27} \textit{Initiale} G  \newline
\line(1,0){75} \newline
\textbf{1} mac wol] mach I wol mach O (M) (Q) (R) (Z) (Fr40) wol nach L  $\cdot$ minnen] minne I O (L) Q Z Fr47 \textbf{2} ir sult] so sult ir G  $\cdot$ Liazen] [h]: laszen L laszin M lẏassen Q lyazen R liazzen Z lazzen \textit{nachträglich korrigiert zu:} liazzen Fr40 :::n Fr47  $\cdot$ weren] geweren Q (Fr47) \textbf{3} der] die R  $\cdot$ schœnen] schone Q \textbf{5} jâmer danne] dann iamer I iammer da mite M  $\cdot$ entsagen] entslahen Fr47 \textbf{6} des] desn I (O) (R) (Z) (Fr40)  $\cdot$ niht langer] so vil niht O L (M) (Q) (R) Z (Fr40) (Fr47) \textbf{7} nam] nam do I \textbf{9} ze al der] zuͤ der I al zeder O \textbf{10} des vürsten jâmers] des werden fursten G Des fvͤrsten iamer O Die fuͯrsten iamers L \textbf{11} riwic] trurig R  $\cdot$ quater] vierd R \textbf{12} vierden] werdin M (Fr47) vierde R  $\cdot$ het] hat L \textbf{13} dannen] ÷Annen O (Fr47) o\textit{m. } R  $\cdot$ schiet dô] schiet da M schid Q \textit{om.} R schiet sus Z  $\cdot$ Parcival] parzival G Parzifal I (R) (Fr40) Parcifal O (L) (Z) Fr47 parzeval M partzifal Q \textbf{14} mâl] wal Fr47 \textbf{15} lîp mit zühten] leypt mit zuchte Q \textbf{16} owê] Awe O Fr47  $\cdot$ wan] ven Q wa R  $\cdot$ daz in] daz ir G in das R  $\cdot$ ruorte] roͮte G rurt Fr47 \textbf{17} vil] \textit{om.} Z  $\cdot$ unsüeziu] vnsuͯsses R \textbf{18} zenge] gar zv enge Z \textbf{19} ouch] \textit{om.} L \textbf{20} elliu] Elle Fr47  $\cdot$ grüene] grunde M guͯtte R  $\cdot$ in dûhte val] idoch zeual I \textbf{21} sîn] Ein Z  $\cdot$ rôt harnasch] harnasch rot Fr40  $\cdot$ dûhte in] in duhte I (O) (Q) (Z) (Fr40) duchte R  $\cdot$ blanc] wis R [val]: blanc Z \textbf{22} diu ougen] siniu augen I sin oͯgen in R  $\cdot$ betwanc] bewist R \textbf{24} sône] don I (O) (Q) (Fr40) Do L Fr47 Da en M (Z) Dene R  $\cdot$ Gahmuretes] gahmurets G Gamvretes O gamuͯretis M gamúretes Q gahmurtes R gamuretes Z gahmuretes Fr40 Gahmuretez Fr47 \textbf{25} gedenkens] Denchens O (L) (Q) (Z) (Fr40) Fr47 Den cheyns M Denkes R  $\cdot$ erlâzen] vorgasszin M \textbf{26} schœnen] werden I (O) (M) (Q) (R) (Fr40)  $\cdot$ Liazen] lyaszen L liazin M lyassen Q lyazen R liazzen Z Fr40 Fr47 \textbf{27} rîche] richen I (Q) (Fr40) \textbf{28} diu] De Fr47  $\cdot$ geselliclîche] minnenkliche R \textbf{29} êre] vnde ere O michel ere L \textbf{30} swar] War L (Q) R Swas M Swa Fr47 \newline
\end{minipage}
\hspace{0.5cm}
\begin{minipage}[t]{0.5\linewidth}
\small
\begin{center}*T
\end{center}
\begin{tabular}{rl}
 & \textbf{iemer gewinnen}, sô daz ich \textbf{mac} minnen gern,\\ 
 & \textbf{ir sult} mich Lyazen \textbf{wern},\\ 
 & iuwerre tohter, der schœnen magt.\\ 
 & ir hât mir alze vil geklagt.\\ 
5 & mag ich \textbf{danne jâmer} entsagen,\\ 
 & des \textbf{en}lâze ich iuch \textbf{sô vil} niht tragen."\\ 
 & urloup nam der junge man\\ 
 & \textbf{ze} dem getriuwen vürsten sân\\ 
 & unde zalder massenîe.\\ 
10 & des vürsten jâmers drîe\\ 
 & was riuwic \textbf{unzem} quater komen,\\ 
 & die vierde vlust heter genomen.\\ 
 & \textbf{\begin{large}V\end{large}on Greharz} schiet \textbf{sus} Parcifal.\\ 
 & rîters site unde rîters \textbf{mâl}\\ 
15 & sîn lîp mit zühten vuorte.\\ 
 & ouwî, wan daz in ruorte\\ 
 & manec unsüeze strenge!\\ 
 & \textbf{im} was di\textit{u} wîte zenge\\ 
 & unde ouch di\textit{u} breite gar ze smal.\\ 
20 & alle grüene in dûhte val.\\ 
 & sîn rôt harnasch \textbf{in dûhte} blanc.\\ 
 & sîn herze di\textit{u} ougen des betwanc.\\ 
 & sît er tumpheite âne wart,\\ 
 & \textbf{sô}ne woltin Gahmuretes art\\ 
25 & \textbf{denkens} niht erlâzen\\ 
 & nâch der \textbf{werden} Lyazen,\\ 
 & der megde sældenrîche,\\ 
 & diu im geselleclîche\\ 
 & sunder minne bôt \textbf{unde} êre.\\ 
30 & \textbf{war} sîn ors nû kêre,\\ 
\end{tabular}
\scriptsize
\line(1,0){75} \newline
T U V W \newline
\line(1,0){75} \newline
\textbf{7} \textit{Initiale} W  \textbf{13} \textit{Großinitiale} T U   $\cdot$ \textit{Initiale} V  \newline
\line(1,0){75} \newline
\textbf{1} [*minne gern]: So daz ich wol mag minne gern V · So das ich wol mag froͤden gern W  $\cdot$ minnen] min U \textbf{2} ir sult] [J* soll*]: So soͤllent ir V  $\cdot$ Lyazen] Lŷazen T lyasen W \textbf{3} der] die W \textbf{4} alze] kumbers W \textbf{5} danne jâmer] [*]: úch iamers danne V eúch danne ya mer W \textbf{6} enlâze] las W  $\cdot$ iuch] îv T \textbf{9} zalder] zuͦ aller W \textbf{11} riuwic unzem quater] rv́wig vnz an ein quater V nun hie zuͦ dem vierden W \textbf{12} vierde] vierden V grossen W  $\cdot$ heter] \textit{om.} W \textbf{13} Greharz] grahars W  $\cdot$ sus] \textit{om.} V  $\cdot$ Parcifal] parzifal V her partzifal W \textbf{18} diu] die T \textbf{19} diu] die T  $\cdot$ gar ze smal] zuͦ schmla W \textbf{20} in dûhte] duchte in W \textbf{21} rôt] roß W \textbf{22} diu] die T  $\cdot$ betwanc] [*]: betang V \textbf{24} sône woltin] So in wolte U  $\cdot$ Gahmuretes] Gahmvretes T Gahmuͦretes U gamuretes V gamurettes W \textbf{25} denkens] Gedenckens W \textbf{26} werden] schoͤnen W  $\cdot$ Lyazen] lŷazen T lyassen W \textbf{28} im] [*nnre]: imme V \textbf{29} unde êre] [*e ere]: ere V \textbf{30} war] Swar V  $\cdot$ nû] \textit{om.} W \newline
\end{minipage}
\end{table}
\end{document}
