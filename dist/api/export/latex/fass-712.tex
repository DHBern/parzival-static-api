\documentclass[8pt,a4paper,notitlepage]{article}
\usepackage{fullpage}
\usepackage{ulem}
\usepackage{xltxtra}
\usepackage{datetime}
\renewcommand{\dateseparator}{.}
\dmyyyydate
\usepackage{fancyhdr}
\usepackage{ifthen}
\pagestyle{fancy}
\fancyhf{}
\renewcommand{\headrulewidth}{0pt}
\fancyfoot[L]{\ifthenelse{\value{page}=1}{\today, \currenttime{} Uhr}{}}
\begin{document}
\begin{table}[ht]
\begin{minipage}[t]{0.5\linewidth}
\small
\begin{center}*D
\end{center}
\begin{tabular}{rl}
\textbf{712} & "nû \textbf{denket}, ob ir mîn œheim sît;\\ 
 & durch triwe scheidet disen strît."\\ 
 & Artus ûz wîsem munde\\ 
 & sprach an der selben stunde:\\ 
5 & "Owî, \textbf{liebiu niftel} \textbf{mîn},\\ 
 & daz dîn jugent \textbf{sô} hôher minne schîn\\ 
 & tuot, daz muoz dir werden sûr!\\ 
 & als tet dîn swester Surdamur\\ 
 & durch \textbf{der} Kriechen lampriure.\\ 
10 & süeziu magt gehiure,\\ 
 & den kampf m\textit{ö}ht ich wol scheiden,\\ 
 & wesse ich daz an iu beiden,\\ 
 & ob \textbf{sîn} herze \textbf{und daz dîne} gesamnet sint.\\ 
 & Gramoflanz, Irotes kint,\\ 
15 & vert mit sô \textbf{manlîchen} siten,\\ 
 & daz der kampf wirt gestriten,\\ 
 & ez \textbf{en}understê diu minne dîn.\\ 
 & gesach er dînen liehten schîn\\ 
 & bî vreude ie ze \textbf{keiner} stunt\\ 
20 & unt dînen \textbf{rôten süezen} munt?"\\ 
 & Si sprach: "\textbf{des} \textbf{en}ist niht \textbf{geschehen}.\\ 
 & wir minnen ein ander \textbf{âne sehen}.\\ 
 & er hât \textbf{aber mir} durch liebe kraft\\ 
 & unt durch rehte geselleschaft\\ 
25 & sînes kleinôdes vil gesant.\\ 
 & er enpfie\textit{n}g ouch von mîner hant\\ 
 & daz z\textbf{er wâren} liebe hôrte\\ 
 & und uns \textbf{beiden} zwîvel stôrte.\\ 
 & der künec ist an mir stæte,\\ 
30 & âne \textbf{valsches} herzen ræte."\\ 
\end{tabular}
\scriptsize
\line(1,0){75} \newline
D \newline
\line(1,0){75} \newline
\textbf{5} \textit{Majuskel} D  \textbf{21} \textit{Majuskel} D  \newline
\line(1,0){75} \newline
\textbf{8} Surdamur] Svrdamvͦr D \textbf{9} Kriechen] chriechen D \textbf{11} möht] moht D \textbf{14} Irotes] Jrots D \textbf{26} enpfieng] enpfieg D \newline
\end{minipage}
\hspace{0.5cm}
\begin{minipage}[t]{0.5\linewidth}
\small
\begin{center}*m
\end{center}
\begin{tabular}{rl}
 & "nû \textbf{gedenkt}, ob ir mîn œhei\textit{m s}ît;\\ 
 & durch triuwe sche\textit{i}det disen strît."\\ 
 & \begin{large}A\end{large}rtus ûz wîsem munde\\ 
 & sprach an der selben stunde:\\ 
5 & "owê, \textbf{liebiu niftel} \textbf{mîn},\\ 
 & daz dîn jugent \textbf{sô} hôher minne sc\textit{h}în\\ 
 & tuot, daz muoz dir werden sûr!\\ 
 & alsô tet dîn swester Surdamur\\ 
 & durch \textbf{der} Kriechen lampriure.\\ 
10 & süeziu maget gehiure,\\ 
 & den kampf möht ich wol scheiden,\\ 
 & w\textit{is}s ich \textit{daz an iu} beiden,\\ 
 & ob \textbf{iuwer} herz gesamet sint.\\ 
 & Gramolantz, Girot\textit{es} kint,\\ 
15 & v\textit{e}r\textit{t} mit sô \textbf{manlîchem} siten,\\ 
 & daz der kampf wirt gestriten,\\ 
 & ez understê \textbf{dan} diu minne dîn.\\ 
 & gesach er dînen liehten schîn\\ 
 & bî vröuden ie zuo \textbf{keiner} stunt\\ 
20 & und dînen \textbf{rôten süezen} munt?"\\ 
 & si sprach: "\textbf{daz} ist niht \textbf{beschehen}.\\ 
 & wir minnen ein ander \textbf{âne sehen}.\\ 
 & er het \textbf{mir aber} durch liebe kraft\\ 
 & und durch rehte geselleschaft\\ 
25 & sînes kleinôtes vil gesant.\\ 
 & er enpfienc ouch von mîner hant\\ 
 & daz z\textbf{er wâren} liebe hôrt\\ 
 & und uns \textbf{beiden} zwîvel stôrt.\\ 
 & der künic ist an mir stæte,\\ 
30 & âne \textbf{valsch} herzen ræte."\\ 
\end{tabular}
\scriptsize
\line(1,0){75} \newline
m n o Fr69 \newline
\line(1,0){75} \newline
\textbf{3} \textit{Initiale} m   $\cdot$ \textit{Capitulumzeichen} n  \newline
\line(1,0){75} \newline
\textbf{1} gedenkt] gedenck o  $\cdot$ sît] sin sit m \textbf{2} triuwe] struwe o  $\cdot$ scheidet] scheitdent m \textbf{6} schîn] schein m \textbf{8} alsô] [Aldo]: Also n  $\cdot$ Surdamur] surdamúr o \textbf{9} der] die Fr69  $\cdot$ Kriechen] kirchen o schrichen Fr69  $\cdot$ lampriure] lammprivren o \textbf{11} möht] mocht o (Fr69) \textbf{12} wiss ich] Wie sich m n o  $\cdot$ daz an iu] \textit{om.} m das an in n \textbf{14} Gramolantz] Gramolancz o  $\cdot$ Girotes] girot m girotz n girots o \textbf{15} vert] [Fattem]: Farttem m Fuͯrt n Firt o  $\cdot$ manlîchem] manigem o \textbf{19} keiner] kommer o \textbf{20} rôten] >roten< o \textbf{22} minnen] manent o \textbf{26} er enpfienc] Erpfing o \textbf{27} wâren] ware o \newline
\end{minipage}
\end{table}
\newpage
\begin{table}[ht]
\begin{minipage}[t]{0.5\linewidth}
\small
\begin{center}*G
\end{center}
\begin{tabular}{rl}
 & "\begin{large}N\end{large}û \textbf{denket}, obe ir mîn œheim sît;\\ 
 & durch triwe scheidet disen strît."\\ 
 & Artus ûz wîsem munde\\ 
 & sprach an der selben stunde:\\ 
5 & "owê, \textbf{liebiu niftel} \textbf{mîn},\\ 
 & daz dîn jugent \textbf{ûz} hôher minne schîn\\ 
 & tuot, daz muoz dir werden sûr!\\ 
 & alsô tet dîn swester Surdamur\\ 
 & durch \textbf{den} Kriechen lampriure.\\ 
10 & süeziu maget gehiure,\\ 
 & den kampf möhte ich wol scheiden,\\ 
 & wesse ich daz an iu beiden,\\ 
 & op \textbf{sîn} herze \textbf{unde dîn} gesament sint.\\ 
 & Gramoflanz, Gyrotes kint,\\ 
15 & vert mit sô \textbf{manlîchen} siten,\\ 
 & daz der kampf wirt gestriten,\\ 
 & ez \textbf{en}understê diu minne dîn.\\ 
 & gesach er dînen liehten schîn\\ 
 & bî vröude ie ze \textbf{deheiner} stunt\\ 
20 & unde dînen \textbf{rôten süezen} munt?"\\ 
 & si sprach: "\textbf{des} \textbf{en}ist niht \textbf{geschehen}.\\ 
 & wir minnen ein ander \textbf{ungesehen}.\\ 
 & er hât \textbf{aber mir} durch liebe kraft\\ 
 & unde durch rehte geselleschaft\\ 
25 & sînes kleinœdes vil gesant.\\ 
 & er enpfienc ouch von mîner hant\\ 
 & daz ze \textbf{wârem} liebe hôrte\\ 
 & unde uns \textbf{den} zwîvel stôrte.\\ 
 & der künic ist ane mir stæte,\\ 
30 & âne \textbf{valsches} herzen ræte."\\ 
\end{tabular}
\scriptsize
\line(1,0){75} \newline
G I L M Z Fr18 Fr22 \newline
\line(1,0){75} \newline
\textbf{1} \textit{Initiale} G L Z Fr18 Fr22  \textbf{11} \textit{Initiale} I  \newline
\line(1,0){75} \newline
\textbf{1} denket] gedenchet L  $\cdot$ obe] daz I \textbf{4} selben] \textit{om.} L \textbf{5} liebiu] vil liebev I \textbf{6} ûz] so I L M Z Fr18  $\cdot$ minne] [pris]: mynne M  $\cdot$ schîn] pin L \textbf{8} Surdamur] surdamuͦr I (Fr18) Sordamvr L surdamuͯr M \textbf{9} den] der M Z Fr22  $\cdot$ Kriechen] chriechen G criechen I kryeschen L krichen M :::e::: Fr22 \textbf{11} möhte] mocht L (M) (Z) (Fr18) \textbf{12} Wuste das ich en beiden M  $\cdot$ daz] \textit{om.} I  $\cdot$ iu] in Fr18 \textbf{13} sîn] si I din L M  $\cdot$ dîn] daz din I Z daz sine L (M) (Fr18) \textbf{14} Gramoflanz] Gramorflanz M Gramoflantz Z :::ntz Fr18  $\cdot$ Gyrotes] gẏrots G Gyrtes I girotes M Gẏrotes Fr18  $\cdot$ kint] [son]: kint M \textbf{15} vert] Wer M  $\cdot$ mit sô] so mit L \textbf{17} enunderstê] vnderste I \textbf{18} liehten] lichten L M \textbf{19} bî vröude] zefreude I Bi vrowen L By vrouden M (Z)  $\cdot$ deheiner] icheiner M keiner Z \textbf{21} enist] ist I L Z Fr18 \textbf{23} liebe] mynen M \textbf{24} rehte] rech L \textbf{27} ze wârem liebe] warn liebes I zuͯ der waren liebe L (M) (Z) (Fr18)  $\cdot$ hôrte] gehorte M Fr18 \textbf{28} den] beiden L (M) Z \newline
\end{minipage}
\hspace{0.5cm}
\begin{minipage}[t]{0.5\linewidth}
\small
\begin{center}*T
\end{center}
\begin{tabular}{rl}
 & "nû \textbf{denket}, ob ir mîn œheim sît;\\ 
 & durch triuwe scheidet disen strît."\\ 
 & Artus ûz wîsem munde\\ 
 & sprach an der selben stunde:\\ 
5 & "owê, \textbf{liebez niftelîn},\\ 
 & daz dîn jugent \textbf{sô} hôher minne schîn\\ 
 & tuot, daz muoz dir werden sûr!\\ 
 & als tet dîn swester Syrdamur\\ 
 & durch \textbf{den} Criechen lampriure.\\ 
10 & süeziu maget gehiure,\\ 
 & den kampf möht ich wol scheiden,\\ 
 & wist ich daz an iu beiden,\\ 
 & ob \textbf{dîn} herze \textbf{und daz sîn} gesament sint.\\ 
 & Gramoflanz, Irotes kint,\\ 
15 & vert mit sô \textbf{manlîchen} siten,\\ 
 & daz der kampf wirt gestriten,\\ 
 & ez \textbf{en}understê diu minne dîn.\\ 
 & gesach er dînen liehten schîn\\ 
 & bî vreuden ie zuo \textbf{einer} stunt\\ 
20 & und dînen \textbf{rôtsüezen} munt?"\\ 
 & si sprach: "\textbf{daz} ist niht \textbf{zuo geschehene}.\\ 
 & wir minnen ein ander \textbf{ungesehene}.\\ 
 & er \textit{het} \textbf{aber mir} durch liebe kraft\\ 
 & und durch rehte geselleschaft\\ 
25 & sînes kleinôdes vil gesant.\\ 
 & er entvienc ouch von mîner hant\\ 
 & daz zuo \textbf{der wâren} liebe hôrte\\ 
 & und uns \textbf{beiden} zwîvel stôrte.\\ 
 & der künec ist an mir stæte,\\ 
30 & âne \textbf{valsche} herzeræte."\\ 
\end{tabular}
\scriptsize
\line(1,0){75} \newline
U V W Q R \newline
\line(1,0){75} \newline
\textbf{3} \textit{Initiale} W R  \newline
\line(1,0){75} \newline
\textbf{1} denket] gedenckent V \textbf{2} triuwe] trúwen R \textbf{3} wîsem] wisen V R seinem weisen W \textbf{5} owê] O wie Q Owi R  $\cdot$ liebez] vil liebe W liebe Q (R)  $\cdot$ niftelîn] nyfftel mein W (Q) (R) \textbf{8} tet] tett s R  $\cdot$ Syrdamur] Surdamuͦr U svrdamur V (W) surdamuͯr Q sordamure R \textbf{9} den] [d*]: den V der Q  $\cdot$ Criechen] kriechen V W R krichen Q \textbf{10} süeziu] Suͯsze R \textbf{11} möht] moch R \textbf{12} ich daz] das ich Q  $\cdot$ an] \textit{om.} W \textbf{13} herze] herr W  $\cdot$ daz] \textit{om.} Q  $\cdot$ sint] sin Q \textbf{14} Gramoflanz] Gramaflantz V Gramoflantz W Q Gramoflanczes R  $\cdot$ Irotes] Jrotes U R ẏrotes V yrotes W \textbf{15} mit sô] mit also W so mit R \textbf{17} ez enunderstê] Ez vnderste V Es vnderste dann W (R) Ezn vnter sten Q \textbf{18} liehten] lichten Q \textbf{19} zuo] \textit{om.} W  $\cdot$ einer] deheiner V (R) keiner W Q \textbf{20} dînen] deinem W deinē Q  $\cdot$ rôtsüezen] roten sussen Q (R) \textbf{21} zuo geschehene] geschehen V W (Q) (R) \textbf{23} het] \textit{om.} U hat W (Q) (R)  $\cdot$ liebe] liebv́ V (R) minne Q \textbf{25} kleinôdes] cleinoͤter V cleinendes R \textbf{26} er entvienc] Ern pfieg Q \textbf{27} liebe] minne R  $\cdot$ hôrte] gehorte W Q (R) \textbf{30} Ane valsches herzen rete V (W) (Q) An falsches ratten stette R \newline
\end{minipage}
\end{table}
\end{document}
