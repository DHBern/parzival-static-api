\documentclass[8pt,a4paper,notitlepage]{article}
\usepackage{fullpage}
\usepackage{ulem}
\usepackage{xltxtra}
\usepackage{datetime}
\renewcommand{\dateseparator}{.}
\dmyyyydate
\usepackage{fancyhdr}
\usepackage{ifthen}
\pagestyle{fancy}
\fancyhf{}
\renewcommand{\headrulewidth}{0pt}
\fancyfoot[L]{\ifthenelse{\value{page}=1}{\today, \currenttime{} Uhr}{}}
\begin{document}
\begin{table}[ht]
\begin{minipage}[t]{0.5\linewidth}
\small
\begin{center}*D
\end{center}
\begin{tabular}{rl}
\textbf{29} & ir \textbf{ougen} dem herzen sân,\\ 
 & daz er wære wolgetân.\\ 
 & si kunde \textbf{ouch} liehte varwe spehen,\\ 
 & \textbf{wan} si hete gesehen\\ 
5 & manegen liehten heiden.\\ 
 & al dâ wart \textbf{under} in beiden\\ 
 & ein vil \textbf{getriulîch} ger.\\ 
 & \textbf{si} sach \textbf{dar} \textbf{unt} \textbf{er} sach her.\\ 
 & dar nâch hiez si schenk\textit{en} sân.\\ 
10 & \textbf{torste} si, daz wære verlân.\\ 
 & \textbf{ez müete si}, deiz niht beleip,\\ 
 & wand ez die ritter ie vertreip,\\ 
 & die \textbf{gerne} sprâchen wider wîp.\\ 
 & doch was ir \textbf{liep} sîn selbes lîp.\\ 
15 & ouch het er ir den muot gegeben:\\ 
 & sîn leben was der vrouwen leben.\\ 
 & \textbf{Dô} stuont \textbf{er} ûf unt sprach:\\ 
 & "vrouwe, ich tuon iu ungemach.\\ 
 & ich kan ze lange sitzen,\\ 
20 & daz \textbf{en}tuon ich niht \textbf{mit} witzen.\\ 
 & mir ist vil dienestlîchen leit,\\ 
 & daz iwer kumber ist sô breit.\\ 
 & \textbf{vrouwe, gebietet} über mich.\\ 
 & swâr ir welt, dâr ist mîn gerich.\\ 
25 & ich dien iu \textbf{allez}, \textbf{daz} ich sol."\\ 
 & \textbf{si sprach}: "hêrre, \textbf{des trûwe ich} wol."\\ 
 & \textit{\begin{large}D\end{large}}er burcgrâve, sîn wir\textit{t},\\ 
 & nû vil wênic des \textbf{enbirt},\\ 
 & er\textbf{n} \textbf{kürze} im \textbf{sîne} stunde.\\ 
30 & \textbf{ze} vrâgen er begunde,\\ 
\end{tabular}
\scriptsize
\line(1,0){75} \newline
D \newline
\line(1,0){75} \newline
\textbf{17} \textit{Majuskel} D  \textbf{27} \textit{Initiale} D  \newline
\line(1,0){75} \newline
\textbf{9} schenken] schench D \textbf{27} Der] ÷er D  $\cdot$ wirt] wir: \textit{nachträglich korrigiert zu:} wirt D \newline
\end{minipage}
\hspace{0.5cm}
\begin{minipage}[t]{0.5\linewidth}
\small
\begin{center}*m
\end{center}
\begin{tabular}{rl}
 & ir \textbf{ouge} dem herzen sân,\\ 
 & daz er wære wol getân.\\ 
 & si k\textit{u}nde \textbf{ouch} liehte varwe spehen,\\ 
 & \textbf{wann} si hete \textbf{ouch} \textbf{ê} gesehen\\ 
5 & manigen liehten heiden.\\ 
 & aldâr wart \textit{\textbf{under}} in beiden\\ 
 & ein vil \textbf{ungetriulîchiu} ger.\\ 
 & \textbf{si} sach \textbf{dar} \textbf{und} \textbf{er} sach her.\\ 
 & dar nâch hiez si schenken sân.\\ 
10 & \textbf{getorste} si, daz wære verlân.\\ 
 & \textbf{ez müete si}, dês niht bleip,\\ 
 & wande e\textit{z} die ritter ie vertreip,\\ 
 & die sprâchen wider \textbf{diu} wîp.\\ 
 & doch was ir \textbf{lîp} sîn selbes lîp.\\ 
15 & ouch het er ir den muot gegeben:\\ 
 & sîn leben was der vrowen leben.\\ 
 & \textbf{\begin{large}G\end{large}ahmuret} stuont ûf und sprach:\\ 
 & "vrowe, ich tuon iu ungemach.\\ 
 & i\textbf{n} kan ze lange sitzen,\\ 
20 & daz tuon ich niht \textbf{mit} witzen.\\ 
 & mir ist vil dienstlîchen leit,\\ 
 & daz iuwer kumber ist sô breit.\\ 
 & \textbf{vrouwe, gebiete\textit{t}} über mich.\\ 
 & wâr ir welt, dâr\textit{s}t mîn gerich.\\ 
25 & ich diene iu \textbf{allez}, \textbf{daz} ich sol."\\ 
 & \textbf{si sprach}: "hêrre, \textbf{daz trûwe ich} wol."\\ 
 & der burcgrâve, sîn wirt,\\ 
 & nû vil wênic des \textbf{verbirt},\\ 
 & er \textbf{en}\textbf{kurzte} ime \textbf{sîne} stunde.\\ 
30 & \textbf{ze} vrâgen er begunde:\\ 
\end{tabular}
\scriptsize
\line(1,0){75} \newline
m n o W \newline
\line(1,0){75} \newline
\textbf{17} \textit{Initiale} m  \newline
\line(1,0){75} \newline
\textbf{1} \textit{Die Verse 26.4-29.1 fehlen} o  \textbf{2} wære wol] wol were n \textbf{3} kunde] kinde m  $\cdot$ liehte] liech n lichte W  $\cdot$ spehen] sprehen n \textbf{4} ê] me W \textbf{6} under] \textit{om.} m \textbf{7} vil] \textit{om.} n o W  $\cdot$ ungetriulîchiu] getruwelich n o (W) \textbf{8} und] \textit{om.} o W \textbf{11} müete] jnnet n (W) muet o  $\cdot$ bleip] bleibt W \textbf{12} ez] er m n o  $\cdot$ ie] E n (o)  $\cdot$ vertreip] vertreibt W \textbf{13} sprâchen] gerne sprochent n (o) gerne sprechent W \textbf{14} sîn] sins o \textbf{15} gegeben] geben W \textbf{17} Gahmuret] Gamúret n Gamuret o W  $\cdot$ ûf] vff vff n \textbf{19} in] Jch n o (W) \textbf{21} vil] so vil W \textbf{23} gebietet] gebietten m (o) \textbf{24} wâr] Was n o Das W  $\cdot$ dârst] darft m das ist n o daz W \textbf{25} diene] diende o \textbf{26} trûwe] getruwe n o W \textbf{28} nû] \textit{om.} W  $\cdot$ des] das n o W \textbf{29} enkurzte] kurtzet n (W) kuͯrcz o \textbf{30} vrâgen] sagen W \newline
\end{minipage}
\end{table}
\newpage
\begin{table}[ht]
\begin{minipage}[t]{0.5\linewidth}
\small
\begin{center}*G
\end{center}
\begin{tabular}{rl}
 & iriu \textbf{ougen} dem herzen sân,\\ 
 & daz er wære wolgetân.\\ 
 & si kunde liehte varwe spehen.\\ 
 & si het \textbf{ouch} \textbf{dâ vor} gesehen\\ 
5 & \textbf{vil} manigen liehten heiden.\\ 
 & al dâ wart \textbf{von} in beiden\\ 
 & ein vil \textbf{getriwiu} ger.\\ 
 & \textbf{er} sach \textbf{dar}, \textbf{si} sach her.\\ 
 & dar nâch hiez si schenken sân.\\ 
10 & \textbf{getorste} si, daz wære verlân.\\ 
 & \textbf{si muote}, daz ez niht beleip,\\ 
 & wan ez die rîter ie vertreip,\\ 
 & die \textbf{gerne} sprâchen wider \textbf{diu} wîp.\\ 
 & doch was ir \textbf{lîp} sîn selbes lîp.\\ 
15 & ouch het er ir den muot gegeben:\\ 
 & sîn leben was der vrouwen leben.\\ 
 & \textbf{dô} stuont \textbf{er} ûf und sprach:\\ 
 & "vrouwe, ich tuon iu ungemach.\\ 
 & ich kan ze lange sitzen,\\ 
20 & daz tuon ich niht \textbf{von} witzen.\\ 
 & \begin{large}M\end{large}ir ist vil dienstlîchen leit,\\ 
 & daz iwer kumber ist sô breit.\\ 
 & \textbf{vrouwe, gebiet} über mich.\\ 
 & swâr ir welt, dârst mîn gerich.\\ 
25 & ich diene iu \textbf{gerne}, \textbf{swaz} ich sol."\\ 
 & "hêrre, \textbf{ich getriwes} \textbf{iu} \textbf{harte} wol."\\ 
 & der burcgrâve, sîn wirt,\\ 
 & nû vil wênic des \textbf{verbirt},\\ 
 & er \textbf{en}\textbf{kürze} im \textbf{die} stunde.\\ 
30 & vrâgen er begunde,\\ 
\end{tabular}
\scriptsize
\line(1,0){75} \newline
G O L M Q R Z Fr29 Fr32 \newline
\line(1,0){75} \newline
\textbf{1} \textit{Initiale} Fr29  \textbf{3} \textit{Initiale} M  \textbf{7} \textit{Versal} Fr32  \textbf{17} \textit{Initiale} M  \textbf{21} \textit{Initiale} G  \textbf{27} \textit{Initiale} O L Q R Z Fr32  \newline
\line(1,0){75} \newline
\textbf{1} iriu ougen] Jr awge Q  $\cdot$ herzen] herren Z \textbf{3} kunde] kund ovch Z  $\cdot$ liehte] lýchte L (M) (Q) \textbf{4} si] Wan sie Q (R) (Fr32)  $\cdot$ ouch] \textit{om.} R  $\cdot$ dâ] do Q \textbf{5} liehten] lýchten L (M) (Q) \textbf{6} von in] vnder in O L (M) (Q) R Z vnder Fr32 \textbf{7} ein] Sin Fr32  $\cdot$ getriwiu] getrevlichiv O (L) (M) (Q) (R) (Z) (Fr32) \textbf{8} dar] hin Q R (Fr32)  $\cdot$ si] er Q vnd sie Z \textbf{10} getorste si] Hiet si getor \textit{nachträglich korrigiert zu:} getorst O  $\cdot$ daz] es Q \textbf{11} ez] er L Q Z Fr32  $\cdot$ beleip] enpleyb Q \textbf{12} ez] daz Fr32  $\cdot$ ie] \textit{om.} O ir R \textbf{13} sprâchen] sprechent R  $\cdot$ diu] \textit{om.} Q \textbf{14} ir lîp sîn] sin lib ir R  $\cdot$ selbes] \textit{om.} Q \textbf{15} muot] munt Z \textbf{17} dô] Er L Da Z  $\cdot$ er] \textit{om.} L \textbf{20} tuon] entvͦn O (M) (R)  $\cdot$ von] mit Z \textbf{24} swâr] War L (M) Swa n\textit{achträglich korrigiert zu: }Swas O War \textit{nachträglich korrigiert zu:} Wasz Q Was R  $\cdot$ dârst] dasz ist Q (R)  $\cdot$ mîn] nun R  $\cdot$ gerich] [gebit]: gerich M \textbf{25} gerne swaz] allez daz O (L) (M) (Q) (R) Z (Fr32) \textbf{26} hêrre] Sie sprach herre Q (R) (Z) (Fr32)  $\cdot$ ich getriwes] des getrawe ich O (L) (M) (Q) (R) (Z) (Fr32)  $\cdot$ iu] \textit{om.} Fr32  $\cdot$ harte] \textit{om.} O L M Q R Z Fr32 \textbf{27} der] ÷er O \textbf{28} des] es Q  $\cdot$ verbirt] enbirt Fr32 \textbf{29} enkürze] kurcze M entkurtzte Q erkurcze R kuͤrtzte Z \textbf{30} vrâgen] Zefragen O (M) (Q) (R) (Z) (Fr32) \newline
\end{minipage}
\hspace{0.5cm}
\begin{minipage}[t]{0.5\linewidth}
\small
\begin{center}*T
\end{center}
\begin{tabular}{rl}
 & ir \textbf{ouge} dem herzen sân,\\ 
 & daz er wære wol getân.\\ 
 & si kunde liehte varwe spehen.\\ 
 & si hete \textbf{ouch} \textbf{dâ vor} gesehen\\ 
5 & \textbf{vil} manegen liehten heiden.\\ 
 & aldâ wart \textbf{under} in beiden\\ 
 & ein vil \textbf{getriuwelîch\textit{iu}} ger.\\ 
 & \textbf{er} sach \textbf{hin}, \textbf{si} sach her.\\ 
 & dar nâch hiez si schenken sân.\\ 
10 & \textbf{getorste} si, daz wære verlân.\\ 
 & \textbf{si muote}, daz ez niht bleip,\\ 
 & wand ez die rîter ie vertreip,\\ 
 & di\textit{e} \textbf{gerne} sprâchen wider \textbf{diu} wîp.\\ 
 & doch was ir \textbf{lîp} sîn selbes lîp.\\ 
15 & ouch het er ir den muot gegeben:\\ 
 & sîn leben was der vrouwen leben.\\ 
 & \textbf{Dô} stuont \textbf{er} ûf und sprach:\\ 
 & "vrouwe, ich tuon iu ungemach.\\ 
 & ich kan ze lange sitzen,\\ 
20 & daz \textbf{en}tuon ich niht \textbf{von} witzen.\\ 
 & mir ist vil dienstlîchen leit,\\ 
 & daz iuwer kumber ist sô breit.\\ 
 & \textbf{gebiet, vrouwe}, über mich.\\ 
 & swâ ir welt, dâr ist mîn gerich.\\ 
25 & ich diene iu \textbf{allez}, \textbf{daz} ich sol."\\ 
 & "Hêrre, \textbf{des getriuwe ich} wol."\\ 
 & \begin{large}D\end{large}er burcgrâve, sîn wirt,\\ 
 & nû vil wênic des \textbf{verbirt},\\ 
 & er \textbf{kurzt}im \textbf{die} stunde.\\ 
30 & \textbf{ze} vrâgenne er begunde,\\ 
\end{tabular}
\scriptsize
\line(1,0){75} \newline
T U V \newline
\line(1,0){75} \newline
\textbf{17} \textit{Majuskel} T  \textbf{26} \textit{Majuskel} T  \textbf{27} \textit{Initiale} T U V  \newline
\line(1,0){75} \newline
\textbf{3} kunde] kunden U \textbf{4} hete] hat V  $\cdot$ gesehen] ê gesehin U (V) \textbf{7} getriuwelîchiu] getriuweliche T \textbf{8} hin] dar U V \textbf{12} wand ez] wan dez T \textbf{13} die] div T \textbf{20} entuon ich] duͦn ich ich U tuͦn ich V \textbf{21} dienstlîchen] innenclichen V \textbf{24} swâ] War U  $\cdot$ gerich] rich U \textbf{26} ich wol] ich [*]: v́ch wol V \textbf{29} Er [kúrtze*]: kúrtzete im ie die stunde V \newline
\end{minipage}
\end{table}
\end{document}
