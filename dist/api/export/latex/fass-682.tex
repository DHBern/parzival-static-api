\documentclass[8pt,a4paper,notitlepage]{article}
\usepackage{fullpage}
\usepackage{ulem}
\usepackage{xltxtra}
\usepackage{datetime}
\renewcommand{\dateseparator}{.}
\dmyyyydate
\usepackage{fancyhdr}
\usepackage{ifthen}
\pagestyle{fancy}
\fancyhf{}
\renewcommand{\headrulewidth}{0pt}
\fancyfoot[L]{\ifthenelse{\value{page}=1}{\today, \currenttime{} Uhr}{}}
\begin{document}
\begin{table}[ht]
\begin{minipage}[t]{0.5\linewidth}
\small
\begin{center}*D
\end{center}
\begin{tabular}{rl}
\textbf{682} & was mit vrouwen umbehalden.\\ 
 & kan ich nû mære walden,\\ 
 & ich sage \textbf{iu}, wer durch in dâ was\\ 
 & geherberget ûf daz gras\\ 
5 & an sîne samnunge komen.\\ 
 & habt ir \textbf{des} ê \textbf{niht} vernomen,\\ 
 & \textbf{sô} lât michz iu machen kunt:\\ 
 & \multicolumn{1}{l}{ - - - }\\ 
 & ûz der \textbf{wazzervesten} stat von \textbf{Punt}\\ 
 & brâhte im der werde œheim sîn,\\ 
10 & der künec Brandelidelin,\\ 
10 & \multicolumn{1}{l}{ - - - }\\ 
 & \textbf{sehs} hundert \textbf{clâre} vrouwen;\\ 
 & ieslîchiu mohte schouwen\\ 
 & gewâpent dâ ir âmîs\\ 
 & durch rîterschaft unt durch prîs.\\ 
15 & die werden Punturteise\\ 
 & wâren wol an dirre reise.\\ 
 & Dâ was, welt ir gelouben mirs,\\ 
 & der clâre \textbf{Bernout} \textbf{de} Riviers;\\ 
 & des rîcher vater Narant\\ 
20 & het im \textbf{lâzen} \textbf{Ukerlant}.\\ 
 & der brâhte \textbf{in} kocken ûf dem mer\\ 
 & ein \textbf{alsô} \textbf{clârez vrouwen} her,\\ 
 & \textbf{den} man dâ \textbf{liehter} varwe jach\\ 
 & unt anders \textbf{niht} dâ von \textbf{in} sprach.\\ 
25 & der wâren \textbf{zwei} hundert\\ 
 & ze meiden \textbf{besundert};\\ 
 & \textbf{zwei} hundert heten dâ ir man,\\ 
 & ob ichz \textbf{geprüevet} rehte \textbf{hân}.\\ 
 & \textbf{\textit{\begin{large}B\end{large}}ernout} fiz cons Narant,\\ 
30 & vünf hundert rîter wert erkant\\ 
\end{tabular}
\scriptsize
\line(1,0){75} \newline
D \newline
\line(1,0){75} \newline
\textbf{17} \textit{Majuskel} D  \textbf{29} \textit{Initiale} D  \newline
\line(1,0){75} \newline
\textbf{15} Punturteise] Pvntvrtêise D \textbf{18} Bernout de Riviers] Bernoͮt de Rivîers D \textbf{20} Ukerlant] Vcherlant D \textbf{29} Bernout] ÷ernoͮt D \newline
\end{minipage}
\hspace{0.5cm}
\begin{minipage}[t]{0.5\linewidth}
\small
\begin{center}*m
\end{center}
\begin{tabular}{rl}
 & was mit vrowen umbehalten.\\ 
 & kan ich nû mære walten,\\ 
 & ich sage \textbf{iu}, wer durch in d\textit{â} was\\ 
 & geherberget ûf daz gras\\ 
5 & an sîn samenunge komen.\\ 
 & habt ir \textbf{daz} ê \textbf{niht} vernomen,\\ 
 & \textbf{sô} lât mich ez iu machen kunt\\ 
 & nû balt ûf der selben stunt:\\ 
 & ûz der \textbf{wazzerveste\textit{n}} stat von \textbf{Pin}\\ 
 & brâht im der werde œheim sîn\\ 
10 & - der künic Brandelidelin\\ 
10 & was ouch bereit sô fîn -\\ 
 & \textbf{sehs} hundert \textbf{clâre} vrouwen;\\ 
 & \textbf{der} ieglîchiu mohte schouwen\\ 
 & gewâpent d\textit{â} ir \textit{â}m\textit{î}s\\ 
 & durch ritterschaft und durch prîs.\\ 
15 & die werden Ponturteise\\ 
 & wâren wol an diser reise.\\ 
 & d\textit{â} was, wolt ir glouben mirs,\\ 
 & der clâre \textbf{Bernou\textit{t}} \textbf{de} Riwirs;\\ 
 & des rîc\textit{h}er vater \textit{Na}r\textit{a}nt\\ 
20 & het im \textbf{gelâzen} \textbf{U\textit{c}herlant}.\\ 
 & der brâht \textbf{in} kocken ûf dem mer\\ 
 & ein \textbf{alsô} \textbf{clârez \textit{vrouwen}} her,\\ 
 & \textbf{den} man d\textit{â} \textbf{liehter} varwe jach\\ 
 & und anders \textbf{niht} d\textit{â} von \textbf{in} sprach.\\ 
25 & der wâren \textbf{zwei} hundert\\ 
 & zuo megden \textbf{ûz gesundert};\\ 
 & \textbf{zwei} hundert heten d\textit{â} ir man,\\ 
 & ob ichz \textbf{gebrüefet} rehte \textbf{hân}.\\ 
 & \textbf{Berno\textit{u}t} fi\textit{z} c\textit{o}ns Narant,\\ 
30 & vünf hundert ritter wert erkant\\ 
\end{tabular}
\scriptsize
\line(1,0){75} \newline
m n o Fr69 \newline
\line(1,0){75} \newline
\newline
\line(1,0){75} \newline
\textbf{2} kan] Kande n kam o \textbf{3} dâ] do m n o \textbf{8} wazzervesten] wasser veste m \textbf{10} Brandelidelin] brandelidin Fr69 \textbf{10} \textit{Vers 682.10¹ fehlt} Fr69  \textbf{12} der ieglîchiu] Die iglich o der iesliche Fr69  $\cdot$ mohte] moͯchte n \textbf{13} dâ] do m n o  $\cdot$ âmîs] mirs m \textbf{17} dâ] Do m n o \textbf{18} Bernout] bernous m bernons n berneus o pernout Fr69  $\cdot$ de Riwirs] de rivirs n der rivirs o derivers Fr69 \textbf{19} des] der Fr69  $\cdot$ rîcher] richtter m  $\cdot$ Narant] worent m o warant n \textbf{20} gelâzen] lazen Fr69  $\cdot$ Ucherlant] utherlant m o icher lant n vckerlant Fr69 \textbf{21} brâht] brach Fr69  $\cdot$ kocken] [kocker]: kocken o  $\cdot$ dem] \textit{om.} n  $\cdot$ mer] wer Fr69 \textbf{22} vrouwen] \textit{om.} m  $\cdot$ her] [sin]: her o \textbf{23} dâ] do m n o \textbf{24} dâ von in] do von in m n von ẏn do o \textbf{26} gesundert] gesmitter o \textbf{27} dâ] do m n o \textbf{29} Bernout] Bernoant m Bernans n Gernont o Bernovt Fr69  $\cdot$ fiz] fir m fúr n o  $\cdot$ cons Narant] kans narant m n kansnarant o \newline
\end{minipage}
\end{table}
\newpage
\begin{table}[ht]
\begin{minipage}[t]{0.5\linewidth}
\small
\begin{center}*G
\end{center}
\begin{tabular}{rl}
 & \begin{large}W\end{large}as mit vrouwen \textbf{al} umbehalden.\\ 
 & kan ich nû mære walden,\\ 
 & ich sage \textbf{iu}, wer durch in dâ was\\ 
 & geherberget ûffez gras\\ 
5 & \textbf{unde} an sîne samenunge komen.\\ 
 & habet ir \textbf{des} ê \textbf{niht} vernomen,\\ 
 & lât michz iu machen kunt:\\ 
 & \multicolumn{1}{l}{ - - - }\\ 
 & ûz der \textbf{vesten} stat von \textbf{Punt}\\ 
 & brâhte im der werde œheim sîn,\\ 
10 & der künic Brandelidelin,\\ 
10 & \multicolumn{1}{l}{ - - - }\\ 
 & \textbf{vier} hundert \textbf{clâre} vrouwen;\\ 
 & \textbf{der} etslîchiu mohte schouwen\\ 
 & gewâpent dâ ir âmîs\\ 
 & durch rîterschaft unde durch prîs.\\ 
15 & die werden Ponturteise\\ 
 & wâren wol an dirre reise.\\ 
 & dâ was, welt i\textit{r} glouben mirs,\\ 
 & der clâre \textbf{Gernout} \textbf{von} Rivirs;\\ 
 & des rîcher vater Narrant\\ 
20 & het im \textbf{verlâzen} \textbf{Ducherlant}.\\ 
 & der brâhte \textbf{im} kocken \textit{ûf dem} mer\\ 
 & ein \textbf{clâre\textit{z} \textit{v}rouwen} her,\\ 
 & \textbf{den} man dâ \textbf{rehter} varwe jach\\ 
 & unde anders \textbf{niht} dâ von \textbf{in} sprach.\\ 
25 & der wâren \textbf{vier} hundert\\ 
 & ze mageden \textbf{ûz gesundert};\\ 
 & \textbf{vier} hundert heten dâ ir man,\\ 
 & ob ichz \textbf{geprüeven} rehte \textbf{kan}.\\ 
 & \textbf{Gernout} viz \textit{c}uns Narant,\\ 
30 & vünf hundert rîter wert erkant\\ 
\end{tabular}
\scriptsize
\line(1,0){75} \newline
G I L M Z Fr18 Fr22 Fr52 \newline
\line(1,0){75} \newline
\textbf{1} \textit{Initiale} G L Fr18 Fr22  \textbf{9} \textit{Initiale} I  \textbf{11} \textit{Initiale} M  \newline
\line(1,0){75} \newline
\textbf{1} vrouwen] freuden I  $\cdot$ al umbehalden] alvmme bihalden M (Fr18) \textbf{2} kan ich] Jch kan L Kan ich uch M Jch kan iv Fr22  $\cdot$ walden] walde Fr18 \textbf{6} ê] noch I y M \textbf{7} lât] So latz Z Zalt Fr22  $\cdot$ michz iu] mich evchz Z \textbf{8} vesten] wazzer vesten Z \textbf{10} Brandelidelin] brandalidelin I Branlidelin L Brandlẏdelin Fr18 Brandlidelin Fr22 \textbf{11} vier] Sehs Z  $\cdot$ clâre] clarer L (Fr22) \textbf{12} etslîchiu] ieslihte Z \textbf{14} prîs] ir pris I \textbf{15} Ponturteise] pontvreise G pontortoeyse I pvntuͯrteise L punturseie M punturteise Z (Fr18) pvntvrtoyse Fr52 \textbf{17} was] was was I  $\cdot$ ir] irz G \textbf{18} Gernout] Gernovt G L Fr18 gernoyt M Bernovt Z gernot Fr52  $\cdot$ von] vnd Fr52  $\cdot$ Rivirs] rimers I Rieviers L ryvirs M [Riu*]: Riuirs Z riuirz Fr52 \textbf{19} rîcher] riche Fr52  $\cdot$ Narrant] narant G narnant Fr52 \textbf{20} het] hat Fr52  $\cdot$ verlâzen] lazzen Z gelazen Fr52  $\cdot$ Ducherlant] duhcherlant I Vngerlant L vcker lant M vcherlant Z ouker lant Fr52 \textbf{21} im] im in I in L Fr52  $\cdot$ uf dem] v̂ber G \textbf{22} ein] ein als I (L) (M) (Z)  $\cdot$ clârez] clare soͮze G \textbf{23} den] Deme M  $\cdot$ rehter] liehter Z \textbf{24} v:d andere von den er sprach Fr52  $\cdot$ in] \textit{om.} M \textbf{25} vier] zwei Z \textbf{26} ze mageden] Zuͯ mageden da L (M) (Fr18) Von landen Fr52 \textbf{27} vier] Zwei Z \textbf{28} op ich rechts gep:::ven kan Fr52  $\cdot$ geprüeven rehte] rechte geprufen L \textbf{29} Gernout] Gernovt G L Fr18 Gernot M Fr52 Bernovt Z  $\cdot$ cuns] Rvns G  $\cdot$ Narant] narrant G I (Fr18) \textbf{30} wert] wirt M  $\cdot$ erkant] bekant Fr52 \newline
\end{minipage}
\hspace{0.5cm}
\begin{minipage}[t]{0.5\linewidth}
\small
\begin{center}*T
\end{center}
\begin{tabular}{rl}
 & was mit vrouwen \textbf{al} umbehalten.\\ 
 & kan ich nû mære walten,\\ 
 & ich sage, wer durch in d\textit{â} was\\ 
 & geher\textit{ber}get ûf daz gras\\ 
5 & \textbf{und} an sîne samenunge komen.\\ 
 & hât ir \textbf{daz} ê \textbf{iht} vernomen,\\ 
 & lât mich\textit{z} iu machen kunt:\\ 
 & \multicolumn{1}{l}{ - - - }\\ 
 & ûz der \textbf{vesten} stat von \textbf{Punt}\\ 
 & brâht im der werde œheim sîn,\\ 
10 & der künec Brandelidelin,\\ 
10 & \multicolumn{1}{l}{ - - - }\\ 
 & \textbf{viere} hundert \textbf{clârer} vrouwen;\\ 
 & \textbf{der} ieglîchiu mohte schouwen\\ 
 & gewâpent dâr ir âmîs\\ 
 & durch rîterschaft und durch prîs.\\ 
15 & die werden Punterteise\\ 
 & wâren wol an dirre reise.\\ 
 & d\textit{â} was, w\textit{e}lt ir geloube\textit{n} mirs,\\ 
 & der clâre \textbf{Bernuot} \textbf{von} Rivirs;\\ 
 & des rîcher vater Narant\\ 
20 & het im \textbf{verlâzen} \textbf{Uckerlant}.\\ 
 & der brâht \textbf{im} \textbf{mit} kocken ûf dem mer\\ 
 & ein \textbf{alsô} \textbf{kreftigez} her,\\ 
 & \textbf{dem} man d\textit{â} \textbf{rehter} varwe jach\\ 
 & und anders dâ von sprach.\\ 
25 & der wâren \textbf{viere} hundert\\ 
 & zuo megden \textbf{d\textit{â}} \textbf{ûz gesundert};\\ 
 & \textbf{viere} hundert heten d\textit{â} ir man,\\ 
 & ob ich ez \textbf{geprüeven} rehte \textbf{kan}.\\ 
 & \textbf{Bernuot} fiz cons Narant,\\ 
30 & vünf hundert rîter wert erkant\\ 
\end{tabular}
\scriptsize
\line(1,0){75} \newline
U V W Q R \newline
\line(1,0){75} \newline
\newline
\line(1,0){75} \newline
\textbf{2} nû] \textit{om.} R \textbf{3} sage] sag v́ch V (W)  $\cdot$ dâ] do U (V) (W) \textbf{4} geherberget] Geherget U  $\cdot$ gras] grvͤne gras V \textbf{5} sîne] \textit{om.} W \textbf{6} daz] dez V (Q) dis W  $\cdot$ iht] niht V (W) (R) ich Q \textbf{7} lât michz] Lant mich U [*ant]: So lant michs V  $\cdot$ machen] e manchen Q \textbf{8} [Vz *]: Vz der wasser vesten stat von pvnt V  $\cdot$ ûz] Vsser R  $\cdot$ vesten] besten W  $\cdot$ von] \textit{om.} R  $\cdot$ Punt] puͦnt U \textbf{10} Brandelidelin] brandelidelein W brandlidelin Q \textbf{11} clârer] klare W Q (R) \textbf{12} der] [D*]: Der V  $\cdot$ ieglîchiu] ieglicher W yegschliche R  $\cdot$ mohte] moͤhte V (W) moche Q \textbf{13} gewâpent] Gafaffett gewapnet R  $\cdot$ dâr] [da*]: da V do W Q R \textbf{14} prîs] hochen pris R \textbf{15} die] Der Q  $\cdot$ Punterteise] puͦnterteise U punterteisen V ponturteise W punturteise Q pontreturtieise R \textbf{16} reise] reisen V \textbf{17} dâ] Do U V Das W Q R  $\cdot$ welt] werlt U  $\cdot$ ir] [*]: ir V irs Q  $\cdot$ gelouben] geleibet U \textbf{18} Bernuot] [ber*]: bernoͮt V gernant Q Sernout R  $\cdot$ Rivirs] Riviris U [*]: Riuirs V riuirs W (R) grűirs Q \textbf{19} rîcher] [rich*]: richer V \textbf{20} Uckerlant] Ockerlant U okerlant V oͤkerlant W ew͑r lant Q och erland R \textbf{21} im mit] im W in in Q im in einen R  $\cdot$ dem] das W \textbf{22} alsô] als ein R  $\cdot$ kreftigez] [clar*]: clares vrowen V creftigen R \textbf{23} dem] Den V W Q  $\cdot$ man] \textit{om.} R  $\cdot$ dâ] do U V W Q doch R  $\cdot$ varwe] frawe Q \textbf{24} dâ von] do niht do von [*]: in V do von in W nicht do von im Q nicht da von im do och R \textbf{25} viere] [*]: zwei V \textbf{26} zuo] Zuͦn W  $\cdot$ dâ] do U W Q R [da*]: da V \textbf{27} viere] [*]: Zwei V  $\cdot$ heten] \textit{om.} R  $\cdot$ dâ] do U V W Q \textbf{28} ez] \textit{om.} W  $\cdot$ geprüeven rehte kan] [geprvͤv*]: geprvͤvet rehte han V recht gepruͯffen kan R \textbf{29} Bernuot] Bernovt V bernayt Q Sernout R  $\cdot$ fiz cons] dez grauen svn V fisz roysz Q \newline
\end{minipage}
\end{table}
\end{document}
