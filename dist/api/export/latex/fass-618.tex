\documentclass[8pt,a4paper,notitlepage]{article}
\usepackage{fullpage}
\usepackage{ulem}
\usepackage{xltxtra}
\usepackage{datetime}
\renewcommand{\dateseparator}{.}
\dmyyyydate
\usepackage{fancyhdr}
\usepackage{ifthen}
\pagestyle{fancy}
\fancyhf{}
\renewcommand{\headrulewidth}{0pt}
\fancyfoot[L]{\ifthenelse{\value{page}=1}{\today, \currenttime{} Uhr}{}}
\begin{document}
\begin{table}[ht]
\begin{minipage}[t]{0.5\linewidth}
\small
\begin{center}*D
\end{center}
\begin{tabular}{rl}
\textbf{618} & \begin{large}C\end{large}linschor \textbf{ist} höfsch und wîs.\\ 
 & der erloubte mir durch sînen prîs\\ 
 & von mîner messenîe erkant\\ 
 & rîterscaft über \textbf{al} sîn lant\\ 
5 & mit manegem stiche und slage.\\ 
 & die ganzen wochen, alle \textbf{ir} tage,\\ 
 & al die wochen in dem jâr,\\ 
 & sunder rote \textbf{ich hân} ze \textbf{vâr},\\ 
 & dise den tac \textbf{unt} jene die naht.\\ 
10 & mit koste ich schaden hân gedâht\\ 
 & Gramoflanze, dem hôch gemuot.\\ 
 & manegen strît er mit \textbf{in} tuot.\\ 
 & waz bewart in ie dar unde?\\ 
 & sînes verhes ich vâren kunde.\\ 
15 & \textbf{Die} wâren ze rîch in mînen solt,\\ 
 & wart mir \textbf{der} decheiner anders holt,\\ 
 & nâch minne ich manegen dienen liez,\\ 
 & dem ich doch lônes niht gehiez.\\ 
 & Mînen lîp gesach nie man,\\ 
20 & i\textbf{ne} m\textit{ö}hte wol \textbf{sînen} dienst hân,\\ 
 & wan einer, der \textbf{truoc} wâpen rôt.\\ 
 & mîn gesinde er brâht in nôt;\\ 
 & vür Logroys er kom geriten.\\ 
 & dâ entworht er si mit \textbf{solhen} siten,\\ 
25 & sîn hant si nider streute,\\ 
 & \textbf{daz} ich mich\textbf{s} wênec vreute.\\ 
 & zwischen Logroys und iwerm urvar\\ 
 & mîner rîter \textbf{im volgeten} vünfe dar.\\ 
 & die enschumpfiert er ûf dem plân\\ 
30 & und gap diu ors dem schifman.\\ 
\end{tabular}
\scriptsize
\line(1,0){75} \newline
D Z Fr68 \newline
\line(1,0){75} \newline
\textbf{1} \textit{Initiale} D Z Fr68  \textbf{15} \textit{Majuskel} D  \textbf{19} \textit{Majuskel} D  \newline
\line(1,0){75} \newline
\textbf{1} Clinschor] Clinscor D Clingezor Z Clinsdior Fr68 \textbf{3} messenîe] maisenie Fr68 \textbf{5} manegem] manegen Fr68 \textbf{6} ir] \textit{om.} Fr68 \textbf{7} alle zit al durh daz iar Fr68 \textbf{8} sunder] sundern Fr68 \textbf{9} unt] \textit{om.} Z  $\cdot$ die] de D \textbf{11} Gramoflanze] Gramoflantz Z \textbf{13} waz] vnd Fr68  $\cdot$ in] sih doh Fr68 \textbf{14} ich] er Z \textbf{15} mit mangen starken ritter stoltz Fr68  $\cdot$ mînen] minnen Z \textbf{16} beide in uelde vnde ce holtz Fr68  $\cdot$ der] \textit{om.} Z \textbf{19} nie man] oh nieman Fr68 \textbf{20} möhte] mohte D Fr68 en moht Z  $\cdot$ sînen] sin Fr68 \textbf{23} Logroys] Logrois Z (Fr68) \textbf{24} si] sih Fr68 \textbf{25} \textit{Versfolge 618.26-25} Z Fr68  \textbf{26} daz] der Fr68  $\cdot$ ich michs] ichs mich Z ih mih Fr68 \textbf{27} Logroys] Logrois Z (Fr68)  $\cdot$ iwerm] uwer Fr68 \textbf{29} enschumpfiert er] entschunfierter Fr68 \newline
\end{minipage}
\hspace{0.5cm}
\begin{minipage}[t]{0.5\linewidth}
\small
\begin{center}*m
\end{center}
\begin{tabular}{rl}
 & Clinsor \textbf{ist} höfsch und wîs.\\ 
 & der er\textit{l}oubt mir durch sînen prîs\\ 
 & von mîne\textit{r} massenîe erkant\\ 
 & ritterschaft über \textbf{alliu} sîn lant\\ 
5 & mit manigem stich und slage.\\ 
 & die ganzen wochen, alle tage,\\ 
 & alle die wochen in dem jâr,\\ 
 & sunder rote \textbf{ich hân} zuo \textbf{vâr},\\ 
 & dise den tac \textbf{und} jene die naht.\\ 
10 & mit kost ich schaden hân gedâht\\ 
 & Gram\textit{o}lanz, dem hôchgemuot.\\ 
 & manigen strît er mit \textbf{in} \dag truoc\dag .\\ 
 & waz bewart  ie dar unde?\\ 
 & sînes verhes ich vâren kunde.\\ 
15 & \textbf{si} wâren ze rîch in mîne\textit{n} solt,\\ 
 & wart mir dekeiner anders holt,\\ 
 & nâch minne ich manigen dienen liez,\\ 
 & dem ich doch lônes niht gehiez.\\ 
 & mînen lîp gesach nie man,\\ 
20 & ich m\textit{ö}hte wol \textbf{sînen} dienst hân,\\ 
 & wan einer, der \textbf{truoc} wâpen rôt.\\ 
 & mîn gesinde er brâht in nôt;\\ 
 & vür L\textit{o}grois \textit{er} kam geriten.\\ 
 & d\textit{â} entworht er si mit \textit{\textbf{solichem}} siten,\\ 
25 & sîn hant si nider ströuwet,\\ 
 & \textbf{daz} ich mich\textbf{s} wênic vröuwet.\\ 
 & zwischen Logrois und iuwerm urvar\\ 
 & mîner ritter \textbf{im volgten} vün\textit{f} dar.\\ 
 & die enschumpfierte  ûf de\textit{m} plân\\ 
30 & und ga\textit{p} diu ros dem schifman.\\ 
\end{tabular}
\scriptsize
\line(1,0){75} \newline
m n o \newline
\line(1,0){75} \newline
\newline
\line(1,0){75} \newline
\textbf{2} erloubt] erbobt m \textbf{3} mîner] mẏne m \textbf{5} stich] schich o \textbf{6} ganzen] gancze m (n) o  $\cdot$ tage] ir tage n o \textbf{11} Gramolanz] Gramonlancz m o Gramolantz n \textbf{12} in] jme n \textbf{14} vâren] forende n \textbf{15} mînen] mẏnnem m mẏnē o \textbf{16} dekeiner] do keiner n dekeiners o \textbf{18} niht] nie o \textbf{20} möhte] mohtte m \textbf{22} gesinde] gesunde o \textbf{23} Logrois] ligrois m  $\cdot$ er] \textit{om.} m >er< o \textbf{24} dâ] Do m n o  $\cdot$ solichem] \textit{om.} m \textbf{26} ich] ichs n \textbf{28} volgten] volgen n  $\cdot$ vünf] fuͯns m \textbf{29} enschumpfierte] entschiempfiete o  $\cdot$ dem] den m n o \textbf{30} gap] gapi m \newline
\end{minipage}
\end{table}
\newpage
\begin{table}[ht]
\begin{minipage}[t]{0.5\linewidth}
\small
\begin{center}*G
\end{center}
\begin{tabular}{rl}
 & \begin{large}C\end{large}linsor \textbf{ist} hövesch unde wîs.\\ 
 & der erloubet mir durch sînen prîs\\ 
 & von mîner massenîe erkant\\ 
 & rîterschaft über \textbf{al} sîn lant\\ 
5 & mit manigem stiche unde slage.\\ 
 & die ganzen wochen, alle \textbf{ir} tage,\\ 
 & al die wochen in dem jâre,\\ 
 & sunder roten \textbf{hân ich} ze \textbf{wâre},\\ 
 & dise den tac, jene die naht.\\ 
10 & mit koste ich schaden hân gedâht\\ 
 & Gramoflanz, deme hôch gemuot.\\ 
 & manigen strît er mit \textbf{mir} tuot.\\ 
 & waz bewart in ie drunde?\\ 
 & sînes verhes ich vâren kunde.\\ 
15 & \textbf{die} wâren ze rîche in mînen solt,\\ 
 & wart mir \textbf{ir} deheiner anders holt,\\ 
 & nâch minne ich manege\textit{n} dienen liez,\\ 
 & dem ich doch lônes niht gehiez.\\ 
 & mînen lîp gesach nie man,\\ 
20 & ich \textbf{en}m\textit{ö}ht wol \textbf{sînen} dienst hân,\\ 
 & wan einer, der \textbf{treit} wâpen rôt.\\ 
 & mîn gesinde er brâhte in nôt;\\ 
 & vür Logroys er kom geriten.\\ 
 & dâ entworhte er si mit \textbf{solhen} siten,\\ 
25 & sîn hant si nider ströute,\\ 
 & \textbf{daz} ich mich\textbf{s} wênic vröute.\\ 
 & zwischen Logroys unde iuwerm urvar\\ 
 & mîner rîter \textbf{volgeten im} vünf\textit{e} dar.\\ 
 & die enschumpfierter ûf de\textit{m} plân\\ 
30 & unde gap diu ors dem schifman.\\ 
\end{tabular}
\scriptsize
\line(1,0){75} \newline
G I L M Z \newline
\line(1,0){75} \newline
\textbf{1} \textit{Initiale} G I L Z  \textbf{17} \textit{Initiale} I  \newline
\line(1,0){75} \newline
\textbf{1} Clinsor] Clinshor G Clinisor L Clingezor Z \textbf{2} erloubet] erlobite M \textbf{4} al] alle M \textbf{5} \textit{Versfolge 618.6-5} M   $\cdot$ mit] Von L \textbf{6} ir] \textit{om.} M \textbf{8} sunder roten] Svnder riten L sunder rotte Z  $\cdot$ hân ich] ich han I M Z \textit{om.} L  $\cdot$ wâre] vare L Z \textbf{11} Gramoflanz] Gramorflanze M Gramoflantz Z \textbf{12} mir] in Z \textbf{13} ie] e M \textbf{14} ich] ich diche I er Z \textbf{15} mînen] minnen Z \textbf{16} ir] \textit{om.} L M Z  $\cdot$ deheiner] dehein G \textbf{17} minne] minnen I  $\cdot$ manegen] manegem G \textbf{19} gesach] \textit{om.} I  $\cdot$ man] nieman L \textbf{20} ich enmöht] Ih en moht G (L) (M) (Z) ich moht I  $\cdot$ sînen] sin G \textbf{21} treit] truͯch L (M) (Z) \textbf{22} mîn gesinde] Wan myn sinde M \textbf{23} Logroys] ligois G logroýs L logrois M \textbf{24} si] sich I \textit{om.} M \textbf{25} \textit{Verfolge 618.26-25} Z  \textbf{26} daz] wenc I  $\cdot$ ich] ichs L Z  $\cdot$ michs] mich I Z  $\cdot$ wênic] sin I \textbf{27} Logroys] logroẏs G Logroýs L logrois M \textbf{28} mîner] Myne M  $\cdot$ volgeten im] yme volgiten M (Z)  $\cdot$ vünfe] funver G \textbf{29} enschumpfierter] entschumpfiert er Z  $\cdot$ dem] den G \textbf{30} diu orse gab er dem [spilman]: shifman I \newline
\end{minipage}
\hspace{0.5cm}
\begin{minipage}[t]{0.5\linewidth}
\small
\begin{center}*T
\end{center}
\begin{tabular}{rl}
 & \begin{large}C\end{large}lynsor, hövesch und wîs,\\ 
 & der erloubete mir durch sînen prîs\\ 
 & von mîner massenîe erkant\\ 
 & rîterschaft über \textbf{al} sîn lant\\ 
5 & mit manegem stiche und slage.\\ 
 & die ganzen wochen, al \textbf{ir} tage,\\ 
 & alle die wochen in dem jâre,\\ 
 & sunder roten \textbf{ich hân} z\textbf{wâre},\\ 
 & dise den tac, jene die naht.\\ 
10 & mit kost ich schaden hân gedâht\\ 
 & Gramoflanz, dem hôch gemuot.\\ 
 & manegen strît er mit \textbf{in} tuot.\\ 
 & waz bewart \textit{in} ie drunde?\\ 
 & sînes verhes ich vâren kunde.\\ 
15 & \textbf{die} wâren zuo rîche in mînen solt,\\ 
 & wart mir dekeiner anders holt,\\ 
 & nâch minne ich manegen dienen liez,\\ 
 & dem ich doch lônes niht gehiez.\\ 
 & mînen lîp gesach nie man,\\ 
20 & ich m\textit{ö}hte wol \textbf{ir} dienst hân,\\ 
 & wan einer, der \textbf{truoc} wâpen rôt.\\ 
 & mîn gesinde er brâhte in nôt;\\ 
 & vür Logrois er kam geriten.\\ 
 & d\textit{â} entworhte\textit{r} si mit \textbf{solichen} siten,\\ 
25 & sîn hant si nider streute,\\ 
 & \textbf{des} ich mich wênic vreute.\\ 
 & zwischen Logrois und iuwerm urvar\\ 
 & mîner rîter \textbf{im volgeten} vünfe dar.\\ 
 & die enschumpfiereter ûf dem plân\\ 
30 & und gap diu ors dem schifman.\\ 
\end{tabular}
\scriptsize
\line(1,0){75} \newline
U V W Q R Fr39 \newline
\line(1,0){75} \newline
\textbf{1} \textit{Initiale} U V W Q Fr39   $\cdot$ \textit{Capitulumzeichen} R  \newline
\line(1,0){75} \newline
\textbf{1} Clynsor] Clinsor V KLynshor W Clinshor Q (R) Fr39  $\cdot$ hövesch] ist hovesch V (W) (Q) (Fr39) ist hofflich R \textbf{2} erloubete] erlaubt W \textbf{4} al sîn] alles sein W alle >seyn< Q \textbf{6} ganzen] ganze V (Q)  $\cdot$ wochen] woche V wuchen vnd R  $\cdot$ ir] \textit{om.} V \textbf{7} die] \textit{om.} V \textbf{8} zwâre] zevare V (Q) (R) (Fr39) \textbf{10} hân] \textit{om.} W \textbf{11} Gramoflanz] Gramaflanz V Gramoflantze W Gramoflantz Q Gramoflancz R \textbf{12} mit in] mir W mit im Q  $\cdot$ tuot] tuͦtte R \textbf{13} waz] Ward R  $\cdot$ in ie] ie U (Q) [*]: in ie  V \textbf{15} wâren] frawen Q  $\cdot$ mînen] dienen Q \textbf{16} anders] ander Q \textbf{18} lônes] lon R lon: Fr39 \textbf{20} ich möhte] Jch mochte U (Fr39) Ich enmoͤcht W Jchn mach Q  $\cdot$ wol] \textit{om.} R  $\cdot$ ir] [*]: sin V seinen Q (R) (Fr39) \textbf{21} einer] einen V  $\cdot$ rôt] rock Q \textbf{23} Logrois] Logroys U (V) [lygroẏs]: logroẏs  Q \textbf{24} dâ] Do U V W Q R Fr39  $\cdot$ entworhter] entworten U  $\cdot$ si] \textit{om.} Q  $\cdot$ mit] nit R  $\cdot$ solichen] solichē V (R) solchem W \textbf{26} des] Das W Q R (Fr39)  $\cdot$ mich] michs Q R Fr39 \textbf{27} Logrois] Lygrois U logroys V Q logris R  $\cdot$ iuwerm] úwern R  $\cdot$ urvar] vberuar W \textbf{28} rîter] \textit{om.} Q  $\cdot$ im volgeten] volgeten W [vogten]: volgten im Q  $\cdot$ vünfe] [*]: fv́nfe V \textbf{29} enschumpfiereter] entschumpffiert er W (Q) (R) (Fr39)  $\cdot$ dem] dē W Q \newline
\end{minipage}
\end{table}
\end{document}
