\documentclass[8pt,a4paper,notitlepage]{article}
\usepackage{fullpage}
\usepackage{ulem}
\usepackage{xltxtra}
\usepackage{datetime}
\renewcommand{\dateseparator}{.}
\dmyyyydate
\usepackage{fancyhdr}
\usepackage{ifthen}
\pagestyle{fancy}
\fancyhf{}
\renewcommand{\headrulewidth}{0pt}
\fancyfoot[L]{\ifthenelse{\value{page}=1}{\today, \currenttime{} Uhr}{}}
\begin{document}
\begin{table}[ht]
\begin{minipage}[t]{0.5\linewidth}
\small
\begin{center}*D
\end{center}
\begin{tabular}{rl}
\textbf{132} & \textbf{\begin{large}E\end{large}rn ruochte}, wâ diu wirtîn saz.\\ 
 & einen guoten kropf er az,\\ 
 & dar nâch er swære trünke tranc.\\ 
 & die vrouwen dûhte gar ze lanc\\ 
5 & sînes wesens in dem poulûn.\\ 
 & si wânde, \textbf{er} wære ein garzûn\\ 
 & gescheiden von den witzen.\\ 
 & ir scham begunde switzen.\\ 
 & iedoch sprach diu herzogîn:\\ 
10 & "junchêrre, ir sult mîn vingerlîn\\ 
 & hie lâzen unt mîn vürspan.\\ 
 & \textbf{hebt} iuch enwec! wan kumt mîn man,\\ 
 & ir müezet \textbf{zürnen} lîden,\\ 
 & daz ir \textbf{gerner} m\textit{ö}htet mîden."\\ 
15 & Dô sprach der knappe wol geborn:\\ 
 & "\textbf{owê}, waz vürht ich iwers mannes zorn?\\ 
 & wan schadet ez iu an êren,\\ 
 & sô wil ich hinnen kêren."\\ 
 & dô \textbf{gieng} er zuo dem bette sân,\\ 
20 & ein ander kus \textbf{dâ wart} getân.\\ 
 & daz was der herzoginne leit.\\ 
 & \textbf{der knappe} ân urloup dannen reit,\\ 
 & iedoch sprach er: "got hüete dîn,\\ 
 & \textbf{alsus} riet \textbf{mir} diu muoter mîn."\\ 
25 & Der knappe des roubes was gemeit.\\ 
 & dô er eine wîle von dan gereit,\\ 
 & \textbf{wol} nâch gein der mîle zil,\\ 
 & dô kom, von dem ich sprechen wil.\\ 
 & \textbf{der} spurte an dem touwe,\\ 
30 & daz gesuochet was sîn vrouwe.\\ 
\end{tabular}
\scriptsize
\line(1,0){75} \newline
D \newline
\line(1,0){75} \newline
\textbf{1} \textit{Initiale} D  \textbf{15} \textit{Majuskel} D  \textbf{25} \textit{Majuskel} D  \newline
\line(1,0){75} \newline
\textbf{14} möhtet] mohtet D \newline
\end{minipage}
\hspace{0.5cm}
\begin{minipage}[t]{0.5\linewidth}
\small
\begin{center}*m
\end{center}
\begin{tabular}{rl}
 & \textbf{\begin{large}E\end{large}r enruochte}, wâ diu wirtîn saz.\\ 
 & einen guoten kropf er az,\\ 
 & dar nâch er \dag were\dag  trünke tranc.\\ 
 & die vrouwen dûhte gar ze lanc\\ 
5 & sînes wesens in der pavelûn.\\ 
 & si w\textit{â}nde, \textbf{er} wære ein garzûn\\ 
 & gescheiden von den witzen.\\ 
 & ir schame begunde switzen.\\ 
 & iedoch sprach diu herzogîn:\\ 
10 & "junchêrre, ir sult mîn vingerlîn\\ 
 & hie lâzen und mîn vürspan.\\ 
 & \textbf{hebt} iuch enwec! wenne kumet mîn man,\\ 
 & ir müezet \textbf{zürnen} lîden,\\ 
 & daz ir \textbf{gerne} m\textit{ö}hte\textit{t} mîden."\\ 
15 & dô sprach der knappe wol geborn:\\ 
 & "waz vörhte ich iuwers mannes zorn?\\ 
 & wan schadet ez iu an \textbf{den} êren,\\ 
 & sô wil ich hinnen kêren."\\ 
 & dô \textbf{gienc} er zuo dem bette sân,\\ 
20 & ein ander kus \textbf{wart d\textit{â}} getân.\\ 
 & daz was der herzoginne leit.\\ 
 & \textbf{der knappe} âne urloup dannen reit,\\ 
 & iedoch sprach er: "got hüete dîn,\\ 
 & \textbf{alsus} riet \textbf{mir} diu muoter mîn."\\ 
25 & \begin{large}D\end{large}er knappe des roubes was gemeit.\\ 
 & dô er ein wîle von dan gereit,\\ 
 & \textbf{wol} nâch gegen der mîle zil,\\ 
 & dô kam, von dem ich sprechen wil.\\ 
 & \textbf{der} sp\textit{ur}te an dem touwe,\\ 
30 & daz gesuochet was sîn vrouwe.\\ 
\end{tabular}
\scriptsize
\line(1,0){75} \newline
m n o \newline
\line(1,0){75} \newline
\textbf{1} \textit{Initiale} m  \textbf{25} \textit{Illustration mit Überschrift:} Also der knappe mit den cleinoͯtern vngesegent von der froͯwen schiet n   $\cdot$ \textit{Initiale} m n o  \newline
\line(1,0){75} \newline
\textbf{2} az] do as n \textbf{3} were] vier n o \textbf{4} vrouwen] frouwe n (o) \textbf{6} wânde] wunde m  $\cdot$ garzûn] ganczurn o \textbf{10} sult] súllen n \textbf{11} vürspan] furspang o \textbf{12} enwec] hin weg n \textbf{13} \textit{Die Verse 132.13-24 fehlen} o  \textbf{14} möhtet] mochtten m \textbf{20} dâ] do m n \textbf{22} dannen] dennan n \textbf{26} gereit] reit o \textbf{27} mîle] milen n o \textbf{29} spurte] sprute m \newline
\end{minipage}
\end{table}
\newpage
\begin{table}[ht]
\begin{minipage}[t]{0.5\linewidth}
\small
\begin{center}*G
\end{center}
\begin{tabular}{rl}
 & \textbf{dône ruohter}, wâ diu wirtîn saz.\\ 
 & einen guoten kropf er az,\\ 
 & dar nâch er swære trünke tranc.\\ 
 & die vrouwen dûhte gar ze lanc\\ 
5 & sînes wesenes in dem pavelûn.\\ 
 & si wânde, \textbf{ez} wære ein garzûn\\ 
 & gescheiden von den witzen.\\ 
 & ir schame begunde switzen.\\ 
 & iedoch sprach diu herzogîn:\\ 
10 & "junchêrre, ir sult mîn vingerlîn\\ 
 & hie lâzen und mîn vürspan.\\ 
 & \textbf{hevet} iuch \textit{e}nwec! wan kumt mîn man,\\ 
 & ir müezet \textbf{zürnen} lîden,\\ 
 & daz ir \textbf{gerne} m\textit{ö}ht mîden."\\ 
15 & dô sprach der knappe wolgeborn:\\ 
 & "\textbf{wie}, waz vürhte ich iwers mannes zorn?\\ 
 & \textit{wan }schade\textit{t} \textit{e}z iu an êren,\\ 
 & sô wil ich hinnen kêren."\\ 
 & dô \textbf{sprang} er gein dem bette sân,\\ 
20 & ein ander kus \textbf{dâ wart} getân.\\ 
 & daz was der herzoginne leit.\\ 
 & âne urloup \textbf{er} dannen reit,\\ 
 & iedoch sprach er: "got hüete dîn,\\ 
 & \textbf{alsô} riet diu muoter mîn."\\ 
25 & der k\textit{n}appe des roubes was gemeit.\\ 
 & dô er eine wîle von dan gereit,\\ 
 & \textbf{vil} nâhen gein der mîle zil,\\ 
 & \begin{large}D\end{large}ô kom, von dem ich sprechen wil.\\ 
 & \textbf{er} spurt an dem touwe,\\ 
30 & daz ges\textit{uoch}et was sîn vrouwe.\\ 
\end{tabular}
\scriptsize
\line(1,0){75} \newline
G I O L M Q R Z Fr35 \newline
\line(1,0){75} \newline
\textbf{1} \textit{Initiale} I L R Z  \textbf{19} \textit{Initiale} I  \textbf{28} \textit{Initiale} G  \newline
\line(1,0){75} \newline
\textbf{1} dône ruohter] Do enrucht er I (Q) Er enruͯchte L Da en ruchter M Nun enruͦchet Jn R Ern rvht Z  $\cdot$ wirtîn] frauwe L \textbf{2} kropf] crofp I chrofpf O craph M  $\cdot$ az] im az O \textbf{4} die vrouwen] diu frowe I (Q)  $\cdot$ gar] al O (M) (Q) es vil R \textbf{5} sînes] Eines Q  $\cdot$ in dem] vnd dem I vnder der L Jn den R \textbf{6} ez] er I O L Q R Z Fr35  $\cdot$ ein] ein wer Z \textbf{8} ir] Jn Z  $\cdot$ switzen] zu swiczen Q \textbf{9} \textit{Versfolge 132.10-9} Z   $\cdot$ iedoch] Jedoch so Z \textbf{11} vürspan] fᵫrspang R \textbf{12} vnde heuet evch hin dan I  $\cdot$ enwec] den wech G \textbf{13} wan chumt min man ir muͤzet zorn liden I  $\cdot$ müezet] muͯssen R  $\cdot$ zürnen] zorn O zwurne Q \textbf{14} ir] irs L  $\cdot$ gerne] gerner I O L (R) Z  $\cdot$ möht] moht G mohtet Z  $\cdot$ mîden] vermiden I \textbf{15} dô] Da M \textbf{16} wie] we I (R) (Z) o\textit{m. } O L M Q Fr35  $\cdot$ vürhte] fúrch R \textbf{17} wan schadet ez] schadet aber ez G Wan schadet úch R  $\cdot$ an] anden I \textbf{19} dô] Da M Z  $\cdot$ gein] zuͯ L \textbf{20} kus dâ] chussen I kusz do Q (R) \textbf{21} daz] Die M  $\cdot$ herzoginne] frawen O (M) herczoginen R kuniginne Fr35 \textbf{22} âne urloup er] Der [chappe]: chnappe an vrlovb O Der knappe ane vrlop L (Q) (Z) (Fr35) Dy knape an orlop M Der knappe oͯne vrlol R \textbf{24} alsô] Alsvs O (L) (M) (Q) (R) (Z) (Fr35)  $\cdot$ riet] riet mir Q Fr35 \textbf{25} knappe] chappe G  $\cdot$ roubes] weybes Q Robens R  $\cdot$ was] wart M \textbf{26} dô] Da M Z  $\cdot$ eine wîle von] wille von Q eine mile von R [weil]: von Z  $\cdot$ gereit] reit I O M (Q) (R) \textbf{27} vil nâhen] Wol L Nache R  $\cdot$ der] einer L (Q) (R) Fr35  $\cdot$ mîle] [*]: wile R \textbf{28} Dô] Da M Z \textbf{29} er] Der L  $\cdot$ spurt] spuͯrte L (M) (R) (Fr35) \textbf{30} gesuochet] geschoͮwet G  $\cdot$ sîn] die R \newline
\end{minipage}
\hspace{0.5cm}
\begin{minipage}[t]{0.5\linewidth}
\small
\begin{center}*T (U)
\end{center}
\begin{tabular}{rl}
 & \textbf{d\textit{ô} enruocht er}, wâ diu wirtîn saz.\\ 
 & einen guoten kropf er az,\\ 
 & dar nâch er swære trünke tranc.\\ 
 & die vrouwen dûhte gar zuo lanc\\ 
5 & sînes wesens in dem pavelûn.\\ 
 & si wânt, \textbf{ez} wære ein garzûn\\ 
 & gescheiden von den witzen.\\ 
 & ir schame begunde switzen.\\ 
 & iedoch sprach diu herzogîn:\\ 
10 & "junchêrre, ir solt mîn vingerlîn\\ 
 & hie lâzen und mîn vürspan.\\ 
 & \textbf{habt} iuch e\textit{n}we\textit{c}! wan kumt mîn man,\\ 
 & ir müezet \textbf{zorn} lîden,\\ 
 & daz ir \textbf{gerner} möhtet mîden."\\ 
15 & dô sprach der knappe wol geborn:\\ 
 & "\textbf{wie}, waz \textit{vörhte ich} iuwers mannes zorn?\\ 
 & wan schadet ez iu an êren,\\ 
 & sô wil ich hinnen kêren."\\ 
 & dô \textbf{sprang} er gein dem bette sân,\\ 
20 & ein ander kus \textbf{dâ wart} getân.\\ 
 & daz was der herzogîn leit.\\ 
 & âne urloup \textbf{er} dannen reit,\\ 
 & iedoch sprach er: "got hüete dîn,\\ 
 & \textbf{alsus} riet diu muoter mîn."\\ 
25 & \begin{large}D\end{large}er knabe des roubes was gemeit.\\ 
 & dô er eine wîle von dan gereit,\\ 
 & \textbf{vil} nâhe gein der mîlen zil,\\ 
 & dô kam, von \textit{dem} ich spr\textit{e}chen wil.\\ 
 & \textbf{er} spurte anme touwe,\\ 
30 & daz gesuochet was sîn vrouwe.\\ 
\end{tabular}
\scriptsize
\line(1,0){75} \newline
U V W T \newline
\line(1,0){75} \newline
\textbf{1} \textit{Initiale} W T  \textbf{4} \textit{Majuskel} T  \textbf{6} \textit{Majuskel} T  \textbf{9} \textit{Majuskel} T  \textbf{16} \textit{Majuskel} T  \textbf{19} \textit{Majuskel} T  \textbf{21} \textit{Majuskel} T  \textbf{23} \textit{Majuskel} T  \textbf{25} \textit{Initiale} U V   $\cdot$ \textit{Majuskel} T  \newline
\line(1,0){75} \newline
\textbf{1} \textit{Vers 132.1 wohl vom Schreiber nachträglich ergänzt} U   $\cdot$ dô] Da U  $\cdot$ enruocht er] enrvͦcht in V (W) (T)  $\cdot$ wirtîn] fv́rstin V \textbf{2} az] doch aß W \textbf{4} vrouwen] frowe V \textbf{5} wesens] wesen W \textbf{6} ez] [*]: er V er W \textbf{9} sprach] so sprach V \textbf{10} solt] soͤllen V \textbf{12} habt] Hebent W (T)  $\cdot$ enwec] ein wer U  $\cdot$ wan] vnd W \textbf{13} müezet] muͤssen W \textbf{14} ir gerner] irs gerne V  $\cdot$ möhtet] mochtet U (T) \textbf{15} dô] so T \textbf{16} wie] We V W \textit{om.} T  $\cdot$ vörhte ich] \textit{om.} U \textbf{17} êren] [*]: eren V eúwern ern W \textbf{20} dâ wart] wart do V (W) \textbf{22} der knappe ane [vr*]: vrloͮp dannen reit T \textbf{24} riet] riet mir V hieß mich W \textbf{25} Der was des ruͦbes gemeit T  $\cdot$ des roubes] [*]: dez roͮbes V seins raubes W \textbf{26} dô] da T \textbf{27} der] [*]: einer V einer W T \textbf{28} dem] \textit{om.} U  $\cdot$ sprechen] sprchen U nun sprechen W \newline
\end{minipage}
\end{table}
\end{document}
