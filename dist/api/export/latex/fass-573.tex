\documentclass[8pt,a4paper,notitlepage]{article}
\usepackage{fullpage}
\usepackage{ulem}
\usepackage{xltxtra}
\usepackage{datetime}
\renewcommand{\dateseparator}{.}
\dmyyyydate
\usepackage{fancyhdr}
\usepackage{ifthen}
\pagestyle{fancy}
\fancyhf{}
\renewcommand{\headrulewidth}{0pt}
\fancyfoot[L]{\ifthenelse{\value{page}=1}{\today, \currenttime{} Uhr}{}}
\begin{document}
\begin{table}[ht]
\begin{minipage}[t]{0.5\linewidth}
\small
\begin{center}*D
\end{center}
\begin{tabular}{rl}
\textbf{573} & \textit{\begin{large}N\end{large}}û was im sîn houbet\\ 
 & mit würfen sô betoubet,\\ 
 & unt dô sîne wunden\\ 
 & \textbf{sô} bluoten begunden,\\ 
5 & daz in sîn snellîch kraft\\ 
 & gar liez mit \textbf{ir} geselleschaft:\\ 
 & Durch swindeln er strûchens pflac;\\ 
 & daz \textbf{houbt im} ûf dem lewen lac.\\ 
 & der schilt viel nider under in.\\ 
10 & gewan er ie kraft oder sin,\\ 
 & diu wâren im beidiu enpfüeret;\\ 
 & unsanfte er \textbf{was} gerüeret.\\ 
 & Aller s\textit{i}n tet im entwîch.\\ 
 & sîn wanküssen ungelîch\\ 
15 & was dem, daz Gymele\\ 
 & von Monte Rybele,\\ 
 & diu süeze \textbf{unt} diu wîse,\\ 
 & legete Kahenise,\\ 
 & dâr ûffe er sînen prîs verslief.\\ 
20 & der prîs gein disem manne lief,\\ 
 & wande \textbf{ir habt daz} wol vernomen,\\ 
 & wâ mit er was von witzen komen,\\ 
 & daz er lac unversunnen,\\ 
 & wie des wart begunnen.\\ 
25 & verholne \textbf{ez} wart beschouwet,\\ 
 & daz mit bluote \textbf{was} \textbf{betouwet}\\ 
 & der kemenâten estrîch.\\ 
 & si bêde dem tôde wâren gelîch,\\ 
 & der lewe unt Gawan.\\ 
30 & ein juncvrouwe wolgetân\\ 
\end{tabular}
\scriptsize
\line(1,0){75} \newline
D \newline
\line(1,0){75} \newline
\textbf{1} \textit{Initiale} D  \textbf{7} \textit{Majuskel} D  \textbf{13} \textit{Majuskel} D  \newline
\line(1,0){75} \newline
\textbf{1} Nû] ÷v D \textbf{13} sin] sîn D \textbf{15} Gymele] Gymêle D \textbf{16} Rybele] Rybêle D \textbf{18} Kahenise] Kahenîse D \newline
\end{minipage}
\hspace{0.5cm}
\begin{minipage}[t]{0.5\linewidth}
\small
\begin{center}*m
\end{center}
\begin{tabular}{rl}
 & \begin{large}N\end{large}û was ime sîn houbet\\ 
 & mit würfen sô be\textit{toubet},\\ 
 & und dô sîne wunden\\ 
 & \textbf{sô} bluoten begunden,\\ 
5 & daz in sîn snellîch kraft\\ 
 & gar liez mit \textbf{ir} geselleschaft:\\ 
 & durch swindeln er strûchen\textit{s} pflac,\\ 
 & daz \textbf{im daz houbt} ûf dem lewen lac.\\ 
 & der schilt viel nider under in.\\ 
10 & gewan er ie kraft oder sin,\\ 
 & diu wâren im beidiu enpfüeret;\\ 
 & unsanft er \textbf{wart} gerüeret,\\ 
 & \textbf{wan} aller sin tete im entwîch.\\ 
 & sîn wa\textit{n}küssen ungelîch\\ 
15 & was dem, daz Gymmele\\ 
 & von Munt Ribbele,\\ 
 & diu süeze, diu wîse,\\ 
 & legte Kahenise,\\ 
 & dâr ûf er sînen prîs verslief.\\ 
20 & der prîs gegen disem manne lief,\\ 
 & wan \textbf{\textit{i}r habt daz} wol vernomen,\\ 
 & wâ mit er was von witzen komen,\\ 
 & daz er lac unversunnen,\\ 
 & wie des wart begunnen.\\ 
25 & verholn wart beschouwet,\\ 
 & daz mit bluote \textbf{wart} \textbf{betouwet}\\ 
 & der kemenâten estrîch.\\ 
 & si beide dem tôde wâren gelîch,\\ 
 & der lewe und Gawan.\\ 
30 & ein juncvrouwe wolgetân\\ 
\end{tabular}
\scriptsize
\line(1,0){75} \newline
m n o \newline
\line(1,0){75} \newline
\textbf{1} \textit{Initiale} m n  \newline
\line(1,0){75} \newline
\textbf{2} betoubet] bedaht m bet:ibt o \textbf{4} sô] Sie o  $\cdot$ begunden] begúnden o \textbf{6} liez] liesse n \textbf{7} swindeln] swindelin n  $\cdot$ strûchens] struchen m n o \textbf{10} gewan] Gawan o \textbf{11} enpfüeret] enphoret o \textbf{12} wart] was n (o) \textbf{14} wanküssen] want kuͯssen m (n) (o) \textbf{15} Gymmele] gẏmmele m gimmele n gammuͯle o \textbf{16} Ribbele] ribbile o \textbf{21} ir habt] erhabt m \textbf{26} wart] was n (o) \newline
\end{minipage}
\end{table}
\newpage
\begin{table}[ht]
\begin{minipage}[t]{0.5\linewidth}
\small
\begin{center}*G
\end{center}
\begin{tabular}{rl}
 & \begin{large}N\end{large}û was im sîn houbet\\ 
 & mit würfen sô betoubet,\\ 
 & unde dô sîne wunden\\ 
 & \textbf{sô} bluoten begunden,\\ 
5 & daz in sîn snelliclîch kraft\\ 
 & gar liez mit \textbf{ir} geselleschaft:\\ 
 & durch swindel\textit{n} er strûchens pflac;\\ 
 & daz \textbf{houbet im} ûf dem lewen lac.\\ 
 & der schilt viel nider under in.\\ 
10 & gewan er ie kraft ode sin,\\ 
 & diu wâren i\textit{m} beidiu enpfüeret;\\ 
 & unsanfter \textbf{was} gerüeret.\\ 
 & aller sin tet im entwîc\textit{h}.\\ 
 & sîn wanküsse ungelîch\\ 
15 & was dem, daz Gimmele\\ 
 & von Monte Rippele,\\ 
 & diu süeze \textbf{unde} diu wîse,\\ 
 & leite Keinise,\\ 
 & dâr ûf er sînen brîs verslief.\\ 
20 & der brîs gein disem manne lief,\\ 
 & wan \textbf{ir habet daz} wol vernomen,\\ 
 & wâ mit er was von witzen komen,\\ 
 & daz er lac unversunnen,\\ 
 & wie des wart begunnen.\\ 
25 & verholne \textbf{ez} wart beschouwet,\\ 
 & daz mit bluote \textbf{was} \textbf{betouwet}\\ 
 & der kemenâten estrîch.\\ 
 & si bêde dem tôde wâren gelîch,\\ 
 & der lewe unde Gawan.\\ 
30 & ein juncvrouwe wolgetân\\ 
\end{tabular}
\scriptsize
\line(1,0){75} \newline
G I L M Z Fr23 \newline
\line(1,0){75} \newline
\textbf{1} \textit{Initiale} G L Z Fr23  \textbf{9} \textit{Initiale} I  \newline
\line(1,0){75} \newline
\textbf{1} Nû] Do Fr23 \textbf{2} sô] also I \textbf{3} dô] da M Z  $\cdot$ sîne] si Fr23 \textbf{4} sô] \textit{om.} Fr23 \textbf{5} in] im Fr23  $\cdot$ snelliclîch] Menliche M \textbf{6} gar liez mit] Lie gar bi L  $\cdot$ ir] \textit{om.} M Fr23 \textbf{7} swindeln] swindelns G  $\cdot$ strûchens] struches M \textbf{8} im] \textit{om.} Z in Fr23  $\cdot$ dem] den Fr23  $\cdot$ lac] bilac M \textbf{11} im beidiu] in beidiv G (Fr23) bede im I  $\cdot$ enpfüeret] einpfuͯret L \textbf{12} unsanfter] vnsanfte I \textbf{13} aller sin] aller sin sîn I Al sin sin L (M) (Fr23) Alle sin Z  $\cdot$ tet] was I  $\cdot$ entwîch] entwic G entwichen I \textbf{14} wanküsse] vanchnisze waz L wanchvssen Z  $\cdot$ ungelîch] vngeblichen I \textbf{15} was] \textit{om.} L  $\cdot$ dem] \textit{om.} Fr23  $\cdot$ daz] \textit{om.} M  $\cdot$ Gimmele] Gamille I Gýmmele L Gymmile M giminele Z Gininule Fr23 \textbf{16} von Monte] von munt I Vomonte Fr23  $\cdot$ Rippele] ribille I Ribbele L Z kybile M Ribale Fr23 \textbf{18} leite] Lech L  $\cdot$ Keinise] keimise I Kachenise L kahanise M kahenise Z (Fr23) \textbf{19} sînen] sin Fr23  $\cdot$ verslief] verslief danne Fr23 \textbf{20} lief] \textit{om.} Fr23 \textbf{21} wan] Wa M  $\cdot$ wol] \textit{om.} Fr23 \textbf{22} wâ mit] Wan Fr23  $\cdot$ witzen] witze Fr23 \newline
\end{minipage}
\hspace{0.5cm}
\begin{minipage}[t]{0.5\linewidth}
\small
\begin{center}*T
\end{center}
\begin{tabular}{rl}
 & Nû was im sîn houbt\\ 
 & mit würfen sô betoubt,\\ 
 & und dô sîne wunden\\ 
 & bluoten begunden,\\ 
5 & daz in sîn snellîch kraft\\ 
 & gar liez mit geselleschaft:\\ 
 & durc\textit{h s}windele\textit{n} er strûchens pflac;\\ 
 & daz \textbf{houbt im} ûf dem lewen lac.\\ 
 & der schilt viel nider under in.\\ 
10 & g\textit{e}wan er \textit{i}e kraft oder sin,\\ 
 & diu wâren im beidiu enpfüert;\\ 
 & unsanft er \textbf{was} gerüert.\\ 
 & al\textbf{sîn} sin tet im entwîc\textit{h}.\\ 
 & sîn wanküssen ungelîch\\ 
15 & was dem, daz Gimile\\ 
 & von Monte Ribile,\\ 
 & diu süeze \textbf{und} diu wîse,\\ 
 & legte Kahenise,\\ 
 & dâr ûf er sînen prîs verslief.\\ 
20 & der prîs gên disem manne lief,\\ 
 & wan \textbf{daz habt ir} wol vernomen,\\ 
 & wâ mit er was von witzen komen,\\ 
 & daz er lac unversunnen,\\ 
 & wie des wart begunnen.\\ 
25 & verholen \textbf{e\textit{z}} wart beschouwet,\\ 
 & daz mit bluot \textbf{was} \textbf{getouwet}\\ 
 & der kemenâten estrîch.\\ 
 & si bêde dem tôde wâren glîch,\\ 
 & der lewe und Gawan.\\ 
30 & ein juncvrou wol getân\\ 
\end{tabular}
\scriptsize
\line(1,0){75} \newline
Q R W V U \newline
\line(1,0){75} \newline
\textbf{1} \textit{Initiale} Q R W V  \newline
\line(1,0){75} \newline
\textbf{1} \textit{Die Verse 553.1-599.30 fehlen} U  \textbf{2} sô] svz V \textbf{4} bluoten] So bluͯtten R (W) (V) \textbf{5} in] im R  $\cdot$ snellîch] manlich R \textbf{7} durch swindelen] Durch sweigens swindelem Q  $\cdot$ strûchens] struches R \textbf{8} houbt im] [*]: imz hoͮbet V  $\cdot$ lac] [*]: gelag V \textbf{10} gewan er ie] Gawan er nie Q \textbf{11} im] \textit{om.} R \textbf{13} alsîn sin tet] Alle sine sin tatten R Alsin sin [*]: tet V Alle sein sinn thet W  $\cdot$ entwîch] entweichen Q (R) \textbf{14} wanküssen] banckússen W  $\cdot$ ungelîch] vngelichen R \textbf{15} daz Gimile] gimille R grayle W da Gẏmele V \textbf{16} Ribile] rabile R rybile W ribele V \textbf{18} Kahenise] kahaneise W [kanhe*]: kanhenise V \textbf{21} daz habt ir] ir habt das R (W) (V) \textbf{22} mit] von R \textbf{23} unversunnen] vnuernomen W \textbf{24} des] das R W (V) \textbf{25} ez] er Q  $\cdot$ beschouwet] geschowet R \textbf{26} getouwet] betowet R (W) V \textbf{28} bêde] beidu R \textbf{29} und] vnd auch W \newline
\end{minipage}
\end{table}
\end{document}
