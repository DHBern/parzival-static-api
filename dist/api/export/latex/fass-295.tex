\documentclass[8pt,a4paper,notitlepage]{article}
\usepackage{fullpage}
\usepackage{ulem}
\usepackage{xltxtra}
\usepackage{datetime}
\renewcommand{\dateseparator}{.}
\dmyyyydate
\usepackage{fancyhdr}
\usepackage{ifthen}
\pagestyle{fancy}
\fancyhf{}
\renewcommand{\headrulewidth}{0pt}
\fancyfoot[L]{\ifthenelse{\value{page}=1}{\today, \currenttime{} Uhr}{}}
\begin{document}
\begin{table}[ht]
\begin{minipage}[t]{0.5\linewidth}
\small
\begin{center}*D
\end{center}
\begin{tabular}{rl}
\textbf{295} & \begin{large}K\end{large}eie hurte vaste an in\\ 
 & \textbf{unt} drang imz or\textit{s} alumbe hin,\\ 
 & \textbf{unze} daz der Waleis übersach\\ 
 & sîn süeze \textbf{\textit{s}ûrez} ungemach,\\ 
5 & sînes wîbes \textbf{gelîchen} schîn,\\ 
 & von Pelrapeire der künegîn,\\ 
 & ich meine den geparrierten snê.\\ 
 & dô kom aber \textbf{vrou} witze als ê,\\ 
 & diu im \textbf{den sin her} wider gap.\\ 
10 & Keie \textbf{daz ors liez} in den walap;\\ 
 & \textbf{der} kom durch tjustieren her.\\ 
 & von rabîne sancten si diu sper.\\ 
 & Keie sîne tjoste brâhte,\\ 
 & als im der ougen mez gedâhte,\\ 
15 & durch \textbf{Parzivalen} schilt ein venster wît.\\ 
 & im wart vergolden dirre strît.\\ 
 & Keie, Artuses scheneschalt,\\ 
 & ze gegentjoste wart gevalt\\ 
 & über den ronen, dâ diu gans entran,\\ 
20 & sô daz daz ors unt der man\\ 
 & \textbf{liten beidiu samt} nôt:\\ 
 & der man \textbf{wart} wunt, daz ors lac tôt.\\ 
 & zwischen dem satelbogen unt eime stein\\ 
 & \textbf{Keie} \textbf{der} zeswe arm \textbf{unt} \textbf{daz} winster bein\\ 
25 & \textbf{zerbrach} von disem gevelle.\\ 
 & surzengel, satel, geschelle\\ 
 & von dirre hurt gar zerbrast.\\ 
 & sus galt zwei bliwen \textbf{der} gast:\\ 
 & daz eine leit ein maget durch in,\\ 
30 & mit dem andern muoser selbe sîn.\\ 
\end{tabular}
\scriptsize
\line(1,0){75} \newline
D \newline
\line(1,0){75} \newline
\textbf{1} \textit{Initiale} D  \newline
\line(1,0){75} \newline
\textbf{2} ors] ore D \textbf{4} sûrez] fv̂rez D \textbf{17} Artuses] Artvs D \newline
\end{minipage}
\hspace{0.5cm}
\begin{minipage}[t]{0.5\linewidth}
\small
\begin{center}*m
\end{center}
\begin{tabular}{rl}
 & \begin{large}K\end{large}eie h\textit{u}rte vaste an in\\ 
 & \textbf{und} dranc imz ros alumbe hin,\\ 
 & \textbf{unz} daz der Waleis übersach\\ 
 & sîn süeze \textbf{sûrez} ungemach,\\ 
5 & sînes wîbes \textbf{glîchen} schîn,\\ 
 & von Pelraperie der künigîn,\\ 
 & ich meine den geparrierten snê.\\ 
 & dô kam aber \textbf{vrouwe} witze als ê,\\ 
 & diu im \textbf{den sin her} wider gap.\\ 
10 & Keie \textbf{daz ros liez} in den walap;\\ 
 & \textbf{der} kam durch justieren her.\\ 
 & von rabbîne san\textit{c}ten si diu sper.\\ 
 & Keie sîne juste brâhte,\\ 
 & als i\textit{m} der ougen mez gedâhte,\\ 
15 & durch \textbf{des Waleises} schilt ein venster wît.\\ 
 & im wart vergolten dirre strît.\\ 
 & Keie, Artuses sc\textit{i}n\textit{isca}lt,\\ 
 & ze gegenjuste wart gevalt\\ 
 & über den ronen, dar diu gans entran,\\ 
20 & sô daz daz ros und der man\\ 
 & \textbf{liten beid\textit{iu} sament} nôt:\\ 
 & der man \textbf{wart} wunt, daz ros lac tôt.\\ 
 & zwischen dem satelbogen und eine\textit{m} stein\\ 
 & \textbf{Keien} \textbf{der} zesewe arm, \textbf{daz} winster bein\\ 
25 & \textbf{zerbrach} von disem ge\textit{v}elle.\\ 
 & surzengel, satel, geschelle\\ 
 & von dirre h\textit{u}rte gar zerbr\textit{a}st.\\ 
 & sus galt zwei bl\textit{i}u\textit{w}en \textbf{dô} \textbf{der} gast:\\ 
 & daz eine leit ein maget durch in,\\ 
30 & mit dem anderen muose er selbe sîn.\\ 
\end{tabular}
\scriptsize
\line(1,0){75} \newline
m n o Fr69 \newline
\line(1,0){75} \newline
\textbf{1} \textit{Initiale} m   $\cdot$ \textit{Capitulumzeichen} n  \newline
\line(1,0){75} \newline
\textbf{1} Keie] Keẏe n Keẏ o  $\cdot$ hurte] herte m kerte n o \textbf{2} imz] vmmb o \textbf{6} Pelraperie] pelraberie m pelrapeir n pelrapier o \textbf{7} geparrierten] parieten o \textbf{9} diu] Vnd n o \textbf{10} Keie] Keẏe n Keẏ o  $\cdot$ in] ẏm o \textbf{12} sancten] santen m n santten o  $\cdot$ si diu] vch myn o \textbf{13} Keie] Keye n Keẏ o \textbf{14} im] ẏ m \textbf{15} des] dasz o  $\cdot$ Waleises] waleis m waleisz n o \textbf{17} Keie] Keẏe n Keẏ o  $\cdot$ Artuses] artus m n artuͯs o  $\cdot$ sciniscalt] scanfanlt m scuͯntscalt o \textbf{19} über] Vwer o Ob Fr69  $\cdot$ ronen] ronen das n roner o  $\cdot$ gans] gancz o \textbf{21} beidiu sament] beiden sament m \textbf{23} einem] einen m \textbf{24} Keien] Keẏe n Keẏ o \textbf{25} gevelle] geselle m \textbf{26} surzengel] Gurtzúgel n Gurczingel o \textbf{27} hurte] herte m  $\cdot$ gar zerbrast] garzerbrachst m (n) \textbf{28} bliuwen] blumen m  $\cdot$ gast] glast o \textbf{30} dem] den o  $\cdot$ muose] muͯsse m múste n (o)  $\cdot$ selbe] selbes n \newline
\end{minipage}
\end{table}
\newpage
\begin{table}[ht]
\begin{minipage}[t]{0.5\linewidth}
\small
\begin{center}*G
\end{center}
\begin{tabular}{rl}
 & Kay hurte vaste an in;\\ 
 & \textbf{er} dranc imz ors alumbe hin,\\ 
 & \textbf{sô} daz der Waleis übersach\\ 
 & sîn süeze \textbf{sûrez} ungemach,\\ 
5 & sînes wîbes \textbf{gelîchen} schîn,\\ 
 & von Pelrapeire der künigîn,\\ 
 & ich meine den geparrierten snê.\\ 
 & dô kom \textbf{im} aber witze als ê,\\ 
 & diu im \textbf{den sin her} wider gap.\\ 
10 & Kay \textbf{lie daz ors} in den walap;\\ 
 & \textbf{der} kom durch tjostieren her.\\ 
 & von rabîne sancten si diu sper.\\ 
 & Kay sîne tjost brâhte,\\ 
 & als im der ougen mez gedâhte,\\ 
15 & \begin{large}D\end{large}urch \textbf{des Waleises} schilt ein venster wît.\\ 
 & im wart vergolten dirre strît.\\ 
 & Kay, Artuses seneschalt,\\ 
 & ze gegentjoste wart gevalt\\ 
 & über den ronen, dâ diu gans entran,\\ 
20 & sô daz daz ors unde der man\\ 
 & \textbf{beidiu sament liten} nôt:\\ 
 & der man \textbf{was} wunt, daz ors lac tôt.\\ 
 & zwischeme satelbogen \textit{unde} einem stein\\ 
 & \textbf{Kay} zeswer arm \textbf{unde} \textbf{sîn} winster bein\\ 
25 & \textbf{zerbrast} von disem gevelle.\\ 
 & surzengel \textbf{unde} satel, geschelle\\ 
 & von dirre hurte gar zerbrast.\\ 
 & sus galt zwei bliwen \textbf{der} gast:\\ 
 & daz eine leit ein maget durch in,\\ 
30 & mit dem anderen muoser \textit{selbe} sîn.\\ 
\end{tabular}
\scriptsize
\line(1,0){75} \newline
G I O L M Q R Z \newline
\line(1,0){75} \newline
\textbf{1} \textit{Initiale} L  \textbf{7} \textit{Initiale} Z  \textbf{9} \textit{Überschrift:} Da streit parczifal mit keẏ nach dem vns er wider zu Im selbs kam Von versinen siner amis R   $\cdot$ \textit{Initiale} O R  \textbf{15} \textit{Initiale} G  \textbf{17} \textit{Initiale} I   $\cdot$ \textit{Capitulumzeichen} L  \textbf{29} \textit{Initiale} R  \newline
\line(1,0){75} \newline
\textbf{1} Kay] kaẏ G kain I Key O M R Z Kayet Q  $\cdot$ hurte] hvrt O (Z) hertte R \textbf{2} er] Vnd Z  $\cdot$ dranc] karte M  $\cdot$ imz] das M \textbf{3} daz] daz es L  $\cdot$ Waleis] waleise O waleisz L  $\cdot$ übersach] daz úbersach R \textbf{4} sûrez] sulersz M [sofers]: sofres Q sines R \textbf{5} sînes] Sein Q  $\cdot$ gelîchen] gelicher R \textbf{6} Pelrapeire] pailrapeir I Pelrapæire O  $\cdot$ der] [diu]: der I div O (L) (Q) \textbf{7} ich meine den] Jn dem Z  $\cdot$ geparrierten] parrierten O gepartirten Q \textbf{8} im] \textit{om.} O L M Q R Z  $\cdot$ witze] wisser M frowe witze Z \textbf{9} diu] do diu I ÷iv O  $\cdot$ den] \textit{om.} O  $\cdot$ her] er Q \textbf{10} Kay] kaẏ G Key O Q Z Keie M Keẏ R  $\cdot$ daz] sin Z \textbf{11} tjostieren] strite R \textbf{12} rabîne] raben O  $\cdot$ sancten] sagite M \textbf{13} Kay] kaẏ G kain I Key O Q Z Keie M Keẏ R \textbf{14} mez] maz M (Q) Z  $\cdot$ gedâhte] gidachten M \textbf{15} des] \textit{om.} Q R  $\cdot$ Waleises] [w*]: waleis G waleisen I waleis O M Q R Z waleýsz L \textbf{16} wart] \textit{om.} Z  $\cdot$ dirre] der O L M Q (R) \textbf{17} Kay] kaẏ G Kayn I Key O Q R Z Keie M  $\cdot$ Artuses] artus M Q (R)  $\cdot$ seneschalt] sineschalt G schinishalt I senetschalt O sinatshalt L sinetscalt M senecschafft Q sy entschalt R sinehtschalt Z \textbf{18} ze gegentjoste] wart zegege Tiost I Ze gegen strit R \textbf{19} den] die L \textit{om.} Z  $\cdot$ ronen] runnen M  $\cdot$ dâ] do Q \textbf{20} sô daz] Da M  $\cdot$ unde der] vnder dem I \textbf{21} liten] liden O M Q Z \textbf{22} was] lac Z  $\cdot$ lac] was O (R) Z \textbf{23} zwischeme] zwishen den I (M) Zwischen L (Q)  $\cdot$ satelbogen] efaltelbogen Q  $\cdot$ unde] \textit{om.} G I  $\cdot$ einem] ein I \textbf{24} Kay] kaẏ G kain I [Kys]: Keys O Kayen L Keyn M Z Key R  $\cdot$ zeswer] ins Rechtter R  $\cdot$ arm] hant Q  $\cdot$ unde] \textit{om.} O Q R  $\cdot$ sîn] \textit{om.} I daz L  $\cdot$ winster] linges R \textbf{25} zerbrast] Brast L Ezn brast M Zu brach Q \textbf{26} surzengel] Suͯrzingen L Fur zingel Q  $\cdot$ unde] \textit{om.} R \textbf{27} dirre] dir L  $\cdot$ hurte] burde M  $\cdot$ zerbrast] zu brach Q \textbf{28} der] diser I (Z) \textbf{29} eine leit ein maget] ein maget leit I \textbf{30} mit] \textit{om.} O  $\cdot$ dem] den M  $\cdot$ muoser] mvͦse O  $\cdot$ selbe] \textit{om.} G I selben M selber R \newline
\end{minipage}
\hspace{0.5cm}
\begin{minipage}[t]{0.5\linewidth}
\small
\begin{center}*T
\end{center}
\begin{tabular}{rl}
 & Key hurte vaste an in;\\ 
 & \textbf{er} dranc im daz ors alumbe hin,\\ 
 & \textbf{sô} daz der Waleis übersach\\ 
 & sîn süeze \textbf{senftez} ungemach,\\ 
5 & sînes wîbes \textbf{glîcher} schîn,\\ 
 & von Peilrapere der künegîn,\\ 
 & ich meine den geparrierten snê.\\ 
 & dô kom aber witze als ê,\\ 
 & di\textit{u} im \textbf{sîn herze} wider gap.\\ 
10 & Key \textbf{lie daz ors} in den walap\\ 
 & \textbf{unde} kom durch tjostieren her.\\ 
 & von rabîne sancten si di\textit{u} sper.\\ 
 & Key sîne tjost brâhte,\\ 
 & als im der ougen mez gedâhte,\\ 
15 & durch \textbf{des Waleises} schilt ein venster wît.\\ 
 & im wart vergolten dirre strît.\\ 
 & Key, Artuses seneschalt,\\ 
 & ze gegentjost wart gevalt\\ 
 & über den ronen, dâ diu gans entran,\\ 
20 & sô daz daz ors unde der man\\ 
 & \textbf{beid\textit{iu} samt liten} nôt:\\ 
 & der man \textbf{wart} wunt, daz ors lac tôt.\\ 
 & zwischen dem satelbogen unde einem stein\\ 
 & \textbf{Keys} zesewe arm, \textbf{sîn} winster bein\\ 
25 & \textbf{zerbrast} von disem gevelle.\\ 
 & surzengel \textbf{unde} satel, geschelle\\ 
 & von dirre hurt gar zerbrast.\\ 
 & Sus galt zwei bliuwen \textbf{dirre} gast:\\ 
 & daz eine leit ein maget durch in,\\ 
30 & mit dem andern muose \textit{er selbe} sîn.\\ 
\end{tabular}
\scriptsize
\line(1,0){75} \newline
T U V W \newline
\line(1,0){75} \newline
\textbf{21} \textit{Initiale} V  \textbf{28} \textit{Majuskel} T  \newline
\line(1,0){75} \newline
\textbf{1} Key] Kêy T Kein V  $\cdot$ hurte] horte U \textbf{2} im daz] iz im U in vnd W  $\cdot$ alumbe hin] alhin W \textbf{3} Waleis] walleiz V \textbf{4} senftez] [*]: sures V saures W \textbf{5} glîcher] glichen V \textbf{6} Peilrapere] Peilrapeire T [pe*]: pelrepere V pelrapeir W  $\cdot$ der] die W \textbf{7} geparrierten] gepartierten W \textbf{8} witze] vroͮ witze V \textbf{9} diu] die T  $\cdot$ sîn herze wider] sin her wider U den sin her wider V (W) \textbf{10} Key] Keẏn V \textbf{11} unde] Der W \textbf{12} sancten] santen U  $\cdot$ diu] die T [di*]: die V \textbf{13} Key] Kein V  $\cdot$ sîne] sin U V \textbf{14} im] ims W \textbf{15} Waleises] waleis T walleisen V waleisen W \textbf{17} Key] Kein V  $\cdot$ Artuses] artus W \textbf{18} gegentjost] der tyost W \textbf{19} dâ] do U V W \textbf{21} beidiu samt] beide samt T (V) (W) Beisamit U  $\cdot$ liten] leident W \textbf{23} dem] den U  $\cdot$ satelbogen unde einem] sattel vnd dem W \textbf{24} Keys] keyns T (U) Keins V Key W  $\cdot$ zesewe] zeswer V W  $\cdot$ sîn] vnd dem W \textbf{25} zerbrast] Zerbrach V Brachen ab W  $\cdot$ disem] dem W \textbf{26} surzengel] Darengv́rtel V Gurtzengel W \textbf{27} zerbrast] zer [bra*]: brast V \textbf{28} dirre] [*]: da der V \textbf{30} muose] mvese T  $\cdot$ er selbe] \textit{om.} T selbe U [*]: er selbe V \newline
\end{minipage}
\end{table}
\end{document}
