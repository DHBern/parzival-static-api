\documentclass[8pt,a4paper,notitlepage]{article}
\usepackage{fullpage}
\usepackage{ulem}
\usepackage{xltxtra}
\usepackage{datetime}
\renewcommand{\dateseparator}{.}
\dmyyyydate
\usepackage{fancyhdr}
\usepackage{ifthen}
\pagestyle{fancy}
\fancyhf{}
\renewcommand{\headrulewidth}{0pt}
\fancyfoot[L]{\ifthenelse{\value{page}=1}{\today, \currenttime{} Uhr}{}}
\begin{document}
\begin{table}[ht]
\begin{minipage}[t]{0.5\linewidth}
\small
\begin{center}*D
\end{center}
\begin{tabular}{rl}
\textbf{304} & \textbf{\begin{large}D\end{large}ô sprach er}: "\textbf{bistûz}, Gawan?\\ 
 & \textbf{wie} kranken prîs ich des hân,\\ 
 & ob dû mir\textbf{z} \textbf{wol} \textbf{erbiutes} hie.\\ 
 & ich hôrte von dir sprechen ie,\\ 
5 & dû \textbf{erbiute}\textbf{z} allen liuten wol.\\ 
 & dîn dienst ich \textbf{doch} enpfâhen sol\\ 
 & niwan ûf \textbf{gegendienstes} gelt.\\ 
 & Nû sage mir, wes sint \textbf{diu} \textbf{gezelt},\\ 
 & der \textbf{dort ist manegez} ûf geslagen?\\ 
10 & lît Artus dâ, sô muoz ich klagen,\\ 
 & daz ich in \textbf{niht} mit êren mîn\\ 
 & mac gesehen \textbf{noch} die künegîn.\\ 
 & ich sol \textbf{rechen ê} ein bliuwen,\\ 
 & \textbf{dâ von} ich sît mit riuwen\\ 
15 & vuor, \textbf{von} solhen sachen:\\ 
 & ein werdiu magt \textbf{mir} lachen\\ 
 & bôt; die \textbf{blou} \textbf{der} scheneschalt\\ 
 & durch mich, daz von ir \textbf{reis} der walt."\\ 
 & "\textbf{unsanfte} \textbf{ist daz} gerochen",\\ 
20 & Sprach Gawan. "im ist zerbrochen\\ 
 & der zeswe arm unt \textbf{daz} winster bein.\\ 
 & \textbf{rît} her, \textbf{schouwe} ors unt \textbf{ouch} den stein.\\ 
 & hie ligent ouch trunzûne ûf dem snê\\ 
 & \textbf{dînes} spers, nâch dem \textbf{\textit{dû} vrâgtest} ê."\\ 
25 & Dô Parzival die wârheit sach,\\ 
 & dô \textbf{vrâgete}r \textbf{vürbaz} unt sprach:\\ 
 & "\textbf{diz} lâze ich an dich, Gawan,\\ 
 & op daz sî der selbe man,\\ 
 & der mir hât laster vor gezilt,\\ 
30 & sô rît ich mit dir, swar dû wilt."\\ 
\end{tabular}
\scriptsize
\line(1,0){75} \newline
D \newline
\line(1,0){75} \newline
\textbf{1} \textit{Initiale} D  \textbf{8} \textit{Majuskel} D  \textbf{20} \textit{Majuskel} D  \textbf{25} \textit{Majuskel} D  \newline
\line(1,0){75} \newline
\textbf{24} dû] \textit{om.} D \newline
\end{minipage}
\hspace{0.5cm}
\begin{minipage}[t]{0.5\linewidth}
\small
\begin{center}*m
\end{center}
\begin{tabular}{rl}
 & \textbf{\begin{large}D\end{large}er Waleis sprach}: "\textbf{dû bist} Gawan.\\ 
 & \textbf{vil} kr\textit{a}nken prîs ich des hân,\\ 
 & ob dû mir \textbf{es} \textbf{schône} \textbf{biutest} hie.\\ 
 & ich hôrte von dir sprechen ie,\\ 
5 & dû \textbf{enbiutest} allen liuten wol.\\ 
 & dîn dienst ich \textbf{doch} enpfâhen sol\\ 
 & niuwan ûf \textbf{gegendienstes} gelt.\\ 
 & nû sage mir, wes sint \textbf{diu} \textbf{zelt},\\ 
 & der \textbf{dort ist manigez} ûf geslagen?\\ 
10 & lît Artus d\textit{â}, sô muoz ich klagen,\\ 
 & daz ich in \textbf{niht} mit êren \textit{m}î\textit{n}\\ 
 & mac gesehen \textbf{noch} die künigîn.\\ 
 & ich sol \textbf{rechen ê} ein bliuwen,\\ 
 & \textbf{dâ von} ich sît mit riuwen\\ 
15 & vuor, \textbf{von} solichen sachen:\\ 
 & ein werdiu magt \textbf{mir} lachen\\ 
 & bôt; die \textbf{bl\textit{ou}} \textbf{der} schiniscalt\\ 
 & durch mich, daz von ir \textbf{reis} der walt."\\ 
 & "\textbf{unsanfte} \textbf{ist daz} gerochen",\\ 
20 & sprach Gawan. "im ist zerbrochen\\ 
 & der zeswe arm und \textbf{daz} winster bein.\\ 
 & \textbf{rît} \dag er schône\dag  ros und \textbf{ouch} den stein.\\ 
 & hie ligent ouch trunz\textit{ûn}e ûf dem snê\\ 
 & \textbf{dînes} spers, nâch dem \textbf{dû vrâg\textit{et}est} ê."\\ 
25 & dô Parcifal die wârheit sach,\\ 
 & dô \textbf{vrâgete} er \textbf{vürbaz} und sprach:\\ 
 & "\textbf{diz} lâz ich an dich, Gawan,\\ 
 & ob daz sî der selbe man,\\ 
 & der mir hât laster vor gezilt,\\ 
30 & sô rît ich mit dir, war dû wilt."\\ 
\end{tabular}
\scriptsize
\line(1,0){75} \newline
m n o \newline
\line(1,0){75} \newline
\textbf{1} \textit{Initiale} m o   $\cdot$ \textit{Capitulumzeichen} n  \newline
\line(1,0){75} \newline
\textbf{1} Gawan] gewan o \textbf{2} kranken] krencken m \textbf{5} enbiutest] erbutest n [erbus]: erbuttest o \textbf{8} wes] wasz o  $\cdot$ diu zelt] die gezelt n gezelt o \textbf{10} dâ sô] doso m (n) (o) \textbf{11} mîn] nim m (o) \textbf{13} ein] min n (o) \textbf{15} von] mit n o \textbf{17} blou] blevo m blouwe n bluͯ o  $\cdot$ der] \textit{om.} n o  $\cdot$ schiniscalt] sciniscant o \textbf{18} reis] res n \textbf{20} Gawan] gewan o  $\cdot$ im ist zerbrochen] jme ist gebrochen n ist nuͯ gebrochen o \textbf{21} der] Das n  $\cdot$ und] \textit{om.} n o \textbf{22} ros] rosse n \textbf{23} trunzûne] trunzime m truntzẏm o \textbf{24} nâch dem dû vrâgetest] nach [de*]: dem du fragest m noch du frogest n noch du frogetest o \textbf{25} dô] Do sprach o  $\cdot$ die] dú o \textbf{27} Gawan] gewan o \textbf{29} gezilt] gezelt n o \newline
\end{minipage}
\end{table}
\newpage
\begin{table}[ht]
\begin{minipage}[t]{0.5\linewidth}
\small
\begin{center}*G
\end{center}
\begin{tabular}{rl}
 & \textbf{dô sprach er}: "\textbf{bistûz}, Gawan?\\ 
 & \textbf{vil} kranken brîs ich des hân,\\ 
 & op dû mir\textbf{z} \textbf{wol} \textbf{erbiutest} hie.\\ 
 & ich hôrt von dir sprechen ie,\\ 
5 & dû \textbf{\textit{er}bi\textit{u}test} \textbf{ez} allen liuten wol.\\ 
 & dîn dienst ich \textbf{doch} enpfâhen sol\\ 
 & niwan ûf \textbf{dienstes} gelt.\\ 
 & nû sage \textit{mir}, wes sint \textbf{diu} \textbf{gezelt},\\ 
 & der \textbf{manigez ist dort} ûf geslagen?\\ 
10 & lît Artus dâ, sô muoz ich klagen,\\ 
 & daz ich in mit \textbf{den} êren mîn\\ 
 & \textbf{niht} mac gesehen \textbf{noch} die künigîn.\\ 
 & ich sol \textbf{rechen \textit{ê}} ein bliuwen,\\ 
 & \textbf{dar umbe} ich sît mit riuwen\\ 
15 & vuor, \textbf{von} solhen sachen:\\ 
 & ein werdiu maget \textbf{ir} lachen\\ 
 & \textbf{mir} bôt; die \textbf{sluoc} \textbf{der} seneschalt\\ 
 & durch mich, daz von i\textit{r} \textbf{\textit{s}wâret} der walt."\\ 
 & "\textbf{unsanfte} \textbf{daz ist} gerochen",\\ 
20 & sprach Gawan. "im ist zerbrochen\\ 
 & der zeswe arm unde \textbf{daz} winster bein.\\ 
 & \textbf{rît} her, \textbf{schouwe} ors unt den stein.\\ 
 & hie ligent ouch trunzûne ûf dem snê\\ 
 & \textbf{dînes} spers, \textit{nâch} dem \textbf{dû vrâgtest} ê."\\ 
25 & dô Parzival die wârheit sach,\\ 
 & dô \textbf{dâhte}r \textbf{mêr} unde sprach:\\ 
 & "\textbf{daz} lâze ich an dich, Gawan,\\ 
 & op daz sî der selbe man,\\ 
 & der mir hât laster vor gezilt,\\ 
30 & sô rîte ich mit dir, swar dû wilt."\\ 
\end{tabular}
\scriptsize
\line(1,0){75} \newline
G I O L M Q R Z \newline
\line(1,0){75} \newline
\textbf{1} \textit{Initiale} L Z   $\cdot$ \textit{Capitulumzeichen} R  \textbf{11} \textit{Initiale} O Q  \textbf{25} \textit{Initiale} I R   $\cdot$ \textit{Capitulumzeichen} L  \newline
\line(1,0){75} \newline
\textbf{1} bistûz] bistu L  $\cdot$ Gawan] chawan I \textbf{3} op] Nu M Do R  $\cdot$ erbiutest] buͯtest L \textbf{4} sprechen] sprech R \textbf{5} erbiutest] bittest G erbietst O (R)  $\cdot$ ez] \textit{om.} Q \textbf{6} dîn] Dinen L Z  $\cdot$ doch] \textit{om.} L M \textbf{7} ûf] uff gein M (Q) (R) (Z)  $\cdot$ dienstes] dinst M  $\cdot$ gelt] widergelt I \textbf{8} mir] \textit{om.} G  $\cdot$ diu] disiv O \textbf{9} manigez ist dort] manges dort ist L (R) dort manigez ist Z \textbf{10} dâ] do O Q  $\cdot$ muoz ich] mich M \textbf{11} daz] ÷az O  $\cdot$ in] en nicht M \textit{om.} Q im R  $\cdot$ êren] augen I \textbf{12} niht] \textit{om.} M  $\cdot$ gesehen] gischouwen M  $\cdot$ noch] vnde ovch O (M) (Z) vnd Q R \textbf{13} rechen ê] rechen noch G E rechen L (R) rechin er M \textbf{14} dar umbe] Daz O  $\cdot$ riuwen] triwen I \textbf{15} vuor] Vor M \textbf{16} ein] Er Z  $\cdot$ ir] mit M \textbf{17} sluoc] schleg R  $\cdot$ der seneschalt] der sinschalt G der shinishalt I der senetschalt O der sinetshalt L der sinetscalt M senecschalt Q der sy mit schalt R der sinetschalt Z \textbf{18} von] durch Q  $\cdot$ ir swâret der walt] ir der >swaret der< walt G ir der walt I ir reis der walt O (L) (Z) ir ir der walt M ir reisz erwalt Q irs reis der walt R \textbf{19} unsanfte] ershal vnsanfte I  $\cdot$ daz ist] ist daz O L (M) (Q) (R) Z \textbf{20} zerbrochen] gestochen O gebrochen L (M) Z \textbf{21} zeswe] [zem]: zesen Q Rechtte R  $\cdot$ unde] \textit{om.} O R  $\cdot$ winster] ling R \textbf{22} rît her] riter I Rit er Q  $\cdot$ schouwe] schawet O  $\cdot$ ors] daz ros L  $\cdot$ den] \textit{om.} M dasz Q \textbf{23} trunzûne] drvnzel O schafftt R \textbf{24} nâch] von G  $\cdot$ vrâgtest] uragest O (L) (M) (Q) (Z) \textbf{25} dô] Da Z  $\cdot$ Parzival] parzifal I L M Barcifal O partzifal Q parczifal R parcifal Z \textbf{26} dô] Da Z  $\cdot$ dâhter] fraget er Q (Z) fragte er R \textbf{27} daz] Ditze O (Z) Disz L M Q \textbf{28} daz] ez I \textbf{29} mir] mich L  $\cdot$ hât laster] laster hat R  $\cdot$ vor] her O \textbf{30} swar] war L (M) (Q) R Z \newline
\end{minipage}
\hspace{0.5cm}
\begin{minipage}[t]{0.5\linewidth}
\small
\begin{center}*T
\end{center}
\begin{tabular}{rl}
 & \textbf{\begin{large}D\end{large}ô sprach er}: "\textbf{bistûz}, Gawan?\\ 
 & \textbf{vil} kranken prîs ich des hân,\\ 
 & ob dû mir\textbf{s} \textbf{wol} \textbf{erbiutest} hie.\\ 
 & ich hôrte von dir \textbf{daz} sprechen ie,\\ 
5 & dû \textbf{erbiute}\textbf{z} allen liuten wol.\\ 
 & dînen dienst ich enpfâhen sol\\ 
 & niuwan ûf \textbf{gegendienstes} gelt.\\ 
 & Nû sage mir, wes sint \textbf{jene} \textbf{gezelt},\\ 
 & der \textbf{manegez lît dort} ûf geslagen?\\ 
10 & lît Artus dâ, sô muoz ich klagen,\\ 
 & daz ich in \textbf{niht} mit \textbf{den} êren mîn\\ 
 & mac gesehen \textbf{unde} die künegîn.\\ 
 & ich sol \textbf{ê rechen} ein bliuwen,\\ 
 & \textbf{dar umb} ich sît mit riuwen\\ 
15 & vuor, \textbf{mit} sölhen sachen:\\ 
 & ein werd\textit{iu} maget \textbf{ir} lachen\\ 
 & \textbf{mir} bôt; die \textbf{sluoc} \textbf{sîn} seneschalt\\ 
 & durch mich, daz von ir \textbf{reis} der walt."\\ 
 & "\textbf{Ungevuoge} \textbf{ist daz} gerochen",\\ 
20 & sprach Gawan. "im ist zerbrochen\\ 
 & der zesewe arm unde \textbf{sîn} winster bein.\\ 
 & \textbf{rîtet} her, \textbf{schouwet} \textbf{des} ors unde den stein.\\ 
 & hie ligent ouch \textbf{die} trunzûn ûf dem snê\\ 
 & \textbf{des} spers, nâch dem \textbf{ir vrâgetet} ê."\\ 
25 & Dô Parcifal di\textit{e} wârheit sach,\\ 
 & dô \textbf{gedâhte}r \textbf{mêr} unde sprach:\\ 
 & "\textbf{diz} lâzich an dich, Gawan,\\ 
 & ob daz sî der selbe man,\\ 
 & der mir hât laster vor gezilt,\\ 
30 & sô rîtich mit dir, swar dû wilt."\\ 
\end{tabular}
\scriptsize
\line(1,0){75} \newline
T U V W \newline
\line(1,0){75} \newline
\textbf{1} \textit{Initiale} T U V W  \textbf{8} \textit{Majuskel} T  \textbf{19} \textit{Majuskel} T  \textbf{25} \textit{Majuskel} T  \newline
\line(1,0){75} \newline
\textbf{1} Dô] [D*]: Der waleis V  $\cdot$ er bistûz] [*]: bist du V er bistu W \textbf{2} kranken prîs] cranker prise U \textbf{3} mirs] mir iz U \textbf{5} erbiutez] [*]: erbivt ez T erbutest iz U [erb*test]: erbietest V beútestes W  $\cdot$ allen] ellend W \textbf{6} dînen] \textit{om.} W  $\cdot$ ich] ich doch U V W \textbf{7} niuwan] Nit dan U  $\cdot$ gegendienstes] dinstes U dienst gegen W \textbf{8} jene] dise U V W \textbf{9} lît dort] ist dort U V dort ist W \textbf{10} dâ] do U V W \textbf{11} niht] \textit{om.} W  $\cdot$ mit den] [*]: mit V \textbf{12} mac] Nit mag W  $\cdot$ die] \textit{om.} U \textbf{14} mit] [*]: mit V in W \textbf{15} vuor] Fúrwar W  $\cdot$ mit] von U V W \textbf{16} werdiu] werde T \textbf{17} sluoc] [sluͦc]: fluͦc U  $\cdot$ sîn seneschalt] [*]: sin scheneschalt V schinischalt W \textbf{18} von] [*]: von V \textbf{19} gerochen] zuͦ broch U \textbf{21} sîn] daz U V (W)  $\cdot$ winster] linke V \textbf{22} rîtet] Reit W  $\cdot$ schouwet] schawe W  $\cdot$ des] daz V (W) \textbf{23} die] dru U \textit{om.} W  $\cdot$ dem] den U \textbf{24} des spers] Dein sper W  $\cdot$ ir] du W \textbf{25} Parcifal] [*]: parzifal T parzifal V partzifal W  $\cdot$ die] div T \textbf{26} gedâhter mêr] gedacht me U [*]: frageter fúrbas V \textbf{27} diz] Das W  $\cdot$ Gawan] gaban W \textbf{28} ob] Oder U \textbf{30} swar] war U ob W \newline
\end{minipage}
\end{table}
\end{document}
