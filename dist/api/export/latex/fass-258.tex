\documentclass[8pt,a4paper,notitlepage]{article}
\usepackage{fullpage}
\usepackage{ulem}
\usepackage{xltxtra}
\usepackage{datetime}
\renewcommand{\dateseparator}{.}
\dmyyyydate
\usepackage{fancyhdr}
\usepackage{ifthen}
\pagestyle{fancy}
\fancyhf{}
\renewcommand{\headrulewidth}{0pt}
\fancyfoot[L]{\ifthenelse{\value{page}=1}{\today, \currenttime{} Uhr}{}}
\begin{document}
\begin{table}[ht]
\begin{minipage}[t]{0.5\linewidth}
\small
\begin{center}*D
\end{center}
\begin{tabular}{rl}
\textbf{258} & \begin{large}D\end{large}ô Parzival gruoz gein ir sprach,\\ 
 & an in si erkenneclîchen sach.\\ 
 & er was der schœnste über elliu lant;\\ 
 & dâ von si in schiere \textbf{het} erkant.\\ 
5 & Si \textbf{sagete}: "ich hân iuch ê gesehen,\\ 
 & dâ von \textbf{ist leide mir} geschehen.\\ 
 & doch müez iu vreude unt êre\\ 
 & got \textbf{immer geben} mêre,\\ 
 & denn ir umbe mich \textbf{gedienet} hât.\\ 
10 & \textbf{des} ist \textbf{nû ermer} mîn wât,\\ 
 & denne ir si jungest sâhet.\\ 
 & wæret ir niht genâhet\\ 
 & mir an der selben zît,\\ 
 & sô het ich êre âne strît."\\ 
15 & Dô sprach er: "vrouwe, \textbf{merket} \textbf{daz},\\ 
 & gein wem ir kêret iwern haz.\\ 
 & \textbf{jâ}ne wart von mîme lîbe\\ 
 & iu noch decheinem wîbe\\ 
 & laster nie gemêret\\ 
20 & - sô het ich mich geunêret -,\\ 
 & sît ich den schilt von êrst gewan\\ 
 & unt rîters vuore mich versan.\\ 
 & mir ist \textbf{ander} iwer kumber leit."\\ 
 & Al weinende diu vrouwe reit,\\ 
25 & daz si begôz ir brüstelîn.\\ 
 & als si gedræt solden sîn,\\ 
 & \textbf{diu stuonden} blanc, hôch, sinwel.\\ 
 & \textbf{jâ}ne wart nie dræhsel \textbf{sô} snel,\\ 
 & der si gedræt hete baz.\\ 
30 & swie minneclîch diu vrouwe saz,\\ 
\end{tabular}
\scriptsize
\line(1,0){75} \newline
D \newline
\line(1,0){75} \newline
\textbf{1} \textit{Initiale} D  \textbf{5} \textit{Majuskel} D  \textbf{15} \textit{Majuskel} D  \textbf{24} \textit{Majuskel} D  \newline
\line(1,0){75} \newline
\newline
\end{minipage}
\hspace{0.5cm}
\begin{minipage}[t]{0.5\linewidth}
\small
\begin{center}*m
\end{center}
\begin{tabular}{rl}
 & \begin{large}D\end{large}ô Parcifal gruoz gegen ir sprach,\\ 
 & an in si erkenneclîch sach.\\ 
 & er was der schœneste über alliu lant;\\ 
 & dâ von si in schiere erkant.\\ 
5 & si \textbf{sprach}: "ich hân iuch ê gesehen,\\ 
 & dâ von \textbf{is\textit{t} \textit{l}eide \textit{mir}} ges\textit{ch}ehen.\\ 
 & doch müeze iu vröude und êre\\ 
 & got \textbf{iemer geben} mêre,\\ 
 & danne ir umb mich \textbf{geschuldet} hât.\\ 
10 & \textbf{d\textit{e}s} ist \textbf{armer nû} mîn wât,\\ 
 & danne ir si jungest sâhet.\\ 
 & wæret ir niht genâhet\\ 
 & mir an der selben zît,\\ 
 & sô hete ich êre âne \textit{str}ît."\\ 
15 & dô sprach er: "vrouwe, \textbf{merket} \textbf{baz},\\ 
 & gegen wem ir kêret iuweren haz.\\ 
 & \textbf{jâ} enwart von mînem lîbe\\ 
 & iu noch dekeinem wîbe\\ 
 & laster nie gemêret\\ 
20 & - sô hete ich mich geunêret -,\\ 
 & sît ich den schilt von êrst gewan\\ 
 & und ritters vuore mich versan.\\ 
 & mir ist \textbf{ander} iuwer kumber leit."\\ 
 & al weinende diu vrouwe reit,\\ 
25 & daz si begôz ir brüstelîn.\\ 
 & als si gedræt solten sîn,\\ 
 & \textbf{diu stuo\textit{n}den} blanc, hôch, sinwel.\\ 
 & \textbf{jâ} enwart nie drehsel \textbf{sô} snel,\\ 
 & der si gedræt hete baz.\\ 
30 & wie minneclîch diu vrouwe saz,\\ 
\end{tabular}
\scriptsize
\line(1,0){75} \newline
m n o Fr69 \newline
\line(1,0){75} \newline
\textbf{1} \textit{Initiale} m n Fr69  \newline
\line(1,0){75} \newline
\textbf{2} erkenneclîch] erkentlichen n \textbf{3} schœneste über alliu] schonffe vber >alle< o \textbf{4} in] \textit{om.} o  $\cdot$ erkant] hette erkant n o \textbf{5} gesehen] ersehen n \textbf{6} ist leide mir] ist ẏme leide m  $\cdot$ geschehen] gesehen m beschehen n (o) \textbf{7} vröude und êre] [e gesehen]: freide vnd ere o \textbf{8} iemer geben] geben iemer Fr69  $\cdot$ mêre] mir o \textbf{9} Dan ir mich vmb mich verschulde hat Fr69  $\cdot$ geschuldet] beschuldet n beschuͯlden o \textbf{10} des] Das m  $\cdot$ armer] armuͯt n (o) \textbf{11} si] vmmb mich sie o \textbf{12} wæret] Woret o \textbf{14} strît] zit m \textbf{16} iuweren] iren m vwer o \textbf{17} jâ] Jo n o \textbf{18} iu] Auch o  $\cdot$ dekeinem] do keinem n \textbf{21} schilt] schut o \textbf{23} ander] anders n o \textbf{26} gedræt] getroget o \textbf{27} stuonden] [stunt]: stuntden m \textbf{28} jâ] Jo n o  $\cdot$ drehsel] tresel n o \textbf{30} wie] So n o Swie Fr69 \newline
\end{minipage}
\end{table}
\newpage
\begin{table}[ht]
\begin{minipage}[t]{0.5\linewidth}
\small
\begin{center}*G
\end{center}
\begin{tabular}{rl}
 & dô Parzival gruoz gein ir \textit{s}prach,\\ 
 & an in si erkenniclîchen sach.\\ 
 & er was der schœnste über elliu lant;\\ 
 & dâ von sin schiere \textbf{het} erkant.\\ 
5 & si \textbf{sprach}: "ich hân iuch ê gesehen,\\ 
 & dâ von \textbf{ist leide mir} geschehen.\\ 
 & doch muoze iu vröude und êre\\ 
 & got \textbf{imer geben} mêre,\\ 
 & danne ir umbe mich \textbf{gedienet} hât.\\ 
10 & \textbf{ez} ist \textbf{nû ermer} mîn wât,\\ 
 & danne ir si \textit{jung}est sâhet.\\ 
 & w\textit{æ}ret i\textit{r n}iht genâhet\\ 
 & \textit{mir} an der selben zît,\\ 
 & sô hete ich êre âne strît."\\ 
15 & dô sprach er: "vrouwe, \textbf{wizzet} \textbf{daz},\\ 
 & gein wem ir kêret iuwern ha\textit{z}.\\ 
 & \textbf{jâ}ne wart von mînem lîbe\\ 
 & iu noch deheinem wîbe\\ 
 & laster nie gemêret\\ 
20 & - sô het ich mich geunêret -,\\ 
 & sî\textit{t} ich den schilt von êrst gewan\\ 
 & unde rîters vuore mich versan.\\ 
 & mirst \textbf{ander} iuwer kumber leit."\\ 
 & al w\textit{ei}nde diu vrouwe reit,\\ 
25 & daz si begôz ir brüstelîn.\\ 
 & als si gedræt solten sîn,\\ 
 & \textbf{diu stuonden} \textit{blanc}, \textit{hôch}, sinewel.\\ 
 & \textbf{jâ}ne wart nie dræhsel \textbf{sô} snel,\\ 
 & der si gedræt het baz.\\ 
30 & swie minniclîch diu vrouwe saz,\\ 
\end{tabular}
\scriptsize
\line(1,0){75} \newline
G I O L M Q R Z Fr21 Fr60 \newline
\line(1,0){75} \newline
\textbf{1} \textit{Initiale} R  \textbf{3} \textit{Initiale} I  \textbf{23} \textit{Initiale} I  \textbf{27} \textit{Initiale} Z  \textbf{29} \textit{Initiale} O Fr21  \newline
\line(1,0){75} \newline
\textbf{1} dô] Da Z  $\cdot$ Parzival] parzifal I L M Parcifals O partzifal Q parczifal R parcifal Z Fr21  $\cdot$ gruoz] \textit{om.} I  $\cdot$ sprach] gesprach G \textbf{2} in] im O (R)  $\cdot$ si erkenniclîchen] erkenneclich sý L \textbf{4} sin schiere het] het si in shier I sy schier het R \textbf{6} ist leide mir] mir leide ist Z \textbf{7} doch muoze iu] Got muͯsz úch R \textbf{8} got] Doch R  $\cdot$ imer geben] geben yemmer L (M) (Z) \textbf{9} gedienet] verdienet I (M) \textbf{10} ez] Des R Fr21  $\cdot$ mîn] mir myn L mit R \textbf{11} danne] dan do I (Fr21)  $\cdot$ ir si] irs Fr21  $\cdot$ jungest] nahest G iungste I zuͯ iuͯngest L (R) \textbf{12} waret ir mir niht genahet G  $\cdot$ ir] ir mir O (R) \textbf{13} mir] do G mir nih I \textbf{14} strît] widerstrit R \textbf{15} dô sprach er] Da sprach her M Er sprach R  $\cdot$ wizzet daz] wiszent basz L merket baz Z \textbf{16} ir] \textit{om.} Fr21  $\cdot$ kêret] keren Q  $\cdot$ haz] han G \textbf{17} jâne wart] ion wart I (M) Ja wart O Ez wart L Es enwart R \textbf{19} nie] ane Q \textbf{20} ich mich] mich selde Z  $\cdot$ geunêret] vngeerret R \textbf{21} sît] sin G  $\cdot$ von êrst] alrest O \textbf{22} unde] Von Fr21  $\cdot$ vuore] fvͦr an O fuͦrren R \textbf{23} ander] aber R  $\cdot$ kumber] kumen Q \textbf{24} Die frowe weinende reit Z  $\cdot$ weinde] wiende G  $\cdot$ diu vrouwe] sy mit Im R \textbf{26} als] Als ob R  $\cdot$ gedræt] treit R \textbf{27} diu stuonden] Stuͯnden sie L  $\cdot$ blanc hôch] hoch blanch G blanc ioch I wis R \textbf{28} jâne] ion I (M) Ja O Ez L (R)  $\cdot$ nie] \textit{om.} Q  $\cdot$ sô] also O M Q Fr21 \textbf{29} der] ÷er O  $\cdot$ gedræt] grat I \textbf{30} swie] Wie L (Q) R  $\cdot$ minniclîch] mindiglich Q  $\cdot$ diu] \textit{om.} O  $\cdot$ saz] was R \newline
\end{minipage}
\hspace{0.5cm}
\begin{minipage}[t]{0.5\linewidth}
\small
\begin{center}*T
\end{center}
\begin{tabular}{rl}
 & \begin{large}D\end{large}ô Parcifal gruoz gegen ir sprach,\\ 
 & An in si erkenneclîche sach.\\ 
 & er was der schœneste über alliu lant;\\ 
 & dâ von sin schiere \textbf{hete} erkant.\\ 
5 & Si \textbf{sprach}: "ich hân iuch ê gesehen,\\ 
 & dâ von \textbf{mir leide ist} geschehen.\\ 
 & doch müez iu vröude unde êre\\ 
 & got \textbf{geben iemer} mêre,\\ 
 & dannir umbe mich \textbf{gedient} hât.\\ 
10 & \textbf{ez} ist \textbf{nû ermer} mîn wât,\\ 
 & danne ir si \textbf{ze} jungest sâhet.\\ 
 & wæret ir niht genâhet\\ 
 & mir an der selben zît,\\ 
 & sô hetich êre âne strît."\\ 
15 & Dô sprach er: "vrouwe, \textbf{merket} \textbf{baz},\\ 
 & gegen wem ir kêret iuwern haz.\\ 
 & \textbf{jô}ne wart von mînem lîbe\\ 
 & iu noch deheinem wîbe\\ 
 & laster nie gemêret\\ 
20 & - sô hetich mich geunêret -,\\ 
 & sît ich den schilt von êrst gewan\\ 
 & unde rîters vuore mich versan.\\ 
 & mirst \textbf{anders} iuwer kumber leit."\\ 
 & Al weinende diu vrouwe reit,\\ 
25 & daz si begôz ir brüstelîn.\\ 
 & als si gedrât solten sîn,\\ 
 & \textbf{stuonden si} blanc, hôch, sinewel.\\ 
 & \textbf{jô}ne wart nie dre\textit{hs}el \textbf{als} snel,\\ 
 & der si gedrât hete baz.\\ 
30 & swie minneclîch diu vrouwe saz,\\ 
\end{tabular}
\scriptsize
\line(1,0){75} \newline
T U V W \newline
\line(1,0){75} \newline
\textbf{1} \textit{Initiale} T U V W  \textbf{2} \textit{Majuskel} T  \textbf{5} \textit{Majuskel} T  \textbf{15} \textit{Initiale} W   $\cdot$ \textit{Majuskel} T  \textbf{24} \textit{Majuskel} T  \newline
\line(1,0){75} \newline
\textbf{1} Parcifal] parzifal T V partzifal W  $\cdot$ gruoz gegen ir] gen ir gruͦs W \textbf{2} in] im W \textbf{5} iuch] îv T  $\cdot$ ê] me V \textbf{6} mir leide ist] ist leide mir U (W) [*]: ist leide mir  V \textbf{7} müez iu] muͦz vch U muͦß ich W  $\cdot$ unde] \textit{om.} U \textbf{10} [*]: Dez ist armer nv min wat V \textbf{11} si] \textit{om.} W \textbf{12} ir] ir mir W \textbf{13} mir an] Do zuͦ W \textbf{14} êre âne strît] groß ere seit W \textbf{16} kêret] keret ir U \textbf{17} jône] Es W \textbf{18} iu] Eúcb W \textbf{19} gemêret] von mir gemeret W \textbf{20} Wann so wer ich guneret W \textbf{24} weinende] zuͦ weinde U \textbf{25} ir] die W \textbf{26} solten] solte U \textbf{27} stuonden si] Die stuͦnden U (V) Die warent W \textbf{28} jône wart] Es enward W  $\cdot$ drehsel] drescel T trescheler V  $\cdot$ als] so U V W \textbf{29} baz] braz U \textbf{30} swie] Wie U W \newline
\end{minipage}
\end{table}
\end{document}
