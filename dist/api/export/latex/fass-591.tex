\documentclass[8pt,a4paper,notitlepage]{article}
\usepackage{fullpage}
\usepackage{ulem}
\usepackage{xltxtra}
\usepackage{datetime}
\renewcommand{\dateseparator}{.}
\dmyyyydate
\usepackage{fancyhdr}
\usepackage{ifthen}
\pagestyle{fancy}
\fancyhf{}
\renewcommand{\headrulewidth}{0pt}
\fancyfoot[L]{\ifthenelse{\value{page}=1}{\today, \currenttime{} Uhr}{}}
\begin{document}
\begin{table}[ht]
\begin{minipage}[t]{0.5\linewidth}
\small
\begin{center}*D
\end{center}
\begin{tabular}{rl}
\textbf{591} & \textit{\begin{large}D\end{large}}iu künegîn sprach: "\textbf{muoz} ich \textbf{sô} spehen,\\ 
 & daz ir \textbf{mir, hêrre}, habt verjehen,\\ 
 & daz ich iwer meisterinne sî,\\ 
 & sô küsset \textbf{dise} vrouwen \textbf{alle} drî.\\ 
5 & dâ sît ir lasters \textbf{an} bewart:\\ 
 & si \textbf{sint} \textbf{erborn} von \textbf{küneges} art."\\ 
 & Dirre bete was er vrô.\\ 
 & die clâren vrouwen kuster dô,\\ 
 & Sangiven unt Itonje\\ 
10 & unt die süezen Cundrie.\\ 
 & Gawan saz selbe \textbf{vünfter} nider.\\ 
 & dô sach er vür und wider\\ 
 & \textbf{an} der clâren meide lîp.\\ 
 & iedoch twang in \textbf{des} \textbf{ein} wîp,\\ 
15 & diu in sîme herzen lac.\\ 
 & dirre meide blic ein nebeltac\\ 
 & was bî Orgelusen gar.\\ 
 & diu dûht \textbf{êt in sô} wol gevar,\\ 
 & von Logroys diu herzogîn;\\ 
20 & dâ jagete in \textbf{sîn herze hin}.\\ 
 & Nû \textbf{diz} was ergangen,\\ 
 & \textbf{daz} Gawan was enpfangen\\ 
 & von den vrouwen allen drîn.\\ 
 & die truogen sô liehten, \textbf{süezen} schîn,\\ 
25 & des lîhte ein herze wære versniten,\\ 
 & daz ê niht kumbers het erliten.\\ 
 & Zuo sîner meisterinne er sprach\\ 
 & umbe die sûl, die er \textbf{dâ} sach,\\ 
 & daz si im sagete mære,\\ 
30 & \textbf{von} welher \textbf{art} diu wære.\\ 
\end{tabular}
\scriptsize
\line(1,0){75} \newline
D Z \newline
\line(1,0){75} \newline
\textbf{1} \textit{Initiale} D Z  \textbf{7} \textit{Majuskel} D  \textbf{21} \textit{Majuskel} D  \textbf{27} \textit{Majuskel} D  \newline
\line(1,0){75} \newline
\textbf{1} Diu] ÷iv D \textbf{2} daz] Des Z \textbf{6} erborn] geborn Z \textbf{8} kuster] kust er Z \textbf{9} Sangiven] Sangîven D Seiven Z  $\cdot$ Itonje] Jtonîe D Jconie Z \textbf{10} Cundrie] Cvndrîe D kundrie Z \textbf{11} vünfter] funfte Z \textbf{12} dô] Da Z \textbf{18} êt] \textit{om.} Z \textbf{19} Logroys] Logrois Z  $\cdot$ herzogîn] kvnigin Z \textbf{24} liehten süezen] werden liehten Z \textbf{27} Zuo] Hin zv Z \textbf{30} diu] sie Z \newline
\end{minipage}
\hspace{0.5cm}
\begin{minipage}[t]{0.5\linewidth}
\small
\begin{center}*m
\end{center}
\begin{tabular}{rl}
 & diu künigîn sprach: "\textbf{sol} ich \textbf{daz} spehen,\\ 
 & daz ir \textbf{mir, hêrre}, hât verjehen,\\ 
 & daz ich iuwer meisterîn sî,\\ 
 & sô küsset \textbf{die} vrouwe\textit{n} \textbf{al} drî.\\ 
5 & d\textit{â} sît ir lasters \textbf{wol} bewart:\\ 
 & si \textbf{sint} \textbf{erborn} von \textbf{küniges} art."\\ 
 & diser bete was er vrô.\\ 
 & die clâren vrouwe\textit{n} kuste er dô,\\ 
 & Sang\textit{i}ven und \textit{I}thonie\\ 
10 & und die süezen Condrie.\\ 
 & Gawan saz selbe \textbf{\textit{vü}nfte} nider.\\ 
 & dô sach er vür und wider\\ 
 & \textbf{zuo} der clâren megde lîp.\\ 
 & iedoch twanc in \textbf{daz} wîp,\\ 
15 & diu in sînem herzen lac.\\ 
 & diser megde bli\textit{c} ein nebeltac\\ 
 & was bî Urgeluse gar.\\ 
 & diu dûhte \textbf{eht in sô} wol gevar,\\ 
 & von Logrois diu herzogîn;\\ 
20 & dâ jagete i\textit{n} \textbf{sîn herz \textit{h}in}.\\ 
 & \begin{large}N\end{large}û \textbf{diz} was ergangen,\\ 
 & \textbf{daz} Gawan was enpfangen\\ 
 & von den vrouwen allen drîn.\\ 
 & die truogen sô liehte\textit{n} schîn,\\ 
25 & des lîht ein herz w\textit{ære} versniten,\\ 
 & daz ê niht kumbers het erliten.\\ 
 & zuo sîner meisterîn \textit{er} sprach\\ 
 & umb die sûle, die er \textbf{d\textit{â}} sach,\\ 
 & daz si im sagte mær,\\ 
30 & \textit{w}elicher \textbf{hant} \textit{diu} wær.\\ 
\end{tabular}
\scriptsize
\line(1,0){75} \newline
m n o \newline
\line(1,0){75} \newline
\textbf{21} \textit{Initiale} m n  \newline
\line(1,0){75} \newline
\textbf{3} iuwer] uwerin n \textbf{4} vrouwen] frouwe m \textbf{5} dâ] Do m n o  $\cdot$ wol] an n (o) \textbf{8} vrouwen] frouͯwe m \textbf{9} Sangiven] Sangwen m n o  $\cdot$ Ithonie] thonie m Jthonien o \textbf{10} Condrie] Cẏmdrie o \textbf{11} selbe] sebbe o  $\cdot$ vünfte] senffte m \textbf{15} herzen] hercze m \textbf{16} blic] blib m \textbf{17} gar] [*]: dar o \textbf{20} in] ẏm m  $\cdot$ hin] sin m \textbf{24} liehten] liehtte m \textbf{25} des] Das o  $\cdot$ wære] was m \textbf{26} ê] do E n  $\cdot$ kumbers] komber n \textbf{27} er] vnd m \textbf{28} die sûle] die súle die súle n  $\cdot$ dâ] do m n o \textbf{30} Von wellicher art die were n (o)  $\cdot$ diu] komer m \newline
\end{minipage}
\end{table}
\newpage
\begin{table}[ht]
\begin{minipage}[t]{0.5\linewidth}
\small
\begin{center}*G
\end{center}
\begin{tabular}{rl}
 & diu künegîn sprach: "\textbf{muoz} ich \textbf{sô} spehen,\\ 
 & daz ir \textbf{mir, hêrre}, habet verjehen,\\ 
 & daz ich iuwer meisterinne sî,\\ 
 & sô küsset \textbf{dise} vrouwen drî.\\ 
5 & dâ sît ir lasters \textbf{ane} bewart:\\ 
 & si \textbf{sint} \textbf{geborn} von \textbf{hôher} art."\\ 
 & dirre bete was er vrô.\\ 
 & die clâren vrouwen kuste er dô,\\ 
 & Sagiven unde Itonien\\ 
10 & unde die süezen Gundrien.\\ 
 & Gawan saz selbe \textbf{vierde} nider.\\ 
 & dô sach er vür unde wider\\ 
 & \textbf{an} der clâren meide lîp.\\ 
 & iedoch dwanc in \textbf{des} \textbf{ein} wîp,\\ 
15 & diu in sînem herzen lac.\\ 
 & dirre meide blic ein nebels tac\\ 
 & was bî Orgelusen gar.\\ 
 & diu dûht \textbf{êt in vil} wol gevar,\\ 
 & von Logroys diu herzogîn;\\ 
20 & dar jagete in \textbf{sînes herzen sin}.\\ 
 & nû \textbf{diz} was ergangen,\\ 
 & Gawan was enpfangen\\ 
 & von den vrouwen allen drîn.\\ 
 & die truogen sô liehten, \textbf{werden} schîn,\\ 
25 & des lîht ein herze wære versniten,\\ 
 & daz ê \textit{niht kumbers} het erliten.\\ 
 & \textbf{hin} ze sîner meisterinne er sprach\\ 
 & umb die sûl, die er sach,\\ 
 & daz si im sagete mære,\\ 
30 & \textbf{von} welher \textbf{art} diu \textbf{sûle} wære.\\ 
\end{tabular}
\scriptsize
\line(1,0){75} \newline
G I L M Fr23 \newline
\line(1,0){75} \newline
\textbf{1} \textit{Initiale} L Fr23  \textbf{17} \textit{Initiale} I  \newline
\line(1,0){75} \newline
\textbf{1} sô] \textit{om.} L M Fr23 \textbf{2} daz] Des Fr23  $\cdot$ verjehen] veriet M \textbf{4} sô] \textit{om.} I  $\cdot$ drî] alle dry M (Fr23) \textbf{5} ir] \textit{om.} L \textbf{6} von] vnde M \textbf{8} kuste] chust I (L) Fr23  $\cdot$ dô] da M \textbf{9} Sagiven] Sâiven G saiwen I Setven L Seiven M Saiven Fr23  $\cdot$ Itonien] Jtonien L M Fr23 \textbf{10} Gundrien] Kvndrien L (M) (Fr23) \textbf{11} vierde] vierder I \textbf{12} dô] Da M  $\cdot$ er] \textit{om.} L  $\cdot$ vür] vort M \textbf{13} meide] magt Fr23 \textbf{14} dwanc] bedwanch Fr23  $\cdot$ des ein wîp] [der einen lip]: daz eine wip I \textbf{16} nebels] nebel I \textbf{17} Orgelusen] Orgelýsen L orgelusin M orgilusen Fr23 \textbf{18} Nu duht er in vil wol gevar Fr23  $\cdot$ êt] \textit{om.} I M \textbf{19} Logroys] logrois G I M logroýs L ligr::s Fr23 \textbf{20} dar jagete] dar iagtan I Das jagit M  $\cdot$ sin] pin Fr23 \textbf{21} was] ist I \textbf{24} sô] al Fr23  $\cdot$ werden] \textit{om.} Fr23 \textbf{26} ê] her M  $\cdot$ niht kumbers] chumbers niht G \textbf{27} hin] \textit{om.} I \textbf{28} sach] da sach I \textbf{30} diu sûle] die L (M) \newline
\end{minipage}
\hspace{0.5cm}
\begin{minipage}[t]{0.5\linewidth}
\small
\begin{center}*T
\end{center}
\begin{tabular}{rl}
 & diu künigîn sprach: "\textbf{muoz} ich \textbf{sô} spehen,\\ 
 & daz ir, \textbf{hêrre, mir} hât verjehen,\\ 
 & daz ich iuwer meisterîn s\textit{î},\\ 
 & sô küsset \textbf{dise} vrouwen \textbf{alle} drî.\\ 
5 & dâ sît ir lasters \textbf{an} bewart:\\ 
 & si \textbf{sîn} \textbf{geborn} von \textbf{hôher} art."\\ 
 & diser bet was er vrô.\\ 
 & die klâren vrouwen kust er dô,\\ 
 & Seyven und Itonie\\ 
10 & und die süezen Kundrie.\\ 
 & Gawan saz selbe \textbf{vünfte} nider.\\ 
 & dô sach er vür und wider\\ 
 & \textbf{an} der klâren meide lîp.\\ 
 & iedoch twanc in \textbf{des} \textbf{ein} wîp,\\ 
15 & diu \textbf{mit stæte} in sînem herzen lac.\\ 
 & diser meide blic ein nebeltac\\ 
 & was bî Orgelusen gar.\\ 
 & diu dûht \textbf{in ouch vil} wol gevar,\\ 
 & von Logrois diu herzogîn;\\ 
20 & dar jagte in \textbf{sînes herzen sin}.\\ 
 & nû \textbf{daz} was ergangen,\\ 
 & Gawan was enpfangen\\ 
 & von den vrouwen allen drîn.\\ 
 & die truogen sô liehten, \textbf{werden} schîn,\\ 
25 & des lîht ein herze wære versniten,\\ 
 & daz \textit{ê} niht kumbers het erliten.\\ 
 & \textbf{hin} zuo sîner meisterîn er sprach\\ 
 & umb die sûl, die er \textbf{d\textit{â}} sach,\\ 
 & daz si im sagte mære,\\ 
30 & \textbf{von} welher \textbf{art} diu wære.\\ 
\end{tabular}
\scriptsize
\line(1,0){75} \newline
Q R W V U \newline
\line(1,0){75} \newline
\textbf{1} \textit{Initiale} Q R  \textbf{11} \textit{Initiale} W  \textbf{21} \textit{Initiale} V  \newline
\line(1,0){75} \newline
\textbf{1} \textit{Die Verse 553.1-599.30 fehlen} U   $\cdot$ sprach] \textit{om.} W  $\cdot$ muoz] [*]: sol V  $\cdot$ sô] \textit{om.} W \textbf{2} hêrre mir] mir herre R V (W)  $\cdot$ verjehen] [*iehen]: veriehen V \textbf{3} sî] sein Q \textbf{6} sîn] sind R (V)  $\cdot$ geborn] gerborn R  $\cdot$ hôher] hober W [*]: kvnigez V \textbf{8} klâren] \textit{om.} R \textbf{9} Seyven] Seyúen Q Seyuen R Seiuen W  $\cdot$ Itonie] Jtonie R ytonien V \textbf{10} Kundrie] kondrie R kvndrien V \textbf{11} Gawan] Gawin R \textbf{12} vür und] hin vnd her R \textbf{13} meide] megden R \textbf{15} stæte] stritte R  $\cdot$ herzen] hertze Q \textbf{17} Orgelusen] orgulusen R \textbf{18} in ouch vil] in echt R echt in vil W (V) \textbf{19} Logrois] logroys Q V lorogis R \textbf{20} sînes herzen sin] [sin* herze*]: sins herzen sin V \textbf{21} daz] dis R W (V)  $\cdot$ was] was alles W \textbf{22} Gawan] Gawin R \textbf{25} des] Das R \textbf{28} dâ] do Q W V \textbf{30} art diu] schlacht sy R \newline
\end{minipage}
\end{table}
\end{document}
