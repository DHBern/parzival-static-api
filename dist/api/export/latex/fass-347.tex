\documentclass[8pt,a4paper,notitlepage]{article}
\usepackage{fullpage}
\usepackage{ulem}
\usepackage{xltxtra}
\usepackage{datetime}
\renewcommand{\dateseparator}{.}
\dmyyyydate
\usepackage{fancyhdr}
\usepackage{ifthen}
\pagestyle{fancy}
\fancyhf{}
\renewcommand{\headrulewidth}{0pt}
\fancyfoot[L]{\ifthenelse{\value{page}=1}{\today, \currenttime{} Uhr}{}}
\begin{document}
\begin{table}[ht]
\begin{minipage}[t]{0.5\linewidth}
\small
\begin{center}*D
\end{center}
\begin{tabular}{rl}
\textbf{347} & "\begin{large}S\end{large}wem ir \textbf{iht} lîhet, der \textbf{diene} ouch daz",\\ 
 & sprach si. "mîn zil sich hœhet baz.\\ 
 & i\textbf{ne} wil von niemen lêhen hân.\\ 
 & mîn vrîheit ist \textbf{sô} getân,\\ 
5 & ieslîcher krône hôch genuoc,\\ 
 & die irdisch houbet ie getruoc."\\ 
 & Er sprach: "ir sît\textbf{z} gelêret,\\ 
 & daz ir hôchvart sus mêret.\\ 
 & sît iwer vater gap den rât,\\ 
10 & er wandelt mir die missetât.\\ 
 & ich sol hie wâpen alsô tragen,\\ 
 & daz wirt gestochen unt geslagen.\\ 
 & ez sî \textbf{strîten} oder turnei,\\ 
 & hie belîbet vil der sper enzwei."\\ 
15 & Mit zorne schiet er von der magt.\\ 
 & sîn \textbf{zurnen} sêre wart geklagt\\ 
 & von alder massenîe.\\ 
 & \textbf{in} klagt ouch Obie.\\ 
 & gegen dirre unges\textit{ch}ihte\\ 
20 & bôt sîn gerihte\\ 
 & unt anders wandels genuoc\\ 
 & Lyppaut, der unschulde truoc.\\ 
 & ez wære krump oder sleht,\\ 
 & er gerte sîner genôze reht\\ 
25 & \textbf{hof}, dâ die vürsten wæren,\\ 
 & unt er wære zuo disen mæren\\ 
 & komen âne schulde.\\ 
 & genædeclîcher hulde\\ 
 & er vaste sînen hêrren bat.\\ 
30 & dem tet \textbf{der} zorn ûf vreuden mat.\\ 
\end{tabular}
\scriptsize
\line(1,0){75} \newline
D \newline
\line(1,0){75} \newline
\textbf{1} \textit{Initiale} D  \textbf{7} \textit{Majuskel} D  \textbf{15} \textit{Majuskel} D  \newline
\line(1,0){75} \newline
\textbf{18} Obie] Obîe D \textbf{19} ungeschihte] vngesîhte D \textbf{22} Lyppaut] Lyppaot D \newline
\end{minipage}
\hspace{0.5cm}
\begin{minipage}[t]{0.5\linewidth}
\small
\begin{center}*m
\end{center}
\begin{tabular}{rl}
 & "wem ir \textbf{iht} lîhet, der \textbf{habe} ouch daz",\\ 
 & sprach si. "mîn zil sich hœhet baz.\\ 
 & ich wil von nieman lêh\textit{e}n hân.\\ 
 & mîn vrîheit ist \textbf{alsô} getân,\\ 
5 & ieglîcher krône hôch genuoc,\\ 
 & die irdensch houbet ie getruoc."\\ 
 & er sprach: "ir sît gelêret,\\ 
 & daz ir hôchvart sus mêret.\\ 
 & sît iuwer vater gap den rât,\\ 
10 & er wandelt mir die missetât.\\ 
 & ich sol hie wâpen alsô tragen,\\ 
 & daz wirt gestochen und geslagen.\\ 
 & ez sî \textbf{strîten} oder turnei,\\ 
 & hie belîbet vil der sper enzwei."\\ 
15 & mit zorne schiet er von der maget.\\ 
 & sîn \textbf{zürnen} sêre wart geklaget\\ 
 & von al der massenîe.\\ 
 & \textbf{in} klagete ouch Obie.\\ 
 & gegen dirre ungeschihte\\ 
20 & bôt sîn gerihte\\ 
 & \hspace*{-.7em}\big| Lipp\textit{ou}t, der unschulde truoc,\\ 
 & \hspace*{-.7em}\big| und anders wandels genuoc.\\ 
 & ez wære krump oder sleht,\\ 
 & er gerte sîner genôze reht\\ 
25 & \textbf{hof}, dâ die vürsten wæren,\\ 
 & und er wære zuo disen mæren\\ 
 & komen âne schuld\textit{e}.\\ 
 & genæ\textit{de}clîcher hulde\\ 
 & er vaste sînen hêrren bat.\\ 
30 & dem tet \textbf{der} zorn ûf vröuden mat.\\ 
\end{tabular}
\scriptsize
\line(1,0){75} \newline
m n o \newline
\line(1,0){75} \newline
\newline
\line(1,0){75} \newline
\textbf{1} ouch] \textit{om.} n o \textbf{2} mîn] ẏme o \textbf{3} lêhen] lehan m \textbf{5} krône] knone o \textbf{12} wirt] wart o \textbf{13} strîten] zuͯ strite n strite o \textbf{14} sper] spor o \textbf{17} al der] alter n alder der o \textbf{18} ouch] uch o  $\cdot$ Obie] obẏe n \textbf{22} Lippout] Lippaot m n Luppoot o  $\cdot$ unschulde] von schulden n o \textbf{23} krump] kuͯmph o \textbf{24} sîner] sine o  $\cdot$ genôze] genossen n o \textbf{25} dâ] do n o  $\cdot$ vürsten] furster o \textbf{27} schulde] schulden m \textbf{28} genædeclîcher] Geneklicher m \textbf{29} sînen] [siner]: sinen m \textbf{30} der] er n  $\cdot$ ûf] an n o \newline
\end{minipage}
\end{table}
\newpage
\begin{table}[ht]
\begin{minipage}[t]{0.5\linewidth}
\small
\begin{center}*G
\end{center}
\begin{tabular}{rl}
 & "swem ir \textbf{iht} lîhet, der \textbf{diene} \textit{ouch} daz",\\ 
 & sprach si. "mîn zil sich hœhet baz.\\ 
 & ich wil von niemen lêhen hân.\\ 
 & mîn vrîheit ist \textbf{\textit{s}ô} getân,\\ 
5 & ieslîcher krône hôch genuoc,\\ 
 & die irdesch houbet ie getruoc."\\ 
 & er sprach: "ir sît gelêret,\\ 
 & daz ir hôchvart sus mêret.\\ 
 & sît iwer vater gap den rât,\\ 
10 & er wandelt mir die missetât.\\ 
 & ich sol hie wâpen alsô tragen,\\ 
 & daz wirt gestochen unt geslagen.\\ 
 & ez sî \textbf{strît\textit{en}} oder turnei,\\ 
 & hie belîbet vil der sper enzwei."\\ 
15 & mit zorne schiet er von der maget.\\ 
 & sîn \textbf{zürnen} sêre wart geklaget\\ 
 & von al der massenîe.\\ 
 & \textbf{in} klagte ouch Obie.\\ 
 & gein dirre ungeschihte\\ 
20 & bôt sîn gerihte\\ 
 & unde anders wandels genuoc\\ 
 & Libaut, der unschulde truoc.\\ 
 & ez wære krump oder sleht,\\ 
 & er gerte sîner genôze reht\\ 
25 & \textbf{hof}, dâ die vürsten wæren,\\ 
 & \textit{und} er wære ze disen mæren\\ 
 & komen \textit{\textbf{gar}} âne schulde.\\ 
 & genædiclîcher hulde\\ 
 & er \textit{vast}e sînen hêrren bat.\\ 
30 & dem tet \textbf{der} zorn ûf vröuden mat.\\ 
\end{tabular}
\scriptsize
\line(1,0){75} \newline
G I O L M Q R Z \newline
\line(1,0){75} \newline
\textbf{1} \textit{Initiale} I O L M Z   $\cdot$ \textit{Capitulumzeichen} R  \textbf{7} \textit{Capitulumzeichen} R  \textbf{15} \textit{Initiale} I  \newline
\line(1,0){75} \newline
\textbf{1} swem] ÷wem O Wem L (M) Q R  $\cdot$ ir] er Z  $\cdot$ lîhet] lihe Z  $\cdot$ der] dem R  $\cdot$ ouch] \textit{om.} G \textbf{2} sprach] sprach sprach I  $\cdot$ si] \textit{om.} Z  $\cdot$ sich hœhet] si hohet O Z hoͯcht sich R \textbf{3} wil] enwil L (M) (Q) Z \textbf{4} ist] div ist O  $\cdot$ sô] also G \textbf{5} hôch] \textit{om.} I \textbf{6} die] diu I  $\cdot$ irdesch] ir dicke M \textbf{8} hôchvart sus] suͯs hohvart L  $\cdot$ mêret] gemert I \textbf{10} er] Jr O L M \textbf{11} hie wâpen] wapfen hie O \textbf{12} daz] Do Q  $\cdot$ gestochen] gesochen L \textbf{13} strîten] strit G  $\cdot$ turnei] tvrne Z \textbf{14} vil der] wil der Q manic Z \textbf{15} er] \textit{om.} Q \textbf{16} zürnen] zcorn M \textbf{17} al der] der L alter Q \textbf{18} klagte] clagt I (L) (R) Z  $\cdot$ Obie] froͮ Obie I Obŷe O abie M oblye Q obye R Z \textbf{19} gein] Gen im R \textbf{21} wandels] wandes R \textbf{22} Libaut] Lybavt O Lybavch L Lybaut Q Lybant R Lybayt Z \textbf{24} gerte] gert Z  $\cdot$ genôze] [grossez]: grossen Q gnosen R (Z) \textbf{25} dâ] do O Q \textbf{26} und er] er G Vnd Z \textbf{27} gar âne] ane G an alle L  $\cdot$ schulde] schuldin M \textbf{29} vaste] ditche G \textbf{30} dem] Den Q  $\cdot$ der] er L M den Q  $\cdot$ vröuden] freude I \newline
\end{minipage}
\hspace{0.5cm}
\begin{minipage}[t]{0.5\linewidth}
\small
\begin{center}*T
\end{center}
\begin{tabular}{rl}
 & "\begin{large}S\end{large}wem ir lîhet, der \textbf{gediene} ouch daz",\\ 
 & sprach si. "mîn zil sich hœhet baz.\\ 
 & ich wil von niemanne lêhen hân.\\ 
 & mîn vrîheit, \textbf{diu} ist \textbf{sô} getân,\\ 
5 & ieslîcher krône hôch genuoc,\\ 
 & die irdesch houbet ie getruoc."\\ 
 & er sprach: "ir sît \textbf{ez} gelêret,\\ 
 & daz ir hôchvart sus mêret.\\ 
 & sît iuwer vater gap den rât,\\ 
10 & er wandelt mir die missetât.\\ 
 & ich sol hie wâpen alsô tragen,\\ 
 & daz wirt gestochen unde geslagen.\\ 
 & ez sî \textbf{strît} oder turnei,\\ 
 & hie blîbet vil der sper enzwei."\\ 
15 & mit zorne schiet er von der maget.\\ 
 & sîn \textbf{zorn} sêre wart geklaget\\ 
 & von alder massenîe.\\ 
 & \textbf{daz} klaget ouch Obie.\\ 
 & gegen dirre ungeschihte\\ 
20 & bôt sîn gerihte\\ 
 & unde anders wandels genuoc\\ 
 & Lybaut, der unschulde truoc.\\ 
 & ez wære krump oder sleht,\\ 
 & er gerte sîner genôze reht\\ 
25 & \textbf{ze h\textit{o}ve}, d\textit{â} d\textit{ie} vürsten wæren,\\ 
 & unde er wære ze disen mæren\\ 
 & komen âne schulde.\\ 
 & gnædeclîcher hulde\\ 
 & er vaste sînen hêrren bat.\\ 
30 & dem tet zorn ûf vreuden mat.\\ 
\end{tabular}
\scriptsize
\line(1,0){75} \newline
T V W \newline
\line(1,0){75} \newline
\textbf{1} \textit{Initiale} T W  \newline
\line(1,0){75} \newline
\textbf{1} Swem] WEm W  $\cdot$ ir] ir iht V  $\cdot$ gediene] habe V diene W \textbf{4} diu ist sô] ist also W \textbf{5} hôch] \textit{om.} W \textbf{7} ez] \textit{om.} W \textbf{9} gap] v́ch gap V \textbf{13} strît] striten V \textbf{14} der] \textit{om.} W \textbf{16} zorn] zúrnen W \textbf{17} alder] aller W \textbf{18} klaget] klagete W  $\cdot$ Obie] obŷe T \textbf{20} sîn] sy ir W \textbf{22} Lybaut] Lẏbaut V Lybot W  $\cdot$ unschulde] vnschuldig W \textbf{25} zehve die da vursten wêren T  $\cdot$ dâ] do V W \textbf{27} âne] gar ane W \textbf{30} dem] Den W  $\cdot$ zorn ûf] er zorn an V \newline
\end{minipage}
\end{table}
\end{document}
