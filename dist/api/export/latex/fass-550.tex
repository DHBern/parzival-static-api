\documentclass[8pt,a4paper,notitlepage]{article}
\usepackage{fullpage}
\usepackage{ulem}
\usepackage{xltxtra}
\usepackage{datetime}
\renewcommand{\dateseparator}{.}
\dmyyyydate
\usepackage{fancyhdr}
\usepackage{ifthen}
\pagestyle{fancy}
\fancyhf{}
\renewcommand{\headrulewidth}{0pt}
\fancyfoot[L]{\ifthenelse{\value{page}=1}{\today, \currenttime{} Uhr}{}}
\begin{document}
\begin{table}[ht]
\begin{minipage}[t]{0.5\linewidth}
\small
\begin{center}*D
\end{center}
\begin{tabular}{rl}
\textbf{550} & \begin{large}D\end{large}em wirte ein bette ouch wart geleit.\\ 
 & dar nâ\textit{c}h \textbf{er}, ein ander knappe, treit\\ 
 & dar \textbf{vür} tischlachen und brôt.\\ 
 & der wirt \textbf{den} bêden daz gebôt;\\ 
5 & dâ gienc diu \textbf{hûsvrouwe} nâch.\\ 
 & d\textit{ô} diu Gawanen \textbf{sach},\\ 
 & si enpfienc in \textbf{herzenlîche}.\\ 
 & \textbf{si} sprach: "ir hât uns rîche\\ 
 & \textbf{nû alrêst} gemachet.\\ 
10 & hêrre, unser sælde wachet."\\ 
 & Der wirt \textbf{kom}. daz wazzer man dar truoc;\\ 
 & dô \textbf{sich} Gawan getwuoc,\\ 
 & eine bete er niht vermeit;\\ 
 & er bat den wirt gesellecheit:\\ 
15 & "lât mit mir ezzen dise magt."\\ 
 & "Hêrre, ez ist si gar verdagt,\\ 
 & daz si mit hêrren æze\\ 
 & oder in sô nâhe sæze;\\ 
 & si würde \textbf{lîhte} mir ze hêr.\\ 
20 & doch habe wir iwer genozzen mêr.\\ 
 & tohter, leiste al sîne ger;\\ 
 & des bin ich mit der volge wer."\\ 
 & Diu süeze wart von scheme rôt,\\ 
 & doch tet si, daz der wirt gebôt:\\ 
25 & zuo Gawane saz vrou Bene.\\ 
 & starker süne zwêne\\ 
 & hete der wirt \textbf{ouch} erzogen.\\ 
 & \textbf{nû} hete daz sprinzelîn ervlogen\\ 
 & \textbf{des âbents} drî galander;\\ 
30 & die \textbf{hiez er} mit ein ander\\ 
\end{tabular}
\scriptsize
\line(1,0){75} \newline
D \newline
\line(1,0){75} \newline
\textbf{1} \textit{Initiale} D  \textbf{11} \textit{Majuskel} D  \textbf{16} \textit{Majuskel} D  \textbf{23} \textit{Majuskel} D  \newline
\line(1,0){75} \newline
\textbf{2} nâch er] naher D \textbf{6} dô] di D \newline
\end{minipage}
\hspace{0.5cm}
\begin{minipage}[t]{0.5\linewidth}
\small
\begin{center}*m
\end{center}
\begin{tabular}{rl}
 & dem wirte ein bette ouch wart geleit.\\ 
 & dar nâch ein ander knappe treit\\ 
 & dâ \textbf{vür} \textit{t}i\textit{sch}lachen und brôt.\\ 
 & der wirt \textbf{den} beiden daz gebôt;\\ 
5 & dô gienc diu \textbf{hûsvrouwe} nâch.\\ 
 & dô diu Gawanen \textbf{sach},\\ 
 & si enpfienc in \textbf{herzelîch}.\\ 
 & \textbf{si} sprach: "ir habt uns rîch\\ 
 & \textbf{nû allerêrst} gemachet.\\ 
10 & hêrre, unser sæl\textit{d}e wac\textit{h}et."\\ 
 & \begin{large}D\end{large}er wirt \textbf{kam}. daz wazzer \textit{man dar truoc};\\ 
 & dô \textbf{sich} Gawan getwuo\textit{c},\\ 
 & ein bete er niht vermeit;\\ 
 & er bat den wirt gesellicheit:\\ 
15 & "lât mit mir ezzen dise maget."\\ 
 & "hêrre, ez ist \textit{si} gar ver\textit{d}aget,\\ 
 & daz si mit hêrre\textit{n} \textit{æ}ze\\ 
 & oder in sô nâhe \textbf{\textit{ih}t} sæze;\\ 
 & si würde \textbf{lîht} mir zuo hêr.\\ 
20 & doch hân wir iuwer \dag grôzen\dag  mêr.\\ 
 & tohter, leiste alle sîn ger;\\ 
 & des bin ich mit der volge wer."\\ 
 & diu süeze wart von schame rôt,\\ 
 & doch tet si, daz der wirt gebôt:\\ 
25 & zuo Gawan saz vrouwe Bene.\\ 
 & starker süne zwêne\\ 
 & hete der wirt erzogen.\\ 
 & \textbf{nû} het daz sprinzelîn er\textit{vl}ogen\\ 
 & \textbf{des âbendes} drî galander;\\ 
30 & die \textbf{hiez er} mit ein ander\\ 
\end{tabular}
\scriptsize
\line(1,0){75} \newline
m n o \newline
\line(1,0){75} \newline
\textbf{11} \textit{Initiale} m  \newline
\line(1,0){75} \newline
\textbf{1} ouch] \textit{om.} n  $\cdot$ wart] gewart o \textbf{3} tischlachen] die flachen m dieslachen o \textbf{7} herzelîch] herczeliclich o \textbf{10} sælde wachet] selbe wachset m \textbf{11} man dar truoc] truͯg man dar m \textbf{12} getwuoc] gettwuͯg gar m \textbf{13} er] ir o \textbf{14} gesellicheit] geselleclicheit o \textbf{16} si gar verdaget] hie gar verzaget m \textbf{17} hêrren æze] herren aussen vnd esse m \textbf{18} iht] nit m \textbf{26} [D*]: Starcker suͯn [suͯ*]: suͯn zwene o \textbf{27} erzogen] auch erzoigen o \textbf{28} ervlogen] erzogen m \textbf{29} drî] dru o  $\cdot$ galander] galinder o \newline
\end{minipage}
\end{table}
\newpage
\begin{table}[ht]
\begin{minipage}[t]{0.5\linewidth}
\small
\begin{center}*G
\end{center}
\begin{tabular}{rl}
 & \begin{large}D\end{large}em wirte ein bet ouch wart geleit.\\ 
 & dar nâch ein ander knappe treit\\ 
 & dar \textbf{vür} tischlachen unde brôt.\\ 
 & der wirt \textbf{den} bêden daz gebôt;\\ 
5 & dâ gienc diu \textbf{hûsvrouwe} nâch.\\ 
 & dô diu Gawanen \textbf{sach},\\ 
 & si enpfienc in \textbf{herzenlîche}.\\ 
 & \textbf{si} sprach: "ir habet uns rîche\\ 
 & \textbf{nû alrêrst} gemachet.\\ 
10 & hêrre, unser sælde wachet."\\ 
 & der wirt \textbf{kom}. daz wazzer man dar truoc;\\ 
 & dô \textbf{sich} Gawan getwuoc,\\ 
 & eine bet er niht vermeit;\\ 
 & er bat den wirt gesellicheit:\\ 
15 & "lât mit mir ezzen dise maget."\\ 
 & "hêrre, ez ist si gar verdaget,\\ 
 & daz si mit hêrren æze,\\ 
 & o\textit{d}e\textit{r} in sô nâhen sæze;\\ 
 & si würde \textit{\textbf{lîhte}} \textit{mir} ze hêr.\\ 
20 & doch habe wir iuwer genozzen mêr.\\ 
 & tohter, leist al sîne ger;\\ 
 & des bin ich mit der volge wer."\\ 
 & diu süeze wart von scham rôt,\\ 
 & doch tet \textit{si}, daz der wirt gebôt:\\ 
25 & zuo Gawane saz vrô Bene.\\ 
 & starker süne zwêne\\ 
 & het der wirt \textbf{ouch} erzogen.\\ 
 & \textbf{nû} hete daz sprinzelîn ervlogen\\ 
 & \textbf{des âbendes} drî galander;\\ 
30 & die \textbf{hiez er} mit ein ander\\ 
\end{tabular}
\scriptsize
\line(1,0){75} \newline
G I L M Z \newline
\line(1,0){75} \newline
\textbf{1} \textit{Initiale} G I L Z  \textbf{23} \textit{Initiale} I  \newline
\line(1,0){75} \newline
\textbf{1} ouch] \textit{om.} M Z  $\cdot$ geleit] bereit Z \textbf{5} dâ] do I (L) \textbf{6} dô] Da L M Z \textbf{10} unser] vnse M \textbf{11} wirt] \textit{om.} Z \textbf{12} dô] Da M  $\cdot$ getwuoc] twuc M \textbf{14} gesellicheit] durc Gesellecheit I \textbf{16} ez] des Z  $\cdot$ ist si] si \sout{ev} I ist So M  $\cdot$ verdaget] vorsagit M \textbf{17} æze] esszin M \textbf{18} oder] obe G \textbf{19} lîhte mir] mir lihte G \textbf{21} leist] du laist I \textbf{23} von] mit I vor M \textbf{24} si] er G  $\cdot$ daz] waz L (M) \textbf{25} Gawane] Gawan I (Z) \textbf{28} ervlogen] ervolgen M \newline
\end{minipage}
\hspace{0.5cm}
\begin{minipage}[t]{0.5\linewidth}
\small
\begin{center}*T
\end{center}
\begin{tabular}{rl}
 & \textit{\begin{large}D\end{large}}em wirte ein bette ouch wart geleit.\\ 
 & dar nâch ein ander knappe treit\\ 
 & dar \textbf{ûf} tischl\textit{a}chen unde brôt.\\ 
 & der wirt \textbf{in} beiden daz gebôt;\\ 
5 & dô gie diu \textbf{wirtinne} nâch.\\ 
 & dô diu Gawanen \textbf{ersach},\\ 
 & si enpfienc in \textbf{hêrlîche}\\ 
 & \textbf{unde} sprach: "ir habt uns rîche\\ 
 & \textbf{alrêrst nû} gemachet.\\ 
10 & hêrre, unser sælde wachet."\\ 
 & der wirt \textbf{gienc în}. daz wazzer man dar truoc;\\ 
 & dô \textbf{im} Gawan getwuoc,\\ 
 & eine bete er niht vermeit;\\ 
 & er bat den wirt gesellecheit:\\ 
15 & "Lât mit mir ezzen dise maget."\\ 
 & "hêrre, ez ist si gar verdaget,\\ 
 & daz si mit hêrren æze\\ 
 & oder in sô nâhe sæze;\\ 
 & si würde mir \textbf{gar} ze hêr.\\ 
20 & doch hân wir iuwer genozzen mêr.\\ 
 & tohter, leiste alsîne ger;\\ 
 & des bin ich mit der volge wer."\\ 
 & \textit{\begin{large}D\end{large}}iu süeze wart von schame rôt,\\ 
 & doch tet si, daz der wirt gebôt:\\ 
25 & Zuo Gawane saz vrou Bene.\\ 
 & starker süne zwêne\\ 
 & hete der wirt \textbf{ouch} erzogen.\\ 
 & \textbf{ouch} hete daz sprinz\textit{e}lîn ervlogen\\ 
 & drîe galander;\\ 
30 & die \textbf{wurden} mit ein ander\\ 
\end{tabular}
\scriptsize
\line(1,0){75} \newline
T U V W O Q R \newline
\line(1,0){75} \newline
\textbf{1} \textit{Initiale} T U V  \textbf{11} \textit{Capitulumzeichen} R  \textbf{15} \textit{Majuskel} T  \textbf{23} \textit{Initiale} T U V W  \textbf{25} \textit{Majuskel} T  \newline
\line(1,0){75} \newline
\textbf{1} \textit{teilsweise Textverlust 549.21-550.7 (Blatt teils abgeschnitten)} O   $\cdot$ Dem] ÷em T  $\cdot$ ein bette ouch wart] ein bette wart auch U (V) ::: ander bette wart O ein bette wart Q och ward R \textbf{2} ein] der R  $\cdot$ treit] reit Q \textbf{3} ûf] fúr V  $\cdot$ tischlachen] tischlaichen T \textbf{4} daz] \textit{om.} R \textbf{5} dô] Da R \textbf{6} dô] :a O  $\cdot$ diu] sy R  $\cdot$ Gawanen] Gawan V geiwanen Q her Gawinen R  $\cdot$ ersach] sach R \textbf{7} hêrlîche] hertzeleich W (O) (Q) erliche R \textbf{8} unde] Si O  $\cdot$ rîche] gemacht Riche R \textbf{9} nû] \textit{om.} O \textbf{11} gienc în daz] kam V W (O) (Q) R  $\cdot$ dar] do Q \textbf{12} im Gawan] Gawan die hende V im Gawin R  $\cdot$ getwuoc] tuͦg R \textbf{13} \textit{Vers 550.13 fehlt} Q  \textbf{15} dise] die V \textbf{16} ez] daz O  $\cdot$ si gar] úch R  $\cdot$ verdaget] vnuersagt R \textbf{18} in] im V \textbf{19} würde] wuͦrden U wurde lihte V (O) (Q) (R) wúrd vil leicht W  $\cdot$ mir gar] mir V W O R \textit{om.} Q \textbf{20} iuwer] tiwer O \textbf{21} alsîne] sine O \textbf{22} des] Das Q Der R  $\cdot$ mit] \textit{om.} U \textbf{23} Diu] ÷iv T  $\cdot$ süeze] dochtter R  $\cdot$ von] vor O Q \textbf{24} daz] swaz O \textbf{25} Gawane] gawan W (O) Gawin R \textbf{27} ouch] \textit{om.} U  $\cdot$ erzogen] gezogen W \textit{om.} R \textbf{28} ouch] Nu U (V) (W) (O) (Q) (R)  $\cdot$ daz sprinzelîn ervlogen] daz sprinzerlin ervlogen T U der wirt auch erflohen W \textbf{29} Dez abendez drie galander V \newline
\end{minipage}
\end{table}
\end{document}
