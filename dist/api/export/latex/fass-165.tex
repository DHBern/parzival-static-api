\documentclass[8pt,a4paper,notitlepage]{article}
\usepackage{fullpage}
\usepackage{ulem}
\usepackage{xltxtra}
\usepackage{datetime}
\renewcommand{\dateseparator}{.}
\dmyyyydate
\usepackage{fancyhdr}
\usepackage{ifthen}
\pagestyle{fancy}
\fancyhf{}
\renewcommand{\headrulewidth}{0pt}
\fancyfoot[L]{\ifthenelse{\value{page}=1}{\today, \currenttime{} Uhr}{}}
\begin{document}
\begin{table}[ht]
\begin{minipage}[t]{0.5\linewidth}
\small
\begin{center}*D
\end{center}
\begin{tabular}{rl}
\textbf{165} & daz si sîn dienst næme.\\ 
 & sîn varwe der minne zæme."\\ 
 & \textbf{der wirt sprach}: "nû sule wir sehen,\\ 
 & an \textbf{des wæte} \textbf{ein} wunder ist geschehen."\\ 
5 & \textbf{\begin{large}S\end{large}i giengen}, dâ si vunden\\ 
 & Parzivalen, den wunden\\ 
 & von eime sper, daz beleip \textbf{doch} ganz.\\ 
 & \textbf{sîn underwant sich} Gurnemanz.\\ 
 & sölh was sîn underwinden,\\ 
10 & daz ein vater sînen kinden,\\ 
 & der sich triwe kunde nieten,\\ 
 & \textbf{möhtez in niht baz} erbieten.\\ 
 & sîne wunden wuosch und bant\\ 
 & der wirt mit sîn selbes hant.\\ 
15 & \textbf{Dô} was ouch ûf geleit daz brôt.\\ 
 & des was dem jungen gaste nôt,\\ 
 & wand in grôz hunger niht vermeit.\\ 
 & al vastende er des morgens reit\\ 
 & von dem vischære.\\ 
20 & sîn wunde unt harnasch swære,\\ 
 & die \textbf{vor Nantes er} bejagete,\\ 
 & im müede unt hunger sagete\\ 
 & \textbf{unt} diu verre tagereise\\ 
 & von Artuse dem Berteneise,\\ 
25 & dâ man in allenthalben vasten liez.\\ 
 & der wirt in mit im ezzen hiez.\\ 
 & \textbf{der gast} sich dâ gelabte.\\ 
 & in den \textbf{barn} er sich \textbf{sô} habte,\\ 
 & \textbf{daz} er der spîse swande vil.\\ 
30 & daz nam der wirt gar zeime spil.\\ 
\end{tabular}
\scriptsize
\line(1,0){75} \newline
D \newline
\line(1,0){75} \newline
\textbf{5} \textit{Initiale} D  \textbf{15} \textit{Majuskel} D  \newline
\line(1,0){75} \newline
\textbf{24} Berteneise] beriteneise D \newline
\end{minipage}
\hspace{0.5cm}
\begin{minipage}[t]{0.5\linewidth}
\small
\begin{center}*m
\end{center}
\begin{tabular}{rl}
 & daz si sînen dienst næm\textit{e}.\\ 
 & sîn varwe der minnen zæm\textit{e}."\\ 
 & \textbf{der wirt \textit{sprach}}: "nû sullen wir sehen,\\ 
 & an \textbf{des wæte} \textbf{ein} wunder ist geschehen."\\ 
5 & \textbf{sus giengen si}, dâ si vunden\\ 
 & Parcifalen, den wunden\\ 
 & von einem sper, daz beleip ganz.\\ 
 & \textbf{sîn underwant}  G\textit{ur}nemanz.\\ 
 & solich was sîn underwinden,\\ 
10 & daz ein vater sînen kinden,\\ 
 & der sich triuwe kunde nieten,\\ 
 & \textbf{niht \textit{b}az m\textit{ö}hte} erbieten.\\ 
 & sîne wunden wuosch und bant\\ 
 & der wirt mit sîn selbes hant.\\ 
15 & \textbf{dô} was ouch ûf geleit daz brôt.\\ 
 & des was dem \textit{j}u\textit{n}gen gaste nôt,\\ 
 & wand in grôz hunger niht vermeit.\\ 
 & al vastende er des morgens reit\\ 
 & von \textbf{sînem wirt}, dem vischære.\\ 
20 & sîn wunde und harnasch swære,\\ 
 & die \textbf{er vor Nantes} bejagete,\\ 
 & ime müede und hunger sagete\\ 
 & \textbf{und} diu verre tagereise\\ 
 & von Artuse dem Br\textit{i}t\textit{u}neise,\\ 
25 & d\textit{â} man in allenthalben vasten lie\textit{z}.\\ 
 & der wirt in mit im ezzen hiez.\\ 
 & \textbf{der gast} sich dâ gel\textit{a}bete.\\ 
 & \textit{in den \textbf{gebæren} er sich \textbf{sô} habete,}\\ 
 & \textbf{daz} er der spîse swande vil.\\ 
30 & daz nam der wirt gar ze einem spil.\\ 
\end{tabular}
\scriptsize
\line(1,0){75} \newline
m n o Fr69 \newline
\line(1,0){75} \newline
\newline
\line(1,0){75} \newline
\textbf{1} sînen dienst] sin dienste n o  $\cdot$ næme] nemen m \textbf{2} minnen] mynne n (o)  $\cdot$ zæme] zemen m \textbf{3} sprach] \textit{om.} m \textbf{4} ist] der ist n  $\cdot$ geschehen] beschehen n o \textbf{5} dâ] do n o \textbf{6} Parcifalen] Parcifaln o  $\cdot$ wunden] sie wuͯnden o \textbf{7} sper] sprer o  $\cdot$ ganz] do gantz n doch gancz o \textbf{8} Gurnemanz] Grunemancz m gurnamantz n guͯrnamancz o \textbf{11} triuwe kunde] truwen kinden o \textbf{12} baz] das m  $\cdot$ möhte] mochte m kunde n (o) \textbf{16} des] Das o  $\cdot$ jungen] mugen m \textbf{17} grôz] grossen n \textbf{20} sîn wunde] Vrkúnde n (o) \textbf{21} die er] Do n o  $\cdot$ vor] wor o \textbf{22} ime müede und] Muͯde vnd in n o \textbf{23} und diu] Alle die n (o) \textbf{24} von] An n An dem o  $\cdot$ Artuse] artusen o artus Fr69  $\cdot$ Brituneise] brutteneise m britaneise n o Britoneise Fr69 \textbf{25} dâ] Do m n o  $\cdot$ liez] liesse m \textbf{26} hiez] heis o \textbf{27} dâ] do n o  $\cdot$ gelabete] gelobette m \textbf{28} \textit{Vers 165.28 fehlt} m  \textbf{30} gar] \textit{om.} n o  $\cdot$ einem] eẏnen o \newline
\end{minipage}
\end{table}
\newpage
\begin{table}[ht]
\begin{minipage}[t]{0.5\linewidth}
\small
\begin{center}*G
\end{center}
\begin{tabular}{rl}
 & daz si sîn dienst næme.\\ 
 & sîn varwe der minne zæme."\\ 
 & \textbf{dô sprach der wirt}: "nû sulen wir sehen,\\ 
 & an \textbf{dem} \textbf{solch} wunder ist geschehen."\\ 
5 & \textbf{si giengen}, dâ si vunden\\ 
 & Parzivalen, den wunden\\ 
 & von einem sper, daz beleip \textbf{doch} ganz.\\ 
 & \textbf{sîn underwant sich} Gurnomanz.\\ 
 & solch was sîn underwinden,\\ 
10 & daz ein vater sînen kinden,\\ 
 & der sich triwe kunde nieten,\\ 
 & \textbf{m\textit{ö}ht i\textit{m}z niht baz} erbieten.\\ 
 & sîne wunden wuosch unde bant\\ 
 & der wirt mit sîn selbes hant.\\ 
15 & \textbf{nû} was ouch ûf geleit daz brôt.\\ 
 & des was dem jungen gaste nôt,\\ 
 & wan in grôz hunger niht vermeit.\\ 
 & al vastende er des morgens reit\\ 
 & von dem vischære.\\ 
20 & sîn wunde unde harnasch swære,\\ 
 & die \textbf{vor Nantis er} bejagete,\\ 
 & im müede unde hunger sagete\\ 
 & \textbf{\begin{large}U\end{large}nt} diu verre tagereise\\ 
 & von Artuse dem Britaneise,\\ 
25 & dô man in allenthalben vasten liez.\\ 
 & der wirt in mit im ezzen hiez.\\ 
 & \textbf{der gast} sich dô gelabte.\\ 
 & in den \textbf{barn} er sich \textbf{sô} habte,\\ 
 & \textbf{daz} er der spîse swande vil.\\ 
30 & daz nam der wirt gar zeinem spil.\\ 
\end{tabular}
\scriptsize
\line(1,0){75} \newline
G I O L M Q R Z Fr17 \newline
\line(1,0){75} \newline
\textbf{5} \textit{Initiale} O L R Z  \textbf{13} \textit{Initiale} I  \textbf{23} \textit{Initiale} G  \newline
\line(1,0){75} \newline
\textbf{2} varwe] frawe Q varwe ist R \textbf{3} dô] Da M Z  $\cdot$ wirt] wir Z  $\cdot$ sulen] soͯlt R \textbf{4} An des wat ist ein wuͯnder geschehen L  $\cdot$ dem solch] des varwe ein I (Fr17) des wat ein O (M) (Q) (Z) des watte R \textbf{5} si] ÷i O  $\cdot$ dâ] do Q \textbf{6} Parzivalen] [parzifaln]: Parzifaln I Parcifaln O Parcifalen L Z Partzival M Partzifaln Q Parczifaln R \textbf{7} \textit{nach 165.7:} \sout{Sich wurd wûnd} Q   $\cdot$ einem] einē Q einen R \textbf{8} sîn] Sich O Q R Sit L  $\cdot$ sich] sin O L (Q) R  $\cdot$ Gurnomanz] Garnomanz I cvrnamanz O gurnemanz M gurnomantz Q Gurnamanz R (Fr17) gvrnemantz Z \textbf{10} ein] sein Q  $\cdot$ sînen kinden] sinem chinde I \textbf{11} triwe] trwen Q \textbf{12} möht imz] moht inz G moht imz I (O) (L) (Fr17) Mochte vns M Mocht esz Q Er mocht Jms R Moht ez im Z  $\cdot$ niht baz] nicht daz M bas nit R \textbf{13} wunden] wúnde Q  $\cdot$ wuosch] wusk man I wusch her Q \textbf{14} \textit{nach 165.14:} \sout{der n::: mit sin selbes hant } Fr17  \textbf{15} nû] Do O L Q  $\cdot$ ouch] \textit{om.} I Q \textbf{16} des] daz I (M) (Q) (R)  $\cdot$ dem jungen gaste] Parcifale L dem junge gaste Q \textbf{18} al vastende] Auastunde I \textbf{20} sîn wunde] sinen wun I Sine wvnden L (Q) (R) (Z)  $\cdot$ unde] vnd sin I (O) (L) sin R \textbf{21} die] die er I (M)  $\cdot$ Nantis] nantes I (L) Natis R  $\cdot$ er] \textit{om.} I M \textbf{22} im] Jn L  $\cdot$ hunger] hvngerig L \textbf{24} Artuse] artus I M Q (R) artusen Z  $\cdot$ dem] \textit{om.} Z  $\cdot$ Britaneise] pritoneise I pritaneise O Brittonaise L britoneise Q Fr17 brytonayse R briteneise Z \textbf{25} dô] da I (O) (L) (M) (R) (Z) Fr17  $\cdot$ allenthalben] allez I Fr17 \textbf{26} im] \textit{om.} Z \textbf{27} dô gelabte] da Gelabte I (M) (Z) tagelabete Fr17 \textbf{28} barn] bran Fr17  $\cdot$ sô] do O Q R \textit{om.} L \textbf{29} daz] der I  $\cdot$ swande] wante M \textbf{30} wirt] \textit{om.} M  $\cdot$ gar] \textit{om.} I L M Q R im O  $\cdot$ zeinem] vuͤr ein I zeheinem Fr17 \newline
\end{minipage}
\hspace{0.5cm}
\begin{minipage}[t]{0.5\linewidth}
\small
\begin{center}*T
\end{center}
\begin{tabular}{rl}
 & daz si sîn dienst næme.\\ 
 & sîn varwe der minnen zæme."\\ 
 & \textbf{\begin{large}D\end{large}ô sprach der wirt}: "nû sule wir sehen,\\ 
 & an \textbf{des wæte} \textbf{ein} wunder ist geschehen."\\ 
5 & \textbf{si giengen}, dâ si vunden\\ 
 & Parcifal, den wunden\\ 
 & von einem sper, daz bleip \textbf{doch} ganz.\\ 
 & \textbf{sich underwant sîn} Gurnemanz.\\ 
 & Solch was sîn underwinden,\\ 
10 & daz ein vater sînen kinden,\\ 
 & der sich triuwen kunde nieten,\\ 
 & \textbf{niht baz m\textit{ö}hte} erbieten.\\ 
 & Sîne wunden wuosch unde bant\\ 
 & der wirt mit sîn selbes hant.\\ 
15 & \textbf{nû} was ouch ûf geleit daz brôt.\\ 
 & des was dem jungen gaste nôt,\\ 
 & wandin grôz hunger niht verm\textit{e}it.\\ 
 & alvastender des morgens reit\\ 
 & vonme vischære.\\ 
20 & sîn wunde unde \textbf{der} harnasch swære,\\ 
 & die\textbf{r vor Nantes} bejagete,\\ 
 & im müede unde hunger sagete,\\ 
 & di\textit{u} verre tagereise\\ 
 & von Artuse dem Brituneise,\\ 
25 & dâ man in allenthalben vasten liez.\\ 
 & der wirt in mit im ezzen hiez.\\ 
 & \textbf{\begin{large}D\end{large}ô er} sich dâ gelabete,\\ 
 & in den \textbf{barn} er sich \textbf{dô} habete,\\ 
 & \textbf{dâ} er der spîse swante vil.\\ 
30 & daz nam der wirt gar zeime spil.\\ 
\end{tabular}
\scriptsize
\line(1,0){75} \newline
T U V W \newline
\line(1,0){75} \newline
\textbf{3} \textit{Initiale} T U V  \textbf{6} \textit{Majuskel} T  \textbf{9} \textit{Majuskel} T  \textbf{13} \textit{Majuskel} T  \textbf{27} \textit{Initiale} T U W  \newline
\line(1,0){75} \newline
\textbf{2} minnen] minne U \textbf{3} wir] \textit{om.} W \textbf{4} des] der W \textbf{5} dâ] do U V W \textbf{6} Parcifal] Parcifaln U Parzifal V Partzifal W \textbf{7} bleip] was W \textbf{8} Gurnemanz] Guͦrnemanz U gurnamantz W \textbf{9} Solch] Selig U Soͤlchs W \textbf{11} triuwen] truͦwe U (V) \textbf{12} niht baz möhte] niht baz mohte T (U) (V) Moͤchtens nicht bas W \textbf{14} sîn] \textit{om.} W \textbf{17} vermeit] vermeheit T \textbf{18} alvastender] Alwege vastende U \textbf{20} wunde] wunden W \textbf{21} Nantes] nantis W \textbf{22} im] In W  $\cdot$ müede] muͦte U \textbf{23} diu] die T \textbf{24} Brituneise] Brituͦneise U \textbf{25} dâ] Do U V Das W  $\cdot$ vasten] vaste U \textbf{27} [D*]: Der gast sich do gelabete V  $\cdot$ dâ] [d*]: do U do W \textbf{28} den barn] dem barn U der wile V  $\cdot$ dô] \textit{om.} U V \textbf{29} dâ] Do U W [D*]: Daz V  $\cdot$ swante] wante U \textbf{30} gar] \textit{om.} U V \newline
\end{minipage}
\end{table}
\end{document}
