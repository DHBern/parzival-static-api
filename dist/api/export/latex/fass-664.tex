\documentclass[8pt,a4paper,notitlepage]{article}
\usepackage{fullpage}
\usepackage{ulem}
\usepackage{xltxtra}
\usepackage{datetime}
\renewcommand{\dateseparator}{.}
\dmyyyydate
\usepackage{fancyhdr}
\usepackage{ifthen}
\pagestyle{fancy}
\fancyhf{}
\renewcommand{\headrulewidth}{0pt}
\fancyfoot[L]{\ifthenelse{\value{page}=1}{\today, \currenttime{} Uhr}{}}
\begin{document}
\begin{table}[ht]
\begin{minipage}[t]{0.5\linewidth}
\small
\begin{center}*D
\end{center}
\begin{tabular}{rl}
\textbf{664} & \begin{large}D\end{large}az lobten si \textbf{al} gelîche.\\ 
 & die herzoginne rîche\\ 
 & si vrâgten, ob daz her wære ir.\\ 
 & \textbf{diu} sprach: "ir sult gelouben mir,\\ 
5 & ich erkenne dâ \textbf{weder} schilt noch man.\\ 
 & der mir \textbf{ê} \textbf{schaden hât} getân,\\ 
 & der ist lîhte in mîn lant geriten\\ 
 & unt hât vor Logroys gestriten.\\ 
 & ich wæne, die vant er doch ze wer.\\ 
10 & si heten strît wol disem her\\ 
 & an \textbf{zingeln} unt an barbigân.\\ 
 & hât dâ rîterschaft getân\\ 
 & der zornige künec Gramoflanz,\\ 
 & sô suochter gelt vür sînen kranz;\\ 
15 & oder swer si sint, \textbf{die} muosen sper\\ 
 & ûf geriht sehen durch \textbf{tjoste} ger."\\ 
 & Ir munt \textbf{in louc dâ} wênec an.\\ 
 & Artus schaden vil gewan,\\ 
 & ê daz er kœme vür Logroys.\\ 
20 & des wart etslîch Bertenoys\\ 
 & ze rehter tjost ab gevalt.\\ 
 & Artuses her \textbf{ouch} wider galt\\ 
 & market, den man in dâ bôt.\\ 
 & si kômen\textbf{s} ze bêder sît in nôt.\\ 
25 & Man sach die \textbf{strîtmüeden} komen,\\ 
 & \textbf{von den sô} dicke ist vernomen,\\ 
 & daz si ir kotzen gerne werten;\\ 
 & si wâren gegen strîte die herten.\\ 
 & \begin{large}B\end{large}eidenthalp ez mit schaden stêt.\\ 
30 & Garel unt Gaherjet\\ 
\end{tabular}
\scriptsize
\line(1,0){75} \newline
D \newline
\line(1,0){75} \newline
\textbf{1} \textit{Initiale} D  \textbf{17} \textit{Majuskel} D  \textbf{25} \textit{Majuskel} D  \textbf{29} \textit{Initiale} D  \newline
\line(1,0){75} \newline
\textbf{11} barbigân] Barbegan D \textbf{22} Artuses] Artvs D \textbf{30} Gaherjet] Gaherîet D \newline
\end{minipage}
\hspace{0.5cm}
\begin{minipage}[t]{0.5\linewidth}
\small
\begin{center}*m
\end{center}
\begin{tabular}{rl}
 & daz lobte\textit{n}s \textbf{al}gelîche.\\ 
 & die herzogîn rîche\\ 
 & si vrâgten, ob daz her wær ir.\\ 
 & \textbf{si} sprach: "ir solt gelouben mir,\\ 
5 & ich erkenne d\textit{â} schilt noch man.\\ 
 & der mir \textbf{ie} \textbf{hât schaden} getân,\\ 
 & der ist lîht in mîn lant geriten\\ 
 & und het vor Logrois gestriten.\\ 
 & ich wæne, die vant er doch zuo wer.\\ 
10 & si heten strît wol disem her\\ 
 & an \textbf{zingeln} und an barbigân.\\ 
 & het d\textit{â} ritterschaft getân\\ 
 & der zornige künic Gramolanz,\\ 
 & sô suochter gelt vür sînen kranz;\\ 
15 & oder wer si sint, \textbf{si} muosen sper\\ 
 & ûf gerihtet sehen durch \textbf{juste} ger."\\ 
 & ir munt \textbf{in \dag lanc\dag  d\textit{â}} wênic an.\\ 
 & Artus schaden vil g\textit{e}wan,\\ 
 & ê daz er kæme vür Logrois.\\ 
20 & des w\textit{a}rt etlîch Britun\textit{o}is\\ 
 & zuo rehter juste ab gevalt.\\ 
 & Artuses her \textbf{ouch} wider galt\\ 
 & m\textit{a}rket, den man in d\textit{â} bôt.\\ 
 & si kômen zuo beider sît i\textit{n} nôt.\\ 
25 & man sach die \textbf{strîtmüede} komen,\\ 
 & \textbf{von den sô} dicke ist vernomen,\\ 
 & daz si ir \dag küssen\dag  gerne werten;\\ 
 & si wâren gegen strîte die herten.\\ 
 & beidenthalp ez mit schaden stêt.\\ 
30 & Gar\textit{el} und Ga\textit{h}er\textit{i}et\\ 
\end{tabular}
\scriptsize
\line(1,0){75} \newline
m n o Fr69 \newline
\line(1,0){75} \newline
\newline
\line(1,0){75} \newline
\textbf{1} lobtens] [b]: lobtte us m lobetens sú n lopte us o  $\cdot$ algelîche] alle glich n o (Fr69) \textbf{5} dâ] do m n o ::: Fr69 \textbf{6} ie] ê Fr69 \textbf{7} lîht] liecht o [lic*]: lichhe Fr69 \textbf{8} Logrois] ::: Fr69 \textbf{9} wæne die vant] fant die zwene n \textbf{11} barbigân] barbigon m \textbf{12} het] Hette n  $\cdot$ dâ] do m n o \textbf{13} zornige] zoznige o  $\cdot$ Gramolanz] gramolancz m o gramolantz n \textbf{15} si muosen] sẏ muͯssen m (n) misse o \textbf{16} gerihtet] richtent o \textbf{17} dâ] do m n o \textbf{18} vil] wil o  $\cdot$ gewan] gawan m \textbf{20} wart] wert m  $\cdot$ Britunois] brittunis m britunois n breitomis o \textbf{22} Artuses] Artus m n o \textbf{23} market] murket m  $\cdot$ dâ] do m n o \textbf{24} kômen] koment m n (o)  $\cdot$ in] ir m \textbf{28} die herten] geherten n \textbf{30} Garel] Garolt m n Galrot o  $\cdot$ Gaheriet] Gahieret m gahierielat n gahieriet o \newline
\end{minipage}
\end{table}
\newpage
\begin{table}[ht]
\begin{minipage}[t]{0.5\linewidth}
\small
\begin{center}*G
\end{center}
\begin{tabular}{rl}
 & \begin{large}D\end{large}az lobten si \textbf{al}gelîche.\\ 
 & die herzoginne rîche\\ 
 & si vrâgten, obe daz her wære ir.\\ 
 & \textbf{si} sprach: "ir sult glouben mir,\\ 
5 & ich erkenne dâ \textbf{weder} schilt noch \textit{man}.\\ 
 & der mir \textbf{ê} \textbf{schaden hât} getân,\\ 
 & derst lîhte in mîn lant geriten\\ 
 & unde hât vor Logroys gestriten.\\ 
 & ich wæne, die vant er doch ze wer.\\ 
10 & si heten strît wol disem her\\ 
 & ane \textbf{zingel} unde ane barbigân.\\ 
 & hât dâ rîterschaft getân\\ 
 & der zornige künic Gramoflanz,\\ 
 & sô suochte er gelt vür sînen kranz;\\ 
15 & oder swer si sint, \textbf{si} muosen sper\\ 
 & ûf gerihtiu sehen durch \textbf{strîtes} ger."\\ 
 & ir munt \textbf{in louc dâ} wênic an.\\ 
 & Artus schaden vil gewan,\\ 
 & ê daz er kœme vür Logroys.\\ 
20 & des wart etslîch Britanoys\\ 
 & ze rehter tjost abe gevalt.\\ 
 & Artuses her \textbf{in} wider galt\\ 
 & market, den man in dâ bôt.\\ 
 & si kômen ze bêder sîte in nôt.\\ 
25 & man sach die \textbf{strîtmüeden} komen,\\ 
 & \textbf{dâ von vil} dicke ist vernomen,\\ 
 & daz si ir kotzen gerne werten;\\ 
 & si wâren gein strîte die herten.\\ 
 & bêdenthalb ez mit schaden stêt.\\ 
30 & Garel unde Gaharet\\ 
\end{tabular}
\scriptsize
\line(1,0){75} \newline
G I L M Z Fr45 \newline
\line(1,0){75} \newline
\textbf{1} \textit{Initiale} G L Z  \textbf{5} \textit{Initiale} I  \textbf{25} \textit{Initiale} I  \newline
\line(1,0){75} \newline
\textbf{1} lobten] geloptan I  $\cdot$ algelîche] alle gliche L (M) (Z) \textbf{2} die] diu I \textbf{3} vrâgten] fragte I  $\cdot$ daz] isz M \textbf{4} si] Die L (M) \textbf{5} ich] Jr Z  $\cdot$ erkenne] erchende I en erkenne M enkenne Z  $\cdot$ weder] \textit{om.} L M  $\cdot$ man] sper G \textbf{6} schaden] scheiden M \textbf{8} vor] von M (Z)  $\cdot$ Logroys] logroẏs G logrois M (Z) \textbf{10} wol] wol zuͤ I \textit{om.} M \textbf{11} zingel] cingeln L (Z) \textbf{13} zornige] zcorniger M  $\cdot$ Gramoflanz] gramorflanz M gramoflantz Z \textbf{14} suochte] sucht I (L) (M) (Z) \textbf{15} swer] wer L M  $\cdot$ si muosen] die muͦssten I (M) sý myͯsen L die muste Z \textbf{16} gerihtiu] gerihtet I  $\cdot$ strîtes] Tiostes I tiost L (M) (Z) \textbf{17} in] [ir]: in L  $\cdot$ dâ wênic] wenc dar I \textbf{19} Logroys] logroẏs G Fr45 Logroýs L Logrois M Z \textbf{20} etslîch] ieslich Fr45  $\cdot$ Britanoys] britanoẏs G pritonoys I Brittanoys L britaneis M britunois Z brittenoẏs Fr45 \textbf{22} Artuses] Artus G Z Fr45 Alture M  $\cdot$ her] he M  $\cdot$ in] hin I Z \textbf{23} dâ] do L \textbf{25} strîtmüeden] stritenden muͤde I \textbf{27} si] \textit{om.} Fr45 \textbf{28} strîte] strite ýe L \textbf{30} Garel] Karel L  $\cdot$ Gaharet] Gaheret L (M) gaheiret Z \newline
\end{minipage}
\hspace{0.5cm}
\begin{minipage}[t]{0.5\linewidth}
\small
\begin{center}*T
\end{center}
\begin{tabular}{rl}
 & daz lobten si \textbf{alle} glîche.\\ 
 & die herzogîn rîche\\ 
 & si vrâgten, ob daz her wære i\textit{r}.\\ 
 & \textbf{diu} sprach: "ir solt glouben mir,\\ 
5 & ich erkenne d\textit{â} \textbf{w\textit{e}der} schilt noch man.\\ 
 & de\textit{r} mir \textbf{ê} \textbf{schaden hât} getân,\\ 
 & der ist lîht in mîn lant geriten\\ 
 & und hât vor Logrois gestriten.\\ 
 & ich wæne, die vant er doch zuo wer.\\ 
10 & si heten strît wol disem her\\ 
 & ane \textbf{zingeln} und ane barbigân.\\ 
 & hât d\textit{â} ritterschaft getân\\ 
 & der zornige künic Gramoflanz,\\ 
 & sô suocht er gelt vür sînen kr\textit{a}nz;\\ 
15 & oder wer si sint, \textbf{die} muosen sper\\ 
 & ûf gerihtiu sehen durch \textbf{tjost} ger."\\ 
 & ir munt \textbf{dô louc in} wênic an.\\ 
 & Artus schaden vil g\textit{e}wan,\\ 
 & ê daz er k\textit{æ}m vür Logrois.\\ 
20 & des wart etslîch Britunois\\ 
 & zuo rehter tjost abe gevalt.\\ 
 & Artuses he\textit{r} \textbf{in} wider galt\\ 
 & market, den man in d\textit{â} b\textit{ô}t.\\ 
 & si kômen zuo bêder sîten in nôt.\\ 
25 & man sach die \textbf{strîte müede} komen,\\ 
 & \textbf{dâ von vil} dicke ist vernomen,\\ 
 & daz \textit{si} ir kotzen gern werten;\\ 
 & si wâren gên strîte die herten.\\ 
 & bêdenthal\textit{p} ez mit schaden stêt.\\ 
30 & Garel und Gaheriet\\ 
\end{tabular}
\scriptsize
\line(1,0){75} \newline
Q R W V \newline
\line(1,0){75} \newline
\textbf{3} \textit{Initiale} W  \newline
\line(1,0){75} \newline
\textbf{1} daz] Des W  $\cdot$ si alle glîche] s:::le >geliche< V \textbf{3} vrâgten] [vraget*]: vrageten V  $\cdot$ ir] in Q \textbf{4} diu] Sy W \textbf{5} dâ] do Q W V  $\cdot$ weder] werder Q  $\cdot$ man] deheinen man R [*]: man V \textbf{6} der] Des Q  $\cdot$ ê] ye R  $\cdot$ hât] hab R \textbf{8} Logrois] logroys Q V logris R lugrois W \textbf{9} Ich mein er vand sy doch zuͦ wer W \textbf{11} zingeln] zigel R  $\cdot$ und ane] vnd R vnd [an*]: ane V \textbf{12} dâ] do Q W V die R \textbf{13} Gramoflanz] gramoflantz Q W Gramoflancz R gramaflanz V \textbf{14} sô] Da R Do W  $\cdot$ kranz] krentz Q \textbf{15} wer] wier R swer V  $\cdot$ si] die R  $\cdot$ die] sv́ V  $\cdot$ muosen] muͯssent brechen manig R muͤssen W (V) \textbf{16} gerihtiu] gerichte R  $\cdot$ durch] nach W  $\cdot$ ger] voller ger R \textbf{17} dô louc in] in log da R in loug do W loͮg in do V \textbf{18} schaden] schadens W  $\cdot$ gewan] gawan Q \textbf{19} kæm] kam W  $\cdot$ Logrois] logroys Q logris R logroẏs V \textbf{20} Britunois] brittonois Q Britenis R britois W [britt*n*]: brittvnoẏs V \textbf{22} Artuses] Artus Q R W  $\cdot$ her] hett Q har R  $\cdot$ in] [*]: oͮch V \textbf{23} dâ] do Q R W V  $\cdot$ bôt] bat Q \textbf{24} sîten] sit R V \textbf{25} strîte müede] stritmuͯden R (W) (V) \textbf{26} [*]: Von den so dikke ist vil vernomen V \textbf{27} si] \textit{om.} Q \textbf{28} Sv́ warent gegen [*]: strite die herten V \textbf{29} bêdenthalp] bedenthalt Q \textbf{30} Garel] Karel R V Kaheb W  $\cdot$ Gaheriet] gaheret Q kacheret R kaheriet W V \newline
\end{minipage}
\end{table}
\end{document}
