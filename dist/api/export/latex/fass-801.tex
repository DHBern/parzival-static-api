\documentclass[8pt,a4paper,notitlepage]{article}
\usepackage{fullpage}
\usepackage{ulem}
\usepackage{xltxtra}
\usepackage{datetime}
\renewcommand{\dateseparator}{.}
\dmyyyydate
\usepackage{fancyhdr}
\usepackage{ifthen}
\pagestyle{fancy}
\fancyhf{}
\renewcommand{\headrulewidth}{0pt}
\fancyfoot[L]{\ifthenelse{\value{page}=1}{\today, \currenttime{} Uhr}{}}
\begin{document}
\begin{table}[ht]
\begin{minipage}[t]{0.5\linewidth}
\small
\begin{center}*D
\end{center}
\begin{tabular}{rl}
\textbf{801} & umbe sich siz deckelachen swanc,\\ 
 & \textbf{vürz bette ûfen teppech} spranc\\ 
 & Condwiramurs, diu lieht gemâl.\\ 
 & ouch umbevienc si Parzival.\\ 
5 & man sagte mir, si kusten sich.\\ 
 & Si sprach: "mir hât gelücke dich\\ 
 & gesendet, \textbf{herzen vreude} mîn!"\\ 
 & si bat in willekomen sîn.\\ 
 & "Nû solt ich zürnen. ine mac.\\ 
10 & \textbf{geêrt} sî \textbf{diu} wîle unt \textbf{dirre} tac,\\ 
 & der mir brâhte disen umbevanc,\\ 
 & dâ von mîn trûren wirdet kranc.\\ 
 & ich hân nû, des mîn herze gert.\\ 
 & sorge ist \textbf{an mir vil} ungewert."\\ 
15 & Nû erwachten ouch diu kindelîn,\\ 
 & Kardeiz unt Loherangrin,\\ 
 & \textbf{die} lâgen ûf dem bette \textbf{al} blôz.\\ 
 & Parzivalen des niht verdrôz,\\ 
 & er\textbf{n} kuste si minneclîche.\\ 
20 & Kyot, der \textbf{zühte} rîche,\\ 
 & bat die knaben dannen tragen.\\ 
 & \textbf{er} begunde ouch \textbf{al den} vrouwen sagen,\\ 
 & daz si ûzem gezelte giengen.\\ 
 & si tâtenz, dô si enpfiengen\\ 
25 & ir hêrren von langer reise.\\ 
 & Kyot, der kurteise,\\ 
 & bevalh der künegîn ir man.\\ 
 & al die juncvrouwen \textbf{er vuorte} dan.\\ 
 & dennoch was ez harte vruo.\\ 
30 & kamerære sluogen \textbf{die winden} zuo.\\ 
\end{tabular}
\scriptsize
\line(1,0){75} \newline
D \newline
\line(1,0){75} \newline
\textbf{6} \textit{Majuskel} D  \textbf{9} \textit{Majuskel} D  \textbf{15} \textit{Majuskel} D  \newline
\line(1,0){75} \newline
\textbf{3} Condwiramurs] Cvndwir amvrs D \textbf{4} Parzival] Parcifal D \textbf{17} die] di D \textbf{18} Parzivalen] Parcifalen D \newline
\end{minipage}
\hspace{0.5cm}
\begin{minipage}[t]{0.5\linewidth}
\small
\begin{center}*m
\end{center}
\begin{tabular}{rl}
 & umb sich si daz declachen swanc,\\ 
 & \textbf{vür daz bette ûf den teppich} spranc\\ 
 & Condwier amurs, diu lieht gemâl.\\ 
 & ouch umbevienc si Parcifal.\\ 
5 & man seite mir, si kusten sich.\\ 
 & si sprach: "mir hât glücke dich\\ 
 & gesendet, \textbf{herzevröude} mîn!"\\ 
 & si bat in willekomen sîn.\\ 
 & "nû solt ich zürnen. ich enmac.\\ 
10 & \textbf{geêret} sî \textbf{diu} wîle und \textbf{diser} tac,\\ 
 & der mir brâht disen umbevanc,\\ 
 & dâ von mîn trûren wirt kranc.\\ 
 & ich hân nû, des mîn herze gert.\\ 
 & sorge ist \textbf{an mir} ungewert."\\ 
15 & \textit{n}û erwachten ouch diu kindelîn,\\ 
 & Cardeiz und Lohelangrin,\\ 
 & \textbf{die} lâgen ûf dem bette \textbf{al}blôz.\\ 
 & Parcifaln des niht verdrôz,\\ 
 & er kuste si minneclîch.\\ 
20 & Kyot, der \textbf{zühten} rîch,\\ 
 & bat die knaben dannen tragen\\ 
 & \textbf{und} begunde ouch \textbf{allen} vrowen sagen,\\ 
 & daz si ûz dem gezelt giengen.\\ 
 & si tâtenz, dô si enpfiengen\\ 
25 & ir hêrren von langer reise.\\ 
 & Kyot, der kurteise,\\ 
 & bevalh der künigîn ir man.\\ 
 & alle die juncvrouwen \textbf{\textit{er} vuort\textit{e}} dan.\\ 
 & dannoch was ez harte vruo.\\ 
30 & \textbf{die} kamerer sluogen \textbf{die winden} zuo.\\ 
\end{tabular}
\scriptsize
\line(1,0){75} \newline
m n o V V' W \newline
\line(1,0){75} \newline
\textbf{5} \textit{Initiale} W  \textbf{15} \textit{Initiale} V  \newline
\line(1,0){75} \newline
\textbf{1} si daz declachen] sich das tegeliches o sie ez tegelich V' \textbf{2} ûf den teppich] vffes teppit V vf [daz b]: den tepich V' \textbf{3} Condwier amurs] Cundewier amúrs n Cunwir anurs o Kvndewiramurs V V' (W) \textbf{4} Parcifal] Parzefal V parzifal V' partzifal W \textbf{5} seite] saget V V' \textbf{7} herzevröude] [her*]: herze froide V her zuͦ froͤden W \textbf{9} solt ich] solt ir o  $\cdot$ ich enmac] in enmag V \textbf{10} diser] der W \textbf{12} mîn] mir V V'  $\cdot$ wirt] wart V' \textbf{14} ungewert] vil vngewert V vil vngemert V' \textbf{15} Nû] Suͯ m (n) (o) \textbf{16} Cardeiz] Cardeis m (n) o V V' W  $\cdot$ Lohelangrin] logelangrin o lohelangrein W \textbf{18} Parcifaln] Parcifalen n Parzefalen V Parzifal V' Herr partzifal W \textbf{19} kuste] enkvste V \textbf{20} Kyot] Kẏot m o \textbf{22} und] Er V V'  $\cdot$ begunde] beguͯnnen o  $\cdot$ allen] allen den V (V') \textbf{24} si] Die V' \textbf{25} langer] der langen V' \textbf{26} Kyot] Kẏot n  $\cdot$ kurteise] [kẏot]: kurteise o \textbf{27} bevalh] Befalsch o  $\cdot$ künigîn] konige o \textbf{28} die] \textit{om.} W  $\cdot$ er vuorte] furten m furte er V' \textbf{30} Do slugen sie die wende zu V'  $\cdot$ kamerer] [k*]: kamerere V kameren W  $\cdot$ winden] winde V W \newline
\end{minipage}
\end{table}
\newpage
\begin{table}[ht]
\begin{minipage}[t]{0.5\linewidth}
\small
\begin{center}*G
\end{center}
\begin{tabular}{rl}
 & umbe sich siz declachen swanc,\\ 
 & \textbf{ûfen tepech vür daz bette} spranc\\ 
 & Condwiramurs, diu lieht gemâl.\\ 
 & ouch umbevienc si Parzival.\\ 
5 & man sagte mir, si kusten sich.\\ 
 & si sprach: "mir hât gelücke dich\\ 
 & gesendet \textbf{her zen vröuden} mîn."\\ 
 & si bat in willekomen sîn.\\ 
 & "nû solde ich zürnen. ichne mac.\\ 
10 & \textbf{sælic} sî wîle unde tac,\\ 
 & der mir brâht disen umbevanc,\\ 
 & dâ von mîn trûren wirdet kranc.\\ 
 & ich hân nû, des mîn herze gert.\\ 
 & sorge ist \textbf{mînhalp} ungewert."\\ 
15 & nû erwacheten ouch diu kindelîn,\\ 
 & Kardeiz unde Loherangrin,\\ 
 & \textbf{diu} lâgen ûf dem bette \textbf{al}blôz.\\ 
 & Parzivalen des niht verdrôz,\\ 
 & er\textbf{ne} kuste si minneclîche.\\ 
20 & Kiot, der \textbf{zühte} rîche,\\ 
 & bat die knaben danne tragen.\\ 
 & \textbf{er} begunde ouch \textbf{al den} vrouwen sagen,\\ 
 & daz si ûzzem gezelte giengen.\\ 
 & si tâtenz, dô si enpfiengen\\ 
25 & ir hêrren von langer reise.\\ 
 & Kiot, der kurteise,\\ 
 & bevalh der künigîn ir man.\\ 
 & al die juncvrouwen \textbf{vuorte er} dan.\\ 
 & dannoch was ez harte vruo.\\ 
30 & kamerære sluogen zuo.\\ 
\end{tabular}
\scriptsize
\line(1,0){75} \newline
G I L M Z \newline
\line(1,0){75} \newline
\textbf{1} \textit{Initiale} L Z  \textbf{5} \textit{Initiale} M  \textbf{13} \textit{Initiale} I  \newline
\line(1,0){75} \newline
\textbf{1} ir dechelachen vmbe sich si swanc I  $\cdot$ sich siz] sie sich das M \textbf{2} Fuͯr das bette vf einen tepich spranch L (M) (Z)  $\cdot$ spranc] si spranc I \textbf{3} Condwiramurs] koͮndwiramvrs G Conduwiramurs I Condwir Amvrs L Kundwir Amuͯrs M Kvndwiramvrs Z  $\cdot$ lieht] licht M \textbf{4} Parzival] parcifal G Z Parzifal I (L) (M) \textbf{5} sagte] sagt I L  $\cdot$ sich] mich M \textbf{6} gelücke] gilucket M \textbf{7} gesendet her] Her gesant L  $\cdot$ her zen vröuden] her zcu den vrowen M herre frevde Z \textbf{8} si bat in] Dv solt mir L \textbf{10} sælic sî] Gert sý die L (M) (Z)  $\cdot$ tac] der tach L (M) (Z) \textbf{12} kranc] so kranch L \textbf{13} des] wes Z \textbf{14} mînhalp] an mir vil L (M) Z  $\cdot$ ungewert] vnwert L \textbf{16} Kardeiz] Kardeisz M  $\cdot$ Loherangrin] Leherangrin I joherangrin L johangrin M Lohagrin Z \textbf{17} ûf dem bette] \textit{om.} M  $\cdot$ alblôz] bloz L also blosz M \textbf{18} Parzivalen] parcivalen G Parzifal I Parcifal L Parzifaln M Parcifaln Z  $\cdot$ des] das M \textbf{19} erne] Er L M \textbf{20} Kiot] Kýot L Kyot M Z \textbf{23} ûzzem gezelte] vz den gezelten L vsz dē gezcelten M \textbf{24} si tâtenz] D:::atenz M  $\cdot$ dô si] doch si in I da sie en M da sie Z \textbf{26} Kiot] Kýot L Kyot Z \textbf{27} der] die L \textbf{28} al] \textit{om.} L  $\cdot$ juncvrouwen] iuncfro I  $\cdot$ vuorte] vurt I (L) (Z) \textbf{29} ez] her M \textbf{30} kamerære] die chamerere I  $\cdot$ zuo] die winden zv Z \newline
\end{minipage}
\hspace{0.5cm}
\begin{minipage}[t]{0.5\linewidth}
\small
\begin{center}*T
\end{center}
\begin{tabular}{rl}
 & umb sich si daz deckelachen swanc,\\ 
 & \textbf{vür daz bette ûf einen teppich} spranc\\ 
 & Kundewiramurs, diu lieht gemâl.\\ 
 & ouch umbevienc si Parcifal.\\ 
5 & man sagete mir, si kusten sich.\\ 
 & si sprach: "mir het gelücke dich\\ 
 & gesendet \textbf{her zuo vreuden} mîn."\\ 
 & si bat in willekomen sîn.\\ 
 & "nû solt ich zürnen. ich enmac.\\ 
10 & \textbf{geêrt} sî \textbf{diu} wîle und \textbf{der} tac,\\ 
 & der mir brâhte disen umbevanc,\\ 
 & dâ von mîn trûren wirdet kranc.\\ 
 & ich hân nû, des mîn herze gert.\\ 
 & sorge ist \textbf{noch an mir vil} ungewert."\\ 
15 & nû erwacheten ouch diu kindelîn,\\ 
 & Kardeiz und Lohrangrin\\ 
 & lâgen ûf dem bette blôz.\\ 
 & Parcifaln des niht verdrôz,\\ 
 & er kuste si minneclîche.\\ 
20 & Kyot, der \textbf{zühte} rîche,\\ 
 & bat die knaben dannen tragen.\\ 
 & \textbf{er} begunde ouch \textbf{al \textit{d}en} vrouwen sagen,\\ 
 & daz si ûz dem gezelte giengen.\\ 
 & si tâtenz, dô si enpfiengen\\ 
25 & ir hêrren von langer reise.\\ 
 & Kyot, der kurteise,\\ 
 & bevalh der küneginne ir man.\\ 
 & alle die juncvrouwen \textbf{vuorter} dan.\\ 
 & dannoch was ez harte vruo.\\ 
30 & kamerære sluogen \textbf{die winden} zuo.\\ 
\end{tabular}
\scriptsize
\line(1,0){75} \newline
U Q R \newline
\line(1,0){75} \newline
\textbf{29} \textit{Initiale} R  \newline
\line(1,0){75} \newline
\textbf{2} einen] ein R \textbf{3} Kundewiramurs] Kuͦndewiramuͦrs U kondwiramurs Q Kondwiramuͦrs R  $\cdot$ lieht] licht Q liech R \textbf{4} Parcifal] Parzifal U partzifal Q parczifal R \textbf{5} sagete] sagt R \textbf{7} her zuo vreuden] her zún frewden Q hercz froͯde R \textbf{10} diu wîle] der wille R \textbf{11} brâhte] brach R \textbf{12} mîn] mir Q  $\cdot$ wirdet] wurde R \textbf{13} nû] \textit{om.} R \textbf{14} noch] \textit{om.} Q R \textbf{16} Kardeiz] Kardeis U (Q) R  $\cdot$ Lohrangrin] Lorangrin U Lochrangrin R \textbf{17} lâgen] Die lagen Q Do lagen R \textbf{18} Parcifaln] Parzifaln U Partzifalen Q Parczifaln R \textbf{19} minneclîche] munnicklichen Q \textbf{20} Kyot] Koyt Q  $\cdot$ zühte] zuchten Q zuchen R \textbf{21} bat] Hie R \textbf{22} al den] alten U \textbf{23} si] \textit{om.} Q \textbf{24} tâtenz] tattent R \textbf{25} von] \textit{om.} R \textbf{26} Kyot] Koyt U Q \textbf{27} küneginne] kunginnen R \textbf{29} ez] er R \textbf{30} sluogen] schluͦg R  $\cdot$ winden] wúrden Q winde R \newline
\end{minipage}
\end{table}
\end{document}
