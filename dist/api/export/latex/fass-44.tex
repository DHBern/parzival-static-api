\documentclass[8pt,a4paper,notitlepage]{article}
\usepackage{fullpage}
\usepackage{ulem}
\usepackage{xltxtra}
\usepackage{datetime}
\renewcommand{\dateseparator}{.}
\dmyyyydate
\usepackage{fancyhdr}
\usepackage{ifthen}
\pagestyle{fancy}
\fancyhf{}
\renewcommand{\headrulewidth}{0pt}
\fancyfoot[L]{\ifthenelse{\value{page}=1}{\today, \currenttime{} Uhr}{}}
\begin{document}
\begin{table}[ht]
\begin{minipage}[t]{0.5\linewidth}
\small
\begin{center}*D
\end{center}
\begin{tabular}{rl}
\textbf{44} & \begin{large}E\end{large}r vuorten \textbf{în}. daz was im leit.\\ 
 & diu küneginne \textbf{im widerreit}.\\ 
 & \textbf{sînen zoum nam si} \textbf{mit} \textbf{ir} hant;\\ 
 & si entstricte der vintâlen bant.\\ 
5 & der wirt in muose lâzen.\\ 
 & \textbf{sîne} knappen niht vergâzen,\\ 
 & si\textbf{ne} kêrten vaste ir hêrren nâch.\\ 
 & durch die stat man vüeren sach\\ 
 & ir gast die küneginne wîs,\\ 
10 & der dâ \textbf{behalden} het den prîs.\\ 
 & si erbeizte, al dâ s\textit{i}s dûhte zît.\\ 
 & "wê, wie getriwe ir knappen sît!\\ 
 & ir wænet verliesen disen man.\\ 
 & \textbf{dem} \textbf{wirt} \textbf{ân iuch gemach} getân.\\ 
15 & nemt sîn ors unt vüeret \textbf{daz} hin.\\ 
 & sîn geselle ich hie bin."\\ 
 & vil vrouwen er dort ûfe vant.\\ 
 & entwâpent mit swarzer hant\\ 
 & wart \textbf{er von} der künegîn.\\ 
20 & ein declachen zobelîn\\ 
 & unt ein bette wol gehêret,\\ 
 & \textbf{gar} an \textbf{im wart} gemêret\\ 
 & ein heinlîchiu êre.\\ 
 & \textbf{al} dâ was \textbf{niemen} mêre.\\ 
25 & \textbf{die} juncvrouwen giengen vür\\ 
 & unt sluzzen \textbf{nâch in zuo} die tür.\\ 
 & dô pflac diu küneginne\\ 
 & einer \textbf{werden}, \textbf{süezer} minne\\ 
 & unt Gahmuret, ir herzen trût.\\ 
30 & ungelîch was \textbf{doch} ir \textbf{zweier} hût.\\ 
\end{tabular}
\scriptsize
\line(1,0){75} \newline
D Fr14 \newline
\line(1,0){75} \newline
\textbf{1} \textit{Initiale} D  \textbf{3} \textit{Initiale} Fr14  \newline
\line(1,0){75} \newline
\textbf{11} sis] sihs D sis Fr14 \textbf{20} declachen] deche:::in Fr14 \textbf{26} in] hin Fr14 \textbf{29} Gahmuret] Gahmvret D Fr14 \newline
\end{minipage}
\hspace{0.5cm}
\begin{minipage}[t]{0.5\linewidth}
\small
\begin{center}*m
\end{center}
\begin{tabular}{rl}
 & er vuorte in \textbf{dan}. daz was im leit.\\ 
 & diu künigîn \textbf{im widerreit}.\\ 
 & \textbf{sînen zoum nam si} \textbf{mit} \textbf{der} hant;\\ 
 & si entstricket \textbf{ime} der vintailen bant.\\ 
5 & der wirt in muoste lâzen.\\ 
 & \textbf{sîne} knappen niht vergâzen,\\ 
 & si kêrten vaste i\textit{r} hêrren nâch.\\ 
 & durch die stat man vüeren sach\\ 
 & ir gast die künigîn wîs,\\ 
10 & der dô \textbf{behalten} hette den prîs.\\ 
 & si erbeizte, aldâ sis dûhte zît.\\ 
 & "wê, wie getriuwe ir knappen sît!\\ 
 & ir wænet verliesen disen man.\\ 
 & \textbf{dem} \textbf{wirt} \textbf{gemach âne iuch} getân.\\ 
15 & neme\textit{t} sîn ros und vüeret \textbf{ez} hin.\\ 
 & sîn geselle ich hie bin."\\ 
 & vil vrouwen er dort \textbf{d}ûfe vant.\\ 
 & \textbf{die} entwâp\textit{e}nt \textbf{in} mit swarzer hant.\\ 
 & \textbf{dô} wart \textbf{ime und} der künigîn\\ 
20 & ein deckelachen zobelîn\\ 
 & und ein bette wol gehêret,\\ 
 & \textbf{gar} an \textbf{ime wart} gemêret\\ 
 & ein heimlichiu êre.\\ 
 & \textbf{al}dâ was \textbf{niemen} mêre.\\ 
25 & \textbf{die} juncvrouwen giengen vür\\ 
 & und sluzzen \textbf{nâch in} die tür.\\ 
 & \begin{large}D\end{large}ô pflac diu küniginne\\ 
 & einer \textbf{werden}, \textbf{süezen} minne\\ 
 & und Gahmuret, ir herzetrût.\\ 
30 & u\textit{n}gelîch was \textbf{doch} ir \textbf{beider} hût.\\ 
\end{tabular}
\scriptsize
\line(1,0){75} \newline
m n o \newline
\line(1,0){75} \newline
\textbf{27} \textit{Initiale} m   $\cdot$ \textit{Capitulumzeichen} n  \newline
\line(1,0){75} \newline
\textbf{1} dan daz] [das]: dan das m das n das dan o  $\cdot$ im] \textit{om.} o \textbf{4} entstricket] entricket o \textbf{5} muoste] muͯste m n miste o \textbf{6} niht] sin nit n  $\cdot$ vergâzen] vergessen n \textbf{7} ir] ire m \textbf{11} sis] sú n (o) \textbf{15} nemet] Nemen m \textbf{16} sîn] Din o \textbf{17} dort dûfe] do vff n dort o \textbf{18} entwâpent] entwappettend m  $\cdot$ in] im o \textbf{20} zobelîn] er zobelin o \textbf{25} juncvrouwen giengen] koͯnnigin ging n jungfrowe ging o \textbf{26} sluzzen] slússe n slosse o  $\cdot$ in] ir n o  $\cdot$ tür] tor n o \textbf{29} Gahmuret] gamúret n gamuret o  $\cdot$ ir] ir \textit{nachträglich korrigiert zu:} irsz m ires n (o)  $\cdot$ herzetrût] hercze truͯt \textit{nachträglich korrigiert zu:} herczen truͯt m hertzen drut n (o) \textbf{30} ungelîch] Vnd gelich m  $\cdot$ beider] zweẏer n o \newline
\end{minipage}
\end{table}
\newpage
\begin{table}[ht]
\begin{minipage}[t]{0.5\linewidth}
\small
\begin{center}*G
\end{center}
\begin{tabular}{rl}
 & er vuort in \textbf{în}. daz was im leit.\\ 
 & diu künigîn \textbf{im widerreit}\\ 
 & \textbf{unde nam in selbe} \textbf{mit} \textbf{ir} hant;\\ 
 & si entstrickt \textbf{im} der vinteilen bant.\\ 
5 & der wirt in muose lâzen.\\ 
 & \textbf{die} knappen niht vergâzen,\\ 
 & si kêrten vaste ir hêrren nâch.\\ 
 & durch die stat man vüeren sach\\ 
 & ir gast die küniginne wîs,\\ 
10 & der dâ \textbf{behalten} het den brîs.\\ 
 & si erbeizt, al dâ sis dûhte zît.\\ 
 & "wê, wie getriuw ir knappen sît!\\ 
 & ir wænet verliesen disen man.\\ 
 & \textbf{im} \textbf{wirt} \textbf{ân iuch gemach} getân.\\ 
15 & \begin{large}N\end{large}emet sîn ors unde vüeret \textbf{ez} hin.\\ 
 & sîn geselle ich hie bin."\\ 
 & vil vrouwen er dort ûffe vant.\\ 
 & entwâpent mit swarzer hant\\ 
 & wart \textbf{er von} der künigîn.\\ 
20 & ein declachen zobelîn\\ 
 & unde ein bette wol gehêret,\\ 
 & \textbf{dâr} an \textbf{wart im} gemêret\\ 
 & ein heinlîchiu êre.\\ 
 & dâ was \textbf{ouch wünne} mêre.\\ 
25 & juncvrouwen giengen \textbf{von in} vür\\ 
 & unde sluzzen \textbf{nâch in zuo} die tür.\\ 
 & dô pflac diu küniginne\\ 
 & einer \textbf{stolzen}, \textbf{werden} minne\\ 
 & unde Gahmuret, ir herzen trût.\\ 
30 & ungelîch was \textbf{doch} ir \textbf{beider} hût.\\ 
\end{tabular}
\scriptsize
\line(1,0){75} \newline
G O L M Q R Z Fr21 \newline
\line(1,0){75} \newline
\textbf{1} \textit{Initiale} M  \textbf{2} \textit{Initiale} O  \textbf{11} \textit{Initiale} L Q R Fr21  \textbf{15} \textit{Initiale} G  \newline
\line(1,0){75} \newline
\textbf{1} in în] on M \textbf{2} diu] ÷iv O  $\cdot$ widerreit] onkegin reit M (R) \textbf{3} unde] Sinen zovm O L (M) (Q) (R) (Z) (Fr21)  $\cdot$ nam in selbe] si O nam sie L (M) Q (R) Z (Fr21)  $\cdot$ mit ir] mit der Q in Ir R \textbf{4} entstrickt] enstrihte O (L) (M) (R) (Fr21) enstrickten Q enstritte Z  $\cdot$ im der] der O Q R Z Fr21 \textit{om.} M  $\cdot$ vinteilen] wemteylen Q \textbf{5} der] Sin Z  $\cdot$ in muose] múst in Q \textbf{6} die] Sine O L M (Q) R Z (Fr21)  $\cdot$ niht] in nicht Q \textbf{7} \textit{Die Verse 44.7-51.12 fehlen} Z   $\cdot$ kêrten] enkerten L (M) (Fr21)  $\cdot$ vaste] \textit{om.} L  $\cdot$ ir] iren L \textbf{8} man] man on M \textbf{9} ir] Erin M  $\cdot$ küniginne] kúnginen R \textbf{10} dâ] do Q  $\cdot$ behalten] behalte Fr21  $\cdot$ het] hat L R \textbf{11} si] Die M  $\cdot$ erbeizt] erbæizte O (L) (M) (R) erbeisten Q  $\cdot$ al dâ] da L do Q  $\cdot$ sis] sie L M Q \textbf{12} Wie gerne ir [kapin]: knapin set M  $\cdot$ wê] Awi O Owi Fr21  $\cdot$ getriuw ir] trúwe sine R \textbf{16} geselle] gesellin M  $\cdot$ hie] die R \textbf{17} vil] Wil Q  $\cdot$ dort ûffe] daruffe R \textbf{20} zobelîn] [sidin]: zoͯbelin R \textbf{22} dâr an] Darynne M  $\cdot$ wart im] im wart O Q (R) Fr21 wart M \textbf{23} ein] Er Q \textbf{24} dâ] Al da O (Q) (R) (Fr21)  $\cdot$ ouch wünne] niemen O (L) (M) (Q) (R) (Fr21) \textbf{25} von in vür] fvͦr O (L) (Q) (R) (Fr21) hen for M \textbf{26} in zuo] in O R Fr21 im Q \textbf{27} dô] Da M R Fr21 \textbf{28} einer] \textit{om.} M  $\cdot$ stolzen werden] werden svͦzen O (M) (Q) (Fr21) Werder suszer L suͯssen werden R \textbf{29} Gahmuret] Gahmvret G Gamvret O Gahmuͯret L gamuraten M gamúert Q Gahmoret Fr21  $\cdot$ ir] ores M \textbf{30} beider] zweier O L (M) (Q) (R) (Fr21) \newline
\end{minipage}
\hspace{0.5cm}
\begin{minipage}[t]{0.5\linewidth}
\small
\begin{center}*T (U)
\end{center}
\begin{tabular}{rl}
 & er vuorte in \textbf{hin}. daz was im leit.\\ 
 & diu küneginne \textbf{engein im reit}.\\ 
 & \textbf{sînen zoum den vienc si} \textbf{bî} \textbf{der} hant;\\ 
 & si enstricte der vintelien bant.\\ 
5 & der wirt in muose lâzen.\\ 
 & \textbf{sîne} knappen niht vergâzen,\\ 
 & si\textbf{ne} kârten vaste i\textit{r} hêrren nâch.\\ 
 & \textbf{al} durch die stat man vüeren sach\\ 
 & ir gast die küneginne wîs,\\ 
10 & der dô \textbf{behabet} hete den prîs.\\ 
 & si erbeizte, al dâ si es dûhte zît.\\ 
 & "w\textit{ê}, wie getriuwe ir knappen sît!\\ 
 & ir wænet verliesen disen man.\\ 
 & \textbf{im} \textbf{wære} \textbf{âne iuwer gemach} getân.\\ 
15 & nemet sîn ors und vüeret \textbf{ez} hin.\\ 
 & sîn geselle ich hie bin."\\ 
 & vil vrouwen er dort ûffe vant.\\ 
 & entwâpen\textit{t} mit swarzer hant\\ 
 & wart \textbf{er von} der künegîn.\\ 
20 & ein deckelachen zobelîn\\ 
 & und ein bette wol geh\textit{ê}ret,\\ 
 & \textbf{dâr} an \textbf{im wart} gemêret\\ 
 & eine heimlîchiu êre.\\ 
 & \textbf{al} dâ was \textbf{nieman} mêre.\\ 
25 & juncvrouwen, \textbf{die} giengen \textbf{dar} vür\\ 
 & und sluzzen \textbf{zuo nâc\textit{h} in} d\textit{ie} tür.\\ 
 & dô pflac diu küneginne\\ 
 & einer \textbf{werden}, \textbf{süezen} minne\\ 
 & und Gahmuret, ir herzetrût.\\ 
30 & ungelîch was ir \textbf{beider} hût.\\ 
\end{tabular}
\scriptsize
\line(1,0){75} \newline
U V W T \newline
\line(1,0){75} \newline
\textbf{5} \textit{Majuskel} T  \textbf{11} \textit{Initiale} T  \textbf{17} \textit{Majuskel} T  \textbf{25} \textit{Majuskel} T  \textbf{27} \textit{Majuskel} T  \newline
\line(1,0){75} \newline
\textbf{1} vuorte] [*]: zoch V  $\cdot$ hin] dan T \textbf{2} engein im] im wider W (T) \textbf{3} sînen] sinem T  $\cdot$ den vienc] ving W nam T  $\cdot$ bî] mit V T \textbf{4} vinteilen] vinculen W \textbf{5} muose] muͦz U muͤste V (T) \textbf{6} sîne] die T \textbf{7} sine] [Si]: Sú V Sy W  $\cdot$ vaste ir hêrren] vaste irn hern U irm herren vaste W \textbf{8} al] \textit{om.} T  $\cdot$ stat man] man in W \textbf{10} dô] da V T  $\cdot$ behabet] behalten W T  $\cdot$ hete] hat W \textbf{11} erbeizte] [*]: erbeissetent V  $\cdot$ si es] es [*]: sv́ V siz T \textbf{12} wê] Wie U \textbf{14} wære] wirt V W T  $\cdot$ âne iuwer] ane v́ch V (W) (T) \textbf{17} ûffe] [v̂ze]: v̂fe T \textbf{18} entwâpent] Entwapente U  $\cdot$ swarzer] [s*]: svͤzzer V \textbf{21} gehêret] gehoret U \textbf{25} juncvrouwen die] [*]: Die iungfrowen V Iunckfrauwen W (T)  $\cdot$ dar] \textit{om.} W T \textbf{26} zuo nâch in] zuͦ nacht in U nach in zuͦ W (T)  $\cdot$ die] den U \textbf{27} dô] Da T \textbf{28} süezen] [suͤssen]: suͤsser V núwer W \textbf{29} Gahmuret] Gahmuͦret U Gamuret V (W)  $\cdot$ ir] irs V  $\cdot$ herzetrût] hertzen trut V (T) \textbf{30} was] was doch W T  $\cdot$ beider] zweir T \newline
\end{minipage}
\end{table}
\end{document}
