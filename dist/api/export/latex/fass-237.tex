\documentclass[8pt,a4paper,notitlepage]{article}
\usepackage{fullpage}
\usepackage{ulem}
\usepackage{xltxtra}
\usepackage{datetime}
\renewcommand{\dateseparator}{.}
\dmyyyydate
\usepackage{fancyhdr}
\usepackage{ifthen}
\pagestyle{fancy}
\fancyhf{}
\renewcommand{\headrulewidth}{0pt}
\fancyfoot[L]{\ifthenelse{\value{page}=1}{\today, \currenttime{} Uhr}{}}
\begin{document}
\begin{table}[ht]
\begin{minipage}[t]{0.5\linewidth}
\small
\begin{center}*D
\end{center}
\begin{tabular}{rl}
\textbf{237} & \begin{large}D\end{large}er taveln \textbf{hundert muosten} sîn,\\ 
 & die man dô truoc zer tür \textbf{dar în}.\\ 
 & man sazte ieslîche schiere\\ 
 & vür \textbf{werder} ritter viere.\\ 
5 & tischlachen \textbf{var} nâch wîze\\ 
 & wurden drûf geleit mit vlîze.\\ 
 & der wir\textit{t} dô selbe wazzer nam,\\ 
 & der was an \textbf{hôchmuote} lam.\\ 
 & mit im twuoc sich Parzival.\\ 
10 & eine sîdîne tweheln wol gemâl,\\ 
 & \textbf{die} bôt eines grâven sun dar nâch,\\ 
 & dem was ze \textbf{knien} \textbf{vür si} gâch.\\ 
 & Swâ \textbf{dô} der taveln decheiniu stuont,\\ 
 & dâ tet man vier knappen kunt,\\ 
15 & daz si ir dienstes niht vergæzen\\ 
 & den, die \textbf{drob} sæzen.\\ 
 & zwêne knieten unt sniten.\\ 
 & die anderen zwêne niht vermiten,\\ 
 & si\textbf{ne} trüegen \textbf{trinken} unt \textbf{ezzen} dar,\\ 
20 & unt nâmen ir mit dienste war.\\ 
 & hœret mêr von rîcheite sagen:\\ 
 & vier karrâschen muosen tragen\\ 
 & manec tiwer goltvaz\\ 
 & ieslîchem ritter, der dâ saz.\\ 
25 & \textbf{man zôch si} ze\textbf{n} vier wenden.\\ 
 & vier ritter mit ir henden\\ 
 & mans ûf die taveln setzen sach.\\ 
 & ieslîchem gie ein schrîbære nâch,\\ 
 & der sich dar zuo arbeite\\ 
30 & unt \textbf{es} wider ûf bereite,\\ 
\end{tabular}
\scriptsize
\line(1,0){75} \newline
D Fr3 \newline
\line(1,0){75} \newline
\textbf{1} \textit{Initiale} D  \textbf{13} \textit{Majuskel} D  \newline
\line(1,0){75} \newline
\textbf{5} var] gevar Fr3 \textbf{7} wirt] wir D \textbf{9} Parzival] par Fr3 \textbf{10} tweheln] twehel Fr3 \textbf{30} es] si Fr3 \newline
\end{minipage}
\hspace{0.5cm}
\begin{minipage}[t]{0.5\linewidth}
\small
\begin{center}*m
\end{center}
\begin{tabular}{rl}
 & der tavelen \textbf{muosen hundert} sîn,\\ 
 & die man dô truoc zuo der tür \textbf{d\textit{â}} \textbf{her în}.\\ 
 & man saste ieglîche schiere\\ 
 & vür \textbf{werde\textit{r}} ritter viere.\\ 
5 & tischlach \textbf{var} nâch wîze\\ 
 & wurden drûf geleit mit vlîze.\\ 
 & der wirt dô selbe wazzer nam,\\ 
 & der was an \textbf{hôhem muote} lam.\\ 
 & mit \textit{ime} twuoc sich Parcifal.\\ 
10 & eine sîdîne twehel wol gemâl,\\ 
 & \textbf{die} bôt eines grâven sun dar nâch,\\ 
 & dem was ze \textbf{kniene} gâch.\\ 
 & wâ \textbf{d\textit{ô}} der tavelen keiniu stuont,\\ 
 & d\textit{â} tet man vier knaben kunt,\\ 
15 & daz si ir dienstes niht vergæzen\\ 
 & den, die \textbf{dâr abe} sæzen.\\ 
 & zwêne knieten und sniten.\\ 
 & die andern zwêne niht vermiten,\\ 
 & si trüegen \textbf{trinken} und \textbf{ezzen} dar,\\ 
20 & und nâmen ir mit dienste war.\\ 
 & \begin{large}H\end{large}œret mê von rîcheit sagen:\\ 
 & vier karrâschen muosen tragen\\ 
 & manic tiure goltvaz\\ 
 & ieglîchem ritter, der d\textit{â} saz.\\ 
25 & \textbf{man zôch si} ze\textbf{n} vier wenden.\\ 
 & vier ritter mit ir henden\\ 
 & mans ûf die tavelen setzen sach.\\ 
 & ieglîchem gienc ein schrîber nâch,\\ 
 & der sich dar zuo arbeite\\ 
30 & und \textbf{si} wider ûf bereite,\\ 
\end{tabular}
\scriptsize
\line(1,0){75} \newline
m n o Fr69 \newline
\line(1,0){75} \newline
\textbf{21} \textit{Initiale} m n o Fr69  \newline
\line(1,0){75} \newline
\textbf{1} muosen] muͯstent n (o) \textbf{2} die] Do n  $\cdot$ tür] túren n (o)  $\cdot$ dâ her în] do herin m in n o \textbf{3} ieglîche] jegelichen n \textbf{4} werder] werden m \textbf{8} muote] milte n \textbf{9} ime] \textit{om.} m  $\cdot$ twuoc] truͦg o \textbf{10} eine sîdîne twehel] Ein siden twehelin Fr69 \textbf{11} dar] do n \textbf{13} dô] da m o \textbf{14} dâ] Do m n o \textbf{16} abe] obe n (o) \textbf{17} sniten] snẏden o \textbf{18} vermiten] vermiden o \textbf{19} si] Die o \textbf{21} mê] jne n \textbf{22} muosen] muͯsten n muze::: Fr69 \textbf{24} dâ] do m n o \textbf{25} zen vier] zim fierden o \textbf{26} ir] den n o \textbf{28} gienc] [truͦg]: gienc Fr69 \textbf{30} bereite] bereiten o \newline
\end{minipage}
\end{table}
\newpage
\begin{table}[ht]
\begin{minipage}[t]{0.5\linewidth}
\small
\begin{center}*G
\end{center}
\begin{tabular}{rl}
 & der tavelen \textbf{muosen hundert} sîn,\\ 
 & die man dâ truoc zer tür \textbf{her în}.\\ 
 & man sazte ieslîche schiere\\ 
 & vür \textbf{werder} rîter viere.\\ 
5 & tischlachen \textbf{var} nâch wîze\\ 
 & wurden drûf geleit mit vlîze.\\ 
 & der wirt dô selbe wazzer nam,\\ 
 & der was an \textbf{hôhem muote} lam.\\ 
 & mit im twuoc sich Parzival.\\ 
10 & ein sîdîn twehel wol gemâl\\ 
 & bôt eines grâven sun dar nâch,\\ 
 & dem was ze \textbf{komene} \textbf{vür si} gâch.\\ 
 & swâ der tavelen deheiniu stuont,\\ 
 & dâ tet man vier knappen kunt,\\ 
15 & daz si ir dienstes niht verg\textit{æ}zen\\ 
 & den, die \textbf{drobe} s\textit{æ}zen.\\ 
 & zwêne knieten unde sniten.\\ 
 & die anderen zwêne niht vermiten,\\ 
 & si\textbf{ne} trüegen \textbf{spîse} unde \textbf{trinken} dar,\\ 
20 & unde nâmen ir mit dienste war.\\ 
 & hœrt mê von rîcheit sagen:\\ 
 & vier k\textit{ar}râsche\textit{n} muosen tragen\\ 
 & manic tiure goltvaz\\ 
 & ieslîch\textit{em} rîter, der dâ saz.\\ 
25 & \textbf{si zugen} ze vier wenden\\ 
 & vier rîter mit ir henden.\\ 
 & mans ûf die tavelen setzen sach.\\ 
 & \begin{large}I\end{large}eslîchem gieng ein schrîbære nâch,\\ 
 & der sich dar zuo \textit{arb}e\textit{i}te\\ 
30 & unde \textbf{si} wider ûf bereite,\\ 
\end{tabular}
\scriptsize
\line(1,0){75} \newline
G I O L M Q R Z Fr40 Fr51 \newline
\line(1,0){75} \newline
\textbf{1} \textit{Initiale} L M Q Z  \textbf{9} \textit{Initiale} R  \textbf{13} \textit{Initiale} I  \textbf{21} \textit{Initiale} Fr51   $\cdot$ \textit{Capitulumzeichen} L  \textbf{25} \textit{Initiale} O  \textbf{28} \textit{Initiale} G  \newline
\line(1,0){75} \newline
\textbf{1} muosen] muͤsten I (L) (Q) \textbf{2} dâ] \textit{om.} L M do Q  $\cdot$ truoc] \textit{om.} I  $\cdot$ zer tür] zvͤ den turn I  $\cdot$ her în] in I R Fr51 dar in L Z er yn M \textbf{3} sazte] sazta I satze Fr51  $\cdot$ ieslîche] yeglichem R jo eyne Fr51 \textbf{4} werder] werde Fr51  $\cdot$ rîter viere] [viere rittere]: rittere viere O \textbf{5} tischlachen] tislachten Q  $\cdot$ var nâch] geuar nach I waren nach R \textit{om.} Fr51 \textbf{7} dô selbe wazzer] daz wazzer selbe do O da selben das wasszir M das wasser selber Q selb das messer R da selbe daz wazzer Z do selber watzer Fr51 \textbf{8} der] Das Q Er R (Fr51)  $\cdot$ hôhem muote] homode Fr51 \textbf{9} Parzival] parzifal I M Barcifal O parcifal L Z partzifal Q parczifal R parzẏual Fr51 \textbf{10} sîdîn] sin R  $\cdot$ twehel] tweln M (Q) (R) (Z) \textbf{11} eines] Ims R \textbf{12} ze komene] zechm Q keiner R zv knien Z ze::: Fr40 zo denste Fr51  $\cdot$ vür si] vur sich I fur dy Q fᵫr in R harde Fr51  $\cdot$ gâch] ga Fr51 \textbf{13} swâ] Swa da O (Z) Wo so L Wa da M Wa Q Wa do R :::wa Fr40 War zo Fr51  $\cdot$ tavelen] tavel M (R)  $\cdot$ deheiniu] einev O deheine R \textbf{14} dâ] Do Q \textbf{15} si] \textit{om.} Q  $\cdot$ dienstes] dienst I  $\cdot$ niht] iht O L  $\cdot$ vergæzen] vergazen G (L) (Z) (Fr51) \textbf{16} den] Wan R  $\cdot$ drobe] dar abe M dar obene Fr51  $\cdot$ sæzen] sazen G (L) (Z) (Fr51) ezen I \textbf{17} knieten] chnieten nider O \textbf{18} die anderen zwêne] Zwe ander Fr51  $\cdot$ niht] das nicht M (Z) (Fr51) \textbf{19} sine] Si O (L) (Q) R Zcwene M :::i Fr40  $\cdot$ spîse unde trinken] trinchen vnde ezzen O (L) (M) (Q) (R) (Z) (Fr40) (Fr51) \textbf{20} nâmen] nam Q næmen Fr40 \textbf{21} mê] mær O \textbf{22} muosen] muͤsen I (L) (Q)  $\cdot$ karrâschen] craschenære G \textbf{23} goltvaz] goltvar M \textbf{24} ieslîchem] ieslich G (I) Jegelicher L Jclichin M (Q) Juwelichen Fr51  $\cdot$ der] dy M  $\cdot$ dâ] do Q  $\cdot$ saz] sar M az Fr51 \textbf{25} si zugen] ÷i zvgen O Die zcogin M Man zoch sie Q (R) Z (Fr40)  $\cdot$ ze] zuͯ den L (Z) (Fr51) \textbf{26} ir] den L \textbf{27} [*]: Vf die taflen man sie setten sach Fr51  $\cdot$ mans] Man M  $\cdot$ tavelen] tavel O \textbf{28} Ieslîchem] Jewelich Fr51  $\cdot$ schrîbære] schivbær O \textbf{29} Die daz arbeyte Fr51  $\cdot$ arbeite] zeigte G [bereite]: arbeite O \textbf{30} ûf] \textit{om.} L \newline
\end{minipage}
\hspace{0.5cm}
\begin{minipage}[t]{0.5\linewidth}
\small
\begin{center}*T
\end{center}
\begin{tabular}{rl}
 & \begin{large}D\end{large}er taveln \textbf{muosen hundert} sîn,\\ 
 & die man dô truoc zer tür \textbf{dar în}.\\ 
 & man sazte ieglîche schiere\\ 
 & vür \textbf{werde} rîter viere.\\ 
5 & tischlachen \textbf{gevar} nâch wîze\\ 
 & wurden drûf geleit mit vlîze.\\ 
 & Der wirt dô selbe wazzer nam,\\ 
 & der was an \textbf{hôhem muote} lam.\\ 
 & mit im twuoc sich Parcifal.\\ 
10 & eine sîdîne twehele wol gemâl,\\ 
 & \textbf{die} bôt eines græven sun dar nâch,\\ 
 & dem was ze \textbf{kniwene} \textbf{vür si} gâch.\\ 
 & Swâ \textbf{dô} der taveln deheiniu stuont,\\ 
 & dâ tet man vier knappen kunt,\\ 
15 & daz si\textit{r} dienstes niht vergæzen\\ 
 & den, die \textbf{drobe} sæzen.\\ 
 & Zwêne knieten unde sniten.\\ 
 & die andern zwêne niht vermiten,\\ 
 & si\textbf{ne} trüegen \textbf{trinken} unde \textbf{ezzen} dar,\\ 
20 & unde nâmen ir mit dienste war.\\ 
 & \textbf{\begin{large}N\end{large}û} hœret mê von rîcheit sagen:\\ 
 & vier karratschen muosen tragen\\ 
 & manec tiure goltvaz\\ 
 & i\textit{e}clîchem rîter, der dâ saz.\\ 
25 & \textbf{man zôch si} ze\textbf{n} vier wenden.\\ 
 & vier rîter mit ir henden\\ 
 & man sûf die taveln setzen sach.\\ 
 & ieglîchem gienc ein schrîbære nâch,\\ 
 & der sich dar zuo arbeite\\ 
30 & unde \textbf{si} wider ûf bereite,\\ 
\end{tabular}
\scriptsize
\line(1,0){75} \newline
T U V W \newline
\line(1,0){75} \newline
\textbf{1} \textit{Initiale} T U V W  \textbf{7} \textit{Majuskel} T  \textbf{13} \textit{Majuskel} T  \textbf{17} \textit{Majuskel} T  \textbf{21} \textit{Initiale} T U V  \newline
\line(1,0){75} \newline
\textbf{1} muosen] mvesen T \textbf{2} dô] \textit{om.} W  $\cdot$ dar în] her in U in V hin ein W \textbf{4} werde] werder W \textbf{5} tischlachen] Die tischlachen W \textbf{7} selbe] selber V W \textbf{9} Parcifal] parzifal T partzifal W \textbf{11} die] Do W \textbf{12} kniwene] kinwene U \textbf{13} Swâ] Wa U (W)  $\cdot$ dô] \textit{om.} V \textbf{14} dâ] Do U W \textbf{15} sir] sirz T \textbf{17} sniten] [sntten]: sniten U \textbf{19} sine trüegen] Sie truͦgen U (V) (W) \textbf{20} ir mit] mit irm W \textbf{21} Nû] \textit{om.} U V W \textbf{22} karratschen] karren W  $\cdot$ muosen] mvesen T muͤssen W \textbf{24} ieclîchem] iesclichem T  $\cdot$ dâ] do U V W \textbf{25} vier] viere U \textbf{27} die] den U  $\cdot$ setzen] ezzen U \textbf{28} Ein schriber ieglichem gie nach W \textbf{29} arbeite] \sout{bereite} arbeite T \textbf{30} si] \textit{om.} W \newline
\end{minipage}
\end{table}
\end{document}
