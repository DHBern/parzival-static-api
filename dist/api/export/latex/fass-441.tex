\documentclass[8pt,a4paper,notitlepage]{article}
\usepackage{fullpage}
\usepackage{ulem}
\usepackage{xltxtra}
\usepackage{datetime}
\renewcommand{\dateseparator}{.}
\dmyyyydate
\usepackage{fancyhdr}
\usepackage{ifthen}
\pagestyle{fancy}
\fancyhf{}
\renewcommand{\headrulewidth}{0pt}
\fancyfoot[L]{\ifthenelse{\value{page}=1}{\today, \currenttime{} Uhr}{}}
\begin{document}
\begin{table}[ht]
\begin{minipage}[t]{0.5\linewidth}
\small
\begin{center}*D
\end{center}
\begin{tabular}{rl}
\textbf{441} & \begin{large}H\end{large}abt ir geprüevet noch sînen art\\ 
 & oder wie ist bewendet iwer vart?"\\ 
 & er sprach zer meide wol geborn:\\ 
 & "dâ hân ich vreude vil verlorn.\\ 
5 & der Grâl mir \textbf{sorgen} gît genuoc.\\ 
 & ich liez ein lant, dâ ich krône truoc,\\ 
 & dar zuo daz minneclîchste wîp.\\ 
 & ûf erde nie sô schœner lîp\\ 
 & wart geborn von \textbf{menschlîcher} vruht.\\ 
10 & ich sen mich nâch ir kiuschen zuht.\\ 
 & nâch ir minne ich trûre vil\\ 
 & unt mêr nâch dem hôhen zil,\\ 
 & wie ich Munsalvæsche mege \textbf{gesehen}\\ 
 & unt den Grâl. daz ist noch ungeschehen.\\ 
15 & Niftel Sigune, dû tuost gewalt,\\ 
 & sît dû mînen kumber manecvalt\\ 
 & erkennest, daz dû vêhest mich."\\ 
 & diu magt sprach: "al mîn gerich\\ 
 & sol ûf dich, neve, sîn verkorn.\\ 
20 & dû hâst \textbf{doch} vreuden vil verlorn,\\ 
 & sît dû lieze dich betrâgen\\ 
 & umbe daz \textbf{werdeclîche} vrâgen\\ 
 & unt \textbf{dô} der süeze Anfortas\\ 
 & dîn wirt unt dîn gelücke was.\\ 
25 & dâ hete dir vrâgen wunsch bejagt.\\ 
 & nû muoz dîn vröude sîn verzagt\\ 
 & unt al dîn hôher muot erlemt.\\ 
 & dîn herze sorge hât gezemt,\\ 
 & diu dir vil \textbf{wilde} wære,\\ 
30 & \textbf{hetes gevrâget dû} der mære."\\ 
\end{tabular}
\scriptsize
\line(1,0){75} \newline
D Fr5 Fr31 \newline
\line(1,0){75} \newline
\textbf{1} \textit{Initiale} D  \textbf{15} \textit{Majuskel} D  \newline
\line(1,0){75} \newline
\textbf{1} sînen] sin Fr31 \textbf{2} bewendet] gewendet Fr31 \textbf{4} verlorn] verlor Fr31 \textbf{5} sorgen] kvmbers Fr31 \textbf{9} menschlîcher] menschen Fr31 \textbf{10} kiuschen] rainen Fr31 \textbf{12} dem] den Fr31 \textbf{13} wie ich Mvnsælvæsce mege gesehn D  $\cdot$ :::n gral mvg gesehen Fr31 \textbf{14} unt den Grâl] :::unshaluatsh Fr31 \textbf{15} Sigune] Sigvͦne D ::: Fr31  $\cdot$ gewalt] giwal Fr31 \textbf{30} gevrâget dû] du givragit Fr5 \newline
\end{minipage}
\hspace{0.5cm}
\begin{minipage}[t]{0.5\linewidth}
\small
\begin{center}*m
\end{center}
\begin{tabular}{rl}
 & habet ir gebrüefet noch sîn art\\ 
 & oder wie ist bewendet iuwer vart?"\\ 
 & \begin{large}E\end{large}r sprach zuo der megde wol geborn:\\ 
 & "d\textit{â} hân ich vröude vil verlorn.\\ 
5 & der Grâl mir \textbf{sorgen} gît genuoc.\\ 
 & ich liez ein lant, d\textit{â} ich krône truoc,\\ 
 & dar zuo daz minneclîcheste wîp.\\ 
 & ûf erde nie sô schœner lîp\\ 
 & wart geborn von \textbf{menschlîcher} vruht.\\ 
10 & ich sene mich nâch ir kiuschen zuht.\\ 
 & nâch ir minne ich trû\textit{re} vil\\ 
 & und mêr nâch dem hôhen zil,\\ 
 & wie ich Mun\textit{t}salvasche müge \textbf{sehen}\\ 
 & und den Grâl. daz ist noch ungeschehen.\\ 
15 & niftel Sigune, dû tuost gewalt,\\ 
 & sît dû mînen kumber manicvalt\\ 
 & erkennest, daz dû v\textit{ê}hest mich."\\ 
 & diu maget sprach: "al mîn gerich\\ 
 & sol ûf di\textit{ch}, neve, sîn verkorn.\\ 
20 & dû hâst \textbf{doch} vröuden vil verlorn,\\ 
 & sît dû lieze dich betrâgen\\ 
 & umb daz \textbf{werdeclîche} vrâgen\\ 
 & und \textbf{dô} der süeze Anfortas\\ 
 & dîn wirt und dîn gelücke was.\\ 
25 & dô hete dir vrâgen wunsch bejaget.\\ 
 & nû muoz dîn vröude sîn verzaget\\ 
 & und alle\textit{r} dîn hôher muot erlemet.\\ 
 & dîn herze \textit{s}o\textit{rg}e hât gezemet,\\ 
 & diu dir vil \textbf{vr\textit{ö}mede} wære,\\ 
30 & \textbf{hetest dû gevrâget} der mære."\\ 
\end{tabular}
\scriptsize
\line(1,0){75} \newline
m n o \newline
\line(1,0){75} \newline
\textbf{3} \textit{Initiale} m   $\cdot$ \textit{Capitulumzeichen} n  \newline
\line(1,0){75} \newline
\textbf{2} ist] er o  $\cdot$ bewendet] erwendet n o \textbf{3} wol geborn] hoch geborn n (o) \textbf{4} dâ] Do m n \textbf{5} Grâl] grole n \textbf{6} dâ] do m n o \textbf{8} erde] erden n o  $\cdot$ sô] kein o \textbf{9} menschlîcher] menschen n o \textbf{10} \textit{Vers 441.10 fehlt} o  \textbf{11} trûre] truwen m truge o \textbf{13} Muntsalvasche] munsaluasce m múntsaluasce n muntsaluasce o  $\cdot$ sehen] gesehen n (o) \textbf{14} Grâl] grole n \textbf{15} Sigune] sigun n o \textbf{16} dû mînen kumber] da myn komer o \textbf{17} vêhest] vohest m (n) fahest o \textbf{18} al mîn] alvmmb o \textbf{19} dich] die m din o  $\cdot$ verkorn] erkorn n verkern o \textbf{23} Anfortas] an fortas n \textbf{25} dô] So o \textbf{26} dîn] dir n o  $\cdot$ vröude] freiden o \textbf{27} aller] alle m o  $\cdot$ hôher] hoͯhen o \textbf{28} sorge] frouͯwe m \textbf{29} diu] Dir o  $\cdot$ vrömede] framede m \textbf{30} der] \textit{om.} n \newline
\end{minipage}
\end{table}
\newpage
\begin{table}[ht]
\begin{minipage}[t]{0.5\linewidth}
\small
\begin{center}*G
\end{center}
\begin{tabular}{rl}
 & \begin{large}H\end{large}abe\textit{t} ir gebrüevet noch sînen art\\ 
 & oder wie ist bewendet iuwer vart?"\\ 
 & er sprach ze der meide wol geborn:\\ 
 & "dâ hân ich vröude vil ve\textit{r}lorn.\\ 
5 & der Grâl mir \textbf{sorge} gî\textit{t} genuoc.\\ 
 & ich liez ein lant, dâ ich krôn truoc,\\ 
 & dâ zuo daz minneclîcheste wîp.\\ 
 & ûf erde nie sô schœner lîp\\ 
 & wart geborn von \textbf{menschlîcher} vruht.\\ 
10 & ich sene mich nâch ir kiuschen zuht.\\ 
 & nâch ir minne ich trûre vil\\ 
 & unde mêr nâch dem hôhem zil,\\ 
 & wie ich Muntsalvatsche müge \textbf{gesehen}\\ 
 & und de\textit{n} Grâl. daz ist noch ungeschehen.\\ 
15 & niftel Sigune, dû tuost gewalt,\\ 
 & sît dû mînen kumber manicvalt\\ 
 & erkennest, daz dû vêhes mich."\\ 
 & diu maget sprach: "al mîn geric\textit{h}\\ 
 & sol ûf dich, neve, sîn verkorn.\\ 
20 & dû hâst \textbf{doch} vröuden vil verlorn,\\ 
 & sît dû lieze dich betrâgen\\ 
 & umb daz \textbf{werdelîche} vrâgen\\ 
 & unt \textbf{dô} der süeze Anfortas\\ 
 & dîn wirt unt dîn gelücke was.\\ 
25 & dâ het dir vrâgen wunsch bejaget.\\ 
 & nû muoz dîn vröude sîn verzaget\\ 
 & unt al dîn hôher muot erlemet.\\ 
 & dîn herze sorgen hât gezemet,\\ 
 & diu dir vil \textbf{wilde} wære,\\ 
30 & \textbf{hetestû dô gevrâget} der mære."\\ 
\end{tabular}
\scriptsize
\line(1,0){75} \newline
G I O L M Z Fr25 \newline
\line(1,0){75} \newline
\textbf{1} \textit{Initiale} G O L Z Fr25  \textbf{15} \textit{Initiale} I  \newline
\line(1,0){75} \newline
\textbf{1} Habet] Habe G  $\cdot$ sînen] sin I O Z Fr25 sine M \textbf{2} bewendet] gewendet I (O) (Fr25) bewant L \textbf{4} hân ich] hain I  $\cdot$ verlorn] uirflorn G \textbf{5} Grâl] Glar Fr25  $\cdot$ sorge gît] sorge gite G sorgen lie I sorgen git O L (M) Fr25 git sorgen Z \textbf{6} dâ] do O \textbf{7} daz minneclîcheste] ein minechlichez I daz minnechliche O (Fr25) \textbf{8} erde] der erdin M  $\cdot$ sô] \textit{om.} I  $\cdot$ schœner] minnenclicher Z \textbf{9} wart] \textit{om.} I  $\cdot$ menschlîcher] menschen her I menschen O Fr25 \textbf{11} trûre] truͯbe L \textbf{12} hôhem] hohen O (M) Z Fr25 hohsten L \textbf{13} Muntsalvatsche] muntshalsche I Munsalvatsche M Montsalvatsche Z  $\cdot$ müge] sul I \textbf{14} den] dem G \textbf{15} Sigune] Sýgvne L \textbf{16} manicvalt] so mancualt I \textbf{18} al] \textit{om.} L  $\cdot$ gerich] geriht G \textbf{19} neve] \textit{om.} O M \textbf{20} dû] diu I  $\cdot$ doch] ouch Z \textbf{21} sît dû] Sie M \textbf{23} dô] da M Z  $\cdot$ Anfortas] Amfortas L \textbf{25} dâ] Do O  $\cdot$ dir] din I  $\cdot$ vrâgen wunsch] vrage vnz L \textbf{26} sîn] sie M \textbf{27} al] \textit{om.} M  $\cdot$ erlemet] ellende M \textbf{28} herze] herzen I  $\cdot$ sorgen] sorge I O L Z  $\cdot$ gezemet] gezend M \textbf{30} dô] \textit{om.} O L M Z \newline
\end{minipage}
\hspace{0.5cm}
\begin{minipage}[t]{0.5\linewidth}
\small
\begin{center}*T
\end{center}
\begin{tabular}{rl}
 & habt ir geprüevet noch sînen art\\ 
 & oder wie ist bewendet iuwer vart?"\\ 
 & Er sprach zer megde wol geborn:\\ 
 & "dâ hân ich vröuden vil verlorn.\\ 
5 & der Grâl mir \textbf{sorgen} gît genuoc.\\ 
 & ich liez ein lant, dâ ich krône truoc,\\ 
 & dar zuo daz minn\textit{ec}lîcheste wîp.\\ 
 & ûf erde nie sô schœner lîp\\ 
 & wart geborn von \textbf{menschen} vruht.\\ 
10 & ich sene mich nâch ir kiuschen zuht.\\ 
 & nâch ir minne ich trûre vil\\ 
 & unde mêr nâch dem hôhen zil,\\ 
 & wie ich Munsalvasche müge \textbf{gesehen}\\ 
 & unde den Grâl. daz ist noch ungeschehen.\\ 
15 & \begin{large}N\end{large}iftel Sygune, dû tuost gewalt,\\ 
 & sît dû mînen kumber manecvalt\\ 
 & erkennest, daz dû v\textit{ê}hest mich."\\ 
 & Diu maget sprach: "al mîn gerich\\ 
 & sol ûf dich, neve, sîn verkorn.\\ 
20 & dû hâst \textbf{ouch} vröuden vil verlorn,\\ 
 & sît dû lieze dich betrâgen\\ 
 & umbe daz \textbf{wærlîche} vrâgen\\ 
 & unde der süeze Anfortas\\ 
 & dîn wirt und\textit{e d}în gelücke was.\\ 
25 & dô hete dir vrâgen wunsch bejaget.\\ 
 & nû muoz dîn vröude sîn verzaget\\ 
 & unde aldîn hôher muot erlemt.\\ 
 & d\textit{în} herze sorge hât gezemt,\\ 
 & di\textit{u} dir vil \textbf{wilde} wære,\\ 
30 & \textbf{hetestû gevrâget} der mære."\\ 
\end{tabular}
\scriptsize
\line(1,0){75} \newline
T U V W Q R \newline
\line(1,0){75} \newline
\textbf{1} \textit{Initiale} V  \textbf{3} \textit{Initiale} W Q   $\cdot$ \textit{Majuskel} T  \textbf{15} \textit{Initiale} T U  \textbf{18} \textit{Majuskel} T  \newline
\line(1,0){75} \newline
\textbf{1} sînen] sin V (R) \textbf{2} bewendet] gewendet W  $\cdot$ iuwer] [s*]: ewr Q \textbf{3} zer] ze R \textbf{4} dâ] Do U V W Q  $\cdot$ vröuden] frewe Q \textbf{5} mir sorgen gît] sorgen git mir V mit sorgen gibt W mir gibt sorgen Q git mir sorgen R \textbf{6} ein] ein ein U  $\cdot$ dâ] do U V W Q  $\cdot$ krône] kronen W \textbf{7} minneclîcheste] minneliclicheste T \textbf{8} erde] erden W (Q) R  $\cdot$ sô] \textit{om.} V  $\cdot$ schœner lîp] scho͑n ein leip Q \textbf{10} mich] \textit{om.} R  $\cdot$ kiuschen] wiplich R \textbf{11} ich trûre] ich [trvr*]: trvre V trurre ich R \textbf{13} ich] \textit{om.} U  $\cdot$ Munsalvasche] mvnsalvasce T Muͦntsalvatsche U muntschavasche V montsaluatz W muntsalvasche Q Munsaluasche R \textbf{15} Sygune] Syguͦne U sigúne Q \textbf{17} vêhest] [wehest]: vehest Q \textbf{18} Diu] Du R \textbf{19} sol] Sold Q  $\cdot$ verkorn] erkorn Q \textbf{20} ouch] doch U W Q R  $\cdot$ vröuden] frewde Q \textbf{21} lieze dich] liesset dich W dich liest R \textbf{22} wærlîche] werdekliche V (W) (Q) (R) \textbf{23} der] do der V W Q  $\cdot$ Anfortas] antefortas R \textbf{24} unde dîn] vnde vnde din T \textbf{25} dô] Da V \textbf{27} aldîn] aller dein Q \textbf{28} dîn] der T Dem Q  $\cdot$ gezemt] verzemt Q \textbf{29} diu] die T Daz V  $\cdot$ wilde] [wide]: wilde R  $\cdot$ wære] weren Q \textbf{30} hetestû] Hestu R \newline
\end{minipage}
\end{table}
\end{document}
