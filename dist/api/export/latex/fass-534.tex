\documentclass[8pt,a4paper,notitlepage]{article}
\usepackage{fullpage}
\usepackage{ulem}
\usepackage{xltxtra}
\usepackage{datetime}
\renewcommand{\dateseparator}{.}
\dmyyyydate
\usepackage{fancyhdr}
\usepackage{ifthen}
\pagestyle{fancy}
\fancyhf{}
\renewcommand{\headrulewidth}{0pt}
\fancyfoot[L]{\ifthenelse{\value{page}=1}{\today, \currenttime{} Uhr}{}}
\begin{document}
\begin{table}[ht]
\begin{minipage}[t]{0.5\linewidth}
\small
\begin{center}*D
\end{center}
\begin{tabular}{rl}
\textbf{534} & \begin{large}S\end{large}wie gern ich in næme dan,\\ 
 & \textbf{doch} mac mîn hêr Gawan\\ 
 & der minne \textbf{des} niht entwenken,\\ 
 & si\textbf{ne} welle \textbf{in} vreude krenken.\\ 
5 & waz hilfet denne mîn underslac,\\ 
 & swaz ich dâ von gesprechen mac?\\ 
 & wert man sol sich niht minne wern,\\ 
 & wan \textbf{den} muoz minne \textbf{helfen} nern.\\ 
 & Gawan durch minne \textbf{arbeit} enpfienc;\\ 
10 & sîn vrouwe reit, ze vuoz er gienc.\\ 
 & Orgeluse unt der degen balt,\\ 
 & \textbf{die} kômen in einen grôzen walt.\\ 
 & dennoch muoser gêns wonen.\\ 
 & er \textbf{zôch} daz pfert zuo zeime ronen.\\ 
15 & sînen schilt, der ê drûfe lac,\\ 
 & des er durch \textbf{schildes} ambet pflac,\\ 
 & nam er ze halse; ûfz pfert er saz.\\ 
 & ez \textbf{truog} in kûme vürbaz\\ 
 & anderhalben ûz in erbouwen lant.\\ 
20 & eine burc er mit den ougen vant;\\ 
 & sîn herze unt \textbf{di\textit{u}} ougen jâhen,\\ 
 & daz si erkanten \textbf{noch} gesâhen\\ 
 & dekeine \textbf{burc} \textbf{nie} der gelîch.\\ 
 & si was alumbe \textbf{rîterlîch}.\\ 
25 & türne und palas\\ 
 & manegez ûf der bürge was.\\ 
 & dar zuo muoser schouwen\\ 
 & in den ve\textit{n}stern manege vrouwen;\\ 
 & der \textbf{was} vier hundert oder mêr,\\ 
30 & viere under in von arde hêr.\\ 
\end{tabular}
\scriptsize
\line(1,0){75} \newline
D Fr7 Fr31 \newline
\line(1,0){75} \newline
\textbf{1} \textit{Initiale} D Fr31  \newline
\line(1,0){75} \newline
\textbf{1} Swie ich in næme gerne dan Fr31 \textbf{2} doch mac] Sone mac doch Fr31 \textbf{4} in] im Fr7 Fr31 \textbf{5} hilfet] hilfet in Fr31  $\cdot$ underslac] widerslach Fr31 \textbf{7} Werdem man sol ich die minne nih wern Fr31  $\cdot$ minne] \textit{om.} Fr7 \textbf{8} muoz minne] minne mvͦz Fr31  $\cdot$ helfen] helfe Fr7 \textbf{9} arbeite] arbeit Fr7 (Fr31)  $\cdot$ enpfienc] enphiech Fr31 \textbf{10} vuoz] fuͤzzen Fr7 \textbf{11} Orgeluse] Orgelv̂se D Orluse Fr7 \textbf{13} muoser] muͤse er Fr7 \textbf{14} zeime] zainer Fr7 \textbf{18} ez] er Fr7 \textbf{19} in] in en Fr7 \textbf{20} er mit den ougen] mit den ovgen er Fr7 \textbf{21} diu] di D \textbf{24} rîterlîch] riterliche Fr31 \textbf{28} venstern] vestern D \newline
\end{minipage}
\hspace{0.5cm}
\begin{minipage}[t]{0.5\linewidth}
\small
\begin{center}*m
\end{center}
\begin{tabular}{rl}
 & wie gerne ich in næme dan,\\ 
 & \textbf{doch} mac mîn hêrre Gaw\textit{a}n\\ 
 & der minne \textbf{des} niht entwe\textit{n}ke\textit{n},\\ 
 & si welle \textbf{im} vröude krenken.\\ 
5 & waz hilfet den mîn underslac,\\ 
 & waz ich dâ von gesprechen mac?\\ 
 & wert man sol sich niht minne wern,\\ 
 & wan \textbf{den} muoz minne \textbf{helfen} nern.\\ 
 & \begin{large}G\end{large}awan durch minne \textbf{arbeit} enpfienc;\\ 
10 & sîn vrouwe reit, zuo vuoz er gienc.\\ 
 & Urgeluse und der degen balt,\\ 
 & \textbf{die} kômen \textit{i}n eine\textit{n} grôzen walt.\\ 
 & dennoch muos er \dag gênse\dag  wonen.\\ 
 & er \textbf{zôch} daz pfert zuo einem ronen.\\ 
15 & sînen schilt, der ê dâr ûf lac,\\ 
 & des er durch \textbf{ritters} ambet pflac,\\ 
 & nam er zuo halse; ûf daz pfert er saz.\\ 
 & ez \textbf{truoc} in \dag kumber\dag  vürbaz\\ 
 & anderhalben ûz in erbûwen lant.\\ 
20 & ein burc er mit den ougen vant;\\ 
 & sîn herz und \textbf{diu} ougen jâhen,\\ 
 & daz si erkanden \textbf{und} gesâhen\\ 
 & dekein \textbf{nie} der glîch.\\ 
 & si was alumb \textbf{ritterlîch}.\\ 
25 & \textbf{beide} türne und palas\\ 
 & manigez ûf der bürge was.\\ 
 & dar zuo muos er schouwen\\ 
 & in den venstern manige vrouwen;\\ 
 & der \textbf{was} vier hundert oder mêr,\\ 
30 & vier under \textit{in} von art hêr.\\ 
\end{tabular}
\scriptsize
\line(1,0){75} \newline
m n o \newline
\line(1,0){75} \newline
\textbf{9} \textit{Illustration mit Überschrift:} Also gawan mit liscosie vaht an des wassers staden m (n)   $\cdot$ \textit{Großinitiale} n   $\cdot$ \textit{Initiale} m o  \newline
\line(1,0){75} \newline
\textbf{1} \textit{Die Verse 534.1-8 fehlen} o  \textbf{2} Gawan] gawen m her gawan n \textbf{3} entwenken] entwecket m \textbf{7} wert] \textit{om.} n \textbf{8} muoz] muͯs m \textbf{11} Urgeluse] Vrgeluͯse m o \textbf{12} in einen] ein einem m \textbf{13} muos] muͯsz n muͦsz o \textbf{18} ez] Er o \textbf{19} in] \textit{om.} o \textbf{20} den] dem o \textbf{21} ougen] auge o \textbf{22} und] noch n (o) \textbf{23} dekein] Do kein burg n Dekein buͯrg o  $\cdot$ der] [ger]: der n \textbf{24} si] Suͯs o \textbf{27} muos] muͯsz n o \textbf{30} in] \textit{om.} m \newline
\end{minipage}
\end{table}
\newpage
\begin{table}[ht]
\begin{minipage}[t]{0.5\linewidth}
\small
\begin{center}*G
\end{center}
\begin{tabular}{rl}
 & \begin{large}S\end{large}wie gerne ich in \textbf{nû} næme dan,\\ 
 & \textbf{doch} mag mîn hêrre Gawan\\ 
 & der minne \textbf{des} niht entwenken,\\ 
 & si\textbf{ne} welle \textbf{im} vröude krenken.\\ 
5 & waz hilfet dane mîn underslac,\\ 
 & swaz ich dâ von gesprechen mac?\\ 
 & wert man sol sich niht minnen weren,\\ 
 & wan \textbf{den} muoz minne \textbf{helfen} neren.\\ 
 & Gawan durch minne \textbf{arbeit} enpfienc;\\ 
10 & sîn vrouwe reit, ze vuoze er gienc.\\ 
 & Orgeluse unde der degen balt,\\ 
 & \textbf{die} kômen in einen grôzen walt.\\ 
 & dannoch muos er gêns wonen.\\ 
 & er \textbf{zôch} daz pfert zuo eine\textit{m} ronen.\\ 
15 & sînen schilt, der ê drûf lac,\\ 
 & des er durch \textbf{schiltes} ambet pflac,\\ 
 & nam er ze halse; ûfez pfert er saz.\\ 
 & ez \textbf{trüege} in kûme vürbaz\\ 
 & anderhalp ûz in erbûwen lant.\\ 
20 & ein burc er mit den ougen vant;\\ 
 & sîn herze unde \textbf{diu} ougen jâhen,\\ 
 & daz si erkanden \textbf{noch} gesâhen\\ 
 & deheine \textbf{burc} \textbf{nie} der gelîch.\\ 
 & si was alumbe \textbf{rîterlîch}.\\ 
25 & türne unde palas\\ 
 & manigez ûf der bürge was.\\ 
 & dar zuo muos er schouwen\\ 
 & in den venstern manige vrouwen;\\ 
 & der \textbf{was} vier hundert ode mêr,\\ 
30 & viere under in von arde hêr.\\ 
\end{tabular}
\scriptsize
\line(1,0){75} \newline
G I L M Z Fr19 \newline
\line(1,0){75} \newline
\textbf{1} \textit{Initiale} G L M Z Fr19  \textbf{9} \textit{Initiale} I  \newline
\line(1,0){75} \newline
\textbf{1} Swie] Wie L Owe M  $\cdot$ in] \textit{om.} I  $\cdot$ nû] \textit{om.} L M Z Fr19 \textbf{2} mag] enmach L  $\cdot$ hêrre Gawan] ergawan M \textbf{3} minne] libe M \textbf{4} sine welle] Si enwollin M Sie wolle Z \textbf{6} swaz] Waz L (M) Z \textbf{7} man sol sich] sol ich Fr19  $\cdot$ minnen] mynne M (Z) (Fr19) \textbf{8} den] der M  $\cdot$ muoz] sol L  $\cdot$ helfen] selbe L \textbf{9} minne] libe M \textbf{10} ze vuoze] vnd I \textbf{11} Orgeluse] orgenluse I Orgelýse L Orgiloise M [Orgelvge]: Orgelvse Fr19  $\cdot$ degen] helt Fr19 \textbf{12} die] \textit{om.} M \textbf{13} er] \textit{om.} L  $\cdot$ gêns] gennes L gense Z \textbf{14} einem] einen G eyme M einer Fr19 \textbf{18} trüege] truc M Z (Fr19)  $\cdot$ kûme] mit kummer M kvmber Fr19 \textbf{19} ûz] \textit{om.} M Z vnz Fr19  $\cdot$ in] in ein L  $\cdot$ erbûwen] irbuwet M \textbf{21} unde] vnd ouch L  $\cdot$ diu] Sin M \textbf{23} deheine] Ny ichein M  $\cdot$ nie] \textit{om.} I M \textbf{28} in den venstern] indem venster I  $\cdot$ vrouwen] shone frowen I vrowe L \newline
\end{minipage}
\hspace{0.5cm}
\begin{minipage}[t]{0.5\linewidth}
\small
\begin{center}*T
\end{center}
\begin{tabular}{rl}
 & Swie gerne ich in \textbf{nû} næme dan,\\ 
 & \textbf{Sône} mac mîn hêr Gawan\\ 
 & der minne niht entwenken,\\ 
 & si\textbf{ne} welle \textbf{im} vröude krenken.\\ 
5 & waz hilfet dan mîn underslac,\\ 
 & swaz ich dâ von gesprechen mac?\\ 
 & wert man sol sich niht minne wern,\\ 
 & wan \textbf{in} muoz minne \textbf{helfe} nern.\\ 
 & Gawan durch minne \textbf{nôt} enpfienc;\\ 
10 & sîn vrouwe reit, zuo vuoz er gienc.\\ 
 & \textit{\begin{large}O\end{large}}rgeluse unde der degen balt\\ 
 & kômen in einen grôzen walt.\\ 
 & dannoch muoser gênes wonen.\\ 
 & er \textbf{vuorte} daz pfert zuo einem ronen.\\ 
15 & sînen schilt, der ê drûffe lac,\\ 
 & des er durch \textbf{schiltes} ambet pflac,\\ 
 & \textbf{den} nam er ze halse; ûf daz pfert er saz.\\ 
 & ez \textbf{truoc} in kûme vürbaz\\ 
 & anderthalp ûz in \textbf{ein} erbûwen lant.\\ 
20 & eine burc er mit den ougen vant;\\ 
 & sîn herze unde \textbf{sîniu} ougen jâhen,\\ 
 & daz si erkanten \textbf{noch} gesâhen\\ 
 & deheine \textbf{burc} der glîc\textit{h}.\\ 
 & si was alumbe \textbf{rîchelîch}.\\ 
25 & Türne unde palas\\ 
 & manegez ûf der bürge was.\\ 
 & dar zuo muoser s\textit{ch}ouwen\\ 
 & in den venstern manege \textbf{schœne} vrouwen;\\ 
 & der \textbf{wâren} vier hundert \textbf{unde} oder mêr,\\ 
30 & viere under in von arde hêr.\\ 
\end{tabular}
\scriptsize
\line(1,0){75} \newline
T U V W O Q R Fr40 \newline
\line(1,0){75} \newline
\textbf{1} \textit{Majuskel} T  \textbf{2} \textit{Majuskel} T  \textbf{4} \textit{Überschrift:} Wie Gawan mit lischosen vaht V  \textbf{11} \textit{Überschrift:} \sout{Wie Gawan mit lischosen vaht} V   $\cdot$ \textit{Initiale} T U V W O R  \textbf{25} \textit{Majuskel} T  \newline
\line(1,0){75} \newline
\textbf{1} Swie] Wie U W Q R  $\cdot$ in nû] \textit{om.} W in O Q R Fr40 \textbf{2} Sône mac] Sonen mag V So ist R so emak Fr40  $\cdot$ mîn] meine Q  $\cdot$ hêr] herze U \textbf{4} sine welle] Sinen welle O Seine welle Q Sy welle R (Fr40) \textbf{6} swaz] Waz U (W) (Q) (R) \textbf{7} wert] welrt Fr40  $\cdot$ sich] ich U  $\cdot$ niht] \textit{om.} W \textbf{8} minne] minnen U (V) W (Q) (Fr40)  $\cdot$ helfe] [s*]: helfe T helfen Q  $\cdot$ nern] mern U \textbf{9} Gawan] Gawin R  $\cdot$ nôt] arbeit U (W) O Q R (Fr40) [*]: arbeit  V \textbf{10} reit] reit er Q  $\cdot$ er] \textit{om.} Q \textbf{11} Orgeluse] ÷Rgelvse T ÷rgelvse O Argeluse Q \textbf{13} dannoch] Da R  $\cdot$ muoser] mveser T muͦz er U (W) (O) \textbf{14} vuorte] zoch U V W O Q R  $\cdot$ pfert] ros R  $\cdot$ einem] einer O Q (R) \textbf{15} ê] \textit{om.} O \textbf{16} durch] [*]: durch V doch R \textbf{17} Den nam] [*]: Den nam V  $\cdot$ ûf daz pfert] [*]: vffes pfert V dar vf O \textbf{18} ez] Er W O Q  $\cdot$ in kûme] [*]: in kvme V \textbf{19} ûz] [*]: vs V \textit{om.} O vor R  $\cdot$ ein erbûwen] [*]: ein erbuen V \textbf{20} burc] [*]: burg V bruc Q \textbf{21} sîn] [*]: Sine V  $\cdot$ sîniu] [*]: sine V \textbf{22} daz si] [*]: Daz sú V Der sie Q Das R \textbf{23} der] nie der V W (O) Q  $\cdot$ glîch] glic T \textbf{24} rîchelîch] riterlich U (V) (W) (O) (Q) (R) \textbf{25} Türne] Tᵫrnen R \textbf{27} dar zvͦ mveser souwen T  $\cdot$ Dar zuͦ muͦz er scheuͦwen U (O) \textbf{29} wâren] worden Q  $\cdot$ unde] \textit{om.} U V W Q R  $\cdot$ oder] \textit{om.} O \textbf{30} under in] darunder R \newline
\end{minipage}
\end{table}
\end{document}
