\documentclass[8pt,a4paper,notitlepage]{article}
\usepackage{fullpage}
\usepackage{ulem}
\usepackage{xltxtra}
\usepackage{datetime}
\renewcommand{\dateseparator}{.}
\dmyyyydate
\usepackage{fancyhdr}
\usepackage{ifthen}
\pagestyle{fancy}
\fancyhf{}
\renewcommand{\headrulewidth}{0pt}
\fancyfoot[L]{\ifthenelse{\value{page}=1}{\today, \currenttime{} Uhr}{}}
\begin{document}
\begin{table}[ht]
\begin{minipage}[t]{0.5\linewidth}
\small
\begin{center}*D
\end{center}
\begin{tabular}{rl}
\textbf{298} & \begin{large}Û\end{large}f dem Plimizœles plân.\\ 
 & Keie wart geholt sân,\\ 
 & in Artuses poulûn getragen.\\ 
 & sîne vriunt begunden in dâ klagen,\\ 
5 & vil vrouwen unt manec man.\\ 
 & dô kom ouch \textbf{mîn} hêr Gawan\\ 
 & über \textbf{in, dâ Keie} lac.\\ 
 & er sprach: "owê unsælic tac,\\ 
 & daz disiu tjost ie wart getân,\\ 
10 & dâ von ich vriunt verlorn hân!"\\ 
 & Er klaget in senlîche.\\ 
 & Keie, der zornes rîche,\\ 
 & sprach: "hêrre, erbarmet iuch mîn lîp?\\ 
 & sus solten klagen altiu wîp.\\ 
15 & ir sît mînes hêrren \textbf{swester} sun.\\ 
 & m\textit{ö}ht ich iu dienst nû getuon,\\ 
 & als \textbf{iwer wille gerte},\\ 
 & dô mich got der lide werte,\\ 
 & sô\textbf{ne} hât \textbf{daz mîn hant} niht vermiten,\\ 
20 & sine habe vil durch iu\textit{ch} gestriten.\\ 
 & \textbf{ich} tæte ouch noch, \textbf{unt} \textbf{solt} ez sîn.\\ 
 & \textbf{nû}\textbf{ne} klagt \textbf{nie} mêre, lât mir den pîn.\\ 
 & iwer œheim, der künec hêr,\\ 
 & gewinnet nimmer \textbf{sölhen} Keien mêr.\\ 
25 & Ir sît mir \textbf{râche} ze \textbf{wol} geborn.\\ 
 & het \textbf{aber ir} einen vinger \textbf{dort} verlorn,\\ 
 & dâ \textbf{wâgte ich gegen} mîn houbet.\\ 
 & seht, ob ir mir\textbf{z} geloubet.\\ 
 & kêrt iuch niht an mîn hetzen.\\ 
30 & er kan unsanfte letzen,\\ 
\end{tabular}
\scriptsize
\line(1,0){75} \newline
D \newline
\line(1,0){75} \newline
\textbf{1} \textit{Initiale} D  \textbf{11} \textit{Majuskel} D  \textbf{25} \textit{Majuskel} D  \newline
\line(1,0){75} \newline
\textbf{1} Plimizœles] Plimizoͤls D \textbf{3} Artuses] Artvs D \textbf{16} möht] moht D \textbf{20} iuch] iv D \newline
\end{minipage}
\hspace{0.5cm}
\begin{minipage}[t]{0.5\linewidth}
\small
\begin{center}*m
\end{center}
\begin{tabular}{rl}
 & ûf dem Plimizoles plân.\\ 
 & Keie wart geholt sân,\\ 
 & in Artuses pavelûn getragen.\\ 
 & sîne vriunt begunden in dô klagen,\\ 
5 & vil vrouwen und manic man.\\ 
 & dô kam ouch \textbf{mîn} hêrre Gawan\\ 
 & über \textbf{in, d\textit{â} Keie} lac.\\ 
 & er sprach: "ouwê unsælic tac,\\ 
 & daz disiu just ie wart getân,\\ 
10 & dâ von ich vriunt verlorn hân!"\\ 
 & er klagete in senlîche.\\ 
 & Keie, der zornes rîche,\\ 
 & sprach: "hêrre, erbarmet iuch mîn lîp?\\ 
 & sus solten klagen altiu wîp.\\ 
15 & ir sît mînes hêrren sun.\\ 
 & m\textit{ö}ht ich iu dienst nû getuon,\\ 
 & als \textbf{iuwer wille gerte},\\ 
 & dô mich got der lide werte.\\ 
 & sô \textbf{en}hât \textbf{mî\textit{n} hant \textit{daz}} niht vermiten,\\ 
20 & sine habe vil durch iuch ges\textit{tr}iten\\ 
 & \textbf{und} tæte ouc\textit{h} \textit{n}och, \textbf{solte}z sîn.\\ 
 & \textbf{nû} klagt \textit{\textbf{nie}} mêre, lât mir den pîn.\\ 
 & iuwer œheim, der künic hêr,\\ 
 & gewinnet nimer \textbf{solhen} Keien mêr.\\ 
25 & ir sît mir \textbf{rehte} ze \textbf{wol} geborn.\\ 
 & hete\textit{t} \textbf{aber ir} einen vinger verlorn,\\ 
 & dâ \textbf{wâget \textit{i}ch gegen} mîn houbet.\\ 
 & seht, ob ir mir\textbf{s} geloubet.\\ 
 & \textbf{doch} kêrt iuch niht an mîn hetzen.\\ 
30 & er kan unsanfte letzen,\\ 
\end{tabular}
\scriptsize
\line(1,0){75} \newline
m n o Fr69 \newline
\line(1,0){75} \newline
\newline
\line(1,0){75} \newline
\textbf{1} Plimizoles] blimizols m plimors n plinors o plimiszolles Fr69 \textbf{2} Keie] Keẏe n Keẏn o \textbf{3} Artuses] artus m n o \textbf{4} in] jme n (o) \textbf{6} Gawan] gewann o \textbf{7} dâ] do m n o  $\cdot$ Keie] keẏe n keẏ o \textbf{8} unsælic] onseligen o \textbf{9} ie] e o \textbf{11} klagete] sagte o  $\cdot$ in] \textit{om.} n o \textbf{12} Keie] Keẏe n Keẏ o \textbf{13} iuch] vber o \textbf{15} hêrren] hertzen herren n (o) \textbf{16} möht] Mocht m (o)  $\cdot$ dienst] [nuͯ]: dienst o \textbf{19} sô enhât] Do hette n Da het o  $\cdot$ mîn] mit m  $\cdot$ daz] \textit{om.} m \textbf{20} sine habe] Sú hat n Sie hab o  $\cdot$ gestriten] gesnitten m \textbf{21} tæte] e dete o  $\cdot$ ouch noch] ouch noch noch m das noch n das [*]: noch o  $\cdot$ sîn] din o \textbf{22} klagt] clagte n o  $\cdot$ nie mêre] in mere m min ere n (o)  $\cdot$ den] die n \textbf{23} œheim] hohem o  $\cdot$ der] den n das o \textbf{24} Keien] keẏe n o \textbf{25} rehte] rich n rach o  $\cdot$ ze] so o \textbf{26} hetet aber ir] Hette aber ir m Hettent ir aber n \textbf{27} wâget] noͯge n voget o  $\cdot$ ich] uͯch m \textbf{29} kêrt] kerte n o  $\cdot$ hetzen] herczen o \newline
\end{minipage}
\end{table}
\newpage
\begin{table}[ht]
\begin{minipage}[t]{0.5\linewidth}
\small
\begin{center}*G
\end{center}
\begin{tabular}{rl}
 & ûf dem Blimzoles plân.\\ 
 & Kay wart geholt sân,\\ 
 & in Artuses pavelûn getragen.\\ 
 & sîne vriunt begunden in dô klagen,\\ 
5 & vil vrouwen unde manic man.\\ 
 & dar kom ouch hêr Gawan\\ 
 & über \textbf{Kayn, \textit{al}dâ er} lac.\\ 
 & er sprach: "owê unsælic tac,\\ 
 & daz disiu tjost ie wart getân,\\ 
10 & dâ von ich vriunt verloren hân!"\\ 
 & er klagte in senlîche.\\ 
 & Kay, der zornes rîche,\\ 
 & \begin{large}S\end{large}prach: "hêrre, erbarmet iuch mîn lîp?\\ 
 & sus solten klagen altiu wîp.\\ 
15 & ir sît mînes hêrren \textbf{swester} sun.\\ 
 & m\textit{ö}ht ich iu dienst nû getuon,\\ 
 & als \textbf{i\textit{r} etswenne gert},\\ 
 & dô mich got der lide wert,\\ 
 & sô\textbf{ne} hât \textbf{mîn hant daz} niht vermiten,\\ 
20 & sine habe vil durch iuch gestriten.\\ 
 & \textbf{si} tæte ouch noch, \textbf{m\textit{ö}ht} ez sîn.\\ 
 & klaget \textbf{niht} mê, lât mir den pîn.\\ 
25 & \hspace*{-.7em}\big| ir sît mir \textbf{râche} ze \textbf{hôch} geboren.\\ 
 & \hspace*{-.7em}\big| het \textbf{aber ir} einen vinger verloren,\\ 
 & \hspace*{-.7em}\big| d\textit{â} \textbf{\textit{w}âget ich \textit{gein}} mîn houbet.\\ 
 & \hspace*{-.7em}\big| sehet, obe ir mir\textbf{z} geloubet.\\ 
 & \hspace*{-.7em}\big| kêrt iuch niht an mîn hetzen.\\ 
30 & \hspace*{-.7em}\big| er kan unsanfte letzen,\\ 
 & \hspace*{-.7em}\big| iwer œheim, der künic hêr,\\ 
 & \hspace*{-.7em}\big| gewinnet nimer \textbf{deheinen} Kay mêr.\\ 
\end{tabular}
\scriptsize
\line(1,0){75} \newline
G I O L M Q R Z \newline
\line(1,0){75} \newline
\textbf{5} \textit{Überschrift:} Wie parcifal frowen iescuten hulde gewan Vnd fvr sie swuͦr Vnd wie in die minne von sinnen brahte Vnd wie er Segremors vnd keyn nider stach Lis vor zwei bleter Z   $\cdot$ \textit{Initiale} Z  \textbf{7} \textit{Initiale} I  \textbf{11} \textit{Initiale} O Q  \textbf{13} \textit{Initiale} G  \textbf{15} \textit{Capitulumzeichen} L  \textbf{23} \textit{Initiale} I  \newline
\line(1,0){75} \newline
\textbf{1} dem] den I des Q  $\cdot$ Blimzoles] plimizol I M Brimizols O Plimizolles L plimizols Q R plimizoles Z \textbf{2} Kay] kai G kain I Key O Q R Z Keie M  $\cdot$ geholt] gefuͤret I gihort M  $\cdot$ sân] dan O \textbf{3} Artuses] artus G Q (R) Z  $\cdot$ pavelûn] geczelt R \textbf{4} dô] da O L M Z \textbf{5} vrouwen] fraw Q  $\cdot$ manic] vil manic Z \textbf{6} dar] Do Q R  $\cdot$ hêr] min O mýn her L (M) (Q) (R) (Z)  $\cdot$ Gawan] gewan Q \textbf{7} Kayn] kain G I keyn O M Q Z key R  $\cdot$ aldâ] da G \textbf{8} owê] we I  $\cdot$ unsælic] vnde salich M vnselig sy der R \textbf{10} ich] der ich I \textbf{11} er] ÷r O  $\cdot$ klagte] clagt I (O) Q (R) \textbf{12} Kay] kai G kain I Key O Q Z Keie M Keẏ R \textbf{13} Sprach] Er sprach R \textbf{14} sus] Den R  $\cdot$ altiu] allú R \textbf{16} möht] moht G I (O) (L) (M) (Q) (Z)  $\cdot$ dienst nû] die dienste L min hilffe R  $\cdot$ getuon] tvͯn L (R) \textbf{17} Als ewer wille gerte Z  $\cdot$ ir] irs G ich O \textbf{18} dô] So O Da M Z  $\cdot$ der] ouch der M \textit{om.} Z  $\cdot$ wert] werte Z \textbf{19} sône hât] So hat O L R Sine M  $\cdot$ mîn] [mit]: min O  $\cdot$ daz] des Q \textit{om.} R \textbf{20} sine habe] Si hab O Sy en habin M Seine hab Q Sy hatt R  $\cdot$ vil durch iuch] durc evch vil I (R) vil duͯrch L \textbf{21} si] Jch Q R  $\cdot$ tæte] tetin M  $\cdot$ ouch] euch Q es R  $\cdot$ möht ez] moht ez G O (L) (M) (Q) (Z) ob ez mohte I \textbf{22} mir] mich I \textit{om.} Z  $\cdot$ den] die L R \textbf{25} ir] han ir I  $\cdot$ mir râche] mir zerache I mir zcu re M miner Rache R  $\cdot$ ze] \textit{om.} M  $\cdot$ hôch] wol I O L M Q Z  $\cdot$ geboren] erborn R \textbf{26} aber ir] ir I ir aber O Q  $\cdot$ einen] doͯrt en R  $\cdot$ verloren] dort verlorn I (L) M (Q) \textbf{27} dâ] Do Q  $\cdot$ wâget] engene waget G  $\cdot$ gein] \textit{om.} G gein im L gern R \textbf{28} mirz] mir R \textbf{29} hetzen] heisszin M herczen R \textbf{24} Gewýnnet solchen kaýen niemer mere L  $\cdot$ deheinen] sulchen M Q (Z) soͯlich R  $\cdot$ Kay mêr] kai mer G kain mer I key mer O keyn mer M keyme Q keẏn wer R [k*]: keyn mer Z \newline
\end{minipage}
\hspace{0.5cm}
\begin{minipage}[t]{0.5\linewidth}
\small
\begin{center}*T
\end{center}
\begin{tabular}{rl}
 & ûf dem Plymizoles plân.\\ 
 & Key wart geholt sân,\\ 
 & in Artuses pavelûn getragen.\\ 
 & sîne vriunt begunden in dô klagen,\\ 
5 & vil vrouwen unde manec man.\\ 
 & dar kom ouch \textbf{mîn} hêr Gawan\\ 
 & über \textbf{Key, dâ er} lac.\\ 
 & er sprach: "ouwê unsælic tac,\\ 
 & daz dis\textit{iu} tjost ie wart getân,\\ 
10 & dâ von ich vriunt verlorn hân!"\\ 
 & er klagete in senelîche.\\ 
 & Key, der zornes rîche,\\ 
 & sprach: "hêrre, erbarmet iuch mîn lîp?\\ 
 & sus solten klagen altiu wîp.\\ 
15 & ir sît mînes hêrren \textbf{swester} sun.\\ 
 & m\textit{ö}ht ich iu dienst nû getuon,\\ 
 & als \textbf{ich etswenne gerte},\\ 
 & dô mich got der lide werte.\\ 
 & sô hât \textbf{mîn hant daz} niht vermiten,\\ 
20 & sine habe vil durch iuch gestriten.\\ 
 & \textbf{si} tæt ouch noch, \textbf{m\textit{ö}ht} ez sîn.\\ 
 & klaget \textbf{niht} mêr, lât mir den pîn.\\ 
25 & \hspace*{-.7em}\big| ir sît mir \textbf{râche} ze \textbf{wol} geborn.\\ 
 & \hspace*{-.7em}\big| het \textbf{ir aber} einen vinger \textbf{dort} verlorn,\\ 
 & \hspace*{-.7em}\big| dâ \textbf{engegen wâgetich} mîn houbet.\\ 
 & \hspace*{-.7em}\big| seht, ob ir mir\textbf{s} geloubet.\\ 
 & \hspace*{-.7em}\big| kêrt iuch niht an mîn hetzen.\\ 
30 & \hspace*{-.7em}\big| er kan unsanfte letzen,\\ 
 & \hspace*{-.7em}\big| iuwer œheim, der künec hêr,\\ 
 & \hspace*{-.7em}\big| gewinnet niemer \textbf{solhen} Key mêr.\\ 
\end{tabular}
\scriptsize
\line(1,0){75} \newline
T U V W \newline
\line(1,0){75} \newline
\textbf{1} \textit{Initiale} W  \newline
\line(1,0){75} \newline
\textbf{1} Plymizoles] [pla*]: plymizolz T plimizols U W [*]: plimezolz V \textbf{2} Key] Kein V \textbf{3} pavelûn] gezelt V \textbf{6} Gawan] gewan U \textbf{7} Key] kein V  $\cdot$ dâ] do U V W  $\cdot$ lac] do lag W \textbf{8} ouwê] \textit{om.} W \textbf{9} disiu] dise T \textbf{10} vriunt] dich W \textbf{11} \textit{Versfolge 298.12-11} V   $\cdot$ senelîche] senecliche V (W) \textbf{12} Key] Kein V \textbf{13} iuch] îv T \textbf{16} möht] moht T (U) [Moht]: Moͤht  V  $\cdot$ dienst nû] nuͦ dinst U dienstes icht W \textbf{19} daz niht] nit daz U (V) (W) \textbf{20} sine] Sinen W  $\cdot$ iuch] îv T \textbf{21} si] [V*]: Vnde V  $\cdot$ ouch] \textit{om.} W  $\cdot$ möht] moht T (U)  $\cdot$ sîn] gesein W \textbf{22} niht] mich W  $\cdot$ den] die U W \textbf{25} râche] recht W \textbf{26} ir aber] [aber]: ir aber T abir ir U (V) ir W  $\cdot$ dort] dorn U \textbf{27} dâ engegen wâgetich] Do waget ich gein U (V) (W) \textbf{28} mirs] mir W \textbf{29} iuch] iv T \textbf{24} Key] keẏn V keyen W \newline
\end{minipage}
\end{table}
\end{document}
