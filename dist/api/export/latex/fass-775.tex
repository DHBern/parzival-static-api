\documentclass[8pt,a4paper,notitlepage]{article}
\usepackage{fullpage}
\usepackage{ulem}
\usepackage{xltxtra}
\usepackage{datetime}
\renewcommand{\dateseparator}{.}
\dmyyyydate
\usepackage{fancyhdr}
\usepackage{ifthen}
\pagestyle{fancy}
\fancyhf{}
\renewcommand{\headrulewidth}{0pt}
\fancyfoot[L]{\ifthenelse{\value{page}=1}{\today, \currenttime{} Uhr}{}}
\begin{document}
\begin{table}[ht]
\begin{minipage}[t]{0.5\linewidth}
\small
\begin{center}*D
\end{center}
\begin{tabular}{rl}
\textbf{775} & \textbf{\begin{large}U\end{large}tepandraguns sun,}\\ 
 & \textbf{Artusen, sach man alsus tuon:}\\ 
 & \textbf{er} prüevete \textbf{kostenlîche}\\ 
 & eine tavelrunde rîche\\ 
5 & ûz eime drîanthasmê.\\ 
 & ir habt wol gehôrt ê,\\ 
 & wie ûf dem Plimizœles plân\\ 
 & \textbf{einer} tavelrunde wart getân.\\ 
 & \textbf{nâch der} disiu \textbf{wart} gesniten,\\ 
10 & sinwel, \textbf{mit} \textbf{solhen} siten,\\ 
 & si erzeigete rîlîchiu dinc.\\ 
 & sinwel man drumbe nam den rinc\\ 
 & ûf \textbf{einem} \textbf{touwec} \textbf{grüenem} gras,\\ 
 & \textbf{daz} wol ein poynder \textbf{landes} was\\ 
15 & \textbf{vome} sedel \textbf{an} tavelrunder;\\ 
 & \textbf{diu stuont dâ mitten} \textbf{sunder},\\ 
 & niht durch den nutz, \textbf{êt} \textbf{durch} den namen.\\ 
 & sich moht ein bœse man wol schamen,\\ 
 & ob er \textbf{dâ} bî den werden saz.\\ 
20 & die spîse sîn munt mit sünden az.\\ 
 & Der rinc wart bî der schœnen naht\\ 
 & gemezzen unt \textbf{vor bedâht}\\ 
 & wol nâch \textbf{rîlîchen} ziln.\\ 
 & es mohte einen armen künec beviln,\\ 
25 & als man den rinc gezieret vant,\\ 
 & dô der mitte morgen wart \textbf{erkant},\\ 
 & Gramoflanz unt Gawan,\\ 
 & von in diu koste wart getân.\\ 
 & Artus was des landes gast.\\ 
30 & sîner koste \textbf{iedoch} dâ niht gebrast.\\ 
\end{tabular}
\scriptsize
\line(1,0){75} \newline
D Fr2 \newline
\line(1,0){75} \newline
\textbf{1} \textit{Initiale} D Fr2  \textbf{21} \textit{Majuskel} D  \newline
\line(1,0){75} \newline
\textbf{1} Utepandraguns] ÷tepandragvns Fr2 \textbf{7} Plimizœles] Plimizoͤls D plimi::: Fr2 \textbf{8} einer] Eine Fr2 \textbf{12} man drumbe] drvmbe man Fr2 \textbf{13} einem] ein Fr2  $\cdot$ grüenem] grvne Fr2 \textbf{24} mohte] moht wol Fr2 \textbf{26} mitte] mitten Fr2 \textbf{27} Gramoflanz] Gramoflantz Fr2 \textbf{30} :::ost da doch ::: Fr2 \newline
\end{minipage}
\hspace{0.5cm}
\begin{minipage}[t]{0.5\linewidth}
\small
\begin{center}*m
\end{center}
\begin{tabular}{rl}
 & \textbf{waz Artus des morgens tuo?}\\ 
 & \textbf{dô greif er vrœlîch zuo}\\ 
 & \textbf{und} brüefte \textbf{kostlîch}\\ 
 & ein tavelrunde rîch\\ 
5 & ûz einem drîanthasmê.\\ 
 & ir habet wol gehœret ê,\\ 
 & wie ûf dem Plimizoles plân\\ 
 & \textbf{ein} tavelrunde wart getân.\\ 
 & \textbf{nâch der} disiu \textbf{wart} gesniten,\\ 
10 & sinwel, \textbf{nâch} \textbf{solhem} siten,\\ 
 & si erz\textit{öu}gte rîlîchiu dinc.\\ 
 & sinwel man dar umb nam den rinc\\ 
 & ûf \textbf{eine\textit{m}} \textbf{grüenen} gras,\\ 
 & \textbf{daz} wol ein ponder \textbf{langez} was\\ 
15 & \textbf{vom} sedel \textbf{an} tavelrunder;\\ 
 & \textbf{diu stuont d\textit{â} mitten} \textbf{sunder},\\ 
 & niht durch den nutz, \textbf{durch} den namen.\\ 
 & sich moht ein bœser man wol schamen,\\ 
 & ob er \textbf{d\textit{â}} bî den werden saz.\\ 
20 & die spîse sîn munt mit sünden az.\\ 
 & der rinc wart bî der schœnen naht\\ 
 & gemezzen und \textbf{vor bedâht}\\ 
 & wol nâch \textbf{rîchen} ziln.\\ 
 & es moht einen armen künic beviln,\\ 
25 & als man den rinc gezieret vant,\\ 
 & dô der mitte morgen wart \textbf{erkant},\\ 
 & Gram\textit{o}lanz und Gawan,\\ 
 & von in diu koste wart getân.\\ 
 & Artus was des landes gast.\\ 
30 & sîner kost d\textit{â} niht \textit{g}ebrast.\\ 
\end{tabular}
\scriptsize
\line(1,0){75} \newline
m n o V V' W \newline
\line(1,0){75} \newline
\textbf{1} \textit{Initiale} V  \newline
\line(1,0){75} \newline
\textbf{4} tavelrunde] tauelrunder W \textbf{5} ûz] Auff W  $\cdot$ drîanthasmê] drianthasine n (o) dẏantasme V (V') \textbf{6} \textit{statt 775.6-9:} Eine tauelrunde wart gesniten e V'  \textbf{7} Plimizoles] plimizols m n W plimzols o plimenzol V \textbf{8} ein] Einre V \textbf{9} disiu] wise n \textbf{10} \textit{Verse 775.10-11 kontrahiert zu:} Sinewel sie erzeigen riliche ding V'   $\cdot$ solhem] solichen W \textbf{11} erzöugte] erzogtte m erzeigete n V erzaiget W \textbf{13} einem] einen m  $\cdot$ grüenen] toͮwig gruͤneme V towen grunē V' \textbf{14} daz] Do W  $\cdot$ ein ponder langez] ein ponder landes V (V') eins ponders lenge W \textbf{15} an] der W \textbf{16} dâ] do m n o V V' W \textbf{17} niht] \textit{om.} o  $\cdot$ den nutz] nutz W  $\cdot$ durch] eht durch V al durch V' noch durch W \textbf{18} moht] moͯchte n (V) (W)  $\cdot$ man] \textit{om.} V'  $\cdot$ wol] \textit{om.} W \textbf{19} dâ] do m n o V W \textit{om.} V'  $\cdot$ bî] hei W \textbf{21} \textit{Die Verse 775.21-776.24 fehlen} V'  \textbf{23} rîchen] rilichen n o V (W) \textbf{24} moht] moͯchte n (V) (W) \textbf{26} mitte] mitten W  $\cdot$ erkant] bekant W \textbf{27} Gramolanz] Gramulantz m Gramolantz n Gramulancz o Gramaflanz V Gramoflantz W  $\cdot$ Gawan] gawann o \textbf{29} was] wart o \textbf{30} sîner] Einer W  $\cdot$ dâ] do m n o W ie doch do V  $\cdot$ gebrast] vergebrast m \newline
\end{minipage}
\end{table}
\newpage
\begin{table}[ht]
\begin{minipage}[t]{0.5\linewidth}
\small
\begin{center}*G
\end{center}
\begin{tabular}{rl}
 & \textbf{\begin{large}U\end{large}tpandraguns sun,}\\ 
 & \textbf{Artus, sprach: "man sol sus tuon."}\\ 
 & \textbf{er} prüevet \textbf{kosteclîche}\\ 
 & ein tavelrunder rîche\\ 
5 & ûz einem d\textit{r}îanthasmê.\\ 
 & ir habet wol gehœret ê,\\ 
 & wie ûf dem Blimzoles plân\\ 
 & \textbf{ein} tavelrunder wart getân.\\ 
 & \textbf{dâ wart} disiu \textbf{nâch} gesniten,\\ 
10 & sinewel, \textbf{mit} \textbf{solchen} siten,\\ 
 & si erzeigte rîchlîchiu dinc.\\ 
 & sinewel \textit{m}a\textit{n} drumbe nam den rinc\\ 
 & ûf \textbf{ein} \textbf{touwec} \textbf{grüene} gras,\\ 
 & \textbf{dâ} wol ein poynder \textbf{landes} was.\\ 
15 & \textbf{von} sedel \textbf{ein} tavelrunder\\ 
 & \textbf{dâ enmitten stuont} \textbf{besunder} -\\ 
 & niht durch den nutz, \textbf{êr} den namen.\\ 
 & sich mohte ein bœse man wol schamen,\\ 
 & ob er \textbf{dâ} bî den werden saz.\\ 
20 & die spîse sîn munt mit sünden az.\\ 
 & der rinc wart bî der schœnen naht\\ 
 & \textbf{wol} gemezzen unde \textbf{vor bedâht}\\ 
 & wol nâch \textbf{rîchlîchen} zilen.\\ 
 & es mohte einen arm\textit{en} \textit{künic} bevilen,\\ 
25 & alse man den rinc geziert vant,\\ 
 & dô der mitter morgen wart \textbf{bekant},\\ 
 & Gramoflanz unde Gawan,\\ 
 & von in diu koste wart getân.\\ 
 & Artus was des landes gast.\\ 
30 & sîner koste \textbf{doch} dâ niht gebrast.\\ 
\end{tabular}
\scriptsize
\line(1,0){75} \newline
G I L M Z Fr18 Fr48 \newline
\line(1,0){75} \newline
\textbf{1} \textit{Initiale} G I L M Z Fr18 Fr48  \newline
\line(1,0){75} \newline
\textbf{1} Utpandraguns] Vtpandragvns G Vter pantroGunes I [Utprandrag*nz]: Utprandragvͯnz L Vter pandraguͯns M \textbf{2} Artus] Artusen M (Z) (Fr18) (Fr48)  $\cdot$ sprach] sach L M Z Fr18 Fr48  $\cdot$ sol sus] alsus L M (Fr18) svs Z also Fr48 \textbf{3} prüevet] vͦpte I pruͯfte L (M) (Z) \textbf{4} tavelrunder] Tauelrunde I (M) \textbf{5} drîanthasmê] dianthasme G (M) diatasme I Sarantasme L \textbf{7} ûf] \textit{om.} L  $\cdot$ Blimzoles] plimzoles G plimizols I M Z plýmizols L plimizol Fr48 \textbf{8} ein] Einer Z (Fr48)  $\cdot$ tavelrunder] Tavelrvnde L \textbf{11} erzeigte] erGazte I erzeýgent L erzeigt Fr48 \textbf{12} man] nam G \textbf{13} touwec] gitouwet M \textbf{15} sedel] gesidel L dem sedel Z Fr48  $\cdot$ ein] an M Z Fr48 \textbf{16} Die stunt da mitten sunder L (M) (Z) (Fr48) \textbf{17} êr] durch Z vnd durch Fr48 \textbf{19} dâ] do Fr48  $\cdot$ den] deme M \textbf{22} wol] \textit{om.} L M Z Fr48  $\cdot$ vor bedâht] Gedaht I \textbf{23} rîchlîchen] riterlichen I (L) \textbf{24} einen] einem Fr48  $\cdot$ armen künic] arm man G  $\cdot$ bevilen] wol beuiln I \textbf{26} dô] Da M  $\cdot$ der] \textit{om.} L  $\cdot$ mitter] mitte I M Z Fr48  $\cdot$ wart] was I \textbf{27} Gramoflanz] Gramoflantz Z \textbf{28} in] den L \textbf{30} doch] iedoch Z (Fr48)  $\cdot$ dâ] do Fr48 \newline
\end{minipage}
\hspace{0.5cm}
\begin{minipage}[t]{0.5\linewidth}
\small
\begin{center}*T
\end{center}
\begin{tabular}{rl}
 & \textbf{\begin{Large}U\end{Large}tpandraguns sun,}\\ 
 & \textbf{Artusen, sach man sus tuon:}\\ 
 & \textbf{er} prüevete \textbf{stolzlîche}\\ 
 & ein tavelrunder rîche\\ 
5 & ûz eime d\textit{r}îanthasmê.\\ 
 & ir habet wol gehœret ê,\\ 
 & wie ûf dem Plymizoles plân\\ 
 & \textbf{ein} tavelrunder wart getân.\\ 
 & \textbf{dâ wart} disiu \textbf{nâch} gesniten,\\ 
10 & sinewel, \textbf{mit} \textbf{solichen} siten,\\ 
 & \textit{si} \textit{er}zeigete rîchlîchiu dinc.\\ 
 & sinewel man drumbe \textit{n}a\textit{m} den rinc\\ 
 & ûf \textbf{ein} \textbf{getouwet} \textbf{grüene} gras,\\ 
 & \textbf{d\textit{â}} wol ein poynder \textbf{landes} was\\ 
15 & \textbf{vonme} sedel \textbf{an} tavelrunder;\\ 
 & \textbf{diu stuont dâ \textit{e}nmitten} \textbf{sunder}\\ 
 & \textit{niht durch den nutz, \textbf{ouch} \textbf{durch} den namen.}\\ 
 & sich mohte ein bœse man wol schamen,\\ 
 & ob er bî den werden saz.\\ 
20 & die spîse sîn munt mit sünden az.\\ 
 & der rinc wart bî der schœnen naht\\ 
 & gemezzen und \textbf{verdâht}\\ 
 & wol nâch \textbf{rîchlîchen} ziln.\\ 
 & es mohte einen armen künec beviln,\\ 
25 & als man den rinc gezieret vant,\\ 
 & dô der mitten morgen \textbf{d\textit{â}} wart \textbf{bekant},\\ 
 & Gramoflanz und Gawan,\\ 
 & von in diu kost wart getân.\\ 
 & Artus was des landes gast.\\ 
30 & sîner kost \textbf{iedoch} dâ niht gebrast.\\ 
\end{tabular}
\scriptsize
\line(1,0){75} \newline
U Q R \newline
\line(1,0){75} \newline
\textbf{1} \textit{Überschrift:} Hie strewet man den rinck do gawan vnd gramoflantz in solten streiten Q   $\cdot$ \textit{Großinitiale} U Q  \newline
\line(1,0){75} \newline
\textbf{1} Utpandraguns] Vtprandraguͦns U Ut pandragen Q Vnd prandaguns R \textbf{2} sus tuon] alsus thu Q \textbf{3} er] Der er Q  $\cdot$ stolzlîche] kostliche Q R \textbf{5} drîanthasmê] thianthasme U dyathasme Q dyantasme R \textbf{7} dem] des Q  $\cdot$ Plymizoles] plimozols U plimizols Q plimiszols R \textbf{8} ein] Einem R \textbf{10} sinewel] Seine wel Q \textbf{11} si erzeigete] Wer zeigete U  $\cdot$ rîchlîchiu] kostliche Q \textbf{12} man] \textit{om.} R  $\cdot$ nam] man U  $\cdot$ den] \textit{om.} Q \textbf{13} getouwet] tawwick Q (R) \textbf{14} dâ] Do U Q R  $\cdot$ poynder] poyder Q pondynder R \textbf{15} vonme] Von mer R \textbf{16} diu] Da R  $\cdot$ dâ enmitten] da ein miten U do mitten Q da mitten R \textbf{17} \textit{Vers 775.17 fehlt} U   $\cdot$ ouch] nun R \textbf{18} sich] Sie Q  $\cdot$ wol] des gar wol R \textbf{19} bî] do bey Q  $\cdot$ werden] wisen R \textbf{20} sîn munt] in sim munde R  $\cdot$ sünden] den sunden Q \textbf{21} rinc] [tisch]: Ring R \textbf{22} verdâht] [wor]: vor gedacht Q vor bracht R \textbf{23} rîchlîchen] Ritterlichen R \textbf{24} es mohte] Esn mocht Q Es moͯchte R \textbf{26} dâ] do U \textit{om.} Q R \textbf{27} Gramoflanz] Gramoflantz Q Gramoflancz R \textbf{28} in] in in Q \textbf{30} dâ] do Q \newline
\end{minipage}
\end{table}
\end{document}
