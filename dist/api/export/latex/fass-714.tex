\documentclass[8pt,a4paper,notitlepage]{article}
\usepackage{fullpage}
\usepackage{ulem}
\usepackage{xltxtra}
\usepackage{datetime}
\renewcommand{\dateseparator}{.}
\dmyyyydate
\usepackage{fancyhdr}
\usepackage{ifthen}
\pagestyle{fancy}
\fancyhf{}
\renewcommand{\headrulewidth}{0pt}
\fancyfoot[L]{\ifthenelse{\value{page}=1}{\today, \currenttime{} Uhr}{}}
\begin{document}
\begin{table}[ht]
\begin{minipage}[t]{0.5\linewidth}
\small
\begin{center}*D
\end{center}
\begin{tabular}{rl}
\textbf{714} & \begin{large}A\end{large}rtus sprach: "sint \textbf{ez} die knaben,\\ 
 & die ich an den rinc \textbf{nâch mir sach} draben?\\ 
 & daz sint von hôher art zwei kint.\\ 
 & waz, ob si sô gevüege sint,\\ 
5 & gar bewart vor missetât,\\ 
 & daz si wol gênt an disen rât?\\ 
 & eintweder pfligt der sinne,\\ 
 & daz er sînes hêrren minne\\ 
 & an mîner nifteln wol siht."\\ 
10 & Bene sprach: "des \textbf{en}weiz ich niht.\\ 
 & hêrre, mag ez mit hulden sîn,\\ 
 & der künec hât \textbf{diz} vingerlîn\\ 
 & dâ her gesant und disen brief.\\ 
 & dô ich nû vürz poulûn lief,\\ 
15 & der kinde einez gab in mir.\\ 
 & vrouwe, sêt, den nemt ir."\\ 
 & Dô wart der brief vil gekust.\\ 
 & \textbf{Itonje} \textbf{druct} in an ir brust.\\ 
 & \textbf{dô sprach si}: "\textbf{hêrre}, \textbf{nû} seht \textbf{hie} an,\\ 
20 & ob mich der künec \textbf{minne man}."\\ 
 & Artus \textbf{nam den brief} in die hant,\\ 
 & \textbf{dâr an er} geschriben vant\\ 
 & von dem, der minnen kunde,\\ 
 & \textbf{waz} ûz sîn selbes munde\\ 
25 & Gramoflanz, der stæte, sprach.\\ 
 & Artus an dem brieve sach,\\ 
 & daz er mit sîme sinne\\ 
 & \textbf{sô endehafte} minne\\ 
 & bî \textbf{sînen zîten} nie vernam.\\ 
30 & dâ stuont daz minne wol gezam:\\ 
\end{tabular}
\scriptsize
\line(1,0){75} \newline
D \newline
\line(1,0){75} \newline
\textbf{1} \textit{Initiale} D  \textbf{17} \textit{Majuskel} D  \newline
\line(1,0){75} \newline
\textbf{18} Itonje] Jtonîe D \textbf{20} man] [mant]: man D \newline
\end{minipage}
\hspace{0.5cm}
\begin{minipage}[t]{0.5\linewidth}
\small
\begin{center}*m
\end{center}
\begin{tabular}{rl}
 & \begin{large}A\end{large}rtus sprach: "sin\textit{t} \textbf{\textit{d}iz} die knaben,\\ 
 & die ich an den rinc \textbf{sach nâch mir} draben?\\ 
 & daz sint von hôher art zwei kint.\\ 
 & waz, ob si sô gevüege sint,\\ 
5 & gar bewart vor missetât,\\ 
 & daz si wol gânt an disen rât?\\ 
 & eintweder pfliget der sinne,\\ 
 & daz er sînes hêrren minne\\ 
 & an mîner nifteln wol siht."\\ 
10 & Bene sprach: "des weiz ich niht.\\ 
 & hêrre, mac ez mit hulden sîn,\\ 
 & der künic hât \textbf{daz} vingerlîn\\ 
 & d\textit{â} her gesant \dag an\dag  disen brief.\\ 
 & dô ich nû vür daz p\textit{a}velûn lief,\\ 
15 & der kinde einez gap in mir.\\ 
 & vrowe, sêt, den nemet ir."\\ 
 & dô wart der brief vil gekust.\\ 
 & \textbf{Ithonie} \textbf{druht} in an ir brust.\\ 
 & \textbf{si sprach}: "\textbf{hêrre}, \textbf{nû} seht \textbf{her} an,\\ 
20 & ob mich der künic \textbf{minnen man}."\\ 
 & Artus \textbf{den brief \textit{n}a\textit{m}} in die hant,\\ 
 & \textbf{dâr an er} geschriben vant\\ 
 & von dem, der minnen kunde.\\ 
 & ûz sîn selbes munde\\ 
25 & Gramolantz, der stæte, sprach.\\ 
 & Artus an dem brieve sach,\\ 
 & daz er mit sînem sinne\\ 
 & \textbf{sô endehafte} minne\\ 
 & bî \textbf{sînen zîten} nie vernam.\\ 
30 & d\textit{â} stuont daz minne wol gezam:\\ 
\end{tabular}
\scriptsize
\line(1,0){75} \newline
m n o Fr69 \newline
\line(1,0){75} \newline
\textbf{1} \textit{Initiale} m   $\cdot$ \textit{Capitulumzeichen} n  \newline
\line(1,0){75} \newline
\textbf{1} sint] sint sint m \textbf{2} den] dem o \textbf{3} zwei] zwein o \textbf{4} ob] >ob< o \textbf{6} an] in o \textbf{10} des] das o \textbf{13} dâ] Do m n o \textbf{14} pavelûn] peuelún m panuelun o \textbf{16} nemet] nenuͯent o  $\cdot$ ir] [wir]: ir Fr69 \textbf{18} Ithonie] Jthonie n Jtonie o  $\cdot$ druht] truͦg o \textbf{21} Artus] Artuͯs nam o  $\cdot$ nam] man m \textbf{23} minnen] mẏnne o \textbf{25} Gramolantz] Gramolancz o Gramoflanz Fr69 \textbf{26} an] in n \textbf{30} dâ] Do m n o \newline
\end{minipage}
\end{table}
\newpage
\begin{table}[ht]
\begin{minipage}[t]{0.5\linewidth}
\small
\begin{center}*G
\end{center}
\begin{tabular}{rl}
 & \begin{large}A\end{large}rtus sprach: "sint \textbf{ez} die knaben,\\ 
 & die ich an den rinc \textbf{sach nâch mir} draben?\\ 
 & daz sint von hôher art zwei kint.\\ 
 & waz, op si sô gevüege sint,\\ 
5 & gar bewart vor missetât,\\ 
 & daz si wol gênt an disen rât?\\ 
 & eintweder pfliget der sinne,\\ 
 & daz er sînes hêrren minne\\ 
 & an mîner nifteln wol siht."\\ 
10 & Bene sprach: "des\textbf{ne} weiz ich niht.\\ 
 & hêrre, mag ez mit hulden sîn,\\ 
 & der künec hât \textbf{ditze} vingerlîn\\ 
 & dâ her gesant unde disen brief.\\ 
 & dô ich nû vür daz pavelûn lief,\\ 
15 & der kinde einez gab in mir.\\ 
 & vrouw\textit{e}, \textit{s}eht, den nemt ir."\\ 
 & dô wart der brief vil gekust.\\ 
 & \textbf{si} \textbf{dructe} in an ir brust.\\ 
 & \textbf{dô sprach si}: "\textbf{nû} seht \textbf{hie} an,\\ 
20 & obe mich der künec \textbf{minne man}."\\ 
 & Artus \textbf{den brief nam} in die hant,\\ 
 & \textbf{dâr an er} geschriben vant\\ 
 & von dem, der minnen kunde,\\ 
 & \textbf{waz} ûz sîn selbes munde\\ 
25 & Gramoflanz, der stæte, sprach.\\ 
 & Artus an dem brieve sach,\\ 
 & daz er mit sînem sinne\\ 
 & \textbf{endehafter} minne\\ 
 & bî \textbf{sîner zîte} nie vernam.\\ 
30 & dâ stuont daz minnen wol gezam:\\ 
\end{tabular}
\scriptsize
\line(1,0){75} \newline
G I L M Z Fr18 Fr22 \newline
\line(1,0){75} \newline
\textbf{1} \textit{Initiale} G L Z Fr18  \textbf{15} \textit{Initiale} I  \newline
\line(1,0){75} \newline
\textbf{1} sint] \textit{om.} L \textbf{2} ich] \textit{om.} M  $\cdot$ sach nâch mir] nach mir sach L (Fr22) sich noch mir M  $\cdot$ draben] haben M \textbf{6} wol] so wol Z  $\cdot$ an] in L \textbf{7} eintweder] Eintwerre L  $\cdot$ der] er Z \textbf{8} hêrren] herczen M \textbf{9} nifteln] niftel L Z \textbf{10} Bene] Bên Fr18  $\cdot$ desne] daz L \textbf{13} dâ] \textit{om.} Z \textbf{14} dô] Da M Z  $\cdot$ pavelûn] gezelt L \textbf{16} vrouwe seht] Froͮwe nemt seht G \textbf{17} der brief wart wol gekust I  $\cdot$ dô] Da M Z \textbf{18} si] Jconie Z  $\cdot$ dructe] [dr*]: druct I (Fr18) [war]: drukt  Z  $\cdot$ an] vaste an L nahe an M (Fr18) \textbf{19} dô sprach si] si sprach I Sie sprach sie L Da sprach sie M  $\cdot$ nû] oheim Z \textbf{20} mich] \textit{om.} L  $\cdot$ man] min man I \textbf{21} den brief nam] nam den brief L \textbf{24} sîn] si I sines M \textbf{25} Gramoflanz] Gramoflantz Z  $\cdot$ sprach] [sach]: sprach I \textbf{29} sîner zîte] sinen ziten I (L) (M) (Z) \textbf{30} minnen wol] minne wol I (M) Z wol mýnne L  $\cdot$ gezam] zam L \newline
\end{minipage}
\hspace{0.5cm}
\begin{minipage}[t]{0.5\linewidth}
\small
\begin{center}*T
\end{center}
\begin{tabular}{rl}
 & \begin{large}A\end{large}rtus sprach: "sint \textbf{ez} die knaben,\\ 
 & die ich an den rinc \textbf{sach nâch mir} draben?\\ 
 & daz sint von hôher art zwei kint.\\ 
 & waz, ob si sô gevüege sint,\\ 
5 & gar bewart vor missetât,\\ 
 & daz si wol gânt an disen rât?\\ 
 & \textit{ei}ntwe\textit{d}e\textit{r} pfliget der sinne,\\ 
 & daz er sînes hêrren minne\\ 
 & an mîner nifteln wol siht."\\ 
10 & Bene sprach: "des \textbf{en}weiz ich niht.\\ 
 & hêrre, mag ez mit hulden sîn,\\ 
 & der künec hât \textbf{diz} vingerlîn\\ 
 & d\textit{â} her gesant und disen brief.\\ 
 & dô ich nû vür daz pavelûn lief,\\ 
15 & der kinde einez gap in mir.\\ 
 & vrouwe, sêt, den nemt ir."\\ 
 & dô wart der brief vil gekust.\\ 
 & \textbf{si} \textbf{twanc} in \textbf{vaste} an ir brust.\\ 
 & \textbf{dô sprach si}: "\textbf{hêrre}, sehet \textbf{her} an,\\ 
20 & o\textit{b} mich der künec \textbf{minnen kan}."\\ 
 & Artus \textbf{den brief nam} in die hant,\\ 
 & \textbf{daz er dâr an} geschriben vant\\ 
 & von dem, der minnen kunde,\\ 
 & \textbf{waz} ûz sîn selbes munde\\ 
25 & Gramoflanz, der stæte, sprach.\\ 
 & Artus an dem brieve sach,\\ 
 & daz er mit sîme sinne\\ 
 & \textbf{endehafter} minne\\ 
 & bî \textbf{sînen zîten} nie vernam.\\ 
30 & d\textit{â} stuont daz minne wol gezam:\\ 
\end{tabular}
\scriptsize
\line(1,0){75} \newline
U V W Q R \newline
\line(1,0){75} \newline
\textbf{1} \textit{Initiale} U V R  \newline
\line(1,0){75} \newline
\textbf{2} Die ich an dem Ringe nach mir sach traben R  $\cdot$ an den rinc] an ring W \textit{om.} Q \textbf{3} zwei] zwey schoͯne R \textbf{4} sô] also Q \textbf{5} gar bewart] Vnd gantz bewert W \textbf{7} eintweder] Antwerte U \textbf{8} er] ers Q  $\cdot$ hêrren] hertzen W \textbf{9} siht] an siht V \textbf{10} des] das W \textbf{12} hât] het V R  $\cdot$ diz] daz V (Q) \textbf{13} dâ] Do U V W Q \textbf{14} daz] den W \textbf{15} kinde] kindes R \textbf{18} si twanc] [*]: Jtonie trvht V \textbf{19} sehet] nv sehent V (Q) (R)  $\cdot$ her] hie Q R \textbf{20} ob] Oder U  $\cdot$ minnen] minne W R meine Q  $\cdot$ kan] [*]: mane V man W (Q) R \textbf{22} daz er dâr an] Dar an er V (W) (Q) (R) \textbf{23} minnen] minne W (Q) R \textbf{24} sîn] seinsz Q \textbf{25} Gramoflanz] Gramaflanz V Gramoflantz W Q Gramoflancz R  $\cdot$ stæte] werde R \textbf{27} sîme] sinnen R \textbf{28} [*]: So Endehafter minne V \textbf{30} dâ] Do U V W Q R  $\cdot$ minne] minnen Q \newline
\end{minipage}
\end{table}
\end{document}
