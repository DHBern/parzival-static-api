\documentclass[8pt,a4paper,notitlepage]{article}
\usepackage{fullpage}
\usepackage{ulem}
\usepackage{xltxtra}
\usepackage{datetime}
\renewcommand{\dateseparator}{.}
\dmyyyydate
\usepackage{fancyhdr}
\usepackage{ifthen}
\pagestyle{fancy}
\fancyhf{}
\renewcommand{\headrulewidth}{0pt}
\fancyfoot[L]{\ifthenelse{\value{page}=1}{\today, \currenttime{} Uhr}{}}
\begin{document}
\begin{table}[ht]
\begin{minipage}[t]{0.5\linewidth}
\small
\begin{center}*D
\end{center}
\begin{tabular}{rl}
\textbf{524} & \begin{large}G\end{large}arzûn oder vilân.\\ 
 & swaz ir \textbf{spottes hât gein mir} getân,\\ 
 & dâ mite ir sünde enpfâhet,\\ 
 & ob ir mîn dienst \textbf{smâhet}.\\ 
5 & Solt ich dienstes geniezen,\\ 
 & i\textit{u}ch m\textit{ö}hte spottes verdriezen.\\ 
 & ob ez mir nimmer würde leit,\\ 
 & ez krenket doch iwer \textbf{werdecheit}."\\ 
 & wider \textbf{zuo zin reit} der wunde man.\\ 
10 & \textbf{dô} sprach \textbf{er}: "bistû\textbf{z}, Gawan?\\ 
 & hâstû iht geborget mir,\\ 
 & daz ist \textbf{nû gar} vergolten dir,\\ 
 & dô mich dîn manlîchiu kraft\\ 
 & vie in herter rîterschaft\\ 
15 & unt \textbf{dô} dû \textbf{bræhte mich} ze hûs\\ 
 & dînem œheim Artus.\\ 
 & vier wochen er des niht vergaz:\\ 
 & die zît ich mit den hunden az."\\ 
 & Dô sprach er: "bistû\textbf{z}, \textbf{Urjans}?\\ 
20 & ob dû mir nû schaden gans,\\ 
 & den trag ich âne schulde;\\ 
 & ich erwarp dir sküneges hulde.\\ 
 & ein swach sin \textbf{half dir und} riet.\\ 
 & \textbf{von schildes ambet man dich} schiet\\ 
25 & unt sagte dich \textbf{gar rehtlôs},\\ 
 & durch daz ein magt von dir verlôs\\ 
 & ir reht, dar zuo des landes vride.\\ 
 & \textbf{der} künec Artus mit einer wide\\ 
 & woltez gerne hân gerochen,\\ 
30 & het ich dich niht versprochen."\\ 
\end{tabular}
\scriptsize
\line(1,0){75} \newline
D Fr11 \newline
\line(1,0){75} \newline
\textbf{1} \textit{Initiale} D  \textbf{5} \textit{Majuskel} D  \textbf{19} \textit{Majuskel} D  \newline
\line(1,0){75} \newline
\textbf{5} dienstes] diens D \textbf{6} iuch möhte] ich mohte D \textbf{16} Artus] Artv̂s D Artuͯ::: Fr11 \textbf{19} Urjans] Vrians D Vr::: Fr11 \textbf{24} von] vnd Fr11 \textbf{25} sagte] sagt Fr11 \textbf{30} ich] \textit{om.} Fr11 \newline
\end{minipage}
\hspace{0.5cm}
\begin{minipage}[t]{0.5\linewidth}
\small
\begin{center}*m
\end{center}
\begin{tabular}{rl}
 & garzûn oder vilâ\textit{n}.\\ 
 & waz ir \textbf{mir spottes habt} getân,\\ 
 & dâ mit ir sünde enpfâhet,\\ 
 & ob ir mîn dienst \textbf{smâhet}.\\ 
5 & solt ich dienstes geniezen,\\ 
 & iuch m\textit{ö}hte spottes verdriezen.\\ 
 & ob ez mir nimmer würde leit,\\ 
 & ez krenket doch iuwer \textbf{wirdicheit}."\\ 
 & \begin{large}W\end{large}ider \textbf{zuo im reit} der wunde man\\ 
10 & \textbf{und} sprach: "bistû Gawan?\\ 
 & hâstû iht geborget mir,\\ 
 & daz ist \textbf{ein teil} vergolten dir,\\ 
 & dô mich dîn manlîchiu kraft\\ 
 & vienc \textit{i}n herter ritterschaft\\ 
15 & und dû \textbf{mich bræhte} zuo hûse\\ 
 & dînem œheim Artuse.\\ 
 & vier wochen er des niht vergaz:\\ 
 & die zît ich mit den hunden az."\\ 
 & dô sprach er: "bistû \textbf{Frians}?\\ 
20 & ob dû mir nû schaden gans,\\ 
 & den trag ich âne schulde;\\ 
 & ich erwarp dirs küniges hulde,\\ 
 & \textbf{dô dir} ein swacher sin riet,\\ 
 & \textbf{daz man dich von schiltes ambet} schiet\\ 
25 & und sagte dich \textbf{der rehte lôs},\\ 
 & durch daz ein maget von dir verlôs\\ 
 & ir reht, dar zuo des landes vride.\\ 
 & \textbf{der} künic Artus mit einer wide\\ 
 & wolt ez gern hân gerochen,\\ 
30 & het ich \textbf{des} dich niht versprochen."\\ 
\end{tabular}
\scriptsize
\line(1,0){75} \newline
m n o \newline
\line(1,0){75} \newline
\textbf{9} \textit{Illustration mit Überschrift:} Also der wunde man wider zuͦ her gawan kam geritten n   $\cdot$ \textit{Großinitiale} n   $\cdot$ \textit{Initiale} m  \newline
\line(1,0){75} \newline
\textbf{1} garzûn] Garczẏm o  $\cdot$ vilân] vilam m (o) \textbf{5} dienstes] dienest n (o) \textbf{6} möhte] mohtte m (o) \textbf{8} ez] Er n \textbf{9} \textit{Die Verse 524.9-30 fehlen} o  \textbf{14} in] ein m \textbf{16} dînem] Dinen n \textbf{23} riet] geriet n \textbf{25} der] gar n \textbf{28} Artus] \textit{om.} n \newline
\end{minipage}
\end{table}
\newpage
\begin{table}[ht]
\begin{minipage}[t]{0.5\linewidth}
\small
\begin{center}*G
\end{center}
\begin{tabular}{rl}
 & \begin{large}G\end{large}arzûn ode vilân.\\ 
 & swaz ir \textbf{spottes habet gein mir} getân,\\ 
 & dâ mit ir sünde enpfâhet,\\ 
 & ob ir mîn dienst \textbf{smâhet}.\\ 
5 & solt ic\textit{h} \textit{d}ienstes geniezen,\\ 
 & iuch m\textit{ö}ht\textit{e} spottes verdriezen.\\ 
 & ob ez mir niemer würde leit,\\ 
 & ez krenket doch iuwer \textbf{werdecheit}."\\ 
 & wide\textit{r} \textbf{zuo zin reit} der wunde man\\ 
10 & \textbf{unde} sprach: "bistû\textbf{z}, Gawan?\\ 
 & hâstû iht geborget mir,\\ 
 & daz ist \textbf{nû gar} vergolten dir,\\ 
 & dô mich dîn manlîchiu kraft\\ 
 & vienc in herter rîterschaft\\ 
15 & unde \textbf{dô} dû \textbf{bræhte mich} ze hûs\\ 
 & dîne\textit{m} œheim Artus.\\ 
 & vier wochen er des niht vergaz:\\ 
 & die zît ich mit den hunden az."\\ 
 & dô sprach er: "bistû\textbf{z}, \textbf{Vrians}?\\ 
20 & ob dû mir nû schaden gans,\\ 
 & den trag ich âne schulde;\\ 
 & ich erwa\textit{r}p dir des küniges hulde.\\ 
 & ein swach sin \textbf{half dir unde} riet.\\ 
 & \textbf{von schiltes ambete man dich} schiet\\ 
25 & unde saget dich \textbf{gar rehtlôs},\\ 
 & durch daz ein maget von dir verlôs\\ 
 & ir reht, dar zuo des landes vride.\\ 
 & \textbf{der} künic Artus mit einer wide\\ 
 & wold ez gerne hân gerochen,\\ 
30 & het ich dich niht versprochen."\\ 
\end{tabular}
\scriptsize
\line(1,0){75} \newline
G I L M Z Fr28 Fr62 \newline
\line(1,0){75} \newline
\textbf{1} \textit{Initiale} G L Z  \textbf{9} \textit{Initiale} Fr62  \textbf{17} \textit{Initiale} I  \newline
\line(1,0){75} \newline
\textbf{1} vilân] Filian L \textbf{2} swaz] Waz L (M)  $\cdot$ spottes habet gein mir] spotes Gein mir habt I (L) (M) gein mir hat spottes Fr62  $\cdot$ getân] tan Z \textbf{3} ir] \textit{om.} L \textbf{4} ir] ev Z (Fr62)  $\cdot$ smâhet] versmahet Z \textbf{5} solt] sol I  $\cdot$ ich dienstes] ih min dienstes G \textbf{6} möhte] mohtes G mochte L M (Z) (Fr62)  $\cdot$ spottes] spottens Z  $\cdot$ verdriezen] erdrieszen L \textbf{7} ob ez mir] Ob ez Z vnd ob mirz ioh Fr62  $\cdot$ würde] wolde M \textbf{8} ez] ih Fr62 \textbf{9} wider] wide G Wder Fr62  $\cdot$ zuo zin] zvͤ in I (Z) zuͯ ým L (M) tzucke Fr62  $\cdot$ wunde] gwunte Fr62 \textbf{11} iht] ich M dan iht Fr62 \textbf{13} dô] Da M Z \textbf{15} dô dû] dy M da du Z  $\cdot$ bræhte mich] mich brehte I (L) mi:: bre:::es Fr28 \textbf{16} dînem] Dinen G  $\cdot$ Artus] Artuͯs L arthuse Fr62 \textbf{19} dô sprach er] Da sprach her M Er sprach Z  $\cdot$ bistûz] bistu L  $\cdot$ Vrians] Vrianz L frians Fr28 urians Fr62 \textbf{20} schaden] schadens M \textbf{21} den] Dy M \textbf{22} erwarp] erwap G  $\cdot$ dir des] driz L \textbf{23} Do dih din boser sin verriet Fr62  $\cdot$ swach] sweche I  $\cdot$ sin half dir] dir sin half I sin dir halff M sin halp Z sin dir daz Fr28  $\cdot$ unde] \textit{om.} Fr28 \textbf{24} ambete] ammocht M \textbf{25} saget] Sagite M (Fr28) (Fr62)  $\cdot$ dich] dir I \textbf{26} von dir] durch dich M  $\cdot$ verlôs] verkoz L \textbf{27} des] ir M \textbf{28} Artus] :rtus Fr28  $\cdot$ mit] an L  $\cdot$ wide] kunges wide I \textbf{29} gerne] \textit{om.} Fr62 \textbf{30} ::t ichz nicht vnder sprochen Fr28 \newline
\end{minipage}
\hspace{0.5cm}
\begin{minipage}[t]{0.5\linewidth}
\small
\begin{center}*T
\end{center}
\begin{tabular}{rl}
 & garzûn oder vilân.\\ 
 & swaz ir \textbf{spottes gegen mir hât} getân,\\ 
 & dâ mit ir sünde enpfâhet,\\ 
 & ob ir mîn dienst \textbf{versmâhet}.\\ 
5 & soltich dienstes geniezen,\\ 
 & iuch m\textit{ö}hte spottes verdrie\textit{ze}n.\\ 
 & ob ez mir niemer würde leit,\\ 
 & ez krenket doch iuwer \textbf{wîpheit}."\\ 
 & \textit{\begin{large}W\end{large}}ider \textbf{reit zim} der wunde man\\ 
10 & \textbf{unde} sprach: "bistû\textbf{z}, Gawan?\\ 
 & hâstû \textbf{nû} iht geborget mir,\\ 
 & daz ist \textbf{nû gar} vergolten dir,\\ 
 & dô mich dîn manlîchiu kraft\\ 
 & vienc in herter rîterschaft\\ 
15 & unde \textbf{dô} dû \textbf{bræhte mich} ze hûs\\ 
 & dînem œheim Artus.\\ 
 & vier wochen er des niht vergaz:\\ 
 & die zît ich mit den hunde\textit{n} az."\\ 
 & Dô sprach er: "bistû\textbf{z}, \textbf{Vryans}?\\ 
20 & ob dû mir nû sch\textit{a}den gans,\\ 
 & den trag ich âne schulde;\\ 
 & ich erwarp dir des küneges hulde.\\ 
 & ein swach sin \textbf{half dir unde} riet.\\ 
 & \textbf{von schiltes ambet man dich} schiet\\ 
25 & unde sagete dich \textbf{gar rehte lôs},\\ 
 & durch daz ein maget von dir verlôs\\ 
 & ir reht, dar zuo des landes vride.\\ 
 & Künec Artus mit einer wide\\ 
 & woltez gerne hân gerochen,\\ 
30 & hettich dich niht versprochen."\\ 
\end{tabular}
\scriptsize
\line(1,0){75} \newline
T U V W O Q R Fr40 \newline
\line(1,0){75} \newline
\textbf{1} \textit{Initiale} O Fr40  \textbf{9} \textit{Initiale} T U V W  \textbf{19} \textit{Majuskel} T  \textbf{28} \textit{Majuskel} T  \newline
\line(1,0){75} \newline
\textbf{1} garzûn] ÷arzvn O \textbf{2} [*]: Swaz ir spottez gegen mir hant getan V  $\cdot$ Was ir mir spotes hand geton R  $\cdot$ swaz] Waz U (W) (Q) \textbf{4} versmâhet] smahet U (V) Fr40 empfachent R \textbf{5} soltich dienstes] Sol ich dinste Fr40 \textbf{6} iuch] iv T  $\cdot$ möhte] mohte T (U) V (O) (Q) Fr40 moͯch R  $\cdot$ spottes] iedoch W O ydoch spottes Q (R) (Fr40)  $\cdot$ verdriezen] verdrien T \textbf{7} niemer] vmmer Q (R) \textbf{8} wîpheit] [*it]: wipheit V \textbf{9} Wider reit zim] ÷ider reit zim T Vider reit zuͦ im U DO widerreit zim V WIder zuͦ im rait W (O) (Q) (R) (Fr40) \textbf{10} unde] Er O  $\cdot$ bistûz] wistu Q  $\cdot$ Gawan] gewain R \textbf{11} nû] \textit{om.} U V W O Q R Fr40 \textbf{12} nû gar] [*]: ein teil V \textbf{13} dô] Da O  $\cdot$ manlîchiu] menschliche Q \textbf{14} in herter] inrehter O (Q) (Fr40) \textbf{15} bræhte mich] mich brechte U (Q) (R) \textbf{16} Artus] artous Fr40 \textbf{17} \textit{Die Verse 524.17-18 fehlen} W  \textbf{18} hunden] hvnde T \textbf{19} bistûz] wistu Q  $\cdot$ Vryans] vrians U W O R Fr40 vrianst V frians Q \textbf{20} mir nû] mir V nun mir R  $\cdot$ schaden] schanden T schadens Q \textbf{21} âne] on alle mein W \textbf{22} dir] úch W \textbf{23} [* s*h]: Do dir ein swacher sin geriet V  $\cdot$ Jch half dir eine vnde riet O  $\cdot$ Er swanck sein helter vnd reit Q  $\cdot$ er sprach ich half dir vnde riet Fr40 \textbf{24} [*]: Daz man dich von schiltez ambt schiet V  $\cdot$ schiltes] schides Q \textbf{25} sagete] sagt O Q  $\cdot$ gar] \textit{om.} U \textbf{27} dar zuo] vnd auch W \newline
\end{minipage}
\end{table}
\end{document}
