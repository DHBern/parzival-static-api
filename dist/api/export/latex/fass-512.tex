\documentclass[8pt,a4paper,notitlepage]{article}
\usepackage{fullpage}
\usepackage{ulem}
\usepackage{xltxtra}
\usepackage{datetime}
\renewcommand{\dateseparator}{.}
\dmyyyydate
\usepackage{fancyhdr}
\usepackage{ifthen}
\pagestyle{fancy}
\fancyhf{}
\renewcommand{\headrulewidth}{0pt}
\fancyfoot[L]{\ifthenelse{\value{page}=1}{\today, \currenttime{} Uhr}{}}
\begin{document}
\begin{table}[ht]
\begin{minipage}[t]{0.5\linewidth}
\small
\begin{center}*D
\end{center}
\begin{tabular}{rl}
\textbf{512} & \begin{large}G\end{large}awan von dem orse spranc.\\ 
 & dô het er manegen \textbf{gedanc},\\ 
 & wie daz ors sîn erbite.\\ 
 & dem brunnen wonte \textbf{ninder} mite,\\ 
5 & \textbf{dâ} erz geheften m\textit{ö}hte.\\ 
 & er dâhte, ob daz t\textit{ö}hte,\\ 
 & daz siz ze behalten næme,\\ 
 & ob im diu bete \textbf{gezæme}.\\ 
 & "Ich sihe wol, wes ir angest hât",\\ 
10 & sprach si. "diz ors mir stên hie lât;\\ 
 & daz behalt ich, unz ir wider kumt.\\ 
 & mîn dienst \textbf{iu} \textbf{doch} \textbf{vil} kleine vrumt."\\ 
 & Dô nam mîn hêr Gawan\\ 
 & den zügel von dem orse \textbf{dan}.\\ 
15 & er sprach: "nû habt mirz, vrouwe."\\ 
 & "bî tumpheit ich iuch schouwe",\\ 
 & sprach si, "wan dâ lac iwer hant,\\ 
 & der grif sol mir sîn unbekant."\\ 
 & Dô sprach der \textbf{minne gerende} man:\\ 
20 & "vrouwe, ine greif nie vorne dran."\\ 
 & "nû, \textbf{dâ} wil ichz enpfâhen",\\ 
 & sprach si, "nû sult ir gâhen,\\ 
 & \textbf{und} bringet \textbf{mir balde} mîn pfert;\\ 
 & mîner reise ir sît mit \textbf{iu} gewert."\\ 
25 & daz dûhte in vreudehaft gewin.\\ 
 & dô gâht er balde \textbf{vor} ir hin\\ 
 & über den stec zer porten în.\\ 
 & dâ sa\textit{ch} er \textbf{manegen} vrouwen schîn\\ 
 & unt manegen rîter jungen,\\ 
30 & die tanzten unde \textbf{sungen}.\\ 
\end{tabular}
\scriptsize
\line(1,0){75} \newline
D \newline
\line(1,0){75} \newline
\textbf{1} \textit{Initiale} D  \textbf{9} \textit{Majuskel} D  \textbf{13} \textit{Majuskel} D  \textbf{19} \textit{Majuskel} D  \newline
\line(1,0){75} \newline
\textbf{5} möhte] mohte D \textbf{6} töhte] tohte D \textbf{28} sach er] sager D \newline
\end{minipage}
\hspace{0.5cm}
\begin{minipage}[t]{0.5\linewidth}
\small
\begin{center}*m
\end{center}
\begin{tabular}{rl}
 & \begin{large}G\end{large}awan von dem rosse spranc.\\ 
 & dô het er manigen \textbf{danc},\\ 
 & wie daz ros sîn \textbf{d\textit{â}} erbit.\\ 
 & dem brunnen wo\textit{n}et \textbf{niendert} mit,\\ 
5 & \textbf{d\textit{â}} erz geheften möhte.\\ 
 & er dâht, ob \textbf{ime} daz t\textit{ö}hte,\\ 
 & daz si ez zuo behalten næme,\\ 
 & ob ime diu bete \textbf{gezæme}.\\ 
 & "ich sihe wol, wes ir angest hât",\\ 
10 & sprach si. "diz ros mir stân hie lât;\\ 
 & daz behalt ich, unz ir wider komt.\\ 
 & mîn dienst \textbf{iuch} \textbf{des} kleine vromt."\\ 
 & dô nam mîn hêr Gawan\\ 
 & den zügel von dem rosse \textbf{dan}.\\ 
15 & er sprach: "nû habt mirz, vrouwe."\\ 
 & "bî tumpheit ich iuch schouwe",\\ 
 & sprach si \textbf{dan}, "wan d\textit{â} lac iuwer hant,\\ 
 & der grif sol mir sîn unbekant."\\ 
 & dô sprach der \textbf{minnebernde} man:\\ 
20 & "vrouwe, ich engr\textit{ei}f nie vorn dâr an."\\ 
 & "nû, \textbf{dâ} wil ichz enpfâhen",\\ 
 & sprach si, "nû solt ir gâhen,\\ 
 & \textbf{und} bringt \textbf{mir balde} mîn pfert;\\ 
 & mîner reise ir sît mit \textbf{iu} gewert."\\ 
25 & daz dûht in vröudenhaft gewin.\\ 
 & dô gâhet er balde \textbf{von} ir hin\\ 
 & über den stec zuor porten în.\\ 
 & dô sach er \textbf{maniger} vrouwen schîn\\ 
 & und manigen ritter jungen,\\ 
30 & die tanz\textit{t}en und \textbf{s\textit{u}ngen}.\\ 
\end{tabular}
\scriptsize
\line(1,0){75} \newline
m n o \newline
\line(1,0){75} \newline
\textbf{1} \textit{Initiale} m   $\cdot$ \textit{Capitulumzeichen} n  \newline
\line(1,0){75} \newline
\textbf{2} danc] gedang n o \textbf{3} sîn dâ] sin do m o do sin do n  $\cdot$ erbit] arbeit o \textbf{4} brunnen] bronden o  $\cdot$ wonet] woret m wondet o  $\cdot$ niendert] nieman o \textbf{5} dâ] Do m n o  $\cdot$ möhte] mochte o \textbf{6} töhte] dohtte m (o) \textbf{10} diz] dasz o  $\cdot$ hie] \textit{om.} n \textbf{11} unz] als ir vncz o \textbf{12} des] doch n o \textbf{13} dô] [Don]: Do o  $\cdot$ hêr] herre her n \textbf{17} dan wan dâ] dan wan do m wenne do n (o) \textbf{19} minnebernde] mynne gerende n (o) \textbf{20} engreif] engrieff m \textbf{21} dâ] do n o \textbf{22} nû] so n ::: o \textbf{25} dûht] dúchte o  $\cdot$ vröudenhaft] freiden halp o \textbf{26} gâhet] gohete n \textbf{30} tanzten] tanczen m (n)  $\cdot$ sungen] singen m s:ngent o \newline
\end{minipage}
\end{table}
\newpage
\begin{table}[ht]
\begin{minipage}[t]{0.5\linewidth}
\small
\begin{center}*G
\end{center}
\begin{tabular}{rl}
 & \begin{large}G\end{large}awan von dem orse spranc.\\ 
 & dô het er manigen \textbf{gedanc},\\ 
 & wie daz ors sî\textit{n} erbite.\\ 
 & dem brunnen wonet \textbf{niemer} mite,\\ 
5 & \textbf{dâ} erz geheften m\textit{ö}hte.\\ 
 & er dâhte, ob \textbf{im} daz t\textit{ö}hte,\\ 
 & daz siz ze behalten næme,\\ 
 & obe im diu bete \textbf{gezæme}.\\ 
 & "ich sihe wol, wes ir angest hât",\\ 
10 & sprach si. "ditze ors mir stên hie lât;\\ 
 & daz behalt ich, unze ir wider komet.\\ 
 & mîn dienst \textbf{iuch} \textbf{doch} \textbf{vil} \textit{kl}e\textit{ine} vromet."\\ 
 & dô nam mîn hêrre Gawan\\ 
 & den zügel von dem ors \textbf{sân}.\\ 
15 & er sprach: "nû habet mirz, vrouwe."\\ 
 & "bî tumpheit ich iu\textit{ch} schouwe",\\ 
 & sprach si, "wan dâ lac iuwer hant,\\ 
 & der grif sol mir sîn unbekant."\\ 
 & dô sprach der \textbf{minne gernde} man:\\ 
20 & "vrouwe, ich engreif nie vorn dran."\\ 
 & "nû, \textbf{dâ} wil ichz enpfâhen",\\ 
 & sprach si, "nû sult ir gâhen,\\ 
 & \textbf{unde} bringet \textbf{balde mir} mîn pfert;\\ 
 & mîner reise ir sît mit \textbf{iu} gewert."\\ 
25 & daz dûhte in vröudehaf\textit{t} gewin.\\ 
 & dô gâht er balde \textbf{von} ir hin\\ 
 & über den stec ze der porten în.\\ 
 & dô sach er \textbf{maniger} vrouwen schîn\\ 
 & unde manigen rîter jungen,\\ 
30 & die tanzeten unde \textbf{sprungen}.\\ 
\end{tabular}
\scriptsize
\line(1,0){75} \newline
G I L M Z \newline
\line(1,0){75} \newline
\textbf{1} \textit{Initiale} G L Z  \textbf{11} \textit{Initiale} I  \textbf{21} \textit{Initiale} M  \newline
\line(1,0){75} \newline
\textbf{2} dô] Da M Z \textbf{3} sîn] siner G \textbf{4} niemer] ninder I (L) (Z) nirgen M \textbf{5} dâ] Das M  $\cdot$ erz] er Z  $\cdot$ möhte] mohte G I (L) (M) Z \textbf{6} dâhte] tochte M  $\cdot$ töhte] dohte G (I) (L) (M) (Z) \textbf{7} ze behalten] zuͯ halten L behaldin M  $\cdot$ næme] nemen M \textbf{8} gezæme] zeme I \textbf{10} ditze] das M  $\cdot$ hie] \textit{om.} M \textbf{11} daz] Da Z  $\cdot$ behalt] behilt M bezal Z  $\cdot$ unze] bisz M \textbf{12} iuch] ev I Z  $\cdot$ vil] \textit{om.} M  $\cdot$ kleine] wenic G \textbf{13} dô] Da M  $\cdot$ hêrre] er M \textbf{14} sân] dan L M Z \textbf{15} er sprach] \textit{om.} I  $\cdot$ mirz] ir mirz I \textbf{16} iuch] iv G \textbf{19} dô] Dar M  $\cdot$ gernde] gernder I \textbf{20} engreif] graif I (M) (Z)  $\cdot$ vorn] vor M \textbf{21} ichz] ich M \textbf{22} sult] svluͯt L \textbf{23} balde mir] \textit{om.} I mir balde L M Z \textbf{24} mit iu] nu mit mir I \textbf{25} vröudehaft] froͮde hafte G fridehafft M \textbf{26} dô] Da M Z  $\cdot$ gâht] ylte M  $\cdot$ von ir hin] vor ir hin I vor yn yn M \textbf{28} dô] da I (M) (Z)  $\cdot$ maniger] mangen I  $\cdot$ vrouwen schîn] vroden lehin L \textbf{30} tanzeten] tanczin M  $\cdot$ sprungen] svngen L (M) \newline
\end{minipage}
\hspace{0.5cm}
\begin{minipage}[t]{0.5\linewidth}
\small
\begin{center}*T
\end{center}
\begin{tabular}{rl}
 & Gawan von dem orse spranc.\\ 
 & dô het er manegen \textbf{gedanc},\\ 
 & wie daz ors sîn erbite.\\ 
 & dem brunnen wonte \textbf{niender} mite,\\ 
5 & \textbf{daz} erz geheften m\textit{ö}hte.\\ 
 & er dâhte, ob \textbf{im} daz t\textit{ö}hte,\\ 
 & daz siz ze behaltenne næme,\\ 
 & ob im di\textit{u} bete \textbf{zæme}.\\ 
 & "Ich sihe wol, wes ir angest hât",\\ 
10 & sprach si. "diz ors mir stân hie lât;\\ 
 & daz behalt ich, unz ir wider komet.\\ 
 & mîn dienst \textbf{iu} \textbf{doch} \textbf{vil} kleine vromet."\\ 
 & Dô nam mîn hêr Gawan\\ 
 & den zügel von dem orse \textbf{dan}.\\ 
15 & er sprach: "nû habt mirz, vrouwe."\\ 
 & "bî tumpheit ich iuch schouwe",\\ 
 & sprach si, "wan dâ lac iuwer hant,\\ 
 & der grif sol mir sîn unbekant."\\ 
 & Dô sprach der \textbf{minne gernde} man:\\ 
20 & "vrouwe, ine greif nie vorne dran."\\ 
 & "Nû wil ichz enpfâhen",\\ 
 & sprach si, "nû sult ir gâhen.\\ 
 & \textbf{nû} bringet \textbf{mir balde} mîn pfert;\\ 
 & mîner reise ir sît mit \textbf{mir} gewert."\\ 
25 & Daz dûht in vröudehaft gewin.\\ 
 & dô gâhet er balde \textbf{von} ir hin\\ 
 & übern stec zer porten în.\\ 
 & dâ sach er \textbf{maneger} vrouwen schîn\\ 
 & unde manegen rîter jungen,\\ 
30 & die tanzten unde \textbf{sungen}.\\ 
\end{tabular}
\scriptsize
\line(1,0){75} \newline
T U V W O Q R Fr40 \newline
\line(1,0){75} \newline
\textbf{1} \textit{Initiale} W O R Fr40  \textbf{9} \textit{Majuskel} T  \textbf{13} \textit{Majuskel} T  \textbf{19} \textit{Majuskel} T  \textbf{21} \textit{Majuskel} T  \textbf{25} \textit{Majuskel} T  \newline
\line(1,0){75} \newline
\textbf{1} Gawan] ÷awan O Gawain R  $\cdot$ dem] den R \textbf{2} het] her W \textbf{3} sîn] sin [*]: do V \textbf{4} niender] nider U nyeman R \textbf{5} daz] Da O (Q) (Fr40) Daran R  $\cdot$ geheften] gehelfen U an gehften O beheften Q  $\cdot$ möhte] mohte T (U) O (Q) (R) Fr40 [*]: mohte V \textbf{6} er dâhte] Er gedacht W Q R (Fr40) Da gedaht er O  $\cdot$ töhte] tohte T (U) (V) O (Q) Fr40 gedochte R \textbf{7} ze] \textit{om.} R  $\cdot$ behaltenne] halten W \textbf{8} diu] die T  $\cdot$ zæme] gezeme W (O) (R) \textbf{9} wol] wol wol O \textbf{10} sprach] Sparch V  $\cdot$ stân hie] stende U sten [*]: hie V \textit{om.} O hie sten R \textbf{11} behalt] beczalt R  $\cdot$ unz] mit U \textbf{12} iu] euch Fr40  $\cdot$ doch] \textit{om.} W  $\cdot$ vil] [wil]: vil Q \textit{om.} R  $\cdot$ kleine] wening V (O) \textbf{13} nam] man U  $\cdot$ Gawan] Gawain R \textbf{14} orse] Rose sin R \textbf{15} mirz] irz Fr40 \textbf{16} iuch] îv T \textbf{17} wan] \textit{om.} O  $\cdot$ dâ lac] daz latt R  $\cdot$ iuwer] úber W \textbf{18} sol] so W \textbf{20} ine] ich R Fr40 \textbf{21} \textit{Versdoppelung 513.21-28 nach 512.21} O   $\cdot$ Nû] Nu do U (V) (W) (O) (Q) (Fr40) \textbf{22} nû sult ir] ir súlt nun R \textbf{23} nû] Vnd U (V) (W) (O) Q R (Fr40)  $\cdot$ mir balde] balde mir V O \textbf{24} mir] vch U (V) (W) (O) (Q) ew:: Fr40 \textbf{26} balde von ir] von ir balde O bade von ir R \textbf{27} übern] [V́*]: V́ber den V Vber W \textbf{30} unde] vnd auch W  $\cdot$ sungen] sprungen R \newline
\end{minipage}
\end{table}
\end{document}
