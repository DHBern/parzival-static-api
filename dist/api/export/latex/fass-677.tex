\documentclass[8pt,a4paper,notitlepage]{article}
\usepackage{fullpage}
\usepackage{ulem}
\usepackage{xltxtra}
\usepackage{datetime}
\renewcommand{\dateseparator}{.}
\dmyyyydate
\usepackage{fancyhdr}
\usepackage{ifthen}
\pagestyle{fancy}
\fancyhf{}
\renewcommand{\headrulewidth}{0pt}
\fancyfoot[L]{\ifthenelse{\value{page}=1}{\today, \currenttime{} Uhr}{}}
\begin{document}
\begin{table}[ht]
\begin{minipage}[t]{0.5\linewidth}
\small
\begin{center}*D
\end{center}
\begin{tabular}{rl}
\textbf{677} & \begin{large}A\end{large}rtus, der \textbf{prîses} erkante,\\ 
 & \textbf{sîne} boten sante\\ 
 & ze Rosche Sabbins in die stat.\\ 
 & den künec Gramoflanz er bat:\\ 
5 & "sît daz unwendec \textbf{nû} sol sîn,\\ 
 & daz er gein dem neven mîn\\ 
 & sînen kampf niht wil \textbf{verbern},\\ 
 & des sol \textbf{in} mîn neve wern.\\ 
 & bit in gein uns schiere komen,\\ 
10 & sît sîn gewalt ist sus vernomen,\\ 
 & daz erz niht vermîden wil.\\ 
 & es wære einem andern man ze vil."\\ 
 & \textbf{Artuses boten vuoren} dan.\\ 
 & dô nam mîn hêr Gawan\\ 
15 & Lischoysen unt Floranden.\\ 
 & die von manegen landen\\ 
 & minnen soldiere\\ 
 & bat \textbf{er im} zeigen schiere,\\ 
 & die der herzogîn ûf hôhen solt\\ 
20 & wâren sô dienstlîche holt.\\ 
 & Er reit zuo \textbf{z}in unt enpfienc si sô,\\ 
 & daz \textbf{si al gelîche sprâchen} dô,\\ 
 & daz der werde Gawan\\ 
 & wære ein manlîch, \textbf{höfsch} man.\\ 
25 & Dâ mite \textbf{kêrt} er von in wider.\\ 
 & sus warb er \textbf{tougenlîche} sider:\\ 
 & in sîne kameren er gienc,\\ 
 & mit harnasche er \textbf{übervienc}\\ 
 & \textbf{den} lîp zen selben stunden\\ 
30 & durch daz, ob sîne wunden\\ 
\end{tabular}
\scriptsize
\line(1,0){75} \newline
D \newline
\line(1,0){75} \newline
\textbf{1} \textit{Initiale} D  \textbf{21} \textit{Majuskel} D  \textbf{25} \textit{Majuskel} D  \newline
\line(1,0){75} \newline
\textbf{3} Rosche Sabbins] Rosce Sabins D \textbf{13} Artuses] Artvss D \textbf{15} Lischoysen] Liscoysen D \newline
\end{minipage}
\hspace{0.5cm}
\begin{minipage}[t]{0.5\linewidth}
\small
\begin{center}*m
\end{center}
\begin{tabular}{rl}
 & Artus, der \textbf{prîs} erkante,\\ 
 & \textbf{sîne} boten \textbf{er dô} sante\\ 
 & zuo Ros\textit{ch}e Sabins in die stat.\\ 
 & den künic Gramolanz er bat:\\ 
5 & "sît daz unwendic sol sîn,\\ 
 & daz er gegen dem neven mîn\\ 
 & sînen kampf niht wil \textbf{verbern},\\ 
 & des sol \textbf{im} mîn neve wern.\\ 
 & bit in gegen uns schier komen,\\ 
10 & sît sîn gewalt ist sus vernomen,\\ 
 & daz erz niht vermîden wil.\\ 
 & es wær einem andern man zuo vil."\\ 
 & \textbf{hie mit vuoren die boten} dan.\\ 
 & dô nam mîn hêr Gawan\\ 
15 & Lischoisen und Floranden.\\ 
 & die von manigen landen\\ 
 & min\textit{n}e\textit{n} soldiere\\ 
 & bat \textbf{er im} zeigen schiere,\\ 
 & die der herzogîn ûf hôhen solt\\ 
20 & wâren sô dienstlîchen holt.\\ 
 & er reit zuo in und enpfienc si sô,\\ 
 & daz \textbf{glîch sprâchen all\textit{e}} \textit{d}ô,\\ 
 & daz der werde Gawan\\ 
 & wære ein manlîch, \textbf{hövesch} man.\\ 
25 & dâ mit \textbf{kêrte} er von in wider.\\ 
 & sus warp er \textbf{tugentlîche\textit{n}} sider:\\ 
 & in sîn kamern er gienc,\\ 
 & mit harnasch er \textbf{übervienc}\\ 
 & \textbf{den} lîp zuo den selben stunden\\ 
30 & durch daz, ob sîn wunden\\ 
\end{tabular}
\scriptsize
\line(1,0){75} \newline
m n o Fr69 \newline
\line(1,0){75} \newline
\newline
\line(1,0){75} \newline
\textbf{1} Artus] Artuͯs o \textbf{2} sîne] Sinen o \textbf{3} Rosche Sabins] rosse Sabbins m n o \textbf{4} Gramolanz] gramolantz m gramonlantz n gramolancz o  $\cdot$ bat] do bat o \textbf{5} sol] nuͯ sol n ẏm sol o \textbf{12} einem] eynnen o \textbf{14} hêr] herre her n \textbf{15} Lischoisen] Liscoisen m n o \textbf{17} minnen] Minem m (o) \textbf{18} bat] Bot n  $\cdot$ im zeigen] zoigen o \textbf{22} glîch] sú glich n (o) algeliche Fr69  $\cdot$ sprâchen] sprach o  $\cdot$ alle dô] alle fro do m do Fr69 \textbf{23} der] \textit{om.} o \textbf{24} hövesch man] hofesman Fr69 \textbf{25} von in] in do o \textbf{26} tugentlîchen] tugentlicher m \textbf{28} harnasch] barnesch o \newline
\end{minipage}
\end{table}
\newpage
\begin{table}[ht]
\begin{minipage}[t]{0.5\linewidth}
\small
\begin{center}*G
\end{center}
\begin{tabular}{rl}
 & \begin{large}A\end{large}rtus, der \textbf{brîs} erkande,\\ 
 & \textbf{sînen} boten sande\\ 
 & ze Roisabins in die stat.\\ 
 & den künic Gramoflanz er bat:\\ 
5 & "sît daz unwendic \textbf{nû} sol sîn,\\ 
 & daz er gein dem neven mîn\\ 
 & sînen kampf niht wil \textbf{verbern},\\ 
 & des sol \textbf{ouch in} mîn neve wern.\\ 
 & bit in gein uns schiere komen,\\ 
10 & sît sîn gewalt ist sus vernomen,\\ 
 & daz erz niht vermîden wil.\\ 
 & es wære einem andern man ze vil."\\ 
 & \textbf{Artuses boten vuoren} dan.\\ 
 & dô nam mîn hêr Gawan\\ 
15 & Lishoisen unde Floranden,\\ 
 & die von manigen landen\\ 
 & minnen soldiere,\\ 
 & bat \textbf{in} zeigen schiere,\\ 
 & die der herzoginne ûf hôhen solt\\ 
20 & wâren sô dienstlîchen holt.\\ 
 & er reit zuo in unde enpfie si sô,\\ 
 & daz \textbf{si alglîche sprâchen} dô,\\ 
 & daz der werde Gawan\\ 
 & wære ein manlîch man.\\ 
25 & dâ mit \textbf{kumt} er von in wider.\\ 
 & sus warp er \textbf{tougenlîchen} sider:\\ 
 & in sîne kamer er gienc,\\ 
 & mit harnasche er \textbf{umbevienc}\\ 
 & \textbf{sînen} lîp zen selben stunden\\ 
30 & durch daz, op sîne wunden\\ 
\end{tabular}
\scriptsize
\line(1,0){75} \newline
G I L M Z Fr18 Fr22 Fr24 Fr61 \newline
\line(1,0){75} \newline
\textbf{1} \textit{Initiale} G I L Z Fr24  \textbf{19} \textit{Initiale} I  \newline
\line(1,0){75} \newline
\textbf{1} Artus] ARt:: Fr24 Artaus Fr61 \textbf{2} boten] luten er I \textbf{3} Roisabins] Roys sabins I Roy sabinsz L rois sabins M :::v sabins Fr24 Roẏ Sabins Fr61 \textbf{4} Gramoflanz] Gramoflanzen I Gramorflanzen M Gramovlantz Fr61 \textbf{5} sît] \textit{om.} I  $\cdot$ daz] iz Fr61  $\cdot$ nû] \textit{om.} Fr61 \textbf{7} sînen kampf] sins chanphes I  $\cdot$ verbern] enbern I irbern M \textbf{8} des] Das M  $\cdot$ ouch in] in oͯch L (M) auch Fr22  $\cdot$ neve] oͤhaim Fr61  $\cdot$ wern] inwern Fr22 \textbf{9} in gein uns schiere] in schire gein vnz L gein vns in schiere Z \textbf{11} vermîden] meẏden Fr61 \textbf{12} andern] andrem I \textbf{13} Artuses] Artus G M Z (Fr22) (Fr24) Artauses Fr61  $\cdot$ boten] bat en M \textbf{14} dô] Da M  $\cdot$ Gawan] ::: Fr18 \textbf{15} Lishoisen] Lishoẏsen G Liscoysen I Lýtschoýsen L Lisoisen M Lẏshoẏsen Fr18 Liscoisin Fr22 Lyshoisen Fr24 Liscoẏsen Fr61  $\cdot$ Floranden] florianden I florandin Fr22 \textbf{17} minnen] Minne L Warn minne Fr61 \textbf{18} bat in] Bat er im Z Fr61 \textbf{20} sô] \textit{om.} L Fr61 \textbf{21} in] yme M \textbf{22} si] \textit{om.} Fr18 Fr24  $\cdot$ alglîche] alle glich M (Z) (Fr18) (Fr61)  $\cdot$ dô] da M \textbf{23} der werde] werden Z \textbf{24} wære] Were benamen L  $\cdot$ manlîch man] hovisch man L menlich hofeman M menlich hofsch man Z (Fr18) mændleich gefueger man Fr61 \textbf{25} kumt] kert Z chom Fr61  $\cdot$ in] im Fr61 \textbf{27} kamer] kammern M  $\cdot$ er] er do I do er Fr61 \textbf{28} umbevienc] vber vienc Z \textbf{29} selben] \textit{om.} L Fr61 \newline
\end{minipage}
\hspace{0.5cm}
\begin{minipage}[t]{0.5\linewidth}
\small
\begin{center}*T
\end{center}
\begin{tabular}{rl}
 & Artus, der \textbf{prîs} erkante,\\ 
 & \textbf{sîne} boten sante\\ 
 & zuo Roitschesabins in die stat.\\ 
 & den künic Gramoflanz er bat:\\ 
5 & "sît daz unwendic \textbf{nû} sol sîn,\\ 
 & daz er gên dem neven mîn\\ 
 & sînen kampf niht wil \textbf{enbern},\\ 
 & des sol \textbf{in ouch} mîn neve wern.\\ 
 & bit in gên uns schiere komen,\\ 
10 & sît sîn gewalt ist sô vernomen,\\ 
 & daz erz niht vermîden wil.\\ 
 & es wære einem andern manne \textbf{gar} zuo vil."\\ 
 & \textbf{Artuses boten vuoren} dan.\\ 
 & dô nam mîn hêr Gawan\\ 
15 & Lyschoysen und Floranden.\\ 
 & die von manegen landen\\ 
 & mi\textit{n}nen soldiere\\ 
 & bat \textbf{er im} zeigen schiere,\\ 
 & die der herzoginne ûf hôhen solt\\ 
20 & wâren sô dienstlîchen holt.\\ 
 & er reit zuo i\textit{n} und enpfienc si \textit{s}ô,\\ 
 & daz \textbf{alle glîche sprâchen} \textit{d}ô,\\ 
 & daz der werde Gawan\\ 
 & wær ein menlîch, \textbf{höfsch} man.\\ 
25 & dâ mit \textbf{kêrt} er von in wider.\\ 
 & sus warp er \textbf{tougenlîchen} sider:\\ 
 & in sîne kamer er gienc,\\ 
 & mit harnasch er \textbf{umbevienc}\\ 
 & \textbf{sînen} lîp \textit{z}en selben stunden\\ 
30 & durch daz, ob sîn wunden\\ 
\end{tabular}
\scriptsize
\line(1,0){75} \newline
Q R W V \newline
\line(1,0){75} \newline
\textbf{1} \textit{Initiale} Q R W V  \newline
\line(1,0){75} \newline
\textbf{1} Artus] [Artus]: ARtus R KVnig artus W \textbf{2} sîne] Einen W Sinen V  $\cdot$ sante] er sante R W er do sande V \textbf{3} Roitschesabins] roirschesabins Q roittschesabins W Roitsche [*abins]: sabins V \textbf{4} Gramoflanz] gramoflantz Q Gramoflancz R gramoflantzen W gramaflanzen V \textbf{7} enbern] verbern R (W) (V) \textbf{8} in ouch] och in R (W) (V) \textbf{9} uns] im W \textbf{10} sô] sust R (W) (V) \textbf{12} andern] andrē R  $\cdot$ gar] \textit{om.} R W V \textbf{13} Artuses] artus Q (R) Kúnig artus W [Hie *]: Artuses V \textbf{14} mîn] meim W \textbf{15} Lyschoysen] Lischoysen Q Lyschoisen R Lyshoien W Lischoien V \textbf{17} minnen] Meinen Q W [*]: Minnen V \textbf{18} im] \textit{om.} W \textbf{19} herzoginne] herczoginen R \textbf{21} in] im Q zin V  $\cdot$ sô] do Q \textbf{22} daz alle] Das sy alle R Do sy alle W [*]: Daz salle V  $\cdot$ glîche] gleichen Q  $\cdot$ dô] so Q \textbf{24} höfsch] hofflich R \textbf{25} Do [*]: mitte kerte er von in wider V \textbf{26} sus] Als Q \textbf{27} er] er do V \textbf{28} umbevienc] úberfieng R \textbf{29} zen] den Q \newline
\end{minipage}
\end{table}
\end{document}
