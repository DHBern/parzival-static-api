\documentclass[8pt,a4paper,notitlepage]{article}
\usepackage{fullpage}
\usepackage{ulem}
\usepackage{xltxtra}
\usepackage{datetime}
\renewcommand{\dateseparator}{.}
\dmyyyydate
\usepackage{fancyhdr}
\usepackage{ifthen}
\pagestyle{fancy}
\fancyhf{}
\renewcommand{\headrulewidth}{0pt}
\fancyfoot[L]{\ifthenelse{\value{page}=1}{\today, \currenttime{} Uhr}{}}
\begin{document}
\begin{table}[ht]
\begin{minipage}[t]{0.5\linewidth}
\small
\begin{center}*D
\end{center}
\begin{tabular}{rl}
\textbf{214} & daz ellenthafter manheit\\ 
 & \textbf{erberme solte} sîn bereit.\\ 
 & sus volgete er dem râte nâch.\\ 
 & hin ze Clamide er sprach:\\ 
5 & "I\textbf{ne} wil dich niht erlâzen,\\ 
 & \textbf{ir vater}, Liazen,\\ 
 & \textbf{dûne bringest} \textbf{im} dîne sicherheit."\\ 
 & "Neinâ, hêrre, \textbf{dem hân ich} herzeleit\\ 
 & getân. ich sluog im sînen sun.\\ 
10 & dû\textbf{ne} solt \textbf{alsô mit mir niht} tuon.\\ 
 & durch Condwiramurs\\ 
 & vaht ouch mit mir Schenteflurs.\\ 
 & ouch wære ich tôt von \textbf{sîner hant},\\ 
 & wan daz mir half mîn scheneschalt.\\ 
15 & \textbf{In} sande in\textbf{z} lant ze Brobarz\\ 
 & Gurnemanz \textbf{de} Graharz\\ 
 & mit werdeclîcher \textbf{heres kraft}.\\ 
 & \textbf{dâ} tâten guote ritterschaft\\ 
 & \textbf{niun} hundert \textbf{ritter}, die \textbf{wol} \textbf{striten}\\ 
20 & - gewâpent ors \textbf{die} alle riten -,\\ 
 & unt \textbf{zwelf} hundert scharjant.\\ 
 & gewâpent ich si in strîte vant.\\ 
 & \textbf{\textit{\begin{large}D\end{large}}en} gebrast niht wan der schilte.\\ 
 & sînes hers mich bevilte.\\ 
25 & ir kom \textbf{ouch} kûme der \textbf{sâme} wider.\\ 
 & mêr helde verlôs ich sider.\\ 
 & nû darbe ich vreude unt êre.\\ 
 & wes gerstû \textbf{von} mir mêre?"\\ 
 & "Ich wil senften dînen vreisen.\\ 
30 & var gein \textbf{den} \textbf{Berteneisen}\\ 
\end{tabular}
\scriptsize
\line(1,0){75} \newline
D \newline
\line(1,0){75} \newline
\textbf{5} \textit{Majuskel} D  \textbf{8} \textit{Majuskel} D  \textbf{15} \textit{Majuskel} D  \textbf{23} \textit{Initiale} D  \textbf{29} \textit{Majuskel} D  \newline
\line(1,0){75} \newline
\textbf{4} Clamide] Chlamide D \textbf{11} Condwiramurs] condwieren amvrs D \textbf{12} Schenteflurs] Scenteflvrs D \textbf{23} den] ÷en D \textbf{30} Berteneisen] Bertteneisen D \newline
\end{minipage}
\hspace{0.5cm}
\begin{minipage}[t]{0.5\linewidth}
\small
\begin{center}*m
\end{center}
\begin{tabular}{rl}
 & daz ellenthafter manheit\\ 
 & \textbf{erberme solte} sîn bereit.\\ 
 & sus volgete er dem râte nâch.\\ 
 & hin ze Clamide er sprach:\\ 
5 & "\textit{in}e wil dich \textit{niht} erlâzen,\\ 
 & \textbf{d\textit{û} enbringest} Liazen\\ 
 & \textbf{vater} dîne sicherheit."\\ 
 & "neinâ, hêrre, \textbf{dem hân ich} herzeleit\\ 
 & getân. ich sluoc ime sînen sun.\\ 
10 & dû \textbf{en}solt \textbf{alsô \textit{mi}t mir niht} tuon.\\ 
 & durch C\textit{o}ndwieramurs\\ 
 & vaht ouch mit mir Schenteflurs.\\ 
 & ouch wære ich tôt von \textbf{sîner hant},\\ 
 & wand daz mir half mîn schinschant.\\ 
15 & \textbf{in} sante in\textbf{z} lant ze Br\textit{o}barz\\ 
 & Gurnemanz \textbf{de} Graharz\\ 
 & mit werdeclîcher \textbf{hers kraft}.\\ 
 & \textbf{die} tâten guote ritterschaft,\\ 
 & \textbf{n\textit{i}un} h\textit{und}e\textit{r}t, die \textbf{wo\textit{l}} \textbf{striten}\\ 
20 & - gewâpent ros \textbf{die} alle riten -,\\ 
 & und \textbf{vünfzehen} hundert sarjant.\\ 
 & gewâpent ich si in strîte vant.\\ 
 & \textbf{den} gebrast niwan der schilte.\\ 
 & sînes hers mich bevilte.\\ 
25 & ir kam \textbf{doch} kûme der \dag sume\dag  wider.\\ 
 & \textit{m}êr helde vlôs ich sider.\\ 
 & nû darbe ich vröude und êre.\\ 
 & wes gerstû \textbf{von} mir mêre?"\\ 
 & "ich wil senften dînen vreisen.\\ 
30 & var gegen \textbf{den} \textbf{Br\textit{i}tuneisen}\\ 
\end{tabular}
\scriptsize
\line(1,0){75} \newline
m n o Fr69 \newline
\line(1,0){75} \newline
\newline
\line(1,0){75} \newline
\textbf{2} erberme] Erbermde n o Fr69 \textbf{3} volgete] volget n (o) (Fr69)  $\cdot$ nâch] [nah]: nach Fr69 \textbf{4} hin] \textit{om.} n  $\cdot$ Clamide] klamide o \textbf{5} ine] Me m Jch n o  $\cdot$ niht] \textit{om.} m \textbf{6} dû] Do m  $\cdot$ enbringest] bringest denne n (o)  $\cdot$ Liazen] lyoszen n \textbf{8} neinâ] Nein n \textbf{9} sluoc] ersluͦg n \textbf{10} ensolt] sold Fr69  $\cdot$ mit mir] niht [tuͯn]: mir m \textbf{11} Condwieramurs] kndwier amurs m kunduwúr amurs n kuͯndwir amuers o \textbf{12} vaht] Vohent n  $\cdot$ Schenteflurs] Scenteflurs m scenteflursz n [scenteflus]: scenteflurs o \textbf{14} mîn] sin n o  $\cdot$ schinschant] scunscant m n scuntscant o schiniscant Fr69 \textbf{15} Brobarz] brabarz m brobartz n brobarcz o brobars Fr69 \textbf{16} Gurnemanz] Gurnemancz m Gurnemantz n Gurnemans o  $\cdot$ de] der n o  $\cdot$ Graharz] graharcz m grahartz n [*]: graharcz o \textbf{18} die] Der n o \textbf{19} niun hundert] Nun hoͯrent m Nẏm hundert o  $\cdot$ wol] wop m \textbf{20} die] sú n (o) \textbf{23} gebrast] brast n  $\cdot$ der schilte] den schilt n (o) \textbf{24} hers] hercz o  $\cdot$ bevilte] befilt n (o) \textbf{25} sume] sumer o \textbf{26} mêr] Nier m Vier n o  $\cdot$ helde] hielde o  $\cdot$ vlôs] slosz n [s*]: slosz o \textbf{29} wil] vil o \textbf{30} Brituneisen] bruttu neisen m britaneisen n o \newline
\end{minipage}
\end{table}
\newpage
\begin{table}[ht]
\begin{minipage}[t]{0.5\linewidth}
\small
\begin{center}*G
\end{center}
\begin{tabular}{rl}
 & \begin{large}D\end{large}az ellenthafter manheit\\ 
 & \textbf{erbermde solte} sîn bereit.\\ 
 & sus volget er dem râte nâch.\\ 
 & hin ze Clamide er sprach:\\ 
5 & "ich\textbf{ne} wil dich niht erlâzen,\\ 
 & \textbf{ir vater}, Liazen,\\ 
 & \textbf{dûne bringest} \textbf{im} dîne sicherheit."\\ 
 & "nein, hêrre, \textbf{ich hân im} herzeleit\\ 
 & getân. ich sluog im sînen sun.\\ 
10 & d\textit{û} solt \textbf{alsô mit mir niht} tuon.\\ 
 & durch Condwiramurs\\ 
 & vaht ouch mit mir Tschentaflurs.\\ 
 & ouch wær ich tôt von \textbf{sîner hant},\\ 
 & wan daz mir half mîn schinschalt.\\ 
15 & \textbf{ich} sande in\textbf{z} lant ze Briubarz\\ 
 & Gurnoman\textit{z} \textbf{von} Graharz\\ 
 & mit werdeclî\textit{ch}er \textbf{herschaft}.\\ 
 & \textbf{die} tâten guote rîterschaft,\\ 
 & \textbf{niun} hundert \textbf{rîter}, die \textbf{striten}\\ 
20 & - gewâpent ors \textbf{si} alle riten -,\\ 
 & unde \textbf{vünfzehen} hundert sarjant.\\ 
 & gewâpent ich si in strîte vant.\\ 
 & \textbf{in} gebrast niwan der schilte.\\ 
 & sînes hers mich bevilte.\\ 
25 & ir kom \textit{\textbf{doch}} kûme der \textbf{sâme} wider.\\ 
 & mêr helde verlôs ich sider.\\ 
 & nû darbe ich vröude unde êre.\\ 
 & wes gerstû \textbf{von} mir mêre?"\\ 
 & "ich wil senften dîne vreis\textit{e}.\\ 
30 & var gein \textit{\textbf{dem}} \textbf{Britaneis\textit{e}}\\ 
\end{tabular}
\scriptsize
\line(1,0){75} \newline
G I O L M Q R Z Fr21 \newline
\line(1,0){75} \newline
\textbf{1} \textit{Initiale} G  \textbf{3} \textit{Initiale} Q  \textbf{19} \textit{Initiale} I  \textbf{23} \textit{Initiale} Z  \textbf{29} \textit{Initiale} O L Q  \newline
\line(1,0){75} \newline
\textbf{1} ellenthafter] ellenthaftiv O erenthaffter Q \textbf{2} erbermde] Zer barmde O Er hemde M \textbf{3} volget] volg O volgete L (M) (Q) \textbf{4} hin] Hie R  $\cdot$ Clamide] Glamide O \textbf{5} ichne] ich I (O) (L) (M) (R) \textbf{6} Duͯ bringest Lyazen L  $\cdot$ Liazen] Lyazzen O liassen Q lyaczen R liazzen Z \textbf{7} dûne bringest im] Dv bringest im O (R) Vatter L  $\cdot$ dîne] \textit{om.} I Q dy M \textbf{8} nein] neina I  $\cdot$ hân] >han< G  $\cdot$ herzeleit] herzenleit O (Z) \textbf{10} dû] dune G  $\cdot$ alsô mit mir niht] mit mir niht also O mit mir so niht L \textbf{11} Condwiramurs] konduwiramurs I kvndwirn amvrs O Condwir amuͯrs L kuntwiramuͯrs \textit{nachträglich korrigiert zu:} kuntwir amuͯrs M kundwiramurs Q kondamuͦres R Gvndewiramvrs Z \textbf{12} Tschentaflurs] schentaflurs I (O) Jentafluͯrs L schentaflur M Jentalfur Q Schentafluͦrs R scentaflvrs Z \textbf{13} sîner hant] ým gevalt L \textbf{14} mîn] \textit{om.} M  $\cdot$ schinschalt] Thsenescalt I schenechant O sinetshalt L sineschalt M senetscant Q ellenthand R smetschalant Z \textbf{15} ich] Jn O L (M) Q R Z  $\cdot$ Briubarz] briafarz I Brvbarz O (M) Z Brvbars L (Q) Bruͯbarz R \textbf{16} Gurnomanz] [kur*]: kurnomanze G Gurnemanz I (O) Gvrnomantz L Gurnemancz M Gúrnomantz Q Gurnamancz R Gvrnemantz Z  $\cdot$ Graharz] grahars Q graharcz R \textbf{17} werdeclîcher] werdechlier G werlicher I werdenlicher R  $\cdot$ herschaft] hers craft I (O) (L) (R) (Z) \textbf{18} guote] guten M \textbf{19} striten] wol striten O (L) (M) (Q) (R) Z \textbf{20} si alle] alle si O \textbf{21} vünfzehen hundert] wol Tusent I \textbf{22} gewâpent] Gewaupttes R  $\cdot$ si] \textit{om.} O sie alle L \textbf{23} niwan] nicht dann Q (Z)  $\cdot$ der] \textit{om.} O \textbf{24} sînes] Jr Q \textbf{25} ir] Jch Q  $\cdot$ doch] vil G \textbf{26} mêr] Mit M \textbf{27} darbe] darff M mangel R \textbf{29} ich] ÷ch O  $\cdot$ senften] dir senfftte R  $\cdot$ dîne] dinen Z  $\cdot$ vreise] freisen G (M) Z Reise R \textbf{30} var] Nu var M Far hin R  $\cdot$ dem] \textit{om.} G den M Z [Der]: dem Fr21  $\cdot$ Britaneise] pritaneisen G pritoneise I [Brttoneisze]: Brittoneisze L britoneise Q (R) brituneisen Z \newline
\end{minipage}
\hspace{0.5cm}
\begin{minipage}[t]{0.5\linewidth}
\small
\begin{center}*T
\end{center}
\begin{tabular}{rl}
 & daz ellenthafter manheit\\ 
 & \textbf{solte erbermede} sîn bereit.\\ 
 & sus volget er dem râte nâch.\\ 
 & hin ze Clamide er sprach:\\ 
5 & "i\textbf{ne} wil dich niht erlâzen,\\ 
 & \textbf{dem vater} Lyazen\\ 
 & \textbf{dûne bringest} dîne sicherheit."\\ 
 & "Nein, hêrre, \textbf{dem hân ich} herzeleit\\ 
 & getân. ich sluoc im sînen sun.\\ 
10 & dû\textbf{ne} solt \textbf{mit mir niht alsô} tuon.\\ 
 & Durch \textbf{vroun} Kundewiramurs\\ 
 & vaht ouch mit mir Schentaflurs.\\ 
 & ouch wære ich tôt von \textbf{im gevalt},\\ 
 & wan daz mir half mîn seneschalt.\\ 
15 & \textbf{in} sante in \textbf{diz} lant ze Breharz\\ 
 & Gurnemanz \textbf{von} Graharz\\ 
 & mit werdeclîcher \textbf{herschaft}.\\ 
 & \textbf{die} tâten guote rîterschaft,\\ 
 & \textbf{vünf} hundert \textbf{rîter}, die \textbf{wolten} \textbf{strîten}\\ 
20 & - gewâpente ors \textbf{si} alle riten -,\\ 
 & unde \textbf{vünfzehen} hundert sarjant.\\ 
 & gewâpent ich si in strîte vant.\\ 
 & \textbf{in} gebrast niht wan der schilte.\\ 
 & sînes hers mich \textbf{gar} bevilte.\\ 
25 & ir kom \textbf{vil} kûme der \textbf{sehste} wider.\\ 
 & mêr helde verlôs ich sider.\\ 
 & nû darbich vröude unde êre.\\ 
 & wes gerstû \textbf{an} mir mêre?"\\ 
 & "Ich wil senften dîne vreise.\\ 
30 & var gegen \textbf{dem} \textbf{Brituneise}\\ 
\end{tabular}
\scriptsize
\line(1,0){75} \newline
T U V W \newline
\line(1,0){75} \newline
\textbf{8} \textit{Majuskel} T  \textbf{11} \textit{Majuskel} T  \textbf{29} \textit{Initiale} U W   $\cdot$ \textit{Majuskel} T  \newline
\line(1,0){75} \newline
\textbf{1} daz ellenthafter] Der ellenthafte U  $\cdot$ manheit] hand W \textbf{2} Erbermde sol sein bekant W \textbf{4} Clamide] klamide W  $\cdot$ sprach] do sprach W \textbf{5} ine] Ich W \textbf{6} dem] Der V  $\cdot$ Lyazen] liazen V lyassen W \textbf{7} dûne] Dv V (W)  $\cdot$ dîne] im dan dein W \textbf{8} herzeleit] leit V \textbf{9} ich sluoc im] dem schluͦg ich W \textbf{10} Dv solt also mit mir niht tvͦn V  $\cdot$ dûne] Du W  $\cdot$ mit mir niht alsô] also nit mir U mir also nicht W \textbf{11} vroun] vrouͦ U willen W  $\cdot$ Kundewiramurs] Cvndewiramvrs T Cuͦndewiramurs U [Cvndewiramu*]: Cvndewiramurs V gundwiramurs W \textbf{12} Schentaflurs] [Schentafluͦs]: Schentafluͦrs U [schenthaflu*]: schenthaflurs V \textbf{13} wære] ward W  $\cdot$ im gevalt] siner [*]: hant V \textbf{14} seneschalt] [senesch*]: seneschant V tscheinschalt W \textbf{15} diz] daz V (W)  $\cdot$ Breharz] brebarz U [Br*]: Brobartz V brebars W \textbf{16} Gurnemanz] Guͦrnemanz U Gurnemantz W  $\cdot$ von] de W  $\cdot$ Graharz] grehartz V grahars W \textbf{17} herschaft] [hersc*aft]: herscraft V \textbf{19} vünf hundert] Nuͦnhuͦndert U (V) Tausent W  $\cdot$ wolten] wol U V W \textbf{21} vünfzehen hundert] vuͦnfhuͦndert U [*]: fúnfzehen hundert V \textbf{22} in] im W \textbf{24} gar] \textit{om.} W \textbf{25} ir] In W  $\cdot$ vil] \textit{om.} V  $\cdot$ der sehste] die summe W \textbf{26} Vil helde vielen do dernider W \textbf{27} darbich] [dar*]: darb ich V \textbf{28} an] nun von W  $\cdot$ mir] mir mir U mich V \textbf{29} dîne vreise] din vreisen U [din*]: dine reise V deine freisen W \textbf{30} dem] den U W  $\cdot$ Brituneise] Brituͦneisen U brituneisen W \newline
\end{minipage}
\end{table}
\end{document}
