\documentclass[8pt,a4paper,notitlepage]{article}
\usepackage{fullpage}
\usepackage{ulem}
\usepackage{xltxtra}
\usepackage{datetime}
\renewcommand{\dateseparator}{.}
\dmyyyydate
\usepackage{fancyhdr}
\usepackage{ifthen}
\pagestyle{fancy}
\fancyhf{}
\renewcommand{\headrulewidth}{0pt}
\fancyfoot[L]{\ifthenelse{\value{page}=1}{\today, \currenttime{} Uhr}{}}
\begin{document}
\begin{table}[ht]
\begin{minipage}[t]{0.5\linewidth}
\small
\begin{center}*D
\end{center}
\begin{tabular}{rl}
\textbf{547} & \begin{large}S\end{large}prach der schifman, des grôzer danc\\ 
 & was mit nîgen niht ze kranc.\\ 
 & dô sprach er: "lieber hêrre mîn,\\ 
 & dar zuo \textbf{ruochet} selbe sîn\\ 
5 & mit mir hînte durch gemach.\\ 
 & \textbf{grœzer} êre nie geschach\\ 
 & decheinem verjen, mîme genôz.\\ 
 & man prüevet mir\textbf{z} vür sælde grôz,\\ 
 & \textbf{behalt} ich \textbf{alsus} werden man."\\ 
10 & Dô sprach mîn hêr Gawan:\\ 
 & "des ir gert, \textbf{des} solt \textbf{ich} biten.\\ 
 & mich hât grôz müede überstriten,\\ 
 & daz mir \textbf{ruowens wære} nôt.\\ 
 & diu mir diz ungemach \textbf{gebôt},\\ 
15 & diu kan wol süeze siuren\\ 
 & \textbf{unt} dem herze\textit{n} vreude \textbf{tiuren}\\ 
 & unt \textbf{der} sorgen machen rîche;\\ 
 & \textbf{s\textit{i} lônet ungelîche}.\\ 
 & \textbf{Owê}, \textbf{vindenlîchiu} vlust,\\ 
20 & \textbf{dû senkest} mir \textbf{die einen} brust,\\ 
 & diu \textbf{ê} der hœhe gerte,\\ 
 & dô mich got vreuden werte!\\ 
 & dâ lac ein herze unden,\\ 
 & ich wæne, daz ist verswunden.\\ 
25 & wâ sol ich nû \textbf{trœsten} holn?\\ 
 & muoz ich âne helfe doln\\ 
 & nâch minne \textbf{alsolhe} riwe?\\ 
 & pfligt si wîplîcher triwe,\\ 
 & si sol mir vreude mêren,\\ 
30 & diu mich \textbf{kan sus} \textbf{versêren}."\\ 
\end{tabular}
\scriptsize
\line(1,0){75} \newline
D Fr7 \newline
\line(1,0){75} \newline
\textbf{1} \textit{Initiale} D  \textbf{10} \textit{Majuskel} D  \textbf{19} \textit{Majuskel} D  \newline
\line(1,0){75} \newline
\textbf{16} herzen] herzem D \textbf{18} si] so D \newline
\end{minipage}
\hspace{0.5cm}
\begin{minipage}[t]{0.5\linewidth}
\small
\begin{center}*m
\end{center}
\begin{tabular}{rl}
 & sprach der schifman, des grôzer danc\\ 
 & was mit nîgen niht zuo kranc.\\ 
 & dô sprach er: "lieber hêrre mîn,\\ 
 & dar zuo \textbf{ruochet} selbe sîn\\ 
5 & mit mir hînt durch gemach.\\ 
 & \textbf{gelîchiu} êre nie geschach\\ 
 & dekeinem verjen, mîm genôz.\\ 
 & man brüefet mir \textbf{daz} vür sælde grôz,\\ 
 & \textbf{behalt} ich \textbf{sus den} werden man."\\ 
10 & dô sprach mîn hêr Gawan:\\ 
 & "des ir gert, \textbf{daz} solt \textbf{ir} biten.\\ 
 & mich het grôz müede überstriten,\\ 
 & \textbf{sô} daz mir \textbf{ruowens wær} nôt.\\ 
 & \dag diz\dag  mir diz ungemach \textbf{bôt},\\ 
15 & diu kan wol süeze sûren\\ 
 & \textbf{und} dem herzen vröude \textbf{trûren}\\ 
 & und \textbf{der} sorgen machen rîch;\\ 
 & \textbf{dem enist ûf erden niht glîch}\\ 
 & \textbf{âne} \textbf{süntlîch} vlust.\\ 
20 & \textbf{diu senket} mir \textit{\textbf{mîn}} brust,\\ 
 & diu \textbf{ê} der hœh\textit{e} gerte,\\ 
 & dô mich got vröude werte.\\ 
 & dâ lac ein herz unden,\\ 
 & ich wæne, daz ist verswunden.\\ 
25 & wâ sol ich nû \textbf{trôs\textit{t}} holn?\\ 
 & muoz ich âne helfe doln\\ 
 & nâch minne \textbf{soliche} \textit{r}iuwe?\\ 
 & pfligt si wîplîcher triuwe,\\ 
 & si sol mir vröuden mêren,\\ 
30 & diu mich \textbf{kan} \textbf{versêren}."\\ 
\end{tabular}
\scriptsize
\line(1,0){75} \newline
m n o \newline
\line(1,0){75} \newline
\newline
\line(1,0){75} \newline
\textbf{1} der schifman des] des schiffman das o \textbf{2} nîgen] jngen o \textbf{4} selbe] selbes n \textbf{6} geschach] gesach o \textbf{7} dekeinem] Do keinem n Dekeinen o  $\cdot$ verjen] feriem n  $\cdot$ mîm] min n mȳ o \textbf{8} sælde] solde m o solt n \textbf{9} sus den] alsus n (o) \textbf{10} hêr] herre her n \textbf{11} daz] des n \textbf{13} mir] mar o \textbf{14} bôt] gebot n o \textbf{16} trûren] [tuͯren]: truͯren m túren n (o) \textbf{17} der] er m \textbf{19} Ouwe sundeclich verlust n  $\cdot$ Owe findeclich fluͯst o \textbf{20} mîn] \textit{om.} m \textbf{21} der] der der o  $\cdot$ hœhe] hohen m \textbf{25} trôst] trosteten m trosteln o \textbf{27} riuwe] truͯwe m \textbf{29} vröuden] freide n (o) \newline
\end{minipage}
\end{table}
\newpage
\begin{table}[ht]
\begin{minipage}[t]{0.5\linewidth}
\small
\begin{center}*G
\end{center}
\begin{tabular}{rl}
 & \begin{large}S\end{large}prach der schifman, des grôzer danc\\ 
 & was mit nîgen niht ze kranc.\\ 
 & dô sprach er: "lieber hêrre mîn,\\ 
 & dar zuo \textbf{ruochet} selbe sîn\\ 
5 & mit mir hînt durch g\textit{e}mach.\\ 
 & \textbf{gelîchiu} êre nie geschach\\ 
 & deheinem verjen, m\textit{îm} genôz.\\ 
 & man prüevet mir\textbf{z} vür sælde grôz,\\ 
 & \textbf{behalte} ich \textbf{alsus} werden man."\\ 
10 & dô sprach mîn hêrre Gawan:\\ 
 & "des ir gert, \textbf{des} solde \textbf{ich} biten.\\ 
 & mich hâ\textit{t} grôz müede überstriten,\\ 
 & daz mir \textbf{ruowens wære} nôt.\\ 
 & diu mir ditze ungemach \textbf{gebôt},\\ 
15 & diu kan wol süeze siuren,\\ 
 & dem herzen vröude \textbf{tiuren}\\ 
 & unde \textbf{der} sorgen machen rîche;\\ 
 & \textbf{si lônet ungelîche}.\\ 
 & \textbf{owê}, \textbf{vindeclîchiu} vlust!\\ 
20 & \textbf{diu senket} mir \textbf{die einen} brust,\\ 
 & diu \textbf{ê} der hœhe gerte,\\ 
 & dô mich got vröuden werte.\\ 
 & dâ lac ein herze unden,\\ 
 & ich wæne, daz ist verswunden.\\ 
25 & wâ sol ich nû \textbf{trœsten} holen?\\ 
 & muoz ich âne helfe dolen\\ 
 & nâch minne \textbf{alsolher} riuwe?\\ 
 & pfliget si wîplîcher triuwe,\\ 
 & si sol mir vröude mêren,\\ 
30 & diu mich \textbf{sus kan} \textbf{versêren}."\\ 
\end{tabular}
\scriptsize
\line(1,0){75} \newline
G I L M Z \newline
\line(1,0){75} \newline
\textbf{1} \textit{Initiale} G L Z  \textbf{3} \textit{Initiale} I  \textbf{19} \textit{Initiale} I  \newline
\line(1,0){75} \newline
\textbf{1} Sprach] Do sprach L  $\cdot$ grôzer] grozzen Z \textbf{2} niht] des niht Z \textbf{3} dô] Da M \textbf{4} ruochet] geruchit M  $\cdot$ sîn] schin M \textbf{5} gemach] g:mach G \textbf{6} gelîchiu] solhev I \textbf{7} verjen] \textit{om.} L  $\cdot$ mîm] m::: G \textbf{8} sælde] seldin M \textbf{10} dô] Da M  $\cdot$ hêrre Gawan] ergawan M \textbf{11} des ir] Das ir M \textbf{12} hât] hate G had uch M  $\cdot$ grôz] groze G grozev I \textbf{15} wol süeze] suͤzen wol I \textbf{16} dem] Vnd den L Vnde deme M (Z) \textbf{17} unde der sorgen] An der sorge L Anden sorgin M \textbf{20} diu] Duͯ L  $\cdot$ senket] schenket Z  $\cdot$ einen] eyn M \textbf{22} dô] Da M Z  $\cdot$ vröuden] frevde Z \textbf{24} ist] si I \textbf{25} trœsten] trost L M Z \textbf{27} alsolher] al solche L (M) (Z) \textbf{29} si] So M \textbf{30} sus kan] kan sus M Z  $\cdot$ versêren] sern I \newline
\end{minipage}
\hspace{0.5cm}
\begin{minipage}[t]{0.5\linewidth}
\small
\begin{center}*T
\end{center}
\begin{tabular}{rl}
 & sprach der schifman, des grôzer danc\\ 
 & was mit nîgenne niht ze kranc.\\ 
 & dô sprach er: "lieber hêrre mîn,\\ 
 & dar zuo \textbf{geruochet} selbe sîn\\ 
5 & mit mir hînt \textbf{sîn} durch gemach.\\ 
 & \textbf{grœzer} êre nie geschach\\ 
 & deheinem verjen, mînem genôz.\\ 
 & man prüevet mir\textbf{z} vür sælde grôz,\\ 
 & \textbf{behielt} ich \textbf{alsus} werden man."\\ 
10 & Dô sprach mîn hêr Gawan:\\ 
 & "des ir gert, \textbf{des} solt \textbf{ich} biten.\\ 
 & mich hât grôz müede überstriten,\\ 
 & daz mir \textbf{wære ruowe} nôt.\\ 
 & diu mir diz ungemach \textbf{gebôt},\\ 
15 & diu kan wol süeze siuren,\\ 
 & dem herzen vröude \textbf{tiuren}\\ 
 & unde \textbf{den} sorge\textit{n} machen rîche;\\ 
 & \textbf{si lônet unglîche}.\\ 
 & \textbf{ouwê}, \textbf{vindenlîch\textit{iu}} ve\textit{r}lust,\\ 
20 & \textbf{dû senkest} mir \textbf{die eine} brust,\\ 
 & di\textit{u} \textbf{ie} der hœhe gerte,\\ 
 & dô mich got vröuden werte!\\ 
 & dâ lac ein herze unden,\\ 
 & ich wæne, daz ist verswunden.\\ 
25 & wâ sol ich nû \textbf{trœsten} holn?\\ 
 & muoz ich âne helfe doln\\ 
 & nâch minne \textbf{sölche} riuwe?\\ 
 & pfliget si wîplîcher triuwe,\\ 
 & si sol mir vröude mêren,\\ 
30 & diu mich \textbf{sus kan} \textbf{verkêren}."\\ 
\end{tabular}
\scriptsize
\line(1,0){75} \newline
T U V W O Q R \newline
\line(1,0){75} \newline
\textbf{10} \textit{Majuskel} T  \textbf{19} \textit{Initiale} W  \newline
\line(1,0){75} \newline
\textbf{1} \textit{Die Verse 547.1-2 fehlen} O   $\cdot$ grôzer] [*]: groser V grossen W (R) \textbf{2} was] Vnd was W  $\cdot$ nîgenne niht ze] nygenden nit R \textbf{4} selbe] selber W \textbf{5} sîn durch gemach] durch gemach W (O) Q (R) [*]: durch gemach  V \textbf{7} deheinem verjen] Deheinem [vergem]: vergen V Keinen frewnd Q  $\cdot$ mînem] minen gast R \textbf{8} sælde] selden W \textbf{9} behielt] Behalt V W O Q R \textbf{11} des ir] Dez ir V Das ir R  $\cdot$ des solt] dez solt V das súlt W (R)  $\cdot$ ich] ir W \textbf{12} mich] Sich Q \textbf{13} wære ruowe] ruͦwe were U (V) (W) (O) (Q) (R) \textbf{14} diz] den O \textit{om.} R  $\cdot$ ungemach] gemach Q  $\cdot$ gebôt] bot W (Q) R \textbf{16} dem] Vnd dem U W (O) Q Vnde in dem V Vnd den R \textbf{17} den sorgen] den sorgenden T der sorgen V W O R der sorge Q \textbf{19} vindenlîchiu] vindenliche T windecliche Q vindenklicher R  $\cdot$ verlust] velust T glust Q \textbf{20} dû senkest] Si senchet O Du schenckest Q (R)  $\cdot$ die] mine V  $\cdot$ eine] \textit{om.} V einen O Q andren R \textbf{21} diu ie] die ie T [D*]: Die ie V Div ê O (Q) (R)  $\cdot$ der hœhe] er [ho*]: hohe Q \textbf{22} mich] mich ee W \textbf{24} verswunden] verschunden R \textbf{25} nû] min R  $\cdot$ trœsten] trost U O (Q) (R) \textbf{27} sölche] alsolche W (O) (Q) als hoͯche R \textbf{28} wîplîcher] wibes O wiplich R \textbf{29} si] \textit{om.} W So Q  $\cdot$ mir vröude] mir vreiden U wir frewden Q \textbf{30} sus kan] kan svz V als kon Q  $\cdot$ verkêren] verseren U V W O Q R \newline
\end{minipage}
\end{table}
\end{document}
