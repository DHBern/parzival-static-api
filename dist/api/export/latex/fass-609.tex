\documentclass[8pt,a4paper,notitlepage]{article}
\usepackage{fullpage}
\usepackage{ulem}
\usepackage{xltxtra}
\usepackage{datetime}
\renewcommand{\dateseparator}{.}
\dmyyyydate
\usepackage{fancyhdr}
\usepackage{ifthen}
\pagestyle{fancy}
\fancyhf{}
\renewcommand{\headrulewidth}{0pt}
\fancyfoot[L]{\ifthenelse{\value{page}=1}{\today, \currenttime{} Uhr}{}}
\begin{document}
\begin{table}[ht]
\begin{minipage}[t]{0.5\linewidth}
\small
\begin{center}*D
\end{center}
\begin{tabular}{rl}
\textbf{609} & \begin{large}D\end{large}ô sprach \textbf{des werden} Lotes sun:\\ 
 & "welt ir \textbf{daz ze liebe} tuon\\ 
 & iwer vriundinne, ob \textbf{ez diu} ist,\\ 
 & daz ir sus valschlîchen list\\ 
5 & von ir vater kunnet sagen\\ 
 & unt dar zuo gern het erslagen\\ 
 & ir bruoder, \textbf{sô ist si} ein übel magt,\\ 
 & daz si den site an iu niht klagt.\\ 
 & Künde si \textbf{tohter unde swester} sîn,\\ 
10 & \textbf{sô wære si} \textbf{ir beider} vögetîn,\\ 
 & daz ir verbæret disen haz.\\ 
 & wie stüende iwerem sweher daz,\\ 
 & het er triwe zerbrochen?\\ 
 & habt ir \textbf{des} niht gerochen,\\ 
15 & daz ir in tôt gein valsche sagt?\\ 
 & sîn sun \textbf{ist des} unverzagt\\ 
 & - in sol des niht verdriezen,\\ 
 & mag er niht geniezen\\ 
 & sîner swester wol gevar -,\\ 
20 & ze pfande \textbf{er gît} \textbf{sich selben} dar.\\ 
 & Hêrre, ich heize Gawan.\\ 
 & swaz iu mîn vater hât getân,\\ 
 & daz rechet an mir - er ist tôt.\\ 
 & ich sol vür \textbf{sîn} lasters nôt,\\ 
25 & hân ich werdeclîchez leben,\\ 
 & ûf kampf vür in ze gîsel geben."\\ 
 & Dô sprach der künec: "sît ir daz,\\ 
 & dar ich trage unverkornen haz,\\ 
 & sô tuot mir iwer werdecheit\\ 
30 & beidiu liebe und leit.\\ 
\end{tabular}
\scriptsize
\line(1,0){75} \newline
D Z \newline
\line(1,0){75} \newline
\textbf{1} \textit{Initiale} D Z  \textbf{9} \textit{Majuskel} D  \textbf{21} \textit{Majuskel} D  \textbf{27} \textit{Majuskel} D  \newline
\line(1,0){75} \newline
\textbf{1} des werden] der kunic Z  $\cdot$ Lotes] Lots D \textbf{8} an iu niht] niht an ev Z \textbf{10} sô wære si] Sie were Z \textbf{18} niht] des niht Z \textbf{24} sîn] sines Z \newline
\end{minipage}
\hspace{0.5cm}
\begin{minipage}[t]{0.5\linewidth}
\small
\begin{center}*m
\end{center}
\begin{tabular}{rl}
 & \begin{large}D\end{large}ô sprach \textbf{der werde} Lotes sun:\\ 
 & "wol\textit{t} ir \textbf{daz zuo liebe} tuon\\ 
 & iuwer vriundîn, ob \textbf{ez diu} ist,\\ 
 & daz ir sus valschlîchen list\\ 
5 & von ir vater k\textit{unn}et sagen\\ 
 & und dar zuo gern het\textbf{e\textit{n}} erslagen,\\ 
 & ir bruoder, \textbf{sô ist si} ein übel maget,\\ 
 & daz si die site an iu niht kl\textit{a}get.\\ 
 & künde si \textbf{tohter oder swester} sîn,\\ 
10 & \textbf{sô wær si} \textbf{\textit{bei}der} vögetîn,\\ 
 & daz ir verbæret disen haz.\\ 
 & wie stüende iuwerm swe\textit{h}er daz,\\ 
 & het er \textbf{die} triuwe zerbrochen?\\ 
 & hab\textit{et} ir \textbf{daz} niht gerochen,\\ 
15 & daz ir in tôt gegen valsch saget?\\ 
 & sîn sun \textbf{des gar ist} unverzaget\\ 
 & - in sol des niht verdriezen,\\ 
 & mac er niht geniezen\\ 
 & sîner swester wol gevar -,\\ 
20 & zuo pfande \textbf{gît er} \textbf{sich selber} dar.\\ 
 & hêrre, ich heize Gawan.\\ 
 & waz iu mîn vater hât getân,\\ 
 & daz rechet an mir - er ist tôt.\\ 
 & ich sol vür \textbf{sîn} lasters nôt,\\ 
25 & hân ich werdeclîchez leben,\\ 
 & ûf kampf vür in zuo \textit{g}î\textit{s}el geben."\\ 
 & \begin{large}D\end{large}ô sprach der künic: "sît ir daz,\\ 
 & dar ich trage u\textit{nver}kornen haz,\\ 
 & sô tuot mir iuwer werdicheit\\ 
30 & beidiu liep und leit.\\ 
\end{tabular}
\scriptsize
\line(1,0){75} \newline
m n o \newline
\line(1,0){75} \newline
\textbf{1} \textit{Initiale} m n  \textbf{27} \textit{Initiale} m   $\cdot$ \textit{Capitulumzeichen} n  \newline
\line(1,0){75} \newline
\textbf{1} \textit{Die Verse 608.18-609.30 fehlen (Blattverlust)} o   $\cdot$ Lotes] lotz m n \textbf{2} wolt] Wol m \textbf{5} kunnet] koment m \textbf{6} heten] hette m \textbf{8} klaget] claclaget m \textbf{10} beider] brieder m ir beider n \textbf{12} sweher] swetter m \textbf{13} het] \sout{Der} het m \textbf{14} habet] Hab m \textbf{23} an] [in]: an m \textbf{26} gîsel] sigel m \textbf{28} unverkornen] vmb kornen m \newline
\end{minipage}
\end{table}
\newpage
\begin{table}[ht]
\begin{minipage}[t]{0.5\linewidth}
\small
\begin{center}*G
\end{center}
\begin{tabular}{rl}
 & \begin{large}D\end{large}ô sprach \textbf{des künic} Lotes sun:\\ 
 & "welt ir \textbf{getriuwelîchen} tuon\\ 
 & iuwer vriundin, ob \textbf{diu daz} ist,\\ 
 & da\textit{z} ir sus valsch\textit{lîch}en list\\ 
5 & von ir vater kunnet sagen\\ 
 & unde dar zuo gerne het erslagen\\ 
 & ir bruoder? \textbf{sist} ein übel maget,\\ 
 & daz si den site an iu niht klaget.\\ 
 & künde si \textbf{tohter unde swester} sîn,\\ 
10 & \textbf{si wære} \textbf{ir bruoder} vögetîn,\\ 
 & daz ir verbæret disen haz.\\ 
 & wie stüende iuwerm sweher daz,\\ 
 & hete \textit{er} \textbf{sîne} triuwe \textit{z}e\textit{r}brochen?\\ 
 & habet ir \textbf{des} niht gerochen,\\ 
15 & daz ir in tôt gein valsche saget?\\ 
 & sîn sun \textbf{ist des gar} unverzaget\\ 
 & - in sol des niht verdriezen,\\ 
 & mag er \textbf{des} niht geniezen\\ 
 & sîner swester wol gevar -,\\ 
20 & ze pfande \textbf{er gît} \textbf{sîn leben} dar.\\ 
 & hêrre, ich heize Gawan.\\ 
 & swaz iu mîn vater \textit{hât} getân,\\ 
 & daz rechet an mir - er ist tôt.\\ 
 & ich sol vür \textbf{sînes} lasters nôt,\\ 
25 & hân ich werdeclîchez leben,\\ 
 & ûf kampf vür in ze gîsel geben."\\ 
 & \begin{large}D\end{large}ô sprach der künic: "sît ir daz,\\ 
 & dar \textit{ich} trage unverkornen haz,\\ 
 & sô tuot mir iuwe\textit{r} werdecheit\\ 
30 & beidiu liep unde leit.\\ 
\end{tabular}
\scriptsize
\line(1,0){75} \newline
G I L M Z Fr34 Fr51 \newline
\line(1,0){75} \newline
\textbf{1} \textit{Initiale} G I L Z Fr51  \textbf{21} \textit{Initiale} I M  \textbf{27} \textit{Initiale} G Fr51  \newline
\line(1,0){75} \newline
\textbf{1} Dô] Da M  $\cdot$ des künic] des kunges I (M) (Fr51) kvnig L der kunic Z  $\cdot$ Lotes] lotis G lottes Fr51 \textbf{2} getriuwelîchen] daz zv liebe Z truͦwelichen Fr51 \textbf{3} vriundin] vreundnen Fr51  $\cdot$ ob diu daz] obz die L (M) (Z) (Fr51) \textbf{4} daz] Dar G  $\cdot$ valschlîchen] valschen G veltslichen L valsliche Fr51 \textbf{5} ir] iren Fr51  $\cdot$ kunnet] kvnnen Fr51 \textbf{6} het] hetten Fr51 \textbf{7} sist] so ist sý L (M) (Z) (Fr51) \textbf{8} site] syten M zite Fr51  $\cdot$ an iu niht] niht an ev Z \textbf{9} swester] sweher I \textbf{10} Vnd daz vor mir niht welt verdagen Fr34  $\cdot$ ir bruoder] ir beider Z irs bruder Fr51 \textbf{11} verbæret] verbaret L ver baren Fr51 \textbf{13} hete er] hete G Hettet ir M  $\cdot$ sîne] \textit{om.} L M Z Fr51  $\cdot$ zerbrochen] gebrochen G ::: Fr34 \textbf{14} habet] Habe Fr51  $\cdot$ des] daz Fr34 (Fr51) \textbf{15} Daz ir nach tode valsch saget Fr34  $\cdot$ tôt] nu toten I  $\cdot$ gein] an Fr51 \textbf{16} des gar] \textit{om.} L des Z Fr51 \textbf{17} in] Jm Fr51  $\cdot$ verdriezen] erdrýeszen L \textbf{18} mag] Ne mach Fr51  $\cdot$ des] \textit{om.} I L M Fr34 Fr51 \textbf{20} ze] So Fr51  $\cdot$ pfande] kamphe M  $\cdot$ sîn leben] sich selben Z \textbf{22} swaz] Waz L (M) (Fr51)  $\cdot$ hât] \textit{om.} G \textbf{24} sol] \textit{om.} Fr51  $\cdot$ sînes lasters] sin laster Fr34 \textbf{25} ich] min Fr34  $\cdot$ werdeclîchez] werdichlichen Fr51 \textbf{26} ûf] Zo Fr51  $\cdot$ in] im Fr51 \textbf{27} Dô] Da M \textbf{28} dar] sit I  $\cdot$ ich] er G  $\cdot$ unverkornen] vnverborgen L vnverholn M \textbf{29} iuwer] iͮwe G \newline
\end{minipage}
\hspace{0.5cm}
\begin{minipage}[t]{0.5\linewidth}
\small
\begin{center}*T
\end{center}
\begin{tabular}{rl}
 & \begin{large}D\end{large}ô sprach \textbf{des küneges} Lotes sun:\\ 
 & "wolt ir \textbf{daz zuo liebe} tuon\\ 
 & iuwer vriundîn, ob \textbf{ez diu} ist,\\ 
 & daz ir sus valschlîchen list\\ 
5 & von ir vater kunnet sagen\\ 
 & und dar zuo gerne hetet erslagen\\ 
 & ir bruoder, \textbf{sô ist si} ein übel maget,\\ 
 & daz si den sit an iu niht klaget.\\ 
 & künde \textit{si} \textbf{swester und tohter} sîn,\\ 
10 & \textbf{si wære} \textbf{ir beider} vögetîn,\\ 
 & daz ir \textit{v}erb\textit{ær}et disen haz.\\ 
 & wie stüende iuwerme sweher daz,\\ 
 & het er triuwe zerbrochen?\\ 
 & hât ir \textbf{des} niht gerochen,\\ 
15 & daz ir in tôt gein valsche saget?\\ 
 & sîn sun \textbf{ist des} unverzaget\\ 
 & - in sol \textit{des} niht verdriezen,\\ 
 & mac er niht geniezen\\ 
 & sîner swester wol gevar -,\\ 
20 & zuo pfande \textbf{er gît} \textbf{sîn leben} dar.\\ 
 & hêrre, ich heize Gawan.\\ 
 & waz iu mîn vater hât getân,\\ 
 & daz rechet an mir - er ist tôt.\\ 
 & ich sol vür \textbf{sînes} lasters nôt,\\ 
25 & hân ich wirdeclîchez leben,\\ 
 & ûf kampf vür in zuo gîsel geben."\\ 
 & \begin{large}D\end{large}ô sprach der künec: "sît ir daz,\\ 
 & dar ich trage unverkornen haz,\\ 
 & sô tuot mir iuwer wirdecheit\\ 
30 & beidiu liep und leit.\\ 
\end{tabular}
\scriptsize
\line(1,0){75} \newline
U V W Q R \newline
\line(1,0){75} \newline
\textbf{1} \textit{Initiale} U V W Q R  \textbf{21} \textit{Initiale} W  \textbf{27} \textit{Initiale} U V R  \newline
\line(1,0){75} \newline
\textbf{1} küneges] kúnig W (R)  $\cdot$ Lotes] lottes W \textbf{2} liebe] eren W \textbf{4} sus] als Q  $\cdot$ valschlîchen] [val*]: valschlichen V valschen W \textbf{8} sit] \textit{om.} R \textbf{9} si] \textit{om.} U [*]: sv́ V  $\cdot$ swester und tohter] swester [*]: oder tohter V tochter vnd schwester W (Q) \textbf{10} si wære ir] [S*]: So were sv́ V \textbf{11} verbæret] werbet U [*]: verberent V \textbf{12} sweher] schecher R \textbf{13} [H*]: Hette er die trvwe zerbrochen V \textbf{14} niht gerochen] [*]: vngerochen V \textbf{15} in] \textit{om.} Q \textbf{16} des] des gar W Q R \textbf{17} des] \textit{om.} U \textbf{18} er] er des R \textbf{19} sîner] Sin R \textbf{20} er gît] gibt er Q  $\cdot$ sîn leben] [*]: sich selben V \textbf{21} ich] Jch bin vnd R  $\cdot$ Gawan] gawann Q \textbf{22} waz] Swaz V \textbf{26} ûf] So wil ich V \textbf{27} Der kúnig sprach herre seit ir das W \textbf{28} dar] Darff Q \textbf{30} und] vnd och R \newline
\end{minipage}
\end{table}
\end{document}
