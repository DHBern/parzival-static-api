\documentclass[8pt,a4paper,notitlepage]{article}
\usepackage{fullpage}
\usepackage{ulem}
\usepackage{xltxtra}
\usepackage{datetime}
\renewcommand{\dateseparator}{.}
\dmyyyydate
\usepackage{fancyhdr}
\usepackage{ifthen}
\pagestyle{fancy}
\fancyhf{}
\renewcommand{\headrulewidth}{0pt}
\fancyfoot[L]{\ifthenelse{\value{page}=1}{\today, \currenttime{} Uhr}{}}
\begin{document}
\begin{table}[ht]
\begin{minipage}[t]{0.5\linewidth}
\small
\begin{center}*D
\end{center}
\begin{tabular}{rl}
\textbf{26} & Der gast zer wirtinne\\ 
 & sprach mit ritters sinne:\\ 
 & "saget \textbf{mir}, ob irs \textbf{geruochet},\\ 
 & durch waz man iuch sô suochet\\ 
5 & zornlîche mit gewalt.\\ 
 & ir habet \textbf{sô} manegen \textbf{degen} balt.\\ 
 & mich müet, daz \textbf{si} sint \textbf{verladen}\\ 
 & mit vîende \textbf{hazze} nâch ir schaden."\\ 
 & "\textbf{daz sage ich} \textbf{iu}, \textbf{hêrre}, sît irs gert.\\ 
10 & mir diende ein ritter, der was wert.\\ 
 & sîn lîp was tugende ein bernde rîs.\\ 
 & der helt was küene unt wîs,\\ 
 & der triwe ein \textbf{reht} \textbf{beklîbeniu} vruht.\\ 
 & sîn zuht \textbf{wac} vür alle zuht.\\ 
15 & \textbf{der} was \textbf{noch} kiuscher denne ein wîp.\\ 
 & vrecheit unt ellen truoc sîn lîp.\\ 
 & sô\textbf{ne} gewuohs an \textbf{ritter} milter hant\\ 
 & vor im \textbf{nie} über elliu lant.\\ 
 & ine weiz, waz nâch uns sule geschehen,\\ 
20 & \textbf{des} lâzen ander liute jehen.\\ 
 & er was \textbf{gein} valscher vuore ein tôr,\\ 
 & \textbf{in} swarzer varwe als ich ein môr.\\ 
 & sîn vater hiez Tankanis,\\ 
 & ein künec, \textbf{der} het ouch \textbf{hôhen} prîs.\\ 
25 & \begin{large}\textit{M}\end{large}în vriunt, der hiez Isenhart.\\ 
 & mîn wîpheit was \textbf{unbewart},\\ 
 & dô ich sîn dienest nâch minne enpfienc,\\ 
 & \textbf{deiz} \textbf{im nâch vröuden} niht ergienc.\\ 
 & des muoz ich immer jâmer tragen.\\ 
30 & si \textbf{wænent}, daz ich in \textbf{schuof} erslagen.\\ 
\end{tabular}
\scriptsize
\line(1,0){75} \newline
D Fr14 \newline
\line(1,0){75} \newline
\textbf{1} \textit{Majuskel} D  \textbf{25} \textit{Initiale} D  \newline
\line(1,0){75} \newline
\textbf{7} si] di Fr14  $\cdot$ verladen] beladen Fr14 \textbf{9} iu] \textit{om.} Fr14 \textbf{23} Tankanis] Tanchanis D \textbf{25} Mîn] ÷in D  $\cdot$ Isenhart] Jsenhart D \newline
\end{minipage}
\hspace{0.5cm}
\begin{minipage}[t]{0.5\linewidth}
\small
\begin{center}*m
\end{center}
\begin{tabular}{rl}
 & \begin{large}D\end{large}er gast zer wirtinne\\ 
 & sprach mit ritters sinne:\\ 
 & "saget \textbf{mir}\textbf{s}, ob irs \textbf{geruochet},\\ 
 & durch waz man iuch sô suochet\\ 
5 & zornlîche mit gewalt.\\ 
 & ir habet \textbf{sô} manigen \textbf{degen} balt.\\ 
 & mich müet, daz \textbf{si} sint \textbf{beladen}\\ 
 & mit vîende \textbf{harte} nâch ir schaden."\\ 
 & "\textbf{daz sage ich}, sît irs gert.\\ 
10 & mir diente ein ritter, der was wert.\\ 
 & sîn lîp was tugende ein ber\textit{n}de rîs.\\ 
 & der helt was küene und wîs,\\ 
 & der triuwe ein \textbf{reht} \textbf{beklîbeniu} vruht.\\ 
 & sîn zuht \textbf{was} vür alle zuht.\\ 
15 & \textbf{er} was \textbf{noch} kiuscher danne ein wîp.\\ 
 & vrecheit und ellen truoc sîn lîp.\\ 
 & \dag sume\dag  gewuohs an \textbf{ritter} milter hant\\ 
 & vor ime \textit{\textbf{nie}} über alliu lant.\\ 
 & \textit{in}e weiz, waz nâch uns sule geschehen.\\ 
20 & \textbf{diz} lâzen ander liute jehen.\\ 
 & er was \textbf{gegen} valscher vuore ein tôr,\\ 
 & \textbf{in} swarzer varwe als ich ein môr.\\ 
 & sîn vater, \textbf{der} hiez Tanckanis,\\ 
 & ein künic, \textbf{der} hete ouch \textbf{hôhen} prîs.\\ 
25 & mîn vriunt, der hiez Ysenhart.\\ 
 & mîn wîpheit was \textbf{unbewart},\\ 
 & dô ich sîn dienst nâch minne enpfienc,\\ 
 & \textbf{daz ez} \textbf{nâch vröude ime} niht ergienc.\\ 
 & des muoz ich iemer jâmer tragen.\\ 
30 & si \textbf{wænent}, daz ich in \textbf{schüefe} erslagen.\\ 
\end{tabular}
\scriptsize
\line(1,0){75} \newline
m n o \newline
\line(1,0){75} \newline
\textbf{1} \textit{Initiale} m o   $\cdot$ \textit{Capitulumzeichen} n  \newline
\line(1,0){75} \newline
\textbf{3} mirs] mir n \textit{om.} o \textbf{4} \textit{Die Verse 26.4-29.1 fehlen} o  \textbf{7} müet] jnnet n \textbf{8} vîende] vigenden n \textbf{9} ich] ich herre n \textbf{10} ein] dem n \textbf{11} bernde] bermde m \textbf{13} triuwe] truͦg n \textbf{16} vrecheit] Friheit n \textbf{17} sume] Sume \textit{nachträglich korrigiert zu:} Kume m Es n \textbf{18} nie] \textit{om.} m \textbf{19} ine] Me m  $\cdot$ sule] sol n  $\cdot$ geschehen] beschehen n \textbf{22} als ich] also n  $\cdot$ môr] [mon]: mor m \textbf{24} hete] [herre]: hette m \textbf{25} Ysenhart] ÿsenhart m \textbf{30} schüefe] schuͦff n \newline
\end{minipage}
\end{table}
\newpage
\begin{table}[ht]
\begin{minipage}[t]{0.5\linewidth}
\small
\begin{center}*G
\end{center}
\begin{tabular}{rl}
 & der gast zer wirtinne\\ 
 & sprach mit rîters sinne:\\ 
 & "saget \textbf{mir}, obe irs \textbf{ruochet},\\ 
 & durch waz man iuch sô suochet\\ 
5 & zorniclîchen mit gewalt.\\ 
 & ir habet \textbf{vil} manigen \textbf{degen} balt.\\ 
 & mich müet, daz \textbf{si} sint \textbf{verladen}\\ 
 & mit vînde \textbf{hazze} nâch ir schaden."\\ 
 & "\textbf{ich sagez} \textbf{iu}, \textbf{hêrre}, sît irs gert.\\ 
10 & mir diente ein rîter, der was wert.\\ 
 & sîn lîp was tugende ein bernde rîs.\\ 
 & der helt was küene und wîs,\\ 
 & der triwe ein \textbf{beklîbendiu} vruht.\\ 
 & sîn zuht \textbf{wac} vür alle zuht.\\ 
15 & \textbf{er} was kiuscher danne ein wîp.\\ 
 & vrecheit und ellen truoc sîn lîp.\\ 
 & sô gewuohs an \textbf{man} \textbf{nie} milter hant\\ 
 & vor im über elliu lant.\\ 
 & ichne weiz, waz nâch uns sule geschehen,\\ 
20 & \textbf{des} lâzen ander liute jehen.\\ 
 & er was \textbf{vor} valscher vuore ein tôr,\\ 
 & \textbf{nâch} swarzer varwe als ich ein môr.\\ 
 & sîn vater, \textbf{der} hiez Tanchanis,\\ 
 & ein künic, \textbf{er} het ouch \textbf{hôhen} brîs.\\ 
25 & mîn vriunt, der hiez Ysenhart.\\ 
 & mîn wîpheit was \textbf{vil} \textbf{unbewart},\\ 
 & dô ich sîn dienst nâch minne enpfie,\\ 
 & \textbf{daz} \textbf{im nâch vröuden} niht ergie.\\ 
 & des muoz ich imer jâmer tragen.\\ 
30 & si \textbf{wænent}, daz ich in \textbf{schüefe} erslagen.\\ 
\end{tabular}
\scriptsize
\line(1,0){75} \newline
G O L M Q R W Z Fr29 Fr32 Fr71 \newline
\line(1,0){75} \newline
\textbf{1} \textit{Initiale} O R Fr29 Fr71  \textbf{3} \textit{Initiale} M   $\cdot$ \textit{Versal} Fr32  \textbf{9} \textit{Initiale} R  \textbf{23} \textit{Initiale} L W  \textbf{25} \textit{Initiale} Q Z Fr32  \newline
\line(1,0){75} \newline
\textbf{1} der] ÷er O  $\cdot$ zer wirtinne] mit ritters sýnne L \textbf{2} mit rîters sinne] zuͯ der kvnigynne L \textbf{3} ruochet] [geruchent]: geruͦchent L geruchet Q (Z) (Fr32) \textbf{4} durch waz] War vmbe L wa::: Fr71  $\cdot$ iuch sô] uch L ir so R eúch suß W \textbf{5} zorniclîchen] Zontliche L Zorlichen Q \textbf{6} vil] hie L so Q Fr32  $\cdot$ degen] helt O L M Q (R) W Z (Fr29) Fr32 \textbf{7} si sint] die sint O L Fr29 Fr32 dy syn M (R) si sin Z  $\cdot$ verladen] úberladen W \textbf{8} vînde] vienden M (R)  $\cdot$ nâch] vff Q mit W \textbf{9} ich sagez iu] Dize sag ich O (Fr29) Daz sage ich L (Z) Das sage ich uch M (W) Disz sag ich euch Q (R) (Fr32)  $\cdot$ irs] ir sin Z (Fr71) ir Fr29 \textbf{10} diente] dient O (Q) (R) Z  $\cdot$ der was] \textit{om.} Q \textbf{11} sîn] Des L W  $\cdot$ bernde] bornde M \textbf{12} helt] \textit{om.} Z \textbf{13} triwe] rechten tugend W triuwen Fr32  $\cdot$ beklîbendiu] reht belibnev O beclybene L rechte bekleben M (Q) Recht berrende R bernde W reht beklibene Z (Fr29) (Fr32) \textbf{14} wac] was O (Q) R W Fr29 wugk M  $\cdot$ zuht] [frucht]: zucht R \textbf{15} er] Der Z  $\cdot$ kiuscher] noch chevscher O (L) (M) (W) (Z) (Fr29) \textbf{16} vrecheit] Freyheyt Q  $\cdot$ ellen] elle \textit{nachträglich korrigiert zu:} ere Q  $\cdot$ truoc] rich O \textbf{17} An ritter gewuͯhs nie milter hant L  $\cdot$ sô] Son O (M) (Fr29) Nie W  $\cdot$ gewuohs] gewusch Q  $\cdot$ man] ritter O M Q (R) W Z (Fr29) ritters Fr32  $\cdot$ nie] nie so O \textit{om.} M Q R Z Fr29 Fr32  $\cdot$ milter] miltiv O (Q) \textbf{18} vor] Von L Q  $\cdot$ im] mir W  $\cdot$ über] nie vber O (M) (Q) (R) Z (Fr29) Fr32 \textbf{19} ichne weiz] En weisz M ihn Fr32  $\cdot$ nâch uns] vnsz noch Q (Z) mir R vns W  $\cdot$ sule] schvl O (L) (M) (Q) (W) (Fr29) (Fr32) so R \textbf{20} des] Das R (Fr32)  $\cdot$ lâzen] lasz in wie M laz wir Z \textbf{21} was vor] was gein O L (M) (Q) R (Fr29) (Fr32) gieng W was an Z  $\cdot$ valscher vuore] valscheit L vascher ge fúre R \textbf{22} nâch] Jn Q R Z (Fr32)  $\cdot$ als ich ein] als ein O L (Q) W also y keyn M \textbf{23} vater] \textit{om.} Q  $\cdot$ der] \textit{om.} O M W Z Fr29  $\cdot$ Tanchanis] dankanis L M W tankonis Q Tankanis R Fr29 Fr32 tankaris Z \textbf{24} Der het auch werdeclichen priß W  $\cdot$ er] der O L M (Q) R Z Fr29 (Fr32)  $\cdot$ het] hat L  $\cdot$ ouch] \textit{om.} M Q \textbf{25} der] \textit{om.} M  $\cdot$ Ysenhart] ẏsenhart G isenhart O eyszenhart Q Jsenbart R ẏsenhârt Fr32 \textbf{26} wîpheit] wiplicheit R  $\cdot$ vil] \textit{om.} O L M Q R W Z Fr29 Fr32 \textbf{27} dô ich] Dúch W Da ich Z  $\cdot$ minne] im Q \textbf{28} daz] Daz es L (M) (Fr29) Do es Q Da ez Z \textbf{30} wænent] weynen M  $\cdot$ daz] \textit{om.} W  $\cdot$ in schüefe] in schuff Q (R) (Z) (Fr71) schuͦff in W  $\cdot$ erslagen] erslahen O \newline
\end{minipage}
\hspace{0.5cm}
\begin{minipage}[t]{0.5\linewidth}
\small
\begin{center}*T
\end{center}
\begin{tabular}{rl}
 & Der gast zer wirtinne\\ 
 & sprach mit rîters sinne:\\ 
 & "\textbf{Nû} saget, ob irs \textbf{geruochet},\\ 
 & durch waz man iuch sô suochet\\ 
5 & zornlîche mit gewalt.\\ 
 & ir habt \textbf{hie} manegen \textbf{helt} balt.\\ 
 & mich müet, daz \textbf{die} sint \textbf{beladen}\\ 
 & mit vîende \textbf{hazze} nâch ir schaden."\\ 
 & "\textbf{daz sag ich} \textbf{iu}, \textbf{hêrre}, sît irs gert.\\ 
10 & mir diende ein rîter, der was wert.\\ 
 & sîn lîp was tugende ein bernde rîs.\\ 
 & der helt was küene und wîs,\\ 
 & der triuwen ein \textbf{beklîbeniu} vruht.\\ 
 & sîn zuht \textbf{wac} vür alle zuht.\\ 
15 & \textbf{er} was \textbf{noch} kiuscher danne ein wîp.\\ 
 & vrecheit und ellen truoc sîn lîp.\\ 
 & sô\textbf{ne} gewuohs an \textbf{rîter} milter hant\\ 
 & vor im \textbf{nie} über alliu lant.\\ 
 & ine weiz, waz nâch uns sul geschehen,\\ 
20 & \textbf{des} lâzen ander liute jehen.\\ 
 & er was \textbf{gegen} valscher vuor ein tôr,\\ 
 & \textbf{nâch} swarzer varwe als ich ein môr.\\ 
 & Sîn vater, \textbf{der} hiez Tankenis,\\ 
 & ein künec, \textbf{der} hete ouch \textbf{den} prîs.\\ 
25 & Mîn vriunt, der hiez Isenhart.\\ 
 & mîn wîp\textit{heit} was \textbf{vil} \textbf{ungespart},\\ 
 & dô ich sînen dienst nâch minne enpfienc,\\ 
 & \textbf{daz ez} \textbf{im nâch vröuden} niht ergienc.\\ 
 & des muoz ich iemer jâmer tragen.\\ 
30 & si \textbf{wânden}, daz ich in \textbf{schüefe} erslagen.\\ 
\end{tabular}
\scriptsize
\line(1,0){75} \newline
T U V \newline
\line(1,0){75} \newline
\textbf{1} \textit{Initiale} U V   $\cdot$ \textit{Majuskel} T  \textbf{3} \textit{Majuskel} T  \textbf{23} \textit{Majuskel} T  \textbf{25} \textit{Majuskel} T  \newline
\line(1,0){75} \newline
\textbf{2} sprach] Srach U \textbf{4} iuch] îv T \textbf{7} beladen] verladin U \textbf{8} vîende] viendes V \textbf{9} iu hêrre] herre iu U herre V  $\cdot$ irs] ir V \textbf{11} sîn] Des V  $\cdot$ bernde] bornde U \textbf{13} der triuwen] Die truͦwe U Der truwe V  $\cdot$ beklîbeniu vruht] gebenediete vruͦth U [*]: beclibene fruht V \textbf{14} wac] waz V \textbf{15} ein] \textit{om.} U \textbf{17} Men vant vnder rittern milter hant V  $\cdot$ sône] Schone U \textbf{19} nâch uns] vns noch V  $\cdot$ geschehen] beschehen V \textbf{20} lâzen] lazen wir U (V) \textbf{22} nâch] Jn V  $\cdot$ varwe] varwen U  $\cdot$ als ich ein môr] als [eim more]: ein mor V \textbf{24} ouch den] hohen U V \textbf{25} Isenhart] Jsenhart T U V \textbf{26} wîpheit] wip T  $\cdot$ was vil ungespart] die was vnverspart U die waz [*]: vnbewart V \textbf{27} dô ich sînen dienst] Do in sime dinste U [D*]: Do ich sin dienest V  $\cdot$ minne] minnen U \textbf{28} daz ez] Daz U (V)  $\cdot$ niht] sit U \textbf{30} wânden] wante U [w*]: wenent V \newline
\end{minipage}
\end{table}
\end{document}
