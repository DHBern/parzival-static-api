\documentclass[8pt,a4paper,notitlepage]{article}
\usepackage{fullpage}
\usepackage{ulem}
\usepackage{xltxtra}
\usepackage{datetime}
\renewcommand{\dateseparator}{.}
\dmyyyydate
\usepackage{fancyhdr}
\usepackage{ifthen}
\pagestyle{fancy}
\fancyhf{}
\renewcommand{\headrulewidth}{0pt}
\fancyfoot[L]{\ifthenelse{\value{page}=1}{\today, \currenttime{} Uhr}{}}
\begin{document}
\begin{table}[ht]
\begin{minipage}[t]{0.5\linewidth}
\small
\begin{center}*D
\end{center}
\begin{tabular}{rl}
\textbf{517} & er mac sich harte wol bejagen,\\ 
 & \textbf{gelernet} er bühsen veile tragen."\\ 
 & \begin{large}Z\end{large}er vrouwen sprach Gawans munt:\\ 
 & "ich reit vür einen rîter wunt,\\ 
5 & \textbf{des} dach ist ein linde.\\ 
 & ob ich den noch vinde,\\ 
 & disiu würze sol in wol ernern\\ 
 & unt al sîn unkraft erwern."\\ 
 & Si sprach: "daz sihe ich gerne.\\ 
10 & waz, ob ich kunst gelerne?"\\ 
 & Dô vuor \textbf{im} balde ein knappe nâch;\\ 
 & dem was zer botschefte gâch,\\ 
 & die er werben solte.\\ 
 & Gawan sîn \textbf{beiten} wolte;\\ 
15 & dô dûht ern ungehiure.\\ 
 & Malcreatiure\\ 
 & hiez der knappe fiere;\\ 
 & Cundrie la surziere\\ 
 & was sîn swester wolgetân.\\ 
20 & er muose ir antlütze hân\\ 
 & gar, wan daz er was ein man.\\ 
 & im \textbf{stuon\textit{t}} ouch ietsweder zan\\ 
 & als einem eber wilde,\\ 
 & ungelîch menschen bilde.\\ 
25 & Im was \textbf{daz hâr ouch} niht sô lanc,\\ 
 & als ez Cundrien ûf \textbf{den} mûl dort swanc.\\ 
 & kurz, scharf als \textbf{igels hût} ez was.\\ 
 & bî dem wazzer \textbf{Ganjas}\\ 
 & ime lande ze Tribalibot\\ 
30 & wahsent liute alsus durch nôt.\\ 
\end{tabular}
\scriptsize
\line(1,0){75} \newline
D \newline
\line(1,0){75} \newline
\textbf{3} \textit{Initiale} D  \textbf{9} \textit{Majuskel} D  \textbf{11} \textit{Majuskel} D  \textbf{25} \textit{Majuskel} D  \newline
\line(1,0){75} \newline
\textbf{18} Cundrie la surziere] Cvndrîe Lasvrzîere D \textbf{22} stuont] stvͦnde D \textbf{28} Ganjas] Ganias D \newline
\end{minipage}
\hspace{0.5cm}
\begin{minipage}[t]{0.5\linewidth}
\small
\begin{center}*m
\end{center}
\begin{tabular}{rl}
 & er mac sich harte wol bejagen,\\ 
 & \textbf{gelêret} er bühsen veil tragen."\\ 
 & zuor vrouwen sprach Gawanes munt:\\ 
 & "ich reit vür einen ritter wunt.\\ 
5 & \textbf{daz} dach ist ein linde.\\ 
 & ob ich den noch vinde,\\ 
 & disiu würz sol in wol ernern\\ 
 & und al sîn unkraft erwern."\\ 
 & si sprach: "daz sihe ich gerne.\\ 
10 & waz, ob ich kunst gelerne?"\\ 
 & dô vuo\textit{r} \textbf{in} balde ein knappe nâch;\\ 
 & dem was zuor botschafte gâch,\\ 
 & die er werben solte.\\ 
 & Gawan sîn \textbf{beiten} wolte;\\ 
15 & dô dûhte ern ungehiur.\\ 
 & Mala creatiur\\ 
 & hiez der knappe viere;\\ 
 & Condrie la \textit{s}urziere\\ 
 & was sîn swester wol getân.\\ 
20 & er muoste ir antlitze hân\\ 
 & gar, wan daz er was ein man.\\ 
 & im \textbf{stuont} ouch ietweder zan\\ 
 & als einem eber wilde,\\ 
 & ungelîch menschen bilde.\\ 
25 & im was \textbf{daz hâr ouch} niht sô lanc,\\ 
 & als ez Condrien ûf \textbf{den} mûl dort swanc.\\ 
 & kurz, scharpf als \textbf{ein igelhût} ez was.\\ 
 & bî dem wazzer \textbf{Gamas}\\ 
 & in dem lande zuo Tribalibot\\ 
30 & wahsent liute alsus durch nôt.\\ 
\end{tabular}
\scriptsize
\line(1,0){75} \newline
m n o \newline
\line(1,0){75} \newline
\newline
\line(1,0){75} \newline
\textbf{2} bühsen] [buͯche]: buͯchssen m \textbf{3} Gawanes] gewannes o \textbf{6} vinde] fúnde o \textbf{9} sihe ich] [ich sic]: sihe ich n \textbf{10} ob] \textit{om.} o \textbf{11} vuor] fuͯre m n  $\cdot$ balde] alle o \textbf{14} beiten] werben n [beieen]: beiden o \textbf{15} dûhte ern] duͯchte er o \textbf{16} Mala creatur m  $\cdot$ Mala creatúre n  $\cdot$ Malo creatúre o \textbf{18} Condrie lazurzier m n  $\cdot$ Cuͯndrie lacuͯrcier o \textbf{20} antlitze] anczlit o \textbf{22} stuont] stuͯn o  $\cdot$ zan] [zaln]: zan m \textbf{23} eber] erber o \textbf{26} Condrien] kuͯndrien o  $\cdot$ mûl dort] múldert n (o) \textbf{27} scharpf] schafft o \textbf{28} Gamas] gemasz n \textbf{29} Tribalibot] [s]: tribalibot o \newline
\end{minipage}
\end{table}
\newpage
\begin{table}[ht]
\begin{minipage}[t]{0.5\linewidth}
\small
\begin{center}*G
\end{center}
\begin{tabular}{rl}
 & \begin{large}E\end{large}r mac sich harte wol bejagen,\\ 
 & \textbf{gelernt} er bühsen veil tragen."\\ 
 & ze der vrouwen sprach Gawans munt:\\ 
 & "ich reit vür einen rîter wunt,\\ 
5 & \textbf{des} dach ist ein linde.\\ 
 & ob ich den noch vinde,\\ 
 & disiu würz sol in wol ernern\\ 
 & unt al sîn unkraft erwern."\\ 
 & si sprach: "daz sich ich gerne.\\ 
10 & waz, ob ich kunst gelerne?"\\ 
 & dô vuor \textbf{in} balde ein knappe nâch;\\ 
 & dem was zer botschefte gâch,\\ 
 & die er werben solde.\\ 
 & Gawan sîn \textbf{beiten} wolde;\\ 
15 & dô dûht ern ungehiure.\\ 
 & Mal creatiure\\ 
 & hiez der knappe fiere;\\ 
 & Gundrie lasurziere\\ 
 & was sîn swester wolgetân.\\ 
20 & er muose ir antlütze hân\\ 
 & gar, wan daz er was ein man.\\ 
 & im \textbf{stuont} ouch ietwederre zan\\ 
 & als einem eber wilde,\\ 
 & ungelîch menschen bilde.\\ 
25 & im was \textbf{daz hâr ouch} niht sô lanc,\\ 
 & als ez Gundrien ûf \textbf{dem} mûl dort swanc.\\ 
 & kurz, scharpfe als \textbf{igelshût} ez was.\\ 
 & bî dem wazzer \textbf{Ganjas}\\ 
 & in dem lande ze Tribalibot\\ 
30 & wahsent liute alsus durch nôt.\\ 
\end{tabular}
\scriptsize
\line(1,0){75} \newline
G I L M Z Fr23 \newline
\line(1,0){75} \newline
\textbf{1} \textit{Initiale} G I Z Fr23   $\cdot$ \textit{Capitulumzeichen} L  \textbf{17} \textit{Initiale} I  \newline
\line(1,0){75} \newline
\textbf{2} gelernt] Kan M  $\cdot$ bühsen] buhse I (L) bussen Fr23  $\cdot$ veil] vaste M \textbf{3} Gawans] Gawanez L \textbf{4} vür] vf Fr23 \textbf{7} ernern] nern I \textbf{8} unt] \textit{om.} M  $\cdot$ unkraft] cechraft Fr23  $\cdot$ erwern] wern M \textbf{11} dô] Da L M Z Fr23  $\cdot$ in] yme M (Fr23) \textbf{13} werben] erwerben I da wirken M \textbf{14} sîn] niht Fr23 \textbf{15} dô] Da Z  $\cdot$ ern] en M \textbf{16} Malcreature G  $\cdot$ Malcrature I  $\cdot$ Malcreature L  $\cdot$ Mal creature M  $\cdot$ Mal createvre Z  $\cdot$ Mala creatuͦr Fr23 \textbf{18} Gvndrie lansvrziere G  $\cdot$ Kvndrie Lasvrziere L  $\cdot$ Cvndrie lasurziere Z \textbf{19} was] [wol]: was G \textbf{20} muose] muͤst I Musz M \textbf{22} ietwederre] ieslichir M \textbf{23} einem] ein Fr23 \textbf{25} daz] sin I \textbf{26} Gundrien] kvndrien L Cundrien Z Gundri::: Fr23  $\cdot$ dem] den L  $\cdot$ dort] da I M \textit{om.} L \textbf{27} scharpfe] swartz scharf Z schraf Fr23  $\cdot$ igelshût] igelhut I eyn egils hued M \textbf{28} Ganjas] ganias G Granias I Ganiaz L ganayas M Gamas Z \textbf{29} ze Tribalibot] zetribalibôt G zebripalibot I \textbf{30} nôt] got M \newline
\end{minipage}
\hspace{0.5cm}
\begin{minipage}[t]{0.5\linewidth}
\small
\begin{center}*T
\end{center}
\begin{tabular}{rl}
 & er mac sich harte wol bejagen,\\ 
 & \textbf{gelernet} er bühsen veile tragen."\\ 
 & \textit{\begin{large}Z\end{large}}er vrouwen sprach Gawans munt:\\ 
 & "ich reit vür einen rîter wunt,\\ 
5 & \textbf{des} dach ist ein linde.\\ 
 & ob ich den noch vinde,\\ 
 & dis\textit{iu} würz sol in wol ernern\\ 
 & unde alsîn unkraft e\textit{r}wern."\\ 
 & Si sprach: "daz sich ich gerne.\\ 
10 & waz, ob ich kunst gelerne?"\\ 
 & Dô vuor \textbf{in} balde ein knappe nâch;\\ 
 & dem was zer botschefte gâch,\\ 
 & die er \textbf{dâ} werben solte.\\ 
 & Gawan sîn \textbf{bîten} wolte;\\ 
15 & dô dûhtern ungehiure.\\ 
 & Malacreatiure\\ 
 & hiez der knappe fiere;\\ 
 & Kundrie Lasursiere\\ 
 & was sîn swester wolgetân.\\ 
20 & \multicolumn{1}{l}{ - - - }\\ 
 & \multicolumn{1}{l}{ - - - }\\ 
 & im \textbf{was} ouch ietweder zan\\ 
 & als einem eber wilde,\\ 
 & unglîch menschen bilde.\\ 
25 & im was \textbf{ouch daz hâr} niht sô lanc,\\ 
 & als ez Kundrien ûf \textbf{den} mûl dort swanc.\\ 
 & kurz, scharpf als \textbf{eines igels hût} ez was.\\ 
 & bî dem wazzer \textbf{Ganyas}\\ 
 & in dem lande ze Tribalibot\\ 
30 & wahsent liute alsu\textit{s} durch nôt.\\ 
\end{tabular}
\scriptsize
\line(1,0){75} \newline
T U V W O Q R Fr40 \newline
\line(1,0){75} \newline
\textbf{1} \textit{Initiale} V O Fr40  \textbf{3} \textit{Initiale} T U W  \textbf{9} \textit{Majuskel} T  \textbf{11} \textit{Majuskel} T  \newline
\line(1,0){75} \newline
\textbf{1} er] ÷r O Der Q Fr40  $\cdot$ harte] armvͤte V  $\cdot$ bejagen] entsagen V betragen R \textbf{3} Zer] Der T U  $\cdot$ Gawans] gawains R \textbf{4} vür einen] vor einem Q \textbf{6} den noch] dannoch W \textbf{7} disiu] dise T (R)  $\cdot$ sol] so R  $\cdot$ wol ernern] nern O \textbf{8} alsîn] als in U  $\cdot$ erwern] ewern T erwere W wern R \textbf{9} sich] sehe U \textbf{10} ob] \textit{om.} R  $\cdot$ gelerne] lerne Q \textbf{11} Dô] Da R  $\cdot$ balde ein knappe] ein knappe [nach]: balde V \textbf{12} zer] zu R \textbf{13} dâ] do U V W Q \textbf{14} Gawan] Gawain R  $\cdot$ sîn] sy W (Fr40)  $\cdot$ bîten] beiden U Q (R) (Fr40) bitten W \textbf{16} Malacreatvre T (U)  $\cdot$ Mal creatv́re V (R)  $\cdot$ Mala creatúre W  $\cdot$ Mal crativre O  $\cdot$ Malcreateure Q (Fr40) \textbf{17} der] der selbe V \textbf{18} Kvndrye Lasursyere T  $\cdot$ Kuͦndrie lasurziere U  $\cdot$ Kvndrie lazvrzier V  $\cdot$ Kundrie lasursiere W  $\cdot$ Kundrie lazuriere Q  $\cdot$ Kundrik lasurziere R \textbf{20} \textit{Die Verse 517.20-21 fehlen} T U   $\cdot$ Er mvͤste (muͦst W [ O Q R ]) ir antlitze han (\textit{om.} Q ) V (W) (O) (Q) (R) \textbf{21} Gar wan (denne R ) daz er waz ein man V (W) (Q) (R) \textbf{22} was] stuͦnt U (V) (W) (O) (Q) (R)  $\cdot$ ietweder] ietwedrez O \textbf{24} \textit{Vers 517.24 fehlt} Q   $\cdot$ menschen] menschlichem W R \textbf{25} ouch] anch W  $\cdot$ daz] sin O  $\cdot$ sô] also V (R) \textbf{26} Kundrien] kvndryen T kuͦndrie U kundrie W (O) kondrien Q kundriern R  $\cdot$ ûf den mûl dort] dort vf dem mvle V (R) auff den mule W \textbf{27} kurz] es was kurtzt Q Kurczes schwarcz R  $\cdot$ eines] ein W O Q R  $\cdot$ igels hût] ygels baut W ygehaut Q  $\cdot$ ez was] \textit{om.} Q \textbf{28} Ganyas] [g*]: gamas V ganias W (R) gamas Q kamas O \textbf{29} dem] dem was Q  $\cdot$ ze] \textit{om.} R  $\cdot$ Tribalibot] Trybalibot T tribabilot W trabalibot R \textbf{30} wahsent] Im wachsent W  $\cdot$ alsus] alsv T als Q \newline
\end{minipage}
\end{table}
\end{document}
