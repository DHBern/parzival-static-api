\documentclass[8pt,a4paper,notitlepage]{article}
\usepackage{fullpage}
\usepackage{ulem}
\usepackage{xltxtra}
\usepackage{datetime}
\renewcommand{\dateseparator}{.}
\dmyyyydate
\usepackage{fancyhdr}
\usepackage{ifthen}
\pagestyle{fancy}
\fancyhf{}
\renewcommand{\headrulewidth}{0pt}
\fancyfoot[L]{\ifthenelse{\value{page}=1}{\today, \currenttime{} Uhr}{}}
\begin{document}
\begin{table}[ht]
\begin{minipage}[t]{0.5\linewidth}
\small
\begin{center}*D
\end{center}
\begin{tabular}{rl}
\textbf{260} & swenne ich vliehen lerne,\\ 
 & sô stirb ich als gerne."\\ 
 & \textit{\begin{large}D\end{large}}ô sprach diu \textbf{blôze} herzogîn:\\ 
 & "er hât hie \textbf{niemen} denne mîn.\\ 
5 & der trôst ist kranc gein strîtes sige."\\ 
 & niht wan knoden unt der rige\\ 
 & was an der vrouwen hemde ganz.\\ 
 & wîplîcher kiusche lobes kranz\\ 
 & truoc si mit armüete.\\ 
10 & si pflac der wâren güete,\\ 
 & \textbf{sô} daz der valsch an ir verswant.\\ 
 & die vintâlen er vür sich bant;\\ 
 & gein strîte er wolde \textbf{vüeren}.\\ 
 & den helm \textbf{er} mit den snüeren\\ 
15 & ebene ze sehene \textbf{ructe}.\\ 
 & innen des daz ors sich bucte,\\ 
 & \textbf{gein} dem pferde ez schrîen niht vermeit.\\ 
 & der vor Parzivale dâ reit\\ 
 & unt vor der blôzen vrouwen,\\ 
20 & der \textbf{erhôrte}\textbf{z} unt wolde schouwen,\\ 
 & wer bî sîme wîbe rite.\\ 
 & daz ors warf er mit \textbf{zornes site}\\ 
 & \textbf{vaste} \textbf{ûz} dem stîge.\\ 
 & gein strîteclîchem wîge\\ 
25 & hielt der herzoge Orilus,\\ 
 & \textbf{gereit} \textbf{z}einer tjost \textbf{alsus}\\ 
 & mit rehter manlîcher ger\\ 
 & von Gaheviez mit eime sper.\\ 
 & daz was geverwet genuoc,\\ 
30 & reht als er \textbf{sîniu} wâpen truoc.\\ 
\end{tabular}
\scriptsize
\line(1,0){75} \newline
D \newline
\line(1,0){75} \newline
\textbf{3} \textit{Initiale} D  \newline
\line(1,0){75} \newline
\textbf{3} Dô] ÷o D \newline
\end{minipage}
\hspace{0.5cm}
\begin{minipage}[t]{0.5\linewidth}
\small
\begin{center}*m
\end{center}
\begin{tabular}{rl}
 & \textbf{wanne} wenne ich vliehen lerne,\\ 
 & sô stirbe ich alsô gerne."\\ 
 & dô sprach diu \textbf{grôze} herzogîn:\\ 
 & "er hât hie \textbf{niemen} danne mîn.\\ 
5 & der trôst ist kranc gegen strîtes \textit{sig}e."\\ 
 & niht wan knoden und der rige\\ 
 & was an der vrouwen hemede ga\textit{n}z.\\ 
 & wîplîcher kiusche lobes kran\textit{z}\\ 
 & truoc si mit armüete.\\ 
10 & si pfl\textit{a}c der wâren güete,\\ 
 & \textbf{sô} daz der valsch an ir verswant.\\ 
 & die fantailen er vür sich bant;\\ 
 & gegen strîte er wolt \textbf{rüeren}.\\ 
 & den helm \textbf{er} mit den snüeren\\ 
15 & ebene ze sehen \textbf{ru\textit{ck}ete}.\\ 
 & innen des daz ros sich buckete,\\ 
 & \textbf{gegen} dem pferde ez schrîen niht vermeit.\\ 
 & der vor Parcifal dâ reit\\ 
 & und vor der blôzen vrouwen,\\ 
20 & der \textbf{erhôrte} \textbf{ez} und wolte schouwen,\\ 
 & wer bî sînem wîbe rite.\\ 
 & daz ros warf er mit \textbf{zornes site}\\ 
 & \textbf{vaste} \textbf{ûz} dem stîge.\\ 
 & gegen strîteclîche\textit{m} wîge\\ 
25 & hielt der herzoge Orilus,\\ 
 & \textbf{gereit} \textbf{ze} einer just \textbf{alsus}\\ 
 & mit rehter manlîcher ger\\ 
 & von Gaheviez mit einem sper.\\ 
 & daz was geverwet genuoc,\\ 
30 & reht alsô e\textit{r} \textbf{\textit{s}îniu} wâpen truoc.\\ 
\end{tabular}
\scriptsize
\line(1,0){75} \newline
m n o Fr69 \newline
\line(1,0){75} \newline
\newline
\line(1,0){75} \newline
\textbf{1} wenne] wo n war o \textbf{2} stirbe] stir n sterbe o \textbf{3} diu grôze] \textit{om.} n die blosse o \textbf{4} danne] denne den n dan [de]: dan o \textbf{5} sige] were m sẏgin o \textbf{7} ganz] gangcz m \textbf{8} wîplîcher] Wipliches o  $\cdot$ kranz] krang m \textbf{10} pflac] pfleg m \textbf{12} fantailen] fantalen o \textbf{15} ruckete] rutte m rurte o \textbf{18} der] Do o  $\cdot$ dâ] do n o \textbf{19} vor] fur o  $\cdot$ blôzen] bossen n \textbf{20} erhôrte] erhort n o \textbf{22} ros] ras o  $\cdot$ er] er vmb n \textbf{24} strîteclîchem] stritteclicher m \textbf{25} Orilus] orelus o \textbf{28} Gaheviez] gahe vies m o gohe viesz n \textbf{30} er sîniu] er einer sine m \newline
\end{minipage}
\end{table}
\newpage
\begin{table}[ht]
\begin{minipage}[t]{0.5\linewidth}
\small
\begin{center}*G
\end{center}
\begin{tabular}{rl}
 & swenne ich vliehen lerne,\\ 
 & sô stirbe ich als gerne."\\ 
 & dô sprach diu \textbf{blôze} herzogîn:\\ 
 & "er hât hie \textbf{niemens} danne mîn.\\ 
5 & der trôst ist kranc gein strîtes sige."\\ 
 & niwan knoden und\textit{e d}er rige\\ 
 & was an der vrouwen hemde ganz.\\ 
 & wîplîcher kiusche lobes kranz\\ 
 & truoc si mit armüete.\\ 
10 & si pflac der wâren güete,\\ 
 & \textbf{sô} daz der valsch an ir verswant.\\ 
 & die vinteilen er vür sich bant;\\ 
 & gein strîter wolte \textbf{vüeren}.\\ 
 & den helm mit den snüeren\\ 
15 & \textbf{er} ebene ze sehenne \textbf{ructe}.\\ 
 & innen des daz ors sich bucte,\\ 
 & \textbf{mit} dem pferde ez schrîen niht vermeit.\\ 
 & de\textit{r} \textit{v}or Parzivale \textit{dâ} reit\\ 
 & unde vor der blôzen vrouwen,\\ 
20 & der \textbf{hôrte} unde wolt schouwen,\\ 
 & wer bî sînem wîbe rite.\\ 
 & daz ors warf er mit \textbf{zornes site}\\ 
 & \textbf{vaste} \textbf{ûz} dem stîge.\\ 
 & gein strîticlîche\textit{m} wîge\\ 
25 & hielt der herzoge Orillus,\\ 
 & \textbf{gere\textit{i}t} \textbf{z}einer tjoste \textbf{sus}\\ 
 & mit rehter manlîche\textit{r} ger\\ 
 & von Kahaviez mit einem sper.\\ 
 & daz was gevärwet genuoc,\\ 
30 & reht als er \textbf{sîniu} wâpen truoc.\\ 
\end{tabular}
\scriptsize
\line(1,0){75} \newline
G I O L M Q R Z Fr21 \newline
\line(1,0){75} \newline
\textbf{3} \textit{Initiale} I R  \textbf{23} \textit{Initiale} I  \textbf{27} \textit{Überschrift:} Hie kumt parcifal vnd hertzog Orilus zvsammen mit strite Z   $\cdot$ \textit{Initiale} L Q Z Fr21  \textbf{29} \textit{Initiale} M   $\cdot$ \textit{Capitulumzeichen} R  \newline
\line(1,0){75} \newline
\textbf{1} swenne] wan swenn I (O) (Fr21) Wan wenne L Z Wann Q Wend itr R  $\cdot$ ich] mich R  $\cdot$ lerne] lernen R \textbf{2} stirbe] sturbe M  $\cdot$ ich] \textit{om.} O  $\cdot$ als] \textit{om.} Q \textbf{3} dô] Da M Z  $\cdot$ diu] \textit{om.} O  $\cdot$ blôze] blozen I \textbf{4} er] ern I (M) (Q) (Z) (Fr21)  $\cdot$ niemens] nieman I (L) (Q) R  $\cdot$ danne] wan L (M) (Q) R Fr21 \textbf{5} gein] nach M \textbf{6} niwan knoden] Nymant konden Q  $\cdot$ unde der] vnde an der G vnz der O vnder R \textbf{7} vrouwen hemde] frowen [lib]: hemede I frowem hemde O \textbf{8} wîplîcher] Weyplich Q  $\cdot$ kiusche] zvhte O  $\cdot$ lobes] blos R \textbf{9} mit] mit ir I Q \textbf{10} der wâren] wibes I \textbf{11} der] \textit{om.} I  $\cdot$ ir] im I in O \textbf{12} vinteilen] fantailen L sein teylen Q synttelen R  $\cdot$ bant] vant Q \textbf{13} strîter] strite M strit er in Z  $\cdot$ wolte] volte L \textbf{15} ze sehenne] zesamen I zesenen Fr21  $\cdot$ ructe] dructe I geruchte M er ructe Q Ruͦrtte R \textbf{16} innen des] inendes I Jnner dez L (Q) Jenne R  $\cdot$ bucte] nider bvhte Fr21 \textbf{17} mit] Gen Q  $\cdot$ ez] er Q  $\cdot$ vermeit] meit Z \textbf{18} der vor] der da vor G Dar vor O  $\cdot$ Parzivale] parzifal I L M Parcifal O (Z) (Fr21) partzifale Q parczifaln R  $\cdot$ dâ] \textit{om.} G \textbf{19} blôzen] blanchen L \textbf{20} der hôrte] derhort I (O) (Z) (Fr21) Der herre L Do hort Q Er hort R  $\cdot$ unde] \textit{om.} L ez vnd Z \textbf{21} bî sînem wîbe] mit siner frowen R  $\cdot$ rite] rette M \textbf{22} daz] Sin R  $\cdot$ mit] vmb in R \textbf{23} ûz] vf L vsser R \textbf{24} gein] Jn R  $\cdot$ strîticlîchem] stritchlichen G stritlicher R  $\cdot$ wîge] vuge R \textbf{25} herzoge] helt O  $\cdot$ Orillus] orilus I (O) M Q (R) Z (Fr21) \textbf{26} gereit] gereht G Bereit O (R)  $\cdot$ zeiner tjoste] zu einem strit R gein einer tiost Z  $\cdot$ sus] alsvs O (L) (M) (Q) (R) (Z) (Fr21) \textbf{27} mit] Sit Fr21  $\cdot$ rehter] ritter L rechtin M  $\cdot$ manlîcher] manlichen G  $\cdot$ ger] wer Fr21 \textbf{28} Kahaviez] kahviez G Gahauiez I Gahaviez O kaheviez L gahevis M kahevitz Q kaheweis R gaheviez Z \textbf{30} sîniu] sin I M Z \newline
\end{minipage}
\hspace{0.5cm}
\begin{minipage}[t]{0.5\linewidth}
\small
\begin{center}*T
\end{center}
\begin{tabular}{rl}
 & swennich vliehen lerne,\\ 
 & sô stirbich als gerne."\\ 
 & Dô sprach diu \textbf{blôze} herzogîn:\\ 
 & "er hât hie \textbf{niemen} danne mîn.\\ 
5 & Der trôst ist kranc gegen strîtes sige."\\ 
 & niht wan knoden unde der rige\\ 
 & was an der vrouwen hemde gan\textit{z}.\\ 
 & wîplîcher kiusche lobes kranz\\ 
 & truoc si mit armüete.\\ 
10 & si pflac der wâren güete,\\ 
 & \textbf{sît} daz der valsch an ir verswant.\\ 
 & Die vinteilen er vür sich bant;\\ 
 & gegen strîte er wolte \textbf{rüeren}.\\ 
 & Den helm mit den snüeren\\ 
15 & \textbf{er} eben ze sehene \textbf{dructe}.\\ 
 & innen des daz ors sich bucte,\\ 
 & \textbf{mit} dem pferde ez schrîen niht vermeit.\\ 
 & Der vor Parcifale dâ reit\\ 
 & unde vor der blôzen vrouwen,\\ 
20 & der \textbf{hôrte} unde wolte schouwen,\\ 
 & wer \textbf{dâ} bî sînem wîbe \textit{rite}.\\ 
 & daz ors warf er mit \textbf{unsite}\\ 
 & \textbf{ûzer} dem stîge.\\ 
 & gegen strîteclîchem wîge\\ 
25 & hielt der herzoge Orilus,\\ 
 & \textbf{der reit} \textbf{gegen} einer tjost \textbf{alsus}\\ 
 & mit rehter manlîcher ger\\ 
 & von Kaheviez mit einem sper.\\ 
 & daz was geverwet genuoc,\\ 
30 & reht als er \textbf{sîn} wâpen truoc.\\ 
\end{tabular}
\scriptsize
\line(1,0){75} \newline
T U V W \newline
\line(1,0){75} \newline
\textbf{3} \textit{Initiale} U V W   $\cdot$ \textit{Majuskel} T  \textbf{5} \textit{Majuskel} T  \textbf{12} \textit{Majuskel} T  \textbf{14} \textit{Majuskel} T  \textbf{18} \textit{Majuskel} T  \newline
\line(1,0){75} \newline
\textbf{1} swennich] Wenic U Wan wenne ich W  $\cdot$ lerne] gelerne W \textbf{2} stirbich] storbe ich U stúrbe ich W  $\cdot$ als] wisse es got gar W \textbf{4} danne] wann W \textbf{7} ganz] ganst T \textbf{11} sît] So V W  $\cdot$ der valsch] die valsheit U \textbf{12} vinteilen] fintellen T U vintellen V vintelen W \textbf{13} gegen] Als ers gen W  $\cdot$ er] \textit{om.} W  $\cdot$ rüeren] vuͦren U (W) [rvͤren]: fvͤren  V \textbf{15} er] Er sy W  $\cdot$ ze sehene] zuͦ samene U sehen W  $\cdot$ dructe] [*]: ructe V geruͦchte W \textbf{16} bucte] behute W \textbf{17} mit] [*]: Gegen V  $\cdot$ ez] er W \textbf{18} Parcifale] parzifale T parzifal V partzifalen W  $\cdot$ dâ] do V \textit{om.} W \textbf{19} \textit{Versfolge 260.20-19} W   $\cdot$ Wer do ritte bei seiner frawen W \textbf{20} hôrte] [*]: erhort ez V erhort es W  $\cdot$ wolte] wol U \textbf{21} Er wolte sehen wer do ritte W  $\cdot$ dâ] do U V  $\cdot$ rite] \textit{om.} T \textbf{22} er] \textit{om.} U  $\cdot$ mit] vmb mit W  $\cdot$ unsite] [*]: zornez sitte V \textbf{23} ûzer] [*]: Vaste vz V Ausserhalb W \textbf{26} alsus] sus W \textbf{28} Kaheviez] Caheveiz U [ka*]: kaheuies V gahafies W \textbf{29} geverwet] geveruͦwet U \textbf{30} er] \textit{om.} W  $\cdot$ sîn] sine U V \newline
\end{minipage}
\end{table}
\end{document}
