\documentclass[8pt,a4paper,notitlepage]{article}
\usepackage{fullpage}
\usepackage{ulem}
\usepackage{xltxtra}
\usepackage{datetime}
\renewcommand{\dateseparator}{.}
\dmyyyydate
\usepackage{fancyhdr}
\usepackage{ifthen}
\pagestyle{fancy}
\fancyhf{}
\renewcommand{\headrulewidth}{0pt}
\fancyfoot[L]{\ifthenelse{\value{page}=1}{\today, \currenttime{} Uhr}{}}
\begin{document}
\begin{table}[ht]
\begin{minipage}[t]{0.5\linewidth}
\small
\begin{center}*D
\end{center}
\begin{tabular}{rl}
\textbf{57} & wem hât sîn manlîchiu zuht\\ 
 & hie lâzen sîner minne vruht?\\ 
 & owê, lieplîch geselleschaft!\\ 
 & sol \textbf{mir} nû riwe mit ir kraft\\ 
5 & immer twingen mînen lîp?\\ 
 & sîme gote ze êren", sprach daz wîp,\\ 
 & "ich mich gerne toufen solte\\ 
 & unt leben, swie er wolte."\\ 
 & Der jâmer gap ir herzen wîc.\\ 
10 & ir \textbf{vreude} vant den dürren zwîc,\\ 
 & als noch diu turteltûbe tuot.\\ 
 & diu het ie den selben muot.\\ 
 & swenne ir an \textbf{trûtscheft} gebrast,\\ 
 & ir triwe kôs den dürren ast.\\ 
15 & Diu vrouwe an rehter zît genas\\ 
 & eines suns, der zweier varwe was,\\ 
 & an dem got \textbf{anders} wart enein:\\ 
 & wîz unt swarzer varwe er schein.\\ 
 & diu künegîn kust in sunder twâl\\ 
20 & vil dicke an sîniu blanken mâl.\\ 
 & diu muoter hiez ir kindelîn\\ 
 & Feirefiz Anschevin.\\ 
 & \textbf{der} wart ein waltswende.\\ 
 & die tjoste sîner hende\\ 
25 & manec sper zerbrâchen,\\ 
 & \textbf{die} schilde \textbf{dürkel stâchen}.\\ 
 & \textit{\begin{large}A\end{large}}ls ein agelster wart \textbf{gevar}\\ 
 & sîn \textbf{hâr} unt ouch sîn \textbf{vel vil gar}.\\ 
 & \textbf{dô} was \textbf{ez} \textbf{ouch} über des jâres zil,\\ 
30 & daz Gahmuret geprîset vil\\ 
\end{tabular}
\scriptsize
\line(1,0){75} \newline
D \newline
\line(1,0){75} \newline
\textbf{9} \textit{Majuskel} D  \textbf{15} \textit{Majuskel} D  \textbf{27} \textit{Initiale} D  \newline
\line(1,0){75} \newline
\textbf{22} Anschevin] Anscevin D \textbf{27} Als] ÷ls D \textbf{30} Gahmuret] Gahmvret D \newline
\end{minipage}
\hspace{0.5cm}
\begin{minipage}[t]{0.5\linewidth}
\small
\begin{center}*m
\end{center}
\begin{tabular}{rl}
 & wem hât sîn manlîchiu zuht\\ 
 & hie lâzen sîner minne vruht?\\ 
 & owê, lie\textit{p}lîch geselleschaft!\\ 
 & sol \textbf{mir} nû riuwe mit ir kraft\\ 
5 & iemer twingen mînen lîp?\\ 
 & sînem gote ze êren", sprach daz wîp,\\ 
 & "ich mich gerne toufen solte\\ 
 & und leben, wie er wolte."\\ 
 & der jâmer ga\textit{p} i\textit{r} herzen wîc.\\ 
10 & i\textit{r} \textbf{vröude} vant den dürren zwîc,\\ 
 & alsô noch diu turteltûbe tuot.\\ 
 & diu het ie den selben muot.\\ 
 & wanne \textit{i}r an \textbf{trûtschaft} gebrast,\\ 
 & ir triuwe kôs den dürren \textit{a}st.\\ 
15 & \begin{large}D\end{large}iu vrowe an rehter zît genas\\ 
 & eines su\textit{n}s, der zweier varwe was,\\ 
 & an dem got \textbf{wunder} wart in ein:\\ 
 & wîz und swarzer varwe er sch\textit{e}in.\\ 
 & diu künigîn kuste in sunder twâl\\ 
20 & vil dicke an \dag ime\dag  blanken mâl.\\ 
 & diu muoter hiez ir kindelîn\\ 
 & Feref\textit{i}z A\textit{n}schevin.\\ 
 & \textbf{er} wart ein \dag waltwende\dag .\\ 
 & die juste sîner hende\\ 
25 & manic sper zerbrâchen,\\ 
 & \textbf{die} schilte \textbf{durchst\textit{â}chen}.\\ 
 & als ein agelster wart \textbf{gevar}\\ 
 & sîn \textbf{vel} und ouch sîn \textbf{\textit{h}âr}.\\ 
 & \textbf{\begin{large}N\end{large}û} was \textbf{ez} über des jâres zil,\\ 
30 & daz Gahmuret geprîset vil\\ 
\end{tabular}
\scriptsize
\line(1,0){75} \newline
m n o \newline
\line(1,0){75} \newline
\textbf{15} \textit{Initiale} m o   $\cdot$ \textit{Capitulumzeichen} n  \textbf{29} \textit{Initiale} m   $\cdot$ \textit{Capitulumzeichen} n  \newline
\line(1,0){75} \newline
\textbf{1} hât] hette o \textbf{2} lâzen] gelossen n o  $\cdot$ sîner] mẏner o \textbf{3} lieplîch] liechtlich m  $\cdot$ geselleschaft] [gestalt]: geselleschafft o \textbf{5} mînen] sinen n \textbf{6} sînem] Sinen o  $\cdot$ sprach] do sprach n \textbf{7} \textit{Verse 57.7-8 kontrahiert zu:} Jch mich gerne tauffen wolte o  \textbf{9} gap] gag \textit{nachträglich korrigiert zu:} jagt m  $\cdot$ ir] irs m iren o \textbf{10} ir] Jch m \textbf{11} turteltûbe] turczel túbe o \textbf{12} het ie den] hat jeden n \textbf{13} ir] mir \textit{nachträglich korrigiert zu:} ir m \textbf{14} ast] gast m \textbf{15} an] in n o \textbf{16} suns] sus \textit{nachträglich korrigiert zu:} suͯns m sús o \textbf{18} swarzer] swarcz er m swartze n (o)  $\cdot$ er schein] erschin m erscheine n o \textbf{19} kuste] kust n k:st o \textbf{20} ime blanken] imeblancken \textit{nachträglich korrigiert zu:} dieblancken m  $\cdot$ mâl] mannl o \textbf{22} Ferefiz] Ferefs m Ferresis n Feresis o  $\cdot$ Anschevin] ausceuin m auscefin n anste vin o \textbf{25} zerbrâchen] zerbrochen n \textbf{26} durchstâchen] durch stochen m n \textbf{28} hâr] gar \textit{nachträglich korrigiert zu:} har m \textbf{29} des] das o  $\cdot$ zil] [zit]: zil n \textbf{30} Gahmuret] gamiret n \newline
\end{minipage}
\end{table}
\newpage
\begin{table}[ht]
\begin{minipage}[t]{0.5\linewidth}
\small
\begin{center}*G
\end{center}
\begin{tabular}{rl}
 & wem hât sîn manl\textit{î}chiu zuht\\ 
 & hie lâzen \textit{s}îner minn\textit{e} vruht?\\ 
 & owê, lieplîch geselleschaft!\\ 
 & sol \textbf{mir} nû riwe mit ir kraft\\ 
5 & imer dwingen mînen lîp?\\ 
 & sînem got zêren", sprach daz wîp,\\ 
 & "ich mich gerne toufen solte\\ 
 & unde leben, swier wolte."\\ 
 & der jâmer gap ir herzen wîc.\\ 
10 & ir \textbf{vröude} vant den dürren zwîc,\\ 
 & als noch diu turteltûbe tuot.\\ 
 & diu het ie den selben muot.\\ 
 & swenne ir an \textbf{vriuntschaft} gebrast,\\ 
 & ir triwe kôs den dürren ast.\\ 
15 & diu vrouwe an rehter zît genas\\ 
 & eines sunes, der zweier varwe was,\\ 
 & an dem got \textbf{wunders} wart enein:\\ 
 & wîz und swarzer varwe er schein.\\ 
 & diu künigîn kust in sunder twâl\\ 
20 & vil dicke an sîniu blanken mâl.\\ 
 & diu muoter hiez ir kindelîn\\ 
 & Feirafiz Antschevin.\\ 
 & \textbf{der} wart ein waltswende.\\ 
 & die t\textit{j}ost \textbf{ze} sîner hende\\ 
25 & manic sper zerbrâchen\\ 
 & \textbf{\begin{large}U\end{large}nd} schilte \textbf{dürkel stâchen}.\\ 
 & als ein agelster wart \textbf{gevar}\\ 
 & sîn \textbf{hâr} und ouch sîn \textbf{vel vil gar}.\\ 
 & \textbf{nû} was \textbf{ez} über des jâres zil,\\ 
30 & daz Gahmuret gebrîset vil\\ 
\end{tabular}
\scriptsize
\line(1,0){75} \newline
G I O L M Q R Z Fr21 Fr37 Fr44 \newline
\line(1,0){75} \newline
\textbf{1} \textit{Initiale} O  \textbf{3} \textit{Initiale} I  \textbf{15} \textit{Initiale} L R  \textbf{19} \textit{Initiale} I M  \textbf{26} \textit{Initiale} G  \textbf{27} \textit{Initiale} R Z Fr21 Fr37  \textbf{29} \textit{Initiale} L M Fr44  \newline
\line(1,0){75} \newline
\textbf{1} wem] ÷em O  $\cdot$ manlîchiu] manlchiv G \textbf{2} hie lâzen] verlazzen hie I (O) (M) (Q) (Fr21) Verlaszen L (R) (Fr37)  $\cdot$ sîner minne] miner minnen G sine O M (Q) R (Fr21) \textbf{3} owê] Awe vil I Awe O \textbf{4} sol] So R  $\cdot$ mit] vnde O  $\cdot$ ir] \textit{om.} R \textbf{6} sînem got] Sinen goten O \textbf{7} ich mich gerne] Gerne ich mich Fr44 \textbf{8} swier] wie er L (M) Q R Z \textbf{9} der] Jr R Durch Fr37  $\cdot$ wîc] dik R \textbf{10} vant] chos I (L) \textbf{12} diu] der Fr37 Sie Fr44  $\cdot$ het] hat I M  $\cdot$ ie] auch I hie Fr37 Fr44  $\cdot$ selben] seben I \textbf{13} swenne] Wenne L (M) (Q) (R)  $\cdot$ ir] her M ie Fr21  $\cdot$ vriuntschaft] ir travtschaft O (Fr21) trutschaft L (M) (Q) (R) trewschefte Z ritterschefte Fr44 \textbf{14} den] einen O (L) (Fr21) Fr37 ie den Z  $\cdot$ dürren] durret Z \textbf{15} an] ze I (Q) \textbf{16} varwe] uarwen Fr44 \textbf{17} dem] den Fr21  $\cdot$ wunders wart] was wunders M  $\cdot$ enein] ein ein I \textbf{18} swarzer] schwarcz R swarz an Fr44  $\cdot$ er] an Im R \textbf{19} \textit{Versfolge 57.27-28.19-26} L   $\cdot$ kust in] in chuste Fr37  $\cdot$ sunder] ane L \textbf{20} vil] \textit{om.} I  $\cdot$ sîniu] synem M sin R  $\cdot$ blanken] blancew I (Fr37) blankil M blanke Q Fr44 wises R \textbf{22} Feirafiz] feirefiz G (O) (Fr21) Ferefis L Feirefisz M Fewefisz Q Ferefeis R Ferefiz Z ferafeiz Fr37 Ferefiez Fr44  $\cdot$ Antschevin] antscheuin I ansevin O Anschevin L (M) (R) anszheuin Q anshevin Z Anschvin Fr21 enschauein Fr37 Anscheuin Fr44 \textbf{23} waltswende] walt [wende]: swende Q waltschwede R \textbf{24} tjost] tost G M zost Fr44  $\cdot$ sîner] sinen I \textbf{25} manic] Vil manich O (L) (M) (Q) (R) (Z) Fr21 (Fr37) (Fr44)  $\cdot$ zerbrâchen] zu brach Q zerbrechen R \textbf{26} Und] Die O L M Q R Z Fr21 (Fr37) Fr44  $\cdot$ dürkel stâchen] dvrch stochen O er durch stach Q gar durchstochen R \textbf{27} ein] \textit{om.} O  $\cdot$ wart] wart er I L  $\cdot$ gevar] gewar R \textbf{28} und ouch] vnd L och R  $\cdot$ sîn vel] vel O  $\cdot$ vil] \textit{om.} L \textbf{29} ez] oz ouch M (R) (Z) (Fr21) (Fr44) er auch Q  $\cdot$ des] \textit{om.} M Fr21 \textbf{30} Gahmuret] Gahmvret \sout{wa} G Gamvret O Gahmuͯret L gamuret M Z (Fr44) Gamúert Q Gahmoret Fr21 \newline
\end{minipage}
\hspace{0.5cm}
\begin{minipage}[t]{0.5\linewidth}
\small
\begin{center}*T (U)
\end{center}
\begin{tabular}{rl}
 & wem hât sîn menlîch zuht\\ 
 & hie gelâzen sîner minne vruht?\\ 
 & owê, lieplîche geselleschaft!\\ 
 & sol nû riuwe mi\textit{t} ir kraft\\ 
5 & iemer twingen mînen lîp?\\ 
 & sîme gote zuo êren", sprach daz wîp,\\ 
 & "ich mich gerne toufen solte\\ 
 & und leben, wie er wolte."\\ 
 & de\textit{r} jâmer gap ir herze\textit{n} wîc.\\ 
10 & ir \textbf{triuwe} vant den dürren zwîc,\\ 
 & als noch diu turteltûbe tuot.\\ 
 & diu hete ie den selben muot.\\ 
 & we\textit{nn} ir an \textbf{trûtschefte} gebrast,\\ 
 & ir triuwe kôs den dürren ast.\\ 
15 & diu vrouwe an rehter zît genas\\ 
 & eines sunes, der zweier varwe was,\\ 
 & an dem got \textbf{wunders} wart in ein:\\ 
 & wîz und swarz\textit{er} varwe er schein.\\ 
 & diu künegîn kustin sunder twâl\\ 
20 & vil dicke an sîniu blanken mâl.\\ 
 & diu muoter hiez ir kindelîn\\ 
 & Ferefis Anschevin.\\ 
 & \textbf{d\textit{e}r} wart ein waltsw\textit{e}nde.\\ 
 & die joste \textbf{von} sîner hende\\ 
25 & \textbf{vil} manec sper zerbrâchen,\\ 
 & \textbf{die} schilte \textbf{durchstâchen}.\\ 
 & \begin{large}A\end{large}ls ein agelster wart \textbf{er} \textbf{var},\\ 
 & sîn \textbf{vel} und ouch sîn \textbf{hâr vil gar}.\\ 
 & \textbf{nû} was \textbf{ouch} über des jâres zil,\\ 
30 & daz Gahmuret geprîset vil\\ 
\end{tabular}
\scriptsize
\line(1,0){75} \newline
U V W T \newline
\line(1,0){75} \newline
\textbf{9} \textit{Majuskel} T  \textbf{15} \textit{Überschrift:} Hie ward ferafis gamurettes sun von antschowe geborn W   $\cdot$ \textit{Platz für Illustration ausgespart} W   $\cdot$ \textit{Initiale} W T  \textbf{19} \textit{Majuskel} T  \textbf{27} \textit{Initiale} U W  \textbf{29} \textit{Initiale} V   $\cdot$ \textit{Majuskel} T  \newline
\line(1,0){75} \newline
\textbf{1} wem] Wenn W \textbf{2} hie gelâzen] verlâzen T  $\cdot$ minne] minnen T \textbf{3} lieplîche] [wiplich]: lieplich T \textbf{4} nû riuwe] mir nun reúwe W mir triuwe T  $\cdot$ mit] mir U \textbf{5} \textit{Versfolge 57.6-5} W  \textbf{7} enpfah ich sines tôvfes ê T \textbf{8} got gebe daz >ez< mir wol ergê T \textbf{9} der] Den U  $\cdot$ herzen] herze U \textbf{10} triuwe] vrovde T  $\cdot$ vant] [*]: kos V \textbf{12} hete] hat T \textbf{13} wenn] Wem U Swenne V (T) \textbf{14} ir triuwe kôs] so kôs si îe T  $\cdot$ ast] nast W \textbf{15} an] zuͦ V (T) \textbf{16} varwe] varwen V \textbf{17} wart] wirt W \textbf{18} wîz] Weisser W  $\cdot$ swarzer] swarz U  $\cdot$ varwe] warwe T \textbf{19} \textit{Versfolge 57.27-28-19} V  \textbf{21} ir] daz T \textbf{22} Ferefis] Feirefiz U Fereuis V Alsus ferafiß W Fereifiz T  $\cdot$ Anschevin] Anscheuin V antscheuin W Anscevin T \textbf{23} der] Dir U  $\cdot$ waltswende] walt swinde U \textbf{24} von] \textit{om.} W T \textbf{25} vil manec sper] der sper vil T \textbf{26} durchstâchen] durch stochen V (W) dvrkel stachen T \textbf{27} er] \sout{er} T  $\cdot$ var] [*]: gevar V geuar W \textbf{28} sîn vel und ouch sîn hâr] sin har vnd oͮch sin vel V (T)  $\cdot$ vil] \textit{om.} W \textbf{29} ouch] ez ovch T  $\cdot$ des] der W \textbf{30} Gahmuret] Gahmuͦret U Gamuret V (W) \newline
\end{minipage}
\end{table}
\end{document}
