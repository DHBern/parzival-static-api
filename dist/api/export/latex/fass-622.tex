\documentclass[8pt,a4paper,notitlepage]{article}
\usepackage{fullpage}
\usepackage{ulem}
\usepackage{xltxtra}
\usepackage{datetime}
\renewcommand{\dateseparator}{.}
\dmyyyydate
\usepackage{fancyhdr}
\usepackage{ifthen}
\pagestyle{fancy}
\fancyhf{}
\renewcommand{\headrulewidth}{0pt}
\fancyfoot[L]{\ifthenelse{\value{page}=1}{\today, \currenttime{} Uhr}{}}
\begin{document}
\begin{table}[ht]
\begin{minipage}[t]{0.5\linewidth}
\small
\begin{center}*D
\end{center}
\begin{tabular}{rl}
\textbf{622} & \begin{large}D\end{large}es was im nôt an der zît.\\ 
 & ir mantel unt \textbf{sîn} kursît\\ 
 & leit an sich hêr Gawan;\\ 
 & si truogez harnasch her dan.\\ 
5 & allêrst diu herzoginne clâr\\ 
 & nam sînes antlützes war,\\ 
 & dâ si sâzen bî ein ander.\\ 
 & \textbf{zwêne} gebrâtene galander,\\ 
 & mit wîne ein glesîn barel\\ 
10 & unt zwei blankiu wastel\\ 
 & diu süeze magt dar \textbf{nâher} truoc\\ 
 & ûf einer \textbf{tweheln wîz} genuoc.\\ 
 & \textbf{Dise} spîse ervloug ein sprinzelîn.\\ 
 & Gawan unt diu herzogîn\\ 
15 & mohtenz wazzer \textbf{selbe} nemen,\\ 
 & ob \textbf{twahens wolde si} \textbf{gezemen};\\ 
 & daz si \textbf{doch} bêdiu tâten.\\ 
 & mit vreude er was berâten,\\ 
 & daz er mit ir ezzen solde,\\ 
20 & durch die er lîden wolde\\ 
 & beidiu vreude und nôt.\\ 
 & swenne si \textbf{daz barel im} \textbf{gebôt},\\ 
 & daz gerüeret het ir munt,\\ 
 & sô wart im \textbf{niwe} vreude kunt,\\ 
25 & daz er \textbf{dâ} nâch solde trinken.\\ 
 & sîn ruwe begunde hinken\\ 
 & unt wart sîn hôch gemüete snel.\\ 
 & ir \textbf{süezer} munt, ir liehtez vel\\ 
 & in \textbf{sô} von kumber jagete,\\ 
30 & daz er neheine wunden klagete.\\ 
\end{tabular}
\scriptsize
\line(1,0){75} \newline
D Z Fr16 Fr68 \newline
\line(1,0){75} \newline
\textbf{1} \textit{Initiale} D Z Fr68  \textbf{13} \textit{Majuskel} D  \newline
\line(1,0){75} \newline
\textbf{1} Des] Iz Fr68  $\cdot$ der] dirre Fr68 \textbf{3} leit] leite Fr68 \textbf{6} sînes] gawans Fr16 \textbf{8} zwêne] Dri Z \textbf{11} nâher] nach Z \textbf{12} tweheln wîz] wizzen tweheln Z \textbf{13} Dise] Die Z (Fr16) (Fr68) \textbf{15} mohtenz] Mohtens Fr16 muͦzen Fr68  $\cdot$ selbe] selben Fr68 \textbf{17} doch] \textit{om.} Z \textbf{18} vreude] freuden Z \textbf{20} lîden] le::: Fr16 \textbf{22} s::enne si ir ::: Fr16 \textbf{23} daz] daz da Fr16 \textbf{24} \textit{nach 622.24:} er besach die s::: / ir guetlich s::: / er clagete nih::: / er vand an ir::: Fr16  \textbf{25} dâ nâch] nach ir Fr16 \textbf{26} \textit{nach 622.26:} als der einen v::: / wurfe an sin ::: / sin ougen be::: / sin herze geg::: Fr16  \newline
\end{minipage}
\hspace{0.5cm}
\begin{minipage}[t]{0.5\linewidth}
\small
\begin{center}*m
\end{center}
\begin{tabular}{rl}
 & des was im nôt an der zît.\\ 
 & ir mantel und \textbf{sîn} kursît\\ 
 & leit an sich hêr Gawan;\\ 
 & si truoc daz harnasch her dan.\\ 
5 & allerêrst diu herz\textit{o}gîn clâr\\ 
 & nam \dag Gawan\dag  ant\textit{litzes} war,\\ 
 & dô si sâzen bî ein ander.\\ 
 & \textbf{zwêne} gebrâten galander,\\ 
 & mit \textit{wîn} ein gle\textit{s}în barel\\ 
10 & und zwei blankiu wastel\\ 
 & diu süeze maget dar \textbf{nâher} truoc\\ 
 & ûf einer \textbf{tweheln wîz} genuoc.\\ 
 & \textbf{die} spîse ervlouc ein sprinzelîn.\\ 
 & Gawan und diu herzogîn\\ 
15 & mohtenz wazzer \textbf{selbe} nemen,\\ 
 & ob \textbf{twahens si wolte} \textbf{gezemen};\\ 
 & daz si \textbf{doch} beidiu tât\textit{e}n.\\ 
 & mit vröude er was berâten,\\ 
 & daz er mit ir ezze\textit{n} solte,\\ 
20 & durch die er lîden wolte\\ 
 & beidiu vröude und nôt.\\ 
 & wan si \textbf{im daz barel} \textbf{bôt},\\ 
 & daz gerüeret he\textit{t} \textit{i}r munt,\\ 
 & sô wart im \textbf{niuwe} vröude kunt,\\ 
25 & daz er nâch \textbf{ir} solte trinken.\\ 
 & sîn riuwe begunde hinken\\ 
 & und wart sîn hôch gemüete snel.\\ 
 & ir \textbf{süezer} munt, ir liehtez vel\\ 
 & in von kumber jagete,\\ 
30 & daz er dekein wunde klagete.\\ 
\end{tabular}
\scriptsize
\line(1,0){75} \newline
m n o \newline
\line(1,0){75} \newline
\newline
\line(1,0){75} \newline
\textbf{2} sîn] ir o \textbf{5} herzogîn] herczigin m \textbf{6} antlitzes] antwurt m antlitz n anczlit o \textbf{9} Mit ein gleslin barel m \textbf{15} mohtenz wazzer] Moͯchtens wasser n Mochtens wassers o \textbf{16} \textit{Verse 622.16-19 kontrahiert zu:} Ob twahens essen [wolte]: solte o  \textbf{17} tâten] tattan m \textbf{18} vröude] freiden n \textbf{19} ezzen] essel m \textbf{23} het ir] het in ir m \textbf{30} dekein] do keine n  $\cdot$ wunde] wurde o \newline
\end{minipage}
\end{table}
\newpage
\begin{table}[ht]
\begin{minipage}[t]{0.5\linewidth}
\small
\begin{center}*G
\end{center}
\begin{tabular}{rl}
 & des was im nôt an der zît.\\ 
 & ir mandel unde \textbf{ir} kursît\\ 
 & leit an sich hêr Gawan;\\ 
 & si truoc daz harnasch her dan.\\ 
5 & alrêrst diu herzoginne clâr\\ 
 & nam sînes antlitzes war,\\ 
 & dâ si sâzen bî ein ander.\\ 
 & \textbf{drî} gebrâten galander,\\ 
 & mit wîn ein glesîn barel\\ 
10 & unde zwei blankiu wastel\\ 
 & diu süeziu maget dar \textbf{nâch} truoc\\ 
 & ûf einer \textbf{wîzen dwehelen} genuoc.\\ 
 & \textbf{die} spîse ervlouc ein sprinzelîn.\\ 
 & Gawan unde diu herzogîn\\ 
15 & mohtenz wazzer \textbf{selbe} nemen,\\ 
 & ob \textbf{si dwahens wolde} \textbf{zemen};\\ 
 & daz si \textbf{dô} beidiu tâten.\\ 
 & mit vröude er was berâten,\\ 
 & daz er mit ir ezzen solde,\\ 
20 & durch die er lîden wolde\\ 
 & beidiu vröude unde nôt.\\ 
 & swenne si \textbf{daz parel im} \textbf{gebôt},\\ 
 & daz gerüeret het ir munt,\\ 
 & sô wart im \textbf{niuwan} vröude kunt,\\ 
25 & daz er \textbf{dar} nâch solde trinken.\\ 
 & sîn riuwe begunde hinken\\ 
 & unde wart sîn hôch gemüete snel.\\ 
 & ir \textbf{süezer} munt, ir liehtez vel\\ 
 & in \textbf{sô} von kumber jagete,\\ 
30 & daz er dehein wunden klagete.\\ 
\end{tabular}
\scriptsize
\line(1,0){75} \newline
G I L M Z \newline
\line(1,0){75} \newline
\textbf{1} \textit{Initiale} I L Z  \textbf{21} \textit{Initiale} I  \newline
\line(1,0){75} \newline
\textbf{2} unde ir] vnd sin L (M) \textbf{3} leit] Leite M  $\cdot$ hêr Gawan] ergawan M \textbf{9} glesîn] gleselin L \textbf{10} blankiu wastel] planckie pastel L \textbf{12} ûf] in I  $\cdot$ wîzen dwehelen] [t*]: twehel wiz I twahel wisz L \textbf{15} mohtenz] Musten das M  $\cdot$ selbe] selben M \textbf{16} Ob twahensz wolte sie gezemen L (M) (Z)  $\cdot$ wolde] wil I \textbf{17} dô] daz L da M \textit{om.} Z \textbf{18} vröude] vroudin M (Z) \textbf{20} lîden] haben L \textbf{22} swenne] Wenne L (M)  $\cdot$ daz parel im gebôt] daz parel bot I ymsz barel bot L \textbf{23} daz] daz ir da I  $\cdot$ ir] den L \textbf{24} sô] do I  $\cdot$ niuwan] nuͯwe L (M) (Z) \textbf{26} riuwe] triwe I \textbf{28} liehtez] lýchtes L (M) \textbf{29} sô] sol L \newline
\end{minipage}
\hspace{0.5cm}
\begin{minipage}[t]{0.5\linewidth}
\small
\begin{center}*T
\end{center}
\begin{tabular}{rl}
 & \begin{large}D\end{large}es was im nôt an der zît.\\ 
 & ir mantel und \textbf{sîn} kursît\\ 
 & leget an sich hêr Gawan;\\ 
 & si truoc daz harnasch her dan.\\ 
5 & alrêst diu herzoginne clâr\\ 
 & nam sînes antlitzes war,\\ 
 & dô si sâzen bî ein ander.\\ 
 & \textbf{zwêne} gebrâtene galander,\\ 
 & mit wîne ein glesîn barel\\ 
10 & und zwei blankiu wastel\\ 
 & diu süeze maget dâ \textbf{nâch} truoc\\ 
 & ûf einer \textbf{tweheln wîz} genuoc.\\ 
 & \textbf{die} spîse ervlouc ein sprinzelîn.\\ 
 & Gawan und diu herzogîn\\ 
15 & mohten daz wazzer \textbf{gerne} nemen,\\ 
 & ob \textbf{twahens si wolt\textit{e}} \textbf{gezemen};\\ 
 & daz si \textbf{dô} beidiu tâten.\\ 
 & mit vreuden er was berâten,\\ 
 & daz er mit ir ezzen solte,\\ 
20 & durch die er lîden wolte\\ 
 & beidiu vreude und nôt.\\ 
 & wan si \textbf{daz barel im} \textbf{bôt},\\ 
 & daz geruort hâte ir munt,\\ 
 & sô wart im \textbf{niuwe} vreude kunt,\\ 
25 & daz er \textbf{dâ} nâch solte trinken.\\ 
 & sîn \textit{riuw}e begunde hinken\\ 
 & und wart sîn hôchgemüete snel.\\ 
 & ir \textbf{rôter} munt, ir liehtez vel\\ 
 & in \textbf{sô} von kumber jagete,\\ 
30 & daz er keine wunde klagete.\\ 
\end{tabular}
\scriptsize
\line(1,0){75} \newline
U V W Q R Fr39 \newline
\line(1,0){75} \newline
\textbf{1} \textit{Initiale} U V W Q Fr39  \newline
\line(1,0){75} \newline
\textbf{1} Des] Es Q \textbf{2} sîn] ir Q \textbf{3} leget] Leite V (W) (Q) (Fr39)  $\cdot$ hêr] \textit{om.} R  $\cdot$ Gawan] gawann Q \textbf{4} daz] [*]: daz V des W Fr39 den R \textbf{6} nam sînes] [N*]: Nam gawanz V \textbf{7} dô] Da R Fr39 \textbf{9} barel] [bar*lin]: ein barel R \textbf{14} Gawan] Gawin R \textbf{15} mohten daz] Moͤhtens V Mochten Q Mohtens Fr39  $\cdot$ gerne] selber W Q (R) (Fr39) \textbf{16} twahens] twahen Q  $\cdot$ si wolte] sie wolten U wolte sv́ V (W) (Q) (R) (Fr39) \textbf{17} dô] da V  $\cdot$ beidiu] beten Q \textbf{18} vreuden] froͤide V (W)  $\cdot$ er was] waz [*]: er V was er R  $\cdot$ berâten] [*]: beraten V gebraten W \textbf{19} solte] solden Q \textbf{20} durch] duch Fr39  $\cdot$ wolte] wolden Q \textbf{22} wan] Swenne V R (Fr39)  $\cdot$ daz] des R  $\cdot$ im] ir W  $\cdot$ bôt] gebot W Q R Fr39 \textbf{24} vreude] frewden Q \textbf{25} dâ nâch] [na*]: nach ir V \textbf{26} sîn] si Fr39  $\cdot$ riuwe] vreide U \textbf{28} liehtez] lichtes Q liettes R \textbf{29} sô] also W \textbf{30} keine wunde] keine wunden V (W) (Fr39) deheinen kumber R \newline
\end{minipage}
\end{table}
\end{document}
