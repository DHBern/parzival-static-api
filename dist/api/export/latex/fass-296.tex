\documentclass[8pt,a4paper,notitlepage]{article}
\usepackage{fullpage}
\usepackage{ulem}
\usepackage{xltxtra}
\usepackage{datetime}
\renewcommand{\dateseparator}{.}
\dmyyyydate
\usepackage{fancyhdr}
\usepackage{ifthen}
\pagestyle{fancy}
\fancyhf{}
\renewcommand{\headrulewidth}{0pt}
\fancyfoot[L]{\ifthenelse{\value{page}=1}{\today, \currenttime{} Uhr}{}}
\begin{document}
\begin{table}[ht]
\begin{minipage}[t]{0.5\linewidth}
\small
\begin{center}*D
\end{center}
\begin{tabular}{rl}
\textbf{296} & \begin{large}P\end{large}arzival, der valscheit swant,\\ 
 & sîn triwe \textbf{in lêrte}, daz er vant\\ 
 & \textbf{snêwec} bluotes zeher drî,\\ 
 & die in \textbf{vor witzen machten} vrî.\\ 
5 & \textbf{sîne gedanke} umben Grâl\\ 
 & unt der küneginne glîchiu mâl,\\ 
 & iewederz was ein strengiu nôt.\\ 
 & an im wac vür der minnen lôt.\\ 
 & \textbf{wan} trûren und minne\\ 
10 & brichet zæhe sinne.\\ 
 & sol \textbf{diz} âventiure sîn?\\ 
 & \textbf{si} m\textit{ö}hten bêde heizen pîn.\\ 
 & Küene liute solten Keien nôt\\ 
 & klagen. sîn manheit im gebôt\\ 
15 & genendeclîche an manegen strît.\\ 
 & man saget in manegen landen wît,\\ 
 & daz Keie, Artuses scheneschalt,\\ 
 & mit siten wære ein ribalt.\\ 
 & des \textbf{sagent} in \textbf{mîniu} mære blôz:\\ 
20 & er was der werdecheit genôz.\\ 
 & swie kleine ich des die volge hân,\\ 
 & getriwe unt ellenthaft \textbf{ein} man\\ 
 & was Keie, des giht mîn munt.\\ 
 & ich tuon \textbf{ouch mêre von im} kunt.\\ 
25 & Artuses hof was ein zil,\\ 
 & dar \textbf{kom} vremder \textbf{ritter} vil,\\ 
 & die werden unt die smæhen,\\ 
 & mit siten die wæhen.\\ 
 & \begin{large}S\end{large}welher partierens pflac,\\ 
30 & \textbf{der selbe Keien} ringe wac;\\ 
\end{tabular}
\scriptsize
\line(1,0){75} \newline
D \newline
\line(1,0){75} \newline
\textbf{1} \textit{Initiale} D  \textbf{13} \textit{Majuskel} D  \textbf{29} \textit{Initiale} D  \newline
\line(1,0){75} \newline
\textbf{12} möhten] mohten D \textbf{17} Artuses] Artvs D \textbf{25} Artuses] Artvs D \newline
\end{minipage}
\hspace{0.5cm}
\begin{minipage}[t]{0.5\linewidth}
\small
\begin{center}*m
\end{center}
\begin{tabular}{rl}
 & \begin{large}P\end{large}arcifal, der valscheit swan\textit{t},\\ 
 & sîn triuwe \textbf{lêrt in}, daz er vant\\ 
 & \textbf{sîne} \textbf{snêwige} bluotes zeher drîe,\\ 
 & die in \textbf{vor witze mach\textit{t}en} vrîe.\\ 
5 & \textbf{sîne gedank} umb den Grâl\\ 
 & und der künig\textit{în} glîchiu mâl,\\ 
 & ietwederz was ein strengiu nôt.\\ 
 & an ime wac vür der min\textit{n}en lôt.\\ 
 & trûren und minne\\ 
10 & brichet zæhe sinne.\\ 
 & sol \textbf{diz} âventiure sîn?\\ 
 & \textbf{si} m\textit{ö}hte\textit{n} beide heizen pîn.\\ 
 & küene liute solten Keien nôt\\ 
 & klagen. sîn manheit im gebôt\\ 
15 & genendeclîch an manige\textit{n} strît.\\ 
 & man saget in manige\textit{n} lan\textit{den} \textit{w}ît,\\ 
 & daz Keie, Artuses schinschalt,\\ 
 & mit siten wære ein ribalt.\\ 
 & des \textbf{sagent} i\textit{n} \textbf{mîniu} mære blôz:\\ 
20 & e\textit{r} was der werdecheit genôz.\\ 
 & wie kleine ich des die volge hân,\\ 
 & getriuwe und ellenthaft \textbf{ein} man\\ 
 & was Keie, des giht \textbf{im} mîn munt.\\ 
 & ich tuon \textbf{iu mêre von ime} kunt.\\ 
25 & Artuses hof was ein zil,\\ 
 & dar \textbf{kôme\textit{n}} vr\textit{ö}meder \textbf{liute} vil,\\ 
 & die w\textit{e}rden und die smæhen,\\ 
 & mit siten die wæhen.\\ 
 & welher partierens pflac,\\ 
30 & \textbf{der selbe Keien} ringe wac;\\ 
\end{tabular}
\scriptsize
\line(1,0){75} \newline
m n o \newline
\line(1,0){75} \newline
\textbf{1} \textit{Initiale} m n o  \newline
\line(1,0){75} \newline
\textbf{1} swant] swang m \textbf{3} sîne snêwige] Sin sne n o \textbf{4} witze] witzen n  $\cdot$ machten] machen m \textbf{5} sîne] Sin n o \textbf{6} künigîn] kunige m  $\cdot$ glîchiu] glich n o \textbf{8} vür] der fúr n  $\cdot$ minnen] minen m \textbf{11} diz] [die]: dis m dise n o \textbf{12} möhten] mohtens m mochten o \textbf{13} Keien] keẏen n kerren o \textbf{14} klagen] Clage o \textbf{15} manigen] manigem m \textbf{16} manigen landen wît] manigem lant vnd srit m \textbf{17} Keie] keẏ n o  $\cdot$ Artuses] artus m n artuͯs o  $\cdot$ schinschalt] scun scalt m scym scalt n (o) \textbf{19} des] Das n o  $\cdot$ in] ym m \textbf{20} er] Es m \textbf{23} Keie] key n keyn o  $\cdot$ des] das n o  $\cdot$ giht] git o  $\cdot$ mîn] sin o \textbf{24} iu] jme n \textbf{25} Artuses] Artus m n Artuͯs o \textbf{26} kômen] koment m n kommer o  $\cdot$ vrömeder] frammeder m \textbf{27} werden] worden m \textbf{29} partierens] partiers n \textbf{30} selbe] selben o  $\cdot$ Keien] keẏen n (o) \newline
\end{minipage}
\end{table}
\newpage
\begin{table}[ht]
\begin{minipage}[t]{0.5\linewidth}
\small
\begin{center}*G
\end{center}
\begin{tabular}{rl}
 & \textit{P}arziva\textit{l}, \textit{der} valscheit swant,\\ 
 & sîn triwe \textbf{in lêrte}, daz er vant\\ 
 & \textbf{sîne} bluotes zeher drî,\\ 
 & die in \textbf{machten witze} vrî.\\ 
5 & \textbf{sîn pensieren} umbe den Grâl\\ 
 & unt der künigîn gelîchiu mâl,\\ 
 & ietwederz was ein strengiu nôt.\\ 
 & an im wac vür der minnen lôt.\\ 
 & trûren unde minne\\ 
10 & brichet zæhe sinne.\\ 
 & sol \textbf{daz} âventiure sîn?\\ 
 & \textbf{si} m\textit{ö}hten beidiu heizen pîn.\\ 
 & küene liute solten Kays nôt\\ 
 & klagen. sîn manheit im gebôt\\ 
15 & genendiclîche an manigen strît.\\ 
 & man saget in manigen landen wît,\\ 
 & daz Kay, Artuses seneschalt,\\ 
 & mit siten wære ein ribalt.\\ 
 & des \textbf{saget} in \textbf{mîn} mære blôz:\\ 
20 & er was der werdicheit genôz.\\ 
 & swie kleine ich des die volge hân,\\ 
 & getriwe unde ellenthafter man\\ 
 & was Kay, des giht \textbf{im} mîn munt.\\ 
 & ich tuon \textbf{noch mêr von im} kunt.\\ 
25 & Artuses hof was ein zil,\\ 
 & dar \textbf{kom} vremder \textbf{liute} vil,\\ 
 & \begin{large}D\end{large}ie werden unde die smæhen,\\ 
 & mit \textit{sit}en die wæhen.\\ 
 & swelher partierns pflac,\\ 
30 & \textbf{der selbe Kay} ringe wac;\\ 
\end{tabular}
\scriptsize
\line(1,0){75} \newline
G I O L M Q R Z \newline
\line(1,0){75} \newline
\textbf{3} \textit{Initiale} L  \textbf{5} \textit{Initiale} I Z  \textbf{9} \textit{Initiale} O  \textbf{27} \textit{Initiale} G  \newline
\line(1,0){75} \newline
\textbf{1} Parzival] ane parzivale G Parzifal I M Barzifal O Parcifal L Z Partzifal Q Parczifal R  $\cdot$ der] \textit{om.} G  $\cdot$ swant] verswant I \textbf{3} sîne] Snebich O (L) (Z) Ane M Schnewe R  $\cdot$ bluotes] bluͯtich L \textbf{4} machten witze] machte witze M Q vor witzen mahten Z \textbf{5} sîn pensieren] Sinin pensieren M Ein gedanc in pavsieren Z  $\cdot$ den] \textit{om.} R \textbf{6} gelîchiu] licht gelichte L  $\cdot$ mâl] gemal Q \textbf{7} ietwederz] Jetwerdez O  $\cdot$ was] wart Q  $\cdot$ strengiu] gestrenge M \textbf{8} vür] \textit{om.} Q  $\cdot$ minnen] minne I (M) (Q) R Z  $\cdot$ lôt] tot I (R) \textbf{9} trûren] ÷rvͤten O Vroűren Q \textbf{10} brichet] brechent I  $\cdot$ zæhe sinne] zahen shin I zuͯchte synne L zwey sinne Q \textbf{11} daz] dis L (M) (Z) \textbf{12} si möhten] si mohten G I Si mohtenz O (L) (M) (R) Sie mochtes Q So mohtenz Z  $\cdot$ beidiu heizen pîn] \textit{om.} M \textbf{13} solten] soltan clagen I solte L  $\cdot$ Kays] kaẏs G kains I [cha]: chein O kaýen L gein M key Q keyen R Z \textbf{14} Clagen wan im sin manheit gebot Z  $\cdot$ klagen] \textit{om.} I  $\cdot$ sîn manheit] sie machten M \textbf{15} genendiclîche] Genediglich thun Q  $\cdot$ manigen] mangem I \textbf{16} saget] sach L  $\cdot$ manigen landen] mangem lande I (Q) \textbf{17} Kay] kaẏ G kain I Key O (R) (Z) keie M  $\cdot$ Artuses] artus G M Q (R)  $\cdot$ seneschalt] sinschalt G schinishalt I senetschalt O sinetshalt L sinetscalt M senecschalt Q sine schalt R sinehtschalt Z \textbf{18} ribalt] [rýba]: rýbabalt L \textbf{19} des] Das Q R  $\cdot$ saget] sagiten M sagent Z  $\cdot$ in] Jm R  $\cdot$ mîn] >dis< O diz L (Q) (R) o\textit{m. } M mine Z \textbf{21} swie] Wie L (M) Q R Z  $\cdot$ volge] vogile M \textbf{22} getriwe] Getrewen Q  $\cdot$ ellenthafter] ellenthaften L [ellenthaffter]: erenthaffter Q ellenthaft ein Z \textbf{23} Kay] kaẏ G kain I Key O (R) (Z) kaý L keie M  $\cdot$ des] \textit{om.} L  $\cdot$ giht] iet M  $\cdot$ im] \textit{om.} O L M Q R Z \textbf{24} tuon] tuͤn ev I (R) tun ir Q  $\cdot$ mêr] mere O L R Z  $\cdot$ von im] \textit{om.} R \textbf{25} Artuses] Artus I Q R Z \textbf{26} kom] qwamen M (R)  $\cdot$ vremder liute] froden lútten R \textbf{27} unde] \textit{om.} R \textbf{28} siten] varwen G \textbf{29} swelher] Welcher L (M) Q (R) \textbf{30} selbe] selben O \textit{om.} Z  $\cdot$ Kay] kaẏ G chain I key O R kaý L gein M keyn Q (Z)  $\cdot$ ringe] [krîegens]: krîegen O kriegen L \newline
\end{minipage}
\hspace{0.5cm}
\begin{minipage}[t]{0.5\linewidth}
\small
\begin{center}*T
\end{center}
\begin{tabular}{rl}
 & Parcifal, der valscheit swant,\\ 
 & sîn triuwe \textbf{lêrtin}, daz er vant\\ 
 & \multicolumn{1}{l}{ - - - }\\ 
 & \multicolumn{1}{l}{ - - - }\\ 
5 & \textbf{sîn pensieren} umbe den Grâl\\ 
 & unde der küneginne glîchiu mâl,\\ 
 & ietwederz was ein streng\textit{iu} nôt.\\ 
 & an im wac vür der minnen lôt.\\ 
 & trûren unde minne\\ 
10 & brichet zæhe sinne.\\ 
 & sol \textbf{diz} âventiure sîn?\\ 
 & \textbf{sô} m\textit{ö}hten beide \textbf{wol} heizen \textit{pîn}.\\ 
 & \begin{large}K\end{large}üene liute solten Keys nôt\\ 
 & klagen. sîn manheit im gebôt\\ 
15 & genendeclîche an manegen strît.\\ 
 & man saget in manegen landen wît,\\ 
 & daz Key, Artuses seneschalt,\\ 
 & mit siten wære ein ribalt.\\ 
 & des \textbf{saget} in \textbf{diz} mære blôz:\\ 
20 & er was der werdecheit genôz.\\ 
 & swie kleine ich des di\textit{e} volge hân,\\ 
 & getriuwe unde ellenthaf\textit{t} \textbf{ein} man\\ 
 & was Key, des giht \textbf{im} mîn munt.\\ 
 & ich tuon \textbf{von im die mære} kunt.\\ 
25 & Artuses hof was ein zil,\\ 
 & dar \textbf{kom} vremder \textbf{liute} vil,\\ 
 & die werden unde die smæhen,\\ 
 & mit siten die wæhen.\\ 
 & swelher partiernes pflac,\\ 
30 & \textbf{den selben Key} ringe wac;\\ 
\end{tabular}
\scriptsize
\line(1,0){75} \newline
T U V W \newline
\line(1,0){75} \newline
\textbf{13} \textit{Initiale} T U W  \newline
\line(1,0){75} \newline
\textbf{1} Parcifal] parzifal T (V) Partzifal W  $\cdot$ swant] \sout{vrî} swant T vri U [*nt]: swant V \textbf{2} Dannoch wonete beide bi U  $\cdot$ lêrtin] [*]: lert in V lert in W \textbf{3} Snewig blvͦtig tropfen dri V  $\cdot$ Schne bluͦtig vnd zehere drei W \textbf{4} Die in mahten witzen (witze W ) fri V (W) \textbf{5} \textit{Versfolge 296.6-5} V   $\cdot$ pensieren] gedenken V bansinieren W \textbf{7} ietwederz] Jeweders V  $\cdot$ strengiu] strenge T \textbf{8} wac] was W  $\cdot$ der minnen lôt] den minen lot U der [minn*]: minne tot V \textbf{10} zæhe] [*]: zehe V \textbf{11} diz] dise U dis ein W \textbf{12} sô] Sie U (V) (W)  $\cdot$ möhten] mohten T (U) moͤchtens W  $\cdot$ beide wol] wol beide V baide W  $\cdot$ heizen] hertzen W  $\cdot$ pîn] \textit{om.} T \textbf{13} Keys] kein U V keine W \textbf{15} genendeclîche] Genenneclich U [G*]: Genendecliche V Gnediglichen W  $\cdot$ manegen] mangem W \textbf{16} landen] luten U \textbf{17} Key] kein V  $\cdot$ Artuses] artvs T (W) \textbf{19} des] Das W  $\cdot$ saget] sagent V  $\cdot$ diz] dise U [*]: dise V mein W \textbf{21} swie] Wie U W  $\cdot$ des] es W  $\cdot$ die] div T \textbf{22} ellenthaft] ellenthafte T ellentschafter U ellenthaffter W  $\cdot$ ein] \textit{om.} U W \textbf{23} Key] keyn V \textbf{24} die] dis W \textbf{25} Artuses] Artuͦses U \textbf{26} kom] kamen W \textbf{29} swelher] Welcher U W  $\cdot$ partiernes] parrierens V \textbf{30} Key] kein V \newline
\end{minipage}
\end{table}
\end{document}
