\documentclass[8pt,a4paper,notitlepage]{article}
\usepackage{fullpage}
\usepackage{ulem}
\usepackage{xltxtra}
\usepackage{datetime}
\renewcommand{\dateseparator}{.}
\dmyyyydate
\usepackage{fancyhdr}
\usepackage{ifthen}
\pagestyle{fancy}
\fancyhf{}
\renewcommand{\headrulewidth}{0pt}
\fancyfoot[L]{\ifthenelse{\value{page}=1}{\today, \currenttime{} Uhr}{}}
\begin{document}
\begin{table}[ht]
\begin{minipage}[t]{0.5\linewidth}
\small
\begin{center}*D
\end{center}
\begin{tabular}{rl}
\textbf{503} & \begin{large}E\end{large}z næht nû wilden mæren,\\ 
 & die vreuden kunnen læren\\ 
 & unt die hôch gemüete bringent;\\ 
 & mit den bêden si ringent.\\ 
5 & nû was ez \textbf{ouch} über des jâres zît.\\ 
 & gescheiden was des kampfes strît,\\ 
 & den der \textbf{lantgrâve} zem Plimizœl\\ 
 & erwarp. der was ze Barbigœl\\ 
 & von Schanpfanzun gesprochen;\\ 
10 & dâ beleip ungerochen\\ 
 & der künec Kingrisin.\\ 
 & Vergulaht, der sun sîn,\\ 
 & kom gein Gawane dar.\\ 
 & dô nam diu werlt ir sippe war\\ 
15 & und schiet den kampf ir \textbf{sippe} maht,\\ 
 & wand ouch der grâve Ehkunaht\\ 
 & ûf im die \textbf{grôzen} schulde truoc,\\ 
 & der man \textbf{Gawanen zêch} genuoc.\\ 
 & \textbf{des} verkôs Kingrimursel\\ 
20 & ûf Gawanen, den degen snel.\\ 
 & si vuoren beide sunder dan,\\ 
 & Vergulaht unt Gawan,\\ 
 & an dem selbem mâle\\ 
 & durch vorschen nâch dem Grâle.\\ 
25 & \textbf{al dâ} si mit ir henden\\ 
 & manege tjoste muosen senden.\\ 
 & wan swers Grâles gerte,\\ 
 & d\textit{er} muose mit dem swerte\\ 
 & sich dem prîse nâhen.\\ 
30 & sus sol man prîses gâhen.\\ 
\end{tabular}
\scriptsize
\line(1,0){75} \newline
D \newline
\line(1,0){75} \newline
\textbf{1} \textit{Großinitiale} D  \newline
\line(1,0){75} \newline
\textbf{7} Plimizœl] Plimizol D \textbf{8} Barbigœl] Barbigol D \textbf{9} Schanpfanzun] [T*hanfanzvn]: Tschanfanzvn D \textbf{16} Ehkunaht] Ehcvnaht D \textbf{28} der] do D \newline
\end{minipage}
\hspace{0.5cm}
\begin{minipage}[t]{0.5\linewidth}
\small
\begin{center}*m
\end{center}
\begin{tabular}{rl}
 & \textit{\begin{large}E\end{large}}z nâhet nû wilden mæren,\\ 
 & die vröude ku\textit{nn}en læren\\ 
 & und die hôchgemüete bringen\textit{t};\\ 
 & mit den beiden si ringen\textit{t}.\\ 
5 & nû was ez \textbf{ouch} über des jâres zît.\\ 
 & gescheiden was des kampfes strît,\\ 
 & den der \textbf{lantgrâve} zuom Plimizol\\ 
 & erwarp. der was zuo Barbigol\\ 
 & von Schanf\textit{a}nzu\textit{n} gesprochen;\\ 
10 & dô beleip ungerochen\\ 
 & der \textbf{werde} künic Kingrisin.\\ 
 & Vergul\textit{a}ht, der sun sîn,\\ 
 & kam gegen Gawane dar.\\ 
 & dô nam diu werlt ir sippe war\\ 
15 & und schiet den kampf ir \textbf{site} maht,\\ 
 & \textit{wenne ouch der grâve} \textit{Ehkunaht}\\ 
 & ûf im die \textbf{grôzen} schulde truoc,\\ 
 & der man \textbf{zêch Gawan} genuoc.\\ 
 & \textbf{daz} verkôs Kingrimursel\\ 
20 & ûf Gawanen, den degen snel.\\ 
 & si vuoren beide sunder dan,\\ 
 & Vergulaht und Gawan,\\ 
 & an dem selben mâl\\ 
 & durch vorschen nâch dem Grâl.\\ 
25 & \textbf{alsô} si mit ir henden\\ 
 & manige juste muosten senden.\\ 
 & wan wer des Grâles gerte,\\ 
 & de\textit{r} muoste mit dem swerte\\ 
 & sich dem prîse nâhen.\\ 
30 & sus sol man prîses gâhen.\\ 
\end{tabular}
\scriptsize
\line(1,0){75} \newline
m n o \newline
\line(1,0){75} \newline
\textbf{1} \textit{Überschrift:} Hie vohet an wunderlich offentuͯre also gawan gon orgeleise (origeleyse n  ) kam m (n) (o)   $\cdot$ \textit{Illustration} m o   $\cdot$ \textit{Initiale} m n o  \newline
\line(1,0){75} \newline
\textbf{1} \textit{Versdoppelung 503.1-2 (²m) nach 503.1 (gestrichen); Lesarten der vorausgehenden Verse mit ¹m bezeichnet} m   $\cdot$ Ez] EEs \textsuperscript{1}\hspace{-1.3mm} m DEs o  $\cdot$ wilden] wilder o \textbf{2} kunnen] koment \textsuperscript{1}\hspace{-1.3mm} m \sout{koment} \textsuperscript{2}\hspace{-1.3mm} m \textbf{3} bringent] bringen m n (o) \textbf{4} ringent] ringen m n o \textbf{5} ez] \textit{om.} n  $\cdot$ des] das o \textbf{7} den der] Den den n  $\cdot$ Plimizol] plúmzol n \textbf{9} Schanfanzun] scanffenzvͯm m scanfanzun n scanffancÿm o \textbf{12} Vergulaht] Verguleht m Verguhelacht n Vergulat o \textbf{16} \textit{Vers 503.16 fehlt} m   $\cdot$ Ehkunaht] ecknacht n echkunacht o \textbf{17} grôzen] grosse o \textbf{18} Gawan] gawanen n gawannen o \textbf{19} Kingrimursel] kingrumúrsel n konigrumel o \textbf{22} Vergulaht] Vergulacht n Vergulat o \textbf{23} selben] sellen n \textbf{24} vorschen] furchten o \textbf{25} alsô si] Aldo do sú n Aldo o \textbf{26} muosten] muͯste o \textbf{27} des] das o  $\cdot$ gerte] gert o \textbf{28} der] Dem m \newline
\end{minipage}
\end{table}
\newpage
\begin{table}[ht]
\begin{minipage}[t]{0.5\linewidth}
\small
\begin{center}*G
\end{center}
\begin{tabular}{rl}
 & \begin{large}E\end{large}z nâh\textit{e}t nû wilden mæren,\\ 
 & die vröuden kunnen læren\\ 
 & unde die hôchgemüete bringent;\\ 
 & mit den beiden si ringent.\\ 
5 & nû was ez \textbf{ouch} über des jâres zît,\\ 
 & \textbf{daz} gescheiden was des kampfes strît,\\ 
 & den der \textbf{lantgrâve} ze dem Blimzol\\ 
 & erwarp. der was ze Barbigol\\ 
 & von Tschanfenzun gesprochen;\\ 
10 & dâ beleip ungerochen\\ 
 & der künic Kingrisin.\\ 
 & Vergulaht, der sun sîn,\\ 
 & kom gein Gawane dar.\\ 
 & dô nam diu werlt ir sippe war\\ 
15 & unde schiet den kampf ir \textbf{sippe} maht,\\ 
 & wan ouch der grâve Ehkunaht\\ 
 & ûf im die \textbf{grôzen} schulde truoc,\\ 
 & der man \textbf{Gawanen zêch} genuoc.\\ 
 & \textbf{dô} verkôs Kingrimursel\\ 
20 & ûf Gawanen, den degen snel.\\ 
 & si vuoren bêde sunder dan,\\ 
 & Vergulaht unde Gawan,\\ 
 & an dem selben mâle\\ 
 & durch vorschen nâch dem Grâle.\\ 
25 & \textbf{al dâ} si mit ir henden\\ 
 & manige tjoste muosen senden.\\ 
 & wan swers Grâles gerte,\\ 
 & der muose mit dem swerte\\ 
 & sich dem brîse nâhen.\\ 
30 & sus sol man brîses gâhen.\\ 
\end{tabular}
\scriptsize
\line(1,0){75} \newline
G I L M Z Fr57 \newline
\line(1,0){75} \newline
\textbf{1} \textit{Überschrift:} Hie scheidet parcifal von siner muter bruder der ein heilic man was Vnd het im alle sine svnde gesagt Vnd het sie der selb man vf sich genomen zv bvzzen fvr in Vnd was bi im in der Closen gewesen vntz an den fvnfzehenden tac Vnd het newer krvt vnd wurtzelin gezzen Vnd ritet er nv von im Vnd hat rat genomen vmb den gral vnd vmb sache war er nu qvam daz lese man hin fvrbaz Z   $\cdot$ \textit{Initiale} G L M Z Fr57  \textbf{17} \textit{Initiale} I  \newline
\line(1,0){75} \newline
\textbf{1} nâhet] nahent G nehet M Z (Fr57)  $\cdot$ nû] \textit{om.} M \textbf{2} die] diu I  $\cdot$ vröuden kunnen] vreude chunnen I (L) (Z) (Fr57) kunen vrouden M \textbf{3} die] div Fr57  $\cdot$ hôchgemüete] hochmut M hochen mvͦt Fr57 \textbf{4} beiden] liden M \textbf{5} ez ouch] \textit{om.} M  $\cdot$ des] die I \textit{om.} L M \textbf{6} daz] \textit{om.} L M Z  $\cdot$ was] wart I \textbf{7} ze dem Blimzol] zuͤ dem blimizol I zvͯm plimizol L (M) (Z) zeblimizœl Fr57 \textbf{8} der] des Z  $\cdot$ ze Barbigol] zebarbigol G zvr barbigol Z ze barbigœl Fr57 \textbf{9} Tschanfenzun] tschanfanzun G (Z) schanfanzun I Tscanfenzvn L Schanphenzcun M tschempfezv́n Fr57 \textbf{10} dâ] do I \textbf{12} Vergulaht] Virgulat I Verguͯlaht L Vergulacht M \textbf{13} Gawane] Gawan I gawanen Z \textbf{14} dô] Da M  $\cdot$ werlt] wert L \textbf{16} Ehkunaht] ehkunat I echuͯnacht M Ehcunaht Z \textbf{17} grôzen] groze I \textbf{18} Gawanen zêch] Gawane zcoch M \textbf{19} dô] [Des]: Do G Des L M Z  $\cdot$ Kingrimursel] kingrymursel M \textbf{20} Gawanen] gawan M \textbf{22} Vergulaht] virgulaht I Verguͯlaht L Vergulacht M \textbf{25} al] Adir M \textbf{27} swers] wer des L (Z) so wer osz M \textbf{28} muose] muese G \textit{om.} M \textbf{30} brîses] pris L \newline
\end{minipage}
\hspace{0.5cm}
\begin{minipage}[t]{0.5\linewidth}
\small
\begin{center}*T
\end{center}
\begin{tabular}{rl}
 & \begin{Large}E\end{Large}z nâhet nû wilden mæren,\\ 
 & Die vröude kunnen læren\\ 
 & unde die hôchgemüete bringent;\\ 
 & mit den beiden si ringent.\\ 
5 & nû was ez über des \textit{j}â\textit{r}es zît.\\ 
 & gescheiden was des kampfes strît,\\ 
 & den der \textbf{grâve} zem Plymizol\\ 
 & erwarp. der was ze Barbigol\\ 
 & von Tschampfenzun gesprochen;\\ 
10 & dâ bleip ungerochen\\ 
 & der künec Kyngrisin.\\ 
 & Vergulaht, der sun sîn,\\ 
 & \textbf{der} kom gegen Gawane dar.\\ 
 & dô nam diu werlt ir sippe war\\ 
15 & unde schiet den kampf ir \textbf{sippe} maht,\\ 
 & wande ouch der grâve Ehkunaht\\ 
 & ûf i\textit{m} die \textbf{swæren} schulde truoc,\\ 
 & d\textit{er} man \textbf{Gawanen zêch} genuoc.\\ 
 & \textbf{daz} verkôs Kyngrimursel\\ 
20 & ûffe Gawanen, den degen snel.\\ 
 & si vuoren beide sunder dan,\\ 
 & Vergulaht unde Gawan,\\ 
 & an dem selben mâle\\ 
 & durch vorschen nâch dem Grâle.\\ 
25 & \textbf{aldâ} si mit ir henden\\ 
 & manege tjost muosen senden.\\ 
 & wande swer des Grâles gerte,\\ 
 & der muose mit dem swerte\\ 
 & sich dem prîse nâhen.\\ 
30 & sus sol man prîses gâhen.\\ 
\end{tabular}
\scriptsize
\line(1,0){75} \newline
T U V W O Q R Fr39 \newline
\line(1,0){75} \newline
\textbf{1} \textit{Überschrift:} Hie wúrt Gawan vnde der lantgrave versvͤnet mit [kyngri*]: kyngrisin vnde sime svne virgvlat vmbe den kamph Den sv́ soltent han getan ze schamfenzvn / Hie gan die wilden auentv́ren an vnde wie Gawan zvͦ orgelvsen kam V   $\cdot$ \textit{Überschrift:} Hie schiet partzifal von dem einsydel treuerissent vnd kumt dis mere wider an gawan W   $\cdot$ \textit{Großinitiale} T U R   $\cdot$ \textit{Initiale} V W O Fr39  \textbf{2} \textit{Majuskel} T  \textbf{5} \textit{Überschrift:} Nun ernerte her gawan den wúnden ritter der sint vntrewe an ym begieg Q  · Initiale Q  \newline
\line(1,0){75} \newline
\textbf{1} Ez] ÷z O \textbf{2} vröude kunnen] froͤden kúnnen W chvnnen frevde O frewden kunden Q \textbf{3} hôchgemüete] hohc gemvͦten O \textbf{4} beiden] besten R  $\cdot$ ringent] ringet V Q \textbf{5} ez] ovch O es auch Q (R) (Fr39)  $\cdot$ über] \textit{om.} R  $\cdot$ jâres] kampfes T (U) Fr39 \textbf{6} gescheiden] Geschehen U Daz gescheiden O \textbf{7} den] Do Q  $\cdot$ grâve] lantgrave U (V) O (R) (Fr39) [gr*]: lantgraffe  Q  $\cdot$ Plymizol] Plŷmôzol T plimizol V W Q R Fr39 brimizol O \textbf{8} der] er Q  $\cdot$ ze] zem V  $\cdot$ Barbigol] [Bargol]: Barbigol T \textbf{9} von] Vor V  $\cdot$ Tschampfenzun] Tschamfenzuͦn U schampfenzvn V tschampffenzun W tschamphezvn O schanpfen zún Q schanfenzun R Tschanfenzvn Fr39 \textbf{10} dâ] Do U W Q R (Fr39) \textbf{11} Kyngrisin] kingrisin V W (Fr39) kẏngrisin R \textbf{12} Vergulaht] Vergulacht U W Q Virgulat V Vorgulacht R ::rgulaht Fr39 \textbf{13} gegen] ge Q  $\cdot$ Gawane] Gawan U grawane R \textbf{14} dô] Da O \textbf{15} den] vom R der Fr39 \textbf{16} ouch] \textit{om.} O  $\cdot$ Ehkunaht] Echcuͦnach U [*]: ekunaht V ekunacht W Q Ehkunacht R \textbf{17} im] in T U  $\cdot$ swæren] swere U grossen W (O) (Fr39) grosse Q R \textbf{18} der] die T Do U  $\cdot$ Gawanen] gawane V Gawan O gawain R \textbf{19} daz] Des W R Fr39 Der Q  $\cdot$ Kyngrimursel] kingrimvrsel V (W) Kyngrumursel R kyn:rimursel Fr39 \textbf{20} Gawanen] Gawan O Gawain R \textbf{21} vuoren] fvͦrten O \textbf{22} Vergulaht] Vergulacht U W Q R Virgulat V  $\cdot$ Gawan] her gawan V Gawain R \textbf{23} selben] selbe Q [selb*m]: selbem Fr39 \textbf{24} vorschen] verschen U vorsehen O \textbf{26} tjost] strit R  $\cdot$ muosen] mvͤste V  $\cdot$ senden] seden R \textbf{27} swer] wer U W Q R \textbf{28} muose] mvese T muͦze U mvͤste V muͦß W  $\cdot$ dem] \textit{om.} R \textbf{30} sus] Als Q  $\cdot$ prîses gâhen] [*]: zvͦ prise gahen V preis entpfahen W strites gahen O pris gachen R \newline
\end{minipage}
\end{table}
\end{document}
