\documentclass[8pt,a4paper,notitlepage]{article}
\usepackage{fullpage}
\usepackage{ulem}
\usepackage{xltxtra}
\usepackage{datetime}
\renewcommand{\dateseparator}{.}
\dmyyyydate
\usepackage{fancyhdr}
\usepackage{ifthen}
\pagestyle{fancy}
\fancyhf{}
\renewcommand{\headrulewidth}{0pt}
\fancyfoot[L]{\ifthenelse{\value{page}=1}{\today, \currenttime{} Uhr}{}}
\begin{document}
\begin{table}[ht]
\begin{minipage}[t]{0.5\linewidth}
\small
\begin{center}*D
\end{center}
\begin{tabular}{rl}
\textbf{351} & \begin{large}D\end{large}â durch in \textbf{starkiu} angest sneit.\\ 
 & Gawan mitten durch si reit.\\ 
 & doch ieslîch zeltsnuor die andern dranc,\\ 
 & ir her was wît unde lanc.\\ 
5 & dô sach er, \textbf{wie} si lâgen,\\ 
 & wes dise unt jene pflâgen.\\ 
 & swer 'byen sey venûz' dâ sprach,\\ 
 & 'gramærzys' er wider jach.\\ 
 & grôz rotte an einem orte lac,\\ 
10 & sarjande von Semblidac;\\ 
 & \textbf{den lac} dâ sunder nâhen bî\\ 
 & turkopele \textbf{von} \textbf{Kaheti}.\\ 
 & unkünde dicke unminne sint.\\ 
 & sus reit des \textbf{künec} Lotes kint.\\ 
15 & \textbf{belîbens bete in} niemen bat.\\ 
 & \textbf{Gawan kêrte} gein der stat.\\ 
 & er dâhte: "sol ich \textbf{kipper} wesen,\\ 
 & ich mac vor \textbf{vlüste} \textbf{baz} genesen\\ 
 & dort in der stat dan hie bî in.\\ 
20 & i\textbf{ne} kêre mich an deheinen gewin,\\ 
 & wan \textbf{wie ich} daz mîne behalde,\\ 
 & sô \textbf{daz es} gelücke walde."\\ 
 & Gawan gein einer porten reit.\\ 
 & der burgære site was im leit:\\ 
25 & \textbf{si}\textbf{ne} hete niht betûwert,\\ 
 & alle ir porten wâren vermûwert\\ 
 & unt al ir wîchûs werlîch;\\ 
 & dar zuo der zinnen ieslîch\\ 
 & mit armbrüste ein schütze pflac,\\ 
30 & der sich schiezens her ûz bewac.\\ 
\end{tabular}
\scriptsize
\line(1,0){75} \newline
D \newline
\line(1,0){75} \newline
\textbf{1} \textit{Initiale} D  \newline
\line(1,0){75} \newline
\textbf{10} Semblidac] Semblidach D \textbf{14} Lotes] Lots D \newline
\end{minipage}
\hspace{0.5cm}
\begin{minipage}[t]{0.5\linewidth}
\small
\begin{center}*m
\end{center}
\begin{tabular}{rl}
 & dâ durch \textit{i}n \textbf{starkiu} angest sneit.\\ 
 & Gawan mitten durch si reit.\\ 
 & doch iegelîch zeltsnuor die anderen dranc,\\ 
 & ir her was wît und lanc.\\ 
5 & d\textit{ô} sach er, \textbf{wâ} si lâgen,\\ 
 & wes dise und jene pflâgen.\\ 
 & wer 'biense venûz' dô sprach,\\ 
 & 'gramerzîs' er wider jach.\\ 
 & grôz rote an einem orte lac,\\ 
10 & sarjande von Sembli\textit{d}ac;\\ 
 & \textbf{den lac} d\textit{â} sunder nâhe bî\\ 
 & turkopele \textbf{von} \textbf{Kaheti}.\\ 
 & unkünde dicke unminne sint.\\ 
 & sus reit de\textit{s} \textbf{künic} Lotes kint,\\ 
15 & \textbf{daz in belîben d\textit{â}} niemen bat.\\ 
 & \textbf{sus kêrt er} \textbf{reht} gegen der stat.\\ 
 & er dâhte: "sol ich \textbf{kempfer} wesen,\\ 
 & ich mac vor \textbf{vlüste} \textbf{baz} genesen\\ 
 & dort in der stat danne hie bî in.\\ 
20 & ich kêre mich an keinen gewin,\\ 
 & wanne \textbf{wie ich} dâ\textit{z} \textit{mî}ne behalte,\\ 
 & sô \textbf{daz es} glücke walte."\\ 
 & Gawan gegen einer porte reit.\\ 
 & der burgære site was ime leit:\\ 
25 & \textbf{si} hete niht betûret,\\ 
 & alle ir porten wâren vermûret\\ 
 & und alliu ir wîchûs werlîch;\\ 
 & dar zuo der zinnen iegelîch\\ 
 & mit armbrust ein schütze pflac,\\ 
30 & der \textit{sich} schiezens her ûz bewac.\\ 
\end{tabular}
\scriptsize
\line(1,0){75} \newline
m n o \newline
\line(1,0){75} \newline
\newline
\line(1,0){75} \newline
\textbf{1} \textit{Versfolge 351.25-30, 352.1-16 (Bl. 225v), 351.1-23 (Bl. 226r), 351.24, 352.17-30 (Bl. 226v)} m   $\cdot$ in] ein m sin n \textbf{3} zeltsnuor] zit snúr o  $\cdot$ anderen] ander n (o)  $\cdot$ dranc] twang n o \textbf{5} dô] Da m  $\cdot$ wâ] wie n o \textbf{7} biense venûz] binseuemis n bien seuenis o \textbf{8} gramerzîs] Gramerczig o  $\cdot$ jach] sach o \textbf{9} rote] rate o \textbf{10} Semblidac] sembliag m semblidag n o \textbf{11} den] Der n o  $\cdot$ dâ] do m n o \textbf{12} turkopele] Turkopere o  $\cdot$ Kaheti] kahetẏ m kahakẏ o \textbf{13} unkünde] Vnkunden o \textbf{14} des künic] der kunig m des koniges o  $\cdot$ Lotes] lates o \textbf{15} dâ] do m n o \textbf{17} dâhte] gedochte n (o) \textbf{18} vor vlüste] fuͯr fluͯcht o \textbf{19} hie] dort o \textbf{21} daz mîne] dar jnne m (n) (o) \textbf{23} porte] porten n o \textbf{24} burgære] burge o \textbf{28} zinnen] zuͦ o \textbf{29} pflac] lag n o \textbf{30} sich] \textit{om.} m  $\cdot$ her ûz] \textit{om.} n o \newline
\end{minipage}
\end{table}
\newpage
\begin{table}[ht]
\begin{minipage}[t]{0.5\linewidth}
\small
\begin{center}*G
\end{center}
\begin{tabular}{rl}
 & dâ durch in \textbf{grôziu} angest sneit.\\ 
 & Gawan enmitten durch si reit.\\ 
 & doch ieslîch zeltsnuor die anderen dranc,\\ 
 & ir her was wît unde lanc.\\ 
5 & dô sach er, \textbf{wie} si lâgen,\\ 
 & wes dise unde jene pflâgen.\\ 
 & swer 'biensevenûz' dâ sprach,\\ 
 & 'grantmerzîs' er wider jach.\\ 
 & grôz rote an einem orte lac,\\ 
10 & sarjande von Semlidac;\\ 
 & \textbf{den lac} dô sunder nâhen bî\\ 
 & turkopel \textbf{von} \textbf{Kabadi}.\\ 
 & unkünde dick\textit{e} \textit{u}nminne sint.\\ 
 & sus reit des \textbf{künic} Lotes kint.\\ 
15 & \textbf{belîbenes bet in} niemen bat.\\ 
 & \textbf{Gawan kêrte} gein der stat.\\ 
 & er dâhte: "sol ich \textbf{kipper} wesen,\\ 
 & ich mac vor \textbf{vlüste} \textbf{baz} genesen\\ 
 & dort in der stat dane hie bî in.\\ 
20 & ich kêre mich an deheinen gewin,\\ 
 & wan \textbf{deich} daz mîn behalte,\\ 
 & sô \textbf{deis} gelücke walte."\\ 
 & Gawan gein einer porte reit.\\ 
 & der burgære site was im leit:\\ 
25 & \textbf{si}\textbf{ne} hete \textbf{des} niht betûret,\\ 
 & al ir borte wâren vermûret\\ 
 & unde elliu iriu wîchûs werlîch;\\ 
 & dâ zuo der zinnen iegelîch\\ 
 & mit armbrüste ein schütze pflac,\\ 
30 & der sich schiezens her ûz bewac.\\ 
\end{tabular}
\scriptsize
\line(1,0){75} \newline
G I O L M Q R Z Fr39 \newline
\line(1,0){75} \newline
\textbf{1} \textit{Initiale} I O L Z Fr39  \textbf{3} \textit{Capitulumzeichen} R  \textbf{13} \textit{Initiale} I  \newline
\line(1,0){75} \newline
\textbf{1} dâ] ÷A O  $\cdot$ in grôziu] ein groszer L ein groziv Fr39 \textbf{3} ieslîch zeltsnuor] iegliches gezeltsnuͦre I  $\cdot$ die] de G  $\cdot$ anderen] ander L  $\cdot$ dranc] zwank R \textbf{5} dô] Da L M Z \textbf{6} dise unde jene] dise vnd ene I iene vnd dise Z \textbf{7} swer] Wer L M Q R  $\cdot$ biensevenûz] bensarenvs Q biensavevnz Z  $\cdot$ dâ] do Q R \textbf{8} \textit{Vers 351.8 fehlt} M   $\cdot$ er] er dar L er do R \textbf{9} orte] ende L M R Fr39 \textbf{10} Semlidac] semlidach G O L semilidac I sem lidac M semlidac Q (Z) Fr39 Semblidag R \textbf{11} dô] da O L Z \textbf{12} Kabadi] kabali I kahadi O L M Z Fr39 gahadi Q kahedi R \textbf{13} dicke unminne] ditche ditche vnminne G doch wunne R \textbf{14} künic] chvniges O (L) (M) (Q) (R) (Fr39)  $\cdot$ Lotes] lotis M \textbf{15} belîbenes bet] belibens I (L) (Fr39) [Belibens bet]: Belibens  O Belibens [bat]: bete M  $\cdot$ in] in da I \textbf{16} kêrte] kert Z \textbf{17} dâhte] gedahte L (Q) Fr39 \textbf{18} baz] wol I \textbf{19} dort] \textit{om.} I \textbf{20} \textit{Vers 351.20 fehlt} Q   $\cdot$ kêre] en kere M (Z) \textbf{21} deich] wie ich O L (M) Q R Z Fr39  $\cdot$ behalte] behalten R \textbf{22} sô] Sal M (R)  $\cdot$ deis] daz ich I des O (L) M Q R Fr39 daz ez Z  $\cdot$ gelücke] geluches I  $\cdot$ walte] warte I walten R \textbf{23} einer] der einen I  $\cdot$ porte] porten I O Z phorten M (Q) \textbf{24} burgære] burge M \textbf{25} sine hete] Si en hetten M (Z) Seine hete Q Sy hettent R  $\cdot$ betûret] getruwert R \textbf{26} borte] porten I O R Z phorten M (Q) \textbf{27} elliu iriu wîchûs] ir wichus ellev I al ir bowurff Q alle ir wchus R \textbf{28} \textit{Versfolge 351.28-27} M  \textbf{29} armbrüste] arm brest R \textbf{30} schiezens] schisses Q \newline
\end{minipage}
\hspace{0.5cm}
\begin{minipage}[t]{0.5\linewidth}
\small
\begin{center}*T
\end{center}
\begin{tabular}{rl}
 & dâ durch in \textbf{grôziu} angest sneit.\\ 
 & Gawan mitten durch si reit.\\ 
 & doch ieslîche zeltsnuor die andern dranc,\\ 
 & ir her was wît unde lanc.\\ 
5 & dô sach er, \textbf{wie} si lâgen,\\ 
 & wes dise unde jene pflâgen.\\ 
 & Swer 'bensevenûs' dô sprach,\\ 
 & 'gramerzîs' er wider jach.\\ 
 & grôz rotte an einem orte lac,\\ 
10 & Sarjande von Semblidac,\\ 
 & \hspace*{-.7em}\big| turkopel \textbf{unde} \textbf{Gahadi},\\ 
 & \hspace*{-.7em}\big| \textbf{die lâgen} dâ sunder nâhe bî.\\ 
 & unkünde dicke unminne sint.\\ 
 & \textit{sus reit des \textbf{küneges} Lotes kint.}\\ 
15 & \textbf{blîbens in dâ} nieman bat.\\ 
 & \textbf{Gawan kêrte} gegen der stat.\\ 
 & er dâhte: "sol ich \textbf{kapfer} wesen,\\ 
 & ich mac vor \textbf{lust} \textbf{niht} genesen\\ 
 & dort in der stat, danne hie bî in.\\ 
20 & ich kêre mich an keinen gewin,\\ 
 & wan \textbf{wie ich} daz mîne behalte,\\ 
 & sô \textbf{daz des} glücke walte."\\ 
 & \begin{large}G\end{large}awan gegen einer porte reit.\\ 
 & der burgære site was im leit:\\ 
25 & \textbf{die} hete \textbf{des} niht betûret,\\ 
 & allir porten wâren vermûret\\ 
 & unde allir wîchûs werlîch;\\ 
 & dar zuo der zinnen iegeslîch\\ 
 & mit armbrust ein schütze pflac,\\ 
30 & der sich schiezens her ûz bewac.\\ 
\end{tabular}
\scriptsize
\line(1,0){75} \newline
T V W \newline
\line(1,0){75} \newline
\textbf{1} \textit{Initiale} W  \textbf{7} \textit{Majuskel} T  \textbf{10} \textit{Majuskel} T  \textbf{23} \textit{Initiale} T V  \newline
\line(1,0){75} \newline
\textbf{3} doch] Durch V \textit{om.} W \textbf{7} swer] Wer W  $\cdot$ bensevenûs] beschevenv́z V benseuemus W \textbf{8} gramerzîs] Gramersi V \textbf{10} Semblidac] semblidag V seblide dag W \textbf{12} \textit{Versfolge 351.11-12} W   $\cdot$ turkopel] [Turk*]: Turkopele V Turopel W  $\cdot$ unde] von V  $\cdot$ Gahadi] kaheti V gabardei W \textbf{11} die] Den V W  $\cdot$ dâ] do V W \textbf{13} unminne] vninne W \textbf{14} \textit{Vers 351.14 fehlt (Zeile ausgespart)} T   $\cdot$ des] \textit{om.} W  $\cdot$ Lotes] lottes W \textbf{15} Beleibens bete in niemans do bat W  $\cdot$ blîbens in dâ] Das in bliben do V \textbf{17} kapfer] ein kampher W \textbf{18} vor lust niht] vor verluste bas V verlust bas W \textbf{21} ich] das ich V \textbf{22} des] es W \textbf{23} porte] porten W \textbf{25} des] doch W \textbf{26} wâren] man W \textbf{27} werlîch] wercklich W \textbf{30} schiezens her ûz] schiessen dez har vs V schiessens auß W \newline
\end{minipage}
\end{table}
\end{document}
