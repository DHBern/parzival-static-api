\documentclass[8pt,a4paper,notitlepage]{article}
\usepackage{fullpage}
\usepackage{ulem}
\usepackage{xltxtra}
\usepackage{datetime}
\renewcommand{\dateseparator}{.}
\dmyyyydate
\usepackage{fancyhdr}
\usepackage{ifthen}
\pagestyle{fancy}
\fancyhf{}
\renewcommand{\headrulewidth}{0pt}
\fancyfoot[L]{\ifthenelse{\value{page}=1}{\today, \currenttime{} Uhr}{}}
\begin{document}
\begin{table}[ht]
\begin{minipage}[t]{0.5\linewidth}
\small
\begin{center}*D
\end{center}
\begin{tabular}{rl}
\textbf{10} & vünf ors \textbf{erwelt und} erkant,\\ 
 & \textbf{diu besten} über \textbf{al} sîn lant,\\ 
 & \textbf{küene}, starc, niht ze laz,\\ 
 & \textbf{manec tiwer goltvaz}\\ 
5 & \textbf{und} manegen guldînen klôz.\\ 
 & den künec \textbf{wênec} des verdrôz,\\ 
 & er envultes im vier soumschrîn.\\ 
 & gesteines muose \textbf{ouch} \textbf{vil} dar în.\\ 
 & Dô \textbf{si} gevüllet lâgen,\\ 
10 & knappen, die \textbf{des} pflâgen,\\ 
 & wâren \textbf{wol} gekleidet und geriten.\\ 
 & dâ\textbf{ne} wart jâmer niht vermiten,\\ 
 & dô er vür sîne muoter gienc\\ 
 & \textbf{und} si in \textbf{sô vaste} zuo ir \textbf{vienc}.\\ 
15 & "filliiroy Gandin,\\ 
 & wil dû niht langer bî mir sîn?",\\ 
 & sprach daz wîplîche wîp.\\ 
 & "owê, nû truoc dich \textbf{doch} mîn lîp;\\ 
 & \textbf{dû} bist ouch Gandins kint!\\ 
20 & ist got an sîner helfe blint\\ 
 & oder ist er \textbf{dran} \textbf{betoubet},\\ 
 & daz er mir niht geloubet?\\ 
 & \textbf{sol ich} nû niwen kumber haben?\\ 
 & \textbf{ich hân mînes herzen kraft} \textbf{begraben},\\ 
25 & \textbf{die} süeze mîner ougen.\\ 
 & wil \textbf{er} mich vürbaz \textbf{rouben}\\ 
 & und ist doch ein rihtære,\\ 
 & sô liuget mir \textbf{daz} mære,\\ 
 & \textbf{als} man von sîner helfe saget,\\ 
30 & sît er \textbf{an mir ist sus} verzaget."\\ 
\end{tabular}
\scriptsize
\line(1,0){75} \newline
D \newline
\line(1,0){75} \newline
\textbf{9} \textit{Versal} D  \newline
\line(1,0){75} \newline
\textbf{2} diu] de D \textbf{15} Gandin] Gaudin D \textbf{16} wil] wil \textit{nachträglich korrigiert zu:} wilt D \newline
\end{minipage}
\hspace{0.5cm}
\begin{minipage}[t]{0.5\linewidth}
\small
\begin{center}*m
\end{center}
\begin{tabular}{rl}
 & vünf ros \textbf{erwelt und} erkant,\\ 
 & \textbf{diu besten} über \textbf{alliu} sîniu lant,\\ 
 & \textbf{kurz}, starc, niht zuo \textbf{lanc und} laz\\ 
 & - \textbf{sîn gewant goltvar was} -,\\ 
5 & \textbf{und} me\textit{n}igen guldînen klôz.\\ 
 & den künic \textbf{wênic} des verdrôz,\\ 
 & er \dag bulieret\dag  im viere soums\textit{chrîn}.\\ 
 & gesteines muose \textbf{ouch} \textbf{vil} dar în.\\ 
 & dô \textbf{si} \textbf{dâ} gevüllet lâgen,\\ 
10 & knappen, die \textbf{des} pflâgen,\\ 
 & wâren \textbf{wol} gekleit und ge\textit{r}iten.\\ 
 & d\textit{â} wart jâmer niht vermiten,\\ 
 & dô er vür sîne muoter gienc\\ 
 & \textbf{und} si in \textbf{sô vaste} zuo ir \textbf{vienc}.\\ 
15 & "fillin my Gandin,\\ 
 & wiltû niht langer bî mir sîn?",\\ 
 & sprach d\textit{az} wîplîche wîp.\\ 
 & "owê, nû truoc dich mî\textit{n} lîp!\\ 
 & \multicolumn{1}{l}{ - - - }\\ 
20 & \multicolumn{1}{l}{ - - - }\\ 
 & oder ist er \textbf{dâr an} \textbf{ertoubet},\\ 
 & daz er mir niht geloubet?\\ 
 & \textbf{durch} nû niuwen kumber habe\textit{n}\\ 
 & \textbf{ich hân mîner herzen kraft} \textbf{vergraben},\\ 
25 & \textbf{daz} süeze mîner ougen.\\ 
 & wil\textbf{tû} mich vürbaz \textbf{touben}?\\ 
 & und ist doch ein rihtære,\\ 
 & sô liuget mir \textbf{disiu} mære,\\ 
 & \textbf{als} man von sîner helfe saget,\\ 
30 & sît er \textbf{an mir ist sust} verzaget."\\ 
\end{tabular}
\scriptsize
\line(1,0){75} \newline
m n o W \newline
\line(1,0){75} \newline
\newline
\line(1,0){75} \newline
\textbf{1} erwelt] er welt er n (o) \textbf{2} alliu sîniu] all sein W \textbf{3} niht zuo lanc und] lang vnd nit W \textbf{4} sîn] Ein W  $\cdot$ goltvar] goltfarwe o \textbf{5} menigen guldînen] memichen guldinen m dine goltedinen o mangen guldin W \textbf{6} den] Der o  $\cdot$ des] das o \textbf{7} bulieret] puluert n polieret o (W)  $\cdot$ soumschrîn] soumserin \textit{nachträglich korrigiert zu:} [soumrosz fin]: soumschrin m [pu*]: suͯluerin n sum:rin o saumerin W \textbf{8} muose] muͯsse m \textbf{9} dâ] \textit{om.} n o W \textbf{11} geriten] geneitten \textit{nachträglich korrigiert zu:} geritten m genitten n genietet o \textbf{12} dâ] Do m n o \textbf{14} zuo ir] an sich n o \textbf{15} fillin my] Fruͯ min n Fili myn o \textbf{17} daz] die \textit{nachträglich korrigiert zu:} daz m  $\cdot$ wîplîche] wiplich n \textbf{18} owê nû truoc] Owe truͦg nuͦ n  $\cdot$ mîn] mich \textit{nachträglich korrigiert zu:} ye min m doch min n (o) \textbf{19} \textit{Die Verse 10.19-20 fehlen} m n o  \textbf{21} ertoubet] toubet n erthobet o \textbf{22} Das ir mir mit glaubent o \textbf{23} nû] vil n o  $\cdot$ kumber] komern n  $\cdot$ haben] habent m \textbf{24} Jch han min hertze crafft begraben n (o) \textbf{28} liuget] lenget o \textbf{29} sîner] miner n \newline
\end{minipage}
\end{table}
\newpage
\begin{table}[ht]
\begin{minipage}[t]{0.5\linewidth}
\small
\begin{center}*G
\end{center}
\begin{tabular}{rl}
 & vünf ors \textbf{erwelt und} erkant,\\ 
 & \textbf{diu besten} über \textbf{al} sîn lant,\\ 
 & \textbf{küene}, starc, niht ze laz,\\ 
 & \textbf{manic tiure goltvaz},\\ 
5 & manigen guldînen klôz.\\ 
 & den künic \textbf{lützel} des verdrôz,\\ 
 & er envultes im vier soumschrîn.\\ 
 & gesteines muose \textbf{vil} dar în.\\ 
 & dô \textbf{die} gevüllet lâgen,\\ 
10 & knappen, die \textbf{der} pflâgen,\\ 
 & wâren \textbf{wol} gekleit und geriten.\\ 
 & \textbf{al} dâ wart jâmer niht vermiten,\\ 
 & dô er vür sîne muoter gienc.\\ 
 & \textbf{vil nâhen} sin zuo \textbf{z}ir \textbf{gevienc}.\\ 
15 & "filliroys Gandin,\\ 
 & wil dû niht langer bî mir sîn?",\\ 
 & sprach daz wîplîche wîp.\\ 
 & "owê, nû truoc dich \textbf{doch} mîn lîp;\\ 
 & \textbf{dû} bist ouch Gandines kint!\\ 
20 & ist got an sîner helfe blint?\\ 
 & \begin{large}O\end{large}der ist er \textbf{drane} \textbf{betoubet},\\ 
 & daz er mir niht geloubet?\\ 
 & \textbf{sol ich} nû niwen kumber haben?\\ 
 & \textbf{mînes herzen kraft hân ich} \textbf{begraben}\\ 
25 & \textbf{unt} \textbf{die} süeze mîner ougen.\\ 
 & wil \textbf{er} mich vürbaz \textbf{rouben}\\ 
 & unde ist doch ein rihtære,\\ 
 & sô liuget mir \textbf{daz} mære,\\ 
 & \textbf{daz} man von sîner helfe saget,\\ 
30 & sît er \textbf{an mir ist sus} verzaget."\\ 
\end{tabular}
\scriptsize
\line(1,0){75} \newline
G O L M Q W Z Fr29 Fr32 \newline
\line(1,0){75} \newline
\textbf{1} \textit{Initiale} O M  \textbf{9} \textit{Versal} Fr32  \textbf{21} \textit{Initiale} G  \newline
\line(1,0){75} \newline
\textbf{1} vünf] ÷vnf O  $\cdot$ ors] usz M \textbf{2} al] alle O M Q \textbf{4} manic] Vnd manich O (L) (M) (Q) (Z) (Fr32) \textbf{5} \textit{Vers 10.5 fehlt} Q   $\cdot$ manigen] Vil manigen O (L) (M) (Fr32) Vnd manigen Z  $\cdot$ guldînen] [thuren]: guldin M \textbf{6} lützel] wennig M \textbf{7} er envultes] Ern fvlt O Er fvlte L Er fultisz M Er fult Q (Z) ern vuͦlte Fr32 \textbf{8} Vil gesteins must ovch dar in Z  $\cdot$ muose] msuste L  $\cdot$ vil] avch O  $\cdot$ în] in \textit{nachträglich korrigiert zu:} sin O inne sein Q \textbf{9} dô] Da L Z  $\cdot$ die] si O (L) (M) (Q) (Z) Fr32 \textbf{10} der] des O L M Q W Z Fr32 \textbf{12} al] \textit{om.} L  $\cdot$ jâmer] iamers W \textit{om.} Fr32 \textbf{13} dô] Da M Z \textbf{14} sin] si O  $\cdot$ zir] ir O L M Q W Z zim Fr32  $\cdot$ gevienc] viͯngk M (Z) \textbf{15} \sout{Sie uff gandin} Filiroys gandin M  $\cdot$ filliroys] Fili foris Z  $\cdot$ Gandin] gaudin W \textbf{17} sprach] So sprach O L M Sust sprach Q (Fr32) Da sprach Z \textbf{18} owê nû] Owý nv L Awe nun Q So W  $\cdot$ doch] doch nun Q \textit{om.} Fr32  $\cdot$ mîn] mein selbes W \textbf{19} dû bist] Vnd bistu Q Vnd bist W (Fr32)  $\cdot$ ouch] doch W  $\cdot$ Gandines] kandines O gamurettes W Gandins Z Fr32 Gaudins Fr29 \textbf{20} an] in Fr32 \textbf{21} er] \textit{om.} Z  $\cdot$ betoubet] beravbet O (Fr32) \textbf{22} er] \textit{om.} O \textbf{23} niwen kumber] muͯwen kummphirrn M  $\cdot$ haben] tragen O Fr32 \textbf{24} herzen kraft] hrrtzen freud W  $\cdot$ hân ich] ich han L (Q) (W) (Fr32)  $\cdot$ begraben] vergeben Q \textbf{25} die] der Z div Fr32 \textbf{26} vürbaz] Nu M \textbf{28} daz] dar \textit{nachträglich korrigiert zu:} das O die L diz Fr32 \textbf{29} daz] Als Q (Fr32) \textbf{30} an mir ist sus] an mir svs ist O L (M) Fr29 sust an mir ist Q an mir suß W an mir ist Z \newline
\end{minipage}
\hspace{0.5cm}
\begin{minipage}[t]{0.5\linewidth}
\small
\begin{center}*T
\end{center}
\begin{tabular}{rl}
 & vünf ors, \textbf{die besten} erkant\\ 
 & \textbf{und ûzerwelt} über \textbf{al} sîn lant,\\ 
 & \textbf{küene}, starc \textbf{und} niht ze laz,\\ 
 & \textbf{vil} \textbf{manec tiure goltvaz}\\ 
5 & \textbf{und} manegen guldînen klôz.\\ 
 & den künec \textbf{lützel} des verdrôz,\\ 
 & ern vulte\textit{s} im vier soumschrîn.\\ 
 & gesteines muose \textbf{ouch} \textbf{mê} dar în.\\ 
 & Dô \textbf{si} gevüllet lâgen,\\ 
10 & knappen, die \textbf{des} pflâgen,\\ 
 & \textbf{die} wâren gekleit und \textbf{wol} geriten.\\ 
 & dâ\textbf{ne} wart \textbf{grôz} jâmer niht vermiten,\\ 
 & dô er vür sîne muoter gienc.\\ 
 & \textbf{vil nâhe} si in zuo \textbf{z}ir \textbf{gevienc}.\\ 
15 & "Fillyrois Gandin,\\ 
 & wiltû niht langer bî mir sîn?",\\ 
 & \textbf{sô} sprach daz wîplîche wîp.\\ 
 & "ouwî, nû truoc dich \textbf{doch} mîn lîp,\\ 
 & \textbf{und} bist ouch Gandines kint!\\ 
20 & ist got an sîner helfe blint\\ 
 & oder \textbf{wie} ist er \textbf{sus} \textbf{betoubet},\\ 
 & daz er mir niht geloubet?\\ 
 & \textbf{sol ich} nû niuwen kumber haben?\\ 
 & \textbf{mînes herzen vröude ich hân} \textbf{begraben}\\ 
25 & \textbf{und} \textbf{die} süeze mîner ougen.\\ 
 & wil \textbf{er} mich vürbaz \textbf{rouben}\\ 
 & und ist doch ein rihtære,\\ 
 & sô liuget mir \textbf{daz} mære,\\ 
 & \textbf{daz} man von sîner helfe saget,\\ 
30 & sît er \textbf{alsus ist} verzaget."\\ 
\end{tabular}
\scriptsize
\line(1,0){75} \newline
T U V \newline
\line(1,0){75} \newline
\textbf{9} \textit{Majuskel} T  \newline
\line(1,0){75} \newline
\textbf{2} al] alles V \textbf{5} guldînen] guͦlden U \textbf{7} ern] Er U V  $\cdot$ vultes] vultez T (U) [fulte*]: fulte  V  $\cdot$ soumschrîn] schone schrin U \textbf{8} muose] mvese T \textbf{10} knappen] die knappen V \textbf{12} dâne] da V  $\cdot$ grôz] \textit{om.} U V \textbf{14} in] \textit{om.} U  $\cdot$ zir] ir V \textbf{15} Gandin] Gaudin U \textbf{19} Gandines] gaudines U \textbf{21} betoubet] betruͦbet U \textbf{30} alsus ist verzaget] alsus ist vnverzaget U [*]: an mir sus hat verzaget V \newline
\end{minipage}
\end{table}
\end{document}
