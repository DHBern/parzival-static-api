\documentclass[8pt,a4paper,notitlepage]{article}
\usepackage{fullpage}
\usepackage{ulem}
\usepackage{xltxtra}
\usepackage{datetime}
\renewcommand{\dateseparator}{.}
\dmyyyydate
\usepackage{fancyhdr}
\usepackage{ifthen}
\pagestyle{fancy}
\fancyhf{}
\renewcommand{\headrulewidth}{0pt}
\fancyfoot[L]{\ifthenelse{\value{page}=1}{\today, \currenttime{} Uhr}{}}
\begin{document}
\begin{table}[ht]
\begin{minipage}[t]{0.5\linewidth}
\small
\begin{center}*D
\end{center}
\begin{tabular}{rl}
\textbf{259} & si muose in doch erbarmen.\\ 
 & mit henden unt mit armen\\ 
 & begunde si sich decken\\ 
 & vor Parzival, dem recken.\\ 
5 & \begin{large}D\end{large}ô sprach er: "vrouwe, nemt durch got\\ 
 & ûf rehten dienst sunder spot\\ 
 & an iwern lîp mîn kursît."\\ 
 & "\textbf{hêrre}, wære daz âne strît,\\ 
 & daz al mîn vreude læge dran,\\ 
10 & sô get\textit{ö}rst ichz \textbf{doch} niht \textbf{grîfen} an.\\ 
 & Welt ir uns \textbf{tœtens machen} vrî,\\ 
 & sô rîtet, daz ich iu verre sî.\\ 
 & doch klagte ich wênec \textbf{mîn nôt},\\ 
 & wan daz ich vürhte, ir \textbf{kiest den tôt}."\\ 
15 & \textbf{Dô sprach er}: "vrouwe, wer næme \textbf{uns}\textbf{z} leben?\\ 
 & daz hât uns gotes kraft gegeben.\\ 
 & ob \textbf{des} gerte ein ganzez her,\\ 
 & man sæhe mich vür \textbf{uns} ze wer."\\ 
 & Si sprach: "\textbf{es} gert ein werder degen,\\ 
20 & der hât sich strîtes sô bewegen,\\ 
 & iwer sehse \textbf{kœmen si} in arbeit.\\ 
 & mir ist iwer rîten bî mir leit.\\ 
 & ich was etswenne sîn wîp.\\ 
 & nû m\textit{ö}hte mîn vertwâlet lîp\\ 
25 & des heldes diern niht gesîn.\\ 
 & sus tuot er gein mir \textbf{zürnen} schîn."\\ 
 & Dô sprach er zuo der vrouwen sân:\\ 
 & "wer ist hie mit iwerem man?\\ 
 & \textbf{wan} vlühe ich nû durch iwern rât,\\ 
30 & daz diuhte iuch lîhte \textbf{ein} missetât.\\ 
\end{tabular}
\scriptsize
\line(1,0){75} \newline
D \newline
\line(1,0){75} \newline
\textbf{5} \textit{Initiale} D  \textbf{11} \textit{Majuskel} D  \textbf{15} \textit{Majuskel} D  \textbf{19} \textit{Majuskel} D  \textbf{27} \textit{Majuskel} D  \newline
\line(1,0){75} \newline
\textbf{10} getörst] getorst D \textbf{24} möhte] mohte D \newline
\end{minipage}
\hspace{0.5cm}
\begin{minipage}[t]{0.5\linewidth}
\small
\begin{center}*m
\end{center}
\begin{tabular}{rl}
 & si muos in doch erbarmen.\\ 
 & mit henden und mit armen\\ 
 & begunde si sich d\textit{e}cken\\ 
 & vor Parcifal, dem recken.\\ 
5 & dô sprach er: "vrouwe, nemt durch got\\ 
 & ûf rehten dienst sunder spot\\ 
 & an iuweren lîp mî\textit{n} kursît."\\ 
 & "\textbf{hêrre}, wære daz âne strît,\\ 
 & daz alliu mîn vröude \textit{læge} dran,\\ 
10 & sô getörst ichz \textbf{doch} niht \textbf{grîfen} an.\\ 
 & welt ir uns \textbf{machen tôdes} vrî,\\ 
 & sô rîtet, daz ich iu verre sî.\\ 
 & doch klagete ich wênic \textbf{mînen tôt},\\ 
 & want daz ich vürhte, ir \textbf{komets in nôt}."\\ 
15 & \textbf{dô sprach er}: "vrouwe, wer næme \textbf{unser} leben?\\ 
 & daz hât uns gotes kraft gegeben.\\ 
 & ob \textbf{daz} gerte ein ganze\textit{z} her,\\ 
 & man sæhe mich vür ze wer."\\ 
 & si sprach: "\textbf{es} gerte ein werder degen,\\ 
20 & der hât sich strîtes sô bewegen,\\ 
 & iuwer sehs \textbf{k\textit{œ}mens} in arbeit.\\ 
 & mir ist iuwer rîte\textit{n} bî \textit{m}ir leit.\\ 
 & ich was etwanne sîn wîp.\\ 
 & nû m\textit{ö}hte mîn vertwâlet lîp\\ 
25 & des heldes dirne niht gesîn.\\ 
 & sus tuot \textit{er} engegen mir \textbf{zürnen} schîn."\\ 
 & dô sprach er zuo de\textit{r v}rouwen sâ\textit{n}:\\ 
 & "wer ist hie mit iuwerm man?\\ 
 & \textbf{wanne} vlü\textit{h}e ich nû durch iuwern rât,\\ 
30 & daz d\textit{iu}hte iuch lîhte \textbf{ein} missetât,\\ 
\end{tabular}
\scriptsize
\line(1,0){75} \newline
m n o Fr69 \newline
\line(1,0){75} \newline
\newline
\line(1,0){75} \newline
\textbf{1} muos] muͯsz n muͤst Fr69 \textbf{3} decken] dencken m \textbf{4} vor] Dem vor o \textbf{7} iuweren] iren m  $\cdot$ mîn] mit m \textbf{9} læge] \textit{om.} m \textbf{10} getörst] gedurste n gedorste o \textbf{14} komets] komen n (o) \textbf{15} unser] vnsz o \textbf{16} hât] hette n \textbf{17} daz] des n (o)  $\cdot$ ganzez] ganczer m  $\cdot$ her] [z]: har o \textbf{18} sæhe] siehe o  $\cdot$ vür] noch o \textbf{20} hât] hette n  $\cdot$ sô bewegen] gar erwegen n o \textbf{21} kœmens] komens m \textbf{22} rîten] ritter m  $\cdot$ bî mir] bẏ ir m \textit{om.} n \textbf{24} möhte] mohte m (o)  $\cdot$ vertwâlet] verquelet n o \textbf{25} dirne] dirnen o \textbf{26} er] \textit{om.} m  $\cdot$ engegen] gegen n o \textbf{27} der vrouwen sân] der der frouwen sant m \textbf{28} mit iuwerm] nit vwer o \textbf{29} vlühe] fluhde m sluͦge o  $\cdot$ iuwern] vwer o \textbf{30} diuhte] dohte m duchte n duͯte o \newline
\end{minipage}
\end{table}
\newpage
\begin{table}[ht]
\begin{minipage}[t]{0.5\linewidth}
\small
\begin{center}*G
\end{center}
\begin{tabular}{rl}
 & si muose in doch erbarmen.\\ 
 & mit henden und mit armen\\ 
 & begunde si sich decken\\ 
 & vor Parzival, dem recken.\\ 
5 & dô sprach er: "vrouwe, nemet durch got\\ 
 & ûf rehten dienst sunder spot\\ 
 & an iuwern lîp mîn kursît."\\ 
 & "\textbf{hêrre}, wære daz âne strît,\\ 
 & daz al mîn vröude læge dran,\\ 
10 & sô\textbf{ne} get\textit{ö}rste ichz \textbf{doch} niht \textbf{nemen} an.\\ 
 & welt ir uns \textbf{machen tôdes} vrî,\\ 
 & sô rîtet, daz ich iu verre sî.\\ 
 & doch klagte ich wênic \textbf{mînen tôt},\\ 
 & wan daz ich vürhte, ir \textbf{komts in nôt}."\\ 
15 & "vrouwe, wer næme \textbf{uns} \textbf{unser} leben?\\ 
 & daz hât uns gotes kraft gegeben.\\ 
 & obe \textbf{des} gerte ein ganzez her,\\ 
 & man sæhe mich vür \textbf{uns} ze wer."\\ 
 & si sprach: "\textbf{es} gert ein werder degen,\\ 
20 & der hât sich strîtes sô bewegen,\\ 
 & iuwer sehse \textbf{k\textit{œ}mens} in arbeit.\\ 
 & mirst iuwer rîten bî mir leit.\\ 
 & ich was etswenne sîn wîp.\\ 
 & nû\textbf{ne} m\textit{ö}hte mîn vertwâlet lîp\\ 
25 & des heldes dirne niht gesîn.\\ 
 & sus tuot er gein mir \textbf{zornes} schîn."\\ 
 & dô sprach er zuo der vrouwen sân:\\ 
 & "wer ist hie mit iuwerem man?\\ 
 & \textit{\textbf{wan}} vlühe ich nû durch iuweren rât,\\ 
30 & \begin{large}D\end{large}az d\textit{i}uht iuch lîhte missetât.\\ 
\end{tabular}
\scriptsize
\line(1,0){75} \newline
G I O L M Q R Z Fr21 Fr60 \newline
\line(1,0){75} \newline
\textbf{5} \textit{Initiale} L R  \textbf{13} \textit{Initiale} I  \textbf{27} \textit{Initiale} I O L Q Z Fr21  \textbf{30} \textit{Initiale} G  \newline
\line(1,0){75} \newline
\textbf{1} muose in] mvͦse O Fr21 musten en M begund in R  $\cdot$ doch] idoch I (O) \textbf{4} vor] von I  $\cdot$ Parzival] parzifal I (L) Parcifal O (Z) (Fr21) parzifale M partzifal Q parczifal R  $\cdot$ dem] den R  $\cdot$ recken] Rechem Fr21 \textbf{5} dô sprach er] ER sprach L Da sprach her M (Z)  $\cdot$ vrouwe] \textit{om.} R  $\cdot$ nemet] memt I \textbf{6} rehten] rechtte R \textbf{7} iuwern] úwer R (Z)  $\cdot$ mîn] ditz Z \textbf{9} vröude] frevnde Z  $\cdot$ læge] legent R (Z) \textbf{10} sône] So O L R Z  $\cdot$ getörste] getorste G (I) (O) L (Z) torste M (Fr21) getrost Q  $\cdot$ doch] \textit{om.} I  $\cdot$ nemen] legen I grifen O (L) (M) (Q) (R) (Z) Fr21 \textbf{11} machen tôdes] todes machen L \textbf{12} rîtet] ritet ir L  $\cdot$ iu] \textit{om.} O M Fr21 \textbf{13} doch] Do Q  $\cdot$ klagte] clagt I (O) (M) Q R (Z) (Fr21)  $\cdot$ mînen tôt] myn [not]: tot M \textbf{14} wan] Denne R  $\cdot$ komts] chomt sin I (L) kemit M koment R (Z) \textbf{15} vrouwe] do sprach er vrowe I Er sprach frowe O (L) (Q) (R) (Fr21) (Fr60) Da sprach er vrouwe M (Z)  $\cdot$ uns] \textit{om.} I  $\cdot$ unser] daz O L (M) (Q) (R) Z Fr21 Fr60 \textbf{17} obe] Vff M  $\cdot$ des] sin I das R  $\cdot$ gerte] gert I (O) (L) (Z) (Fr21) (Fr60)  $\cdot$ ganzez] ganze Fr21 \textbf{18} uns] evch I \textbf{19} es gert ein werder] iener gerter I sin gert ein werder O \textbf{20} der] \textit{om.} I  $\cdot$ sô] her L  $\cdot$ bewegen] verwegen O (R) \textbf{21} iuwer] V́we R  $\cdot$ kœmens] chomens G (Z) (Fr21) chomen sin I kemen dez L komen osz M komes Q kement R \textbf{22} mirst] Jm ist Q \textbf{23} sîn] si I \textbf{24} nûne] nu I  $\cdot$ möhte] mohte G (I) (O) (L) (M) (Q) (Z) (Fr21) moch R  $\cdot$ vertwâlet] vertailter I (R) \textbf{25} heldes] haldes Q  $\cdot$ dirne] frowe R \textbf{26} gein mir zornes] zuͯrnen gein mir L (Z) gen mir hilfe Q gen mir zúrnen R \textbf{27} \textit{Versfolge 259.28-27} M   $\cdot$ dô] ÷o O Da M \textbf{29} wan vlühe] flvhe G wan flie M  $\cdot$ ich nû] >nv< ich L \textbf{30} diuht] dvht G (I) (O) (M) (R) Fr21  $\cdot$ iuch] mich R \textit{om.} Z  $\cdot$ lîhte] liht ein I O (Q) Z Fr21 eyn M (R) \newline
\end{minipage}
\hspace{0.5cm}
\begin{minipage}[t]{0.5\linewidth}
\small
\begin{center}*T
\end{center}
\begin{tabular}{rl}
 & si muosin doch erbarmen.\\ 
 & mit henden unde mit armen\\ 
 & begunde si sich decken\\ 
 & vor Parcifale, dem recken.\\ 
5 & \begin{large}D\end{large}ô sprach er: "vrouwe, nemt durch got\\ 
 & ûf rehten dienst sunder spot\\ 
 & an iuwern lîp mîn kursît."\\ 
 & \textbf{Si sprach}: "\textbf{unde} wære daz âne strît,\\ 
 & daz al mîn vröude læge dran,\\ 
10 & sô\textbf{ne} get\textit{ö}rstichz niht \textbf{grîfen} an.\\ 
 & welt ir uns \textbf{machen tôdes} vrî,\\ 
 & sô rîtet, daz ich iu verre sî.\\ 
 & doch klaget ich wênic \textbf{mînen tôt},\\ 
 & wan daz ich vorhte, ir \textbf{komet\textit{s} in nôt}."\\ 
15 & \textbf{Dô sprach er}: "vrouwe, wer næm \textbf{uns} \textbf{daz} leben?\\ 
 & daz hât uns gotes kraft gegeben,\\ 
 & \textbf{unde} ob \textbf{des} gerte ein ganzez her,\\ 
 & man sæhe mich vür \textbf{in} ze wer."\\ 
 & Si sprach: "\textbf{des} gert ein werder degen,\\ 
20 & der hât sich strîtes sô bewegen,\\ 
 & iuwer sehse \textbf{k\textit{œ}met e\textit{s}} in arbeit.\\ 
 & mir ist iuwer rîten bî mir leit.\\ 
 & ich was etswenne sîn wîp.\\ 
 & nû\textbf{ne} m\textit{ö}hte mîn vertwâlet lîp\\ 
25 & des heldes dirne niht gesîn.\\ 
 & sus tuot er gegen mir \textbf{zürnen} schîn."\\ 
 & \begin{large}D\end{large}ô sprach er zuo der vrouwen sân:\\ 
 & "wer ist hie mit iuwerm man?\\ 
 & vlühe ich nû durch iuwern rât,\\ 
30 & daz d\textit{i}uhte iuch lîhte \textbf{ein} missetât.\\ 
\end{tabular}
\scriptsize
\line(1,0){75} \newline
T U V W \newline
\line(1,0){75} \newline
\textbf{5} \textit{Initiale} T U W  \textbf{8} \textit{Majuskel} T  \textbf{15} \textit{Majuskel} T  \textbf{19} \textit{Majuskel} T  \textbf{27} \textit{Initiale} T  \newline
\line(1,0){75} \newline
\textbf{1} si muosin] si mvesin T So muͦz iz in U [So]: Sv́ mvͤste [*]: in V Sy muͦß W  $\cdot$ doch] iedoch W \textbf{4} Parcifale] Parzifale T (V) partzifali W \textbf{5} Dô sprach er] ER sprach W \textbf{8} Si sprach unde] [*]: Herre V Sy sprach ia W \textbf{9} mîn vröude læge] mine vruͦnde legen U \textbf{10} sône] Sonen V So W  $\cdot$ getörstichz] getorstichz T getorst ich iz doch U (V) toͤrst ich es doch W  $\cdot$ an] \textit{om.} U \textbf{11} ir] \textit{om.} U \textbf{13} klaget] klage W  $\cdot$ mînen tôt] meine not W \textbf{14} komets] komentz T comt U eúweren W  $\cdot$ in nôt] tot W \textbf{16} hât uns] [*]: vns hat V vns hat W \textbf{17} ob] \textit{om.} U  $\cdot$ des] das W \textbf{18} vür] gegen W \textbf{19} des] ez V  $\cdot$ ein] em W \textbf{20} sô] har V \textbf{21} kœmet es] cômet ez T quamen is U hettens W  $\cdot$ in] \textit{om.} W \textbf{24} nûne möhte] nvne mohte T (U) Nv moͤhte V \textbf{26} zürnen] zorne U zornes W \textbf{30} diuhte] dvhte T (U) (V)  $\cdot$ iuch] îv T  $\cdot$ lîhte] lecht W  $\cdot$ ein] \textit{om.} U \newline
\end{minipage}
\end{table}
\end{document}
