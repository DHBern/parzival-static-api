\documentclass[8pt,a4paper,notitlepage]{article}
\usepackage{fullpage}
\usepackage{ulem}
\usepackage{xltxtra}
\usepackage{datetime}
\renewcommand{\dateseparator}{.}
\dmyyyydate
\usepackage{fancyhdr}
\usepackage{ifthen}
\pagestyle{fancy}
\fancyhf{}
\renewcommand{\headrulewidth}{0pt}
\fancyfoot[L]{\ifthenelse{\value{page}=1}{\today, \currenttime{} Uhr}{}}
\begin{document}
\begin{table}[ht]
\begin{minipage}[t]{0.5\linewidth}
\small
\begin{center}*D
\end{center}
\begin{tabular}{rl}
\textbf{558} & \begin{large}G\end{large}awan, der \textbf{prîses} erkant,\\ 
 & an \textbf{die vorhte} sich niht want;\\ 
 & er sprach: "nû gebt mir \textbf{strîtes} rât.\\ 
 & ob ir gebiet, rîters tât\\ 
5 & sol ich \textbf{hie} leisten, ruochets got.\\ 
 & iwern rât unt iwer gebot\\ 
 & wil ich immer gerne hân.\\ 
 & hêr wirt, ez wære missetân,\\ 
 & solt ich sus \textbf{hinnen} scheiden.\\ 
10 & die lieben unt die leiden\\ 
 & heten mich vür einen zagen."\\ 
 & Alrêst \textbf{der wirt begunde} klagen,\\ 
 & wand im sô leide nie geschach.\\ 
 & \textbf{hin} ze \textbf{sîme gaste} er sprach:\\ 
15 & "Ob daz got erzeige,\\ 
 & daz ir \textbf{niht sît} veige,\\ 
 & sô werdet ir hêrre dises landes.\\ 
 & swaz vrouwen hie stêt pfandes,\\ 
 & die \textbf{starkez} wunder \textbf{her} \textbf{betwanc},\\ 
20 & \textbf{daz} noch nie rîters prîs \textbf{erranc},\\ 
 & manec \textbf{sarjant}, edeliu rîterschaft,\\ 
 & ob die \textbf{hie} \textbf{erlœset} iwer kraft,\\ 
 & sô sît ir prîses gehêret\\ 
 & unt hât iuch got wol geêret.\\ 
25 & ir \textit{\textbf{muget}} mit vreuden hêrre sîn\\ 
 & über \textbf{manegen liehten} schîn,\\ 
 & vrouwen von manegen landen.\\ 
 & wer jæhe iu des ze schanden,\\ 
 & ob ir \textbf{hinnen schiedet} \textbf{alsus},\\ 
30 & sît Lischoys Gwelljus\\ 
\end{tabular}
\scriptsize
\line(1,0){75} \newline
D \newline
\line(1,0){75} \newline
\textbf{1} \textit{Initiale} D  \textbf{12} \textit{Majuskel} D  \textbf{15} \textit{Majuskel} D  \newline
\line(1,0){75} \newline
\textbf{25} muget] \textit{om.} D \textbf{30} sit Liscoys gwellivs D \newline
\end{minipage}
\hspace{0.5cm}
\begin{minipage}[t]{0.5\linewidth}
\small
\begin{center}*m
\end{center}
\begin{tabular}{rl}
 & Gawan, der \textbf{prîs} erkante,\\ 
 & an \textbf{die vorht} \textbf{er} sich niht wante;\\ 
 & er sprach: "nû gebt mir \textbf{strîtes} rât.\\ 
 & ob ir gebietet, ritters tât\\ 
5 & sol ich \textbf{hie} leisten, ruochet es got.\\ 
 & iuwern rât und iuwer gebot\\ 
 & wil ich iemer gerne hân.\\ 
 & hêr wirt, ez wær missetân,\\ 
 & solt ich sus \textbf{hinnen} scheiden.\\ 
10 & die lieben und die leiden\\ 
 & heten mich vür einen zagen."\\ 
 & allerêrst \textbf{begund der wirt} klagen,\\ 
 & wan im sô leide nie geschach.\\ 
 & zuo \textbf{Gawan} er \textbf{dô} sprach:\\ 
15 & "ob daz got erzeige,\\ 
 & daz ir \textbf{niht sît} veige,\\ 
 & sô werdet ir hêrre dises .\\ 
 & waz vrouwen hie stât pfandes,\\ 
 & die \textbf{starkez} wunder \textbf{hie} \textbf{betwanc},\\ 
20 & \textbf{daz} noch nie ritters prîs \textbf{erklanc},\\ 
 & manic \textbf{sarjant}, edeliu ritterschaft,\\ 
 & ob die \textbf{erlœset} iuwer kraft,\\ 
 & sô sît ir prîses gehêret\\ 
 & und het iuch got wol geêret.\\ 
25 & ir \textbf{m\textit{ö}ht} mit vröuden hêrre sîn\\ 
 & über \textbf{manigen liehten} schîn,\\ 
 & vrouwen von manigen landen.\\ 
 & wer jæhe iu des zuo schanden,\\ 
 & ob ir \textbf{hinnen scheidet} \textbf{sus},\\ 
30 & sît Lischois Gwellius\\ 
\end{tabular}
\scriptsize
\line(1,0){75} \newline
m n o \newline
\line(1,0){75} \newline
\newline
\line(1,0){75} \newline
\textbf{1} Do gawan den prisz erkante n \textbf{2} niht] [w]: nút n \textit{om.} o \textbf{3} gebt] gebp o \textbf{6} iuwern] Vwer o \textbf{7} wil] Vil o \textbf{10} leiden] lieden o \textbf{12} begund der wirt] der wurt begunde n o  $\cdot$ klagen] :agen o \textbf{15} erzeige] erzeigete n \textbf{16} niht] sint o \textbf{19} hie] her n o \textbf{20} ritters] strites n  $\cdot$ erklanc] errang n o \textbf{21} sarjant] sariant vnd n \textbf{23} gehêret] gehoret o \textbf{24} geêret] geheret n geferrt o \textbf{25} möht] moht m (o)  $\cdot$ vröuden] freuͯide o \textbf{28} \textit{Vers 558.28 fehlt} o  \textbf{29} scheidet] scheiden n o \textbf{30} Lischois] liscois m n loscois o  $\cdot$ Gwellius] giwellivs n gewellius o \newline
\end{minipage}
\end{table}
\newpage
\begin{table}[ht]
\begin{minipage}[t]{0.5\linewidth}
\small
\begin{center}*G
\end{center}
\begin{tabular}{rl}
 & \begin{large}G\end{large}awan, der \textbf{brîs} erkande,\\ 
 & an \textbf{die vorhte} sich niht wande;\\ 
 & er sprach: "nû gebet mir \textbf{strîtes} rât.\\ 
 & ob ir gebietet, rîters tât\\ 
5 & sol ich \textbf{hie} leisten, ruochet es got.\\ 
 & iuwern rât unde iuwe\textit{r} gebot\\ 
 & wil ich immer gerne hân.\\ 
 & hêr wirt, ez wære missetân,\\ 
 & solt ich sus \textbf{hinnen} scheiden.\\ 
10 & die lieben unde die leiden\\ 
 & heten mich vür einen zagen."\\ 
 & alrêrst \textbf{der wirt begunde} klagen,\\ 
 & wan im sô leide nie geschach.\\ 
 & \textbf{hin} ze \textbf{sînem gaste} er sprach:\\ 
15 & "op daz got erzeige,\\ 
 & daz ir \textbf{niht sît} veige,\\ 
 & sô werdet ir hêrre dises landes.\\ 
 & swaz vrouwen hie stêt pfandes,\\ 
 & die \textbf{starkez} wunder \textbf{her} \textbf{betwanc},\\ 
20 & \textbf{daz} noch nie rîters brîs \textbf{erranc},\\ 
 & manic \textbf{sarjant}, edeliu rîterschaft,\\ 
 & op die \textbf{hie} \textbf{lœset} iuwer kraft,\\ 
 & sô sît ir brîses gehêret\\ 
 & unde hât iuch got wol geêret.\\ 
25 & ir \textbf{muget} mit vröuden hêrre sîn\\ 
 & über \textbf{manigen liehten} schîn,\\ 
 & vrouwen von manigen landen.\\ 
 & wer jæhe iu des ze schanden,\\ 
 & ob ir \textbf{hinnen schiet} \textbf{alsus},\\ 
30 & sît Lishois Gewellius\\ 
\end{tabular}
\scriptsize
\line(1,0){75} \newline
G I L M Z Fr23 \newline
\line(1,0){75} \newline
\textbf{1} \textit{Initiale} G I L Z Fr23  \textbf{21} \textit{Initiale} I  \newline
\line(1,0){75} \newline
\textbf{1} brîs] brîse G \textbf{2} niht] \textit{om.} L \textbf{3} nû] \textit{om.} M Fr23  $\cdot$ gebet] gep I \textbf{4} gebietet] gebirtet Fr23 \textbf{5} sol ich] ich sol I  $\cdot$ hie leisten] g::lter Fr23  $\cdot$ ruochet] geruchet M  $\cdot$ es] sin I \textbf{6} iuwer gebot] iuwern [rat]: gebot G \textbf{8} wære] war Fr23 \textbf{9} solt] Ab M \textbf{10} die lieben] Div lieben Fr23 \textbf{13} nie] \textit{om.} M \textbf{14} er] er do I \textbf{18} swaz] Waz L (M)  $\cdot$ vrouwen] freuden I \textbf{20} erranc] entwanch Fr23 \textbf{22} hie lœset] hie erloset L Z Fr23 erloset M \textbf{23} gehêret] geeret Fr23 \textbf{24} wol] \textit{om.} M vil Fr23  $\cdot$ geêret] gemeret Fr23 \textbf{25} ir muget] Mit magit M  $\cdot$ vröuden] vrouwen M \textbf{26} liehten] lichten L M Fr23  $\cdot$ schîn] frowen shin I \textbf{27} vrouwen] \textit{om.} I  $\cdot$ von] vnd Fr23 \textbf{28} jæhe] Sprichet M \textbf{29} ir] hir M  $\cdot$ schiet] scheidet Fr23 \textbf{30} Lishois] liscoys I Lytschoýs L Lisois M lihsovs Fr23  $\cdot$ Gewellius] gewellius G (I) Gwelluͯs L gwelluͯs M \newline
\end{minipage}
\hspace{0.5cm}
\begin{minipage}[t]{0.5\linewidth}
\small
\begin{center}*T
\end{center}
\begin{tabular}{rl}
 & Gawan, der \textbf{prîs} erkante,\\ 
 & an \textbf{diu wort} \textbf{er} sich niht wante;\\ 
 & er sprach: "\textbf{hêrre}, nû gebt mir rât.\\ 
 & ob ir gebiet, rîters tât\\ 
5 & sol ich leisten, ruochets got.\\ 
 & iuwern rât unde iuwer gebot\\ 
 & wil ich iemer gerne hân.\\ 
 & hêr wirt, ez wæ\textit{r}e missetân,\\ 
 & solt ich sus \textbf{hin} scheiden.\\ 
10 & die lieben unde die leiden\\ 
 & heten mich vür einen zagen."\\ 
 & Alrêrst \textbf{der wirt begunde} klagen,\\ 
 & wand im sô leide nie geschach.\\ 
 & \textbf{hin} zuo \textbf{sînem gaste} er sprach:\\ 
15 & "ob daz got erzeige,\\ 
 & Daz ir \textbf{sît niht} veige,\\ 
 & sô werdet ir hêrre disses landes.\\ 
 & swaz vrouwen hie stât pfandes,\\ 
 & die \textbf{grôz} wunder \textbf{her} \textbf{twanc}\\ 
20 & \textbf{unde} noch nie rîters prîs \textbf{erranc},\\ 
 & manec \textbf{vrouwe}, edel\textit{iu} rîterschaft,\\ 
 & ob die \textbf{erlœset} iuwer kraft,\\ 
 & sô sît ir prîses gehêret\\ 
 & unde hât iuch got wol geêret.\\ 
25 & ir \textbf{muget} mit vröude hêrre sîn\\ 
 & über \textbf{maneger liehter} \textbf{vrouwen} schîn,\\ 
 & vrouwen von manegen landen.\\ 
 & wer jæhe iu des ze schanden,\\ 
 & ob ir \textbf{scheidet hin} \textbf{alsus},\\ 
30 & sît Lyschoys Gewellius\\ 
\end{tabular}
\scriptsize
\line(1,0){75} \newline
T U V W Q R Fr25 Fr39 Fr40 \newline
\line(1,0){75} \newline
\textbf{1} \textit{Initiale} Fr39   $\cdot$ \textit{Capitulumzeichen} R  \textbf{12} \textit{Majuskel} T  \textbf{15} \textit{Initiale} Fr40  \textbf{16} \textit{Majuskel} T  \textbf{17} \textit{Initiale} Q Fr39  \newline
\line(1,0){75} \newline
\textbf{1} \textit{Die Verse 553.1-599.30 fehlen} U   $\cdot$ der] den R \textbf{2} wort] vorhte V (W) (Q) (R) (Fr39) (Fr40) \textbf{3} hêrre] \textit{om.} V  $\cdot$ rât] [r*]: strittes rat V \textbf{4} ir] iv Fr40 \textbf{5} [S*]: Sol ich hie leisten ruͦchet ez got V  $\cdot$ ruochets] ruchest Q \textbf{6} iuwern] Júwer R Jwe::: Fr25 \textbf{8} wirt] mir Fr40  $\cdot$ wære] werre T \textbf{9} ich sus] ichs als Q \textbf{13} im] in Fr25  $\cdot$ nie] mer Q \textbf{14} hin zuo sînem] Hin zem R Wider sinen Fr25  $\cdot$ er] er do Fr25 \textbf{15} daz got] got daz Fr25  $\cdot$ erzeige] [erzei*]: erzeige T ertzeigte Q \textbf{16} sît niht] niht sit V (W) (Q) (R) (Fr25) (Fr40)  $\cdot$ veige] der veige V \textbf{18} swaz] Was W R Wo Q \textbf{19} twanc] betwang V ertwang W (Q) (R) (Fr40) :::twanc Fr25 \textbf{20} unde noch] [*]: Daz noch V Noch W Q R (Fr39) (Fr40)  $\cdot$ prîs] hand R \textbf{21} vrouwe] seriant V (W) (Q) (R) (Fr25) (Fr39) (Fr40)  $\cdot$ edeliu] edele T \textbf{22} die] die hie V W R Fr39 de Fr25 \textbf{23} gehêret] gehoͯret R \textbf{24} hât iuch got] hat iv got T hant úch R \textbf{25} vröude] froͤiden wol V erre R \textbf{26} maneger liehter vrouwen] manigen liehten V (W) (R) Fr39 (Fr40) manchen lichten Q \textbf{29} ob] Oder R O: Fr25  $\cdot$ scheidet hin alsus] schiedent hinnan sus V schiedet hin alsus W (R) hinnen schiedet alsvs Fr25 \textbf{30} sît] Nu Q  $\cdot$ Lyschoys] Lyscoys T lischoẏs V lishoys W Lyshois Q Lyschois R Lishoẏs Fr25 Lẏshois Fr39 lishois Fr40  $\cdot$ Gewellius] gwellius Q R Fr39 Fr40 Gwellivs Fr25 \newline
\end{minipage}
\end{table}
\end{document}
