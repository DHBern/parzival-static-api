\documentclass[8pt,a4paper,notitlepage]{article}
\usepackage{fullpage}
\usepackage{ulem}
\usepackage{xltxtra}
\usepackage{datetime}
\renewcommand{\dateseparator}{.}
\dmyyyydate
\usepackage{fancyhdr}
\usepackage{ifthen}
\pagestyle{fancy}
\fancyhf{}
\renewcommand{\headrulewidth}{0pt}
\fancyfoot[L]{\ifthenelse{\value{page}=1}{\today, \currenttime{} Uhr}{}}
\begin{document}
\begin{table}[ht]
\begin{minipage}[t]{0.5\linewidth}
\small
\begin{center}*D
\end{center}
\begin{tabular}{rl}
\textbf{222} & unt \textbf{daz} mîn hêrre im siges jach,\\ 
 & den man gein im in kampfe sach.\\ 
 & der selbe hât betwungen mich\\ 
 & gar âne \textbf{hælingen} slich.\\ 
5 & man sach dâ viwer ûz helmen wæn\\ 
 & unt swert in henden umbe \textbf{dræn}."\\ 
 & \begin{large}D\end{large}ô \textbf{sprâchen si} alle gelîche,\\ 
 & beide arme und rîche,\\ 
 & daz Keie hete missetân.\\ 
10 & hie sule wir diz mære lân\\ 
 & unt \textbf{komens} wider an die vart.\\ 
 & daz wüeste lant erbûwen wart,\\ 
 & dâ krône truoc Parzival.\\ 
 & man sach dâ vreude unt schal.\\ 
15 & sîn sweher Tampenteire\\ 
 & liez im ûf Pelrapeire\\ 
 & lieht gesteine \textbf{unt} rôte\textit{z} golt.\\ 
 & daz teilter, sô daz man im holt\\ 
 & was durch sîne milte.\\ 
20 & vil baniere, \textbf{niwe} schilte,\\ 
 & \textbf{des} wart sîn lant gezieret\\ 
 & unt vil geturnieret\\ 
 & von im unt \textbf{von} den sînen.\\ 
 & er liez dicke ellen schînen\\ 
25 & an der marke sînes landes ort,\\ 
 & der junge degen \textbf{unvervorht}.\\ 
 & sîn \textbf{tât} was \textbf{gein} den gesten\\ 
 & geprüevet vür \textbf{die} besten.\\ 
 & \begin{large}N\end{large}û \textbf{hœret} \textbf{ouch} von der künegîn.\\ 
30 & wie \textbf{m\textit{ö}ht der} imer baz gesîn?\\ 
\end{tabular}
\scriptsize
\line(1,0){75} \newline
D \newline
\line(1,0){75} \newline
\textbf{7} \textit{Initiale} D  \textbf{29} \textit{Initiale} D  \newline
\line(1,0){75} \newline
\textbf{15} Tampenteire] Tampentêyre D \textbf{16} Pelrapeire] Pelrapêyre D \textbf{17} rôtez] rotes D \textbf{30} möht] moht D \newline
\end{minipage}
\hspace{0.5cm}
\begin{minipage}[t]{0.5\linewidth}
\small
\begin{center}*m
\end{center}
\begin{tabular}{rl}
 & und \textbf{daz} mîn hêrre ime siges jach,\\ 
 & den man gegen im in kampfe sach.\\ 
 & der selbe hât betwungen mich\\ 
 & gar âne \textbf{hælingen} slich.\\ 
5 & man sach d\textit{â} viur ûz helmen wæn\\ 
 & und swert in henden umbe \textbf{dræn}."\\ 
 & \begin{large}D\end{large}ô \textbf{sprâchen} alle gelîche,\\ 
 & beide arme und rîche,\\ 
 & daz K\textit{ei}e hete missetân.\\ 
10 & hie sullen wir diz mære lân\\ 
 & und \textbf{komens} wider an die vart.\\ 
 & daz wüeste lant erbûwen wart,\\ 
 & dâ krône truoc Parcifal.\\ 
 & man sach d\textit{â} vröude und schal.\\ 
15 & sîn sweher T\textit{a}mpenterie\\ 
 & liez ime ûf Pelraperie\\ 
 & lieht gesteine \textbf{und} rôtez golt.\\ 
 & daz teilt er, sô daz man im holt\\ 
 & was durch sîne milte.\\ 
20 & vil baniere, \textbf{niuwe} schilte,\\ 
 & \textbf{der} wart sîn lant gezieret\\ 
 & und vil geturnieret\\ 
 & von ime und den sînen.\\ 
 & er liez dicke ellen schînen\\ 
25 & an der marke sînes landes ort,\\ 
 & der junge degen \textbf{unervorht}.\\ 
 & sîn \textbf{tât} was \textbf{gegen} den gesten\\ 
 & gebrüefet vür \textbf{den} besten.\\ 
 & \begin{large}N\end{large}û \textbf{hœret} \textbf{ouch} von der künigîn.\\ 
30 & wie \textbf{möhte der} i\textit{e}mer baz gesîn?\\ 
\end{tabular}
\scriptsize
\line(1,0){75} \newline
m n o Fr69 \newline
\line(1,0){75} \newline
\textbf{7} \textit{Initiale} m n  \textbf{29} \textit{Initiale} m   $\cdot$ \textit{Capitulumzeichen} n  \newline
\line(1,0){75} \newline
\textbf{1} \textit{Versfolge 222.2-1} o  \textbf{2} den] Die n o \textbf{4} slich] [al slich]: alglich o \textbf{5} dâ] do m n o \textbf{7} Dô] Die o \textbf{9} Keie] komere m keiner n o  $\cdot$ missetân] missetat o \textbf{10} diz] dise n das o \textbf{11} vart] farte o \textbf{12} wüeste] vuͯste o fúrsten Fr69  $\cdot$ wart] wartte o \textbf{13} dâ] Do n o  $\cdot$ truoc] trig o  $\cdot$ Parcifal] herre parcifal Fr69 \textbf{14} man sach dâ] Man sach do m n o Da sach man Fr69 \textbf{15} Tampenterie] Tumpenterie m tampanteir n tampantier o \textbf{16} Pelraperie] pelrapeir n pelrapier o \textbf{17} lieht] [Liep]: Liecht m \textbf{20} niuwe] núwer Fr69 \textbf{23} den] von den n \textbf{24} dicke] deck o \textbf{27} tât] stat n o \textbf{28} gebrüefet] Gepriset n (o) \textbf{30} möhte der] mochte dir o  $\cdot$ iemer] iamer m \newline
\end{minipage}
\end{table}
\newpage
\begin{table}[ht]
\begin{minipage}[t]{0.5\linewidth}
\small
\begin{center}*G
\end{center}
\begin{tabular}{rl}
 & \textit{und} \textbf{daz} mîn hêrre im siges jach,\\ 
 & den man gein im in kampfe sach.\\ 
 & der selbe hât betwungen mich\\ 
 & gar âne \textbf{hælingen} slich.\\ 
5 & man sach dâ viur ûz helmen wæn\\ 
 & unde swert in handen umbe \textbf{dræn}."\\ 
 & \begin{large}D\end{large}ô \textbf{jâhens} algelîche,\\ 
 & beidiu arm unde rîche,\\ 
 & daz Kay het missetân.\\ 
10 & hie sulen wir diz mære lân\\ 
 & unde \textbf{komens} wider an die vart.\\ 
 & daz wüeste lant erbûwen wart,\\ 
 & dâ krône truoc Parzival.\\ 
 & man sach dâ vröude unde schal.\\ 
15 & sîn sweher Tampunteire\\ 
 & liez im ûf Pelrapeire\\ 
 & lieht gesteine, rôtez golt.\\ 
 & daz teilter, sô daz man im holt\\ 
 & was durch sîne milte.\\ 
20 & vil banier, \textbf{niwer} schilte\\ 
 & wart sîn lant gezieret\\ 
 & unde vil geturnieret\\ 
 & von im unde \textbf{von} den sînen.\\ 
 & er lie dicke ellen schînen\\ 
25 & an der marke sînes landes ort,\\ 
 & der junge degen \textbf{unervorht}.\\ 
 & sîn \textbf{\textit{ge}tât} wa\textit{s} \textbf{von} den gesten\\ 
 & gebrüevet vür \textbf{die} besten.\\ 
 & nû \textbf{sprechet} von der künigîn.\\ 
30 & wie \textbf{m\textit{ö}ht der} imer baz gesîn?\\ 
\end{tabular}
\scriptsize
\line(1,0){75} \newline
G I O L M Q R Z Fr21 Fr23 \newline
\line(1,0){75} \newline
\textbf{3} \textit{Initiale} Fr21  \textbf{7} \textit{Initiale} G  \textbf{11} \textit{Initiale} Q  \textbf{15} \textit{Initiale} I  \textbf{29} \textit{Initiale} O L R Fr21  \newline
\line(1,0){75} \newline
\textbf{1} und] \textit{om.} G  $\cdot$ im] in L  $\cdot$ siges] sigen M \textbf{2} den] Dem Z  $\cdot$ in] mit I  $\cdot$ kampfe] strite R \textbf{4} gar oͯn alle ander geschaht R  $\cdot$ hælingen] helinge I M hælichen O Fr21 heiligen Q \textbf{5} dâ] daz L do Q  $\cdot$ helmen] helm I (O) (Q) (R) Fr21  $\cdot$ wæn] varn L \textbf{6} dræn] dran I (M) darn L den R \textbf{7} Dô] Da M Z  $\cdot$ jâhens] iahen I iageten sie M  $\cdot$ algelîche] alle geliche O (L) M (Q) (R) Z Fr21 \textbf{8} beidiu] \textit{om.} O M Q Fr21 \textbf{9} Kay] kai G key O Z (Fr23) kaý L keye M kyͯ R keẏ Fr21  $\cdot$ het] hat Fr21 \textbf{10} hie] nu I  $\cdot$ diz] die I das M R (Fr23)  $\cdot$ mære] rede I \textbf{11} komens] chomen O (L) (M) (Q) (Z) Fr23 \textbf{12} daz] da I  $\cdot$ erbûwen] erbowet Q \textbf{13} dâ] Do O L Q  $\cdot$ Parzival] Parzifal I (M) Parcifal O L (Z) Fr21 partzifal Q parczifal R parzi:::al Fr23 \textbf{14} man sach dâ] da sach man selten I Man sach da selten O (M) (Fr21) (Fr23) Man vant da L Man sag dicke Q Man sach da dicke R  $\cdot$ vröude unde] freuden I (O) (M) (Q) (Fr21) froͯwde R \textbf{15} sîn] Min R  $\cdot$ sweher] swester L  $\cdot$ Tampunteire] Tanpuntaire I Tampvnteir O (Fr21) Tampvntaiere L tampunterre M tempunteire Q [Tampuͦnteire]: Tampuͦrnteire R \textbf{16} im ûf] in uff M auff im Q  $\cdot$ Pelrapeire] pailrapier I Pelrapeir O pelrapere M pelerapeire R peilrapeir Fr21 \textbf{17} lieht] Liht O (L) lich:: Q  $\cdot$ gesteine] gesteine vnd L (M) (Q) (R) Fr21 stein vnd Z \textbf{18} teilter] teilt er O R Z Fr21 te:t er Q  $\cdot$ daz] \textit{om.} Q  $\cdot$ holt] was holt I \textbf{19} was] \textit{om.} I  $\cdot$ durch] drich Fr21 \textbf{20} niwer] niwe I \textbf{21} wart] Dez wart L (Q) (R) \textbf{22} unde] \textit{om.} Z \textbf{23} unde von] von O L vnd R \textbf{24} er] Es Q  $\cdot$ ellen] eren Q \textbf{25} sînes] des Q \textbf{26} der junge] Den iungen Q  $\cdot$ unervorht] hy vnd dort Q \textbf{27} \textit{Die Verse 222.27-28 fehlen} R   $\cdot$ getât was] tat wart G gerete was M getat Q  $\cdot$ von] vor O Fr21 gein Z  $\cdot$ gesten] [besten]: Gesten I \textbf{28} vür die] ze der O (Fr21) \textbf{29} nû] ÷v O An Q  $\cdot$ sprechet] horet auch Q (R)  $\cdot$ von] vor Q \textbf{30} möht] moht G I O (L) (M) (Q) Z Fr21  $\cdot$ der] ir L dir M  $\cdot$ gesîn] geschehen sin L \newline
\end{minipage}
\hspace{0.5cm}
\begin{minipage}[t]{0.5\linewidth}
\small
\begin{center}*T
\end{center}
\begin{tabular}{rl}
 & unde mîn hêrre im siges jach,\\ 
 & den man gegen im in kampfe sach.\\ 
 & der selbe hât betwungen mich\\ 
 & gar âne \textbf{hælinges} slich.\\ 
5 & man sach dâ viur ûz helmen wæn\\ 
 & unde swert in henden umbe \textbf{gên}."\\ 
 & \begin{large}D\end{large}ô \textbf{sprâchens} alle glîche,\\ 
 & beide arm unde rîche,\\ 
 & daz Key hete missetân.\\ 
10 & Hie suln wir diz mære lân\\ 
 & unde \textbf{komen} wider an die vart.\\ 
 & daz wüeste lant erbûwen wart,\\ 
 & dâ krône truoc Parcifal.\\ 
 & man sach dâ vröude unde schal.\\ 
15 & Sîn sweher Tampuntere\\ 
 & liez im ûf Peilrapeire\\ 
 & lieht gesteine \textbf{unde} rôte\textit{z} golt.\\ 
 & daz teilter, sô daz man im holt\\ 
 & was durch sîne milte.\\ 
20 & vil baniere, \textbf{niuwer} schilte,\\ 
 & \textbf{des} wart sîn lant gezieret\\ 
 & unde vil geturnieret\\ 
 & von im unde \textbf{von} den sînen.\\ 
 & er lie dicke ellen schînen\\ 
25 & an der marke sînes landes ort,\\ 
 & der junge degen \textbf{unervorht}.\\ 
 & sîn \textbf{tât} was \textbf{gegen} den gesten\\ 
 & geprüevet vür \textbf{die} besten.\\ 
 & \begin{large}N\end{large}û \textbf{sprechet} \textbf{ouch} von der künegîn.\\ 
30 & wie \textbf{m\textit{ö}htir} iemer baz gesîn?\\ 
\end{tabular}
\scriptsize
\line(1,0){75} \newline
T U V W \newline
\line(1,0){75} \newline
\textbf{7} \textit{Initiale} T U V W  \textbf{10} \textit{Majuskel} T  \textbf{15} \textit{Majuskel} T  \textbf{29} \textit{Initiale} T U V W  \newline
\line(1,0){75} \newline
\textbf{1} unde] [V*]: Vnde daz V Vnd das W \textbf{2} in kampfe sach] [kampe*]: kampen san U in streite sach W \textbf{5} dâ] do W  $\cdot$ wæn] varn V \textbf{6} umbe gên] wening sparn V vmbdreen W \textbf{7} alle glîche] algeliche V (W) \textbf{9} Key] keyn V \textbf{10} diz] dise U dir W \textbf{11} komen] [kvmen*]: kvmenz V \textbf{12} daz] [D*]: Daz V  $\cdot$ erbûwen] erbuͦwet U \textbf{13} dâ] Die U Do W  $\cdot$ Parcifal] parzifal T V partzifal W \textbf{14} Men sach [*]: da froide vnde schal V  $\cdot$ dâ] do W \textbf{15} Tampuntere] tampvntêre T tamputere V tampenteir W \textbf{16} im] in W  $\cdot$ Peilrapeire] pelrapeire T Peylrapere U belrepere V pelrapeir W \textbf{17} rôtez] rotes T \textbf{18} teilter] leiter U  $\cdot$ im] in U \textbf{20} baniere] bauͦyere U  $\cdot$ niuwer] in vwer U \textbf{24} ellen] allen U ellent V (W) \textbf{26} unervorht] vnd ervort U \textbf{27} tât] [*]: getat V getat W \textbf{28} die] [d*]: den V \textbf{29} Nû] No U  $\cdot$ sprechet ouch von] sprechet von U [*]: hoͤrent oͮch von V sprechen vor W \textbf{30} möhtir] mohtir T (U) moͤht [*]: der  V \newline
\end{minipage}
\end{table}
\end{document}
