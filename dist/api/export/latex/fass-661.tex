\documentclass[8pt,a4paper,notitlepage]{article}
\usepackage{fullpage}
\usepackage{ulem}
\usepackage{xltxtra}
\usepackage{datetime}
\renewcommand{\dateseparator}{.}
\dmyyyydate
\usepackage{fancyhdr}
\usepackage{ifthen}
\pagestyle{fancy}
\fancyhf{}
\renewcommand{\headrulewidth}{0pt}
\fancyfoot[L]{\ifthenelse{\value{page}=1}{\today, \currenttime{} Uhr}{}}
\begin{document}
\begin{table}[ht]
\begin{minipage}[t]{0.5\linewidth}
\small
\begin{center}*D
\end{center}
\begin{tabular}{rl}
\textbf{661} & \textbf{kom her}, der mich erkande,\\ 
 & der \textbf{mir sorge} wande."\\ 
 & \textit{\begin{large}D\end{large}}ô sprach \textbf{mîn hêr} Gawan:\\ 
 & "vrouwe, muoz ich mîn leben hân,\\ 
5 & sô wirt noch vreude an iu vernomen."\\ 
 & des selben tages solt ouch komen\\ 
 & mit her Artus der Bertun,\\ 
 & der \textbf{klagenden} Arniven sun,\\ 
 & durch \textbf{sippe} unt durch triwe.\\ 
10 & manege baniere niwe\\ 
 & sach Gawan gein im \textbf{trecken},\\ 
 & mit \textbf{rotte}\textbf{z} velt \textbf{verdecken},\\ 
 & von Logroys die strâzen her,\\ 
 & mit manegem \textbf{lieht} \textbf{gemâlem} sper.\\ 
15 & Gawane tet ir komen wol.\\ 
 & swer samnunge warten sol,\\ 
 & \textbf{den} lêrt \textbf{sûmen} \textbf{den} gedanc:\\ 
 & er vürhtet, sîn helfe \textbf{werde} kra\textit{n}c.\\ 
 & Artus Gawane \textbf{den} zwîvel brach.\\ 
20 & \textbf{âvoy}, wie man den komen sach!\\ 
 & Gawan sich \textbf{hal des} tougen,\\ 
 & daz sîniu liehten ougen\\ 
 & weinen muosen lernen.\\ 
 & zeiner zisternen\\ 
25 & \textbf{wæren} si beidiu \textbf{dô} enwiht,\\ 
 & wan si \textbf{habten}\textbf{s} wazzers niht.\\ 
 & Von der liebe was daz weinen,\\ 
 & daz Artus kunde erscheinen.\\ 
 & \multicolumn{1}{l}{ - - - }\\ 
 & \multicolumn{1}{l}{ - - - }\\ 
 & von kinde het er in erzogen.\\ 
30 & ir bêder triwe unerlogen\\ 
\end{tabular}
\scriptsize
\line(1,0){75} \newline
D \newline
\line(1,0){75} \newline
\textbf{3} \textit{Initiale} D  \textbf{27} \textit{Majuskel} D  \newline
\line(1,0){75} \newline
\textbf{3} Dô] ÷o D \textbf{18} kranc] chrach D \newline
\end{minipage}
\hspace{0.5cm}
\begin{minipage}[t]{0.5\linewidth}
\small
\begin{center}*m
\end{center}
\begin{tabular}{rl}
 & \textbf{kam}, der mich erkante\\ 
 & \textbf{oder} der \textbf{mir sorge} wante."\\ 
 & \begin{large}D\end{large}ô sprach \textbf{mîn hêrre} Gawan:\\ 
 & "vrowe, muoz ich mîn leben hân,\\ 
5 & sô wirt noch vröude an iu vernomen."\\ 
 & des selben tages solt ouch komen\\ 
 & mit her Artus de\textit{r} Britu\textit{n},\\ 
 & der \textbf{klagende} Ar\textit{niv}en sun,\\ 
 & durch \textbf{sippe} und durch triuwe.\\ 
10 & manic banier niuwe\\ 
 & sach Gawan gegen im \textbf{strecken},\\ 
 & mit \textbf{rotte}\textbf{z} velt \textbf{verdecken}.\\ 
 & \multicolumn{1}{l}{ - - - }\\ 
 & \multicolumn{1}{l}{ - - - }\\ 
15 & Gawan tet ir komen wol.\\ 
 & wer samenunge warten sol,\\ 
 & \textbf{der} l\textit{ê}ret \textbf{sûmen} \textbf{den} gedanc:\\ 
 & er vörhtet, sîn helfe \textbf{wær} kranc.\\ 
 & Artus Gawan \textbf{den} zwîvel brach.\\ 
20 & \textbf{owê}, wie man den komen sach!\\ 
 & Gawan sich \textbf{hal des} tougen,\\ 
 & daz sîniu liehten ougen\\ 
 & weinen muosen lernen.\\ 
 & zuo einer zisternen\\ 
25 & \textbf{wæren} si beidiu \textbf{dô} enwiht,\\ 
 & wan si \textbf{behabten} wazzers niht.\\ 
 & von der lieb\textit{e} was daz weinen,\\ 
 & daz Artus kunde ersch\textit{e}inen\\ 
 & triuwe, die an den beiden was,\\ 
 & ganz und lûter \textit{als}a\textit{m} ein glas.\\ 
 & von kinde het er in erzogen.\\ 
30 & ir beider triuwe unerlogen\\ 
\end{tabular}
\scriptsize
\line(1,0){75} \newline
m n o Fr69 \newline
\line(1,0){75} \newline
\textbf{3} \textit{Initiale} m   $\cdot$ \textit{Capitulumzeichen} n  \newline
\line(1,0){75} \newline
\textbf{1} mich] nich o \textbf{2} sorge] sorgen o  $\cdot$ wante] erwante n \textbf{3} Gawan] her [r]: gawan n \textbf{7} Artus] artuͯs o  $\cdot$ der] dem m  $\cdot$ Britun] brituͯm m brituͦn n britẏm o \textbf{8} Arniven] arunen m arniwen n \textbf{11} im] jmen n \textbf{13} \textit{Die Verse 661.13-14 fehlen} m n o  \textbf{14} ::: Fr69 \textbf{15} Gawan] Gawane Fr69 \textbf{16} wer] Vwer o Swer Fr69 \textbf{17} der lêret] Der latret m Der latet o den leret Fr69 \textbf{23} muosen] mussen m o muͯsse n mvͦse Fr69 \textbf{25} beidiu dô] beide de Fr69 \textbf{27} liebe] liebin m \textbf{28} kunde] kunne n [ku*]: kunde o ::: Fr69  $\cdot$ erscheinen] er schinen m ::: Fr69 \textbf{28} den] in n (o) Fr69 \textbf{28} alsam] dan m sam o Fr69 \textbf{29} er in] \textit{om.} n ::: Fr69 \newline
\end{minipage}
\end{table}
\newpage
\begin{table}[ht]
\begin{minipage}[t]{0.5\linewidth}
\small
\begin{center}*G
\end{center}
\begin{tabular}{rl}
 & \textbf{her kom}, der mich erkande,\\ 
 & der \textbf{mînen kumber} wande."\\ 
 & \begin{large}D\end{large}ô sprach \textbf{der werde} Gawan:\\ 
 & "vrouwe, muoz ich mîn leben hân,\\ 
5 & sô wirt noch vröude an iu vernomen."\\ 
 & des selben tages solde ouch komen\\ 
 & mit her Artus der Britun,\\ 
 & der \textbf{klagenden} Arniven sun,\\ 
 & durch \textbf{klage} unde durch triwe.\\ 
10 & manige banier niwe\\ 
 & sach Gawan gein im \textbf{trecke\textit{n}},\\ 
 & mit \textbf{rîtern} velt \textbf{verdecke\textit{n}},\\ 
 & von Logroys die strâze her,\\ 
 & mit manigem \textbf{lieht} \textbf{gemâltem} sper.\\ 
15 & Gawan tet ir komen wol.\\ 
 & swer samnunge warten sol,\\ 
 & \textbf{den} lêret \textbf{sunder} gedanc:\\ 
 & er vürhtet, sîn helfe \textbf{werde} kranc.\\ 
 & Artus Gawan \textbf{den} zwîvel brach.\\ 
20 & \textbf{âvoy}, wie man den komen sach!\\ 
 & Gawan sich \textbf{hal des} tougen,\\ 
 & daz sîniu liehten ougen\\ 
 & weinen muos\textit{t}e\textit{n} lernen.\\ 
 & zeiner zisternen\\ 
25 & \textbf{wâren} si beidiu \textbf{dô} enwiht,\\ 
 & wan si\textbf{ne} \textbf{behielten} \textbf{des} wazzers niht.\\ 
 & von der liebe was daz weinen,\\ 
 & daz Artus kunde erscheinen.\\ 
 & \multicolumn{1}{l}{ - - - }\\ 
 & \multicolumn{1}{l}{ - - - }\\ 
 & von kinde het er in erzogen.\\ 
30 & ir bêder triwe unerlogen\\ 
\end{tabular}
\scriptsize
\line(1,0){75} \newline
G I L M Z Fr45 \newline
\line(1,0){75} \newline
\textbf{3} \textit{Initiale} G I L Z  \textbf{19} \textit{Initiale} I  \newline
\line(1,0){75} \newline
\textbf{2} der] Vnd L (M) (Z) \textbf{3} Dô] Da M \textbf{4} mîn leben] minen lîp I \textbf{5} noch] auch I \textbf{6} selben] selbes L \textbf{7} mit] min I (M) (Fr45)  $\cdot$ Britun] prituͦn I Brittvn L brittuͦn Fr45 \textbf{8} klagenden] claginde M  $\cdot$ Arniven] arniuen I Arnive L arnẏuen Fr45 \textbf{9} klage] sippe Z \textbf{11} trecken] trechet G getrecken Z \textbf{12} rîtern] rotten Z  $\cdot$ verdecken] verdecchet G bedecken L (M) Z Fr45 \textbf{13} Logroys] logroẏs G Fr45 logrois M Z  $\cdot$ die] diu I  $\cdot$ strâze] strazzen Z \textbf{14} lieht] \textit{om.} I liht L lichten M (Fr45)  $\cdot$ gemâltem] gemalten L (M) \textbf{15} Gawan] Gawane M \textbf{16} swer] Wer L M \textbf{17} den] Der M  $\cdot$ lêret] lerte M  $\cdot$ sunder] svmen den Z \textbf{18} vürhtet] forcht L [worchte]: vorchte M  $\cdot$ werde] wuͯrde L \textbf{19} Gawan] Gawane L gewan M  $\cdot$ den] tet Z \textbf{20} âvoy] Owe L M  $\cdot$ den] \textit{om.} I \textbf{21} hal des] ez hal L des hal M des Z \textbf{22} liehten] lichten L (M) \textbf{23} muosten] moͮse G \textbf{25} \textit{Versdoppelung} Z   $\cdot$ dô] da M Z \textbf{26} sine behielten des] si behaltent des I sý behabtens L (Z) sie hatten des M \textbf{27} daz] da I \textbf{28} Artus] Artuͯs L  $\cdot$ erscheinen] erschinen M \textbf{29} Triwe wande er in von chinde het erzogen I  $\cdot$ in erzogen] yngezcogen M \textbf{30} unerlogen] vngelogen Z \newline
\end{minipage}
\hspace{0.5cm}
\begin{minipage}[t]{0.5\linewidth}
\small
\begin{center}*T
\end{center}
\begin{tabular}{rl}
 & \textbf{her kam}, der mich erkante,\\ 
 & der \textbf{mir sorgen} wante."\\ 
 & dô sprach \textbf{der werde} Gawan:\\ 
 & "vrou, muoz ich mîn leben hân,\\ 
5 & sô wirt noch vreude an iu vernomen."\\ 
 & des selben ta\textit{g}es solt ouch komen\\ 
 & mit her Artus der Britun,\\ 
 & der \textbf{klagenden} Arnyven sun,\\ 
 & durch \textbf{sippe} und durch triuwe.\\ 
10 & manec baniere niuwe\\ 
 & sach Gawan gên \textit{im} \textbf{trecken},\\ 
 & mit \textbf{rotte} \textbf{daz} velt \textbf{bedecken},\\ 
 & von Logrois die strâze her,\\ 
 & mit manegem \textbf{liehtem}, \textbf{gemâlem} sper.\\ 
15 & Gawan tet ir komen wol.\\ 
 & wer samenu\textit{n}g\textit{e} warten sol,\\ 
 & \textbf{der} lêrt \textbf{sûmen} \textbf{den} gedanc:\\ 
 & er vürhtet, sîn helfe \textbf{werde} kranc.\\ 
 & Artus Gawane \textbf{sîn} zwîvel brach.\\ 
20 & \textbf{âvoy}, wie man den \textbf{d\textit{â}} komen sach!\\ 
 & Gawan sich \textbf{des hal} tougen,\\ 
 & daz sîniu liehten ougen\\ 
 & weinen muosten lernen.\\ 
 & zuo einer zisternen\\ 
25 & \textbf{wâren} si beidiu \textbf{doch} enwiht,\\ 
 & wan si \textbf{behalten} \textbf{des} wazzers niht.\\ 
 & von der liebe was daz weinen,\\ 
 & daz Artus kunde erscheinen.\\ 
 & \multicolumn{1}{l}{ - - - }\\ 
 & \multicolumn{1}{l}{ - - - }\\ 
 & von kinde het er in erzogen.\\ 
30 & ir beider triuwe unerlogen\\ 
\end{tabular}
\scriptsize
\line(1,0){75} \newline
Q R W V \newline
\line(1,0){75} \newline
\textbf{3} \textit{Initiale} R W V  \newline
\line(1,0){75} \newline
\textbf{2} sorgen] sorge V \textbf{3} Gawan] gawann Q \textbf{5} vreude] froden R \textbf{6} tages] tates Q \textbf{7} mit] Min R (W)  $\cdot$ der] von R  $\cdot$ Britun] brittum Q Brituͦn R brittvn V \textbf{8} klagenden] clagende R  $\cdot$ Arnyven] arniuen Q V Arnẏuen R arnyuen W \textbf{10} manec] Mangen R \textbf{11} Gawan] gawin R  $\cdot$ im] \textit{om.} Q [*]: im V  $\cdot$ trecken] streken R (V) \textbf{13} \textit{Versfolge 661.14-13} W   $\cdot$ Logrois] logroys Q W Logris R [*is]: logrois V \textbf{14} liehtem] lichtem Q liechtten R (W) lieht V \textbf{15} Gawan] Gawin R Gawane W V \textbf{16} wer] Swer V  $\cdot$ samenunge] samenugen Q \textbf{17} der] Den R W V  $\cdot$ sûmen] sume R  $\cdot$ gedanc] ganck W \textbf{18} vürhtet] vorcht W [*]: voͤrhtet V  $\cdot$ helfe] froͯde R  $\cdot$ werde] [w*]: werde V \textbf{19} Gawane] gawan W  $\cdot$ sîn] den R W V \textbf{20} den dâ] den do Q do R den W V \textbf{21} Gawan] Gawin R  $\cdot$ sich] sach R  $\cdot$ des hal] des hals R hal dez V \textbf{22} sîniu liehten] seine lichten Q sine liechtú R \textbf{23} muosten] muͤssen W (V) \textbf{25} wâren] Weren R W (V)  $\cdot$ si] sy do R  $\cdot$ beidiu doch] do beidu R baide do W (V) \textbf{26} behalten] behaptten R behielten V \textbf{27} liebe] liebu R [lieb*]: liebe V  $\cdot$ daz] \textit{om.} W \textbf{28} Artus kunde] kunde artus W \textbf{29} het] hat W  $\cdot$ erzogen] gezogen W [*]: erzogen V \newline
\end{minipage}
\end{table}
\end{document}
