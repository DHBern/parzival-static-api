\documentclass[8pt,a4paper,notitlepage]{article}
\usepackage{fullpage}
\usepackage{ulem}
\usepackage{xltxtra}
\usepackage{datetime}
\renewcommand{\dateseparator}{.}
\dmyyyydate
\usepackage{fancyhdr}
\usepackage{ifthen}
\pagestyle{fancy}
\fancyhf{}
\renewcommand{\headrulewidth}{0pt}
\fancyfoot[L]{\ifthenelse{\value{page}=1}{\today, \currenttime{} Uhr}{}}
\begin{document}
\begin{table}[ht]
\begin{minipage}[t]{0.5\linewidth}
\small
\begin{center}*D
\end{center}
\begin{tabular}{rl}
\textbf{53} & er \textbf{stât} \textbf{hie} selbe \textbf{ouch} ame rê.\\ 
 & unvergolten dienest im tet \textbf{ze} wê."\\ 
 & ûf erde niht sô guotes was,\\ 
 & der helm von arde ein adamas,\\ 
5 & dicke unde herte,\\ 
 & \textbf{ame} strîte ein guot geverte.\\ 
 & dô lobte Hiutegers hant,\\ 
 & swenne er kœme in sînes hêrren lant,\\ 
 & \textbf{daz} er\textbf{z wolde} erwerben gar\\ 
10 & unt \textbf{senden wider} wol gevar.\\ 
 & daz tet er unbetwungen.\\ 
 & nâch urloube drungen\\ 
 & \textbf{zem} künege, swaz dâ vürsten was.\\ 
 & \textbf{dô} rûmten si den palas.\\ 
15 & swie \textbf{verwüestet wære} sî\textit{n} lant,\\ 
 & doch kunde Gahmuretes hant\\ 
 & swenken sölher gâbe solt,\\ 
 & als al \textbf{die} boume tr\textit{üe}ge\textit{n} go\textit{l}t.\\ 
 & \begin{large}E\end{large}r teilte \textbf{grôze gâbe}.\\ 
20 & sîne man, sîne mâge\\ 
 & nâmen \textbf{von im} des \textbf{heldes} guot.\\ 
 & daz was der küneginne muot.\\ 
 & der brûtloufte hôchgezît\\ 
 & hete dâ vor \textbf{manegen} grôzen strît.\\ 
25 & \textbf{die wurden} \textbf{sus} ze suone brâht.\\ 
 & i\textbf{ne} hân mir\textbf{s} \textbf{selbe niht} erdâht:\\ 
 & Man sagete \textbf{mir}, daz Isenhart\\ 
 & küneclîche bestatet wart.\\ 
 & daz tâten, die in erkanden.\\ 
30 & den zins von sînen landen,\\ 
\end{tabular}
\scriptsize
\line(1,0){75} \newline
D Fr9 \newline
\line(1,0){75} \newline
\textbf{19} \textit{Initiale} D  \textbf{27} \textit{Majuskel} D  \newline
\line(1,0){75} \newline
\textbf{1} selbe ouch] ouch selbe Fr9 \textbf{3} Gewẏnnet ouch im den helm der was Fr9 \textbf{4} der helm von arde] Von siner givte Fr9 \textbf{6} ame] Jn Fr9 \textbf{7} lobte] gelobete Fr9  $\cdot$ Hiutegers] Hv̂tegers D hutegeres Fr9 \textbf{12} drungen] do trvngen Fr9 \textbf{15} sîn] sint D \textbf{16} Gahmuretes] Gahmvretes D gamvretes Fr9 \textbf{18} trüegen] trvͦgent D (Fr9)  $\cdot$ golt] got \textit{nachträglich korrigiert zu:} golt D \textbf{19} Er] ÷r D \textbf{20} sîne mâge] die habe Fr9 \textbf{21} von im] vnde Fr9 \textbf{27} Isenhart] Jsenhart D \newline
\end{minipage}
\hspace{0.5cm}
\begin{minipage}[t]{0.5\linewidth}
\small
\begin{center}*m
\end{center}
\begin{tabular}{rl}
 & er \textbf{stât} \textbf{hie} selbe \textbf{ouch} anme rê.\\ 
 & unvergolten dienst im tet \textbf{ze} wê."\\ 
 & ûf erde niht sô guotes was,\\ 
 & der helme von art ein adamas,\\ 
5 & dicke und herte,\\ 
 & \textbf{anme} strîte ein guot geverte.\\ 
 & \begin{large}D\end{large}ô lobete Hutegers hant,\\ 
 & wenne er kæme in sînes hêrren lant,\\ 
 & \textbf{daz} er \textbf{ez wolte} erwerben gar\\ 
10 & und \textbf{senden wider} wol gevar.\\ 
 & daz tet er unbetwungen.\\ 
 & nâch urloube drungen\\ 
 & \textbf{zem} künige, waz d\textit{â} vürsten was.\\ 
 & \textbf{dô} rûmten si den palas.\\ 
15 & wie \textbf{verwüestet wære} sîn lant,\\ 
 & doch kunde Gahmuretes hant\\ 
 & \dag wenken\dag  solicher gâbe solt,\\ 
 & als alle boume trüegen golt.\\ 
 & er teilte\textbf{s} \textbf{sunder wâge}.\\ 
20 & sîne man \textbf{und} sîne mâge\\ 
 & nâmen \textbf{d\textit{â}} des \textbf{heldes} guot.\\ 
 & daz was der küniginne muot.\\ 
 & der brûtlouf hôchgezît\\ 
 & hete dâ vor grôzen strît.\\ 
25 & \textbf{der was} \textbf{alsus} ze suone brâht.\\ 
 & \textit{i}\textbf{\textit{n}e} hâ\textit{n} \textit{m}ir\textbf{s} \textbf{selbe niht} \textit{erdâht}:\\ 
 & man sagete \textbf{mir}, daz Ysenhart\\ 
 & küniclîchen bestatet wart.\\ 
 & daz tâten, die in erkanden.\\ 
30 & den zins von sînen landen,\\ 
\end{tabular}
\scriptsize
\line(1,0){75} \newline
m n o \newline
\line(1,0){75} \newline
\textbf{7} \textit{Initiale} m n o  \newline
\line(1,0){75} \newline
\textbf{1} er] Es n  $\cdot$ hie selbe ouch] ouch hie selb n  $\cdot$ anme rê] an mere \textit{nachträglich korrigiert zu:} mee m an me n o \textbf{2} ze] so o \textbf{3} erde] erden n o \textbf{4} der] Dem o  $\cdot$ adamas] adamast m \textbf{6} anme] An o \textbf{7} lobete] gelobete n  $\cdot$ Hutegers] huttegers m huttigers n huͯttigers o \textbf{8} kæme] kein o \textbf{12} drungen] rungen n (o) \textbf{13} dâ] do m n o \textbf{15} verwüestet] werwuͯstet o \textbf{16} Gahmuretes] Gahmurettes m gamiretes n gamuͯretes o \textbf{17} gâbe solt] boume solt n gobe [dort]: solt o \textbf{18} als] Als obe n  $\cdot$ trüegen] triegent n \textbf{21} dâ] do m n o  $\cdot$ heldes] helles o \textbf{22} küniginne] konig o \textbf{25} was] wart n o  $\cdot$ alsus ze suone] als suͯs zuͦsyme o \textbf{26} ine hân mirs] Me han ich mirs m Jch han mirs n o  $\cdot$ selbe] selber n  $\cdot$ niht erdâht] nit \textit{nachträglich korrigiert zu:} nit erdacht m nit bedacht n nit [ge]: bedaht o \textbf{27} Ysenhart] jsenhart m n ẏsenhart o \textbf{30} landen] handen n o \newline
\end{minipage}
\end{table}
\newpage
\begin{table}[ht]
\begin{minipage}[t]{0.5\linewidth}
\small
\begin{center}*G
\end{center}
\begin{tabular}{rl}
 & er \textbf{stêt} \textbf{hie} selbe an dem rê.\\ 
 & unvergolten dienst im tet wê."\\ 
 & ûf erde niht sô guotes was,\\ 
 & der helm von arde ein adamas,\\ 
5 & dick und herte,\\ 
 & \textbf{an} strîte ein guot geverte.\\ 
 & dô lobte Hutegers hant,\\ 
 & swenner k\textit{œ}me in sînes hêrren lant,\\ 
 & \textbf{\begin{large}D\end{large}az} er\textbf{z wolde} erwe\textit{r}ben gar\\ 
10 & und \textbf{wider senden} wol gevar.\\ 
 & daz teter unbetwungen.\\ 
 & nâch urloube drungen\\ 
 & \textbf{zem} künige, swaz dâ vürsten was.\\ 
 & \textbf{dô} rûmten si den palas.\\ 
15 & swie \textbf{verwüest wære} \textit{sîn} lant,\\ 
 & doch kunde Gahmuretes hant\\ 
 & swenken solher gâbe solt,\\ 
 & als al \textbf{die} boume tr\textit{üe}gen golt.\\ 
 & er teilte \textbf{grôze gâbe}.\\ 
20 & sîne man, sîne mâge\\ 
 & nâmen \textbf{dâ} des \textbf{küniges} guot.\\ 
 & daz was der küniginne muot.\\ 
 & \multicolumn{1}{l}{ - - - }\\ 
 & \multicolumn{1}{l}{ - - - }\\ 
25 & \multicolumn{1}{l}{ - - - }\\ 
 & \multicolumn{1}{l}{ - - - }\\ 
 & man saget \textbf{uns}, daz Ysenhart\\ 
 & küniclîche bestatet wart.\\ 
 & daz tâten, die in erkanden.\\ 
30 & den zins von sînen landen,\\ 
\end{tabular}
\scriptsize
\line(1,0){75} \newline
G I O L M Q R Z \newline
\line(1,0){75} \newline
\textbf{1} \textit{Initiale} O M  \textbf{9} \textit{Initiale} G  \textbf{27} \textit{Initiale} I L Q Z  \newline
\line(1,0){75} \newline
\textbf{1} \textit{Die Verse 48.21-54.6 fehlen} R   $\cdot$ er] ÷r O Der M Es Q  $\cdot$ hie] \textit{om.} Q  $\cdot$ selbe] selbe nach O (Z) noch selbe L selben noch M noch selber Q \textbf{2} im tet] det ým L  $\cdot$ wê] ze we O (M) (Z) \textbf{3} erde] erdin M (Q) \textbf{6} an] An dem O (M) (Z) Ym Q  $\cdot$ guot] gvͦte O \textbf{7} dô] Nv O L (M) (Q)  $\cdot$ lobte] lobt O  $\cdot$ Hutegers] huͤtegers I hvͦtegers O huttegeres L Z hutegeres Q \textbf{8} swenner] Wenne er L (M) (Q)  $\cdot$ kœme] chome G chom I O (Q) \textbf{9} erz] er Q  $\cdot$ erwerben] erweben G \textbf{10} wider senden] senden wider O L (M) Q Z \textbf{12} drungen] do drungen I gedrvngen L \textbf{13} swaz] waz L (Q) Z  $\cdot$ dâ] do O Q \textbf{14} \textit{Die Verse 52.3-8 folgen auf 53.14} Z   $\cdot$ dô] Da Z \textbf{15} swie] Swie gar O Wie L (M) Q  $\cdot$ verwüest] verwuͤhtstet I verwæiset O verbustet Q  $\cdot$ sîn] daz G \textbf{16} Gahmuretes] Gahmurets G Gamvretes O Gahmuͯretes L gamuretis M gamúretes Q Gamuretes Z \textbf{17} So digke sulde her geben solt M  $\cdot$ swenken] swenden I \textbf{18} als] Sam O L M Q Z  $\cdot$ die] \textit{om.} Z  $\cdot$ trüegen] troͮgen G (M) (Q) (Z) \textbf{20} sîne] sinen I Syneme M Seinē Q  $\cdot$ man] mannen I  $\cdot$ sîne mâge] vnd sinen magen I syneme mage M vnd seine mage Q \textbf{21} dâ] von im O (L) Q Z von M  $\cdot$ des küniges] des herren O (Q) Z [de*]: dez heldes L den herren M \textbf{23} \textit{Die Verse 53.23-26 fehlen} G I   $\cdot$ Der brvͦfte (bruͯtlaufte L [ M Z ]) hoh gezit (hochczyd M [ Z ]) O (L) (M) (Z)  $\cdot$ Dy vor der brautlaufte hochzeit Q \textbf{24} Het da vor mangen (mannigen groszin M [ Z ]) strit O (L) (M) (Z)  $\cdot$ Dor vor heten manchen streyt Q \textbf{25} Die (Do M ) wurden svs (sie M ) ze svͦne (syne M sunne Q ) braht O (L) (M) (Q) (Z) \textbf{26} Jch (Jchen Q [ Z ]) han mirs (mir Q Z ) selbe (selben Q selbern M ) niht gedaht (erdaht L [ M Z ]) O (L) (M) (Q) (Z) \textbf{27} saget] sagete L (Q)  $\cdot$ uns] mir O L (M) Q Z  $\cdot$ Ysenhart] ẏsenhart G Jsenhart M eyszenhart Q isenhart Z \textbf{28} bestatet] bestanden M \textbf{30} sînen] den M \newline
\end{minipage}
\hspace{0.5cm}
\begin{minipage}[t]{0.5\linewidth}
\small
\begin{center}*T (U)
\end{center}
\begin{tabular}{rl}
 & er \textbf{stuont} \textbf{noch} selbe an dem rê.\\ 
 & unvergolten dienst im tet wê."\\ 
 & ûf erd\textit{e n}iht sô guotes was,\\ 
 & der helm von arde ein adamas,\\ 
5 & dicke und herte,\\ 
 & \textbf{an} strîte ein guot geverte.\\ 
 & dô lobet Hutegers hant,\\ 
 & wan er k\textit{æ}m in sînes hêrren lant,\\ 
 & er \textbf{woltez im} erwerben gar\\ 
10 & und \textbf{senden wider} wol gevar.\\ 
 & daz tet er unbetwungen.\\ 
 & nâch urloube \textbf{dô} drungen\\ 
 & \textbf{zuo einem} künege, waz d\textit{â} vürsten was.\\ 
 & \textbf{sus} rûmten si den palas.\\ 
15 & "wie \textbf{verwüesteten wir} sîn lant!"\\ 
 & doch kunde Gahmuretes hant\\ 
 & swenken sol\textit{h}er g\textit{â}b\textit{e} solt,\\ 
 & als alle \textbf{die} boume trüegen golt.\\ 
 & er teilt \textbf{golt ân wâge}.\\ 
20 & sîn man \textbf{und} sîne mâge\\ 
 & nâmen \textbf{von im} des \textbf{landes} guot.\\ 
 & daz was der küneginne muot.\\ 
 & d\textit{er} brûtloufte hôchzît\\ 
 & het dâ vo\textit{r} \textbf{\textit{m}anegen} grôzen strît.\\ 
25 & \textbf{die wurden} \textbf{sô} zuo suone brâht.\\ 
 & ich hân mir \textbf{ez} \textbf{niht selbe} erdâht:\\ 
 & man saget \textbf{mir}, daz Isenhart\\ 
 & küneclîche bestatet wart.\\ 
 & daz tâten, die in erkanden.\\ 
30 & den zins von sî\textit{n}e\textit{n} lande\textit{n},\\ 
\end{tabular}
\scriptsize
\line(1,0){75} \newline
U V W T \newline
\line(1,0){75} \newline
\textbf{7} \textit{Majuskel} T  \textbf{12} \textit{Majuskel} T  \textbf{14} \textit{Majuskel} T  \textbf{15} \textit{Majuskel} T  \textbf{27} \textit{Initiale} V W T  \newline
\line(1,0){75} \newline
\textbf{1} stuont] stot V (W) (T)  $\cdot$ selbe] selber V selbes W  $\cdot$ an dem] hie am W vf dem T \textbf{2} im tet] thet im W (T) \textbf{3} vf der erde was niht so gvͦt T  $\cdot$ erde] erden V  $\cdot$ niht] nit nit U \textbf{4} der helm eins adamas flvͦt T  $\cdot$ helm] \textit{om.} W \textbf{6} an] an dem T  $\cdot$ guot] gvͦte T \textbf{7} dô] Nv T  $\cdot$ lobet] lobete V W lobete ouch T  $\cdot$ Hutegers] Hútigers V (W) Hvtegeres T \textbf{8} wan er] swen [e*]: er V swenner T  $\cdot$ kæm] quam U  $\cdot$ hêrren] vatters W \textbf{9} daz erz wolte werben gar T  $\cdot$ im] in W \textbf{10} senden] [*]: senden V in senden T \textbf{12} dô] da V \textit{om.} T \textbf{13} zuo einem] Zuͦm V (T) Zuͦ dem W  $\cdot$ waz] swas V (T)  $\cdot$ dâ] do U V W \textbf{14} sus] Do T  $\cdot$ den] des W \textbf{15} wie] Swie V Swie doch T  $\cdot$ verwüesteten wir] verwuͦstet were V (W) (T) \textbf{16} Gahmuretes] Gahmuͦretes U Gamuretes V (W) \textbf{17} swenken] Swenden V  $\cdot$ solher gâbe] sol er gebin U \textbf{18} als] sam T  $\cdot$ die] \textit{om.} T  $\cdot$ boume] [*]: berge V \textbf{19} teilt] teilte V (W) T  $\cdot$ golt ân wâge] [g*]: grosze gobe V grôze gabe T \textbf{20} sîn man] sine [man*]: manne V  $\cdot$ sîne mâge] [*]: vnde moge V \textbf{21} nâmen] Nam W  $\cdot$ von im des landes] von im des [*]: herren V von im des herren W da des vursten T \textbf{23} der] Die U Des W  $\cdot$ brûtloufte] [*]: brvnloͮfte V brautlauffes W \textbf{24} vor manegen] vor in manigen U vor [*]: vor manigen V  $\cdot$ grôzen] \textit{om.} V T \textbf{25} sô] da V svs T \textbf{26} niht selbe] selben niht V T selber nit W \textbf{27} saget] [*]: sagete V sagete T  $\cdot$ Isenhart] Jsenhart U T Jsinhart V ysenhart W \textbf{28} wart] ware W \textbf{30} sînen landen] sime lande U \newline
\end{minipage}
\end{table}
\end{document}
