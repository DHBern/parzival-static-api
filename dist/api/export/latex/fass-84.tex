\documentclass[8pt,a4paper,notitlepage]{article}
\usepackage{fullpage}
\usepackage{ulem}
\usepackage{xltxtra}
\usepackage{datetime}
\renewcommand{\dateseparator}{.}
\dmyyyydate
\usepackage{fancyhdr}
\usepackage{ifthen}
\pagestyle{fancy}
\fancyhf{}
\renewcommand{\headrulewidth}{0pt}
\fancyfoot[L]{\ifthenelse{\value{page}=1}{\today, \currenttime{} Uhr}{}}
\begin{document}
\begin{table}[ht]
\begin{minipage}[t]{0.5\linewidth}
\small
\begin{center}*D
\end{center}
\begin{tabular}{rl}
\textbf{84} & diu werde Waleisinne.\\ 
 & si twanc iedoch sîn minne.\\ 
 & Er saz \textbf{vür} \textbf{si} sô nâhe nider,\\ 
 & daz si in begreif unt zôch in wider\\ 
5 & anderhalp \textbf{vaste} an ir lîp.\\ 
 & si was ein maget \textbf{unt} niht ein wîp,\\ 
 & diu in sô nâhen sitzen liez.\\ 
 & \textbf{welt ir nû hœren}, wie \textbf{si} hiez?\\ 
 & diu küneginne Herzeloyde,\\ 
10 & \textbf{unt} ir base Rischoyde,\\ 
 & die hete der künec Kaylet.\\ 
 & des muomen sun was Gahmuret.\\ 
 & \begin{large}V\end{large}rou Herzeloyde gap den schîn,\\ 
 & wæren \textbf{erloschen gar} die kerzen sîn,\\ 
15 & \textbf{dâ} wære doch lieht von ir genuoc.\\ 
 & wan daz \textbf{grôz} jâmer undersluoc\\ 
 & die hœhe an sîner vreude breit,\\ 
 & sîn minne wære ir vil bereit.\\ 
 & Si sprâchen gruoz nâch \textbf{zühte} kür.\\ 
20 & bî einer wîle giengen schenken vür\\ 
 & mit gezierde von Azagouc,\\ 
 & dâr an grôziu rîcheit niemen trouc.\\ 
 & die truogen junchêrren în.\\ 
 & \textbf{daz} muosen tiure nepfe sîn\\ 
25 & von edelem gesteine,\\ 
 & \textbf{wît}, niht ze kleine.\\ 
 & si wâren alle sunder golt.\\ 
 & \textbf{ez} was des \textbf{landes zinses} solt,\\ 
 & \textbf{daz} Isenhart vil dicke bôt\\ 
30 & vrôn Belakanen vür grôze nôt.\\ 
\end{tabular}
\scriptsize
\line(1,0){75} \newline
D \newline
\line(1,0){75} \newline
\textbf{3} \textit{Majuskel} D  \textbf{13} \textit{Initiale} D  \textbf{19} \textit{Majuskel} D  \newline
\line(1,0){75} \newline
\textbf{9} Herzeloyde] Herzelôyde D \textbf{10} Rischoyde] Riscôyde D \textbf{12} Gahmuret] Gahmvret D \textbf{13} Herzeloyde] Herzelôyde D \textbf{21} Azagouc] Azagoͮch D \textbf{29} Isenhart] Jsenhart D \newline
\end{minipage}
\hspace{0.5cm}
\begin{minipage}[t]{0.5\linewidth}
\small
\begin{center}*m
\end{center}
\begin{tabular}{rl}
 & diu werde Waleis\textit{i}nne.\\ 
 & si twanc iedoch sîn minne.\\ 
 & er saz \textbf{vür} \textbf{si} sô nâhe nider,\\ 
 & daz si in begreif und zôch in wider\\ 
5 & anderhalp \textbf{vaste} an ir lîp.\\ 
 & si was ein maget, niht ein wîp,\\ 
 & diu in sô nâhen sitzen liez.\\ 
 & \textbf{wollet ir nû hœren}, wie \textbf{si} hiez?\\ 
 & \textbf{si hiez} diu künigîn Herczeloide.\\ 
10 & ir base \textbf{hiez} Ritschoide,\\ 
 & die hete der künic Kailet.\\ 
 & des muomen sun was Gahmuret.\\ 
 & vrowe Herczeloide gap den schîn,\\ 
 & wæren \textbf{erloschen gar} die kerzen sîn,\\ 
15 & \textbf{d\textit{â}} wære doch lieht von ir genuoc.\\ 
 & wanne daz \textbf{grô\textit{z}} jâmer undersluoc\\ 
 & die hœhe an sîner vröude breit,\\ 
 & sîn minne wær ir vil bereit.\\ 
 & si sprâchen gruoz nâch \textbf{zühte} kür.\\ 
20 & bî einer wîle giengen schenken vür\\ 
 & mit gezierde von A\textit{z}agouc,\\ 
 & dâr an grôz rîcheit niemen trouc.\\ 
 & die truogen junchêrren în.\\ 
 & \textbf{daz} muosen tiure nepfe sîn\\ 
25 & von edelem gesteine,\\ 
 & \textbf{grôz}, niht ze kleine.\\ 
 & si wâren alle sunder golt.\\ 
 & \textbf{ez} was des \textbf{lan\textit{t}zin\textit{s}es} solt,\\ 
 & \textbf{daz} Ysenhart vil dicke bôt\\ 
30 & vrouwen Belakanen vür grôze nôt.\\ 
\end{tabular}
\scriptsize
\line(1,0){75} \newline
m n o \newline
\line(1,0){75} \newline
\newline
\line(1,0){75} \newline
\textbf{1} Waleisinne] waleisunne m walesinne n \textbf{2} sîn] \textit{om.} o \textbf{6} niht] vnd nit n \textbf{8} nû] im o \textbf{9} Herczeloide] hertzoleide n herczeleide o \textbf{10} Ritschoide] ritter scheide o \textbf{11} Kailet] gaẏlet n o \textbf{12} Gahmuret] gahmmuret m gamúret n gamuͯret o \textbf{13} vrowe] Freide o  $\cdot$ Herczeloide] hertzeloide n herczeleide o \textbf{14} kerzen sîn] kertzin n (o) \textbf{15} dâ] Do m n o \textbf{16} grôz] grosse m n o  $\cdot$ jâmer] lant o \textbf{17} vröude] freuͯiden o \textbf{18} ir] in n \textbf{19} si] [Di]: Si m \textbf{21} Azagouc] aragoug m n aragant o \textbf{23} junchêrren] juncker o \textbf{24} \textit{Vers 84.24 fehlt} o   $\cdot$ muosen] muͯssent n \textbf{27} si] Die o \textbf{28} lantzinses] lanczines m lanczensses o \textbf{29} Ysenhart] Isenhart m ẏsenhart n o \textbf{30} vrouwen] Frouwe m n (o)  $\cdot$ Belakanen] belakannen m  $\cdot$ vür] \textit{om.} o \newline
\end{minipage}
\end{table}
\newpage
\begin{table}[ht]
\begin{minipage}[t]{0.5\linewidth}
\small
\begin{center}*G
\end{center}
\begin{tabular}{rl}
 & diu werde Waleisinne.\\ 
 & si twanc iedoch sîn minne.\\ 
 & er saz \textbf{vor} \textbf{ir} sô nâhen nider,\\ 
 & daz sin begreif und zôch in wider\\ 
5 & anderhalp an ir lîp.\\ 
 & si was ein maget, niht ein wîp,\\ 
 & diu in sô nâhen sitzen liez.\\ 
 & \textbf{nû hœre\textit{t}}, \textit{w}ie \textbf{diu} hiez,\\ 
 & diu künigîn Herzeloide -\\ 
10 & \textbf{unde} ir base Ritschoide.\\ 
 & \multicolumn{1}{l}{ - - - }\\ 
 & \multicolumn{1}{l}{ - - - }\\ 
 & vrô Herzeloide gap den schîn,\\ 
 & wæren \textbf{erloschen gar} die kerzen sîn,\\ 
15 & \textbf{dâ} wære doch lieht von ir genuoc.\\ 
 & wan daz \textbf{grôz} jâmer undersluoc\\ 
 & die hœhe an sîner vröude breit,\\ 
 & sîn minne wære ir vil bereit.\\ 
 & si sprâchen gruoz nâch \textbf{zühten} kür.\\ 
20 & bî einer wîle giengen schenken vür\\ 
 & mit gezierde von Azagouc,\\ 
 & dâr an grôz rîcheit niemen trouc.\\ 
 & die truogen junchêrren în.\\ 
 & \textbf{ez} muosen tiure nepfe sîn\\ 
25 & von edelem gesteine,\\ 
 & \textbf{wît}, niht ze kleine.\\ 
 & si wâren alle sunder golt.\\ 
 & \textbf{\begin{large}D\end{large}az} was des \textbf{landes zinses} solt,\\ 
 & \textbf{den} Ysenhart vil dicke bôt\\ 
30 & vrôn Belacanen vür grôze nôt.\\ 
\end{tabular}
\scriptsize
\line(1,0){75} \newline
G I O L M Q R Z Fr50 \newline
\line(1,0){75} \newline
\textbf{1} \textit{Initiale} O  \textbf{3} \textit{Initiale} L R Z  \textbf{9} \textit{Initiale} I  \textbf{28} \textit{Initiale} G  \textbf{29} \textit{Initiale} I  \newline
\line(1,0){75} \newline
\textbf{1} diu] ÷iv O  $\cdot$ Waleisinne] waleisẏnne L waleisynne M waleysinne Q walassine R \textbf{2} si] Dy Q  $\cdot$ sîn minne] sine sinne R \textbf{3} er] der I Es Q  $\cdot$ ir] si Fr50 \textbf{4} daz sin] Daz si O Das in Q Da sy in R  $\cdot$ wider] nider O (Q) \textbf{5} an] hin an O L Q Z vaste an Fr50 \textbf{6} niht] vnd niht I (O) (M) (Q) (Fr50) \textbf{8} hœret wie] horet reht wie G mvgt ir horen wie O (L) (M) (Q) (R) (Z)  $\cdot$ diu] si O (Z) \textbf{9} Herzeloide] herzelaude I herzelavde O Hertzelayde L herczeloide M herzeloúde Q herczelaude R herzelovde Z \textbf{10} Ritschoide] rischoide I Richavde O Ritschauͯde L richowde Q Ritschaude R ritschovde Z \textbf{11} \textit{Die Verse 84.11-12 fehlen} G I   $\cdot$ Die het der chunich kaylet (kailet M kalet Q [Kayler]: Kaylet R Gailet Z ) O (L) (M) (Q) (R) (Z) \textbf{12} Des mvͦmen svn was [Gamoret]: Gamvret (Gahmuͯret L gamuret M Q Z gahmuret R ) O (L) (M) (Q) (R) (Z) \textbf{13} Herzeloide] herzenlaude I herzelavde O Hertzelauͯde L herczeloide M herzeloude Q herczelaude R herzelovde Z  $\cdot$ gap den] gabt den I gab O \textbf{14} wæren] warn I Were M  $\cdot$ erloschen gar] gar erloschen O Q R erloschen L \textbf{15} dâ] Do Q  $\cdot$ doch] auch I  $\cdot$ lieht] liehtes Z  $\cdot$ von ir] von in I vor in O vor ir Q \textbf{16} grôz] ez grozzer I iz groz O \textbf{17} vröude] vreuden I \textbf{18} wære] warn I  $\cdot$ ir] in O om M ir gar Q  $\cdot$ vil] wol R \textbf{19} sprâchen] sprach L R  $\cdot$ gruoz] groz O  $\cdot$ zühten] zuhte I (L) (M) (R) \textbf{20} giengen] gie O \textbf{21} mit] Mir Q  $\cdot$ gezierde] [getierde]: gezierde G  $\cdot$ Azagouc] azagoͮch G azagoͮc I azagovch O (L) azagauck Q argavo R azagovc Z \textbf{22} rîcheit] Richart R  $\cdot$ niemen] nienen I niht O  $\cdot$ trouc] betravch O \textbf{23} truogen] trúngen Q \textbf{24} ez] Daz Z  $\cdot$ nepfe] nepfin Q \textbf{26} niht] vnd niht L (Q) \textbf{28} \textit{Versfolge 84.29-30-28} Q   $\cdot$ Daz] Ez L  $\cdot$ landes] lant O L \textbf{29} Ysenhart] ẏsenhart G jsenhart L (R) Jsenart M eysjnhart Q isenhart Z \textbf{30} vrôn] Vrov L (Q) (R) Vor en M  $\cdot$ Belacanen] bellicanen I Belanen O Belecanen L belacane M Bellaconen R belacan Z \newline
\end{minipage}
\hspace{0.5cm}
\begin{minipage}[t]{0.5\linewidth}
\small
\begin{center}*T (U)
\end{center}
\begin{tabular}{rl}
 & diu werde Waleisinne.\\ 
 & si twanc iedoch sîn minne.\\ 
 & er saz \textbf{von} \textbf{ir} sô nâhe nider,\\ 
 & daz si in begreif und zôch in wider\\ 
5 & \textbf{hin} anderhalben an ir lîp.\\ 
 & si was ein maget, niht ein wîp,\\ 
 & diu in sô nâhe sitzen liez.\\ 
 & \textbf{wolt ir nû hœren}, wie \textbf{diu} hiez?\\ 
 & diu küneginne Herzeloyde,\\ 
10 & \textbf{und} ir base Rischoyde,\\ 
 & die hete der künec Kaylet.\\ 
 & des muomen sun was Gahmuret.\\ 
 & vrô Herzeloyde gap den schîn,\\ 
 & wæren \textbf{gar erleschet} die kerzen sîn,\\ 
15 & \textbf{sô} wære doch lieht von ir genuoc.\\ 
 & wan daz \textbf{sîn} jâmer undersluoc\\ 
 & die hœhe an sîner vreude breit,\\ 
 & sîn minne wære ir vil bereit.\\ 
 & si sprâchen gruoz nâch \textbf{zühte} kür.\\ 
20 & bî einer wîlen giengen schenken vür\\ 
 & mit gezierde von Azagouc,\\ 
 & dâr an grôz rîcheit nieman trouc.\\ 
 & die truogen junchêrren în.\\ 
 & \textbf{ez} muosen tiure nepfe sîn\\ 
25 & von edeleme gesteine,\\ 
 & \textbf{wît}, niht zuo kleine.\\ 
 & si wâren alle sunder golt.\\ 
 & \textbf{ez} was des \textbf{landes zinses} \textit{s}olt,\\ 
 & \textbf{den} Isenhart vil dicke bôt\\ 
30 & v\textit{rou}n Belakanen vür grôze nôt.\\ 
\end{tabular}
\scriptsize
\line(1,0){75} \newline
U V W T \newline
\line(1,0){75} \newline
\textbf{5} \textit{Initiale} W  \textbf{9} \textit{Majuskel} T  \newline
\line(1,0){75} \newline
\textbf{1} Waleisinne] wallessinne V waleisynne W \textbf{3} Er satzte so nahe von ir nyder W  $\cdot$ von] vor V T \textbf{4} si] \textit{om.} W  $\cdot$ zôch in] zoch [*]: in wider V [z*]: zoch T \textbf{5} hin] \textit{om.} W T \textbf{6} niht] vnde niht V \textbf{8} wolt ir nû] nv mvget ir T  $\cdot$ diu] [si]: sv́ V sy W (T) \textbf{9} Herzeloyde] herzeleide U hertzelaude V hertzeloyde W \textbf{10} Rischoyde] ritterscheide U Ritschaude V ritsoyde W Ritschôye T \textbf{11} \textit{Versdoppelung:} Do hete der kuͦnec kaylet U   $\cdot$ Kaylet] kaẏlet V gaylet W \textbf{12} Gahmuret] Gahmuͦret U Gamuret V (W) \textbf{13} Herzeloyde] herzeleit U Hertzelaude V hertzeloyde W \textbf{14} gar erleschet] gar erloschen W erloscen T \textbf{15} sô] da T \textbf{16} sîn] ein V W ez grôz T \textbf{18} mir wart svnder wân geseit T \textbf{19} sprâchen] sprach W sprêche T  $\cdot$ zühte] zúchten W \textbf{20} bî einer wîlen] bi einer wile V (W) dar nach T \textbf{21} gezierde] zierde W  $\cdot$ Azagouc] azaguͦc U Azagoͮg V azagoug W Azagôvc T \textbf{22} dâr an] Dar in W an den T  $\cdot$ grôz rîcheit nieman] [*]: nieman gros richeit V \textbf{23} junchêrren] Juͦncherre U \textbf{26} niht] vnd nicht W (T) \textbf{28} solt] golt U \textbf{29} Isenhart] Jsenhart U T Jsinhart V ysenhart W \textbf{30} vroun] Vor in U Frauw W  $\cdot$ Belakanen] Balacanen U Belecanen V Belacanen T \newline
\end{minipage}
\end{table}
\end{document}
