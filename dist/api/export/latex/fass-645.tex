\documentclass[8pt,a4paper,notitlepage]{article}
\usepackage{fullpage}
\usepackage{ulem}
\usepackage{xltxtra}
\usepackage{datetime}
\renewcommand{\dateseparator}{.}
\dmyyyydate
\usepackage{fancyhdr}
\usepackage{ifthen}
\pagestyle{fancy}
\fancyhf{}
\renewcommand{\headrulewidth}{0pt}
\fancyfoot[L]{\ifthenelse{\value{page}=1}{\today, \currenttime{} Uhr}{}}
\begin{document}
\begin{table}[ht]
\begin{minipage}[t]{0.5\linewidth}
\small
\begin{center}*D
\end{center}
\begin{tabular}{rl}
\textbf{645} & \begin{large}D\end{large}er knappe, den si \textbf{knien dâ} sach.\\ 
 & diu künegîn zem brieve sprach:\\ 
 & "\textbf{Ôwol} der hant, diu dich schreip!\\ 
 & âne sorge ich nie beleip\\ 
5 & sît des tages, daz ich sach\\ 
 & \textbf{die} hant, \textbf{von der diu} schrift geschach."\\ 
 & si weinde \textbf{sêre} und was doch vrô.\\ 
 & hin zem knappen sprach si dô:\\ 
 & "Dû bist Gawans kneht."\\ 
10 & "jâ, vrouwe, \textbf{der} enbiutet iu sîn reht,\\ 
 & dienstlîche triwe ân allen wanc\\ 
 & unt dâ bî sîne vreude kranc,\\ 
 & ir \textbf{en}wellet im vreude machen hôch.\\ 
 & sô kumberlîch ez sich gezôch\\ 
15 & nie umb al sîn êre.\\ 
 & Vrouwe, er enbiut iu mêre,\\ 
 & daz er mit werden vreuden lebe,\\ 
 & \textbf{unt} vreischet \textbf{iwer} trôstes gebe.\\ 
 & ir mugt wol an dem brieve sehen\\ 
20 & mêre, denn ich\textbf{s} \textbf{iu} \textbf{kunne} \textbf{jehen}."\\ 
 & Si sprach: "ich hân vür wâr erkant,\\ 
 & durch waz dû \textbf{zuo mir bist} gesant.\\ 
 & ich tuon im \textbf{werdiu} dienst dar\\ 
 & mit \textbf{wünneclîcher} vrouwen schar,\\ 
25 & die vür wâr bî mîner zît\\ 
 & \textbf{an prîse vor ûz hânt} den strît.\\ 
 & âne Parzivals wîp\\ 
 & unt âne Orgelusen lîp\\ 
 & sô\textbf{ne} \textbf{erkenne} ich ûf der erde\\ 
30 & bî toufe \textbf{keine} sô \textbf{werde}.\\ 
\end{tabular}
\scriptsize
\line(1,0){75} \newline
D \newline
\line(1,0){75} \newline
\textbf{1} \textit{Initiale} D  \textbf{3} \textit{Majuskel} D  \textbf{9} \textit{Majuskel} D  \textbf{16} \textit{Majuskel} D  \textbf{21} \textit{Majuskel} D  \newline
\line(1,0){75} \newline
\textbf{27} Parzivals] Parcifals D \newline
\end{minipage}
\hspace{0.5cm}
\begin{minipage}[t]{0.5\linewidth}
\small
\begin{center}*m
\end{center}
\begin{tabular}{rl}
 & der knappe, den si \textbf{d\textit{â} knien} sach.\\ 
 & diu künig\textit{în} zuom brief sprach:\\ 
 & "\textbf{ôwol} der hant, diu dich schreip!\\ 
 & âne sorge ich nie bleip\\ 
5 & sît des tages, daz ich \textbf{in} sach,\\ 
 & \textbf{von des} hant \textbf{disiu} schrift geschach."\\ 
 & si weinte und was doch vrô.\\ 
 & hin zem knappen sprach si dô:\\ 
 & "dû bist Gawanes kneht."\\ 
10 & "j\textit{â}, vrowe, \textbf{der} enbiutet iu sîn reht,\\ 
 & dienstlîch triuwe âne allen wanc\\ 
 & und dâ bî sîn vröude kranc,\\ 
 & ir wellet im vröude machen hôch.\\ 
 & sô kumberlîch ez sich gez\textit{ô}ch\\ 
15 & nie umb alle sîn êre.\\ 
 & vrouwe, er enbiutet iu mêre,\\ 
 & daz er mit \textit{wer}den \textit{vröu}den lebe,\\ 
 & \textbf{und} vre\textit{is}ch\textit{et} \textbf{er} \textbf{iuwers} trôstes gebe.\\ 
 & ir moget wol an dem brieve sehen\\ 
20 & mê, dan ich \textbf{iu} \textbf{kan} \textbf{verjehen}."\\ 
 & si sprach: "ich hân vür wâr erkant,\\ 
 & durch waz \textit{dû} \textbf{zuo mir bist} gesant.\\ 
 & ich tuon \textit{im} \textbf{werden} diens\textit{t} dar\\ 
 & mit \textbf{minneclîcher} vro\textit{w}e\textit{n} schar,\\ 
25 & die vür wâr bî mîner zît\\ 
 & \textbf{vor ûz an prîse hânt} de\textit{n s}trît.\\ 
 & âne Parcifals wîp\\ 
 & und âne Urgelusen lîp\\ 
 & sô \textbf{erkenne} ich ûf der erde\\ 
30 & bî touf \textbf{k\textit{ein}} sô \textbf{werde}.\\ 
\end{tabular}
\scriptsize
\line(1,0){75} \newline
m n o \newline
\line(1,0){75} \newline
\newline
\line(1,0){75} \newline
\textbf{1} der knappe] Den knappen n  $\cdot$ dâ] do m n o \textbf{2} künigîn] kunnig m \textbf{5} des] das o \textbf{6} schrift] geschrifft n \textbf{7} vrô] so fro n \textbf{10} jâ] Jr m \textbf{12} vröude] frowe o \textbf{14} gezôch] gezeh m \textbf{15} umb alle] al vmb n \textbf{17} werden vröuden] froͯden werden m \textbf{18} vreischet] frefeclich m \textbf{22} dû] \textit{om.} m \textbf{23} im] uͯch m  $\cdot$ dienst] dienste m dar dienste o \textbf{24} minneclîcher] wuͯneclichen o  $\cdot$ vrowen] froͯder m \textbf{26} den strît] den pris vnd strit m \textbf{27} Parcifals] parcifales n \textbf{30} kein] kam m o \newline
\end{minipage}
\end{table}
\newpage
\begin{table}[ht]
\begin{minipage}[t]{0.5\linewidth}
\small
\begin{center}*G
\end{center}
\begin{tabular}{rl}
 & \begin{large}D\end{large}er knappe, den si \textbf{dâ knien} sach.\\ 
 & di\textit{u} künegîn ze dem brieve sprach:\\ 
 & "\textbf{wol} der hant, diu dich schreip!\\ 
 & âne sorge ich nie beleip\\ 
5 & sît des tages, daz ich sach\\ 
 & \textbf{die} hant, \textbf{von der diu} schrift geschach."\\ 
 & si weinde \textbf{sêre} unde was doch vrô.\\ 
 & hin ze dem knappen sprach si dô:\\ 
 & "dû bist Gawans kneht."\\ 
10 & "jâ, vrouwe, \textbf{er} enbiut iu sîn reht,\\ 
 & dienstlîch triuwe ân allen wanc\\ 
 & unde dâ bî sîne vröude kranc,\\ 
 & ir\textbf{ne} welt im vröude machen hôch.\\ 
 & sô kumberlîch ez sich gezôch\\ 
15 & nie \textit{umbe al} sîn êre.\\ 
 & vrouwe, er enbiut iu mêre,\\ 
 & daz er mit werden vröuden lebe,\\ 
 & vreischet \textbf{er} \textbf{iuwers} trôstes gebe.\\ 
 & ir muget wol an dem brieve sehen\\ 
20 & mêre, danne ich \textbf{kunne} \textbf{jehen}."\\ 
 & si sprach: "ich hân vür wâr erkant,\\ 
 & durch waz dû \textbf{zuo mir bist} gesant.\\ 
 & ich tuon im \textbf{werden} dienst dar\\ 
 & mit \textbf{werdeclîcher} vrouwen schar,\\ 
25 & die vür wâr bî mîner zît\\ 
 & \textbf{hânt vor ûz} den \textbf{besten} strît.\\ 
 & âne Parcivales wîp\\ 
 & unde ân Orgelusen lîp\\ 
 & sô\textbf{ne} \textbf{erkenne} ich ûf der erde\\ 
30 & bî toufe \textbf{deheine} sô \textbf{werde}.\\ 
\end{tabular}
\scriptsize
\line(1,0){75} \newline
G I L M Z Fr18 \newline
\line(1,0){75} \newline
\textbf{1} \textit{Initiale} G L Z Fr18  \textbf{13} \textit{Initiale} I  \newline
\line(1,0){75} \newline
\textbf{1} Der knappe den] den chnappen den I Den L  $\cdot$ dâ knien] da kniende I da vor ir knien L knien Fr18 \textbf{2} diu] Die G Z \textbf{3} wol] O wol Z \textbf{4} sorge] sorgen L \textbf{7} weinde] wende M \textbf{8} dô] so M \textbf{9} Gawans] Gawansz L gawanes Z \textbf{10} er] der Z \textbf{11} Dinst vnd truͯwe ane wanch L \textbf{12} bî sîne vröude] zuͤ siner freuden I \textbf{13} im vröude machen] [ir]: in machen freude I \textbf{15} umbe al] al vmbe G \textbf{16} enbiut] bevt Z \textbf{17} lebe] leben M \textbf{18} vreischet] Gefraishet I \textbf{19} brieve] brývelin L \textbf{20} ich] ichz M ichs Z  $\cdot$ jehen] veryehen L \textbf{21} wâr] \textit{om.} L M \textbf{22} zuo mir bist] bist zuͯ mir L (M) \textbf{24} werdeclîcher] wertlicher I wvͯnneclicher L (M) (Z) \textbf{26} An pris vor vsz hant den strit L (M) (Z) \textbf{27} Parcivales] parzivales G parzifales I parzifals L M parcifals Z \textbf{28} ân Orgelusen] Orgelisen L \textbf{29} sône] So M Z  $\cdot$ erde] erden M \newline
\end{minipage}
\hspace{0.5cm}
\begin{minipage}[t]{0.5\linewidth}
\small
\begin{center}*T
\end{center}
\begin{tabular}{rl}
 & de\textit{r} knabe, den si \textbf{d\textit{â} knien} sach.\\ 
 & diu künigîn zuom brief sprach:\\ 
 & "\textbf{wol} der hant, diu dich schreip!\\ 
 & âne sorge ich nie bleip\\ 
5 & sît des tages, daz ich sach\\ 
 & \textbf{die} hant, \textbf{von der diu} schrift geschach."\\ 
 & si weinte \textbf{sêre} und was doch vrô.\\ 
 & hin zuom knaben sprach si dô:\\ 
 & "dû bist Gawans kneht."\\ 
10 & "jâ, vrou, \textbf{er} enbiut iu sîn reht,\\ 
 & dienstlîche triuwe ân allen wanc\\ 
 & und dâ bî sîne vreude kranc,\\ 
 & ir\textbf{ne} wolt im vreuden machen hôch.\\ 
 & sô kumberlîch ez sich gezôch\\ 
15 & nie umb alle sîn êre.\\ 
 & vrou, er enbiut iu mêre,\\ 
 & daz er mit werden vreuden lebe,\\ 
 & vreischet \textbf{\textit{e}r} \textbf{iuwers} trôstes gebe.\\ 
 & ir mogt wol an dem brieve sehen\\ 
20 & mêr, dann ich\textbf{s} \textbf{künne} \textbf{jehen}."\\ 
 & si sprach: "ich hân vür wâr erkant,\\ 
 & durch waz dû \textbf{bist zuo mir} gesant.\\ 
 & ich tuon im \textbf{werden} dienst dar\\ 
 & mit \textbf{wünneclîcher} vrouwen schar,\\ 
25 & die vür wâr bî mîner zît\\ 
 & \textbf{an prîse vo\textit{r} \textit{û}z hân\textit{t}} den strît.\\ 
 & âne Parcifales wîp\\ 
 & und ân Orgelusen lîp\\ 
 & sô \textbf{en}\textbf{kenne} ich ûf der erden\\ 
30 & bî toufe \textbf{deheinen} sô \textbf{werden}.\\ 
\end{tabular}
\scriptsize
\line(1,0){75} \newline
Q R W V \newline
\line(1,0){75} \newline
\textbf{21} \textit{Initiale} W V  \newline
\line(1,0){75} \newline
\textbf{1} der] Den Q [*]: Zem V  $\cdot$ dâ] do Q W V \textbf{3} wol] Wol sei W \textbf{4} sorge] sorgen W \textbf{5} sach] [*]: in sach V \textbf{6} [*]: Von dez hant dise schrift geschach V  $\cdot$ schrift] geschriff R \textbf{7} weinte] weinet W \textbf{8} knaben] knechte W  $\cdot$ dô] also W \textbf{9} Gawans] Gawines R gawanes V \textbf{10} enbiut] bút R \textbf{11} allen] \textit{om.} W \textbf{13} irne] Jr R (W) [Jm]: Jrn  V  $\cdot$ im] in dann W  $\cdot$ vreuden] froͯde R (V) \textbf{14} gezôch] zoch R \textbf{16} enbiut] bút R [*]: enbútet V \textbf{17} vreuden] froude R \textbf{18} vreischet] Pruͤfet W [*]: Vnd freischet V  $\cdot$ er] ir Q \textbf{20} ichs] ich eúch W ich [*]: v́ch V  $\cdot$ künne jehen] [*]: kv́nne iehen V \textbf{23} werden dienst] werde dienste R \textbf{25} bî] an W [*]: bi V \textbf{26} vor ûz] von vns Q vor [*]: vz V  $\cdot$ hânt] hon Q \textbf{27} Parcifales] pfartzifales Q parczifals R partzifals W [parzifal*]: parzifalez V \textbf{28} Orgelusen] vrgulusen R \textbf{29} enkenne] bekenne R erkenne W V  $\cdot$ erden] erde W V \textbf{30} deheinen] deheinne R (W) (V)  $\cdot$ werden] werde W V \newline
\end{minipage}
\end{table}
\end{document}
