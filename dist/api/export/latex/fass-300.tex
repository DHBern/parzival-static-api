\documentclass[8pt,a4paper,notitlepage]{article}
\usepackage{fullpage}
\usepackage{ulem}
\usepackage{xltxtra}
\usepackage{datetime}
\renewcommand{\dateseparator}{.}
\dmyyyydate
\usepackage{fancyhdr}
\usepackage{ifthen}
\pagestyle{fancy}
\fancyhf{}
\renewcommand{\headrulewidth}{0pt}
\fancyfoot[L]{\ifthenelse{\value{page}=1}{\today, \currenttime{} Uhr}{}}
\begin{document}
\begin{table}[ht]
\begin{minipage}[t]{0.5\linewidth}
\small
\begin{center}*D
\end{center}
\begin{tabular}{rl}
\textbf{300} & \begin{large}E\end{large}r kêrt ûz, dâ er den Waleis vant,\\ 
 & des witze was der minnen pfant.\\ 
 & \textbf{er} truoc \textbf{drî tjoste} durch den schilt,\\ 
 & mit heldes handen dar gezilt.\\ 
5 & \textbf{ouch} het in Orilus versniten.\\ 
 & \textbf{Sus kom Gawan} zuo \textbf{z}im geriten,\\ 
 & sunder kalopieren\\ 
 & unt âne punieren.\\ 
 & er wolde güetlîche ersehen,\\ 
10 & \textbf{von wem der strît} \textbf{dâ} wære geschehen.\\ 
 & Dô sprach er grüezenlîche dar\\ 
 & \textbf{ze Parzivale, der}\textbf{s} kleine war\\ 
 & nam. daz muose êt alsô sîn:\\ 
 & dâ tet vrou Minne ir ellen schîn\\ 
15 & an \textbf{dem}, den Herzeloyde \textbf{bar}.\\ 
 & ungezaltiu sippe in \textbf{gar}\\ 
 & schiet von den witzen sîn\\ 
 & unt ûf geerbeter pîn\\ 
 & von vater unt von muoter art.\\ 
20 & der Waleis wênec innen wart,\\ 
 & waz \textbf{mînes} hêrn Gawans munt\\ 
 & \textbf{mit worten im dâ tæte} kunt.\\ 
 & Dô sprach des \textbf{künec} Lotes sun:\\ 
 & "hêrre, ir welt gewalt nû tuon,\\ 
25 & sît ir mir grüezen \textbf{widersagt}.\\ 
 & i\textbf{ne} bin \textbf{doch} niht \textbf{sô} \textbf{gar} verzagt,\\ 
 & ine bringe\textbf{z} \textbf{an} \textbf{ander vrâge}.\\ 
 & ir habt man und mâge\\ 
 & unt den künec selben entêret,\\ 
30 & unser laster \textbf{hie} gemêret.\\ 
\end{tabular}
\scriptsize
\line(1,0){75} \newline
D \newline
\line(1,0){75} \newline
\textbf{1} \textit{Initiale} D  \textbf{6} \textit{Majuskel} D  \textbf{11} \textit{Majuskel} D  \textbf{23} \textit{Majuskel} D  \newline
\line(1,0){75} \newline
\textbf{23} Lotes] loths D \newline
\end{minipage}
\hspace{0.5cm}
\begin{minipage}[t]{0.5\linewidth}
\small
\begin{center}*m
\end{center}
\begin{tabular}{rl}
 & er kêrte ûz, d\textit{â} er den Waleis vant,\\ 
 & des witze was der minnen pfant.\\ 
 & \textbf{er} truoc \textbf{drîe juste} durch den schilt,\\ 
 & mit heldes handen dar gezilt.\\ 
5 & \textbf{ouch} hete in Orilus versniten.\\ 
 & \textbf{sus kam Gawan} zuo im geriten,\\ 
 & sunder kalop\textit{i}eren\\ 
 & und âne punieren.\\ 
 & er wolte güetlîchen ersehen,\\ 
10 & \textbf{von wem der strît} wære geschehen.\\ 
 & dô sprach er grüezenlîchen da\textit{r}\\ 
 & \textbf{ze Parcifalen, der} \textbf{es} kleine wa\textit{r}\\ 
 & nam. daz muos eht alsô sîn:\\ 
 & dô tet vrouwe Minne \textit{ir ellen} schîn\\ 
15 & an \textbf{dem}, den Herczeloide \textbf{bar}.\\ 
 & ungezaltiu sippe in \textbf{dâr}\\ 
 & schiet von den witzen sîn\\ 
 & und ûf geerbeter pîn\\ 
 & von vater und von muoter art.\\ 
20 & der Walei\textit{s} wênic innen wart,\\ 
 & waz hêrren Gawans munt\\ 
 & \textbf{mit worten ime d\textit{â} tæt\textit{e}} kunt.\\ 
 & dô sprach des \textbf{künic} Lotes sun:\\ 
 & "hêrre, ir welt gewalt nû tuon,\\ 
25 & sît ir mir grüezen \textbf{widersaget}.\\ 
 & ich \textbf{en}bin \textbf{noch} niht verzaget,\\ 
 & ich enbringe \textbf{es} \textbf{andere vrâgen}.\\ 
 & ir habt man unde mâgen\\ 
 & und den künic selbe entêret,\\ 
30 & unser laster \textbf{hie} gemêret.\\ 
\end{tabular}
\scriptsize
\line(1,0){75} \newline
m n o Fr69 \newline
\line(1,0){75} \newline
\newline
\line(1,0){75} \newline
\textbf{1} dâ] do m n o \textbf{4} gezilt] gezelt o \textbf{5} Orilus] oriluͯs o \textbf{6} Gawan] gewan o \textbf{7} kalopieren] caluperen m kalaperen o \textbf{8} punieren] pẏmeren o \textbf{10} wære] do were n da were Fr69  $\cdot$ geschehen] beschehen n (o) \textbf{11} er] >er< Fr69  $\cdot$ grüezenlîchen] gruͯselichen m (o) gr:::slichen Fr69  $\cdot$ dar] dan m \textbf{12} Parcifalen] parcifal n o ::: Fr69  $\cdot$ war] wan m \textbf{13} eht] es o \textbf{14} \textit{Versdoppelung 300.14-17 (²o) nach 300.17; Lesarten der vorausgehenden Verse mit ¹o bezeichnet} o   $\cdot$ vrouwe] farwe \textsuperscript{2}\hspace{-1.3mm} o  $\cdot$ ir ellen] \textit{om.} m jr aller \textsuperscript{2}\hspace{-1.3mm} o \textbf{15} dem] \textit{om.} \textsuperscript{2}\hspace{-1.3mm} o  $\cdot$ den] der do n der \textsuperscript{1}\hspace{-1.3mm} o den die \textsuperscript{2}\hspace{-1.3mm} o  $\cdot$ Herczeloide] hertzeleide n herczeleide \textsuperscript{1}\hspace{-1.3mm} o \textsuperscript{2}\hspace{-1.3mm} o \textbf{16} dâr] gar n \textsuperscript{1}\hspace{-1.3mm} o \textsuperscript{2}\hspace{-1.3mm} o \textbf{17} witzen] [st]: wiczen o \textbf{20} Waleis] waleise m waleisz n \textbf{21} hêrren] her n horn o  $\cdot$ Gawans] gewans o \textbf{22} dâ] do m n \textit{om.} o  $\cdot$ tæte] tetten m \textbf{23} künic] kv́niges n (o)  $\cdot$ Lotes] loths m n >lothez< o \textbf{24} welt] wellent n \textbf{25} mir grüezen] [min]: mir grussent o \textbf{26} enbin noch] bin doch n o  $\cdot$ verzaget] so verzaget n (o) \textbf{27} enbringe] bringe n o  $\cdot$ andere vrâgen] ander froge n ander fro o \textbf{28} mâgen] moge n o \textbf{29} selbe] selber o  $\cdot$ entêret] eneret n \textbf{30} unser] Vnd vnser n Vnde vns o \newline
\end{minipage}
\end{table}
\newpage
\begin{table}[ht]
\begin{minipage}[t]{0.5\linewidth}
\small
\begin{center}*G
\end{center}
\begin{tabular}{rl}
 & er kêrte ûz, dâ er den Waleis vant,\\ 
 & des witze was der minnen pfant.\\ 
 & \textbf{\textit{d}er} truoc \textbf{driu venster} durch den schilt,\\ 
 & mit heldes handen dar gezilt;\\ 
5 & \textbf{sus} het in Orillus versniten.\\ 
 & \textbf{Gawan kom} zuo im geriten,\\ 
 & sunder galopieren\\ 
 & unde âne punieren.\\ 
 & er wolte güetlîche ersehen,\\ 
10 & \textbf{von wem der strît} wære geschehen.\\ 
 & \begin{large}D\end{large}ô sprach er grüezlîche dar.\\ 
 & \textbf{Parzival} \textbf{des} kleine war\\ 
 & nam. daz muosêt alsô sîn:\\ 
 & dô tet vrou Minne ir ellen schîn\\ 
15 & an \textbf{im}, den Herzeloide \textbf{bar}.\\ 
 & ungezaltiu sippe in \textbf{gar}\\ 
 & schiet von den witzen sîn\\ 
 & unde ûf geerbeter pîn\\ 
 & von vater unde von muoter art.\\ 
20 & der Waleis wênic innen wart,\\ 
 & waz \textbf{des} hêrn Gawans munt\\ 
 & \textbf{mit worten im dâ tæte} kunt.\\ 
 & dô sprach des \textbf{künic} Lotes sun:\\ 
 & "hêrre, ir welt gewalt nû tuon,\\ 
25 & sît ir mir grüezen \textbf{widersaget}.\\ 
 & ich bin \textbf{doch} niht \textbf{sô} \textbf{gar} verzaget,\\ 
 & ich enbringe\textbf{z} \textbf{an} \textbf{ander vrâge}.\\ 
 & ir habet man unde mâge\\ 
 & unt den künic selbe entêret,\\ 
30 & unser laster gemêret.\\ 
\end{tabular}
\scriptsize
\line(1,0){75} \newline
G I O L M Q R Z \newline
\line(1,0){75} \newline
\textbf{1} \textit{Initiale} I  \textbf{3} \textit{Capitulumzeichen} L  \textbf{11} \textit{Initiale} G O M R  \textbf{21} \textit{Initiale} I  \textbf{23} \textit{Initiale} L  \textbf{27} \textit{Überschrift:} Hie hept parczifal by den schne Vnd wie zu Jm kam Her gewan R   $\cdot$ \textit{Initiale} R  \newline
\line(1,0){75} \newline
\textbf{1} ûz] \textit{om.} I  $\cdot$ dâ] do Q  $\cdot$ Waleis] waleiýsz L \textbf{2} des] sin I  $\cdot$ minnen] minne I O (L) (M) Q R \textbf{3} der] er G  $\cdot$ driu] diu I (L) drin Q  $\cdot$ venster] tiost Z \textbf{4} heldes] hendes O Q  $\cdot$ handen] hant O L M \textbf{5} sus] Ouch Z  $\cdot$ in] Jm R  $\cdot$ Orillus] Orilus I (O) (M) (Q) R (Z) \textbf{6} Gawan] Sewan Q  $\cdot$ zuo] vsz Q  $\cdot$ im] \textit{om.} L \textbf{7} Sund galopierten R \textbf{8} punieren] pvmeren L \textbf{9} \textit{Versfolge 300.10-9} L   $\cdot$ Daz wolte er gesuͯchtliche spehen L  $\cdot$ wolte] wolt in Q  $\cdot$ ersehen] sehen Q \textbf{10} wære] da were L (M) (R) (Z) do wer Q \textbf{11} Dô] ÷o O Da Z  $\cdot$ grüezlîche] gutlichen Q (R) \textbf{12} Parzival] [parzifal]: Parzifal I M Barcifal O Parcifal L Partzifal Q parczifal R Zv parcifal Z  $\cdot$ kleine] clneine I kleiner Z \textbf{13} muosêt] muͤst I muste M (Z)  $\cdot$ alsô] alsus L (M) (Z) \textbf{14} dô] da I (L) (M) (Z)  $\cdot$ ir] \textit{om.} I  $\cdot$ ellen] eren Q \textbf{15} im] in I  $\cdot$ den] der O L [der]: den M  $\cdot$ Herzeloide] herzelaude I herzen lavde O Hertzelaude L herczeloide M herzeloude Q herczelaude R hertzenlovde Z  $\cdot$ bar] gebar L \textbf{16} in] \textit{om.} O Z \textbf{18} ûf] vf in ein L  $\cdot$ geerbeter] geborner I erbeter Q geerbtte R \textbf{20} Waleis] waleýs L wales Q  $\cdot$ wênic] do I \textbf{21} waz] Daz L  $\cdot$ des] \textit{om.} I mines O (L) (M) (Q) (R) (Z)  $\cdot$ Gawans] Gawanes O Gawanez L gawans M \textbf{22} im] in Q (R)  $\cdot$ dâ tæte] tet da I do tet Q R \textbf{23} dô] Da M Z  $\cdot$ künic] chunges I (O) (L) (Q) (R)  $\cdot$ Lotes] lotis M \textbf{24} gewalt nû] Nu giwalt M gewalt hie R \textbf{25} grüezen] gewalt O gruͦses R \textbf{26} Daz han ich nýman geclagt L  $\cdot$ doch niht sô gar] doch niht so I so gar doch niht O (Q) dach so gar nicht M so gar doch nie R \textbf{27} enbringez] bringz O (M) (R)  $\cdot$ an] an >an< I \textbf{28} habet] habt hie I \textbf{29} selbe] selben M selber R \textbf{30} unser] Vnd vnser R  $\cdot$ gemêret] hie gemeret Z \newline
\end{minipage}
\hspace{0.5cm}
\begin{minipage}[t]{0.5\linewidth}
\small
\begin{center}*T
\end{center}
\begin{tabular}{rl}
 & er kêrte ûz, dâ er den Waleis vant,\\ 
 & des witze was der minne pfant.\\ 
 & \textbf{der} truoc \textbf{driu venster} durch den schilt,\\ 
 & mit heldes handen dar gezilt;\\ 
5 & \textbf{sus} hetin Orilus versniten.\\ 
 & \textbf{Gawan kom} zuo \textbf{z}im geriten,\\ 
 & sunder galopieren\\ 
 & unde âne punieren.\\ 
 & er wolte güetlîche ersehen,\\ 
10 & \textbf{wâ im der schade} wære geschehen.\\ 
 & dô sprach er grüezlîchen dar.\\ 
 & \textbf{Parcifal} \textbf{des} kleine war\\ 
 & nam. daz muose eht alsô sîn:\\ 
 & dâ tet vrou Minne ir ellen schîn\\ 
15 & an \textbf{dem}, den Herzeloyde \textbf{gebar}.\\ 
 & ungezalt\textit{iu} sippe in \textbf{gar}\\ 
 & schiet von den witzen sîn\\ 
 & unde ûf geerbeter pîn\\ 
 & von vater unde von muoter art.\\ 
20 & Der Waleis wênic innen wart,\\ 
 & waz \textbf{mînes} hêrn Gawans munt\\ 
 & \textbf{im dâ tæt mit worten} kunt.\\ 
 & \begin{large}D\end{large}ô sprach des \textbf{küneges} Lotes sun:\\ 
 & "hêrre, ir welt gewalt nû tuon,\\ 
25 & sît ir mir grüezen \textbf{versaget}.\\ 
 & ich bin \textbf{iedoch} \textbf{des} niht verzaget,\\ 
 & ine bringe\textbf{z} \textbf{an} \textbf{ein} \textbf{ander vrâge}.\\ 
 & ir habt man unde mâge\\ 
 & und den künec selbe \textit{entêret}\\ 
30 & \textbf{unde} unser laster gemêret.\\ 
\end{tabular}
\scriptsize
\line(1,0){75} \newline
T U V W \newline
\line(1,0){75} \newline
\textbf{20} \textit{Majuskel} T  \textbf{23} \textit{Initiale} T U V  \newline
\line(1,0){75} \newline
\textbf{1} kêrte] kert V  $\cdot$ dâ] do V W  $\cdot$ Waleis] walleis V waleisen W \textbf{3} driu] die U  $\cdot$ venster] [*]: iuste V \textbf{6} zim] im U W \textbf{9} ersehen] [*]: ersehen V \textbf{10} wâ im der schade wære] [*]: Von wem der strit do were V Von wem der streit wer W \textbf{11} er] \textit{om.} U W \textbf{12} Parcifal] Parzifal T [*]: Ze parzefale V Gawan W  $\cdot$ des] [der]: des T [*]: ders V des vil W \textbf{13} muose eht] mvese eht T mvͤste V \textbf{14} dâ] Do U V W  $\cdot$ Minne] mine U venus W  $\cdot$ ellen] essel U \textbf{15} dem den] im der W  $\cdot$ Herzeloyde] herzeleide U [herzelauden]: herzelaude V hertzeloyde W  $\cdot$ gebar] barn W \textbf{16} Wann in die minne also warn W  $\cdot$ ungezaltiu] vngezalte T \textbf{18} ûf geerbeter] vf [ge*]: geerbter V \textbf{20} Waleis] walleis V \textbf{21} waz] [*]: Was V \textbf{22} Mit worten im daten (im do tet V [ W ]) kuͦnt U (V) (W)  $\cdot$ tæt] tet T \textbf{23} küneges] [kvnes]: kvneges T  $\cdot$ Lotes] lottes W \textbf{25} grüezen versaget] gruͦß widersaget W \textbf{26} des] so U V W  $\cdot$ verzaget] [ver*]: verzaget U [v*aget]: verzaget V vertaget W \textbf{27} ine bringez] Jch bringen iz U Ich bring es W  $\cdot$ ein] \textit{om.} W \textbf{29} selbe entêret] selbe :: T selben v́nteret V selber enteret W \newline
\end{minipage}
\end{table}
\end{document}
