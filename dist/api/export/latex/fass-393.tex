\documentclass[8pt,a4paper,notitlepage]{article}
\usepackage{fullpage}
\usepackage{ulem}
\usepackage{xltxtra}
\usepackage{datetime}
\renewcommand{\dateseparator}{.}
\dmyyyydate
\usepackage{fancyhdr}
\usepackage{ifthen}
\pagestyle{fancy}
\fancyhf{}
\renewcommand{\headrulewidth}{0pt}
\fancyfoot[L]{\ifthenelse{\value{page}=1}{\today, \currenttime{} Uhr}{}}
\begin{document}
\begin{table}[ht]
\begin{minipage}[t]{0.5\linewidth}
\small
\begin{center}*D
\end{center}
\begin{tabular}{rl}
\textbf{393} & \begin{large}D\end{large}az \textbf{got} ir strîtes \textbf{gegenniet}\\ 
 & des tages von ein ander schiet.\\ 
 & des was ir helendiu zuht ein pfant,\\ 
 & daz ir neweder wart \textbf{genant}.\\ 
5 & si\textbf{ne} \textbf{erkande} ouch niemen dâ.\\ 
 & daz tet man aber anderswâ.\\ 
 & Zuo Meljanze sprach Scherules:\\ 
 & "hêrre, muoz ich iuch bitten des,\\ 
 & sô \textbf{ruochet} mînen hêrren sehen.\\ 
10 & swes vriwent dâ bêdenthalben jehen,\\ 
 & des sult ir gerne volgen\\ 
 & unt sît im niht erbolgen."\\ 
 & daz dûhte si guot über al.\\ 
 & dô vuoren si ûf des \textbf{küneges} sal,\\ 
15 & daz innere her von der stat.\\ 
 & \textbf{des vürsten} marschalc si des bat.\\ 
 & Dô nam mîn hêr Gawan\\ 
 & den grâven Laheduman\\ 
 & unt ander sîne gevangen.\\ 
20 & \textbf{die kômen} dar zuo gegangen.\\ 
 & \textbf{er} bat si geben sicherheit,\\ 
 & die er des tages ab in erstreit,\\ 
 & Scherulese, sîme wirt.\\ 
 & mannegelîch nû niht verbirt,\\ 
25 & \textbf{sine v\textit{üe}ren}, als dâ \textbf{gelobt} was,\\ 
 & ze Bearosche ûfen palas.\\ 
 & Melyanze gap diu burcgrævîn\\ 
 & rîchiu kleider unt ein rîselîn,\\ 
 & dâ er sînen wunden arm în hienc,\\ 
30 & dâ Gawans tjoste durch \textbf{gienc}.\\ 
\end{tabular}
\scriptsize
\line(1,0){75} \newline
D \newline
\line(1,0){75} \newline
\textbf{1} \textit{Initiale} D  \textbf{7} \textit{Majuskel} D  \textbf{17} \textit{Majuskel} D  \newline
\line(1,0){75} \newline
\textbf{7} Meljanze] Melianze D  $\cdot$ Scherules] Scervles D \textbf{23} Scherulese] Scervlese D \textbf{25} vüeren] fvͦren D \textbf{26} Bearosche] Bearoscê D \newline
\end{minipage}
\hspace{0.5cm}
\begin{minipage}[t]{0.5\linewidth}
\small
\begin{center}*m
\end{center}
\begin{tabular}{rl}
 & daz \textbf{got} ir strîtes \textbf{gegenniet}\\ 
 & des tage\textit{s} von ein ander schiet.\\ 
 & des wa\textit{s i}r helendiu zuht ein pfant,\\ 
 & daz ir enweder wart \textbf{genant}.\\ 
5 & si \textbf{en}\textbf{kante} ouch niemen \textbf{anders} d\textit{â}.\\ 
 & daz \textit{tet} man aber anderswâ.\\ 
 & ze M\textit{e}l\textit{i}anze sprach Scherules:\\ 
 & "hêrre, muoz ich iuch bitten des,\\ 
 & sô \textbf{ruochet} mînen hêrren sehen.\\ 
10 & wes vriunt d\textit{â} beidenthalben jehen,\\ 
 & des sult ir gerne volgen\\ 
 & und sît ime niht erbolgen."\\ 
 & daz dûhte si guot über al.\\ 
 & dô vuorens ûf des \textbf{küniges} sal,\\ 
15 & daz inner her von der stat.\\ 
 & \textbf{des vürsten} marschalc si des bat.\\ 
 & dô nam mîn hêr Gawan\\ 
 & den grâven Lahed\textit{u}man\\ 
 & und ander sîne gevangen.\\ 
20 & \textbf{die kômen} dar zuo gegangen.\\ 
 & \textbf{er} bat si geben sicherheit,\\ 
 & die \textit{er} des tages ab in erstreit,\\ 
 & Scherules, sînem wirt.\\ 
 & mannegelîch nû niht verbirt,\\ 
25 & \textbf{si envüeren}, als dâ \textbf{gelobet} was,\\ 
 & ze Bearosche ûf den palas.\\ 
 & Melianz gap diu burcgrævîn\\ 
 & rîchiu kleider und ein rîselîn,\\ 
 & dâ er sînen wunden arm în hienc,\\ 
30 & dâ Gawan\textit{e}s juste durch \textbf{ergienc}.\\ 
\end{tabular}
\scriptsize
\line(1,0){75} \newline
m n o \newline
\line(1,0){75} \newline
\newline
\line(1,0){75} \newline
\textbf{1} gegenniet] gegen [mit]: miet n gegen miet o \textbf{2} des] Das o  $\cdot$ tages] tage m \textbf{3} was ir] was ein ir m \textbf{4} enweder] ẏetweder n (o) \textbf{5} enkante] enkuͯnde o  $\cdot$ dâ] do m n \textbf{6} daz] Do n  $\cdot$ tet] \textit{om.} m  $\cdot$ anderswâ] anders swo o \textbf{7} Melianze] merelancze m meliantz n meliancz o  $\cdot$ Scherules] scerules m n sterules o \textbf{10} dâ] do m n o \textbf{11} des] Das n o \textbf{15} daz] Des o \textbf{18} Laheduman] [h*]: lahediman m leheduman n lehelmduman o \textbf{20} kômen] koment n \textbf{21} geben] gegen o \textbf{22} er] \textit{om.} m  $\cdot$ erstreit] streit n o \textbf{23} Scherules] Scerules m Sterulese n Strenlese o \textbf{24} mannegelîch nû] Menlich ẏm o \textbf{25} envüeren] fẏren n (o)  $\cdot$ dâ] do n o \textbf{26} Bearosche] bearosce m bearosc n bearost o  $\cdot$ den] dem o \textbf{27} Melianz] Meliancz m o Meliantz n  $\cdot$ burcgrævîn] burgrafen o \textbf{30} Gawanes] gawanens m \newline
\end{minipage}
\end{table}
\newpage
\begin{table}[ht]
\begin{minipage}[t]{0.5\linewidth}
\small
\begin{center}*G
\end{center}
\begin{tabular}{rl}
 & daz \textbf{got} ir strîtes \textbf{gegenbiet}\\ 
 & des tages von ein ander schiet.\\ 
 & des was ir helendiu zuht ein pfant,\\ 
 & daz ir dewe\textit{de}r wart \textbf{bekant}.\\ 
5 & si\textbf{ne} \textbf{erkande} ouch niemen dâ.\\ 
 & daz tet man aber anderswâ.\\ 
 & ze Melianze sprach Tscherules:\\ 
 & "hêrre, \textit{muoz ich} iuch biten des,\\ 
 & \textit{sô} \textbf{ruochet} mînen \textit{hêrren} sehen.\\ 
10 & swes vriunt dâ bêden\textit{t}h\textit{alb}en jehen,\\ 
 & des sult ir gerne volgen\\ 
 & unde sît im niht erbolgen."\\ 
 & daz dûhte si guot über al.\\ 
 & dô vuo\textit{r}en s\textit{i} ûf des \textbf{vürsten} sal\\ 
15 & \textbf{\begin{large}U\end{large}nde} daz inner her von der stat.\\ 
 & \textbf{Libautes} marschalc si des bat.\\ 
 & dô nam mîn hêr Gawan\\ 
 & den grâven Lachdoman\\ 
 & unde ander sîne gevangen.\\ 
20 & \textbf{er kom} dar zuo gegangen\\ 
 & \textbf{unde} bat si geben sicherheit,\\ 
 & die er des tages abe in erstreit,\\ 
 & Tscherules, sînem wirt.\\ 
 & mannegelîch nû niht verbirt,\\ 
25 & \textbf{er enk\textit{œ}m}, als dâ \textbf{geboten} was,\\ 
 & ze Bearotsche ûf den palas.\\ 
 & Melianze gap diu burcgrævîn\\ 
 & rîchiu kleider unde ein rîselîn,\\ 
 & dâr sînen wunden arm în hienc,\\ 
30 & dâ Gawans tjost durch \textbf{gienc}.\\ 
\end{tabular}
\scriptsize
\line(1,0){75} \newline
G I O L M Q R Z Fr28 \newline
\line(1,0){75} \newline
\textbf{3} \textit{Initiale} I O L Z   $\cdot$ \textit{Capitulumzeichen} R  \textbf{7} \textit{Capitulumzeichen} R  \textbf{15} \textit{Initiale} G  \textbf{17} \textit{Initiale} I   $\cdot$ \textit{Capitulumzeichen} R  \newline
\line(1,0){75} \newline
\textbf{1} \textit{Die Verse 370.13-412.12 fehlen} Q   $\cdot$ got] \textit{om.} I  $\cdot$ gegenbiet] gein niet Z \textbf{2} ander] andren R \textbf{3} des] ÷es O Das R  $\cdot$ helendiu] ellen diu I (O) (M) heluͯde L  $\cdot$ ein] ir R \textbf{4} deweder] dewere G weder M newe der Fr28  $\cdot$ bekant] gemant I genant O L M R Z (Fr28) \textbf{5} sine] Si O L (R) (Z)  $\cdot$ erkande] erkantent R (Z) \textbf{7} Melianze] Melianzen I Melyanz O Meliantze L Meliancze R meliantz Z  $\cdot$ Tscherules] Schurles I Tschervles O Tshervles L scerules M scherules R \textbf{8} muoz ich] ih wil G ich mvͦz O \textbf{9} sô] \textit{om.} G  $\cdot$ ruochet] ruͤchet auch I geruchet M  $\cdot$ hêrren] \textit{om.} G \textbf{10} swes] Wez L (M) Was R  $\cdot$ bêdenthalben] bedenchen G \textbf{12} sît] en sit O M (Z) Fr28  $\cdot$ erbolgen] verbolgen R \textbf{14} dô] Da O M Z  $\cdot$ vuoren si] foͮrten sin G schwúrencz R \textbf{15} Unde daz] daz I Vntz L Vnd R  $\cdot$ her] \textit{om.} Z \textbf{16} Libautes] libauts G Lybavtes O Z [Lýbavͯ*]: Lýbavͯt L Lybantes R Lẏbautes Fr28 \textbf{17} dô] Da O M  $\cdot$ hêr] \textit{om.} M \textbf{18} Lachdoman] [h]: lohdoman I Lahdoman O R Latoman L lahedoman Z \textbf{21} bat] bat er I \textbf{22} abe in] ab im O yn abe M \textbf{23} Tscherules] Schureles I Tschervles O Tsheruͯles L Scerules M Scherules R tscherulese Fr28  $\cdot$ sînem] sinen L \textbf{24} mannegelîch] Menliche M (R) (Fr28)  $\cdot$ nû] noch O ouch L \textbf{25} er enkœm] eren chom G (I) (M) (Z) Er chom O (R) (Fr28)  $\cdot$ dâ] do O R  $\cdot$ geboten] gelobt O (L) (M) (R) Z Fr28 \textbf{26} ze Bearotsche] zebearotsche G zebeatrotsce I Ze Bearotsch O (L) (M) Ze Bearosze R zvͦ bearosche Fr28 \textbf{27} Melianze] Melianz I L Melyanzen O Meliancz R Meliantze Z  $\cdot$ diu burcgrævîn] den Burgruvin R die kunigin Z \textbf{28} rîchiu] Riche R  $\cdot$ kleider] cleit L \textbf{29} sînen wunden] sine wunden M sinen gwunden Fr28  $\cdot$ în] an Fr28 \textbf{30} diu von Gawans Tiost ergienc I  $\cdot$ Gawans] Gawanes O  $\cdot$ tjost] sper R \newline
\end{minipage}
\hspace{0.5cm}
\begin{minipage}[t]{0.5\linewidth}
\small
\begin{center}*T
\end{center}
\begin{tabular}{rl}
 & daz \textbf{er} ir strîtes \textbf{gegenbiet}\\ 
 & des tages von ein ander schiet.\\ 
 & des was ir helnde zuht ein pfant,\\ 
 & daz ir dewederre wart \textbf{genant}.\\ 
5 & si \textbf{erkande} ouch nieman dâ.\\ 
 & daz tet man aber anderswâ.\\ 
 & \begin{large}Z\end{large}e Melyanze sprach Tscherules:\\ 
 & "hêrre, muoz ich iuch bitten des,\\ 
 & sô \textbf{geruochet} mînen hêrren sehen.\\ 
10 & swes vriunt dâ beidenthalben jehen,\\ 
 & des sult ir gerne volgen\\ 
 & unde sît im niht erbolgen."\\ 
 & Daz dûhte si guot über al.\\ 
 & dô vuoren si ûf des \textbf{vürsten} sal\\ 
15 & \textbf{unde} daz inner her von der stat.\\ 
 & \textbf{Lybautes} marschalc si des bat.\\ 
 & Dô nam mîn hêr Gawan\\ 
 & den grâven Lachdoman\\ 
 & unde ander sîne gevangen.\\ 
20 & \textbf{der kom} dar zuo gegangen.\\ 
 & \textbf{er} bat si geben sicherheit,\\ 
 & die er des tages ab in erstreit,\\ 
 & Tscherulese, sînem wirt.\\ 
 & mannegelîch nû niht verbirt,\\ 
25 & \textbf{sine k\textit{œ}men}, als dâ \textbf{gelobet} was,\\ 
 & ze Bearosch ûf den palas.\\ 
 & Melyanze gap diu burcgrævîn\\ 
 & rîchiu kleider unde ein rîselîn,\\ 
 & dâ er sînen wunden arm în hienc,\\ 
30 & dâ Gawans tjost durch \textbf{gienc}.\\ 
\end{tabular}
\scriptsize
\line(1,0){75} \newline
T V W \newline
\line(1,0){75} \newline
\textbf{3} \textit{Initiale} W  \textbf{7} \textit{Initiale} T  \textbf{13} \textit{Majuskel} T  \textbf{17} \textit{Majuskel} T  \newline
\line(1,0){75} \newline
\textbf{1} er] got V W \textbf{5} nieman] nieman anders V  $\cdot$ dâ] do W \textbf{6} tet man] [tat*]: tatenz V \textbf{7} Melyanze] melianze V melianz W  $\cdot$ Tscherules] Tscerulos T scherules V tscherules W \textbf{8} muoz] wuͦß W  $\cdot$ iuch] iv T \textbf{9} geruochet] ruͦchent W \textbf{10} swes] Wes W  $\cdot$ dâ] do W \textbf{16} Lybautes] Dez fúrsten V Lyboutes W \textbf{18} Lachdoman] laheduman V lohdoman W \textbf{20} der kom dar] Die komen do V Er kam do W \textbf{21} er] Vnd W \textbf{22} erstreit] strait W \textbf{23} Tscherulese] Tscervlese T Scherulez V Tscherules W  $\cdot$ sînem] seinen W \textbf{24} mannegelîch] Menlich V \textbf{25} sine kœmen] sine comen T Sú enfuͤren V Er kam W  $\cdot$ dâ] do V W \textbf{26} Bearosch] Bearosc T bearotsche V beaorsche W  $\cdot$ den] [*]: dem V \textbf{27} Melyanze] Melianze V Melyanz W \textbf{28} rîchiu] \textit{om.} W \textbf{30} gienc] ergieng V \newline
\end{minipage}
\end{table}
\end{document}
