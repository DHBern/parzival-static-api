\documentclass[8pt,a4paper,notitlepage]{article}
\usepackage{fullpage}
\usepackage{ulem}
\usepackage{xltxtra}
\usepackage{datetime}
\renewcommand{\dateseparator}{.}
\dmyyyydate
\usepackage{fancyhdr}
\usepackage{ifthen}
\pagestyle{fancy}
\fancyhf{}
\renewcommand{\headrulewidth}{0pt}
\fancyfoot[L]{\ifthenelse{\value{page}=1}{\today, \currenttime{} Uhr}{}}
\begin{document}
\begin{table}[ht]
\begin{minipage}[t]{0.5\linewidth}
\small
\begin{center}*D
\end{center}
\begin{tabular}{rl}
\textbf{169} & im ist minne unt gruoz bereit,\\ 
 & m\textit{a}g er geniezen werdecheit."\\ 
 & ieslîcher im des \textbf{dâ} \textbf{verjach}\\ 
 & unt dar nâch, swer in ie gesach.\\ 
5 & \begin{large}D\end{large}er wirt in \textbf{mit} der hant gevienc,\\ 
 & geselleclîche er dannen gienc.\\ 
 & in vrâgete der vürste mære,\\ 
 & \textbf{welch} sîn ruowe wære\\ 
 & des nahtes dâ bî im gewesen.\\ 
10 & "\textbf{dâ}\textbf{ne} wære ich niht genesen,\\ 
 & wan daz \textbf{mîn muoter her mir} riet\\ 
 & des tages, dô ich von ir schiet."\\ 
 & "got müeze lônen iu und ir.\\ 
 & hêrre, ir tuot genâde an mir."\\ 
15 & \textbf{Dô} gie der helt \textbf{mit} witzen kranc,\\ 
 & dâ man got unt dem wirte sanc.\\ 
 & der wirt zer messe in lêrte,\\ 
 & \textbf{daz noch die} sælde mêrte,\\ 
 & opfern unt segnen sich\\ 
20 & unt gein dem tiuvele kêren gerich.\\ 
 & \textbf{dô} giengen si ûf den palas,\\ 
 & \textbf{al} dâ der tisch \textbf{gedecket} was.\\ 
 & der gast ze sîme wirte saz,\\ 
 & die spîse er ungesmæhet az.\\ 
25 & der wirt sprach durch höfscheit:\\ 
 & "hêrre, iu \textbf{en}sol niht wesen leit,\\ 
 & ob ich iuch \textbf{vrâge} mære,\\ 
 & wannen iwer reise wære."\\ 
 & Er seit im \textbf{gar} \textbf{diu} underscheit,\\ 
30 & \textbf{unt} wie er von sîner muoter reit,\\ 
\end{tabular}
\scriptsize
\line(1,0){75} \newline
D \newline
\line(1,0){75} \newline
\textbf{5} \textit{Initiale} D  \textbf{15} \textit{Majuskel} D  \textbf{29} \textit{Majuskel} D  \newline
\line(1,0){75} \newline
\textbf{2} mag er] manger D \newline
\end{minipage}
\hspace{0.5cm}
\begin{minipage}[t]{0.5\linewidth}
\small
\begin{center}*m
\end{center}
\begin{tabular}{rl}
 & im ist minne und gruoz bereit,\\ 
 & mac er geniezen werdicheit."\\ 
 & ieslîcher ime des \textbf{d\textit{â}} \textbf{verjach}\\ 
 & und dar nâch, wer in ie gesach.\\ 
5 & der wirt in \textbf{mit} der hant gevienc,\\ 
 & geselleclîchen er dannen gienc.\\ 
 & in vrâgete der vürste mære,\\ 
 & \textbf{wie lîht} sîn ruowe wære\\ 
 & des nahtes d\textit{â} bî ime gewesen.\\ 
10 & "\textbf{hêrre}, \textbf{dan}\textbf{ne} wære ich niht genesen,\\ 
 & wan daz \textbf{\textit{mir} here mîn muoter} r\textit{ie}t\\ 
 & des tages, dô ich von ir schiet."\\ 
 & "got müeze lônen iu und ir.\\ 
 & hêrre, ir tuot gnâde an mir."\\ 
15 & \textbf{dô} gienc der helt \textbf{mit} witzen kranc,\\ 
 & d\textit{â} man got und dem wirte sanc.\\ 
 & der wirt zer messe in lêrte,\\ 
 & \textbf{daz ie die} sælde \textit{m}êrte,\\ 
 & opfern und segnen sich\\ 
20 & und gegen dem tiuvel kêren gerich.\\ 
 & \textbf{dô} giengens ûf den palas,\\ 
 & d\textit{â} der tisch \textbf{berihtet} was.\\ 
 & der gast ze sînem wirte saz,\\ 
 & die spîse er ungesmâhet az.\\ 
25 & der wirt sprach durch hövescheit:\\ 
 & "hêrre, iu sol niht wesen leit,\\ 
 & ob ich iuch \textbf{vrâge} mære,\\ 
 & wannen iuwer reise wære."\\ 
 & er seit im \textbf{gar} \textbf{die} underscheit,\\ 
30 & wie er von sîner muoter reit,\\ 
\end{tabular}
\scriptsize
\line(1,0){75} \newline
m n o Fr69 \newline
\line(1,0){75} \newline
\newline
\line(1,0){75} \newline
\textbf{1} und] vnd do n \textbf{3} des dâ] des do m n do das o \textbf{6} dannen] von dennan n \textbf{7} vürste] fursten o \textbf{9} nahtes] nahstes o  $\cdot$ dâ] do m n o \textbf{11} mir] \textit{om.} m  $\cdot$ here] herre m n (o)  $\cdot$ mîn] [mit]: min m  $\cdot$ riet] reit m \textbf{13} ir] [mir]: ir o \textbf{15} kranc] \textit{om.} n o \textbf{16} dâ] Do m n o  $\cdot$ sanc] sitzen n (o) \textbf{17} zer] zú o \textbf{18} sælde] selbe o  $\cdot$ mêrte] werte m \textbf{19} segnen] keren n \textbf{21} dô] Vnd o \textbf{22} dâ] Do m n  $\cdot$ berihtet] vff berichtet n \textbf{28} wannen] Wennan n \newline
\end{minipage}
\end{table}
\newpage
\begin{table}[ht]
\begin{minipage}[t]{0.5\linewidth}
\small
\begin{center}*G
\end{center}
\begin{tabular}{rl}
 & im ist minne unde gruoz bereit,\\ 
 & mag er geniezen werdicheit."\\ 
 & ieslîcher im des \textbf{dâ} \textbf{jach}\\ 
 & unt dar nâch, swer in ie gesach.\\ 
5 & der wirt in \textbf{bî} der hant gevienc,\\ 
 & geselliclîcher dannen gienc.\\ 
 & in vrâgte der vürste mære,\\ 
 & \textbf{welch} sîn ruowe wære\\ 
 & des nahtes dâ bî im gewesen.\\ 
10 & "\textbf{hêrre}, \textbf{dâ}\textbf{ne} wære ich niht genesen,\\ 
 & wan daz \textbf{mîn muoter her mir} r\textit{ie}t\\ 
 & des tages, dô ich von ir schiet."\\ 
 & "got müeze lônen iu unde ir.\\ 
 & hêrre, ir tuot genâde an mir."\\ 
15 & \textbf{sus} gienc der helt \textbf{mit} witzen kranc,\\ 
 & dâ man gote unde dem wirte sanc.\\ 
 & der wirt zer messe in lêrte,\\ 
 & \textbf{daz noch die} sælde mêrte,\\ 
 & opferen unde segenen sich\\ 
20 & \textit{und} gein dem tievel kêren gerich.\\ 
 & \textbf{dô} giengens ûf den palas,\\ 
 & dâ der tisch \textbf{verdecket} was.\\ 
 & der gast zuo sînem wirte saz,\\ 
 & die spîser ungesmæhet az.\\ 
25 & der wirt sprach durch höfscheit:\\ 
 & "hêrre, iu sol niht wesen leit,\\ 
 & obe ich iuch \textbf{vrâge} mære,\\ 
 & wa\textit{n}nen iwer reise wære."\\ 
 & er saget im \textbf{die} underscheit,\\ 
30 & wier von sîner muoter reit,\\ 
\end{tabular}
\scriptsize
\line(1,0){75} \newline
G I O L M Q R Z Fr21 \newline
\line(1,0){75} \newline
\textbf{1} \textit{Initiale} Q  \textbf{5} \textit{Initiale} L  \textbf{7} \textit{Initiale} I O R Z Fr21  \textbf{15} \textit{Initiale} M  \textbf{21} \textit{Capitulumzeichen} L  \textbf{25} \textit{Initiale} I  \newline
\line(1,0){75} \newline
\textbf{2} geniezen] gemessen R \textbf{3} des] \textit{om.} R  $\cdot$ dâ] do O L Q \textit{om.} Fr21  $\cdot$ jach] veriach I (O) (M) (Q) (Z) (Fr21) gegen veriach R \textbf{4} swer] wer L M Q R Z  $\cdot$ ie gesach] sach I ýe ersach L [geschach]: gesach Fr21 \textbf{5} gevienc] vie O geviek Q \textbf{6} geselliclîcher] Jclicher M \textbf{7} in] ÷n O  $\cdot$ vrâgte] fragt Z Fr21  $\cdot$ vürste] wirt O L M Fr21 \textbf{8} welch] Wie O Z  $\cdot$ ruowe] gemach L fraw Q [muͦtter]: Ruͦwe R \textbf{9} dâ] \textit{om.} O Fr21  $\cdot$ im] \textit{om.} R \textbf{10} dâne] do Q  $\cdot$ ich] \textit{om.} O \textbf{11} daz] das mir R  $\cdot$ her mir] mir her L her R  $\cdot$ riet] reit G \textbf{12} dô] da M Z \textbf{13} müeze] muͦsz R \textbf{15} \textit{Die Verse 169.15-16 fehlen} R   $\cdot$ sus] Ausz Q Da Z \textbf{16} dâ] Do Q \textbf{17} dir wirt in zuͤ der messe lerte I \textbf{18} sælde mêrte] salde meret G sele nerte O Fr21 \textbf{19} segenen] gesegen Z \textbf{20} und] \textit{om.} G  $\cdot$ kêren] \textit{om.} Fr21 \textbf{21} dô] dar nach I (O) (M) (Q) (R) (Z) (Fr21)  $\cdot$ den] dem O dē M Q \textbf{22} dâ] Al da O (M) (Q) (R) (Z) (Fr21)  $\cdot$ verdecket] Gedechet I (O) (L) (M) (Q) (Z) (Fr21) \textbf{23} sînem] dem I (O) \textbf{24} ungesmæhet] vngeschmeche R \textbf{26} iu sol] iv ensol O (M) ensol úch R \textbf{27} iuch] \textit{om.} Q  $\cdot$ vrâge] fragte R \textbf{28} wannen] wanenen G \textbf{29} saget] sagte L  $\cdot$ im] im gar O L (M) Q R Z  $\cdot$ underscheit] warheit I \newline
\end{minipage}
\hspace{0.5cm}
\begin{minipage}[t]{0.5\linewidth}
\small
\begin{center}*T
\end{center}
\begin{tabular}{rl}
 & im ist minne unde gruoz bereit,\\ 
 & mag er geniezen werdecheit."\\ 
 & \textbf{ir} iegelîcher im des \textbf{verjach}\\ 
 & unde dar nâch, swer in ie gesach.\\ 
5 & Der wirt in \textbf{bî} der hant gevienc,\\ 
 & geselleclîche er dannen gienc.\\ 
 & in vrâgete der vürste mære,\\ 
 & \textbf{welich} sîn ruowe wære\\ 
 & des nahtes dâ bî im gewesen.\\ 
10 & "\textbf{Hêrre}, \textbf{dâ} wære ich niht ge\textit{n}esen,\\ 
 & wan daz \textbf{mîn muoter mir her} riet\\ 
 & des tages, dô ich von ir schiet."\\ 
 & "got müeze lônen iu unde ir.\\ 
 & hêrre, ir tuot genâde an mir."\\ 
15 & \textbf{\begin{large}S\end{large}us} gienc der helt \textbf{an} witzen kranc,\\ 
 & dâ man gote unde dem wirte sanc.\\ 
 & der wirt zer messe in lêrte\\ 
 & \textbf{unde ouch sîn} sælde mêrte,\\ 
 & opfern unde segenen sich\\ 
20 & unde gegen dem tievel kêren gerich.\\ 
 & \textbf{Dar nâch} giengens ûf den palas,\\ 
 & \textbf{al} dâ der tisch \textbf{gedecket} was.\\ 
 & Der gast zuo sînem wirte saz,\\ 
 & die spîse er ungesmæhet az.\\ 
25 & Der wirt sprach durch hövescheit:\\ 
 & "hêrre, iu sol niht wesen leit,\\ 
 & ob ich iuch \textbf{vrâgete} mære,\\ 
 & wannen iuwer reise wære."\\ 
 & Er seitim \textbf{gar} \textbf{die} underscheit,\\ 
30 & wie er von sîner muoter reit,\\ 
\end{tabular}
\scriptsize
\line(1,0){75} \newline
T U V W \newline
\line(1,0){75} \newline
\textbf{5} \textit{Initiale} W   $\cdot$ \textit{Majuskel} T  \textbf{10} \textit{Majuskel} T  \textbf{15} \textit{Initiale} T U V  \textbf{21} \textit{Majuskel} T  \textbf{23} \textit{Majuskel} T  \textbf{25} \textit{Majuskel} T  \textbf{29} \textit{Majuskel} T  \newline
\line(1,0){75} \newline
\textbf{3} ir] \textit{om.} W \textbf{4} swer in ie] wer in ie U seiner wer in W \textbf{6} er] \textit{om.} U \textbf{9} dâ] do U V \textit{om.} W \textbf{10} dâ] do U W [*]: dane V  $\cdot$ genesen] gewesen T \textbf{11} mir her] har mir V \textbf{15} Sus] Pvs U  $\cdot$ an] mit W \textbf{16} dâ] Do U V W \textbf{17} in] auch W \textbf{18} unde ouch sîn] [*]: Daz ie die V Do mit er im die W \textbf{19} segenen] schoͤne segnen W \textbf{23} zuo] bei W \textbf{24} Speise er guͤtlichen aß W \textbf{27} iuch] îv T  $\cdot$ vrâgete] vrage U V (W) \textbf{29} seitim] seit in V  $\cdot$ die] den W \newline
\end{minipage}
\end{table}
\end{document}
