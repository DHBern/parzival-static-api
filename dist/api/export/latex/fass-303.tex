\documentclass[8pt,a4paper,notitlepage]{article}
\usepackage{fullpage}
\usepackage{ulem}
\usepackage{xltxtra}
\usepackage{datetime}
\renewcommand{\dateseparator}{.}
\dmyyyydate
\usepackage{fancyhdr}
\usepackage{ifthen}
\pagestyle{fancy}
\fancyhf{}
\renewcommand{\headrulewidth}{0pt}
\fancyfoot[L]{\ifthenelse{\value{page}=1}{\today, \currenttime{} Uhr}{}}
\begin{document}
\begin{table}[ht]
\begin{minipage}[t]{0.5\linewidth}
\small
\begin{center}*D
\end{center}
\begin{tabular}{rl}
\textbf{303} & \textbf{\begin{large}M\end{large}în hêr Gawan} \textbf{dô} sprach:\\ 
 & "swaz hie \textbf{mit rede gein iu} geschach,\\ 
 & \textbf{diu} ist lûter unt minneclîch\\ 
 & \textbf{unt} niht mit \textbf{stæter} trüebe rîch.\\ 
5 & Ich ger, als ich\textbf{z} gedienen wil.\\ 
 & hie \textbf{lît} ein künec unt rîter vil\\ 
 & \textbf{unt manec vrouwe wol gevar}.\\ 
 & geselleschaft gib ich iu dar,\\ 
 & lât \textbf{ir} mich mit iu rîten;\\ 
10 & \textbf{dâ bewar ich iuch} vor strîten."\\ 
 & "\textbf{Iwer genâde}, hêrre, ir sprechet wol,\\ 
 & daz ich vil gerne \textbf{dienen} sol,\\ 
 & sît ir kumpânîe bietet mir,\\ 
 & \textbf{nû} wer ist iwer hêrre oder ir?"\\ 
15 & "ich heize '\textbf{hêrre}' einen man,\\ 
 & von dem ich manec urbor hân.\\ 
 & ein teil ich \textbf{der} \textbf{benenne} hie.\\ 
 & er \textbf{was} gein mir des willen ie,\\ 
 & daz er \textbf{mirz} \textbf{rîterlîche} bôt.\\ 
20 & sîne swester het der künec Lot,\\ 
 & diu mich zer werlde brâhte.\\ 
 & swes got an mir gedâhte,\\ 
 & daz biutet dienst sîner hant.\\ 
 & der künec Artus ist er genant.\\ 
25 & \textbf{mîn nam ist ouch} \textbf{vil} \textbf{unverholn},\\ 
 & an \textbf{allen steten} \textbf{unverstoln}.\\ 
 & liute, die mich erkennent,\\ 
 & Gawan \textbf{mich die} nennent.\\ 
 & iu dient \textbf{mîn} lîp unt \textbf{der} name,\\ 
30 & welt irz kêren mir von schame."\\ 
\end{tabular}
\scriptsize
\line(1,0){75} \newline
D \newline
\line(1,0){75} \newline
\textbf{1} \textit{Initiale} D  \textbf{5} \textit{Majuskel} D  \textbf{11} \textit{Majuskel} D  \newline
\line(1,0){75} \newline
\newline
\end{minipage}
\hspace{0.5cm}
\begin{minipage}[t]{0.5\linewidth}
\small
\begin{center}*m
\end{center}
\begin{tabular}{rl}
 & \textbf{mîn hêrre Gawan} sprach:\\ 
 & "waz hie \textbf{mit rede gegen iu} geschach,\\ 
 & \textbf{daz} ist lûter und minneclîch\\ 
 & \textbf{und} niht mit \textbf{st\textit{æ}ter} trüebe rîch.\\ 
5 & ich ger, als ich gedienen wil.\\ 
 & hie \textbf{lît} ein künic und ritter vil\\ 
 & \textbf{und manic vrouwe wol gevar}.\\ 
 & geselleschaft gib ich iu dar,\\ 
 & lât mich \textbf{dar} mit iu rîten;\\ 
10 & \textbf{ich bewar iuch d\textit{â}} vor strîten."\\ 
 & "\textbf{iuwer gnâde}, hêr\textit{r}e, ir sprechet wol,\\ 
 & daz ich vil gerne \textbf{dienen} sol,\\ 
 & sît ir k\textit{o}mp\textit{â}nîe bi\textit{e}tet mir,\\ 
 & \textbf{nû} wer i\textit{st} iuwer hêrre oder ir?"\\ 
15 & "ich heize '\textbf{hêrre}' einen man,\\ 
 & von dem ich manige urbor hân.\\ 
 & ein teil ich \textbf{der} \textbf{benennen wil} hie.\\ 
 & er \textbf{was} gegen mir des willen ie,\\ 
 & daz er \textbf{mirz} \textbf{ritterlîch} bôt.\\ 
20 & sîne swester het der künic Lot,\\ 
 & diu mich zer werlt brâhte.\\ 
 & wes got an mir gedâhte,\\ 
 & daz biutet dienest sîner hant.\\ 
 & der künic Artus ist er genant.\\ 
25 & \textbf{mîn name ist ouch} \textbf{vil} \textbf{unverholn},\\ 
 & an \textbf{allen steten} \textbf{unverstoln}.\\ 
 & liute, die mich erkennent,\\ 
 & Gawan \textbf{mich die} nennent.\\ 
 & iu dienet lîp und \textbf{der} name,\\ 
30 & welt irz kêren mir von schame."\\ 
\end{tabular}
\scriptsize
\line(1,0){75} \newline
m n o \newline
\line(1,0){75} \newline
\newline
\line(1,0){75} \newline
\textbf{1} mîn] Der n o  $\cdot$ Gawan] gewan o  $\cdot$ sprach] do sprach n o \textbf{2} rede] reden o  $\cdot$ geschach] beschach n o \textbf{4} stæter] stritter m  $\cdot$ trüebe] truwen n o \textbf{5} ich] ichs n (o) \textbf{10} dâ] do m n \textbf{11} hêrre] herte m \textbf{13} kompânîe] campenie m  $\cdot$ bietet] bittet m bieten o \textbf{14} ist] ich m \textbf{17} wil] \textit{om.} n o \textbf{18} er] Es o \textbf{22} gedâhte] hat gedacht o \textbf{24} der] \textit{om.} n o \textbf{28} mich die] die mich n \textbf{29} lîp] min lip n der lip o \textbf{30} welt] Wolten n Welte o  $\cdot$ irz kêren mir] ir mirs keren n  $\cdot$ von] fúr n \newline
\end{minipage}
\end{table}
\newpage
\begin{table}[ht]
\begin{minipage}[t]{0.5\linewidth}
\small
\begin{center}*G
\end{center}
\begin{tabular}{rl}
 & \textbf{des küniges Lotes sun} \textbf{dô} sprach:\\ 
 & "swaz hie \textbf{mit rede gein iu} geschach,\\ 
 & \textbf{d\textit{iu}} ist lûter unde minniclîch,\\ 
 & niht mit \textbf{valscher} trüebe rîch.\\ 
5 & ich ger, als ich \textbf{ez} gedienen wil.\\ 
 & hie \textbf{lît} ein künic unde rîter vil\\ 
 & \textbf{mit wünniclîcher vrouwen schar}.\\ 
 & geselleschaft gibe ich iu dar,\\ 
 & lât mich mit iu rîten,\\ 
10 & \textbf{sô bewar ich iuch} vor strîten."\\ 
 & "\textbf{got lône iu}, hêrre, ir sprecht wol,\\ 
 & daz ich vil gerne \textbf{dienen} sol,\\ 
 & sît ir kompânîe bietet mir,\\ 
 & wer ist iuwer hêrre oder ir?"\\ 
15 & "ich heize \textbf{hêrren} einen man,\\ 
 & von dem ich manic urbor hân.\\ 
 & \begin{large}E\end{large}in teil ich \textbf{der} \textbf{b\textit{en}enne} hie.\\ 
 & er \textbf{pflac} gein mir des willen ie,\\ 
 & daz er \textbf{mirz} \textbf{williclîchen} bôt.\\ 
20 & sîne swester hât der künic Lot,\\ 
 & diu mich zer werlde brâhte.\\ 
 & swes got an mir gedâhte,\\ 
 & daz biut dienst sîner hant.\\ 
 & der künic Artus ist er genant.\\ 
25 & \textbf{ouch ist mîn name} \textbf{unverstolen},\\ 
 & an \textbf{maniger stat} \textbf{vil} \textbf{unverholen}.\\ 
 & liute, die mich erkennent,\\ 
 & Gawa\textit{n} \textbf{mich die} nennent.\\ 
 & iu dient \textbf{mîn} lîp unde \textbf{ouch mîn} name,\\ 
30 & welt irz kêren mir von schame."\\ 
\end{tabular}
\scriptsize
\line(1,0){75} \newline
G I O L M Q R Z \newline
\line(1,0){75} \newline
\textbf{1} \textit{Initiale} L R  \textbf{5} \textit{Initiale} Z  \textbf{11} \textit{Initiale} O Q  \textbf{13} \textit{Initiale} I  \textbf{15} \textit{Initiale} M  \textbf{17} \textit{Initiale} G  \textbf{27} \textit{Initiale} I  \newline
\line(1,0){75} \newline
\textbf{1} Min her Gawan do sprach Z  $\cdot$ küniges] konnick M  $\cdot$ Lotes] lotis M  $\cdot$ dô] da M \textit{om.} Q \textbf{2} swaz] Waz L (M) (Q) (R)  $\cdot$ hie] ie Z  $\cdot$ mit] \textit{om.} I  $\cdot$ rede] reden Q (R)  $\cdot$ gein iu] \textit{om.} L \textbf{3} diu] daz G  $\cdot$ minniclîch] mundigklich Q \textbf{4} niht mit] Vnd niht L  $\cdot$ valscher trüebe] valsche betrube M falscher truben Q falscher trewe Z \textbf{5} ez] \textit{om.} R  $\cdot$ gedienen] genden O genyszen L \textbf{7} Vnd manic frowe wol gevar Z \textbf{9} lât] lat ir I (O) (L) (M) (Q) (R) \textbf{10} vor] von I \textbf{11} got] ÷ot O \textbf{12} dienen] vmb úch diennen R \textbf{14} wer] Nv wer Z  $\cdot$ ist] \textit{om.} O  $\cdot$ ir] wer sit ir Z \textbf{15} heize] haiz in I  $\cdot$ hêrren] herre I (O) (M) (R) (Z)  $\cdot$ einen] vnde bin sin I \textbf{17} benenne] benne G  $\cdot$ hie] alhie Z \textbf{18} gein mir des willen] des willen gein myr M \textbf{19} er mirz] ersz mir Q er mirs ye R  $\cdot$ williclîchen] riterliche O (L) (M) (Q) (R) (Z) \textbf{20} der] \textit{om.} R  $\cdot$ Lot] lott R \textbf{22} swes] Wes L M Q R  $\cdot$ mir] mich I O L Q R \textbf{23} sîner] der sinen Q von siner R \textbf{24} der] \textit{om.} L R Z \textbf{25} unverstolen] vil vnuerholn I vil vnverstoln O (R) gar vnuerstolen Q \textbf{26} vil] \textit{om.} I gar Q  $\cdot$ unverholen] vnuerstoln I \textbf{28} Gawan] Gawanen G  $\cdot$ mich die nennent] die mich nennet Q bin ich genennet R \textbf{29} iu] vnd I  $\cdot$ ouch] \textit{om.} I L M \textbf{30} irz kêren mir] ir myrs keren M  $\cdot$ von] fur Q zu R \newline
\end{minipage}
\hspace{0.5cm}
\begin{minipage}[t]{0.5\linewidth}
\small
\begin{center}*T
\end{center}
\begin{tabular}{rl}
 & \textbf{\begin{large}D\end{large}es küneges Lotes sun} \textbf{dô} sprach:\\ 
 & "swaz hie \textbf{gegen iu mit rede} geschach,\\ 
 & \textbf{diu}st lûter unde minneclîch,\\ 
 & niht mit \textbf{valscher} trüebe rîch.\\ 
5 & ich ger, als ich\textbf{z} gedienen wil.\\ 
 & hie \textbf{ist} ein künec unde rîter vil\\ 
 & \textbf{mit wünneclîcher vrouwen schar}.\\ 
 & geselleschaft gib ich iu dar,\\ 
 & lât \textbf{ir} mich mit iu rîten,\\ 
10 & \textbf{sô bewar ich iuch} vor strîten."\\ 
 & "\textbf{Got lône iu}, hêrre, ir sprechet wol,\\ 
 & daz ich vil \textit{gerne} \textbf{gedienen} sol,\\ 
 & sît ir kumpânîe bietet mir,\\ 
 & wer ist iuwer hêrre oder ir?"\\ 
15 & "Ich heize \textbf{hêrren} einen man,\\ 
 & von dem ich manec urbor hân.\\ 
 & ein teil \textit{ich} \textbf{iu} \textbf{benenne} hie.\\ 
 & er \textbf{pflac} gegen mir des willen ie,\\ 
 & daz er\textbf{z mir} \textbf{rîterlîche} bôt.\\ 
20 & sîne swester hât der künec Lot,\\ 
 & diu mich zer werlte brâhte.\\ 
 & swes got an mir gedâhte,\\ 
 & daz biutet dienst sîner hant.\\ 
 & der künec Artus ist er genant.\\ 
25 & \textbf{ouch ist mîn name} \textbf{vil} \textbf{unverstoln},\\ 
 & an \textbf{maneger stat} \textbf{vil} \textbf{unverholn}.\\ 
 & liute, die mich erkenne\textit{n}t,\\ 
 & Gawan \textbf{si mich} nenne\textit{n}t.\\ 
 & iu dient \textbf{mîn} lîp unde \textbf{ouch mîn} name,\\ 
30 & welt irz kêren mir von schame."\\ 
\end{tabular}
\scriptsize
\line(1,0){75} \newline
T U V W \newline
\line(1,0){75} \newline
\textbf{1} \textit{Initiale} T U V  \textbf{11} \textit{Initiale} W   $\cdot$ \textit{Majuskel} T  \textbf{15} \textit{Majuskel} T  \newline
\line(1,0){75} \newline
\textbf{1} küneges Lotes] kúnig lottes W \textbf{2} Was mit rede gen eúch beschach W  $\cdot$ swaz] Waz U \textbf{3} diust] [*]: Die ist V Die W  $\cdot$ unde] ist vnd W \textbf{4} trüebe] treúwe W \textbf{6} ist] [*]: lit V \textbf{7} wünneclîcher] minniglicher W \textbf{8} gib ich] git U \textbf{9} lât ir mich] Lant mich [*]: dar V \textbf{10} iuch] îv T  $\cdot$ vor] wol vor W \textbf{11} lône] lan W \textbf{12} vil] eúch W  $\cdot$ gerne] \textit{om.} T  $\cdot$ gedienen] [*dienen]: gedienen V dienen W \textbf{13} bietet] mit W \textbf{15} heize hêrren] hiez herren U herre weis got W \textbf{16} urbor] húbe W \textbf{17} ein teil] Den W  $\cdot$ ich] \textit{om.} T  $\cdot$ iu] [*]: der V dir W \textbf{18} er] Der W \textbf{19} erz] er U V W  $\cdot$ mir] mir iz U [*]: mirs V \textbf{20} der] [de*]: der U den V  $\cdot$ Lot] lôt T \textbf{22} swes] Wie iz U Wes W  $\cdot$ mir] mich W \textbf{25} ist] \textit{om.} U \textbf{26} an] In W \textbf{27} erkennent] erkennet T \textbf{28} si] [*]: sie U  $\cdot$ nennent] nennet T \textbf{29} ouch mîn] \textit{om.} W \textbf{30} irz] ir mirs W  $\cdot$ mir von] nit zuͦ W \newline
\end{minipage}
\end{table}
\end{document}
