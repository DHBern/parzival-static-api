\documentclass[8pt,a4paper,notitlepage]{article}
\usepackage{fullpage}
\usepackage{ulem}
\usepackage{xltxtra}
\usepackage{datetime}
\renewcommand{\dateseparator}{.}
\dmyyyydate
\usepackage{fancyhdr}
\usepackage{ifthen}
\pagestyle{fancy}
\fancyhf{}
\renewcommand{\headrulewidth}{0pt}
\fancyfoot[L]{\ifthenelse{\value{page}=1}{\today, \currenttime{} Uhr}{}}
\begin{document}
\begin{table}[ht]
\begin{minipage}[t]{0.5\linewidth}
\small
\begin{center}*D
\end{center}
\begin{tabular}{rl}
\textbf{83} & mit \textbf{maneger} \textbf{werden} \textbf{vrouwen}.\\ 
 & si wolte gerne schouwen\\ 
 & den werden künec von Zazamanc.\\ 
 & vil müeder ritter nâch ir dranc.\\ 
5 & diu tischlachen wâren ab genomen,\\ 
 & ê si \textbf{in}z poulûn wære komen.\\ 
 & \textbf{Ûf spranc der wirt} \textbf{vil} schiere\\ 
 & unt gevangener künege viere.\\ 
 & den vuor ouch etslîch vürste mite.\\ 
10 & dô enpfieng er si \textbf{nâch} zühte site.\\ 
 & er geviel ir wol, dô si in \textbf{ersach}.\\ 
 & diu \textbf{Waleisinne} mit \textbf{vreuden} sprach:\\ 
 & "\textit{\begin{large}I\end{large}}r sît hie wirt, dâ ich iuch \textbf{vant},\\ 
 & sô bin ich wirtîn überz lant.\\ 
15 & \textbf{ruochet} ir\textbf{s}, daz ich iuch küssen sol,\\ 
 & \textbf{daz} ist mit mînem willen wol."\\ 
 & \textbf{Er sprach}: "iwer kus sol wesen mîn,\\ 
 & \textbf{sulen} \textbf{dise} hêrren geküsset sîn.\\ 
 & sol künec \textbf{oder} vürste des enbern,\\ 
20 & sô\textbf{ne} getar \textbf{ouch} ich\textbf{s} von iu niht \textbf{gern}."\\ 
 & "\textbf{deiswar}, daz sol ouch geschehen.\\ 
 & i\textbf{ne} hân ir keinen ê gesehen."\\ 
 & si \textbf{kuste}, die es \textbf{dâ} wâren wert.\\ 
 & des hete Gahmuret gegert.\\ 
25 & er bat sitzen die künegîn.\\ 
 & mîn hêr Brandelidelin\\ 
 & mit zühten \textbf{zuo} der vrouwen saz.\\ 
 & grüene binz, von touwe naz,\\ 
 & dünne ûf \textbf{die teppeche} \textbf{was} gestrœt,\\ 
30 & \textbf{dâ saz ûf}, \textbf{des} sich hie vrœt,\\ 
\end{tabular}
\scriptsize
\line(1,0){75} \newline
D \newline
\line(1,0){75} \newline
\textbf{7} \textit{Majuskel} D  \textbf{13} \textit{Initiale} D  \textbf{17} \textit{Majuskel} D  \newline
\line(1,0){75} \newline
\textbf{3} Zazamanc] Zazamanch D \textbf{13} Ir] ÷R D \textbf{24} Gahmuret] Gahmvret D \newline
\end{minipage}
\hspace{0.5cm}
\begin{minipage}[t]{0.5\linewidth}
\small
\begin{center}*m
\end{center}
\begin{tabular}{rl}
 & mit \textbf{maniger} \textbf{werden} \textbf{vrouwen}.\\ 
 & si wolte gerne schouwen\\ 
 & de\textit{n} werden künic von Zazamanc.\\ 
 & vil müeder ritter nâch ir dranc.\\ 
5 & diu tischlachen wâren ab genomen,\\ 
 & ê si \textbf{in} daz pavelûn wære komen.\\ 
 & \textbf{ûf spranc der wirt} schiere\\ 
 & und gevangener künige viere.\\ 
 & den vuor ouch etlîch vürste mite.\\ 
10 & dô enpfienc er si \textbf{nâch} zühte site.\\ 
 & er geviel ir wol, dô si in \textbf{gesach}.\\ 
 & diu \textbf{W\textit{a}l\textit{ei}s\textit{inn}e} mit \textbf{vröuden} sprach:\\ 
 & "ir sît hie wirt, d\textit{â} ich iuch \textbf{mant},\\ 
 & sô b\textit{in} ich wirtinne über daz lant.\\ 
15 & \textbf{ruochet} ir\textbf{s}, daz ich iuch küssen sol,\\ 
 & \textbf{daz} ist mit mînem willen wol."\\ 
 & \textbf{er sprach}: "iuwer kus sol wesen mîn,\\ 
 & \textbf{sullen} \textbf{die} hêrren geküsset sîn.\\ 
 & sol künic, vürste des enbern,\\ 
20 & sô getar \textbf{ouch} ich\textbf{s} von iu niht \textbf{gern}."\\ 
 & \multicolumn{1}{l}{ - - - }\\ 
 & \multicolumn{1}{l}{ - - - }\\ 
 & si \textbf{kusten}, die es \textbf{dâ} wâren wert.\\ 
 & des hete Gahmuret gegert.\\ 
25 & \begin{large}E\end{large}r bat sitzen die künigîn.\\ 
 & mîn hêrre Bra\textit{n}delidelin\\ 
 & mit zühten \textbf{zuo} der vrowen saz.\\ 
 & grüene \dag bin ichs\dag , von touwe naz,\\ 
 & dünne ûf \textbf{die teppiche} \textbf{wâren} geströuwet,\\ 
30 & \textbf{dâ saz ûf}, \textbf{des} sich hie vröuwet,\\ 
\end{tabular}
\scriptsize
\line(1,0){75} \newline
m n o \newline
\line(1,0){75} \newline
\textbf{25} \textit{Illustration mit Überschrift:} Also gamiret (gamuͯret o  ) die koͯnnigin (juͯnge konigin o  ) bat sitzen mit iren jungfrouwen n (o)   $\cdot$ \textit{Initiale} m n o  \newline
\line(1,0){75} \newline
\textbf{2} wolte] woltent n (o) \textbf{3} den] Der m  $\cdot$ Zazamanc] zazamang m n o \textbf{4} müeder] mutter o \textbf{6} pavelûn] penelún o  $\cdot$ wære] waren n \textbf{9} vuor] fuͦrt n \textbf{10} nâch] mit o  $\cdot$ zühte] zúchtem n (o) \textbf{11} ir] [in]: ir ir o \textbf{12} Waleisinne] wolsame m n wol same o \textbf{13} hie] noch hie o  $\cdot$ dâ] do m n o  $\cdot$ mant] nante n o \textbf{14} bin ich] bich \textit{nachträglich korrigiert zu:} bin ich m  $\cdot$ daz] dis n o \textbf{16} mînem] mẏnnen o \textbf{20} von iu niht] nit von vch o \textbf{21} \textit{Die Verse 83.21-22 fehlen} m n o  \textbf{23} si] Die o  $\cdot$ dâ] do n o  $\cdot$ wâren] [wert]: weren o \textbf{24} Gahmuret] gamiret n gemuͯret o  $\cdot$ gegert] begert n o \textbf{25} Er] EEr o  $\cdot$ die] do die n \textbf{26} Brandelidelin] Bradelidelin m \textbf{28} ichs] ich n o \textbf{29} wâren] was n o  $\cdot$ geströuwet] getrowet o \newline
\end{minipage}
\end{table}
\newpage
\begin{table}[ht]
\begin{minipage}[t]{0.5\linewidth}
\small
\begin{center}*G
\end{center}
\begin{tabular}{rl}
 & mit \textbf{maniger} \textbf{juncvrouwen}.\\ 
 & si wolte gerne schouwen\\ 
 & den werden künic von Zazamanc.\\ 
 & vil müeder rîter nâch ir dranc.\\ 
5 & diu tischlachen wâren abe genomen,\\ 
 & ê si \textbf{under}z pavelûn wære komen.\\ 
 & \textbf{der wirt spranc ûf} \textbf{vil} schiere\\ 
 & unde gevangener künige viere.\\ 
 & den vuor ouch etslîch vürste mite.\\ 
10 & dô enpfieng er si \textbf{mit} zühte site.\\ 
 & er geviel ir wol, dô sin \textbf{gesach}.\\ 
 & diu \textbf{künigîn} mit \textbf{\textit{vröud}en} sprach:\\ 
 & "ir sît hie wirt, dâ ich iuch \textbf{vant},\\ 
 & sô bin ich wirtîn überz lant.\\ 
15 & \textbf{ge\textit{ruoch}t} ir, daz ich iuch küssen sol,\\ 
 & \textbf{ez} ist mit mînem willen wol."\\ 
 & "\begin{large}I\end{large}wer kus sol wesen mîn,\\ 
 & \textbf{mugen} \textbf{dise} hêrren geküsset sîn.\\ 
 & sol künic \textbf{oder} vürste des enberen,\\ 
20 & sô\textbf{ne} geta\textit{r} \textit{i}ch \textbf{es} von iu niht \textbf{gegeren}."\\ 
 & \textbf{si sprach}: "daz sol ouch geschehen.\\ 
 & ich hân ir deheinen ê gesehen."\\ 
 & si \textbf{kuste} \textbf{al}, dies \textbf{dâ} wâren wert.\\ 
 & des hete Gahmuret gegert.\\ 
25 & er bat sitzen die künigîn.\\ 
 & mîn hêr Brandelidelin\\ 
 & mit zühten \textbf{zuo} der vrouwen saz.\\ 
 & grüene binz, von touwe naz,\\ 
 & dünne ûf \textbf{den tepch} \textbf{was} geströut,\\ 
30 & \textbf{dâr ûf saz}, \textbf{des} sich hie vröut,\\ 
\end{tabular}
\scriptsize
\line(1,0){75} \newline
G I O L M Q R Z \newline
\line(1,0){75} \newline
\textbf{1} \textit{Initiale} O M  \textbf{5} \textit{Initiale} I Q R Z  \textbf{11} \textit{Initiale} L  \textbf{17} \textit{Initiale} G  \textbf{19} \textit{Initiale} I  \newline
\line(1,0){75} \newline
\textbf{1} mit] ÷it O  $\cdot$ maniger] liehten O (R) (Z) rittern vnd L lichte M Q  $\cdot$ juncvrouwen] mit frowen L vrouwen M \textbf{2} si] So R  $\cdot$ wolte] woldin M (Q) (R) (Z) \textbf{3} Zazamanc] zazamanch G O L zazamant Q zasmanc R \textbf{4} vil müeder rîter] Mvder ritter vil Z  $\cdot$ ir] im R \textbf{6} ê] do I  $\cdot$ underz] vnder die I (M) widerz O in die L  $\cdot$ wære] warn I (M) (Q) werren R (Z) \textbf{8} gevangener] geuangern I gefangen Q  $\cdot$ künige] riter I \textbf{9} den] Jm O L Jnt M Vnd Z  $\cdot$ vuor] \textit{om.} Z  $\cdot$ vürste] kunigk Q fᵫrsten R (Z) \textbf{10} Er enpfienc sie nach zvhte site Z  $\cdot$ dô] Da M Die R  $\cdot$ er si] siv O sie Q er R  $\cdot$ mit] nach ir L mit mit R  $\cdot$ zühte] [zvht]: zvhte G zuhtechlichem I  $\cdot$ site] mitte R \textbf{11} dô] da M Z  $\cdot$ gesach] shach I ersach O sach L Q \textbf{12} vröuden] zuhten G \textbf{13} dâ] \textit{om.} M do Q \textbf{14} überz] vber diz L \textbf{15} geruocht] gebiet G Geruͦchen R  $\cdot$ ir] \textit{om.} L irs M Q R Z \textbf{16} ez] Daz Z \textbf{17} Iwer kus sol] Er sprach der kusz sol Q Er sprach úwer [s]: kus sol R \textbf{18} mugen] Svllen Z  $\cdot$ hêrren] herre R  $\cdot$ geküsset] gek:ssz M gezogen Q \textbf{20} sône] So L Q R Z  $\cdot$ getar ich es] getar och ihes G tar ichz M [kar]: kan ich Q  $\cdot$ gegeren] gern I L M gewern R \textbf{22} ich] ichn I (O) (L) (M) (R) (Z)  $\cdot$ ir] iren Q  $\cdot$ gesehen] ge ien M \textbf{23} kuste] kusten Z  $\cdot$ dies dâ wâren] die sin warn I (Z) die des woren O die ez waren L (M) (R) disze worren Q \textbf{24} hete] hat M  $\cdot$ Gahmuret] Gamvret O Gahmuͯret L gamuret M Q Z \textbf{25} sitzen] susze M \textbf{26} Brandelidelin] brandalidelin I Brantlidelin L Brandelin M brandlidelin Q \textbf{28} binz] pimz I O byn ich M semden Z \textbf{29} dünne] \textit{om.} O  $\cdot$ den] die O L Q dem R  $\cdot$ geströut] gestipt M \textbf{30} des] der I  $\cdot$ hie] dy M \newline
\end{minipage}
\hspace{0.5cm}
\begin{minipage}[t]{0.5\linewidth}
\small
\begin{center}*T (U)
\end{center}
\begin{tabular}{rl}
 & mit \textbf{liehten} \textbf{juncvrouwen}.\\ 
 & si wolte gerne schouwen\\ 
 & den werden künec von Zazamanc.\\ 
 & vil müeder ritter nâch ir dranc.\\ 
5 & diu tischlachen wâren abe genomen,\\ 
 & ê si \textbf{under}\textit{z} pavelûn wære komen.\\ 
 & \textit{\textbf{der wirt spranc ûf}} \textit{\textbf{vil}} \textit{schiere}\\ 
 & und gevange\textit{ne}r künige viere.\\ 
 & den vuor ouch etslîch vürste mite.\\ 
10 & dô entvienc er si \textbf{mit} zühte site.\\ 
 & er geviel ir wol, dô si in \textbf{gesach}.\\ 
 & diu \textbf{künigîn} mit \textbf{zühten} sprach:\\ 
 & "ir sît hie wirt, d\textit{â} ich iuch \textbf{vant},\\ 
 & sô bin ich wirtîn über daz lant.\\ 
15 & \textbf{geruochet} ir, daz ich iuch küssen sol,\\ 
 & \textbf{daz} ist mit mîme willen wol."\\ 
 & "iuwer kus sol wesen mîn,\\ 
 & \textbf{mugen} \textbf{dise} hêrren geküsset sîn.\\ 
 & sol künec \textbf{oder} vürste des enbern,\\ 
20 & sô g\textit{et}ar ich, \textbf{vrouwe}, von iu niht \textbf{gern}."\\ 
 & \textbf{si sprach}: "daz sol ouch geschehen.\\ 
 & ich \textbf{en}hân ir dekeinen ê gesehen."\\ 
 & si \textbf{kuste} \textbf{al}, die es wâren wert.\\ 
 & des hete Gahmuret gegert.\\ 
25 & er bat sitzen die künigîn.\\ 
 & mîn hêrre Brandelidelin\\ 
 & mit zühten \textbf{vor} de\textit{r} vrouwen saz.\\ 
 & grüene binzen, von touwe naz,\\ 
 & \textit{d}ünne \textit{ûf} \textbf{\textit{daz} \textit{teppic}} \textbf{wâ\textit{ren}} geströuwet.\\ 
30 & \textbf{dâr ûffe saz}, \textbf{der} sich hie vröuwet,\\ 
\end{tabular}
\scriptsize
\line(1,0){75} \newline
U V W T \newline
\line(1,0){75} \newline
\textbf{5} \textit{Initiale} V W T  \textbf{7} \textit{Majuskel} T  \textbf{15} \textit{Majuskel} T  \textbf{17} \textit{Majuskel} T  \textbf{21} \textit{Majuskel} T  \newline
\line(1,0){75} \newline
\textbf{1} [Mi*h*gf*]: Mit rittern vnd mit frowen V  $\cdot$ liehten] schonen W \textbf{2} si wolte] die wolten T \textbf{3} Zazamanc] zazamang V W \textbf{4} vil] gnvͦc T  $\cdot$ ir] [*]: in V \textbf{5} abe] dann W \textbf{6} underz] vnder U zem T  $\cdot$ wære] waz V \textbf{7} \textit{Vers 83.7 fehlt} U  \textbf{8} gevangener] gevanger U \textbf{9} den] jr T  $\cdot$ ouch] \textit{om.} T  $\cdot$ vürste] fúrstein W \textbf{10} Den entpfieng sy mit zúchtten sitte W  $\cdot$ entvienc er] enpfienger er T  $\cdot$ zühte] goͮtem V \textbf{11} gesach] sach W \textbf{12} zühten] freúden W \textbf{13} dâ] do U V W \textbf{14} wirtîn] frauw W  $\cdot$ daz] all diß W \textbf{16} daz] ez T \textbf{18} mugen] mvget T  $\cdot$ sîn] \textit{om.} T \textbf{20} getar ich] gent dar ich U getar ichs V getar ich des T  $\cdot$ vrouwe] \textit{om.} W T \textbf{21} ouch] eúch W \textbf{22} enhân] han W T  $\cdot$ ir] \textit{om.} W \textbf{23} al die es] alle die es da (do W ) V die des T \textbf{24} Gahmuret] Gahmuͦret U Gamuret V (W) \textbf{27} zühten] zuht V  $\cdot$ vor der] vor den U [*]: er zvͦ der V [*]: zvͦ der T \textbf{28} binzen] bintz W \textbf{29} Vf die dvnne was gestreuͦwet U · Dúnne auff das was gestraut W · dvnne vf den teppic was gestrôvt T \textbf{30} der] [*]: dez V die W (T)  $\cdot$ hie] do W \newline
\end{minipage}
\end{table}
\end{document}
