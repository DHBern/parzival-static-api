\documentclass[8pt,a4paper,notitlepage]{article}
\usepackage{fullpage}
\usepackage{ulem}
\usepackage{xltxtra}
\usepackage{datetime}
\renewcommand{\dateseparator}{.}
\dmyyyydate
\usepackage{fancyhdr}
\usepackage{ifthen}
\pagestyle{fancy}
\fancyhf{}
\renewcommand{\headrulewidth}{0pt}
\fancyfoot[L]{\ifthenelse{\value{page}=1}{\today, \currenttime{} Uhr}{}}
\begin{document}
\begin{table}[ht]
\begin{minipage}[t]{0.5\linewidth}
\small
\begin{center}*D
\end{center}
\begin{tabular}{rl}
\textbf{826} & \textbf{\begin{large}D\end{large}ie} naht sîn lîp ir minne \textbf{enpfant};\\ 
 & dô wart er vürste in Brabant.\\ 
 & diu hôchgezît rîlîche ergienc.\\ 
 & manec hêrre von sîner hende enpfienc\\ 
5 & \textbf{ir} lêhen, \textbf{die daz} solten hân.\\ 
 & guot rihtære wart der selbe man.\\ 
 & er tet ouch dicke rîterschaft,\\ 
 & daz er den prîs \textbf{behielt} mit kraft.\\ 
 & Si gewunnen \textbf{samt} schœniu kint.\\ 
10 & \textbf{vil liute} in Brabant noch sint,\\ 
 & die wol wizzen von in beiden,\\ 
 & ir enpfâhen, sîn danscheiden,\\ 
 & \textbf{daz} in ir vrâge \textbf{dan} \textbf{vertreip},\\ 
 & unt wie lange er dâ beleip.\\ 
15 & er schiet \textbf{ouch} ungerne dan.\\ 
 & \textbf{Nû} brâht \textbf{im} aber sîn \textbf{vriwent, der} swan,\\ 
 & ein kleine \textbf{gevüege} seitiez.\\ 
 & sînes kleinœdes er dâ liez\\ 
 & \textbf{ein swert, ein horn}, ein vingerlîn.\\ 
20 & hin \textbf{vuor} Loherangrin.\\ 
 & wel wir dem mære rehte tuon,\\ 
 & sô was \textbf{er} Parzivals sun.\\ 
 & der vuor wazzer und wege\\ 
 & \textbf{unz} wider ins Grâles pflege.\\ 
25 & Durch \textbf{waz} verlôs daz guote wîp\\ 
 & werdes \textbf{vriunts} minneclîchen lîp?\\ 
 & er \textbf{widerriet ir} \textbf{vrâgen} ê,\\ 
 & dô er \textbf{vür si} gienc vome sê.\\ 
 & hie \textbf{solte} Ereck nû sprechen,\\ 
30 & der kunde mit rede \textbf{sich} rechen.\\ 
\end{tabular}
\scriptsize
\line(1,0){75} \newline
D \newline
\line(1,0){75} \newline
\textbf{1} \textit{Initiale} D  \textbf{9} \textit{Majuskel} D  \textbf{16} \textit{Majuskel} D  \textbf{25} \textit{Majuskel} D  \newline
\line(1,0){75} \newline
\textbf{22} Parzivals] Parcifals D \newline
\end{minipage}
\hspace{0.5cm}
\begin{minipage}[t]{0.5\linewidth}
\small
\begin{center}*m
\end{center}
\begin{tabular}{rl}
 & \textbf{bî} naht sî\textit{n} lîp ir minne \textbf{enpfant};\\ 
 & dô wart er vürste in Brabant.\\ 
 & \begin{large}D\end{large}iu hôchgezît rîchelîch ergienc.\\ 
 & manic hêrre von sîner hende enpfienc\\ 
5 & \textbf{ir} \textit{l}êhen, \textbf{die daz} solten hân.\\ 
 & guot rihter wart der selbe man.\\ 
 & er tet ouch dicke ritterschaft,\\ 
 & daz er den prîs \textbf{behielt} mit kraft.\\ 
 & si gewunnen \textbf{samen} schœniu kint.\\ 
10 & \textbf{vil liute} in Bra\textit{b}ant noch sint,\\ 
 & die wol wizzen von in beiden,\\ 
 & ir enpfâhen, \textit{s}în danscheiden,\\ 
 & \textbf{daz} in ir vrâge \textbf{dan} \textbf{vertreip},\\ 
 & und wie lange er dâ bleip.\\ 
15 & er schiet ungerne dan.\\ 
 & \textbf{nû} brâht \textbf{im} aber sîn \textbf{vriunt, der} swan,\\ 
 & ein kleine \textbf{vüege} seitiez.\\ 
 & sînes kleinœtes er d\textit{â} liez\\ 
 & \textbf{ein swert, ein horn}, ein vingerlîn.\\ 
20 & hin \textbf{vuor} Lohelangrin.\\ 
 & wellen wir dem mære rehte tuon,\\ 
 & \textit{sô was \textbf{er} Parcifals sun}.\\ 
 & der vuor wazzer und wege\\ 
 & \textbf{unz} wider in des Grâles pflege.\\ 
25 & durch \textbf{waz} verlôs \textbf{in} daz guote wîp\\ 
 & werdes \textbf{vriundes} minneclîche\textit{n} lîp?\\ 
 & er \textbf{widerriet ir} \textbf{vrâgens} ê,\\ 
 & dô er \textbf{\textit{zuo} i\textit{r}} gienc \textit{vo}n dem sê.\\ 
 & hie \textbf{solte} Ere\textit{c} nû sprechen,\\ 
30 & der kunde mit rede \textbf{sich} rechen.\\ 
\end{tabular}
\scriptsize
\line(1,0){75} \newline
m n V V' W \newline
\line(1,0){75} \newline
\textbf{3} \textit{Initiale} m V   $\cdot$ \textit{Capitulumzeichen} n  \textbf{21} \textit{Initiale} W  \textbf{29} \textit{Initiale} V'  \newline
\line(1,0){75} \newline
\textbf{1} \textit{statt 824.23-826.2:} Wie er zu der herzoginnen gein brabant quam (vgl. 825.15: herzogîn) / Vnd die zu einer amyen nam (Fortsetzung von 824.2; weiterer Text in 826.23) V'   $\cdot$ bî] Die V  $\cdot$ sîn] sẏ m  $\cdot$ ir] \textit{om.} W \textbf{2} Brabant] brobant n probrant V probant W \textbf{3} \textit{Die Verse 826.3-22 fehlen} V'  \textbf{5} lêhen] hehen m \textbf{6} rihter] ritter V \textbf{9} gewunnen] gewunnet V  $\cdot$ samen] samet V \textit{om.} W \textbf{10} Brabant] brabrant m brobant n probrant V probant W \textbf{12} sîn] schin m \textbf{13} daz in ir vrâge] Das in froge n Vnd das ir frage in W \textbf{14} dâ] do n V W \textbf{15} ungerne] oͮch vngerne V \textbf{16} aber] \textit{om.} W \textbf{17} ein] Eine V  $\cdot$ vüege] gefuͤge V W \textbf{18} kleinœtes] cleinoͤter V  $\cdot$ er dâ] erdo m (n) (V) (W) \textbf{21} wellen] [Weilen]: Wellen m \textbf{22} \textit{Vers 826.22 fehlt} m   $\cdot$ er] es W  $\cdot$ Parcifals] parzefals V herr partzifals W \textbf{23} \textit{statt 826.23-24:} Vnd dar nach wider zu dem grol fur also (Fortsetzung von 824.23; vgl. 826.20: vuor) V'   $\cdot$ der vuor] Do fuͦr er W \textbf{24} des] der W \textbf{25} \textit{statt 826.25-28:} Do uon wil ich nit sagen nv / Wan daz wer zu vil / Do uon ich nv swigen wil V'   $\cdot$ durch] Doch W  $\cdot$ verlôs in] verlos V verlassen W \textbf{26} werdes vriundes] Von frúnden W  $\cdot$ minneclîchen] mẏnnecliches m (W) \textbf{27} vrâgens] frogen n V (W) \textbf{28} zuo ir] von in m von ir n fúr sv́ V  $\cdot$ von] an m n \textbf{29} Die solt man sprechen W  $\cdot$ Erec] ere m n [erec]: ereg V crig V' \textbf{30} rechen] gerechen V \newline
\end{minipage}
\end{table}
\newpage
\begin{table}[ht]
\begin{minipage}[t]{0.5\linewidth}
\small
\begin{center}*G
\end{center}
\begin{tabular}{rl}
 & \textbf{\begin{large}D\end{large}ie} naht sîn lîp ir minne \textbf{enpfant};\\ 
 & dô wart er vürste in Brabant.\\ 
 & diu hôchzît rîchelîch ergienc.\\ 
 & manic hêrre von sîner hende enpfienc\\ 
5 & \textbf{grôz} lêhen, \textbf{daz si} solden hân.\\ 
 & guot rihtære wart der selbe man.\\ 
 & er tet ouch dicke rîterschaft,\\ 
 & daz er den brîs \textbf{behielt} mit kraft.\\ 
 & si gewunnen \textbf{ensament} schœniu kint.\\ 
10 & \textbf{vil liute} in Brabant noch sint,\\ 
 & die wol wizzen von in beiden,\\ 
 & ir enpfâhen, sîn \textbf{von} dan scheiden,\\ 
 & \textbf{daz} in ir vrâge \textbf{dâ} \textbf{vertreip},\\ 
 & unde wie lange er dâ beleip.\\ 
15 & er schiet \textbf{doch} ungerne dan.\\ 
 & \textbf{dô} brâhte \textbf{im} aber sîn \textbf{vriunt, der} swan,\\ 
 & ein kleine \textbf{gevüege} seitiez.\\ 
 & sînes kleinœdes er dâ liez\\ 
 & \textbf{ein horn, ein swert}, ein vingerlîn.\\ 
20 & hin \textbf{vuor} Loherangrin.\\ 
 & welle wir dem mære rehte tuon,\\ 
 & sô was \textbf{ez} Parzivals sun.\\ 
 & der vuor \textbf{sît} wazzer unde wege\\ 
 & wider in des Grâles pflege.\\ 
25 & durch \textbf{waz} verlôs daz guote wîp\\ 
 & werdes \textbf{mannes} minniclîchen lîp?\\ 
 & er \textbf{hete sis} \textbf{gewarnet} ê,\\ 
 & dô er \textbf{vür si} gienc von dem sê.\\ 
 & hie \textbf{sol} Erec nû sprechen,\\ 
30 & der kunde mit rede \textbf{si} rechen.\\ 
\end{tabular}
\scriptsize
\line(1,0){75} \newline
G I L Z \newline
\line(1,0){75} \newline
\textbf{1} \textit{Initiale} G L Z  \textbf{15} \textit{Initiale} I  \newline
\line(1,0){75} \newline
\textbf{2} dô] Da Z  $\cdot$ Brabant] prabant Z \textbf{4} enpfienc] epfie L \textbf{6} rihtære] ritter L \textbf{7} er] daz er I \textbf{8} daz] \sout{eͤ} daz I \textbf{9} ensament] mit samt L \textbf{10} vil lute noch in brabant sint I  $\cdot$ Brabant] prabant Z  $\cdot$ sint] sin L \textbf{12} sîn von dan scheiden] sin danne scheiden L si dan gescheiden Z \textbf{13} ir] \textit{om.} L Z \textbf{16} dô] Nv L Z  $\cdot$ brâhte] sprach I  $\cdot$ im] in I L \textbf{17} gevüege] Gevugez I (L) \textbf{20} Loherangrin] Loheragrin I joherangrin L Lohagrin Z \textbf{21} dem] der L \textbf{22} ez] er L  $\cdot$ Parzivals] parcifals G L Z parzifals I \textbf{23} wege] wegen I \textbf{24} wider] Vnz wider L (Z)  $\cdot$ in] an L  $\cdot$ pflege] phle:en I \textbf{27} hete] \textit{om.} I  $\cdot$ sis] si I siz doch L (Z) \textbf{28} dô] Da Z  $\cdot$ si gienc] do Gie si I \textbf{29} sol] solt Z  $\cdot$ Erec] erech G (L) (Z) ereche I  $\cdot$ nû] \textit{om.} L \newline
\end{minipage}
\hspace{0.5cm}
\begin{minipage}[t]{0.5\linewidth}
\small
\begin{center}*T
\end{center}
\begin{tabular}{rl}
 & \textbf{die} naht sîn lîp ir minne \textbf{ervant};\\ 
 & dô wart er vürste in Brabant.\\ 
 & diu hôchgezît rîlîche ergienc.\\ 
 & manec hêrre von sîner hant enpfienc\\ 
5 & \textbf{grôziu} lêhen, \textbf{daz si} solten hân.\\ 
 & guot rihter wart der selbe man.\\ 
 & er tet ouch dicke rîterschaft,\\ 
 & daz er den prîs \textbf{behielte} mit kraft.\\ 
 & si gewunnen \textbf{samen} schœniu kint.\\ 
10 & \textbf{der geslehte} in Brabant noch sint,\\ 
 & die wol wizzen von in beiden,\\ 
 & i\textit{r} enpfâhen, \textbf{sîn komen}, sîn danscheiden,\\ 
 & \textbf{wan} in ir vrâge \textbf{dannen} \textbf{treip},\\ 
 & und wie lange er dâ bleip.\\ 
15 & er schiet ungerne dan.\\ 
 & \textbf{dô} brâht \textbf{in} aber sîn \textbf{wîzer} swan\\ 
 & \multicolumn{1}{l}{ - - - }\\ 
 & \multicolumn{1}{l}{ - - - }\\ 
 & \textbf{ein horn, ein swert}, ein vingerlîn.\\ 
20 & hin \textbf{schiet dô} Lohrangrin.\\ 
 & wellen wir dem mære rehte tuon,\\ 
 & \textit{s}ô was \textbf{er} Parcifals sun.\\ 
 & der v\textit{u}or \textbf{sît} wazzer und wege\\ 
 & wider in des Grâles pflege.\\ 
25 & durch \textbf{daz} verlôs daz guote wîp\\ 
 & werdes \textbf{mannes} minneclîchen lîp.\\ 
 & er \textbf{hete si es} \textbf{doch} \textbf{gewarnet} ê,\\ 
 & dô er \textbf{vür \textit{si}} gienc von dem sê.\\ 
 & hie \textbf{sol} Erec nû sprechen,\\ 
30 & der kunde mit rede \textbf{si} rechen.\\ 
\end{tabular}
\scriptsize
\line(1,0){75} \newline
U Q R \newline
\line(1,0){75} \newline
\textbf{1} \textit{Initiale} R  \textbf{25} \textit{Initiale} R  \newline
\line(1,0){75} \newline
\textbf{1} \textit{Die Verse 821.21-826.30 fehlen} Q   $\cdot$ ervant] empfieng R \textbf{2} Brabant] brabrant R \textbf{3} hôchgezît] hochzitt R \textbf{5} grôziu] Gros R \textbf{6} rihter] riter U \textbf{8} behielte] behielt R \textbf{9} samen] ze sament R  $\cdot$ schœniu] Schoͯne R \textbf{10} der geslehte] Vil lútte R  $\cdot$ Brabant] [*rabant]: brabant U \textbf{11} wizzen] wisen R \textbf{12} ir] Je U  $\cdot$ sîn komen] \textit{om.} R \textbf{13} Das In frage do vertreb R \textbf{15} schiet] schied doch R \textbf{16} Nun bracht im aber sin frund der schwan R \textbf{17} \textit{Die Verse 826.17-18 fehlen} U   $\cdot$ Ein cleines gefuͯgtes seities R \textbf{18} Sins cleinondes er da lies R \textbf{20} hin schiet dô] Hin fᵫr R  $\cdot$ Lohrangrin] Lorangrin U lohgragrin R \textbf{22} sô] Do U  $\cdot$ er] es R  $\cdot$ Parcifals] Parzifals U parczifals R \textbf{23} vuor] vor U  $\cdot$ wege] wage R \textbf{24} wider] Vncz wider R \textbf{25} durch daz] Durch was R  $\cdot$ guote] werde R \textbf{28} si] \textit{om.} U \textbf{29} Erec] Erich R \newline
\end{minipage}
\end{table}
\end{document}
