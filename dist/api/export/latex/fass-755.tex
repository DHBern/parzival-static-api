\documentclass[8pt,a4paper,notitlepage]{article}
\usepackage{fullpage}
\usepackage{ulem}
\usepackage{xltxtra}
\usepackage{datetime}
\renewcommand{\dateseparator}{.}
\dmyyyydate
\usepackage{fancyhdr}
\usepackage{ifthen}
\pagestyle{fancy}
\fancyhf{}
\renewcommand{\headrulewidth}{0pt}
\fancyfoot[L]{\ifthenelse{\value{page}=1}{\today, \currenttime{} Uhr}{}}
\begin{document}
\begin{table}[ht]
\begin{minipage}[t]{0.5\linewidth}
\small
\begin{center}*D
\end{center}
\begin{tabular}{rl}
\textbf{755} & \textbf{Dô} was bî dem selben tage\\ 
 & über al daz her gemeiniu klage,\\ 
 & daz Parzival, der werde man,\\ 
 & \textbf{von in was} gescheiden dan.\\ 
5 & Artus mit râte sich bewac,\\ 
 & daz er unz an den \textbf{ahten} tac\\ 
 & Parzivals dâ wolde bîten\\ 
 & unt von der stat niht rîten.\\ 
 & Gramoflanzes her was \textbf{ouch} komen.\\ 
10 & dem was manec wîter rinc genomen\\ 
 & mit \textbf{zelten} wol gezieret.\\ 
 & dâ was geloschieret\\ 
 & den stolzen werden liuten.\\ 
 & man m\textit{ö}htez de\textit{n} vier briuten\\ 
15 & \textbf{niht baz erbieten} mit vreude siten.\\ 
 & von Schastel Marvale geriten\\ 
 & kom ein man zer selben zît.\\ 
 & der seite alsus, ez wære ein strît\\ 
 & ûfem warthûs in der sûl gesehen.\\ 
20 & swaz ie \textbf{mit} swerten \textbf{wære} geschehen,\\ 
 & "daz ist gein disem strîte ein niht".\\ 
 & vor Gawane er des mæres giht,\\ 
 & dâ der bî Artuse saz.\\ 
 & manec \textbf{rîter} dâ mit rede maz,\\ 
25 & von wem der strît \textbf{dâ} wære getân.\\ 
 & Artus, der künec, sprach dô sân:\\ 
 & "\begin{large}D\end{large}en strît ich einhalp wol weiz.\\ 
 & in \textbf{streit} mîn neve von Kanvoleiz,\\ 
 & der von uns schiet hiute vruo."\\ 
30 & dô riten ouch \textbf{dise} zwêne zuo.\\ 
\end{tabular}
\scriptsize
\line(1,0){75} \newline
D Fr12 \newline
\line(1,0){75} \newline
\textbf{1} \textit{Majuskel} D  \textbf{27} \textit{Initiale} D  \newline
\line(1,0){75} \newline
\textbf{3} Parzival] Parcifal D Fr12 \textbf{5} bewac] des bewac Fr12 \textbf{6} an] \textit{om.} Fr12 \textbf{7} Parzivals] Parcifals D Fr12 \textbf{9} Gramoflanzes] Gramoflanzs D Gramanzes Fr12 \textbf{11} zelten] gezelten Fr12 \textbf{14} möhtez den] mohtez der D \textbf{15} vreude] vrouden Fr12 \textbf{16} Schastel Marvale] Scastel marvâle D Scastelmarvale Fr12 \textbf{18} alsus] \textit{om.} Fr12 \newline
\end{minipage}
\hspace{0.5cm}
\begin{minipage}[t]{0.5\linewidth}
\small
\begin{center}*m
\end{center}
\begin{tabular}{rl}
 & \textbf{ouch} \textit{was} bî dem selben tage\\ 
 & über al daz her gemeiniu klage,\\ 
 & daz Parcifal, der werde man,\\ 
 & \textbf{von in was sus} gescheiden dan.\\ 
5 & \begin{large}A\end{large}rtus mit râte sich bewac,\\ 
 & daz er unz an den \textbf{ahten} tac\\ 
 & Parcifals d\textit{â} wolte bîten\\ 
 & und von der stat niht rîten.\\ 
 & Gramolanzes her was komen.\\ 
10 & dem was manic wîter rinc genomen\\ 
 & mit \textbf{zelten} wol gezieret.\\ 
 & d\textit{â} was geloschieret\\ 
 & den stolzen werden liuten.\\ 
 & man m\textit{ö}hte ez den vier briuten\\ 
15 & \textbf{niht baz erbieten} mit vröuden siten.\\ 
 & vo\textit{n} Schahtel Ma\textit{r}v\textit{e}ile geriten\\ 
 & kam ein man zer selben zît.\\ 
 & der seite alsus, ez wære ein strît\\ 
 & ûf dem warthûs in der sûl gesehen.\\ 
20 & waz ie \textbf{mit} swerten \textbf{was} geschehen,\\ 
 & "daz ist gegen \textit{disem} strît ein niht".\\ 
 & vor Gawan er des mæres giht,\\ 
 & d\textit{â} der bî Artuse saz.\\ 
 & manic \textbf{man} d\textit{â} mit red\textit{e} \textit{m}az,\\ 
25 & von wem der strît \textbf{d\textit{â}} wær getân.\\ 
 & Artus, der künic, sprach dô sân:\\ 
 & "den strît ich einhalp wol weiz.\\ 
 & in \textbf{tet} mîn neve \textit{von} Kanvoleiz,\\ 
 & der von uns schiet hiute vruo."\\ 
30 & dô riten ouch \textbf{die} zwêne zuo.\\ 
\end{tabular}
\scriptsize
\line(1,0){75} \newline
m n o V V' \newline
\line(1,0){75} \newline
\textbf{5} \textit{Initiale} n V  \textbf{9} \textit{Initiale} V'  \newline
\line(1,0){75} \newline
\textbf{1} was] \textit{om.} m  $\cdot$ bî] an V' \textbf{2} al] ale n  $\cdot$ her] mer V' \textbf{3} daz] Do V'  $\cdot$ Parcifal] [*]: parzefal V parzifal V' \textbf{4} [*]: Von in was sus gescheiden dan V \textbf{6} ahten] achsten o \textbf{7} Parcifals] Parzefals V Parzifals V'  $\cdot$ dâ] do m n o V V' \textbf{9} Gramolanzes] Gramolantz m n Gramolancz o Gramaflanz V Gramaflantz V'  $\cdot$ was] was ouch n (o) (V) (V') \textbf{10} Deme waz manig [*]: witer ring [*enomen]: genomen V \textbf{11} zelten] zette: o gezelten V V' \textbf{12} dâ] Das m Do n o V V'  $\cdot$ geloschieret] beloschieret o \textbf{13} den stolzen] Men moͤhte V (V') \textbf{14} Als ich hie wil betúten V (V')  $\cdot$ möhte] mohtte m (n) \textbf{15} erbieten] erbiten V'  $\cdot$ vröuden] \textit{om.} n \textbf{16} \textit{Versdoppelung 755.16-756.20 (²o) nach 756.20; Lesarten der vorausgehenden Verse mit ¹o bezeichnet} o   $\cdot$ Von] Vo m Wan \textsuperscript{1}\hspace{-1.3mm} o  $\cdot$ Schahtel Marveile] schahttel mauile m schathel maueile n schattel marueile \textsuperscript{1}\hspace{-1.3mm} o schathel marueile \textsuperscript{2}\hspace{-1.3mm} o Schatel marveile V (V') \textbf{18} ez] er \textsuperscript{1}\hspace{-1.3mm} o \textbf{19} ûf] Vffeme V' \textbf{20} swerten] swerte \textsuperscript{2}\hspace{-1.3mm} o \textbf{21} daz] Do \textsuperscript{2}\hspace{-1.3mm} o  $\cdot$ disem] \textit{om.} m \textbf{22} Gawan] Gawane V (V') \textbf{23} dâ] Do m n o V V' \textbf{24} dâ] do m n o V V'  $\cdot$ rede maz] rede was vnd mas m \textbf{25} dâ] do m n \textsuperscript{1}\hspace{-1.3mm} o V V' \textit{om.} \textsuperscript{2}\hspace{-1.3mm} o \textbf{28} in] Jch wene in V V'  $\cdot$ von] \textit{om.} m  $\cdot$ Kanvoleiz] canvoleis m V V' kaufaleis n kanfoleis \textsuperscript{1}\hspace{-1.3mm} o kanfoleisz \textsuperscript{2}\hspace{-1.3mm} o \textbf{29} hiute] hútte morgen V (V') \textbf{30} ouch] \textit{om.} \textsuperscript{1}\hspace{-1.3mm} o  $\cdot$ die] dise n \textsuperscript{1}\hspace{-1.3mm} o \textsuperscript{2}\hspace{-1.3mm} o V \newline
\end{minipage}
\end{table}
\newpage
\begin{table}[ht]
\begin{minipage}[t]{0.5\linewidth}
\small
\begin{center}*G
\end{center}
\begin{tabular}{rl}
 & \textbf{\begin{large}D\end{large}ô} was bî dem selben tage\\ 
 & über al daz her gemeiniu klage,\\ 
 & daz Parzival, der werde man,\\ 
 & \textbf{sus was von in} gescheiden dan.\\ 
5 & Artus mit râte sich bewac,\\ 
 & daz er unze an den \textbf{vierden} tac\\ 
 & Parzivals dâ wolde bîten\\ 
 & unde von der stat niht rîten.\\ 
 & Gramoflanzes her was komen.\\ 
10 & dem was manic wîte rinc genomen\\ 
 & mit \textbf{gezelten} wol geziert.\\ 
 & dâ was geloisiert\\ 
 & den stolzen werden liuten.\\ 
 & man möhtez den vier briuten\\ 
15 & \textbf{baz erbieten niht} mit vröuden siten.\\ 
 & von Tschastel marveile geriten\\ 
 & kom ein man zer selben zît.\\ 
 & der seite alsus, ez wære ein strît\\ 
 & ûf dem warthûse in der sûl gesehen.\\ 
20 & swaz ie \textbf{von} swerten \textbf{wære} geschehen,\\ 
 & "daz ist gein disem strîte ein niht".\\ 
 & vor Gawan er des mæres giht,\\ 
 & dâ der bî Artuse saz.\\ 
 & manic \textbf{rîter} dâ mit rede maz,\\ 
25 & von wem der strît \textbf{dâ} wære getân.\\ 
 & Artus, der künic, sprach dô sân:\\ 
 & "den strît ich einhalp wol weiz.\\ 
 & in \textbf{streit} mîn neve von Kanvoleiz,\\ 
 & der von uns schiet hiute vruo."\\ 
30 & dô riten ouch \textbf{dise} zwêne zuo.\\ 
\end{tabular}
\scriptsize
\line(1,0){75} \newline
G I L M Z Fr43 Fr48 \newline
\line(1,0){75} \newline
\textbf{1} \textit{Initiale} G L Z  \textbf{17} \textit{Initiale} I  \textbf{29} \textit{Initiale} Z  \newline
\line(1,0){75} \newline
\textbf{1} Dô] Ez L Nu M Da Z \textbf{3} Parzival] parcifal G Z Parzifal I L M \textbf{4} von in gescheiden] gischeiden von yn so M \textbf{6} unze] bisz M \textbf{7} Parzivals] parcifals G (Z) Parzifals I L Parzifal M Parz::: Fr43  $\cdot$ bîten] betin M \textbf{9} Gramoflanzes] Gramoflanzis M Gramoflantz Z Gramonflanzes Fr43  $\cdot$ her was] her waz ouch L (M) (Fr43) was ouch here Z \textbf{10} dem] den I  $\cdot$ manic wîte rinc] ein witer rinc I manig rinch wit L manic witer rinc Fr43  $\cdot$ genomen] benomen G \textbf{12} dâ] do I \textbf{13} liuten] liute Fr43 \textbf{14} möhtez] moht ez I (L) (M) (Z) (Fr43) (Fr48) \textbf{15} baz erbieten niht] nih baz erbieten I (L) Das erbiten nicht M  $\cdot$ vröuden] frevde Z \textbf{16} Tschastel marveile] shatel morueile I kastel Marveile L schastel marfeile M tschahtel Marveil Z schastel Marueile Fr43 Schahtel Marueil Fr48 \textbf{18} der] er I  $\cdot$ seite] seit I (L) (Z) (Fr48) \textbf{19} in] in in I  $\cdot$ gesehen] ersehen Z \textbf{20} swaz] waz L (M)  $\cdot$ von] mit L M Z Fr43 Fr48  $\cdot$ swerten] swerte M  $\cdot$ wære] was I ist L \textbf{22} vor] Von M  $\cdot$ Gawan] gawane L M Fr43  $\cdot$ des mæres] die mere L \textbf{23} dâ] Do Z  $\cdot$ Artuse] Artuͯse L artus Z (Fr48) \textbf{24} dâ] do Fr48  $\cdot$ maz] waz L [sasz]: masz M \textbf{25} wem] dem L Fr48  $\cdot$ dâ] \textit{om.} L M do Fr48 \textbf{26} Do sprach der kvnig Artus san L  $\cdot$ dô] da M Fr43 \textbf{27} weiz] wei* Fr48 \textbf{28} mîn] [mit]: myn M  $\cdot$ Kanvoleiz] Ganfoleiz I kamvoleiz L Fr48 kanoleisz M kamfoleiz Z kamuol::: Fr43 \textbf{29} hiute] hvten morgen L (M) (Fr43) \textbf{30} dô] Da M Z Fr43  $\cdot$ dise] die I \newline
\end{minipage}
\hspace{0.5cm}
\begin{minipage}[t]{0.5\linewidth}
\small
\begin{center}*T
\end{center}
\begin{tabular}{rl}
 & \textbf{\begin{large}D\end{large}ô} was bî dem selben tage\\ 
 & über al daz her gemeiniu klage,\\ 
 & daz Parcifal, der werde man,\\ 
 & \textbf{sus was von in} gescheiden dan.\\ 
5 & Artus mit râte sich bewac,\\ 
 & daz er unz an den \textbf{vierden} tac\\ 
 & Parcifals d\textit{â} wolte bîten\\ 
 & und von der stat niht rîten.\\ 
 & Gramoflanzes her was \textbf{ouch} komen.\\ 
10 & dem was manec w\textit{î}ter rinc genomen\\ 
 & mit \textbf{gezelten} wol gezieret.\\ 
 & d\textit{â} was geloschieret\\ 
 & den stolzen werden liuten.\\ 
 & man m\textit{ö}ht ez den vier briuten\\ 
15 & \textbf{niht baz erbieten} mit vreude siten.\\ 
 & von Tschahtel Marvele geriten\\ 
 & kom ein man zuo der selben zît.\\ 
 & der sagete alsus, ez wære ein strît\\ 
 & ûf dem warthûse in der s\textit{û}l gesehen.\\ 
20 & waz ie \textbf{mit} swerten \textbf{wære} geschehen,\\ 
 & "daz ist gein disem strîte ein niht".\\ 
 & vor Gawane er der mære giht,\\ 
 & d\textit{â} der bî Artuse saz.\\ 
 & manec \textbf{rîter} d\textit{â} mit rede maz,\\ 
25 & von wem der strît wære getân.\\ 
 & Artus, der künec, sprach dô sân:\\ 
 & "den strît ich einhalp wol weiz.\\ 
 & in \textbf{streit} mîn neve von Kanvoleiz,\\ 
 & der von uns schiet hiute \textbf{morgen} vruo."\\ 
30 & dô riten ouch \textbf{dise} zwêne zuo.\\ 
\end{tabular}
\scriptsize
\line(1,0){75} \newline
U W Q R \newline
\line(1,0){75} \newline
\textbf{1} \textit{Initiale} U W  \newline
\line(1,0){75} \newline
\textbf{1} dem] den R  $\cdot$ tage] tagen R \textbf{2} klage] klagen Q \textbf{3} Parcifal] herr partzifal W partzifal Q parczifal R  $\cdot$ werde] Junge R \textbf{4} Als was geschiden von in dann Q \textbf{6} daz] Do W  $\cdot$ unz] mit U  $\cdot$ den] der W \textbf{7} Parcifals] Partzifals W Q Parczifals R  $\cdot$ dâ] do U W Q \textbf{9} Gramoflanzes] Gramoflantzes W Gramoflanszes Q Gramaflanczes R \textbf{10} wîter rinc] weterrinc U \textbf{11} wol] schon W \textbf{12} dâ] Do U W R Das Q \textbf{14} möht] mocht U Q macht R  $\cdot$ den] dann Q \textbf{15} vreude] froͤden W (R) \textbf{16} Tschahtel Marvele] schatermarveile U kastel marfeile W schachtel marueile Q schatel marveile R \textbf{18} sagete] sagt W  $\cdot$ alsus] als Q  $\cdot$ ez] er R \textbf{19} in] vff Q  $\cdot$ sûl] sol U  $\cdot$ gesehen] geschehn Q (R) \textbf{20} mit] von W  $\cdot$ geschehen] gesehen W \textbf{21} niht] wicht R \textbf{22} Gawane] gawan Q gewan R  $\cdot$ er] der Q  $\cdot$ der] des W Q R  $\cdot$ mære] meres W Q \textbf{23} dâ der] Do der U Der do W  $\cdot$ Artuse] artus R  $\cdot$ saz] sach vnd sas R \textbf{24} dâ] do U Q R  $\cdot$ rede maz] mere sas R \textbf{25} von] Vom R  $\cdot$ wære] do were W (Q) da were R \textbf{27} ich] \textit{om.} R  $\cdot$ wol] von Q \textbf{28} neve] neuen R  $\cdot$ von] \textit{om.} W  $\cdot$ Kanvoleiz] kanuoleiß W kanvoleisz Q kanvoleis R \textbf{29} morgen] enmorgen R \textbf{30} ouch] \textit{om.} R \newline
\end{minipage}
\end{table}
\end{document}
