\documentclass[8pt,a4paper,notitlepage]{article}
\usepackage{fullpage}
\usepackage{ulem}
\usepackage{xltxtra}
\usepackage{datetime}
\renewcommand{\dateseparator}{.}
\dmyyyydate
\usepackage{fancyhdr}
\usepackage{ifthen}
\pagestyle{fancy}
\fancyhf{}
\renewcommand{\headrulewidth}{0pt}
\fancyfoot[L]{\ifthenelse{\value{page}=1}{\today, \currenttime{} Uhr}{}}
\begin{document}
\begin{table}[ht]
\begin{minipage}[t]{0.5\linewidth}
\small
\begin{center}*D
\end{center}
\begin{tabular}{rl}
\textbf{796} & von tôde \textbf{lebendec} \textbf{dan hiez} gên\\ 
 & unt \textbf{der} Lazarum \textbf{bat} ûf stên,\\ 
 & der selbe half, daz Anfortas\\ 
 & wart gesunt unt wol genas.\\ 
5 & \textbf{Swaz der Franzoys heizet flôrî},\\ 
 & der glast kom sînem velle bî.\\ 
 & Parzival\textit{s} schœne was \textbf{nû} ein wint\\ 
 & \textbf{unt} \textbf{Absalon}, Davides kint,\\ 
 & von Ascalun Vergulaht\\ 
10 & unt \textbf{al}, \textbf{den schœne} \textbf{was} geslaht,\\ 
 & \textbf{unt} \textbf{des} man Gahmurete jach,\\ 
 & dô man in în zogen sach\\ 
 & ze Kanvoleiz sô \textbf{wünneclîch},\\ 
 & ir decheines schœne was der gelîch,\\ 
15 & die Anfortas ûz siecheit truoc.\\ 
 & got noch künste kan genuoc.\\ 
 & Dâ ergienc \textbf{dô} dehein ander wal,\\ 
 & wan \textbf{die} diu schrift ame Grâl\\ 
 & hete ze hêrren in benant.\\ 
20 & Parzival wart schiere \textbf{bekant}\\ 
 & ze künege unt ze hêrren dâ.\\ 
 & ich wæne, iemen anderswâ\\ 
 & vünde zwêne \textbf{als} rîche man\\ 
 & \multicolumn{1}{l}{ - - - }\\ 
25 & als Parzival unt Feirefiz.\\ 
 & man bôt \textbf{vil} dienstlîchen vlîz\\ 
 & dem wirte unt \textbf{sîme gaste}.\\ 
 & \textbf{Ine weiz, wie manege raste}\\ 
 & \multicolumn{1}{l}{ - - - }\\ 
 & \multicolumn{1}{l}{ - - - }\\ 
 & \multicolumn{1}{l}{ - - - }\\ 
 & \multicolumn{1}{l}{ - - - }\\ 
 & \multicolumn{1}{l}{ - - - }\\ 
 & Condwiramurs \textbf{dô was} geriten\\ 
30 & \textbf{gein} Munsalvæsche mit vreude siten.\\ 
\end{tabular}
\scriptsize
\line(1,0){75} \newline
D \newline
\line(1,0){75} \newline
\textbf{5} \textit{Majuskel} D  \textbf{17} \textit{Majuskel} D  \textbf{28} \textit{Majuskel} D  \newline
\line(1,0){75} \newline
\textbf{7} Parzivals] Parcifal D \textbf{11} Gahmurete] Gahmvrete D \textbf{12} man in] mann D \textbf{20} Parzival] Parcifal D \textbf{24} \textit{Vers 796.24 fehlt} D  \textbf{25} Parzival] Parcifal D \textbf{30} Munsalvæsche] Mvnsalvæsce D \newline
\end{minipage}
\hspace{0.5cm}
\begin{minipage}[t]{0.5\linewidth}
\small
\begin{center}*m
\end{center}
\begin{tabular}{rl}
 & von tôde \textbf{lebendic} \textbf{dan hie} gên\\ 
 & und \textbf{\textit{d}er} Laz\textit{a}r\textit{um} \textbf{bat} ûf stên,\\ 
 & der selbe half, daz Anfortas\\ 
 & wart gesunt und wol genas.\\ 
5 & \textbf{waz der Franzois heizet flôrî},\\ 
 & der glast kam sînem velle bî.\\ 
 & Parcifals schœne was \textbf{nû} ein wint\\ 
 & \textbf{und} \textbf{Absolones}, Davides kint,\\ 
 & vo\textit{n} Ascal\textit{un} Vergulaht\\ 
10 & und \textbf{al} \textbf{der schône} \textbf{wart} geslaht,\\ 
 & \textbf{und} \textbf{daz} \textit{m}an Gahmuret\textit{e} jach,\\ 
 & dô man in în zogen sach\\ 
 & zuo Kanvoleiz sô \textbf{wünneclîch},\\ 
 & ir dekeines schœne was der gelîch,\\ 
15 & die Anfortas ûz siecheit truoc.\\ 
 & got noch künste kan genuoc.\\ 
 & d\textit{â} ergienc \textbf{dô} kein ander wal,\\ 
 & wan diu schrift an dem Grâl\\ 
 & het zuo hêrren in benant.\\ 
20 & Parcifal wart schier \textbf{bekant}\\ 
 & zuo künige und zuo hêrren dâ.\\ 
 & ich wæne, ieman anderswâ\\ 
 & vünde zwêne \textbf{alsô} rîche man,\\ 
 & ob ich rîcheit \textbf{gebrüefen} kan,\\ 
25 & alsô Parcifal und Ferefiz.\\ 
 & man bôt \textbf{vil} dienstlîchen vlîz\\ 
 & dem wirt und \textbf{dem gaste sîn}.\\ 
 & \textbf{daz ist der gloube mîn}.\\ 
 & \begin{large}N\end{large}û \textit{si} alsô sint gesezzen\\ 
 & und ir sorge hânt vergezzen,\\ 
 & dô seite man in mære,\\ 
 & diu wâren vröudebære,\\ 
 & diu wâren vröidebære,\\ 
 & \textbf{daz} Condwieramurs \textbf{kam} geriten\\ 
30 & \textbf{gegen} Muntsalvasche mit vröuden siten.\\ 
\end{tabular}
\scriptsize
\line(1,0){75} \newline
m n o V V' W \newline
\line(1,0){75} \newline
\textbf{28} \textit{Illustration mit Überschrift:} Also parcifal des groles herre wart vnd anfortas erlost mit siner froge die do geschah m (n) (o)   $\cdot$ \textit{Initiale} m n o V V' W  \newline
\line(1,0){75} \newline
\textbf{1} lebendic] lebende V'  $\cdot$ dan hie] hieß dann W \textbf{2} und der] Vnder m  $\cdot$ Lazarum] lazzer in m [fra]: lazaruͯm o lasarun V lazarun W  $\cdot$ bat] hiez V' thet W \textbf{3} half] helffe n  $\cdot$ Anfortas] anefortas o \textbf{5} \textit{Die Verse 796.5-16 fehlen} V'   $\cdot$ waz] Swaz V  $\cdot$ Franzois] frantzois m n franczois o frantzoys W \textbf{7} Parcifals] Parzefals V Partzifals W  $\cdot$ nû] \textit{om.} W \textbf{8} Absolones] absolons m n o [absolo*]: absolonez V absolon W  $\cdot$ Davides] dauides m n o V W \textbf{9} von] Vo m  $\cdot$ Ascalun] [*]: ascalẏm m ascelun n aschalun V astalune W  $\cdot$ Vergulaht] vergelacht n verglacht o [vergula*]: vergulaht V vergulacht W \textbf{10} al] do aller n alle o V W  $\cdot$ der] den V die der W  $\cdot$ wart] was n (o) (V) (W) \textbf{11} daz] des V  $\cdot$ man] gawan m  $\cdot$ Gahmurete] gamurettette m gamúrete n jahmueretez o Gamerette V gamurete W \textbf{12} în zogen] zeigen n zoigen o \textbf{13} Kanvoleiz] kanfoleis m V convoleis n (o) kanuoleiß W  $\cdot$ sô] \textit{om.} W \textbf{14} dekeines] do keines n (W)  $\cdot$ der] ir W \textbf{17} dâ] Do m n o V V' W  $\cdot$ dô] da o \textbf{18} schrift] geschrifft n \textbf{19} het] Hat W \textbf{20} Parcifal] Parzefal V Parzifal V' Partzifal W \textbf{21} dâ] do n V V' W \textbf{23} zwêne alsô] drie alse V (V') also zwen W \textbf{24} gebrüefen] prufen V' \textbf{25} Parcifal] parzefal artus V artus parzifal V' partzifal W  $\cdot$ Ferefiz] ferefis m o ferrefis n ferevis V fereuis V' ferafiß W \textbf{26} vlîz] [pris]: flis o \textbf{27} und] vnd auch W  $\cdot$ dem gaste] den gesten V V' \textbf{28} ist] ist auch W  $\cdot$ der] der der V' \textbf{28} Nû si alsô] NV also m (o) ALso sú nuͯ n NVn als sy W  $\cdot$ sint] sein W \textbf{28} sorge] sorgen V V' \textbf{28} in] im W \textbf{28} diu] Do fúr o \textbf{28} diu] Do fúr o \textbf{29} daz] Wie W  $\cdot$ Condwieramurs] condwir amurs m conduwier amirs n Cundwir amurs o kvndewiramurs V V' (W) \textbf{30} Muntsalvasche] muntsaluasce m n o Mvntschalfasche V munschalfasche V' montsaluatsch W  $\cdot$ vröuden] froide o \newline
\end{minipage}
\end{table}
\newpage
\begin{table}[ht]
\begin{minipage}[t]{0.5\linewidth}
\small
\begin{center}*G
\end{center}
\begin{tabular}{rl}
 & von tôde \textbf{lebende} \textbf{hiez hine} gên\\ 
 & unde Lazarum \textbf{hiez} ûf stên,\\ 
 & der selbe half, daz Anfortas\\ 
 & wart gesunt unde wol genas.\\ 
5 & \textbf{er was vor ungemache vrî},\\ 
 & der glast kom sînem velle bî.\\ 
 & Parzivals schœne was \textbf{nû} ein wint,\\ 
 & \textbf{Apsolon}, Davides kint,\\ 
 & \textbf{unde} von Aschalun Vergulaht\\ 
10 & unde \textbf{allen}, \textbf{den schœne} \textbf{was} geslaht,\\ 
 & \textbf{oder} \textbf{des} man Gahmuret jach,\\ 
 & dô man in în zogen sach\\ 
 & ze Kanvoleis sô \textbf{wünniclîch},\\ 
 & ir deheines schœne was der gelîch,\\ 
15 & die Anfortas ûz siecheit truoc.\\ 
 & got noch künste kan genuoc.\\ 
 & dâ ergienc \textbf{dô} dehein ander wal,\\ 
 & wan \textbf{die} diu schrift anme Grâl\\ 
 & het ze hêrren in benant.\\ 
20 & Parzival wart schiere \textbf{erkant}\\ 
 & ze künige unde ze hêrren dâ.\\ 
 & ich wæne, iemen anderswâ\\ 
 & vünde zwêne \textbf{als} rîche man,\\ 
 & obe ich rîcheit \textbf{prüeven} kan,\\ 
25 & als Parzival unde Feirafiz.\\ 
 & man bôt \textbf{in} dienstlîchen vlîz,\\ 
 & dem wirte unde \textbf{sînem gaste}.\\ 
 & \textbf{ichne weiz, wie manige raste}\\ 
 & \multicolumn{1}{l}{ - - - }\\ 
 & \multicolumn{1}{l}{ - - - }\\ 
 & \multicolumn{1}{l}{ - - - }\\ 
 & \multicolumn{1}{l}{ - - - }\\ 
 & \multicolumn{1}{l}{ - - - }\\ 
 & Condwiramurs \textbf{dô was} geriten\\ 
30 & \textbf{ze} Muntsalfatsche mit vröuden siten.\\ 
\end{tabular}
\scriptsize
\line(1,0){75} \newline
G I L M Z \newline
\line(1,0){75} \newline
\textbf{1} \textit{Initiale} L  \textbf{5} \textit{Initiale} Z  \textbf{7} \textit{Initiale} I  \newline
\line(1,0){75} \newline
\textbf{1} von] von dem I (L) (Z)  $\cdot$ lebende] lebendic I (M) Z  $\cdot$ hiez hine] hiez I dan hiz L (M) (Z) \textbf{2} unde] von dem Tode I  $\cdot$ Lazarum] [Laz*rvm]: Lazarvm L lasarum M  $\cdot$ hiez ûf] hiez vf vf I bat vf L uff hiesz M vf bat Z \textbf{3} Anfortas] Amfortas L \textbf{4} unde wol] wol vnd Z \textbf{5} \textit{Die Verse 796.5-6 fehlen} M   $\cdot$ Gotes craft wart an im shin I  $\cdot$ Waz (Daz Z ) der franzois (frantzois Z ) heiszet flori L (Z) \textbf{7} Parzivals] Parzifals I M Parcifalz L Parcifals Z \textbf{8} Apsolon] Absolon I L Z Absalon M  $\cdot$ Davides] dauides I M Z \textbf{9} \textit{Die Verse 796.9-12 fehlen} L   $\cdot$ Aschalun] ascalun M (Z)  $\cdot$ Vergulaht] verGulat I virgulacht M \textbf{10} was] ist Z \textbf{11} oder] Olde G  $\cdot$ Gahmuret] Gahmv̂ret G Gahmureten I Gamurete M gamureten Z \textbf{12} dô] Da M Z \textbf{13} Waz vnnach so wvͯnneclich L  $\cdot$ Kanvoleis] kanvoleiz G Ganfoleis I kantvoleis M kamfoleis Z \textbf{14} der] in L so M \textbf{15} \textit{Die Verse 796.15-16 fehlen} L   $\cdot$ siecheit] sich M \textbf{16} kan] hat I \textbf{17} dô] \textit{om.} I da M Z \textbf{18} die diu] diu die diu I den die L \textbf{19} het] heten I  $\cdot$ benant] genant L \textbf{20} Parzival] parcifal G (Z) parzifal I (L) (M) \textbf{21} künige] konnigen M  $\cdot$ unde] vnz I \textbf{22} iemen] ninder I \textbf{23} vünde] man funde I \textbf{25} Parzival] parcifal G Z Parzifal I (L) (M)  $\cdot$ Feirafiz] ferefiz L feirefisz M feirefiz Z \textbf{26} bôt] bat M  $\cdot$ in] vil L M Z \textbf{29} Condwiramurs] koͮndwiramvrs G Conduwiramurs I Condwir Amvrs L Kundwir Amvrs M (Z)  $\cdot$ dô] da M Z \textbf{30} ze] Gein L (M) Z  $\cdot$ Muntsalfatsche] mvntsalvatsche G (L) (M) muntshaluasce I montsalvatsch Z \newline
\end{minipage}
\hspace{0.5cm}
\begin{minipage}[t]{0.5\linewidth}
\small
\begin{center}*T
\end{center}
\begin{tabular}{rl}
 & von tôde \textbf{lebendic} \textbf{hiez dan} gên\\ 
 & und Lazarum \textbf{bat} ûf stên,\\ 
 & der selbe half, daz Anfortas\\ 
 & wart gesunt und wol genas.\\ 
5 & \textbf{waz der Franzoyser heizet flôrî},\\ 
 & der glast kam sîme velle bî.\\ 
 & \begin{large}P\end{large}arcifals schœne was ein wint,\\ 
 & \textbf{Absalon}, Davides kint,\\ 
 & \textbf{und} von Ascalun Vergulaht\\ 
10 & und \textbf{alle}, \textbf{den schœne} \textbf{was} geslaht,\\ 
 & \textbf{oder} \textbf{des} man Gahmurete jach,\\ 
 & dô man in în zogen sach\\ 
 & zuo Kanvoleiz sô \textbf{minneclîch},\\ 
 & ir dekeines schœne was der glîch,\\ 
15 & die Anfortas ûz \textit{siecheit truoc}.\\ 
 & got noch künste kan genuoc.\\ 
 & dô ergienc \textbf{ouch} kein ander wal,\\ 
 & wan \textbf{den} diu schrift an dem Grâl\\ 
 & hete zuo hêrren in benant.\\ 
20 & Parcifal wart schiere \textbf{erkant}\\ 
 & zuo künege und zuo hêrren dâ.\\ 
 & ich wæne, ieman anderswâ\\ 
 & vünd\textit{e} zwêne \textbf{sô} rîche man,\\ 
 & ob ich rîcheit \textbf{prüeven} kan,\\ 
25 & als Parcifal und Ferefis.\\ 
 & man bôt \textbf{vil} dienstlîchen vlîz\\ 
 & dem wirte und \textbf{sîme gaste}.\\ 
 & \textbf{ich enweiz, wie manege raste}\\ 
 & \multicolumn{1}{l}{ - - - }\\ 
 & \multicolumn{1}{l}{ - - - }\\ 
 & \multicolumn{1}{l}{ - - - }\\ 
 & \multicolumn{1}{l}{ - - - }\\ 
 & \multicolumn{1}{l}{ - - - }\\ 
 & Kundewiramurs \textbf{was} geriten\\ 
30 & \textbf{gein} Munsalvasche mit vreuden siten.\\ 
\end{tabular}
\scriptsize
\line(1,0){75} \newline
U Q R \newline
\line(1,0){75} \newline
\textbf{7} \textit{Initiale} U  \newline
\line(1,0){75} \newline
\textbf{1} hiez dan] dann hiesz Q hies R \textbf{2} Lazarum] lazaruͦm U laszarum R  $\cdot$ bat] hiesz Q (R) \textbf{5} Franzoyser] franzoiser U franczosier R \textbf{7} Parcifals] Parzifals U Partzifals Q Parczifals R  $\cdot$ was] was nun Q \textbf{8} Absalon] Alsolon R  $\cdot$ Davides] dauides Q R \textbf{9} Ascalun] aschaluͦn U ascalún Q  $\cdot$ Vergulaht] vergolacht U vergulacht Q R \textbf{10} alle den schœne] alle die der schone Q allú schoͯne R \textbf{11} Gahmurete] Gahmuͦrete U gamúreten Q Gahmurten R \textbf{12} în zogen] zogen R \textbf{13} Kanvoleiz] kanvoleis Q R  $\cdot$ minneclîch] munnenclich Q \textbf{15} die] Do Q  $\cdot$ Anfortas] Anfortes R  $\cdot$ siecheit truoc] \textit{om.} U \textbf{17} ouch] do Q R \textbf{18} den diu schrift] di die schrifft Q dú die geschrifft R \textbf{20} Parcifal] Parzifal U Partzifal Q Parczifal R \textbf{21} dâ] do Q \textbf{22} ieman] nieman R \textbf{23} vünde] Vuͦnden U  $\cdot$ sô] als Q R \textbf{25} Parcifal] partzifal Q parczifal R  $\cdot$ und] vn R  $\cdot$ Ferefis] feirefisz Q feriefis R \textbf{28} enweiz] weis R \textbf{29} Kundewiramurs] Kuͦndewiramuͦrs U kúndwiramurs Q Kundwiramurs R  $\cdot$ was] do was Q waz do \textit{(abweichende Reklamante)} R \textbf{30} gein] Ge R  $\cdot$ Munsalvasche] muͦnsalvatsche U muntsaluasche Q nunschaualasche R  $\cdot$ mit vreuden] gen frewde Q \newline
\end{minipage}
\end{table}
\end{document}
