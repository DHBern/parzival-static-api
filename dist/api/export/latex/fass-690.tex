\documentclass[8pt,a4paper,notitlepage]{article}
\usepackage{fullpage}
\usepackage{ulem}
\usepackage{xltxtra}
\usepackage{datetime}
\renewcommand{\dateseparator}{.}
\dmyyyydate
\usepackage{fancyhdr}
\usepackage{ifthen}
\pagestyle{fancy}
\fancyhf{}
\renewcommand{\headrulewidth}{0pt}
\fancyfoot[L]{\ifthenelse{\value{page}=1}{\today, \currenttime{} Uhr}{}}
\begin{document}
\begin{table}[ht]
\begin{minipage}[t]{0.5\linewidth}
\small
\begin{center}*D
\end{center}
\begin{tabular}{rl}
\textbf{690} & \begin{large}D\end{large}û hâst dir \textbf{selben} an gesigt,\\ 
 & ob dîn herze triwen pfligt."\\ 
 & Dô disiu rede was getân,\\ 
 & dô\textbf{ne} moht \textbf{ouch} mîn hêr Gawan\\ 
5 & \textbf{vor} unkraft niht langer stên.\\ 
 & er begunde \textbf{al} \textbf{swindel\textit{n}de} gên,\\ 
 & wand im\textbf{z} houbet erschellet was;\\ 
 & \textbf{er strûchte} nider \textbf{an}z gras.\\ 
 & Artuses junchêrrelîn\\ 
10 & spranc einez \textbf{underz houbet} sîn.\\ 
 & dô bant im daz süeze kint\\ 
 & \textbf{ab den helm} unt swanc den wint\\ 
 & \multicolumn{1}{l}{ - - - }\\ 
 & \multicolumn{1}{l}{ - - - }\\ 
 & \textbf{mit eime huote} pfæwîn wîz\\ 
 & \textbf{under dougen}. dirre \textbf{kindes} vlîz\\ 
15 & lêrte Gawanen niwe kraft.\\ 
 & \textbf{Ûz} beiden \textbf{hern} geselleschaft\\ 
 & mit \textbf{rotte} kômen hie unt dort,\\ 
 & \textbf{iewederer} \textbf{her} \textbf{an} sînen ort,\\ 
 & dâ ir zil wâren gestôzen\\ 
20 & mit \textbf{gespiegelten} ronen grôzen.\\ 
 & Gramoflanz die koste gap\\ 
 & durch sînes kampfes urhap.\\ 
 & der boume hundert wâren\\ 
 & mit \textbf{liehten blicken} clâren.\\ 
25 & dâne solte niemen zwischen komen.\\ 
 & \textbf{si} stuonden, \textbf{sus hân ichz} vernomen,\\ 
 & vierzec poynder von ein ander\\ 
 & mit \textbf{gevärweten blicken} glander,\\ 
 & vünfzec ieweder sît.\\ 
30 & dâ zwischen solt ergên der strît.\\ 
\end{tabular}
\scriptsize
\line(1,0){75} \newline
D \newline
\line(1,0){75} \newline
\textbf{1} \textit{Initiale} D  \textbf{3} \textit{Majuskel} D  \textbf{16} \textit{Majuskel} D  \newline
\line(1,0){75} \newline
\textbf{6} swindelnde] swindelde D \textbf{9} Artuses] Artvss D \textbf{15} Gawanen] Gawann D \newline
\end{minipage}
\hspace{0.5cm}
\begin{minipage}[t]{0.5\linewidth}
\small
\begin{center}*m
\end{center}
\begin{tabular}{rl}
 & dû hâst dir \textbf{selben} an gesiget,\\ 
 & ob dîn herze triuwen pfliget."\\ 
 & dô disiu rede was getân,\\ 
 & dô mohte mîn hêr Gawan\\ 
5 & \textbf{von} unkraft niht langer stân.\\ 
 & er begunde \textbf{al} \dag swindeln\dag  gân,\\ 
 & wan im \textbf{daz} houbt erschellet was;\\ 
 & \textbf{er strûch\textit{et}} nider \textbf{an} daz gras.\\ 
 & Artuses junchêrrelîn\\ 
10 & spranc einez \textbf{under daz houbt} sîn.\\ 
 & dô bant im daz süeze kint\\ 
 & \textbf{ab den helm} und swanc den wint\\ 
 & \multicolumn{1}{l}{ - - - }\\ 
 & \multicolumn{1}{l}{ - - - }\\ 
 & \textbf{mit einem huote} pfæwîn wîz\\ 
 & \textbf{under diu ougen}. diser \textbf{kindes} vlîz\\ 
15 & lêrt Gawanen niuwe kraft.\\ 
 & \textbf{ûz} beiden \textit{\textbf{hern}} geselleschaft\\ 
 & mit \textbf{ro\textit{t}e} kômen hie und dort,\\ 
 & \textbf{ietweder} \textbf{here} \textbf{in} sînen ort,\\ 
 & d\textit{â} ir zil wâren gestôzen\\ 
20 & mit \textbf{gespiegelten} ronen grôzen.\\ 
 & Gramolanz die koste gap\\ 
 & durch sînes kampfes urhap.\\ 
 & der boum hundert wâren\\ 
 & mit \textbf{liehten blicken} clâren.\\ 
25 & dâ ensolte nieman zwischen komen.\\ 
 & \textbf{si} stuonden, \textbf{als ichs hân} vernomen,\\ 
 & vierzic ponder von ein ander\\ 
 & mit \textbf{geverwete\textit{m} blic} glander,\\ 
 & vünfzi\textit{c i}etweder sît.\\ 
30 & dâ zwischen solte \textit{e}rgên der strît.\\ 
\end{tabular}
\scriptsize
\line(1,0){75} \newline
m n o \newline
\line(1,0){75} \newline
\newline
\line(1,0){75} \newline
\textbf{1} hâst] hast hast o  $\cdot$ selben] selber n o \textbf{4} dô] Da o  $\cdot$ mohte] moͯchte ouch n (o)  $\cdot$ hêr] herre her n \textbf{5} von] Vor n o \textbf{6} al swindeln] alle swindelin n \textbf{8} strûchet] struch m  $\cdot$ an] in o \textbf{10} einez] ein m \textbf{13} huote] guten o \textbf{16} hern] \textit{om.} m \textbf{17} rote] rore m n o \textbf{18} here] herre o \textbf{19} dâ] Do m n o \textbf{21} Gramolanz] Gramolantz m n Gramolancz o \textbf{23} wâren] weren n werent o \textbf{25} zwischen] zuischen o \textbf{26} ichs] ich o \textbf{28} geverwetem] geferwetten m (n)  $\cdot$ glander] galander n \textbf{29} Funfftzig ponder ietweder sit m \textbf{30} ergên] argen m \newline
\end{minipage}
\end{table}
\newpage
\begin{table}[ht]
\begin{minipage}[t]{0.5\linewidth}
\small
\begin{center}*G
\end{center}
\begin{tabular}{rl}
 & \begin{large}D\end{large}û hâst dir \textbf{selbem} an gesiget,\\ 
 & obe dîn herze triwen pfliget."\\ 
 & dô disiu rede was getân,\\ 
 & dô\textbf{ne} mohte mîn hêr Gawan\\ 
5 & \textbf{vor} unkraft niht langer stên.\\ 
 & er begunde \textbf{sweibelnde} gên,\\ 
 & wan im \textbf{sîn} houbet erschellet was;\\ 
 & \textbf{si sâzen} nider \textbf{ûf} daz gras.\\ 
 & Artuses junchêrrelîn\\ 
10 & spra\textit{nc} einez \textbf{an de\textit{n} rücke} sîn.\\ 
 & dô bant im daz süeze kint\\ 
 & \textbf{den helm abe} unde swanc den wint\\ 
 & \multicolumn{1}{l}{ - - - }\\ 
 & \multicolumn{1}{l}{ - - - }\\ 
 & \textbf{mit einem huote} pfæwîn wîz\\ 
 & \textbf{under sîn ougen}. dirre \textbf{kinde} vlîz\\ 
15 & lêrte Gawan niwe kraft.\\ 
 & \textbf{von} bêden \textbf{hêrren} geselleschaft\\ 
 & mit \textbf{storîen} kômen hie unde dort,\\ 
 & \textbf{ietweders} \textbf{her} \textbf{an} sînen ort,\\ 
 & dâ ir zil wâren gestôzen\\ 
20 & mit \textbf{gesp\textit{a}lten} ronen grôzen.\\ 
 & Gramoflanz die koste gap\\ 
 & durch sînes kampfes urhap.\\ 
 & der bou\textit{m} hundert wâren\\ 
 & mit \textbf{spæhen varwen} clâren.\\ 
25 & dâne solde niemen zwischen komen.\\ 
 & \textbf{die} stuonden, \textbf{sus hân ichz} vernomen,\\ 
 & vierzic poinder von ein ander\\ 
 & mit \textbf{liehten blicken} glander,\\ 
 & vünfzic ietweder sît.\\ 
30 & dâ enzwischen solde ergên der strît.\\ 
\end{tabular}
\scriptsize
\line(1,0){75} \newline
G I L M Z Fr20 Fr52 \newline
\line(1,0){75} \newline
\textbf{1} \textit{Initiale} G I L Z Fr20  \textbf{21} \textit{Initiale} I  \textbf{27} \textit{Initiale} M  \newline
\line(1,0){75} \newline
\textbf{1} Dû] ÷u Fr20  $\cdot$ selbem] selber I (Z) (Fr52) selben L (M) (Fr20) \textbf{2} pfliget] phlige: Fr52 \textbf{3} dô] Da M  $\cdot$ disiu] div Fr20  $\cdot$ was] wart I L  $\cdot$ getân] [gawan]: getan Fr52 \textbf{4} dône] Da en M \textbf{5} vor] von I Fr20  $\cdot$ niht] \textit{om.} I \textbf{6} er begunde] do begunde er I  $\cdot$ sweibelnde] swaibunde I swindelen L (M) al swindelnde Z \textbf{7} im sîn] im daz L (M) Z (Fr20) \textbf{8} si sâzen] Dy saszin M Er strvhte Z  $\cdot$ ûf daz] andaz Fr20 \textbf{9} Artuses] Artvs G (I) (Z) (Fr20) \textbf{10} spranc] sprah G  $\cdot$ einez] \textit{om.} I  $\cdot$ den] dem G \textbf{11} dô] Da M Z \textbf{12} swanc] swanc im I \textbf{14} sîn] sinen Fr20  $\cdot$ dirre] dir M \textbf{15} lêrte] lert I  $\cdot$ Gawan] Gawanen L (M) (Z) Fr20 \textbf{16} hêrren] hern L (M) Fr20 \textbf{17} chomen mit storien hie und dort I  $\cdot$ storîen] stivren L (M) stvrierin Fr20 \textbf{18} ietweders] Jclichir M Jetweder Z  $\cdot$ her] herre M \textbf{20} gespalten] gespielten G spiegelinen L gespigilten M (Z) \textbf{21} Gramoflanz] [Grmoflanz]: Gramoflanz L Gramorflanz M Gramoflantz Z \textbf{23} boum] boͮbe G \textbf{24} spæhen varwen] speher varwe I \textbf{25} dâne solde] da sol I \textbf{26} stuonden] stunden stunden I  $\cdot$ ichz] ich I L M \textbf{27} von] gei: Fr20  $\cdot$ ein] an I \textbf{28} liehten blicken] lichten bliche L lichteme blic M liehtem blic Z  $\cdot$ glander] galander M Z \textbf{29} ietweder] ietwedere G \textbf{30} dâ enzwischen] dar inne I  $\cdot$ strît] [ster]: strit Z \newline
\end{minipage}
\hspace{0.5cm}
\begin{minipage}[t]{0.5\linewidth}
\small
\begin{center}*T
\end{center}
\begin{tabular}{rl}
 & dû hâst dir \textbf{selber} an gesiget,\\ 
 & ob dîn herze triuwen pfliget."\\ 
 & \begin{large}D\end{large}ô disiu rede was getân,\\ 
 & dô \textbf{en}mohte mîn hêr Gawan\\ 
5 & \textbf{vor} unkraft niht langer stân.\\ 
 & er begunde \textbf{swindelnde} gân,\\ 
 & wan im \textbf{daz} houbet erschellet was;\\ 
 & \textbf{si sâzen} nider \textbf{ûf} daz gras.\\ 
 & Artuses junchêrrelîn\\ 
10 & spranc einez \textbf{an den rücke} sîn.\\ 
 & dô bant im daz süeze kint\\ 
 & \textbf{den helm abe} und swanc den wint\\ 
 & under sîne ougen\\ 
 & mit eime huote tougen.\\ 
 & \textbf{der selbe was} pfæwîn wîz.\\ 
 & dirre \textbf{junchêrrelîn} vlîz\\ 
15 & lêrte Gawanen niuwe kraft.\\ 
 & \textbf{ûz} beiden \textbf{hern} geselleschaft\\ 
 & mit \textbf{sturme} \textit{kômen} hie und dort,\\ 
 & \textbf{ietweder} \textbf{her} \textbf{an} sînen ort,\\ 
 & d\textit{â} ir zil wâren gestôzen\\ 
20 & mit \textbf{gespiegelten} ronen grôzen.\\ 
 & Gramoflanz die kost gap\\ 
 & durch sînes kampfes urhap.\\ 
 & der boume hundert wâren\\ 
 & mit \textbf{liehten blicken} clâren.\\ 
25 & dâ ensolte nieman zwischen komen.\\ 
 & \textbf{die} stuonden, \textbf{sus hân ich ez} vernomen,\\ 
 & vierzic poynder von ein ander\\ 
 & mit \textbf{liehtem blicke} glander,\\ 
 & vünfzic ietweder sît.\\ 
30 & dâ enzwischen solte ergân der strît.\\ 
\end{tabular}
\scriptsize
\line(1,0){75} \newline
U V W Q R \newline
\line(1,0){75} \newline
\textbf{3} \textit{Initiale} U W  \newline
\line(1,0){75} \newline
\textbf{1} selber] selben V (Q)  $\cdot$ an gesiget] angesigen R \textbf{2} triuwen] trv́we V (W) (Q) (R) \textbf{4} enmohte] moͤhte V moch R \textbf{5} vor] Von V (Q)  $\cdot$ stân] gesten W \textbf{6} swindelnde] schwindelen W swindelunge Q schwindelinge R \textbf{8} si] Die Q \textbf{9} Artuses] Kúnig artus W Artus Q R \textbf{10} Sprang [eine*]: eines vnders hoͮbet sin V  $\cdot$ spranc einez] Sprach einer W  $\cdot$ rücke] rucken W (R) \textbf{12} swanc] schanktt R \textbf{12} \textit{Die Verse 690.12¹-12² fehlen} W Q R  \textbf{13} Mit einem (einē Q einen R ) huͦte pfewin weiß W (Q) (R) \textbf{14} [*]: Disses [iuncherrelin*]: iuncherrelins vlis V  $\cdot$ Vnder (Vnd Q ) seine augen diser (dise Q ) kinde fleis W (Q) (R) \textbf{15} Gawanen] herr gawan W gawin R \textbf{17} sturme] storie V W Q storien R  $\cdot$ kômen] \textit{om.} U \textbf{18} ietweder] Ytweders Q (R)  $\cdot$ sînen] sin V sine R \textbf{19} dâ] Do U W Q  $\cdot$ zil] \textit{om.} Q \textbf{20} \textit{Vers 690.20 fehlt} R   $\cdot$ gespiegelten] gespalten V  $\cdot$ ronen] rouen W \textbf{21} Gramoflanz] Gramaflanz V Kúnig gramoflantz W Gramoflantz Q Gramoflancz R \textbf{22} urhap] virhab R \textbf{25} ensolte] solte V (R) \textbf{26} Die stvͦnden [*]: als ich han vernomen V  $\cdot$ sus] als Q  $\cdot$ ez] \textit{om.} W R \textbf{28} liehtem] liehten V (R) lichtem Q  $\cdot$ blicke] blikken V \textbf{29} ietweder] yettwedre R \newline
\end{minipage}
\end{table}
\end{document}
