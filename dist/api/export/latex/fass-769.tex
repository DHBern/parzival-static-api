\documentclass[8pt,a4paper,notitlepage]{article}
\usepackage{fullpage}
\usepackage{ulem}
\usepackage{xltxtra}
\usepackage{datetime}
\renewcommand{\dateseparator}{.}
\dmyyyydate
\usepackage{fancyhdr}
\usepackage{ifthen}
\pagestyle{fancy}
\fancyhf{}
\renewcommand{\headrulewidth}{0pt}
\fancyfoot[L]{\ifthenelse{\value{page}=1}{\today, \currenttime{} Uhr}{}}
\begin{document}
\begin{table}[ht]
\begin{minipage}[t]{0.5\linewidth}
\small
\begin{center}*D
\end{center}
\begin{tabular}{rl}
\textbf{769} & \begin{large}A\end{large}rtus sprach: "von dem vater dîn,\\ 
 & Gahmurete, dem neven mîn,\\ 
 & ist ez dîn volleclîcher art,\\ 
 & in \textbf{wîbe} dienste dîn verriu vart.\\ 
5 & Ich wil dich dienst wizzen lân,\\ 
 & daz selten grœzer\textbf{z} ist getân\\ 
 & ûf erde decheinem wîbe,\\ 
 & \textbf{ir} \textbf{wünneclîchem} lîbe.\\ 
 & ich meine die herzoginne,\\ 
10 & diu hie sitzet. nâch ir minne\\ 
 & ist waldes vil \textbf{verswendet}.\\ 
 & ir minne hât gepfendet\\ 
 & an vreuden manegen rîter guot\\ 
 & unt in erwendet hôhen muot."\\ 
15 & Er sagete \textbf{ir urliuge} gar\\ 
 & unt ouch von der Clinschors schar,\\ 
 & \textbf{die dâ} sâzen \textbf{an} allen sîten,\\ 
 & unt von den zwein strîten,\\ 
 & die Parzival, sîn bruoder, streit\\ 
20 & ze Joflanze ûf dem anger breit,\\ 
 & unt swaz er anders hât ervaren,\\ 
 & dâ er den lîp niht kunde sparen.\\ 
 & "Er sol \textbf{dirz} selbe machen kunt.\\ 
 & er suochet einen hôhen vunt:\\ 
25 & nâch dem Grâle \textbf{wirbet} er.\\ 
 & von iu beiden \textbf{sampt} ist daz mîn ger,\\ 
 & \textbf{ir saget mir} liute und lant,\\ 
 & \textbf{die} iu mit \textbf{strîte} \textbf{sîn} bekant."\\ 
 & Der heiden sprach: "ich nenne \textbf{sie},\\ 
30 & die mir die rîter vüerent hie:\\ 
\end{tabular}
\scriptsize
\line(1,0){75} \newline
D Fr2 \newline
\line(1,0){75} \newline
\textbf{1} \textit{Initiale} D Fr2  \textbf{5} \textit{Majuskel} D  \textbf{15} \textit{Majuskel} D  \textbf{23} \textit{Majuskel} D  \textbf{29} \textit{Majuskel} D  \newline
\line(1,0){75} \newline
\textbf{1} Artus] ÷Rtvs Fr2 \textbf{2} Gahmurete] Gahmvrete D Gamvreten Fr2  $\cdot$ dem] den Fr2 \textbf{14} in] im Fr2 \textbf{15} sagete] saget Fr2 \textbf{16} Clinschors] Clinscors D Clinsors Fr2 \textbf{17} an] in Fr2 \textbf{19} Parzival] Parcifal D Partzefal Fr2 \textbf{20} Joflanze] Joflantze Fr2  $\cdot$ ûf] \textit{om.} Fr2 \textbf{23} selbe] selben Fr2 \textbf{28} sîn bekant] sint erkant Fr2 \textbf{29} sprach] \textit{om.} Fr2 \newline
\end{minipage}
\hspace{0.5cm}
\begin{minipage}[t]{0.5\linewidth}
\small
\begin{center}*m
\end{center}
\begin{tabular}{rl}
 & Artus sprach: "von dem vater dîn,\\ 
 & Gahmuret, dem neven mîn,\\ 
 & ist ez dîn volleclîcher art,\\ 
 & in \textbf{wîbe} dienst dîn verriu vart.\\ 
5 & ich wil dich dienst wizzen lân,\\ 
 & daz selten grœzer ist getân\\ 
 & ûf erde keinem wîbe,\\ 
 & \textbf{ir} \textbf{wünneclîchem} lîbe.\\ 
 & ich mein die herzoginne,\\ 
10 & diu hie sitzet. nâch ir minne\\ 
 & ist waldes vil \textbf{geswendet}.\\ 
 & ir minne het gepfendet\\ 
 & an vröude manigen ritter guot\\ 
 & und in erwendet hôhen muot."\\ 
15 & er sagte \textbf{im} \textbf{ir urliuge} gar\\ 
 & und ouch von der Clinsors schar,\\ 
 & \textbf{die dâ} sâzen \textbf{an} allen sîten,\\ 
 & und von den zwein strîten,\\ 
 & die Parcifal, sîn bruoder, streit\\ 
20 & zuo Joflanze ûf dem anger breit,\\ 
 & und waz er anders het ervarn,\\ 
 & d\textit{â} er den lîp niht kunde s\textit{p}arn.\\ 
 & "er sol \textbf{dirz} selbe machen kunt.\\ 
 & er suochet einen hôhen \textit{v}unt:\\ 
25 & nâch dem Grâl \textbf{wir\textit{b}et} er.\\ 
 & von iu beiden ist daz mî\textit{n ge}r,\\ 
 & \textbf{daz ir mir sagt} liute und lant,\\ 
 & \textbf{die} iu mit \textbf{strît} \textbf{sint} bekant."\\ 
 & der heiden sprach: "ich nenne \textbf{iu} \textbf{die},\\ 
30 & die mir die ritter vüerent hie:\\ 
\end{tabular}
\scriptsize
\line(1,0){75} \newline
m n o V V' W \newline
\line(1,0){75} \newline
\textbf{1} \textit{Initiale} V W  \textbf{29} \textit{Initiale} W  \newline
\line(1,0){75} \newline
\textbf{1} \textit{Die Verse 769.1-28 fehlen} V'  \textbf{2} Gahmuret] Gamuͯret n o Gammerette V Gamuret W \textbf{3} dîn] dine V  $\cdot$ volleclîcher] folleclichen o [volleclicher]: vollecliche V \textbf{4} \textit{Verse 769.4-5 kontrahiert zu:} Jn wibe dienst wissen lan o   $\cdot$ wîbe] weibes W \textbf{5} dienst] \textit{om.} W \textbf{6} daz] Des n  $\cdot$ grœzer] groͤssers V \textbf{7} erde] erden n o V W  $\cdot$ keinem] keinen o \textbf{8} wünneclîchem] [wuͯnneclichem]: wuͯnneclichen o \textbf{11} geswendet] [*swendet]: verswendet V \textbf{13} vröude] froͤuden V (W) \textbf{14} in erwendet] ir erwendent W  $\cdot$ hôhen] hohe o \textbf{15} sagte] saget W  $\cdot$ ir] \textit{om.} o \textbf{16} der] des W  $\cdot$ Clinsors] clinsor o klynshors W \textbf{17} dâ] do n o V W \textbf{19} Parcifal] parzefal V herr partzifal W \textbf{20} Joflanze] joflantz m n jofalcz o tschoflantz W \textbf{21} waz] swas V  $\cdot$ het] hette n V \textbf{22} dâ] Do m n o V W  $\cdot$ sparn] sprarn m \textbf{23} sol dirz] so das W \textbf{24} suochet] [suͦchete]: suͦchet V  $\cdot$ vunt] muͯnt m \textbf{25} wirbet] wirdet m \textbf{26} mîn ger] mir m \textbf{27} mir] mit o \textbf{28} \textit{nach 769.28:} Sv́ rettent mitteinander vil gar / Heimliche vnde offenbar / Dar nach ruͦfte kv́nig [art*]: artus do / Den helden allen dar ieso / Hartte betteliche er sv́ bat / Daz ieglicher seite uf der stat / Bi dem eide den er imme hette getan / Waz imme wider varn were svnder wan / Die wile iederman vsser lande waz / Sv́ gelobetent alle zetuͦnde daz / Sv́ seitent die worheit alle glich / Es were in schande oder erlich / Boors vnd lýonel an den stvnden / Seitent wie sv́ einander funden / Vnde wie ietwederre (iclicher V'  ) mit dem andern vaht / Vnde der werde kolagrenans geslaht / Wart do von ir eime erslagen / Daz begundent sv́ von ende sagen (mit eyden clagen V'  ) / Vmbe kolagrenans tot tugenthaft / Trurte der kv́nig vnde alle (Trurte alle V'  ) die ritterschaft / Dar nach sprach der heiden rich / Kv́nig Artus nv hoͤre mich (Einschub entspr. 'Troisième Continuation', Ep. 29, V. 42335-42354) V (V')   $\cdot$ iu] ich o \textbf{29} Jch wil dir nennen alle die V (V') \textbf{30} \textit{nach 769.30:} An den mir ist gelungen / Die ich alle han betwungen V  Die ich alle han betwungen V'   $\cdot$ Kv́nige herzogen Grauen alhie V (V') \newline
\end{minipage}
\end{table}
\newpage
\begin{table}[ht]
\begin{minipage}[t]{0.5\linewidth}
\small
\begin{center}*G
\end{center}
\begin{tabular}{rl}
 & \begin{large}A\end{large}rtus sprach: "von dem vater dîn,\\ 
 & Gahmuret, dem neven mîn,\\ 
 & ist ez dîn volleclîcher art,\\ 
 & in \textbf{wîbes} dienst dîn verriu vart.\\ 
5 & ich wil dich dienst wizzen lân,\\ 
 & daz selten grœzer ist getân\\ 
 & ûf erde deheinem wîbe,\\ 
 & \textbf{ir} \textbf{minneclîchem} lîbe.\\ 
 & ich meine die herzoginne,\\ 
10 & diu hie sitzet. nâch ir minne\\ 
 & ist waldes vil \textbf{verswendet}.\\ 
 & ir minne hât gepfendet\\ 
 & an vröuden manigen rîter guot\\ 
 & unde in erwendet hôhen muot."\\ 
15 & er saget \textbf{im} \textbf{âne lougen} gar\\ 
 & unde ouch von der Clinsores schar,\\ 
 & \textbf{si} sâzen \textbf{in} allen sîten,\\ 
 & unde von den zwein strîten,\\ 
 & die Parzival, sîn bruoder, streit\\ 
20 & \textit{ze} Tschofflanz ûf dem anger breit,\\ 
 & unde swaz er anders het ervarn,\\ 
 & dâ er den lîp niht kunde sparn.\\ 
 & "er sol \textbf{dirz} selbe machen kunt.\\ 
 & er suochte einen hôhen vunt:\\ 
25 & nâch dem Grâle \textbf{suocht}er.\\ 
 & von iu bêden \textbf{sament} ist daz mîn ger,\\ 
 & \textbf{nû saget mir} liute unde lant,\\ 
 & \textbf{diu} iu mit \textbf{state} \textbf{sîn} bekant."\\ 
 & der heiden sprach: "ich nenne \textbf{sie},\\ 
30 & die mir die rîter vüerent hie:\\ 
\end{tabular}
\scriptsize
\line(1,0){75} \newline
G I L M Z Fr18 Fr72 \newline
\line(1,0){75} \newline
\textbf{1} \textit{Initiale} G I L Z Fr18  \textbf{29} \textit{Initiale} Z  \newline
\line(1,0){75} \newline
\textbf{1} Artus] Rtv: Fr72 \textbf{2} Gahmuret] Gahmureten I (Fr18) Gahmvͯret L Gamurete M Gamuret Z Ga:::t Fr72 \textbf{3} ist ez] :st d::: Fr72 \textbf{4} wîbes] wibe I  $\cdot$ verriu] verie L \textbf{6} grœzer] groͤzerz Fr18 \textbf{7} erde] erdin M (Z)  $\cdot$ deheinem] dehainen I \textbf{11} waldes] wandels Z \textbf{14} in] im I \textbf{15} saget im âne lougen] sagite yme ir vrlougen M (Fr72) sagt ir vͦr livgen Fr18 \textbf{16} von der] des I  $\cdot$ Clinsores] Glinsors I Clinisors L clinsors M Clingsors Z Clẏnshors Fr18 :::ors Fr72 \textbf{17} si] Die da L Z Die M Fr18 \textbf{19} Parzival] parcifal G Z (Fr18) parzifal I L M \textbf{20} ze] \textit{om.} G  $\cdot$ Tschofflanz] Tschofflanz G shoffanze I Tschoflanze L scofflanze M Tschoflanz Z Tschoflantz Fr18 schaffians Fr72  $\cdot$ dem] dein L den Fr72 \textbf{21} swaz] waz L (M)  $\cdot$ het] hette M (Fr72) \textbf{23} dirz] ditz I daz L (M) Z Fr18 \textbf{24} suochte] shucht I (L) (M) (Z) (Fr18) \textbf{25} suochter] sucht er I (L) wirbet her M (Z) (Fr18) \textbf{26} iu] ir Fr18  $\cdot$ sament] \textit{om.} I L  $\cdot$ daz] \textit{om.} L M \textbf{27} nû saget mir] vnd sagt mir I Das y mir seit M \textbf{28} iu] \textit{om.} Fr18  $\cdot$ state] stâte G strite M Z Fr18  $\cdot$ sîn] sint L M Z \textbf{29} heiden] heide M \textbf{30} die rîter] da ritter L Ritter M \newline
\end{minipage}
\hspace{0.5cm}
\begin{minipage}[t]{0.5\linewidth}
\small
\begin{center}*T
\end{center}
\begin{tabular}{rl}
 & \begin{large}A\end{large}rtus sprach: "\textit{von} dem vater dîn,\\ 
 & Gahmureten, dem neven mîn,\\ 
 & ist ez dîn volleclîcher art,\\ 
 & in \textbf{wîbes} dienste dîn verriu vart.\\ 
5 & ich wil dich dienst wizzen lân,\\ 
 & daz selten grœzer ist getân\\ 
 & ûf erde dekeime wîbe\\ 
 & \textbf{an} \textbf{minneclîchem} lîbe.\\ 
 & ich meine die herzoginne,\\ 
10 & diu hie sitzet. nâch ir minne\\ 
 & ist waldes vil \textbf{verswendet}.\\ 
 & ir minne hât gepfendet\\ 
 & an vreuden manegen rîter guot\\ 
 & und in erwendet hôhen muot."\\ 
15 & er sagete \textbf{ir urliuge} gar\\ 
 & und ouch von der Clynsors schar,\\ 
 & \textbf{die dâ} sâzen \textbf{in} alle\textit{n} sîten,\\ 
 & und von den zwein strîten,\\ 
 & \textit{die Parcifal, sîn bruoder, streit}\\ 
20 & zuo Tschoflanze ûf dem anger breit,\\ 
 & und waz er ander\textit{s} het ervarn,\\ 
 & dâ er den lîp niht kunde sparn.\\ 
 & "er sol \textbf{daz} selbe machen kunt.\\ 
 & er suochet einen hôhen vunt:\\ 
25 & nâch dem Grâle \textbf{wirbet} er.\\ 
 & von iu beiden \textbf{samt} ist daz mîn ger,\\ 
 & \textbf{nû saget mir} liute und lant,\\ 
 & \textbf{die} iu mit \textbf{strîte} \textbf{sint} bekant."\\ 
 & der heiden sprach: "ich nenne \textbf{sie},\\ 
30 & die mir die rîter vüere\textit{n}t hie:\\ 
\end{tabular}
\scriptsize
\line(1,0){75} \newline
U Q R Fr53 \newline
\line(1,0){75} \newline
\textbf{1} \textit{Initiale} U  \newline
\line(1,0){75} \newline
\textbf{1} \textit{Die Verse 764.13-774.30 fehlen} R   $\cdot$ von] \textit{om.} U \textbf{2} Gahmureten] Gamuret Q \textbf{3} volleclîcher] willicklicher Q \textbf{4} dîn] dem Q \textbf{6} grœzer] grozest Q \textbf{7} dekeime] kleinem \textit{nachträglich korrigiert zu:} keinem Q \textbf{8} an minneclîchem] Jrem mindiglichen Q \textbf{15} sagete] sagt Q  $\cdot$ urliuge] vrlawbe Q \textbf{16} der] des Q  $\cdot$ Clynsors] Clinshor \textit{nachträglich korrigiert zu:} Clinshors Q \textbf{17} dâ] do Q  $\cdot$ in allen] in alle U mallen Q an allen Fr53 \textbf{19} \textit{Vers 769.19 fehlt} U   $\cdot$ Parcifal] partzifal Q \textbf{20} Tschoflanze] schoflansze Q schoflanze Fr53  $\cdot$ dem] den Q \textbf{21} und waz] Swaz Fr53  $\cdot$ anders] ander U \textbf{22} dâ] Do Q \textbf{23} daz] dirz Fr53 \textbf{26} samt ist daz] ist Fr53 \textbf{28} sint] sein Q (Fr53)  $\cdot$ bekant] erkant Q (Fr53) \textbf{30} vüerent] vuͦret U \newline
\end{minipage}
\end{table}
\end{document}
