\documentclass[8pt,a4paper,notitlepage]{article}
\usepackage{fullpage}
\usepackage{ulem}
\usepackage{xltxtra}
\usepackage{datetime}
\renewcommand{\dateseparator}{.}
\dmyyyydate
\usepackage{fancyhdr}
\usepackage{ifthen}
\pagestyle{fancy}
\fancyhf{}
\renewcommand{\headrulewidth}{0pt}
\fancyfoot[L]{\ifthenelse{\value{page}=1}{\today, \currenttime{} Uhr}{}}
\begin{document}
\begin{table}[ht]
\begin{minipage}[t]{0.5\linewidth}
\small
\begin{center}*D
\end{center}
\begin{tabular}{rl}
\textbf{752} & "\begin{large}O\end{large}wê der unergetzten nôt!",\\ 
 & sprach der heiden. "ist mîn vater tôt?\\ 
 & ich mac wol \textbf{vreuden} vlüste jehen\\ 
 & unt vreuden vunt mit wârheit spehen.\\ 
5 & ich hân an disen stunden\\ 
 & vreude vloren unt vreude vunden.\\ 
 & Wil ich der wârheit grîfen zuo,\\ 
 & beidiu mîn vater unt \textbf{ouch} dû\\ 
 & unt ich, \textbf{wir} wâren \textbf{gar} al ein,\\ 
10 & doch ez \textbf{an} drîen stücken schein.\\ 
 & swâ man siht den wîsen man,\\ 
 & der enzelt decheine sippe dan,\\ 
 & zwischen vater unt \textbf{des} kinden,\\ 
 & wil er die wârheit vinden.\\ 
15 & mit dir selben hâstû \textbf{hie} gestriten,\\ 
 & gein mir selben \textbf{ich kom} ûf strît geriten,\\ 
 & \textbf{mich selben het ich} gern erslagen.\\ 
 & dône kundestû des niht \textbf{verzagen},\\ 
 & dû\textbf{ne} wertes mir mîn selbes lîp.\\ 
20 & Jupiter, \textbf{diz} wunder schrîp!\\ 
 & dîn kraft tet uns helfe kunt,\\ 
 & daz \textbf{si} unser sterben understuont."\\ 
 & Er lachete unde weinde tougen.\\ 
 & sîniu heidenschiu ougen\\ 
25 & begunden wazzer rêren\\ 
 & al \textbf{nâch} des toufes êren.\\ 
 & der touf \textbf{sol lêren} triwe,\\ 
 & sît unser ê diu niwe\\ 
 & nâch Kriste wart genennet.\\ 
30 & an Kriste ist triwe erkennet.\\ 
\end{tabular}
\scriptsize
\line(1,0){75} \newline
D \newline
\line(1,0){75} \newline
\textbf{1} \textit{Initiale} D  \textbf{7} \textit{Majuskel} D  \textbf{23} \textit{Majuskel} D  \newline
\line(1,0){75} \newline
\textbf{29} Kriste] christe D \newline
\end{minipage}
\hspace{0.5cm}
\begin{minipage}[t]{0.5\linewidth}
\small
\begin{center}*m
\end{center}
\begin{tabular}{rl}
 & "owê der unergetzten nôt!",\\ 
 & sprach der heiden. "ist mîn vater tôt?\\ 
 & ich mac wol \textbf{vröuden} vlust jehen\\ 
 & und vröuden \textit{v}unt mit wârheit spehen.\\ 
5 & ich hân an disen stunden\\ 
 & \textit{vröude verloren und vröude vunden}.\\ 
 & wil ich der wârheit grîfen zuo,\\ 
 & beidiu mîn vater und \textbf{ouch} dû\\ 
 & und ich, \textbf{wir} wâren \textbf{gar} alein,\\ 
10 & doch ez \textbf{an} drîn stücken schein.\\ 
 & wâ man siht den wîsen man,\\ 
 & de\textit{r} enz\textit{e}lt dekein sippe \textit{d}an,\\ 
 & zwischen vater und \textbf{den} kinden,\\ 
 & wil er die wârheit vinden.\\ 
15 & mit dir selbe hâstû gestriten,\\ 
 & gegen mir selbe \textbf{kam ich} û\textit{f} strît geriten\\ 
 & \textbf{und het mich selb} gern erslagen.\\ 
 & dô enkundestu des niht \textbf{vertragen},\\ 
 & dû wertest mir mîn selbe\textit{s l}î\textit{p}.\\ 
20 & Jupiter, \textbf{diz} wunder schrî\textit{p}!\\ 
 & dîn kraft tet uns helfe kuont,\\ 
 & daz \textbf{si} un\textit{s}er sterben \textit{un}de\textit{rst}uont."\\ 
 & er \textit{l}achte und weinte tougen.\\ 
 & sîniu heidenschiu ougen\\ 
25 & begunden wazzer rêren\\ 
 & al \textbf{nâch} des toufes êren.\\ 
 & der touf \textbf{sol lêren} triuwe,\\ 
 & sît unser ê diu niuwe\\ 
 & nâch Krist wart genennet.\\ 
30 & an Krist ist triuwe erkennet.\\ 
\end{tabular}
\scriptsize
\line(1,0){75} \newline
m n o V V' Fr69 \newline
\line(1,0){75} \newline
\textbf{1} \textit{Initiale} V  \textbf{2} \textit{Capitulumzeichen} V'  \newline
\line(1,0){75} \newline
\textbf{1} unergetzten] vierczgetzten o \textbf{3} \textit{Die Verse 752.3-4 fehlen} V'   $\cdot$ vröuden vlust jehen] freuͯiden fluͯst johen o froͮede [flust*]: fluste iehen V \textbf{4} vunt] muͯnt m frint n \textbf{6} \textit{Vers 752.6 fehlt} m   $\cdot$ vröude vunden] disen fuͯnden o \textbf{9} Jch han verlorn den vater min V' \textbf{10} So han ich funden >dich< bruder fin (\textit{vgl. 752.14:} vinden) V' \textbf{11} \textit{Die Verse 752.11-22 fehlen} V'   $\cdot$ wâ] Do n Swo V (Fr69) \textbf{12} der enzelt] Den enzilt m n o  $\cdot$ dekein] do kein n  $\cdot$ dan] an m \textbf{13} vater] vatteren V  $\cdot$ den] des n o V Fr69 \textbf{15} selbe] selben V \textbf{16} selbe] selben V  $\cdot$ ûf] vs m \textbf{17} selb] selben V \textbf{18} vertragen] vertagen n o \textbf{19} wertest] enwertest V  $\cdot$ selbes lîp] selbes hant lipt m \textbf{20} Jupiter] Juppiter n (V) J:piter o  $\cdot$ schrîp] schript m (n) \textbf{22} unser] vnder m  $\cdot$ understuont] det kunt m \textbf{23} lachte] dahtte m lachet V'  $\cdot$ tougen] ouch tovgen V' \textbf{24} Mit sinen heidenischen ovgen V'  $\cdot$ heidenschiu] heidenschen n o (V) \textbf{25} \textit{Die Verse 752.25-30 fehlen} V'  \textbf{26} al] Alle n \textbf{28} diu] de o \textbf{29} Krist] crist m n o criste V  $\cdot$ genennet] gemennet o \textbf{30} Krist] crist m n o criste V  $\cdot$ triuwe] vnser truwe V \newline
\end{minipage}
\end{table}
\newpage
\begin{table}[ht]
\begin{minipage}[t]{0.5\linewidth}
\small
\begin{center}*G
\end{center}
\begin{tabular}{rl}
 & "\begin{large}O\end{large}wê der unergetzeten nôt!",\\ 
 & sprach der heiden. "ist mîn vater tôt?\\ 
 & ich mac wol \textbf{vröude unde} vlüste jehen\\ 
 & unde vröuden vunt mit wârheit spehen.\\ 
5 & ich hân an disen stunden\\ 
 & vröude verlorn unde vröude vunden.\\ 
 & wil ich der wârheit grîfen zuo,\\ 
 & beidiu mîn vater unde dû\\ 
 & unde ich wâren al ein,\\ 
10 & doch ez \textbf{in} drîn stucken schein.\\ 
 & swâ man siht den wîsen man,\\ 
 & derne zelt neheine sippe dan,\\ 
 & zwischen vater unde \textbf{den} kinden,\\ 
 & wil er die wârheit vinden.\\ 
15 & mit dir selben hâstû \textbf{hie} gestriten,\\ 
 & gein mir selben \textbf{ich kom} ûf strît geriten,\\ 
 & \textbf{mich selben ich hete} gerne erslagen.\\ 
 & dône kundestû des niht \textbf{verzagen},\\ 
 & dû\textbf{ne} wertest mir mîn selbes lîp.\\ 
20 & Juppiter, \textbf{daz} wunder schrîp!\\ 
 & dîn kraft tet uns helfe kunt,\\ 
 & daz \textbf{si} unser sterben understuont."\\ 
 & er lachete unde weinde tougen.\\ 
 & sîniu heidenschiu ougen\\ 
25 & begunden wazzer rêren\\ 
 & al \textbf{nâch} des toufes êren.\\ 
 & der tou\textit{f} \textbf{\textit{pfligt} solher} triwe,\\ 
 & sît unser ê diu niwe\\ 
 & nâch Kristen wart genennet.\\ 
30 & an Kriste ist triwe erkennet.\\ 
\end{tabular}
\scriptsize
\line(1,0){75} \newline
G I L M Z \newline
\line(1,0){75} \newline
\textbf{1} \textit{Initiale} G I L M Z  \textbf{21} \textit{Initiale} I  \newline
\line(1,0){75} \newline
\textbf{1} unergetzeten] vnuerzagten I \textbf{2} heiden] heide M \textbf{3} jehen] nu sehen I \textbf{4} vröuden vunt] auch I  $\cdot$ mit wârheit] von schulden Z  $\cdot$ spehen] iehen I \textbf{6} unde vröude] vnd L \textbf{7} wil ich] ich wil I  $\cdot$ wârheit] frevde vnd warheit Z \textbf{9} wâren al ein] dach was alleine M waren gar alleine Z \textbf{10} in drîn] an drin I yn daryn M  $\cdot$ schein] scheine Z \textbf{11} swâ] Wa L M \textbf{12} derne zelt] Der enzcelte M \textbf{13} den] \textit{om.} L \textbf{15} selben] selber L  $\cdot$ hie] die L \textbf{16} selben] selber L  $\cdot$ ich kom] chom ich I  $\cdot$ ûf strît] \textit{om.} L \textbf{17} Jch het mich selber gerne erslagen L  $\cdot$ ich hete gerne] ich hete nih gerne I gerne hette M \textbf{18} dône] Da M Da en Z  $\cdot$ des] \textit{om.} I \textbf{19} dûne wertest] Du werst M \textbf{20} Juppiter] Ivppiter G Jupiter I (L) (M) (Z)  $\cdot$ schrîp] schript L \textbf{22} si] \textit{om.} I Z er L  $\cdot$ sterben] [craft]: sterben I \textbf{24} heidenschiu] haidnischen I \textbf{27} touf] toͮffe G (M) (Z)  $\cdot$ pfligt solher] solher G sal leren M (Z) \textbf{28} sît] Sin Z \textbf{29} Kristen] christen G kriste I cristen M (Z) \textbf{30} Kriste] christe G crist M Z \newline
\end{minipage}
\hspace{0.5cm}
\begin{minipage}[t]{0.5\linewidth}
\small
\begin{center}*T
\end{center}
\begin{tabular}{rl}
 & "\begin{large}O\end{large}wê der unergetzeten nôt!",\\ 
 & sprach der heiden. "ist mîn vater tôt?\\ 
 & ich mac wol \textbf{vreude und} verlust jehen\\ 
 & und vreuden vunt mit wârheit \textit{sp}ehen.\\ 
5 & ich hân an disen stunden\\ 
 & vreude verlorn und vreude vunden.\\ 
 & wil ich der wârheit grîfen zuo,\\ 
 & beidiu mîn vater und dû\\ 
 & \textit{und} ich \textit{w}âren \textbf{doch} alein,\\ 
10 & doch ez \textbf{in} drîn stücken schein.\\ 
 & wâ man sihet den wîsen man,\\ 
 & der enzelet dekeine sippe dan,\\ 
 & zwischen \textbf{dem} vater und \textbf{de\textit{n}} kinde\textit{n},\\ 
 & wil er die wârheit vinden.\\ 
15 & mit dir selber hâstû \textbf{hie} gestriten,\\ 
 & gein mir selbe \textbf{ich kam} ûf strît geriten,\\ 
 & \textbf{mich selber het ich} gerne erslagen.\\ 
 & dô enkundestû des niht \textbf{verzagen},\\ 
 & dû werte\textit{st} mir mîn selbes lîp.\\ 
20 & Jupiter, \textbf{daz} wunder schrîp!\\ 
 & dîn kraft tet uns helfe kunt,\\ 
 & daz unser sterbe\textit{n} understuont."\\ 
 & er lachete und weinde tougen.\\ 
 & sîniu heidenschiu ougen\\ 
25 & begunden wazzer rêren\\ 
 & al \textbf{durch} des toufes êren.\\ 
 & der touf \textbf{sol lêren} triuwe,\\ 
 & sît unser ê diu niuwe\\ 
 & nâch Kriste wart genennet.\\ 
30 & an Kriste ist triuwe erkennet.\\ 
\end{tabular}
\scriptsize
\line(1,0){75} \newline
U W Q R \newline
\line(1,0){75} \newline
\textbf{1} \textit{Initiale} U W R  \newline
\line(1,0){75} \newline
\textbf{3} wol] \textit{om.} W  $\cdot$ jehen] sehen W \textbf{4} und] Vn R  $\cdot$ spehen] iehen U W \textbf{6} und vreude] vnd Q \textbf{8} dû] auch du W \textbf{9} und] \textit{om.} U  $\cdot$ wâren] varen U \textbf{10} schein] erscheine R \textbf{11} sihet] sich Q \textbf{12} enzelet] zelt R \textbf{13} den kinden] dem kinde U \textbf{15} selber] selben Q \textbf{16} selbe] \textit{om.} W selber Q R  $\cdot$ kam] han R \textbf{18} dô enkundestû] Dennoch kúndestu Q Do kundestu R  $\cdot$ verzagen] vertagen R \textbf{19} wertest] werte U entwertest W werdest Q  $\cdot$ mîn] dein W \textbf{20} Jupiter] Juͦpiter U Iupiter W Juppiter Q  $\cdot$ daz] dis R \textbf{22} unser] sy vnser R  $\cdot$ sterben] sterber U starben W \textbf{23} und weinde tougen] weinde vnd taugen W \textbf{26} durch] nach W Q R \textbf{27} touf] tawfe Q Ruͦff R  $\cdot$ lêren] lernen R  $\cdot$ triuwe] trewen Q \textbf{28} niuwe] newen Q \textbf{29} Kriste] criste U kristo W cristo Q (R) \textbf{30} Kriste] criste U kristo W cristo Q (R) \newline
\end{minipage}
\end{table}
\end{document}
