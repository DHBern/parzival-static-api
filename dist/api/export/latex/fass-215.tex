\documentclass[8pt,a4paper,notitlepage]{article}
\usepackage{fullpage}
\usepackage{ulem}
\usepackage{xltxtra}
\usepackage{datetime}
\renewcommand{\dateseparator}{.}
\dmyyyydate
\usepackage{fancyhdr}
\usepackage{ifthen}
\pagestyle{fancy}
\fancyhf{}
\renewcommand{\headrulewidth}{0pt}
\fancyfoot[L]{\ifthenelse{\value{page}=1}{\today, \currenttime{} Uhr}{}}
\begin{document}
\begin{table}[ht]
\begin{minipage}[t]{0.5\linewidth}
\small
\begin{center}*D
\end{center}
\begin{tabular}{rl}
\textbf{215} & - dâ vert ouch \textbf{vor dir} Kingrun -,\\ 
 & gein Artuse dem Bertun.\\ 
 & dem soltû mînen dienest sagen.\\ 
 & bit in, daz er mir helfe klagen\\ 
5 & laster, daz ich vuorte dan.\\ 
 & ein juncvrouwe mich lachte an.\\ 
 & daz man die durch mich zerblou,\\ 
 & sô sêre mich nie dinc gerou.\\ 
 & der selben sage, ez sî mir leit,\\ 
10 & unt bring ir dîne sicherheit,\\ 
 & sô daz dû leistes \textbf{sîn} gebot,\\ 
 & oder nim alhie den tôt."\\ 
 & "Sol daz geteilte gelten,\\ 
 & sône wil ich\textbf{z} niht \textbf{beschelten}."\\ 
15 & \textbf{sus} sprach der künec von Brandigan:\\ 
 & "ich wil die \textbf{vart} von hinnen hân."\\ 
 & Mit \textbf{gelübde} \textbf{dô} dannen schiet,\\ 
 & den \textbf{ê sîn hôchvart} verriet.\\ 
 & Parzival, der wîgant,\\ 
20 & gienc, dâ er sîn ors al müede vant.\\ 
 & sîn vuoz dar \textbf{nâch} nie gegreif:\\ 
 & er spranc drûf âne stegreif,\\ 
 & daz \textbf{al}umbe \textbf{begunden} \textbf{zirben}\\ 
 & \textbf{sîne} \textbf{verhouwene} \textbf{schildes schirben}.\\ 
25 & \textbf{\begin{large}D\end{large}es} wâren die burgære gemeit.\\ 
 & daz ûzer her sach herzeleit.\\ 
 & brât unt lide \textbf{im} tâten wê.\\ 
 & \textbf{man leite den künec} Clamide,\\ 
 & dâ sîne helfære wâren.\\ 
30 & die tôten mit den bâren\\ 
\end{tabular}
\scriptsize
\line(1,0){75} \newline
D \newline
\line(1,0){75} \newline
\textbf{13} \textit{Majuskel} D  \textbf{17} \textit{Majuskel} D  \textbf{25} \textit{Initiale} D  \newline
\line(1,0){75} \newline
\textbf{1} Kingrun] kingrvͦn D \textbf{2} Bertun] Beritvͦn D \textbf{28} Clamide] Chlamidê D \newline
\end{minipage}
\hspace{0.5cm}
\begin{minipage}[t]{0.5\linewidth}
\small
\begin{center}*m
\end{center}
\begin{tabular}{rl}
 & - dar vert ouch \textbf{vo\textit{r} dir} Kingr\textit{un} -,\\ 
 & gegen Artuse dem Britu\textit{n}.\\ 
 & dem solt\textit{û} mînen dienest sagen.\\ 
 & bite in, daz er mir helfe klagen\\ 
5 & laster, daz ich vuorte dan.\\ 
 & ein juncvr\textit{o}uwe mich lachete an.\\ 
 & daz man die durch mich zerblou,\\ 
 & sô sêre mich nie dinc gerou.\\ 
 & der selben sage, ez sî mir leit,\\ 
10 & und bring ir dî\textit{n} sicherheit,\\ 
 & sô daz dû leis\textit{t}est \textbf{ir} gebot,\\ 
 & oder nim alhie \textbf{von mir} den tôt."\\ 
 & "sol daz geteilte gelten,\\ 
 & sô enwil ich\textbf{z} niht \textbf{beschelten}",\\ 
15 & sprach der künic von Brandigan.\\ 
 & "ich wil die \textbf{vart} von hinnen hân."\\ 
 & mit \textbf{gelübde} \textbf{dô} dannen schiet,\\ 
 & den \textbf{sî\textit{n} hôchvart ê} verriet.\\ 
 & Parcifal, der wîgant,\\ 
20 & gienc, d\textit{â} er sîn ros almüede vant.\\ 
 & sîn vuoz dar \textbf{nâher} nie gegreif:\\ 
 & er spranc drûf âne stegreif,\\ 
 & daz \textbf{al}umbe \textbf{begunden} \textbf{zirben}\\ 
 & \textbf{sîn} \textbf{verhouwene} \textbf{schiltes schirben}.\\ 
25 & \textbf{des} wâren die burgære gemeit.\\ 
 & daz ûzer her sach herzeleit.\\ 
 & \hspace*{-.7em}\big| \textbf{man vuorte den künic} Clamide,\\ 
 & \hspace*{-.7em}\big| \textbf{dem} brât und lide tâten wê,\\ 
 & dâ sîne helfære wâren.\\ 
30 & die tôten mit den bâren\\ 
\end{tabular}
\scriptsize
\line(1,0){75} \newline
m n o Fr69 \newline
\line(1,0){75} \newline
\newline
\line(1,0){75} \newline
\textbf{1} vor] uol m  $\cdot$ dir] die o  $\cdot$ Kingrun] kingrim m konigrim o \textbf{2} Britun] brittuͯm m britym o \textbf{3} soltû] solte m  $\cdot$ mînen] niemen o \textbf{6} juncvrouwe] jungfroruwe m  $\cdot$ lachete] lachet n o Fr69 \textbf{9} sage] sage ich o \textbf{10} dîn] dis m \textbf{11} leistest] leiststest m \textbf{14} beschelten] schelten n \textbf{15} Brandigan] brandigon n brandian o \textbf{16} hinnen] hunden o \textbf{17} dô] \textit{om.} n o er do Fr69 \textbf{18} sîn] sins m  $\cdot$ ê] ie o \textbf{20} dâ] do m n o  $\cdot$ almüede] [almuͯt]: almuͯde n \textbf{23} begunden] begunde n o (Fr69) \textbf{24} schiltes] schilites n  $\cdot$ schirben] schieben o \textbf{25} des] Das n \textbf{28} Clamide] klamide o \textbf{29} dâ] Do n o \textbf{30} die] Sie o \newline
\end{minipage}
\end{table}
\newpage
\begin{table}[ht]
\begin{minipage}[t]{0.5\linewidth}
\small
\begin{center}*G
\end{center}
\begin{tabular}{rl}
 & - dâ vert ouch \textbf{vor dir} Kingrun -,\\ 
 & gein Artuse dem Britun.\\ 
 & dem soltû mîn dienst sagen.\\ 
 & bit in, daz er mir helfe klagen\\ 
5 & laster, daz ich vuorte dan.\\ 
 & ein juncvrouwe mich lachte an.\\ 
 & daz man die durch mich zerblou,\\ 
 & sô sêre mich nie dinc gerou.\\ 
 & der selben sage, ez sî mir leit,\\ 
10 & unde brinc ir dîne sicherheit,\\ 
 & sô daz dû leistest \textbf{ir} gebot,\\ 
 & oder \textbf{dû}, nim alhie den tôt."\\ 
 & "sol daz geteilt gelten,\\ 
 & sône wil ich \textbf{ez} niht \textbf{beschelten}."\\ 
15 & \textbf{dô} sprach der künic von Brandigan:\\ 
 & "ich wil die \textbf{reise} von hinnen hân."\\ 
 & mit \textbf{urloube} dannen schiet,\\ 
 & den \textbf{ê sîn hôher muot} verriet.\\ 
 & Parzival, der wîgant,\\ 
20 & \textit{gie}, \textit{dâ er} sîn or\textit{s} \textit{a}lmüede vant.\\ 
 & sîn vuoz dar \textbf{nâher} nie gegreif:\\ 
 & er spranc drûf âne stegreif,\\ 
 & daz umbe \textbf{in} \textbf{begunden} \textbf{zirben}\\ 
 & \textbf{sînes} \textbf{schiltes schirben}.\\ 
25 & \textbf{des} wâren die burgære gemeit.\\ 
 & daz ûzer her sach herzeleit.\\ 
 & brât unde lide \textbf{im} tâten wê.\\ 
 & \textbf{man leite den künic} Clamide,\\ 
 & dâ sîne helfære wâren.\\ 
30 & die tôten mit den bâren\\ 
\end{tabular}
\scriptsize
\line(1,0){75} \newline
G I O L M Q R Z Fr21 \newline
\line(1,0){75} \newline
\textbf{15} \textit{Initiale} R  \textbf{19} \textit{Initiale} L M Z  \textbf{29} \textit{Initiale} I O Fr21  \newline
\line(1,0){75} \newline
\textbf{1} dâ] Do O Q Das M  $\cdot$ ouch] \textit{om.} L  $\cdot$ Kingrun] kyngrvn O (R) kýngrvn L kyngruͯn Q \textbf{2} Artuse] artus I Q R Z Artuͯse L  $\cdot$ dem] den Q  $\cdot$ Britun] Brittvn L brittum Q Briton R \textbf{3} mîn] minen I (L) (Q) R Z \textbf{4} klagen] clage M tragen R \textbf{5} laster] Das laster Q \textbf{6} lachte] lachet I O L Q R \textbf{7} daz] Di O  $\cdot$ die] \textit{om.} O R \textbf{8} sô] Zu Q  $\cdot$ nie dinc] nie dehain dinc I dick nie Q \textbf{9} selben] selbe M  $\cdot$ sage] sage auch I \textbf{10} ir] in L \textbf{11} daz dû] dagstu Q  $\cdot$ leistest] stest zuͤ I haltest R \textbf{12} nim] nymmest M  $\cdot$ alhie] hie I O alhe Fr21 \textbf{13} sol] So R  $\cdot$ geteilt] getellte Q \textbf{14} sône] So O L R  $\cdot$ ez] \textit{om.} I O R  $\cdot$ beschelten] schelten Z \textbf{15} dô] \textit{om.} I Svs O (L) (Z) Fr21 So M Es Q  $\cdot$ Brandigan] Brandegan L [Brang]: Brandigan R \textbf{16} wil] \textit{om.} I  $\cdot$ reise] vart L  $\cdot$ von hinnen] hinne I (O) (M) (Q) (Z) (Fr21)  $\cdot$ hân] an Q farn R \textbf{17} urloube] vrlaube er I (O) geluͯbede er L dem gelubte Q (R) gelvbde da Z \textbf{18} ê] hie M  $\cdot$ hôher muot] hohvart O (L) (M) (Q) (R) (Z) (Fr21) \textbf{19} Parzival] Parzifal I (M) Parcifal O (L) Z Fr21 Partzifal Q Parczifal R \textbf{20} sin ors er almoͮde vant G  $\cdot$ dâ] do Q R  $\cdot$ almüede] \textit{om.} I alleine R mvde Z \textbf{21} sîn] Sins L  $\cdot$ nâher] nach O M Fr21 nachen R  $\cdot$ gegreif] begreiff Q R \textbf{22} stegreif] stegeraf Q \textbf{23} umbe in begunden] vmbe gegvnden O ez vmbe beguͯnde L vmmbe bigonde M im begonde Q es begund vmbe R vmbe gunde Z (Fr21) \textbf{24} sînes] Sines verhowen L Sein vorhawen Q Sins verhowe R Sin verhowenz Z  $\cdot$ schiltes] schilde Z \textbf{26} ûzer her] vnszer herre Q  $\cdot$ sach] Gwan I sin O het R  $\cdot$ herzeleit] herzenleit O (Z) \textbf{27} brât] Braten Z  $\cdot$ lide] bein O lider Z  $\cdot$ im tâten] in taten I Fr21 taten im Z \textbf{28} leite] lert O fuͯrte L leyd Q (R) (Fr21)  $\cdot$ Clamide] [clmide]: clamide G klamide I Glamide O \textbf{29} dâ] ÷a O Do Q Di da Fr21  $\cdot$ helfære] helffe M \textbf{30} tôten] taten Q \newline
\end{minipage}
\hspace{0.5cm}
\begin{minipage}[t]{0.5\linewidth}
\small
\begin{center}*T
\end{center}
\begin{tabular}{rl}
 & - dâ vert ouch \textbf{dîn} Kyngrun -,\\ 
 & gegen Artuse dem Britun.\\ 
 & dem soltû mînen dienst sagen.\\ 
 & bitin, daz er mir helfe klagen\\ 
5 & laster, daz ich vuorte dan.\\ 
 & ein juncvrouwe mich lachete an.\\ 
 & daz man die durch mich zerblou,\\ 
 & sô sêre mich nie dinc gerou.\\ 
 & der selben sage, ez sî mir leit,\\ 
10 & unde bringir dîne sicherheit,\\ 
 & sô daz dû leistest \textbf{ir} gebot,\\ 
 & oder nim alhie \textbf{von mir} den tôt."\\ 
 & "Sol daz geteil\textit{t}e gelten,\\ 
 & sône wil ich \textbf{sîn} niht \textbf{schelten}",\\ 
15 & sprach der künec von Brandigan.\\ 
 & "ich wil die \textbf{vart} von hinnen hân."\\ 
 & mit \textbf{gelübede} dannan schiet,\\ 
 & den \textbf{ê sîn stolzheit} verriet.\\ 
 & Parcifal, der wîgant,\\ 
20 & gie, dâ er sîn ors almüedez vant.\\ 
 & sîn vuoz dar \textbf{nâch} nie gegreif:\\ 
 & er spranc drûf âne stegreif,\\ 
 & daz \textbf{ez} umbe \textbf{begunde} \textbf{zwirben}.\\ 
 & \textbf{sîner} \textbf{verhouwenen} \textbf{schiltschirben}\\ 
25 & wâren die burgære gemeit.\\ 
 & Daz ûzer her sach herzeleit\\ 
 & \hspace*{-.7em}\big| \textbf{an dem künege} Clamide.\\ 
 & \hspace*{-.7em}\big| brât unde lide \textbf{im} tâten wê.\\ 
 & \textbf{man vuortin}, dâ sîne helfære wâren.\\ 
30 & die tôten mit den bâren\\ 
\end{tabular}
\scriptsize
\line(1,0){75} \newline
T U V W \newline
\line(1,0){75} \newline
\textbf{13} \textit{Majuskel} T  \textbf{26} \textit{Majuskel} T  \newline
\line(1,0){75} \newline
\textbf{1} dâ] [D*]: Dar V Do W  $\cdot$ dîn] dir U [*]: uor dir V vor dir W  $\cdot$ Kyngrun] kyngruͦn U [*]: kingrun V kingrun W \textbf{2} Britun] Brituͦn U brittvn V \textbf{7} die] ir W  $\cdot$ zerblou] zuͦ der blo U \textbf{8} nie dinc] ding nie W  $\cdot$ gerou] beroͮ V \textbf{12} alhie von mir] von mir alhie V alhie W \textbf{13} geteilte] [geleite]: geteile T \textbf{14} sône] So W  $\cdot$ ich sîn niht schelten] ich nit beschelten U [*]: ichz niht beschelten V ich es nit beschelten W \textbf{17} gelübede] vrlobe er V \textbf{18} verriet] verreit W \textbf{19} Parcifal] Parzifal V Partzifal W \textbf{20} dâ] do W  $\cdot$ almüedez] alle muͦde U \textbf{21} vuoz] mit U tuͦs W  $\cdot$ dar nâch] [darna*]: darnaher V \textbf{22} spranc] sprach V \textbf{23} ez] [*]: ez V \textit{om.} W \textbf{24} sîner] [Sin*]: Sins V Sein W  $\cdot$ verhouwenen] verhowene V zerhawen W  $\cdot$ schiltschirben] [*]: schiltez schirben V \textbf{25} wâren] [D*]: Dez warent V Des waren W \textbf{28} \textit{Versfolge 215.27-28} W   $\cdot$ an] Man laite W  $\cdot$ Clamide] klamide W \textbf{27} brât] Bracht U [D*]: Dem brat V  $\cdot$ im] \textit{om.} V \textbf{29} man vuortin] \textit{om.} W  $\cdot$ dâ] do V (W) \textbf{30} tôten] taten W \newline
\end{minipage}
\end{table}
\end{document}
