\documentclass[8pt,a4paper,notitlepage]{article}
\usepackage{fullpage}
\usepackage{ulem}
\usepackage{xltxtra}
\usepackage{datetime}
\renewcommand{\dateseparator}{.}
\dmyyyydate
\usepackage{fancyhdr}
\usepackage{ifthen}
\pagestyle{fancy}
\fancyhf{}
\renewcommand{\headrulewidth}{0pt}
\fancyfoot[L]{\ifthenelse{\value{page}=1}{\today, \currenttime{} Uhr}{}}
\begin{document}
\begin{table}[ht]
\begin{minipage}[t]{0.5\linewidth}
\small
\begin{center}*D
\end{center}
\begin{tabular}{rl}
\textbf{651} & \begin{large}G\end{large}awans mâc, der rîche\\ 
 & Artus, warp \textbf{herzenlîche}\\ 
 & zer messenîe dise vart.\\ 
 & vor sûmen \textbf{het ouch sich} bewart\\ 
5 & Gynover, diu kurteise,\\ 
 & warp zen vrouwen \textbf{dise} \textbf{stolzen} reise.\\ 
 & Keie sprach \textbf{in sîme} zorn:\\ 
 & "wart aber ie sô werder man geborn,\\ 
 & get\textit{ö}rste ich des gelouben hân,\\ 
10 & sô von Norwæge Gawan.\\ 
 & ziu dar \textbf{nâher}, holt in dâ,\\ 
 & sô ist er lîhte anderswâ!\\ 
 & wil er wenken als ein eichorn,\\ 
 & ir mugt in schiere hân verlorn."\\ 
15 & Der knappe sprach zer künegîn:\\ 
 & "vrouwe, gein dem hêrren mîn\\ 
 & muoz ich balde kêren.\\ 
 & werbet sîn dinc nâch iweren êren."\\ 
 & Zeinem ir kamerære si sprach:\\ 
20 & "\textbf{schaffe} disem knappen \textbf{guot} gemach.\\ 
 & sîn \textbf{ors} solt dû schouwen.\\ 
 & sî daz mit sporn \textbf{verhouwen},\\ 
 & gib imz beste, daz \textbf{hie veile} sî.\\ 
 & won im ander kumber bî,\\ 
25 & ez sî pfantlôse oder kleit,\\ 
 & des sol er alles sîn bereit."\\ 
 & \textbf{Si sprach}: "\textbf{nû} sage Gawan,\\ 
 & \textbf{im} sî \textbf{mîn dienst} undertân.\\ 
 & urloup \textbf{ich dir zem künege} nim.\\ 
30 & dîme hêrren sag ouch dienst von im."\\ 
\end{tabular}
\scriptsize
\line(1,0){75} \newline
D \newline
\line(1,0){75} \newline
\textbf{1} \textit{Initiale} D  \textbf{15} \textit{Majuskel} D  \textbf{19} \textit{Majuskel} D  \textbf{27} \textit{Majuskel} D  \newline
\line(1,0){75} \newline
\textbf{9} getörste] getorste D \newline
\end{minipage}
\hspace{0.5cm}
\begin{minipage}[t]{0.5\linewidth}
\small
\begin{center}*m
\end{center}
\begin{tabular}{rl}
 & Gawans mâge, der rîch\textit{e}\\ 
 & Artus, warp \textbf{herzeclîche}\\ 
 & zer massenîe dise vart.\\ 
 & vor sûmen \textbf{het ouch si\textit{ch}} bewart\\ 
5 & G\textit{i}nover, diu kurteise,\\ 
 & \textbf{si} warp zen vrowen \textbf{dise} reise.\\ 
 & \begin{large}K\end{large}eie sprach \textbf{in sînem} zorn:\\ 
 & "wart aber ie sô werder man \textit{geb}orn,\\ 
 & getörste ich des glouben hân,\\ 
10 & sô von Norwæge Gawan.\\ 
 & ziu dar \textbf{nâher}, holt in dâ,\\ 
 & sô ist er lîht anderswâ!\\ 
 & wil er wenken als ein eichorn,\\ 
 & ir müget in schier hân verlorn."\\ 
15 & der knappe sprach zer künigîn:\\ 
 & "vrowe, gegen dem hêrren mîn\\ 
 & muoz ich balde kêren.\\ 
 & werbet sîn dinc nâch iuwern êren."\\ 
 & zuo eim ir kamerer \textit{si} sprach:\\ 
20 & "\textbf{schaffe} disem knappen \textbf{guot} gemach.\\ 
 & sîn \textbf{runzît} soltû schouwen.\\ 
 & sî daz mit sporn \textbf{zerhouwen},\\ 
 & gip imz beste, daz \textbf{dâ} sî.\\ 
 & won im \textbf{kein} ander kumber \textit{bî},\\ 
25 & ez sî pfantlôse oder kleit,\\ 
 & des sol er alles sîn bereit."\\ 
 & \textbf{si sprach}: "\textbf{nû} sage Gawan,\\ 
 & \textbf{im} sî \textbf{mîn dienst} undertân.\\ 
 & urloup \textbf{zem künige ich dir} nim.\\ 
30 & dînem hêrren sage ouch \textit{dienst} von im."\\ 
\end{tabular}
\scriptsize
\line(1,0){75} \newline
m n o Fr69 \newline
\line(1,0){75} \newline
\textbf{7} \textit{Initiale} m   $\cdot$ \textit{Capitulumzeichen} n  \newline
\line(1,0){75} \newline
\textbf{1} Gawans] :::wans Fr69  $\cdot$ rîche] richen m \textbf{2} Artus] :::us Fr69  $\cdot$ warp] [wart]: warp o \textbf{4} sûmen] sinem o  $\cdot$ ouch] er o  $\cdot$ sich] sẏ m \textbf{5} Ginover] Genofer m Genouer n o  $\cdot$ kurteise] corteise m \textbf{6} vrowen] frowe o \textbf{7} Keie] Keẏe n o \textbf{8} geborn] verlorn m \textbf{9} getörste] Geturste n (o) \textbf{10} Norwæge] norwege m n o \textbf{11} dâ] do n \textbf{12} lîht] liecht o \textbf{15} zer] derzeder Fr69 \textbf{16} hêrren] herrem Fr69 \textbf{19} ir] irem m n (o)  $\cdot$ si] er m \textit{om.} o \textbf{20} guot gemach] [vngemach]: guͦt gemach o \textbf{21} runzît] ruiczit o \textbf{22} zerhouwen] verhouwen n (o) \textbf{23} dâ] hie n o \textbf{24} won] Wenne n  $\cdot$ bî] \textit{om.} m \textbf{30} dienst] \textit{om.} m \newline
\end{minipage}
\end{table}
\newpage
\begin{table}[ht]
\begin{minipage}[t]{0.5\linewidth}
\small
\begin{center}*G
\end{center}
\begin{tabular}{rl}
 & \begin{large}G\end{large}awans mâc, der rîche\\ 
 & Artus, warp \textbf{höfschlîche}\\ 
 & zer messenîe dise vart.\\ 
 & \textbf{ouch was} vor sûmen \textbf{gar} bewart\\ 
5 & Schinover, diu kurteise,\\ 
 & warp ze den vrouwen \textbf{dise} reise.\\ 
 & Kay sprach \textbf{in sînem} zorn:\\ 
 & "wart aber ie sô wert ma\textit{n g}eborn,\\ 
 & get\textit{ö}rst ich des glouben hân,\\ 
10 & sô von Norwæge Gawan.\\ 
 & zehû dâ \textbf{hin}, \textbf{nû} holt in dâ,\\ 
 & sô ist er lîhte anderswâ!\\ 
 & wil er wenken als ein eichorn,\\ 
 & ir muget in schier hân verlorn."\\ 
15 & der knappe sprach ze der künegîn:\\ 
 & "vrouwe, gein dem hêrren mîn\\ 
 & muoz ich balde kêren.\\ 
 & werbet sîn dinc nâch iuwern êren."\\ 
 & zeinem ir kamerære si sprach:\\ 
20 & "\textbf{schaffen} disem knappen \textbf{guo\textit{t}} gemach.\\ 
 & sîn \textbf{ors} soltû schouwen.\\ 
 & sî daz mit sporen \textbf{verhouwen},\\ 
 & gip im daz beste, daz \textbf{hie veile} sî.\\ 
 & wone im ander kumber bî,\\ 
25 & ez sî pfantlôse oder kleit,\\ 
 & des sol er alles sîn bereit.\\ 
 & \textbf{geselle}, sage Gawan,\\ 
 & \textbf{ich} sî \textbf{im an dienste} undertân.\\ 
 & urloup \textbf{ich dir \textit{zuo} dem künige} nim.\\ 
30 & dînem hêrren sage ouch dienst von im."\\ 
\end{tabular}
\scriptsize
\line(1,0){75} \newline
G I L M Z Fr48 \newline
\line(1,0){75} \newline
\textbf{1} \textit{Initiale} G I L Z  \textbf{17} \textit{Initiale} I  \newline
\line(1,0){75} \newline
\textbf{1} Gawans] Gawansz L \textbf{2} Artus] Artuͯs L  $\cdot$ höfschlîche] helffeliche M herzenliche Z \textbf{4} vor sûmen] versumen I (Z) \textbf{5} kynover diu korteise G  $\cdot$ Schinover] Ginofer I Gýnover L Ginover M Gynofer Z Gynouer Fr48 \textbf{6} den] der Fr48  $\cdot$ dise] die stolzen L (M) (Fr48) dise stoltzen Z \textbf{7} Kay] kain I Keý L Keye M Key Z Fr48 \textbf{8} man geborn] man ie geborn G geborn Z \textbf{9} getörst] getorst G I L (M) Fr48 \textbf{10} Norwæge] norwage G I Norwege L M Z Fr48  $\cdot$ Gawan] her gawan Z \textbf{11} zehû] zuͦ I Zahiv L \textit{om.} M  $\cdot$ dâ hin] nu hin Z hin nuͦ her Fr48 \textbf{12} lîhte] lychte L leiht Fr48 \textbf{13} eichorn] korn M a::horn Fr48 \textbf{18} werbet] Werb L Werben Z  $\cdot$ sîn dinc] sint ich M  $\cdot$ iuwern] \textit{om.} Z \textbf{19} ir] \textit{om.} L Fr48 \textbf{20} schaffen] shafe I (M) (Z) Schaffet L  $\cdot$ guot] guetin G \textbf{21} ors] orse I \textbf{22} sî] Sit Fr48 \textbf{23} gip] So gib Fr48  $\cdot$ veile] \textit{om.} M \textbf{24} wone] Wont L Wan Z \textbf{26} des] daz I (M)  $\cdot$ er] im I  $\cdot$ alles] alle M \textbf{29} zuo] uon G  $\cdot$ nim] myn M \textbf{30} im] yn M \newline
\end{minipage}
\hspace{0.5cm}
\begin{minipage}[t]{0.5\linewidth}
\small
\begin{center}*T
\end{center}
\begin{tabular}{rl}
 & Gawans mâc, der rîche\\ 
 & Artus, warp \textbf{herzenlîche}\\ 
 & zuor massenîe dise vart.\\ 
 & vor sûmen \textbf{het ouch sich} bewart\\ 
5 & Gynover, diu kurteise,\\ 
 & warp zuo\textit{n} vrouwen \textbf{die} \textbf{stolzen} reise.\\ 
 & Key sprach \textbf{durch sînen} zorn:\\ 
 & "wart aber ie sô werder man geborn,\\ 
 & get\textit{ör}st ich des glouben hân,\\ 
10 & sô von Norwæge Gawan.\\ 
 & zâhiu dâ \textbf{hin}, \textbf{nû} holt in dâ,\\ 
 & sô ist er lîhte anderswâ!\\ 
 & wil er wenken als ein eichorn,\\ 
 & ir mo\textit{g}t in schier hân verlorn."\\ 
15 & der knabe sprach zuor künigîn:\\ 
 & "vrou, gên dem hêrren mîn\\ 
 & muoz ich balde kêren.\\ 
 & werbet sîn dinc nâch iuweren êren."\\ 
 & zuo einem ir kamerer si sprach:\\ 
20 & "\textbf{schaffe} disem knaben gemach.\\ 
 & sîn \textbf{ros} soltû schouwen.\\ 
 & sî daz mit sporn \textbf{verhouwen},\\ 
 & gip im daz beste, daz \textbf{hie veile} sî.\\ 
 & won im ander kumber bî,\\ 
25 & ez sî pfantlôse oder kleit,\\ 
 & des sol er alles sîn bereit."\\ 
 & \textbf{si sprach}: "\textbf{nû} sage Gawan,\\ 
 & \textbf{im} sî \textbf{mîn dienst} undertân.\\ 
 & urloup \textbf{ich dir zuom künige} nim.\\ 
30 & dînem hêrren sag ouch dienst von im."\\ 
\end{tabular}
\scriptsize
\line(1,0){75} \newline
Q R W V Fr40 \newline
\line(1,0){75} \newline
\textbf{1} \textit{Initiale} W V  \textbf{27} \textit{Initiale} R  \newline
\line(1,0){75} \newline
\textbf{1} Gawans] Gawins R \textbf{3} zuor] Ze R \textbf{4} Vor svmvnge hette oͮch [b*]: sich bewart V \textbf{5} Gynover] Gẏnouer Q Tschynouer W  $\cdot$ kurteise] [*]: kvrteise V \textbf{6} zuon] zwr Q  $\cdot$ die] dise V  $\cdot$ stolzen] stok R stoltze W (V) \textbf{7} Key] Kei Q Kay W [K*]: Kegin V  $\cdot$ durch sînen] [*]: in sinem V \textbf{9} getörst] Getrost Q Getorst W V  $\cdot$ des] daz V \textbf{10} Norwæge] norwege Q R norwaige W [norw*]: norwege V \textbf{11} Zahi dahin nv holnt [i*]: in da V  $\cdot$ zâhiu] Ziecht W  $\cdot$ dâ hin] \textit{om.} R  $\cdot$ in dâ] in do W \textbf{13} ein] in R  $\cdot$ eichorn] [*]: eichorn V \textbf{14} mogt] mocht Q \textbf{16} hêrren] herzen Fr40 \textbf{19} ir] irm R (V) o\textit{m. } W Fr40 \textbf{20} schaffe] Schaffet W  $\cdot$ knaben] knappen guͯt R (W) (V) (Fr40) \textbf{21} ros] roß das W  $\cdot$ schouwen] beschowen V \textbf{22} verhouwen] gehoͯwen R \textbf{23} veile sî] [weile]: veile sey Q [*]: si V \textbf{24} won im] Wonet im auch W \textbf{26} des sol er] Das sol im R (V) \textbf{27} nû sage] vnd auch W \textbf{30} ouch] den W \newline
\end{minipage}
\end{table}
\end{document}
