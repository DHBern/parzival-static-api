\documentclass[8pt,a4paper,notitlepage]{article}
\usepackage{fullpage}
\usepackage{ulem}
\usepackage{xltxtra}
\usepackage{datetime}
\renewcommand{\dateseparator}{.}
\dmyyyydate
\usepackage{fancyhdr}
\usepackage{ifthen}
\pagestyle{fancy}
\fancyhf{}
\renewcommand{\headrulewidth}{0pt}
\fancyfoot[L]{\ifthenelse{\value{page}=1}{\today, \currenttime{} Uhr}{}}
\begin{document}
\begin{table}[ht]
\begin{minipage}[t]{0.5\linewidth}
\small
\begin{center}*D
\end{center}
\begin{tabular}{rl}
\textbf{274} & \begin{large}D\end{large}ô zôch man der vrouwen wert\\ 
 & starc, wol gênde, \textbf{ein schœne} pfert,\\ 
 & gesatelt unt gezoumet wol.\\ 
 & man huop si drûf, diu rîten sol\\ 
5 & dannen mit ir küenem man.\\ 
 & sîn ors \textbf{wart} gewâpent sân,\\ 
 & reht als erz \textbf{gein} strîte reit.\\ 
 & sîn swert, dâ mit er des tages streit,\\ 
 & man \textbf{vorn} an den satel hienc.\\ 
10 & von vuoz ûf gewâpent gienc\\ 
 & Orilus zem orse sîn.\\ 
 & er spranc drûf vor der herzogîn.\\ 
 & \textbf{Jeschute} unt \textbf{er} vuoren dan zehant.\\ 
 & \textbf{sîne} messenîe gein Lalant\\ 
15 & \textbf{bat er alle} kêren,\\ 
 & wan ein rîter solt in lêren\\ 
 & gein Artuse rîten.\\ 
 & er bat daz volc des bîten.\\ 
 & Si kômen \textbf{Artuse} sô nâhen,\\ 
20 & daz si \textbf{sîniu} poulûn sâhen,\\ 
 & \textbf{vil nâhe} eine mîle daz wazzer nider.\\ 
 & der vürste sant den rîter wider,\\ 
 & der in gewîset hete dar.\\ 
 & vrou Jeschute, diu wol gevar,\\ 
25 & was sîn gesinde unt niemen mêr.\\ 
 & der unlôse Artus, niht ze hêr,\\ 
 & was gegangen, dô er \textbf{des âbents} geaz,\\ 
 & ûf \textbf{einen} plân. umb in dâ saz\\ 
 & \begin{large}D\end{large}iu werde massenîe.\\ 
30 & Orilus, der valsches vrîe,\\ 
\end{tabular}
\scriptsize
\line(1,0){75} \newline
D \newline
\line(1,0){75} \newline
\textbf{1} \textit{Initiale} D  \textbf{19} \textit{Majuskel} D  \textbf{29} \textit{Initiale} D  \newline
\line(1,0){75} \newline
\textbf{13} Jeschute] Jescv̂te D \textbf{24} Jeschute] Jescvte D \newline
\end{minipage}
\hspace{0.5cm}
\begin{minipage}[t]{0.5\linewidth}
\small
\begin{center}*m
\end{center}
\begin{tabular}{rl}
 & \begin{large}D\end{large}ô zôch man \textbf{dar} der vrouwen wert\\ 
 & star\textit{c}, wolgênde, \textbf{ein schœne} pfert,\\ 
 & gesatelt und gezoumet wol.\\ 
 & man huop si drûf, diu rîten sol\\ 
5 & dannen mit ir küenen man.\\ 
 & sîn ros \textbf{was} gewâpent sân,\\ 
 & reht als erz \textbf{in} strîte reit.\\ 
 & sîn swert, dâ mite er des tages streit,\\ 
 & man \textbf{vornân} an den satel hienc.\\ 
10 & von vuoze ûf gewâpent gienc\\ 
 & Orilus zem rosse sîn.\\ 
 & er spranc drûf vor der herzogîn.\\ 
 & \textbf{Jeschut\textit{e}} und \textbf{er} vuoren danne zehant.\\ 
 & \textbf{sîne} massenîe gegen Lalant\\ 
15 & \textbf{bat er alle} kêren,\\ 
 & wanne ein ritter solt in lêren\\ 
 & gegen Artuse rîten.\\ 
 & er bat daz volc des bîten.\\ 
 & \textbf{dô} si kômen sô nâhen,\\ 
20 & daz si \textbf{Artuses} pavelûne sâhen,\\ 
 & \textbf{vil nâch} ein mîle daz wa\textit{zz}er nider,\\ 
 & der vürste sante den ritter wider,\\ 
 & der in gewîset hete dar.\\ 
 & vrouwe Jeschute, diu wol gevar,\\ 
25 & was sîn gesinde und niemen mêr.\\ 
 & der unlôse Artus, niht ze hêr,\\ 
 & was gegangen, dô er geaz.\\ 
 & ûf \textbf{einem} plân umb in dô saz\\ 
 & diu werde massenîe.\\ 
30 & Orilus, der valsches vrîe,\\ 
\end{tabular}
\scriptsize
\line(1,0){75} \newline
m n o Fr69 \newline
\line(1,0){75} \newline
\textbf{1} \textit{Initiale} m   $\cdot$ \textit{Capitulumzeichen} n  \newline
\line(1,0){75} \newline
\textbf{2} starc] Starcke m \textbf{4} rîten] do riten n \textbf{5} man] [mach]: man m \textbf{7} in] zuͯ n (o) \textbf{8} er] do er o \textbf{9} vornân] furnam o \textbf{11} Orilus] Orelus o \textbf{13} Jeschute] Jescutten m Jescuten n Juscuͯte o \textbf{18} volc] fock o  $\cdot$ des] \textit{om.} n o \textbf{20} Artuses] artus m (n) (o) \textbf{21} wazzer] was er m n (o) \textbf{24} Jeschute] jescutte m jescute n o  $\cdot$ gevar] [gewar]: gevar o \textbf{25} was] Das n o \textbf{26} ze] so o \textbf{27} gegangen] gegangangen o \textbf{28} einem] einen n ein o \textbf{29} diu] Der o \newline
\end{minipage}
\end{table}
\newpage
\begin{table}[ht]
\begin{minipage}[t]{0.5\linewidth}
\small
\begin{center}*G
\end{center}
\begin{tabular}{rl}
 & dô zôch man der vrouwen wert\\ 
 & starc, wol gênde, \textbf{schône ein} pfert,\\ 
 & gesatelt unde gezoumet wol.\\ 
 & man huop si drûf, diu rîten sol\\ 
5 & dannen mit ir küenen man.\\ 
 & sîn ors \textbf{was} gewâpent sân,\\ 
 & reht als erz \textbf{gein} strîte reit.\\ 
 & sîn swert, dâ mit ers tages streit,\\ 
 & man \textbf{vor} an den satel hienc.\\ 
10 & von vuoze ûf gewâpent gienc\\ 
 & Orillus zem orse sîn.\\ 
 & er spranc drûf vor der herzogîn.\\ 
 & \multicolumn{1}{l}{ - - - }\\ 
 & \multicolumn{1}{l}{ - - - }\\ 
15 & \textbf{sîn gesinde er wider bat} kêren,\\ 
 & wan ein rîter solt in lêren\\ 
 & gein Artuse rîten.\\ 
 & er bat daz volc des bîten.\\ 
 & si kômen \textbf{Artus} sô nâhen,\\ 
20 & daz si \textbf{sîn} pavelûn sâhen,\\ 
 & \textbf{vil nâch} eine mîle daz wazzer nider.\\ 
 & der vürste sande den rîter wider,\\ 
 & der in gewîset hete dar.\\ 
 & vrou Jeschute, diu wolgevar,\\ 
25 & was sîn gesinde unde niemen mêr.\\ 
 & der unlôse Artus, niht ze hêr,\\ 
 & was gegangen, dô er\textbf{s âbendes} geaz,\\ 
 & ûf \textbf{einen} plân. umbe in dâ saz\\ 
 & diu werde messenîe.\\ 
30 & Orillus, der valsches vrîe,\\ 
\end{tabular}
\scriptsize
\line(1,0){75} \newline
G I O L M Q R Z Fr36 \newline
\line(1,0){75} \newline
\textbf{1} \textit{Initiale} I  \textbf{13} \textit{Initiale} I  \textbf{25} \textit{Initiale} Z  \textbf{29} \textit{Initiale} O Q Fr36  \newline
\line(1,0){75} \newline
\textbf{1} dô] Da O Z  $\cdot$ der] dar der L (R) Fr36 \textbf{2} starc sheͦn wol gende ein phert I  $\cdot$ Ein wol schone gende pfert O (M)  $\cdot$ Starch vnd schone gande ein phert L  $\cdot$ Starc wol gende ein schone pfert Q (R) (Z)  $\cdot$ ein wolgetan schoͤnes pfært Fr36 \textbf{3} gesatelt] Besaltet Fr36  $\cdot$ gezoumet] geczynieret M gezemet R \textbf{4} man huop] man hube Q so sas Fr36 \textbf{5} dannen] Da Fr36  $\cdot$ küenen] chvͦne O (Fr36) \textbf{6} was] wart O L M Q R Fr36 \textbf{7} erz] er I (M) er daz L  $\cdot$ reit] Rittett R \textbf{8} dâ mit ers] do er mit des Q \textbf{9} man] man ým L  $\cdot$ vor] wol Q \textbf{10} vuoze] fuͤzzen I  $\cdot$ gienc] er giench Fr36 \textbf{11} Orillus] Orilus I (O) M Q R Z Fr36 \textbf{12} vor] durch vor Q \textit{om.} Z  $\cdot$ herzogîn] chungin I (O) \textbf{13} \textit{Die Verse 274.13-14 fehlen} G   $\cdot$ Si shieden dannen alzehant I Jeschvͦte (Jescuͯte L Jescute M Q Z Jscute R ) vnde er (\textit{om.} Q ) schieden sich (\textit{om.} L M Q R Z ) dan zehant O (L) (M) (Q) (R) (Z) \textbf{14} die andren wider in ir lant I Die messenie gein lalant (delalant R ) O (L) (M) (R) (Z) Die massenie delalant Q \textbf{15} bat er alle chern I (O) (L) (M) (Q) (R) (Z) \textbf{16} ein rîter solt in] in solte ein ritter L eyn ritter yn sulde M ein ritter sol in Q \textbf{17} Artuse] artus I Q R Artuͯse L \textbf{18} daz volc des] des volches I isz volc des M dasz wolck Q \textbf{19} si] ein teil si I  $\cdot$ Artus] Artvse O (M) (Z) Artuͯse L :rtusen Fr36 \textbf{20} sîn pavelûn] sin geczeltten R die pavelun Z \textbf{21} nâch] nahen I (Q) (R) Fr36  $\cdot$ eine] einer L \textbf{22} den] da den I der Q \textit{om.} Fr36 \textbf{23} hete] hat R \textbf{24} Jeschute] ieschute G ieskute I Jeschvͦte O Jescuͯte L iescute M Q Z Juscute R jescute Fr36 \textbf{26} unlôse] [eder]: edel L  $\cdot$ Artus] Artvse O Artuͯs L  $\cdot$ niht] vnde nicht M \textbf{27} dô] da O Z Fr36 \textit{om.} Q  $\cdot$ ers] des Q \textbf{28} einen] eyn M  $\cdot$ plân] palen Q  $\cdot$ in dâ] in do L Q vnde M im do R \textbf{30} Orillus] Orilus I (O) M Q R Z (Fr36)  $\cdot$ der] des R  $\cdot$ valsches] ualkes I \newline
\end{minipage}
\hspace{0.5cm}
\begin{minipage}[t]{0.5\linewidth}
\small
\begin{center}*T
\end{center}
\begin{tabular}{rl}
 & Dô zôch man \textbf{dar} der vrouwen wert\\ 
 & starc, wol gânde, \textbf{schône ein} pfert,\\ 
 & gesatelt unde gezoumet wol.\\ 
 & man huop si drûf, diu rîten sol\\ 
5 & dannen mit ir küenen man.\\ 
 & sîn ors \textbf{was} gewâpent sân,\\ 
 & reht als erz \textbf{gegen} \textbf{dem} strîte reit.\\ 
 & sîn swert, dâ mit er des tages streit,\\ 
 & man \textbf{vorne} an den satel hienc.\\ 
10 & von vuoze ûf gewâpent gienc\\ 
 & Orilus zuo dem orse sîn.\\ 
 & er spranc drûf vor der herzogîn\\ 
 & \textbf{Jeschuten}, unde vuoren dan zehant.\\ 
 & \textbf{die} massenîe gegen Lalant\\ 
15 & \textbf{bat er alle} kêren,\\ 
 & wan ein rîter soltin lêren\\ 
 & gegen Artuse rîten.\\ 
 & er bat daz volc des bîten.\\ 
 & \begin{large}S\end{large}i kômen \textbf{Artuse} sô nâhen,\\ 
20 & daz si \textbf{sîne} pavelûn sâhen,\\ 
 & \textbf{wol} eine mîle daz wazzer nider.\\ 
 & Der vürste sante den \textit{rît}e\textit{r} wider,\\ 
 & der in gewîset hete dar.\\ 
 & vrou Jeschute, diu wol gevar,\\ 
25 & was sîn gesinde unde niemen mêr.\\ 
 & der unlôse Artus, niht ze hêr,\\ 
 & was gegangen, dôr \textbf{des âbendes} geaz,\\ 
 & ûf \textbf{einen} plân. umbin dâ saz\\ 
 & diu werde massenîe.\\ 
30 & Orilus, der valsches vrîe,\\ 
\end{tabular}
\scriptsize
\line(1,0){75} \newline
T U V W \newline
\line(1,0){75} \newline
\textbf{1} \textit{Initiale} W   $\cdot$ \textit{Majuskel} T  \textbf{19} \textit{Überschrift: } Hie kam orilus zvͦ kv́nig artus hof do er mit sinem wibe iescuten versvͤnet wart vnde parzifal in des twang V   $\cdot$ \textit{Initiale} T U V  \textbf{22} \textit{Majuskel} T  \newline
\line(1,0){75} \newline
\textbf{1} man] \textit{om.} U  $\cdot$ dar] [*]: dar V \textit{om.} W \textbf{2} wol gânde schône] wol schone gande ein U (V) wolgond ein schons W \textbf{4} diu] dies W \textbf{5} küenen] kvͤnem V \textbf{6} was] wart V \textbf{7} als] sam W  $\cdot$ erz] er iz U er W  $\cdot$ gegen dem] [*]: in V \textbf{9} man] Da V  $\cdot$ vorne] vor in U  $\cdot$ den] dem U V \textbf{10} vuoze] vuͦzen U \textbf{11} zuo] ginc zu U [*]: aldo zvͦ V \textbf{13} Jeschuten] Jescvten T Jescute U V Iestute W  $\cdot$ unde] [*]: vnde er V vnd er W \textbf{14} die] [*]: Sine V \textbf{15} kêren] widerkeren W \textbf{17} Artuse] Artusse U \textbf{20} pavelûn] gezelt V \textbf{22} den] drei W  $\cdot$ rîter] vursten T \textbf{24} Jeschute] Jescvte T Jescuͦte U iescute V iestute W  $\cdot$ diu] \textit{om.} U [*]: die V \textbf{26} niht ze hêr] vnd nit [*]: zuͦ her U [*]: niht zeher V \textbf{27} des âbendes geaz] do saß W \textbf{28} umbin dâ saz] do mit im des abends aß W  $\cdot$ dâ] do V \textbf{29} werde] húbsche werde W \textbf{30} der] des W  $\cdot$ valsches] valsche U \newline
\end{minipage}
\end{table}
\end{document}
