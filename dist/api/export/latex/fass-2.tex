\documentclass[8pt,a4paper,notitlepage]{article}
\usepackage{fullpage}
\usepackage{ulem}
\usepackage{xltxtra}
\usepackage{datetime}
\renewcommand{\dateseparator}{.}
\dmyyyydate
\usepackage{fancyhdr}
\usepackage{ifthen}
\pagestyle{fancy}
\fancyhf{}
\renewcommand{\headrulewidth}{0pt}
\fancyfoot[L]{\ifthenelse{\value{page}=1}{\today, \currenttime{} Uhr}{}}
\begin{document}
\begin{table}[ht]
\begin{minipage}[t]{0.5\linewidth}
\small
\begin{center}*D
\end{center}
\begin{tabular}{rl}
\textbf{2} & wil ich triwe vinden,\\ 
 & al dâ si kan verswinden\\ 
 & \textbf{als} viwer in \textbf{den} brunnen\\ 
 & und daz tou \textbf{an} der sunnen?\\ 
5 & \textbf{noch} erkante ich nie sô wîsen man,\\ 
 & \textbf{de\textit{r}} m\textit{ö}hte gerne \textit{kun}de hân,\\ 
 & welher \textbf{stiure} \textbf{disiu} \textbf{mære} gernt\\ 
 & und waz si guoter lê\textit{re} wernt.\\ 
 & dâr an si ni\textit{mer} \textbf{des} verzagent,\\ 
10 & beide si vliehent unde jagent,\\ 
 & si \textbf{entwîchent} unde kêrent,\\ 
 & si lasternt und \textit{êre}nt.\\ 
 & swer mit \textbf{disen} schan\textit{z}en allen kan,\\ 
 & an dem hât witze wol getân,\\ 
15 & der sich niht \textbf{versitzet} \textbf{noch} \textbf{vergêt}\\ 
 & und sich anders wol \textbf{verstêt}.\\ 
 & valsch geselleclîcher muot\\ 
 & ist \textbf{dem} helle viure guot\\ 
 & unde hôher werdecheit ein hagel.\\ 
20 & \textbf{sîn} \textbf{triuwe} hât sô kurzen zagel,\\ 
 & daz si den dritten biz niht galt,\\ 
 & v\textit{üe}r si \textbf{bî} bremen in den walt.\\ 
 & \textit{\begin{large}D\end{large}}is\textit{e} maneger slahte \textit{underbint}\\ 
 & \textbf{iedoch} niht gar von \textbf{m\textit{an}ne} sint.\\ 
25 & vür diu wîp stôze ich di\textit{siu zil}:\\ 
 & swelhiu \textbf{mîn râten merken} wil,\\ 
 & diu sol w\textit{iz}zen, war si kêre\\ 
 & ir prîs \textbf{und ir êre}\\ 
 & und wem si \textbf{dâ nâch} sî bereit\\ 
30 & \textbf{minne} und ir werdecheit,\\ 
\end{tabular}
\scriptsize
\line(1,0){75} \newline
D \newline
\line(1,0){75} \newline
\textbf{23} \textit{Initiale} D  \newline
\line(1,0){75} \newline
\textbf{1} triwe vinden] tri*nden \textit{nachträglich korrigiert zu:} triwe finden D \textbf{2} verswinden] ve*winden \textit{nachträglich korrigiert zu:} verswinden D \textbf{4} sunnen] sunn* \textit{nachträglich korrigiert zu:} sunnen D \textbf{5} noch] *ôh \textit{nachträglich korrigiert zu:} nôh D \textbf{6} der] *n \textit{nachträglich korrigiert zu:} den D  $\cdot$ möhte] mohte D  $\cdot$ kunde] :::de D \textbf{8} lêre] le:: D \textbf{9} nimer] ni::: D \textbf{12} êrent] :::nt D \textbf{13} schanzen] schan*n \textit{nachträglich korrigiert zu:} schanden D \textbf{14} hât] ha* \textit{nachträglich korrigiert zu:} hat D \textbf{16} wol] *l \textit{nachträglich korrigiert zu:} wol D \textbf{22} vüer] fvͦr D \textbf{23} Dise] ÷isen D  $\cdot$ underbint] ::: D \textbf{24} manne] m::ne D \textbf{25} disiu zil] di::: D \textbf{27} wizzen] wîzen D  $\cdot$ kêre] chê:e \textit{nachträglich korrigiert zu:} chêre D \newline
\end{minipage}
\hspace{0.5cm}
\begin{minipage}[t]{0.5\linewidth}
\small
\begin{center}*m
\end{center}
\begin{tabular}{rl}
 & wil ich triuwe vinden,\\ 
 & aldâ si kan verswinden\\ 
 & \textbf{alsô} viur in \textbf{den} brunnen\\ 
 & und der tou \textbf{von} der sunnen?\\ 
5 & \textbf{doch} erkant ich nie sô wîsen man,\\ 
 & \textbf{er} möhte gerne k\textit{u}nde hân,\\ 
 & welher \textbf{t\textit{iu}re} \textbf{die} \textbf{vr\textit{o}wen} gerent\\ 
 & und waz si guoter lêre werent.\\ 
 & dâr an si niemer verzagent,\\ 
10 & beide si vliehent und jagent,\\ 
 & si \textbf{entwîchent} und kêrent,\\ 
 & si lasterent und êrent.\\ 
 & wer mit \textbf{disen} schanzen allen kan,\\ 
 & an dem hât witze wol getân,\\ 
15 & der sich niht \textbf{verstrêt} \textbf{und} \textbf{verstât}\\ 
 & und sich anders wol \textbf{vergât}.\\ 
 & valsch geselleclîcher muot\\ 
 & ist \textbf{zuo} \textbf{der} helle viure guot\\ 
 & und \textbf{ist} hôher wirdicheit ein hagel.\\ 
20 & \textbf{untriuwe} het sô kurzen zagel,\\ 
 & daz si den dritten biz niht galt,\\ 
 & vüere si \textbf{mit} bre\textit{m}en in den walt.\\ 
 & dise meniger slahte underbint\\ 
 & \textbf{doch} niht gar von \textbf{mannen} sint.\\ 
25 & vür diu wîp stôze ich disiu zil:\\ 
 & weliu \textbf{hie} \textbf{mîn râten merken} wil,\\ 
 & diu sol wizzen, war si kêre\\ 
 & ir brîs \textbf{und ir êre}\\ 
 & und wem si \textbf{dâ nâch} sî bereit\\ 
30 & \textbf{minne} und ir wirdicheit,\\ 
\end{tabular}
\scriptsize
\line(1,0){75} \newline
m n o W \newline
\line(1,0){75} \newline
\textbf{23} \textit{Initiale} n o W  \newline
\line(1,0){75} \newline
\textbf{2} si kan] kan sú n (o) (W) \textbf{3} alsô] [Aldo]: Also m  $\cdot$ viur] vor o  $\cdot$ den] dem W \textbf{6} kunde] kinde m W \textbf{7} tiure] toͯre m  $\cdot$ vrowen] frewen m \textbf{9} niemer] nẏemer des n (o) (W) \textbf{12} êrent] lerent o \textbf{13} allen] wol n \textbf{15} verstrêt] verstet n ustert o versinnet W  $\cdot$ und] noch n o \textbf{16} wol] nit W \textbf{18} helle viure] hellen fúr n (o) hellen W \textbf{20} het] hette o \textbf{21} den] der o \textbf{22} vüere] Euͤr W  $\cdot$ bremen] brenen m \textbf{23} meniger] manige n (W) \textbf{25} diu] dise o  $\cdot$ disiu] hie disz n \textbf{26} merken] volgen n \textbf{27} diu] Dise n \textbf{29} dâ nâch] noch do W \textbf{30} ir] \textit{om.} n o W \newline
\end{minipage}
\end{table}
\newpage
\begin{table}[ht]
\begin{minipage}[t]{0.5\linewidth}
\small
\begin{center}*G
\end{center}
\begin{tabular}{rl}
 & wil ich triwe vinden,\\ 
 & al dâ si kan verswinden\\ 
 & \textbf{sam} \textbf{daz} viur in \textbf{dem} brunnen\\ 
 & unt daz tou \textbf{von} der sunnen?\\ 
5 & \textbf{ouch} erkande ich nie sô wîsen man,\\ 
 & \textbf{er} \textbf{en}m\textit{ö}hte gerne kunde hân,\\ 
 & welher \textbf{stiure} \textbf{disiu} \textbf{mære} gernt\\ 
 & unde waz si guoter lêre wernt.\\ 
 & dâr an si nime\textit{r} \textbf{des} verzagent,\\ 
10 & beidiu si vliehent und jagent,\\ 
 & s\textit{i} \textbf{\textit{e}ntwîchent} und kêrent,\\ 
 & si lasterent und êrent.\\ 
 & swer mit \textbf{den} schanzen allen kan,\\ 
 & an dem hât witze wol getân,\\ 
15 & der sich niht \textbf{versitzet} \textbf{noch} \textbf{vergêt}\\ 
 & unde sich \textbf{doch} anders wol \textbf{verstêt}.\\ 
 & valsch geselleclîcher muot\\ 
 & ist \textbf{z}\textbf{em} helle viure guot\\ 
 & unde \textbf{ist} hôher werdecheit ein hagel.\\ 
20 & \textbf{sîn} \textbf{triwe} hât sô kurzen zagel,\\ 
 & daz si den driten biz niht galt,\\ 
 & vüere si \textbf{mit} bremen in den walt.\\ 
 & dise maniger slahte underbint\\ 
 & \textbf{iedoch} niht gar von \textbf{manne} sint.\\ 
25 & vür diu wîp stôze ich disiu zil:\\ 
 & swelhiu \textbf{mîn râten \textit{m}erken} wil,\\ 
 & diu sol wizzen, war si kêre\\ 
 & ir brîs \textbf{und ir êre}\\ 
 & unde wem si \textbf{dâ nâch} sî bereit\\ 
30 & \textbf{minne} und ir werdecheit,\\ 
\end{tabular}
\scriptsize
\line(1,0){75} \newline
G O L M Q Z Fr58 \newline
\line(1,0){75} \newline
\textbf{1} \textit{Initiale} O L M Z Fr58  \textbf{25} \textit{Initiale} L  \newline
\line(1,0){75} \newline
\textbf{1} wil] ÷il O Fr58 Als Q Uil Z \textbf{2} al] \textit{om.} L \textbf{3} Seyn vor den juͤden brunnen M  $\cdot$ daz] \textit{om.} O L Q Z Fr58 \textbf{4} daz] der L Q Fr58 clar M  $\cdot$ von] an Q vor Z Fr58  $\cdot$ sunnen] sunden Q \textbf{5} ouch] Doch Q  $\cdot$ erkande] [ny]: en kunde M  $\cdot$ nie] mer \textit{nachträglich korrigiert zu:} nie Q \textbf{6} er enmöhte] eren mohte G (O) (Q) Er mohte L (M) Z Fr58  $\cdot$ kunde] chvnden O \textbf{7} welher] Wil ir M \textbf{8} lêre] mere L  $\cdot$ wernt] weret Q \textbf{9} dâr] Das Q  $\cdot$ nimer] nime: G mynner M  $\cdot$ verzagent] vszagen M verzaget Q \textbf{10} vliehent] fliegent O  $\cdot$ jagent] sie iagent Z \textbf{11} si entwîchent] si ent si entwichent G \textbf{13} swer] Der O L M Z Wer Q  $\cdot$ den schanzen allen] allen schantzen L \textbf{14} dem] den M  $\cdot$ witze] wirde L \textbf{15} niht] \textit{om.} L \textbf{16} doch anders] anders dez L anders Q Z  $\cdot$ wol] vol Q \textbf{17} valsch] Valchs O Als Q \textbf{18} zem] zcu der M (Q)  $\cdot$ viure] vil M \textbf{20} sô] eynen M (Q) \textbf{22} vüere] Fvͦr O (Q)  $\cdot$ bremen] brennen Q (Z) \textbf{23} dise] Dieser Q  $\cdot$ maniger slahte] manigeslahte O \textbf{24} iedoch] Dy doch Q  $\cdot$ manne] minnen O mannen Q (Z) \textbf{25} diu] dyn Q \textbf{26} swelhiu] Welche L Q So wer M  $\cdot$ râten] rat O L (M) Z rede Q  $\cdot$ merken] :erchen G \textbf{27} sol wizzen] wisse L  $\cdot$ kêre] keren Q \textbf{29} wem] wan Q  $\cdot$ dâ nâch sî] sý dar nach L \textbf{30} minne] Synne M \newline
\end{minipage}
\hspace{0.5cm}
\begin{minipage}[t]{0.5\linewidth}
\small
\begin{center}*T
\end{center}
\begin{tabular}{rl}
 & Wil ich triuwe vinden,\\ 
 & aldâ si kan verswinden\\ 
 & \textbf{sam} viur in \textbf{dem} brunnen\\ 
 & und der tou \textbf{vor} der sunnen?\\ 
5 & \textbf{Ouch} erkand ich nie sô wîsen man,\\ 
 & \textbf{er}\textbf{n} m\textit{ö}hte gerne k\textit{u}nde hân,\\ 
 & welher \textit{\textbf{stiure}} \textbf{dise} \textbf{mære} gernt\\ 
 & und waz s\textit{i} \textbf{\textit{ouc}h} guoter lêre wernt.\\ 
 & dâr an si niemer \textbf{des} verzagent,\\ 
10 & beide si vliehent und jagent,\\ 
 & si \textbf{entwenkent} und kêrent,\\ 
 & si lesternt und êrent.\\ 
 & swer mit \textbf{den} schanz\textit{e}n allen kan,\\ 
 & an dem hât witze wol getân,\\ 
15 & der sich niht \textbf{versitzet} \textbf{noch} \textbf{vergêt}\\ 
 & und sich \textbf{doch} anders wol \textbf{verstêt}.\\ 
 & Valsch geselleclîcher muot\\ 
 & ist \textbf{z}\textbf{em} helle viure guot\\ 
 & und \textbf{ist} hôher werdecheit ein hagel.\\ 
20 & \textbf{sîn} \textbf{triuw\textit{e}} hât sô kurzen zagel,\\ 
 & daz si den driten biz niht galt,\\ 
 & vüer si \textbf{mit} bremen in den walt.\\ 
 & \begin{large}D\end{large}ise maneger slahte underbint\\ 
 & \textbf{iedoch} niht gar von \textbf{mannen} sint.\\ 
25 & vür di\textit{u} wîp stôz ich disiu zil:\\ 
 & swelhiu \textbf{mînen rât hœren} wil,\\ 
 & diu sol wizzen, war si kêre,\\ 
 & \textbf{sô daz s}ir prîs \textbf{gemêre},\\ 
30 & \hspace*{-.7em}\big| \textbf{ir êre} und ir werdecheit,\\ 
 & \hspace*{-.7em}\big| und wem s\textbf{ir minne} sî bereit,\\ 
\end{tabular}
\scriptsize
\line(1,0){75} \newline
T U V Fr32 \newline
\line(1,0){75} \newline
\textbf{1} \textit{Majuskel} T  \textbf{5} \textit{Majuskel} T  \textbf{9} \textit{Versal} Fr32  \textbf{11} \textit{Initiale} Fr32  \textbf{17} \textit{Majuskel} T  \textbf{23} \textit{Initiale} T U V Fr32  \newline
\line(1,0){75} \newline
\textbf{1} Wil ich] Ob ich wil V  $\cdot$ triuwe] [truwen]: truwe V \textbf{3} brunnen] bruͦnne U \textbf{4} vor] von U Fr32 [vo*]: von V \textbf{6} ern möhte] ern mohte T Er mohte U (Fr32)  $\cdot$ kunde] k:nde T \textbf{7} welher] Wil er U  $\cdot$ stiure] \textit{om.} T \textbf{8} si ouch] s:::h T  $\cdot$ wernt] gernt U [gernt]: wernt Fr32 \textbf{10} beide si] beidiv Fr32 \textbf{11} si] Die Fr32  $\cdot$ und] noch Fr32 \textbf{13} schanzen] schanz:n T \textbf{15} niht] \textit{om.} Fr32 \textbf{16} anders] \textit{om.} U anderz Fr32 \textbf{18} zem] zuͦ eim U zer V \textbf{20} triuwe] triuwer T \textbf{21} driten] [triten]: driten T \textbf{22} vüer] Vur U \textbf{23} Dise] Diseu Fr32  $\cdot$ maneger slahte] maniger slahten U (Fr32) \textbf{24} gar] \textit{om.} U \textbf{25} diu] die T  $\cdot$ disiu] dise V \textbf{26} swelhiu] Welhe U (V) swehiv Fr32 \textbf{28} Jren pris vnd ire ere V  $\cdot$ ir lip ir pris vnde ir ere Fr32  $\cdot$ sô daz sir] Da sie ir U \textbf{30} \textit{Versfolge 2.29-30} V Fr32   $\cdot$ Jrre minne irre werdickeit V  $\cdot$ ir minne vnd ir werdekeit Fr32 \textbf{29} Vnd wemme sv́ do nach si bereit V (Fr32)  $\cdot$ sir minne] uͦr minne dannoch U \newline
\end{minipage}
\end{table}
\end{document}
