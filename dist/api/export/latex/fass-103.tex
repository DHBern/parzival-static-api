\documentclass[8pt,a4paper,notitlepage]{article}
\usepackage{fullpage}
\usepackage{ulem}
\usepackage{xltxtra}
\usepackage{datetime}
\renewcommand{\dateseparator}{.}
\dmyyyydate
\usepackage{fancyhdr}
\usepackage{ifthen}
\pagestyle{fancy}
\fancyhf{}
\renewcommand{\headrulewidth}{0pt}
\fancyfoot[L]{\ifthenelse{\value{page}=1}{\today, \currenttime{} Uhr}{}}
\begin{document}
\begin{table}[ht]
\begin{minipage}[t]{0.5\linewidth}
\small
\begin{center}*D
\end{center}
\begin{tabular}{rl}
\textbf{103} & si kêrte ir herze an guote kunst.\\ 
 & des bejagete si der \textbf{werlde} gunst.\\ 
 & \multicolumn{1}{l}{ - - - }\\ 
 & \multicolumn{1}{l}{ - - - }\\ 
5 & ir kiusche was vür prîs erkant.\\ 
 & küneginne über driu lant,\\ 
 & Waleis und Anschouwe,\\ 
 & dâr über was si vrouwe.\\ 
 & si truog ouch krône ze Norgals\\ 
10 & in der houbtstat ze Kingrivals.\\ 
 & ir was ouch wol sô \textbf{liep ir} man,\\ 
 & ob ie kein vrouwe \textbf{mêr} gewan\\ 
 & sô \textbf{werden} vriunt, waz war ir daz?\\ 
 & si \textbf{mohte}z lâzen âne haz.\\ 
15 & dô er ûze beleip ein halbez jâr,\\ 
 & sînes komens warte si vür wâr.\\ 
 & \textbf{daz} \textbf{wart} ir lîpgedinge.\\ 
 & dô brast ir vreuden klinge\\ 
 & \textbf{mitten} ime hefte enzwei.\\ 
20 & owê unt heiâ hei,\\ 
 & daz \textbf{güete} al sölhen \textbf{kumber} tregt\\ 
 & unt immer \textbf{triwe} jâmer \textbf{regt}!\\ 
 & alsus vert diu menscheit:\\ 
 & hiute \textbf{vreude}, morgen leit.\\ 
25 & \begin{large}D\end{large}iu vrouwe umb einen mitten tac\\ 
 & eines angestlîchen slâfes pflac.\\ 
 & ir kom ein vorhtlîcher schric.\\ 
 & si dûhte, wie \textbf{ein sternenblic}\\ 
 & si gein den lüften vuorte,\\ 
30 & dâ si mit kreften ruorte\\ 
\end{tabular}
\scriptsize
\line(1,0){75} \newline
D \newline
\line(1,0){75} \newline
\textbf{25} \textit{Initiale} D  \newline
\line(1,0){75} \newline
\textbf{3} \textit{Die Verse 103.3-4 fehlen} D  \textbf{7} Waleis] Waleys D  $\cdot$ Anschouwe] Anscoͮwe D \newline
\end{minipage}
\hspace{0.5cm}
\begin{minipage}[t]{0.5\linewidth}
\small
\begin{center}*m
\end{center}
\begin{tabular}{rl}
 & si kêrte ir herze an guote kunst.\\ 
 & des bejagete si der \textbf{werde} gunst.\\ 
 & \multicolumn{1}{l}{ - - - }\\ 
 & \multicolumn{1}{l}{ - - - }\\ 
5 & ir kiusche was vür prîs erkant.\\ 
 & küniginne über driu lant,\\ 
 & Waleis und Anschouwe,\\ 
 & dâr über was si vrouwe.\\ 
 & si truoc ouch krône ze Norgals\\ 
10 & in der houbetstat ze \textit{K}ingr\textit{i}vals.\\ 
 & ir was ouch wol sô \textbf{liep ir} man,\\ 
 & obe ie kein vrouwe \textbf{mêr} gewan\\ 
 & sô \textbf{werden} vriunt, waz war ir daz?\\ 
 & si \textbf{möhte} ez lâzen âne haz.\\ 
15 & dô er ûze bleip ein halbez jâr,\\ 
 & sînes komens warte si vür wâr.\\ 
 & \textbf{daz} \textbf{was} ir lîpgedinge.\\ 
 & dô brast ir vröuden klinge\\ 
 & \textbf{mitte\textit{n}} ime hefte enzwei.\\ 
20 & owê und heiâ hei,\\ 
 & daz \textbf{minne} alsolichen \textbf{kumber} treget\\ 
 & und iemer \textbf{triuwe} jâmer \textbf{reget}!\\ 
 & alsus vert diu menscheit:\\ 
 & hiute \textbf{liep} \textbf{und} morgen leit.\\ 
25 & \begin{large}D\end{large}iu vrouwe umb einen mitten tac\\ 
 & eines angestlîche\textit{n} sl\textit{âf}es pflac.\\ 
 & ir kam ein vorhtlîcher schric.\\ 
 & si dûhte, wie \textbf{ein sternenblic}\\ 
 & si gegen de\textit{n} \textit{l}üften vuorte,\\ 
30 & dâ si mit kreften ruorte\\ 
\end{tabular}
\scriptsize
\line(1,0){75} \newline
m n o \newline
\line(1,0){75} \newline
\textbf{25} \textit{Initiale} m   $\cdot$ \textit{Capitulumzeichen} n  \newline
\line(1,0){75} \newline
\textbf{2} des] Das n o  $\cdot$ bejagete si der werde] bejaget sider werden n (o) \textbf{3} \textit{Die Verse 103.3-4 fehlen} m n o  \textbf{5} ir] Jr p n \textbf{7} Waleis] Wileis o  $\cdot$ Anschouwe] an schouwe m anschouwe n anschowe o \textbf{10} Kingrivals] kringruals m krungruals n konigen als o \textbf{12} vrouwe] man vnd frouwe n \textbf{17} was] [wart]: was m \textbf{19} mitten] Mittem m \textbf{20} heiâ] [hercze]: heẏa o \textbf{21} alsolichen] sollichen n (o)  $\cdot$ treget] tragent o \textbf{22} \textit{Versdoppelung (mit Anteil aus Vers 103.23):} Vnd iemer truͯwe iamer regent / Alsus obene iemer truwe reget o  \textbf{24} und] \textit{om.} n o  $\cdot$ morgen] [morn]: morngen o \textbf{25} mitten tac] mittag o \textbf{26} angestlîchen slâfes] angeslichens slosses m \textbf{28} dûhte] duͯhte o \textbf{29} den lüften] den ge luftten m  $\cdot$ vuorte] [furste]: furte o \textbf{30} dâ] Do n o \newline
\end{minipage}
\end{table}
\newpage
\begin{table}[ht]
\begin{minipage}[t]{0.5\linewidth}
\small
\begin{center}*G
\end{center}
\begin{tabular}{rl}
 & si kêrte ir herze an guote kunst.\\ 
 & des bejagte si der \textbf{werlte} gunst,\\ 
 & \textit{vrô} \textit{Herzeloide}, diu \textit{k}ünigîn.\\ 
 & ir site an lobe vant gewin.\\ 
5 & ir \textit{k}i\textit{usch}e was vür brîs erkant,\\ 
 & \textbf{der} künigîn über driu lant,\\ 
 & Waleis und Anschouwe,\\ 
 & dâr über was si vrouwe.\\ 
 & si truoc ouch krône ze Nurgals\\ 
10 & in der houbetstat ze Kinrivals.\\ 
 & ir was ouch wol sô \textbf{liep ir} man,\\ 
 & obe ie dehein vrouwe \textbf{mê} gewan\\ 
 & sô \textbf{lieben} vriunt, waz war ir daz?\\ 
 & si \textbf{mohte}z lâzen âne haz.\\ 
15 & dô er ûze beleip ein halbez jâr,\\ 
 & sînes komens wart si vür wâr.\\ 
 & \textbf{daz} \textbf{was} ir lîpgedinge.\\ 
 & dô brast ir vröuden klinge\\ 
 & \textbf{enmitten} in dem hefte enzwei.\\ 
20 & owê und heiâ hei,\\ 
 & daz \textbf{guot} alsolhen \textbf{kumber} treit\\ 
 & unde imer \textbf{triwe} jâmer \textbf{weit}!\\ 
 & alsus vert diu menscheit:\\ 
 & hiute \textbf{liep}, morgen leit.\\ 
25 & \begin{large}D\end{large}iu vrouwe umbe einen mitten tac\\ 
 & eines angestlîchen slâfes pflac.\\ 
 & ir kom ein vorhtlîcher schric.\\ 
 & si dûhte, wie \textbf{ein sternenblic}\\ 
 & si gein den lüften vuorte,\\ 
30 & dâ si mit kreften ruorte.\\ 
\end{tabular}
\scriptsize
\line(1,0){75} \newline
G I O L M Q R Z Fr21 Fr36 \newline
\line(1,0){75} \newline
\textbf{1} \textit{Initiale} O L  \textbf{5} \textit{Initiale} I  \textbf{15} \textit{Capitulumzeichen} L  \textbf{25} \textit{Initiale} G I L M Q R Z Fr21  \newline
\line(1,0){75} \newline
\textbf{1} si] ÷i O  $\cdot$ kêrte] chert I O (Fr21) \textbf{2} bejagte] beiagt Z beiach Fr21 \textbf{3} div vil reine chungin G  $\cdot$ Herzeloide] herzelaude I herzenlavde O Hertzelauͯde L herczeloide M herzeloude Q herczeleide R herzelovde Z Herzeloͮde Fr21 Fr36 \textbf{4} site] seyt Q hertze Z  $\cdot$ vant] hant I \textbf{5} kiusche] site G  $\cdot$ was] wart L (Q)  $\cdot$ brîs] pris pris L \textbf{6} der] \textit{om.} I Z Dîv O (L)  $\cdot$ driu] [diuͤ]: driuͤ I \textbf{7} Waleis] walois I Walaẏs L  $\cdot$ Anschouwe] anschoͮwe G antschoͮe I anschawe O Antschowe L ascouwe M anshowe Q Z anschowe R (Fr21) \textbf{8} \textit{Vers 103.8 fehlt} I   $\cdot$ si] sin M \textbf{9} si] Zu Q  $\cdot$ ze] zer I  $\cdot$ Nurgals] norgals I (Z) Nvͦrgals O Norgalis L Nurgilas M nűrgals Q \textbf{10} Kinrivals] [kungrigals]: kungrigvals I Gingrivals O kingrivals L Fr21 kingervals M kingriuals Q Kingriwals R kingrvals Z \textbf{11} ouch wol] auch I (Z) o\textit{m. } R \textbf{12} ie] her y M \textit{om.} R  $\cdot$ vrouwe] vrouwen M  $\cdot$ mê] E L men Q \textit{om.} Fr21 \textbf{13} lieben] werden O L (M) Q R Z Fr21  $\cdot$ vriunt] frvnde L  $\cdot$ war ir] wirret I war iz M \textbf{14} mohtez] moͮhtez G mochtiz wol M (R) (Z) (Fr21) mochten wol Q \textbf{15} dô] swen I Da Z  $\cdot$ ûze] vssen R (Fr21) \textbf{16} komens] komendes R  $\cdot$ wart] warte M Q Z \textbf{18} dô] Da M Z  $\cdot$ brast ir] braste irre Q  $\cdot$ klinge] klingen Q gelinge Fr21 \textbf{19} enmitten] Miten O (M) (Q) (Z) (Fr21)  $\cdot$ in] vz Fr21 \textbf{20} owê] Awi I Awe O Owý L (M) (R)  $\cdot$ heiâ] hera L eia M (Q) \textbf{21} guot] gute L Q (R) (Z)  $\cdot$ alsolhen] alsule M soͯlichen R ansolchen Fr36 \textbf{22} \textit{Vers 103.22 fehlt} Z   $\cdot$ imer] reiner I keyner M iamer Q  $\cdot$ weit] weiget G reget I (O) L (M) (Q) (Fr21) Fr36 wegt R \textbf{23} vert] stat Fr21  $\cdot$ diu] disev Fr21 \textbf{24} liep] vro I L vrevde O (Z) (Fr21) (Fr36) fry M \textbf{26} angestlîchen] anslichen I angstilichs Fr21  $\cdot$ slâfes] slafin M  $\cdot$ pflac] si phlach O pfack Q \textbf{28} ein sternenblic] ein stern blic I (L) (M) (Z) ein sternes blich O eines sternes plick Q (Fr21) eins sternen blik R \textbf{29} den] \textit{om.} M Z Fr21 \textbf{30} dâ] Do Q Daz Z  $\cdot$ ruorte] [fur]: rurte Q \newline
\end{minipage}
\hspace{0.5cm}
\begin{minipage}[t]{0.5\linewidth}
\small
\begin{center}*T (U)
\end{center}
\begin{tabular}{rl}
 & si kêrte ir herze an guote kunst.\\ 
 & des bejagete si der \textbf{werlte} gunst,\\ 
 & vrou Herzeloyde, diu künegîn.\\ 
 & ir site an lobe vant gewin.\\ 
5 & ir kiusche was vür prîs erkant,\\ 
 & \textbf{der} küniginne über driu lant,\\ 
 & Waleis und Anschouwe,\\ 
 & dâr über was s\textit{i} vrouwe.\\ 
 & si truoc ouch krônen zuo Nurgals\\ 
10 & in der houbetstat zuo Kingrivals.\\ 
 & ir was ouch wol sô \textbf{lieber} man,\\ 
 & ob ie dekein vrouwe gewan\\ 
 & sô \textbf{werden} vriunt, waz war ir daz?\\ 
 & si \textbf{moht} ez \textbf{wol} lâzen âne haz.\\ 
15 & dô er ûz bleip ein halbez jâr,\\ 
 & sînes komens warte si vür wâr.\\ 
 & \textbf{diz} \textbf{was} ir lîpgedinge.\\ 
 & dô brast ir vreuden klinge\\ 
 & \textbf{d\textit{â}} \textbf{mitten} in dem hefte enzwei.\\ 
20 & owê und heiâ hei,\\ 
 & daz \textbf{güete} alsolichen \textbf{jâmer} treget\\ 
 & und imer \textbf{trûren} jâmer \textbf{reget}!\\ 
 & alsus vert diu menscheit:\\ 
 & hiute \textbf{vreude}, morgen leit.\\ 
25 & \begin{large}D\end{large}iu vrouwe umb einen mitten tac\\ 
 & eines angestlîchen slâfes pflac.\\ 
 & ir kam ein vorhteclîcher schric.\\ 
 & si dûhte, wie \textbf{eines sternen blic}\\ 
 & si gein den lüften vuorte,\\ 
30 & d\textit{â} si mit kreften ruorte\\ 
\end{tabular}
\scriptsize
\line(1,0){75} \newline
U V W T \newline
\line(1,0){75} \newline
\textbf{9} \textit{Majuskel} T  \textbf{15} \textit{Initiale} T  \textbf{25} \textit{Initiale} U V W   $\cdot$ \textit{Majuskel} T  \textbf{30} \textit{Majuskel} T  \newline
\line(1,0){75} \newline
\textbf{1} kêrte] kert W  $\cdot$ herze] sinne T  $\cdot$ guote] keúsche W \textbf{2} der werlte] der [werke]: werlte U die waren T \textbf{3} Herzeloyde] Herzeleide U hertzelaude V hertzeloyde W  $\cdot$ diu] ein V \textbf{4} lobe] hohe T \textbf{7} Waleis] Walleis V  $\cdot$ Anschouwe] Anschowe U V antschowe W \textbf{8} si] sin U \textbf{9} krônen] crone V (W) (T)  $\cdot$ Nurgals] Norgals U V (W) \textbf{10} Kingrivals] Kingriuals V (W) kyngrivals T \textbf{11} wol sô lieber] so liep ir V lieb ir W wol so liep ir T \textbf{12} ie] ir T  $\cdot$ gewan] mer gewan W (T) \textbf{13} war] wer W \textbf{14} si] So W  $\cdot$ moht] moͤht V (W)  $\cdot$ wol] \textit{om.} W T \textbf{15} ûz] \textit{om.} W \textbf{16} sînes komens] sines kummendes V sin comen T  $\cdot$ warte] wartet W \textbf{17} diz] daz T \textbf{18} brast] brach V W  $\cdot$ vreuden] ftoͤden W \textbf{19} dâ mitten] Do mitten U Enmitten V (W) mitten T \textbf{21} güete] guͦt W (T)  $\cdot$ alsolichen] als soͤlchen W  $\cdot$ treget] weget W \textbf{22} trûren] trurens V triuwe T  $\cdot$ jâmer reget] vnfroͤde veget W \textbf{23} vert] stat W \textbf{24} vreude] lieb W \textbf{27} vorhteclîcher] angestlicher T \textbf{28} eines] ein T  $\cdot$ sternen] sternes W \textbf{30} dâ] Do U das V (W) \newline
\end{minipage}
\end{table}
\end{document}
