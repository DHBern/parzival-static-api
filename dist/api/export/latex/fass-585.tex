\documentclass[8pt,a4paper,notitlepage]{article}
\usepackage{fullpage}
\usepackage{ulem}
\usepackage{xltxtra}
\usepackage{datetime}
\renewcommand{\dateseparator}{.}
\dmyyyydate
\usepackage{fancyhdr}
\usepackage{ifthen}
\pagestyle{fancy}
\fancyhf{}
\renewcommand{\headrulewidth}{0pt}
\fancyfoot[L]{\ifthenelse{\value{page}=1}{\today, \currenttime{} Uhr}{}}
\begin{document}
\begin{table}[ht]
\begin{minipage}[t]{0.5\linewidth}
\small
\begin{center}*D
\end{center}
\begin{tabular}{rl}
\textbf{585} & solte \textbf{si gewaltes} verdriezen.\\ 
 & er mohte \textbf{doch} \textbf{des} geniezen,\\ 
 & daz si in âne sînen danc\\ 
 & wol gesunden \textbf{ê} betwanc.\\ 
5 & \begin{large}V\end{large}rou \textbf{minne}, welt ir prîs bejagen,\\ 
 & \textbf{m\textit{ö}ht} ir iu \textbf{doch} lâzen sagen:\\ 
 & iu ist ân êre dirre strît.\\ 
 & Gawan lebte \textbf{ie} sîne zît,\\ 
 & als iwer hulde im gebôt.\\ 
10 & \textbf{daz} tet \textbf{ouch} sîn vater Lot.\\ 
 & muoterhalp \textbf{al} sîn geslehte,\\ 
 & \textbf{daz} stuont iu gar ze rehte\\ 
 & \textbf{sît} her von Mazadane,\\ 
 & \textbf{den} \textbf{ze} Famorgane\\ 
15 & Terredelaschoye vuorte,\\ 
 & den iwer kraft dô ruorte.\\ 
 & Mazadans nâchkomen,\\ 
 & von den \textbf{ist dicke sît} vernomen,\\ 
 & daz \textbf{ir} encheiner \textbf{iuch} \textbf{nie} \textbf{verliez}.\\ 
20 & Ither von Gaheviez\\ 
 & iwer insigel truoc.\\ 
 & \textbf{swâ} man vor wîben sîn gewuoc,\\ 
 & \textbf{des} wolte \textbf{sich ir keiniu} schamen,\\ 
 & \textbf{swâ} man nante sînen namen,\\ 
25 & \textbf{ob si der minne} ir krefte jach.\\ 
 & nû prüevet \textbf{denne}, diu in sach;\\ 
 & der wâren diu \textbf{rehten} mære komen,\\ 
 & \textbf{an dem iu dienst wart benomen}.\\ 
 & \begin{large}N\end{large}û tuot ouch Gawane den tôt,\\ 
30 & als sîme neven Ilynot,\\ 
\end{tabular}
\scriptsize
\line(1,0){75} \newline
D Z \newline
\line(1,0){75} \newline
\textbf{5} \textit{Initiale} D  \textbf{29} \textit{Initiale} D  \newline
\line(1,0){75} \newline
\textbf{6} möht] moht D \textbf{14} Famorgane] Famvrgane D (Z) \textbf{15} Terredelaschoye] Terre delascoye D Terre delatschoie Z \textbf{16} dô] da Z \textbf{20} Ither] Jther D Jcher Z  $\cdot$ Gaheviez] Caheviez Z \textbf{29} Gawane] gawan Z \textbf{30} Ilynot] Jlynot D Jbnot Z \newline
\end{minipage}
\hspace{0.5cm}
\begin{minipage}[t]{0.5\linewidth}
\small
\begin{center}*m
\end{center}
\begin{tabular}{rl}
 & solt \textbf{si gewaltes} verdriezen.\\ 
 & er mohte \textbf{doch} \textbf{des} \textbf{niht} geniezen,\\ 
 & daz si in ân sînen danc\\ 
 & wol gesunden \textbf{ê} betwanc.\\ 
5 & vrouwe, wellet ir prîs bejagen,\\ 
 & \textbf{m\textit{ö}ht} ir iu \textbf{daz} lâzen sagen:\\ 
 & \textit{iu} ist âne êre diser strît.\\ 
 & Gawan lebte sîn\textit{e} zît,\\ 
 & als iuwer hulde ime gebôt.\\ 
10 & \textbf{daz} tet \textbf{ouch} sîn vater Lot.\\ 
 & muoterhalp \textbf{al} sîn gesleht,\\ 
 & \textbf{daz} stuont iu gar zuo reht\\ 
 & \textbf{dô} her von Mazadane,\\ 
 & \textbf{den} \textbf{diu reine} Murgane\\ 
15 & \textbf{in} Terred\textit{e}laschoie vuorte,\\ 
 & den iuwer kraft dô ruorte.\\ 
 & Mazadans nâchkomen,\\ 
 & von den \textbf{ist dicke sît} vernomen,\\ 
 & daz \textbf{ir} keiner \textbf{iuch} \textbf{nie} \textbf{verliez}.\\ 
20 & I\textit{t}her von Gaheviez\\ 
 & iuwer ingesigel \textbf{ouch} truoc.\\ 
 & \textbf{wâ} man vor wîben sîn gewuoc,\\ 
 & \textbf{daz} wolte \textbf{ir dekeiniu} schamen,\\ 
 & \textbf{wâ} man nan\textit{t}e sînen namen,\\ 
25 & \textbf{ob \textit{si} der minne} ir krefte jach.\\ 
 & nû br\textit{ü}efet \textbf{dan}, diu in sach;\\ 
 & der wâren diu \textbf{rehten} mær\textit{e k}omen,\\ 
 & \textbf{an dem iu dienst wart benomen}.\\ 
 & nû tuot ouch Gawan den tôt,\\ 
30 & als sînem neven Il\textit{in}ot,\\ 
\end{tabular}
\scriptsize
\line(1,0){75} \newline
m n o \newline
\line(1,0){75} \newline
\newline
\line(1,0){75} \newline
\textbf{2} mohte] moͯchte n  $\cdot$ niht] \textit{om.} n o \textbf{4} ê] \textit{om.} o \textbf{6} möht ir] Maht ir m Mocht >ir< o  $\cdot$ daz] des n \textbf{7} iu] V̂ich m \textbf{8} sîne] siner m \textbf{12} iu] auch o \textbf{13} Mazadane] mazedane n \textbf{14} reine] weine o  $\cdot$ Murgane] morgane o \textbf{15} Terredelaschoie] tere do laschoie m terre do laschoie n terre do laschorie o \textbf{18} vernomen] genomen o \textbf{19} ir keiner iuch] úch keiner n \textbf{20} Ither] Icher m Jther n o  $\cdot$ Gaheviez] gahevies m o gaheviesz n \textbf{21} iuwer] Jr m n o \textbf{23} dekeiniu] do keine n dekeinen o \textbf{24} nante] nan::e m \textbf{25} si] \textit{om.} m \textbf{27} rehten] rehtte m (n) (o)  $\cdot$ mære komen] mere jach komen m \textbf{28} wart] were n \textbf{30} neven] neren o  $\cdot$ Ilinot] jlmot m o ilmot n \newline
\end{minipage}
\end{table}
\newpage
\begin{table}[ht]
\begin{minipage}[t]{0.5\linewidth}
\small
\begin{center}*G
\end{center}
\begin{tabular}{rl}
 & solde \textbf{gewaltes si} verdriezen.\\ 
 & er mohte \textbf{iedoch} geniezen,\\ 
 & daz si \textit{in} ân sînen danc\\ 
 & wol gesunden betwanc.\\ 
5 & vrouwe \textbf{minne}, welt ir prîs bejagen,\\ 
 & \textbf{muget} ir iu \textbf{doch} lâzen sagen:\\ 
 & iu ist ân êre dirre strît,\\ 
 & \textbf{wan} Gawan lebet \textbf{ie} sîn\textit{e} zît,\\ 
 & als iuwer hulde im gebôt.\\ 
10 & \textbf{alsô} tet sîn vater Lot.\\ 
 & muoterhalp sî\textit{n} geslehte\\ 
 & stuont iu gar ze rehte\\ 
 & \textbf{sît} her von Mazadan,\\ 
 & \textbf{den} Phimurgan\\ 
15 & Der Deilatschoy \textit{v}uort\textit{e},\\ 
 & den iuwer kraft dô ruorte.\\ 
 & Mazadans nâchkomen,\\ 
 & \textbf{dâ} von \textbf{sô dicke ist} vernomen,\\ 
 & daz \textbf{ir} deheiner \textbf{niht} \textbf{en}\textbf{liez}.\\ 
20 & Ither von Kahaviez\\ 
 & iuwer insigel truoc.\\ 
 & \textbf{swâ} man vor wîben sîn gewuoc,\\ 
 & \hspace*{-.7em}\big| \textbf{dâ} man nande sînen namen,\\ 
 & \hspace*{-.7em}\big| \textbf{des} wolde \textbf{ir deheiniu sich} schamen;\\ 
25 & \textbf{der minne si} ir krefte jach.\\ 
 & nû prüevet \textbf{di\textit{e} vrowe\textit{n}}, diu in \textbf{dô} sach;\\ 
 & der wâren diu \textbf{wâren} mære \textbf{dô} komen,\\ 
 & \textbf{alsir ê wol habet vernomen}.\\ 
 & nû tuot ouch Gawane den tôt,\\ 
30 & als sînem neven Ilinot,\\ 
\end{tabular}
\scriptsize
\line(1,0){75} \newline
G I L M Fr19 \newline
\line(1,0){75} \newline
\textbf{5} \textit{Initiale} I L Fr19  \textbf{17} \textit{Initiale} M  \textbf{25} \textit{Initiale} I  \newline
\line(1,0){75} \newline
\textbf{1} gewaltes si] si gewaltes I \textbf{3} in] \textit{om.} G \textbf{4} betwanc] twanch L (M) (Fr19) \textbf{5} minne] libe M \textbf{8} ie] alle I  $\cdot$ sîne] sin G \textbf{11} sîn] sine G \textbf{12} iu] ýe L \textbf{13} Mazadan] mazadam I \textbf{14} Phimurgan] pfeimvrgan G faimurgan I zuͯ Feý mvͯrgan L zcu feymorgan M cefemorgan Fr19 \textbf{15} Der Deilatschoy] der deilatschoͮy G der die lascoyge I Der da Latschoy L Der thelaschoy M Der delashoẏ Fr19  $\cdot$ vuorte] gefuͦrt G \textbf{16} kraft] kamph I  $\cdot$ dô] da M \textbf{17} Mazadans] Mazedams I Mazadansz L \textbf{18} vernomen] komen Fr19 \textbf{20} Ither] Jthern I M Jther L  $\cdot$ Kahaviez] kahauiez I kaheviez L Fr19 kahaviesz M \textbf{22} swâ] Wo L (M)  $\cdot$ vor] von I \textbf{23} wolde] enwolte L (M) (Fr19)  $\cdot$ ir deheiniu sich] sich dehainev I sich ir deheine L (M) (Fr19) \textbf{25} minne] libe M  $\cdot$ jach] sprach M \textbf{26} die vrowen] diu froͮwe G (Fr19) die vrowe L (M)  $\cdot$ dô] \textit{om.} L M Fr19 \textbf{27} wâren] \textit{om.} I L M  $\cdot$ dô] \textit{om.} I L M Fr19 \textbf{28} alsir] Also M  $\cdot$ ê wol] Also wol ir hat ehir vernommen M wol E L (Fr19) \textbf{29} Gawane] Gawan I (M) Fr19 \textbf{30} sînem] sinen L  $\cdot$ Ilinot] imelot I Jlinot L illinot M Linot Fr19 \newline
\end{minipage}
\hspace{0.5cm}
\begin{minipage}[t]{0.5\linewidth}
\small
\begin{center}*T
\end{center}
\begin{tabular}{rl}
 & solte \textbf{gewaltes si} verdriezen.\\ 
 & er mohte \textbf{iedoch} geniezen,\\ 
 & daz sin ân sînen danc\\ 
 & wol gesunden \textbf{ê} betwanc.\\ 
5 & vrou \textbf{minne}, wolt ir prîs bejagen,\\ 
 & \textbf{m\textit{ö}ht} ir iu \textbf{doch} lâzen sagen:\\ 
 & iu ist ân êre diser strît,\\ 
 & \textbf{wan} Gawan lebte \textbf{ie} sîne zît,\\ 
 & als iuwer hulde im gebôt.\\ 
10 & \textbf{als} tet sîn vater Lot.\\ 
 & muoterhalp sîn ges\textit{l}ehte\\ 
 & stuont iu gar zuo rehte\\ 
 & \textbf{sît} her von Mazadan,\\ 
 & \textbf{der} Feimorgan\\ 
15 & Terredelaschoie vuorte,\\ 
 & den iuwer kraft dô ruorte.\\ 
 & Mazadans nâchkomen,\\ 
 & von den \textbf{dicke ist} vernomen,\\ 
 & daz deheiner \textbf{nie} \textbf{liez}.\\ 
20 & Ither von Kaheviez\\ 
 & iuwer insigel truoc.\\ 
 & \textbf{wen} man vor wîben sîn gewuoc,\\ 
 & \hspace*{-.7em}\big| \textbf{dô} man \textit{n}ante sînen namen,\\ 
 & \hspace*{-.7em}\big| \textbf{des} \textbf{en}wolt \textbf{sich deheiniu} schamen;\\ 
25 & \textbf{der minne si} ir krefte jach.\\ 
 & nû prüevet \textbf{danne}, diu in sach;\\ 
 & der \textit{wâr}en diu \textbf{rehten} mære k\textit{o}men,\\ 
 & \textbf{an \textit{dem} i\textit{u} dienst wart benomen}.\\ 
 & nû tuot ouch Gawane den tôt,\\ 
30 & als sînem neven Ilinot,\\ 
\end{tabular}
\scriptsize
\line(1,0){75} \newline
Q R W V U \newline
\line(1,0){75} \newline
\textbf{5} \textit{Initiale} R W V  \newline
\line(1,0){75} \newline
\textbf{1} \textit{Die Verse 553.1-599.30 fehlen} U   $\cdot$ gewaltes] gewalte R \textbf{2} mohte] moͯchtte R (W) \textbf{3} sin] sy Ime R \textbf{4} gesunden] gesund R \textbf{6} möht] Mocht Q Muͯgt R (W) (V) \textbf{8} Gawan lebte] Gawin lebt R \textbf{10} tet] thet auch W  $\cdot$ Lot] lott R \textbf{11} geslehte] geschegte Q \textbf{12} iu] ye R \textbf{13} sît] [D*]: Sit V  $\cdot$ Mazadan] Mazadon R mazadane V \textbf{14} Den ze feimorgon R  $\cdot$ Den zuͦ feimorgan W  $\cdot$ [*]: den die fene morgane V \textbf{15} Terredelaschoie] Der delaszhoy Q Der delashoi R Der delashoie W [J*]: Der delaschoi V \textbf{16} kraft dô] [*]: craft do V \textbf{17} \textit{Versdoppelung 582.24 nach 585.17} R   $\cdot$ Mazadans] Mazedans V \textbf{19} daz] Das ir R W (V)  $\cdot$ nie liez] niene ließ W [*]: v́ch nie verliez V \textbf{20} Ither] Jchern Q Jchter R Yter V  $\cdot$ Kaheviez] kaherisz Q kacheuies R kahafies W kahevies V \textbf{21} V́wer Jngesigel troͧg R  $\cdot$ [*]: uwer ingesigel oͮch trvͦc V \textbf{22} wen] Wa R (W) Swa V \textbf{24} \textit{Versdoppelung 585.24 (²R) nach 585.17; Lesarten der vorausgehenden Verse mit ¹R bezeichnet} R   $\cdot$ dô] Da \textsuperscript{2}\hspace{-1.3mm} R  $\cdot$ nante] mante Q \textbf{23} des] Da R  $\cdot$ deheiniu] deheiner R ir keine W \textbf{25} [D*]: Ob sv́ der minne ir crefte iach V \textbf{26} danne diu in sach] dú in denne sach R danne die [*]: in sach V \textbf{27} der] dar R  $\cdot$ wâren] frawen Q  $\cdot$ rehten] Rette R  $\cdot$ mære komen] mere kamen Q [m*]: mere komen V \textbf{28} dem iu] ewr Q den v́ch V  $\cdot$ benomen] genomen V \textbf{29} Gawane] Gawine R \textbf{30} \textit{Vers 586.2 ist am Rand nachgetragen und später radiert:} Daz der iunge s:e rang V   $\cdot$ sînem] sinen R  $\cdot$ Ilinot] Jlinot R ilmot W \newline
\end{minipage}
\end{table}
\end{document}
