\documentclass[8pt,a4paper,notitlepage]{article}
\usepackage{fullpage}
\usepackage{ulem}
\usepackage{xltxtra}
\usepackage{datetime}
\renewcommand{\dateseparator}{.}
\dmyyyydate
\usepackage{fancyhdr}
\usepackage{ifthen}
\pagestyle{fancy}
\fancyhf{}
\renewcommand{\headrulewidth}{0pt}
\fancyfoot[L]{\ifthenelse{\value{page}=1}{\today, \currenttime{} Uhr}{}}
\begin{document}
\begin{table}[ht]
\begin{minipage}[t]{0.5\linewidth}
\small
\begin{center}*D
\end{center}
\begin{tabular}{rl}
\textbf{42} & aht vanen sweimen gein der stat,\\ 
 & die er \textbf{balde wenden} bat\\ 
 & \textit{\begin{large}D\end{large}}en küenen, sig\textit{e}lôsen man.\\ 
 & dar nâch gebôt er im \textbf{dô} sân,\\ 
5 & daz er kêrte nâch im în.\\ 
 & daz tet er, wan ez \textbf{solt} \textbf{et} sîn.\\ 
 & Gaschier sîn kumen \textbf{ouch} niht verbirt.\\ 
 & an dem innen wart \textbf{der} wirt,\\ 
 & daz sîn gast was komen ûz.\\ 
10 & daz er niht îsen als ein strûz\\ 
 & \textbf{unt} \textbf{starke} vlinse verslant,\\ 
 & \textbf{daz machete, daz er niht envant}.\\ 
 & sîn \textbf{zorn} begunde limmen\\ 
 & unt als ein lewe brimmen.\\ 
15 & \textbf{dô brach er} ûz sîn eigen hâr.\\ 
 & er sprach: "nû sint mir mîniu jâr\\ 
 & \textbf{nâch grôzer} tumpheit \textbf{bewant}.\\ 
 & die gote heten mir gesant\\ 
 & einen küenen, werden gast.\\ 
20 & ist er \textbf{verladen} mit strîtes last,\\ 
 & sô\textbf{ne} \textbf{mag} ich nimmer werden wert.\\ 
 & waz \textbf{touc} mir schilt \textbf{unt} swert?\\ 
 & er \textbf{sol} mich schelten, swer michs mane."\\ 
 & dô kêrt er \textbf{vor} den sînen dane.\\ 
25 & \textbf{gein} der porte er \textbf{vaste} ruorte.\\ 
 & ein knappe \textbf{im widervuorte}\\ 
 & einen schilt, \textbf{ûzen} und innen dran\\ 
 & \textbf{gemâlt} als ein durchstochen man,\\ 
 & geworht in Isenhartes lant.\\ 
30 & \textbf{einen helm er vuorte ouch in der hant}\\ 
\end{tabular}
\scriptsize
\line(1,0){75} \newline
D Fr14 \newline
\line(1,0){75} \newline
\textbf{3} \textit{Initiale} D Fr14  \newline
\line(1,0){75} \newline
\textbf{3} Den] ÷en D  $\cdot$ sigelôsen] sigolosen D \textbf{7} Gaschier] Gascîer D Gascier Fr14 \textbf{29} Isenhartes] Jsenhartes D \newline
\end{minipage}
\hspace{0.5cm}
\begin{minipage}[t]{0.5\linewidth}
\small
\begin{center}*m
\end{center}
\begin{tabular}{rl}
 & aht vanen swei\textit{m}en gegen der stat,\\ 
 & die er \textbf{wenden balde} bat\\ 
 & den küenen, sigelôsen man.\\ 
 & dar nâch gebôt er im \textbf{dô} sân,\\ 
5 & daz er kêrte nâch im în.\\ 
 & daz tet er, wande e\textit{z} \textbf{solte} sîn.\\ 
 & \begin{large}G\end{large}aschier sîn komen \textbf{ouch} niht verbirt.\\ 
 & an dem innen wart \textbf{der} wirt,\\ 
 & daz sîn gast was komen ûz.\\ 
10 & daz \textit{er} niht îsen als ein strûz\\ 
 & \textbf{und} \textbf{st\textit{ark}e} vlinse vers\textit{l}ant,\\ 
 & \textbf{daz meinde, daz er ir iht vant}.\\ 
 & sîn \textbf{zorn} begunde l\textit{imm}en\\ 
 & und als ein lewe bri\textit{mm}en.\\ 
15 & \textbf{dô brach er} ûz sîn eigen hâr.\\ 
 & er sprach: "nû sint mir mîniu jâr\\ 
 & \textbf{nâch grôzer} tumpheit \textbf{bewant}.\\ 
 & die gote heten mir gesant\\ 
 & einen küenen, werden gast.\\ 
20 & ist er \textbf{beladen} mit strîtes last,\\ 
 & sô \textbf{mac} ich \textit{niem}er werden wert.\\ 
 & waz \textbf{tuot} mir schilt \textbf{und} swert?\\ 
 & er \textbf{sol} mich\textbf{s} schelten, wer michs mane."\\ 
 & dô kêrt er \textbf{von} den sînen dane.\\ 
25 & \textbf{gegen} der porte er \textbf{vaste} ruorte.\\ 
 & ein knappe \textbf{ime widervuorte}\\ 
 & einen schilt, \textbf{vern} und innen dran\\ 
 & \textbf{gemâlet} als ein durchstochen man,\\ 
 & \dag geworden\dag  in Ysenhartes lant.\\ 
30 & \textbf{einer vuorte ouch in der hant}\\ 
\end{tabular}
\scriptsize
\line(1,0){75} \newline
m n o \newline
\line(1,0){75} \newline
\textbf{7} \textit{Initiale} m   $\cdot$ \textit{Capitulumzeichen} n  \newline
\line(1,0){75} \newline
\textbf{1} vanen] faren o  $\cdot$ sweimen] sweinen m o \textbf{5} în] hin n o \textbf{6} ez] er m n o \textbf{7} Gaschier] GAscier m (n) (o) \textbf{9} ûz] [ios]: vs m \textbf{10} er] \textit{om.} m det er o \textbf{11} starke] strackte m  $\cdot$ verslant] versbant m \textbf{12} vant] enfant n \textbf{13} limmen] lutten m \textbf{14} brimmen] brinen m \textbf{20} er] E n (o) \textbf{21} niemer] meiner m (o) \textbf{23} michs] mich n o  $\cdot$ wer] vnd [s]: wer o \textbf{26} ein] Eim o \textbf{27} vern] fren n o \textbf{29} geworden] Gewerden n o  $\cdot$ Ysenhartes] jsenharttes m jsenhartes n isenhartes o \textbf{30} einer] Er n  $\cdot$ der hant] den [stat]: strat o \newline
\end{minipage}
\end{table}
\newpage
\begin{table}[ht]
\begin{minipage}[t]{0.5\linewidth}
\small
\begin{center}*G
\end{center}
\begin{tabular}{rl}
 & ahte vanen sweimen gein der stat,\\ 
 & dier \textbf{vil} \textbf{balde wenden} bat\\ 
 & den küenen, sigelôsen man.\\ 
 & dar nâch gebôt er im sân,\\ 
5 & daz er kêrte nâch im în.\\ 
 & daz tet er, wan ez \textbf{muose} sîn.\\ 
 & Gatschier sîn komen niht verbirt.\\ 
 & an dem innen wart \textbf{der} wirt,\\ 
 & daz sîn gast was komen ûz.\\ 
10 & daz er niht îsen als ein strûz\\ 
 & \textbf{\begin{large}U\end{large}nd} \textbf{grôze} vlinse verslant,\\ 
 & \textbf{daz machte, daz er ir niene vant}.\\ 
 & sîn \textbf{munt} begunde limmen\\ 
 & und als ein leu brimmen.\\ 
15 & \textbf{er brach} ûz sîn eigen hâr.\\ 
 & er sprach: "nû sint mir mîniu jâr\\ 
 & \textbf{mit grôzer} tumpheit \textbf{bewant}.\\ 
 & die gote heten mir gesant\\ 
 & einen küenen, werden gast.\\ 
20 & ist er \textbf{verladen} mit strîtes last,\\ 
 & sô\textbf{ne} \textbf{wil} ich nimer werden wert.\\ 
 & waz \textbf{touc} mir schilt \textbf{und} swert?\\ 
 & er \textbf{mac} mich schelten, swer michs mane."\\ 
 & dô kêrter \textbf{von} den sînen dane.\\ 
25 & \textbf{hin ûz} der borter ruorte.\\ 
 & ein knappe \textbf{im widervuorte}\\ 
 & einen schilt, \textbf{ûzen} und innen dran\\ 
 & \textbf{gemâl} als ein durchstochen man,\\ 
 & geworht in Ysenhartes lant.\\ 
30 & \textbf{einen helm vuorter in der hant}\\ 
\end{tabular}
\scriptsize
\line(1,0){75} \newline
G O L M Q R Z Fr21 \newline
\line(1,0){75} \newline
\textbf{1} \textit{Initiale} O M  \textbf{7} \textit{Initiale} L Q R Z Fr21  \textbf{11} \textit{Initiale} G  \newline
\line(1,0){75} \newline
\textbf{1} \textit{Vers 42.1 fehlt} Q   $\cdot$ ahte] ÷ht O Alhie R  $\cdot$ vanen] wannan L  $\cdot$ sweimen] sweiben M schwumen R \textbf{3} den] Deme M \textbf{4} im] in da Z \textbf{6} ez] daz L \textbf{7} Gatschier] Gotischier M Gatschir Q Catschier R Gatsier Fr21  $\cdot$ niht] ovch niht Z  $\cdot$ verbirt] verlies R \textbf{8} innen wart] dingen frogt Q  $\cdot$ der wirt] er geheies R sin wirt Fr21 \textbf{10} îsen] Jsen asz M (Q) ysen ist R \textbf{11} grôze] starche O (L) (Q) (R) (Z) (Fr21) starkin l\sout{ } M  $\cdot$ verslant] vor slangk M \textbf{12} machte] machet O L R (Z) Fr21  $\cdot$ er ir] er O (M) er in Q ers Fr21  $\cdot$ niene] niht O L (M) (Q) (R) Z Fr21  $\cdot$ vant] envant L (M) \textbf{13} sîn] Das sient M  $\cdot$ munt] zorn Z  $\cdot$ limmen] zu iúngen Q \textbf{14} und] \textit{om.} O  $\cdot$ brimmen] brinnen O Fr21 brúmen Q (R) \textbf{15} er brach] Do brach er O L (Q) R Fr21 Da brach her M (Z)  $\cdot$ sîn] si O syneme M  $\cdot$ hâr] hore M \textbf{16} mir] \textit{om.} O  $\cdot$ mîniu] mine R \textbf{17} mit] Nach O L (M) (Q) R Z Fr21 \textbf{18} gote] goten O  $\cdot$ heten] hette Q  $\cdot$ mir] in L \textbf{19} werden] wenden L \textbf{20} \textit{Versdoppelung 1.1, 1.3, 1.7 (²O) nach 42.20} O   $\cdot$ verladen] uberladen R  $\cdot$ strîtes] ritter M ritters Q \textbf{21} sône] So L Q \textbf{22} touc] tavgt O solt Q  $\cdot$ und] oder L (M) \textbf{23} mac] sol L Z  $\cdot$ schelten] selde M [stellen]: scheltten R  $\cdot$ swer michs mane] der mich es man L adir mane M wer mich seyn man Q swer micl man Fr21 \textbf{24} dô] Da M Z  $\cdot$ kêrter] kert er Q Z Fr21 kertte R  $\cdot$ von] vor Fr21  $\cdot$ den] \textit{om.} Z \textbf{25} hin ûz] Gein O (Q) (R) Z Fr21 Keyn M  $\cdot$ der] dem R  $\cdot$ borter] porten er vaste O (M) pforte er faste Q tor er faste R porte er vaste Z (Fr21) \textbf{26} im] om da M in R Fr21 \textbf{27} ûzen und innen] innen vnd avzen O (L) (Q) (Fr21) vnd vzzen vnd innen Z \textbf{28} gemâl] Mal O (M) Fr21 Gemeld Q gemalt R Z  $\cdot$ durchstochen] irstochin M \textbf{29} Ysenhartes] ẏsenhartes G Jsenhartes L Z ysenarts M eysenhartes Q Jsenharttes R ysenharts Fr21 \textbf{30} \textit{Vers 42.30 fehlt} Q   $\cdot$ in der] ander O (L) M (Z) (Fr21) \newline
\end{minipage}
\hspace{0.5cm}
\begin{minipage}[t]{0.5\linewidth}
\small
\begin{center}*T (U)
\end{center}
\begin{tabular}{rl}
 & ahte vanen sweimen gein der stat,\\ 
 & d\textit{ie} er \textbf{balde wenden} bat\\ 
 & de\textit{n} küenen, sigelôsen man.\\ 
 & dar nâch gebôt er im sân,\\ 
5 & daz er kêrte nâch im în.\\ 
 & daz tet er, wan ez \textbf{muose} sîn.\\ 
 & \textit{G}atschier sîn komen niht verbirt.\\ 
 & an dem \textbf{nû} i\textit{nne}n wart \textbf{sîn} wirt,\\ 
 & daz sîn gast was komen ûz.\\ 
10 & daz er niht îsen als ein strûz\\ 
 & \textbf{noch} \textbf{ganz\textit{e}} vlins verslant,\\ 
 & \textbf{vil kûme er des wart erwant}.\\ 
 & sîn \textbf{muot} begunde limmen\\ 
 & und als ein lew\textit{e} br\textit{i}mmen.\\ 
15 & \textbf{dô brach er} ûz sîn eigen hâr.\\ 
 & er sprach: "nû sint mir mîniu jâr\\ 
 & \textbf{gar zuo} tumpheit \textbf{gewant}.\\ 
 & die gote heten mir gesant\\ 
 & einen küenen, werden gast.\\ 
20 & ist er \textbf{beladen} mit st\textit{r}îtes last,\\ 
 & sô \textbf{mag} ich niemer werden wert.\\ 
 & waz \textbf{touc} mir schilt \textbf{oder} swert?\\ 
 & er \textbf{muoz} mich schelten, swer michs man."\\ 
 & dô kêrte er \textbf{von} den sînen dan.\\ 
25 & \textbf{gein} der porten \textit{er} \textbf{vaste} ruorte.\\ 
 & ein knappe \textbf{gein im vuorte}\\ 
 & einen schilt, \textbf{ûzen} und innen dran\\ 
 & \textit{\textbf{gemâl}} als ein durchstochen man,\\ 
 & geworht in Isenhartes lant,\\ 
30 & \textbf{daz gap ime von Schotten Fridebrant},\\ 
\end{tabular}
\scriptsize
\line(1,0){75} \newline
U V W T \newline
\line(1,0){75} \newline
\textbf{7} \textit{Initiale} V W T  \textbf{13} \textit{Majuskel} T  \textbf{26} \textit{Majuskel} T  \newline
\line(1,0){75} \newline
\textbf{1} sweimen] \textit{om.} W \textbf{2} die er] Der U Die er vil W (T) \textbf{3} den] Dem U  $\cdot$ küenen] kummer W \textbf{5} kêrte] kert W  $\cdot$ în] hin V hin ein W \textbf{6} muose] muͦz U muͤst eht V (W) mvese T \textbf{7} Gatschier] Atschier U Gaschier T \textbf{8} nû] \textit{om.} T  $\cdot$ innen] ieman U  $\cdot$ sîn] [*]: der V der T \textbf{11} noch] vnd T  $\cdot$ ganze vlins] ganz vlins U gantzen flins V gantze felsen W starke vlinse T  $\cdot$ verslant] schlant W (T) \textbf{12} [*]: Daz mahte imme daz er ir niene vant V · Vil kume er doch des erwant W · daz mahte daz er sin niht vant T \textbf{13} muot] munt V \textbf{14} und] \textit{om.} W  $\cdot$ lewe brimmen] lewen bruͦmmen U loͤwen brimmen W leuwe brinnen T \textbf{17} gar zuo] nach grozer T \textbf{18} gote] \textit{om.} T \textbf{20} beladen] verladen W  $\cdot$ strîtes] stites U \textbf{21} mag] enmag V (W) wil T \textbf{22} touc] toͮget V \textbf{23} muoz] [*]: mag (V) mac T  $\cdot$ swer] [*]: wer V der W  $\cdot$ michs] mich W \textbf{24} von] vor T \textbf{25} er] \textit{om.} U \textbf{26} knappe] knappen W  $\cdot$ gein im] engegen im V in wider W im wider T \textbf{27} ûzen und innen] auß vnd innen W innen vnde v̂zen T \textbf{28} gemâl] \textit{om.} U gemolet V (W) \textbf{29} Isenhartes] ysenhartes U W T Jsinhartes V \textbf{30} [*]: Einen helm fvͦrte [*]: er in der hant V · Einen fuͦrt er an der hand W · einen helm er vuͦrte ander hant T \newline
\end{minipage}
\end{table}
\end{document}
