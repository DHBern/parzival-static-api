\documentclass[8pt,a4paper,notitlepage]{article}
\usepackage{fullpage}
\usepackage{ulem}
\usepackage{xltxtra}
\usepackage{datetime}
\renewcommand{\dateseparator}{.}
\dmyyyydate
\usepackage{fancyhdr}
\usepackage{ifthen}
\pagestyle{fancy}
\fancyhf{}
\renewcommand{\headrulewidth}{0pt}
\fancyfoot[L]{\ifthenelse{\value{page}=1}{\today, \currenttime{} Uhr}{}}
\begin{document}
\begin{table}[ht]
\begin{minipage}[t]{0.5\linewidth}
\small
\begin{center}*D
\end{center}
\begin{tabular}{rl}
\textbf{201} & ir \textbf{habe} zwispilte.\\ 
 & die \textbf{koufliute} \textbf{des} bevilte.\\ 
 & sus wart vergolten in ir kouf.\\ 
 & den burgæren in \textbf{die} kolen trouf.\\ 
5 & ich wære \textbf{dâ nû wol} soldier,\\ 
 & \textbf{wan dâ trinket niemen} bier.\\ 
 & si hânt \textbf{wînes unt spîse} vil.\\ 
 & dô warp, als ich iu sagen wil,\\ 
 & Parzival, der reine:\\ 
10 & von êrste die spîse kleine\\ 
 & teilt \textbf{er} mit sîn selbes hant.\\ 
 & \textbf{er sazte die werden}, die er \textbf{dâ} vant.\\ 
 & er wolde niht ir læren magen\\ 
 & überkrüpfe lâzen tragen.\\ 
15 & er gab in rehter mâze teil.\\ 
 & si wurden sînes râtes geil.\\ 
 & \textbf{hin} ze naht schuof er in mêre,\\ 
 & der unlôse, niht ze hêre.\\ 
 & Bîligens wart gevrâget dâ.\\ 
20 & er unt diu künegîn sprâchen: "jâ."\\ 
 & er lac mit sölhen vuogen,\\ 
 & des nû niht wil genuogen\\ 
 & manegiu wîp, \textbf{swer} in sô tuot.\\ 
 & daz si durch arbeitlîchen muot\\ 
25 & ir zuht sus parrierent\\ 
 & unt sich dar gegen zierent!\\ 
 & \begin{large}V\end{large}or gesten sint si an kiuschen siten.\\ 
 & ir \textbf{herzen wille} hât \textbf{versniten},\\ 
 & swaz mac an den gebærden sîn.\\ 
30 & \textbf{ir vriwent} si heinlîchen pîn\\ 
\end{tabular}
\scriptsize
\line(1,0){75} \newline
D \newline
\line(1,0){75} \newline
\textbf{19} \textit{Majuskel} D  \textbf{27} \textit{Initiale} D  \newline
\line(1,0){75} \newline
\newline
\end{minipage}
\hspace{0.5cm}
\begin{minipage}[t]{0.5\linewidth}
\small
\begin{center}*m
\end{center}
\begin{tabular}{rl}
 & ir \textbf{habe} \textbf{ein} zwispilte.\\ 
 & die \textbf{schifliute} \textbf{des} bevilte.\\ 
 & sus wart vergolten in ir kouf.\\ 
 & den burgern in \textbf{die} kolen trouf.\\ 
5 & ich wære \textbf{dâ nû wol} soldier,\\ 
 & \textbf{wanne d\textit{â} trinket nieman} bier.\\ 
 & si hânt \textbf{wînes und \textit{spîse}} vil.\\ 
 & dô warp, als ich iu sagen wil,\\ 
 & Parcifal, der reine:\\ 
10 & von êrst die spîse \textbf{er} kleine\\ 
 & teilte mit sîn selbes hant.\\ 
 & \textbf{er sazete die werden}, die er \textbf{d\textit{â}} vant.\\ 
 & er wolt niht ir læren magen\\ 
 & überkr\textit{ü}pfe lâzen tragen.\\ 
15 & er gap in rehter mâze teil.\\ 
 & si wurden sînes râtes geil.\\ 
 & \textbf{hin} ze naht schuof er in mêre,\\ 
 & der unlôse, niht ze hêre.\\ 
 & bîligens wart gevrâget dâ.\\ 
20 & er und diu künigîn sprâchen: "\textit{j}â."\\ 
 & er lac mit solhen vuogen,\\ 
 & des nû niht wil genuogen\\ 
 & manigiu wîp, \textbf{wer} i\textit{n} sô tuot.\\ 
 & daz si durch arb\textit{ei}tlîchen muot\\ 
25 & ir zuht sus parrierent\\ 
 & und sich dar gegen zierent!\\ 
 & vor gesten sint si an kiuschen siten.\\ 
 & ir \textbf{herzewille} hât \textbf{vermiten},\\ 
 & waz mac an den gebærden sîn.\\ 
30 & \textbf{ir vriunt} si heimlîchen pîn\\ 
\end{tabular}
\scriptsize
\line(1,0){75} \newline
m n o Fr69 \newline
\line(1,0){75} \newline
\newline
\line(1,0){75} \newline
\textbf{1} habe] haben o \textbf{2} des] dasz o es Fr69 \textbf{5} dâ] do n o  $\cdot$ nû] miner o  $\cdot$ soldier] suldeier o \textbf{6} dâ] do m n o \textbf{7} wînes und spîse] winesvnd vnd m spise vnd wines n o Fr69 \textbf{10} die] ein n  $\cdot$ er kleine] die cleine n encleine o \textbf{11} sîn] sins o \textbf{12} dâ] do m n o \textbf{13} ir] ire o \textbf{14} überkrüpfe] V̂ber krapfe m \textbf{19} dâ] do n \textbf{20} Si sprachent beide sament Ja Fr69  $\cdot$ jâ] da m \textbf{22} des] Den n  $\cdot$ nû niht wil] wil ::: Fr69 \textbf{23} wer] der n  $\cdot$ in] im m \textbf{24} arbeitlîchen] arbietlichen m \textbf{27} kiuschen] kúschem n (o) kúschē Fr69 \textbf{28} ir herzewille] Jres hertzen wille n (o)  $\cdot$ vermiten] versnitten n o \textbf{29} den] dent o \newline
\end{minipage}
\end{table}
\newpage
\begin{table}[ht]
\begin{minipage}[t]{0.5\linewidth}
\small
\begin{center}*G
\end{center}
\begin{tabular}{rl}
 & ir \textbf{kouf} zwispilde.\\ 
 & die \textbf{koufliute} \textbf{des} bevilde.\\ 
 & sus wart vergolten in ir kouf.\\ 
 & den burgæren \textbf{dô} in kolen trouf.\\ 
5 & ich wære \textbf{dâ nû wol} soldier,\\ 
 & \textbf{wan dâ trinkt niemen} bier.\\ 
 & \begin{large}S\end{large}i habent \textbf{spîse unde wînes} vil.\\ 
 & dô warb, als ich iu sagen wil,\\ 
 & Parzival, der reine:\\ 
10 & von êrste die spîse kleine\\ 
 & teilt\textbf{er} mit sîn selbes hant.\\ 
 & \textbf{dô satzter alle}, dier vant.\\ 
 & er wolt niht ir læren magen\\ 
 & überkrüpfe lâzen tragen.\\ 
15 & er gab i\textit{n} rehter mâze teil.\\ 
 & si wurden sînes râtes geil.\\ 
 & \textbf{hin} ze naht schuof er in mêre,\\ 
 & der unlôse, niht ze hêre.\\ 
 & bîligens wart gevrâget dâ.\\ 
20 & er unt diu künigîn sprâchen: "jâ."\\ 
 & er lac mit solhen vuogen,\\ 
 & des nû niht wil genuogen\\ 
 & manigiu wîb, \textbf{der} in sô tuot.\\ 
 & daz si durch arbeitlîchen muot\\ 
25 & ir zuht sus parrierent\\ 
 & unde sich dar geine zierent!\\ 
 & vor gesten sint si an kiuschen siten.\\ 
 & ir \textbf{herze willen} hât \textbf{versniten},\\ 
 & swaz mag an den gebærden sîn.\\ 
30 & \textbf{ir vriunt} si heinlîchen pîn\\ 
\end{tabular}
\scriptsize
\line(1,0){75} \newline
G I O L M Q R Z Fr21 \newline
\line(1,0){75} \newline
\textbf{7} \textit{Initiale} G  \textbf{9} \textit{Initiale} R  \textbf{15} \textit{Initiale} I  \textbf{19} \textit{Initiale} M Z  \textbf{21} \textit{Initiale} L  \textbf{27} \textit{Initiale} O Fr21   $\cdot$ \textit{Capitulumzeichen} L  \newline
\line(1,0){75} \newline
\textbf{1} zwispilde] zezwisbilde I mit zwispilde O zwýevelde L zcwisch bilde M zwifaltte vnd zwispilde R \textbf{2} koufliute] seflute I schiflvͦte O (Z)  $\cdot$ des bevilde] es bevilde L des gevilte Q \textbf{3} wart vergolten] was vergolten O vergoltten ward R \textbf{4} Den burgeren Q  $\cdot$ dô] \textit{om.} O nv L da M Z  $\cdot$ in] indie I (M) in ir O  $\cdot$ trouf] [tuf]: tauf I troff R (Z) \textbf{5} dâ nû wol] nu da wol I nv da L da Nu M do nú wol Q nun wol da R \textbf{6} dâ] do Q denne R  $\cdot$ trinkt] entrinket Q  $\cdot$ bier] hir M \textbf{7} spîse unde wînes] win vnd spise I spise wines O spise vnde winere M \textbf{8} dô] Da O M Z  $\cdot$ als] al L  $\cdot$ ich iu] ich ev nu I ich L Q ev Z \textbf{9} Parzival] Parcifal O L Z Parzifal M Partzifal Q PArczifal R \textbf{10} êrste] erst er I  $\cdot$ kleine] er cleine L \textbf{11} teilter] Tailte I (L) Teilt er O (Q) (R) [Tel*]: Teilt er  Z  $\cdot$ sîn selbes] siner I sines selbis M (Z) \textbf{12} Er satzt die werden die er da vant Z  $\cdot$ dô] Da M  $\cdot$ satzter alle] seit er allen I sazt er O (Q) (R)  $\cdot$ dier] die er da I (M) die er do Q \textbf{13} er] ern I (O) (M) (Z) Orn Q  $\cdot$ ir læren] erlern ir I ir eren Q ir lere Z  $\cdot$ magen] machen Q \textbf{15} in rehter] ir rehter G recht R \textbf{16} wurden] wardin M  $\cdot$ râtes] ratens Z \textbf{17} schuof] do schuͦff R \textbf{18} unlôse] vnsose L  $\cdot$ niht ze] vnd nih der I \textbf{19} bîligens] Di ligens M  $\cdot$ dâ] do Q \textbf{20} diu künigîn sprâchen] kungin sprachet R  $\cdot$ jâ] [wa]: ia I \textbf{21} solhen] sulen M \textbf{22} des] Daz O Fr21 Da R  $\cdot$ wil] vil Q wol R  $\cdot$ genuogen] benuͦgen R (Z) \textbf{23} in] Nu M \textbf{24} arbeitlîchen] arbaitlichem I abbrechenlchen R \textbf{25} sus] \textit{om.} I ausz Q  $\cdot$ parrierent] [parrîernen]: parrîernt G parrierten O barrierten Fr21 \textbf{26} zierent] [ziernen]: ziernt G zierten O Fr21 \textbf{27} vor gesten] ewer gest I ÷or gesten O  $\cdot$ sint si] sint I sin sie L (M)  $\cdot$ an] in R \textbf{28} herze willen] herzen wille O (L) Fr21 herczen willen M (Z) hertze wille Q (R) \textbf{29} swaz] waz I (L) (M) (Q) (R) [*az]: Waz  O  $\cdot$ gebærden] Gebarn I \textbf{30} vriunt si] froͯde R  $\cdot$ heinlîchen] herzenlichen O heimliche Q R \newline
\end{minipage}
\hspace{0.5cm}
\begin{minipage}[t]{0.5\linewidth}
\small
\begin{center}*T
\end{center}
\begin{tabular}{rl}
 & ir \textbf{habe} zwis\textit{p}ilde.\\ 
 & die \textbf{schifliute} \textbf{es} bevilde.\\ 
 & Sus wart vergolten in ir kouf.\\ 
 & den burgæren in \textbf{die} kolen trouf.\\ 
5 & ich wære \textbf{nû wol dâ} soldier,\\ 
 & \textbf{man trinket dâ nimer} bier.\\ 
 & si hânt \textbf{spîse unde wînes} vil.\\ 
 & Dô warp, als ich iu sagen wil,\\ 
 & Parcifal, der reine:\\ 
10 & von êrst die spîse kleine\\ 
 & teilt\textbf{er} mit sîn selbes hant.\\ 
 & \textbf{er sazte die werden}, die er \textbf{dâ} vant.\\ 
 & er wolte niht ir læren magen\\ 
 & überkrüpfe lâzen tragen.\\ 
15 & er gab in rehter mâze teil.\\ 
 & si wurden sînes râtes geil.\\ 
 & ze naht schuof er in mêre,\\ 
 & der unlôse, niht ze hêre.\\ 
 & Bîligens wart gevrâget dâ.\\ 
20 & er unde diu künegîn sprâchen: "jâ."\\ 
 & \begin{large}E\end{large}r lac mit solhen vuogen,\\ 
 & des nû niht wil genuogen\\ 
 & manec wîp, \textbf{der} in sô tuot.\\ 
 & daz si durch arbeitlîchen muot\\ 
25 & ir zuht sus parrierent\\ 
 & unde sich dar gegene zierent!\\ 
 & vor gesten sint si an kiuschen siten.\\ 
 & ir \textbf{herze den willen} hât \textbf{versniten},\\ 
 & waz mac an den gebærden sîn.\\ 
30 & \textbf{ir vriunden} si heinlîchen pîn\\ 
\end{tabular}
\scriptsize
\line(1,0){75} \newline
T U V W \newline
\line(1,0){75} \newline
\textbf{3} \textit{Majuskel} T  \textbf{8} \textit{Majuskel} T  \textbf{19} \textit{Majuskel} T  \textbf{21} \textit{Initiale} T U V W  \newline
\line(1,0){75} \newline
\textbf{1} zwispilde] [zwisv*]: zwisbilde T zwivilde U (W) [zwi*]: zwispilte  V \textbf{3} ir] der W \textbf{5} wol dâ] do wol ein W \textbf{6} Wann do trincket nun niemant bier W  $\cdot$ dâ nimer] do nuͦme U da nv́me V \textbf{7} spîse unde wînes] weines vnd speise W \textbf{9} Parcifal] Parzifal V Partzifal W \textbf{11} teilter] Deilet er U [T*]: Teilt er V  $\cdot$ sîn selbes] siner U \textbf{12} werden] selben W  $\cdot$ dâ] [*]: do V do W \textbf{18} der unlôse] Den vnlosen V \textbf{26} zierent] ir zierent U \textbf{27} sint si] [*]: sintz V sein sy W \textbf{30} vriunden] [*]: frúnde V frúnde W  $\cdot$ heinlîchen] haymliche W \newline
\end{minipage}
\end{table}
\end{document}
