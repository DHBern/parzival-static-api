\documentclass[8pt,a4paper,notitlepage]{article}
\usepackage{fullpage}
\usepackage{ulem}
\usepackage{xltxtra}
\usepackage{datetime}
\renewcommand{\dateseparator}{.}
\dmyyyydate
\usepackage{fancyhdr}
\usepackage{ifthen}
\pagestyle{fancy}
\fancyhf{}
\renewcommand{\headrulewidth}{0pt}
\fancyfoot[L]{\ifthenelse{\value{page}=1}{\today, \currenttime{} Uhr}{}}
\begin{document}
\begin{table}[ht]
\begin{minipage}[t]{0.5\linewidth}
\small
\begin{center}*D
\end{center}
\begin{tabular}{rl}
\textbf{206} & \textbf{ir} ebenhœhe unt ir mangen,\\ 
 & swaz ûf redern kom gegangen,\\ 
 & \textbf{igel}, \textbf{katzen} in den graben,\\ 
 & \textbf{die} kunde \textbf{daz} viwer \textbf{hin} dan wol schaben.\\ 
5 & \textit{\begin{large}K\end{large}}ingrun \textbf{scheneschalt}\\ 
 & \textbf{was} \textbf{komen} ze Bertane in daz lant\\ 
 & \textbf{und} vant den künec Artus\\ 
 & \textbf{in} Brizljan \textbf{ze}m weidehûs,\\ 
 & daz was geheizen Karminal.\\ 
10 & dô warb er, als in Parzival\\ 
 & gevangenen hete dar gesant.\\ 
 & vroun Cunnewaren de Lalant\\ 
 & brâht er sîne sicherheit.\\ 
 & \textbf{diu juncvrouwe was} gemeit,\\ 
15 & daz mit triwen klagt ir nôt,\\ 
 & den man dâ hiez '\textbf{der} ritter rôt'.\\ 
 & Über \textbf{al} \textbf{diz} mære wart vernomen.\\ 
 & \textbf{dô} was ouch vür den künec komen\\ 
 & der betwungene werde man.\\ 
20 & im unt der messenîde sân\\ 
 & saget er, wa\textit{z} \textbf{in} was enboten.\\ 
 & Keie erschrac und begunde rôten.\\ 
 & \textbf{Doch} sprach er: "bistûz, Kingrun?\\ 
 & âvoy, wie manegen Bertun\\ 
25 & hât enschumpfieret dîn hant,\\ 
 & \textbf{dû, Clamides scheneschalt!}\\ 
 & wirt mir dîn meister nimmer holt,\\ 
 & dînes \textbf{amtes} dû doch geniezen solt.\\ 
 & \textbf{der kezzel ist} uns undertân,\\ 
30 & mir hie unt dir ze Brandigan.\\ 
\end{tabular}
\scriptsize
\line(1,0){75} \newline
D \newline
\line(1,0){75} \newline
\textbf{5} \textit{Initiale} D  \textbf{17} \textit{Majuskel} D  \textbf{23} \textit{Majuskel} D  \newline
\line(1,0){75} \newline
\textbf{5} Kingrun] ÷ingrvͦn \textit{nachträglich korrigiert zu:} Kingrvͦn D \textbf{8} Brizljan] Prizlian D \textbf{21} waz] was D \textbf{22} Keie] kêie D \textbf{23} Kingrun] kingrvͦn D \textbf{26} Clamides] Chlamides D \newline
\end{minipage}
\hspace{0.5cm}
\begin{minipage}[t]{0.5\linewidth}
\small
\begin{center}*m
\end{center}
\begin{tabular}{rl}
 & \textbf{ir} ebenhœhe und ir ma\textit{n}gen,\\ 
 & waz ûf rederen kam gegangen,\\ 
 & \textbf{igele}, \textbf{katzen} in den graben,\\ 
 & \textbf{die} kunde \textbf{daz} viur \textbf{hin} dan wol schaben.\\ 
5 & \textbf{\begin{large}N\end{large}û was ouch} Kingrun \textbf{schinschant}\\ 
 & \textbf{komen} ze Britanie in daz lant\\ 
 & \textbf{und} vant den künic Artu\textit{s}\\ 
 & \textbf{in} Pricilan \textbf{ze}m weidehûs,\\ 
 & daz \textit{was} geheizen Carminal.\\ 
10 & dô warp er, als in Parcifal\\ 
 & gevangen hete dar gesant.\\ 
 & vrouwen Cunnewaren de Lalant\\ 
 & brâhte er sîne sicherheit.\\ 
 & \textbf{diu juncvrouwe was} gemeit,\\ 
15 & daz mit triuwen klagete ir nôt,\\ 
 & den man d\textit{â} hiez '\textbf{der} ritter rôt'.\\ 
 & über \textbf{alle} \textbf{disiu} mære wart vernomen.\\ 
 & \textbf{dô} was ouch vür den künic komen\\ 
 & der betwungene werde man.\\ 
20 & im und der massenîe sân\\ 
 & sagete er, waz \textbf{in} was enboten.\\ 
 & Keie erschrac und begunde rôten.\\ 
 & \textbf{doch} sprach er: "bist dûz, K\textit{i}ngr\textit{u}n?\\ 
 & â\textit{v}oy, wie manigen Br\textit{i}t\textit{u}n\\ 
25 & hât entschumpfieret dîn hant\\ 
 & \textbf{dô, Cla\textit{m}ides schinschant!}\\ 
 & wirt mir dîn meister niemer holt,\\ 
 & dînes \textbf{amptes} dû doch geniezen solt.\\ 
 & \textbf{der \textit{k}ezzel ist} uns undertân,\\ 
30 & mir hie und dir ze Brandigan.\\ 
\end{tabular}
\scriptsize
\line(1,0){75} \newline
m n o Fr69 \newline
\line(1,0){75} \newline
\textbf{5} \textit{Initiale} m   $\cdot$ \textit{Capitulumzeichen} n  \newline
\line(1,0){75} \newline
\textbf{1} ir ebenhœhe] Jn eben hoͯhe o  $\cdot$ mangen] manigen m o \textbf{4} schaben] schaber o \textbf{5} Kingrun] konigrẏm o  $\cdot$ schinschant] scunscant m n sconscant o \textbf{6} Britanie] britane n o \textbf{7} Artus] artuͯse m artuͯs o \textbf{8} in] Ze Fr69  $\cdot$ Pricilan] parcilan o prisclian Fr69 \textbf{9} was] \textit{om.} m  $\cdot$ Carminal] carmuͯal n Corninal o Karminal Fr69 \textbf{10} in] eyn o \textbf{12} vrouwen] Frouwe m (n) (o)  $\cdot$ Cunnewaren] connewaren n konne waren o \textbf{15} nôt] [leit]: not Fr69 \textbf{16} dâ] do m n o  $\cdot$ hiez] heisset Fr69  $\cdot$ der] den n o \textbf{22} Keie] Keye n \textbf{23} doch] Do n o  $\cdot$ bist dûz] bistu o  $\cdot$ Kingrun] kungrin m kingruͦn n konigrin o \textbf{24} âvoy] Anoi m Owe n o  $\cdot$ Britun] bruͯtten m brituͦn n britym o \textbf{25} hât] Hette n \textbf{26} dô] One n o  $\cdot$ Clamides] klanides m clamedez o  $\cdot$ schinschant] scuntscant m scunscant n o \textbf{28} doch] noch o \textbf{29} kezzel] leczel m \newline
\end{minipage}
\end{table}
\newpage
\begin{table}[ht]
\begin{minipage}[t]{0.5\linewidth}
\small
\begin{center}*G
\end{center}
\begin{tabular}{rl}
 & ebenhœhe unde \textit{ir} mangen,\\ 
 & swaz ûf rederen kom gegangen,\\ 
 & \textbf{igele}, \textbf{katzen} in den graben,\\ 
 & \textbf{daz} kunde \textbf{daz} viur \textbf{her} dan wol schaben.\\ 
5 & Kingrun \textbf{schinschalt}\\ 
 & \textbf{was} \textbf{komen} ze Britanie in daz lant\\ 
 & \textbf{unde} vant den künic Artus\\ 
 & \textbf{ze} Brizilan \textbf{ze}\textit{m} weidehûs,\\ 
 & daz was geheizen Karminal.\\ 
10 & dô warp er, als in Parzival\\ 
 & gevangen hete dar gesant.\\ 
 & vroun Kunewaren de Lalant\\ 
 & brâhter sîne sicherheit.\\ 
 & \textbf{diu juncvrouwe was} gemeit,\\ 
15 & daz mit triwen klaget ir nôt,\\ 
 & den man dâ hiez \textbf{den} rîter rôt.\\ 
 & über \textbf{al} \textbf{daz} mære wart vernomen.\\ 
 & \textbf{dô} was ouch vür den künic komen\\ 
 & der betwungene werde man.\\ 
20 & im unt der messenîe sân\\ 
 & sagter, waz \textbf{in} was enboten.\\ 
 & Kay erschrac unde begunde rôten.\\ 
 & \textbf{dô} sprach er: "bistûz, Kingrun?\\ 
 & âvoy, wie manigen Britun\\ 
25 & hât entschunpfiert dîn hant,\\ 
 & \textbf{dû, Clamides schinschalt!}\\ 
 & wirt mir dîn meister nimer holt,\\ 
 & dînes \textbf{ambetes} dû doch geniezen solt.\\ 
 & \textbf{die kezzele sint} uns undertân,\\ 
30 & mir hie unt dir ze Brandigan.\\ 
\end{tabular}
\scriptsize
\line(1,0){75} \newline
G I O L M Q R Z Fr21 \newline
\line(1,0){75} \newline
\textbf{1} \textit{Initiale} I  \textbf{17} \textit{Initiale} O L M Z Fr21  \textbf{23} \textit{Initiale} I  \newline
\line(1,0){75} \newline
\textbf{1} ebenhœhe] Ebine hoe M  $\cdot$ unde] \textit{om.} O Fr21  $\cdot$ ir mangen] mangen G magen Q mengen R \textbf{2} swaz] Soz O Waz L (Q) (R)  $\cdot$ rederen] rdren R \textbf{3} den] dem I (Q) \textbf{4} her dan wol] har dan R wol her dan Z \textbf{5} Nv waz och kýngrvn dan gewant L  $\cdot$ Kingrun] Kyngrvn O (M) Kyngrún Q Kyngurnt R  $\cdot$ schinschalt] schineschalt I schenetscant O senetscalt M senetscant Q schinetschant R smetschalant Z seneschant Fr21 \textbf{6} was komen] was I Vnd kom L Was kome R  $\cdot$ Britanie] [britaie]: britanie G Brittanie L britangen Q Brytanie R  $\cdot$ in] chomen in I \textbf{7} vant] van Z  $\cdot$ Artus] Artuͯs L \textbf{8} ze] in I (O) (M) (R) (Z) (Fr21)  $\cdot$ Brizilan] brizzian I brezian O Breszilian L Bricilia M bressilian Q Breczilian R brezzilian Z Brezalian Fr21  $\cdot$ zem] zen G ze sinem I zu eime Q in dem Z Fr21  $\cdot$ weidehûs] hus I \textbf{9} was] ward R  $\cdot$ Karminal] charminal I kvrnvnal O Fr21 kvrnwal L kvrnuͯal M carminal Q karmial R karmival Z \textbf{10} dô] Da Z  $\cdot$ in] im Q  $\cdot$ Parzival] parzifal I M Parcifal O (L) (Z) (Fr21) partzifal Q parczifal R \textbf{11} hete] hat R \textbf{12} vroun] Vrow L (M) (Q) (R)  $\cdot$ Kunewaren] kunuwarn I kvnewarn O kunwarn M konwaren Q Cuͦnwaren R kvnnewaren Z Cvnwaren Fr21  $\cdot$ de Lalant] dalalant I von lalant R \textbf{13} brâhter] Braht O (L) borcht er Q \textbf{14} was] was des R \textbf{15} klaget] clagte L (M) \textbf{16} dâ] \textit{om.} R \textbf{17} V́ber all waurent die mer komen R  $\cdot$ über] ÷ber O  $\cdot$ daz] die L  $\cdot$ wart] da wart Fr21 \textbf{18} Nun hett es der kúng vernomen R  $\cdot$ dô] nu I Da Z \textbf{19} der] \textit{om.} Q \textbf{20} sân] nam M \textbf{21} sagter] sagt er I (O) (Q) (Z) (Fr21) Sagiten M Saget Jn R  $\cdot$ waz] das Q  $\cdot$ in] yme M (Q) im e Fr21 \textbf{22} Kay] kai G Key O R Z Keye M Keẏ Fr21  $\cdot$ erschrac] sprach O erscac Q  $\cdot$ begunde] gund R \textbf{23} dô] Dach M Da Z  $\cdot$ bistûz] bistu L Q (R)  $\cdot$ Kingrun] kyngrvn O M (R) kýngrvn L kyngrún Q \textbf{24} âvoy] Awe O Owý L Ovoy M  $\cdot$ Britun] pritun I Brittvn L brittúm Q Briton R \textbf{25} hât] Han M \textbf{26} Gnvger pris gein dir verswant L  $\cdot$ dû] durch I Do Q Dv bist Fr21  $\cdot$ Clamides] clamide I Glamides O  $\cdot$ schinschalt] shinischalt I senechant O sinetschalt M senec scant Q Sinetschant R smetschalant Z senetschant Fr21 \textbf{27} mir] dy M \textbf{28} doch] \textit{om.} O Fr21 \textbf{29} uns] vnz L \textbf{30} Brandigan] Brandigon R \newline
\end{minipage}
\hspace{0.5cm}
\begin{minipage}[t]{0.5\linewidth}
\small
\begin{center}*T
\end{center}
\begin{tabular}{rl}
 & \textbf{ir} ebenhœhe unde ir mangen,\\ 
 & swaz ûf redern ko\textit{m} gegangen,\\ 
 & \textbf{katzen}, \textbf{igele} in den graben,\\ 
 & \textbf{daz} kunde viur \textbf{hin} dan wol schaben.\\ 
5 & \textbf{\begin{large}N\end{large}û was ouch} Kyngrun \textbf{dan gewant}\\ 
 & \textbf{unde} \textbf{kom} ze Britanie in daz lant.\\ 
 & \textbf{er} vant den künec Artus\\ 
 & \textbf{in} Brecilian \textbf{in} dem weidehûs,\\ 
 & daz was geheizen Carminal.\\ 
10 & dô warb er, als in Parcifal\\ 
 & gevangen hete dar gesant.\\ 
 & vroun Cunnewaren de Lalant\\ 
 & brâhter sîne sicherheit.\\ 
 & \textbf{Dô was diu juncvrouwe} gemeit,\\ 
15 & daz mit triuwen klagete ir nôt,\\ 
 & den man dâ hiez \textbf{den} rîter rôt.\\ 
 & Über \textbf{al} \textbf{daz lant} \textbf{daz} mære wart vernomen.\\ 
 & \textbf{Nû} was ouch vür den künec komen\\ 
 & der betwungene werde man.\\ 
20 & im unde der massenîe sân\\ 
 & sageter, waz \textbf{im} was enboten.\\ 
 & Key erschrac unde begunde rôten.\\ 
 & \textbf{dô} sprach er: "bist dûz, Kyngrun?\\ 
 & Âvoy, wie manegen Britun\\ 
25 & hât \textbf{uns} entschumpfieret dîn hant!\\ 
 & \textbf{gevuoger prîs gegen dir verswant.}\\ 
 & wirt mir dîn meister niemer holt,\\ 
 & dînes \textbf{dienstes} dû doch geniezen solt.\\ 
 & \textbf{uns ist der kezze\textit{l}} undertân,\\ 
30 & mir hie unde dir ze Brandigan.\\ 
\end{tabular}
\scriptsize
\line(1,0){75} \newline
T U V W \newline
\line(1,0){75} \newline
\textbf{5} \textit{Initiale} T U  \textbf{14} \textit{Majuskel} T  \textbf{17} \textit{Majuskel} T  \textbf{18} \textit{Majuskel} T  \textbf{24} \textit{Majuskel} T  \newline
\line(1,0){75} \newline
\textbf{1} ebenhœhe] ebenhohen U (V) heben hohe W  $\cdot$ unde] \textit{om.} W \textbf{2} swaz] Waz U (W)  $\cdot$ redern] den redern V  $\cdot$ kom] con T kan V \textbf{3} den] die W \textbf{4} daz] Die W  $\cdot$ viur] daz vuͦr U (V) (W)  $\cdot$ wol] \textit{om.} W  $\cdot$ schaben] geschaben U (V) \textbf{5} Nû] Das W  $\cdot$ Kyngrun] kyngruͦn U kingrun V W  $\cdot$ dan gewant] [*]: schineschant V \textbf{6} unde kom ze] [*]: kvmmen V Vnd kam W  $\cdot$ Britanie] pritanie V britania W \textbf{7} Er vand do haime kúnig artus W  $\cdot$ er] [*]: Vnde V  $\cdot$ Artus] artuͦs U \textbf{8} in Brecilian] Jn Brizilian U [Jm]: Jn brezilian V Zuͦ prezilian W  $\cdot$ in dem] zuͦ dem U (V) \textbf{9} Carminal] [*arminal]: carminal V karminal W \textbf{10} Parcifal] parzifal V partzifal W \textbf{12} vroun] Vro U  $\cdot$ Cunnewaren] kvnnevaren V kunewarn W \textbf{16} dâ] do U \textit{om.} V W  $\cdot$ hiez] haist W \textbf{17} al] \textit{om.} W  $\cdot$ daz mære] die mere U dis mer W \textbf{21} waz] was T \textbf{22} Key] [k*]: keẏn V \textbf{23} Kyngrun] kŷngrvn T kyngruͦn U kingrun W \textbf{24} Britun] Brituͦn U brittvn V \textbf{26} [*]: Dv clamides schineschant V  $\cdot$ gevuoger] Gnuͦger W \textbf{28} dienstes] [*]: amtes V amptes W  $\cdot$ doch] \textit{om.} W \textbf{29} kezzel] kezzer T U \newline
\end{minipage}
\end{table}
\end{document}
