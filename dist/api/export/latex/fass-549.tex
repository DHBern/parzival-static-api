\documentclass[8pt,a4paper,notitlepage]{article}
\usepackage{fullpage}
\usepackage{ulem}
\usepackage{xltxtra}
\usepackage{datetime}
\renewcommand{\dateseparator}{.}
\dmyyyydate
\usepackage{fancyhdr}
\usepackage{ifthen}
\pagestyle{fancy}
\fancyhf{}
\renewcommand{\headrulewidth}{0pt}
\fancyfoot[L]{\ifthenelse{\value{page}=1}{\today, \currenttime{} Uhr}{}}
\begin{document}
\begin{table}[ht]
\begin{minipage}[t]{0.5\linewidth}
\small
\begin{center}*D
\end{center}
\begin{tabular}{rl}
\textbf{549} & \begin{large}D\end{large}er wirt ze sîner tohter sprach:\\ 
 & "dû solt schaffen guot gemach\\ 
 & mîme hêrren, der hie stêt.\\ 
 & ir zwei mit ein ander gêt.\\ 
5 & \textbf{nû diene} im unverdrozzen;\\ 
 & wir hân sîn vil genozzen."\\ 
 & Sîme sune \textbf{bevalch} er Gringuljeten.\\ 
 & des diu magt was gebeten,\\ 
 & mit grôzer zuht daz \textbf{wart} getân.\\ 
10 & mit der meide Gawan\\ 
 & ûf eine kemenâten gienc.\\ 
 & den estrîch \textbf{al} übervienc\\ 
 & \textbf{niwer binz} \textbf{unt} bluomen \textbf{wol gevar}\\ 
 & wâren drûf gesniten \textbf{dar}.\\ 
15 & dâ entwâpende in diu süeze.\\ 
 & "got iu des \textbf{danken} müeze",\\ 
 & sprach Gawan. "\textbf{vrouwe}, \textbf{es ist mir} nôt.\\ 
 & wan daz manz iu von hove gebôt,\\ 
 & sô dient ir mir ze sêre."\\ 
20 & Si sprach: "ich diene iu mêre,\\ 
 & hêrre, nâch iweren hulden\\ 
 & denne von andern schulden."\\ 
 & \textbf{Des wirtes sun, ein knappe}, truoc\\ 
 & senfter bette \textbf{dar} genuoc.\\ 
25 & an die want gein der tür\\ 
 & ein teppech wart geleit dar vür;\\ 
 & dâ \textbf{solte Gawan} sitzen.\\ 
 & der knappe truoc mit witzen\\ 
 & eine kultern sô gemâl\\ 
30 & \textbf{ûfe}\textbf{z} \textbf{bette} von rôtem zindâl.\\ 
\end{tabular}
\scriptsize
\line(1,0){75} \newline
D \newline
\line(1,0){75} \newline
\textbf{1} \textit{Initiale} D  \textbf{7} \textit{Majuskel} D  \textbf{20} \textit{Majuskel} D  \textbf{23} \textit{Majuskel} D  \newline
\line(1,0){75} \newline
\textbf{7} Gringuljeten] Gringvlieten D \newline
\end{minipage}
\hspace{0.5cm}
\begin{minipage}[t]{0.5\linewidth}
\small
\begin{center}*m
\end{center}
\begin{tabular}{rl}
 & der wirt zuo sîner tohter sprach:\\ 
 & "dû solt schaffen guot gemach\\ 
 & mînem hêrre\textit{n}, \textit{d}er hie stêt.\\ 
 & ir zwei mit ein ander gêt.\\ 
5 & \textbf{nû diene} im unverdrozzen;\\ 
 & wir hân sîn vil genozzen."\\ 
 & sîne\textit{m} sun \textbf{enpfalch} er Gringuleten.\\ 
 & d\textit{e}s diu maget was gebeten,\\ 
 & mit grôzer zuht daz \textbf{wart} getân.\\ 
10 & mit der megde Gawan\\ 
 & ûf ein kemenâten gienc.\\ 
 & den \textit{e}strîch übervienc\\ 
 & \textbf{niu binz}, bluomen \dag und\dag  \textbf{wol gevar}\\ 
 & wâren dâr ûf gesniten \textbf{dar}.\\ 
15 & dô entwâpent in diu süeze.\\ 
 & "got iu des \textbf{danken} müeze",\\ 
 & sprach Gawan. "\textbf{es ist mir} nôt.\\ 
 & wan daz manz iu von hove gebôt,\\ 
 & s\textit{ô} \textit{d}ienet ir mir zuo sêre."\\ 
20 & si sprach: "ich diene iu mêre,\\ 
 & hêrre, nâch iuwern hulden\\ 
 & dan von andern schulden."\\ 
 & \textbf{des wirtes sun, ein knappe}, truoc\\ 
 & senfter bette \textbf{dar} genuoc.\\ 
25 & an die want gege\textit{n} \textit{d}er tür\\ 
 & ein teppich wart geleit dar vür;\\ 
 & d\textit{â} \textbf{solte Gawan} sitzen.\\ 
 & der knappe truoc mit witzen\\ 
 & ein kulter sô gemâl\\ 
30 & \textbf{ûf} \textbf{daz} \textbf{bette} von rôtem zendâl.\\ 
\end{tabular}
\scriptsize
\line(1,0){75} \newline
m n o \newline
\line(1,0){75} \newline
\newline
\line(1,0){75} \newline
\textbf{3} hêrren der] herren der der m \textbf{5} unverdrozzen] vnertroszen o \textbf{7} sînem] Sinen m o  $\cdot$ enpfalch] enpflach o  $\cdot$ Gringuleten] gringuletten m \textbf{8} des] Das m \textbf{9} wart] wasz o \textbf{12} estrîch] enstrich m \textbf{13} binz] wisz n \textbf{15} entwâpent] entwoppen n \textbf{16} got] [Dot]: Got o \textbf{19} sô dienet] So sẏ dienent m \textbf{20} diene] die o \textbf{23} wirtes] wurte o \textbf{25} gegen der] gegen ym der m \textbf{27} dâ] Do m n o \newline
\end{minipage}
\end{table}
\newpage
\begin{table}[ht]
\begin{minipage}[t]{0.5\linewidth}
\small
\begin{center}*G
\end{center}
\begin{tabular}{rl}
 & \begin{large}D\end{large}er wirt ze sîner tohter sprach:\\ 
 & "dû solt schaffen guo\textit{t} gemach\\ 
 & mînem hêrren, der hie stêt.\\ 
 & ir zwei mit ein ander gêt.\\ 
5 & \textbf{nû diene} im unverdrozzen;\\ 
 & wir hân sîn vil genozzen."\\ 
 & sînem sune \textbf{bevalch} er Gringulieten.\\ 
 & des diu maget was gebeten,\\ 
 & mit grôzer zuht daz \textbf{was} getân.\\ 
10 & mit der meide Gawan\\ 
 & ûf eine kemenâten gienc.\\ 
 & den estrîch \textbf{al} übervienc\\ 
 & \textbf{niuwer binez} \textbf{unde} bluomen \textbf{wol gevar}\\ 
 & wâren drûf gesniten \textbf{dar}.\\ 
15 & dô entwâpent in diu süeze.\\ 
 & "got iu des \textbf{danken} müeze",\\ 
 & sprach Gawan. "\textbf{vrouwe}, \textbf{ez ist mir} nôt.\\ 
 & wan daz manz iu von hove gebôt,\\ 
 & sô dient ir mir ze sêre."\\ 
20 & si sprach: "ich diene iu mêre,\\ 
 & hêrre, nâch iuwern hulde\textit{n}\\ 
 & danne von andern schulden."\\ 
 & \textbf{des wirtes sun, ein knappe}, truoc\\ 
 & senfter bette \textbf{dar} genuoc.\\ 
25 & an die want gein der tür\\ 
 & ein tepich wart geleit dar vür;\\ 
 & dâ \textbf{solde Gawan} sitzen.\\ 
 & der knappe truoc mit witzen\\ 
 & einen kulter sô gemâl\\ 
30 & \textbf{über} \textbf{bette} \textit{von} rôtem zendâl.\\ 
\end{tabular}
\scriptsize
\line(1,0){75} \newline
G I L M Z \newline
\line(1,0){75} \newline
\textbf{1} \textit{Initiale} G I L Z  \textbf{13} \textit{Initiale} I  \newline
\line(1,0){75} \newline
\textbf{2} guot] guͦte G \textbf{5} diene] dient I Z \textbf{6} hân] hab:: I  $\cdot$ vil] wol M \textbf{7} Gringulieten] Gringuͯlgeten L Gringvleten Z \textbf{9} was] wart I L Z \textbf{11} gienc] \textit{om.} Z \textbf{12} al] er al Z \textbf{13} binez] bimez G (I) (M) binzen L (Z) \textbf{14} dar] [gar]: dar Z \textbf{15} dô] Da L M Z \textbf{16} des] \textit{om.} L \textbf{17} ez] sin I \textbf{21} nâch] von Z  $\cdot$ hulden] hulde G \textbf{22} danne] da I \textbf{24} senfter] Senfte L (M) \textbf{28} knappe] [cha]: chnappe G  $\cdot$ witzen] wizzen M \textbf{29} einen] Eyn M \textbf{30} über] Vf daz L (M) (Z)  $\cdot$ von] mit G \newline
\end{minipage}
\hspace{0.5cm}
\begin{minipage}[t]{0.5\linewidth}
\small
\begin{center}*T
\end{center}
\begin{tabular}{rl}
 & \textit{\begin{large}D\end{large}}er wirt zuo sîner tohter sprach:\\ 
 & "dû solt schaffen guot gemach\\ 
 & mînem hêrren, der hie stêt.\\ 
 & ir zwei mit ein ander gêt\\ 
5 & \textbf{unde dient} im unverdrozzen;\\ 
 & wir haben sîn vil genozzen."\\ 
 & sînem sun \textbf{bevalch} er Kryngulieten.\\ 
 & Des diu maget was gebeten,\\ 
 & mit grôzer zuht daz \textbf{wart} getân.\\ 
10 & \textit{mit der megde Gawan}\\ 
 & ûf eine kemenâten gienc.\\ 
 & den estrîch \textbf{al} übervienc\\ 
 & \textbf{niuwe binzen} \textbf{und\textit{e}} \textit{b}luomen \textbf{clâr}\\ 
 & wâren drûf gesniten \textbf{gar}.\\ 
15 & Dô entwâpent in diu süeze.\\ 
 & "got iu des \textbf{lônen} müeze",\\ 
 & sprach Gawan. "\textbf{mir ist sîn} nôt.\\ 
 & wan daz manz iu von hove gebôt,\\ 
 & sô dient ir mir ze sêre."\\ 
20 & Si sprach: "ich dieniu mêre,\\ 
 & hêrre, nâch iuwern hulden\\ 
 & denne von andern schulden."\\ 
 & \textbf{Ein knappe, des wirtes sun}, \textbf{dar} truoc\\ 
 & senfter bette genuoc.\\ 
25 & an die want gein der tür\\ 
 & ein teppich wart geleit dar vür;\\ 
 & dâ \textbf{Gawan solte} sitzen.\\ 
 & der knappe truoc \textbf{dar} mit witzen\\ 
 & eine kulter sô gemâl\\ 
30 & \textbf{dar ûf} von rôtem zindâl.\\ 
\end{tabular}
\scriptsize
\line(1,0){75} \newline
T U V W O Q R \newline
\line(1,0){75} \newline
\textbf{1} \textit{Initiale} T U W  \textbf{8} \textit{Majuskel} T  \textbf{9} \textit{Initiale} O   $\cdot$ \textit{Capitulumzeichen} R  \textbf{15} \textit{Majuskel} T  \textbf{20} \textit{Majuskel} T  \textbf{23} \textit{Initiale} W   $\cdot$ \textit{Majuskel} T  \newline
\line(1,0){75} \newline
\textbf{1} Der] ÷er T \textbf{3} hêrren] herren gut gemach Q \textbf{6} vil] wol R \textbf{7} sînem] Sinen V Mein Q  $\cdot$ bevalch] enpfalh Q  $\cdot$ Kryngulieten] Kryngvlieten T kringulieren U gringuleten V kringuleten W kryngvleten O kinguheten Q kringuletten R \textbf{9} mit] ÷it O  $\cdot$ daz wart] was das W Q wart daz O (R) \textbf{10} \textit{Vers 549.10 fehlt, von späterer Hand nachgetragen:} Mit der megde her Gawan T  \textbf{11} ûf] Jn R  $\cdot$ kemenâten] kemenate U (Q) \textbf{12} al übervienc] allvmbe vieng V (W) al vm er vinc Q \textbf{13} niuwe] Neúwer W (O) (R) Hevwer Q  $\cdot$ binzen] bintz W (Q) bimz O  $\cdot$ unde bluomen] vnde vnde blvͦmen T  $\cdot$ clâr] wolgeuar W (O) (Q) (R) \textbf{14} wâren drûf] Varn druff Q Daruff wauren R  $\cdot$ gesniten] gestreut Q  $\cdot$ gar] dar W Q \textbf{17} Gawan] Gawain R  $\cdot$ mir] vrowe mir V (O) (R) fraw des Q  $\cdot$ sîn] mir Q \textbf{18} manz iu] man vch iz U man úch R  $\cdot$ gebôt] bot R \textbf{19} sô dient] Do mit Q \textbf{20} dieniu] diende v́ch V  $\cdot$ mêre] gere Q \textbf{21} \textit{teilsweise Textverlust 549.21-550.7 (Blatt teils abgeschnitten)} O   $\cdot$ nâch] von R \textbf{23} Ein] SEin W \textbf{24} senfter] Senfte Q \textbf{25} an] Jn Q \textbf{27} dâ] Do U V W Q  $\cdot$ Gawan] Gawin R \textbf{28} der] Ein R  $\cdot$ dar] \textit{om.} U V W O Q R \textbf{29} eine] Ein W Einen Q R  $\cdot$ kulter] gurtel Q \textbf{30} dar ûf] Vffens bette V \newline
\end{minipage}
\end{table}
\end{document}
