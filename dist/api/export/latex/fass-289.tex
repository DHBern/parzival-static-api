\documentclass[8pt,a4paper,notitlepage]{article}
\usepackage{fullpage}
\usepackage{ulem}
\usepackage{xltxtra}
\usepackage{datetime}
\renewcommand{\dateseparator}{.}
\dmyyyydate
\usepackage{fancyhdr}
\usepackage{ifthen}
\pagestyle{fancy}
\fancyhf{}
\renewcommand{\headrulewidth}{0pt}
\fancyfoot[L]{\ifthenelse{\value{page}=1}{\today, \currenttime{} Uhr}{}}
\begin{document}
\begin{table}[ht]
\begin{minipage}[t]{0.5\linewidth}
\small
\begin{center}*D
\end{center}
\begin{tabular}{rl}
\textbf{289} & \textbf{\begin{large}W\end{large}eder ern sprach dô} sus noch sô,\\ 
 & wan er schiet von \textbf{den} \textbf{witzen} dô.\\ 
 & Segramorses kastelân\\ 
 & huop sich gein sînem barn sân.\\ 
5 & er muose ûf durch \textbf{ruowen} stên,\\ 
 & ober \textbf{inder} wolde gên.\\ 
 & \textbf{sich legent genuoge} durch \textbf{ruowen} nider,\\ 
 & \textbf{daz} habt ir \textbf{dicke} vreischet sider.\\ 
 & waz ruowe kôs er in dem snê?\\ 
10 & mir tæte ein ligen drinne wê.\\ 
 & der schadehafte \textbf{erwarb} ie spot,\\ 
 & sælden pflihtære, \textbf{dem} half got.\\ 
 & \textbf{Daz} her lac wol sô nâhen,\\ 
 & daz si Parzivalen sâhen\\ 
15 & \textbf{haben}, als im was geschehen.\\ 
 & der minne, \textbf{er muose} \textbf{ir} siges jehen,\\ 
 & diu Salomonen ouch betwanc.\\ 
 & dâ nâch was \textbf{dô} niht ze lanc,\\ 
 & \textbf{ê} Segramors \textbf{dort} zuo \textbf{z}\textbf{in} gienc.\\ 
20 & \textbf{swer} in \textbf{hazzete oder der in wol} enpfienc,\\ 
 & den was er \textbf{al} gelîche holt.\\ 
 & sus teilt er bâgens grôzen solt.\\ 
 & Er sprach: "ir habt \textbf{des} vreischet vil,\\ 
 & rîterschaft ist topelspil\\ 
25 & \textbf{unt} daz ein man \textbf{von} tjoste viel.\\ 
 & ez sinket \textbf{halt} \textbf{ein} mers kiel.\\ 
 & lât \textbf{mich} nimmer niht gestrîten,\\ 
 & daz er mîn get\textit{ö}rste bîten,\\ 
 & ob er \textbf{bekande} mînen schilt.\\ 
30 & des hât mich gar an im bevilt,\\ 
\end{tabular}
\scriptsize
\line(1,0){75} \newline
D \newline
\line(1,0){75} \newline
\textbf{1} \textit{Initiale} D  \textbf{13} \textit{Majuskel} D  \textbf{23} \textit{Majuskel} D  \newline
\line(1,0){75} \newline
\textbf{3} Segramorses] Segramors D \textbf{28} getörste] getorste D \newline
\end{minipage}
\hspace{0.5cm}
\begin{minipage}[t]{0.5\linewidth}
\small
\begin{center}*m
\end{center}
\begin{tabular}{rl}
 & \textbf{weder er ensprach dô} sus noch sô,\\ 
 & wanne er schiet \textbf{aber} von \textbf{herzen} dô.\\ 
 & Segramorses kastelân\\ 
 & huop sich gegen sînem barn sân.\\ 
5 & er muose ûf durch \textbf{ruowen} stên,\\ 
 & ob er \textbf{iemer} wolte gên.\\ 
 & \textbf{sich} \dag gent\dag  \textbf{genuoge} durch \textbf{ruowen} nider,\\ 
 & \textbf{daz} habet ir \textbf{dicke} \textit{ge}vreische\textit{t} sider.\\ 
 & waz ruowe kôs er in dem snê?\\ 
10 & mir tæte ein ligen drinne wê.\\ 
 & der schadehafte \textbf{erwarp} i\textit{e} spot,\\ 
 & sæl\textit{d}en pflihtære, \textbf{dem} half got.\\ 
 & \textbf{daz} her lac wol sô nâhen,\\ 
 & daz si Parcifale\textit{n} \textit{s}âhen\\ 
15 & \textbf{haben}, als ime was geschehen.\\ 
 & der minne \textbf{er muose} siges jehen,\\ 
 & diu Salomonen ouch betwanc.\\ 
 & dâ nâch was \textbf{dô} niht ze lanc,\\ 
 & \textbf{ê} Segramors \textbf{dort} zu\textit{o} \textbf{z}\textbf{im} gienc.\\ 
20 & \textbf{wer} in \textbf{hazzete oder} enpfienc,\\ 
 & den was er \textbf{als} gelîch holt.\\ 
 & sus teilt er bâgens grôzen solt.\\ 
 & er sprach: "ir habt \textbf{des} \textit{ge}vrei\textit{s}che\textit{t} vil\\ 
 & - ritterschaft ist topelspil -,\\ 
25 & daz ein man \textbf{von} juste viel.\\ 
 & ez sinket \textbf{joch} \textbf{eines} mers kiel.\\ 
 & lât \textbf{in} nieme\textit{r} niht gestrîten,\\ 
 & daz er mîn getörste bî\textit{t}en,\\ 
 & ob er \textbf{bekante} mînen schilt.\\ 
30 & des hât mich gar an ime bevilt,\\ 
\end{tabular}
\scriptsize
\line(1,0){75} \newline
m n o \newline
\line(1,0){75} \newline
\newline
\line(1,0){75} \newline
\textbf{1} weder] Wieder o  $\cdot$ dô] weder n \textit{om.} o \textbf{2} herzen] witzen n (o) \textbf{3} Segramorses] Segramors m n o \textbf{5} muose] muͯsse m muͯste n o \textbf{7} gent] clagen n lagen o  $\cdot$ ruowen] ruͦwe n (o) \textbf{8} habet ir] hab ich ir o  $\cdot$ gevreischet] freissen m gefreiset n o \textbf{10} mir tæte ein] Nit dette er o \textbf{11} ie] ir m \textbf{12} sælden pflihtære] Selten pflicht ere m Selten pfliget ere (erer o ) n (o)  $\cdot$ half] helffe n \textbf{14} Parcifalen sâhen] parcifalen iohen vnd sohen m parcifaln sohen o \textbf{15} geschehen] beschehen n o \textbf{16} muose] musse m muͯste n o \textbf{17} Salomonen] salamonen o \textbf{18} dô] \textit{om.} n o \textbf{19} Segramors] segromorsz n  $\cdot$ zuo] zuͯcz m  $\cdot$ zim] jme n (o) \textbf{20} hazzete] hasset n o  $\cdot$ enpfienc] wol enpfing n o \textbf{21} den] Dem o  $\cdot$ als gelîch] alle glich n alglich o \textbf{22} teilt] teilte n (o)  $\cdot$ bâgens] bagans n \textbf{23} des] \textit{om.} n o  $\cdot$ gevreischet] freiches m gefreiset o \textbf{25} von] mit n o \textbf{26} joch] doch n o \textbf{27} lât] [hat]: lat m  $\cdot$ niemer] niemen m  $\cdot$ niht] \textit{om.} n \textbf{28} getörste] gedurste n (o)  $\cdot$ bîten] bitten m \newline
\end{minipage}
\end{table}
\newpage
\begin{table}[ht]
\begin{minipage}[t]{0.5\linewidth}
\small
\begin{center}*G
\end{center}
\begin{tabular}{rl}
 & \textbf{sîn munt sprach weder} sus noch sô,\\ 
 & wan er schiet von \textbf{witzen} dô.\\ 
 & Segremorses kastelân\\ 
 & huop sich gein sînem barne sân.\\ 
5 & er muose ûf durch \textbf{ruowe} stên,\\ 
 & ober \textbf{inder} wolte gên.\\ 
 & \textbf{sich legent genuoge} durch \textbf{ruowe} nider,\\ 
 & \textbf{des} habt ir \textbf{vil} gevreischet sider.\\ 
 & waz ruowe kôs er in dem snê?\\ 
10 & mir tæte ein ligen drinne wê.\\ 
 & der schadehafte \textbf{warb} ie spot,\\ 
 & sælden pflihtær, \textbf{dem} half got.\\ 
 & \textbf{daz} her lac wol sô nâhen,\\ 
 & daz si Parzivalen sâhen\\ 
15 & \textbf{halten}, als im was geschehen.\\ 
 & der minne \textbf{muoser} siges jehen,\\ 
 & diu Salmonen ouch betwanc.\\ 
 & dar nâch was \textbf{ouch} niht ze lanc,\\ 
 & Segremors \textbf{dar} zuo \textbf{in} gienc.\\ 
20 & \textbf{der} \textit{in} \textbf{wol oder übel} enpfienc,\\ 
 & den was er \textbf{al}gelîche holt.\\ 
 & sus teilt er bâgens grôzen solt.\\ 
 & er sprach: "ir habet gevreischet vil,\\ 
 & rîterschaft ist topelspil\\ 
25 & \textbf{unt} daz ein man \textbf{von} tjoste viel.\\ 
 & ez sinket \textbf{halt} \textbf{ein} mers kiel.\\ 
 & lât \textbf{mich} nimer niht gestrîten,\\ 
 & daz er mîn get\textit{ö}rste bîten,\\ 
 & ob er \textbf{erkande} mînen schilt.\\ 
30 & des hât mich gar an im bevilt,\\ 
\end{tabular}
\scriptsize
\line(1,0){75} \newline
G I O L M Q R Z Fr40 \newline
\line(1,0){75} \newline
\textbf{3} \textit{Initiale} L Z  \textbf{5} \textit{Initiale} O Q  \textbf{13} \textit{Initiale} I   $\cdot$ \textit{Capitulumzeichen} L  \newline
\line(1,0){75} \newline
\textbf{1} \textit{Die Verse 288.15-293.2 fehlen} R   $\cdot$ Weder er sprach do sus noch so Z  $\cdot$ sîn] si I  $\cdot$ weder] \textit{om.} M  $\cdot$ sus] hy Q \textbf{2} von] von den O M Z \textbf{3} Segremorses] segremors G (I) (Q) (Z) SAýgremors L Segremors es M \textbf{4} sînem] dem O  $\cdot$ barne] barnen L  $\cdot$ sân] dan I (M) Q \textbf{5} er] ÷r O  $\cdot$ muose] muͤs I  $\cdot$ ruowe] triwe O rwue Q ruwen Z \textbf{6} inder] iender O L irgen M \textbf{7} ruowe] ruwen M (Z) \textbf{8} des] Das M \textbf{10} mir] Mit Q \textbf{11} warb ie] er warb îe O (Z) wirbet L \textbf{12} pflihtær] phlichtern M  $\cdot$ dem] \textit{om.} O L M Q Fr40  $\cdot$ half] hilfet I half îe O (L) (M) (Q) hal ie Fr40 \textbf{13} daz] Ditze O (L) (M)  $\cdot$ her] er Q Z \textbf{14} Parzivalen] parzifal I parcifaln O Z parcifalen L parzifaln M partzifal Q ::rzifalen Fr40 \textbf{16} muoser] muͤst er I er must ir Z \textbf{17} Salmonen] salomon I (M) salomonen O (L) (Z) [salamol]: salamon  Q \textbf{18} dar nâch] Dar nacha O  $\cdot$ ouch] do L Q Fr40 da M Z  $\cdot$ ze] \textit{om.} L \textbf{19} Segremors] Ê Segremors O (Z) Saýgremors L E sigremors M  $\cdot$ dar] dort Q Fr40  $\cdot$ in] ým L (Q) (Z) (Fr40) \textbf{20} Dern hetz oder wol enphiench O  $\cdot$ Der in haszet (hazzite M [ Q Z Fr40 ]) oder wol enphie L (M) (Q) (Z) (Fr40)  $\cdot$ in] \textit{om.} G \textbf{21} den] Dem L (M)  $\cdot$ er] \textit{om.} O Q  $\cdot$ algelîche] allen Geliche I (Fr40) \textbf{22} teilt] teilte L M  $\cdot$ bâgens] hagens Q hazzens Fr40 \textbf{23} habet] habt ez I habt des Z \textbf{24} \textit{Die Verse 289.24-25 fehlen} Z   $\cdot$ topelspil] tobis spil M \textbf{26} ez] Ezn O  $\cdot$ halt] ioch M halt als Q doch Z \textbf{27} gestrîten] streiten Q \textbf{28} mîn getörste] min getorste G I (L) Z (Fr40) torst min O myn torste M mein getroste Q  $\cdot$ bîten] en biten O beiten Fr40 \textbf{29} erkande] bechande O (M) (Q) kande Z \textbf{30} hât] het I  $\cdot$ an] am O \newline
\end{minipage}
\hspace{0.5cm}
\begin{minipage}[t]{0.5\linewidth}
\small
\begin{center}*T
\end{center}
\begin{tabular}{rl}
 & \textbf{sîn munt sprach weder} sus noch sô,\\ 
 & wand er schiet von \textbf{witzen} dô.\\ 
 & Segremorses kastelân\\ 
 & huop sich gegen sînem barn sân.\\ 
5 & er muose ûf durch \textbf{ruowe} stân,\\ 
 & ober \textbf{iender} wolte gân.\\ 
 & \textbf{Genuoge legent sich} durch \textbf{ruowe} nider,\\ 
 & \textbf{des} hât ir \textbf{vil} gevreischet sider.\\ 
 & waz ruowe kôs er in dem snê?\\ 
10 & mir tæte ein ligen drinne wê.\\ 
 & Der schadehafte \textbf{erwarp} ie spot,\\ 
 & sælden pflihtær half \textbf{ie} got.\\ 
 & \textbf{\begin{large}D\end{large}iz} her lac wol sô nâhen,\\ 
 & daz si Parcifaln sâhen\\ 
15 & \textbf{halten}, alse im was geschehen.\\ 
 & der minne \textbf{muoser} siges jehen,\\ 
 & diu Salomonem ouch betwanc.\\ 
 & dar nâch was \textbf{dô} niht ze lanc,\\ 
 & \textbf{Unz} Segremors \textbf{dar} zuo \textbf{z}\textbf{in} gienc.\\ 
20 & \textbf{der} in \textbf{hazzete oder wol} enpfienc,\\ 
 & den was er \textbf{al}glîche holt.\\ 
 & sus teilter bâgens grôzen solt.\\ 
 & er sprach: "ir habt gevreischet vil,\\ 
 & rîterschaft ist topelspil\\ 
25 & \textbf{unde} daz ein man \textbf{zer} tjost viel.\\ 
 & ez sinket \textbf{halt} \textbf{ein} mers kiel.\\ 
 & lât \textbf{mich} niemer niht gestrîten,\\ 
 & daz er mîn get\textit{ö}rste bîten,\\ 
 & ob er \textbf{bekande} mînen schilt.\\ 
30 & des hât mich gar an im bevilt,\\ 
\end{tabular}
\scriptsize
\line(1,0){75} \newline
T U V W \newline
\line(1,0){75} \newline
\textbf{7} \textit{Majuskel} T  \textbf{11} \textit{Majuskel} T  \textbf{13} \textit{Initiale} T U W  \textbf{19} \textit{Majuskel} T  \newline
\line(1,0){75} \newline
\textbf{1} sprach weder] was wider W \textbf{2} er] der U  $\cdot$ von] von den W \textbf{3} Segremorses] Segremors T W [S*]: Sagremors V \textbf{5} muose] mvese T [*]: muͤste V \textbf{6} ober] Vber U \textbf{7} \textit{Versfolge 289.8-7} V   $\cdot$ Genuoge legent sich] Sich legent gnvͦge V (W) \textbf{8} gevreischet] ervaren V \textbf{10} tæte] thet W \textbf{12} half] [haf*]: haft U dem half V \textbf{13} Diz] Daz V \textbf{14} Parcifaln] parzifaln T parzifalen V partzifaln W \textbf{16} minne] minnen U  $\cdot$ muoser] mveser T \textbf{17} Salomonem] salamonen V \textbf{18} dô] doch W \textbf{19} Unz] Mit U [D*]: Daz V  $\cdot$ Segremors] [*]: sagremors V  $\cdot$ zin] in U W [zim]: zin V \textbf{20} hazzete] hasset W \textbf{21} alglîche] alle gliche U allen gleiche W \textbf{22} teilter] teilt er V (W)  $\cdot$ bâgens] beiagenes W \textbf{23} habt] habet das W  $\cdot$ gevreischet] ervaren V erfreischet W \textbf{25} unde daz] [*]: Daz V  $\cdot$ zer] von V \textbf{26} halt] ioch V \textbf{28} getörste] getorste T U  $\cdot$ bîten] beiten U W \textbf{30} des] Das W \newline
\end{minipage}
\end{table}
\end{document}
