\documentclass[8pt,a4paper,notitlepage]{article}
\usepackage{fullpage}
\usepackage{ulem}
\usepackage{xltxtra}
\usepackage{datetime}
\renewcommand{\dateseparator}{.}
\dmyyyydate
\usepackage{fancyhdr}
\usepackage{ifthen}
\pagestyle{fancy}
\fancyhf{}
\renewcommand{\headrulewidth}{0pt}
\fancyfoot[L]{\ifthenelse{\value{page}=1}{\today, \currenttime{} Uhr}{}}
\begin{document}
\begin{table}[ht]
\begin{minipage}[t]{0.5\linewidth}
\small
\begin{center}*D
\end{center}
\begin{tabular}{rl}
\textbf{706} & \begin{large}D\end{large}az her was komen ze bêder sît\\ 
 & ûf \textbf{den} grüenen anger wît,\\ 
 & iewederhalp an \textbf{sîniu} zil.\\ 
 & si prüeveten \textbf{diz} nîtspil.\\ 
5 & den küenen wîganden\\ 
 & dors wâren gestanden.\\ 
 & dô striten sus die werden\\ 
 & ze vuoz ûf der erden\\ 
 & einen herten strît scharpf erkant.\\ 
10 & diu swert \textbf{ûf} hôhe ûz der hant\\ 
 & wurfen \textbf{dicke} \textbf{dise} recken;\\ 
 & \textbf{si} wandelten die ecken.\\ 
 & Sus enpfienc \textbf{der künec} Gramoflanz\\ 
 & sûren zins vür sînen kranz.\\ 
15 & sîner vriwendinne künne\\ 
 & leit ouch bî im swache wünne.\\ 
 & sus engalt der werde Parzival\\ 
 & Itonjen, der lieht gemâl,\\ 
 & der er geniezen solde,\\ 
20 & ob reht ze rehte wolde.\\ 
 & Nâch prîse die vil gevarnen\\ 
 & mit strîte \textbf{muosen} arnen,\\ 
 & einer streit vür vriwendes nôt,\\ 
 & dem andern minne daz gebôt,\\ 
25 & daz er was minne undertân.\\ 
 & Dô kom \textbf{ouch} mîn hêr Gawan,\\ 
 & dô ez \textbf{alsus vil nâch} was komen,\\ 
 & daz den sig \textbf{hete aldâ} genomen\\ 
 & der \textbf{stolze, küene} Waleis.\\ 
30 & Brandelidelin von Punturteis\\ 
\end{tabular}
\scriptsize
\line(1,0){75} \newline
D Fr66 \newline
\line(1,0){75} \newline
\textbf{1} \textit{Initiale} D  \textbf{13} \textit{Majuskel} D  \textbf{21} \textit{Majuskel} D  \textbf{26} \textit{Majuskel} D  \newline
\line(1,0){75} \newline
\textbf{17} Parzival] Parcifal D \textbf{18} Itonjen] Jtonîen D \textbf{30} Punturteis] Pvntvrtêis D \newline
\end{minipage}
\hspace{0.5cm}
\begin{minipage}[t]{0.5\linewidth}
\small
\begin{center}*m
\end{center}
\begin{tabular}{rl}
 & \dag der\dag  \textit{her} was komen zuo beider sît\\ 
 & ûf \textbf{dem} grüenen anger wît,\\ 
 & ietwederhalp an \textbf{sîn} zil.\\ 
 & si bruoften \textbf{daz} nîtspil.\\ 
5 & den k\textit{üen}en wîganden\\ 
 & diu ros wâren gestanden.\\ 
 & dô striten sus die werden\\ 
 & zuo vuoz ûf der erden\\ 
 & einen herten \textit{st}rî\textit{t} scharpf erkant.\\ 
10 & diu swert hôhe ûz der hant\\ 
 & wurfen \textbf{dô} \textbf{die} recken\\ 
 & \textbf{und} wandelten die ecken.\\ 
 & sus enpfienc \textbf{der künic} Gramolanz\\ 
 & sûren zins vür sînen kranz.\\ 
15 & sîner vriundinne künne\\ 
 & \textit{l}eit ouch bî ime swache wünne.\\ 
 & sus engalt der werde Parcifal\\ 
 & Ithonien, der lieht gemâl,\\ 
 & der er geniezen solte,\\ 
20 & ob reht zuo rehte wolte.\\ 
 & nâch prîse die vil gevarnen\\ 
 & mit strît \textbf{prîs} \textbf{müezen} arnen:\\ 
 & einer str\textit{e}it vür vriundes nôt,\\ 
 & dem ander\textit{n} minne daz gebôt,\\ 
25 & daz er wa\textit{s} \textit{m}inne undertân.\\ 
 & dô kam mîn hêr Gawan,\\ 
 & d\textit{ô} ez \textbf{vil nâch alsus} was komen,\\ 
 & daz den sic \textbf{aldâ het} genomen\\ 
 & der \textbf{küene, stolze} Waleis.\\ 
30 & Brandelidelin von Ponturt\textit{e}is\\ 
\end{tabular}
\scriptsize
\line(1,0){75} \newline
m n o \newline
\line(1,0){75} \newline
\newline
\line(1,0){75} \newline
\textbf{1} her] \textit{om.} m \textbf{5} küenen] komen m \textbf{9} strît] pris m \textbf{11} wurfen] :urffen o \textbf{13} Gramolanz] gramolantz m n gramolancz o \textbf{15} vriundinne künne] fruͯindeme kommen o \textbf{16} leit] Keit m  $\cdot$ ime] in n \textbf{18} Ithonien] Jtonien m o Jthonẏen n \textbf{21} gevarnen] genaren n \textbf{22} strît prîs] prisz strit n >strit< pris o \textbf{23} streit] strit m \textbf{24} andern] ander m \textbf{25} was minne] was [in]: mẏnne mynne m was mẏnne mine o \textbf{26} kam] kam ouch n (o)  $\cdot$ hêr] herre her n \textbf{27} dô] Da m o  $\cdot$ was] w:: o \textbf{29} küene stolze] stoltze kuͯne n (o) \textbf{30} Brandelidelin] Prandelidelin n Brandeledelin o  $\cdot$ Ponturteis] ponturtois m [*]: purturteis n pontúrteis o \newline
\end{minipage}
\end{table}
\newpage
\begin{table}[ht]
\begin{minipage}[t]{0.5\linewidth}
\small
\begin{center}*G
\end{center}
\begin{tabular}{rl}
 & \begin{large}D\end{large}az her was komen ze bê\textit{de}r sît\\ 
 & ûf \textbf{den} grüenen anger wît,\\ 
 & ietwederhalp an \textbf{sîniu} zil.\\ 
 & s\textit{i} prüeveten \textbf{d\textit{it}z} nîtspil.\\ 
5 & den küenen wîganden\\ 
 & diu ors wâren gestanden.\\ 
 & dô striten sus die werden\\ 
 & ze vuoz ûf der erden\\ 
 & einen herten strît scharf erkant.\\ 
10 & diu swert hôch ûz der hant\\ 
 & wurfen \textbf{dicke} \textbf{die} recken;\\ 
 & \textbf{si} wandelten die ecken.\\ 
 & sus enpfie \textbf{der künic} Gramoflanz\\ 
 & sûren zins vür sînen kranz.\\ 
15 & sîner vriundinne künne\\ 
 & leit ouch bî im swache wünne.\\ 
 & sus engalt der werde Parcival\\ 
 & Itonien, der lieht gemâl,\\ 
 & der er geniezen solde,\\ 
20 & obe reht ze rehte wolde.\\ 
 & nâch prîse die \textit{vi}l gevarnen\\ 
 & mit strîte \textbf{muosen} arnen,\\ 
 & einer streit vür vriundes nôt,\\ 
 & dem andern minne daz gebôt,\\ 
25 & daz er was minne undertân.\\ 
 & dô kom \textbf{ouch} mîn hêr Gawan,\\ 
 & dô ez \textbf{vil nâch sus} was komen,\\ 
 & daz den sic \textbf{hete} genomen\\ 
 & der \textbf{stolze, küene} Waleis.\\ 
30 & Brandelidelin von Ponturteis\\ 
\end{tabular}
\scriptsize
\line(1,0){75} \newline
G I L M Z Fr18 \newline
\line(1,0){75} \newline
\textbf{1} \textit{Initiale} G I L Z Fr18  \textbf{21} \textit{Initiale} I  \newline
\line(1,0){75} \newline
\textbf{1} komen] \textit{om.} M Fr18  $\cdot$ ze bêder sît] zeber sit G zuͯ beiden siten L (M) \textbf{2} wît] witen L (M) \textbf{3} ietwederhalp] Jcslichir M  $\cdot$ an] \textit{om.} Fr18  $\cdot$ sîniu] sin I sinem M \textbf{4} si] sit G  $\cdot$ ditz] daz G  $\cdot$ nîtspil] mit spil I M \textbf{6} Warn die rosz gestanden L \textbf{7} dô] Da M Z Fr18  $\cdot$ sus] sẏ L \textbf{8} vuoz] uuͤzen I \textbf{9} herten strît scharf] herten sharphen strit I strit scharff M \textbf{11} dicke] \textit{om.} L \textbf{13} enpfie] einphie L  $\cdot$ Gramoflanz] gramorflanz M gramoflantz Z (Fr18) \textbf{14} sûren] swern I \textbf{15} vriundinne] vriunde I \textbf{16} ouch] \textit{om.} I \textbf{17} Parcival] parcifal G Z Fr18 parzifal I L M \textbf{18} Itonien] Jconien Z Itonẏen Fr18  $\cdot$ lieht] licht L M \textbf{19} geniezen] [engelten so]: Geniezen I \textbf{21} nâch] Nac I  $\cdot$ vil] wol G \textbf{22} muosen] die muͦsen I \textbf{23} nôt] tot I \textbf{24} andern] ander L \textbf{25} er] \textit{om.} M \textbf{26} dô] Da M Z  $\cdot$ ouch] \textit{om.} M  $\cdot$ Gawan] [*]: gawan Z \textbf{27} dô] Da M Z  $\cdot$ sus] alsus L M Z (Fr18) \textbf{30} Brandelidelin] brandalidelin I Branlidelin L Brandeliden M Brandlidelin Fr18  $\cdot$ Ponturteis] ponturtoys I pvntvrteiz L punturteis M (Z) (Fr18) \newline
\end{minipage}
\hspace{0.5cm}
\begin{minipage}[t]{0.5\linewidth}
\small
\begin{center}*T
\end{center}
\begin{tabular}{rl}
 & \begin{large}D\end{large}az her was komen zuo beider sît\\ 
 & ûf \textbf{den} grüenen anger wît,\\ 
 & ietwederhalp an \textbf{sîn} zil.\\ 
 & si prüeveten \textbf{diz} nîtspil.\\ 
5 & den küenen wîganden\\ 
 & diu ors wâren gestanden.\\ 
 & dô striten sus die werden\\ 
 & zuo vuoz ûf der erden\\ 
 & einen herten strît scharpf erkant.\\ 
10 & diu swert hôch ûz der hant\\ 
 & wurfen \textbf{dicke} \textbf{die} recken;\\ 
 & \textbf{si} wandelten die ecken.\\ 
 & sus entvienc Gramoflanz\\ 
 & sûren zins vür sînen kranz.\\ 
15 & sîner vriundinne künne\\ 
 & leit ouch bî im swache wünne.\\ 
 & sus engalt der werde Parcifal\\ 
 & Itonien, der lieht gemâl,\\ 
 & der er geniezen solte,\\ 
20 & o\textit{b} reht zuo rehte wolte.\\ 
 & nâch prîse die vil gevarnen\\ 
 & mit strîte \textbf{muosen} \textit{a}rnen,\\ 
 & einer streit vür vriundes nôt,\\ 
 & dem andern minne daz gebôt,\\ 
25 & daz er was minnen undertân.\\ 
 & dô kam \textbf{ouch} mîn hêr Gawan,\\ 
 & dô ez \textbf{vil nâch alsus} was komen,\\ 
 & daz den sic \textbf{hete} genomen\\ 
 & der \textbf{stolze, küene} Waleis.\\ 
30 & Brandelidelin von Punterteis\\ 
\end{tabular}
\scriptsize
\line(1,0){75} \newline
U V W Q R \newline
\line(1,0){75} \newline
\textbf{1} \textit{Initiale} U W R  \newline
\line(1,0){75} \newline
\textbf{1} her] er Q \textbf{3} sîn] seine W Q (R) \textbf{4} diz] daz V  $\cdot$ nîtspil] [*]: nit spil V mit spil W \textbf{6} wâren] worden Q \textbf{7} sus] ausz Q \textbf{9} herten] starcken W \textbf{11} dicke] \textit{om.} R \textbf{13} sus] Als Q  $\cdot$ Gramoflanz] Gramaflanz V gramoflantz W der konigk gramoflantz Q der kung Gramiflancz R \textbf{16} \textit{Vers 706.16 fehlt} R  \textbf{17} sus] Als Q  $\cdot$ Parcifal] Parzifal U parzefal V partzifal W Q parczifal R \textbf{18} Itonien] Jtonien U R ytonien V (W) (Q) \textbf{20} ob] Oder U \textbf{21} gevarnen] gefaren R \textbf{22} Mit strite [*]: pris mvͤstent arnen V  $\cdot$ muosen] muͤssen W (R)  $\cdot$ arnen] garnen U \textbf{23} streit vür] strit fᵫr sin R \textbf{25} minnen] minne W R meine Q \textbf{27} dô] Da R  $\cdot$ alsus] sus W als Q \textbf{28} hete genomen] [ha*]: hatte aldo genomen V \textbf{29} Waleis] walleis V maleiß W Valeis R \textbf{30} Brandelidelin] Brandelidelein W brandlidelin Q (R)  $\cdot$ Punterteis] Ponterteis U ponturteis W puntorteis Q puͯntuͯrteis R \newline
\end{minipage}
\end{table}
\end{document}
