\documentclass[8pt,a4paper,notitlepage]{article}
\usepackage{fullpage}
\usepackage{ulem}
\usepackage{xltxtra}
\usepackage{datetime}
\renewcommand{\dateseparator}{.}
\dmyyyydate
\usepackage{fancyhdr}
\usepackage{ifthen}
\pagestyle{fancy}
\fancyhf{}
\renewcommand{\headrulewidth}{0pt}
\fancyfoot[L]{\ifthenelse{\value{page}=1}{\today, \currenttime{} Uhr}{}}
\begin{document}
\begin{table}[ht]
\begin{minipage}[t]{0.5\linewidth}
\small
\begin{center}*D
\end{center}
\begin{tabular}{rl}
\textbf{358} & "\textbf{\begin{large}N\end{large}û sich", sprach si}, "swester mîn,\\ 
 & deiswâr, mîn ritter unt der dîn\\ 
 & begênt hie ungelîchiu werc.\\ 
 & der dîne wænet, daz wir den berc\\ 
5 & unt die burc sulen verliesen.\\ 
 & ander wer wir muozen kiesen."\\ 
 & Diu junge muose ir spotten doln.\\ 
 & \textbf{diu} sprach: "er mac sichs wol erholn.\\ 
 & ich gib im \textbf{noch} gegen ellen trôst,\\ 
10 & daz er dînes spottes wirt erlôst.\\ 
 & er sol dienst gein mir kêren\\ 
 & \textbf{unt} ich \textbf{wil} im vreude mêren.\\ 
 & sît dû gihst, er sî ein koufman,\\ 
 & er sol mînes lônes market hân."\\ 
15 & Ir bêder strît der worte\\ 
 & Gawan ze merke hôrte.\\ 
 & als ez im dô getohte,\\ 
 & \textbf{übersaz} \textbf{er}z, swie er mohte.\\ 
 & \textbf{sol lûter herze sich niht schemen},\\ 
20 & daz \textbf{müeze der tôt dâr von ê nemen}.\\ 
 & Daz grôze her al stille lac,\\ 
 & des Poydiconjunz \textbf{dort} pflac,\\ 
 & wan ein werder jungelinc\\ 
 & was ime strîte unt \textbf{al}sîn rinc,\\ 
25 & der herzoge \textbf{von} Lanverunz.\\ 
 & dô kom Poydiconjunz.\\ 
 & \textbf{ouch nam} der altwîse man\\ 
 & \textbf{die} eine unt die andern dan.\\ 
 & \textbf{diu vesperîe was erliten}\\ 
30 & unt wol durch \textbf{werdiu} wîp gestriten.\\ 
\end{tabular}
\scriptsize
\line(1,0){75} \newline
D \newline
\line(1,0){75} \newline
\textbf{1} \textit{Initiale} D  \textbf{7} \textit{Majuskel} D  \textbf{15} \textit{Majuskel} D  \textbf{21} \textit{Majuskel} D  \newline
\line(1,0){75} \newline
\textbf{22} Poydiconjunz] Poydiconivnz D \textbf{26} Poydiconjunz] Poydiconivnz D \textbf{27} altwîse] alt wise D \newline
\end{minipage}
\hspace{0.5cm}
\begin{minipage}[t]{0.5\linewidth}
\small
\begin{center}*m
\end{center}
\begin{tabular}{rl}
 & "\textbf{nû sich", sprach si}, "swester mîn,\\ 
 & deiswâr, mîn ritter und der dîn,\\ 
 & \textbf{und} begânt hie ungelîchiu werc.\\ 
 & der dîne wænet, daz wir den berc\\ 
5 & und die burc sullen verliesen.\\ 
 & ander wer \textit{w}i\textit{r} muozen kiesen."\\ 
 & diu junge muose ir spotten doln.\\ 
 & \textbf{diu} sprach: "er mac sichs wol erholn.\\ 
 & ich gib ime \textbf{noch} gegen ellen trôst,\\ 
10 & daz er dînes spottes wirt erlôst.\\ 
 & er sol dienest gegen mir kêren,\\ 
 & ich \textbf{wil} ime vröude mêren.\\ 
 & \textit{sît dû gihst, er sî ein koufman,}\\ 
 & \textit{er sol mînes lônes market hân."}\\ 
15 & \textit{ir beider strît der worte}\\ 
 & \textit{Gawan zuo merke hôrte}\\ 
 & \hspace*{-.7em}\big| \textbf{und} \textbf{übersaz} ez, wie er mohte,\\ 
 & \hspace*{-.7em}\big| als ez ime dô getohte,\\ 
 & \textbf{wand er was schamlîch und doch wîs}.\\ 
20 & daz \textbf{vuogete ime dicke hôhen prîs}.\\ 
 & \begin{large}D\end{large}az grôze her a\textit{l s}tille lac,\\ 
 & des Poidiconiunz \textbf{dort} pflac,\\ 
 & wanne ein werder jungelinc\\ 
 & was in \textit{dem} strîte und \textbf{als} sîn rinc,\\ 
25 & der herzoge \textbf{von} Laverunz.\\ 
 & dô kam Poidic\textit{o}niunz,\\ 
 & der \textbf{werde}, alte, wîse man,\\ 
 & \textbf{und nam} eine und die anderen dan.\\ 
 & \textbf{doch was diu vesperîe vermiten}\\ 
30 & und wol durch \textbf{werdiu} wîp gestriten.\\ 
\end{tabular}
\scriptsize
\line(1,0){75} \newline
m n o \newline
\line(1,0){75} \newline
\textbf{21} \textit{Initiale} m   $\cdot$ \textit{Capitulumzeichen} n  \newline
\line(1,0){75} \newline
\textbf{3} ungelîchiu] vnglich n \textbf{4} wir] [mir]: wir o \textbf{6} wir] iv m \textbf{7} diu] Der o  $\cdot$ muose] muͯsse m \textbf{8} diu] Sú n (o) \textbf{13} \textit{Die Verse 358.13-16 fehlen} m  \textbf{18} er] sie o  $\cdot$ mohte] moͯchte n \textbf{17} getohte] doͯchte n o \textbf{19} wîs] wise m n o \textbf{20} prîs] prise m n o \textbf{21} al stille] all alle stille m alle stille n o \textbf{22} des] Das n (o)  $\cdot$ Poidiconiunz] poidiconivncz m poidicomitz n poidicamuͯncz o \textbf{24} dem strîte] mastrite m \textbf{25} Laverunz] lauerúncz m lauerúntze n laneruͯncze o \textbf{26} Poidiconiunz] poidicunivnz m poidocomitz n podicaniuncz o \textbf{27} werde alte] alte werde n \textbf{28} anderen] ander o \textbf{29} vermiten] erlitten n o \textbf{30} wîp] vmb n  $\cdot$ gestriten] stritten n erstritten o \newline
\end{minipage}
\end{table}
\newpage
\begin{table}[ht]
\begin{minipage}[t]{0.5\linewidth}
\small
\begin{center}*G
\end{center}
\begin{tabular}{rl}
 & \textbf{dô sprach si: "sihestû}, swester mîn,\\ 
 & dêswâr, mîn rîter unde der dîn\\ 
 & begênt hie ungelîchiu werc.\\ 
 & der dîne wænet, daz wir den berc\\ 
5 & unt die burc sulen verliesen.\\ 
 & ander wer wir müezen kiesen."\\ 
 & diu junge muose ir spoten dolen.\\ 
 & \textbf{si} sprach: "er mac sich es wol erholen.\\ 
 & ich gibe im \textbf{doch} gein ellen trôst,\\ 
10 & daz er dînes spotes wirt erlôst.\\ 
 & er sol dienst gein mir kêren\\ 
 & \textbf{unde} ich \textbf{sol} im vröide mêren.\\ 
 & sît dû gihest, er sî ein koufman,\\ 
 & er sol mînes lônes market hân."\\ 
15 & ir beider strît der worte\\ 
 & Gawan ze merke hôrte.\\ 
 & als ez im dô getohte,\\ 
 & \textbf{versaz} \textbf{er}z, swie er mohte.\\ 
 & \textbf{sol lûter herze sich niht schemen},\\ 
20 & daz \textbf{muoz der tôt dar von ê nemen}.\\ 
 & daz grôze her al stille lac,\\ 
 & des Poydekoniunz \textbf{dâ} pflac,\\ 
 & wan ein werder jungelinc\\ 
 & was ime strîte unde \textbf{al} sîn rinc,\\ 
25 & der herzoge \textbf{von} Lanvarunz.\\ 
 & dô kom Poydeconiunz.\\ 
 & \textbf{ouch nam} der altwîse man\\ 
 & \textbf{die} einen unde die anderen dan.\\ 
 & \textbf{diu vesperîe was erliten}\\ 
30 & unde wol durch \textbf{werdiu} wîp gestriten.\\ 
\end{tabular}
\scriptsize
\line(1,0){75} \newline
G I O L M Q R Z Fr39 \newline
\line(1,0){75} \newline
\textbf{11} \textit{Initiale} I O L Q Z Fr39   $\cdot$ \textit{Capitulumzeichen} R  \textbf{21} \textit{Initiale} I   $\cdot$ \textit{Capitulumzeichen} R  \newline
\line(1,0){75} \newline
\textbf{1} dô] Da O M  $\cdot$ si] \textit{om.} Z \textbf{2} deswar der riter min vnd tin I  $\cdot$ Das ist war din Ritter vnd der min R  $\cdot$ dêswâr] Zwar Q Z \textbf{3} ungelîchiu] vngliche R \textbf{4} wænet] weninc M werit Q  $\cdot$ wir] wir wir L \textbf{6} ander] Jn der L  $\cdot$ wir müezen] muͤzen wir vns I \textbf{7} diu] Die Fr39  $\cdot$ muose] muͤse Fr39  $\cdot$ ir] irn R  $\cdot$ spoten] spotte Q (R)  $\cdot$ dolen] [lan]: doln Fr39 \textbf{8} sich es] sis I Z sich O L M Q R Fr39 \textbf{9} im] in Q R  $\cdot$ doch] noch O L M Q R (Z) Fr39  $\cdot$ gein] \textit{om.} O  $\cdot$ ellen] allen Q (Fr39) \textbf{10} spotes] spottens I L Z Fr39 \textbf{11} er] ÷r O  $\cdot$ sol] solt Q \textbf{12} sol] wil O L M Q R Z Fr39 \textbf{13} \textit{Versfolge 358.14-13} I   $\cdot$ sît] Sider R  $\cdot$ gihest] sprichst M \textbf{14} lônes] lotes I \textbf{15} der] \textit{om.} O  $\cdot$ worte] gerte I worchte L wortten R \textbf{16} ze merke] dye markt vnd R \textbf{17} dô] da M R Z  $\cdot$ getohte] gidachte M (Q) \textbf{18} versaz] Vber saz O (L) (M) (Q) (Fr39) V́ber sach R (Z)  $\cdot$ swie] wie O L (M) Q R Z \textbf{19} sol] Da O  $\cdot$ schemen] smehen Q \textbf{20} dar von ê] e im I ê da von O da von ym M  $\cdot$ nemen] benemen I \textbf{21} al] da I also M (R) \textbf{22} des] daz I De R  $\cdot$ Poydekoniunz] [poydokomunz]: Poydokomunz I Poydekvmvnz O poýde Conjvnz L poide kvnivnz M poydekoműnz Q poydekonivires R poidekonivnz Z poy de Conivntz Fr39  $\cdot$ dâ] do Q R Fr39 \textbf{25} Lanvarunz] lauarunz I Lanveronz O Lvnvarvnz L (Fr39) lanvaruͯnz M lonvarúns Q Lonfarvnz R Lonvarvnz Z \textbf{26} dô] Da M  $\cdot$ Poydeconiunz] poydokomunz I der Poydekomvͦnz O Poý de Cvnivnz L poide konivunz M poydekomvnz Q pondekamevnz R poidekonivnz Z poy de Conivnz Fr39 \textbf{27} altwîse] alt wise G alte wiser I \textbf{29} diu] Dú R \textbf{30} werdiu] werde R \newline
\end{minipage}
\hspace{0.5cm}
\begin{minipage}[t]{0.5\linewidth}
\small
\begin{center}*T
\end{center}
\begin{tabular}{rl}
 & \textbf{Dô sprach si: "sihestû}, swester mîn,\\ 
 & deiswâr, mîn rîter unde der dîn\\ 
 & begânt hie ungelîchiu werc.\\ 
 & der dîne wænet, daz wir den berc\\ 
5 & unde die burc suln verliesen.\\ 
 & ander wer \textit{w}ir müezen kiesen."\\ 
 & \begin{large}D\end{large}iu junge muose ir spotten doln.\\ 
 & \textbf{si} sprach: "er mac sichs wol erholn.\\ 
 & ich gib im \textbf{noch} gegen ellen trôst,\\ 
10 & daz er dînes spottes wirt erlôst.\\ 
 & er sol dienst gegen mir kêren\\ 
 & \textbf{unde} ich im vröude mêren.\\ 
 & sît dû gihst, er sî ein koufman,\\ 
 & er sol mînes lônes market hân."\\ 
15 & ir beider strît der worte\\ 
 & Gawan ze merke hôrte,\\ 
 & \textit{als ez im dô getohte,}\\ 
 & \textit{\textbf{unde} \textbf{übersaz} ez, swie er mohte.}\\ 
 & \textbf{sol lûter herze sich niht schemen},\\ 
20 & daz \textbf{muoz der tôt dar von ê nemen}.\\ 
 & Daz grôze her alstille lac,\\ 
 & des Poydekuniunz \textbf{dâ} pflac,\\ 
 & wan \textbf{er} ein werder jungelinc\\ 
 & was \textit{i}m\textit{e} strîte unde \textbf{al}sîn rinc,\\ 
25 & Der herzoge Lunveruns.\\ 
 & Dô kom Poydekuniuns.\\ 
 & \textbf{ouch nam} der alte, wîse man\\ 
 & \textbf{die} einen unde die andern dan.\\ 
 & \textbf{diu vesperîe was erliten}\\ 
30 & unde wol durch \textbf{werde} wîp gestriten.\\ 
\end{tabular}
\scriptsize
\line(1,0){75} \newline
T V W \newline
\line(1,0){75} \newline
\textbf{1} \textit{Majuskel} T  \textbf{7} \textit{Initiale} T V  \textbf{21} \textit{Majuskel} T  \textbf{25} \textit{Majuskel} T  \textbf{26} \textit{Majuskel} T  \newline
\line(1,0){75} \newline
\textbf{1} sihestû] nv sich V \textbf{4} dîne] dein der W \textbf{6} wir] mir T \textbf{7} muose] mvese T muͤste V \textbf{8} sichs wol] sich noch W \textbf{9} im] \textit{om.} W  $\cdot$ ellen] \textit{om.} W \textbf{10} spottes] spottens V W \textbf{12} unde ich] Jch wil im V Vnd ich wil W \textbf{17} \textit{Die Verse 358.17-18 fehlen (Zeilen ausgespart)} T   $\cdot$ Wann das im so gedachte W \textbf{18} Do versas er wie er machte W \textbf{19} Wande er was schamlich vnde doch wis V  $\cdot$ niht] \textit{om.} W \textbf{20} Das fuͦgete im dicke hohen pris V  $\cdot$ ê] \textit{om.} W \textbf{22} Poydekuniunz] poydekvmvns V poyde guniunses W  $\cdot$ dâ] do V W \textbf{23} er] \textit{om.} W \textbf{24} Was komen in den ring W  $\cdot$ ime] dem T \textbf{25} herzoge] herzoge von V (W)  $\cdot$ Lunveruns] [lunuenuns]: lunueruns V luniueruns W \textbf{26} Poydekuniuns] poydekvmuns V poyde guniuns W \textbf{30} Vnd durch die weib wol gestritten W \newline
\end{minipage}
\end{table}
\end{document}
