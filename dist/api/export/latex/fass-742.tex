\documentclass[8pt,a4paper,notitlepage]{article}
\usepackage{fullpage}
\usepackage{ulem}
\usepackage{xltxtra}
\usepackage{datetime}
\renewcommand{\dateseparator}{.}
\dmyyyydate
\usepackage{fancyhdr}
\usepackage{ifthen}
\pagestyle{fancy}
\fancyhf{}
\renewcommand{\headrulewidth}{0pt}
\fancyfoot[L]{\ifthenelse{\value{page}=1}{\today, \currenttime{} Uhr}{}}
\begin{document}
\begin{table}[ht]
\begin{minipage}[t]{0.5\linewidth}
\small
\begin{center}*D
\end{center}
\begin{tabular}{rl}
\textbf{742} & \begin{large}D\end{large}er heiden truog \textbf{êt} starkiu lit.\\ 
 & swenner schrîte Thabronit\\ 
 & - \textbf{dâ} \textbf{diu} küneginne Secundille was\\ 
 & \textbf{vor} der montâne \textbf{ze} Koukesas -,\\ 
5 & \textbf{sô} gewan er \textbf{niwen} hôhen muot\\ 
 & gein dem, der \textbf{ie} was behuot\\ 
 & \textbf{vor} \textbf{solhem} strîtes überlast.\\ 
 & er was schumpfentiure ein gast,\\ 
 & daz er \textbf{si} nie gedolte,\\ 
10 & doch \textbf{si} \textbf{maneger} \textbf{zim} \textbf{erholte}.\\ 
 & Mit kunst si die arme \textbf{erswungen},\\ 
 & viwers blicke ûz helmen \textbf{sprungen},\\ 
 & von ir swerten gienc der sûre wint.\\ 
 & got \textbf{ner} dâ Gahmuretes kint!\\ 
15 & \textbf{der} \textbf{wunsch} wirt in beiden,\\ 
 & dem getouften unt dem heiden;\\ 
 & Die nante ich \textbf{ê} vür einen.\\ 
 & sus begunden siz \textbf{ouch} \textbf{meinen},\\ 
 & wæren si ein ander \textbf{baz} \textbf{bekant},\\ 
20 & si\textbf{ne} sazten niht sô hôhiu pfant.\\ 
 & Ir strît galt niht mêre\\ 
 & wan vreude, sælde und êre.\\ 
 & swer dâ den prîs gewinnet,\\ 
 & \textbf{ob} der triwe minnet,\\ 
25 & \textbf{werltlîche} vreude er hât verlorn\\ 
 & unt immer herzen riwe erkorn.\\ 
 & Wes sûmestû dich, Parzival,\\ 
 & daz dû an die kiuschen lieht gemâl\\ 
 & \textbf{niht} \textbf{denkest} - \textbf{ich meine} dîn wîp -,\\ 
30 & wiltû behalten hie den lîp?\\ 
\end{tabular}
\scriptsize
\line(1,0){75} \newline
D \newline
\line(1,0){75} \newline
\textbf{1} \textit{Initiale} D  \textbf{11} \textit{Majuskel} D  \textbf{17} \textit{Majuskel} D  \textbf{21} \textit{Majuskel} D  \textbf{27} \textit{Majuskel} D  \newline
\line(1,0){75} \newline
\textbf{3} diu] de D \textbf{4} Koukesas] koͮkesas D \textbf{14} Gahmuretes] Gahmvrets D \textbf{27} Parzival] Parcifal D \newline
\end{minipage}
\hspace{0.5cm}
\begin{minipage}[t]{0.5\linewidth}
\small
\begin{center}*m
\end{center}
\begin{tabular}{rl}
 & \begin{large}D\end{large}er heiden truoc \textbf{eht} starkiu lit.\\ 
 & \textit{we}n\textit{ne} er schrîte Tabronit\\ 
 & - \textbf{d\textit{â}} \textbf{diu} künigîn Secundille was\\ 
 & \textbf{vor} der montâne \textbf{zuo} Kaukasas -,\\ 
5 & \textbf{dô} gewan er \textbf{niuwen} hôhen muot\\ 
 & gegen dem, der \textbf{ê} was behuot\\ 
 & \textbf{vor} \textbf{solichem} strîtes überlast.\\ 
 & er was schumpfentiur ein gast,\\ 
 & daz er \textbf{si} nie gedolt,\\ 
10 & doch \textbf{si} \textbf{maniger} \textbf{zu\textit{o} im} \textbf{holt}.\\ 
 & mit kunst si die arm \textbf{erswungen},\\ 
 & viures blicke ûz helmen \textbf{sprungen},\\ 
 & von ir swerten gienc der sûr wint.\\ 
 & got \textbf{ner} dâ Gahmuretes kint!\\ 
15 & \textbf{der} \textit{\textbf{wunsch} wirt} in beiden,\\ 
 & dem getouften und dem heiden;\\ 
 & die nante ich \textbf{ê} vür einen.\\ 
 & sus begunden siz \textbf{noch} \textbf{meinen},\\ 
 & wæren si ein ander \textbf{baz} \textbf{erkant},\\ 
20 & si sasten niht sô hôhiu pfant.\\ 
 & ir strît galt niht mêre\\ 
 & wan vröude, sælde und êre.\\ 
 & wer d\textit{â} den prîs gewinnet,\\ 
 & \textbf{ob} der triuwe minnet,\\ 
25 & \textbf{we\textit{l}ch} vröude er het verlorn\\ 
 & und iemer herzen riuwen erkorn.\\ 
 & \begin{large}W\end{large}es sûmest dû dich, Parcifal,\\ 
 & daz dû an d\textit{ie} kiuschen lieht gemâl\\ 
 & \textbf{niht} \textbf{gedenkest} - \textbf{ich mein} dîn wîp -,\\ 
30 & wiltû behalten hie den lîp?\\ 
\end{tabular}
\scriptsize
\line(1,0){75} \newline
m n o V V' \newline
\line(1,0){75} \newline
\textbf{1} \textit{Initiale} m V V'   $\cdot$ \textit{Capitulumzeichen} n  \textbf{27} \textit{Initiale} m n V  \newline
\line(1,0){75} \newline
\textbf{1} truoc] strúg o  $\cdot$ eht] oͮch V (V') \textbf{2} wenne er schrîte] Man erschritte m Swenne er schriete V  $\cdot$ Tabronit] thabronit m n o \textbf{3} dâ] Do m n o V V'  $\cdot$ Secundille] secundile m \textbf{4} zuo] \textit{om.} n  $\cdot$ Kaukasas] kaucasas n kancasas o [kvncke*]: kvnckesas V kunkesaz V' \textbf{5} dô] So n o V V' \textbf{6} der] er V \textit{om.} V'  $\cdot$ ê] ie V V' \textbf{7} \textit{Die Verse 742.7-10 fehlen} V'  \textbf{10} zuo] zuht m hin zvͦ V \textbf{11} arm] armen o \textbf{13} der] \sout{ir} der n \textbf{14} dâ] do n V V'  $\cdot$ Gahmuretes] [gahmuͯr*]: gahmuͯrettes m gamiretes n gahmuuretes o Gameretes V (V') \textbf{15} wunsch wirt] wirt wunsch m \textbf{17} \textit{Die Verse 742.17-26 fehlen} V'   $\cdot$ ê vür einen] [*]: e fúr einen V \textbf{18} noch meinen] ouch meinen n (o) [*ch*einen]: sv́z oͮch meinen  V \textbf{20} sasten] ensattent V \textbf{23} wer] Swer V  $\cdot$ dâ] do m n o V \textbf{25} welch] Wech m Wertlich V \textbf{26} herzen riuwen] hertzen ruwe n (V) herczen truwe o  $\cdot$ erkorn] erkarn o \textbf{27} sûmest dû] sin vesten o  $\cdot$ Parcifal] parzefal V parzifal V' \textbf{28} an] in n [in]: an o  $\cdot$ die] den m  $\cdot$ kiuschen lieht] liechten kúschen n kúschen liechte o kúsche lieht V  $\cdot$ gemâl] mal n o \textbf{29} gedenkest] denkest V \newline
\end{minipage}
\end{table}
\newpage
\begin{table}[ht]
\begin{minipage}[t]{0.5\linewidth}
\small
\begin{center}*G
\end{center}
\begin{tabular}{rl}
 & \begin{large}D\end{large}er heiden truoc starkiu lit.\\ 
 & swenner schrîte Tabrunit,\\ 
 & \textbf{daz} \textbf{der} künigîn Secundillen was\\ 
 & \textbf{von} der montâne Kausakas,\\ 
5 & \textbf{sô} gewan er \textbf{niwen} hôhen muot\\ 
 & gein dem, der \textbf{ie} was behuot\\ 
 & \textbf{vor} \textbf{solhes} strîtes überlast.\\ 
 & er was schumpfentiure ein gast,\\ 
 & daz er nie gedolte,\\ 
10 & doch \textbf{si} \textbf{maniges} \textbf{zimierde} \textbf{holte}.\\ 
 & mit kunst si die arme \textbf{swungen},\\ 
 & viures blicke ûz helmen \textbf{sprungen},\\ 
 & von ir swerten gie der sûre wint.\\ 
 & got \textbf{ner} dâ Gahmuretes kint!\\ 
15 & \textbf{daz} \textbf{wunschen} wirt in beiden,\\ 
 & dem getouften unde dem heiden;\\ 
 & die nante ich \textbf{ie} vür einen.\\ 
 & sus begunden siz \textbf{ouch} \textbf{meinen},\\ 
 & wæren si \textbf{ê} ein ander \textbf{bekant},\\ 
20 & si\textbf{ne} satzten niht sô hôhiu pfant.\\ 
 & ir strît galt niht mêre\\ 
 & wan vröude, sælde unde êre.\\ 
 & swer dâ den brîs gewinnet,\\ 
 & \textbf{op} der triwe minnet,\\ 
25 & \textbf{werltlîch} vröude er hât verlorn\\ 
 & unde imer herzeriwe erkorn.\\ 
 & wes sûmestû dich, Parcival,\\ 
 & daz dû an die kiuschen lieht gemâl\\ 
 & \textbf{niht} \textbf{gedenkest}, \textbf{an} dîn wîp,\\ 
30 & wil dû \textit{behalten hie} den lîp?\\ 
\end{tabular}
\scriptsize
\line(1,0){75} \newline
G I L M Z Fr24 Fr50 \newline
\line(1,0){75} \newline
\textbf{1} \textit{Initiale} G I L Z Fr24 Fr50  \textbf{15} \textit{Initiale} I  \newline
\line(1,0){75} \newline
\textbf{1} starkiu] starken M zv starke Z \textbf{2} swenner] Wenne er L (M) (Fr50)  $\cdot$ schrîte] schrit Fr50  $\cdot$ Tabrunit] Tanprunit I trabruͯnit M \textbf{3} Secundillen] secuntillen I secundille M \textbf{4} von der] Vor dem Z  $\cdot$ montâne] montanien I montane in Z minne Fr50  $\cdot$ Kausakas] kavchasas G Fr24 Gauchasas I Cavcasas L [kauwasas]: kaukasas M kavkasas Z Fr50 \textbf{5} sô] Do Fr24  $\cdot$ er] er ie I  $\cdot$ niwen] Niwan M \textbf{6} dem] \textit{om.} I \textbf{7} vor solhes strîtes] von solhem strite I \textbf{8} schumpfentiure] schunphentuͯres M \textbf{9} er] er die L \textbf{10} doch si] Da sich M Doch sich Z  $\cdot$ maniges] mazineges I manger L (Z) manich Fr50  $\cdot$ zimierde] mire I von im L dazcu yme M zv im Z amîe Fr24  $\cdot$ holte] erholte M (Z) Fr24 \textbf{11} swungen] [twuͯgen]: twungen L erswngen M \textbf{12} viures] daz fiͮwers Fr50  $\cdot$ helmen] deme helmen M helm Fr24  $\cdot$ sprungen] clungen M drungen Z \textbf{13} swerten] swerte M  $\cdot$ der sûre] vil viwers I da svre Z \textbf{14} ner] erner M riet Fr50  $\cdot$ dâ] daz I  $\cdot$ Gahmuretes] Gahmvretes G Fr24 (Fr50) Gahmuͯretes L Gamuretis M gamuretes Z \textbf{15} daz wunschen] Der wunsh I (M)  $\cdot$ wirt] wir Fr50 \textbf{16} getouften] Getauftem I gitoufftē M \textbf{17} ie] ê I (L) (Z) Fr24 \textit{om.} Fr50 \textbf{19} wæren si] Warensz L waren Fr50  $\cdot$ ê ein ander] ein ander I (L) (Fr50) eyn ander basz M ein ander e Z \textbf{20} hôhiu] hohen M \textbf{22} vröude sælde] selde frevde Z vrovde vnd sælde Fr50 \textbf{23} swer] Wer L M  $\cdot$ dâ] dan M \textbf{24} der] er I \textbf{25} werltlîch] werlich Fr50  $\cdot$ er] \textit{om.} Fr50 \textbf{26} imer] myner M \textbf{27} Parcival] parcifal G Z (Fr24) Fr50 Parzifal I (L) (M) \textbf{28} dû] dir M  $\cdot$ lieht] lieh I liht L (M) \textbf{29} an dîn wîp] ich meyne andy wip M ich mein din wip Z an div wip Fr50 \textbf{30} behalten hie] hie behalten G  $\cdot$ den] dinen M \newline
\end{minipage}
\hspace{0.5cm}
\begin{minipage}[t]{0.5\linewidth}
\small
\begin{center}*T
\end{center}
\begin{tabular}{rl}
 & der heiden truoc starkiu lit.\\ 
 & wan er schrîte Tabrunit,\\ 
 & \textbf{daz} \textbf{der} küneginne Secundille was\\ 
 & \textbf{von} der montâne \textbf{in} Kaukasas,\\ 
5 & \textbf{sô} gewan er \textbf{niht wan} hôhen muot\\ 
 & gein dem, der \textbf{ie} was behuot\\ 
 & \textbf{von} \textbf{soliches} strîtes über\textit{l}a\textit{s}t.\\ 
 & er was schumpfentiure ein gast,\\ 
 & daz er nie gedolte,\\ 
10 & doch \textbf{sich} \textbf{maneger} \textbf{zuo im} \textbf{dâ} \textbf{erholte}.\\ 
 & mit kunst si die arme \textbf{erswungen},\\ 
 & viures blicke ûz \textbf{den} helmen \textbf{drungen},\\ 
 & von ir swerten gienc der sûr\textit{e} wint.\\ 
 & got \textbf{genere} dâ Gahmuretes kint!\\ 
15 & \textbf{daz} \textbf{wunschen} wirt in beiden,\\ 
 & dem getouften und dem heiden;\\ 
 & die nante ich \textbf{ê} vür einen.\\ 
 & sus begunden si ez \textbf{ouch} \textbf{vereinen},\\ 
 & wæren si ein ander \textbf{baz} \textbf{bekant},\\ 
20 & si \textbf{en}satzten niht sô hôhiu pfant.\\ 
 & ir strît galt niht mêre\\ 
 & wan vreude, sælde und êre.\\ 
 & wer dâ den prîs gewinnet,\\ 
 & \textbf{oder} der triuwe \textit{m}i\textit{nnet},\\ 
25 & \textbf{werltlîche} vreude er hât verlorn\\ 
 & und imer herzeriuwe erkorn.\\ 
 & \begin{large}W\end{large}es sûmestû dich, Parcifal,\\ 
 & daz dû an die kiuschen lieht gemâl\\ 
 & \textbf{iht} \textbf{gedenkest} - \textbf{ich meine} dîn wîp -,\\ 
30 & wiltû behalten hie den lîp?\\ 
\end{tabular}
\scriptsize
\line(1,0){75} \newline
U W Q R \newline
\line(1,0){75} \newline
\textbf{1} \textit{Initiale} W  \textbf{27} \textit{Initiale} U W  \newline
\line(1,0){75} \newline
\textbf{2} schrîte] schrey W schriet R  $\cdot$ Tabrunit] Tanbruͦmit U tabrűnit Q tarburnit R \textbf{3} Secundille] secűndille Q \textbf{4} in] \textit{om.} R  $\cdot$ Kaukasas] kankasas W R \textbf{5} niht wan] neúwen W (Q) (R) \textbf{6} der] er R \textbf{7} von] Vor W Q R  $\cdot$ soliches] sulchest Q  $\cdot$ überlast] vberkraft U \textbf{8} er was] Ein Q \textbf{10} doch sich] Do sy W  $\cdot$ dâ] \textit{om.} W Q R  $\cdot$ erholte] holte W [holte]: erholte Q \textbf{12} den] \textit{om.} W R  $\cdot$ drungen] sprungen W (Q) (R) \textbf{13} swerten] schwert R  $\cdot$ sûre] suͦren U \textbf{14} genere dâ] ner do W Q ner [*]: da R  $\cdot$ Gahmuretes] Gahmuͦretes U gamuretes W gamúretes Q \textbf{15} daz wunschen] Der wunsch W \textbf{18} sus] Als Q  $\cdot$ begunden] begund R  $\cdot$ vereinen] meinen Q (R) \textbf{19} bekant] erkant W \textbf{20} ensatzten] ensassen W satzten Q (R) \textbf{23} dâ] do W Q  $\cdot$ den] \textit{om.} W \textbf{24} oder] Ob W Q R  $\cdot$ triuwe] in trúwen R  $\cdot$ minnet] nimmet U \textbf{25} werltlîche] Welche W Werliche Q  $\cdot$ er hât] hat er W \textbf{26} herzeriuwe] hertze trewe Q \textbf{27} Parcifal] partzifal W Q parczifal R \textbf{28} kiuschen] kúsche W \textbf{29} iht] Nit W (Q) R  $\cdot$ dîn wîp] [die]: din wip U din schones wib R \textbf{30} hie den] dinen R \newline
\end{minipage}
\end{table}
\end{document}
