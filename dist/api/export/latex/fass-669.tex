\documentclass[8pt,a4paper,notitlepage]{article}
\usepackage{fullpage}
\usepackage{ulem}
\usepackage{xltxtra}
\usepackage{datetime}
\renewcommand{\dateseparator}{.}
\dmyyyydate
\usepackage{fancyhdr}
\usepackage{ifthen}
\pagestyle{fancy}
\fancyhf{}
\renewcommand{\headrulewidth}{0pt}
\fancyfoot[L]{\ifthenelse{\value{page}=1}{\today, \currenttime{} Uhr}{}}
\begin{document}
\begin{table}[ht]
\begin{minipage}[t]{0.5\linewidth}
\small
\begin{center}*D
\end{center}
\begin{tabular}{rl}
\textbf{669} & \begin{large}V\end{large}on hûs sich rottierte;\\ 
 & sîne reise \textbf{er} \textbf{alsus} \textbf{zierte},\\ 
 & dâ von \textbf{m\textit{ö}hte ich iu wunder} sagen.\\ 
 & manec soumære muose tragen\\ 
5 & \textbf{kappeln} unt \textbf{kamergewant}.\\ 
 & manec soum mit harnasche erkant\\ 
 & giengen ouch dâr unden,\\ 
 & \textbf{helme} oben drûf gebunden\\ 
 & bî manegem schilde wolgetân.\\ 
10 & manec schœne kastelân\\ 
 & man bî \textbf{den soumen} ziehen sach.\\ 
 & rîter unt vrouwen \textbf{hinden} nâch\\ 
 & riten an ein ander vaste.\\ 
 & daz gezoc wol eine raste\\ 
15 & an der lenge \textbf{was} gemezzen.\\ 
 & dône \textbf{wart} \textbf{dâ} niht vergezzen,\\ 
 & Gawan einen rîter \textbf{wol gevar}\\ 
 & immer schuof zeiner vrouwen clâr.\\ 
 & daz w\textit{æ}ren kranke sinne,\\ 
20 & \textbf{ob die spr\textit{æ}chen iht} von minne.\\ 
 & Der Turkote Florant\\ 
 & zeime gesellen wart erkant\\ 
 & \textbf{Sangive} von Norwæge.\\ 
 & Lischoys, der \textbf{gar} untræge,\\ 
25 & reit bî der süezen Cundrie.\\ 
 & sîn swester Itonje\\ 
 & bî Gawane solde rîten.\\ 
 & an den selben zîten\\ 
 & Arnive unt diu herzogîn\\ 
30 & ouch gesellen wolden sîn.\\ 
\end{tabular}
\scriptsize
\line(1,0){75} \newline
D Fr8 Fr10 \newline
\line(1,0){75} \newline
\textbf{1} \textit{Initiale} D Fr10  \textbf{21} \textit{Majuskel} D  \newline
\line(1,0){75} \newline
\textbf{2} alsus] also Fr10 \textbf{3} möhte] mohte D muht Fr10 \textbf{6} soum mit] saumer niht Fr10 \textbf{7} giengen] Gieng Fr8  $\cdot$ unden] under Fr10 \textbf{8} Mit helmen druf gebunden Fr8 \textbf{12} hinden] \textit{om.} Fr8 \textbf{15} wart] was Fr8 \textbf{17} Gawan einen] :::n Ain Fr10 \textbf{18} immer schuof] Schuͦf imber Fr8  $\cdot$ clâr] dar Fr8 schar Fr10 \textbf{19} wæren] waren D Fr8 (Fr10) \textbf{20} Sprachen die nicht von minne Fr8  $\cdot$ spræchen] sprahen D redtun Fr10 \textbf{21} Der Tukoẏte Florant Fr8  $\cdot$ :::rkoit Florunt Fr10 \textbf{23} Sangive] Sangîve D Sangẏuen Fr8 :::gwinen Fr10  $\cdot$ Norwæge] Norwege Fr8 Norweg Fr10 \textbf{24} Lischoys] Lyscoẏs D Lẏsclẏois Fr8 :::ors Fr10 \textbf{25} Cundrie] Cvndrîe D Gundrẏe Fr8 Cundrei Fr10 \textbf{26} Itonje] Jtonîe D ẏthonẏe Fr8 Jtonei Fr10 \textbf{27} Gawane] Gawan Fr10 \textbf{29} Arnive] Arnîve D Arnẏue Fr8 :::e Fr10 \textbf{30} gesellen] gesellinn Fr10 \newline
\end{minipage}
\hspace{0.5cm}
\begin{minipage}[t]{0.5\linewidth}
\small
\begin{center}*m
\end{center}
\begin{tabular}{rl}
 & von hûs sich r\textit{o}ttierte;\\ 
 & sîn reise \textbf{er} \textbf{sus} \textbf{zierte},\\ 
 & dâ von \textbf{m\textit{ö}ht ich iu wunder} sagen.\\ 
 & manic soumer muoste tragen\\ 
5 & \textbf{kappeln} und \textbf{krâmgewant}.\\ 
 & manic soum mit harnasch erkant\\ 
 & giengen ouch dâr unden,\\ 
 & \textbf{helm} oben dâr ûf gebunden\\ 
 & bî manigem schilt wol getân.\\ 
10 & manic schœne kastelân\\ 
 & man bî \textbf{den \textit{s}oumen} ziehen sach.\\ 
 & ritter und vrowen nâch\\ 
 & riten ane ein ander vaste.\\ 
 & daz gezoc wol ein raste\\ 
15 & an der lenge \textbf{wart} gemezzen.\\ 
 & dô en\textbf{wart} niht vergezzen,\\ 
 & Gawan einen ritter \textbf{wol gevar}\\ 
 & iemer schuo\textit{f zuo }e\textit{i}ner vrowen clâr.\\ 
 & daz wæren kranke sinne,\\ 
20 & \textbf{spræchen die niht} von minne.\\ 
 & der Turkoite Florant\\ 
 & zuo einem gesellen wart erkant\\ 
 & \textbf{Sang\textit{iv}en} von Norwæge.\\ 
 & Lischois, der untræge,\\ 
25 & reit bî der süezen Condrie.\\ 
 & sîn swester Ithonie\\ 
 & bî Gawane solt\textit{e} rîten.\\ 
 & an den selben zîten\\ 
 & Ar\textit{niv}e und diu herzogîn\\ 
30 & ouch gesellen wolten sîn.\\ 
\end{tabular}
\scriptsize
\line(1,0){75} \newline
m n o \newline
\line(1,0){75} \newline
\newline
\line(1,0){75} \newline
\textbf{1} rottierte] rettierte m \textbf{3} möht] moht m (o) so moͯchte n \textbf{6} soum] foume n \textbf{9} manigem] manigen o \textbf{10} kastelân] ca:::ellan o \textbf{11} soumen] zumen m (n) (o) \textbf{15} wart] was n o \textbf{18} iemer schuof zuo einer] Yemer schus jener m  $\cdot$ clâr] schar o \textbf{19} wæren] vnerent n \textbf{21} Turkoite] turkoitte m \textbf{23} Sangiven] Sangwen m n Sanwen o  $\cdot$ Norwæge] norwege m n o \textbf{24} Lischois] Liscois m n o  $\cdot$ untræge] gar vntrege n o \textbf{25} Condrie] Cundrie o \textbf{26} Ithonie] jtonie m (o) ẏtonie n \textbf{27} solte] solten m  $\cdot$ rîten] ligen o \textbf{29} Arnive] Aruͯne m Arniwe n \newline
\end{minipage}
\end{table}
\newpage
\begin{table}[ht]
\begin{minipage}[t]{0.5\linewidth}
\small
\begin{center}*G
\end{center}
\begin{tabular}{rl}
 & \begin{large}V\end{large}on hûse sich rotierte;\\ 
 & sîne reise \textbf{er} \textbf{alsô} \textbf{vierte},\\ 
 & dâ von \textbf{ich wunder möhte} sagen.\\ 
 & manic soumære muose tragen\\ 
5 & \textbf{kappeln} unde \textbf{kamergewant}.\\ 
 & manic soum mit harnasche erkant\\ 
 & giengen ouch dâr unden,\\ 
 & \textbf{helm} oben drûf gebunden\\ 
 & bî manigem schilte wolgetân.\\ 
10 & manic schœne kastelân\\ 
 & man bî \textbf{dem soume} ziehen sach.\\ 
 & rîter unde vrouwen nâch\\ 
 & riten an ein ander vaste.\\ 
 & daz gezoc wol eine raste\\ 
15 & an der lenge \textbf{wart} gemezzen.\\ 
 & dône \textbf{wart} niht vergezzen,\\ 
 & Gawan einen rîter \textbf{dar}\\ 
 & imer schuof zuo einer vrouwen clâr.\\ 
 & daz w\textit{æ}ren kranke sinne,\\ 
20 & \textbf{op die niht spr\textit{æ}chen} von minne.\\ 
 & der Turkoite Florant\\ 
 & zeinem gesellen wart erkant\\ 
 & \textbf{Sagiven} von Norwæge.\\ 
 & Lishois, der \textbf{gar} untræge,\\ 
25 & reit bî der süezen Gundrie.\\ 
 & sîn swester Itonie\\ 
 & bî Gawane solde rîten.\\ 
 & an den selben zîten\\ 
 & Arnive unde diu herzogîn\\ 
30 & ouch \textbf{dâ} gesellen wolden sîn.\\ 
\end{tabular}
\scriptsize
\line(1,0){75} \newline
G I L M Z Fr61 \newline
\line(1,0){75} \newline
\textbf{1} \textit{Initiale} G L M Z  \textbf{13} \textit{Initiale} I  \newline
\line(1,0){75} \newline
\textbf{1} sich] er sich I  $\cdot$ rotierte] rotiert M (Fr61) \textbf{2} Jn reise also geviert M  $\cdot$ alsô] allez I  $\cdot$ vierte] wierte L zierte Z tziert Fr61 \textbf{3} ich] man I  $\cdot$ möhte] mohte I L (M) (Z) \textbf{5} kappeln] Chappelle Fr61 \textbf{6} soum] saumere I \textbf{8} helm] helme L Z Fr61  $\cdot$ oben] \textit{om.} I Fr61 \textbf{9} manigem] ::: mangen I \textbf{11} dem soume] den sovmen L (Z) (Fr61) \textbf{12} nâch] hinden nach Z \textbf{15} wart] waz L (M) (Z) (Fr61) \textbf{16} dône wart] Da vone wart M Da enwart da Z Da enward Fr61 \textbf{17} dar] wol gevar Z \textbf{18} imer] \textit{om.} Fr61  $\cdot$ vrouwen] [shar]: frowen clar I vrouwen dar M \textbf{19} wæren] waren G L (M) Z (Fr61) \textbf{20} niht spræchen] niht sprachen G iht sprachen L sprechin nicht M (Fr61) sprachen niht Z \textbf{21} Turkoite] tvͦrkoite G Turchoyde I Tuͯrkoite L tirkoite M Tvrkoit Z Turkoyt Fr61  $\cdot$ Florant] floriant G I \textbf{22} zeinem] seinen Fr61 \textbf{23} Sagiven] sagîven G sagiwen I Sagýven L Saiven M (Fr61) Seyven Z  $\cdot$ Norwæge] norwage G I (L) Norwege M Z Norweg Fr61 \textbf{24} Lishois] Liscoys I Fr61 Lytschoýs L Lisois M  $\cdot$ untræge] vngetrege Z \textbf{25} der süezen] \textit{om.} Fr61  $\cdot$ Gundrie] gvndrîe G Cvndrien L kundrie M Z gundrien Fr61 \textbf{26} Itonie] Jtonie G M ytonie I jtonien L Jconie Z ẏtonien Fr61 \textbf{27} Gawane] Gawan I (M) (Z)  $\cdot$ solde] hiez man Fr61 \textbf{29} Arnive] Arnuue I  $\cdot$ unde] vnde vnde M \textbf{30} die wolten auch da sin I  $\cdot$ gesellen] gesellinne Fr61 \newline
\end{minipage}
\hspace{0.5cm}
\begin{minipage}[t]{0.5\linewidth}
\small
\begin{center}*T
\end{center}
\begin{tabular}{rl}
 & von hûse sich rottierte,\\ 
 & sîne reise \textbf{alsô} \textbf{vierte},\\ 
 & dâ von \textbf{ich wunder möhte} sagen.\\ 
 & manic soumer muoste tragen\\ 
5 & \textbf{kappel} und \textbf{kamergewant}.\\ 
 & manic soum mit harnasch erkant\\ 
 & giengen ouch dâr unden,\\ 
 & \textbf{helm} oben drûf gebunden\\ 
 & bî manegem schilde wol getân.\\ 
10 & manic schœne kastelân\\ 
 & man \textit{b}î \textbf{\textit{den} soumen} ziehen sach.\\ 
 & ritter und vrouwen \textbf{hinden} nâch\\ 
 & riten \textit{an} ein ander vaste.\\ 
 & daz gezoc wol ein raste\\ 
15 & an der lenge \textbf{was} gemezzen.\\ 
 & d\textit{ôn}e \textbf{was} niht vergezzen,\\ 
 & Gawan einen ritter \textbf{wol gevar}\\ 
 & immer schuof zuo einer vrouwen clâr.\\ 
 & daz w\textit{æ}ren kranke sinne,\\ 
20 & \textbf{ob die spr\textit{æ}c\textit{h}en iht} von minne.\\ 
 & der Turkoyte Florant\\ 
 & zuo einem gesellen wart erkant\\ 
 & \textbf{Seyven} von Norwæge.\\ 
 & Lyschoys, der \textbf{gar} untræge,\\ 
25 & reit bî der süezen Kundrie.\\ 
 & sîn swester Itonie\\ 
 & bî Gawan solte rîten.\\ 
 & an den selben zîten\\ 
 & Arnyve und diu herzogîn\\ 
30 & \textit{ou}ch \textbf{dâ} gesellen wolten sîn.\\ 
\end{tabular}
\scriptsize
\line(1,0){75} \newline
Q R W V \newline
\line(1,0){75} \newline
\textbf{17} \textit{Initiale} W  \newline
\line(1,0){75} \newline
\textbf{1} hûse] hvse er V  $\cdot$ rottierte] rottieret R rittieret W \textbf{2} reise] Reise er R (W) (V)  $\cdot$ alsô] [als*]: alsvz V  $\cdot$ vierte] vieret R W [*]: zierte V \textbf{3} von ich] [*]: von ich V \textbf{4} muoste] mvͤste V \textbf{5} kappel] Kapplen R (W) (V) \textbf{6} soum] [samit]: sauͯm Q \textbf{7} unden] vnder R \textbf{8} helm] Helme W V  $\cdot$ oben] \textit{om.} R \textbf{9} manegem] mangen W \textbf{11} bî den] sie Q bi [dem]: den V  $\cdot$ soumen] soͯmer R (V) \textbf{12} hinden] hin R \textbf{13} an] nach Q \textbf{15} lenge] lengen R [len*]: lenge V  $\cdot$ was] wol R \textbf{16} dône] Danne Q Do R W V  $\cdot$ was] ward R enward W (V) \textbf{17} Gawan] Gawin R  $\cdot$ einen] ye einen R \textbf{18} immer] \textit{om.} R  $\cdot$ zuo] ye [*u]: zu R  $\cdot$ clâr] dar R [*]: dar V \textbf{19} wæren] warren Q (R) (W) (V) \textbf{20} spræchen] sprachten Q sprecht R [sprachen]: sprachen V  $\cdot$ iht] nit R [nht]: niht V \textbf{21} Turkoyte] kurkoit Q Turkeite R tvrkoẏte V \textbf{22} erkant] benant V \textbf{23} Seyven] Seyuen R W (V)  $\cdot$ von] vnd W  $\cdot$ Norwæge] norwegen Q Norwege R (W) (V) \textbf{24} Lyschoys] Lyszhois Q Lyschois R Lyshois W Lyscois V \textbf{26} Itonie] ytonie Q R W V \textbf{27} Gawan] gawin R gawane W (V) \textbf{28} an] Zuͦ W \textbf{29} Arnyve] Arnoue Q Arnyue R W Arniue V \textbf{30} ouch] Nach Q  $\cdot$ dâ] do R W V \newline
\end{minipage}
\end{table}
\end{document}
