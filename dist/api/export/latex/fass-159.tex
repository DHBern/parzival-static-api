\documentclass[8pt,a4paper,notitlepage]{article}
\usepackage{fullpage}
\usepackage{ulem}
\usepackage{xltxtra}
\usepackage{datetime}
\renewcommand{\dateseparator}{.}
\dmyyyydate
\usepackage{fancyhdr}
\usepackage{ifthen}
\pagestyle{fancy}
\fancyhf{}
\renewcommand{\headrulewidth}{0pt}
\fancyfoot[L]{\ifthenelse{\value{page}=1}{\today, \currenttime{} Uhr}{}}
\begin{document}
\begin{table}[ht]
\begin{minipage}[t]{0.5\linewidth}
\small
\begin{center}*D
\end{center}
\begin{tabular}{rl}
\textbf{159} & \textbf{\begin{large}N\end{large}û} tuo ez durch dîne gesellecheit\\ 
 & \textbf{unt} lâz dir sîn mîn laster leit.\\ 
 & got \textbf{hüete} dîn - ich wil von dir varn -,\\ 
 & der mag uns \textbf{bêde} wol bewarn."\\ 
5 & Ithern v\textit{on} Gaheviez\\ 
 & er jæmerlîche ligen \textbf{hiez}.\\ 
 & der was doch \textbf{tôt} sô minneclîch,\\ 
 & \textbf{lebende} was er sælden rîch.\\ 
 & wære ritterschaft sîn endes wer\\ 
10 & ze\textbf{r tjost} durch schilt mit eime sper,\\ 
 & wer klage\textit{t} denne die wunders nôt?\\ 
 & er starp \textbf{von} eime gabylôt.\\ 
 & Iwanet ûf in dô brach\\ 
 & der liehten bluomen zeime dach.\\ 
15 & er stiez \textbf{den gabylôtes} stil\\ 
 & zuo \textbf{z}im. nâch der marter zil\\ 
 & der knappe kiusche unt stolz\\ 
 & dructe \textbf{in kriuzes wîs} ein holz\\ 
 & durch des gabylôtes snîden.\\ 
20 & \textbf{dône wolder} niht vermîden,\\ 
 & \textbf{hin} in die stat er sagete,\\ 
 & des manec wîp verzagete\\ 
 & unt \textbf{des} manec ritter weinde,\\ 
 & der klagende triwe erscheinde.\\ 
25 & Dâ wart jâmers vil gedolt.\\ 
 & der tôte schône wart geholt.\\ 
 & diu küneginne reit ûz der stat.\\ 
 & daz \textbf{heilictuom} si vüeren bat\\ 
 & ob dem künege von Kukumerlant,\\ 
30 & den tôte Parzivales hant.\\ 
\end{tabular}
\scriptsize
\line(1,0){75} \newline
D \newline
\line(1,0){75} \newline
\textbf{1} \textit{Initiale} D  \textbf{25} \textit{Majuskel} D  \newline
\line(1,0){75} \newline
\textbf{5} Ithern] Jthern D  $\cdot$ von] vnt D \textbf{11} klaget] claget:v \textit{nachträglich korrigiert zu:} chlagetiv D \textbf{13} Iwanet] Jwanet D \textbf{29} Kukumerlant] Chvchvmerlant D \newline
\end{minipage}
\hspace{0.5cm}
\begin{minipage}[t]{0.5\linewidth}
\small
\begin{center}*m
\end{center}
\begin{tabular}{rl}
 & \textbf{nû} tuoz durch dîne gesellecheit\\ 
 & \textbf{und} lâz dir sîn mîn laster leit.\\ 
 & got \textbf{hüete} dîn - ich wil von dir varn -,\\ 
 & der mac uns \textbf{balde} wol bewarn."\\ 
5 & I\textit{t}hern von Gaheviez\\ 
 & er jâmerlîchen ligen \textbf{liez}.\\ 
 & der was doch \textbf{tôt} sô minneclîch,\\ 
 & \textbf{lebendic} was er sælden rîch.\\ 
 & wær ritterschaft sîn endes wer\\ 
10 & ze \textbf{jungest} durch schilt mit einem sper,\\ 
 & wer klagete danne die wunders nôt?\\ 
 & er starp \textbf{mit} einem gabilôt.\\ 
 & Iwanet ûf in dô brach\\ 
 & der liehten bluomen ze einem dach.\\ 
15 & er stiez \textbf{den gabilôtes} stil\\ 
 & \dag sus\dag  im nâch der martel zil.\\ 
 & der knappe kiusch und stolz\\ 
 & druh\textit{te} \textbf{in kriuze wîse} ein holz\\ 
 & durch des gabilôtes snîden.\\ 
20 & \textbf{dô} \dag enwoltent\dag  niht vermîden,\\ 
 & \textbf{hin} in die stat er sagete,\\ 
 & des manic wîp verzagete\\ 
 & und manic ritter weinde,\\ 
 & der klagende triuwe erscheinde.\\ 
25 & d\textit{â} wart jâmers vil gedolt.\\ 
 & der tôte schône wart geholt.\\ 
 & diu künigîn reit ûz der stat.\\ 
 & das \textbf{heilictuom} si vüeren bat\\ 
 & ob dem künic von Kukumerlant,\\ 
30 & den tôte Parcifals hant.\\ 
\end{tabular}
\scriptsize
\line(1,0){75} \newline
m n o \newline
\line(1,0){75} \newline
\newline
\line(1,0){75} \newline
\textbf{1} \textit{Die Verse 158.18-160.3 fehlen (Blattverlust)} o  \textbf{5} Ithern] Jchern m Jthern n  $\cdot$ Gaheviez] gaviesz n \textbf{7} minneclîch] jemerlich n \textbf{13} Iwanet] Jwanet m n \textbf{16} zil] zit n \textbf{18} druhte] Druch m  $\cdot$ in kriuze wîse ein] ein crútz wis von n \textbf{21} sagete] saget n \textbf{22} des] Das n  $\cdot$ verzagete] verzaget n \textbf{25} dâ] Do m n  $\cdot$ gedolt] gedult n \textbf{28} heilictuom] heillituͯm m heiltuͯm n \textbf{29} Kukumerlant] cucumerlant m kucumer lant n \textbf{30} Parcifals] parcifales n \newline
\end{minipage}
\end{table}
\newpage
\begin{table}[ht]
\begin{minipage}[t]{0.5\linewidth}
\small
\begin{center}*G
\end{center}
\begin{tabular}{rl}
 & tuo ez durch \textit{dîn} gesellecheit;\\ 
 & lâ dir sîn mîn laster leit.\\ 
 & \multicolumn{1}{l}{ - - - }\\ 
 & \multicolumn{1}{l}{ - - - }\\ 
5 & Itheren von Kaheviez\\ 
 & er jæmerlîchen ligen \textbf{liez}.\\ 
 & der was doch \textbf{wol} sô minniclîch,\\ 
 & \textbf{lebende} was er sælden rîch.\\ 
 & wære rîterschaft sînes endes wer\\ 
10 & ze\textbf{r tjost} durch schilt mit einem sper,\\ 
 & wer klagte danne die wunders nôt?\\ 
 & er starp \textbf{von} einem gabilôt.\\ 
 & Ywanet ûf in dô brach\\ 
 & der li\textit{e}hten bluomen zeinem dach.\\ 
15 & er stiez \textbf{den gabilôtes} stil\\ 
 & zuo im. nâch der marter zil\\ 
 & der knappe kiusche und stolz\\ 
 & dructe \textbf{in kriuze wîs} ein holz\\ 
 & durch d\textit{es} gabilôtes snîden.\\ 
20 & \textbf{er enwolt} niht vermîden,\\ 
 & in die stat er sagte,\\ 
 & des manic wîp verzagte\\ 
 & unde manic rîter weinde,\\ 
 & der klagende triwe erscheinde.\\ 
25 & dâ wart jâmers vil gedolt.\\ 
 & der tôte schône wart geholt.\\ 
 & diu künigîn reit ûz der stat.\\ 
 & daz \textbf{heilictuom} si vüeren bat\\ 
 & obe dem künige von Kukumerlant,\\ 
30 & den tôte Parzivales hant.\\ 
\end{tabular}
\scriptsize
\line(1,0){75} \newline
G I O L M Q R Z Fr36 \newline
\line(1,0){75} \newline
\textbf{1} \textit{Initiale} I O Q Z  \textbf{5} \textit{Initiale} L R  \textbf{17} \textit{Initiale} I  \newline
\line(1,0){75} \newline
\textbf{1} tuo ez] ÷vͦ ez O Tvn es Q Nv tvz Z  $\cdot$ dîn] \textit{om.} G \textbf{2} lâ] vnde la O (L) (M) (Q) (R) (Z)  $\cdot$ sîn mîn laster] min laster wesen I R \textbf{3} \textit{Die Verse 159.3-4 fehlen} G I O L M Q R   $\cdot$ Got hvͤt din ich wil von dir varn Z \textbf{4} Der mac vns bede wol bewarn Z \textbf{5} Itheren] Jtern I Jthern O JHtern L (R) Jchern M Q Z  $\cdot$ Kaheviez] kahaviez G I O kankaviez M gahefies Q kaheveis R \textbf{6} er] Ern Q \textbf{7} der] \textit{om.} Q Er R  $\cdot$ wol] \textit{om.} O L M Q R tot Z  $\cdot$ minniclîch] manlich I \textbf{8} lebende] Lebendich Z  $\cdot$ sælden] vroudin M \textbf{9} endes wer] \textit{om.} M ende were R \textbf{10} zer tjost] Zuͯ týost L (Z) Der strit R  $\cdot$ durch] mit Q  $\cdot$ einem] dem R \textbf{11} wer] Vor M  $\cdot$ klagte] chlagt I O (M)  $\cdot$ die] diu I \textbf{12} er starp] Vrstarp M  $\cdot$ einem] einen I \textbf{13} Ywanet] ẏwanet G Juuanet I Jwanet O L M Jwan R  $\cdot$ ûf in dô] uff yme da M da vf in Z \textbf{14} liehten] liehehten G lichten L (M) (Q)  $\cdot$ zeinem] ein L R \textbf{15} stiez] sasz M  $\cdot$ den] des R  $\cdot$ gabilôtes] gibelotes R \textbf{16} zuo im] Zcu zcim M  $\cdot$ nâch der] zer R  $\cdot$ marter] marters Z \textbf{18} ein cruzwis ein holz I  $\cdot$ in kriuze] en chrevzen O in Cruͯces L (Q) ein Crúcz wis R \textbf{19} durch] stiez durc I  $\cdot$ des] die G \textit{om.} M Q R \textbf{20} enwolt] wolte I \textbf{21} er] er do Q \textbf{22} des] Daz O (Q) (Fr36)  $\cdot$ wîp] fraw Q (R) \textbf{23} weinde] wainende I \textbf{24} der] \textit{om.} I  $\cdot$ erscheinde] bescheinde O \textbf{25} dâ] Do Q R  $\cdot$ gedolt] verdolt Z \textbf{26} tôte] rote L  $\cdot$ schône] schoner I \textbf{27} ûz] zcu M \textbf{28} heilictuom] heiltvͦm O (L) (Q) [holtum]: heltum  R \textbf{29} dem] den I  $\cdot$ von] \textit{om.} M  $\cdot$ Kukumerlant] kvcumerlant O (L) kucuͯmerlant M kukúmerlant Q kukumerland R kvnkvmerlant Z \textbf{30} tôte] torte Q  $\cdot$ Parzivales] [parzifals]: Parzifals I Parcifals O (Z) Parcifales L parzevales M portzifales Q parczifales R \newline
\end{minipage}
\hspace{0.5cm}
\begin{minipage}[t]{0.5\linewidth}
\small
\begin{center}*T
\end{center}
\begin{tabular}{rl}
 & tuoz durch dîne gesellecheit\\ 
 & \textbf{und} lâ dir sîn mîn laster leit.\\ 
 & got \textbf{pflege} dîn - ich wil von dir varn -,\\ 
 & der mac uns \textbf{beide} wol bewarn."\\ 
5 & \textit{\begin{large}I\end{large}}theren von Kaheviez\\ 
 & er jæmerlîchen ligen \textbf{liez}.\\ 
 & der was doch \textbf{tôt} sô minneclîch,\\ 
 & \textbf{lebende} was er sældenrîch.\\ 
 & wære rîterschaft sînes endes wer\\ 
10 & ze\textbf{r tjost} durch schilt mit einem sper,\\ 
 & wer klagete danne die wunders nôt?\\ 
 & er starp \textbf{von} einem gabylôt.\\ 
 & Ywanet ûf in dô brach\\ 
 & der liehten bluomen zeine\textit{m} dach.\\ 
15 & er stiez \textbf{der gabylôte} stil\\ 
 & zuo im nâch der marter zil.\\ 
 & der knappe kiusche und stolz\\ 
 & druhte \textbf{kriuzewîs} ein holz\\ 
 & durchs gabylôtes snîden.\\ 
20 & \textbf{ern wolte} niht vermîden,\\ 
 & in die stat er\textbf{z} sagete,\\ 
 & des manec wîp verzagete\\ 
 & und \textbf{des} manec rîter weinde,\\ 
 & der klagende triuwe erscheinde.\\ 
25 & dâ wart jâmers vil gedolt.\\ 
 & der tôte schône wart geholt.\\ 
 & Diu künegîn reit ûz der stat.\\ 
 & daz \textbf{heiltuom} si vüeren bat\\ 
 & obeme künege von Kukumerlant,\\ 
30 & den tôte Parcifals hant.\\ 
\end{tabular}
\scriptsize
\line(1,0){75} \newline
T U V W \newline
\line(1,0){75} \newline
\textbf{5} \textit{Initiale} T  \textbf{27} \textit{Majuskel} T  \newline
\line(1,0){75} \newline
\textbf{3} dîn] dem W \textbf{4} beide] \textit{om.} U \textbf{5} Itheren] ÷teren T Tern U Yttern V Ythern W  $\cdot$ Kaheviez] Caheveiz U [*ahevies]: kahevies V gahaifies W \textbf{7} doch tôt] auch W  $\cdot$ minneclîch] iemerlich U \textbf{8} sældenrîch] mynnen reich W \textbf{9} sînes endes wer] sein ende schwer W \textbf{12} einem] einer V W \textbf{13} \textit{Die Verse 159.13-160.30 fehlen} W  \textbf{14} zeinem] zeinen T \textbf{15} gabylôte] gabeloten V  $\cdot$ stil] zil U \textbf{18} kriuzewîs] [*]: in crúcewis V \textbf{19} durchs] Duͦrch U Dvrch [*]: Des V \textbf{20} ern] er V \textbf{21} in] [J*]: Hin in V \textbf{25} dâ] Do U \textbf{26} tôte] rote V \textbf{27} Diu] Do U V  $\cdot$ künegîn reit] reit die kuͦnegin U (V) \textbf{29} Kukumerlant] Cvkvmberlant T Cukuͦmer lant U \textbf{30} Parcifals] Parcifals T parzifales V \newline
\end{minipage}
\end{table}
\end{document}
