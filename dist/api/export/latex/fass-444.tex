\documentclass[8pt,a4paper,notitlepage]{article}
\usepackage{fullpage}
\usepackage{ulem}
\usepackage{xltxtra}
\usepackage{datetime}
\renewcommand{\dateseparator}{.}
\dmyyyydate
\usepackage{fancyhdr}
\usepackage{ifthen}
\pagestyle{fancy}
\fancyhf{}
\renewcommand{\headrulewidth}{0pt}
\fancyfoot[L]{\ifthenelse{\value{page}=1}{\today, \currenttime{} Uhr}{}}
\begin{document}
\begin{table}[ht]
\begin{minipage}[t]{0.5\linewidth}
\small
\begin{center}*D
\end{center}
\begin{tabular}{rl}
\textbf{444} & \begin{large}I\end{large}edoch bereit er sich zer tjost.\\ 
 & Parzival mit solher kost\\ 
 & het ouch sper vil verzert.\\ 
 & er dâhte: "\textbf{ich} wære unernert,\\ 
5 & rit ich über dises mannes sât.\\ 
 & wie würde denne sînes zornes rât?\\ 
 & nû tritte ich hie den wilden varm.\\ 
 & mir \textbf{engeswîchen} hende, \textbf{ieweder} arm,\\ 
 & ich gibe vür mîne \textbf{reise} \textbf{ein} pfant,\\ 
10 & daz ninder bindet mich sîn hant."\\ 
 & Daz wart ze bêder sît getân:\\ 
 & diu ors in den walap verlân,\\ 
 & mit sporn getriben unt \textbf{ouch} gevuort\\ 
 & vast ûf der rabbîne hurt.\\ 
15 & ir \textbf{enweders} tjost dâ misseriet.\\ 
 & maneger tjost ein \textbf{gegenniet}\\ 
 & was Parzivals hôhiu brust.\\ 
 & \textbf{den} lêrte kunst unt sîn gelust,\\ 
 & daz sîn tjost als eben vuor\\ 
20 & reht in den stric der helmsnuor.\\ 
 & er traf in, dâ man \textbf{hæht} den schilt,\\ 
 & sô man ritterschefte spilt,\\ 
 & daz von Munsalvæsche der templeis\\ 
 & von dem orse in eine halden reis\\ 
25 & sô verre hin ab - diu was \textbf{sô} tief -,\\ 
 & daz dâ sîn leger wênec slief.\\ 
 & Parzival der tjoste nâch\\ 
 & volgete. dem orse was \textbf{ze} gâch;\\ 
 & ez viel hin ab, daz ez \textbf{gar} zerbrast.\\ 
30 & Parzival eines zêders ast\\ 
\end{tabular}
\scriptsize
\line(1,0){75} \newline
D Fr5 Fr31 \newline
\line(1,0){75} \newline
\textbf{1} \textit{Initiale} D Fr5  \textbf{11} \textit{Majuskel} D  \newline
\line(1,0){75} \newline
\textbf{1} er] \textit{om.} Fr5 \textbf{2} Parzival] Parcifal D Fr5 \textbf{3} het] Er het Fr5 \textbf{4} wære] were ouch Fr5 \textbf{8} ieweder] vnd Fr5 \textbf{9} vür] vf Fr5 \textbf{11} bêder sît] beiden siten Fr5 \textbf{17} Parzivals] Parcifals D (Fr5) \textbf{18} lêrte] lert Fr5  $\cdot$ sîn] oͮch Fr5 \textbf{21} hæht] henkit Fr5 \textbf{23} Munsalvæsche] Mvnsælvæsce D muntsaluasch Fr5 \textbf{24} eine] die Fr5 \textbf{27} Parzival] Parcifal D Fr5 Parzifal Fr31 \textbf{28} ze] \textit{om.} Fr5 Fr31 \textbf{29} gar] alliz Fr5 \textit{om.} Fr31 \textbf{30} Parzival] Parcifal D Fr5 Parzifal Fr31 \newline
\end{minipage}
\hspace{0.5cm}
\begin{minipage}[t]{0.5\linewidth}
\small
\begin{center}*m
\end{center}
\begin{tabular}{rl}
 & iedoch bereiter sich zer jost.\\ 
 & Parcifal mit solher kost\\ 
 & hete ouch sper vil verzert.\\ 
 & er dâhte: "\textbf{ich} wære u\textit{n}e\textit{r}nert,\\ 
5 & ri\textit{t}e ich über dises mannes sât.\\ 
 & wie würde danne sînes zornes rât?\\ 
 & nû tret ich hie den wi\textit{l}den varm.\\ 
 & mir\textbf{n geswîchen} hende, \textbf{ietwede\textit{r}} arm,\\ 
 & ich gibe vür mîne \textbf{reise} \textbf{ein} pfant,\\ 
10 & daz niende\textit{r} bindet mich sîn hant."\\ 
 & daz wart ze beider sîte getân:\\ 
 & diu ros in den walap verl\textit{â}n,\\ 
 & mit sporn getriben und \textbf{ouch} gevuort\\ 
 & vaste ûf der rabîne hurt.\\ 
15 & ir \textbf{enweders} just dâ misseriet.\\ 
 & maniger juste e\textit{i}n \textbf{gegenniet}\\ 
 & was Parcifals hôhiu brust.\\ 
 & \textbf{den} lêrte kunst und sîne gelust,\\ 
 & daz sîn just als ebe\textit{n} vuor\\ 
20 & reht in den strick der helmsnuor.\\ 
 & er traf in, dâ man \textbf{henket} den schilt,\\ 
 & sô man ritterschafte spilt,\\ 
 & daz von Mun\textit{t}salvasche der templeis\\ 
 & von dem rosse in eine halden reis\\ 
25 & sô \textit{v}er\textit{r}e hin abe - diu was \textbf{sô} tief -,\\ 
 & daz dâ sîn leger wênic slief.\\ 
 & Parcifal, der \textbf{werde}, juste nâch\\ 
 & volgete. dem rosse was gâch;\\ 
 & ez viel hin abe, daz ez \textbf{gar} zerbra\textit{st}.\\ 
30 & Parcifal eines zêders ast\\ 
\end{tabular}
\scriptsize
\line(1,0){75} \newline
m n o \newline
\line(1,0){75} \newline
\newline
\line(1,0){75} \newline
\textbf{1} bereiter] bereit er n o \textbf{3} verzert] zerzert n \textbf{4} dâhte] gedochte n [gedac*]: gedacht o  $\cdot$ ich] \textit{om.} n ach o  $\cdot$ unernert] veneret m \textbf{5} rite] Riche m \textbf{6} würde] winde n \textbf{7} wilden] widen m \textbf{8} mirn] Mir n o  $\cdot$ ietweder] yettwedern m \textbf{10} niender] niende m nyergen n \textbf{12} den] dem o  $\cdot$ verlân] verlorn m \textbf{13} und ouch gevuort] vnd gefuͯrt n vngefurt o \textbf{15} enweders] ietweders n (o)  $\cdot$ dâ] do n o  $\cdot$ misseriet] [mitte]: misse riet o \textbf{16} ein gegenniet] engegen niet m (n) (o) \textbf{18} sîne] sin n o \textbf{19} just] suͯst o  $\cdot$ als eben] als ebe m alle eben n aleben o \textbf{21} dâ] do n o \textbf{22} sô] Do n \textbf{23} Muntsalvasche] munsaluasce m monsaluasce n muntsaluasce o  $\cdot$ der templeis] templis o \textbf{25} verre] were m \textbf{26} dâ] do n o \textbf{27} werde] \textit{om.} n o \textbf{28} volgete] Volget n (o)  $\cdot$ gâch] so goch n (o) \textbf{29} ez] \textit{om.} o  $\cdot$ zerbrast] zerbrach m \textbf{30} zêders] deders o \newline
\end{minipage}
\end{table}
\newpage
\begin{table}[ht]
\begin{minipage}[t]{0.5\linewidth}
\small
\begin{center}*G
\end{center}
\begin{tabular}{rl}
 & \begin{large}I\end{large}edoch bereit er sich ze \textit{der} tjost.\\ 
 & Parzival mit solher kost\\ 
 & het ouch sper vil verzert.\\ 
 & er dâhte: "\textbf{ich} wære unerner\textit{t},\\ 
5 & rite ich über disses mannes sât.\\ 
 & wie würde denne sînes zornes rât?\\ 
 & nû trit ich hie de\textit{n} wilden varm.\\ 
 & mir\textbf{n ge\textit{s}w\textit{î}chen} hende \textbf{unde} arm,\\ 
 & ich gibe vür mîn \textbf{reise} \textbf{ein} pfant,\\ 
10 & daz niender bindet mich sîn hant."\\ 
 & daz wart ze beider sîte getân:\\ 
 & diu ors in den walap verlân,\\ 
 & mit sporn getriben unde \textbf{ouch} gevuort\\ 
 & vaste ûf der rabîne hurt.\\ 
15 & ir \textbf{dewe\textit{de}rs} tjost dâ misseriet.\\ 
 & maniger tjoste ein \textbf{geinniet}\\ 
 & was Parzivales hôhiu brust.\\ 
 & \textbf{den} lêrte kunst unde sîn gelust,\\ 
 & daz sîn tjost als ebene vuor\\ 
20 & rehte in den stric der helmsnuor.\\ 
 & er traf in, dâ man \textbf{helt} den schilt,\\ 
 & sô man rîterschefte spilt,\\ 
 & daz von Muntsalvatsche der te\textit{m}peleis\\ 
 & von dem orse in eine halde reis\\ 
25 & sô verre hin abe - diu was tief -,\\ 
 & daz dâ sîn leger wênic slief.\\ 
 & Parzival der tjost nâch\\ 
 & volget. dem orse was \textbf{ouch} gâch;\\ 
 & ez vi\textit{e}l hin abe, daz ez zerbrast.\\ 
30 & Parzival eines zêders ast\\ 
\end{tabular}
\scriptsize
\line(1,0){75} \newline
G I O L M Z \newline
\line(1,0){75} \newline
\textbf{1} \textit{Initiale} G I O L M Z  \textbf{17} \textit{Initiale} I  \newline
\line(1,0){75} \newline
\textbf{1} Iedoch] ÷edoch O  $\cdot$ er] \textit{om.} I  $\cdot$ der] \textit{om.} G \textbf{2} Parzival] Parcival G parzifal I (L) (M) Barcifal O Parcifal Z \textbf{3} het] Hat M \textbf{4} dâhte] gedaht L  $\cdot$ wære] wer auch I  $\cdot$ unernert] vnernerte G \textbf{5} disses] dise O  $\cdot$ mannes] mannas L \textbf{6} denne sînes zornes] duses mannes M \textbf{7} trit] trette I (O) L (M) (Z)  $\cdot$ den] dem G  $\cdot$ varm] warm L \textbf{8} Mir engeswiche ýetweder arm L  $\cdot$ mirn geswîchen] Mirn gewischen G mir geswichen denn I Mir geswiche O Mir en geswigen M  $\cdot$ unde] vnde ietweder O (M) ietweder Z \textbf{9} gibe] gib im I gebe M \textbf{10} niender bindet mich] mich nirgen bindet M \textbf{11} sîte] siten M \textbf{12} in den] indem I (M)  $\cdot$ walap] halap I \textbf{13} getriben] [vnde]: getriben G  $\cdot$ ouch] \textit{om.} I O L M \textbf{15} ir deweders] ir dewers G Jr ýetweders L (M) Jetweders Z \textbf{16} ein geinniet] engegen biet I ein gegen biet L engegin nyet M \textbf{17} Parzivales] parziuales G parzifals I M Barcifals O parzifalz L parcifals Z \textbf{18} lêrte] lert I (L) \textbf{19} als] al O L M \textbf{20} der helmsnuor] des helms snuͤr I \textbf{21} dâ] rechte da M  $\cdot$ helt] hapt I (O) (L) het M heht Z \textbf{22} sô] vnd da I \textbf{23} Muntsalvatsche] mvntsalvasche G muntshaluasche I Munsalvatsche M montsalvatsch Z  $\cdot$ tempeleis] tepeleis G Tepeloys I \textbf{24} eine halde] einden I eine halden O (L) (M) (Z) \textbf{25} tief] so tif Z \textbf{26} dâ] \textit{om.} M  $\cdot$ leger wênic] langer wech I \textbf{27} Parzival] Parziua G parzifal I (L) (M) Barcifal O Parcifal Z \textbf{28} volget] Volgte O L (M)  $\cdot$ ouch gâch] so gach I zegach O (L) (M) (Z) \textbf{29} viel] vil G M Z  $\cdot$ zerbrast] gar zerbrast O L (M) (Z) \textbf{30} Parzival] Parziual G parzifal I (L) (M) Barcifal O Parcifal Z \newline
\end{minipage}
\hspace{0.5cm}
\begin{minipage}[t]{0.5\linewidth}
\small
\begin{center}*T
\end{center}
\begin{tabular}{rl}
 & iedoch bereiter sich zer tjost.\\ 
 & Parcifal mit solher kost\\ 
 & het ouch sper vil verzert.\\ 
 & er dâhte: "\textbf{er} wære \textbf{des} unernert,\\ 
5 & rite ich über disses mannes sât.\\ 
 & wie würde danne sînes zornes rât?\\ 
 & nû tret ich hie den wilden varm.\\ 
 & mir \textbf{geswîche} hende \textbf{unde} \textbf{ietweder} arm,\\ 
 & ich gib\textbf{im} vür mîne \textbf{reite} pfant,\\ 
10 & daz niender bindet mich sîn hant."\\ 
 & \begin{large}D\end{large}az wart ze beider sît getân:\\ 
 & diu ors in den walap verlân,\\ 
 & mit sporn getriben unde gevuort\\ 
 & vaste ûffe der rabîne hurt.\\ 
15 & ir \textbf{deweders} tjost dâ misseriet.\\ 
 & maneger tjost ein \textbf{gegenbiet}\\ 
 & was Parcifals hôh\textit{iu} brust.\\ 
 & \textbf{daz} lêrte kunst unde sîn gelust,\\ 
 & daz sîn tjost als ebene vuor\\ 
20 & reht in den stric der helmsnuor.\\ 
 & er traf in, dâ \textit{man} \textbf{hebet} den schilt,\\ 
 & sô man rîterschefte spilt,\\ 
 & daz von Munsalvasche der templeis\\ 
 & von dem orse in eine halden reis\\ 
25 & sô verre hin abe - diu was \textbf{sô} tief -,\\ 
 & daz dâ sîn leger wênic slief.\\ 
 & Parcifal der tjost nâch\\ 
 & volgete. dem orse was \textbf{ze} gâch;\\ 
 & ez viel hin abe, daz ez \textbf{gar} zerbrast.\\ 
30 & Parcifal eines zêders ast\\ 
\end{tabular}
\scriptsize
\line(1,0){75} \newline
T U V W Q R \newline
\line(1,0){75} \newline
\textbf{1} \textit{Initiale} W Q  \textbf{11} \textit{Initiale} T U  \newline
\line(1,0){75} \newline
\textbf{1} bereiter] bereitet Q \textbf{2} Parcifal] Parzifal V R Partzifal W Q \textbf{3} het] Hat W \textbf{4} er wære] ich were U (V) (W)  $\cdot$ des] \textit{om.} W Q R \textbf{5} disses] des W \textbf{6} zornes] zorne W \textbf{7} tret] dran U teet Q  $\cdot$ den] >den< U \textbf{8} mir] [M*]: Mirn V  $\cdot$ geswîche] in gesweche U gewichen Q schwachent beidú R  $\cdot$ unde] oder U  $\cdot$ ietweder] beide U auch Q \textit{om.} R \textbf{9} gibim] gebe U gibt V gib W R  $\cdot$ reite] reise U V (W) (Q) (R)  $\cdot$ pfant] [ei*]: ein phant V ein pfant W (Q) (R) \textbf{10} niender bindet mich] nider bindet mich U niergent bindet mich V niendert bindet sich W bindet nyendert R  $\cdot$ sîn] mein W \textbf{11} sît] [seide]: side Q \textbf{12} diu] die T  $\cdot$ den walap] dem loff R \textbf{13} gevuort] gehurt V \textbf{14} rabîne] hertten R  $\cdot$ hurt] furt V [sint hurt]: hurt Q \textbf{15} deweders] beider U [*weders]: ieweders V deweder Q  $\cdot$ tjost] stich R  $\cdot$ dâ] do U V W \textit{om.} R \textbf{16} tjost] stich R  $\cdot$ ein gegenbiet] en gegn bîet T (U) (V) engegen niet W R ein gen niet Q \textbf{17} Parcifals] parzifales V partzifals W partzifal Q parczifals R  $\cdot$ hôhiu] hohe T ein hoe Q \textbf{18} daz] Den U V W Q R  $\cdot$ kunst] kursit W \textbf{19} tjost] stich R \textbf{20} den stric] den [stri*]: strig V dem strick W den streit Q \textit{om.} R  $\cdot$ der helmsnuor] helm [svvͦr]: snuͦr T die helm snuͦr U der helme snuͦr V dem helm die schnuͦr R \textbf{21} dâ] do U V W Q  $\cdot$ man hebet] hebet T men habet V man hoͤhet W (R) \textbf{23} Munsalvasche] Mvnsalvasce T Muntsalvatsche U [mvntsch*]: mvntschalvasche V montsaluatz W muntsalvasche Q Munsaluesche R  $\cdot$ templeis] templis R \textbf{24} halden] halde U W Q \textbf{25} sô verre] Verre R  $\cdot$ diu] sie Q  $\cdot$ sô tief] tieff Q \textbf{26} dâ] do V W Q \textbf{27} Parcifal] Parzifal V Partzifal W Q  $\cdot$ der tjost] den stiche R \textbf{28} volgete] Volget W Q \textbf{29} gar] \textit{om.} V  $\cdot$ zerbrast] zu [bra*]: brach U zerbrach R \textbf{30} Parcifal] Parzifal V Partzifal W Q Parczifal R  $\cdot$ zêders] zeder Q  $\cdot$ ast] asch R \newline
\end{minipage}
\end{table}
\end{document}
