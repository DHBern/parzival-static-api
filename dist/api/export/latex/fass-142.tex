\documentclass[8pt,a4paper,notitlepage]{article}
\usepackage{fullpage}
\usepackage{ulem}
\usepackage{xltxtra}
\usepackage{datetime}
\renewcommand{\dateseparator}{.}
\dmyyyydate
\usepackage{fancyhdr}
\usepackage{ifthen}
\pagestyle{fancy}
\fancyhf{}
\renewcommand{\headrulewidth}{0pt}
\fancyfoot[L]{\ifthenelse{\value{page}=1}{\today, \currenttime{} Uhr}{}}
\begin{document}
\begin{table}[ht]
\begin{minipage}[t]{0.5\linewidth}
\small
\begin{center}*D
\end{center}
\begin{tabular}{rl}
\textbf{142} & si vorhte, daz er den lîp verlür\\ 
 & unt daz si \textbf{grôzen} schaden kür.\\ 
 & eine strâze er dô gevienc,\\ 
 & diu gein den Bertenoysen gienc.\\ 
5 & diu was \textbf{geestrîchet} unt breit.\\ 
 & swer im widergienc oder widerreit,\\ 
 & \textbf{ez} wære ritter oder koufman,\\ 
 & die selben gruozter alle sân\\ 
 & unt jach, \textbf{daz} wære sîner muoter rât.\\ 
10 & diu gab in \textbf{ouch} âne missetât.\\ 
 & der âbent begunde nâhen,\\ 
 & grôz müede gein im gâhen.\\ 
 & \begin{large}D\end{large}ô \textbf{ersach} der tumpheit genôz\\ 
 & ein hûs ze \textbf{guoter} mâze grôz.\\ 
15 & dâ \textbf{was inne} ein arger wirt,\\ 
 & als noch ûf \textbf{ungeslehte} birt.\\ 
 & \textbf{daz} was ein vischære\\ 
 & \textbf{unt} \textbf{aller} \textbf{güete} lære.\\ 
 & den knappen hunger lêrte,\\ 
20 & daz er dar gein kêrte\\ 
 & unt klagete dem wirte hungers nôt.\\ 
 & \textbf{der} sprach: "ine gæbe \textbf{iu} ein halbez brôt\\ 
 & niht ze drîzec jâren.\\ 
 & swer mîner milte vâren\\ 
25 & vergeben wil, der sûmet sich.\\ 
 & i\textbf{ne} sorge \textbf{umb niemen danne} umb mich,\\ 
 & dar nâch umb mîniu kindelîn.\\ 
 & ir enkomt tâlanc \textbf{dâ} her în.\\ 
 & het ir pfenninge oder pfant,\\ 
30 & ich behielt iuch al zehant."\\ 
\end{tabular}
\scriptsize
\line(1,0){75} \newline
D \newline
\line(1,0){75} \newline
\textbf{13} \textit{Initiale} D  \newline
\line(1,0){75} \newline
\newline
\end{minipage}
\hspace{0.5cm}
\begin{minipage}[t]{0.5\linewidth}
\small
\begin{center}*m
\end{center}
\begin{tabular}{rl}
 & si vorhte, daz er den lîp verlür\\ 
 & und daz si \textbf{grœzeren} schaden kür.\\ 
 & eine strâze er dô gevienc,\\ 
 & diu gegen den Britunoisen gienc.\\ 
5 & diu was \textbf{gestrichen} und breit.\\ 
 & wer ime widergienc oder widerreit,\\ 
 & \textbf{ez} wære ritter oder koufman,\\ 
 & die selben gruozete er alle sân\\ 
 & und jach, \textbf{daz} wære sîner muoter rât.\\ 
10 & diu gap in \textbf{ouch} ân \textbf{alle} missetât.\\ 
 & \textbf{dô} der âbent begunde nâhen\\ 
 & \textbf{und} grôz müede gegen ime gâhen,\\ 
 & dô \textbf{ersach} der tumpheite genôz\\ 
 & ein hûs, \textbf{daz was} ze mâze grôz.\\ 
15 & dâ \textbf{was inne} ein arger wirt,\\ 
 & als noch ûf \textbf{ungeslehte} birt.\\ 
 & \textbf{daz} was ein vischære\\ 
 & \textbf{aller} \textbf{güeter} lære.\\ 
 & den knappen hunge\textit{r l}êr\textit{t}e,\\ 
20 & daz er dar gegen kêrte\\ 
 & und klagete dem wirte hungers nôt.\\ 
 & \textbf{er} sprach: "ine gæbe ein halbez brôt\\ 
 & \textbf{ie} niht ze drîzic jâren.\\ 
 & wer mîner milte vâren\\ 
25 & vergebene wil, der sûmet sich.\\ 
 & \textit{i}\textbf{\textit{n}e} sorge \textbf{niht wanne} umbe mich,\\ 
 & dar nâch umb mîniu kindelîn.\\ 
 & ir enkomet tâlanc \textbf{dâ} her în.\\ 
 & hetet ir pfennige oder pfant,\\ 
30 & ich behielte iuch a\textit{l} zehant."\\ 
\end{tabular}
\scriptsize
\line(1,0){75} \newline
m n o \newline
\line(1,0){75} \newline
\newline
\line(1,0){75} \newline
\textbf{1} er] ir o  $\cdot$ verlür] verlor o \textbf{2} grœzeren] grossen n o \textbf{4} Britunoisen] britvnoissen m britaneysen n britaneisen o \textbf{6} widergienc] do wider ging o  $\cdot$ oder widerreit] oder reit n vnd reit o \textbf{8} selben] selbe o \textbf{9} daz] es n \textbf{10} alle] \textit{om.} n o \textbf{14} ze mâze] in mossen n (o)  $\cdot$ grôz] gruͦs o \textbf{19} hunger lêrte] hungert serre m \textbf{20} dar gegen] [gegen]: dogegen o \textbf{22} ine] ich ein n \textbf{26} ine] Me m  $\cdot$ wanne] dann o  $\cdot$ mich] [dich]: mich o \textbf{28} enkomet] koment n (o) \textbf{29} hetet] Hetten o \textbf{30} behielte] gehielte n (o)  $\cdot$ al] alle m \newline
\end{minipage}
\end{table}
\newpage
\begin{table}[ht]
\begin{minipage}[t]{0.5\linewidth}
\small
\begin{center}*G
\end{center}
\begin{tabular}{rl}
 & si vorhte, daz er den lîp verlür\\ 
 & unt daz si \textbf{grœzeren} schaden kür.\\ 
 & eine strâze er dô gevienc,\\ 
 & diu gein \textit{den} Britaneisen gienc.\\ 
5 & diu was \textbf{gebert} unde breit.\\ 
 & swer im widergienc oder widerreit,\\ 
 & \textbf{ez} wære rîter oder koufman,\\ 
 & die selben gruozter alle sân\\ 
 & unde jach, \textbf{ez} wære sîner muoter rât.\\ 
10 & diu gab in \textbf{im} âne missetât.\\ 
 & \begin{large}D\end{large}er âbent begunde nâhen,\\ 
 & grôz müede gein im gâhen.\\ 
 & dô \textbf{sach} der tumpheit genôz\\ 
 & ein hûs ze \textbf{guoter} mâze grôz.\\ 
15 & dâr \textbf{inne was} ein arger wirt,\\ 
 & als noch ûf \textbf{ungeslahte} birt.\\ 
 & \textbf{daz} was ein vischære\\ 
 & \textbf{unde} \textbf{maniger} \textbf{güete} lære.\\ 
 & den knappen hunge\textit{r} lêrte,\\ 
20 & daz er dar gegene kêrte\\ 
 & unde klagte dem wirte hungers nôt.\\ 
 & \textbf{er} sprach: "ine gæbe ein halbez brôt\\ 
 & \textbf{iu} niht ze drîzec jâren.\\ 
 & swer mîner milte vâren\\ 
25 & vergebene wil, der sûmet sich.\\ 
 & ich sorge \textbf{umbe niemen wan} umbe mich,\\ 
 & dar nâch umbe mîniu kindelîn.\\ 
 & ir enkomet tâlanc her în.\\ 
 & het ir pfenninge oder pfant,\\ 
30 & ich behielt iuch al zehant."\\ 
\end{tabular}
\scriptsize
\line(1,0){75} \newline
G I O L M Q R Z \newline
\line(1,0){75} \newline
\textbf{11} \textit{Initiale} G I O L Q R Z  \textbf{29} \textit{Initiale} I  \newline
\line(1,0){75} \newline
\textbf{2} daz si] da I  $\cdot$ grœzeren] Grozzen I (M) (Q) (R) (Z)  $\cdot$ schaden] \textit{om.} Z  $\cdot$ kür] [erl]: erkuͤr I \textbf{3} eine] Seine Q  $\cdot$ dô] \textit{om.} I da O M Z  $\cdot$ gevienc] gingk Q \textbf{4} diu] Der M  $\cdot$ den] \textit{om.} G O dem R  $\cdot$ Britaneisen] britoneisen G R pritonoisen I Britaneise O prittonaysen L briteneisin M brittaniesen Q britvneisen Z  $\cdot$ gienc] ge Z \textbf{5} diu] der I  $\cdot$ gebert] Gestrichen I (L) gestrichet O R Z gestricket M geschrigen Q \textbf{6} swer] Wer L Q R  $\cdot$ im] in R  $\cdot$ widergienc] wider fuͤr I  $\cdot$ oder] vnd Q  $\cdot$ widerreit] reit O L M R Z \textbf{8} gruozter] Gruzt er I (Q) (Z)  $\cdot$ alle sân] alsam I allen san Q \textbf{9} jach] sprach M iech R  $\cdot$ sîner] sin I (Q) \textbf{10} in im] im O in im ouch L en ouch M (Z) im auch Q (R) \textbf{11} Der] ÷er O  $\cdot$ nâhen] im nahen Z \textbf{12} müede] mute Q \textbf{13} dô] Da M Z  $\cdot$ tumpheit] tmpheit Q \textbf{14} ein] Es Q \textbf{16} ûf] \textit{om.} R  $\cdot$ birt] brit L \textbf{18} unde] \textit{om.} O L Q R  $\cdot$ maniger] aller I (Z)  $\cdot$ güete] guͯter L (M) \textbf{19} hunger] hungeren G \textbf{21} klagte] chlagt I \textbf{22} er] Der O L M R Z  $\cdot$ ine] ich I O  $\cdot$ gæbe] Geb ev I [*]: gebe iv O gilt R \textbf{23} iu niht] nih I (O) \textbf{24} swer] Wer L Q R  $\cdot$ vâren] wil varen I \textbf{25} vergebene wil] verGeben I Vnde vergeben wil O Wer geben wil Q  $\cdot$ sûmet] sunet Q \textbf{26} sorge] ensorge L (M) (Q) Z  $\cdot$ umbe niemen wan] vmb nieman nuͤn I (O) niht wan L vmb nyeman dan Q (R) (Z) \textbf{27} mîniu] mine R \textbf{28} enkomet] chomet O (M) (Q) (R)  $\cdot$ her] da her I (O) M Z (Q) hie her R \textbf{29} ir] \textit{om.} Q  $\cdot$ pfant] [p*]: pfund R \textbf{30} al] \textit{om.} I M alle R \newline
\end{minipage}
\hspace{0.5cm}
\begin{minipage}[t]{0.5\linewidth}
\small
\begin{center}*T (U)
\end{center}
\begin{tabular}{rl}
 & si vorhte, daz er den lîp verlür\\ 
 & und daz si \textbf{grœzern} schaden kür.\\ 
 & eine strâze er dô gevienc,\\ 
 & diu gein den Brituneisen gienc.\\ 
5 & diu was \textbf{geriten} unde breit.\\ 
 & wer im widergienc oder widerreit,\\ 
 & \textbf{er} wære rîter oder koufman,\\ 
 & die selben gruozter all\textit{e s}ân\\ 
 & und jach, \textbf{ez} wære sîner muoter rât.\\ 
10 & diu gab in \textbf{im} \textbf{ouch} ân missetât.\\ 
 & der âbent begunde nâhen,\\ 
 & grôz müede gein im gâhen.\\ 
 & dô \textbf{sach} der tumpheite genôz\\ 
 & ein hûs zuo \textbf{guoter} mâze grôz.\\ 
15 & dâr \textbf{in was} ein arger wirt,\\ 
 & als noch ûf \textbf{ungeslahte} birt,\\ 
 & \textbf{und} was ein vischære\\ 
 & \textbf{und} \textbf{aller} \textbf{güete} lære.\\ 
 & den knappen hunger lêrte,\\ 
20 & daz er d\textit{a}r gein kêrte\\ 
 & und klagete dem wirte hungers nôt.\\ 
 & \textbf{er} spr\textit{a}c\textit{h}: "in gæbe ein halbez brôt\\ 
 & \textbf{iu} niht zuo drîzic jâren.\\ 
 & wer mîner milte vâren\\ 
25 & vergeben wil, der sûmet sich.\\ 
 & ich \textbf{en}sorge \textbf{umb nieman dan} umb mich,\\ 
 & dar nâch umb mîniu kindelîn.\\ 
 & ir enkomet tâlanc \textbf{dâ} her în.\\ 
 & hetet ir pfenninge oder pfant,\\ 
30 & ich behielt iuch al zuohant."\\ 
\end{tabular}
\scriptsize
\line(1,0){75} \newline
U V W T \newline
\line(1,0){75} \newline
\textbf{3} \textit{Majuskel} T  \textbf{11} \textit{Überschrift:} Hie kvnt Parzifal zvͦ eines vischers [*]: hus der vngetruwe waz jn knaben wis V   $\cdot$ \textit{Initiale} T  \textbf{13} \textit{Majuskel} T  \textbf{24} \textit{Majuskel} T  \newline
\line(1,0){75} \newline
\textbf{1} vorhte] vochtin U \textbf{2} grœzern] grozen V \textbf{4} gein den] gegen [dem]: den V giengen den T  $\cdot$ Brituneisen] Britoneisen U brittuneisen V Britvnêisen T \textbf{5} geriten] getriben W gestrichen T \textbf{6} wer] Swer V (T)  $\cdot$ widerreit] reit V (W) T \textbf{7} er] ez T \textbf{8} alle sân] allez an U T alsan V \textbf{10} in] \textit{om.} W \textbf{12} gein im] im engegene V engegen W engegn im T \textbf{13} sach] gesach T \textbf{14} zuo] in W  $\cdot$ mâze] mazen V \textbf{16} ungeslahte] vngeslehte V (W) geslehte T \textbf{17} und] Dis W daz T \textbf{18} aller] maneger T \textbf{20} dar] dir U \textbf{22} er sprach] Er [spra*]: spranc U der sprach T  $\cdot$ in gæbe] ich geb úch W \textbf{23} iu] \textit{om.} W  $\cdot$ zuo drîzic] in dreissig W zehvndert T \textbf{24} wer] Swer V (T)  $\cdot$ mîner milte] milte an mich wil W \textbf{25} wil] \textit{om.} W  $\cdot$ sûmet] sumet iedoch W \textbf{26} ensorge] sorg W  $\cdot$ umb nieman dan] vmb niemant wann W niht wan T  $\cdot$ umb mich] fúr mich W \textbf{27} kindelîn] klaine kindelin W \textbf{28} enkomet] kvmment V (W)  $\cdot$ dâ] \textit{om.} V \textbf{29} hetet] Habent W \textbf{30} behielt] gehielt V \newline
\end{minipage}
\end{table}
\end{document}
