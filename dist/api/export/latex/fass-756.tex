\documentclass[8pt,a4paper,notitlepage]{article}
\usepackage{fullpage}
\usepackage{ulem}
\usepackage{xltxtra}
\usepackage{datetime}
\renewcommand{\dateseparator}{.}
\dmyyyydate
\usepackage{fancyhdr}
\usepackage{ifthen}
\pagestyle{fancy}
\fancyhf{}
\renewcommand{\headrulewidth}{0pt}
\fancyfoot[L]{\ifthenelse{\value{page}=1}{\today, \currenttime{} Uhr}{}}
\begin{document}
\begin{table}[ht]
\begin{minipage}[t]{0.5\linewidth}
\small
\begin{center}*D
\end{center}
\begin{tabular}{rl}
\textbf{756} & wol nâch strîtes êre\\ 
 & helme unt ir schilde sêre\\ 
 & \textbf{wâren mit swerten} an gerant.\\ 
 & \textbf{ieweder} wol gelêrte hant\\ 
5 & truoc, \textbf{der} diu strîtes mâl \textbf{entwarf}.\\ 
 & in strîte man \textbf{ouch} kunst bedarf.\\ 
 & Bî Artuse\textit{s} ringe hin\\ 
 & \textbf{si} riten. dâ wart vil nâch in\\ 
 & \textbf{geschouwet}, dâ der heiden reit;\\ 
10 & der vuort \textbf{êt} solhe \textbf{rîcheit}.\\ 
 & wol \textbf{beherberget} \textbf{was} daz velt.\\ 
 & si kêrten vür daz \textbf{hôch} gezelt\\ 
 & \textbf{an} Gawans ringe.\\ 
 & ob mans iht innen bringe,\\ 
15 & daz man si gerne sæhe?\\ 
 & ich wæne, daz \textbf{dâ} geschæhe.\\ 
 & Gawan kom snellîche nâch,\\ 
 & wand\textbf{er} \textbf{vor} Artuse sach,\\ 
 & daz si gein sîme gezelte riten.\\ 
20 & \textbf{der} enpfienc si dâ mit vreude siten.\\ 
 & Si hetenz harnasch dennoch an.\\ 
 & Gawan, der höfsche man,\\ 
 & hiez si entwâpen schiere.\\ 
 & ecidemôn, dem tiere,\\ 
25 & was geteilet mit der strît.\\ 
 & der heiden truog ein kursît\\ 
 & - dem \textbf{was von slegen ouch} worden wê -,\\ 
 & daz was ein saranthasmê.\\ 
 & \textbf{dâr an} stuont manec tiwer stein.\\ 
30 & dâr unde ein wâpenroc erschein,\\ 
\end{tabular}
\scriptsize
\line(1,0){75} \newline
D Fr12 \newline
\line(1,0){75} \newline
\textbf{7} \textit{Majuskel} D  \textbf{21} \textit{Majuskel} D  \newline
\line(1,0){75} \newline
\textbf{7} Artuses] Artvse D \textbf{20} vreude] vrouden Fr12 \newline
\end{minipage}
\hspace{0.5cm}
\begin{minipage}[t]{0.5\linewidth}
\small
\begin{center}*m
\end{center}
\begin{tabular}{rl}
 & wol nâch strîtes êre\\ 
 & \textbf{ir} helm und ir schilt sêre\\ 
 & \textbf{mit swerten wâren} an gerant.\\ 
 & \textbf{ietweder} wol gelêrte hant\\ 
5 & truoc, \textbf{der} diu strîtes mâl \textbf{en\textit{t}warf}.\\ 
 & in strîte man \textbf{ouch} kunst bedarf.\\ 
 & bî Artuses ringe \textbf{si} hin\\ 
 & riten. dô wart vil nâch in\\ 
 & \textbf{gewartet}, d\textit{â} der heiden reit;\\ 
10 & der vuort \textbf{ouch} solich \textbf{rîcheit}.\\ 
 & wol \textbf{geherbe\textit{rge}t} \textbf{was} daz velt.\\ 
 & si kêrten vür daz \textbf{hôch} gezelt\\ 
 & \textbf{an} Gawan\textit{e}s ringe.\\ 
 & ob mans ih\textit{t} innen bringe,\\ 
15 & daz man si \textbf{d\textit{â}} gerne sæhe?\\ 
 & ich wæn, daz \textbf{dâ} geschæhe.\\ 
 & Gawan kam snelleclîch nâch,\\ 
 & wan \textbf{der} \textbf{vor} Artu\textit{se} sach,\\ 
 & daz si gegen sînem gezelt riten.\\ 
20 & \textbf{er} enpfienc si d\textit{â} mit vröude siten.\\ 
 & si hetten den harnasch dannoch an.\\ 
 & Gawan, der hübsch man,\\ 
 & hie si entwâpen schier.\\ 
 & ecidemôn, dem tier,\\ 
25 & was geteilet mit der strît.\\ 
 & der heiden truoc ein kursît\\ 
 & - dem \textbf{ouch von slegen was} worden wê -,\\ 
 & daz was ein saranthasmê.\\ 
 & \textbf{an dem} stuont manic tiur stei\textit{n}.\\ 
30 & dâr u\textit{nder} ein wâpenroc erschei\textit{n},\\ 
\end{tabular}
\scriptsize
\line(1,0){75} \newline
m n o V V' \newline
\line(1,0){75} \newline
\newline
\line(1,0){75} \newline
\textbf{1} \textit{Versdoppelung 755.16-756.20 (²o) nach 756.20; Lesarten der vorausgehenden Verse mit ¹o bezeichnet} o  \textbf{5} entwarf] enwarf m (n) \textbf{7} Artuses] artus m n o V V' \textbf{9} dâ] do m n o V V' \textbf{10} ouch] eht V (V') \textbf{11} geherberget] geherbet m \textbf{12} daz] die V' \textbf{13} an] [*]: Gegen V Gegen V'  $\cdot$ Gawanes] gawanens m gawans n \textsuperscript{1}\hspace{-1.3mm} o V' [*]: Gawans V \textbf{14} iht] ihs m \textbf{15} dâ] do m \textit{om.} n o V V' \textbf{16} dâ] do n \textsuperscript{1}\hspace{-1.3mm} o V V' \textbf{18} der] er V V'  $\cdot$ Artuse] artu m \textbf{20} dâ] do m n o V'  $\cdot$ vröude] freiden n (o) (V') \textbf{21} hetten den] hettens V V' \textbf{23} hie] Hiesse n  $\cdot$ entwâpen] entwoppent o  $\cdot$ schier] gar schier o \textbf{24} \textit{Die Verse 756.24-757.7 fehlen} V'  \textbf{28} saranthasmê] saranthasine n saratasine o \textbf{29} manic] mange o  $\cdot$ stein] steine m \textbf{30} under] vmb m  $\cdot$ erschein] erscheine m \newline
\end{minipage}
\end{table}
\newpage
\begin{table}[ht]
\begin{minipage}[t]{0.5\linewidth}
\small
\begin{center}*G
\end{center}
\begin{tabular}{rl}
 & \begin{large}W\end{large}ol nâch strîtes êre\\ 
 & helm unde ir schilte sêre\\ 
 & \textbf{wâren mit swerten} an gerant.\\ 
 & \textbf{ietwederre} wol gelêrte hant\\ 
5 & truoc, diu strîtes mâl \textbf{erwarf}.\\ 
 & in strîte man \textbf{wol} kunst bedarf.\\ 
 & bî Artuses ringe hin\\ 
 & \textbf{si} riten. dô wart vil nâch in\\ 
 & \textbf{geschouwet}, dâ der heiden reit;\\ 
10 & der vuort \textbf{êt} \textbf{an} solh \textbf{kleit}.\\ 
 & wol \textbf{beherberget} \textbf{was} daz velt.\\ 
 & si kêrten vür daz \textbf{hôch} gezelt\\ 
 & \textbf{gein} Gawans ringe.\\ 
 & op mans iht innen bringe,\\ 
15 & daz man si gerne sæhe?\\ 
 & ich wæne \textbf{ouch}, daz geschæhe.\\ 
 & Gawan kom snellîchen nâch,\\ 
 & wan \textbf{er} \textbf{vor} Artuse sach,\\ 
 & daz si gein sînem gezelte riten.\\ 
20 & \textbf{der} enpfienc si dâ mit vröuden siten.\\ 
 & si hetenz harnasch dannoch an.\\ 
 & Gawan, der \textbf{stolze} höfsche man,\\ 
 & hiez si entwâpen schiere.\\ 
 & ecidemôn, dem tiere,\\ 
25 & was geteilt mit der strît.\\ 
 & der heiden truoc ein kursît\\ 
 & - dem \textbf{was von slegen} worden wê -,\\ 
 & daz was ein saranthasmê.\\ 
 & \textbf{dâr an} stuont manic tiwer stein.\\ 
30 & dâr under ein wâpenroc erschein,\\ 
\end{tabular}
\scriptsize
\line(1,0){75} \newline
G I L M Z Fr43 Fr48 \newline
\line(1,0){75} \newline
\textbf{1} \textit{Initiale} G I L Fr43 Fr48  \textbf{7} \textit{Initiale} M  \textbf{25} \textit{Initiale} I  \newline
\line(1,0){75} \newline
\textbf{2} ir] \textit{om.} L  $\cdot$ schilte sêre] shiltes ere I \textbf{4} ietwederre] Jclichers M  $\cdot$ gelêrte] gerte L \textbf{5} diu] des L der die M (Z) (Fr43) (Fr48)  $\cdot$ erwarf] entwarf Z Fr43 Fr48 \textbf{6} wol] och wol L (M) ouch Z (Fr43) Fr48 \textbf{7} Artuses] Artvs G (Z) (Fr48) \textbf{8} dô] da L M Z Fr43  $\cdot$ in] \textit{nachträglich hinzugefügt} Z hin Fr48 \textbf{9} dâ] \textit{om.} L do Fr48 \textbf{10} êt an solh] et an solhev I riliche L an thure M ein solich Z ot an tiuriv Fr43 an solch Fr48 \textbf{11} beherberget] Geherberget I beherget L \textbf{12} gezelt] zelt Fr48 \textbf{13} Gawans] Gawanes L Fr43 \textbf{14} mans iht] man siz I man sý ieht L (M) \textbf{15} gerne] iht Gerne I \textbf{16} ouch] \textit{om.} Z Fr48  $\cdot$ daz] daz da I L Z das das M daz do Fr48 \textbf{17} kom] kom ouch L \textbf{18} Artuse] Artv̂se G artus Z Fr48 \textbf{20} dâ] \textit{om.} I do Fr48  $\cdot$ vröuden] frevde Z (Fr48) \textbf{22} stolze] \textit{om.} L M Z Fr43 Fr48  $\cdot$ höfsche] hofsher I \textbf{23} entwâpen] enphaen M \textbf{26} heiden] heide M \textbf{27} von] vor M  $\cdot$ worden] och worden L (M) (Z) (Fr43) (Fr48) \textbf{28} saranthasmê] sarantasme I L Sarianthasme Fr48 \textbf{30} erschein] shein I \newline
\end{minipage}
\hspace{0.5cm}
\begin{minipage}[t]{0.5\linewidth}
\small
\begin{center}*T
\end{center}
\begin{tabular}{rl}
 & wol nâch strîtes êre\\ 
 & helme und ir schilde sêre\\ 
 & \textbf{wâren mit swerten} an ger\textit{ant}.\\ 
 & \textit{\textbf{ietweders} wol gelêrte hant}\\ 
5 & \textit{truoc, \textbf{der} diu strîtes mâl \textbf{entwarf}.}\\ 
 & \textit{in strît man \textbf{ouch} \textbf{wol} kunst bedarf.}\\ 
 & \textit{bî Artuses ringe hin}\\ 
 & \textit{\textbf{si} riten. dô wart vil nâch in}\\ 
 & \textit{\textbf{beschouwet}}\textit{, d}â \textit{der} \textit{heiden} \textit{reit;}\\ 
10 & \textit{der vuorte \textbf{eht} solhe \textbf{rîcheit}.}\\ 
 & \textit{wol \textbf{beherbergt} \textbf{wart} daz velt.}\\ 
 & \textit{si kêrten vür daz \textbf{grôz} gezelt}\\ 
 & \textit{\textbf{gegen} \textbf{hêrren} Gawans ringe.}\\ 
 & \textit{ob man si iht innen bringe,}\\ 
15 & \textit{daz man si gerne s}\textit{æ}\textit{he?}\\ 
 & \textit{ich wæne \textbf{ouch}, daz \textbf{daz} geschæhe.}\\ 
 & \textit{Gawan kam snelliclîche nâch,}\\ 
 & \textit{wan \textbf{er} \textbf{bî} Artus sach,}\\ 
 & \textit{daz si gein sînem gezelte r}iten.\\ 
20 & \textbf{der} enpfienc si dâ mit vreuden siten.\\ 
 & si heten daz harnasch dannoch an.\\ 
 & Gawan, der hövesch man,\\ 
 & hiez si entwâpen schiere.\\ 
 & ecidemôn, dem tiere,\\ 
25 & was geteilet \textit{mit} der strît.\\ 
 & der heiden truoc ein kursît\\ 
 & - dem \textbf{was von slegen} worden wê -,\\ 
 & daz was ein saranthasmê.\\ 
 & \textbf{dâr} stuont manec tiure stein.\\ 
30 & dâr under ein wâpenroc erschein,\\ 
\end{tabular}
\scriptsize
\line(1,0){75} \newline
U W Q R \newline
\line(1,0){75} \newline
\textbf{7} \textit{Initiale} W  \newline
\line(1,0){75} \newline
\textbf{3} gerant] geriten U \textbf{4} \textit{Die Verse 756.4-19 fehlen} U   $\cdot$ ietweders] Jtweder Q (R) \textbf{5} der diu] den der R  $\cdot$ entwarf] erwarff Q an traff R \textbf{6} wol] \textit{om.} Q R \textbf{7} Artuses] artus Q R \textbf{9} beschouwet] Geschawet Q (R)  $\cdot$ dâ] do W Q \textbf{10} eht] auch Q doch R \textbf{11} wart] wasz Q (R) \textbf{12} grôz] hoch Q R \textbf{13} hêrren] her W \textit{om.} Q R  $\cdot$ Gawans] Gawins R \textbf{14} innen] Inna R \textbf{16} daz daz] das do Q \textbf{17} nâch] nache R \textbf{18} bî] vor Q R  $\cdot$ Artus] artusze Q  $\cdot$ sach] seche R \textbf{19} gezelte] zeltte R \textbf{20} der enpfienc] Dern enphieck Q Er empfieng R  $\cdot$ dâ] do W Q R \textbf{21} harnasch] [harnac*]: harnashe R \textbf{22} Gawan] Gawin R \textbf{23} entwâpen] entwapnett R \textbf{25} geteilet] geteit R  $\cdot$ mit] im U \textit{om.} W \textbf{27} worden] auch worden Q \textbf{29} dâr] Daran W (Q) R  $\cdot$ tiure] edel R  $\cdot$ stein] steẏm Q \newline
\end{minipage}
\end{table}
\end{document}
