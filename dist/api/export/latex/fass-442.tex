\documentclass[8pt,a4paper,notitlepage]{article}
\usepackage{fullpage}
\usepackage{ulem}
\usepackage{xltxtra}
\usepackage{datetime}
\renewcommand{\dateseparator}{.}
\dmyyyydate
\usepackage{fancyhdr}
\usepackage{ifthen}
\pagestyle{fancy}
\fancyhf{}
\renewcommand{\headrulewidth}{0pt}
\fancyfoot[L]{\ifthenelse{\value{page}=1}{\today, \currenttime{} Uhr}{}}
\begin{document}
\begin{table}[ht]
\begin{minipage}[t]{0.5\linewidth}
\small
\begin{center}*D
\end{center}
\begin{tabular}{rl}
\textbf{442} & "\textit{\begin{large}I\end{large}}ch warp, als der den schaden hât",\\ 
 & sprach er, "liebiu niftel, gip mir rât.\\ 
 & gedenke rehter sippe an mir\\ 
 & unt sage mir ouch: wie stêtz dir?\\ 
5 & ich solte trûren umbe dîne klage,\\ 
 & wan daz ich hœhern kumber trage,\\ 
 & den ieman getrüege.\\ 
 & mîn nôt ist \textbf{z}ungevüege."\\ 
 & Si sprach: "nû helfe dir des hant,\\ 
10 & dem aller kumber ist bekant,\\ 
 & ob dir sô wol gelinge,\\ 
 & daz dich ein slâ dar bringe,\\ 
 & al dâ dû Munsalvæsche sihst,\\ 
 & \textbf{dâ} dû mir dîner vröuden gihst.\\ 
15 & Cundrie la surziere reit\\ 
 & \textbf{vil müelîch} hinnen. mir ist leit,\\ 
 & daz ich niht vrâgte, ob si dar\\ 
 & wolte \textbf{kêren} oder anderswar.\\ 
 & immer \textbf{swenne} si kumt, ir mûl dort stêt,\\ 
20 & dâ der brunne \textbf{ûzem} velse gêt.\\ 
 & ich râte, daz dû ir rîtes nâch.\\ 
 & ir ist lîhte vor dir niht sô gâch,\\ 
 & dûne mügest si schiere hân erriten."\\ 
 & dâne wart niht langer dô gebiten.\\ 
25 & urloup nam der helt al dâ.\\ 
 & \textbf{dô} kêrt \textbf{er} ûf die niwen slâ.\\ 
 & Cundrien m\textit{û}l die reise gienc.\\ 
 & daz ungeverte im undervienc\\ 
 & eine slâ, die er het erkorn.\\ 
30 & sus wart aber der Grâl verlorn.\\ 
\end{tabular}
\scriptsize
\line(1,0){75} \newline
D Fr5 Fr31 \newline
\line(1,0){75} \newline
\textbf{1} \textit{Initiale} D Fr5  \textbf{9} \textit{Majuskel} D  \newline
\line(1,0){75} \newline
\textbf{1} Ich] ÷ch D  $\cdot$ den] der Fr5 \textbf{2} gip] nu gib Fr5 \textbf{6} wan] \textit{om.} Fr5 \textbf{13} Munsalvæsche] Mvnsælvæsce D muntsaluasch Fr5  $\cdot$ sihst] \textit{om.} Fr5 \textbf{15} Cvndrie Lasvrzîere reit D  $\cdot$ Kundrie latzurZire reit Fr5 \textbf{16} müelîch] nivwilich Fr5 (Fr31) \textbf{20} ûzem velse] vz der velse Fr5 vz dem velsen Fr31 \textbf{22} vor] von Fr5 Fr31  $\cdot$ sô] ze Fr31 \textbf{24} dâne] Da Fr5 Fr31 \textbf{26} dô] Sa Fr31  $\cdot$ kêrt er] kertir Fr5  $\cdot$ slâ] slage sa Fr5 \textbf{27} Cundrien] Cvndrîen D Kundrien Fr5 (Fr31)  $\cdot$ mûl] mwͦel D \textbf{28} ungeverte] vngerete Fr31  $\cdot$ im] in Fr5 Fr31 \textbf{29} slâ] hufslah Fr5 :::sla Fr31 \textbf{30} Sus w::: der gral abe verlorn Fr31 \newline
\end{minipage}
\hspace{0.5cm}
\begin{minipage}[t]{0.5\linewidth}
\small
\begin{center}*m
\end{center}
\begin{tabular}{rl}
 & "\begin{large}I\end{large}ch warp, als der den schaden hât",\\ 
 & sprach er, "liebiu niftel, gip mir rât.\\ 
 & gedenke rehter sippe an mir\\ 
 & und sage mir ouch: wie stât ez dir?\\ 
5 & ich solte trûren umb dîne klage,\\ 
 & wanne daz ich hœhe\textit{r}n kumber trage,\\ 
 & danne iem\textit{an} getrüege.\\ 
 & mîn nôt ist \textbf{zuo} ungevüege."\\ 
 & si sprach: "nû helfe dir des hant,\\ 
10 & dem aller kumber ist bekant,\\ 
 & ob dir sô wol gelinge,\\ 
 & daz dich ein slâ dar bringe,\\ 
 & aldâ dû Mun\textit{t}salvasche sihest,\\ 
 & \textbf{dâ} dû mir dîner vröuden gihest.\\ 
15 & Condrie la surziere reit\\ 
 & \textbf{vil müelîche} hinnen. mir ist leit,\\ 
 & daz ich niht vrâgete, ob si dar\\ 
 & wolte \textbf{kêren} oder anderswa\textit{r}.\\ 
 & iemer \textbf{sô} si kumt, ir mûl dort stât,\\ 
20 & d\textit{â} der brunne \textbf{ûz dem} velse gât.\\ 
 & ich râte, daz dû ir rîtest nâch.\\ 
 & ir ist lîhte vor dir niht sô gâch,\\ 
 & dû enmügest si schiere hân erriten."\\ 
 & dô enwart niht langer d\textit{â} gebiten.\\ 
25 & urloup nam der helt aldâ.\\ 
 & \textbf{dô} kêrte \textbf{er} ûf die niuwen slâ.\\ 
 & Condrien mûl die reise gienc.\\ 
 & daz ungeverte ime undervienc\\ 
 & eine slâ, die er hete erkorn.\\ 
30 & sus wart aber der Grâl verlorn.\\ 
\end{tabular}
\scriptsize
\line(1,0){75} \newline
m n o \newline
\line(1,0){75} \newline
\textbf{1} \textit{Initiale} m  \newline
\line(1,0){75} \newline
\textbf{3} gedenke] Gedanck o \textbf{4} \textit{Versdoppelung (vermutlich mit Anteil aus Vers 442.3):} Vnd sage mir was richt er sie / Vnd sage mir auch wie stat ez dir o  \textbf{5} dîne] den o \textbf{6} hœhern] hohen m (o)  $\cdot$ trage] klage trage o \textbf{7} ieman] yemer m e >man< o  $\cdot$ getrüege] getruwe o \textbf{8} zuo] so n \textbf{11} dir] [si]: dir m \textbf{13} Muntsalvasche] munsaluasce m muntsaluasce n munt saluasce o \textbf{14} dâ] Vnd n o \textbf{15} Condrie lasurziere reit m  $\cdot$ Condrie lasurtzier reit n  $\cdot$ Condrie lasurczier reit o \textbf{16} müelîche] niuͯlich n (o)  $\cdot$ mir ist] mit ir n o \textbf{18} anderswar] anderswa dar m anderswa o \textbf{19} sô si kumt] in n \textbf{20} dâ] Do m o So n  $\cdot$ velse] felch o \textbf{21} râte] wolte n \textbf{22} lîhte vor dir] fúr dir lichte n (o) \textbf{23} enmügest] muͯgest n (o) \textbf{24} dâ] do m \textit{om.} n o \textbf{26} dô kêrte er] Vnd kerte n o  $\cdot$ die] due o  $\cdot$ niuwen] nuwe n o \textbf{27} Condrien] Condrie n Cundrie o \textbf{28} daz ungeverte] Des vngefuͯrget das o \newline
\end{minipage}
\end{table}
\newpage
\begin{table}[ht]
\begin{minipage}[t]{0.5\linewidth}
\small
\begin{center}*G
\end{center}
\begin{tabular}{rl}
 & "\begin{large}I\end{large}ch warp, al\textit{s d}er den schaden hât",\\ 
 & sprach er, "liebiu niftel, gib mir rât.\\ 
 & gedenke rehter sippe an mir\\ 
 & unde sage mir \textit{ouch}: wie stêtz dir?\\ 
5 & ich solde trûren umbe dîne klage,\\ 
 & wan daz ich hœhern kumber trage,\\ 
 & danne ie man getrüege.\\ 
 & mîn nôt ist \textbf{ze} ungevüege."\\ 
 & si sprach: "nû helfe dir des hant,\\ 
10 & dem aller kumber ist bekant,\\ 
 & obe dir sô wol gelinge,\\ 
 & daz dich ein slâ dar bringe,\\ 
 & al dâ dû Muntsalvatsche sihest,\\ 
 & \textbf{dâ} dû mir dîner vröuden gihest.\\ 
15 & Gundrie lasurziere reit\\ 
 & \textbf{niulîche} hinnen. mir ist leit,\\ 
 & daz ich niht vrâgete, ob si dar\\ 
 & wolde \textbf{kêren} oder anderswar.\\ 
 & immer \textbf{swenne} si kumet, ir mûl dort stêt,\\ 
20 & dâ der brunne \textbf{ûz dem} velse gêt.\\ 
 & ich râte, daz dû ir rîtes nâch.\\ 
 & ir ist lîhte vor dir niht sô gâch,\\ 
 & dûne mügest si schier hân erriten."\\ 
 & dône wart niht lenger dâ gebiten.\\ 
25 & urloup nam der helt al dâ.\\ 
 & \textbf{dô} kêrt \textbf{er} ûf die niuwen slâ.\\ 
 & Gundrien mûl die reise gienc.\\ 
 & daz ungeverte im undervienc\\ 
 & ein slâ, die er hete erkorn.\\ 
30 & sus wart aber der Grâl verlorn.\\ 
\end{tabular}
\scriptsize
\line(1,0){75} \newline
G I O L M Z \newline
\line(1,0){75} \newline
\textbf{1} \textit{Initiale} G O L Z  \textbf{15} \textit{Initiale} I  \newline
\line(1,0){75} \newline
\textbf{1} Ich] ÷ch O  $\cdot$ warp als] wirbe alz L  $\cdot$ als der den] als der der den G alz dern L alse der [degin]: den M \textbf{2} gib mir] nu gip mir dinen I \textbf{3} rehter] reht I \textbf{4} ouch] \textit{om.} G \textbf{6} hœhern] hohin M \textbf{7} ie man getrüege] ieman vertruͯge L \textbf{8} mîn] Mit L  $\cdot$ ist ze] ist O dy ist zcu M \textbf{9} dir] mir L \textbf{10} aller] alle M \textbf{13} Muntsalvatsche] mvntsaltvasche G Muntshalvasche I montsalvatsche Z \textbf{14} dâ] [du]: da I Das M  $\cdot$ dîner] der M  $\cdot$ gihest] vergihst Z \textbf{15} Gvndrîe lasurziere reit G  $\cdot$ Kvndrie lasvrziere reit O (M)  $\cdot$ Kvndrie Lazvrzirie reit L  $\cdot$ Cvndrie lasvrziere reit Z \textbf{16} niulîche] Vil niͮlich O (L) (Z) Vil Mulichin M \textbf{19} swenne] so L wenne M \textbf{20} ûz dem velse] durc den velsen I \textbf{21} ich râte] nu rat ich I  $\cdot$ dû] \textit{om.} M Z  $\cdot$ ir] \textit{om.} L \textbf{22} vor] von L  $\cdot$ sô gâch] zegach I \textbf{23} dûne] Dv O  $\cdot$ si] \textit{om.} L  $\cdot$ erriten] erretten M \textbf{24} dône wart] Da wart O L Danne wart M Da enwart Z  $\cdot$ dâ] \textit{om.} I do L \textbf{25} al] \textit{om.} I L \textbf{26} dô kêrt er] vnde cherte O (L) (M) Da kert er Z \textbf{27} Gundrien] Kvndrien O L (M) Cvndrien Z \textbf{28} ungeverte] vnderverte Z  $\cdot$ undervienc] wideruienc I \textbf{29} erkorn] vorkorn M \newline
\end{minipage}
\hspace{0.5cm}
\begin{minipage}[t]{0.5\linewidth}
\small
\begin{center}*T
\end{center}
\begin{tabular}{rl}
 & "Ich warp, als der den schaden hât",\\ 
 & sprach er, "liebiu niftele, \textbf{nû} gip mir rât.\\ 
 & gedenke rehter sippe an mir\\ 
 & unde sage mir ouch: wie stêt ez dir?\\ 
5 & ich solte trûren umbe dîne klage,\\ 
 & wan daz ich hœhern kumber trage,\\ 
 & danne ie man getrüege.\\ 
 & mîn nôt ist ungevüege."\\ 
 & \begin{large}S\end{large}i sprach: "nû helfe dir des hant,\\ 
10 & dem aller kumber ist bekant,\\ 
 & ob dir sô wol gelinge,\\ 
 & daz dich ein slâ dar bringe,\\ 
 & aldâ dû Munsalvasche sihst,\\ 
 & \textbf{daz} dû mir dîner vröude gihst.\\ 
15 & Kundrie Lasurziere reit\\ 
 & \textbf{niuweclîchen} hinnen. mir ist leit,\\ 
 & daz ich \textbf{si} niht vrâgete, ob si dar\\ 
 & wolte oder anderswar.\\ 
 & iemer \textbf{swenne} si kumt, ir mûl dort stêt,\\ 
20 & dâ der brunne \textbf{ûz einem} velse gêt.\\ 
 & ich râte \textbf{dir}, daz dû ir rîtest nâch.\\ 
 & ir ist lîhte vor dir niht sô gâch,\\ 
 & dûne mügest si schiere hân erriten."\\ 
 & dâne wart niht langer dô gebiten.\\ 
25 & urloup nam der helt aldâ\\ 
 & \textbf{unde} kêrte ûf die niuwen slâ.\\ 
 & Kundrien mûl die reise gienc.\\ 
 & daz ungeverte im undervienc\\ 
 & eine slâ, dier hete erkorn.\\ 
30 & sus wart aber der Grâl verlorn.\\ 
\end{tabular}
\scriptsize
\line(1,0){75} \newline
T U V W Q R \newline
\line(1,0){75} \newline
\textbf{1} \textit{Initiale} Q   $\cdot$ \textit{Majuskel} T  \textbf{9} \textit{Initiale} T U V R  \newline
\line(1,0){75} \newline
\textbf{1} den] \textit{om.} W \textbf{2} liebiu] [liber]: libe Q  $\cdot$ nû] \textit{om.} W R \textbf{3} rehter] rehte V (W) (Q) \textbf{4} mir ouch] [mir*]: mir auch U mir V auch mir W  $\cdot$ stêt] get W \textbf{6} hœhern] horen Q  $\cdot$ trage] clage trage R \textbf{7} ie man] man ye Q  $\cdot$ getrüege] [getr*ge]: getrvͤge V \textbf{8} ungevüege] zuͦ vngevuͦge U (W) (Q) [*]: zevngefvͤge  V so vngefuͯg R \textbf{10} Dem allú ding sind bekant R  $\cdot$ aller] alle U \textbf{12} dich ein slâ] ein sla dich U ich an huͦffschlag R \textbf{13} aldâ] Das R  $\cdot$ dû] die U  $\cdot$ Munsalvasche] mvnsalvasce T muͦntsalvatsche U [mvntsc*]: mvntschalvasche V montsaluatsch W muntsalsche Q munschalualesche R \textbf{14} daz] [Daz]: Da V Do Q R  $\cdot$ mir] mit Q  $\cdot$ vröude] vreiden U (V) (W) (R) \textbf{15} kvndrie salvrziere reit T  $\cdot$ Kvndrie lazvrziere reit V (Q) · Kundrye lasurziere reit R \textbf{16} niuweclîchen] Vil nvͦweliche U (V) (W) (Q) (R) \textbf{17} si niht] nit nit U niht V (W) (Q) (R) \textbf{18} wolte] Wolte keren U V W (Q) (R) \textbf{19} swenne] wanne U (W) (Q) (R) swen V  $\cdot$ dort stêt] stet dort U dort >stet< V \textbf{20} dâ] Do U V W (Q)  $\cdot$ brunne] brunnen W  $\cdot$ ûz einem] vz ieme U vz ginem V auß dem W (Q) vsserm R  $\cdot$ velse] felsen R \textbf{21} dir] \textit{om.} U V W Q R \textbf{22} lîhte] \textit{om.} Q  $\cdot$ sô] ze V (W) R \textbf{23} dûne] Duͦ U (V) (Q) (R)  $\cdot$ si schiere] schiere U schiere sv́ V \textbf{24} dâne] Do in U (V) (W) (Q) Do R  $\cdot$ dô] \textit{om.} U W \textbf{26} ûf] fúr W  $\cdot$ die] den R \textbf{27} Kundrien] Kuͦndrien U Gen kundrien Q Kundryen R \textbf{28} im] in Q  $\cdot$ undervienc] wider vieng W \textbf{29} erkorn] [v]: erkorn R \textbf{30} der] ein Q \newline
\end{minipage}
\end{table}
\end{document}
