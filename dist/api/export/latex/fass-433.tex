\documentclass[8pt,a4paper,notitlepage]{article}
\usepackage{fullpage}
\usepackage{ulem}
\usepackage{xltxtra}
\usepackage{datetime}
\renewcommand{\dateseparator}{.}
\dmyyyydate
\usepackage{fancyhdr}
\usepackage{ifthen}
\pagestyle{fancy}
\fancyhf{}
\renewcommand{\headrulewidth}{0pt}
\fancyfoot[L]{\ifthenelse{\value{page}=1}{\today, \currenttime{} Uhr}{}}
\begin{document}
\begin{table}[ht]
\begin{minipage}[t]{0.5\linewidth}
\small
\begin{center}*D
\end{center}
\begin{tabular}{rl}
\textbf{433} & "\begin{large}T\end{large}uot ûf!" "wem? wer sît ir?"\\ 
 & "ich wil inz herze \textbf{dîn} zuo dir."\\ 
 & "sô gert ir zengem rûme."\\ 
 & "waz denne, belîbe ich kûme?\\ 
5 & mîn dringen soltû selten klagen.\\ 
 & ich wil dir nû von wunder sagen."\\ 
 & "jâ, sît irz, vrou Âventiure?\\ 
 & wie vert der gehiure?\\ 
 & ich meine den \textbf{werden} Parzival,\\ 
10 & den Cundrie nâch dem Grâl\\ 
 & mit unsüezen worten jagete,\\ 
 & dâ manec vrouwe klagete,\\ 
 & daz niht wendec \textbf{wart} sîn reise.\\ 
 & von Artuse dem Berteneise\\ 
15 & huop er sich dô. wie vert er nuo?\\ 
 & den selben mæren \textbf{grîfet} zuo,\\ 
 & ob er an vreuden sî verzagt.\\ 
 & oder hât er hôhen prîs bejagt?\\ 
 & oder ob sîn ganziu werdecheit\\ 
20 & \textbf{ist} beidiu lang unde breit?\\ 
 & oder ist si kurz oder smal?\\ 
 & nû prüevet uns die selben zal,\\ 
 & waz von sînen henden sî geschehen.\\ 
 & hât er Munsalvæsche sît gesehen\\ 
25 & unt den süezen Anfortas,\\ 
 & des herze dô vil siufzec was?\\ 
 & durch iwer güete gebt uns trôst,\\ 
 & ob \textbf{der} von jâmer sî erlôst.\\ 
 & lât hœren \textbf{uns diu} mære,\\ 
30 & ob Parzival dâ wære,\\ 
\end{tabular}
\scriptsize
\line(1,0){75} \newline
D Fr31 \newline
\line(1,0){75} \newline
\textbf{1} \textit{Initiale} D  \newline
\line(1,0){75} \newline
\textbf{9} Parzival] Parzifal D \textbf{10} Cundrie] Cvndrîe D \textbf{15} dô] \textit{om.} Fr31 \textbf{16} Dem selben mâre grife zvͦ Fr31 \textbf{20} ist] Si Fr31 \textbf{24} Munsalvæsche] Mvnsalvæsce D mvntsaluasch Fr31 \textbf{26} siufzec] suͦhtic Fr31 \textbf{27} güete] [*vͤte]: gvͤte D gvͦetie Fr31  $\cdot$ uns] mir Fr31 \textbf{28} der] er Fr31 \textbf{30} Parzival] Parcifal D (Fr31) \newline
\end{minipage}
\hspace{0.5cm}
\begin{minipage}[t]{0.5\linewidth}
\small
\begin{center}*m
\end{center}
\begin{tabular}{rl}
 & "\textit{\begin{large}T\end{large}}uot ûf!" "wem? wer sît ir?"\\ 
 & "ich wil in\textit{z} her\textit{z}e \textbf{dîn} ze dir."\\ 
 & "sô gert ir ze engem rûme."\\ 
 & "waz danne, belîbe ich kûme?\\ 
5 & mîn dringen soltû selten klagen.\\ 
 & ich wil dir nû von wunder sagen."\\ 
 & "jâ, sît irz, vrouwe Âve\textit{n}tiure?\\ 
 & wie vert der gehiure?\\ 
 & ich meine de\textit{n} \textbf{werden} Parcifal,\\ 
10 & den Condrie nâch dem Grâl\\ 
 & mit unsüezen worten jagete,\\ 
 & d\textit{â} manigiu vrouwe klagete,\\ 
 & daz niht wendic \textbf{was} sîn reise.\\ 
 & von Artuse dem Brituneise\\ 
15 & huop er sich dô. wie vert er nû?\\ 
 & den selben mæren \textbf{grîfet} zuo,\\ 
 & ob er an vrœden sî verzaget.\\ 
 & oder hât er hôhen prîs bejaget?\\ 
 & oder ob sîn ganziu werdicheit\\ 
20 & \textbf{sî} beidiu lanc und breit?\\ 
 & oder ist si kurz oder smal?\\ 
 & nû prüevet un\textit{s} die selben zal,\\ 
 & waz von sînen henden sî ges\textit{cheh}en.\\ 
 & hât er Mun\textit{t}salvasche sider gesehen\\ 
25 & und den süezen Anfortas,\\ 
 & des herze dô vil siufzic was?\\ 
 & durch iuwere güete gebet uns trôst,\\ 
 & ob \textbf{er} von jâmer sî erlôst.\\ 
 & lât hœren \textbf{uns diu} mære,\\ 
30 & ob Parcifal dâ wære,\\ 
\end{tabular}
\scriptsize
\line(1,0){75} \newline
m n o \newline
\line(1,0){75} \newline
\textbf{1} \textit{Illustration mit Überschrift:} Wie parcifal Sẏgunen in der clusen vant m  Also gawan zuͯ sigunen kam vor ein cluse vnd er sú frogete vmb parcifalen (parcifaln o  ) n (o)   $\cdot$ \textit{Großinitiale} m n o  \newline
\line(1,0){75} \newline
\textbf{1} Tuot] Guͯt m (n) (o)  $\cdot$ wer] buer n búr o \textbf{2} inz herze dîn] inr herre din m herre min n (o) \textbf{3} engem] eẏnem o \textbf{4} kûme] kuͯne o \textbf{7} jâ] Jo n  $\cdot$ Âventiure] aueture m \textbf{9} meine] menne o  $\cdot$ den] dem m \textbf{10} Condrie] kondrie m comdri n condri o \textbf{11} worten] worte o \textbf{12} dâ] Do m n o  $\cdot$ klagete] clagete das o \textbf{13} wendic] wenig o \textbf{14} dem] gon n gan o  $\cdot$ Brituneise] brittuͯneise m britaneise o \textbf{15} dô wie] nuͯ do wie n wie o \textbf{16} grîfet] griffe n griffen o \textbf{21} oder] vnd n \textbf{22} uns] vnd m  $\cdot$ selben] selbe n o \textbf{23} geschehen] geslagen m \textbf{24} hât] Hette n  $\cdot$ Muntsalvasche] munsaluasce m muntsaluasce n mont saluasce o  $\cdot$ sider] sit n o \textbf{25} Anfortas] anfortes o \textbf{26} des] Der o  $\cdot$ dô] \textit{om.} n \textbf{27} iuwere] yere m ir n o \textbf{28} er] der n o  $\cdot$ sî] ist n \textbf{30} dâ] do n o \newline
\end{minipage}
\end{table}
\newpage
\begin{table}[ht]
\begin{minipage}[t]{0.5\linewidth}
\small
\begin{center}*G
\end{center}
\begin{tabular}{rl}
 & "\begin{large}T\end{large}uot ûf!" "wem? wer sît ir?"\\ 
 & "ich wil inz herze \textbf{dîn} zuo dir."\\ 
 & "sô gert ir zengem rûme."\\ 
 & "waz dane, belîbe ich kûme?\\ 
5 & mîn dringen soltû selten klagen.\\ 
 & ich wil dir nû von wunder sagen."\\ 
 & "jâ, sît irz, vrou Âventiure?\\ 
 & wie vert der gehiure?\\ 
 & ich meine den \textbf{werden} Parzival,\\ 
10 & den Gundrie nâch dem Grâl\\ 
 & mit unsüezen worten jagte,\\ 
 & dâ manic vrouwe klagte,\\ 
 & daz niht wendic \textbf{wart} sîn reise.\\ 
 & von Artuse dem Britaneise\\ 
15 & huop er sich dô. wie vert er nû?\\ 
 & den selben mæren \textbf{grîfet} zuo,\\ 
 & ob er an vröuden sî verzaget.\\ 
 & oder hât er hôhen brîs bejaget?\\ 
 & \textit{oder} obe sîn ganziu werdicheit\\ 
20 & \textbf{sî} beidiu lanc unde breit?\\ 
 & oder ist si kurz oder smal?\\ 
 & nû prüevet uns die selben zal,\\ 
 & waz von sînen handen sî geschehen.\\ 
 & hât er Muntsalvatsche sît gesehen\\ 
25 & unt den süezen Anfortas,\\ 
 & des herze dô \textit{vil} \textit{siufz}ic was?\\ 
 & durch iwer güete gebet uns trôst,\\ 
 & op \textbf{der} von \textit{jâm}er sî erlôst.\\ 
 & lât hœren \textbf{uns diu} mære,\\ 
30 & obe Parzival dâ wære,\\ 
\end{tabular}
\scriptsize
\line(1,0){75} \newline
G I O L M Z \newline
\line(1,0){75} \newline
\textbf{1} \textit{Überschrift:} Van Parzifal auentiwer wie der svͤz vnde gehiwer zvͦ Trevirscent dem ainsidel cherte der in von got beweist vnd lerit hie gent Parzifals auentiwer wider an I   $\cdot$ \textit{Initiale} G I O L Z  \textbf{17} \textit{Initiale} I  \newline
\line(1,0){75} \newline
\textbf{1} Tuot] [Guͤt]: Tuͤt I ÷vͦt O \textbf{2} dîn] hin L \textbf{3} zengem] zeuͯgen L zcu gein M \textbf{4} ich] \textit{om.} M \textbf{5} dringen soltû selten] dingen soltu selte L \textbf{6} wil] \textit{om.} M \textbf{8} der] die Z \textbf{9} Parzival] parzifal I L M Barceval O parcifal Z \textbf{10} den] dem I  $\cdot$ Gundrie] Gvndrîe O kvndrie L (M) Cvndrie Z \textbf{12} dâ] daz I (O) \textbf{13} niht wendic wart] niht wart wendich L gewendit wart M \textbf{14} Artuse] Artus I (Z) Artuͯse L  $\cdot$ Britaneise] pritoneise I Brittoneise L brituneise Z \textbf{15} dô] \textit{om.} I da M \textbf{16} den selben] Von salden L  $\cdot$ grîfet] grifen L \textbf{18} oder] olde G \textbf{19} oder] vnt G \textbf{20} beidiu] \textit{om.} O \textbf{21} Adir korcz adir lanc adir smal M  $\cdot$ oder ist] olde ist G  $\cdot$ oder smal] olde smal G vnd smal L \textbf{24} Muntsalvatsche] montshaluasce I mvntsalvasche O Muntschalvatsche M Montsalvatsch Z  $\cdot$ sît] \textit{om.} I O L M \textbf{25} Anfortas] Amfortas L \textbf{26} dô] da M Z  $\cdot$ vil siufzic] so trurch G vil suffic I vil flvhtech O \textbf{28} der] er I O L  $\cdot$ jâmer] chumber G \textbf{29} uns] \textit{om.} I  $\cdot$ diu] diese L \textbf{30} Parzival] parzifal I L M Barcifal O parcifal Z  $\cdot$ dâ] der I \newline
\end{minipage}
\hspace{0.5cm}
\begin{minipage}[t]{0.5\linewidth}
\small
\begin{center}*T
\end{center}
\begin{tabular}{rl}
 & "\begin{Large}T\end{Large}uot ûf!" "wem? wer sît ir?"\\ 
 & "Ich wil in daz herze \textbf{hin} ze dir."\\ 
 & "Sô gert ir ze engem rûme."\\ 
 & "waz danne, blîbich kûme?\\ 
5 & mîn dringen soltû selten klagen.\\ 
 & ich wil dir nû von wundere sagen."\\ 
 & "Jâ, sît irz, vrou Âventiure?\\ 
 & wie vert der gehiure?\\ 
 & ich meine den \textbf{jungen} Parcifal,\\ 
10 & den Kundrie nâch dem Grâl\\ 
 & mit unsüezen worten jagete,\\ 
 & dâ manec vrouwe klagete,\\ 
 & daz niht wendic \textbf{wart} sîn reise.\\ 
 & von Artuse dem Brituneise\\ 
15 & huop er sich dô. wie vert er nû?\\ 
 & den selben mæren \textbf{grîfen} zuo,\\ 
 & ob er an vröuden sî verzaget.\\ 
 & oder hât er hôhen prîs bejaget?\\ 
 & oder ob sîn ganz\textit{iu} werdecheit\\ 
20 & \textbf{sî} beidiu lanc unde breit?\\ 
 & oder ist si kurz oder smal?\\ 
 & nû prüevet uns die selben zal,\\ 
 & waz von sînen handen sî geschehen.\\ 
 & hât er Munsalvasche sît gesehen\\ 
25 & unde den süezen Anfortas,\\ 
 & des herze dô vil siuftic was?\\ 
 & durch iuwer güete gebt uns trôst,\\ 
 & ob \textbf{der} von jâmer sî erlôst.\\ 
 & lât hœren \textbf{liebiu} mære,\\ 
30 & ob Parcifal  dâ wære,\\ 
\end{tabular}
\scriptsize
\line(1,0){75} \newline
T U V W Q R \newline
\line(1,0){75} \newline
\textbf{1} \textit{Überschrift:} Awentewr wy partzifal quam in den wald do her die klusenerin Q   $\cdot$ \textit{Großinitiale} T U Q R   $\cdot$ \textit{Initiale} V W  \textbf{2} \textit{Majuskel} T  \textbf{3} \textit{Majuskel} T  \textbf{7} \textit{Majuskel} T  \newline
\line(1,0){75} \newline
\textbf{1} wem] wenn W  $\cdot$ wer] wir U \textbf{2} wil] wis R  $\cdot$ herze] her zu R  $\cdot$ hin] din V (Q) dem R \textbf{3} ir] her Q  $\cdot$ engem] einem R  $\cdot$ rûme] roúmen Q \textbf{5} soltû] solte U \textbf{6} nû von wundere] von wuͦnder nuͦ U \textbf{7} irz] ir U \textbf{8} der gehiure] ir so vngehúrre R \textbf{9} jungen] werden U V W (Q) R  $\cdot$ Parcifal] Parzifal T (V) (R) ) partzifal (W) (Q) \textbf{10} den] Dem Q Der R  $\cdot$ Kundrie] kvndrye T kuͦndrie U \textbf{11} worten] worte R \textbf{12} dâ] Do V W R Das Q \textbf{13} sîn] eyn Q \textbf{14} Artuse] Artuͦse U artus W R  $\cdot$ dem] vnd mengen R  $\cdot$ Brituneise] Brituͦneise U [britunes*]: brituneise V briteneise Q Britoneise R \textbf{15} huop er] Vber W  $\cdot$ dô] \textit{om.} Q R \textbf{16} grîfen] grifent V (W) \textbf{17} verzaget] veriaget W \textbf{18} oder ober hochen pris hab beiagt R \textbf{19} ganziu] ganze T \textbf{20} sî beidiu] Jst beide V Sige R  $\cdot$ unde] oder Q \textbf{21} si] \textit{om.} U [si]: sú V  $\cdot$ oder smal] vnde smal V vnd zu smael Q \textbf{23} sînen] sinnen T  $\cdot$ sî] ist R \textbf{24} Munsalvasche] Mvnsalvasce T Muͦntsalvatsche U muntsaluasche V montsaluatz W múntsalualsche Q Munsaluasche R \textbf{25} Anfortas] anfrotas R \textbf{26} des herze] Der herre Q Des herczen R  $\cdot$ vil siuftic] vol seúfftzens W vil sᵫnffczent R  $\cdot$ was] saz Q \textbf{27} güete] gut Q \textbf{28} ob] Oder U  $\cdot$ der] er W R \textbf{29} lât hœren] Hant úch R \textbf{30} Parcifal] Parzifal T V partzifal W Q parczifal R  $\cdot$ dâ] do U V W Q \newline
\end{minipage}
\end{table}
\end{document}
