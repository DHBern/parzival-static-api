\documentclass[8pt,a4paper,notitlepage]{article}
\usepackage{fullpage}
\usepackage{ulem}
\usepackage{xltxtra}
\usepackage{datetime}
\renewcommand{\dateseparator}{.}
\dmyyyydate
\usepackage{fancyhdr}
\usepackage{ifthen}
\pagestyle{fancy}
\fancyhf{}
\renewcommand{\headrulewidth}{0pt}
\fancyfoot[L]{\ifthenelse{\value{page}=1}{\today, \currenttime{} Uhr}{}}
\begin{document}
\begin{table}[ht]
\begin{minipage}[t]{0.5\linewidth}
\small
\begin{center}*D
\end{center}
\begin{tabular}{rl}
\textbf{102} & daz was ein stolz werder man -\\ 
 & niht der von Rome entran\\ 
 & Julius dâ \textbf{bevor}.\\ 
 & de\textit{r} künec Nabuchodonozor\\ 
5 & sîner muoter bruoder was,\\ 
 & der an trügelîchen buochen las,\\ 
 & er \textbf{solte} selbe sîn ein got.\\ 
 & \textbf{daz} wære nû der \textbf{liute} spot.\\ 
 & \textbf{ir lîp}, \textbf{ir} guot was ungespart.\\ 
10 & die \textbf{gebruoder} wâren von hôher art,\\ 
 & von \textbf{Ninus}, der gewaldes pflac,\\ 
 & ê \textbf{würde gestiftet} Baldac.\\ 
 & der selbe stifte ouch Ninnive.\\ 
 & in tet schade unt laster wê.\\ 
15 & der jach der \textbf{bâruc} zurborn.\\ 
 & des \textbf{wart} gewunnen unt verlorn\\ 
 & genuoc ze bêden sîten.\\ 
 & man sach \textbf{dâ} helde strîten.\\ 
 & Dô schift er sich über mer\\ 
20 & unt vant den bâruc mit \textbf{wer}.\\ 
 & mit vreuden er enpfangen wart,\\ 
 & swie mich \textbf{jâmer} \textbf{sîner} vart.\\ 
 & \textbf{\begin{large}W\end{large}az} dâ geschæhe, wie ez dort ergê,\\ 
 & gewin unt vlust, wie daz \textbf{gestê},\\ 
25 & \textbf{den} \textbf{en}weiz vrou Herzeloyde niht.\\ 
 & \textbf{diu} was alsô diu sunne lieht\\ 
 & unt hete minneclîchen lîp.\\ 
 & rîcheit bî \textbf{tugende} pflac daz wîp\\ 
 & unt vreuden mêr denne \textbf{ze} vil.\\ 
30 & si was gar \textbf{ob dem} wunsches zil.\\ 
\end{tabular}
\scriptsize
\line(1,0){75} \newline
D \newline
\line(1,0){75} \newline
\textbf{19} \textit{Majuskel} D  \textbf{23} \textit{Initiale} D  \newline
\line(1,0){75} \newline
\textbf{4} der] de D \textbf{12} Baldac] Baldach D \textbf{13} Ninnive] Ninnivê D \newline
\end{minipage}
\hspace{0.5cm}
\begin{minipage}[t]{0.5\linewidth}
\small
\begin{center}*m
\end{center}
\begin{tabular}{rl}
 & daz was ein stolz werder man -\\ 
 & niht der von Rome entran\\ 
 & Julusse d\textit{â} \textbf{bevor}.\\ 
 & der künic Nabuchodonosor\\ 
5 & sîner muoter bruoder was,\\ 
 & der an trügelîchen buochen las,\\ 
 & er \textbf{solte} selber sîn ein got.\\ 
 & \textbf{daz} wære nû der \textbf{liute} spot.\\ 
 & \textbf{ir lîp}, \textbf{ir} guot was ungespart.\\ 
10 & die \textbf{gebruoder} wâren von hôher art,\\ 
 & von \textbf{Nynus}, der gewaldes pflac,\\ 
 & ê \textbf{würde gestiftet} Baldac.\\ 
 & der selbe stiftete ouch \textit{N}inive.\\ 
 & in tet schade und laster wê.\\ 
15 & der jach der \textbf{künic} zurborn.\\ 
 & des \textbf{was} gewunnen und verlorn\\ 
 & genuoc ze beiden sîten.\\ 
 & man sach \textbf{d\textit{â}} helde strîten.\\ 
 & dô schiffete er sich über mer\\ 
20 & und vant den bâr\textit{uc} mit \textbf{wer}.\\ 
 & mit vröuden er enpfangen wart,\\ 
 & wie mich \textbf{jâmert} \textbf{sîner} vart.\\ 
 & \textbf{\begin{large}D\end{large}az} d\textit{â} geschæhe, wie ez dort ergê,\\ 
 & gewin und verlust, wie daz \textbf{geschê},\\ 
25 & \textbf{des} \textbf{en}weiz vrouwe Herczeloid\textit{e} niht.\\ 
 & \textbf{diu} was alsô diu sunne lieht\\ 
 & und hete minneclîchen lîp.\\ 
 & rîcheit bî \textbf{jugende} pflac daz wîp\\ 
 & und vröuden mêr danne \textbf{ze} vil.\\ 
30 & si was gar \textbf{ob dem} wunsches zil.\\ 
\end{tabular}
\scriptsize
\line(1,0){75} \newline
m n o \newline
\line(1,0){75} \newline
\textbf{23} \textit{Initiale} m   $\cdot$ \textit{Capitulumzeichen} n  \newline
\line(1,0){75} \newline
\textbf{3} Julusse] Jvͯluͯsse m Julúse n o  $\cdot$ dâ] do m n o  $\cdot$ bevor] bẏ vor o \textbf{4} Nabuchodonosor] [nabuchodonoser]: nabuchodonosor n nobuͯche donẏsor o \textbf{6} trügelîchen] trú welichen o  $\cdot$ las] [was]: las o \textbf{8} nû] jme n (o) \textbf{9} was] wart o \textbf{11} Nynus] nÿnuͯs m mẏnes o \textbf{12} gestiftet] gestifftet do o  $\cdot$ Baldac] baldack m baldag n o \textbf{13} stiftete] stifftet n o  $\cdot$ Ninive] mjniwe m mẏnywe n mẏnne we o \textbf{16} des was] Das wart n o \textbf{17} beiden] beder o \textbf{18} dâ] do m n o \textbf{19} schiffete] schifft n o \textbf{20} bâruc] barne m \textbf{22} sîner] sin n (o) \textbf{23} Daz dâ] Das do m n Do das o  $\cdot$ ergê] er gie n (o) \textbf{24} gewin] Gewinne n  $\cdot$ geschê] geschie n [gesche]: geschie o \textbf{25} enweiz] enweis ich o  $\cdot$ Herczeloide] herczeloidez m [b*]: hertzeloid n herczeleide o \textbf{30} si] So o \newline
\end{minipage}
\end{table}
\newpage
\begin{table}[ht]
\begin{minipage}[t]{0.5\linewidth}
\small
\begin{center}*G
\end{center}
\begin{tabular}{rl}
 & daz was ein stolz werder man -\\ 
 & niht der von Rome entran\\ 
 & Iulius dâ \textbf{bevor}.\\ 
 & der künic Nabchodonosor\\ 
5 & sîner muoter bruoder was,\\ 
 & der an trügelîchen buochen las,\\ 
 & er \textbf{wolt} selbe sîn ein got.\\ 
 & \textbf{ez} wære nû der \textbf{liute} spot.\\ 
 & \textbf{ir lîp}, \textbf{ir} guot was ungespart.\\ 
10 & die \textbf{gebruoder} wâren von hôher art,\\ 
 & von \textbf{Linus}, der gewaltes pflac,\\ 
 & ê \textbf{gestiftet würde} Baldac.\\ 
 & der selbe stift ouch Ninve.\\ 
 & in tet schade und laster wê.\\ 
15 & der jach der \textbf{bâruc} ze urboren.\\ 
 & des \textbf{wart} gewunnen und verloren\\ 
 & \begin{large}G\end{large}enuoc ze beiden sîten.\\ 
 & man sach \textbf{dâ} helde strîten.\\ 
 & dô schifte er sich über mer\\ 
20 & und vant den bâruc mit \textbf{wer}.\\ 
 & mit vröuden er enpfangen wart,\\ 
 & swie mich \textbf{jâmer} \textbf{sîner} vart.\\ 
 & \textbf{waz} dâ geschæhe, wiez dort ergê,\\ 
 & gewin und vlust, wie daz \textbf{gestê},\\ 
25 & \textbf{des} weiz vrô Herzeloide niht.\\ 
 & \textbf{diu} was als diu sunne lieht\\ 
 & unde het minniclîchen lîp.\\ 
 & rîcheit bî \textbf{jugent} pflac daz wîp\\ 
 & unde vröuden mêre dane vil.\\ 
30 & si was gar \textbf{obe dem} wunsches zil.\\ 
\end{tabular}
\scriptsize
\line(1,0){75} \newline
G I O L M Q R Z Fr21 Fr36 Fr48 \newline
\line(1,0){75} \newline
\textbf{1} \textit{Initiale} O  \textbf{17} \textit{Initiale} G  \textbf{23} \textit{Überschrift:} Auentiwer wie Gahmuret den leib verlos ze dem barrvch I   $\cdot$ \textit{Initiale} I L R Z Fr21  \newline
\line(1,0){75} \newline
\textbf{1} daz] ÷az O Der L  $\cdot$ stolz werder] stolze werde M \textbf{2} Rome] rom I (O) Z \textbf{3} Iulius] Jvlivs O (M) (Q) (R) (Z) (Fr48) Jvlio L Jvlẏvs Fr21  $\cdot$ bevor] Belior R \textbf{4} künic] \textit{om.} M  $\cdot$ Nabchodonosor] nabochodonosor I nabvchodonosor O (M) (Q) (Fr21) (Fr48) Nabuͯchodonosor L nabachadonosor R nabuchodonsor Z \textbf{6} buochen] buͦche R \textbf{7} wolt] solde Z  $\cdot$ selbe] selben M selber Q (R) \textbf{8} ez wære] Er was M  $\cdot$ liute] werlde O  $\cdot$ spot] gespot Q \textbf{10} gebruoder] brvͦder O \textbf{11} von] \textit{om.} I Vo Z  $\cdot$ Linus] Nino L livns M nẏnűs Q Nynus R Ninus Z  $\cdot$ der] \textit{om.} I der des L  $\cdot$ gewaltes] waltes O \textbf{12} gestiftet würde] wurde gestiftet der O wuͯrde gestiftet L (M) (Q) (R) (Z) (Fr21) (Fr36)  $\cdot$ Baldac] baldach G (O) (L) baldak Fr36 \textbf{13} selbe] \textit{om.} Fr21  $\cdot$ stift] stifte I O L (M) Fr21  $\cdot$ Ninve] [ni*]: niniue I Ninnive O Fr21 Nẏnive L Nÿnive M niniúe Q Nẏnenue R ninive Z Fr36 \textbf{14} in] dem I Jm O (Fr36)  $\cdot$ tet] tvͦt Fr36  $\cdot$ schade und laster] laster vnde scade M \textbf{15} jach der] iach den O  $\cdot$ bâruc] brauc Q  $\cdot$ ze urboren] zurbor I \textbf{16} des] Das M  $\cdot$ gewunnen] gewúnden Q \textbf{18} dâ] die O L M Q R Z Fr21  $\cdot$ strîten] [riten]: striten M \textbf{19} dô] Da O M Z  $\cdot$ schifte] sift I schiffet O (L) Q R Z (Fr21)  $\cdot$ sich] \textit{om.} M \textbf{20} wer] her Q \textbf{22} swie] Wie L (M) Q R Z  $\cdot$ jâmer] iamert I L (R) Z iammerte M  $\cdot$ sîner] seine Q (Fr21) \textbf{23} waz] Vaz I Swaz O (Fr21)  $\cdot$ dâ] dort Q do R  $\cdot$ geschæhe] geschah Fr21  $\cdot$ ergê] ergie L Q R \textbf{24} gestê] gestie L beste M Q \textbf{25} weiz] enweiz I (L) (M) (Q) Z  $\cdot$ Herzeloide] herzelaude I (O) herczeloide M herzeloude Q (Z) herczeleide R herzloͮd Fr21 Belecane L \textbf{26} Noch hertzelauͯde waz hie geschiht L  $\cdot$ lieht] licht M Q (Fr21) \textbf{27} unde] Die L  $\cdot$ minniclîchen] wunenklichen R \textbf{28} jugent] iuͯngen da L \textbf{29} unde] \textit{om.} I An O  $\cdot$ vröuden] frovde I (Q)  $\cdot$ vil] ze vil O (L) (M) (Q) (R) (Z) (Fr21) \textbf{30} obe] uff M  $\cdot$ dem] des I (O) (Q) (R)  $\cdot$ wunsches] wusche M  $\cdot$ zil] spil L \newline
\end{minipage}
\hspace{0.5cm}
\begin{minipage}[t]{0.5\linewidth}
\small
\begin{center}*T (U)
\end{center}
\begin{tabular}{rl}
 & daz was ein stolzer werder man -\\ 
 & niht der von Rome entran\\ 
 & Julius dâ \textbf{vor}.\\ 
 & der künec Nabuchodonosor\\ 
5 & sîner muoter bruoder was,\\ 
 & der an trügelîchen büechern las,\\ 
 & er \textbf{wolte} selbe sîn ein got.\\ 
 & \textbf{ez} wære nû der \textbf{werlde} spot.\\ 
 & \textbf{sîn habe}, \textbf{sîn} guot was ungespart.\\ 
10 & die \textbf{bruoder} wâren von hôher art,\\ 
 & von \textbf{Linus}, der gewalte pflac,\\ 
 & ê \textbf{würde gestiftet} Baldac.\\ 
 & der selbe stifte ouch Ninive.\\ 
 & in tet schade und laste\textit{r} wê.\\ 
15 & der jach der \textbf{bâruc} zuo urborn.\\ 
 & des \textbf{wart} gewunnen und verlorn\\ 
 & genuoc zuo beiden sîten.\\ 
 & man sach \textbf{die} helde strîten.\\ 
 & dô schifte er sich über mer\\ 
20 & und vant den bâruc mit \textbf{her}.\\ 
 & mit vreuden er entvangen wart,\\ 
 & wie mich \textbf{jâmere} \textbf{sîne} vart.\\ 
 & \textbf{\begin{large}W\end{large}az} d\textit{â} geschæhe, wie ez dort ergê,\\ 
 & gewin und verlust, wie daz \textbf{gestê},\\ 
25 & \textbf{des} weiz vrou Herzeloyde niht.\\ 
 & \textbf{si} was als diu sunne lieht\\ 
 & und hete minniclîchen lîp.\\ 
 & rîcheit bî \textbf{der} \textbf{jugent} pflac daz wîp\\ 
 & und vreuden mê dan \textbf{zuo} vil.\\ 
30 & si was gar \textbf{über des} wunsches zil.\\ 
\end{tabular}
\scriptsize
\line(1,0){75} \newline
U V W T \newline
\line(1,0){75} \newline
\textbf{11} \textit{Majuskel} T  \textbf{19} \textit{Majuskel} T  \textbf{23} \textit{Initiale} U V W   $\cdot$ \textit{Majuskel} T  \newline
\line(1,0){75} \newline
\textbf{1} stolzer werder] werder stoltzer V stoltzer W werder T \textbf{3} Julius] Juͦlius U [*]: Julius V Der grosse iulius W Jvlio T  $\cdot$ vor] [*]: belior V \textbf{4} Nabuchodonosor] Nabuͦchodonosor U \textbf{6} trügelîchen] valschen T \textbf{7} selbe] selber V W \textbf{8} ez] Er W  $\cdot$ werlde] livte T \textbf{9} sîn habe sîn] Jr [*spart]: lip ir V ir lip ir T \textbf{11} von] \textit{om.} V  $\cdot$ Linus] linuͦs U [J*]: Jvnus V ninus W  $\cdot$ gewalte] [*]: gewaltes V gewaltes W T \textbf{12} würde gestiftet] dan wúrde gestiftet V das gestifftet wurd W wurde T  $\cdot$ Baldac] Baldag V (W) \textbf{13} stifte ouch] oͮch stiftet V stifte T  $\cdot$ Ninive] Nẏniue V niniue W \textbf{14} in] im T  $\cdot$ laster] laste U \textbf{15} der jach] Den iach V \textbf{19} schifte er] schiffet er V schifften sy W  $\cdot$ sich] \textit{om.} W \textbf{20} und] Er W  $\cdot$ her] gewer W wer T \textbf{22} wie] swie V T  $\cdot$ mich] mit W  $\cdot$ sîne] siner T \textbf{23} dâ] do U V W  $\cdot$ wie ez dort] was do W \textbf{24} und] \textit{om.} T  $\cdot$ daz gestê] es do ste W \textbf{25} des] Das W desen T  $\cdot$ Herzeloyde] Herzeleide U hertzelaude V hertzloyde W \textbf{26} si] div T  $\cdot$ diu] die T  $\cdot$ lieht] licht W \textbf{27} minniclîchen] einen minnenclich V einen minniglichen W \textbf{28} bî der jugent] bi iugent V (T) vnd tugende W \textbf{30} über] obe V (W) (T)  $\cdot$ des wunsches] dem wunsche W dem wunsc ein T \newline
\end{minipage}
\end{table}
\end{document}
