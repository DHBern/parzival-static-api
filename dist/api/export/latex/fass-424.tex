\documentclass[8pt,a4paper,notitlepage]{article}
\usepackage{fullpage}
\usepackage{ulem}
\usepackage{xltxtra}
\usepackage{datetime}
\renewcommand{\dateseparator}{.}
\dmyyyydate
\usepackage{fancyhdr}
\usepackage{ifthen}
\pagestyle{fancy}
\fancyhf{}
\renewcommand{\headrulewidth}{0pt}
\fancyfoot[L]{\ifthenelse{\value{page}=1}{\today, \currenttime{} Uhr}{}}
\begin{document}
\begin{table}[ht]
\begin{minipage}[t]{0.5\linewidth}
\small
\begin{center}*D
\end{center}
\begin{tabular}{rl}
\textbf{424} & ez wâren meide \textbf{als} von der zît,\\ 
 & den man diu besten jâr \textbf{noch} gît.\\ 
 & ich bin des \textbf{unerværet},\\ 
 & heten si geschæret\\ 
5 & als ein valke sîn gevider.\\ 
 & dâ rede ich niht wider.\\ 
 & Nû hœret, \textbf{wie} sich der rât geschiet,\\ 
 & waz man des landes künege riet.\\ 
 & die wîsen het er zim genomen.\\ 
10 & an sînen rât die wâren komen.\\ 
 & etslîcher sînen willen sprach,\\ 
 & als im sîn bester sin verjach.\\ 
 & Dô mâzen siz an manege stat.\\ 
 & der künec sîne rede \textbf{ouch} hœren bat.\\ 
15 & er sprach: "ez wart mit mir gestriten.\\ 
 & ich kom durch âventiwer geriten\\ 
 & inz fôrest Læhtamris.\\ 
 & \textbf{ein} ritter al ze hôhen prîs\\ 
 & in dirre wochen an mir \textbf{sach},\\ 
20 & wander mich vlüglingen stach\\ 
 & hinderz ors \textbf{al} sunder twâl.\\ 
 & er twanc mich des, daz ich den Grâl\\ 
 & gelobte im \textbf{ze}rwerben.\\ 
 & solde ich nû drumbe \textbf{ersterben},\\ 
25 & sô \textbf{muoz} ich leisten sicherheit,\\ 
 & die sîn hant an mir erstreit.\\ 
 & dâ râtet umbe, des ist nôt.\\ 
 & mîn bester schilt was vür den tôt,\\ 
 & daz ich dâr umbe bôt mîne hant,\\ 
30 & als iu mit rede ist \textbf{hie} bekant.\\ 
\end{tabular}
\scriptsize
\line(1,0){75} \newline
D Fr1 Fr5 \newline
\line(1,0){75} \newline
\textbf{1} \textit{Initiale} Fr5  \textbf{7} \textit{Initiale} Fr1   $\cdot$ \textit{Capitulumzeichen} Fr5   $\cdot$ \textit{Majuskel} D  \textbf{13} \textit{Majuskel} D  \newline
\line(1,0){75} \newline
\textbf{2} den man] \textit{om.} Fr5 \textbf{6} dâ] dane Fr1 \textbf{7} wie] e Fr1 Fr5 \textbf{8} künege] herrn Fr1 \textbf{10} komen] \sout{ime} komin Fr5 \textbf{13} mâzen] rieten Fr1 \textbf{14} sîne rede ouch] ouch sin rede Fr5 \textbf{17} inz] zem Fr1  $\cdot$ Læhtamris] Læhtamrîs D Lechtaniris Fr5 \textbf{22} er] vnt Fr1 \textbf{24} ersterben] sterbin Fr5 \textbf{25} leisten] [deisten]: leisten Fr1 \textbf{28} \textit{Vers 424.28 fehlt} Fr5  \textbf{29} daz] Do Fr5  $\cdot$ dâr umbe] [da fvr]: dar vmbe Fr1 \textbf{30} ist hie] hie ist Fr5 \newline
\end{minipage}
\hspace{0.5cm}
\begin{minipage}[t]{0.5\linewidth}
\small
\begin{center}*m
\end{center}
\begin{tabular}{rl}
 & ez wâren megde \textbf{als} von der zît,\\ 
 & den man diu besten jâr \textbf{n\textit{o}ch} gît.\\ 
 & ich bin des \textbf{unverværet},\\ 
 & hetten si geschæret\\ 
5 & als ein valke sîn gevidere.\\ 
 & dô rede ich niht widere.\\ 
 & \begin{large}N\end{large}û hœret, \textbf{ê} sich der rât geschiet,\\ 
 & waz man des landes künige riet.\\ 
 & die wîsen hette er zuo im genomen.\\ 
10 & an sînen rât die wâren k\textit{o}men.\\ 
 & etslîcher sînen willen sprach,\\ 
 & als ime sîn bester si\textit{n} \textit{v}erjach.\\ 
 & dô mâzen \dag sich\dag  an manige stat.\\ 
 & der künic sîne rede hœren bat.\\ 
15 & er sprach: "ez wart mit \textbf{im und} mir gestriten.\\ 
 & ich kam durch âventiure geriten\\ 
 & \dag mi\dag  fôrest Lethamris.\\ 
 & \textbf{ein} ritter alze hôhe\textit{n} prîs\\ 
 & in dirre wochen an mir \textbf{sach},\\ 
20 & wand er mich vlügelin\textit{gen} stach\\ 
 & hinder daz ros \textbf{al} sunder twâl.\\ 
 & er twanc \textit{mich} des, daz ich den Grâl\\ 
 & \textit{ge}l\textit{o}bete ime \textbf{ze} erwerben.\\ 
 & solt ich nû dâr umbe \textbf{sterben},\\ 
25 & sô \textbf{muoz} ich leisten sicherheit,\\ 
 & die sîn hant an mir erstreit.\\ 
 & dâ râtet umb, des ist nôt.\\ 
 & mîn bester schilt was vür den tôt,\\ 
 & daz ich dâr umbe bôt mîne hant,\\ 
30 & als iu mit rede ist \textbf{hie} bekant.\\ 
\end{tabular}
\scriptsize
\line(1,0){75} \newline
m n o \newline
\line(1,0){75} \newline
\textbf{7} \textit{Illustration mit Überschrift:} Also die landes herren mit dem kv́nige zuͯ rote gingent vmb einen kampff n   $\cdot$ \textit{Initiale} m n  \newline
\line(1,0){75} \newline
\textbf{1} ez] Also n  $\cdot$ der] der selben n \textbf{2} noch] nach m o  $\cdot$ gît] giht o \textbf{3} unverværet] vnerferet n (o) \textbf{7} \textit{Die Verse 424.7-30 fehlen} o  \textbf{10} komen] kamen m \textbf{12} bester sin] bester sin bester sin m \textbf{13} dô] Sú n  $\cdot$ sich] sichs n \textbf{14} sîne] sin n \textbf{15} im und] \textit{om.} n \textbf{17} mi fôrest] Zú forderst n  $\cdot$ Lethamris] lachamris n \textbf{18} hôhen] hohem m \textbf{20} wand] Wenne n  $\cdot$ vlügelingen] flugelin m \textbf{21} hinder] Húnder n  $\cdot$ al] alle n \textbf{22} mich] \textit{om.} m \textbf{23} gelobete] Erloubette m [Gelob*]: Gelobet n \textbf{27} dâ] Do n  $\cdot$ umb] vs dar vmb n  $\cdot$ des] das n \textbf{29} mîne] min n \textbf{30} mit] min n \newline
\end{minipage}
\end{table}
\newpage
\begin{table}[ht]
\begin{minipage}[t]{0.5\linewidth}
\small
\begin{center}*G
\end{center}
\begin{tabular}{rl}
 & ez wâren meide von der zît,\\ 
 & den man diu besten jâr \textbf{dâ} gît.\\ 
 & ich bin des \textbf{unerværet},\\ 
 & heten si geschæret\\ 
5 & als ein valke sîn gevidere.\\ 
 & dâ rede ich niht widere.\\ 
 & nû hœrt, \textbf{ê} sich der rât geschiet,\\ 
 & waz man des landes künege riet.\\ 
 & die wîsen heter zim genomen.\\ 
10 & an sînen rât die wâren komen.\\ 
 & ieslîcher sînen willen sprach,\\ 
 & als im sîn bester sin verjach.\\ 
 & dô mâzen siz an manege stat.\\ 
 & der künec sîne rede \textbf{och} hœren bat.\\ 
15 & er sprach: "ez wart mit mir gestriten.\\ 
 & ich kom durch âventiure geriten\\ 
 & inz fôreis Læhtamris.\\ 
 & \textbf{einem} rîter alze hôhen brîs\\ 
 & in dirre wochen an mir \textbf{geschach},\\ 
20 & wan er mich vlügelingen stach\\ 
 & hinderz ors \textbf{al} sunder twâl.\\ 
 & er twanc mich des, daz ich den Grâl\\ 
 & gelobte im erwerben.\\ 
 & solt ich nû drumbe \textbf{sterben},\\ 
25 & sô \textbf{muose} ich leisten sicherheit,\\ 
 & die sîn hant an mir erstreit.\\ 
 & dâ râtet umbe, des ist nôt.\\ 
 & mîn bester schilt was vür den tôt,\\ 
 & daz ich dâr umbe bôt mîne hant,\\ 
30 & als iu mit rede ist \textbf{hie} bekant.\\ 
\end{tabular}
\scriptsize
\line(1,0){75} \newline
G I O L M Q R Z Fr21 \newline
\line(1,0){75} \newline
\textbf{1} \textit{Initiale} I O L Q Z   $\cdot$ \textit{Capitulumzeichen} R  \textbf{15} \textit{Initiale} I  \textbf{27} \textit{Initiale} M  \newline
\line(1,0){75} \newline
\textbf{1} ez] ÷z O  $\cdot$ meide] megden R  $\cdot$ von] als vorn I als von O (L) (M) (Q) R [r]: -*-von Z \textbf{2} den] dem I  $\cdot$ diu] den Z  $\cdot$ besten] beste Q  $\cdot$ dâ] \textit{om.} I noch O L M Q (Z) och R \textbf{3} unerværet] vnerweret Q \textbf{6} dâ] Daz L Do Q Die R  $\cdot$ rede] redt Q \textbf{8} des] dem R da des Fr21  $\cdot$ landes künege] landes chunge chunge I chvniges lande O (Z) landes koniges Q \textbf{9} zim] zuͤ zim I \textbf{10} sînen] sinem R \textbf{11} sînen] sin M \textbf{12} bester] hoster Fr21  $\cdot$ sin] wil R \textbf{13} dô] Da O M Z  $\cdot$ mâzen siz] mosses sie Q \textbf{14} der] Die M  $\cdot$ sîne] \textit{om.} O sin L R Z Fr21 auch sein Q  $\cdot$ och] \textit{om.} Q Z Fr21 \textbf{15} er] ÷R I \textbf{17} inz] Anz L  $\cdot$ Læhtamris] lantampris I Lehtamiris O zuͯ Lehtemrisz L lechtamrisz M zu lechtam preysz Q Lehtambris R Lehtamris Z (Fr21) \textbf{18} einem] [ein]: einem G Ein O L (M) Q R Z Fr21  $\cdot$ alze hôhen] [alzehoher]: alzehohen G alze hohen O (L) (M) (Q) (Z) (Fr21) mit hochem R \textbf{19} mir] mich R  $\cdot$ geschach] sach O L M Q R Z Fr21 \textbf{20} wan] wande I  $\cdot$ stach] \sout{sprach} stach Q \textbf{21} al] \textit{om.} I O L Q R Fr21 \textbf{22} des daz] das R  $\cdot$ Grâl] Grale L \textbf{23} erwerben] zerwerben I (O) (L) (R) (Fr21) zcu werbin M (Q) (Z) \textbf{24} sterben] er sterben O L (M) (Q) (Z) (Fr21) \textbf{25} muose] mvͦz I (O) (M) (Q) (R) (Z) Fr21  $\cdot$ sicherheit] die sicherheit R \textbf{27} dâ] nu I Do Q  $\cdot$ râtet] rauten R  $\cdot$ umbe] drumbe I \textbf{28} bester] bezer I \textbf{29} umbe] fúr R  $\cdot$ mîne] min I O (Q) R Z Fr21 \textbf{30} iu] \textit{om.} O  $\cdot$ mit] min I (Q) Z  $\cdot$ rede] redin M  $\cdot$ ist hie] nu ist I ist R Z \newline
\end{minipage}
\hspace{0.5cm}
\begin{minipage}[t]{0.5\linewidth}
\small
\begin{center}*T
\end{center}
\begin{tabular}{rl}
 & ez wâren megede \textbf{als} von der zît,\\ 
 & den man di\textit{u} beste\textit{n} jâr \textbf{dâ} gît.\\ 
 & ich bin des \textbf{unerværet},\\ 
 & heten si geschæret\\ 
5 & als ein valke sîn gevidere.\\ 
 & dâ redich niht widere.\\ 
 & \begin{large}N\end{large}û hœret, \textbf{ê} sich der rât geschiet,\\ 
 & waz man des landes künege riet.\\ 
 & die wîsen heter zim genomen.\\ 
10 & an sînen rât die wâren komen.\\ 
 & etslîcher sînen willen sprach,\\ 
 & alsim sîn bester sin verjach.\\ 
 & dô mâzen siz an manege stat.\\ 
 & Der künec sîn rede \textbf{ouch} hœren bat.\\ 
15 & er sprach: "ez wart mit mir gestriten.\\ 
 & ich kom durch âventiure geriten\\ 
 & in dez vôr\textit{e}ht Lehtambris.\\ 
 & \textbf{ein} rîter alze hôhen prîs\\ 
 & in dirre wochen an mir \textbf{sach},\\ 
20 & wander mich vlügelingen stach\\ 
 & hinderz ors sunder twâl.\\ 
 & er twanc mich des, daz ich den Grâl\\ 
 & gelobete im \textbf{ze}rwerben.\\ 
 & soltich nû drumbe \textbf{ersterben},\\ 
25 & sô \textbf{muoz} ich leisten sicherheit,\\ 
 & die sîn hant an mir erstreit.\\ 
 & dâ râtet umbe, des ist nôt.\\ 
 & mîn bester schilt was vür den tôt,\\ 
 & daz ich drumbe bôt mîne hant,\\ 
30 & als iu mit rede ist bekant.\\ 
\end{tabular}
\scriptsize
\line(1,0){75} \newline
T U V W \newline
\line(1,0){75} \newline
\textbf{1} \textit{Initiale} V  \textbf{7} \textit{Initiale} T U W  \textbf{14} \textit{Majuskel} T  \newline
\line(1,0){75} \newline
\textbf{1} als] al V \textbf{2} diu besten] die beste T  $\cdot$ dâ] noch U V W \textbf{6} dâ] Do U Der W \textbf{7} geschiet] schiet W \textbf{9} heter] hetten W \textbf{12} sin] wille W \textbf{13} dô] Da V \textbf{14} sîn] sine U \textbf{17} vôreht] vorheht T  $\cdot$ Lehtambris] Lehtambrîs T lethtambris W \textbf{20} wander] Wan er V \textbf{21} hinderz] Hinder U \textbf{24} nû] \textit{om.} V  $\cdot$ ersterben] sterben V W \textbf{27} dâ] Do U W \textbf{28} schilt] sinn W \textbf{30} mit] min V  $\cdot$ bekant] hie bekant U V W \newline
\end{minipage}
\end{table}
\end{document}
