\documentclass[8pt,a4paper,notitlepage]{article}
\usepackage{fullpage}
\usepackage{ulem}
\usepackage{xltxtra}
\usepackage{datetime}
\renewcommand{\dateseparator}{.}
\dmyyyydate
\usepackage{fancyhdr}
\usepackage{ifthen}
\pagestyle{fancy}
\fancyhf{}
\renewcommand{\headrulewidth}{0pt}
\fancyfoot[L]{\ifthenelse{\value{page}=1}{\today, \currenttime{} Uhr}{}}
\begin{document}
\begin{table}[ht]
\begin{minipage}[t]{0.5\linewidth}
\small
\begin{center}*D
\end{center}
\begin{tabular}{rl}
\textbf{724} & \begin{large}S\end{large}i erbeizten, die dâ komen sint.\\ 
 & des \textbf{künec} Gramoflanzes kint\\ 
 & manegiu vor \textbf{im} sprungen,\\ 
 & in\textbf{z} poulûn si \textbf{sich} drungen.\\ 
5 & die kamerære \textbf{wider strît}\\ 
 & rûmten eine strâzen wît\\ 
 & gein \textbf{der Bertenoyse} künegîn.\\ 
 & \textbf{sîn œheim} Brandelidelin\\ 
 & \textbf{vorem künege inz poulûn gienc}.\\ 
10 & \textbf{Ginover den mit kusse enpfienc}.\\ 
 & Der künec wart \textbf{ouch} enpfangen sus.\\ 
 & \textbf{Bernouten} und Affinamus\\ 
 & die künegîn man \textbf{ouch} küssen sach.\\ 
 & Artus ze Gramoflanze sprach:\\ 
15 & "ê ir \textbf{sitzens} beginnet,\\ 
 & seht, ob ir decheine minnet\\ 
 & dirre vrouwen, und küsset \textbf{sie}.\\ 
 & iu beiden sî daz erloubet hie."\\ 
 & \textbf{im sagete}, wer sîn vriwendinne was,\\ 
20 & \textbf{ein} brief, den er ze velde las.\\ 
 & ich meine, daz er ir bruoder sach,\\ 
 & \textbf{diu im} vor alder werlde jach\\ 
 & ir werden minne tougen.\\ 
 & Gramoflanzes ougen\\ 
25 & \textbf{si erkanten}, diu im minne truoc.\\ 
 & \textbf{sîn vreude hôch was} genuoc,\\ 
 & sît Artus \textbf{het erloubet} daz,\\ 
 & daz si beide ein ander âne haz\\ 
 & mit gruoze \textbf{enpfâhen tâten} kunt:\\ 
30 & er kuste Itonjen an den munt.\\ 
\end{tabular}
\scriptsize
\line(1,0){75} \newline
D \newline
\line(1,0){75} \newline
\textbf{1} \textit{Initiale} D  \textbf{11} \textit{Majuskel} D  \newline
\line(1,0){75} \newline
\textbf{2} Gramoflanzes] Gramofranzs D \textbf{12} Bernouten] Bernoͮten D \textbf{24} Gramoflanzes] Gramoflanzs D \textbf{29} tæten] taten D \textbf{30} Itonjen] Jtonien D \newline
\end{minipage}
\hspace{0.5cm}
\begin{minipage}[t]{0.5\linewidth}
\small
\begin{center}*m
\end{center}
\begin{tabular}{rl}
 & si erbeizten, die d\textit{â} komen sint.\\ 
 & des \textbf{künic} Gramolantzes kint\\ 
 & manegiu vor \textbf{in} spr\textit{u}ngen,\\ 
 & in \textbf{die} pavelûn si \textbf{sich} drungen.\\ 
5 & die kamerer \textbf{in} \textbf{wider strît}\\ 
 & \textit{r}û\textit{m}den ein strâze wît\\ 
 & gegen \textbf{der Brittuneise} künigîn.\\ 
 & \textbf{der künic} Brandelidelin\\ 
 & \textbf{vor Gramolantz \dag und\dag  daz pavelûn gienc}.\\ 
10 & \textbf{Genofer den mit kiusch enpfienc}.\\ 
 & der künic wart enpfangen sus.\\ 
 & \textbf{Bernautten} und Offinamus\\ 
 & die künigîn man \textbf{ouch} küssen sach.\\ 
 & Artus zuo Gramolantz sprach:\\ 
15 & "ê ir \textbf{sitzens} beg\textit{i}nnet,\\ 
 & seht, ob ir dekein minnet\\ 
 & diser vrowen, und küsse\textit{t} \textbf{sie}.\\ 
 & iu beiden sî daz erloubet hie."\\ 
 & \textbf{dô sagte im}, wer sîn vriundîn was,\\ 
20 & \textbf{ein} brief, den er zuo velde las.\\ 
 & ich mein, daz er ir bruoder sach,\\ 
 & \textbf{diu im} vor aller der werlte jach\\ 
 & ir werden minne tougen.\\ 
 & Gramolantzes ougen\\ 
25 & \textbf{si erkanten}, diu im minne truoc.\\ 
 & \textbf{sîn vröude was hôch} genuoc,\\ 
 & sît Artus \textbf{het erloubet} daz,\\ 
 & daz si beide ein ander âne haz\\ 
 & mit gruoze \textbf{enpfâhen tâten} kunt:\\ 
30 & er kuste Ithonien an den munt.\\ 
\end{tabular}
\scriptsize
\line(1,0){75} \newline
m n o Fr69 \newline
\line(1,0){75} \newline
\textbf{1} \textit{Initiale} Fr69  \newline
\line(1,0){75} \newline
\textbf{1} dâ] do m n o \textbf{2} künic] koniges o  $\cdot$ Gramolantzes] gramolanczes o Gramoflanzes Fr69 \textbf{3} manegiu] [Maniger]: Manige o  $\cdot$ in] [*]: an o im Fr69  $\cdot$ sprungen] springen m [*]: springen n \textbf{4} in die pavelûn] Jnz [pol]: poveln Fr69 \textbf{5} in] niv Fr69 \textbf{6} rûmden] Kunden m Frunde o \textbf{7} Brittuneise] bituneise n britoneise o  $\cdot$ künigîn] konig o \textbf{8} Brandelidelin] brandeliedelin o \textbf{9} Gramolantz] gramolancz o  $\cdot$ daz] des n  $\cdot$ gienc] do ging n \textbf{12} Bernautten] Bernonten n Bernuͯten o bernoͮten Fr69  $\cdot$ Offinamus] affmanius n auernamus o affinamurs Fr69 \textbf{14} zuo] zuͦ zuͦ o  $\cdot$ Gramolantz] gramolancz o \textbf{15} beginnet] begunnent m o \textbf{16} dekein] do keine n dekeinen o \textbf{17} küsset] kussen m o \textbf{19} vriundîn] fruͯnde o \textbf{20} den] do o \textbf{23} werden] werde n \textbf{24} Gramolantzes] Gramolanczes o \textbf{25} erkanten] erkante o \textbf{28} âne] >on< o \textbf{30} Ithonien] jthonien m n jtonien o \newline
\end{minipage}
\end{table}
\newpage
\begin{table}[ht]
\begin{minipage}[t]{0.5\linewidth}
\small
\begin{center}*G
\end{center}
\begin{tabular}{rl}
 & \begin{large}S\end{large}i erbeizten, die dâ komen sint.\\ 
 & des \textbf{künic} Gramoflanzes kint\\ 
 & manigiu vor \textbf{im} sprungen,\\ 
 & in \textbf{daz} pavelûn si drungen.\\ 
5 & die kamerære \textbf{ze bêder sît}\\ 
 & rûmden ein\textit{e} strâze wît\\ 
 & gein \textbf{britânischer} künigîn.\\ 
 & \textbf{sîn œheim} Brandelidelin\\ 
 & \textbf{hiez er vor im gên dar în}.\\ 
10 & \textbf{den kuste Schinover, diu künigîn}.\\ 
 & der künic wart \textbf{ouch} enpfangen sus.\\ 
 & \textbf{Beakurs} unde Affinamus\\ 
 & die künigîn man küssen sach.\\ 
 & Artus ze Gramoflanze sprach:\\ 
15 & "ê ir \textbf{sitzen} beginnet,\\ 
 & sehet, obe ir deheine minnet\\ 
 & dirre vrouwen, unde küsset \textbf{die}.\\ 
 & iu beiden sîz erloubet hie."\\ 
 & \textbf{im sagte}, wer sîn vriundinne was,\\ 
20 & \textbf{einen} brief, den er ze velde las.\\ 
 & ich meine, daz er ir bruoder sach,\\ 
 & \textbf{dem si} vor al der werlde jach\\ 
 & ir werden minne tougen.\\ 
 & Gramoflanzes ougen\\ 
25 & \textbf{erkanden si}, diu im minne truoc.\\ 
 & \textbf{des was sîn vröude hôch} genuoc,\\ 
 & sît Artus \textbf{erloubte} daz,\\ 
 & daz si beidiu ein ander âne haz\\ 
 & mit gruoze \textbf{enpfâhen tâten} kunt:\\ 
30 & er kuste Itonien an den munt.\\ 
\end{tabular}
\scriptsize
\line(1,0){75} \newline
G I L M Z Fr20 Fr45 \newline
\line(1,0){75} \newline
\textbf{1} \textit{Initiale} G I L Z Fr20  \textbf{19} \textit{Initiale} I  \newline
\line(1,0){75} \newline
\textbf{1} Si] Hie L ÷ie Fr20  $\cdot$ erbeizten] erbeiten L \textbf{2} künic] chunges I (L)  $\cdot$ Gramoflanzes] Gramoflanzez L gramorflanz M gramoflantzes Z Gramuͦflanzes Fr45 \textbf{3} im] yn M (Fr45) \textbf{4} in] dur Fr20 \textbf{5} ze bêder sît] zeber sit G Fr20 czu beiden siten M \textbf{6} rûmden] rvnden G (Fr20) Kuͯmit yn M  $\cdot$ eine] einen G \textbf{7} gein] Gein der I (M) (Z) (Fr45)  $\cdot$ britânischer] pritonisher I Brittanoýser L britaneisir M brituneiser Z pritaneischer Fr20 britteneẏser Fr45 \textbf{8} œheim] neven Fr20  $\cdot$ Brandelidelin] brandalidelin I Branlidelin L brandeliden Z [*]: brandelidelin Fr20 \textbf{10} Schinover] Ginover G (M) (Fr20) Ginouer I (Fr45) Gẏnover L Z \textbf{12} Beakurs] beakv̂rs G [ba]: beacurs I Beakuͯrs L Gernot M Fr45 beachvrs Fr20  $\cdot$ Affinamus] affinamvs L [affim]: affinamus Z \textbf{13} künigîn] [manigen]: konigyn M \textbf{14} Gramoflanze] gramoflanzen M gramoflantzen Z \textbf{15} \textit{Vers 724.15 fehlt (Zeile ausgespart)} Fr20  \textbf{16} deheine] icheine M keine Z \textbf{17} die] sie Fr45 \textbf{18} sîz] sy das M (Z) (Fr45)  $\cdot$ erloubet] geloͮbint Fr20 \textbf{19} sagte] sagt L Z \textbf{20} einen] ein I L M Z Fr20 Fr45 \textbf{21} er] \textit{om.} M  $\cdot$ ir bruoder sach] [si daz er]: si da sach Fr20 \textbf{22} al der] aller I (M) [al*]: alder  L [alde]: al der Fr20  $\cdot$ jach] sprach M \textbf{23} werden] werde M \textbf{24} Gramoflanzes] Gramoflantz Z gramoflanzis Fr20 \textbf{26} des] Do L (Fr45) Da M Z si Fr20 \textbf{29} enpfâhen tâten] taten enpfahen L (M) (Z) \textbf{30} Itonien] Itônîen G yconie Z \newline
\end{minipage}
\hspace{0.5cm}
\begin{minipage}[t]{0.5\linewidth}
\small
\begin{center}*T
\end{center}
\begin{tabular}{rl}
 & si erbeizten, die dâ komen sint.\\ 
 & des \textbf{küneges} Gramoflanzes kint\\ 
 & manegiu vor \textbf{in} sprungen,\\ 
 & in \textbf{daz} pavelûn si drungen.\\ 
5 & die kamerære \textbf{in} \textbf{wider strît}\\ 
 & rûmten eine strâze wît\\ 
 & gein \textbf{der Brituneiser} künegîn.\\ 
 & \textbf{sîn œheim} Brandelidelin\\ 
 & \textbf{vorme künege in daz pavelûn gienc}.\\ 
10 & \textbf{Gynover den mit kusse entvienc}.\\ 
 & der künec wart \textbf{ouch} entvangen sus.\\ 
 & \textbf{Bernout} und Affinamous\\ 
 & die küneginne man küssen sach.\\ 
 & Artus zuo Gramoflanze sprach:\\ 
15 & "ê ir \textbf{sitzen} beginnet,\\ 
 & sehet, ob ir dekeine minnet\\ 
 & dirre vrouwen, und küsset \textbf{die}.\\ 
 & iu beiden sî daz erloubet hie."\\ 
 & \textbf{im sagete}, wer sîn vriundîn was,\\ 
20 & \textbf{ein} brief, den er zuo velde las.\\ 
 & ich meine, daz er ir bruoder sach,\\ 
 & \textbf{diu im} vor aller der werlde jach\\ 
 & ir werden minne tougen.\\ 
 & Gramoflanzes ougen\\ 
25 & \textbf{erkanten s\textit{i}}, \textit{d}iu im minne truoc.\\ 
 & \textbf{dô was sîn vreude hôch} genuoc,\\ 
 & sît Artus \textbf{erloubete} daz,\\ 
 & daz si beide ein ander âne haz\\ 
 & mit gruoze \textbf{tâten entvâhen} kunt:\\ 
30 & er kuste Itonien an den munt.\\ 
\end{tabular}
\scriptsize
\line(1,0){75} \newline
U V W Q R \newline
\line(1,0){75} \newline
\newline
\line(1,0){75} \newline
\textbf{1} erbeizten] bezeigten Q  $\cdot$ dâ] do V W Q \textbf{2} Gramoflanzes] Gramaflanzes V gramoflantzes W Q Gramoflanczes R \textbf{4} pavelûn] gezelt V  $\cdot$ drungen] [d*]: sich drungen V do trúngen Q \textbf{5} in wider] enwider Q \textbf{6} rûmten] Konde Q Rumtent in R \textbf{7} \textit{Versfolge 724.8-7} W   $\cdot$ Brituneiser] Brituͦneiser U bituneser W britoneiser Q britteneister R \textbf{8} sîn] Sy R  $\cdot$ Brandelidelin] brandlidelin Q \textbf{9} [V*]: Vor gramaflanze Jns gezelt gieng V  $\cdot$ in daz] in Q \textbf{10} Gynover] Tschinouer W Synouer Q Gynouer R  $\cdot$ mit kusse] do mit grusse Q \textbf{12} Bernout] Bernovt U Bernvt V Bernuͦt W Bernoút Q Bernouet R  $\cdot$ und] vt W  $\cdot$ Affinamous] affinamuͦs U affinamus V W (Q) R \textbf{13} küssen] [*]: oͮch cv́ssen V \textbf{14} Gramoflanze] Gramaflancze V gramoflantzen W gramoflantze Q Gramoflanczen R \textbf{15} sitzen] siczent R \textbf{16} dekeine] deheine V R keine W Q \textbf{17} küsset] minnet Q \textbf{18} daz] \textit{om.} Q \textbf{19} vriundîn] frúnde R \textbf{20} ein] [Einen]: Ein V \textbf{21} er] es R \textbf{22} aller der] der \sout{alle} R \textbf{23} werden] frewde Q \textbf{24} Gramoflanzes] Gramflanzes V Kúnig gramoflantzes W Gramoflantzes Q [Ggramoflancz]: Grramoflancz R \textbf{25} erkanten] Erkante Q  $\cdot$ si] sie im U die R \textbf{26} vreude] frawe Q  $\cdot$ hôch genuoc] hochgemuͦt V (Q) (R) \textbf{27} Sit [artu*]: artus hette erloͮbet daz V \textbf{28} ander] andren R \textbf{29} gruoze] gruͯszen R  $\cdot$ tâten] taͤten V (R) tatte W \textbf{30} Itonien] Jtonien U Jconien V ytonien W (Q) Jtonie R  $\cdot$ den] de W \newline
\end{minipage}
\end{table}
\end{document}
