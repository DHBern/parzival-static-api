\documentclass[8pt,a4paper,notitlepage]{article}
\usepackage{fullpage}
\usepackage{ulem}
\usepackage{xltxtra}
\usepackage{datetime}
\renewcommand{\dateseparator}{.}
\dmyyyydate
\usepackage{fancyhdr}
\usepackage{ifthen}
\pagestyle{fancy}
\fancyhf{}
\renewcommand{\headrulewidth}{0pt}
\fancyfoot[L]{\ifthenelse{\value{page}=1}{\today, \currenttime{} Uhr}{}}
\begin{document}
\begin{table}[ht]
\begin{minipage}[t]{0.5\linewidth}
\small
\begin{center}*D
\end{center}
\begin{tabular}{rl}
\textbf{630} & \textit{\begin{large}D\end{large}}iu selben kleider leiten an\\ 
 & die zwêne und Gawan.\\ 
 & si giengen ûf den palas,\\ 
 & dâ einhalp manec rîter was,\\ 
5 & anderhalp die \textbf{clâren} vrouwen.\\ 
 & swer rehte kunde schouwen,\\ 
 & von Logroys diu herzogîn\\ 
 & truoc vor \textbf{ûz} den besten schîn.\\ 
 & Der wirt unt die geste\\ 
10 & stuonden vür si, diu \textbf{dâ} gleste,\\ 
 & diu Orgeluse was genant.\\ 
 & der Turkote Florant\\ 
 & unt Lischoys, der clâre,\\ 
 & wurden ledec âne vâre,\\ 
15 & die \textbf{zwêne} vürsten kurtoys,\\ 
 & durch die herzoginne von Logroys.\\ 
 & Si dankte Gawane drumbe,\\ 
 & gein valscheit diu tumbe\\ 
 & \textbf{unt diu} herzelîche wîse\\ 
20 & gein wîplîchem prîse.\\ 
 & Dô disiu rede geschach,\\ 
 & Gawan vier küneginne sach\\ 
 & bî der herzoginne stên.\\ 
 & er bat \textbf{die zwêne} nâher gên\\ 
25 & durch sîne kurtôsîe;\\ 
 & die \textbf{jungern drîe}\\ 
 & hiez er küssen \textbf{dise} zwêne.\\ 
 & Nû was ouch vrou Bene\\ 
 & mit Gawane dar gegangen;\\ 
30 & diu wart dâ wol enpfangen.\\ 
\end{tabular}
\scriptsize
\line(1,0){75} \newline
D Z Fr16 Fr63 \newline
\line(1,0){75} \newline
\textbf{1} \textit{Initiale} D  \textbf{9} \textit{Majuskel} D  \textbf{17} \textit{Majuskel} D  \textbf{21} \textit{Majuskel} D  \textbf{28} \textit{Majuskel} D  \newline
\line(1,0){75} \newline
\textbf{1} Diu] ÷iv D \textbf{5} die clâren] klare Z \textbf{7} Logroys] Logrois Z \textbf{10} si diu dâ] die Z \textbf{12} Turkote] Tvrkoite Z \textbf{13} Lischoys] Liscoys D Lishois Z lisc::: Fr16 \textbf{16} dur di her:::grois Fr16  $\cdot$ Logroys] Logrois Z Logrôis Fr63 \textbf{17} si dankte] Sie dancten Z [d*]: si da::: Fr16 danchte Fr63  $\cdot$ Gawane] gawan Z ::: Fr16 \textbf{21} Dô] Da Z \textbf{22} Gawan] ::: Fr16 \textbf{28} Bene] Bêne D \textbf{29} Gawane] gawan Z ::: Fr16 \newline
\end{minipage}
\hspace{0.5cm}
\begin{minipage}[t]{0.5\linewidth}
\small
\begin{center}*m
\end{center}
\begin{tabular}{rl}
 & diu selben kleider leiten an\\ 
 & die zwêne und Gawan.\\ 
 & si giengen ûf den palas,\\ 
 & d\textit{â} einhalp manic ritter was,\\ 
5 & anderhalp die \textbf{clâr\textit{en}} vrouwen.\\ 
 & wer rehte konde schouwen,\\ 
 & von Logrois diu herzogîn\\ 
 & truoc vor \textbf{ûz} den besten schîn.\\ 
 & der wirt und die geste\\ 
10 & stuonden vür si, diu \textbf{d\textit{â}} gleste,\\ 
 & diu Urgeluse was genant.\\ 
 & der Turkoite Florant\\ 
 & und Lischois, der clâre,\\ 
 & wurden ledic âne vâre,\\ 
15 & die \textbf{zwên} vürsten kurtois,\\ 
 & durch die herzogîn \textit{von} L\textit{o}grois.\\ 
 & si dankte Gawan dâr umbe,\\ 
 & gegen valscheit diu tumbe\\ 
 & \textbf{und \dag durch\dag  diu} herzelîchen wîse\\ 
20 & \dag durch die wîplîchen\dag  prîse.\\ 
 & dô disiu rede geschach,\\ 
 & Gawan vier künigîn sach\\ 
 & bî der herzogîn stên.\\ 
 & er bat \textbf{die zwên} nâher gên\\ 
25 & durch sîn kurtois\textit{î}e;\\ 
 & die \textbf{junger\textit{n} drîe}\\ 
 & hiez er küssen \textbf{dise} zwêne.\\ 
 & nû wa\textit{s} \textit{o}uch vrowe Bene\\ 
 & mit Gawan dar gegangen;\\ 
30 & diu wart d\textit{â} wol enpfangen.\\ 
\end{tabular}
\scriptsize
\line(1,0){75} \newline
m n o \newline
\line(1,0){75} \newline
\newline
\line(1,0){75} \newline
\textbf{1} \textit{Vers 630.1 fehlt} n  \textbf{4} dâ] Do m n o \textbf{5} clâren] clar m \textbf{10} si] in o  $\cdot$ dâ] do m n o \textbf{12} Turkoite] turkoitte m \textbf{13} Lischois] liscois m n o \textbf{16} durch] Vor o  $\cdot$ von] \textit{om.} m  $\cdot$ Logrois] ligrois m Logroeis o \textbf{19} herzelîchen] hertzeclichem n \textbf{21} dô] Die o \textbf{25} kurtoisîe] cortoise m \textbf{26} jungern] junger m \textbf{27} hiez] Hiesse n  $\cdot$ er] ers o \textbf{28} was ouch] was uͯch ouch m \textbf{30} dâ] do m n o \newline
\end{minipage}
\end{table}
\newpage
\begin{table}[ht]
\begin{minipage}[t]{0.5\linewidth}
\small
\begin{center}*G
\end{center}
\begin{tabular}{rl}
 & diu selben kleider leit\textit{en} an\\ 
 & die zwêne unde Gawan.\\ 
 & si giengen ûf den palas,\\ 
 & dâ \textit{ein}halp manic rîter was,\\ 
5 & anderhalp die vrouwen.\\ 
 & swer rehte kunde schouwen,\\ 
 & \begin{large}V\end{large}on Logroys diu herzogîn\\ 
 & truoc \textbf{dâ} vor den besten schîn.\\ 
 & der wirt unde die geste\\ 
10 & stuonden vür si, diu \textbf{dâ} gleste,\\ 
 & diu Orgeluse was genant.\\ 
 & \dag von\dag  Turkoite Florant\\ 
 & unde Lishois, der clâre,\\ 
 & wurden ledic âne vâre,\\ 
15 & die vürsten kurtoys,\\ 
 & durch die herzogîn von Logroys.\\ 
 & si dankete Gawane drumbe,\\ 
 & gein valscheit diu tumbe,\\ 
 & \textbf{diu} herzelîche wîse\\ 
20 & gein wîplîche\textit{m} prîse.\\ 
 & dô disiu rede \textbf{alsô} geschach,\\ 
 & Gawan vier küneginne sach\\ 
 & bî der herzoginne stên.\\ 
 & er bat \textit{\textbf{si}} nâher \textbf{zime} gên\\ 
25 & durch sîne kur\textit{t}oisîe;\\ 
 & die \textbf{jungen Arnive}\\ 
 & hiez er küssen, \textbf{die} zwêne.\\ 
 & nû was ouch vrouwe Bene\\ 
 & mit Gawane dar gegangen;\\ 
30 & diu wart dâ wol enpfangen.\\ 
\end{tabular}
\scriptsize
\line(1,0){75} \newline
G I L M Z Fr51 \newline
\line(1,0){75} \newline
\textbf{1} \textit{Initiale} L Fr51  \textbf{7} \textit{Initiale} G I  \textbf{29} \textit{Initiale} I  \newline
\line(1,0){75} \newline
\textbf{1} leiten] leit G \textbf{2} Gawan] min her Gawan I \textbf{3} si] Die M \textbf{4} einhalp] iene halp G \textbf{5} die] clare L (Z) dy claren M (Fr51) \textbf{6} swer] Wer L M Fr51 \textbf{7} Logroys] logrois G I Fr51 logroýs L ligrois M \textbf{8} truoc dâ vor] Truͯch vor in L Truc vor M Trvc vor vz Z De troch vor Fr51 \textbf{10} si diu] die Z  $\cdot$ dâ] \textit{om.} L M Z Fr51 \textbf{11} diu] Do Fr51  $\cdot$ Orgeluse] Orguluse I Orgelýse L \textbf{12} von] [:on]: von G \textit{om.} L Der M Z (Fr51)  $\cdot$ Turkoite] turchoite G Turchoyde I Tuͯrkoýten L tirkoite M  $\cdot$ Florant] floriant I [Flora*]: Florant M \textbf{13} Lishois] lyshois G Liscoys I Lýtschoys L lisois M lithois Fr51  $\cdot$ der] der degen L \textbf{15} die] Die drý L (M) (Fr51) Die zwene Z \textbf{16} herzogîn] herzoginnen Fr51  $\cdot$ von] de L  $\cdot$ Logroys] logrois G M Fr51 Logroýs L \textbf{17} dankete] danchten L (M) (Z) (Fr51)  $\cdot$ Gawane] Gawan I L (M) \textbf{18} diu] der Fr51 \textbf{19} diu] Vnd die Z Der Fr51 \textbf{20} wîplîchem] wipplichen G (Fr51) \textbf{21} dô] Da M Z  $\cdot$ alsô] \textit{om.} L M Z Fr51 \textbf{22} küneginne] konnige M \textbf{23} herzoginne] herzoginnen Fr51 \textbf{24} bat si] bat G die zwene L bat dy zcwene M (Z) (Fr51)  $\cdot$ zime] \textit{om.} L (M) (Z) (Fr51)  $\cdot$ gên] ien M \textbf{25} kurtoisîe] chursoisie G kuͯrtosien L (Fr51) \textbf{26} jungen] iungern Z  $\cdot$ Arnive] Arniue I jotonien L drie Z arniven Fr51 \textbf{27} hiez] heic Fr51  $\cdot$ die] diese L (M) (Z) (Fr51) \textbf{29} Gawane] Gawan I L (M)  $\cdot$ gegangen] chomen I \textbf{30} ir enphahen wart da wol vernomen I \newline
\end{minipage}
\hspace{0.5cm}
\begin{minipage}[t]{0.5\linewidth}
\small
\begin{center}*T
\end{center}
\begin{tabular}{rl}
 & diu selben kleider legeten an\\ 
 & die zwêne und Gawan.\\ 
 & si giengen ûf den palas,\\ 
 & d\textit{â} einhalp manec rîter was,\\ 
5 & anderhalp die \textbf{clâren} vrouwen.\\ 
 & wer rehte kunde schouwen,\\ 
 & von Logrois diu herzogîn\\ 
 & truoc vor \textbf{ûz} den besten schîn.\\ 
 & \begin{large}D\end{large}er wirt und die geste\\ 
10 & stuonden vür si, diu gleste,\\ 
 & diu Orgeluse was genant.\\ 
 & der Turkoyte Florant\\ 
 & und Lyschoys, d\textit{er} clâre,\\ 
 & wurden ledic âne vâre,\\ 
15 & die \textbf{zwêne} vürsten kurtois,\\ 
 & durch \textit{die} herzoginne von Logrois.\\ 
 & si dankete Gawan dâr umbe,\\ 
 & gein valscheit diu tumbe\\ 
 & \textbf{und} herzenlîchen wîse\\ 
20 & gein wîplîchem prîse.\\ 
 & dô disiu rede geschach,\\ 
 & Gawan viere küneginne sach\\ 
 & bî der herzoginne stên.\\ 
 & er bat \textbf{die zwêne} nâher gên\\ 
25 & durch sîne kurtôsîe;\\ 
 & die \textbf{jungesten drîe}\\ 
 & hiez er küssen \textbf{die} zwêne.\\ 
 & nû was ouch vrou Bene\\ 
 & mit Gawane dar gegangen;\\ 
30 & diu wart d\textit{â} wol entvangen.\\ 
\end{tabular}
\scriptsize
\line(1,0){75} \newline
U V W Q R \newline
\line(1,0){75} \newline
\textbf{1} \textit{Initiale} R  \textbf{9} \textit{Initiale} U V  \newline
\line(1,0){75} \newline
\textbf{1} \textit{Versfolge 630.1-2-629.30} U   $\cdot$ legeten] leit er V (W) leget R \textbf{4} dâ] Do U V W  $\cdot$ einhalp] innenhalb W  $\cdot$ was] sas W R \textbf{5} die clâren] clare V \textbf{6} wer] Swer V \textbf{7} Logrois] Logroys U (V) Ligroys Q \textbf{8} truoc] Die trvͦg V  $\cdot$ besten] liechten W \textbf{10} si] \textit{om.} R  $\cdot$ diu] [d*]: die do V \textbf{11} Orgeluse] orgelusze Q orguluse R  $\cdot$ was] wart Q \textbf{12} der] Zer \textit{nachträglich korrigiert zu:} Der R  $\cdot$ Turkoyte] turkoite U turkoyte vnd W \textbf{13} Lyschoys] lyshois W liszhoys Q Lyschois R  $\cdot$ der] die U \textbf{15} kurtois] turkoys Q \textbf{16} die] \textit{om.} U  $\cdot$ Logrois] logroys V W [logris]: logrois R \textbf{17} dankete] dauchte W dancten R  $\cdot$ Gawan] gawane V gawine R \textbf{19} und] [*]: Vnde die V \textbf{21} disiu] dise R \textbf{22} Gawan] Gawin R  $\cdot$ küneginne] kúnginnen R \textbf{23} herzoginne] herczoginen R \textbf{25} kurtôsîe] kurchosie Q \textbf{26} jungesten] iungeren V (W) (Q) Jungen R \textbf{27} die] [di*]: die V diese Q (R) \textbf{29} Gawane] gawan W Q Gawin R \textbf{30} dâ] do U V W Q \textit{om.} R  $\cdot$ wol] schone W \newline
\end{minipage}
\end{table}
\end{document}
