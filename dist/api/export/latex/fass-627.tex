\documentclass[8pt,a4paper,notitlepage]{article}
\usepackage{fullpage}
\usepackage{ulem}
\usepackage{xltxtra}
\usepackage{datetime}
\renewcommand{\dateseparator}{.}
\dmyyyydate
\usepackage{fancyhdr}
\usepackage{ifthen}
\pagestyle{fancy}
\fancyhf{}
\renewcommand{\headrulewidth}{0pt}
\fancyfoot[L]{\ifthenelse{\value{page}=1}{\today, \currenttime{} Uhr}{}}
\begin{document}
\begin{table}[ht]
\begin{minipage}[t]{0.5\linewidth}
\small
\begin{center}*D
\end{center}
\begin{tabular}{rl}
\textbf{627} & \begin{large}A\end{large}rnive zorn \textbf{bejagete},\\ 
 & daz der knappe ir niht \textbf{en}sagete\\ 
 & \textbf{alsus getâniu} mære,\\ 
 & war er gesendet wære.\\ 
5 & Si bat den, der d\textit{e}r porten pflac:\\ 
 & "ez sî naht oder tac,\\ 
 & sô der knappe wider rîte,\\ 
 & vüege, daz er mîn bîte,\\ 
 & unz ich in gespreche;\\ 
10 & mit dîner kunst daz zeche."\\ 
 & Doch truoc si ûfen knappen haz.\\ 
 & wider în durch vrâgen baz\\ 
 & gienc si zer herzoginne.\\ 
 & diu pflag ouch der sinne,\\ 
15 & daz ir munt des niht gewuoc,\\ 
 & welhen namen Gawan truoc.\\ 
 & sîn bete \textbf{hete an ir} bewart,\\ 
 & si versweic sînen namen unt sînen art.\\ 
 & \textbf{Pûsîne} unt ander schal\\ 
20 & ûf dem palase erhal\\ 
 & mit \textbf{vrœlîchen} sachen.\\ 
 & manec rückelachen\\ 
 & in \textbf{dem palase} wart gehangen.\\ 
 & al dâ wart niht gegangen\\ 
25 & wan ûf \textbf{teppichen} wol geworht.\\ 
 & ez het ein armer wirt ervorht.\\ 
 & al umbe \textbf{an} allen sîten,\\ 
 & mit senften pflûmîten\\ 
 & manec \textbf{gesiz} \textbf{dâ wart} geleit,\\ 
30 & \textbf{dar ûf man tiure kultern treit}.\\ 
\end{tabular}
\scriptsize
\line(1,0){75} \newline
D Z Fr16 \newline
\line(1,0){75} \newline
\textbf{1} \textit{Großinitiale} D Z   $\cdot$ \textit{Initiale} Fr16  \textbf{5} \textit{Majuskel} D  \textbf{11} \textit{Majuskel} D  \textbf{19} \textit{Majuskel} D  \newline
\line(1,0){75} \newline
\textbf{1} Arnive] ARnîve D Arniven Fr16 \textbf{2} ensagete] sagte Z \textbf{3} Vnd si versweic der mere Fr16 \textbf{5} der der] der dir D \textbf{9} unz] Vntz daz Z  $\cdot$ gespreche] bespreche Z \textbf{11} Doch] Da Z \textbf{17} hete an ir] wart dar an Z \textbf{18} sînen art] sin art Z \textbf{19} Pûsîne] Busvnen Z \textbf{23} dem palase] den palas Z \textbf{29} gesiz] geseez Z \textbf{30} kultern] kulter Z \newline
\end{minipage}
\hspace{0.5cm}
\begin{minipage}[t]{0.5\linewidth}
\small
\begin{center}*m
\end{center}
\begin{tabular}{rl}
 & \begin{large}A\end{large}\textit{r}n\textit{iv}e zorn \textbf{bejagete},\\ 
 & daz der knappe ir niht sagete\\ 
 & \textbf{und ir versweic der} mære,\\ 
 & war er gesendet wære.\\ 
5 & si bat den, der der porten pflac:\\ 
 & "ez sî naht oder tac,\\ 
 & sô der knappe wider rîte,\\ 
 & \textbf{sô} vüege, daz er mîn bîte,\\ 
 & unz \textbf{daz} ich in gespreche;\\ 
10 & mit dîner kunst daz \textit{z}eche."\\ 
 & doch truoc si ûf den knappen haz.\\ 
 & wider în durch vrâgen baz\\ 
 & gienc si zer herzoginne.\\ 
 & diu pflac ouch der sinne,\\ 
15 & daz ir munt des niht gewuoc,\\ 
 & welhen namen Gawan truoc.\\ 
 & sîn bete \textbf{het an i\textit{r}} bewart,\\ 
 & si versweic sînen namen und sîn art.\\ 
 & \textbf{busûn} und ander schal\\ 
20 & ûf dem palas erhal\\ 
 & mit \textbf{vröuderîchen} sachen.\\ 
 & manic rückelachen\\ 
 & in \textbf{den palas} wart gehangen.\\ 
 & aldâ wart \textit{niht} gegangen\\ 
25 & wan ûf \textbf{teppichen} wol geworht.\\ 
 & ez het ein armer wirt ervorht.\\ 
 & alumb \textbf{an} allen sîten,\\ 
 & mit senften pl\textit{û}mîten\\ 
 & manic \textbf{siz} \textbf{wart d\textit{â}} geleit\\ 
30 & \textbf{und rîch kulter dar ûf gespreit}.\\ 
\end{tabular}
\scriptsize
\line(1,0){75} \newline
m n o \newline
\line(1,0){75} \newline
\textbf{1} \textit{Initiale} m n  \newline
\line(1,0){75} \newline
\textbf{1} Arnive] ARune m ARniwe n Arnwe o \textbf{2} niht] wit o \textbf{3} versweic] wersweig o \textbf{5} der porten] porten o \textbf{7} rîte] [mag]: rite o \textbf{10} zeche] reche m n o \textbf{11} doch] Do n \textbf{17} ir] ẏm m ẏn o \textbf{21} vröuderîchen] froiden richen o \textbf{24} niht] \textit{om.} m \textbf{26} ervorht] geforcht n \textbf{28} plûmîten] plimitten m (n) plamiten o \textbf{29} siz] secz o  $\cdot$ dâ] do m n o \textbf{30} kulter] kultern n o  $\cdot$ gespreit] gespriet o \newline
\end{minipage}
\end{table}
\newpage
\begin{table}[ht]
\begin{minipage}[t]{0.5\linewidth}
\small
\begin{center}*G
\end{center}
\begin{tabular}{rl}
 & Arnive zorn \textbf{bejaget\textit{e}},\\ 
 & daz der knappe ir niht saget\textit{e}\\ 
 & \textbf{alsus getâniu} mære,\\ 
 & war er gesendet wære.\\ 
5 & si bat den, der der porten pflac:\\ 
 & "ez sî naht oder tac,\\ 
 & sô der knappe wider rîte,\\ 
 & vüege, daz er mîn bîte,\\ 
 & un\textit{z} \textbf{daz} ich in gespreche;\\ 
10 & mit dîner kunst daz zeche."\\ 
 & doch truoc si ûf den knappen haz.\\ 
 & wider în durch vrâge\textit{n} baz\\ 
 & gienc si zer herzoginne.\\ 
 & diu pflag ouch der sinne,\\ 
15 & daz ir munt des niht gewuoc,\\ 
 & welhen namen Gawan truoc.\\ 
 & sîn bete \textbf{wart dar ane} bewart.\\ 
 & si versweic sînen namen unde sînen art.\\ 
 & \textbf{busûnær} unde ander schal\\ 
20 & ûf dem palas erhal\\ 
 & mit \textbf{vrœlîchen} sachen.\\ 
 & manic rückelachen\\ 
 & in \textbf{dem palas} wart gehangen.\\ 
 & al dâ wart niht gegangen\\ 
25 & wan ûf \textbf{teppich} wol geworht.\\ 
 & ez het ein armer w\textit{ir}t ervorht.\\ 
 & al umbe, \textbf{ze} allen sîten,\\ 
 & mit senften plûmîten\\ 
 & manic \textbf{gesez} \textbf{dâ wart} geleit,\\ 
30 & \textbf{dar ûf manic tiur kulter breit}.\\ 
\end{tabular}
\scriptsize
\line(1,0){75} \newline
G I L M Z Fr51 \newline
\line(1,0){75} \newline
\textbf{1} \textit{Initiale} I L Z  \textbf{21} \textit{Initiale} I  \newline
\line(1,0){75} \newline
\textbf{1} Arnive] ARniue I  $\cdot$ bejagete] beiagit G \textbf{2} der knappe ir] ir der knappe L (Fr51)  $\cdot$ sagete] sagit G ensagete I (Fr51) \textbf{3} getâniu] getaner L \textbf{5} bat] bas Fr51  $\cdot$ porten] phorten M (Fr51) \textbf{7} sô] Als Fr51 \textbf{8} mîn] miner Fr51 \textbf{9} unz] vnde G Want Fr51  $\cdot$ gespreche] bespreche Z spreche Fr51 \textbf{10} Mit] Min L  $\cdot$ zeche] cechne Fr51 \textbf{11} doch] Do L Da Z \textbf{12} vrâgen] vrage G \textbf{13} si] \textit{om.} I so Fr51 \textbf{14} diu] Do Fr51  $\cdot$ sinne] sinnen Fr51 \textbf{16} welhen] Welche L  $\cdot$ namen] namen namen Fr51 \textbf{18} sînen namen] den namen Fr51  $\cdot$ sînen art] sinnart G sin art I Z (M) (Fr51) \textbf{19} busûnær] Busuͯne M (Fr51) Busvnen Z \textbf{20} dem] den Fr51 \textbf{23} in dem] Jn L Jndē M Jn den Z Fr51 \textbf{24} al] \textit{om.} Fr51 \textbf{25} wan] Mer Fr51  $\cdot$ teppich] tepichen L (Z) (Fr51) \textbf{26} wirt] wunt G  $\cdot$ ervorht] erworht I (M) \textbf{27} ze] an Z  $\cdot$ sîten] sitten G zcyten M (Fr51) \textbf{29} manic gesez] mangez I Manich sitz L (Fr51) \textbf{30} dar vf manc Teppich breit I  $\cdot$ Dar vf man tývr kuͯlter treit L (M) (Z) \newline
\end{minipage}
\hspace{0.5cm}
\begin{minipage}[t]{0.5\linewidth}
\small
\begin{center}*T
\end{center}
\begin{tabular}{rl}
 & \begin{large}A\end{large}rnyve zorn \textbf{behagete},\\ 
 & daz der knappe ir niht sagete\\ 
 & \textbf{alsus getâniu} mære,\\ 
 & war er gesant wære.\\ 
5 & si bat den, der der porten pflac:\\ 
 & "ez sî naht oder tac,\\ 
 & sô der knappe wider rîte,\\ 
 & vüege, daz er mîn bîte,\\ 
 & unz \textbf{daz} ich in gespreche;\\ 
10 & mit dîner kunst daz zeche."\\ 
 & doch truoc si ûf den knappen haz.\\ 
 & wider în durch vrâgen baz\\ 
 & gienc si zuo der herzoginne.\\ 
 & diu pflac ouch der sinne,\\ 
15 & daz ir munt d\textit{es} niht gewuoc,\\ 
 & welhen namen Gawan truoc.\\ 
 & sîne bete \textbf{wart dar an} bewart.\\ 
 & si versweic sînen namen und sînen art.\\ 
 & \textbf{\begin{large}B\end{large}usû\textit{næ}r} und ande\textit{r} schal\\ 
20 & ûf dem palase erhal\\ 
 & mit \textbf{vrœlîchen} sachen.\\ 
 & manec rückelachen\\ 
 & in \textbf{den palas} wart gehangen.\\ 
 & al dâ wart niht gegangen\\ 
25 & wan ûf \textbf{teppich} wol gewo\textit{r}ht.\\ 
 & ez hete ein armer wirt ervorht.\\ 
 & al umb, \textbf{zuo} allen sîten,\\ 
 & mit senften plûmîten\\ 
 & manec \textbf{gesetze} \textbf{d\textit{â} wart} geleit,\\ 
30 & \textbf{dar ûf man tiure kultern treit}.\\ 
\end{tabular}
\scriptsize
\line(1,0){75} \newline
U V W Q R \newline
\line(1,0){75} \newline
\textbf{1} \textit{Initiale} U V W Q  \textbf{19} \textit{Initiale} U  \newline
\line(1,0){75} \newline
\textbf{1} Arnyve] Arniue V Q ARnyue W (R)  $\cdot$ behagete] beiagete V (W) (Q) (R) \textbf{2} daz] \textit{om.} Q  $\cdot$ ir] \textit{om.} V  $\cdot$ sagete] ensagete V \textbf{3} mære] [lere]: mere Q \textbf{5} bat] hat W  $\cdot$ porten] porte V pforten Q \textbf{9} unz] Mit U  $\cdot$ daz] \textit{om.} V  $\cdot$ gespreche] bespreche W \textbf{10} kunst] kunt Q  $\cdot$ daz] das ich Q \textbf{11} doch] Do Q \textbf{12} vrâgen] frage R \textbf{13} herzoginne] herczoginen R \textbf{14} pflac ouch] pfagk Q \textbf{15} des] daz U V R  $\cdot$ gewuoc] gewuͦt R \textbf{16} Gawan] herr gawan W \textbf{18} sînen art] sin art V (W) [sin*]: sin art  R \textbf{19} Busûnær] Besuͦnder U Bvsvne V Busunen R  $\cdot$ ander] andern U \textbf{20} ûf] Do auff W \textbf{23} den] dem Q  $\cdot$ gehangen] uff gehangen Q \textbf{24} wart] ward anders W \textbf{25} teppich] toͤpiden V (W) (Q)  $\cdot$ wol] schon W  $\cdot$ geworht] gewocht U \textbf{26} ervorht] geuorcht W (R) \textbf{27} al] Alle R  $\cdot$ sîten] ziten V \textbf{29} gesetze] gesitz V (W) (R)  $\cdot$ dâ] do U V W Q \textit{om.} R \textbf{30} Dar vf [*]: man túre cultern treit V  $\cdot$ treit] spreit R \newline
\end{minipage}
\end{table}
\end{document}
