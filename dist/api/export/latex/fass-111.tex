\documentclass[8pt,a4paper,notitlepage]{article}
\usepackage{fullpage}
\usepackage{ulem}
\usepackage{xltxtra}
\usepackage{datetime}
\renewcommand{\dateseparator}{.}
\dmyyyydate
\usepackage{fancyhdr}
\usepackage{ifthen}
\pagestyle{fancy}
\fancyhf{}
\renewcommand{\headrulewidth}{0pt}
\fancyfoot[L]{\ifthenelse{\value{page}=1}{\today, \currenttime{} Uhr}{}}
\begin{document}
\begin{table}[ht]
\begin{minipage}[t]{0.5\linewidth}
\small
\begin{center}*D
\end{center}
\begin{tabular}{rl}
\textbf{111} & die hât ez vor im her gesant,\\ 
 & sît ichz \textbf{lebendic ime} lîbe \textit{v}an\textit{t}."\\ 
 & \textit{\begin{large}D\end{large}}iu vrouwe ir willen dâr an sach,\\ 
 & daz diu spîse was ir herzen dach,\\ 
5 & \textbf{diu} milch in ir \textbf{tüttelîn}.\\ 
 & \textbf{die} dructe drûz diu künegîn.\\ 
 & si sprach: "dû bist von \textbf{triwen} komen.\\ 
 & het ich des toufes niht genomen,\\ 
 & dû wærest wol mînes toufes zil.\\ 
10 & ich sol mich begiezen vil\\ 
 & mit dir unt mit den ougen,\\ 
 & offenlîch unt tougen,\\ 
 & wande ich wil Gahmureten klagen."\\ 
 & diu vrouwe hiez dar nâher tragen\\ 
15 & ein hemde nâch bluot var,\\ 
 & dâr inne \textbf{an}s bâruckes schar\\ 
 & Gahmuret den lîp verlôs,\\ 
 & \textbf{der werlîchen} ende kôs\\ 
 & mit rehter manlîcher ger.\\ 
20 & diu vrouwe \textbf{vrâgete ouch} nâch dem sper,\\ 
 & daz Gahmurete gap den rê.\\ 
 & Ipomidon von Ninnive\\ 
 & gap alsus werlîchen lôn,\\ 
 & der stolze, werde Babylon.\\ 
25 & daz hemede ein hader was von slegen.\\ 
 & diu vrouwe woldez an sich legen,\\ 
 & als si dâ vor hete getân,\\ 
 & \textbf{sô} \textbf{von ritterschaft kom} ir man.\\ 
 & dô \textbf{nâmen} si\textbf{z} ir ûzer hant.\\ 
30 & die besten über \textbf{al daz} lant\\ 
\end{tabular}
\scriptsize
\line(1,0){75} \newline
D Fr33 \newline
\line(1,0){75} \newline
\textbf{3} \textit{Initiale} D  \textbf{25} \textit{Initiale} Fr33  \newline
\line(1,0){75} \newline
\textbf{1} die hât ez] Daz hat dich Fr33 \textbf{2} vant] han D \textbf{3} diu] ÷iv D \textbf{6} die] \textit{om.} Fr33 \textbf{13} Gahmureten] Gahmvreten D Gamureten Fr33 \textbf{15} var] gevar Fr33 \textbf{17} Gahmuret] Gahmvret D Gamuret Fr33 \textbf{18} werlîchen] ritterlichen Fr33 \textbf{21} Gahmurete] Gahmvrete D \textbf{22} Ipomidon] Jpomidon D Jpomid:n Fr33  $\cdot$ Ninnive] Ninnivê D Ninive Fr33 \textbf{24} Babylon] Bab::: Fr33 \textbf{25} hader] hadel Fr33 \textbf{27} hete] heten Fr33 \textbf{28} von ritterschaft kom] kom::: ritterschaft Fr33 \textbf{29} ûzer] vzer der Fr33 \textbf{30} daz] sin Fr33 \newline
\end{minipage}
\hspace{0.5cm}
\begin{minipage}[t]{0.5\linewidth}
\small
\begin{center}*m
\end{center}
\begin{tabular}{rl}
 & die hât ez vor ime her gesant,\\ 
 & sît ich ez \textbf{lebendic ime} lîbe vant."\\ 
 & diu vrouwe ir willen dâr an sach,\\ 
 & daz diu spîse was ir he\textit{r}zen dach,\\ 
5 & \textbf{diu} milch in ir \textbf{tüttelîn}.\\ 
 & \textbf{die} dructe drûz diu künigîn.\\ 
 & si sprach: "dû bist von \textbf{ir nû} komen.\\ 
 & hete ich des toufes niht genomen,\\ 
 & dû wærest wol mînes toufes zil.\\ 
10 & ich \textit{sol} mich begiezen vil\\ 
 & mit dir und mit den ougen,\\ 
 & offenlîche und tougen,\\ 
 & wand ich wil Gahmureten klagen."\\ 
 & diu vrouwe hiez dar nâher tragen\\ 
15 & ein hemede nâch bluote var,\\ 
 & dâr inne \textbf{an}s bâruckes schar\\ 
 & Gahmuret den lîp verlôs,\\ 
 & \textbf{der werlîches} ende kôs\\ 
 & mit rehter manlîcher ger.\\ 
20 & \dag owê\dag  vrouwe \textbf{vrâgete ouch} nâch dem sper,\\ 
 & daz Gahmuret gap \dag dere\dag .\\ 
 & Ypomedo\textit{n} von Ninive\\ 
 & gap alsus werlîchen lôn,\\ 
 & der stolze, werde Babilon.\\ 
25 & daz hemede ein hader was von slegen.\\ 
 & diu vrouwe wolt ez an sich legen,\\ 
 & als si dâ vor hete getân,\\ 
 & \textbf{sô} \textbf{kam von ritterschaft} ir man.\\ 
 & dô \textbf{nâmen} si \textbf{daz} ir ûz der hant.\\ 
30 & die besten über \textbf{alliu} lant\\ 
\end{tabular}
\scriptsize
\line(1,0){75} \newline
m n o \newline
\line(1,0){75} \newline
\newline
\line(1,0){75} \newline
\textbf{1} hât] hette n \textbf{4} ir] irs m n o  $\cdot$ herzen] heiczen m \textbf{6} drûz] das us o \textbf{9} toufes] touffel n \textbf{10} sol] \textit{om.} m \textbf{13} Gahmureten] gahmuretten m gamireten n gamuͯreten o \textbf{15} var] farwa o \textbf{16} bâruckes] beruckes n \textbf{17} Gahmuret] Gamiret n Gamuret o \textbf{21} Gahmuret] gamiret n gamuͯret o  $\cdot$ dere] dem re n o \textbf{22} Ypomedon] Ypomedo m Jpomiden n Jpomidon o  $\cdot$ Ninive] ninniue m mýnne n (o) \textbf{23} alsus] [also]: alsvs m  $\cdot$ werlîchen] verlichen o \textbf{25} von] mit o \textbf{28} sô] Do n o \textbf{29} nâmen] mann o  $\cdot$ si daz ir] ir das n o \textbf{30} alliu] alle das n o \newline
\end{minipage}
\end{table}
\newpage
\begin{table}[ht]
\begin{minipage}[t]{0.5\linewidth}
\small
\begin{center}*G
\end{center}
\begin{tabular}{rl}
 & die hât ez vor im her gesant,\\ 
 & sît ich ez \textbf{lebende in dem} lîbe vant."\\ 
 & \begin{large}D\end{large}iu vrouwe ir willen dâr an sach,\\ 
 & daz diu spîse was ir herzen dach.\\ 
5 & \textbf{die} milch in ir \textbf{tütelîn}\\ 
 & dructe drûz diu künigîn.\\ 
 & si sprach: "dû bist von \textbf{triwen} komen.\\ 
 & het ich des toufes niht genomen,\\ 
 & dû wærest wol mînes toufes zil.\\ 
10 & ich sol mich begiezen vil\\ 
 & mit dir und mit den ougen,\\ 
 & offenlîche und tougen,\\ 
 & wan ich wil Gahmureten klagen."\\ 
 & diu vrouwe hiez dar nâher tragen\\ 
15 & ein hemde nâch bluote var,\\ 
 & dâr, inne des pâruckes schar,\\ 
 & Gahmuret den lîp verlôs,\\ 
 & \textbf{der werdiclîchen} ende kôs\\ 
 & mit rehter manlîcher ger.\\ 
20 & diu vrouwe \textbf{vrâgete ouch} nâch dem sper,\\ 
 & daz Gahmuret gap den rê.\\ 
 & Ipomidon von Ninve\\ 
 & gap alsus werlîchen lôn,\\ 
 & der \textit{stolz}e, \textit{werd}e Babilon.\\ 
25 & daz hemde ein hader was von slegen.\\ 
 & diu vrouwe woltz an sich legen,\\ 
 & alsi dâ vor hete getân,\\ 
 & \textbf{swenne} \textbf{kom von rîterschaft} ir man.\\ 
 & dô \textbf{brâchen} si\textbf{z} ir ûz der hant.\\ 
30 & die besten über \textbf{al daz} lant\\ 
\end{tabular}
\scriptsize
\line(1,0){75} \newline
G I O L M Q R Z \newline
\line(1,0){75} \newline
\textbf{1} \textit{Initiale} O  \textbf{3} \textit{Initiale} G L R Z  \textbf{13} \textit{Überschrift:} Hie gebirt die frow den barczifal Gahmuretes sun von dem dis buͯch seit vnd sinen tetten R  \textbf{19} \textit{Initiale} I  \newline
\line(1,0){75} \newline
\textbf{1} die] ÷ie O  $\cdot$ ez] er O  $\cdot$ vor im her] her vor im I [von]: vor im her O vor her L \textbf{2} ich] \textit{om.} I  $\cdot$ ez] \textit{om.} M  $\cdot$ lebende] lebndich O (L) (Q) (R) (Z)  $\cdot$ dem] minem O (Q)  $\cdot$ vant] fanck Q empfand R \textbf{3} an] nach Z  $\cdot$ sach] ershach I \textbf{4} diu] ir O  $\cdot$ ir] des I irs Q  $\cdot$ herzen] chindes I \textbf{5} die] diu I  $\cdot$ in] \textit{om.} I  $\cdot$ tütelîn] cziczen M \textbf{6} dructe] Druckt R  $\cdot$ drûz] ir usz M \textbf{7} \textit{Vers 111.7 fehlt} Q  \textbf{8} des toufes] den tauf I \textbf{9} mînes toufes] min tavfel O \textbf{10} ich sol] [dich]: ich sol I [Sich]: [Mich]: Jch sol O Du solt Q \textbf{13} Gahmureten] Gamvreten O Gahmuͯreten L gamureten M (Z) gamúreten Q gahmuretten R \textbf{14} diu vrouwe] si I Dy fraw die Q  $\cdot$ nâher] \textit{om.} M nahen Q \textbf{15} ein] Eym M  $\cdot$ var] dar Q \textbf{16} inne des] er indes I inne indes O (L) (M) inne eins Q Z Jnne in R  $\cdot$ pâruckes] brauches Q (R) \textbf{17} Gahmuret] Gamvret O Gamuret M Z Gahmuͯret L Gamuͯret Q \textbf{18} der] Den R  $\cdot$ werdiclîchen] riterlichez I werlichen O (L) (M) (Q) R Z \textbf{19} rehter] riter O \textbf{20} vrâgete ouch] vraget auch I (M) (Q) (Z) avch fragte O [vraget ouch]: ouch vraget L och fragt R  $\cdot$ nâch] \textit{om.} L \textbf{21} Gahmuret] Gahmureten I Gamvreten O Gahmuͯret L gamuret M Q Z  $\cdot$ gap] gabt I  $\cdot$ den rê] die rer R \textbf{22} Ipomidon] ypomidon I (M) (Q) Jpomidon O Z Jhpomidon L Jhpomiden R  $\cdot$ Ninve] niniue I R Ninnive O Ninive L (Z) Nynive M nyniűe Q \textbf{24} stolze werde] werde stolze G \textbf{25} ein hader was] was ein [ha*der]: hader O \textbf{26} sich] \textit{om.} I Q \textbf{27} dâ vor] da vor diche ê I do von Q  $\cdot$ hete] hat O (Q) \textbf{28} swenne kom] so ie I Wenne kom L (Q) (R)  $\cdot$ rîterschaft] strite was chomen I \textbf{29} dô] Da M Z  $\cdot$ siz ir] sie irs L si ez M \textbf{30} über al daz] uͯber alles ir L vbir alle M vbel an dasz Q v́ber das R vber allez Z \newline
\end{minipage}
\hspace{0.5cm}
\begin{minipage}[t]{0.5\linewidth}
\small
\begin{center}*T (U)
\end{center}
\begin{tabular}{rl}
 & die hât ez vor im her gesant,\\ 
 & sît ich ez \textbf{lebendic in mîme} lîbe vant."\\ 
 & diu vrouwe ir willen dâ an sach,\\ 
 & daz diu spîse was ir herzen dach.\\ 
5 & \textbf{die} milch in ir \textbf{brüstelîn}\\ 
 & dructe drûz diu künegîn.\\ 
 & si sprach: "dû bist von \textbf{triuwen} komen.\\ 
 & hete ich des touf\textit{es} niht genomen,\\ 
 & dû wæres wol mînes toufes zil.\\ 
10 & ich sol mich begiezen vil\\ 
 & mit dir und mit den ougen,\\ 
 & offenlîche und tougen,\\ 
 & wan ich wil Gahmureten klagen."\\ 
 & diu vrouwe hiez dar nâher tragen\\ 
15 & ein hemede nâch bluote var,\\ 
 & dâr inne \textbf{in} des bâruckes schar\\ 
 & Gahmuret den lîp verlôs\\ 
 & \textbf{und daz werlîch} ende kôs\\ 
 & mit rehter manlîcher ger.\\ 
20 & diu vrouwe \textbf{ouch vrâgete} nâch dem sper,\\ 
 & daz Gahmurete gap den rê.\\ 
 & Ihpomidon \textit{von} Ninive\\ 
 & gap alsus werlîchen lôn,\\ 
 & \textit{d}er stolze, werde Babylon.\\ 
25 & daz hemede ein hader was von slegen.\\ 
 & diu vrouwe wolt ez an sich legen,\\ 
 & als si dâ vor hete getân,\\ 
 & \textbf{sô} \textbf{von ritterschaft kam} ir man.\\ 
 & dô \textbf{brâchen} si \textbf{ez} ir ûz der hant.\\ 
30 & die besten über \textbf{al daz} lant\\ 
\end{tabular}
\scriptsize
\line(1,0){75} \newline
U V W T \newline
\line(1,0){75} \newline
\textbf{3} \textit{Initiale} W   $\cdot$ \textit{Majuskel} T  \textbf{20} \textit{Majuskel} T  \textbf{22} \textit{Majuskel} T  \textbf{25} \textit{Majuskel} T  \textbf{29} \textit{Majuskel} T  \textbf{30} \textit{Majuskel} T  \newline
\line(1,0){75} \newline
\textbf{1} die] div T  $\cdot$ gesant] benant W \textbf{2} sît ichs mit lebendem libe enpfant T  $\cdot$ lebendic] lebende W  $\cdot$ mîme] dem V (W)  $\cdot$ vant] befant W \textbf{4} ir] irz T \textbf{5} \textit{Vers nachträglich weitgehend radiert} T   $\cdot$ in] aus W  $\cdot$ brüstelîn] tittelin V (W) t:telin T \textbf{8} des toufes] des deuͦf \textit{nachträglich korrigiert zu:} den deuͦf U \textbf{9} wol] doch W T  $\cdot$ mînes toufes] mein iamers W \textbf{13} Gahmureten] Gahmuͦreten U Gamureten V (W) \textbf{14} dar] in W \textbf{16} in] \textit{om.} W \textbf{17} Gahmuret] Gahmuͦret U Gamuret V W \textbf{18} und daz werlîch] der wêrliches T \textbf{19} rehter] ritter W \textbf{20} vrouwe ouch vrâgete] fragt auch W \textbf{21} \textit{Versfolge 111.22-21} W   $\cdot$ Gahmurete] Gahmuͦrete U Gamurette V gamuret W \textbf{22} Ihpomidon] Jpomidon U Ẏpomidon V Ypomidon W Jhpomidon T  $\cdot$ von] vnd U (W) [*]: von  V  $\cdot$ Ninive] Niniue V (W) Ninivê T \textbf{23} werlîchen] werden T \textbf{24} der] Vnd der U (W)  $\cdot$ stolze werde] werde stoltze W  $\cdot$ Babylon] [*]: babilon V babilon W Babylôn T \textbf{25} ein hader] gar ein hudel W \textbf{26} sich] \textit{om.} W \textbf{27} hete] hat W \textbf{29} ez ir] irs W T \textbf{30} al daz] alles des W \newline
\end{minipage}
\end{table}
\end{document}
