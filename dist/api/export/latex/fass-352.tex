\documentclass[8pt,a4paper,notitlepage]{article}
\usepackage{fullpage}
\usepackage{ulem}
\usepackage{xltxtra}
\usepackage{datetime}
\renewcommand{\dateseparator}{.}
\dmyyyydate
\usepackage{fancyhdr}
\usepackage{ifthen}
\pagestyle{fancy}
\fancyhf{}
\renewcommand{\headrulewidth}{0pt}
\fancyfoot[L]{\ifthenelse{\value{page}=1}{\today, \currenttime{} Uhr}{}}
\begin{document}
\begin{table}[ht]
\begin{minipage}[t]{0.5\linewidth}
\small
\begin{center}*D
\end{center}
\begin{tabular}{rl}
\textbf{352} & \begin{large}S\end{large}i vlizzen \textbf{sich} \textbf{gein} strîtes werc.\\ 
 & Gawan \textbf{reit} ûf an den berc.\\ 
 & swie wênec er dâ wære bekant,\\ 
 & er \textbf{reit} \textbf{ûf}, dâ er die burc vant.\\ 
5 & sîn ougen muosen schouwen\\ 
 & manege werde vrouwen.\\ 
 & diu wirtîn selbe komen was\\ 
 & durch warten ûf \textbf{den} palas\\ 
 & mit ir schœnen tohtern zwein,\\ 
10 & von den vil \textbf{liehter} varwe schein.\\ 
 & \textbf{Schiere het er von in} vernomen.\\ 
 & si \textbf{sprâchen}: "wer \textbf{was} uns hie komen?"\\ 
 & sus sprach diu alte herzogîn:\\ 
 & "waz \textbf{zoges} mac ditze sîn?"\\ 
15 & \textbf{Dô sprach ir elter tohter} sân:\\ 
 & "muoter, ez ist ein koufman."\\ 
 & "nû vüeret man im doch schilde mite."\\ 
 & "daz ist vil koufliute site."\\ 
 & Ir jünger tohter dô sprach:\\ 
20 & "dû zîhest in, daz doch nie geschach.\\ 
 & swester, des \textbf{maht}û dich schamen.\\ 
 & er gewan nie koufmannes namen.\\ 
 & er ist sô minneclîch getân,\\ 
 & ich wil in \textbf{zeime} ritter hân.\\ 
25 & sîn dienst mac hie lônes gern,\\ 
 & des wil ich in durch liebe wern."\\ 
 & Sîne knappen \textbf{nâmen} \textbf{dô} goume,\\ 
 & daz ein linde unt ölboume\\ 
 & \textbf{unden} bî der mûre stuont.\\ 
30 & daz dûhte si ein gæber vunt.\\ 
\end{tabular}
\scriptsize
\line(1,0){75} \newline
D \newline
\line(1,0){75} \newline
\textbf{1} \textit{Initiale} D  \textbf{11} \textit{Majuskel} D  \textbf{15} \textit{Majuskel} D  \textbf{19} \textit{Majuskel} D  \textbf{27} \textit{Majuskel} D  \newline
\line(1,0){75} \newline
\newline
\end{minipage}
\hspace{0.5cm}
\begin{minipage}[t]{0.5\linewidth}
\small
\begin{center}*m
\end{center}
\begin{tabular}{rl}
 & si vlizzen \textbf{gegen} strîtes wer\textit{c}.\\ 
 & Gawan \textbf{reit} ûf an den berc.\\ 
 & wie wênic er dâ wære bekant,\\ 
 & er \textbf{reit} \textbf{ûf}, d\textit{â} er die burc vant.\\ 
5 & sîn ougen muosen schouwen\\ 
 & manige werde vrouwen.\\ 
 & diu wirtîn selbe komen was\\ 
 & durch warten ûf \textbf{den} palas\\ 
 & mit ir schœnen tohteren zwein,\\ 
10 & von den vil \textbf{liehter} varwe schein.\\ 
 & \textbf{schiere hete er \textit{von in}} verno\textit{m}en.\\ 
 & si \textbf{sprac\textit{h}}: "\textit{w}er \textbf{mac} uns hie komen?"\\ 
 & sus sprach diu alte herzogîn:\\ 
 & "waz \textbf{gezoge\textit{s}} mac diz sîn?"\\ 
15 & \textbf{dô sprach ir elter tohter} sân:\\ 
 & "muoter, ez ist ein koufman."\\ 
 & "nû vüeret man ime doch schilte mite."\\ 
 & "daz ist vil koufliute site."\\ 
 & ir jünger tohter dô sprach:\\ 
20 & "dû zîhest in, daz doch nie geschach.\\ 
 & swester, des \textbf{maht}û dich schamen.\\ 
 & er gewan nie koufmans namen.\\ 
 & er ist sô minneclîch getân,\\ 
 & ich wil in \textbf{ze einem} ritter hân.\\ 
25 & sîn dienest mac hie lônes gern,\\ 
 & des wil ich in durch liebe wern."\\ 
 & sîne knappen \textbf{nâmen} \textbf{dô} goume,\\ 
 & daz ein linde und ölboume\\ 
 & \textbf{unden} bî der mûren stuont.\\ 
30 & daz dûhte si ein gæber vunt.\\ 
\end{tabular}
\scriptsize
\line(1,0){75} \newline
m n o \newline
\line(1,0){75} \newline
\newline
\line(1,0){75} \newline
\textbf{1} \textit{Versfolge 351.25-30, 352.1-16 (Bl. 225v), 351.1-23 (Bl. 226r), 351.24, 352.17-30 (Bl. 226v)} m   $\cdot$ vlizzen gegen strîtes] flissent sich zuͯ strite n (o)  $\cdot$ werc] wert m \textbf{3} dâ] do n o \textbf{4} dâ] do m n o \textbf{5} muosen] muͯssen m muͯsten n o \textbf{9} schœnen tohteren zwein] schowen dochtern czweẏ o \textbf{11} von in vernomen] ver nonen m \textbf{12} sprach wer] sprach hie wer m \textbf{13} herzogîn] herczogen o \textbf{14} gezoges] gezoget m gezúges n (o) \textbf{18} daz] Disz o \textbf{19} jünger] junge n (o) \textbf{20} geschach] beschach n o \textbf{21} des] das o \textbf{26} des] Das n o \textbf{30} vunt] kúnt o \newline
\end{minipage}
\end{table}
\newpage
\begin{table}[ht]
\begin{minipage}[t]{0.5\linewidth}
\small
\begin{center}*G
\end{center}
\begin{tabular}{rl}
 & si vlizzen \textbf{sich} \textbf{gein} strîtes werc.\\ 
 & Gawan \textbf{kêrte} ûf an den berc.\\ 
 & swie wênic er dâ wære bekant,\\ 
 & er \textbf{kêrte} \textbf{ûf}, dâ er die burc vant.\\ 
5 & sîniu ougen muosen schouwen\\ 
 & \textit{\textbf{vil}} manige werde vrouwen.\\ 
 & diu wirtîn selbe komen was\\ 
 & durch warten ûf \textbf{dem} palas\\ 
 & mit ir \textit{schœnen} tohteren zwein,\\ 
10 & von den vil \textbf{liehter} varwe schein.\\ 
 & \textbf{von den \textit{het er vil schiere}} vernomen.\\ 
 & si \textbf{vrâgten}: "wer \textbf{mag} uns hie komen?"\\ 
 & s\textit{us} sprach diu alte herzogîn:\\ 
 & "waz \textbf{gezoges} mac diz sîn?"\\ 
15 & \textbf{ir alter tohter sprach dô} sân:\\ 
 & "muoter, ez ist ein koufman."\\ 
 & "nû vüert man im doch schilte mite."\\ 
 & "daz ist vil koufliute site."\\ 
 & ir jünger tohter dô sprach:\\ 
20 & "dû zîhest in, daz doch nie geschach.\\ 
 & swester, des \textbf{maht}û dich schamen.\\ 
 & er gewan nie koufmannes namen.\\ 
 & er ist sô minniclîch getân,\\ 
 & ich wil in \textbf{zeinem} rîter hân.\\ 
25 & \begin{large}S\end{large}în dienst mac hie lônes gern,\\ 
 & des wil ich in durch liebe wern."\\ 
 & sîne knappen \textbf{tâten} goume,\\ 
 & daz ein linde unde ölboume\\ 
 & \textbf{unden} bî der mûre stuont.\\ 
30 & daz dûhte si ein gæber vunt.\\ 
\end{tabular}
\scriptsize
\line(1,0){75} \newline
G I O L M Q R Z Fr39 \newline
\line(1,0){75} \newline
\textbf{1} \textit{Initiale} I L Z Fr39   $\cdot$ \textit{Capitulumzeichen} R  \textbf{15} \textit{Initiale} I  \textbf{25} \textit{Initiale} G  \newline
\line(1,0){75} \newline
\textbf{1} strîtes] streite Q \textbf{2} Gawan] Gaban Q  $\cdot$ kêrte] kert I (O) (Q) (R) (Z)  $\cdot$ an] \textit{om.} I \textbf{3} swie] Wie L (M) Q R  $\cdot$ dâ] do Q R Fr39  $\cdot$ bekant] erchant I O \textbf{4} kêrte] chert I reit O L M Q R Z Fr39  $\cdot$ ûf] \textit{om.} I  $\cdot$ dâ] do Q Fr39 \textbf{5} sîniu] Sine R \textbf{6} vil] \textit{om.} G  $\cdot$ werde] werden Fr39 \textbf{7} selbe] selben Q \textbf{8} ûf] in Fr39  $\cdot$ dem] den I (O) (L) (Q) (Z) (Fr39) dē M \textbf{9} schœnen] \textit{om.} G \textbf{10} den] \textit{om.} L Fr39  $\cdot$ vil] \textit{om.} Z  $\cdot$ liehter] lihtiv O (Q) lýchter L liechtte R (Z)  $\cdot$ varwe] frawen Q \textbf{11} \textit{Versfolge 352.12-11} I   $\cdot$ von] Ay von I  $\cdot$ het er vil schiere] er schiere hete G het er schiere L Fr39 \textbf{12} vrâgten] sprachen Z  $\cdot$ mag] [v]: Mage G \textbf{13} sus] so G \textbf{14} diz] das Q \textbf{15} sprach] die sprach L (M) Z  $\cdot$ dô] al O \textit{om.} L M Q R Z \textbf{17} im] \textit{om.} Q ich R  $\cdot$ doch] auch L \textbf{18} daz] muter ez I Wê daz O (L) (M) (Z) Wie das Q R \textbf{19} ir] Die Z  $\cdot$ jünger] iungen I junge M  $\cdot$ dô] da L M \textbf{20} daz] daz daz O \textbf{21} des] daz L  $\cdot$ schamen] wol schamen I schawm Q \textbf{22} gewan] engewan L (Z) (Fr39) \textbf{23} sô] \textit{om.} O R \textbf{25} hie] wol Z \textbf{26} liebe] lobe R liebiv Fr39 \textbf{27} tâten] namen L Fr39 \textbf{28} ein] \textit{om.} L Fr39  $\cdot$ unde] oder L (M) Fr39 vnd ein Z \textbf{29} bî] in Q  $\cdot$ mûre] Muren M \textbf{30} gæber] geher O [gabert]: gaber L \newline
\end{minipage}
\hspace{0.5cm}
\begin{minipage}[t]{0.5\linewidth}
\small
\begin{center}*T
\end{center}
\begin{tabular}{rl}
 & si vlizzen \textbf{sich} \textbf{des} strîtes werc.\\ 
 & Gawan \textbf{kêrte} ûf an den berc.\\ 
 & swie wênic er dâ wære bekant,\\ 
 & er \textbf{reit} \textbf{doch}, dâ er die burc vant.\\ 
5 & Sîniu ougen muosen schouwen\\ 
 & \textbf{\textit{v}il} manege werde vrouwen.\\ 
 & Diu wirtîn selbe komen was\\ 
 & durch warten ûf \textbf{den} palas\\ 
 & mit ir schœnen tohteren zwein,\\ 
10 & von den vil \textbf{liehte} varwe schein.\\ 
 & \textbf{von den heter schiere} vernomen,\\ 
 & \textbf{daz} si \textbf{vrâgeten}: "wer \textbf{mac} uns hie komen?"\\ 
 & \textit{sus} sprach diu alte herzogîn:\\ 
 & "waz \textbf{gezoges} mac diz sîn?"\\ 
15 & \textbf{Ir elter tohter, diu sprach} sân:\\ 
 & "muoter, ez ist ein koufman."\\ 
 & "Nû vüeret man im doch schilte mite."\\ 
 & "\textbf{Wê}, daz ist vil koufliute site!"\\ 
 & \textit{Ir} jünger tohter dô sprach:\\ 
20 & "dû zîhest in, daz doch nie geschach.\\ 
 & swester, des \textbf{darft} dû dich schamen.\\ 
 & er\textbf{n} gewan nie koufmannes namen.\\ 
 & er ist sô minneclîch getân,\\ 
 & ich wil in \textbf{vür einen} rîter hân.\\ 
25 & sîn dienst mac hie lônes gern,\\ 
 & des wil ich in durch lieb\textit{e} wern."\\ 
 & \begin{large}S\end{large}îne knappen \textbf{tâten} goume,\\ 
 & daz ein linde unde ölboume\\ 
 & \textbf{ûzzen} bî der mûre stuont.\\ 
30 & daz dûhte si ein gæber vunt.\\ 
\end{tabular}
\scriptsize
\line(1,0){75} \newline
T V W \newline
\line(1,0){75} \newline
\textbf{1} \textit{Initiale} W  \textbf{5} \textit{Majuskel} T  \textbf{7} \textit{Majuskel} T  \textbf{15} \textit{Majuskel} T  \textbf{17} \textit{Majuskel} T  \textbf{18} \textit{Majuskel} T  \textbf{19} \textit{Majuskel} T  \textbf{27} \textit{Initiale} T V  \newline
\line(1,0){75} \newline
\textbf{1} des] gegen V (W) \textbf{2} kêrte] kert W \textbf{3} swie] Wie W  $\cdot$ dâ] do V W \textbf{4} doch dâ] doch do V do W  $\cdot$ burc] brucken W \textbf{5} muosen] mvesen T muͤsten V \textbf{6} vil] wil T \textbf{7} selbe] selber W \textbf{10} von] Auff W  $\cdot$ liehte] liehter V (W) \textbf{11} von den] Von in V Man W  $\cdot$ vernomen] do vernomen W \textbf{12} daz si vrâgeten] Das sú sprachen V Sy fragten alle W  $\cdot$ mac] ist W \textbf{13} sus] \textit{om.} T W \textbf{18} wê] \textit{om.} V \textbf{19} Ir] \textit{om.} T Die W \textbf{20} daz] des W \textbf{21} darft dû] magstu W \textbf{22} ern] Er W \textbf{24} vür einen] zeinem V ymmer zuͦ einem W \textbf{26} liebe] [libi]: liebi T \textbf{27} tâten] nomen do V \textbf{28} unde] vnd ein W \textbf{29} Auß bei der maur sahen sy an der stuont W  $\cdot$ mûre] mvren V \textbf{30} gæber] gabe W \newline
\end{minipage}
\end{table}
\end{document}
