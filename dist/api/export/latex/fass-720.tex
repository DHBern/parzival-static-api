\documentclass[8pt,a4paper,notitlepage]{article}
\usepackage{fullpage}
\usepackage{ulem}
\usepackage{xltxtra}
\usepackage{datetime}
\renewcommand{\dateseparator}{.}
\dmyyyydate
\usepackage{fancyhdr}
\usepackage{ifthen}
\pagestyle{fancy}
\fancyhf{}
\renewcommand{\headrulewidth}{0pt}
\fancyfoot[L]{\ifthenelse{\value{page}=1}{\today, \currenttime{} Uhr}{}}
\begin{document}
\begin{table}[ht]
\begin{minipage}[t]{0.5\linewidth}
\small
\begin{center}*D
\end{center}
\begin{tabular}{rl}
\textbf{720} & \begin{large}D\end{large}er kinde einez z\textbf{em künege} sprach:\\ 
 & "hêrre, \textbf{swes} ir vür ungemach\\ 
 & jeht, daz sol mîn hêrre lân,\\ 
 & wil er rehte vuoge hân.\\ 
5 & ir wizzet wol umbe den alten haz.\\ 
 & \textbf{mîme hêrrn stêt} belîben baz,\\ 
 & denne daz er \textbf{dâ her} zuo \textbf{z}iu rîte.\\ 
 & diu herzoginne \textbf{pfligt} noch site,\\ 
 & daz si \textbf{im} \textbf{ir} hulde hât versagt\\ 
10 & unt manegem man \textbf{ab im} geklagt."\\ 
 & "Er sol mit wênec liuten komen",\\ 
 & sprach \textbf{Artus}. "die wîle hân ich \textbf{genomen}\\ 
 & vride vür den selben zorn\\ 
 & von der herzoginne wol geborn.\\ 
15 & Ich wil im guot geleite tuon.\\ 
 & Beacors, mîner swester sun,\\ 
 & \textbf{nimt in dort an halbem} wege.\\ 
 & er sol varn in \textbf{mînes} geleites pflege.\\ 
 & des \textbf{darf} er niht vür laster jehen.\\ 
20 & ich lâze in werde liute \textbf{sehen}."\\ 
 & Mit urloube si \textbf{vuoren} dan.\\ 
 & \textbf{Artus} \textbf{hielt} eine ûf dem plân.\\ 
 & Bene unt diu \textbf{zwei} kindelîn\\ 
 & ze Rosche Sabbins riten în,\\ 
25 & anderthalben \textbf{ûz}, \textbf{dâ} daz her lac.\\ 
 & dô\textbf{ne} gelebte nie sô lieben tac\\ 
 & Gramoflanz, dô in gesprach\\ 
 & Bene unt diu kint. sîn herze jach,\\ 
 & im \textbf{wære} \textbf{al} \textbf{solhiu} mære brâht,\\ 
30 & der \textbf{sælde} \textbf{gein im het} \textbf{erdâht}.\\ 
\end{tabular}
\scriptsize
\line(1,0){75} \newline
D \newline
\line(1,0){75} \newline
\textbf{1} \textit{Initiale} D  \textbf{11} \textit{Majuskel} D  \textbf{15} \textit{Majuskel} D  \textbf{21} \textit{Majuskel} D  \newline
\line(1,0){75} \newline
\textbf{16} Beacors] Beachcors D \textbf{24} Rosche Sabbins] Rosce Sabbins D \newline
\end{minipage}
\hspace{0.5cm}
\begin{minipage}[t]{0.5\linewidth}
\small
\begin{center}*m
\end{center}
\begin{tabular}{rl}
 & \begin{large}D\end{large}er kinde ein\textit{ez} zuo \textbf{Artuse} sprach:\\ 
 & "hêrre, \textbf{wes} ir vür ungemach\\ 
 & jeht, daz sol mîn hêrre lân,\\ 
 & wil er rehte vuoge hân.\\ 
5 & ir wizzet wol umb den alten haz.\\ 
 & \textbf{stât mînem hêrren} blîben baz,\\ 
 & dan daz er \textbf{dâ her} zuo iu rîte?\\ 
 & \textit{diu herzogîn} \textbf{pfliget} noch site,\\ 
 & daz si \textbf{ir} hulde het versaget\\ 
10 & und manigem man \textbf{ab im} geklaget."\\ 
 & "er sol mit wênic liuten komen",\\ 
 & sprach \textbf{Artus}. "die wîle hân ich \textbf{vernomen}\\ 
 & vride vür den selben zorn\\ 
 & von der herzogîn wol geborn.\\ 
15 & ich wil im guot geleite tuon.\\ 
 & Bea\textit{c}urs, mîner swester sun,\\ 
 & \textbf{nimt in an dem halben} wege.\\ 
 & er sol varn in \textbf{mîne\textit{r}} pflege.\\ 
 & des \textbf{sol} er niht vür laster jehen.\\ 
20 & ich lâze in werde liute \textbf{sehen}."\\ 
 & mit urloube si \textbf{vuoren} dan.\\ 
 & \textbf{Artus} \textbf{hielt} \textbf{al}ein ûf dem plân.\\ 
 & Bene und diu \textbf{zwei} kindelîn\\ 
 & zuo Ros\textit{ch}e Sabins riten în,\\ 
25 & \textbf{und} anderhalp daz her \textbf{dô} lac.\\ 
 & dô gelebte nie sô lieben tac\\ 
 & Gramolanz, dô in gesprach\\ 
 & Bene und diu kint. sîn herze jach,\\ 
 & im \textbf{wær} \textbf{al}\textbf{solich} mære brâht,\\ 
30 & der \textbf{sælde} \textbf{hete gegen im} \textbf{dâht}.\\ 
\end{tabular}
\scriptsize
\line(1,0){75} \newline
m n o \newline
\line(1,0){75} \newline
\textbf{1} \textit{Initiale} m   $\cdot$ \textit{Capitulumzeichen} n  \newline
\line(1,0){75} \newline
\textbf{1} einez] ein m \textbf{4} rehte] reht m (o) echt n \textbf{7} Dan das er durch uch rette o \textbf{8} diu herzogîn] \textit{om.} m \textbf{10} manigem] mangen o \textbf{12} hân ich] ich han o \textbf{13} vür] noch fúr o \textbf{14} der] [den]: der o \textbf{15} im] nú o  $\cdot$ tuon] duͯn duͯn o \textbf{16} Beacurs] Beaturs m o Beatus o \textbf{17} nimt] Nement o  $\cdot$ halben] selben o \textbf{18} mîner] minem m \textbf{22} plân] plam o \textbf{24} Rosche] rosse m n o  $\cdot$ Sabins] sabbins m \textbf{26} gelebte] gelepten o \textbf{27} Gramolanz] Gramolantz m n Gramolancz o \textbf{28} diu] sin o \textbf{29} alsolich] also sollich n \textbf{30} dâht] gedacht n \newline
\end{minipage}
\end{table}
\newpage
\begin{table}[ht]
\begin{minipage}[t]{0.5\linewidth}
\small
\begin{center}*G
\end{center}
\begin{tabular}{rl}
 & \begin{large}D\end{large}er kinde einez ze \textbf{dem künige} sprach:\\ 
 & "hêrre, \textbf{swes} ir vür ungemach\\ 
 & jeht, daz sol mîn hêr\textit{r}e lân,\\ 
 & wil er rehte vuoge hân.\\ 
5 & ir wizzet wol umbe den alten haz.\\ 
 & \textbf{mînem hêrren stêt} belîben baz,\\ 
 & danne daz er zuo iu rîte.\\ 
 & \textit{diu herzogîn \textbf{hât} noch site},\\ 
 & daz si\textbf{m} hulde hât versaget\\ 
10 & unde manege\textit{m} man \textbf{über in} klaget."\\ 
 & "er sol mit wênec liuten komen",\\ 
 & sprach \textbf{der künic}. "die wîle hân ich \textbf{genomen}\\ 
 & \textbf{einen} vride vür den selben zorn\\ 
 & von der herzoginne wol geborn.\\ 
15 & ich wil im guot geleite tuon.\\ 
 & Beakurs, mîner swester sun,\\ 
 & \textbf{sende ich im ze halbem} wege.\\ 
 & er sol varn in \textbf{mînes} geleites pflege.\\ 
 & des \textbf{en}\textbf{darf} er niht vür laster jehen.\\ 
20 & ich lâze in werde liute \textbf{sehen}."\\ 
 & mit urloube si \textbf{schieden} dan.\\ 
 & \textbf{der künec} \textbf{beleip} eine ûf dem plân.\\ 
 & Bene unde diu kindelîn\\ 
 & ze Roisabins riten în\\ 
25 & \textbf{unde} anderhalp, \textbf{dâ} daz her lac.\\ 
 & dô\textbf{ne} gelebte nie sô lieben tac\\ 
 & Gramoflanz, dô in gesprach\\ 
 & Bene und di\textit{u} kint. sîn herze jach,\\ 
 & im \textbf{w\textit{æ}ren} \textbf{sölchiu} mære brâht,\\ 
30 & der \textbf{si} \textbf{gein im het} \textbf{erdâht}.\\ 
\end{tabular}
\scriptsize
\line(1,0){75} \newline
G I L M Z Fr20 Fr24 Fr45 \newline
\line(1,0){75} \newline
\textbf{1} \textit{Initiale} G L Z Fr20 Fr24  \textbf{11} \textit{Initiale} I  \textbf{21} \textit{Initiale} Fr45  \newline
\line(1,0){75} \newline
\textbf{1} Der] ÷er Fr20 \textbf{2} swes] wes L M  $\cdot$ ir] [is]: ir M \textbf{3} hêrre] herze G \textbf{6} mînem] Minen L  $\cdot$ hêrren] \textit{om.} Fr20 \textbf{8} \textit{Vers 720.8 fehlt} G  \textbf{9} hulde] ir hulde M \textbf{10} manegem] mangen G maniger M manigē Fr24  $\cdot$ über] vf L von Fr20  $\cdot$ in] im Fr20  $\cdot$ klaget] Gechlagt I (L) (Z) (Fr20) (Fr24) \textbf{11} liuten] \textit{om.} Fr45 \textbf{12} sprach der künic] \textit{om.} L  $\cdot$ hân ich] han auch ich I ich han Z  $\cdot$ genomen] [vernomen]: genomen I \textbf{16} Beakurs] beacurs I Beakuͯrs L Beacuͯrs M beachvrs Fr20 Beacvͦrs Fr24 Beakuͦrs Fr45 \textbf{17} halbem] halben M (Z) (Fr24) \textbf{18} er sol varn] Svs var L \textbf{19} endarf] darf Fr24 \textbf{21} schieden] schienden L \textbf{22} künec] \textit{om.} Fr20  $\cdot$ eine] eyner M alein Fr45 \textbf{23} Bene] ::ne Fr45  $\cdot$ kindelîn] chunigin Fr20 \textbf{24} Roisabins] Roys sabins I Roẏsabins L Roitschesabins Z roisabin Fr20 Roy Sabyns Fr24 Rovsabyne Fr45  $\cdot$ riten] ritens Fr20 \textbf{25} unde] \textit{om.} M  $\cdot$ dâ] usz da M (Z) (Fr45)  $\cdot$ her] mer Fr24 \textbf{26} dône gelebte] done gelep I Da en gelebiten M (Fr24) Da gelebte Z do lebte Fr45  $\cdot$ nie] sy ny M \textbf{27} Gramoflanz zuͯ hant er sprach L  $\cdot$ Gramoflanz] Gramoflantz G Z Gramoflans Fr45  $\cdot$ dô] da M Z \textbf{28} diu] die G sin M \textbf{29} wæren] waren G L  $\cdot$ sölchiu] so liebiv Fr20 \textbf{30} \textit{Vers 720.30 fehlt} M   $\cdot$ si gein] selde gein Z selden vor Fr45  $\cdot$ erdâht] gedaht Z \newline
\end{minipage}
\hspace{0.5cm}
\begin{minipage}[t]{0.5\linewidth}
\small
\begin{center}*T
\end{center}
\begin{tabular}{rl}
 & \begin{large}D\end{large}er kinde einez zuo \textbf{dem künege} sprach:\\ 
 & "hêrre, \textbf{waz} ir vür ungemach\\ 
 & jehet, daz sol mîn hêrre lân,\\ 
 & wil er rehte gevuoge hân.\\ 
5 & ir wizzet wol umb den alten haz.\\ 
 & \textbf{mîme hêrren stêt} blîben baz,\\ 
 & dan daz er zuo iu rîte.\\ 
 & diu herzoginne \textbf{pfliget} noch site,\\ 
 & daz si \textbf{im} \textbf{ir} hulde hât versaget\\ 
10 & und manig\textit{em} man \textbf{über in} klaget."\\ 
 & "er sol mit wênic liuten komen",\\ 
 & sprach \textbf{der künec}. "die wîle hân ich \textbf{vernomen}\\ 
 & \textbf{einen} vriden vür den selben zorn\\ 
 & von der herzoginne wol geborn.\\ 
15 & ich wil im guot geleite tuon.\\ 
 & Beakurs, mîner swester sun,\\ 
 & \textbf{sende ich im zuo halbem} wege.\\ 
 & er sol varn in \textbf{mînes} geleites pflege.\\ 
 & des \textbf{en}\textbf{darf} er niht vür laster jehen.\\ 
20 & ich lâzen werde liute \textbf{spehen}."\\ 
 & mit urloube si \textbf{schieden} dan.\\ 
 & \textbf{der künec} \textbf{bleip} \textbf{al}eine ûf dem plân.\\ 
 & Bene und diu \textbf{zwei} kindelîn\\ 
 & zuo Roitschesabins riten în,\\ 
25 & anderhalp \textbf{ûz}, \textbf{dâ} daz her \textbf{dâ} lac.\\ 
 & dô gele\textit{b}te nie sô lieben tac\\ 
 & Gramoflanz, dô in gesprach\\ 
 & Bene und diu kint. sîn herze jach,\\ 
 & im \textbf{wæren} \textbf{solichiu} mære brâht,\\ 
30 & der \textbf{sælde} \textbf{vür in hete} \textbf{erdâht}.\\ 
\end{tabular}
\scriptsize
\line(1,0){75} \newline
U V W Q R \newline
\line(1,0){75} \newline
\textbf{1} \textit{Initiale} U W R  \textbf{21} \textit{Initiale} W  \newline
\line(1,0){75} \newline
\textbf{2} waz] swas V wes Q R \textbf{4} gevuoge] fuͦge V W (Q) (R) \textbf{6} mîme] Weinem W \textbf{7} Dan [da*]: daz er do her zvͦz v́ch ritte V \textbf{9} daz si im] [*me]: Daz simme V Das ym Q  $\cdot$ ir] \textit{om.} Q \textbf{10} manigem] manig U manigē V (W) (Q)  $\cdot$ über in] [* in]: ab im V  $\cdot$ klaget] geclaget V (W) (Q) (R) \textbf{12} der künec] artus W  $\cdot$ hân ich] vnd ich han R  $\cdot$ vernomen] [*nommen]: genommen V gnomen W (Q) \textbf{13} vriden] fride V (W) Q R \textbf{15} guot] guͦte V \textbf{16} Beakurs] Beakuͦrs U Beakurß W \textbf{17} zuo halbem] zvͦ halbē V zu halben Q zem halben R \textbf{18} varn] frilich farn R \textbf{19} des] Daz V  $\cdot$ endarf] darff W R \textbf{20} spehen] sehen V W (Q) (R) \textbf{21} urloube] vrlawben Q  $\cdot$ si] \textit{om.} W \textbf{22} der künec] Artus W  $\cdot$ aleine] eine V einer Q enig R  $\cdot$ dem] dē W Q den R \textbf{24} Roitschesabins] Botschesabins U [Roitschesabin*]: Roitschesabins V roytschesabins W roitschebins Q  $\cdot$ riten] [*]: rittent V \textbf{25} dâ daz] do daz V (W) (Q)  $\cdot$ dâ lac] do lag V (Q) lag W R \textbf{26} gelebte] gelete U gelaubte Q  $\cdot$ sô] so hohe Q \textbf{27} Gramoflanz] Gramaflancz V Gramoflantz W Q Gramoflancz R  $\cdot$ in gesprach] in gesach W zu In sprach R \textbf{29} solichiu] soͯmliche R \textbf{30} Der selde [*]: gegen im hette erdoht V  $\cdot$ in] hin Q \newline
\end{minipage}
\end{table}
\end{document}
