\documentclass[8pt,a4paper,notitlepage]{article}
\usepackage{fullpage}
\usepackage{ulem}
\usepackage{xltxtra}
\usepackage{datetime}
\renewcommand{\dateseparator}{.}
\dmyyyydate
\usepackage{fancyhdr}
\usepackage{ifthen}
\pagestyle{fancy}
\fancyhf{}
\renewcommand{\headrulewidth}{0pt}
\fancyfoot[L]{\ifthenelse{\value{page}=1}{\today, \currenttime{} Uhr}{}}
\begin{document}
\begin{table}[ht]
\begin{minipage}[t]{0.5\linewidth}
\small
\begin{center}*D
\end{center}
\begin{tabular}{rl}
\textbf{22} & daz er den prîs \textbf{über menegiu \textit{l}ant}\\ 
 & \textbf{hete} al eine \textbf{zuo sîner hant}."\\ 
 & "Nû \textbf{sich} êt, wenne \textbf{oder} wie,\\ 
 & \textbf{unt} vüege, daz er mich \textbf{gespreche} hie.\\ 
5 & \textbf{wir} haben \textbf{doch} vrid \textbf{al disen} tac.\\ 
 & dâ von der helt wol rîten mac\\ 
 & her ûf ze mir. oder sol ich dar?\\ 
 & er ist anders denne wir gevar.\\ 
 & \textbf{owî}, wan tæte im daz \textbf{niht} wê!\\ 
10 & daz \textbf{het} ich gerne \textbf{ervunden ê};\\ 
 & ob \textbf{mirz} die mîne rieten,\\ 
 & ich solt im êr bieten.\\ 
 & geruochet er mir nâhen,\\ 
 & wie sol ich in enpfâhen?\\ 
15 & ist er mir \textbf{dar zuo wol} geborn,\\ 
 & daz mîn kus \textbf{niht} sî verlorn?"\\ 
 & "vrouwe, er ist vür küneges künne erkant.\\ 
 & des sî mîn lîp genennet pfant.\\ 
 & \textbf{\begin{large}V\end{large}rouwe}, ich wil iwern vürsten sagen,\\ 
20 & daz si rîchiu kleider tragen\\ 
 & unt \textbf{daz si} \textbf{vor} iu bîten,\\ 
 & \textbf{unz} d\textit{az} wir zuo \textbf{ziu} rîten.\\ 
 & daz saget \textbf{ir} iweren vrouwen gar,\\ 
 & wan swenne ich nû hin \textbf{nider} var,\\ 
25 & sô bring ich \textbf{iu} den werden gast,\\ 
 & dem \textbf{süezer} tugende nie gebrast."\\ 
 & \textbf{harte} \textbf{wênic} des verdarp.\\ 
 & vil behendeclîchen warp\\ 
 & der marschalc sîner vrouwen bete.\\ 
30 & balde \textbf{wart dô} Gahmurete\\ 
\end{tabular}
\scriptsize
\line(1,0){75} \newline
D Fr9 Fr14 \newline
\line(1,0){75} \newline
\textbf{3} \textit{Initiale} Fr9   $\cdot$ \textit{Majuskel} D  \textbf{19} \textit{Initiale} D Fr14  \textbf{27} \textit{Initiale} Fr9  \newline
\line(1,0){75} \newline
\textbf{1} menegiu] manich Fr9  $\cdot$ lant] sant D \textbf{4} gespreche] spreche Fr9 Fr14 \textbf{5} wir haben doch] Wante wir han Fr9 \textbf{8} denne] [der]: den Fr9 \textbf{9} owî wan] Owe ne Fr9  $\cdot$ im] ẏn Fr9 \textbf{14} in enpfâhen] en phahen Fr14 \textbf{17} vür] von Fr9 \textbf{21} si] \textit{om.} Fr14 \textbf{22} daz] d:: D  $\cdot$ ziu] v̂ Fr9 \textbf{23} daz saget ir] Vnde saget iz ouch Fr9 \textbf{24} wan] \textit{om.} Fr9 \textbf{26} tugende] mẏnne Fr9 \textbf{30} Gahmurete] Gahmvrete D gamvrete Fr9 \newline
\end{minipage}
\hspace{0.5cm}
\begin{minipage}[t]{0.5\linewidth}
\small
\begin{center}*m
\end{center}
\begin{tabular}{rl}
 & daz er den prîs \textbf{über menic lant}\\ 
 & \textbf{hette} aleine \textbf{zuo sîner hant}."\\ 
 & "\textit{\begin{large}N\end{large}}û \textbf{s\textit{e}ht} e\textit{ht}, wenne \textbf{oder} wie,\\ 
 & \dag umb vienge\dag , daz e\textit{r} mich \dag sprache\dag  hie\\ 
5 & \textbf{und} haben vride \textbf{allen} tac.\\ 
 & dâ von der helt wol rîten mac\\ 
 & her ûf zuo mir. oder sol ich dar?\\ 
 & e\textit{r} ist anders denne wir gevar.\\ 
 & \textbf{und} wenne tæte im daz \textbf{noch} wê?\\ 
10 & daz \textbf{hœre} ich gerne \textbf{von min\textit{n}en ê};\\ 
 & ob \textbf{mirz} die mî\textit{n}en rieten,\\ 
 & ich solte ime êre bieten.\\ 
 & geruochet er mir nâhe\textit{n},\\ 
 & wie sol ich in enpfâhen?\\ 
15 & ist er mir \textbf{wol dar zuo} geborn,\\ 
 & daz mîn kus \textbf{niht} sî verlorn?"\\ 
 & "vrowe, er ist vür küniges kün\textit{n}e erkant.\\ 
 & des sî mîn lîp genennet pfant.\\ 
 & ich wil i\textit{uwe}ren vürsten sagen,\\ 
20 & daz si rîch\textit{iu} kleider tragen\\ 
 & und \textbf{daz si} \textbf{vor} i\textit{u} bîten,\\ 
 & \textbf{unz} daz wir zuo \textbf{iu zuo} rîten.\\ 
 & daz saget \textbf{ir} iuweren vrouwen gar,\\ 
 & \textit{w}enne wenn ich nû hin \textbf{wider} var,\\ 
25 & sô bringe ich den werden gast,\\ 
 & dem \textbf{süezer} tugent \textit{n}ie gebrast."\\ 
 & \textbf{harte} \textbf{wênic} des verdarp.\\ 
 & vil behendeclîchen warp\\ 
 & der marschalc sîner vrouwen bete.\\ 
30 & balde \textbf{wurden} Gahmurete\\ 
\end{tabular}
\scriptsize
\line(1,0){75} \newline
m n o \newline
\line(1,0){75} \newline
\textbf{3} \textit{Initiale} m   $\cdot$ \textit{Capitulumzeichen} n  \newline
\line(1,0){75} \newline
\textbf{2} aleine] alle in o \textbf{3} Nû] Nnu m  $\cdot$ seht eht] sichtes m \textbf{4} umb vienge] Jm singe o  $\cdot$ er] es m  $\cdot$ sprache] sprach n o \textbf{5} vride] friden m n freiuͦden o \textbf{8} er] Es m n o  $\cdot$ anders] [andern]: anders o \textbf{9} noch] \textit{om.} o \textbf{10} minnen] mÿnen \textit{nachträglich korrigiert zu:} mÿnnen m \textbf{11} mînen] mÿen \textit{nachträglich korrigiert zu:} mÿnen m iemann o \textbf{13} nâhen] nahet \textit{nachträglich korrigiert zu:} nahen m \textbf{15} wol dar zuo] dar zuͦ wal n (o) \textbf{16} mîn kus] ẏm kusch o \textbf{17} vür] von n o  $\cdot$ künne] kunige \textit{nachträglich korrigiert zu:} kunne m \textbf{19} \textit{Die Verse 22.19-23.6 fehlen} o   $\cdot$ iuweren] iren m n \textbf{20} rîchiu] richen m \textbf{21} iu] ir m n \textbf{22} zuo rîten] riten n \textbf{23} ir] \textit{om.} n  $\cdot$ iuweren] uwer m n \textbf{24} wenne] Nenne m  $\cdot$ nû] \textit{om.} n \textbf{26} nie] ÿe m \textbf{27} des] das n \textbf{28} warp] [wart]: warp m \textbf{30} Gahmurete] [Gahmuret]: Gahmurette m gamúret n \newline
\end{minipage}
\end{table}
\newpage
\begin{table}[ht]
\begin{minipage}[t]{0.5\linewidth}
\small
\begin{center}*G
\end{center}
\begin{tabular}{rl}
 & daz er den prîs \textbf{ze sîner hant}\\ 
 & \textbf{hât} al eine \textbf{über manigiu lant}."\\ 
 & "nû \textbf{sich} êt, wenne \textbf{und} wie,\\ 
 & vüege, daz er mich \textbf{spreche} hie.\\ 
5 & \textbf{wir} hân \textbf{doch} vride \textbf{disen} tac.\\ 
 & dâ von der helt wol rîten mac\\ 
 & \begin{large}H\end{large}er ûf ze mir. oder sol ich dar?\\ 
 & er ist anders danne wir gevar.\\ 
 & \textbf{owê}, wan tæte im daz \textbf{niht} wê!\\ 
10 & daz \textbf{hete} ich gerne \textbf{ervunden ê};\\ 
 & op \textbf{mirz} die mîne rieten,\\ 
 & ich solt im êre bieten.\\ 
 & geruochet er mir nâhen,\\ 
 & wie sol ich in enpfâhen?\\ 
15 & ist er mir \textbf{dar zuo wol} geboren,\\ 
 & daz mîn kus \textbf{iht} sî verloren?"\\ 
 & "vrouwe, erst vür küniges künne erkant.\\ 
 & des sî mîn lîp genennet pfant.\\ 
 & \textbf{vrouwe}, ich wil iweren vürsten sagen,\\ 
20 & daz si rîchiu kleider tragen\\ 
 & unde \textbf{hie} \textbf{vor} iu bîten,\\ 
 & \textbf{biz} daz wir zuo rîten.\\ 
 & daz saget \textbf{ouch} iweren vrouwen gar,\\ 
 & wan swenne ich nû hin \textbf{nider} var,\\ 
25 & sô bringe ich \textbf{iu} den werden gast,\\ 
 & dem \textbf{ganzer} tugende nie gebrast."\\ 
 & \textbf{dâr an ouch} \textbf{lützel} des verdarp.\\ 
 & vil behendeclîchen warp\\ 
 & der marschalc sîner vrouwen bete.\\ 
30 & balde \textbf{wart dô} Gahmurete\\ 
\end{tabular}
\scriptsize
\line(1,0){75} \newline
G O L M Q R W Z Fr29 Fr32 Fr36 Fr55 Fr71 \newline
\line(1,0){75} \newline
\textbf{1} \textit{Initiale} O Fr29  \textbf{3} \textit{Initiale} M Fr71  \textbf{7} \textit{Initiale} G  \textbf{9} \textit{Versal} Fr32  \textbf{17} \textit{Initiale} R  \textbf{19} \textit{Initiale} L W Z Fr32 Fr71  \textbf{30} \textit{Versal} Fr32  \newline
\line(1,0){75} \newline
\textbf{1} ze sîner hant] vber manich lant O (L) (M) (W) vber manche lant Q (Z) (Fr29) (Fr32) úber allu land R \textbf{2} hât] Hette M  $\cdot$ über manigiu lant] ze siner hant O (L) (M) (Q) (R) (Fr29) Fr32 mit siner hant Z in seiner hant W \textbf{3} sich] [sch]: sich O sehet Z  $\cdot$ êt] ouch M \textit{om.} Q Z  $\cdot$ und] oder O L (M) (Q) Z (Fr29) Fr32 (Fr71) \textbf{4} vüege] Vnd fvͦg O (L) (M) (Q) (R) (Z) (Fr29) (Fr32) (Fr55) dv fvͦgest Fr71  $\cdot$ mich] mir R  $\cdot$ spreche] gespreche O L (Fr71) bespreche M Z  $\cdot$ hie] zu R \textbf{5} disen] allen disen O (Q) (W) Z Fr29 (Fr32) \textbf{7} Wil er har zuͦ mir oder sol ich dar W  $\cdot$ sol] \textit{om.} O \textbf{8} danne] wanne M  $\cdot$ gevar] gestalt M \textbf{9} owê] Owi O Fr29 (Fr55) Awe Q  $\cdot$ wan] vnd Q \textbf{10} daz hete ich gerne] Vnd het ich dasz Q daz het gerne Fr71  $\cdot$ ervunden] befunden R \textbf{11} mîne rieten] mýnnen rieten L (R) (Z) mynne rite Q \textbf{12} êre] \textit{om.} M  $\cdot$ bieten] irbiten M (Q) (R) \textbf{13} geruochet] Gervcket Z  $\cdot$ er] der M  $\cdot$ nâhen] genahen W \textbf{14} sol] solt W \textbf{15} ist er] Osz der M Wer er Z  $\cdot$ dar] danne dar Fr32  $\cdot$ geboren] erborn R \textbf{16} kus] kost M chvzzen Fr71  $\cdot$ iht sî] ist niht O niht sý L (M) (R) (Fr32) \textbf{17} vrouwe] \textit{om.} L W  $\cdot$ erst] er ist mir O  $\cdot$ vür küniges künne] von kvniges fruͯcht L (W) vor konniges kint M fur funges kunigk \textit{nachträglich korrigiert zu} fur kunges kunigk Q vuͦr kvneges art Fr32 von chvniges chvnn Fr71  $\cdot$ erkant] erkorn Fr32 \textbf{18} genennet] benemment L fur in ein W gein ev ein Z \textbf{19} vrouwe] Czwore Q  $\cdot$ wil] \textit{om.} Z \textbf{20} si] sy ir W \textit{om.} Fr29  $\cdot$ rîchiu kleider] rechte cleynote Q \textbf{21} hie vor iu] vor uͯch hie L  $\cdot$ bîten] peyden Q \textbf{22} biz] Vntz L W (Fr29) (Fr55)  $\cdot$ wir] wir er Q  $\cdot$ zuo] zv iv O (M) (W) (Z) (Fr29) (Fr32) zu im R (Fr71) \textbf{23} ouch] \textit{om.} M Fr32 \textbf{24} wan swenne] Wenne L (Q) (W) Wan wanne M (R)  $\cdot$ nû] \textit{om.} O (R)  $\cdot$ nider] wider Q (R) \textbf{25} werden] selben Fr32 \textbf{26} ganzer] svͦzer O (L) (M) (R) (W) (Z) (Fr29) (Fr32) doch nye Q  $\cdot$ tugende] tugenden Z  $\cdot$ nie gebrast] gebrach Q \textbf{27} dâr an ouch] Dar an doch O (M) Fr29 Fr36 Der rede L W Harte Q (R) Z (Fr32) Dar an Fr71  $\cdot$ lützel] lúczes R wenic Z  $\cdot$ des] \textit{om.} L do W \textbf{28} behendeclîchen] behendenklich er R (Fr32) (Fr71) Wann behendeckliche W \textbf{29} bete] bat M \textbf{30} wart] wurden W  $\cdot$ dô] da M R Z  $\cdot$ Gahmurete] Gamvret O Gahmvret L Fr36 gammerat M gamúert Q Gahmuret R gamurete W gamuret Z Ga:::t Fr29 gamvͦrete Fr32 Gahmvrete Fr71 \newline
\end{minipage}
\hspace{0.5cm}
\begin{minipage}[t]{0.5\linewidth}
\small
\begin{center}*T
\end{center}
\begin{tabular}{rl}
 & daz er den prîs \textbf{über elliu lant}\\ 
 & \textbf{hete} aleine \textbf{ze sîner hant}."\\ 
 & "\begin{large}N\end{large}û \textbf{sich} eht, wenne \textbf{oder} wie,\\ 
 & \textbf{und} vüege, daz er mich \textbf{gespreche} hie.\\ 
5 & \textbf{wir} hân \textbf{doch} vride \textbf{disen} tac.\\ 
 & dâ von der helt wol rîten mac\\ 
 & her ûf ze mir. oder sol ich dar?\\ 
 & er ist anders danne wir gevar.\\ 
 & \textbf{ouwê}, wan tæt im daz \textbf{niht} wê!\\ 
10 & daz \textbf{het} ich gerne \textbf{ervunden ê};\\ 
 & \hspace*{-.7em}\big| ich solt im êre bieten,\\ 
 & \hspace*{-.7em}\big| ob \textbf{ez mir} die mîne rieten.\\ 
 & geruochet er mir nâhen,\\ 
 & wie sol ich in enpfâhen?\\ 
15 & ist er mir \textbf{dar zuo wol} geborn,\\ 
 & daz mîn kus \textbf{iht} sî verlorn?"\\ 
 & "Vrouwe, er ist vür küneges künne erkant.\\ 
 & des sî mîn lîp genennet pfant.\\ 
 & ich wil iuwern vürsten sagen,\\ 
20 & daz si rîch\textit{iu} kleider tragen\\ 
 & und \textbf{hie} \textbf{bî} iu bîten,\\ 
 & \textbf{biz} daz wir zuo \textbf{iu} rîten.\\ 
 & daz saget \textbf{ouch} iuwern vrouwen gar,\\ 
 & wan swenn ich nû hin \textbf{nider} var,\\ 
25 & sô bring ich \textbf{iu} den werden gast,\\ 
 & dem \textbf{süezer} tugende nie gebrast."\\ 
 & \textbf{\begin{large}D\end{large}âr an dô} \textbf{lützel} des verdarp.\\ 
 & vil behendeclîche \textbf{dô} warp\\ 
 & der marschalc sîner vrouwen bete.\\ 
30 & balde \textbf{wart dô} Gahmurete\\ 
\end{tabular}
\scriptsize
\line(1,0){75} \newline
T U V \newline
\line(1,0){75} \newline
\textbf{3} \textit{Initiale} T  \textbf{17} \textit{Majuskel} T  \textbf{27} \textit{Initiale} T U V  \newline
\line(1,0){75} \newline
\textbf{1} elliu] [*]: manig V \textbf{4} vüege] gevuͦge U \textbf{5} vride] vriden U  $\cdot$ disen] al disen U V \textbf{11} ez mir] mirs U V  $\cdot$ mîne] minen V \textbf{15} ist er] Er ist V \textbf{16} iht] nit U (V) \textbf{17} künne] luͦnne U [*int]: kint V \textbf{18} genennet pfant] [g*pf*]: fv́r in ein pfant V \textbf{20} rîchiu] riche T ir riche V \textbf{21} bî] [in]: bi U  $\cdot$ bîten] beidin U \textbf{24} swenn] so U \textbf{25} bring ich] bringin U \textbf{27} Dâr an] [Da*]: Der rede V  $\cdot$ dô lützel des] lutzel [*]: da V \textbf{28} behendeclîche] bescheidenliche V \textbf{30} wart dô] do wart U  $\cdot$ Gahmurete] Gahmvrete T Gahmuͦret U Gamurette V \newline
\end{minipage}
\end{table}
\end{document}
