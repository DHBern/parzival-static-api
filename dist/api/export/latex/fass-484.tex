\documentclass[8pt,a4paper,notitlepage]{article}
\usepackage{fullpage}
\usepackage{ulem}
\usepackage{xltxtra}
\usepackage{datetime}
\renewcommand{\dateseparator}{.}
\dmyyyydate
\usepackage{fancyhdr}
\usepackage{ifthen}
\pagestyle{fancy}
\fancyhf{}
\renewcommand{\headrulewidth}{0pt}
\fancyfoot[L]{\ifthenelse{\value{page}=1}{\today, \currenttime{} Uhr}{}}
\begin{document}
\begin{table}[ht]
\begin{minipage}[t]{0.5\linewidth}
\small
\begin{center}*D
\end{center}
\begin{tabular}{rl}
\textbf{484} & \begin{large}V\end{large}râgt er niht bî der êrsten naht,\\ 
 & sô zergêt sîner vrâge maht.\\ 
 & wirt sîn vrâge an rehter zît getân,\\ 
 & sô sol erz künicrîche hân\\ 
5 & und hât der kumber ende\\ 
 & von der hœhesten hende.\\ 
 & dâ mit ist Anfortas genesen,\\ 
 & er\textbf{n} sol aber \textbf{niemer} künec wesen.'\\ 
 & sus lâsen wir ame Grâle,\\ 
10 & daz Anfortases quâle\\ 
 & dâ mit ein ende næme,\\ 
 & swenne im diu vrâge kæme.\\ 
 & wir strichen an die wunden,\\ 
 & swâ mit wir senften kunden:\\ 
15 & die guoten salben nardas\\ 
 & \textbf{unt} swaz gedrîakelt was\\ 
 & unt den rouch \textbf{von} lignâlôe.\\ 
 & im was êt \textbf{zallen zîten} wê.\\ 
 & dô zôch ich mich dâ her;\\ 
20 & swachiu wünne ist mîner jâre wer.\\ 
 & sît kom ein rîter dar geriten\\ 
 & - \textbf{der} m\textit{ö}htez \textbf{gerne} hân vermiten -,\\ 
 & von dem ich dir \textbf{ê} sagte.\\ 
 & unprîs \textbf{der} dâ bejagte,\\ 
25 & sît er den rehten \textbf{kumber} sach,\\ 
 & daz er niht zuo dem wirte sprach:\\ 
 & 'hêrre, wie stêt iwer nôt?'\\ 
 & sît im sîn tumpheit daz gebôt,\\ 
 & daz er \textbf{al} dâ niht vrâgte,\\ 
30 & grôzer sælde in dô betrâgte."\\ 
\end{tabular}
\scriptsize
\line(1,0){75} \newline
D \newline
\line(1,0){75} \newline
\textbf{1} \textit{Initiale} D  \newline
\line(1,0){75} \newline
\textbf{22} möhtez] mohtez D \newline
\end{minipage}
\hspace{0.5cm}
\begin{minipage}[t]{0.5\linewidth}
\small
\begin{center}*m
\end{center}
\begin{tabular}{rl}
 & vrâgt er niht bî der êrsten naht,\\ 
 & sô zergât sîner vrâge maht.\\ 
 & wirt sîn vrâge an rehter zît getân,\\ 
 & sô sol er daz künicrîche hân\\ 
5 & und het der ku\textit{mber} ende\\ 
 & von der hœhesten hende.\\ 
 & dâ mit ist Anfortas genesen,\\ 
 & er sol aber \textbf{niht mê} künic wesen.'\\ 
 & sus lâsen wir an dem Grâl,\\ 
10 & daz Anfortases quâl\\ 
 & dâ mit ein ende næme,\\ 
 & wan im diu vrâge kæme.\\ 
 & wir strichen an die wunden,\\ 
 & wâ mit wir senften kunden:\\ 
15 & die guoten salben nard\textit{a}s,\\ 
 & \textbf{daz}, waz getrîakelt was,\\ 
 & und den rouch \textbf{von} lingnum âlôe.\\ 
 & im was eht \textbf{zuo allen zîten} wê.\\ 
 & dô zôch ich mich dâ her;\\ 
20 & swachiu wünne ist mîner jâre wer.\\ 
 & sît kam ein ritter dar geriten\\ 
 & - \textbf{der} m\textit{ö}ht ez \textbf{gern} hân vermiten -,\\ 
 & von dem ich dir \textbf{sô} saget.\\ 
 & unprîs \textbf{er} d\textit{â} bejaget,\\ 
25 & sît er den rehten \textbf{kumber} sach,\\ 
 & daz er niht zuo dem wirte sprach:\\ 
 & 'hêrre, wie stât iuwer nôt?'\\ 
 & sît im sîn tumpheit daz gebôt,\\ 
 & daz er \textbf{in} d\textit{â} niht vrâgete,\\ 
30 & grôzer sælde in dâ betrâgete."\\ 
\end{tabular}
\scriptsize
\line(1,0){75} \newline
m n o \newline
\line(1,0){75} \newline
\newline
\line(1,0){75} \newline
\textbf{5} kumber] kunig m \textbf{7} Anfortas] afortas o \textbf{10} Anfortases] anfortas m n er anfortas o \textbf{12} vrâge] frogte o \textbf{15} guoten] gute o  $\cdot$ salben] salbe n o  $\cdot$ nardas] nardis m \textbf{16} daz] Vnd n o \textbf{19} \textit{Versdoppelung 484.19-20 (²o) nach 484.20; Lesarten der vorausgehenden Verse mit ¹o bezeichnet} o   $\cdot$ ich] [zo]: ich \textsuperscript{2}\hspace{-1.3mm} o \textbf{21} kam] kan o \textbf{22} möht] moht m (o)  $\cdot$ gern] gerner n (o) \textbf{23} sô] do n e von o \textbf{24} er] >er< o  $\cdot$ dâ] do m n \textbf{25} kumber] [komen]: kommer o \textbf{29} dâ] do m n o \textbf{30} dâ] do n o \newline
\end{minipage}
\end{table}
\newpage
\begin{table}[ht]
\begin{minipage}[t]{0.5\linewidth}
\small
\begin{center}*G
\end{center}
\begin{tabular}{rl}
 & \textit{\begin{large}V\end{large}}\textit{r}âget er niht bî der êrsten naht,\\ 
 & sô zergêt sîner vrâge maht.\\ 
 & wirt sîn vrâge an rehter zît getân,\\ 
 & sô sol erz künicrîche hân\\ 
5 & unt hât der kumber ende\\ 
 & von der hœhesten hende.\\ 
 & dâ mit ist Anfortas genesen,\\ 
 & er\textbf{n} sol aber \textbf{nimer} künic wesen.'\\ 
 & sus lâsen wir ame Grâle,\\ 
10 & daz Anfortases quâle\\ 
 & dâ mit ein ende næme,\\ 
 & swenne im \textit{diu} vrâge kæme.\\ 
 & wir strichen an die wunden,\\ 
 & swâ mit wir senften kunden:\\ 
15 & die guoten salben nardas\\ 
 & \textbf{unde} swaz gedrîakelt was\\ 
 & unt den rouch \textbf{von} lignâlôe.\\ 
 & im was êt \textbf{ze allen zîten} wê.\\ 
 & dô zôch ich mich dâ her;\\ 
20 & swachiu wünne ist mîner jâre wer.\\ 
 & sît kom ein rîter dar geriten\\ 
 & - \textbf{der} m\textit{ö}htez \textbf{gerner} hân vermiten -,\\ 
 & von dem ich dir \textbf{ê} sagete.\\ 
 & unprîs \textbf{er} d\textit{â} bejagete,\\ 
25 & sît er den rehten \textbf{kumber} sach,\\ 
 & daz er niht zem wirte sprac\textit{h}:\\ 
 & 'hêrre, wie stêt iuwer nôt?'\\ 
 & sît im sîn tumpheit daz gebôt,\\ 
 & daz er \textbf{al} dâ niht vrâgete,\\ 
30 & grôzer sælden in dâ betrâgete."\\ 
\end{tabular}
\scriptsize
\line(1,0){75} \newline
G I O L M Z \newline
\line(1,0){75} \newline
\textbf{1} \textit{Initiale} G I O L Z  \textbf{17} \textit{Initiale} I  \newline
\line(1,0){75} \newline
\textbf{1} Vrâget] Sagit G ÷ragt O Fragete M  $\cdot$ niht] icht M \textbf{2} vrâge] [vreuden]: vrage I sorgen O  $\cdot$ maht] magt Z \textbf{3} wirt] Wert M  $\cdot$ an rehter] zerehter O (Z) \textbf{7} ist] \textit{om.} I  $\cdot$ Anfortas] amfortas L \textbf{8} ern] Er O  $\cdot$ nimer] niht mer O (L) (M) Z \textbf{9} \textit{Die Verse 484.9-12 fehlen} M   $\cdot$ lâsen] las Z \textbf{10} Anfortases] anfortas G (I) (O) Z [Amfortas]: Amfortases L \textbf{12} Swenne] [Sv*]: Swenne G Wanne L  $\cdot$ diu] \textit{om.} G \textbf{14} swâ] Wa L (M) \textbf{15} salben] salbe L \textbf{16} swaz] waz L (M)  $\cdot$ gedrîakelt] getriachet O getiriackelt L \textbf{17} den] de M  $\cdot$ lignâlôe] lignit aloe I \textbf{19} dô] Da M Z \textbf{20} swachiu] Was M  $\cdot$ wer] ger I \textbf{21} kom ein] kommen M \textbf{22} möhtez] mohtiz G (I) (L) (M)  $\cdot$ gerner] gerne O L (M) Z \textbf{24} er] der O L M Z  $\cdot$ dâ] dran G da er da I \textbf{25} er] irsz M \textbf{26} sprach] sprac G \textbf{29} al] \textit{om.} I O  $\cdot$ vrâgete] enfragte O \textbf{30} grôzer sælden] Groze selde M Grozzer selde Z  $\cdot$ dâ] \textit{om.} I O L \newline
\end{minipage}
\hspace{0.5cm}
\begin{minipage}[t]{0.5\linewidth}
\small
\begin{center}*T
\end{center}
\begin{tabular}{rl}
 & vrâget er niht bî der êrsten naht,\\ 
 & sô zergât sîner vrâge maht.\\ 
 & wirt sîn vrâge an rehter zît getân,\\ 
 & sô sol er daz künecrîche hân\\ 
5 & unde hât der kumber ende\\ 
 & von der hœhesten hende.\\ 
 & dâ mit ist Anfortas genesen,\\ 
 & er sol aber \textbf{niht mêr} künec wesen.'\\ 
 & Sus lâse wir an dem Grâle,\\ 
10 & daz Anfortasses quâle\\ 
 & dâ mit ein ende næme,\\ 
 & swennim diu vrâge kæme.\\ 
 & wir strichen an die wunden,\\ 
 & swâ mite wir senften kunden:\\ 
15 & die guoten salben nardas\\ 
 & \textbf{unde} swaz getrîakelt was\\ 
 & unde den rouch lingnâlôe.\\ 
 & im was eht \textbf{allenthalben} wê.\\ 
 & dô zôch ich mich dâ her;\\ 
20 & swach\textit{iu} wünne ist mîner jâre wer.\\ 
 & Sît kom ein rîter dar geriten\\ 
 & - \textbf{er} m\textit{ö}htez \textbf{gerner} hân vermiten -,\\ 
 & von dem ich dir \textbf{ê} sagete.\\ 
 & unprîs \textbf{er} dâ bejagete,\\ 
25 & sît er den rehten \textbf{jâmer} sach,\\ 
 & daz er niht zuo dem wirte sprach:\\ 
 & 'hêrre, wie stât iuwer nôt?'\\ 
 & sît im sîn tumpheit daz gebôt,\\ 
 & daz er dâ niht \textbf{en}vrâgete,\\ 
30 & grôzer sælden in dô betrâgete."\\ 
\end{tabular}
\scriptsize
\line(1,0){75} \newline
T U V W Q R \newline
\line(1,0){75} \newline
\textbf{1} \textit{Initiale} W  \textbf{9} \textit{Majuskel} T  \textbf{21} \textit{Majuskel} T  \newline
\line(1,0){75} \newline
\textbf{1} \textit{Die Verse 453.1-502.30 fehlen} U   $\cdot$ vrâget] [Vragete]: Vraget V  $\cdot$ er] ir Q  $\cdot$ bî der] die W \textbf{2} sîner] mer Q \textbf{4} er daz künecrîche] es kumenriche R \textbf{5} der] er Q  $\cdot$ ende] ein ende W \textbf{7} Anfortas] anfrotas R \textbf{8} er] Ern Q \textbf{9} lâse] lasz Q \textbf{10} Anfortasses] anfortas W R \textbf{12} swennim] Wenn im W (Q) Wenn R  $\cdot$ kæme] also quaͯme R \textbf{13} strichen] streichen Q \textbf{14} swâ] wo W (Q) (R)  $\cdot$ senften] sy senfftten R \textbf{15} salben] selben V \textbf{16} swaz] was W Q R  $\cdot$ getrîakelt] getrisselt W triakers Q triakelt R \textbf{17} lingnâlôe] [*]: von lignvm aloe V von ligni aloe W von Lignaloe Q von holcz aloie R \textbf{18} eht] auch Q \textit{om.} R  $\cdot$ allenthalben wê] [*]: zvͦ allen ziten we V vmenthalben we R \textbf{20} swachiu] swache T Wache Q \textbf{22} er möhtez] er mohtez T Der moht ez V (Q) (R) Der moͤcht er W  $\cdot$ gerner] gerne V W Q (R) \textbf{23} ê] ie W \textbf{24} er] der W Q  $\cdot$ dâ] do V W Q \textbf{25} jâmer] kvmber V (W) (Q) (R) \textbf{26} er] [es]: er Q  $\cdot$ wirte] ritter Q \textbf{27} stât] stet es W \textbf{29} dâ] do V W Q R  $\cdot$ envrâgete] fragte W (Q) (R) \textbf{30} sælden] selde Q R  $\cdot$ dô] \textit{om.} R \newline
\end{minipage}
\end{table}
\end{document}
