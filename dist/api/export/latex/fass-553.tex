\documentclass[8pt,a4paper,notitlepage]{article}
\usepackage{fullpage}
\usepackage{ulem}
\usepackage{xltxtra}
\usepackage{datetime}
\renewcommand{\dateseparator}{.}
\dmyyyydate
\usepackage{fancyhdr}
\usepackage{ifthen}
\pagestyle{fancy}
\fancyhf{}
\renewcommand{\headrulewidth}{0pt}
\fancyfoot[L]{\ifthenelse{\value{page}=1}{\today, \currenttime{} Uhr}{}}
\begin{document}
\begin{table}[ht]
\begin{minipage}[t]{0.5\linewidth}
\small
\begin{center}*D
\end{center}
\begin{tabular}{rl}
\textbf{553} & \begin{large}G\end{large}rôz müede im zôch diu ougen zuo;\\ 
 & \textbf{sus} slief er unze des morgens vruo.\\ 
 & \multicolumn{1}{l}{ - - - }\\ 
 & \multicolumn{1}{l}{ - - - }\\ 
 & \multicolumn{1}{l}{ - - - }\\ 
 & \multicolumn{1}{l}{ - - - }\\ 
 & \multicolumn{1}{l}{ - - - }\\ 
 & \multicolumn{1}{l}{ - - - }\\ 
 & \multicolumn{1}{l}{ - - - }\\ 
 & \multicolumn{1}{l}{ - - - }\\ 
 & \multicolumn{1}{l}{ - - - }\\ 
 & \multicolumn{1}{l}{ - - - }\\ 
 & \multicolumn{1}{l}{ - - - }\\ 
 & \multicolumn{1}{l}{ - - - }\\ 
 & dô \textbf{erwachete} der wîgant.\\ 
 & einhalp der kemenâten want\\ 
5 & vil venster \textbf{hete}, dâ vor glas.\\ 
 & der venster einez offen was\\ 
 & gegen dem boumgarten.\\ 
 & dar în gie er durch warten,\\ 
 & durch luft und durch \textbf{der} vogel \textbf{sanc}.\\ 
10 & sîn sitzen \textbf{wart dâ niht ze lanc}.\\ 
 & Er kôs eine burc, die er des âbents sach,\\ 
 & dô im diu âventiur geschach,\\ 
 & vil vrouwen ûf dem palas,\\ 
 & manegiu under in vil schœne was.\\ 
15 & \textbf{ez} dûht in ein wunder grôz,\\ 
 & daz die vrouwen niht verdrôz\\ 
 & ir wachens, daz si sliefen niht.\\ 
 & dennoch der tac was niht \textbf{ze} lieht.\\ 
 & er dâhte: "ich wil in zêren\\ 
20 & \textbf{mich an slâfen} kêren."\\ 
 & Wider an sîn bette er gienc.\\ 
 & der meide mantel \textbf{übervienc}\\ 
 & in - daz was sîn decke.\\ 
 & ob man in dâ iht wecke?\\ 
25 & nein, daz wære dem wirte leit.\\ 
 & diu magt durch gesellecheit,\\ 
 & al dâ si vor ir muoter lac,\\ 
 & \textbf{si} brach ir slâf, des si pflac,\\ 
 & unt gienc hin \textbf{ûf} z\textbf{ir} gaste;\\ 
30 & der slief dennoch al vaste.\\ 
\end{tabular}
\scriptsize
\line(1,0){75} \newline
D Fr7 \newline
\line(1,0){75} \newline
\textbf{1} \textit{Überschrift:} Die Auentivre von schastel marvelle Fr7  · Großinitiale D Fr7  \textbf{11} \textit{Majuskel} D  \textbf{21} \textit{Majuskel} D  \newline
\line(1,0){75} \newline
\textbf{3} erwachete] erwachet Fr7 \textbf{5} glas] mit glase Fr7 \textbf{6} was] was: Fr7 \textbf{9} der] den Fr7 \textbf{18} ze lieht] so liht Fr7 \textbf{29} ûf] \textit{om.} Fr7 \textbf{30} al] also Fr7 \newline
\end{minipage}
\hspace{0.5cm}
\begin{minipage}[t]{0.5\linewidth}
\small
\begin{center}*m
\end{center}
\begin{tabular}{rl}
 & grôz müede im zôch diu ougen zuo;\\ 
 & \textbf{sus} slief er unz des morgens vruo.\\ 
 & \begin{large}D\end{large}er nû welle, der verneme,\\ 
 & ob im sîn muot \textit{des} gesteme.\\ 
 & hie slîchet ein âventiur her\\ 
 & - des bin ich Gawanes wer -,\\ 
 & die brüefet man zuo soliche\textit{r} \textit{n}ôt,\\ 
 & d\textit{er} niht glîchet wan der tôt.\\ 
 & si pfliget angestlîcher sit,\\ 
 & doch vert d\textit{â} prîs und êre mit,\\ 
 & wem aldâ gelinget;\\ 
 & dar nâch si vröude bringet.\\ 
 & nû mî\textit{n} hêr Gawan gepflac\\ 
 & guoter ruowe unz an den tac.\\ 
 & dô \textbf{erwa\textit{c}hte} der wîgant.\\ 
 & einhalp der kemenâten want\\ 
5 & vil venster \textbf{het}, dâ v\textit{o}r glas.\\ 
 & der venster einez offen was\\ 
 & gegen dem boumgarten.\\ 
 & dar în gienc er durch warten,\\ 
 & durch luft und durch \textbf{den} vogel \textbf{gesanc}.\\ 
10 & sîn sitzen \textbf{wart dô niht zuo lanc}.\\ 
 & er kôs ein burc, die er des âbents sach,\\ 
 & dô i\textit{m} diu âventiur geschach,\\ 
 & vil vrouwen ûf dem palas,\\ 
 & manigiu under in vil schœne was.\\ 
15 & \textbf{ez} dûht in ein wunder grôz,\\ 
 & daz die vrouwen niht verdrôz\\ 
 & ir wac\textit{h}ens, daz si sliefen niht.\\ 
 & dannoch der tac was niht \textbf{zuo} lieht.\\ 
 & er dâht: "ich wil in zuo êren\\ 
20 & \textbf{von hinnen} kêren."\\ 
 & wider an sîn bet er gienc.\\ 
 & der megde mantel \textbf{übervienc}\\ 
 & in - daz was sîn decke.\\ 
 & ob man in d\textit{â} iht wecke?\\ 
25 & nein, daz wær dem wirte leit.\\ 
 & diu maget durch gesellicheit,\\ 
 & aldâ si vor ir muoter lac,\\ 
 & \dag brâht\dag  ir slâf, des si \textbf{ê} pflac,\\ 
 & und gienc hin zuo \textbf{dem} gaste;\\ 
30 & der slief dannoch al vaste.\\ 
\end{tabular}
\scriptsize
\line(1,0){75} \newline
m n o \newline
\line(1,0){75} \newline
\textbf{2} \textit{Illustration mit Überschrift:} Aufentuͯrr von schachttel marfeilie m  Also her gawan ging in dem boumgarten spatzieren n (o)   $\cdot$ \textit{Großinitiale} n   $\cdot$ \textit{Initiale} m o  \newline
\line(1,0){75} \newline
\textbf{2} des] das o \textbf{2} Der] Er o \textbf{2} des] \textit{om.} \textit{(krit. Text emendiert nach V#'* ͫ)} m n o \textbf{2} des] Das o \textbf{2} die] Do o  $\cdot$ solicher nôt] sollicher art not m \textbf{2} der] Die \textit{(krit. Text emendiert nach V#'* ͫ)} m n o  $\cdot$ wan] ram o \textbf{2} si] So o \textbf{2} dâ] do m n o \textbf{2} aldâ] >al< da o \textbf{2} mîn] mim m  $\cdot$ hêr] herre her n \textbf{3} erwachte] erworhtte m \textbf{5} vor glas] verglas m (n) (o) \textbf{8} er] \textit{om.} n \textbf{11} kôs] kose n \textbf{12} im] in m \textbf{14} manigiu] Maniger n \textbf{15} dûht] duͯcht o \textbf{16} daz] [Die]: Dasz o \textbf{17} wachens] wachssens m  $\cdot$ daz] [dies]: das o \textbf{18} lieht] licht n \textbf{20} von] Ouch von n \textbf{24} dâ iht] do iht m (o) icht do n \textbf{27} aldâ] Aldo do n \textbf{28} ir slâf] der slag o \textbf{30} al] also n \newline
\end{minipage}
\end{table}
\newpage
\begin{table}[ht]
\begin{minipage}[t]{0.5\linewidth}
\small
\begin{center}*G
\end{center}
\begin{tabular}{rl}
 & \begin{large}G\end{large}rôz müede im zôch diu ougen zuo;\\ 
 & \textbf{sus} slief er unze des morgens vruo.\\ 
 & \multicolumn{1}{l}{ - - - }\\ 
 & \multicolumn{1}{l}{ - - - }\\ 
 & \multicolumn{1}{l}{ - - - }\\ 
 & \multicolumn{1}{l}{ - - - }\\ 
 & \multicolumn{1}{l}{ - - - }\\ 
 & \multicolumn{1}{l}{ - - - }\\ 
 & \multicolumn{1}{l}{ - - - }\\ 
 & \multicolumn{1}{l}{ - - - }\\ 
 & \multicolumn{1}{l}{ - - - }\\ 
 & \multicolumn{1}{l}{ - - - }\\ 
 & \multicolumn{1}{l}{ - - - }\\ 
 & \multicolumn{1}{l}{ - - - }\\ 
 & dô \textbf{entwachete} der wîgant.\\ 
 & einhalp der kemenâten want\\ 
5 & vil venster \textbf{het}, dâ vor glas.\\ 
 & der venster einez offen was\\ 
 & gein dem boumgarten.\\ 
 & dar în gienc er durch warten,\\ 
 & durch luft unde dur\textit{ch} \textbf{der} vogel \textbf{sanc}.\\ 
10 & sîn sitzen \textbf{wart dô niht lanc}.\\ 
 & er kôs ein burc, die er des âbendes sach,\\ 
 & dô im diu âventiure geschach,\\ 
 & vil vrouwen ûf dem palas,\\ 
 & manigiu under in vil schœne was.\\ 
15 & \textbf{ez} dûhte in ein wunder grôz,\\ 
 & daz die vrouwen niht verdrôz\\ 
 & ir wachens, daz si sliefen niht.\\ 
 & dannoch der tac was niht \textbf{ze} lieht.\\ 
 & er dâhte: "ich wil in ze êren\\ 
20 & \textbf{mich an slâfen} kêren."\\ 
 & wider an sîn bette er gienc.\\ 
 & der meide mandel \textbf{übervienc}\\ 
 & in - daz was sîn decke.\\ 
 & ob man in dâ iht wecke?\\ 
25 & nein, daz wære dem wirte leit.\\ 
 & diu maget durch gesellecheit,\\ 
 & al dâ si vor ir muoter lac,\\ 
 & \textbf{si} brach ir slâf, des si pflac,\\ 
 & unt gienc hin zuo \textbf{ir} gaste;\\ 
30 & der slief dannoch al vaste.\\ 
\end{tabular}
\scriptsize
\line(1,0){75} \newline
G I L M Z Fr23 Fr62 \newline
\line(1,0){75} \newline
\textbf{1} \textit{Überschrift:} Hie hat her gawan Lishois gwellius gevangen Vnd hat in ge geben einem ferien Vnd hat in der ferie die naht behalten vnd tvt im gvtlich vnd wol dar nach so lese man wie ez im fvrbaz ge Z   $\cdot$ \textit{Initiale} G L Z  \textbf{11} \textit{Initiale} I  \newline
\line(1,0){75} \newline
\textbf{2} slief] flif Z \textbf{3} \textit{Versdoppelung (\textasciicircum2L); Lesarten des vorausgehenden Verses (im Kolophon) mit \textasciicircum1L bezeichnet} L   $\cdot$ dô] [Doch]: Do \textsuperscript{2}\hspace{-1.3mm} L Da M  $\cdot$ entwachete] erwachte I (L) (M) entwaht Z \textbf{5} \textit{Vers 553.5 fehlt} M   $\cdot$ het] heten I \textbf{6} der] [Vil]: Der M \textbf{7} boumgarten] [bogarten]: bovmgarten G \textbf{8} în gienc er] in gie L gienc er in Fr62  $\cdot$ durch] \textit{om.} I \textbf{9} durch] durf G vnd durh Fr62  $\cdot$ der] \textit{om.} I Z Fr62 den M \textbf{10} Do was sin sitcen niht ce lanc Fr62  $\cdot$ dô] da I L M Z  $\cdot$ lanc] ze lanc I (Z) \textbf{11} er kôs] e er kos Fr62  $\cdot$ des âbendes] des nahtes I do Fr62 \textbf{12} dô] Da M Z \textbf{14} manigiu under in] er sah der manec Fr62  $\cdot$ was] [saz]: was Z \textbf{15} ez dûhte] Do duhtez Fr62 \textbf{17} ir wachens] er wachens I Jrs wachens Z \textbf{18} was niht] nicht was M  $\cdot$ lieht] lýcht L (M) (Z) (Fr62) \textbf{21} an] in an L \textbf{22} übervienc] er vber vienc Z umbe uienc Fr62 \textbf{23} in] Vnd Z \textbf{24} man] \textit{om.} Fr62  $\cdot$ iht] ieman Fr62 \textbf{25} wære] [wirt]: wire M \textbf{28} si brach] Brach L  $\cdot$ des si] des sie e Z (Fr62) \textbf{29} hin] \textit{om.} Fr23 \textbf{30} slief] sleif Fr23 \newline
\end{minipage}
\hspace{0.5cm}
\begin{minipage}[t]{0.5\linewidth}
\small
\begin{center}*T
\end{center}
\begin{tabular}{rl}
 & \begin{large}G\end{large}rôz müede im zôch diu ougen zuo;\\ 
 & \textbf{dô} slief er unz des morgens vruo.\\ 
 & \multicolumn{1}{l}{ - - - }\\ 
 & \multicolumn{1}{l}{ - - - }\\ 
 & \multicolumn{1}{l}{ - - - }\\ 
 & \multicolumn{1}{l}{ - - - }\\ 
 & \multicolumn{1}{l}{ - - - }\\ 
 & \multicolumn{1}{l}{ - - - }\\ 
 & \multicolumn{1}{l}{ - - - }\\ 
 & \multicolumn{1}{l}{ - - - }\\ 
 & \multicolumn{1}{l}{ - - - }\\ 
 & \multicolumn{1}{l}{ - - - }\\ 
 & \multicolumn{1}{l}{ - - - }\\ 
 & \multicolumn{1}{l}{ - - - }\\ 
 & dô \textbf{erwachete} der wîgant.\\ 
 & einhalp der kemenâten want\\ 
5 & vil venster dâ vor \textbf{heten} glas.\\ 
 & der venster einez offen was\\ 
 & gegen dem boumgarten.\\ 
 & dar în gienc er durch warten,\\ 
 & durch luft unde durch \textbf{der} vogele \textbf{sanc}.\\ 
10 & \textbf{dâ was} sîn sitzen \textbf{harte unlanc}.\\ 
 & er kôs eine burc, die er des âbendes sach,\\ 
 & dô im diu âventiure geschach,\\ 
 & vil vrouwen ûf de\textit{m} palas,\\ 
 & maneg\textit{iu} under in vil schœne was.\\ 
15 & \textbf{daz} dûht in ein wunder grôz,\\ 
 & daz die vrouwen niht verdrôz\\ 
 & ir wachens, daz si sliefen niht.\\ 
 & dannoch der tac was niht \textbf{sô} lieht.\\ 
 & er dâhte: "ich wil in ze êren\\ 
20 & \textbf{mich an slâfen} kêren."\\ 
 & wider an sîn bette er gienc.\\ 
 & der megede mantel \textbf{umbevienc}\\ 
 & in - daz was sîn decke.\\ 
 & ob man in dâ iht wecke?\\ 
25 & nein, daz wære dem wirte leit.\\ 
 & diu maget durch gesellecheit,\\ 
 & aldâ si vor ir muoter lac,\\ 
 & \textbf{si} brach ir slâf, des si \textbf{dâ} pflac,\\ 
 & unde gie hin \textbf{ûf} z\textbf{ir} gaste;\\ 
30 & der slief dannoch al vaste.\\ 
\end{tabular}
\scriptsize
\line(1,0){75} \newline
T U V W O Q R Fr39 \newline
\line(1,0){75} \newline
\textbf{1} \textit{Initiale} T V W O Q Fr39   $\cdot$ \textit{Capitulumzeichen} R  \textbf{21} \textit{Initiale} W  \newline
\line(1,0){75} \newline
\textbf{1} \textit{Die Verse 553.1-599.30 fehlen} U   $\cdot$ Grôz] ÷roze O  $\cdot$ diu] \textit{om.} O \textbf{2} \textit{nach 553.2:} Wer nv welle der verneme / Ob im sin muͦt dez gesteme / Hie slichet ein aventúre her / Dez bin ich Gawanes wer / Die pruͤfet man zvͦ solher not / Der niht gelichet wan der tot / Sv̂ pfliget engestlicher sitte / Doch vert do pris vnde ere mitte / Swem aldo gelinget / Dar sú froͤide bringet / Nv min her Gawan gepflag / Guͦter ruwe vntz an den tag V   $\cdot$ des] \textit{om.} W Q R \textbf{3} der] aber der O  $\cdot$ erwachete] erwacht R \textbf{4} want] fant R \textbf{5} vor] von R Fr39  $\cdot$ heten] heittern R \textbf{7} dem] den Q \textbf{9} durch der] der W (R) dvrch den O  $\cdot$ vogele] voͤgellin V  $\cdot$ sanc] gesanck W Q \textbf{10} dâ] Do V W Q (Fr39)  $\cdot$ sitzen] sicze R \textbf{13} dem] den T \textbf{14} manegiu] manege T (R) \textbf{18} was niht] niht was O  $\cdot$ sô] zuͦ W \textbf{19} dâhte] gedacht W \textbf{20} slâfen] schlaffe R \textbf{22} umbevienc] [*]: vmbevieng V úberuieng W (O) (Q) (R) (Fr39) \textbf{24} \textit{Vers 553.24 fehlt} R   $\cdot$ dâ] do V W Q Fr39  $\cdot$ iht] nicht Q \textbf{27} aldâ] Da O Als do R  $\cdot$ si] \textit{om.} V \textbf{28} des si] \textit{om.} R  $\cdot$ dâ] e V \textit{om.} W O Q Fr39  $\cdot$ pflac] sy lag R \textbf{29} gie] lief V  $\cdot$ ûf] \textit{om.} V  $\cdot$ zir] zvͦ dem V \textbf{30} dannoch al] aldo viel V dannoch O (R) \newline
\end{minipage}
\end{table}
\end{document}
