\documentclass[8pt,a4paper,notitlepage]{article}
\usepackage{fullpage}
\usepackage{ulem}
\usepackage{xltxtra}
\usepackage{datetime}
\renewcommand{\dateseparator}{.}
\dmyyyydate
\usepackage{fancyhdr}
\usepackage{ifthen}
\pagestyle{fancy}
\fancyhf{}
\renewcommand{\headrulewidth}{0pt}
\fancyfoot[L]{\ifthenelse{\value{page}=1}{\today, \currenttime{} Uhr}{}}
\begin{document}
\begin{table}[ht]
\begin{minipage}[t]{0.5\linewidth}
\small
\begin{center}*D
\end{center}
\begin{tabular}{rl}
\textbf{542} & \begin{large}L\end{large}ischoys Gwelljus,\\ 
 & der junge süeze, warp alsus:\\ 
 & vrechheit unt ellenthaftiu tât,\\ 
 & \textbf{daz} was sînes hôhen herzen rât.\\ 
5 & er vrumte manegen snellen swanc.\\ 
 & dicke er von Gawane spranc\\ 
 & unt aber wider sêre ûf in.\\ 
 & Gawan truoc stætelîchen sin.\\ 
 & Er dâhte: "ergrîfe ich dich zuo mir,\\ 
10 & ich sols vil \textbf{gar} gelônen dir."\\ 
 & man sach dâ viwers blicke\\ 
 & \textbf{unt} diu swert ûf werfen dicke\\ 
 & ûz ellenthaften henden.\\ 
 & si begunden ein ander wenden\\ 
15 & neben, vür unt hinder sich.\\ 
 & ân nôt was ir gerich.\\ 
 & si m\textit{ö}hten\textit{z} âne strîten lân.\\ 
 & Dô \textbf{begreif} in \textbf{mîn} hêr Gawan:\\ 
 & \textbf{er} warf in under sich mit kraft.\\ 
20 & mit halsen solch geselleschaft\\ 
 & \textbf{müeze} mich vermîden;\\ 
 & ine möhte ir niht erlîden.\\ 
 & Gawan \textbf{bat} \textbf{sicherheite}.\\ 
 & der was als \textbf{unbereite}\\ 
25 & Lischoys, der \textbf{dâ} unden lac,\\ 
 & \textbf{als} dô er von êrste strîtes pflac.\\ 
 & Er sprach: "dû sûmest dich ân nôt.\\ 
 & vür sicherheit gip ich den tôt.\\ 
 & lâz enden dîne werden hant,\\ 
30 & swaz mir ie prîses wart \textbf{bekant}.\\ 
\end{tabular}
\scriptsize
\line(1,0){75} \newline
D \newline
\line(1,0){75} \newline
\textbf{1} \textit{Initiale} D  \textbf{9} \textit{Majuskel} D  \textbf{18} \textit{Majuskel} D  \textbf{27} \textit{Majuskel} D  \newline
\line(1,0){75} \newline
\textbf{1} Liscôys gwellivs D \textbf{17} möhtenz] mohtens D \textbf{25} Lischoys] Liscoys D \newline
\end{minipage}
\hspace{0.5cm}
\begin{minipage}[t]{0.5\linewidth}
\small
\begin{center}*m
\end{center}
\begin{tabular}{rl}
 & Lischois Gwellius,\\ 
 & der junge süeze, \textit{w}arp alsus:\\ 
 & vrecheit und ellenthaftiu \textit{t}ât,\\ 
 & \textbf{daz} was sînes hôhen herze\textit{n} rât.\\ 
5 & er vr\textit{o}mte manigen snellen swanc.\\ 
 & dicke er von Gaw\textit{a}nen spranc\\ 
 & und aber wider sêre ûf in.\\ 
 & Gawan truoc stæteclîchen sin.\\ 
 & er dâhte: "ergrîf ich dich zuo \textit{mi}r,\\ 
10 & ich sols vil \textbf{wol} gelônen dir."\\ 
 & man sach d\textit{â} \textit{v}iures blicke\\ 
 & \textbf{und} diu swert ûf werfen dicke\\ 
 & û\textit{z} ellenthaften henden.\\ 
 & si begunden ein ander wenden\\ 
15 & neben, vür un\textit{d h}inder sich.\\ 
 & âne nôt was ir gerich.\\ 
 & si möhtenz âne strîten lân.\\ 
 & dô \textbf{ergr\textit{eif}} in hêr Gawan:\\ 
 & \textbf{er} warf in under sich mit kraft.\\ 
20 & mit halsen solich geselleschaft\\ 
 & \textbf{müeste} mich vermîden;\\ 
 & ich enmöht ir niht erlîden.\\ 
 & Gawan \textbf{bat} \textbf{aber dô} \textbf{sicherheit}.\\ 
 & der was als \textbf{ungereit}\\ 
25 & Lischois, der \textbf{dâ} unden lac,\\ 
 & \textbf{alsô} dô er von êrste strîtes pflac.\\ 
 & er sprach: "dû sûmest dich âne nôt.\\ 
 & vür sicherheit gip ich den tôt.\\ 
 & \textbf{daz} lâz enden dîn werde hant,\\ 
30 & waz mir ie prîses w\textit{a}rt \textbf{erkant}.\\ 
\end{tabular}
\scriptsize
\line(1,0){75} \newline
m n o \newline
\line(1,0){75} \newline
\newline
\line(1,0){75} \newline
\textbf{1} Liscois gwellius m  $\cdot$ Liscois vnd giwellius n  $\cdot$ Liscois gevellens o \textbf{2} warp] starb m  $\cdot$ alsus] also o \textbf{3} tât] rat m \textbf{4} sînes] sin o  $\cdot$ herzen] hercze m \textbf{5} vromte] fremte m freuite o \textbf{6} Gawanen] [gaw]: gawenen m \textbf{9} dâhte] gedochte n  $\cdot$ mir] ymer m \textbf{11} dâ] do m n o  $\cdot$ viures] frures m \textbf{13} ûz] Vff m \textbf{15} und hinder] vnd sin hinder m \textbf{17} möhtenz] mochtencz o \textbf{18} ergreif] er gr m  $\cdot$ hêr] hsr o \textbf{22} enmöht] enmocht o \textbf{23} aber dô] ob er do m ob er o \textbf{25} Lischois] [Kiscois]: Liscois m Liscois n o  $\cdot$ dâ] do n o \textbf{29} daz] \textit{om.} n o \textbf{30} wart] werd m \newline
\end{minipage}
\end{table}
\newpage
\begin{table}[ht]
\begin{minipage}[t]{0.5\linewidth}
\small
\begin{center}*G
\end{center}
\begin{tabular}{rl}
 & \textit{\begin{large}L\end{large}}ishois Gewellius,\\ 
 & der junge süeze, warb alsus:\\ 
 & v\textit{re}cheit unde ellenthaftiu tât,\\ 
 & \textbf{daz} was sînes hôhen herzen rât.\\ 
5 & er vrumete manigen snellen swanc.\\ 
 & dicke er von Gawane spranc\\ 
 & unde aber \textit{wider sêre} ûf in.\\ 
 & Gawan truoc stætelîchen sin.\\ 
 & er dâhte: "ergr\textit{î}f ich dich zuo mir,\\ 
10 & ich sol es vil \textbf{wol} gelônen dir."\\ 
 & man sach dâ viures blicke,\\ 
 & diu swert ûf werfen dicke\\ 
 & ûz ellenthaften henden.\\ 
 & si begunden ein ander wenden\\ 
15 & neben, vür unde hinder sich.\\ 
 & âne nôt was ir gerich.\\ 
 & s\textit{i} m\textit{ö}htenz âne strîten lân.\\ 
 & dô \textbf{begreif} in \textbf{mîn} hêrre Gawan:\\ 
 & \textbf{er} warf in under sich mit kraft.\\ 
20 & mit halsen solche geselleschaft\\ 
 & \textbf{müeze} mich vermîden;\\ 
 & ichne m\textit{ö}ht ir niht erlîden.\\ 
 & Gawan \textbf{bat} \textbf{sicherheit}.\\ 
 & der was als \textbf{unbereit}\\ 
25 & Lishois, der \textbf{dâ} unde lac,\\ 
 & \textbf{als} dô er von êrste strîtes pflac.\\ 
 & er sprach: "dû sûmest dich ân nôt.\\ 
 & vür sicherheit gib ich den tôt.\\ 
 & lâze enden dîne werden hant,\\ 
30 & swaz mir ie brîses wart \textbf{bekant}.\\ 
\end{tabular}
\scriptsize
\line(1,0){75} \newline
G I L M Z \newline
\line(1,0){75} \newline
\textbf{1} \textit{Initiale} G L Z  \textbf{11} \textit{Initiale} I  \newline
\line(1,0){75} \newline
\textbf{1} Lishois] Ẏyshois G Liscois I Litschoýs L Lisois M  $\cdot$ Gewellius] gwellivs L (M) Z \textbf{3} vrecheit] vercheit G  $\cdot$ ellenthaftiu] ellenthafter I  $\cdot$ tât] [hant]: tat G \textbf{4} was] \textit{om.} I \textbf{5} er vrumete] erfruͯmite M \textbf{6} er von Gawane] er von Gawan I von gawan er Z \textbf{7} wider sêre] sere wider G \textbf{9} dâhte] gedachte L  $\cdot$ ergrîf] ergreif G grif I ergriffe L (Z) begrife M \textbf{10} es] \textit{om.} M  $\cdot$ wol] gar L M Z  $\cdot$ gelônen] vergelten L \textbf{11} dâ] \textit{om.} I \textbf{12} diu] Vnd L Vnde dy M (Z) \textbf{17} si] sine G  $\cdot$ möhtenz] mohtinz G (L) (M) (Z) \textbf{18} dô] Da M Z  $\cdot$ hêrre] er M \textbf{20} halsen] helfen Z \textbf{22} ichne] ich I (Z)  $\cdot$ möht] moht G (I) (L) (M) (Z)  $\cdot$ ir] [in]: ir Z \textbf{23} bat] bat sich Z \textbf{24} Der was also e bereit M  $\cdot$ als] allez I al L \textbf{25} Lishois] Liscoys I Lýtschoýs L Lisois M  $\cdot$ dâ] da cla M \textbf{26} dô] da M Z  $\cdot$ strîtes] stritens I \textbf{29} werden] werde I (L) (Z) \textbf{30} swaz] Waz L (M)  $\cdot$ brîses] strites L  $\cdot$ bekant] erchant I \newline
\end{minipage}
\hspace{0.5cm}
\begin{minipage}[t]{0.5\linewidth}
\small
\begin{center}*T
\end{center}
\begin{tabular}{rl}
 & Lyschoys Gewellius,\\ 
 & der junge süeze, warp alsus:\\ 
 & vrecheit unde ellenthaft\textit{iu} tât\\ 
 & was sînes hôhen herzen rât.\\ 
5 & er vrumte manegen snellen swanc.\\ 
 & dicke er von Gawane spranc\\ 
 & unde aber wider sêre ûf in.\\ 
 & Gawan truoc stætelîchen sin.\\ 
 & er dâhte: "ergrîfe ich dich ze mir,\\ 
10 & ich sols vil \textbf{wol} gelônen dir."\\ 
 & man sach dâ viures blicke\\ 
 & \textbf{unde} di\textit{u} swert ûf werfen dicke\\ 
 & ûz ellenthaften henden.\\ 
 & si begunden ein ander wenden\\ 
15 & neben, vür unde hinder sich.\\ 
 & Âne nôt was ir gerich.\\ 
 & si m\textit{ö}htenz âne strîten lân.\\ 
 & Dô \textbf{begreif} in \textbf{mîn} hêr Gawan\\ 
 & \textbf{unde} warf in under sich mit kraft.\\ 
20 & mit halsen sölch geselleschaft\\ 
 & \textbf{muoze} mich vermîden;\\ 
 & ine m\textit{ö}htir niht erlîden.\\ 
 & Gawan \textbf{iesch} \textbf{sicherheit}.\\ 
 & der was \textbf{aber} als \textbf{unbereit}\\ 
25 & Lyschoys, der \textbf{dort} unden lac,\\ 
 & \textbf{sam} dô er von êrst strîtes pflac.\\ 
 & Er sprach: "dû sûmest dich âne nôt.\\ 
 & vür sicherheit gib ich den tôt.\\ 
 & lâz enden dîne werde hant,\\ 
30 & swaz mir ie prîses wart \textbf{bekant}.\\ 
\end{tabular}
\scriptsize
\line(1,0){75} \newline
T U V W O Q R Fr40 \newline
\line(1,0){75} \newline
\textbf{1} \textit{Initiale} W O R Fr40  \textbf{16} \textit{Majuskel} T  \textbf{18} \textit{Majuskel} T  \textbf{27} \textit{Majuskel} T  \newline
\line(1,0){75} \newline
\textbf{1} Lyschoys] Lyscoys T Lyschois U V R LYshois W ÷yhoys O Lishois Q Liskois Fr40  $\cdot$ Gewellius] gewellus U Gwellivs O (R) (Fr40) \textbf{3} vrecheit] Friheit O Frechafft R  $\cdot$ ellenthaftiu] ellenthafte T erenthaffte Q  $\cdot$ tât] getat R \textbf{4} hôhen herzen] hertzen hoher Q \textbf{5} vrumte manegen] fugte manchem Q \textbf{6} von] vor O  $\cdot$ Gawane] Gawan U gawanen Q Gaweinen R  $\cdot$ spranc] schwank R \textbf{9} dâhte] gedachte W  $\cdot$ ergrîfe] begrife O  $\cdot$ dich] \textit{om.} W \textbf{10} ich] Er Q  $\cdot$ sols] solt is U sol V W O Q R Fr40  $\cdot$ gelônen] lonen V glauben Q \textbf{11} dâ] do U V W Q des R \textbf{12} unde] Von R  $\cdot$ diu] die T \textit{om.} W O Q R Fr40 \textbf{13} ûz] Zu Q (Fr40)  $\cdot$ ellenthaften] erenthafften Q \textbf{15} hinder] hinden Q \textbf{17} möhtenz] mohtenz T (U) (V) O mochtes Q mochtent R  $\cdot$ strîten] streite Fr40 \textbf{20} sölch] [sluck]: sulch Q \textbf{21} mich] ich R \textbf{22} ine möhtir] ine mohtir T (U) Jch moht ir O [ich*]: ich moht ir Fr40  $\cdot$ erlîden] liden R \textbf{23} Gawan] Gawin R  $\cdot$ iesch] bat aber do V \textbf{24} der] Der helt W Die Q  $\cdot$ aber] \textit{om.} U V W O Q R Fr40  $\cdot$ als] von im Q \textbf{25} Lyschoys] Lyscoys T Lyschois U R Lichoys V Lyshois W Lẏshoẏs O Lishois Q Liskois Fr40  $\cdot$ dort] da O R Fr40 do Q \textbf{26} sam dô] Als R \textbf{27} sûmest] muͯst R \textbf{28} sicherheit] sicheren V  $\cdot$ gib] gen U \textbf{29} werde] werden V R \textbf{30} Swaz mir ie [*]: prises wart bekant V  $\cdot$ swaz] Waz U (W) (Q) Was R \newline
\end{minipage}
\end{table}
\end{document}
