\documentclass[8pt,a4paper,notitlepage]{article}
\usepackage{fullpage}
\usepackage{ulem}
\usepackage{xltxtra}
\usepackage{datetime}
\renewcommand{\dateseparator}{.}
\dmyyyydate
\usepackage{fancyhdr}
\usepackage{ifthen}
\pagestyle{fancy}
\fancyhf{}
\renewcommand{\headrulewidth}{0pt}
\fancyfoot[L]{\ifthenelse{\value{page}=1}{\today, \currenttime{} Uhr}{}}
\begin{document}
\begin{table}[ht]
\begin{minipage}[t]{0.5\linewidth}
\small
\begin{center}*D
\end{center}
\begin{tabular}{rl}
\textbf{77} & kum wider \textbf{unt} nim von mîner hant\\ 
 & krône, sceptrum unt \textbf{ein} lant.\\ 
 & daz ist mich an erstorben.\\ 
 & daz hât dîn minne erworben.\\ 
5 & \textit{\begin{large}H\end{large}}ab dir \textbf{ouch} ze soldiment\\ 
 & \textbf{dise} rîchen prîsent\\ 
 & in \textbf{den} vier soumschrîn.\\ 
 & dû solt ouch mîn ritter sîn\\ 
 & ime lande ze Waleis,\\ 
10 & \textbf{von} der \textbf{houbtstat} ze Kanvoleis.\\ 
 & ine ruoche, ob ez diu küneginne siht.\\ 
 & ez mac mir geschaden niht.\\ 
 & ich bin schœner unt rîcher\\ 
 & unt kan ouch minneclîcher\\ 
15 & minne enpfâhen unt \textbf{minne} geben.\\ 
 & wiltû nâch werder minne leben,\\ 
 & sô hab dir mîne krône\\ 
 & nâch minne ze lône."\\ 
 & An \textbf{disem} brieve er niht mêre vant.\\ 
20 & sîn hersnier eines knappen hant\\ 
 & wider ûf sîn houbet zôch.\\ 
 & Gahmureten trûren v\textit{l}ôch.\\ 
 & man bant im ûf den adamas,\\ 
 & der dicke und herte was.\\ 
25 & er wolde sich arbeiten.\\ 
 & die boten hiez er \textbf{leiten}\\ 
 & durch \textbf{ruowen} under\textbf{z} poulûn.\\ 
 & swâ gedrenge was, dâ macheter rûn.\\ 
 & Dirre verlôs, \textbf{jener} gewan.\\ 
30 & dâ moht erholn sich ein man,\\ 
\end{tabular}
\scriptsize
\line(1,0){75} \newline
D \newline
\line(1,0){75} \newline
\textbf{5} \textit{Initiale} D  \textbf{19} \textit{Majuskel} D  \textbf{29} \textit{Majuskel} D  \newline
\line(1,0){75} \newline
\textbf{5} Hab] ÷ab D \textbf{10} Kanvoleis] chanvoleis D \textbf{22} Gahmureten] Gahmvreten D  $\cdot$ vlôch] vroͮch D \newline
\end{minipage}
\hspace{0.5cm}
\begin{minipage}[t]{0.5\linewidth}
\small
\begin{center}*m
\end{center}
\begin{tabular}{rl}
 & kum \textit{wider} \textbf{und} nim von mîner hant\\ 
 & krône, cepter und \textbf{ein} lant.\\ 
 & daz ist mich an erstorben.\\ 
 & daz hât dîn minne erworben.\\ 
5 & habe dir \textbf{ouch} ze soldiment\\ 
 & \textbf{disen} rîchen prîsent\\ 
 & in \textbf{den} vier soumschrîn.\\ 
 & dû solt ouch mîn ritter sîn\\ 
 & inme land\textit{e} ze Waleis\\ 
10 & \textbf{vor} der \textbf{houbetstat} ze Kanvoleis.\\ 
 & in ruoche, obez diu künigîn siht.\\ 
 & ez \textbf{en}mac mir \textbf{vil} geschaden niht.\\ 
 & ich bin schœner und rîcher\\ 
 & und kan ouch minneclîcher\\ 
15 & minne enpfâhen und \textbf{minne} geben.\\ 
 & wiltû nâch werder minne leben,\\ 
 & sô hab dir mîne krône\\ 
 & nâch minne \textit{zuo} lône."\\ 
 & \begin{large}A\end{large}n \textbf{disem} brieve  niht mêre vant.\\ 
20 & sîn hersenier eines knappen hant\\ 
 & wider ûf sîn houbet zôch.\\ 
 & Gahmureten trûren vlôch.\\ 
 & man bant i\textit{m} ûf den adamas,\\ 
 & der dicke und herte was.\\ 
25 & er wolte sich arbeiten.\\ 
 & die boten hiez er \textbf{leiten}\\ 
 & durch \textbf{ruowen} under\textbf{z} pavelûn.\\ 
 & wâ gedrenge was, d\textit{â} macht er rû\textit{n}.\\ 
 & dirre verlôs, \textbf{einer} gewan.\\ 
30 & dâ mohte erholn sich ein man,\\ 
\end{tabular}
\scriptsize
\line(1,0){75} \newline
m n o \newline
\line(1,0){75} \newline
\textbf{19} \textit{Initiale} m o   $\cdot$ \textit{Capitulumzeichen} n  \newline
\line(1,0){75} \newline
\textbf{1} wider] \textit{om.} m \textbf{2} ein] \textit{om.} n o \textbf{4} daz] Es n o \textbf{6} rîchen] risen n \textbf{7} vier] viern n o \textbf{8} ritter] vatter o \textbf{9} lande] landes m  $\cdot$ Waleis] weleis m waleise o \textbf{10} Kanvoleis] canvoleis m tanfaleis n tampoleis o \textbf{12} enmac] mag n o \textbf{13} schœner] schone n \textbf{15} geben] genben o \textbf{18} zuo] \textit{om.} m \textbf{20} hersenier] hersernier n \textbf{22} Gahmureten] Gahmuretten m Gamierten n Gamureten o \textbf{23} im] in m  $\cdot$ adamas] adamast m \textbf{28} dâ] do m n o  $\cdot$ rûn] ruͯme m \textbf{29} einer] jenner n (o) \textbf{30} dâ] Do n o  $\cdot$ mohte] moͯchte n \newline
\end{minipage}
\end{table}
\newpage
\begin{table}[ht]
\begin{minipage}[t]{0.5\linewidth}
\small
\begin{center}*G
\end{center}
\begin{tabular}{rl}
 & kum wider, nim von mîner hant\\ 
 & krône, zepter und \textbf{daz} lant.\\ 
 & daz ist mich ane erstorben.\\ 
 & daz hât dîn minne erworben.\\ 
5 & habe dir ze soldimente\\ 
 & \textbf{dise} rîche presente\\ 
 & in \textbf{disen} vier soumschrîn.\\ 
 & dû solt ouch mîn rîter sîn\\ 
 & in dem lande ze Waleis\\ 
10 & \textbf{vor} der \textbf{stat} ze Kanvoleis.\\ 
 & ich enruoche, obez diu künegîn siht.\\ 
 & ez mac mir \textbf{vil} geschaden niht.\\ 
 & ich bin schœner und rîcher\\ 
 & unde kan ouch minniclîcher\\ 
15 & minne enpfâhen und \textbf{minne} geben.\\ 
 & wil dû nâch werder minne leben,\\ 
 & sô habe dir mîne krône\\ 
 & nâch minne ze lône."\\ 
 & an \textbf{dem} brief er niht mêre vant.\\ 
20 & sîn härsenier eines knappen hant\\ 
 & wider ûf sîn houbet zôch.\\ 
 & Gahmureten trûren vlôch.\\ 
 & man bant im ûf den adamas,\\ 
 & der dicke und herte was.\\ 
25 & er wolt sich arbeiten.\\ 
 & die boten hiez er \textbf{leiten}\\ 
 & durch \textbf{ruowe} under\textbf{z} pavelûn.\\ 
 & swâ gedrenge was, dâ machter rûn.\\ 
 & \begin{large}D\end{large}irre vlôs \textbf{und} \textbf{der} gewan.\\ 
30 & dâ maht erholen sich ein man,\\ 
\end{tabular}
\scriptsize
\line(1,0){75} \newline
G I O L M Q R Z Fr56 \newline
\line(1,0){75} \newline
\textbf{1} \textit{Initiale} O R Z Fr56   $\cdot$ \textit{Capitulumzeichen} L  \textbf{3} \textit{Initiale} I  \textbf{29} \textit{Überschrift:} Hie kvmt er wider zv strite Z   $\cdot$ \textit{Initiale} G L R Z Fr56  \newline
\line(1,0){75} \newline
\textbf{1} kum] ÷vm O Fr56  $\cdot$ nim] vnd nim I mynn Q \textbf{2} daz] ein O (M) Q Z Fr56 \textit{om.} L min R \textbf{4} hât] hant Z  $\cdot$ dîn] div O \textbf{5} dir] dir ovch Z \textbf{6} dise rîche] Die richen O L M R Die reiche Q Dise risen Z  $\cdot$ presente] presentie R \textbf{7} in disen] disiu I Nim div O Jn den L M Q Z Jn dem so R  $\cdot$ soumschrîn] schoymschryn M vmschrin R \textbf{9} Waleis] walais I walaiz L valeis R \textbf{10} vor] von I Jn R  $\cdot$ stat] haupt stat I (O) (L) (M) (Q) (R) (Z)  $\cdot$ Kanvoleis] kanvoleiz G ganfolois I canvolais O kanvolaiz L kanroleis Q kamfoleis Z \textbf{11} obez] avch daz iz O das ir M ob ichs Q \textbf{12} ez] ezn I (L) (M) (Q) (Z)  $\cdot$ mir] \textit{om.} O  $\cdot$ geschaden] gewerren I \textbf{14} minniclîcher] myndiglicher Q \textbf{15} minne enpfâhen] Mein enfpfhon Q  $\cdot$ und minne] vnd I L (M) (Fr56) \textbf{16} werder] rechter Q \textbf{17} mîne] mýnne L  $\cdot$ krône] kronen M \textbf{18} lône] lonen M \textbf{19} dem] disem O (M) (Q) R disen Z  $\cdot$ niht mêre] nimmer Z \textbf{20} härsenier] harschemer L helm R  $\cdot$ hant] kant L \textbf{21} sîn] das R \textbf{22} Gahmureten] Gamvreten O Gahmuͯreten L Gamureten M Q (Z) gahmvreten Fr56 \textbf{24} dicke] riche L  $\cdot$ herte] herre L \textbf{25} sich] sy R \textbf{26} boten] betten R \textbf{27} \textit{nach 77.27:} Wa gedreng was an dem feld R   $\cdot$ ruowe] ruen I (R) truwe L  $\cdot$ underz] in die L  $\cdot$ pavelûn] baieluͤn I geczelt R \textbf{28} \textit{nach 77.28} Durch siner frowen bottun R   $\cdot$ Da macht er ain Rum R  $\cdot$ swâ] Wo L Q Wo esz M  $\cdot$ dâ] do O Q  $\cdot$ rûn] in rum I \textbf{29} Dirre] ÷irre Fr56  $\cdot$ und der] einer I der O L M Q R Z Fr56 \textbf{30} dâ] Do Q  $\cdot$ erholen sich] er holt sich O sich erhon R  $\cdot$ ein] en Z \newline
\end{minipage}
\hspace{0.5cm}
\begin{minipage}[t]{0.5\linewidth}
\small
\begin{center}*T (U)
\end{center}
\begin{tabular}{rl}
 & kum wider, nim von mîner hant\\ 
 & krône, scepter und lant.\\ 
 & daz ist mich an erstorben.\\ 
 & daz hât dîne minne erworben.\\ 
5 & habe dir zuo soldimente\\ 
 & \textbf{die} rîchen prîsente\\ 
 & in \textbf{den} vier \textit{s}o\textit{um}schrîn.\\ 
 & dû solt ouch mîn ritter sîn\\ 
 & in dem lande zuo Waleis\\ 
10 & \textbf{vor} der \textbf{houbetstat} zuo Kanvoleis.\\ 
 & ine ruoche, ob ez diu künegîn siht.\\ 
 & ez \textbf{en}mac mir \textbf{vil} geschaden niht.\\ 
 & ich bin schœner und rîcher\\ 
 & und kan ouch minniclîcher\\ 
15 & minne entvâhen und geben.\\ 
 & wiltû nâch werder minne leben,\\ 
 & sô habe dir mîne krône\\ 
 & nâch minne zuo lône."\\ 
 & \begin{large}A\end{large}n \textbf{diseme} brieve er \textit{niht} mêre vant.\\ 
20 & sîn hersenier eines knappen hant\\ 
 & wider ûf sîn houbet zôch.\\ 
 & Gahmureten trûren vlôch.\\ 
 & man bant im ûf den adamas,\\ 
 & der dicke und herte was.\\ 
25 & er wolte sich arbeiten.\\ 
 & die boten hiez er \textbf{geleiten}\\ 
 & durch \textbf{ruowe} under pavelûn.\\ 
 & wâ gedrenge was, d\textit{â} machter rûn.\\ 
 & dirre verlôs, \textbf{der} gewan.\\ 
30 & d\textit{â} mohte erholn sich ein man,\\ 
\end{tabular}
\scriptsize
\line(1,0){75} \newline
U V W T \newline
\line(1,0){75} \newline
\textbf{1} \textit{Initiale} W   $\cdot$ \textit{Majuskel} T  \textbf{17} \textit{Majuskel} T  \textbf{19} \textit{Initiale} U V   $\cdot$ \textit{Majuskel} T  \textbf{29} \textit{Initiale} W T  \newline
\line(1,0){75} \newline
\textbf{2} lant] [*]: ein lant V \textbf{3} erstorben] gestorben W gêrbet T \textbf{4} din minne mich hat verderbet T \textbf{5} habe dir] Hab dirs W vnde habe T \textbf{7} soumschrîn] schoin schrin U saum schreinen W \textbf{8} ouch] \textit{om.} W  $\cdot$ sîn] scheinen W \textbf{9} zuo] \textit{om.} W  $\cdot$ Waleis] [*]: walleis V waleys W \textbf{10} zuo] \textit{om.} W  $\cdot$ Kanvoleis] [*]: kanvoleis V kanuoleys W \textbf{11} jne rvͦche siht ez div kvnegin T \textbf{12} enmac] mag W  $\cdot$ vil geschaden niht] niht vil scade gesin T \textbf{13} ich] wand ich T \textbf{15} geben] [*]: minne geben V minne geben W \textbf{17} dir] \textit{om.} T \textbf{18} minne] minnen W \textbf{19} niht mêre] vmere U númme V \textbf{22} Gahmureten] Gahmuͦreten U Gamuret V W  $\cdot$ trûren] do truren W \textbf{23} im] in W \textbf{26} geleiten] leiten V (W) T \textbf{27} einen knappen hin zem pavelvn T  $\cdot$ pavelûn] daz pauelun V (W) \textbf{28} wâ] swa V T  $\cdot$ gedrenge] getreúwe W  $\cdot$ dâ] do U V \textit{om.} W  $\cdot$ machter] der W \textbf{29} verlôs] floch W \textbf{30} dâ] Do U (V) W  $\cdot$ mohte] moͤht V \newline
\end{minipage}
\end{table}
\end{document}
