\documentclass[8pt,a4paper,notitlepage]{article}
\usepackage{fullpage}
\usepackage{ulem}
\usepackage{xltxtra}
\usepackage{datetime}
\renewcommand{\dateseparator}{.}
\dmyyyydate
\usepackage{fancyhdr}
\usepackage{ifthen}
\pagestyle{fancy}
\fancyhf{}
\renewcommand{\headrulewidth}{0pt}
\fancyfoot[L]{\ifthenelse{\value{page}=1}{\today, \currenttime{} Uhr}{}}
\begin{document}
\begin{table}[ht]
\begin{minipage}[t]{0.5\linewidth}
\small
\begin{center}*D
\end{center}
\begin{tabular}{rl}
\textbf{81} & \textbf{des twanc in} werdiu minne\\ 
 & einer rîchen küneginne.\\ 
 & diu kom ouch sît \textbf{nâch} \textbf{im} in nôt.\\ 
 & \textbf{si lag an klagenden triwen} tôt.\\ 
5 & \textbf{Swie} Gahmuret \textbf{wære} ouch mit klage,\\ 
 & doch het er \textbf{an} dem \textbf{halben} tage\\ 
 & gevrumt \textbf{sô} vil der sper enzwei.\\ 
 & wære worden der turnei,\\ 
 & sô wære verswendet der walt.\\ 
10 & \textbf{geverwet} hundert im wâren gezalt,\\ 
 & \begin{large}D\end{large}i\textit{u} gar vertet der fiere.\\ 
 & sîne liehten baniere\\ 
 & \textbf{wâren den krîgiere\textit{r}n} worden.\\ 
 & daz was in ir orden.\\ 
15 & dô \textbf{reit} er gein dem poulûn.\\ 
 & der \textbf{Wâleisinne} garzûn\\ 
 & huop sich nâch im ûf die vart.\\ 
 & der tiwer wâpenroc im wart\\ 
 & durchstochen unt \textbf{verhouwen}.\\ 
20 & den truog er vür die vrouwen.\\ 
 & \textbf{er} was \textbf{von golde dennoch} guot.\\ 
 & \textbf{er} gleste als ein \textbf{glüendic} gluot.\\ 
 & dâr an kôs man rîcheit.\\ 
 & dô sprach diu künegîn gemeit:\\ 
25 & "dich hât ein werdez wîp gesant\\ 
 & bî disem ritter in ditze lant.\\ 
 & Nû mant mich diu vuoge mîn,\\ 
 & daz die andern \textbf{niht} verkrenket sîn,\\ 
 & die âventiure brâhte \textbf{dar}.\\ 
30 & ieslîcher nem mînes wunsches war,\\ 
\end{tabular}
\scriptsize
\line(1,0){75} \newline
D \newline
\line(1,0){75} \newline
\textbf{5} \textit{Majuskel} D  \textbf{11} \textit{Initiale} D  \textbf{27} \textit{Majuskel} D  \newline
\line(1,0){75} \newline
\textbf{5} Gahmuret] Gahmvret D \textbf{11} Diu] Die D \textbf{13} krîgierern] chrigîren D \newline
\end{minipage}
\hspace{0.5cm}
\begin{minipage}[t]{0.5\linewidth}
\small
\begin{center}*m
\end{center}
\begin{tabular}{rl}
 & \textbf{des twanc in} werdiu minne\\ 
 & einer rîchen küniginne.\\ 
 & diu kam ouch sît \textbf{nâch} \textbf{ime} in nôt.\\ 
 & \textbf{si lac an klagenden triuwen} \textit{tôt}.\\ 
5 & \textbf{wie} Gahmuret \textbf{wær} ouch mit klage,\\ 
 & doch het er \textbf{an} dem \textbf{halben} tage\\ 
 & gevrumt \textbf{sô} vil der sper enzwei.\\ 
 & wære worden der turnei,\\ 
 & sô wære verswendet der walt.\\ 
10 & \textbf{geverwet} hundert im wâren gezalt,\\ 
 & diu gar vertet der fiere.\\ 
 & sîne liehten baniere\\ 
 & \textbf{wâren den krîiereren} worden.\\ 
 & daz \textit{was} \textbf{wol} in ir orden.\\ 
15 & dô \textbf{reit} er gegen dem pavelûn.\\ 
 & der \textbf{Wâleis\textit{i}nne} garzûn\\ 
 & huop sich nâch ime ûf die vart.\\ 
 & der tiure wâpenroc ime wart\\ 
 & durchstochen und \textbf{verhouwen}.\\ 
20 & den truoc er vür die vrouwen.\\ 
 & \textbf{er} was \textbf{von golde dannoch} guot.\\ 
 & \textbf{er} gleste als ein \textbf{glüende} g\textit{l}uot.\\ 
 & dâr an kôs man rîcheit.\\ 
 & dô sprach diu künigîn gemeit:\\ 
25 & "dich hât ein werdez wîp gesant\\ 
 & bî disem ritter in diz lant.\\ 
 & \begin{large}N\end{large}û mant mich diu vuoge mîn,\\ 
 & daz die andern \textbf{niht} verkrenket sî\textit{n},\\ 
 & die \textbf{diu} âventiure brâhte \textbf{dar}.\\ 
30 & ieglîcher \dag nam\dag  mînes wunsches war,\\ 
\end{tabular}
\scriptsize
\line(1,0){75} \newline
m n o \newline
\line(1,0){75} \newline
\textbf{27} \textit{Initiale} m n  \newline
\line(1,0){75} \newline
\textbf{2} küniginne] koniginnen o \textbf{4} an klagenden triuwen tôt] an clagen den truwen m an clagender mynne dot n clagen der trúwen dot o \textbf{5} Gahmuret] gamiret n gamuͯret o \textbf{6} halben] selben o \textbf{10} geverwet] Geferwe n  $\cdot$ hundert] húnder n hinder o  $\cdot$ im] \textit{om.} o  $\cdot$ gezalt] gestalt o \textbf{11} der] die o \textbf{13} krîiereren] kriegern n \textbf{14} was] \textit{om.} m \textbf{15} pavelûn] panalunn o \textbf{16} Wâleisinne] Walleissunne m waleisen n waleisin o  $\cdot$ garzûn] ganczunn o \textbf{21} golde] goldes o \textbf{22} glüende] gluͯgender n (o)  $\cdot$ gluot] guͦt m \textbf{25} gesant] genant o \textbf{26} diz] das o \textbf{28} sîn] sint m \textbf{29} brâhte] brochten n (o) \textbf{30} mînes] sins o \newline
\end{minipage}
\end{table}
\newpage
\begin{table}[ht]
\begin{minipage}[t]{0.5\linewidth}
\small
\begin{center}*G
\end{center}
\begin{tabular}{rl}
 & \textbf{den riet} \textbf{ein} werdiu minne\\ 
 & einer rîchen küniginne.\\ 
 & diu kom ouch sît \textbf{nâch} \textbf{im} in nôt.\\ 
 & \textbf{an klagenden riwen lac si} tôt.\\ 
5 & Gahmuret \textbf{was} ouch mit klage.\\ 
 & doch heter \textbf{an} dem \textbf{selben} tage\\ 
 & gevrumet \textbf{sô} vil der sper enzwei.\\ 
 & wære worden der turnei,\\ 
 & sô wære verswende\textit{t} \textit{d}er walt.\\ 
10 & hundert \textbf{sper} im wâren gezalt,\\ 
 & diu gar vertet der fiere.\\ 
 & sîne lieht baniere\\ 
 & \textbf{den krojiere\textit{r}n wâren} worden.\\ 
 & daz was \textbf{wol} in ir orden.\\ 
15 & dô \textbf{kêrte}r gein dem bavelûn.\\ 
 & der \textbf{küniginne} garzûn\\ 
 & huop sich nâch im ûf die vart.\\ 
 & der tiwere wâpenroc im wart\\ 
 & durchstochen und \textbf{verhouwen}.\\ 
20 & den truog er vür die vrouwen.\\ 
 & \textbf{der} was \textbf{dannoch von golde} guot\\ 
 & \textbf{unde} glaste als ein \textbf{glüende} gluot.\\ 
 & dâr \textit{an} kôs man rîcheit.\\ 
 & dô sprach diu künigîn gemeit:\\ 
25 & "dich hât ein werdez wîp gesant\\ 
 & bî disem rîter in diz lant.\\ 
 & nû manet mich diu vuoge mîn,\\ 
 & daz die anderen \textbf{iht} verkrenket sîn,\\ 
 & die \textbf{diu} âventiure brâhte \textbf{dar}.\\ 
30 & ieslîcher neme mînes wunsches war:\\ 
\end{tabular}
\scriptsize
\line(1,0){75} \newline
G I O L M Q R Z \newline
\line(1,0){75} \newline
\textbf{1} \textit{Initiale} O M  \textbf{5} \textit{Initiale} L Q R Z  \newline
\line(1,0){75} \newline
\textbf{1} den riet] demriet I ÷en riet O Des twanc Z  $\cdot$ ein] div O (L) (M) (Q) (R) in Z  $\cdot$ werdiu] werde R \textbf{2} rîchen] werden O reiche Q  $\cdot$ küniginne] Kúnginen R \textbf{3} diu] Da M  $\cdot$ ouch sît nâch im] auch sit von im I (O) ouch sit duͯrch in L nach im sid R  $\cdot$ nôt] [rat]: not M \textbf{4} Si lach an chlagnden (clagende M ) triwen (trúwe R ) tot O (L) (M) (Q) (R) (Z) \textbf{5} Gahmuret] Gamvret O GAhmuͯret L Gamuret M Q Wie gamuret Z  $\cdot$ was] wer Z \textbf{6} heter] hat er R  $\cdot$ an] in O L (M) Q (R)  $\cdot$ selben] halben O (L) (M) (R) (Z) halbe Q \textbf{7} gevrumet sô vil] Gebrochin M  $\cdot$ enzwei] ein zwai I so vil enzwey Q \textbf{8} wære] vnd were I \textbf{9} verswendet der] verswendet von im der G verschwindet der R \textbf{10} Geverbet hvndert im waren (ým hvndert waren L hundert war im Q Jm hundert warrt R ) gezalt O (L) (M) (Q) (R) (Z)  $\cdot$ wâren] wart I \textbf{11} vertet] wer tett R  $\cdot$ der] \textit{om.} Z  $\cdot$ fiere] fiers M frye R \textbf{12} lieht] liehten O (Z) lychten L (M) lichte Q \textbf{13} den krojierern wâren] den croieren warn G Waren den grogiereren (kroyeren L krieren M grogierer R kroierern Z ) O (L) (M) (R) (Z) Worden den kreyer Q \textbf{14} in ir] im mer I in orin M \textbf{15} dô] Da Z  $\cdot$ kêrter] chert I kert er Q R Z  $\cdot$ dem] der I L M  $\cdot$ bavelûn] pauelin R \textbf{16} küniginne] kúnnigen R \textbf{17} im] \textit{om.} O M Q R Z  $\cdot$ die] der Q R \textbf{18} im wart] [in ben]: im wart I truͦg Jm ward R \textbf{19} verhouwen] zerhowen R durchhowen Z \textbf{21} der] Er Z  $\cdot$ dannoch von golde] von golde dannoch O L (M) (Q) (R) (Z) \textbf{22} unde] Er Z  $\cdot$ glaste] gliszte M  $\cdot$ als] sam O  $\cdot$ glüende gluot] gluͤndiu gloͮt I glvgendiger gluͯt L glute tut Q \textbf{23} an] \textit{om.} G  $\cdot$ rîcheit] reichet Q \textbf{24} dô] Da M Z \textbf{25} gesant] [erchant]: gesant I \textbf{26} diz] daz I (Q) (R) \textbf{27} mich] dich Q \textbf{28} anderen] ander R  $\cdot$ iht] niht I (R)  $\cdot$ verkrenket] [vercher]: verchrenchet G verkenket R \textbf{29} die diu] diu die I Die Q Z  $\cdot$ brâhte] brachten M (Z) \textbf{30} ieslîcher] Jeslichiv O Die L  $\cdot$ neme] nemen L nom R \newline
\end{minipage}
\hspace{0.5cm}
\begin{minipage}[t]{0.5\linewidth}
\small
\begin{center}*T (U)
\end{center}
\begin{tabular}{rl}
 & \textbf{den riet} \textbf{eine} werdiu minne\\ 
 & einer rîchen küniginne.\\ 
 & diu kam ouch sît \textbf{durch} \textbf{in} in nôt.\\ 
 & \textbf{si lac an klagenden triuwen} tôt.\\ 
5 & \begin{large}G\end{large}ahmuret \textbf{was} ouch mit klage.\\ 
 & doch het er \textbf{in} dem \textbf{halben} tage\\ 
 & gevrumt vil der sper enzwei.\\ 
 & wære worden der turnei,\\ 
 & sô wære \textbf{gar} verswendet der walt.\\ 
10 & \textbf{geverwet} hundert im wâren gezalt,\\ 
 & diu gar vertet der fiere.\\ 
 & sî\textit{n}e liehten baniere\\ 
 & \textbf{wâren den krîere\textit{r}n} worden.\\ 
 & daz was \textbf{wol} in ir orden.\\ 
15 & dô \textbf{kêrte} er gein dem pavelûn.\\ 
 & d\textit{er} \textbf{küneginne} garzûn\\ 
 & huop sich nâch im ûf die vart.\\ 
 & der tiure wâpenroc im wart\\ 
 & durchstochen und \textbf{durchhouwen}.\\ 
20 & den truoc er vür die vrouwen.\\ 
 & \textbf{der} was \textbf{von golde dannoch} guot\\ 
 & \textbf{und} glastet als ein \textbf{glüende} gluot.\\ 
 & dâr an kôs man rîcheit.\\ 
 & dô sprach diu künegin gemeit:\\ 
25 & "dich hât ein werdez wîp gesant\\ 
 & bî diseme rîter in diz lant.\\ 
 & nû manet mich diu vuoge mîn,\\ 
 & daz danderen \textbf{êt} verkrenket sîn,\\ 
 & die âventiure brâhte \textbf{har}.\\ 
30 & ieclîcher neme mînes wunsches war:\\ 
\end{tabular}
\scriptsize
\line(1,0){75} \newline
U V W T \newline
\line(1,0){75} \newline
\textbf{5} \textit{Initiale} U V W   $\cdot$ \textit{Majuskel} T  \textbf{12} \textit{Majuskel} T  \textbf{15} \textit{Initiale} T  \textbf{16} \textit{Majuskel} T  \textbf{24} \textit{Majuskel} T  \textbf{27} \textit{Majuskel} T  \newline
\line(1,0){75} \newline
\textbf{1} den riet] [*]: Daz riet im V Dem riet W  $\cdot$ eine] div T  $\cdot$ werdiu] werde T \textbf{2} einer rîchen] Ein rechte W \textbf{3} sît durch in in] nach im seit in W sit von im T \textbf{5} Gahmuret] GAhmuͦret U Gamuret V (W)  $\cdot$ ouch mit] in T \textbf{6} er] \textit{om.} W \textbf{7} vil] so vil W T \textbf{9} gar verswendet] verswendet gar V (W) verswendet T \textbf{10} hundert im] im hundert W \textbf{12} sîne] Sie U  $\cdot$ liehten] liechte W \textbf{13} den krîerern] crieren U (T) do kryen W \textbf{15} kêrte] kert V W \textbf{16} der] die U \textbf{17} im] \textit{om.} W \textbf{21} was] waz da V \textbf{22} als ein glüende] als ein gluͤgender V als sam ein brinnende W von golde als ein T \textbf{24} gemeit] so gemeit W \textbf{25} dich] Dis V \textbf{28} êt] iht V (W) ih T  $\cdot$ verkrenket] gekrencket W \textbf{29} die] Die die V die div T  $\cdot$ brâhte] brahten V  $\cdot$ har] dar W T \textbf{30} ieclîcher] iegeliche V jr ieglicher T  $\cdot$ neme mînes] nam seines W \newline
\end{minipage}
\end{table}
\end{document}
