\documentclass[8pt,a4paper,notitlepage]{article}
\usepackage{fullpage}
\usepackage{ulem}
\usepackage{xltxtra}
\usepackage{datetime}
\renewcommand{\dateseparator}{.}
\dmyyyydate
\usepackage{fancyhdr}
\usepackage{ifthen}
\pagestyle{fancy}
\fancyhf{}
\renewcommand{\headrulewidth}{0pt}
\fancyfoot[L]{\ifthenelse{\value{page}=1}{\today, \currenttime{} Uhr}{}}
\begin{document}
\begin{table}[ht]
\begin{minipage}[t]{0.5\linewidth}
\small
\begin{center}*D
\end{center}
\begin{tabular}{rl}
\textbf{665} & unt \textbf{der künec} Meljanz \textbf{von} Parbigœl\\ 
 & unt Jofreit fiz Idœl,\\ 
 & die sint hin ûf gevangen,\\ 
 & ê der \textbf{bûhurt} wære ergangen.\\ 
5 & \textbf{Ouch} viengen \textbf{si} von Logroys\\ 
 & \textbf{den herzogen} \textbf{Friam} \textbf{von} Vermendoys\\ 
 & unt \textbf{den grâven} Ritschart \textbf{von} \textbf{Navers}.\\ 
 & der \textbf{vertet} niwan eines spers;\\ 
 & gein swem ouch daz sîn hant gebôt,\\ 
10 & der viel \textbf{vor} im durch tjoste nôt.\\ 
 & Artus mit sîn selbes hant\\ 
 & vienc den \textbf{degen wert} erkant.\\ 
 & Dâ wurden unverdrozzen\\ 
 & die poinder sô \textbf{geslozzen},\\ 
15 & des m\textit{ö}hte swenden sich der walt.\\ 
 & manec tjoste ungezalt\\ 
 & \textbf{rêrten} trunzûne.\\ 
 & die werden Bertune\\ 
 & wâren \textbf{ouch} manlîche ze wer\\ 
20 & gein der herzoginne her.\\ 
 & Artuses nâchhuote\\ 
 & muose strîtes sîn ze muote.\\ 
 & man \textbf{hardierte} si den tac,\\ 
 & \textbf{unze} \textbf{dâr} diu vluot des hers \textbf{lac}.\\ 
25 & Ouch solte mîn hêr Gawan\\ 
 & der herzogîn gekündet hân,\\ 
 & daz ein sîn helfære\\ 
 & in ir lande wære;\\ 
 & sô wære des strîtes niht geschehen.\\ 
30 & dô\textbf{ne} wolt er\textbf{s ir noch niemen} \textbf{jehen},\\ 
\end{tabular}
\scriptsize
\line(1,0){75} \newline
D \newline
\line(1,0){75} \newline
\textbf{5} \textit{Majuskel} D  \textbf{13} \textit{Majuskel} D  \textbf{25} \textit{Majuskel} D  \newline
\line(1,0){75} \newline
\textbf{1} Meljanz] Melianz D  $\cdot$ Parbigœl] Parbigoͤl D \textbf{2} Idœl] Jdoͤl D \textbf{5} Logroys] Logrôys D \textbf{6} Vermendoys] Vermendôys D \textbf{7} Ritschart] Ritscart D \textbf{15} möhte] mohte D \textbf{18} Bertune] Bertvͦne D \textbf{21} Artuses] Artvs D \newline
\end{minipage}
\hspace{0.5cm}
\begin{minipage}[t]{0.5\linewidth}
\small
\begin{center}*m
\end{center}
\begin{tabular}{rl}
 & und Melianz \textbf{von} Barbigol\\ 
 & und \textit{Jo}fr\textit{e}it fi\textit{z} Idol,\\ 
 & die sint hin ûf gevangen,\\ 
 & ê der \textbf{strît} wær ergangen.\\ 
5 & \textbf{dô} viengen \textbf{aber} \textbf{si} \textbf{die} von Logrois,\\ 
 & \textbf{den herzogen} \textbf{Friam} \textbf{von} Vermendois\\ 
 & und \textbf{den grâven} Rischart \textbf{de} \textbf{Novers}.\\ 
 & der \textbf{vertet} niht wan eines spers;\\ 
 & gegen wem ouch daz sîn hant gebôt,\\ 
10 & der viel \textbf{vor} im durch juste nôt.\\ 
 & Artus mit sîn selbes hant\\ 
 & vienc den \textbf{werden degen} erkant.\\ 
 & d\textit{â} wurden unverdrozzen\\ 
 & die ponder sô \textbf{beslozzen},\\ 
15 & des m\textit{ö}hte swenden sich der walt.\\ 
 & manic just\textit{e u}ngezalt\\ 
 & \textbf{dô} \textbf{rêrte} \dag runzîne\dag .\\ 
 & die werden Britune\\ 
 & wâren \textbf{ouch} manlîch zuo wer\\ 
20 & gegen der herzogîn her.\\ 
 & Artuses nâchhuote\\ 
 & muoste strîtes sîn zuo muote.\\ 
 & man \textbf{hardierte} si den tac,\\ 
 & \textbf{unz} \textbf{dâr} diu vluot des hers \textbf{lac}.\\ 
25 & ouch solt mîn hêr Gawan\\ 
 & der her\textit{z}ogîn gekündet hân,\\ 
 & daz ein sîn helfære\\ 
 & in ir lande wære;\\ 
 & sô wære des strîtes niht geschehen.\\ 
30 & dô wolt er \textbf{irs noch niemen} \textbf{jehen},\\ 
\end{tabular}
\scriptsize
\line(1,0){75} \newline
m n o \newline
\line(1,0){75} \newline
\newline
\line(1,0){75} \newline
\textbf{1} Melianz] meliantz m o meliancz o  $\cdot$ Barbigol] parbigol n o \textbf{2} Jofreit] gotfrid m o gotfrit n  $\cdot$ fiz Idol] firidol m o furidol n \textbf{5} aber si] sú aber n aber o \textbf{6} herzogen] hertzogin m  $\cdot$ Friam] frẏam n frian o  $\cdot$ von] vnd n  $\cdot$ Vermendois] vermandois o \textbf{7} Rischart] ritschart n ritterschart o  $\cdot$ de Novers] denovers m de nouers n \textbf{11} Artus] Artús o \textbf{13} dâ] Do m n o \textbf{15} möhte] mohtte m (o) \textbf{16} Manig juste vnerkant vnd vngezalt m \textbf{17} runzîne] rinczime o \textbf{18} werden] werde o  $\cdot$ Britune] brittune m britunie n britame o \textbf{21} nâchhuote] noch noch húte o \textbf{23} hardierte] harderte o \textbf{25} hêr] herre her n \textbf{26} herzogîn] herogin m \newline
\end{minipage}
\end{table}
\newpage
\begin{table}[ht]
\begin{minipage}[t]{0.5\linewidth}
\small
\begin{center}*G
\end{center}
\begin{tabular}{rl}
 & \begin{large}U\end{large}nde \textbf{r\textit{o}y\textit{s}} Melianz \textbf{de} Barbigol\\ 
 & unde Jofreit fis Idol,\\ 
 & die sint hin ûf gevangen,\\ 
 & ê der \textbf{bûhurt} wære ergangen.\\ 
5 & \textbf{ouch} viengen \textbf{die} von Logrois\\ 
 & \textbf{duc} \textbf{Firmam} \textbf{de} Frimidois\\ 
 & unde \textbf{\textit{c}un\textit{s}} \textit{R}itschart \textbf{de} \textbf{Ninivers}.\\ 
 & der \textbf{vuorte} \textbf{ouch} niwan eines spers;\\ 
 & gein swem ouch daz sîn hant gebôt,\\ 
10 & der viel \textbf{vor} im durch tjoste nôt.\\ 
 & Artus mit sîn selbes hant\\ 
 & vienc den \textbf{wert} erkant.\\ 
 & dâ wurden unverdrozzen\\ 
 & die poinder sô \textbf{geslozzen},\\ 
15 & des möhte swenden sich der walt.\\ 
 & manic tjost ungezalt\\ 
 & \textbf{rêrt} \textbf{ir} trunzûne.\\ 
 & die werden Britune\\ 
 & wâren \textbf{dâ} manlîch ze wer\\ 
20 & gein der herzoginne her.\\ 
 & Artuses nâchhuote\\ 
 & muose strîtes sîn ze muote.\\ 
 & man \textbf{parrierte} si den tac,\\ 
 & \textbf{ê} \textbf{daz} diu vluot des hers \textbf{gelac}.\\ 
25 & ouch solde mîn hêr Gawan\\ 
 & der herzoginne gekündet hân,\\ 
 & daz ein sîn helfære\\ 
 & in ir lande wære;\\ 
 & sô\textbf{ne} wære des strîtes niht geschehen.\\ 
30 & dô\textbf{ne} wolder\textbf{s niht} \textbf{verjehen},\\ 
\end{tabular}
\scriptsize
\line(1,0){75} \newline
G I L M Z \newline
\line(1,0){75} \newline
\textbf{1} \textit{Initiale} G L Z  \textbf{13} \textit{Initiale} I  \newline
\line(1,0){75} \newline
\textbf{1} roys] rins G roy I  $\cdot$ Melianz] meliantze Z  $\cdot$ de] zcu M \textbf{2} Jofreit] tschfreit G schoͮfreit I schoffreit M Lofreit Z  $\cdot$ fis Idol] fisidol G I M Z Fizedol L \textbf{3} hin] hie Z  $\cdot$ gevangen] gigangen M \textbf{4} bûhurt] bihurt M  $\cdot$ ergangen] zergangen L \textbf{5} viengen die] vie man der L viengen sie Z  $\cdot$ Logrois] logroẏs G logr:ys I Logroýs L lo grois M \textbf{6} Firmam] firmanil I fimam M Friam Z  $\cdot$ de Frimidois] de frimedoys I de Frýmedoýs L defirmedois M von fermendois Z \textbf{7} cuns] Rvns vnde G  $\cdot$ Ritschart] ritshart I Rytschart L  $\cdot$ Ninivers] niniuers I Nivevers L Nunvers M Nivers Z \textbf{8} vuorte ouch] furt I vertet ouch Z \textbf{9} swem] wem L (M)  $\cdot$ daz] er Z  $\cdot$ gebôt] bot I \textbf{10} nôt] tot L \textbf{11} sîn selbes] siner I syns selbins M \textbf{12} den] den degen L (M) Z \textbf{13} unverdrozzen] vnverdrozzen : G \textbf{15} möhte] mohte I (L) (M) (Z)  $\cdot$ swenden] swenden swenden I wenden M \textbf{17} rêrt ir] Kert ir L Rerten Z \textbf{18} Britune] pritune I [B*]: Brittvne L brytuͯne M \textbf{21} Artuses] Artvs G (I) (L) (M) (Z) \textbf{23} parrierte] hardierte Z \textbf{24} vluot] fluht I Z  $\cdot$ des] \textit{om.} L \textbf{30} Da enwolde ers ir noch nieman iehen Z  $\cdot$ dône] Da vone M  $\cdot$ wolders] wolt er irs I \newline
\end{minipage}
\hspace{0.5cm}
\begin{minipage}[t]{0.5\linewidth}
\small
\begin{center}*T
\end{center}
\begin{tabular}{rl}
 & und \textbf{roys} Melyanz \textbf{de} Barbigol\\ 
 & und Jofreit fis Idol,\\ 
 & die sint hin ûf gevangen,\\ 
 & ê der \textbf{bêhurt} wære ergangen.\\ 
5 & \textbf{ouch} viengen \textbf{die} von Logrois\\ 
 & \textbf{den herzogen} \textbf{Friam} \textbf{de} Fermendois\\ 
 & und \textbf{cuns} Ritschart \textbf{de} \textbf{Navers}.\\ 
 & der \textbf{vertet} niuwan eines spers;\\ 
 & gên wem ouch daz sîn hant gebôt,\\ 
10 & der viel \textbf{von} im durch tjoste nôt.\\ 
 & Artus mit sînes selbes hant\\ 
 & vienc den \textbf{degen wert} erkant.\\ 
 & dô wurden unverdrozzen\\ 
 & die poynder sô \textbf{geslozzen},\\ 
15 & d\textit{e}s m\textit{ö}hte swenden sich der walt.\\ 
 & manic tjost ungezalt\\ 
 & \textbf{\textit{r}êrte} \textbf{ir} trunzûne.\\ 
 & die werden Britune\\ 
 & wâren \textbf{d\textit{â}} menlîche zuo wer\\ 
20 & gên der herzogîn her.\\ 
 & Artuses nâchhuote\\ 
 & muoste strîtes sîn ze muote.\\ 
 & man \textbf{par\textit{r}ierte} si den tac,\\ 
 & \textbf{ê} \textbf{daz} diu vlu\textit{o}t des hers \textbf{gelac}.\\ 
25 & ouch solde mîn hêr Gawan\\ 
 & der herzoginne gekündet hân,\\ 
 & daz ein sîn helfæ\textit{r}e\\ 
 & in ir lande wære;\\ 
 & sô \textbf{en}wær des strîtes niht geschehen.\\ 
30 & dô \textbf{en}wolt er\textbf{s niht} \textbf{verjehen},\\ 
\end{tabular}
\scriptsize
\line(1,0){75} \newline
Q R W V \newline
\line(1,0){75} \newline
\textbf{3} \textit{Initiale} Q  \textbf{5} \textit{Initiale} V  \newline
\line(1,0){75} \newline
\textbf{1} roys] kv́nig V  $\cdot$ Melyanz] meliantz Q Meliancz R melyantz W melianz V  $\cdot$ de] von W (V) \textbf{2} Jofreit] jofreiti Q iofrid W jofrid V  $\cdot$ fis Idol] fisidol Q R W visidol V \textbf{4} der] daz [*r]: der V \textbf{5} Oͮch viengen [*]: aber sv́ der von logrois V  $\cdot$ ouch] Vnd W  $\cdot$ Logrois] logroys Q logoris R \textbf{6} Friam de Fermendois] friam defermendeis Q fryam de firmendois R firam fimendois W vriamde firmendois V \textbf{7} [Vn*]: Vnde den graven Ritscart de navers V  $\cdot$ Ritschart] rutschart W  $\cdot$ de Navers] denauers Q de Nauers R (W) \textbf{9} wem] swem V  $\cdot$ ouch daz] daz och R auch der W \textbf{10} von] vor R W \textbf{14} poynder] pondierer R  $\cdot$ sô] do R \textit{om.} W  $\cdot$ geslozzen] beschlossen W (V) \textbf{15} des möhte] Der es mochte Q  $\cdot$ swenden] [senden]: swende Q  $\cdot$ walt] Schwarczwald R \textbf{17} rêrte] Kerte Q [*ert]: do rert V \textbf{18} Britune] brittúne Q brittvne V \textbf{19} dâ] do Q W V  $\cdot$ zuo] zer R \textbf{20} herzogîn] herczoginen R \textbf{21} Artuses] Artus Q R W \textbf{22} muoste] Muͦs R Muͤsse W (V)  $\cdot$ strîtes] streitens W \textbf{23} parrierte] partierte Q [*]: parrierte V \textbf{24} [*]: E [*a*]: daz die flvͦt dez [her*]: heres gelag V  $\cdot$ vluot] vlucht Q \textbf{26} herzoginne] herczoginnen R \textbf{27} sîn] seiner W  $\cdot$ helfære] helferine Q \textbf{28} ir] dem W \textbf{29} enwær] were R (W) (V)  $\cdot$ des] der R  $\cdot$ strîtes] strit R streiten W \textbf{30} Do enwolt [e*]: ers ir noch nieman veriehen V  $\cdot$ ers] er irs W \newline
\end{minipage}
\end{table}
\end{document}
