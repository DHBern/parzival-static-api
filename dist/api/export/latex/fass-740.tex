\documentclass[8pt,a4paper,notitlepage]{article}
\usepackage{fullpage}
\usepackage{ulem}
\usepackage{xltxtra}
\usepackage{datetime}
\renewcommand{\dateseparator}{.}
\dmyyyydate
\usepackage{fancyhdr}
\usepackage{ifthen}
\pagestyle{fancy}
\fancyhf{}
\renewcommand{\headrulewidth}{0pt}
\fancyfoot[L]{\ifthenelse{\value{page}=1}{\today, \currenttime{} Uhr}{}}
\begin{document}
\begin{table}[ht]
\begin{minipage}[t]{0.5\linewidth}
\small
\begin{center}*D
\end{center}
\begin{tabular}{rl}
\textbf{740} & \begin{large}D\end{large}az ich die rede \textbf{mac niht} verdagen,\\ 
 & i\textbf{ne} \textbf{müeze} ir strît mit triwen klagen,\\ 
 & sît ein verch unt ein bluot\\ 
 & solch ungenâde ein ander tuot.\\ 
5 & si wâren \textbf{doch} \textbf{bêde} eines mannes kint,\\ 
 & der geliuterten triwe fundamint.\\ 
 & Den heiden \textbf{minnen} nie verdrôz,\\ 
 & des was sîn herze in strîte grôz.\\ 
 & gein prîse truog er willen\\ 
10 & durch die küneginne Secundillen,\\ 
 & diu daz lant \textbf{ze} Tribalibot\\ 
 & im gap. diu was sîn schilt in nôt.\\ 
 & Der heiden \textit{n}a\textit{m} an strîte zuo.\\ 
 & \textbf{wie} tuon ich dem getouften nuo?\\ 
15 & er\textbf{n} welle an minne \textbf{denken},\\ 
 & sô\textbf{ne} mac er niht entwenken.\\ 
 & \textbf{dirre strît müeze im} erwerben\\ 
 & \textbf{vor}\textbf{s} \textbf{heidens} \textbf{hant ein} sterben.\\ 
 & daz wende, tugenthafter Grâl!\\ 
20 & Condwiramurs, \textbf{diu} lieht gemâl,\\ 
 & hie stêt iwer beider dienstman\\ 
 & in der grœsten nôt, die er ie gewan!\\ 
 & Der heiden warf \textbf{daz} swert ûf hôch.\\ 
 & manec sîn slac sich sus gezôch,\\ 
25 & daz Parzival kom \textbf{ûf} diu knie.\\ 
 & man mac wol \textbf{jehen}, \textbf{sus} striten sie,\\ 
 & der si bêde nennen wil \textbf{ze} zwein.\\ 
 & si wâren doch bêde niht wan ein.\\ 
 & mîn bruoder und ich, daz ist ein lîp,\\ 
30 & als ist guot man unt \textbf{des} \textbf{guot} wîp.\\ 
\end{tabular}
\scriptsize
\line(1,0){75} \newline
D \newline
\line(1,0){75} \newline
\textbf{1} \textit{Initiale} D  \textbf{7} \textit{Majuskel} D  \textbf{13} \textit{Majuskel} D  \textbf{23} \textit{Majuskel} D  \newline
\line(1,0){75} \newline
\textbf{13} nam] man D \textbf{25} Parzival] Parcifal D \newline
\end{minipage}
\hspace{0.5cm}
\begin{minipage}[t]{0.5\linewidth}
\small
\begin{center}*m
\end{center}
\begin{tabular}{rl}
 & daz ich die rede \textbf{niht mac} ver\textit{d}agen,\\ 
 & ich \textbf{muoz} ir strît mit triuwen klagen,\\ 
 & sî\textit{t} ein verch und ein bluot\\ 
 & solich ungenâde ein ander tuot.\\ 
5 & si wâren \textbf{beide} eines manne\textit{s} kint,\\ 
 & der geliuterten triuwe fundamint.\\ 
 & den heiden \textbf{minne} nie verdrôz,\\ 
 & des was sîn herz in strît\textit{e g}rôz.\\ 
 & gegen prîse truoc er willen\\ 
10 & durch die künigîn Secundillen,\\ 
 & diu daz lant \textbf{zuo} Tribalibot\\ 
 & im gap. diu \textit{w}a\textit{s s}în schil\textit{t} \textit{in} nôt.\\ 
 & der heiden nam an strîte zuo.\\ 
 & \textbf{wie} tuon ich dem getouften nû?\\ 
15 & er welle an minne \textbf{\textit{ged}enken},\\ 
 & \textit{s}ô mac er niht entwenken.\\ 
 & \textbf{diser strît muoz im} erwerben\\ 
 & \textbf{von} \textbf{des} \textbf{heidens} \textbf{hant ein} sterben.\\ 
 & daz wende, tugenthafter Grâl!\\ 
20 & Condwieramurs, lieht gemâl,\\ 
 & hie stât iuwer beider dienstman\\ 
 & in der grœsten nôt, die er ie gewan!\\ 
 & der heiden war\textit{f} \textbf{sîn} swert ûf hôch.\\ 
 & manic sîn slac sich sus gezôch,\\ 
25 & daz Parcifal kam \textbf{ûf} diu knie.\\ 
 & man mac wol \textit{\textbf{jehen}, \textbf{dô} striten s}ie,\\ 
 & der si beide nennen wil \textbf{vür} zwein.\\ 
 & si wâren doch beide niht \textit{w}an ein.\\ 
 & mîn bruoder und ich, daz ist ein lîp,\\ 
30 & alsô ist guot man und \textbf{sîn} wîp.\\ 
\end{tabular}
\scriptsize
\line(1,0){75} \newline
m n o V V' Fr69 \newline
\line(1,0){75} \newline
\newline
\line(1,0){75} \newline
\textbf{1} Der rede ich nit mac verdagen V'  $\cdot$ verdagen] vertragen m \textbf{2} ich muoz] Jch muͤsze V (V') Jn mvͤze Fr69 \textbf{3} sît] Sin m  $\cdot$ verch] [vech]: \sout{verch} verch V' \textbf{4} solich] Die V'  $\cdot$ ein ander] an einander V' \textbf{5} si] [Jr s*]: Sie o  $\cdot$ beide] doch beide V V'  $\cdot$ mannes] mannen m \textbf{6} geliuterten] [gel*]: gelúterten V gelidern V'  $\cdot$ triuwe] truwen V V' (Fr69)  $\cdot$ fundamint] firmamint n fundamit V' \textbf{7} den] die Fr69  $\cdot$ nie] \textit{om.} n \textbf{8} strîte grôz] stritte froͯ gros m \textbf{10} die] eige o  $\cdot$ Secundillen] socundille o \textbf{11} Tribalibot] tribabilot V' \textbf{12} was sîn schilt in] kam in schiltte m \textbf{14} tuon] sol V'  $\cdot$ nû] dv V' \textbf{15} welle] enwelle V (V')  $\cdot$ gedenken] wencken m dencken V (V') \textbf{16} sô mac er] Do mager m So enmag er V (V')  $\cdot$ entwenken] [*]: entwenken V \textbf{17} \textit{Vers 740.17 fehlt} o   $\cdot$ muoz] muͤsze V \textbf{18} heidens] heiden m n o \textbf{19} tugenthafter] tugenthafte V' \textbf{20} Condwieramurs] Cunduwier amurs n Cuͯnwier amúrs o Kvndewiramors V V'  $\cdot$ lieht] die Lieht V (V') \textbf{22} grœsten] grossen n \textbf{23} der] Die n  $\cdot$ warf] wars m  $\cdot$ sîn swert ûf] vff sin swert n daz swert uf V (V') \textbf{24} manic sîn slac] Wan sin swert V'  $\cdot$ gezôch] [*]: gezoch V \textbf{25} Parcifal] parzefal V parzifal V'  $\cdot$ kam] vil V'  $\cdot$ knie] [nẏe]: knie n \textbf{26} jehen dô striten sie] stritten jehen hie m \textbf{27} nennen] nemen V' newen Fr69  $\cdot$ vür] zuͯ n (o) (V) (V') (Fr69) \textbf{28} si] Do o  $\cdot$ niht wan] nit mitwan m nuwent V [nyeman]: nyewan V' \textbf{29} ich] [ist]: ich n  $\cdot$ daz ist] das sint n des ist o ist V' \textbf{30} Also >ist< got man >vnd< sin wip o  $\cdot$ guot] \textit{om.} V' \newline
\end{minipage}
\end{table}
\newpage
\begin{table}[ht]
\begin{minipage}[t]{0.5\linewidth}
\small
\begin{center}*G
\end{center}
\begin{tabular}{rl}
 & daz ich die rede \textbf{mac niht} verdagen,\\ 
 & ich \textbf{muoz} ir strît mit triwen klagen,\\ 
 & sît ein verch unde ein bluot\\ 
 & solch ungnâde ein ander tuot.\\ 
5 & si wâren \textbf{doch} eines mannes kint,\\ 
 & der geliuterten triwe fundamint.\\ 
 & den heiden \textbf{minne} nie verdrôz,\\ 
 & des was sîn herze in strîte grôz.\\ 
 & gein prîse truoc er willen\\ 
10 & durch die künigîn Secundillen,\\ 
 & diu daz lant Tribalibot\\ 
 & im gap. diu was sîn schilt in nôt.\\ 
 & der heiden nam an strîte zuo.\\ 
 & \textbf{waz} tuon ich dem getouften nuo?\\ 
15 & er\textbf{ne} welle an minne \textbf{denken},\\ 
 & sô\textbf{ne} mac er niht entwenken,\\ 
 & \textbf{imne müeze dirre strît} erwerben\\ 
 & \textbf{von} \textbf{heidens} \textbf{handen} sterben.\\ 
 & daz wende, tugenthafter Grâl!\\ 
20 & Condwiramurs, \textbf{diu} lieht gemâl,\\ 
 & hie stêt iwer bêder dienstman\\ 
 & in der grœzisten nôt, die er ie gewan!\\ 
 & der heiden warf \textbf{daz} swert ûf hôch.\\ 
 & manic sîn slac sich sus gezôch,\\ 
25 & daz Parzival kom \textbf{ûf} diu knie.\\ 
 & man mac wol \textbf{sprechen}, \textbf{sus} striten sie,\\ 
 & der si bêde nennen wil \textbf{ze} zwein.\\ 
 & si wâren doch bêde niwan ein.\\ 
 & mîn bruoder unde ich, daz ist ein lîp,\\ 
30 & als ist guot man unde \textbf{des} wîp.\\ 
\end{tabular}
\scriptsize
\line(1,0){75} \newline
G I L M Z Fr24 \newline
\line(1,0){75} \newline
\textbf{1} \textit{Initiale} Fr24  \textbf{3} \textit{Initiale} I  \textbf{21} \textit{Initiale} I  \newline
\line(1,0){75} \newline
\textbf{1} die rede] der [rege]: rede I  $\cdot$ mac niht] nich mach L (Fr24) \textbf{2} strît] striten I \textbf{4} ein ander] dem andern Z \textbf{5} eines] beide eines L (M) (Z) (Fr24)  $\cdot$ mannes] vater M \textbf{6} triwe] truͯwen L (Z)  $\cdot$ fundamint] ein fvndament L (M) (Fr24) \textbf{7} verdrôz] bedro:: Fr24 \textbf{8} was] wart I  $\cdot$ in] an I \textbf{9} prîse] strite I \textbf{10} Secundillen] secuntillen I Secu::: Fr24 \textbf{11} lant] lant hat I lant zcu M (Z) (Fr24)  $\cdot$ Tribalibot] tripalipot I \textbf{12} im gap] \textit{om.} I  $\cdot$ in] in der I \textbf{13} heiden] heide M  $\cdot$ an] in L \textbf{15} erne] Er M \textbf{16} entwenken] gewenken M \textbf{17} imne] im I (M)  $\cdot$ müeze] musz M \textbf{18} von heidens] Von des heiden M Vor heidens Z  $\cdot$ handen] hant eyn M (Z) \textbf{19} wende] wendet M \textbf{20} Condwiramurs] koͮndwiramvrs G Gonduwiramurs I Cvndwir Amvͯrs L Kundwir Amuͯrs M Kvndwiramvrs Z (Fr24)  $\cdot$ lieht] lýcht L (M) \textbf{22} die] vnd I \textbf{23} heiden] heide M  $\cdot$ ûf] \textit{om.} L \textbf{24} sîn] \textit{om.} L \textbf{25} Parzival] parcifal G Z Parzifal I (L) (M) ::rcifal Fr24 \textbf{26} man] \textit{om.} L  $\cdot$ sie] hie Z \textbf{27} nennen wil] wil nemen I (Fr24)  $\cdot$ ze] sý L \textbf{29} daz] \textit{om.} Z \textbf{30} wîp] gvt wip Z \newline
\end{minipage}
\hspace{0.5cm}
\begin{minipage}[t]{0.5\linewidth}
\small
\begin{center}*T
\end{center}
\begin{tabular}{rl}
 & \begin{large}D\end{large}az \textit{ich} die rede \textbf{niht mac} verdagen,\\ 
 & ich \textbf{en}\textbf{müeze} ir strît mit triuwen klagen,\\ 
 & sît ein verch und ein bluot\\ 
 & solich ungenâde ein ander tuot.\\ 
5 & si wâren \textbf{doch} \textbf{beide} eines mannes kint,\\ 
 & der geliuterten triuwen \textbf{ein} fundamint.\\ 
 & den heiden \textbf{minne} nie verdrôz,\\ 
 & des was sîn herze in strîte grôz.\\ 
 & gein prîse truoc er willen\\ 
10 & durch die küneginne Secundille\textit{n},\\ 
 & diu daz lant \textbf{zuo} Tribalibot\\ 
 & im gap. diu was sîn schilt in nôt.\\ 
 & der heiden nam an strîte zuo.\\ 
 & \textbf{waz} tuon ich dem getouften nuo?\\ 
15 & er \textbf{en}welle an minne \textbf{denken},\\ 
 & sô \textbf{en}mag er niht entwenken,\\ 
 & \textbf{ime müeze dirre strît} erwerben\\ 
 & \textbf{von} \textbf{heidensch} \textbf{handen} sterben.\\ 
 & daz wende, tugenthafter Grâl!\\ 
20 & Kundewiramurs, \textbf{diu} lieht gemâl,\\ 
 & hie stêt iuwer beider dienstman\\ 
 & in der grœsten nôt, die er ie gewan!\\ 
 & der heiden warf \textbf{daz} swert ûf hôch.\\ 
 & manec sîn slac sich sus gezôch,\\ 
25 & daz Parcifal kam \textbf{an} diu \textit{knie}.\\ 
 & man mac wol \textbf{sprechen}, \textbf{sus} striten sie,\\ 
 & der si beide nennen wil \textbf{\textit{zuo}} zwein.\\ 
 & si wâren doch beide niht wan ein.\\ 
 & mîn bruoder und ich, daz ist ein lîp,\\ 
30 & als ist guot man und \textbf{des} wîp.\\ 
\end{tabular}
\scriptsize
\line(1,0){75} \newline
U W Q R \newline
\line(1,0){75} \newline
\textbf{1} \textit{Initiale} U  \textbf{3} \textit{Illustration mit Überschrift:} Hie strit parczifal mit einem heiden zu ros vnd zu fuͦs vnd brach parczifaln sin schwert vnd wauren bruͦd von dem vatter Gahmuretten R   $\cdot$ \textit{Initiale} R  \newline
\line(1,0){75} \newline
\textbf{1} ich] \textit{om.} U  $\cdot$ niht mac] mag nicht Q nit gern R  $\cdot$ verdagen] [vertragen]: verdagen Q \textbf{2} ich enmüeze] Ich muͦß W (R) Jchen musz Q  $\cdot$ strît] beider strit R  $\cdot$ mit triuwen] \textit{om.} Q mit wauren trúwen R \textbf{3} verch] werck Q (R) \textbf{4} ein ander] ein andren R \textbf{5} doch] \textit{om.} W Q  $\cdot$ beide] beidú R \textbf{6} geliuterten] gelúttrerter R  $\cdot$ triuwen] treúw W (R) \textbf{7} den heiden] Der heide Q  $\cdot$ nie] mer Q \textbf{10} küneginne] werden kúnginnen R  $\cdot$ Secundillen] Secuͦndille U serúndille Q \textbf{11} Tribalibot] Tabilibot R \textbf{12} sîn schilt in] in strit sin R \textbf{15} er enwelle] Ern wolte Q  $\cdot$ denken] gedencken W (Q) (R) \textbf{16} sô enmag] So mag W Sone mackt Q  $\cdot$ entwenken] gewencken W \textbf{17} müeze] muͦß W (R) enmússe Q \textbf{18} Von heidens hant ein sterben Q  $\cdot$ heidensch] haidens W (R) \textbf{20} Kundewiramurs] Kuͦndewiramuͦrs U Kundwiramurs W Condwiramúrs Q Kunduwiramuͦrs R \textbf{22} grœsten] grossen W  $\cdot$ ie] ye sin tag R \textbf{23} der heiden] Er W  $\cdot$ ûf hôch] vff inder hand hoch R \textbf{24} sus] als Q  $\cdot$ gezôch] zoch R \textbf{25} Parcifal] Parzifal U partzifal W Q parczifal R  $\cdot$ knie] \textit{om.} U \textbf{26} sus] als Q  $\cdot$ striten sie] streiten sie Q stritten sy hie R \textbf{27} si] \textit{om.} Q  $\cdot$ zuo] \textit{om.} U \textbf{28} beide] beidú R  $\cdot$ niht wan] nun R \textbf{29} ich] ist R  $\cdot$ lîp] [lip]: leip Q \textbf{30} guot] der R  $\cdot$ des] das W R \newline
\end{minipage}
\end{table}
\end{document}
