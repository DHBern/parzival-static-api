\documentclass[8pt,a4paper,notitlepage]{article}
\usepackage{fullpage}
\usepackage{ulem}
\usepackage{xltxtra}
\usepackage{datetime}
\renewcommand{\dateseparator}{.}
\dmyyyydate
\usepackage{fancyhdr}
\usepackage{ifthen}
\pagestyle{fancy}
\fancyhf{}
\renewcommand{\headrulewidth}{0pt}
\fancyfoot[L]{\ifthenelse{\value{page}=1}{\today, \currenttime{} Uhr}{}}
\begin{document}
\begin{table}[ht]
\begin{minipage}[t]{0.5\linewidth}
\small
\begin{center}*D
\end{center}
\begin{tabular}{rl}
\textbf{773} & \begin{large}D\end{large}er heiden was von herzen vrô,\\ 
 & daz sînes bruoder \textbf{prîs} alsô\\ 
 & stuont, daz sîn hant erstreit\\ 
 & sô manege hôhe werdecheit.\\ 
5 & des dankter im sêre.\\ 
 & er hetes selbe ouch êre.\\ 
 & Innen des hiez tragen Gawan,\\ 
 & als ez unwizzende wære getân,\\ 
 & des heidens zimierde in den rinc.\\ 
10 & \textbf{si} prüeveten\textbf{z} \textbf{dâ} vür \textbf{hôhiu} dinc.\\ 
 & rîter und vrouwen\\ 
 & begunden alle schouwen\\ 
 & den wâpenroc, den schilt, daz kursît.\\ 
 & der helm was zenge noch ze wît.\\ 
15 & Si prîsten al gemeine\\ 
 & die \textbf{tiwern edeln} steine,\\ 
 & die dran verwieret lâgen.\\ 
 & niemen darf mich vrâgen\\ 
 & von ir arde, wie si wæren,\\ 
20 & die \textbf{lîhten} unt die swæren.\\ 
 & Iuch hete baz bescheiden des\\ 
 & Eraclius oder Ercules\\ 
 & unt der Krieche Alexander\\ 
 & unt dennoch ein ander\\ 
25 & \textbf{unt} der wîse Pictagoras,\\ 
 & der ein astronomierre was\\ 
 & unt sô wîs âne strît,\\ 
 & niemen sît Adames zît\\ 
 & \textbf{möhte} im glîchen sin \textbf{getragen}.\\ 
30 & \textbf{der} kunde \textbf{wol} von \textbf{steinen} sagen.\\ 
\end{tabular}
\scriptsize
\line(1,0){75} \newline
D Fr2 \newline
\line(1,0){75} \newline
\textbf{1} \textit{Initiale} D Fr2  \textbf{7} \textit{Majuskel} D  \textbf{15} \textit{Majuskel} D  \textbf{21} \textit{Majuskel} D  \newline
\line(1,0){75} \newline
\textbf{1} Der] ÷er Fr2 \textbf{9} heidens] heiden Fr2 \textbf{22} Ercules] Hercvles Fr2 \textbf{23} krieche] chrieche D \textbf{25} Pictagoras] Pitagoras Fr2 \textbf{26} astronomierre] astronomierer Fr2 \textbf{27} strît] strite: Fr2 \textbf{28} Adames] Adâmes D  $\cdot$ zît] zite: Fr2 \textbf{29} möhte] Moht Fr2 \newline
\end{minipage}
\hspace{0.5cm}
\begin{minipage}[t]{0.5\linewidth}
\small
\begin{center}*m
\end{center}
\begin{tabular}{rl}
 & \begin{large}D\end{large}er heiden was von herzen vrô,\\ 
 & daz sînes bruoder \textbf{prîs} alsô\\ 
 & stuont, daz sî\textit{n} hant erstreit\\ 
 & sô manige hôhe wirdicheit.\\ 
5 & des dankte er im sêre.\\ 
 & er het es selbe ouch êre.\\ 
 & innen des hiez tragen Gawan,\\ 
 & als ez unwizzen\textit{de} wær getân,\\ 
 & des heidens zimierde in den rinc.\\ 
10 & \textbf{daz} brüeften \textbf{si} \textbf{d\textit{â}} vür \textbf{hôhiu} dinc.\\ 
 & ritter und vrowen\\ 
 & begunden alle schouwen\\ 
 & den wâpenroc, den schilt, daz kursît.\\ 
 & der helm was zuo enge noch zuo wît.\\ 
15 & si prîsten alle gemeine\\ 
 & die \textbf{edeln tiuren} steine,\\ 
 & die dâran verwieret \textit{lâg}en.\\ 
 & niemen darf mich vrâgen\\ 
 & von ir arde, wie si wæren,\\ 
20 & die \textbf{liehten} und die swæren.\\ 
 & iuch het baz bescheiden des\\ 
 & Eraclius oder Hercules\\ 
 & und der Krieche Alexander\\ 
 & und dannoch ein ander,\\ 
25 & der wîse Pict\textit{ag}oras,\\ 
 & der ein astronomierre was\\ 
 & und \textbf{ouch} sô wîse âne strît,\\ 
 & \textbf{daz} nieman sît Adam\textit{e}s zît\\ 
 & \textbf{m\textit{ö}hte} im glîch\textit{en} sin \textbf{getragen}.\\ 
30 & \textbf{der} kunde \textbf{wol} von \textbf{steinen} sagen.\\ 
\end{tabular}
\scriptsize
\line(1,0){75} \newline
m n o V V' W Fr6 \newline
\line(1,0){75} \newline
\textbf{1} \textit{Initiale} m V V' W Fr6   $\cdot$ \textit{Capitulumzeichen} n  \textbf{7} \textit{Majuskel} Fr6  \newline
\line(1,0){75} \newline
\textbf{3} sîn] sint m \textbf{5} \textit{Die Verse 773.5-6 fehlen} V'   $\cdot$ dankte] dancket W (Fr6) \textbf{6} selbe] selbes W \textbf{7} innen] Jnnes o  $\cdot$ des] daz V  $\cdot$ tragen Gawan] tragen Gawon V [gawon]: tragen gawon V' tragen herr gawan W \textbf{8} unwizzende] vnwisen m vnwissen n (V') vmwissen o \textbf{9} heidens] haiden W (Fr6) \textbf{10} brüeften] prisete V'  $\cdot$ dâ] do m n o V \textit{om.} V' W  $\cdot$ vür] \textit{om.} o  $\cdot$ hôhiu] grosze V' (W) \textbf{12} begunden] Die begunden n \textbf{13} daz] dar n \textbf{14} zuo enge] noch zuͦ enge o \textbf{15} alle gemeine] algemeine Fr6 \textbf{16} edeln tiuren] túren edeln V (Fr6) turen edil V'  $\cdot$ steine] gesteine V' \textbf{17} verwieret lâgen] verwiret worent m verwircket wol logen n [ver*]: verwurket logen V verwurket logen V' verwircket lagen W (Fr6) \textbf{18} \textit{statt 773.18-30:} Jch magz uch alles nit gesogen V'   $\cdot$ darf] bedarff n \textbf{20} liehten] [*hten]: lihten V lihten Fr6 \textbf{22} Eraclius] eraclivs Fr6  $\cdot$ Hercules] Ercvles V ercves Fr6 \textbf{23} Krieche] kriech m n k:::g o crieche Fr6  $\cdot$ Alexander] allexander n \textbf{25} Pictagoras] pictoras m \textbf{26} der] Das o  $\cdot$ astronomierre] astronimus W \textbf{27} wîse] wit Fr6 \textbf{28} Adames] adamans m adams n o W \textbf{29} möhte] Mohtte m (o)  $\cdot$ glîchen] glich m o gliche n  $\cdot$ sin] sinne n sein W \textbf{30} kunde] kuͯnde m \newline
\end{minipage}
\end{table}
\newpage
\begin{table}[ht]
\begin{minipage}[t]{0.5\linewidth}
\small
\begin{center}*G
\end{center}
\begin{tabular}{rl}
 & der heiden was von herzen vrô,\\ 
 & daz sînes bruoder \textbf{dinc} alsô\\ 
 & stuont, daz sîn hant erstreit\\ 
 & sô manige hôhe werdecheit.\\ 
5 & des danket er im sêre.\\ 
 & er hets selbe ouch êre.\\ 
 & innen des hiez tragen Gawan,\\ 
 & als ez unwizzende wære getân,\\ 
 & des heidens zimierde in den rinc.\\ 
10 & \textbf{si} prüeveten \textbf{si} \textbf{dâ} vür \textbf{grôziu} dinc.\\ 
 & rîter unde vrouwen\\ 
 & begunden alle schouwen\\ 
 & den wâpenroc, \textbf{den helm}, den schilt, daz kursît.\\ 
 & der helm was zenge noch ze wît.\\ 
15 & si brîsten algemeine\\ 
 & die \textbf{tiuren edeln} steine,\\ 
 & die dran verwiert lâgen.\\ 
 & niemen darf mich vrâgen\\ 
 & von ir art, wie si wæren,\\ 
20 & die \textbf{liehten} unde die swæren.\\ 
 & iuch het baz bescheiden des\\ 
 & Eraculis oder Ercules\\ 
 & unde der Krieche Alexander\\ 
 & unde dannoch ein ander,\\ 
25 & der wîse Pitagoras,\\ 
 & der ein astronomierre was\\ 
 & unde \textbf{sus} sô wîse âne strît,\\ 
 & niemen sît Adames zît\\ 
 & \textbf{mohte} im gelîchen sin \textbf{getragen}.\\ 
30 & \textbf{er} kunde \textbf{baz} von \textbf{sternen} sagen.\\ 
\end{tabular}
\scriptsize
\line(1,0){75} \newline
G I L M Z Fr18 \newline
\line(1,0){75} \newline
\textbf{7} \textit{Initiale} I  \newline
\line(1,0){75} \newline
\textbf{3} erstreit] erstret I \textbf{5} des] Der L  $\cdot$ danket] dankete M (Z) \textbf{6} hets selbe ouch] het sin selbe auch I (Z) het ez ouch selbe L hattesz selben ouch M \textbf{7} innen] Bynnen M  $\cdot$ tragen] \textit{om.} Z \textbf{8} als] Tragen als Z \textbf{9} heidens] haiden I \textbf{10} prüeveten si] pruͯftenz L (M) (Z) (Fr18) \textbf{11} unde] vnd auch die I \textbf{12} begunden] Begvndens L \textbf{13} Wapen roch schilt kvrsit L  $\cdot$ den helm] \textit{om.} M Z Fr18  $\cdot$ daz] vnd M \textbf{15} brîsten] sprachen L  $\cdot$ algemeine] alle gemeine Z \textbf{16} steine] Gesteine I (L) \textbf{17} verwiert] verwerret I vervieret M \textbf{18} niemen darf mich] nun darf mich nieman I \textbf{19} wæren] waren L \textbf{20} liehten] lýchten L (M) (Z) \textbf{21} bescheiden] gescheiden Z \textbf{22} Eraculis] Eraclius I (L) (M) Eraklivs Z Eraclys Fr18  $\cdot$ oder] olde G vnd Z  $\cdot$ Ercules] hercuͯles L hercules M Z (Fr18) \textbf{23} Krieche] chrieche G crieche I Z  $\cdot$ Alexander] allexander M \textbf{25} Pitagoras] pictagoras Z Fr18 \textbf{26} astronomierre] astronomigus I wise astronomirre M astronomie Z \textbf{27} sô] \textit{om.} Fr18  $\cdot$ âne strît] an strite M \textbf{28} Adames] adamis M adams Z  $\cdot$ zît] gezcyte M \textbf{29} sin] han Z  $\cdot$ getragen] tragen I \textbf{30} er] Der L (M) Z Fr18  $\cdot$ sternen] steinen I Z sterne M \newline
\end{minipage}
\hspace{0.5cm}
\begin{minipage}[t]{0.5\linewidth}
\small
\begin{center}*T
\end{center}
\begin{tabular}{rl}
 & \begin{large}D\end{large}er heiden was von herzen vrô,\\ 
 & daz sînes bruoder \textbf{dinc} alsô\\ 
 & stuont, daz sîn hant erstreit\\ 
 & s\textit{ô} manege hôhe wirdecheit.\\ 
5 & des danket er im sêre.\\ 
 & er het es selber ouch êre.\\ 
 & indes hiez tragen Gawan,\\ 
 & als ez unwizzende wære getân,\\ 
 & des heidens zimierde in den rinc.\\ 
10 & \textbf{si} prüeveten \textbf{ez} vür \textbf{grôziu} dinc.\\ 
 & rîter und vrouwe\textit{n}\\ 
 & begunden alle schouwen\\ 
 & den wâpenroc, den schilt, daz kursît.\\ 
 & der helm was zuo enge noch zuo wît.\\ 
15 & si prîsten alle gemeine\\ 
 & die \textbf{tiuren edelen} steine,\\ 
 & die dran verwi\textit{e}ret lâgen.\\ 
 & nieman darf mich vrâgen\\ 
 & von ir arde, wie si wæren,\\ 
20 & die \textbf{lîhten} und die swæren.\\ 
 & iuch hete baz bescheiden des\\ 
 & Eraclius oder Ercules\\ 
 & und der Krieche Alexander\\ 
 & und dannoch ein ander,\\ 
25 & der wîse Pictagoras,\\ 
 & der ein astronomierre was\\ 
 & und \textbf{sus} sô wîse âne strît,\\ 
 & nieman sît Adames zît\\ 
 & \textbf{moht} im gelîchen sin \textbf{tragen}.\\ 
30 & \textbf{der} kunde \textbf{baz} von \textbf{steinen} sagen.\\ 
\end{tabular}
\scriptsize
\line(1,0){75} \newline
U Q R Fr53 \newline
\line(1,0){75} \newline
\textbf{1} \textit{Initiale} U Fr53  \newline
\line(1,0){75} \newline
\textbf{1} \textit{Die Verse 764.13-774.30 fehlen} R  \textbf{4} sô] Sol U \textbf{7} indes] Jnne Q  $\cdot$ Gawan] Gauwan U \textbf{10} prüeveten ez] prufftes do Q prvͤftenz da Fr53  $\cdot$ grôziu] hohev Fr53 \textbf{11} vrouwen] vreuͦwe U \textbf{13} den schilt] schilt Fr53  $\cdot$ kursît] kurist Q \textbf{15} alle gemeine] all gemeine Q \textbf{16} edelen] edle Fr53  $\cdot$ steine] gesteine Q \textbf{17} dran] da Fr53  $\cdot$ verwieret] verwirrit U verworcht Q \textbf{22} Eraclius] Eraclus U  $\cdot$ Ercules] Eccules U erculea Q \textbf{23} Oder alexander Fr53  $\cdot$ Krieche] kiche Q  $\cdot$ Alexander] allexander Q \textbf{24} dannoch] noch Fr53 \textbf{25} Pictagoras] pittagoras Q pytagoras Fr53 \textbf{26} astronomierre] astromirre Q \textbf{27} sus] als Q \textbf{28} Adames] adams Q Fr53 \textbf{29} moht] Noch Q  $\cdot$ sin] \textit{om.} Fr53  $\cdot$ tragen] getragen Q (Fr53) \textbf{30} steinen] sterne Q \newline
\end{minipage}
\end{table}
\end{document}
