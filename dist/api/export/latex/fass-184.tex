\documentclass[8pt,a4paper,notitlepage]{article}
\usepackage{fullpage}
\usepackage{ulem}
\usepackage{xltxtra}
\usepackage{datetime}
\renewcommand{\dateseparator}{.}
\dmyyyydate
\usepackage{fancyhdr}
\usepackage{ifthen}
\pagestyle{fancy}
\fancyhf{}
\renewcommand{\headrulewidth}{0pt}
\fancyfoot[L]{\ifthenelse{\value{page}=1}{\today, \currenttime{} Uhr}{}}
\begin{document}
\begin{table}[ht]
\begin{minipage}[t]{0.5\linewidth}
\small
\begin{center}*D
\end{center}
\begin{tabular}{rl}
\textbf{184} & \textbf{Ouch} was diu jæmerlîche schar\\ 
 & elliu nâch aschen \textbf{var}\\ 
 & oder als valwer leim.\\ 
 & \textbf{mîn hêrre, der} grâve von Wertheim,\\ 
5 & wære ungern soldier dâ gewesen.\\ 
 & er m\textit{ö}hte \textbf{ir} soldes niht genesen.\\ 
 & der zadel vuoget in hungers nôt.\\ 
 & si\textbf{ne} heten kæse, vleisch noch brôt.\\ 
 & \multicolumn{1}{l}{ - - - }\\ 
10 & \multicolumn{1}{l}{ - - - }\\ 
 & \multicolumn{1}{l}{ - - - }\\ 
 & \multicolumn{1}{l}{ - - - }\\ 
 & \multicolumn{1}{l}{ - - - }\\ 
 & \multicolumn{1}{l}{ - - - }\\ 
15 & \multicolumn{1}{l}{ - - - }\\ 
 & \multicolumn{1}{l}{ - - - }\\ 
 & \multicolumn{1}{l}{ - - - }\\ 
 & \multicolumn{1}{l}{ - - - }\\ 
 & des \textbf{twanc} si ein \textbf{werder} man,\\ 
20 & der \textbf{stolze} künec von Brandigan.\\ 
 & \multicolumn{1}{l}{ - - - }\\ 
 & \multicolumn{1}{l}{ - - - }\\ 
 & \multicolumn{1}{l}{ - - - }\\ 
 & \multicolumn{1}{l}{ - - - }\\ 
25 & \multicolumn{1}{l}{ - - - }\\ 
 & \multicolumn{1}{l}{ - - - }\\ 
 & \textbf{\begin{large}W\end{large}olt} ich \textbf{nû daz} wîzen in,\\ 
 & sô het ich harte kranken sin.\\ 
 & wan dâ ich dicke bin erbeizet\\ 
30 & \textbf{unt} \textbf{dâ} man mich \textbf{hêrre} heizet,\\ 
\end{tabular}
\scriptsize
\line(1,0){75} \newline
D Fr15 \newline
\line(1,0){75} \newline
\textbf{1} \textit{Initiale} Fr15   $\cdot$ \textit{Majuskel} D  \textbf{27} \textit{Initiale} D  \newline
\line(1,0){75} \newline
\textbf{4} Wertheim] :::heim Fr15 \textbf{6} möhte] mohte D \textbf{9} \textit{Die Verse 184.9-18 fehlen} D  \textbf{20} \textit{nach 184.20:} Der w:::e / er re:::e. Fr15   $\cdot$ Brandigan] :::ndigan Fr15 \textbf{21} \textit{Die Verse 184.21-26 fehlen} D  \textbf{27} \textit{Die Verse 184.27-185.12 fehlen} Fr15  \newline
\end{minipage}
\hspace{0.5cm}
\begin{minipage}[t]{0.5\linewidth}
\small
\begin{center}*m
\end{center}
\begin{tabular}{rl}
 & \textbf{ouch} was diu jâmerlîche schar\\ 
 & elli\textit{u} nâch aschen \textbf{gevar}\\ 
 & oder als valwer leim.\\ 
 & \textbf{mîn hêrre, der} grâve von Werthei\textit{m},\\ 
5 & wære ungerne soldier dâ gewesen.\\ 
 & er m\textit{ö}hte \textbf{ir} soldes niht genesen.\\ 
 & der zadel vuogt in hungers nôt.\\ 
 & si heten kæse, vleisch noch \textbf{daz} brôt.\\ 
 & si liezen \textbf{zenstürn} sîn\\ 
10 & \textbf{und} smalzeten ouch dekeinen wîn\\ 
 & mit ir munde, sô si trunken.\\ 
 & die wamme in nider sunken,\\ 
 & ir hüffe hôch und mager,\\ 
 & \textbf{geru\textit{m}pfen} als ein \textbf{ungersch} zager\\ 
15 & was diu hût \textbf{hin} zuo den riben.\\ 
 & \textbf{der hunger het in} daz vleisch vertriben.\\ 
 & den muosen si durch zadel doln.\\ 
 & in trouf \textbf{vil} \textbf{wênic} in die koln.\\ 
 & des \textbf{betwanc} si ein \textbf{werder} man,\\ 
20 & der \textbf{stolze} künic von Bra\textit{n}digan.\\ 
 & si arnden Cla\textit{m}ides bete.\\ 
 & sich \textbf{vergaz} d\textit{â} selten mit dem mete\\ 
 & zuber oder kanne.\\ 
 & ein \textbf{Truhinger} pfanne\\ 
25 & mit krapfen selten d\textit{â} erschrei.\\ 
 & in was der se\textit{l}be dôn enzwei.\\ 
 & \textbf{wolt} ich \textbf{nû daz} wîzen in,\\ 
 & sô het ich harte kranken sin.\\ 
 & wan dâ ich dicke bin erbeizet,\\ 
30 & \textbf{d\textit{â}} man mich \textbf{hêrre} heizet,\\ 
\end{tabular}
\scriptsize
\line(1,0){75} \newline
m n o Fr69 \newline
\line(1,0){75} \newline
\newline
\line(1,0){75} \newline
\textbf{2} elliu] Elli m \textbf{4} Wertheim] werthein m wertheẏmm o \textbf{5} dâ] do n o \textbf{6} möhte] mochte m (o) \textbf{8} daz] \textit{om.} n o \textbf{12} wamme in nider] wannen in wider o \textbf{13} hüffe] hoff o \textbf{14} gerumpfen] Gerupfen m  $\cdot$ ungersch] vngers n o \textbf{15} was] Was in n o \textbf{17} muosen] músten n  $\cdot$ durch] doch o \textbf{19} betwanc] twang n o \textbf{20} der] Do o  $\cdot$ künic] man vnd koͯnig n  $\cdot$ Brandigan] bradigan m [brang]: brandigran n \textbf{21} si] Sin o  $\cdot$ Clamides] clanides m \textbf{22} dâ selten] do seltten m (n) die solten o \textbf{23} kanne] pfanne n kannen o \textbf{24} Truhinger] drúhunder n druhunger o  $\cdot$ pfanne] kanne n pfannen o \textbf{25} selten] solten o  $\cdot$ dâ] do m n o  $\cdot$ erschrei] er schrein o \textbf{26} selbe] sebe m \textbf{28} sô] Sie o \textbf{29} dâ] do n o \textbf{30} dâ] Do m Vnd n o  $\cdot$ hêrre] here o \newline
\end{minipage}
\end{table}
\newpage
\begin{table}[ht]
\begin{minipage}[t]{0.5\linewidth}
\small
\begin{center}*G
\end{center}
\begin{tabular}{rl}
 & \textbf{ouch} was diu jæmerlîche schar\\ 
 & elliu nâch aschen \textbf{var}\\ 
 & oder alse valwer leim.\\ 
 & grâve \textbf{Poppe} von Wertheim\\ 
5 & wære ungerne soldier dâ gewesen.\\ 
 & er m\textit{ö}ht \textbf{ir} soldes niht genesen.\\ 
 & der zadel vuogte in hungers nôt.\\ 
 & si\textbf{ne} heten kæse, vleisch noch brôt.\\ 
 & si liezen \textbf{zenstüren} sîn\\ 
10 & \textbf{und} smalzten ouch deheinen wîn\\ 
 & mit ir munde, sô si trunken.\\ 
 & die wambe in nider sunken,\\ 
 & ir hüffe hôch unde mager,\\ 
 & \textbf{gerumpfen} als ein \textbf{Ungers} zager\\ 
15 & was \textbf{in} diu hût zuo den riben.\\ 
 & \textbf{der hunger het in} daz vleisch vertriben.\\ 
 & den muosen si durch zadel doln.\\ 
 & in trouf \textbf{vil} \textbf{lützel} in die koln.\\ 
 & des \textbf{twanc} si ein \textbf{stolzer} man,\\ 
20 & der \textbf{werde} künic von Brandigan.\\ 
 & si arnden Clamides bete.\\ 
 & si\textit{ch} \textbf{vergôz} dâ selten mit dem mete\\ 
 & \textbf{der} zuber oder \textbf{diu} kanne.\\ 
 & ein \textbf{Truhendingære} pfanne\\ 
25 & mit krapfen selten dâ erschrei.\\ 
 & in was der selbe dôn enzwei.\\ 
 & \textbf{solt} ich \textbf{daz nû} wîzen in,\\ 
 & sô het ich harte kranken sin.\\ 
 & wan dâ ich dicke bin erbeizet,\\ 
30 & \textbf{dâ} man mich \textbf{hêrren} heizet,\\ 
\end{tabular}
\scriptsize
\line(1,0){75} \newline
G I O L M Q R Z \newline
\line(1,0){75} \newline
\textbf{1} \textit{Initiale} L  \textbf{13} \textit{Initiale} I  \textbf{19} \textit{Initiale} G I Z  \newline
\line(1,0){75} \newline
\textbf{1} ouch] Do L  $\cdot$ jæmerlîche] Iemerlichu R \textbf{2} var] geuar I \textbf{3} alse] als ein I \textit{om.} O \textbf{4} grave ppope von Wertheim G Graue bopbe von werthaim I der grave von wertheim O Graf Boppe da von Werthein L der greffe von wertheym M Der grawe von werkeim Q der graue wertheim R Der grefe von wertheime Z \textbf{5} soldier] soldner Q [sold*]: soldener Z  $\cdot$ dâ] do Q \textbf{6} möht] moht G O (Q) en mochtes L en mochte M (Z)  $\cdot$ niht] da nicht M  $\cdot$ genesen] sin genesen I O (L) (Q) (R) sin M \textbf{7} zadel] saled Q  $\cdot$ vuogte in] vugt in I (Z) in fugte Q \textbf{8} sine] Si O [Hie]: Sie Q  $\cdot$ noch] vnd Q \textbf{9} si liezen] ze ezzen O Die lissen Q  $\cdot$ zenstüren] zene stuͯrgen L zen stoͤren Z \textbf{10} und] Si O  $\cdot$ smalzten] smalzegeten I (O) smahten L smelcztin M (Q) \textbf{11} ir] rin M  $\cdot$ munde] muͯden R  $\cdot$ trunken] trincken Q \textbf{12} die] Jr L Sein Q  $\cdot$ wambe] wamben I (Z) wagin M wangen Q  $\cdot$ in] im Q \textbf{13} hüffe] huf I Q hvffen L huten M  $\cdot$ hôch] hohc O hohe Q \textbf{14} gerumpfen] Verrvmpfen O  $\cdot$ ein] eins I (M) (Q) R  $\cdot$ zager] ager Q \textbf{15} in] im Q  $\cdot$ hût] huff R  $\cdot$ zuo den riben] zeriben O zcu den rebin M \textbf{16} der hunger het in] Jn het der hvnger L Der her vnger hatte M  $\cdot$ daz] ir I \textbf{17} den] Da O  $\cdot$ durch zadel doln] von mangel zoln R \textbf{18} Jn kroppff vil luczel in die keln R  $\cdot$ die] diu I \textbf{19} stolzer] vil stolzer I werder O Q (R) \textbf{20} werde] stolcz Q (R)  $\cdot$ Brandigan] bradigan G prandigan I Brandegan L \textbf{21} si] Die O  $\cdot$ arnden] armten Q  $\cdot$ Clamides] chlamides I klamides M Z Clamites Q  $\cdot$ bete] gebet Q \textbf{22} sich vergôz] si vergoz G Sich vergaz L Sie begosz Q Sich begoz Z  $\cdot$ dâ] do O Q  $\cdot$ dem] \textit{om.} O \textbf{24} ein] der I Noch O Er Q  $\cdot$ Truhendingære] truhendingare G Truͤhender I drvhendinger O truͯhendinger L truhendiger M truhendin gare Q kruͦg hangender R trvͤhendinger Z \textbf{25} Mit crapfen soltten der erschein R  $\cdot$ dâ] do Q \textbf{26} dôn] den M \textbf{27} solt] [Dolte]: Solte L Soͯltt R  $\cdot$ in] nye Q \textbf{29} dâ] do Q  $\cdot$ erbeizet] erbezt M (Q) (R) \textbf{30} dâ] vnd da I (O) (R) (Z) Vnd L (M) Vnd do Q  $\cdot$ mich] \textit{om.} I mich nv O  $\cdot$ hêrren] herre O M Q Z wirt L harte R  $\cdot$ heizet] reiczet R \newline
\end{minipage}
\hspace{0.5cm}
\begin{minipage}[t]{0.5\linewidth}
\small
\begin{center}*T
\end{center}
\begin{tabular}{rl}
 & \textbf{\begin{large}D\end{large}\textit{â}} was diu jâmerlîchiu schar\\ 
 & alliu nâch eschen \textbf{var}\\ 
 & oder als \textbf{ein} va\textit{l}wer lein.\\ 
 & \textbf{Mîn hêrre, der} grâve vo\textit{n} Werthein,\\ 
5 & wære ungerne soldier dâ gewesen.\\ 
 & er m\textit{ö}hte \textbf{des} soldes niht \textbf{sîn} genesen.\\ 
 & der zadel vuogt in hungers nôt.\\ 
 & si\textbf{ne} heten kæse, vleisch noch brôt.\\ 
 & si liezen \textbf{ir zene stürmen} sîn.\\ 
10 & \textbf{si} smalzten ouch dekeinen wîn\\ 
 & mit ir munde, sô si trunken.\\ 
 & die wamben in nider sunken,\\ 
 & ir hüffe hôch unde mager,\\ 
 & \textbf{rumpfen} als ein \textbf{ungersch} zager\\ 
15 & was \textbf{in} di\textit{u} hût \textbf{hin} zuo den riben.\\ 
 & \textbf{in hete der hunger} daz vleisch vertriben.\\ 
 & den muosen si durch zadel doln.\\ 
 & in trouf \textbf{wênic} in die koln.\\ 
 & des \textbf{twanc} si ein \textbf{werde\textit{r}} man,\\ 
20 & der \textbf{stolze} künec von Brandigan.\\ 
 & si arneten Clamides bete.\\ 
 & sich \textbf{vergôz} dâ selten mit dem mete\\ 
 & \textbf{der} zuber oder \textbf{diu} kanne.\\ 
 & ein \textbf{ellenwîtiu} pfanne\\ 
25 & mit krapfen selten dâ erschrei.\\ 
 & in was der selbe dôn enzwei.\\ 
 & \textbf{Solt} ich \textbf{nû daz} wîzen in,\\ 
 & sô heit ich harte kranken sin.\\ 
 & wan dâ ich dicke bin erbeizet\\ 
30 & \textbf{unde} man mich \textbf{wirt} heizet,\\ 
\end{tabular}
\scriptsize
\line(1,0){75} \newline
T U V W \newline
\line(1,0){75} \newline
\textbf{1} \textit{Initiale} T U V W  \textbf{4} \textit{Majuskel} T  \textbf{27} \textit{Initiale} W  \newline
\line(1,0){75} \newline
\textbf{1} Dâ] Do T U (V) (W) \textbf{3} ein] \textit{om.} W  $\cdot$ valwer] valuwer T \textbf{4} hêrre] herze U  $\cdot$ von] vo T  $\cdot$ Werthein] wertheim U V \textbf{5} soldier] soldener V  $\cdot$ dâ] [*]: do V aldo W \textbf{6} möhte] mohte T (U) V  $\cdot$ sîn] \textit{om.} U V W \textbf{7} Mangel der fvͦctin hvngers not V  $\cdot$ vuogt in] vochte U \textbf{8} sine heten] Seinen herren W  $\cdot$ brôt] daz brot W \textbf{9} ir] \textit{om.} W  $\cdot$ stürmen] [st*]: stúrmen V \textbf{10} smalzten] smahten V (W) \textbf{11} ir munde] irn muͦnden U (W) \textbf{12} wamben] warmmen U \textbf{13} hüffe] húffel W \textbf{14} rumpfen] Gervmfen V (W)  $\cdot$ als ein ungersch] alsein [vngeschi]: vngerschi T als [ein]: einz ungers V als ein vnger W \textbf{15} diu] die T  $\cdot$ hin] [*]: hin V \textit{om.} W \textbf{16} vleisch] flaich W \textbf{17} muosen] mvesen T (W) muͦzen U  $\cdot$ zadel] mangel V \textbf{18} wênic] auch lútzel W \textbf{19} werder] werden T \textbf{21} Clamides] Clamedes T U klamides W \textbf{22} vergôz] begoz V  $\cdot$ dâ] do V W \textbf{24} ein ellenwîtiu] Ein al zuͦ wite U In elen weiter W \textbf{25} dâ] do V W  $\cdot$ erschrei] er scherei T \textbf{27} wîzen] verweisen W \textbf{29} dâ] do V W \textbf{30} unde] [W*]: vnde T vnd wan U [*]: Do V \newline
\end{minipage}
\end{table}
\end{document}
