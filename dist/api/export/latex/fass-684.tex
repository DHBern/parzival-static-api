\documentclass[8pt,a4paper,notitlepage]{article}
\usepackage{fullpage}
\usepackage{ulem}
\usepackage{xltxtra}
\usepackage{datetime}
\renewcommand{\dateseparator}{.}
\dmyyyydate
\usepackage{fancyhdr}
\usepackage{ifthen}
\pagestyle{fancy}
\fancyhf{}
\renewcommand{\headrulewidth}{0pt}
\fancyfoot[L]{\ifthenelse{\value{page}=1}{\today, \currenttime{} Uhr}{}}
\begin{document}
\begin{table}[ht]
\begin{minipage}[t]{0.5\linewidth}
\small
\begin{center}*D
\end{center}
\begin{tabular}{rl}
\textbf{684} & \textbf{die} welt ir im \textbf{verkrenken}.\\ 
 & wie \textbf{megt} ir \textbf{des} \textbf{erdenken},\\ 
 & daz ir gein sîner swester sun\\ 
 & solch \textbf{ungenâde} wellet tuon?\\ 
5 & het iu der werde Gawan\\ 
 & grœzer herzeleit getân,\\ 
 & er m\textit{ö}hte der tavelrunder\\ 
 & doch geniezen sunder,\\ 
 & \textbf{wand} in geselleschefte wernt\\ 
10 & alle, die drüber pflihte gernt."\\ 
 & Der künec sprach: "den gelobten strît\\ 
 & mîn unverzagtiu hant sô gît,\\ 
 & daz ich Gawanen bî disem tage\\ 
 & \textbf{gein} prîse oder \textbf{in} laster sage.\\ 
15 & Ich hân \textbf{mit wârheit} vernomen,\\ 
 & \textbf{Artus} sî mit \textbf{storîe} komen\\ 
 & unt \textbf{des} wîp, diu künegîn;\\ 
 & diu sol willekomen sîn.\\ 
 & ob diu arge herzoginne\\ 
20 & \textbf{im} gein mir \textbf{ræt} unminne,\\ 
 & ir kint, daz sult ir understên.\\ 
 & \textbf{dâ}\textbf{ne} mac niht anders an ergên,\\ 
 & wan \textbf{daz} ich den kampf leisten wil.\\ 
 & ich hân rîter \textbf{wol sô} vil,\\ 
25 & daz ich gewalt entsitze niht.\\ 
 & swaz mir von einer hant geschiht,\\ 
 & die nôt wil ich lîden.\\ 
 & solt ich nû vermîden,\\ 
 & des ich mich vermezzen hân,\\ 
30 & sô wolt ich dienst nâch minnen lân.\\ 
\end{tabular}
\scriptsize
\line(1,0){75} \newline
D \newline
\line(1,0){75} \newline
\textbf{11} \textit{Majuskel} D  \textbf{15} \textit{Majuskel} D  \newline
\line(1,0){75} \newline
\textbf{7} möhte] mohte D \textbf{13} Gawanen] Gawann D \newline
\end{minipage}
\hspace{0.5cm}
\begin{minipage}[t]{0.5\linewidth}
\small
\begin{center}*m
\end{center}
\begin{tabular}{rl}
 & \textbf{die} wolt ir im \textbf{versenken}.\\ 
 & wie \textbf{m\textit{ö}ht} ir \textbf{daz} \textbf{erdenken},\\ 
 & daz ir gegen sîner swester sun\\ 
 & solich \textbf{ungenâde} wellet tuon?\\ 
5 & het iu der werde Gawan\\ 
 & grœzer herzeleit getân,\\ 
 & er m\textit{ö}hte der tavelrunder\\ 
 & doch geniezen sunder,\\ 
 & \textbf{wan} in geselleschaft werent\\ 
10 & alle, die dâr über pflihte gerent."\\ 
 & \begin{large}D\end{large}er künic sprach: "den gelobten strît\\ 
 & mîn unverzagtiu hant sô gît,\\ 
 & daz ich Gawanen bî disem tage\\ 
 & \textbf{gegen} prîse oder \textbf{in} laster sage.\\ 
15 & ich hab \textbf{ouch mære} vernomen,\\ 
 & \textbf{Artus} sî mit \textbf{storîe} komen\\ 
 & und \textbf{sîn} wîp, diu künigîn;\\ 
 & diu sol \textbf{uns} willekom sîn.\\ 
 & ob diu arge herzoginne\\ 
20 & \textbf{im} gegen mir \textbf{râtet} unminne,\\ 
 & ir kint, daz solt ir understân.\\ 
 & \textbf{d\textit{â}} mac niht anders an ergân,\\ 
 & wan ich de\textit{n k}ampf leisten wil.\\ 
 & ich hân \textbf{hie} ritter \textbf{alsô} vil,\\ 
25 & daz ich gewalt entsitze niht.\\ 
 & waz mir von einer hant geschiht,\\ 
 & die nôt wil ich lîden.\\ 
 & solt ich nû vermîden,\\ 
 & des ich mich vermezzen hân,\\ 
30 & sô wolt ich dienst nâch minne lân.\\ 
\end{tabular}
\scriptsize
\line(1,0){75} \newline
m n o Fr69 \newline
\line(1,0){75} \newline
\textbf{11} \textit{Initiale} m n  \newline
\line(1,0){75} \newline
\textbf{1} ir] er n o  $\cdot$ versenken] verkrencken n o \textbf{2} möht] moht m (o) mochten n \textbf{7} möhte] mohtte m (o) \textbf{10} pflihte] pflichtig o \textbf{15} ouch] uͯch o \textbf{16} Artus] Artús o \textbf{20} gegen] geben o  $\cdot$ unminne] mynne n \textbf{22} dâ] Do m n o \textbf{23} den kampf] den konig kampf m \textbf{29} des] Das n  $\cdot$ vermezzen] vermesse o \textbf{30} minne] minnen Fr69 \newline
\end{minipage}
\end{table}
\newpage
\begin{table}[ht]
\begin{minipage}[t]{0.5\linewidth}
\small
\begin{center}*G
\end{center}
\begin{tabular}{rl}
 & \textbf{\begin{large}W\end{large}ie} welt ir im \textbf{verkrenken},\\ 
 & wie \textbf{muget} ir \textbf{des} \textbf{gedenken},\\ 
 & daz ir gein sîner swester sun\\ 
 & solhe \textbf{ungevuoge} welt tuon?\\ 
5 & hiet iu der werde Gawan\\ 
 & \textbf{noch} grœzer herzeleit getân,\\ 
 & er m\textit{ö}hte der tavelrunder\\ 
 & \textbf{ie} doch geniezen sunder,\\ 
 & \textbf{sît} in geselleschefte wernt\\ 
10 & alle, die drüber pflihte gernt."\\ 
 & der künic sprach: "den gelobten strît\\ 
 & mîn unverzagetiu hant sô gît,\\ 
 & daz ich Gawan bî disem tage\\ 
 & \textbf{in} brîs oder \textbf{in} laster sage.\\ 
15 & ich hân \textbf{mit wârheit} vernomen,\\ 
 & \textbf{der künic} sî mit \textbf{storîe} komen\\ 
 & unde \textbf{sîn} wîp, diu künigîn;\\ 
 & diu sol \textbf{hie} willekomen sîn.\\ 
 & op diu arge herzoginne\\ 
20 & \textbf{im} gein mir \textbf{râte} unminne,\\ 
 & ir kint, daz solt ir understên.\\ 
 & \textbf{hie}\textbf{ne} mac niht a\textit{nd}e\textit{r}s an ergên,\\ 
 & wan \textbf{daz} ich den kampf leisten wil.\\ 
 & ich hân \textbf{doch} rîter \textbf{wol sô} vil,\\ 
25 & daz ich gewalt entsitze niht.\\ 
 & swaz mir von einer hant geschiht,\\ 
 & die nôt wil ich lîden.\\ 
 & solde ich nû vermîden,\\ 
 & des ich mich vermezzen hân,\\ 
30 & sô wolde ich dienst nâch minne lân.\\ 
\end{tabular}
\scriptsize
\line(1,0){75} \newline
G I L M Z Fr18 Fr20 Fr52 \newline
\line(1,0){75} \newline
\textbf{1} \textit{Initiale} G L Z Fr20  \textbf{11} \textit{Initiale} I  \newline
\line(1,0){75} \newline
\textbf{1} Wie] Die L (M) Z (Fr20) (Fr52)  $\cdot$ ir im] irn sus I  $\cdot$ verkrenken] nv krenken Fr52 \textbf{2} ir] ir ev I  $\cdot$ gedenken] erdenchen I (L) (M) (Z) (Fr52) \textbf{4} ungevuoge] vnfuͯge L (M) (Z) (Fr20) (Fr52)  $\cdot$ welt] muget I \textbf{7} möhte] môhte G moht I (L) (M) (Z) Fr18 (Fr20) (Fr52) \textbf{8} sunder] besvnder Fr52 \textbf{9} sît] sit si I  $\cdot$ in] im Fr52 \textbf{11} den gelobten] der gelobite Fr20 \textbf{13} Gawan] Gawane L (Fr52) Gawanen Fr20  $\cdot$ tage] tate I \textbf{14} pris oder laster sage Fr52 \textbf{15} hân] hat Fr18 \textbf{16} storîe] stvrie G Z stivre L (M) (Fr18) strite Fr20 \textbf{17} sîn] des L M Z Fr18 Fr20 Fr52 \textbf{18} sol] suln M (Fr18) \textbf{20} râte unminne] ratet vnminne I (Fr18) rote vnde mynne M \textbf{21} daz] \textit{om.} M \textbf{22} anders] arges G  $\cdot$ an] \textit{om.} I \textbf{23} daz] als I \textbf{24} wol sô] also I L M \textbf{25} gewalt] den gewalt I \textbf{26} swaz] Waz L (M)  $\cdot$ hant] han Fr18 \textbf{30} nâch] \textit{om.} Fr20  $\cdot$ minne] minnen I Fr18 \newline
\end{minipage}
\hspace{0.5cm}
\begin{minipage}[t]{0.5\linewidth}
\small
\begin{center}*T
\end{center}
\begin{tabular}{rl}
 & \textbf{die} w\textit{e}lt ir \textit{i}me \textbf{verkrenken}.\\ 
 & wie \textbf{moget} ir \textbf{des} \textbf{gedenken},\\ 
 & daz ir gein sîner swester sun\\ 
 & soliche \textbf{ungevuoge} wolt tuon?\\ 
5 & het iu der werde Gawan\\ 
 & \textbf{noch} grœzer herzeleit getân,\\ 
 & er m\textit{ö}hte der tavelrunder\\ 
 & \textbf{ie} doch geniezen sunder,\\ 
 & \textbf{wan} in geselleschefte wernt\\ 
10 & alle, die dâr über pflihte gernt."\\ 
 & der künec sprach: "den gelobeten strît\\ 
 & mîn unverzagetiu hant sô gît,\\ 
 & daz ich Gawanen bî disem tage\\ 
 & \textbf{gein} prîse oder \textbf{gein} laster sage.\\ 
15 & ich hân \textbf{mit wârheit} vernomen,\\ 
 & \textbf{der künec} sî mit \textbf{sturme} komen\\ 
 & und \textbf{des} wîp, diu künegîn;\\ 
 & diu sol \textbf{hie} willekomen sîn.\\ 
 & ob diu arge herzoginne\\ 
20 & gein mir \textbf{râte} unminne,\\ 
 & ir kint, d\textit{az solt} ir understân.\\ 
 & \textbf{hie} \textbf{en}mac niht anders ane ergân,\\ 
 & wan \textbf{daz} ich den kampf leisten wil.\\ 
 & ich hân \textbf{doch} rîter \textbf{wol sô} vil,\\ 
25 & daz ich gewalt entsitze niht.\\ 
 & waz mir von einer hant geschiht,\\ 
 & die nôt wil ich lîden.\\ 
 & solt ich nû vermîden,\\ 
 & des ich mich vermezzen hân,\\ 
30 & sô wolt ich dienst nâch minne lân.\\ 
\end{tabular}
\scriptsize
\line(1,0){75} \newline
U V W Q R \newline
\line(1,0){75} \newline
\textbf{11} \textit{Initiale} R  \newline
\line(1,0){75} \newline
\textbf{1} welt] werlt U  $\cdot$ ime] me U \textbf{2} moget] went V mocht Q  $\cdot$ des] nv V das W (R)  $\cdot$ gedenken] erdenken R \textbf{4} ungevuoge] vnvuͦge V (W) (Q) (R)  $\cdot$ wolt] wolte W wollen Q \textbf{6} noch grœzer] Nach W \textbf{7} er möhte] Er mochte U Q Jr moͯchttent R \textbf{8} Lassen geniessen ye doch besunder R  $\cdot$ sunder] besunder Q \textbf{11} den gelobeten] de lopten R \textbf{12} unverzagetiu] vnuerczagtte R \textbf{13} Gawanen] herr gawan W gaben Q Gawin R \textbf{14} gein laster] in laster V W Q R \textbf{15} mit] mich Q \textbf{16} Artus [*]: si mit storie komen V  $\cdot$ sturme] stosse W storie Q R \textbf{17} des] [*z]: sin V \textbf{18} hie] [*]: vnz V  $\cdot$ willekomen] vil komen Q \textbf{20} gein] [*]: Jm gegen V Im gegen W (Q) (R)  $\cdot$ râte] [*]: ratet V \textbf{21} daz solt] die U \textbf{22} enmac] mag R  $\cdot$ ane] \textit{om.} R \textbf{23} den] [dem]: den V \textbf{24} doch] [*]: hie V \textbf{26} waz] Swaz V  $\cdot$ geschiht] darumb beschicht R \textbf{28} solt] Sol Q \textbf{30} nâch] von Q \newline
\end{minipage}
\end{table}
\end{document}
