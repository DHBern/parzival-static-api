\documentclass[8pt,a4paper,notitlepage]{article}
\usepackage{fullpage}
\usepackage{ulem}
\usepackage{xltxtra}
\usepackage{datetime}
\renewcommand{\dateseparator}{.}
\dmyyyydate
\usepackage{fancyhdr}
\usepackage{ifthen}
\pagestyle{fancy}
\fancyhf{}
\renewcommand{\headrulewidth}{0pt}
\fancyfoot[L]{\ifthenelse{\value{page}=1}{\today, \currenttime{} Uhr}{}}
\begin{document}
\begin{table}[ht]
\begin{minipage}[t]{0.5\linewidth}
\small
\begin{center}*D
\end{center}
\begin{tabular}{rl}
\textbf{455} & \begin{large}S\end{large}us schreip \textbf{dar von} Flegetanis.\\ 
 & Kyot, der meister wîs,\\ 
 & diz mære begunde suochen\\ 
 & in latînischen buochen,\\ 
5 & wâ gewesen wære\\ 
 & ein volc dâ zuo \textbf{gebære},\\ 
 & daz ez des Grâles pflæge\\ 
 & unt der kiusche sich bewæge.\\ 
 & er la\textit{s} der lande chrônica\\ 
10 & ze Brittanie unt anderswâ,\\ 
 & ze Francrîche unt in \textbf{Yrlant}.\\ 
 & ze Anschouwe er diu mære vant.\\ 
 & er las von Mazadan\\ 
 & mit wârheit sunder wân.\\ 
15 & \textbf{umb} allez sîn geslehte\\ 
 & stuont dâ geschriben rehte\\ 
 & unt anderhalp, wie Titurel\\ 
 & unt des sun Frimutel\\ 
 & den Grâl \textbf{bræhte} ûf Anfortas,\\ 
20 & des swester Herzeloyde  was,\\ 
 & bî der Gahmuret ein kint\\ 
 & gewan, des \textbf{disiu} mære sint.\\ 
 & Der \textbf{rîtet} nû ûf \textbf{die} niwen slâ,\\ 
 & \textbf{die} gein im kom der rîter grâ.\\ 
25 & er \textbf{erkande} \textbf{eine} stat, swie læge der snê,\\ 
 & dâ liehte bluomen stuonden ê.\\ 
 & daz was vor eines gebirges want,\\ 
 & \textbf{al} dâ sîn manlîchiu hant\\ 
 & vroun Jeschuten die hulde erwarp\\ 
30 & unt dâ Oriluses zorn verdarp.\\ 
\end{tabular}
\scriptsize
\line(1,0){75} \newline
D \newline
\line(1,0){75} \newline
\textbf{1} \textit{Initiale} D  \textbf{23} \textit{Majuskel} D  \newline
\line(1,0){75} \newline
\textbf{9} las] la: \textit{nachträglich korrigiert zu:} las D \textbf{12} Anschouwe] Anschoͮwe D \textbf{17} Titurel] Tytvrel D \textbf{20} Herzeloyde] Herzeloͮyde D \textbf{21} Gahmuret] Gahmvret D \textbf{29} Jeschuten] Jescvten D \textbf{30} Oriluses] Orilvs D \newline
\end{minipage}
\hspace{0.5cm}
\begin{minipage}[t]{0.5\linewidth}
\small
\begin{center}*m
\end{center}
\begin{tabular}{rl}
 & sus schreip \textbf{der wîse} Fl\textit{e}getanis.\\ 
 & Kiot, der meister wîs,\\ 
 & diz mære begunde suochen\\ 
 & in latînischen buochen,\\ 
5 & wâ gewesen wære\\ 
 & ein volc dar zuo \textbf{bære},\\ 
 & daz ez des Grâles pflæge\\ 
 & und der kiusch sich bewæge.\\ 
 & er las der land\textit{e} chrônica\\ 
10 & zuo Bretanie und anderswâ,\\ 
 & zuo Francrîch und in \textbf{ir lant}.\\ 
 & zuo Anschouwe er diu mære vant.\\ 
 & er las von Mazadan\\ 
 & mit wârheit sunder wân.\\ 
15 & \textbf{umb} allez sîn gesleht\\ 
 & stuont d\textit{â} geschriben reht\\ 
 & und anderhalp, wi\textit{e} \textit{T}iturel\\ 
 & und des sun Frim\textit{u}tel\\ 
 & den Grâl \textbf{brâht} \textit{ûf} Anfortas,\\ 
20 & des swester Herczeloide was,\\ 
 & bî der Gahmuret ein kint\\ 
 & g\textit{e}wan, des \textbf{diu} mære sint.\\ 
 & der \textbf{rîtet} nû ûf niuwe slâ,\\ 
 & \textbf{d\textit{â}} gegen im kam der ritter grâ.\\ 
25 & er \textbf{erkante} \textbf{ein} stat, wie læge der snê,\\ 
 & d\textit{â} liehte bluomen stuonden ê.\\ 
 & daz was vor eines gebirges want,\\ 
 & \textbf{al}dâ sîn manlîchiu hant\\ 
 & vrouwen Jeschuten die hulde erwarp\\ 
30 & und d\textit{â} Oriluses zorn verdarp.\\ 
\end{tabular}
\scriptsize
\line(1,0){75} \newline
m n o \newline
\line(1,0){75} \newline
\newline
\line(1,0){75} \newline
\textbf{1} Flegetanis] flagetanis m pflegetanis o \textbf{2} Kiot] Kẏot m n \textbf{3} diz] Dise n \textbf{6} bære] gebere n o \textbf{9} lande] landes m \textbf{10} Bretanie] britanie n britane o \textbf{11} Francrîch] franckrich m franckenrich n franckerich o  $\cdot$ ir] irem n \textbf{12} Anschouwe] anschowe o \textbf{16} dâ] do m n o \textbf{17} wie Titurel] wie die titúrel m wie titturel n \textbf{18} Frimutel] [frimu*] frimuertel m frumitel n frimuͯtel o \textbf{19} Grâl] grole n  $\cdot$ ûf] \textit{om.} m \textbf{20} Herczeloide] hertzoiloẏde n herczeleide o \textbf{21} Gahmuret] gamúret n gahuͯmet o \textbf{22} gewan] Gawan m o \textbf{24} dâ] Do m n o \textbf{26} dâ] Do m n o \textbf{29} vrouwen] Frouwe m (n) (o)  $\cdot$ Jeschuten] jescutten m jescuten n jescúte o \textbf{30} dâ] do m n o  $\cdot$ Oriluses] orilus m n o \newline
\end{minipage}
\end{table}
\newpage
\begin{table}[ht]
\begin{minipage}[t]{0.5\linewidth}
\small
\begin{center}*G
\end{center}
\begin{tabular}{rl}
 & \begin{large}S\end{large}us schreip \textbf{dar von} Fleigetanis.\\ 
 & Kiot, der meister wîs,\\ 
 & d\textit{i}z mære begunde suochen\\ 
 & in latînischen buochen,\\ 
5 & wâ gewesen wære\\ 
 & ein volc dar zuo \textbf{gebære},\\ 
 & daz ez des Grâles pflæge\\ 
 & unt der kiusche sich bewæge.\\ 
 & \textit{er las der lande chrônica}\\ 
10 & ze Britannia unde anderswâ,\\ 
 & ze Francrîche unde in \textbf{ir lant}.\\ 
 & ze Anschouwe er diu mære vant.\\ 
 & er las von Mazadan\\ 
 & mit wârheit sunder wân.\\ 
15 & \textbf{über} allez sîn geslehte\\ 
 & \textit{stuont dâ geschriben rehte}\\ 
 & unt anderhalp, \textit{wie} Titurel\\ 
 & unt des sun Frimutel\\ 
 & den Grâl \textbf{brâht} ûf Anfortas,\\ 
20 & des swester Herzeloide was,\\ 
 & bî der Gahmuret ein kint\\ 
 & gewan, des \textbf{disiu} mære sint.\\ 
 & der \textbf{rîtet} nû ûf \textbf{die} niuwen slâ,\\ 
 & \textbf{die} gein im kom der rîter grâ.\\ 
25 & er \textbf{erkande} \textbf{ein} stat, swie læge der snê,\\ 
 & dâ liehte bluomen stuonden ê.\\ 
 & daz was vor eines \textit{ge}birges want,\\ 
 & dâ sîn manlîchiu hant\\ 
 & vrôn Jeschuten die hulde erwarp\\ 
30 & unt dâ Orilluses zorn verdarp.\\ 
\end{tabular}
\scriptsize
\line(1,0){75} \newline
G I O L M Z \newline
\line(1,0){75} \newline
\textbf{1} \textit{Initiale} G I O L Z  \textbf{17} \textit{Initiale} I  \newline
\line(1,0){75} \newline
\textbf{1} Sus] ÷vs O  $\cdot$ von] \textit{om.} L  $\cdot$ Fleigetanis] flegitamus I flegetanis O (L) Z flogetanis M \textbf{2} Kiot] kyot O (M) (Z) Kýot L \textbf{3} \textit{Versdoppelung} M   $\cdot$ diz] Daz G  $\cdot$ begunde] bigonde her M \textbf{6} volc] vol M  $\cdot$ dar] daz I \textbf{8} kiusche] vnkuͯsche L \textbf{9} \textit{Vers 455.9 fehlt} G  \textbf{10} ze Britannia] zebritannia G zepritanie I Ze britane O Zuͯ Brittanie L \textbf{11} ze Francrîche] ze franchreche G zefranchenrich I Ze franchriche O Zuͯ Franckriche L Zcu francrich M  $\cdot$ in ir lant] in irgelant I inir lant O in Jrlant L ir lant M in irlant Z \textbf{12} Anschouwe] Anschoͮwę G antshoͮve I anshawe O anscowe M anschowe Z \textbf{15} über] Vmbe O (L) (M) (Z)  $\cdot$ allez] alle M \textbf{16} \textit{Vers 455.16 fehlt} G  \textbf{17} unt] \textit{om.} I  $\cdot$ wie] von G  $\cdot$ Titurel] tutulel I Tytvrel O Týtuͯrel L \textbf{18} Frimutel] frimuntel I frymuͯtel M \textbf{19} brâht] brehte I (O)  $\cdot$ Anfortas] Amfortas L \textbf{20} des] de I  $\cdot$ Herzeloide] herzeloyde G herzenlaude I herzelavde O Hertzelouͯde L herzeloude M herzenlovde Z \textbf{21} der] \textit{om.} M  $\cdot$ Gahmuret] gamuret G Z Gamvret O Gahmvret L gamuͯret M \textbf{22} sint] sin L \textbf{23} rîtet] ritter M Z  $\cdot$ niuwen] Nuwe M \textbf{25} erkande] bechande I  $\cdot$ ein] die O L  $\cdot$ swie] swie da I wie L (M) \textbf{26} dâ] \sout{swie} da I Do O  $\cdot$ liehte] lychte L (M)  $\cdot$ stuonden] stuͤnden I \textbf{27} vor] von M  $\cdot$ gebirges] birges G  $\cdot$ want] vant Z \textbf{28} dâ] Alda O L M Z \textbf{29} vrôn] Vrov L (M)  $\cdot$ Jeschuten] ieschuten G iesscuten I Jescvͦten O Jescuͯten L iescuten M (Z)  $\cdot$ erwarp] [gewan]: erwarp G \textbf{30} dâ] \textit{om.} L  $\cdot$ Orilluses] orillus G Orilus I (O) L M (Z) \newline
\end{minipage}
\hspace{0.5cm}
\begin{minipage}[t]{0.5\linewidth}
\small
\begin{center}*T
\end{center}
\begin{tabular}{rl}
 & \begin{large}S\end{large}us schreip \textbf{dar von} Flegetanis.\\ 
 & Kyot, der meister wîs,\\ 
 & diz mære begunde suochen\\ 
 & in latînischen buochen,\\ 
5 & wâ gewesen wære\\ 
 & ein volc dâ zuo \textbf{gebære},\\ 
 & daz ez des Grâles pflæge\\ 
 & unde der kiusche sich bewæge.\\ 
 & er las der lande chrônica\\ 
10 & ze Britanie unde anderswâ,\\ 
 & ze Francrîche unde in \textbf{Yrlant}.\\ 
 & ze Anschouwe er di\textit{u} mære vant.\\ 
 & e\textit{r} las von Mazadan\\ 
 & mit wârheite sunder wân.\\ 
15 & \textbf{umbe} allez sîn geslehte\\ 
 & stuont dâ geschriben rehte\\ 
 & unde anderhalp, wie Tyturel\\ 
 & unde des sun Frimutel\\ 
 & den Grâl \textbf{brâhte} ûf Anfortas,\\ 
20 & des swester Herzeloyde was,\\ 
 & bî der Gahmuret ein kint\\ 
 & gewan, des \textbf{dis\textit{iu}} mære sint.\\ 
 & der \textbf{reit} nû ûf \textbf{die} niuwen slâ,\\ 
 & \textbf{die} gegen im kom der rîter grâ.\\ 
25 & er \textbf{bekande} \textbf{die} stat, swie læge der snê,\\ 
 & dâ liehte bluomen stuonden ê.\\ 
 & daz was vor eines gebirges want,\\ 
 & \textbf{al}dâ sîn manlîch\textit{iu} hant\\ 
 & vroun Jeschuten die hulde erwarp\\ 
30 & unde dâ Oriluses zorn verdarp.\\ 
\end{tabular}
\scriptsize
\line(1,0){75} \newline
T U V W Q R \newline
\line(1,0){75} \newline
\textbf{1} \textit{Initiale} T V W Q R  \newline
\line(1,0){75} \newline
\textbf{1} \textit{Die Verse 453.1-502.30 fehlen} U   $\cdot$ dar von] der wise V do W  $\cdot$ Flegetanis] Flegetanîs T flegatanis R \textbf{2} Kyot] Koyt Q \textbf{3} diz] Die W \textbf{4} in] Jn den R \textbf{7} ez] er W \textbf{9} lande] landen R  $\cdot$ chrônica] karonica V koronica R \textbf{10} Britanie] brittanie V britania W britange Q (R) \textbf{11} Francrîche] francrich V franckreich W franckreiche Q frankriche R  $\cdot$ in Yrlant] in Jrlant T in ir lant Q Jm ẏrland R \textbf{12} Anschouwe] anschowe V Q R antschowe W  $\cdot$ diu] die T \textbf{13} er] ez T  $\cdot$ Mazadan] Mazadân T mazedan V [mandadan]: mansadan R \textbf{15} geslehte] geschechte W (Q) \textbf{16} dâ] do V W Q das R \textbf{17} Tyturel] titurel Q \textbf{18} Frimutel] Frymvtel T [frim*el]: frimuntel V frumutel R \textbf{19} brâhte] brachten W  $\cdot$ Anfortas] anfortes R \textbf{20} des] Der Q  $\cdot$ Herzeloyde] herzelaude V hertzeloyde W heyszeloude Q herczelaude R \textbf{21} Gahmuret] gamuret V W gamúret Q \textbf{22} disiu] dise T R \textbf{23} reit] [*]: ritet V Rittett R ritter Q \textbf{24} die] Do V W Q  $\cdot$ im] \textit{om.} Q \textbf{25} bekande] erkante V (W) kant Q (R)  $\cdot$ die] [*]: die V  $\cdot$ swie] wie W R wie do Q \textbf{26} dâ] Do V W Q  $\cdot$ liehte] lichte Q \textbf{28} manlîchiu] manliche T \textbf{29} vroun] Fraw W Q (R)  $\cdot$ Jeschuten] Jescvten T (R) iescuten V Q iestuten W \textbf{30} dâ] do V Q \textit{om.} W  $\cdot$ Oriluses] Orilus T (V) (W) (Q) (R)  $\cdot$ zorn] zurnen Q \newline
\end{minipage}
\end{table}
\end{document}
