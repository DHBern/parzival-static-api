\documentclass[8pt,a4paper,notitlepage]{article}
\usepackage{fullpage}
\usepackage{ulem}
\usepackage{xltxtra}
\usepackage{datetime}
\renewcommand{\dateseparator}{.}
\dmyyyydate
\usepackage{fancyhdr}
\usepackage{ifthen}
\pagestyle{fancy}
\fancyhf{}
\renewcommand{\headrulewidth}{0pt}
\fancyfoot[L]{\ifthenelse{\value{page}=1}{\today, \currenttime{} Uhr}{}}
\begin{document}
\begin{table}[ht]
\begin{minipage}[t]{0.5\linewidth}
\small
\begin{center}*D
\end{center}
\begin{tabular}{rl}
\textbf{412} & als tet ouch Kyngrimursel.\\ 
 & gein ellen si bêde wâren snel.\\ 
 & \textit{\begin{large}D\end{large}}er künec mant die sîne:\\ 
 & "wie lange sulen wir pîne\\ 
5 & von \textbf{disen} zwein mannen pflegen?\\ 
 & mînes vetern sun hât sich bewegen,\\ 
 & er \textbf{wil} erneren disen man,\\ 
 & der \textbf{mir} den schaden hât getân,\\ 
 & den er \textbf{billîcher} ræche,\\ 
10 & ob im \textbf{ellens} niht gebræche."\\ 
 & Genuoge, dens ir triwe jach,\\ 
 & kurn einen, der zem künege sprach:\\ 
 & "hêrre, muoze wirz iu sagen,\\ 
 & der lantgrâve ist unerslagen\\ 
15 & hie von maneger hende.\\ 
 & got iuch an site wende,\\ 
 & die man iu vervâhe baz.\\ 
 & werltlîch prîs iu sînen haz\\ 
 & teilet, \textbf{erslahet} ir iwern gast.\\ 
20 & ir ladet ûf iuch der schanden last.\\ 
 & Sô ist der ander iwer mâc,\\ 
 & in des geleite ir disen bâc\\ 
 & \textbf{hebt}. \textbf{daz} sult ir lâzen.\\ 
 & ir sît dâr von verwâzen.\\ 
25 & Nû gebt uns einen vride her,\\ 
 & die wîle daz dirre tac \textbf{gewer}.\\ 
 & der vride sî ouch dise naht.\\ 
 & wes ir iuch drumbe habt bedâht,\\ 
 & daz stêt dannoch \textbf{z}iwerer hant,\\ 
30 & ir sît geprîset oder geschant.\\ 
\end{tabular}
\scriptsize
\line(1,0){75} \newline
D \newline
\line(1,0){75} \newline
\textbf{3} \textit{Initiale} D  \textbf{11} \textit{Majuskel} D  \textbf{21} \textit{Majuskel} D  \textbf{25} \textit{Majuskel} D  \newline
\line(1,0){75} \newline
\textbf{3} Der] ÷er \textit{nachträglich korrigiert zu:} Der D \newline
\end{minipage}
\hspace{0.5cm}
\begin{minipage}[t]{0.5\linewidth}
\small
\begin{center}*m
\end{center}
\begin{tabular}{rl}
 & als tet ouch Ki\textit{n}gri\textit{m}u\textit{r}sel.\\ 
 & gegen ellen si beide wâren snel.\\ 
 & \begin{large}D\end{large}er künic mante die sîne:\\ 
 & "wie lange sullen wir pîne\\ 
5 & von \textbf{den} zwein mannen pflegen?\\ 
 & mînes vettern sun hât sich b\textit{e}wegen,\\ 
 & er \textbf{welle} ernern disen man,\\ 
 & der \textbf{mir} de\textit{n} schaden hât getân,\\ 
 & den er \textbf{billîcher} ræche,\\ 
10 & ob ime \textbf{ellens} niht gebræche."\\ 
 & genuoge, den es ir triuwe jach,\\ 
 & kurn einen, de\textit{r} zuo dem künige sprach:\\ 
 & "hêrre, muozen wir ez iu sagen,\\ 
 & der lantgrâve ist u\textit{n}erslagen\\ 
15 & hie von maniger hende.\\ 
 & got iuch an site wende,\\ 
 & die man iu ver\textit{v}âhe baz.\\ 
 & werltlîch prîs iu sînen haz\\ 
 & teilet, \textbf{erslahet} \textit{ir} iuwern gast.\\ 
20 & ir ladet ûf iuch der schanden last.\\ 
 & sô ist \textit{d}er ander iuwer mâc,\\ 
 & in des geleite \dag er\dag  disen bâc\\ 
 & \textbf{hebet}. \textbf{daz} solt ir lâzen.\\ 
 & ir sît dâ von verwâzen.\\ 
25 & nû gebet uns einen vride her,\\ 
 & die wîle daz dirre tac \textbf{gewer}.\\ 
 & der vride sî ouch dise naht.\\ 
 & wes ir iuch dar umbe habet bedâht,\\ 
 & daz stât dannoch \textit{\textbf{zuo}} iuwerre hant,\\ 
30 & ir sît geprîset oder geschant.\\ 
\end{tabular}
\scriptsize
\line(1,0){75} \newline
m n o \newline
\line(1,0){75} \newline
\textbf{3} \textit{Initiale} m   $\cdot$ \textit{Capitulumzeichen} n  \newline
\line(1,0){75} \newline
\textbf{1} Kingrimursel] kimgrinvsel m kingrumursel n konigrumel o \textbf{2} beide wâren] worent beide n \textbf{4} wir] ir o \textbf{5} von] Wir suͯllen von o  $\cdot$ den] disen n o \textbf{6} hât] hette n  $\cdot$ bewegen] betwegen m \textbf{7} er welle ernern] Erwellen ennere o \textbf{8} den] der m \textbf{12} kurn] Kúren n  $\cdot$ der] den m \textbf{14} unerslagen] vnuerslagen m \textbf{17} vervâhe] verhahe m versehe n \textbf{18} iu] uͯsz o \textbf{19} ir] \textit{om.} m  $\cdot$ iuwern] vwer o \textbf{20} schanden] schande o \textbf{21} der] er m \textbf{23} solt] sol o \textbf{24} dâ] [an]: do n \textbf{25} einen] einem n  $\cdot$ vride] friden o \textbf{27} ouch] úch n (o) \textbf{28} habet] hett o \textbf{29} zuo] \textit{om.} m  $\cdot$ iuwerre] yre m \newline
\end{minipage}
\end{table}
\newpage
\begin{table}[ht]
\begin{minipage}[t]{0.5\linewidth}
\small
\begin{center}*G
\end{center}
\begin{tabular}{rl}
 & als tet ouch Kingrimursel.\\ 
 & gein ellen si bêde wâren snel.\\ 
 & der künic mante die sîne:\\ 
 & "wie lange sulen wir pîne\\ 
5 & von \textbf{disen} zwein mannen pflegen?\\ 
 & mînes veteren sun hât sich bewegen,\\ 
 & er \textbf{wil} ernern disen man,\\ 
 & der \textbf{uns} den schaden hât getân,\\ 
 & den er \textbf{billîchen} ræche,\\ 
10 & obe im \textbf{ellens} niht gebræche."\\ 
 & genuoge, dens ir triwe jach,\\ 
 & kuren einen, der zem künige sprach:\\ 
 & "\begin{large}H\end{large}êrre, muoze wirz iu sagen,\\ 
 & der lantgrâve ist unerslagen\\ 
15 & hie von maniger hende.\\ 
 & got iuch an site wende,\\ 
 & die man iu vervâhe baz.\\ 
 & werltlîch brîs iu sînen haz\\ 
 & teilt, \textbf{erslaht} ir iweren gast.\\ 
20 & ir ladet ûf iuch der schanden last.\\ 
 & sô ist der ander iwer mâc,\\ 
 & in des geleite ir disen bâc\\ 
 & \textbf{hevet}. \textbf{daz} sult ir lâzen.\\ 
 & ir sît dâr von verwâzen.\\ 
25 & \textit{nû} gebet uns einen vride her,\\ 
 & die wîle daz dirre tac \textbf{wer}.\\ 
 & der vride sî ouch dise naht.\\ 
 & wes ir iuch drumbe habet bedâht,\\ 
 & daz stêt dannoch \textbf{dâ} \textbf{ze} iwer hant,\\ 
30 & ir sît gebrîset oder geschant.\\ 
\end{tabular}
\scriptsize
\line(1,0){75} \newline
G I O L M Q R Z \newline
\line(1,0){75} \newline
\textbf{3} \textit{Initiale} I O L Z   $\cdot$ \textit{Capitulumzeichen} R  \textbf{13} \textit{Initiale} G  \textbf{21} \textit{Initiale} I  \newline
\line(1,0){75} \newline
\textbf{1} \textit{Die Verse 370.13-412.12 fehlen} Q   $\cdot$ Kingrimursel] Kyngrimvrsel O kyngrymursel M kungrumursel R \textbf{3} der] ÷er O  $\cdot$ mante] mant I (O) (Z)  $\cdot$ sîne] sinen R \textbf{4} wie] Wir M  $\cdot$ sulen] sul M soͯlt R  $\cdot$ pîne] pinen R \textbf{7} ernern] er wern O nern L \textbf{8} uns] mir O L (M) R Z \textbf{9} billîchen] billicher O (M) R Z \textbf{10} ellens] ellen O L (R)  $\cdot$ gebræche] gebrechen R \textbf{11} genuoge] Gegnvͦgen O \textbf{12} kuren] Nun R Kvm Z  $\cdot$ einen] Einenen L einer R  $\cdot$ der] \textit{om.} R \textbf{13} muoze] suln I  $\cdot$ wirz] wir R \textbf{16} an] von I  $\cdot$ site wende] sitten wandle R \textbf{17} man] nun R  $\cdot$ iu] \textit{om.} M \textbf{18} werltlîch] Wertlichin M  $\cdot$ iu sînen] ich sinnen Q \textbf{19} teilt] Teilt iv O Erteilit M  $\cdot$ erslaht] slaht I O (L) (Q) (R)  $\cdot$ ir] ir nu R \textbf{20} schanden] schaden R \textbf{22} ir] er O L  $\cdot$ disen] ewern I  $\cdot$ bâc] baͮch O tag R \textbf{23} hevet] Hat O L Habit M (R) \textbf{24} von] \textit{om.} R \textbf{25} nû gebet] gebet G Nv L  $\cdot$ vride] friden M \textbf{26} daz] \textit{om.} I  $\cdot$ dirre] dissen Q  $\cdot$ wer] gewer O M Q \textbf{27} sî] \textit{om.} Q  $\cdot$ dise] die R \textbf{28} wes] Wem L  $\cdot$ bedâht] verdaht I (L) gedacht Q \textbf{29} dâ ze] ze I (O) (L) (R) (Z) in Q \newline
\end{minipage}
\hspace{0.5cm}
\begin{minipage}[t]{0.5\linewidth}
\small
\begin{center}*T
\end{center}
\begin{tabular}{rl}
 & als tet ouch Kyngrimursel.\\ 
 & gegen ellen si beide wâren snel.\\ 
 & \begin{large}D\end{large}er künec mante die sîne:\\ 
 & "wie lange suln wir pîne\\ 
5 & von \textbf{disen} zwein mannen pflegen?\\ 
 & mînes vetern sun hât sich bewegen,\\ 
 & er \textbf{wil} ernern disen man,\\ 
 & der \textbf{mir} den schaden hât getân,\\ 
 & den er \textbf{billîcher} ræche,\\ 
10 & ob im \textbf{ellen} niht gebræche."\\ 
 & genuoge, den sir triuwe jach,\\ 
 & kurn einen, der zem künege sprach:\\ 
 & "Hêrre, muozen wirz iu sagen,\\ 
 & der lantgrâve ist unerslagen\\ 
15 & hie von maneger hende.\\ 
 & got iuch an site wende,\\ 
 & die man iu vervâhe baz.\\ 
 & werltlîch prîs iu sînen haz\\ 
 & teilt, \textbf{slaht} ir iuwern gast.\\ 
20 & ir ladet ûf iuch der schanden last.\\ 
 & sôst der ander iuwer mâc,\\ 
 & in des geleite ir disen bâc\\ 
 & \textbf{habet}. \textbf{den} sult ir lâzen.\\ 
 & ir sît dâr von verwâzen.\\ 
25 & nû gebet uns einen vride her,\\ 
 & die wîle daz dirre tac \textbf{wer}.\\ 
 & Der vride sî ouch dise naht.\\ 
 & wes ir iuch drumbe habt bedâht,\\ 
 & daz stêt dannoch \textbf{an} iuwerre hant,\\ 
30 & ir sît geprîset oder geschant.\\ 
\end{tabular}
\scriptsize
\line(1,0){75} \newline
T U V W \newline
\line(1,0){75} \newline
\textbf{3} \textit{Initiale} T U  \textbf{13} \textit{Majuskel} T  \textbf{27} \textit{Majuskel} T  \newline
\line(1,0){75} \newline
\textbf{1} Kyngrimursel] kyngrymvrsel T kingrimursel V W \textbf{2} ellen] allen U  $\cdot$ beide] \textit{om.} W \textbf{7} wil] welle V  $\cdot$ ernern] eren U \textbf{9} billîcher] billiche V \textbf{10} ellen] allen U \textbf{11} den sir] des ir U \textbf{12} kurn] Kvsen V \textbf{13} wirz iu] wir eúchs W \textbf{14} ist] ist vns W \textbf{16} iuch] iv T  $\cdot$ site] sitten V \textbf{17} Die v́ch an eren machen las V \textbf{18} Werltliche preiß nun seinen has W \textbf{19} slaht] erslahent V \textbf{20} iuch] îv T  $\cdot$ schanden] súnden W \textbf{22} ir] er V \textbf{23} habet] Helt U hebet V (W)  $\cdot$ den] das V W \textbf{24} verwâzen] mazen U \textbf{25} gebet] gebt ir W  $\cdot$ vride] vriden U \textbf{26} wer] gewer V W \textbf{28} iuch] îv T  $\cdot$ habt] haht W \textbf{29} an] zuͦ U (V) W \newline
\end{minipage}
\end{table}
\end{document}
