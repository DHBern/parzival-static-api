\documentclass[8pt,a4paper,notitlepage]{article}
\usepackage{fullpage}
\usepackage{ulem}
\usepackage{xltxtra}
\usepackage{datetime}
\renewcommand{\dateseparator}{.}
\dmyyyydate
\usepackage{fancyhdr}
\usepackage{ifthen}
\pagestyle{fancy}
\fancyhf{}
\renewcommand{\headrulewidth}{0pt}
\fancyfoot[L]{\ifthenelse{\value{page}=1}{\today, \currenttime{} Uhr}{}}
\begin{document}
\begin{table}[ht]
\begin{minipage}[t]{0.5\linewidth}
\small
\begin{center}*D
\end{center}
\begin{tabular}{rl}
\textbf{580} & \begin{large}S\end{large}i ist von Munsalvæsche komen."\\ 
 & dô Gawan \textbf{hete} vernomen\\ 
 & Munsalvæsche nennen,\\ 
 & dô begunder vreude \textbf{erkennen};\\ 
5 & er wânde, er wære dâ nâhe bî.\\ 
 & Dô sprach, der ie was valsches vrî,\\ 
 & Gawan, zer küneginne:\\ 
 & "vrouwe, mîne sinne,\\ 
 & die mir wâren entrunnen,\\ 
10 & die habt ir gewunnen\\ 
 & wider in mîn herze;\\ 
 & ouch senftet \textbf{sich mîn} smerze.\\ 
 & swaz ich krefte oder sinne hân,\\ 
 & die hât iwer dienstman\\ 
15 & gar von \textbf{iwern schulden}."\\ 
 & Si sprach: "hêrre, \textbf{iwern hulden}\\ 
 & sul wir uns alle nâhen\\ 
 & unt des mit triwen gâhen.\\ 
 & nû volget mir \textbf{unt} \textbf{en}reit niht vil.\\ 
20 & eine würze ich iu geben wil,\\ 
 & dâ von ir slâfet; deist \textbf{iu} guot.\\ 
 & \textbf{ezzens, trinkens} keinen muot\\ 
 & sult ir haben vor der naht.\\ 
 & sô kumt iu wider iwer maht,\\ 
25 & sô trit ich iu mit spîse zuo,\\ 
 & daz ir wol \textbf{bîtet} unze vruo."\\ 
 & Eine würze si leite in \textbf{sînen} munt:\\ 
 & dô slief er an der selben stunt.\\ 
 & \begin{large}W\end{large}ol si sîn mit decke pflac.\\ 
30 & alsus \textbf{überslief} den tac\\ 
\end{tabular}
\scriptsize
\line(1,0){75} \newline
D Fr7 \newline
\line(1,0){75} \newline
\textbf{1} \textit{Initiale} D  \textbf{6} \textit{Majuskel} D  \textbf{16} \textit{Majuskel} D  \textbf{27} \textit{Majuskel} D  \textbf{29} \textit{Initiale} D  \newline
\line(1,0){75} \newline
\textbf{1} Munsalvæsche] Mvnsalvæsce D \textbf{3} Munsalvæsche] Mvnsalvæsce D \textbf{15} von] won Fr7 \textbf{16} iwern] bi iͮwern Fr7 \textbf{19} enreit] reit Fr7 \textbf{25} trit] trat Fr7 \textbf{26} unze] wenne Fr7 \textbf{29} wol] Vil wol Fr7 \textbf{30} den] er den Fr7 \newline
\end{minipage}
\hspace{0.5cm}
\begin{minipage}[t]{0.5\linewidth}
\small
\begin{center}*m
\end{center}
\begin{tabular}{rl}
 & si ist von Muntsalvasche komen."\\ 
 & dô Gawan \textbf{het} vernomen\\ 
 & \textit{Muntsalvasche nennen},\\ 
 & dô begund\textit{e} er vröude \textbf{erkennen};\\ 
5 & er wânde, er wær dâ nâhe bî.\\ 
 & dô sprach, der ie was valsches vrî,\\ 
 & Gawan, zuor küniginne:\\ 
 & "vrouwe, mîne sinne,\\ 
 & die mir wâren entrunnen,\\ 
10 & die habt ir gewunnen\\ 
 & wider in mîn herze;\\ 
 & ouch \dag stiftet\dag  \textbf{sich mî\textit{n}} \textit{s}merze.\\ 
 & waz ich krefte oder sinne hân,\\ 
 & die het iuwer dienstman\\ 
15 & gar von \textbf{iuwer schulde}."\\ 
 & si sprach: "hêrre, \textbf{iuwer hulde}\\ 
 & sullen wi\textit{r} \textit{u}ns alle nâhen\\ 
 & und des mit triuwen gâhen.\\ 
 & nû volget mir, redet niht \textbf{zuo} vil.\\ 
20 & ein würz ich iu geben wil,\\ 
 & dâ von ir slâfet; daz ist guot.\\ 
 & \textbf{trinkens, ezzens} keinen muot\\ 
 & solt ir haben vor der naht.\\ 
 & sô kumt iu wider iuwer maht,\\ 
25 & sô trit ich iu mit spîse zuo,\\ 
 & daz ir wol \textbf{beite\textit{t}} unz vruo."\\ 
 & ein würz si \textbf{im} leite in \textbf{den} munt:\\ 
 & dô slief er an der selben stunt.\\ 
 & wol si sîn mit decke pflac.\\ 
30 & alsus \textbf{slie\textit{f}} den tac\\ 
\end{tabular}
\scriptsize
\line(1,0){75} \newline
m n o \newline
\line(1,0){75} \newline
\newline
\line(1,0){75} \newline
\textbf{1} Muntsalvasche] muntsaluasce m n munt saluasce o \textbf{2} dô] Da o \textbf{3} \textit{Vers 580.3 fehlt} m   $\cdot$ Montsaluasce nennen n (o) \textbf{4} begunde] begunder m \textbf{5} dâ] do n o \textbf{12} mîn smerze] min hercze vnd smercze m \textbf{17} wir uns] wir uͯwer hulde vns m \textbf{22} ezzens] essen o \textbf{26} beitet] beittens m \textbf{27} würz] wucz o  $\cdot$ in] an o \textbf{29} mit decke] nit dicke o \textbf{30} slief] slies m slieff er o \newline
\end{minipage}
\end{table}
\newpage
\begin{table}[ht]
\begin{minipage}[t]{0.5\linewidth}
\small
\begin{center}*G
\end{center}
\begin{tabular}{rl}
 & \textit{\begin{large}S\end{large}}i ist von Muntsalvatsche komen."\\ 
 & dô \textit{Gawan} \textit{\textbf{hete}} vernomen\\ 
 & Muntsalvatsche nennen;\\ 
 & dô begunde er vröude \textbf{erkennen};\\ 
5 & er wânde, er wære dâ nâhen bî.\\ 
 & dô sprach, der ie was valsches vrî,\\ 
 & Gawan, zuo \textit{der} küneginne:\\ 
 & "vrouwe, mîne sinne,\\ 
 & die mir wâren entrunnen,\\ 
10 & die habet ir gewunnen\\ 
 & wider in mîn herze;\\ 
 & ouch senftet \textbf{sich mîn} smerze.\\ 
 & swaz ich krefte ode sinne hân,\\ 
 & die hât iuwer dienstman\\ 
15 & gar von \textbf{iuwern \textit{sc}hulden}."\\ 
 & si sprach: "hêrre, \textbf{iuweren hulden}\\ 
 & sul wir uns alle nâhen\\ 
 & unde des mit triuwen gâhen.\\ 
 & nû volget mir \textbf{unde} \textit{r}edet niht vil.\\ 
20 & eine würz ich iu geben wil,\\ 
 & dâ von ir slâfet; daz ist \textbf{iu} guot.\\ 
 & \textbf{ezzens noch trinkens} deheinen muot\\ 
 & sult ir haben vor der naht.\\ 
 & sô kumet iu wider iuwer maht,\\ 
25 & sô trit ich iu mit spîs\textit{e z}uo,\\ 
 & daz ir wol \textbf{bîtet} unze vruo."\\ 
 & eine würze si legete in \textbf{sînen} munt:\\ 
 & dô slief er an der selben stunt.\\ 
 & wol si sîn mit decke pflac.\\ 
30 & alsus \textbf{überslief} den tac\\ 
\end{tabular}
\scriptsize
\line(1,0){75} \newline
G I L M Z Fr19 \newline
\line(1,0){75} \newline
\textbf{1} \textit{Initiale} G L Z  \textbf{11} \textit{Initiale} I  \newline
\line(1,0){75} \newline
\textbf{1} Si] Di G  $\cdot$ Muntsalvatsche] muntsaluasche I montsalvatsche Z \textbf{2} dô] Da M Z  $\cdot$ Gawan hete] hete Gawan G gawan hatte M \textbf{3} Muntsalvatsche] muntsaluasce I Montsalvatsche Z \textbf{4} dô] Da M \textbf{5} er] ez L  $\cdot$ dâ nâhen] nahe da M \textbf{6} dô] Da M  $\cdot$ ie was] \textit{om.} I \textbf{7} zuo der küneginne] zechuneginne G \textbf{8} mîne] in myme M \textbf{10} gewunnen] wider gewunnen I \textbf{12} senftet] senfte I  $\cdot$ mîn] mit Z  $\cdot$ smerze] hercze M \textbf{13} swaz] Waz L (M) \textbf{15} schulden] hulden G \textbf{19} unde redet] vnd [er]: enredet G redet L \textbf{20} würz] vrsz M \textbf{22} noch] \textit{om.} L M Z Fr19  $\cdot$ trinkens] trinkes M \textbf{23} ir] \textit{om.} L \textbf{24} maht] craft I \textbf{25} spîse zuo] spise wider zoͮ G \textbf{26} bîtet] belibet I bittet L beitet M Z Fr19  $\cdot$ unze] bisz M \textbf{27} in sînen] im in den I \textbf{28} dô slief er] daz er slief I Da slieff M (Z) \textbf{29} wol] vil wol I  $\cdot$ decke] decken M \textbf{30} alsus] Als L Alse her M (Fr19)  $\cdot$ den] er den I L \newline
\end{minipage}
\hspace{0.5cm}
\begin{minipage}[t]{0.5\linewidth}
\small
\begin{center}*T
\end{center}
\begin{tabular}{rl}
 & Si ist von Munsalvasche komen."\\ 
 & dô Gawan \textbf{het} vernomen\\ 
 & Munsalvasche nennen,\\ 
 & d\textit{ô} begund er vreude \textbf{kennen};\\ 
5 & er wânte, er wære d\textit{â} nâhen bî.\\ 
 & dô sprach, der ie was val\textit{sch}es vrî,\\ 
 & Gawan, zuor küniginne:\\ 
 & "vrouwe, mîne sinne,\\ 
 & die mir wâren entrunnen,\\ 
10 & die habt ir ge\textit{wunn}en\\ 
 & wider in mîn herze;\\ 
 & ouch senftet \textbf{ich mînen} smerze.\\ 
 & waz ich krefte oder sinne hân,\\ 
 & die hât iuwer dienstman\\ 
15 & gar von \textbf{iuweren schulden}."\\ 
 & si sprach: "hêrre, \textbf{iuweren hulden}\\ 
 & sul wir uns all\textit{e} nâhen\\ 
 & und des mit triuwen gâhen.\\ 
 & nû volget mir \textbf{und} redet niht vil.\\ 
20 & ein würz ich iu geben wil,\\ 
 & dâ von ir slâfet; daz ist guot.\\ 
 & \textbf{ezzens, trinkens} keinen muot\\ 
 & solt ir haben vor der naht.\\ 
 & sô kumt iu wider \textit{iuw}er maht,\\ 
25 & sô trit ich iu mit spîs\textit{e} zuo,\\ 
 & daz ir wol \textbf{beitet} unz vruo."\\ 
 & ein würz si legte in \textbf{sînen} munt:\\ 
 & dô slief er an der selben stunt.\\ 
 & wol si sîn mit decke pflac.\\ 
30 & alsô \textbf{überslief er} den tac,\\ 
\end{tabular}
\scriptsize
\line(1,0){75} \newline
Q R W V U \newline
\line(1,0){75} \newline
\textbf{1} \textit{Initiale} Q R  \newline
\line(1,0){75} \newline
\textbf{1} \textit{Die Verse 553.1-599.30 fehlen} U   $\cdot$ Munsalvasche] muntsaluasche Q Munsalualsche R montsaluatschs W [mvnsch*che]: mvnschalvasche V \textbf{2} Gawan] gawin R  $\cdot$ het] hat R \textbf{3} Munsalvasche] Muntsaluasche Q Munsalualsche R Montsaluatschse W Mvntsalvasche V  $\cdot$ nennen] nomen R \textbf{4} dô] Da Q  $\cdot$ kennen] erkennen R W V \textbf{5} dâ] do Q W V \textbf{6} valsches] falchses Q valcsches W \textbf{7} Gawan] Gawin R  $\cdot$ zuor] der W \textbf{10} gewunnen] genomen Q \textbf{12} senftet] senfttret R  $\cdot$ ich] sich R W V \textbf{13} waz] Wo W Swaz V \textbf{16} iuweren] úwer R \textbf{17} alle] allen Q \textbf{18} des] das R  $\cdot$ triuwen] gantzen treúwen W \textbf{19} vil] [*]: zevil V \textbf{21} ist] ist úch R (W) (V) \textbf{22} trinkens] vnde trinkendes V \textbf{24} iuwer] er Q \textbf{25} spîse] speysen Q \textbf{26} daz] Vntz W \textbf{27} würz] wurczen R  $\cdot$ legte] im legtte R leit im V  $\cdot$ sînen] den R V \textbf{29} wol] Wo R  $\cdot$ decke] gedekke V \newline
\end{minipage}
\end{table}
\end{document}
