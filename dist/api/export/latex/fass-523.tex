\documentclass[8pt,a4paper,notitlepage]{article}
\usepackage{fullpage}
\usepackage{ulem}
\usepackage{xltxtra}
\usepackage{datetime}
\renewcommand{\dateseparator}{.}
\dmyyyydate
\usepackage{fancyhdr}
\usepackage{ifthen}
\pagestyle{fancy}
\fancyhf{}
\renewcommand{\headrulewidth}{0pt}
\fancyfoot[L]{\ifthenelse{\value{page}=1}{\today, \currenttime{} Uhr}{}}
\begin{document}
\begin{table}[ht]
\begin{minipage}[t]{0.5\linewidth}
\small
\begin{center}*D
\end{center}
\begin{tabular}{rl}
\textbf{523} & \begin{large}G\end{large}awan \textbf{daz} klagete sêre.\\ 
 & diu vrouwe \textbf{es} lachete mêre,\\ 
 & denn inder schimpfes \textbf{in} gezam.\\ 
 & sît man im daz ors \textbf{genam},\\ 
5 & ir süezer munt hin zim \textbf{dô} sprach:\\ 
 & "vür einen rîter ich iuch sach.\\ 
 & dar nâch in kurzen stunden\\ 
 & wurdet ir arzet vür die wunden.\\ 
 & nû müezet ir ein garzûn wesen.\\ 
10 & sol iemen sîner kunst genesen,\\ 
 & sô trœstet iuch iwerre sinne.\\ 
 & gert ir noch mîner minne?"\\ 
 & "jâ, vrouwe", sprach hêr Gawan,\\ 
 & "m\textit{ö}hte ich iwer minne hân,\\ 
15 & diu wære mir lieber danne iht.\\ 
 & ez \textbf{en}wont ûf erde nihtes niht\\ 
 & under krône unt alle, die krône tragent\\ 
 & unt die vreudehaften prîs bejagent,\\ 
 & der gein iu teilte ir gewin,\\ 
20 & sô ræt mir mînes herzen sin,\\ 
 & daz ich\textbf{z} in lâzen solte.\\ 
 & iwer minne \textbf{ich} haben wolte.\\ 
 & mag ich der niht erwerben,\\ 
 & sô muoz \textbf{ein sûwerez} sterben\\ 
25 & \textbf{sich schiere an mir} erzeigen.\\ 
 & ir wüestet iwer eigen.\\ 
 & ob ich vrîheit ie gewan,\\ 
 & ir sult mich doch vür eigen hân;\\ 
 & daz dunket mich iwer ledec reht.\\ 
30 & nû nennet mich rîter oder kneht,\\ 
\end{tabular}
\scriptsize
\line(1,0){75} \newline
D Fr11 \newline
\line(1,0){75} \newline
\textbf{1} \textit{Großinitiale} D  \newline
\line(1,0){75} \newline
\textbf{3} denn inder] den ninder D \textbf{4} sît] seint daz Fr11 \textbf{14} möhte] mohte D m:::t Fr11 \textbf{16} enwont] want Fr11 \textbf{18} bejagent] beiaget Fr11 \textbf{19} iu] ewer Fr11 \textbf{29} dunket] duͯnkchz Fr11 \newline
\end{minipage}
\hspace{0.5cm}
\begin{minipage}[t]{0.5\linewidth}
\small
\begin{center}*m
\end{center}
\begin{tabular}{rl}
 & Gawan klagte sêre,\\ 
 & diu vrouwe lachete mêre.\\ 
 & \multicolumn{1}{l}{ - - - }\\ 
 & \multicolumn{1}{l}{ - - - }\\ 
5 & ir süezer munt hin zuo \dag ir\dag  sprach:\\ 
 & "v\textit{ü}r eine\textit{n} ritter ich iuch sach.\\ 
 & dar nâch in kurzen stunden\\ 
 & w\textit{u}rdet ir arzet vür die wunden.\\ 
 & nû müezet ir ein garzû\textit{n} wesen.\\ 
10 & so\textit{l} iemen sîner kunst genesen,\\ 
 & sô trœstet iuch iuwer sinne.\\ 
 & gert ir noch mîner minne?"\\ 
 & "jâ, vrouwe", sprach hêr Gawan,\\ 
 & "m\textit{ö}ht ich \textbf{noch} iuwer minne hân,\\ 
15 & diu wær mir lieber dan iht.\\ 
 & ez wonte ûf erden nihtes niht\\ 
 & under krône und alle, die krône tragent\\ 
 & und die vröudehaften prîs bejagent,\\ 
 & der gegen iu teil\textit{t}e ir gewin,\\ 
20 & sô râtet mir mînes herzen sin,\\ 
 & daz ich in lâzen solte.\\ 
 & iuwer minne \textbf{ich} haben wolte.\\ 
 & mac ich der niht erwerben,\\ 
 & sô muoz \textbf{ich schier} sterben.\\ 
25 & \textbf{daz muoz sich balde} \textit{er}z\textit{ei}gen.\\ 
 & ir w\textit{üe}stet iuwer eigen.\\ 
 & ob ich vrîheit ie gewan,\\ 
 & ir solt mich doch vür eigen hân;\\ 
 & daz dunke\textit{t} \textit{m}ich iuwer ledic reht.\\ 
30 & nû nennet mich ritter oder kneht,\\ 
\end{tabular}
\scriptsize
\line(1,0){75} \newline
m n o \newline
\line(1,0){75} \newline
\textbf{1} \textit{Capitulumzeichen} n  \newline
\line(1,0){75} \newline
\textbf{2} vrouwe] \textit{om.} o  $\cdot$ mêre] es mere n o \textbf{3} \textit{Die Verse 523.3-4 fehlen} m n o  \textbf{5} zuo] zuͦ zuͦ o \textbf{6} vür einen] vor einem m \textbf{8} wurdet] Wuͯrdent m (o) \textbf{9} garzûn] garczuͯm m garczẏm o \textbf{10} sol] So m  $\cdot$ iemen] man o \textbf{13} hêr Gawan] hergawan m \textbf{14} möht] Moht m (n) Moch o  $\cdot$ noch] \textit{om.} n \textbf{16} erden] er o \textbf{19} teilte] teile m \textbf{21} ich] ichs n o \textbf{25} erzeigen] beczougen m erzoigen o \textbf{26} wüestet] wistent m \textbf{27} ie] [e]: ie o \textbf{28} solt] soltent n \textbf{29} dunket mich] duncket uncket mich m \textbf{30} oder] vnd o \newline
\end{minipage}
\end{table}
\newpage
\begin{table}[ht]
\begin{minipage}[t]{0.5\linewidth}
\small
\begin{center}*G
\end{center}
\begin{tabular}{rl}
 & \begin{large}G\end{large}awan \textbf{daz} klagete sêre.\\ 
 & diu vrouwe \textbf{es} lachete mêre,\\ 
 & danne iender schimpfes \textbf{in} gezam.\\ 
 & sît \textbf{daz} man im daz ors \textbf{genam},\\ 
5 & ir süezer munt \textit{hin zim} \textit{\textbf{dô}} sprach:\\ 
 & "vür einen rîter ich iuch sach.\\ 
 & dar nâch in kurzen stunden\\ 
 & wurdet ir arzet vür die wunden.\\ 
 & nû müezet ir ein garzûn wesen.\\ 
10 & sol iemen sîner kunst genesen,\\ 
 & sô trœstet \textit{iuch} iuwer sinne.\\ 
 & gert ir noch mîner minne?"\\ 
 & "jâ, vrouwe", sprach hêr Gawan,\\ 
 & "m\textit{ö}hte ich iuwer minne hân,\\ 
15 & diu wære mir lieber danne iht.\\ 
 & ez won\textit{t} ûf erden nihtes niht\\ 
 & under krône unde alle, die krône tragent\\ 
 & unde die vröudehaften brîs bejagent,\\ 
 & der gein iu teilte ir gewin,\\ 
20 & sô rætet mir mînes herzen sin,\\ 
 & daz ich\textbf{z} in lâzen solde.\\ 
 & iuwer minne \textbf{ich} haben wolde.\\ 
 & mag ich der niht erwerben,\\ 
 & sô muoz \textbf{ein sûrez} sterben\\ 
25 & \textbf{sich schiere an mir} erzeigen.\\ 
 & ir wüeste\textit{t} iuwer eigen.\\ 
 & obe ich vrîheit ie gewan,\\ 
 & ir sult mich doch vür eigen hân;\\ 
 & daz dunket mich iuwer ledic reht.\\ 
30 & nû nennet mich rîter oder kneht,\\ 
\end{tabular}
\scriptsize
\line(1,0){75} \newline
G I L M Z Fr62 \newline
\line(1,0){75} \newline
\textbf{1} \textit{Überschrift:} Hie ist her gawan vmb sin orss komen daz hat im ein wunder ritter genomen vnd stet her gawan bi der frowen zv fvzzen Z   $\cdot$ \textit{Initiale} G I L Z Fr62  \textbf{23} \textit{Initiale} I  \newline
\line(1,0){75} \newline
\textbf{1} Gawan] Gauwan I  $\cdot$ klagete] clagt Z \textbf{2} diu] di Fr62  $\cdot$ vrouwe] \textit{om.} L  $\cdot$ es] sin I Z des M \textbf{3} danne iender] Deheines L Der nirgen M den niergen Fr62  $\cdot$ in] im Z \textit{om.} Fr62 \textbf{4} sît daz] sit I  $\cdot$ genam] nam I Z \textbf{5} hin zim dô] mit froͮden G zcu yme da M hin zvim Z \textbf{6} einen] eyme M \textbf{8} arzet] ein artzet L \textbf{9} ir] ir nu Z \textbf{11} iuch] \textit{om.} G \textbf{13} hêr] \textit{om.} M \textbf{14} möhte] Mohte G (I) (L) (M) (Z) moh Fr62  $\cdot$ ich] ihc Z \textbf{15} diu] di Fr62 \textbf{16} wont] wonte G enwont Z  $\cdot$ erden] erde L Z Fr62 der erdin M  $\cdot$ nihtes] des I \textbf{17} under] vnder der I Vnd Z  $\cdot$ unde] adir M \textbf{18} vröudehaften] freudhaf I vroude hochstin M \textbf{19} der] Den M  $\cdot$ teilte] teil ich M \textbf{20} sô] idoch so I  $\cdot$ mir mînes herzen] [mir]: min I mir myn bester M \textbf{21} in] \textit{om.} I \textbf{24} sûrez] swere Fr62 \textbf{26} ir] ich I  $\cdot$ wüestet] woͮste G \textbf{29} ledic] lede M \newline
\end{minipage}
\hspace{0.5cm}
\begin{minipage}[t]{0.5\linewidth}
\small
\begin{center}*T
\end{center}
\begin{tabular}{rl}
 & \textit{\begin{large}G\end{large}}awan \textbf{daz} klagete sêre.\\ 
 & diu vrouwe \textbf{es} lachete mêre,\\ 
 & dan iender schimpfes \textbf{im} gezam.\\ 
 & sît \textbf{daz} man im daz ors \textbf{dâ} \textbf{nam},\\ 
5 & Ir süezer munt hin zim \textbf{dô} sprach:\\ 
 & "vür einen rîter ich iuch \textbf{ê} sach.\\ 
 & dâ nâch in kurzen stunden\\ 
 & wurdet ir arzât vür die wunden.\\ 
 & nû müezet ir ein garzûn wesen.\\ 
10 & sol iemen sîner kunst genesen,\\ 
 & sô trœstet iuch iuwerre sinne.\\ 
 & gert ir noch mîner minne?"\\ 
 & "Jâ, vrouwe", sprach hêr Gawan,\\ 
 & "m\textit{ö}htich iuwer minne hân,\\ 
15 & diu wære mir lieber danne iht.\\ 
 & ez \textbf{en}wont ûf \textbf{der} erden nihtes niht\\ 
 & under krône unde alle, die krône tragent\\ 
 & unde die vreudehaften prîs bejagent,\\ 
 & der gegen iu teilte ir gewin,\\ 
20 & sô râtet mir mînes herzen sin,\\ 
 & daz ich\textbf{z} in lâzen solte\\ 
 & \textbf{unde} iuwer minne haben wolte.\\ 
 & mag ich der niht erwerben,\\ 
 & sô muoz \textbf{ein sûrez} sterben\\ 
25 & \textbf{sich schiere an mir} erzeigen.\\ 
 & ir wüestet iuwer eigen.\\ 
 & ob ich vrîheit ie gewan,\\ 
 & ir sult mich doch vür eigen hân;\\ 
 & daz dunket mich iuwer ledic reht.\\ 
30 & nû nennet mich rîter oder kneht,\\ 
\end{tabular}
\scriptsize
\line(1,0){75} \newline
T U V W O Q R Fr40 \newline
\line(1,0){75} \newline
\textbf{1} \textit{Überschrift:} Hye entreit der wunde man herrn gawan kastelan Q   $\cdot$ \textit{Großinitiale} T U   $\cdot$ \textit{Initiale} V W O Q R  \textbf{5} \textit{Majuskel} T  \textbf{13} \textit{Majuskel} T  \newline
\line(1,0){75} \newline
\textbf{1} Gawan] ÷awan T (O) Gawin R \textbf{2} es] des Q R  $\cdot$ lachete] lachet R \textbf{3} iender schimpfes] ye der schimpt R  $\cdot$ im] in U W O Q [im]: in V Inen R \textbf{4} sît daz] Sit O  $\cdot$ im] \textit{om.} R  $\cdot$ dâ] do U V W \textit{om.} O Q im R  $\cdot$ nam] genam W O R \textbf{5} süezer] suͯsze R  $\cdot$ hin zim dô] me dannoch W zu im do Q hin zu Jm R \textbf{6} iuch] îv T  $\cdot$ ê] \textit{om.} U V W O Q R Fr40 \textbf{8} wurdet] Wúrdet W (O)  $\cdot$ arzât] ein arzt O \textbf{9} müezet] moͤgent V \textbf{10} sol] So R  $\cdot$ iemen] man Fr40 \textbf{11} iuch] iv T \textit{om.} W \textbf{12} noch] nach U W \textbf{13} hêr] der Fr40  $\cdot$ Gawan] gawin R \textbf{14} möhtich] mohtich T (O) (Q) (Fr40) Moch ich U \textbf{15} diu] \textit{om.} W  $\cdot$ iht] et iht O ichtes icht Q \textbf{16} ez enwont] [E*t]: Ez wont V  $\cdot$ der erden] erden U V W Fr40 erde O Q R  $\cdot$ nihtes] \textit{om.} U Fr40 \textbf{17} die krône] di Fr40 \textbf{18} die] \textit{om.} Q Fr40  $\cdot$ vreudehaften] fundschafftte R \textbf{19} gegen] mit R  $\cdot$ ir] úwern R \textbf{20} mir] \textit{om.} O  $\cdot$ mînes herzen] mein selbs Q mein Fr40 \textbf{21} Das ich in lassen solte W  $\cdot$ Das nichttes in lasze soͯlte R \textbf{22} unde] \textit{om.} U V W O Q R Fr40  $\cdot$ haben] ich haben U V W O Q R Fr40  $\cdot$ wolte] soͯlte R \textbf{23} der] die R \textbf{24} ein sûrez sterben] ein swarez sterben O ich sus ersterben R \textbf{25} schiere] \textit{om.} Q  $\cdot$ erzeigen] schir ertzeigen Q mus erczeigen R \textbf{26} ir wüestet] [*]: Jr w#;uestent V Jr west Q \textbf{27} ie] [*]: ie V e Q Fr40 \textbf{28} doch] [*]: doch V \textit{om.} R \textbf{29} reht] [*]: reht V wert W \textbf{30} oder] [*]: oder V \newline
\end{minipage}
\end{table}
\end{document}
