\documentclass[8pt,a4paper,notitlepage]{article}
\usepackage{fullpage}
\usepackage{ulem}
\usepackage{xltxtra}
\usepackage{datetime}
\renewcommand{\dateseparator}{.}
\dmyyyydate
\usepackage{fancyhdr}
\usepackage{ifthen}
\pagestyle{fancy}
\fancyhf{}
\renewcommand{\headrulewidth}{0pt}
\fancyfoot[L]{\ifthenelse{\value{page}=1}{\today, \currenttime{} Uhr}{}}
\begin{document}
\begin{table}[ht]
\begin{minipage}[t]{0.5\linewidth}
\small
\begin{center}*D
\end{center}
\begin{tabular}{rl}
\textbf{66} & ein mære in stichet als ein dorn,\\ 
 & daz er sîn wîp \textbf{hât} verlorn,\\ 
 & diu Artuses muoter was.\\ 
 & ein pfaffe, der \textbf{wol} zouber las,\\ 
5 & mit dem diu vrouwe ist hin gewant,\\ 
 & dem ist Artus nâch gerant.\\ 
 & \textbf{ez} ist nû \textbf{ime dritten jâr},\\ 
 & daz er sun unt wîp verlôs \textbf{vür wâr}.\\ 
 & hie ist ouch sîner tohter man,\\ 
10 & der \textbf{wol mit rîterschefte} kan,\\ 
 & Lot von Norwæge,\\ 
 & gein valscheit der træge\\ 
 & unt der snelle gein dem prîse,\\ 
 & der \textbf{küene} degen wîse.\\ 
15 & hie ist ouch Gawan, des sun,\\ 
 & sô kranc, daz er niht mac getuon\\ 
 & rîterschaft enkeine.\\ 
 & er was bî mir, der kleine.\\ 
 & \textbf{er} \textbf{sprichet}, möht er einen schaft\\ 
20 & \textbf{gebrechen}, trôste in des sîn kraft,\\ 
 & er \textbf{tæte} gerne rîters tât.\\ 
 & wie \textbf{vruo} sîn ger begunnen hât!\\ 
 & Hie \textbf{hât der künec} von Patrigalt\\ 
 & von spern einen ganzen walt.\\ 
25 & \textbf{des} vuore ist \textbf{dâ} \textbf{engein} \textbf{gar} ein wint,\\ 
 & \textbf{wan} die von Portegal hie sint,\\ 
 & die heizen wir die vrechen.\\ 
 & \textbf{si} wellent durch schilde stechen.\\ 
 & \begin{large}H\end{large}ie hânt die Provenzale\\ 
30 & schilde wol gemâle.\\ 
\end{tabular}
\scriptsize
\line(1,0){75} \newline
D Fr9 Fr33 \newline
\line(1,0){75} \newline
\textbf{23} \textit{Majuskel} D  \textbf{29} \textit{Initiale} D Fr9 Fr33  \newline
\line(1,0){75} \newline
\textbf{1} ein] Sẏn Fr9 \textbf{3} Artuses] Artvs D \textbf{6} Artus] :::us Fr33 \textbf{7} ime dritten] in daz dritte Fr9 \textbf{8} sun unt wîp verlôs] vuͦrlos wip vnde svn Fr9 sin wib verlos Fr33 \textbf{11} Lot] Loht D :oth Fr9  $\cdot$ Norwæge] Norwege D (Fr9) \textbf{14} degen] vn: der Fr9 \textbf{15} des] sin Fr9 \textbf{21} er tæte] So teter Fr9  $\cdot$ gerne] gar gerne Fr33 \textbf{23} Patrigalt] :::igalt Fr9 \textbf{25} dâ engein gar] g::: da kegen Fr9 da gein gar Fr33 \textbf{26} Portegal] portigal Fr9 :::ortgal Fr33 \textbf{28} wellent] wilt Fr9 \newline
\end{minipage}
\hspace{0.5cm}
\begin{minipage}[t]{0.5\linewidth}
\small
\begin{center}*m
\end{center}
\begin{tabular}{rl}
 & ein mære in stichet als ein dorn,\\ 
 & daz er sîn wîp \textbf{hât} verlorn,\\ 
 & diu Artuses muoter was.\\ 
 & ein pfaffe, der \textbf{wol} zouber las,\\ 
5 & mit dem diu vrowe ist hin gewant,\\ 
 & dem ist Artus nâch gerant.\\ 
 & \textbf{ez} ist nû \textbf{im dritten jâr},\\ 
 & daz er sun und wîp verlôs \textbf{vür wâr}.\\ 
 & hie ist ouch sîner tohter man,\\ 
10 & der \textbf{wol mit ritterschaf\textit{t}e} kan,\\ 
 & L\textit{ot} von Norw\textit{æ}ge,\\ 
 & gegen valscheit der træge\\ 
 & und der snelle gegen dem prîse,\\ 
 & der \textbf{küene} degen wîse.\\ 
15 & \textit{h}ie ist ouch Gawan, des sun,\\ 
 & sô kranc, daz er niht ma\textit{c} getuon\\ 
 & ritterschaft dekeine.\\ 
 & er was bî mir, der kleine,\\ 
 & \textbf{und} \textbf{sprichet}, m\textit{ö}ht er einen schaft\\ 
20 & \textbf{gebrechen}, trôst in des sîn kraft,\\ 
 & er \textbf{tæte} gerne ritters tât.\\ 
 & wie \textbf{vruo} \textbf{es} sîn ger begunnen hât!\\ 
 & hie \textbf{hât der künic} von Patrigalt\\ 
 & von speren einen ganzen walt.\\ 
25 & \textbf{des} vuore ist \textbf{dâ} \textit{\textbf{wider}} \textbf{gar} ein wint,\\ 
 & \textbf{wan} die von Portigal hie sint,\\ 
 & die heizen wir die vrec\textit{h}en.\\ 
 & \textbf{si} wellent durch schilte stechen.\\ 
 & hie hânt die \textit{P}r\textit{o}venzale\\ 
30 & schilte wol gemâle.\\ 
\end{tabular}
\scriptsize
\line(1,0){75} \newline
m n o \newline
\line(1,0){75} \newline
\newline
\line(1,0){75} \newline
\textbf{2} hât] het o \textbf{3} Artuses] artusas n arstuͯses o \textbf{6} ist] \textit{om.} o  $\cdot$ Artus] artuͯs o \textbf{8} vür wâr] \textit{om.} n o \textbf{10} ritterschafte] ritterschaffe m \textbf{11} Lot] Lach m Lech n o  $\cdot$ Norwæge] norwage m norwege n norwige o \textbf{15} hie] Nie \textit{nachträglich korrigiert zu:} Hie m  $\cdot$ Gawan des] gawandes m n gewandes o \textbf{16} mac] magte m \textbf{17} dekeine] do keine n \textbf{19} möht] mocht m (o) \textbf{20} gebrechen] Zerbrechen n Zuͦ brichen o  $\cdot$ des] \textit{om.} o \textbf{22} es] sie es o \textbf{23} hie hât] Die hette n o  $\cdot$ Patrigalt] partigalt o \textbf{25} wider] \textit{om.} m \textbf{26} Portigal] partigal n \textbf{27} vrechen] frehtten \textit{nachträglich korrigiert zu:} frechtten m \textbf{29} Provenzale] braffenczalle m profenczale n profentzale o \newline
\end{minipage}
\end{table}
\newpage
\begin{table}[ht]
\begin{minipage}[t]{0.5\linewidth}
\small
\begin{center}*G
\end{center}
\begin{tabular}{rl}
 & ein mære in stichet als ein dorn,\\ 
 & daz er sîn wîp \textbf{hât} verlorn,\\ 
 & diu Artuses muoter was.\\ 
 & ein pfaffe, der \textbf{wol} zouber las,\\ 
5 & \begin{large}M\end{large}it dem diu vrouwe ist hin gewant,\\ 
 & dem ist Artus nâch gerant.\\ 
 & \textbf{ez} ist nû \textbf{daz drite jâr},\\ 
 & daz er sun und wîp verlôs \textbf{vür wâr}.\\ 
 & hie ist ouch sîner tohter man,\\ 
10 & der \textbf{wol \textit{mit} rîterschefte} kan,\\ 
 & Lot von Norwæge,\\ 
 & gein valscheit der træge\\ 
 & \textit{unde} der snelle gein dem brîse,\\ 
 & der \textbf{stolze} degen wîse.\\ 
15 & hie ist ouch Gawan, des sun,\\ 
 & sô kranc, daz er niht mac getuon\\ 
 & rîterschaft deheine.\\ 
 & er was bî mir, der kleine,\\ 
 & \textbf{unde} \textbf{giht}, m\textit{ö}hter einen schaft\\ 
20 & \textbf{zerbrechen}, trôst in des sîn kraft,\\ 
 & er \textbf{worhte} gerne rîters tât.\\ 
 & \textit{wie} \textbf{vruo} \textbf{es} sîn ger begunnen hât!\\ 
 & hie \textbf{habent die} von Patrigalt\\ 
 & von speren einen ganzen walt.\\ 
25 & \textbf{der} vuore ist \textbf{wider} \textbf{die} ein wint.\\ 
 & die von Portigal hie sint,\\ 
 & die heizen wir die vrechen.\\ 
 & \textbf{die} wellent durch schilde stechen.\\ 
 & hie habent die Provenzale\\ 
30 & schilde wol gemâle.\\ 
\end{tabular}
\scriptsize
\line(1,0){75} \newline
G I O L M Q R Z Fr44 \newline
\line(1,0){75} \newline
\textbf{1} \textit{Initiale} O  \textbf{5} \textit{Initiale} G  \textbf{19} \textit{Initiale} I  \newline
\line(1,0){75} \newline
\textbf{1} ein mære] ÷in mêr O  $\cdot$ als ein] alz L \textbf{3} Artuses] Artus R \textbf{4} wol] \textit{om.} Fr44  $\cdot$ las] [was]: las R \textbf{5} Mit dem] Da mit Z  $\cdot$ diu vrouwe ist] ist diu frowe I \textbf{6} \textit{Vers 66.6 fehlt} R  \textbf{7} ez] Er M  $\cdot$ nû] \textit{om.} Fr44  $\cdot$ daz drite] indem driten O (L) (M) (Q) (R) (Z) (Fr44) \textbf{8} verlôs] \textit{om.} R \textbf{9} tohter] thoter L \textbf{10} wol] vil R  $\cdot$ mit rîterschefte] zeriterschefte G \textbf{11} Lot] loht I Lozt M Loth Z  $\cdot$ von] uor Fr44  $\cdot$ Norwæge] norwage G (M) norwege I (O) (L) R Z norweigen Q Norwêge Fr44 \textbf{12} valscheit] wiszheit R  $\cdot$ der] vil L \textbf{13} unde] \textit{om.} G  $\cdot$ dem] \textit{om.} L \textbf{14} stolze] chvͦne O (L) (M) (Q) (R) (Z) (Fr44)  $\cdot$ degen] helt O L M Q (R) Fr44 \textbf{15} Gawan] Gawen I Fr44 [Gawan*]: Gawan O Gawein Q Gawin Z \textbf{16} er] er es R  $\cdot$ mac] macht R \textbf{19} giht] ien M  $\cdot$ möhter] mohter G (O) (L) (Q) (Z) (Fr44) mochte M moͯcht R \textbf{20} trôst] vnde trost O  $\cdot$ des] daz L \textbf{21} worhte] tete Z \textbf{22} wie] \textit{om.} G Vil O M Q R Fr44  $\cdot$ es] er I \textit{om.} Fr44  $\cdot$ ger] \textit{om.} I gar Q \textbf{23} habent die] hat der chvnich O (L) (M) (Q) (R) (Z) (Fr44)  $\cdot$ von] \textit{om.} O M Fr44  $\cdot$ Patrigalt] partigal R \textbf{24} ganzen] grozen O Fr44 \textbf{25} der] Des O L M (Q) R Z Fr44  $\cdot$ wider die] wider den O L (M) Fr44 wider dann Q widen den R da gein gar Z  $\cdot$ ein] einen R \textbf{26} die] die hie I Wan die Z  $\cdot$ Portigal] portigale I portegal L portgal R  $\cdot$ hie] \textit{om.} I \textbf{27} wir] hie Q \textbf{28} die] Sie M Q (R) Z Fr44 \textbf{29} hie] Dy Q Sie Z Fr44  $\cdot$ Provenzale] provenzal I provenciale L Prouenczale M (R) prouenzale Q (Fr44) proventzale Z \textbf{30} gemâle] zuͯ male L \newline
\end{minipage}
\hspace{0.5cm}
\begin{minipage}[t]{0.5\linewidth}
\small
\begin{center}*T (U)
\end{center}
\begin{tabular}{rl}
 & eine mære in stichet als ein dorn,\\ 
 & daz er sîn wîp \textbf{hatte} verlorn,\\ 
 & diu Artuses muoter was.\\ 
 & ein pfaffe, der \textbf{von} zouber las,\\ 
5 & mit dem diu vrouwe ist hin gewant,\\ 
 & dem ist Artus nâch gerant.\\ 
 & \textbf{daz} ist nû \textbf{in dem driten jâre},\\ 
 & daz er sun und wîp verlôs \textbf{zwâre}.\\ 
 & hie ist ouch sîner tohter man,\\ 
10 & der \textbf{mit ritterschefte wol} kan,\\ 
 & Lot von Norwæge,\\ 
 & gein valscheit der træge.\\ 
 & \multicolumn{1}{l}{ - - - }\\ 
 & \multicolumn{1}{l}{ - - - }\\ 
15 & hie ist ouch Gawan, des sun,\\ 
 & sô kranc, daz er niht mac getuon\\ 
 & ritterschaft dekeine.\\ 
 & er was bî mir, der kleine.\\ 
 & \textbf{er} \textbf{giht}, \textbf{und} m\textit{ö}hter einen schaft\\ 
20 & \textbf{zerbrechen}, trôstin des sîn kraft,\\ 
 & er \textbf{wo\textit{l}de} gerne ritters \textit{t}ât.\\ 
 & wie \textbf{vrô} sîn ger begunnen hât!\\ 
 & hie \textbf{hât der künec} von Patrigalt\\ 
 & von spern einen ganzen walt.\\ 
25 & \textbf{des} vuore ist \textbf{gein} \textbf{den} ein wint.\\ 
 & die von Portegal hie sint,\\ 
 & die heizen \textit{wi}r die vrechen.\\ 
 & \textbf{si} wellent durch schilt stechen.\\ 
 & \multicolumn{1}{l}{ - - - }\\ 
30 & \multicolumn{1}{l}{ - - - }\\ 
\end{tabular}
\scriptsize
\line(1,0){75} \newline
U V W T \newline
\line(1,0){75} \newline
\textbf{23} \textit{Initiale} W   $\cdot$ \textit{Majuskel} T  \textbf{26} \textit{Majuskel} T  \textbf{28} \textit{Majuskel} T  \textbf{29} \textit{Initiale} V   $\cdot$ \textit{Majuskel} T  \newline
\line(1,0){75} \newline
\textbf{1} eine] ein V W T \textbf{2} hatte] hat V W T \textbf{3} Artuses] artuͦses U \textbf{4} von] wol von W \textbf{7} Ez ist nv wol (\textit{om.} T ) daz dirtte ior V (T)  $\cdot$ nû] \textit{om.} W \textbf{8} zwâre] fúr wor V (T) \textbf{10} mit ritterschefte wol] wol mit ritterschefte V (W) (T) \textbf{11} Lot] Lôt T  $\cdot$ Norwæge] narwege W Norwêge T \textbf{13} \textit{Die Verse 66.13-14 fehlen} U W   $\cdot$ Vnd der snelle gegen (dem T ) prise V (T) \textbf{14} Der kvͤne helt wise V (T) \textbf{15} Gawan] gauwin W \textbf{17} ritterschaft] Rihterschaft V \textbf{18} er] Der V \textbf{19} er giht und] Vnde giht V (T)  $\cdot$ möhter] mochter U (T)  $\cdot$ einen] doch ein W \textbf{20} trôstin] vnde troste in V \textbf{21} er wolde] er worde U So thete er W so worhter T  $\cdot$ gerne] gerne tvͦn V  $\cdot$ tât] rat U \textbf{22} wie vrô] [V*]: Vil frvͦ ez V Wie fruͤ es W wie vruͦz T \textbf{25} gein den] [*]: wider den V gegen den nicht W wider die als T \textbf{26} Portegal] Portigal T \textbf{27} heizen] hiessen W  $\cdot$ wir] vor U \textbf{29} \textit{Die Verse 66.29-30 fehlen} U   $\cdot$ HJe hant die provenzale (von prouenzal W von provenzale T ) V (W) (T) \textbf{30} Schilte wol gemale V (W) (T) \newline
\end{minipage}
\end{table}
\end{document}
