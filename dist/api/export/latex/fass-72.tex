\documentclass[8pt,a4paper,notitlepage]{article}
\usepackage{fullpage}
\usepackage{ulem}
\usepackage{xltxtra}
\usepackage{datetime}
\renewcommand{\dateseparator}{.}
\dmyyyydate
\usepackage{fancyhdr}
\usepackage{ifthen}
\pagestyle{fancy}
\fancyhf{}
\renewcommand{\headrulewidth}{0pt}
\fancyfoot[L]{\ifthenelse{\value{page}=1}{\today, \currenttime{} Uhr}{}}
\begin{document}
\begin{table}[ht]
\begin{minipage}[t]{0.5\linewidth}
\small
\begin{center}*D
\end{center}
\begin{tabular}{rl}
\textbf{72} & gewâpent \textbf{vaste} unz ûf den huof.\\ 
 & hie garzûne, ruofâ ruof!\\ 
 & sîn lîp spranc drûf, wand erz dâ vant.\\ 
 & vil sta\textit{r}ker sper des heldes hant\\ 
5 & mit hurte verswande.\\ 
 & \textbf{die} poynder er zertrande\\ 
 & \textbf{immer durch}, anderthalben ûz.\\ 
 & dem anker volgete \textbf{nâch} der strûz.\\ 
 & \begin{large}G\end{large}ahmuret stach hinderz ors\\ 
10 & Poytwin \textbf{de} \textbf{Prienlascors}\\ 
 & unt anders manegen werden man,\\ 
 & an \textbf{den} \textit{e}r sicherheit gewan.\\ 
 & swaz dâ \textbf{gekriuzter} ritter reit,\\ 
 & die genuzzen des heldes arbeit.\\ 
15 & diu gewunnen ors, diu gab er in.\\ 
 & an im lag \textbf{ir grôz} gewin.\\ 
 & Glîcher baniere\\ 
 & man \textbf{gein im vuorte} \textbf{viere}\\ 
 & - küene rotten riten drunde,\\ 
20 & ir \textbf{hêrre} strîten \textbf{kunde} -\\ 
 & an ieslîcher \textbf{eines} grîfen zagel.\\ 
 & daz \textbf{hinder} teil \textbf{was ouch} ein hagel\\ 
 & \textbf{an} rîterschaft. des wâren die.\\ 
 & daz vorder teil des grîfen hie\\ 
25 & der künec von Gascone truoc\\ 
 & \textbf{ûf} \textbf{sîme} schilde, ein rîter kluoc.\\ 
 & \textbf{gezimieret} was \textbf{sîn} lîp,\\ 
 & \textbf{sô} wol geprüeven kunnen wîp.\\ 
 & \textbf{er} nam sich vor den andern ûz,\\ 
30 & dô er ûf\textbf{em helme} \textbf{ersach} den strûz.\\ 
\end{tabular}
\scriptsize
\line(1,0){75} \newline
D Fr33 \newline
\line(1,0){75} \newline
\textbf{9} \textit{Initiale} D  \textbf{17} \textit{Majuskel} D  \textbf{29} \textit{Initiale} Fr33  \newline
\line(1,0){75} \newline
\textbf{1} ûf den] anden Fr33 \textbf{3} spranc] [sprach]: spranch D \textbf{4} starker] stacher D \textbf{5} verswande] her verswante Fr33 \textbf{6} die] den Fr33  $\cdot$ zertrande] zu rante Fr33 \textbf{7} immer] Hie Fr33 \textbf{9} Gahmuret] Gahmvret D Gamuret Fr33 \textbf{10} Poytwin] poitwinen Fr33  $\cdot$ de] von Fr33  $\cdot$ Prienlascors] [prienlacors]: prienlascors D prelaͤhziors Fr33 \textbf{12} den] dem Fr33  $\cdot$ er] ir D \textbf{13} gekriuzter] gecruzigeter Fr33 \textbf{16} ir] vil Fr33 \textbf{19} rotten] rotte Fr33 \textbf{21} eines] ein Fr33 \textbf{22} was ouch] daz was Fr33 \textbf{23} die] [dri]: die Fr33 \textbf{25} Gascone] Gasconie Fr33 \textbf{27} sîn] des Fr33 \textbf{28} sô] Als Fr33 \textbf{29} er] Der Fr33 \textbf{30} ersach] sach Fr33 \newline
\end{minipage}
\hspace{0.5cm}
\begin{minipage}[t]{0.5\linewidth}
\small
\begin{center}*m
\end{center}
\begin{tabular}{rl}
 & gewâpent \textbf{vaste} unz ûf den huo\textit{f}.\\ 
 & hie garzûne, ruofâ ruof!\\ 
 & sîn lîp spranc drûf, wand er ez d\textit{â} vant.\\ 
 & vil st\textit{a}rker sper des heldes hant\\ 
5 & mit hurte verswande.\\ 
 & \textbf{dis} poinder er zertrande\\ 
 & \textbf{iemer durch}, anderhalben ûz.\\ 
 & dem anker volgete \textbf{nâch} der strûz.\\ 
 & \begin{large}G\end{large}ahmuret stach hinder daz ors\\ 
10 & Pontewinnen \textbf{den} \textbf{Prienlascors}\\ 
 & und anders manigen werden man,\\ 
 & an \textbf{dem} er sicherheit gewan.\\ 
 & waz d\textit{â} \textbf{getriuwer} ritter reit,\\ 
 & die genuzzen des heldes arbeit.\\ 
15 & diu gewunnen ros, diu gab er in.\\ 
 & an ime lac \textbf{grôz ir} gewin.\\ 
 & gelîcher baniere\\ 
 & man \textbf{gegen ime vuorte} \textbf{viere}\\ 
 & - \textit{küe}ne rotten riten drunde,\\ 
20 & ir \textbf{hêrre} strîten \textbf{kunde} -\\ 
 & an ieglîcher \textbf{eines} grîfen zagel.\\ 
 & daz \textbf{hinder} teil \textbf{ouch was} ein \textit{h}agel\\ 
 & \textbf{an} ritterschafte. des wâren die.\\ 
 & daz vorder teil des grîfen hie\\ 
25 & der künic von Gascone truoc\\ 
 & \textbf{ûf} \textbf{einem} schilte, ein ritter kluoc.\\ 
 & \textbf{gezimiert} was \textbf{des heldes} lîp,\\ 
 & \textbf{sô} wol gebrüefen kunnen wîp.\\ 
 & \textbf{er} nam sich vor den andern ûz,\\ 
30 & dô er ûf \textbf{ersach} den strûz.\\ 
\end{tabular}
\scriptsize
\line(1,0){75} \newline
m n o \newline
\line(1,0){75} \newline
\textbf{9} \textit{Initiale} m   $\cdot$ \textit{Capitulumzeichen} n  \newline
\line(1,0){75} \newline
\textbf{1} unz] bitz n (o)  $\cdot$ huof] huͦp \textit{nachträglich korrigiert zu:} huͦff m \textbf{3} spranc] sprag o  $\cdot$ drûf] uͯff o  $\cdot$ dâ] do m n o \textbf{4} starker] stercker m \textbf{5} verswande] dort verswande n \textbf{6} dis] Das o \textbf{8} volgete] er volget n folget o \textbf{9} Gahmuret] Gamiret n Gamuret o \textbf{10} Pontewinnen] Pontewinen n Ponte winen o  $\cdot$ den] de n o  $\cdot$ Prienlascors] [prenlascors]: prienlascors n \textbf{11} anders] \textit{om.} n \textbf{12} gewan] began o \textbf{13} waz] Wa o  $\cdot$ dâ] do m n o \textbf{14} heldes] helles o \textbf{15} diu] \textit{om.} n o \textbf{18} man] Wenne n (o) \textbf{19} küene] [*ine]: Sine m  $\cdot$ riten] ritter o  $\cdot$ drunde] drunder m (n) dar [imder]: vnder  o \textbf{20} kunde] wol kunde n \textbf{22} hagel] zagel m \textbf{23} des] das n o \textbf{24} des] das o \textbf{26} einem] synem n (o) \textbf{27} gezimiert] Gezÿniret o  $\cdot$ lîp] pin o \newline
\end{minipage}
\end{table}
\newpage
\begin{table}[ht]
\begin{minipage}[t]{0.5\linewidth}
\small
\begin{center}*G
\end{center}
\begin{tabular}{rl}
 & gewâpent \textbf{vaste} unze ûf den huof.\\ 
 & hie garzûn, ruofâ ruof!\\ 
 & sîn lîp spra\textit{n}c drûf, wan erz dâ vant.\\ 
 & vil starker sper des heldes hant\\ 
5 & mit hurte verswande.\\ 
 & \textbf{die} ponder er zertrande,\\ 
 & \textbf{imer} anderthalben ûz.\\ 
 & dem anker volgete \textbf{nâch} der strûz.\\ 
 & Gahmuret stach hinderz ors\\ 
10 & Poitewinen \textbf{von} \textbf{Prienlacors}\\ 
 & unde anders manigen werden man,\\ 
 & an \textbf{dem} er sicherheit gewan.\\ 
 & swaz dâ \textbf{gekriu\textit{z}ter} rîter reit,\\ 
 & die genuzzen des heldes arbeit.\\ 
15 & diu gewunnen ors, diu gap er in.\\ 
 & an im lac \textbf{ir grôz} gewin.\\ 
 & \begin{large}G\end{large}elîcher baniere\\ 
 & man \textbf{vuorte gein im} \textbf{viere}\\ 
 & - küene roten riten drunden,\\ 
20 & ir \textbf{hêrren} strîten \textbf{kunden} -\\ 
 & an iegelîcher \textbf{ein} grîfen zagel.\\ 
 & daz \textbf{ander} teil \textbf{ouch was} ein hagel\\ 
 & \textbf{gein} rîterschaft. des wâren die.\\ 
 & daz vorder teil des grîfen hie\\ 
25 & der künic von Gascon truoc\\ 
 & \textbf{an} \textbf{dem} schilt, ei\textit{n} rîter kluoc.\\ 
 & \textbf{gezimiert} was \textbf{sîn} lîp,\\ 
 & \textbf{als} wol geprüeven kunnen wîp.\\ 
 & \textbf{der} nam sich vor den andern ûz,\\ 
30 & dôr ûf \textbf{dem helm\textit{e}} \textbf{\textit{s}ach} den strûz.\\ 
\end{tabular}
\scriptsize
\line(1,0){75} \newline
G I O L M Q R Z Fr21 \newline
\line(1,0){75} \newline
\textbf{1} \textit{Initiale} O  \textbf{9} \textit{Initiale} I  \textbf{17} \textit{Initiale} G  \textbf{19} \textit{Initiale} M  \textbf{29} \textit{Überschrift:} Hie kvmt gamuret vnd kvnic Gascon in dem strite zv samnen Z   $\cdot$ \textit{Initiale} I L R Z Fr21  \newline
\line(1,0){75} \newline
\textbf{1} gewâpent] ÷ewafent O Gampent R  $\cdot$ unze] \textit{om.} M  $\cdot$ den huof] den [vuͤz]: huͤf I die fuͦs R \textbf{2} garzûn] Ruͦffent garczunge R  $\cdot$ ruof] Ruͦs R \textbf{3} sîn lîp] Er Q  $\cdot$ spranc] sprach G  $\cdot$ dâ] do Q  $\cdot$ vant] want I \textbf{4} Vil starker sper er da fand / Die vertett des heldes hand R  $\cdot$ starker] starke M \textbf{5} mit] mit siner I  $\cdot$ verswande] do verswant Q \textbf{6} er] \textit{om.} O \textbf{7} ein halb in anderthalb I  $\cdot$ imer] Jmmer durch O M (Q) Z (Fr21) Hie duͯrch L Siner durch R  $\cdot$ anderthalben] andertalbere Q anderhab Fr21 \textbf{8} volgete] volget I (O) Q (R) Z \textbf{9} Gahmuret] Gamvret O Gahmuͯret L [Gamuet]: Gamuret M Gamuret Q Z Gahmoret Fr21  $\cdot$ hinderz] hinder R \textbf{10} Poitewinen] potewunen I Poytwinen O Fr21 Poytewinen L Q R Beide winen M Portewinen Z  $\cdot$ von] \textit{om.} O de M Z Fr21  $\cdot$ Prienlacors] prinlacors G brinlascors I Prinlahiors O (Fr21) prýelaiors L prilachiors M prienlaiors Q (R) \textbf{11} anders] ander I O R Z  $\cdot$ manigen werden] werde R manige werde Z \textbf{12} dem] den L R Z  $\cdot$ er] \textit{om.} I \textbf{13} swaz] Was L (M) Q (Z) Sus R  $\cdot$ dâ] do Q so R  $\cdot$ gekriuzter] gekruter G armer I kruͯtziter L (Fr21) geziret Q \textbf{14} genuzzen] geniszen M  $\cdot$ des heldes] siner I \textbf{15} diu gap] gab O (L) M Q Fr21 gas R  $\cdot$ in] hin L \textbf{16} ir grôz] goz ir I \textbf{18} Gein im fvrte man fiere O (Fr21)  $\cdot$ man vuorte] Fuͯrte man L (Q)  $\cdot$ gein] nach R  $\cdot$ im] in I  $\cdot$ viere] [mere]: viere M \textbf{19} roten riten] ritter O rotten ritter L rittere ritten M rotte ritten Z \textbf{20} hêrren] herre O L (M) Z Fr21  $\cdot$ strîten] streite Q  $\cdot$ kunden] chvnde O (L) (M) (Q) (Z) (Fr21) kone R \textbf{21} ein] eins I Z  $\cdot$ grîfen] grefen Q \textbf{22} daz] vnd daz I (O) (L) (M) (Q) (Fr21) Vnd des R  $\cdot$ ander teil] hinder teil O L (M) Q (R) (Z) (Fr21)  $\cdot$ ouch was] was avch O (L) (Fr21) was M \textbf{23} gein] An L Z Ein Q R  $\cdot$ des] das Q \textbf{25} Gascon] asconie G Gashoni I Goscon O gaschon M Dascone Q \textbf{26} dem] sim I (M) (R) (Z)  $\cdot$ ein] einen G L (Q) Z \textbf{28} als] Sam O M Q Z Fr21  $\cdot$ geprüeven] bruͤven I (O) geprufet M  $\cdot$ kunnen] kunden I (M) kvnne Fr21  $\cdot$ wîp] div wip O (R) (Fr21) \textbf{29} vor] von I  $\cdot$ den] der O  $\cdot$ andern] ander Q \textbf{30} dôr] Da her M (Z)  $\cdot$ sach] er sach G \newline
\end{minipage}
\hspace{0.5cm}
\begin{minipage}[t]{0.5\linewidth}
\small
\begin{center}*T (U)
\end{center}
\begin{tabular}{rl}
 & gewâpent \textbf{wol} unz ûf den huof.\\ 
 & hie garzûne, ruofâ ruof!\\ 
 & sîn \textit{lîp} spranc dar ûf, wan er ez dâ vant.\\ 
 & vil starker sper des heldes hant\\ 
5 & mit hurte verswande.\\ 
 & \textbf{die} poynder er zertrande\\ 
 & \textbf{eine sîte in}, anderthalp ûz.\\ 
 & dem anker volgete \textbf{ie} der strûz.\\ 
 & Gahmuret stach hinder\textit{z} ors\\ 
10 & Poytewin \textbf{d\textit{e}} \textbf{Prialatscors}\\ 
 & und anders manigen werden man,\\ 
 & an \textbf{den} er sicherheit gewan.\\ 
 & waz dâ \textbf{gekriu\textit{z}eter} ritter reit,\\ 
 & die genuzzen des heldes arbeit.\\ 
15 & diu gewunnen ors, diu gab er in.\\ 
 & an ime lac \textbf{ir grôz} gewin.\\ 
 & gelîcher baniere\\ 
 & man \textbf{vuorte gein im} \textbf{schiere}\\ 
 & - küene rotten riten drunde,\\ 
20 & ir \textbf{hêrre} strîten \textbf{ku\textit{nde}} -\\ 
 & an ieclîcher \textbf{eines} grîfen zagel.\\ 
 & daz \textbf{hinder} teil \textbf{ouch was} ein hagel\\ 
 & \textbf{gein} ritterschefte. des wâren die.\\ 
 & daz vorder teil des grîfen hie\\ 
25 & der künec von Gasgone truoc\\ 
 & \textbf{an} \textbf{sîme} schilte, ein ritter kluoc.\\ 
 & \textbf{gezieret} was \textbf{sîn} lîp,\\ 
 & \textbf{sa\textit{m}} \textit{w}ol geprüeven kunnen wîp.\\ 
 & \textbf{der} nam sich vor den andern ûz,\\ 
30 & dô er ûf \textbf{dem helme} \textbf{sach} den strûz.\\ 
\end{tabular}
\scriptsize
\line(1,0){75} \newline
U V W T \newline
\line(1,0){75} \newline
\textbf{13} \textit{Majuskel} T  \textbf{17} \textit{Majuskel} T  \textbf{20} \textit{Majuskel} T  \textbf{29} \textit{Initiale} W  \newline
\line(1,0){75} \newline
\textbf{1} wol unz] wol bit U vaste vnz T \textbf{2} hie] [*]: hie rieffent V \textbf{3} lîp] \textit{om.} U  $\cdot$ ez] \textit{om.} W  $\cdot$ dâ] do V W  $\cdot$ vant] rant W \textbf{4} starker] starcke W \textbf{5} verswande] da verswande T \textbf{7} nv hie in sa dort ûz T  $\cdot$ eine sîte in] Einsit in vnd V Iennet durch W \textbf{8} volgete] volget W T  $\cdot$ ie] [*]: nach V ie nach W \textbf{9} Gahmuret] Gahmuͦret U Gamuret V W  $\cdot$ hinderz] hinder U \textbf{10} Poytewin] Poẏthewinen V Poteuin W  $\cdot$ de] der U  $\cdot$ Prialatscors] Prielat cors U Preilascors V prialadschorß W \textbf{11} anders] ander W anderz T \textbf{13} waz] Swaz V (T)  $\cdot$ dâ] do V \textit{om.} W  $\cdot$ gekriuzeter] gecruͦteter U getruscheter V gegruͤzter W \textbf{14} genuzzen] geniessen W \textbf{15} diu gab er] gaber T \textbf{18} man vuorte] Fuͦrte man W  $\cdot$ gein im schiere] [*]: noch im viere V gegen im fiere W (T) \textbf{19} drunde] darunder W \textbf{20} Die waren stoltz vnd munder W  $\cdot$ kunde] [kude]: ku U \textbf{21} ieclîcher] ieglichen banier W  $\cdot$ eines] ein V W T \textbf{22} daz] vnd daz V  $\cdot$ ouch was] [*]: waz V was auch W (T) \textbf{23} gein] An V \textbf{24} daz] Iens W des T  $\cdot$ vorder] [*]: wider V \textbf{25} Gasgone] Gaschgonie V \textbf{26} an] Diß an W \textbf{27} gezieret] Gezimieret V W (T) \textbf{28} sam] als T  $\cdot$ wol] die wol U V W  $\cdot$ geprüeven] pruͤuen V  $\cdot$ wîp] div wip T \textbf{29} der] [*]: Er V  $\cdot$ vor] von T \textbf{30} sach] ersach T \newline
\end{minipage}
\end{table}
\end{document}
