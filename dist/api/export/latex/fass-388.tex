\documentclass[8pt,a4paper,notitlepage]{article}
\usepackage{fullpage}
\usepackage{ulem}
\usepackage{xltxtra}
\usepackage{datetime}
\renewcommand{\dateseparator}{.}
\dmyyyydate
\usepackage{fancyhdr}
\usepackage{ifthen}
\pagestyle{fancy}
\fancyhf{}
\renewcommand{\headrulewidth}{0pt}
\fancyfoot[L]{\ifthenelse{\value{page}=1}{\today, \currenttime{} Uhr}{}}
\begin{document}
\begin{table}[ht]
\begin{minipage}[t]{0.5\linewidth}
\small
\begin{center}*D
\end{center}
\begin{tabular}{rl}
\textbf{388} & \textbf{\begin{large}D\end{large}er} dâ nâch prîse \textbf{wol rite}\\ 
 & unt nâch der wîbe lône \textbf{strite},\\ 
 & \textbf{ine m\textit{ö}ht ir niht} erkennen.\\ 
 & solt ich si \textbf{iu} alle nennen,\\ 
5 & ich würde ein unmüezec man.\\ 
 & innerhalp wart ez \textbf{dâ} guot getân\\ 
 & durch die jungen Obilot\\ 
 & unt ûzerhalp ein ritter rôt,\\ 
 & die zwêne behielten dâ den prîs,\\ 
10 & vür si niemen \textbf{decheinen gewîs}.\\ 
 & Dô des ûzeren hers gast\\ 
 & innen wart, daz im gebrast\\ 
 & \textbf{dienstdankes} von dem meister sîn\\ 
 & - der was gevangen hin în -,\\ 
15 & er reit, dâ er sîne knappen sach.\\ 
 & ze sînen gevangen er dô sprach:\\ 
 & "Ir hêrren \textbf{gâbet} mir sicherheit.\\ 
 & \textbf{mir ist hie} widervarn leit:\\ 
 & gevangen ist der künec von Liz.\\ 
20 & nû kêret allen iweren vlîz,\\ 
 & ob er ledec \textbf{müge} sîn,\\ 
 & mag er \textbf{sô vil} geniezen min",\\ 
 & sprach er zem künege von Avendroyn\\ 
 & unt ze Schirniel von Lyrivoyn\\ 
25 & unt zem herzogen Marangliez.\\ 
 & mit spæher gelübde er si liez\\ 
 & von im \textbf{rîten} in die stat.\\ 
 & Melyanzen er si lœsen bat\\ 
 & oder daz si \textbf{erwürben im} den Grâl.\\ 
30 & si\textbf{ne} kunden im ze keinem mâl\\ 
\end{tabular}
\scriptsize
\line(1,0){75} \newline
D \newline
\line(1,0){75} \newline
\textbf{1} \textit{Initiale} D  \textbf{11} \textit{Majuskel} D  \textbf{17} \textit{Majuskel} D  \newline
\line(1,0){75} \newline
\textbf{3} möht] moht D \textbf{19} Liz] Lŷz D \textbf{24} Schirniel] [Scirmel]: Scirniel D \newline
\end{minipage}
\hspace{0.5cm}
\begin{minipage}[t]{0.5\linewidth}
\small
\begin{center}*m
\end{center}
\begin{tabular}{rl}
 & \textbf{wer} dâ nâch prîse \textbf{wolte rîten}\\ 
 & und nâch der wîbe lône \textbf{strîten},\\ 
 & \textbf{der m\textit{ö}ht ich niht} erk\textit{e}nnen.\\ 
 & solt ich si alle nennen,\\ 
5 & ich würde ein unmüezic man.\\ 
 & innerhalp wart ez \textbf{d\textit{â}} guot getân\\ 
 & durch die jungen \textit{O}bilot\\ 
 & und ûzerhalp ein ritter rôt,\\ 
 & die zwêne behielten d\textit{â} den prîs,\\ 
10 & vür si nieman \textbf{dekeine wîs}.\\ 
 & dô des ûzern hers gast\\ 
 & innen wart, daz ime gebrast\\ 
 & \textbf{dienstdankes} von dem meister sîn\\ 
 & - der was gevangen hin în -,\\ 
15 & er reit, d\textit{â} er sîne knappen sach.\\ 
 & ze sîne\textit{n} gevangen er dô sprach:\\ 
 & "ir hêrren, \textbf{gebet} mir sicherheit.\\ 
 & \textbf{nû ist mir} widervarn leit:\\ 
 & gevangen ist der künic von Liz.\\ 
20 & nû \textit{kê}ret allen iuwern vlîz,\\ 
 & ob er ledic \textbf{müeze} sîn,\\ 
 & mac er \textbf{sô vil} geniezen mîn",\\ 
 & sprach er zem künic von Avendr\textit{oin}\\ 
 & und ze \textit{Sch}ir\textit{ni}el von Lirw\textit{oin}\\ 
25 & und zem her\textit{z}oge\textit{n} \textit{M}arangliez.\\ 
 & mit spæher gelübde er si liez\\ 
 & von ime \textbf{rîten} in die stat.\\ 
 & Melianzen er si lœsen bat\\ 
 & oder daz si \textbf{erwürben im} den Grâl.\\ 
30 & si \textbf{en}kunden ime ze keinem mâl\\ 
\end{tabular}
\scriptsize
\line(1,0){75} \newline
m n o \newline
\line(1,0){75} \newline
\newline
\line(1,0){75} \newline
\textbf{1} dâ] do n o \textbf{3} möht] moht m (o)  $\cdot$ erkennen] erkonnen m \textbf{6} dâ] do m \textit{om.} n o  $\cdot$ guot] [wol]: gut o \textbf{7} Obilot] abilot m abilat o \textbf{9} dâ] do m n o \textbf{10} dekeine] do keine n \textbf{13} dienstdankes] Dienstes danckes o \textbf{14} hin] bẏ n o \textbf{15} dâ] do m n o  $\cdot$ sîne] den n sin o \textbf{16} sînen] sinem m \textbf{19} Liz] lisz n lis o \textbf{20} kêret] hoͯrent m \textbf{21} müeze] muse m \textbf{23} Avendroin] auendrvm m avondron n auendrom o \textbf{24} Schirniel] zirwel m scirmel n sarmel o  $\cdot$ Lirwoin] lirwum m lirwon n linwom o \textbf{25} herzogen Marangliez] herhogen von maranglies m hertzogen von marangliesz n herczogen maranglies o \textbf{26} gelübde] gelub o \textbf{28} Melianzen] Melianczen m Meliantzen n Meleanczen o  $\cdot$ lœsen] laszen o \textbf{29} im] in o \textbf{30} enkunden] kunden n (o) \newline
\end{minipage}
\end{table}
\newpage
\begin{table}[ht]
\begin{minipage}[t]{0.5\linewidth}
\small
\begin{center}*G
\end{center}
\begin{tabular}{rl}
 & \textbf{wer} dâ nâch prîse \textbf{wol rite}\\ 
 & unde nâch der wîbe lône \textbf{strite}?\\ 
 & \textbf{lât mich si wol} erkennen.\\ 
 & solt ich si \textbf{iu} alle ne\textit{n}nen,\\ 
5 & ich würde ein u\textit{n}müezic man.\\ 
 & innerhalp wart ez guot getân\\ 
 & durch die jungen Obilot\\ 
 & unt \textit{ûz}er\textit{halp} ein rîter rôt,\\ 
 & die zwêne behielten dâ den prîs\\ 
10 & \textbf{unde} vür si niemen \textbf{deheine wîs}.\\ 
 & dô des ûzern hers gast\\ 
 & innen wart, daz im gebrast\\ 
 & \textbf{dienstdankes} von dem meister sîn\\ 
 & - der was gevangen hin în -,\\ 
15 & er reit, dâ er sîne knappen sach.\\ 
 & ze sînen gevange\textit{n} er dô sprach:\\ 
 & "ir hêrren, \textbf{ir} \textbf{gâbet} mir sicherheit.\\ 
 & \textbf{mir ist hie} widervaren leit:\\ 
 & \begin{large}G\end{large}evangen ist der künic von Liz.\\ 
20 & nû kêrt allen iwern vlîz,\\ 
 & ober ledic \textbf{müge} sîn,\\ 
 & mag er \textbf{dâr an} geniezen mîn",\\ 
 & sprach er zem künige von Avendroyn\\ 
 & unt ze Tschirnel von Liravoyn\\ 
25 & unt zem herzogen Marangliez.\\ 
 & mit spæher gelübede er si liez\\ 
 & von im \textbf{rîten} in die stat.\\ 
 & Melianzen er si lœsen bat\\ 
 & oder daz si\textbf{m würben umbe} den Grâl.\\ 
30 & si kunden im ze deheine\textit{m} mâl\\ 
\end{tabular}
\scriptsize
\line(1,0){75} \newline
G I O L M Q R Z Fr28 \newline
\line(1,0){75} \newline
\textbf{1} \textit{Initiale} I O L Z   $\cdot$ \textit{Capitulumzeichen} R  \textbf{17} \textit{Initiale} I  \textbf{19} \textit{Initiale} G M  \newline
\line(1,0){75} \newline
\textbf{1} \textit{Die Verse 370.13-412.12 fehlen} Q   $\cdot$ wer] ÷er O \textbf{2} wîbe] wiben R \textbf{3} Jch enmoht ir niht erkennen Z  $\cdot$ mich si] michs O L \textbf{4} ich si iu] ich ev si I (R) ichs iv O (L) (Z) ich sie M  $\cdot$ nennen] nenenen G \textbf{5} unmüezic] vmoͮzch G vn Mudic M \textbf{6} innerhalp] innerthalben I  $\cdot$ guot] da wol O da guͯt L (M) (Z) do guͦt R \textbf{7} jungen] Junge R  $\cdot$ Obilot] Obylot O \textbf{8} ûzerhalp] der vor G \textbf{9} dâ] do L \textit{om.} R \textbf{10} deheine] deheinen O (Z) In dehein R \textbf{11} dô] Da M \textbf{12} im] yn M \textbf{13} dienstdankes] Dienstes L Dienstes dankes R \textbf{15} knappen] [gevangene]: chnapen G  $\cdot$ sach] [vant]: sach O fand R \textbf{16} sînen] sim I sinē M [sinem]: sinen Z  $\cdot$ gevangen] [chnapen]: gevangene G gevangenen L  $\cdot$ er dô sprach] sprach er ze hand R \textbf{17} gâbet] habt Z \textbf{19} Gevangen] Gawan R  $\cdot$ Liz] Lýz L lisz M Lis R \textbf{20} nû] \textit{om.} I  $\cdot$ allen] alle M \textbf{21} ober] Ob ir R  $\cdot$ ledic] ledic hie I  $\cdot$ müge] mugen R  $\cdot$ sîn] gesyn M \textbf{22} dâr an] so vil Z  $\cdot$ geniezen] geniesse R \textbf{23} von] \textit{om.} M  $\cdot$ Avendroyn] auetroyn I avendroẏz O Auendroyn R \textbf{24} ze Tschirnel] zetschirnel G scriuiel I zeschirme O zuͯ Schirmel L zcu scirinel M ze schirniel R zv schieniel Z  $\cdot$ Liravoyn] liaravoin G lirauoyn I Liravoẏz O Lýravoýn L liravoyn M Lyravoyn R Lyravoin Z lẏravoim Fr28 \textbf{25} herzogen] herzogen von I (Fr28)  $\cdot$ Marangliez] meriangliez G moragliez O Marangleis R \textbf{26} spæher] spehen I spahem L \textbf{28} Melianzen] Melyanzen O Melyanczen R Meliantzen Z \textbf{29} sim würben umbe] im wurben vmb O sie yme wurden vmmb M sie erwurben im Z \textbf{30} si] sin I (L) (M) (Z) (Fr28)  $\cdot$ ze deheinem] zedeheinen G zdem einen O \newline
\end{minipage}
\hspace{0.5cm}
\begin{minipage}[t]{0.5\linewidth}
\small
\begin{center}*T
\end{center}
\begin{tabular}{rl}
 & \textbf{\begin{large}D\end{large}er} dâ nâch prîse \textbf{wolte rîten}\\ 
 & unde nâch der wîbe lône \textbf{strîten}:\\ 
 & \textbf{Lât mich si wol} erkennen.\\ 
 & soltich si alle nennen,\\ 
5 & ich würde ein unmüezic man.\\ 
 & innerhalp wart ez \textbf{dâ} guot getân\\ 
 & durch die jungen Obylot\\ 
 & unde ûzerhalp ein rîter rôt,\\ 
 & die zwêne behielten dâ den prîs\\ 
10 & \textbf{unde} vür si nieman \textbf{deheine wîs}.\\ 
 & Dô des ûzern hers gast\\ 
 & innen wart, daz im gebrast\\ 
 & \textbf{dienstdankens} von dem meister sîn\\ 
 & - der was gev\textit{a}ngen hin în -,\\ 
15 & er reit, dâ er sîne knappen sach.\\ 
 & ze sînen gevangen er dô sprach:\\ 
 & "ir hêrren, \textbf{ir} \textbf{gâbet} mir sicherheit.\\ 
 & \textbf{mir ist hie} widervarn leit:\\ 
 & gevangen ist der künec von Liz.\\ 
20 & nû kêret allen iuwern vlîz,\\ 
 & ob er ledic \textbf{müge} sîn,\\ 
 & mag er \textbf{dâr an} geniezen mîn",\\ 
 & sprach er zem künege von Evendroyn\\ 
 & unde ze Tschirniel von Lyravoyn\\ 
25 & unde zem herzogen Marangliez.\\ 
 & mit spæher gelübde er si liez.\\ 
 & von im \textbf{si} \textbf{riten} in die stat.\\ 
 & Melyanzen er si lœsen bat\\ 
 & oder daz si\textbf{m würben umb}en Grâl.\\ 
30 & si\textbf{ne} kunden im ze deheinem mâl\\ 
\end{tabular}
\scriptsize
\line(1,0){75} \newline
T V W \newline
\line(1,0){75} \newline
\textbf{1} \textit{Initiale} T V W  \textbf{3} \textit{Majuskel} T  \textbf{11} \textit{Majuskel} T  \newline
\line(1,0){75} \newline
\textbf{1} Der dâ] Wer do V DEr do W  $\cdot$ wolte rîten] wol ritte V wol reite W \textbf{2} lône strîten] lone [*]: stritte V lenge streite W \textbf{3} [*]: Der moͤhte ich niht [*]: erkennen V \textbf{4} si] eúch W \textbf{6} ez dâ guot] [*]: ez wol V es do guͦt W \textbf{7} Obylot] Obylôt T obẏlot V abilot W \textbf{9} dâ] do V W \textbf{13} dienstdankens] Dienest dankes V \textbf{14} gevangen] gevavangen T \textbf{15} dâ] do V W  $\cdot$ sach] vand W \textbf{16} er dô sprach] sprach er zuͦ hand W \textbf{19} Liz] Lŷz T lis V W \textbf{20} kêret] hoͤren W \textbf{22} geniezen] [*]: so vil V \textbf{23} Evendroyn] Cvdrôyn T [*]: auendroẏn V auendroyn W \textbf{24} Tschirniel] Tscirniel T schirniel V schirmel W  $\cdot$ Lyravoyn] Lyravôyn T [*]: lẏravoẏn V lymuoym W \textbf{25} Marangliez] Maranglŷez T maranglies V marangließ W \textbf{26} si] in W \textbf{27} si] \textit{om.} W \textbf{28} Melyanzen] Melẏanzen V Melianzen W \textbf{29} sim würben umben] sv́ [*]: erwúrben im den V \newline
\end{minipage}
\end{table}
\end{document}
