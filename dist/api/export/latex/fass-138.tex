\documentclass[8pt,a4paper,notitlepage]{article}
\usepackage{fullpage}
\usepackage{ulem}
\usepackage{xltxtra}
\usepackage{datetime}
\renewcommand{\dateseparator}{.}
\dmyyyydate
\usepackage{fancyhdr}
\usepackage{ifthen}
\pagestyle{fancy}
\fancyhf{}
\renewcommand{\headrulewidth}{0pt}
\fancyfoot[L]{\ifthenelse{\value{page}=1}{\today, \currenttime{} Uhr}{}}
\begin{document}
\begin{table}[ht]
\begin{minipage}[t]{0.5\linewidth}
\small
\begin{center}*D
\end{center}
\begin{tabular}{rl}
\textbf{138} & Sus riten si ûf der slâ hin nâch.\\ 
 & dem knappen \textbf{vor in ouch was} \textbf{vil} gâch.\\ 
 & \textbf{doch wesse} der unverzagete\\ 
 & niht, daz man in jagete.\\ 
5 & wan swe\textit{n} sîn ougen sâhen,\\ 
 & sô er \textbf{dem} begunde nâhen,\\ 
 & den gruozte der knappe guoter\\ 
 & unt jach: "sus \textbf{riet mir} mîn muoter."\\ 
 & \textbf{\begin{large}S\end{large}us} kom unser \textbf{tœrscher} knabe\\ 
10 & geriten \textit{e}ine halden abe.\\ 
 & wîbes stimme er \textbf{hôrte}\\ 
 & vor eines \textbf{velses} orte.\\ 
 & ein vrouwe ûz rehtem jâmer schrei.\\ 
 & \textbf{ir} was diu wâre vreude enzwei.\\ 
15 & der knappe reit ir balde zuo.\\ 
 & nû hœret, waz diu vrouwe tuo.\\ 
 & \textbf{dâ} brach vrou Sigune\\ 
 & ir langen zöpfe brûne\\ 
 & \textbf{vor} jâmer ûz \textbf{ir} swarten.\\ 
20 & der knappe begunde warten.\\ 
 & Schianatulander,\\ 
 & den vürsten, tôt dâ vander,\\ 
 & der juncvrouwen \textbf{tôt} in ir schôz.\\ 
 & \textbf{aller schimpfe} si verdrôz.\\ 
25 & "Er sî trûric oder vreuden var,\\ 
 & die \textbf{bat} \textbf{mîn} muoter grüezen gar.\\ 
 & got halde iuch", sprach des knappen munt,\\ 
 & "ich hân hie jæmerlîchen vunt\\ 
 & in \textbf{iwerm schôze} vunden.\\ 
30 & wer gap iu den ritter wunden?\\ 
\end{tabular}
\scriptsize
\line(1,0){75} \newline
D \newline
\line(1,0){75} \newline
\textbf{1} \textit{Majuskel} D  \textbf{9} \textit{Initiale} D  \textbf{25} \textit{Majuskel} D  \newline
\line(1,0){75} \newline
\textbf{5} swen] swenne D \textbf{10} eine] enine D \textbf{17} Sigune] Sigv̂ne D \textbf{21} Schianatulander] Scianatvlandr D \newline
\end{minipage}
\hspace{0.5cm}
\begin{minipage}[t]{0.5\linewidth}
\small
\begin{center}*m
\end{center}
\begin{tabular}{rl}
 & sus riten\textit{s} ûf der slage hin nâch.\\ 
 & dem knappen \textbf{vor in ouch was} gâch.\\ 
 & \textbf{daz enweiz} der \textit{un}verzagete\\ 
 & niht, daz man in jagete.\\ 
5 & wanne wen sîn ougen s\textit{â}hen,\\ 
 & sô er \textbf{dem} begunde nâhen,\\ 
 & den gruoz\textit{t}e der knabe guoter\\ 
 & und jach: "sus \textbf{reit} mîn muoter."\\ 
 & \textbf{\begin{large}S\end{large}us} kam unser \textbf{tœrscher} knabe\\ 
10 & geriten einen halden abe.\\ 
 & wîbes stimme er \textbf{hôrte}\\ 
 & vor eines \textbf{velses} orte.\\ 
 & ein vrouwe ûz rehtem jâmer schrei.\\ 
 & \textbf{ir} was diu wâre vröude enzwei.\\ 
15 & der knappe reit ir balde zuo.\\ 
 & nû hœret, waz diu vrouwe tuo.\\ 
 & \textbf{d\textit{â}} brach vrouwe Sigune\\ 
 & ir langen zöpf\textit{e} brûne\\ 
 & \textbf{vor} jâmer ûz \textbf{ir} swarten.\\ 
20 & der knappe begunde warten.\\ 
 & S\textit{ch}ianatulander,\\ 
 & den vürsten, tôt d\textit{â} vant er\\ 
 & der juncvrouwen in ir schôz.\\ 
 & \textbf{aller schimpf} si verdrôz.\\ 
25 & \begin{large}E\end{large}r sî trûric oder vröude var,\\ 
 & die \textbf{bat} \textbf{sîn} muoter grüezen gar.\\ 
 & "got halte iuch", sprach des knaben munt,\\ 
 & "ich hân hie jâmerlîchen vunt\\ 
 & in \textbf{iuwerem schôze} vunden.\\ 
30 & wer gap iu den ritter wunden?"\\ 
\end{tabular}
\scriptsize
\line(1,0){75} \newline
m n o \newline
\line(1,0){75} \newline
\textbf{9} \textit{Überschrift:} Wie (Also n o  ) parcifal von siner muͯtter reit (schiet n  ) vnd Sigunen (sú n  siguͯn o  ) einen toten ritter ime huse vant m (n) (o)   $\cdot$ \textit{Illustration} o   $\cdot$ \textit{Initiale} m n o  \textbf{25} \textit{Initiale} m   $\cdot$ \textit{Capitulumzeichen} n  \newline
\line(1,0){75} \newline
\textbf{1} ritens] rittend m \textbf{2} dem] Den o  $\cdot$ vor in ouch] auch vor in o \textbf{3} daz enweiz] Doch wuste n Ouch wuͯste o  $\cdot$ unverzagete] verzagette m [verzagete]: vnverzagete o \textbf{4} jagete] zagte o \textbf{5} wen] wenne n  $\cdot$ sâhen] sehen m \textbf{7} gruozte] gruͯsse m \textbf{8} reit] riet n \textbf{12} velses] feilsen n felschen o \textbf{17} dâ] Do m n o  $\cdot$ Sigune] siguͦn n siguͯn o \textbf{18} ir] Jrn o  $\cdot$ zöpfe] zopfen m zopff o \textbf{19} vor] Von n o \textbf{20} der] Den o \textbf{21} Schianatulander] Stianatulander m Scian atukalander n [Scian atulander]: Scian atukalander o \textbf{22} den] Der o  $\cdot$ dâ] do m o so n  $\cdot$ vant er] funder o \textbf{24} verdrôz] das verdros o \textbf{26} sîn] min n (o) \textbf{28} vunt] frunt o \textbf{29} iuwerem] irem m vwern o \textbf{30} ritter] ritters o \newline
\end{minipage}
\end{table}
\newpage
\begin{table}[ht]
\begin{minipage}[t]{0.5\linewidth}
\small
\begin{center}*G
\end{center}
\begin{tabular}{rl}
 & sus riten si ûf der slâ hin nâch.\\ 
 & dem knappen \textbf{was ouch vor in} gâch.\\ 
 & \textbf{dône wesse} der unverzagte\\ 
 & niht, daz man in jagete.\\ 
5 & wan swen sîn ougen sâhen,\\ 
 & sô er \textbf{de\textit{m}} begunde nâhen,\\ 
 & den gruozte der knappe guoter\\ 
 & unde jach: "sus \textbf{riet} mîn muoter."\\ 
 & \textbf{alsus} kom unser \textbf{tœrscher} knabe\\ 
10 & geriten ein hal\textit{d}en abe.\\ 
 & wîbes stimme er \textbf{hôrte}\\ 
 & vor eines \textbf{velses} orte.\\ 
 & ein vrouwe ûz rehtem jâmer schrei.\\ 
 & \textbf{ir} was diu wâre vröude enzwei.\\ 
15 & der knappe reit ir balde zuo.\\ 
 & nû hœret, waz diu vrouwe tuo.\\ 
 & \textbf{ez} brach vrô Sigune\\ 
 & ir lange zöpfe brûne\\ 
 & \textbf{vor} jâmer ûz \textbf{der} swarten.\\ 
20 & der knabe begunde warten.\\ 
 & Tschianatulander,\\ 
 & den vürsten, tôt dâ vander\\ 
 & der juncvrouwen in ir schôz.\\ 
 & \textbf{alles schimpfes} si verdrôz.\\ 
25 & "er sî trûrec oder vröuden var,\\ 
 & die \textbf{bat} \textbf{mîn} muoter grüezen gar.\\ 
 & got halde iuch", sprach des knappen munt,\\ 
 & "ich hân hie jæmerlîchen vunt\\ 
 & \begin{large}I\end{large}n \textbf{iwe\textit{r} schôz\textit{e}} vunden.\\ 
30 & wer gap iu den rîter wunden?"\\ 
\end{tabular}
\scriptsize
\line(1,0){75} \newline
G I O L M Q R Z \newline
\line(1,0){75} \newline
\textbf{9} \textit{Initiale} R  \textbf{11} \textit{Initiale} I  \textbf{13} \textit{Initiale} L  \textbf{15} \textit{Initiale} O Z  \textbf{29} \textit{Initiale} G  \newline
\line(1,0){75} \newline
\textbf{1} sus] Ausz Q  $\cdot$ der slâ] seiner sla Q dem schlag R \textbf{2} gâch] >nih< Gach I \textbf{3} dône] Da en M Denne Q  $\cdot$ wesse] west niht I  $\cdot$ unverzagte] vnuertzagt Q \textbf{4} niht] \textit{om.} I  $\cdot$ jagete] so sere iagete I gacht Q \textbf{5} wan swen] wan >swen< I Wan wen L M Wann Q Wen R  $\cdot$ sâhen] ye gesahen Q sagen Z \textbf{6} sô er dem] so er den G Dem er O So e dem R \textbf{7} gruozte] gruͤzt I (Z)  $\cdot$ der] den L  $\cdot$ guoter] [i*]: gvͦter O \textbf{8} unde] er I  $\cdot$ jach] sprach I L R  $\cdot$ sus] vnsz Q  $\cdot$ mîn] mir myn L (R) (Z) \textbf{9} alsus] Suͯs L \textbf{10} ein halden] einhalben G am halden Q \textbf{12} vor eines] Von eyner M Von eines Q  $\cdot$ velses] velsen I R velschis M \textbf{13} ûz] vor Z  $\cdot$ rehtem] \textit{om.} R \textbf{14} ir] Der Q R [Er]: Jr Z  $\cdot$ wâre] rehte O \textit{om.} R \textbf{15} der] ÷er O  $\cdot$ ir] [ein]: er R \textbf{17} ez brach] si hiez I Do [sprach]: prach O Do brach L Q Da brach M R Z  $\cdot$ vrô] div frawe O (M)  $\cdot$ Sigune] sigûne I (O) Sýgvne L sigúne Q Sygunne R \textbf{18} ir lange] si brach ir I Jr langen R \textbf{19} der] ir I M R Z \textbf{20} begunde] [begunge]: begunde G \textbf{21} Tschianatulander] tschinnatulander G shinadulander I * \textit{nachträglich korrigiert zu:} Tschionachtolander O Schýnatvlander L Eschinot lande M Scionotulander Q Shẏnotulander R Tschionatulander Z \textbf{22} tôt dâ] toten I L tot O (Q) (R)  $\cdot$ vander] wander M \textbf{24} schimpfes] schimphens M  $\cdot$ si] [in]: sẏ R \textbf{25} vröuden] frævde O (M) (Q) \textbf{26} gar] dar L \textbf{27} munt] [muͦt]: mund R \textbf{28} hie] \textit{om.} L  $\cdot$ jæmerlîchen] iemmelichen M \textbf{29} iwer schôze] iweren schozen G \textbf{30} rîter] ritten Z \newline
\end{minipage}
\hspace{0.5cm}
\begin{minipage}[t]{0.5\linewidth}
\small
\begin{center}*T (U)
\end{center}
\begin{tabular}{rl}
 & sus riten si ûf der slâ hin nâch.\\ 
 & dem knappen \textbf{vor in was} gâch.\\ 
 & \textbf{doch wiste} der unverzagete\\ 
 & niht, daz man in jagete.\\ 
5 & wan wen sîniu ougen sâhen,\\ 
 & sô er \textbf{in} begunde nâhen,\\ 
 & den gruozte der knappe guoter\\ 
 & und jach: "sus \textbf{riet mir} mîn muoter."\\ 
 & \textbf{\begin{large}A\end{large}lsus} kam unser \textbf{küener} knabe\\ 
10 & geriten ein halde abe.\\ 
 & wîbes stimme er \textbf{erhôrte}\\ 
 & vor eines \textbf{waldes} orte.\\ 
 & eine vrouwe ûz rehtem jâmer schrei.\\ 
 & \textbf{dô} was diu wâre vröude enzwei.\\ 
15 & der knappe reit ir balde zuo.\\ 
 & nû hœret, waz diu vrouwe tuo.\\ 
 & \textbf{d\textit{â}} brach vrouwe Sygune\\ 
 & ir langen zöpfe brûne\\ 
 & \textbf{von} jâmer ûz \textbf{der} swarten.\\ 
20 & der knappe begunde warten.\\ 
 & Schinohtudelander,\\ 
 & den vürsten, tôt d\textit{â} vander\\ 
 & der juncvrouwen in ir schôz.\\ 
 & \textbf{aller schimpf} si verdrôz.\\ 
25 & "er sî trûric oder vröuden var,\\ 
 & die \textbf{tet} \textbf{mîn} muoter grüezen gar.\\ 
 & got halt iuch", sprach des knappen munt,\\ 
 & "ich hân hie jæmerlîchen vunt\\ 
 & in \textbf{iuwern schœzen} vunden.\\ 
30 & wer gab iu den rîter wunden?"\\ 
\end{tabular}
\scriptsize
\line(1,0){75} \newline
U V W T \newline
\line(1,0){75} \newline
\textbf{1} \textit{Initiale} W   $\cdot$ \textit{Majuskel} T  \textbf{3} \textit{Majuskel} T  \textbf{9} \textit{Überschrift:} Hie kvnt parzifal zvͦm ersten male zvͦ sinre [nv́*]: nv́ftelen sigvnen V   $\cdot$ \textit{Großinitiale} T   $\cdot$ \textit{Initiale} U V  \textbf{10} \textit{Majuskel} T  \textbf{11} \textit{Majuskel} T  \textbf{13} \textit{Majuskel} T  \textbf{17} \textit{Majuskel} T  \textbf{20} \textit{Majuskel} T  \textbf{25} \textit{Majuskel} T  \newline
\line(1,0){75} \newline
\textbf{1} der] ir T \textbf{2} vor in was] vor in waz oͮch V (W) was ouch vor in T \textbf{3} doch wiste] Do en wesse T \textbf{4} \textit{nach 138.4:} Er hette gebaitet auff dem plan / Er muͤste in do bestanden han W   $\cdot$ niht] \textit{om.} W  $\cdot$ in] in yetzo W \textbf{5} wen] swen V T wenne W \textbf{6} in] [*]: dem V dem W T  $\cdot$ nâhen] gahen W \textbf{8} riet mir] hies mich W (T) \textbf{9} küener] toͤrscher V (W) (T) \textbf{10} ein halde] eine halden V (T) ein halp W \textbf{11} erhôrte] horte V T gehorte W \textbf{12} waldes] velses T \textbf{13} eine vrouwe ûz] Ein wip von T \textbf{14} Der froͤde was gar enzwei W  $\cdot$ dô] Der V ir T  $\cdot$ wâre] rehte T \textbf{15} balde] baldec T \textbf{16} nû hœret] Vnd sach W \textbf{17} Si hiez mit namen Sygv̂ne T  $\cdot$ dâ] Do U V W  $\cdot$ vrouwe] die frauwe W  $\cdot$ Sygune] Siguͦne U sigvne V (W) \textbf{18} langen] lange W \textbf{19} Zoch si v̂z ir swarten T  $\cdot$ von] vor V \textbf{21} Schinohtudelander] Schinote do lander U Schinoten [*]: de lalander V Schienot de lander W [*delander]: Schinohtvdelander T \textbf{22} dâ] do U V W \textbf{23} ir] irn U \textbf{24} \textit{nach 138.24:} Vmb irn vil schoͤnen lieben man / Den sy do toten muͦste han W   $\cdot$ Klage hette sy groß W  $\cdot$ aller schimpf] allez schimpfes T  $\cdot$ verdrôz] bedrôz T \textbf{26} tet] bat V bot W hiez T \textbf{27} halt iuch] hat úch W \textbf{29} iuwern schœzen] v́werre schossen V eúwern schlossen W îuwer schôz T \newline
\end{minipage}
\end{table}
\end{document}
