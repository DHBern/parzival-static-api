\documentclass[8pt,a4paper,notitlepage]{article}
\usepackage{fullpage}
\usepackage{ulem}
\usepackage{xltxtra}
\usepackage{datetime}
\renewcommand{\dateseparator}{.}
\dmyyyydate
\usepackage{fancyhdr}
\usepackage{ifthen}
\pagestyle{fancy}
\fancyhf{}
\renewcommand{\headrulewidth}{0pt}
\fancyfoot[L]{\ifthenelse{\value{page}=1}{\today, \currenttime{} Uhr}{}}
\begin{document}
\begin{table}[ht]
\begin{minipage}[t]{0.5\linewidth}
\small
\begin{center}*D
\end{center}
\begin{tabular}{rl}
\textbf{470} & \begin{large}E\end{large}z ist hiute der karvrîtac,\\ 
 & daz man vür wâr dâ warten mac:\\ 
 & ein tûbe von himel swinget.\\ 
 & ûf den stein \textbf{diu} bringet\\ 
5 & eine \textbf{kleine}, wîz oblât.\\ 
 & ûf dem steine si die lât.\\ 
 & diu tûbe ist durchliuhtec blanc.\\ 
 & ze himel tuot si widerwanc.\\ 
 & immer \textbf{alle} karvrîtage\\ 
10 & bringet si ûf den stein, als ich \textbf{iu} sage,\\ 
 & dâ von der stein enpfæhet,\\ 
 & swaz guotes ûf erden dræhet\\ 
 & von trinken unt von spîse,\\ 
 & als den wunsch von pardîse:\\ 
15 & ich meine, swaz diu erde \textbf{mac} gebern.\\ 
 & der stein si vürbaz mêr sol wern:\\ 
 & swaz wildes \textbf{underem lufte} lebt,\\ 
 & ez vliege \textbf{ode} loufe \textbf{unt daz} swebt.\\ 
 & der rîterlîchen bruoderschaft,\\ 
20 & die pfrüende \textbf{in} gît des Grâles kraft.\\ 
 & die aber zem Grâle \textbf{sint} benant,\\ 
 & hœret, wie die werdent bekant:\\ 
 & zende an des \textbf{steines} drum\\ 
 & von karacten ein epitafjum\\ 
25 & sagt sînen namen unt sînen art,\\ 
 & swer dar \textbf{tuon sol} \textbf{die} sælden vart,\\ 
 & ez sî von meiden ode von knaben.\\ 
 & die schrift \textbf{darf} niemen danne schaben.\\ 
 & sô man \textbf{den} namen gelesen hât,\\ 
30 & vor ir ougen si zergât.\\ 
\end{tabular}
\scriptsize
\line(1,0){75} \newline
D \newline
\line(1,0){75} \newline
\textbf{1} \textit{Initiale} D  \newline
\line(1,0){75} \newline
\newline
\end{minipage}
\hspace{0.5cm}
\begin{minipage}[t]{0.5\linewidth}
\small
\begin{center}*m
\end{center}
\begin{tabular}{rl}
 & ez ist hiut der karvrîtac,\\ 
 & daz man vür wâr d\textit{â} warten mac:\\ 
 & \begin{large}E\end{large}in tûbe von himel swinget.\\ 
 & ûf den stein \textbf{diu} bringet\\ 
5 & ein \textbf{klein}, wîz oblât.\\ 
 & ûf dem stein si die lât.\\ 
 & diu tûbe ist durchliuhtec blanc.\\ 
 & zuo himel tuot si widerwanc.\\ 
 & iemer \textbf{am} karvrîtage\\ 
10 & bringet si ûf den stein, als ich sage,\\ 
 & dâ von der stein enpfæhet,\\ 
 & waz guotes ûf erden dræhet\\ 
 & von trinken und von spîse,\\ 
 & als den wunsch von paradîse:\\ 
15 & ich mein, waz diu erde \textbf{müge} gebern.\\ 
 & der stein si vürbaz mêr sol wern:\\ 
 & waz wilde\textit{s} \textbf{\textit{u}nder dem lufte} lebet,\\ 
 & ez vliege, loufe \textbf{oder} swebet.\\ 
 & der ritterlîche\textit{n} bruoderschaft\\ 
20 & die pfrüende gît des Grâles kraft.\\ 
 & die aber zem Grâl \textbf{sint} benant,\\ 
 & hœrt, wie die werdent bekant:\\ 
 & zuo ende an des \textbf{steines} drum\\ 
 & von karac\textit{t}en ein ep\textit{i}tafum\\ 
25 & saget sînen namen und sînen art,\\ 
 & wer dar \textbf{tuon sol} \textbf{die} sælden \textit{v}art,\\ 
 & ez sî von megden oder von knaben.\\ 
 & die schrift \textbf{darf} niemen da\textit{nne}n schaben.\\ 
 & sô man \textbf{den} namen gelesen hât,\\ 
30 & vor ir ougen si zergât.\\ 
\end{tabular}
\scriptsize
\line(1,0){75} \newline
m n o \newline
\line(1,0){75} \newline
\textbf{3} \textit{Initiale} m  \newline
\line(1,0){75} \newline
\textbf{2} dâ] do m n o \textbf{9} am] [al]: am m \textbf{10} sage] úch sage n \textbf{14} den] dem o \textbf{15} müge] mag n o \textbf{17} wildes under] wildes vff vnd vnder m \textbf{18} vliege] fliehe o \textbf{19} ritterlîchen] ritterlicher m \textbf{24} karacten] karachen m karathen n karathan o  $\cdot$ epitafum] epẏſtafvͦm m epẏstasúm n epistafum o \textbf{25} sînen art] sin art n \textbf{26} sælden] selben n  $\cdot$ vart] art m \textbf{28} schrift] geschrifft n  $\cdot$ dannen] dar an m \textbf{29} hât] [han]: hat o \newline
\end{minipage}
\end{table}
\newpage
\begin{table}[ht]
\begin{minipage}[t]{0.5\linewidth}
\small
\begin{center}*G
\end{center}
\begin{tabular}{rl}
 & \begin{large}E\end{large}z ist hiute der karvrîtac,\\ 
 & daz man vür wâr dâ warten mac:\\ 
 & ein tûbe von himel swinget.\\ 
 & ûf den stein \textbf{si} bringet\\ 
5 & ein \textbf{klein}, wîz oblât.\\ 
 & ûf dem stein si die lât.\\ 
 & di\textit{u} tûbe ist durchliuhtic blanc.\\ 
 & ze himel tuot si widerwanc.\\ 
 & imer \textbf{an dem} karvrîtage\\ 
10 & bringet si ûf den \textit{stein}, als ich \textbf{iu} sage,\\ 
 & dâ von der stein enpfæhet,\\ 
 & swaz guotes ûf erden dræhet\\ 
 & von trinken unde von spîse,\\ 
 & als den wunsch von pardîse:\\ 
15 & ich mein, swaz diu erde \textbf{mac} gebern.\\ 
 & der stein si vürbaz mêr sol wern:\\ 
 & swaz wildes \textbf{underm lufte} lebet,\\ 
 & ez vliege \textbf{ode} loufe \textbf{unde daz} swebet.\\ 
 & der rîterlîchen bruoderschaft,\\ 
20 & \textit{die pfrüende \textbf{in} gît des Grâles kraft.}\\ 
 & die aber ze dem Grâl \textbf{sint} benant,\\ 
 & hœrt, wie die werdent bekant:\\ 
 & zende an des \textbf{steines} drum\\ 
 & von karacten ein epitafjum\\ 
25 & saget sînen namen unde sînen art,\\ 
 & swer dar \textbf{sol tuon} \textbf{der} sælden vart,\\ 
 & ez sî von meiden ode von knaben.\\ 
 & die schrift \textbf{darf} niemen dannen schaben.\\ 
 & sô man \textbf{den} namen gelesen hât,\\ 
30 & vor ir ougen si zergât.\\ 
\end{tabular}
\scriptsize
\line(1,0){75} \newline
G I O L M Z Fr18 \newline
\line(1,0){75} \newline
\textbf{1} \textit{Initiale} G I O L M Z Fr18  \textbf{15} \textit{Initiale} I  \newline
\line(1,0){75} \newline
\textbf{1} Ez] ÷z O \textbf{3} swinget] singet M \textbf{4} den] dem M  $\cdot$ si] div O (M) (Z) Fr18 \textbf{5} ein klein wîz] eine chlein wîz I Eine chleine wize O (L) (M) (Z) (Fr18)  $\cdot$ oblât] oblaten M \textbf{6} dem] den I  $\cdot$ lât] lac M \textbf{7} diu] die G  $\cdot$ durchliuhtic] durc luhtet I \textbf{8} widerwanc] wider swanc Z \textbf{9} an dem] an den I alle O L M Z Fr18  $\cdot$ karvrîtage] gute fritage M \textbf{10} stein] \textit{om.} G \textbf{11} \textit{Vers 470.11 fehlt} I  \textbf{12} swaz] Waz L (M)  $\cdot$ ûf erden] vf der erde I (O) (L) (Fr18) uff der erdin M  $\cdot$ dræhet] trahet I \textbf{14} als] \textit{om.} I Al L \textbf{15} swaz] waz L (M) Z  $\cdot$ gebern] [gegeben]: gegebern L \textbf{16} si] sin L \textbf{17} swaz] Waz L (M)  $\cdot$ wildes] \textit{om.} L  $\cdot$ underm lufte] vnder den luften I inder luffte M  $\cdot$ lebet] mac leben I wildes lept L \textbf{18} ez vliege luft oder sweben I  $\cdot$ ode] ez M  $\cdot$ unde] oder O L (M) (Fr18) \textbf{20} \textit{Vers 470.20 fehlt} G  \textbf{21} benant] genant Z \textbf{22} bekant] erchant O (L) (M) (Fr18) \textbf{24} von] mit I  $\cdot$ epitafjum] epytafrvm Z \textbf{25} sînen art] sin art I sine art L art M \textbf{26} swer] Wer L M  $\cdot$ sol tuon der] tvͦn sol di O (M) (Z) (Fr18) \textbf{28} darf] mac I  $\cdot$ dannen] drab I ab O  $\cdot$ schaben] geshaben I \newline
\end{minipage}
\hspace{0.5cm}
\begin{minipage}[t]{0.5\linewidth}
\small
\begin{center}*T
\end{center}
\begin{tabular}{rl}
 & \begin{large}E\end{large}z ist hiute der karvrîtac,\\ 
 & daz man vür wâr dâ warten mac:\\ 
 & ein tûbe von himele swinget.\\ 
 & ûf den stein \textbf{diu} bringet\\ 
5 & eine wîze oblât.\\ 
 & ûf dem steine si die lât.\\ 
 & diu tûbe ist durchliuhtic blanc.\\ 
 & ze himele tuot si widerwanc.\\ 
 & iemer \textbf{alle} karvrîtage\\ 
10 & bringet si ûf den \textit{stein}, als ich \textbf{iu} sage,\\ 
 & dâ von der stein enpfæhet,\\ 
 & swaz guotes ûf \textbf{der} erden dræhet\\ 
 & von trinkenne unde von spîse,\\ 
 & alse den wunsch von paradîse:\\ 
15 & ich meine, swaz diu erde \textbf{mac} gebern.\\ 
 & der stein si vürbaz mê sol wern:\\ 
 & swaz wildes \textbf{undern lüften} \textit{l}ebet,\\ 
 & ez vliege \textbf{oder} \textbf{ez} loufe \textbf{oder} \textit{swebet}.\\ 
 & der rîterlîchen bruoderschaft\\ 
20 & die p\textit{f}rü\textit{e}nde gît des Grâles kraft.\\ 
 & die aber zem Grâle \textbf{sîn} benant,\\ 
 & hœret, wie die werdent bekant:\\ 
 & zende an des \textbf{Grâles} drum\\ 
 & von karactern ein epitafjum\\ 
25 & seit sînen namen unde sînen art,\\ 
 & swer dar \textbf{tuon sol} \textbf{der} sælden vart,\\ 
 & ez sî von megden oder von knaben.\\ 
 & die schrift \textbf{sol} niemen dannen schaben.\\ 
 & sô man \textbf{die} namen gelesen hât,\\ 
30 & vor ir ougen si zergât.\\ 
\end{tabular}
\scriptsize
\line(1,0){75} \newline
T U V W Q R Fr42 \newline
\line(1,0){75} \newline
\textbf{1} \textit{Initiale} T W Q Fr42  \newline
\line(1,0){75} \newline
\textbf{1} \textit{Die Verse 453.1-502.30 fehlen} U  \textbf{2} daz] Des W R  $\cdot$ dâ] do V Q \textit{om.} W \textbf{3} tûbe] [toͮ]: tvbe Fr42 \textbf{4} diu] sy R \textbf{5} eine] eine cleine V ein kleine W Q (R) (Fr42)  $\cdot$ wîze] wiz Fr42 \textbf{6} dem] den Q R  $\cdot$ die] div Fr42 \textbf{7} durchliuhtic] durch luchtet Q \textbf{8} widerwanc] wider swang V wider yerren gank R \textbf{9} iemer alle] [J*]: Jemer amme V  $\cdot$ karvrîtage] [v]: kar:rîetage Fr42 \textbf{10} bringet si] Bringet sv́ sv́ V [bringez]: bringensz Q Bringencz R  $\cdot$ stein] \textit{om.} T  $\cdot$ iu] \textit{om.} W \textbf{12} swaz] Was W Q R  $\cdot$ ûf der erden] auff der erde W (R) \textbf{13} unde] \textit{om.} Fr42 \textbf{14} von] vom R \textbf{15} meine] neme Q  $\cdot$ swaz] was W Q (R)  $\cdot$ erde] erden Q  $\cdot$ gebern] gegebern R \textbf{16} vürbaz] [bas]: furbaz Q  $\cdot$ sol] musz Q \textbf{17} swaz] Was W Q R  $\cdot$ wildes] [wid*]: wilden R  $\cdot$ undern lüften] vnderm luffte W Q (Fr42)  $\cdot$ lebet] swebet T lebe W \textbf{18} oder ez loufe oder swebet] oder ez lovfe oder T ez loͮffe oder ez swebet V (R) (Fr42) es lauffe oder es schwe be W es [floufe]: loufe oder das swebt Q \textbf{19} bruoderschaft] [brvͦderchaft]: brvͦderschaft T \textbf{20} \textit{Vers 470.20 fehlt} R   $\cdot$ pfrüende] prvevende T  $\cdot$ gît] in geit W (Q) in ::: Fr42 \textbf{22} bekant] erkant Q R \textbf{23} an] als Q  $\cdot$ Grâles] steines V W Q R (Fr42) \textbf{24} epitafjum] Epytavivn T \textbf{25} sînen art] seine art W Q sin art R \textbf{26} swer] Wer W Q R  $\cdot$ tuon] kumen R  $\cdot$ sol] so Q  $\cdot$ vart] [*]: fart R \textbf{28} schrift sol] [*]: schrift darf V schrifft darff W (Q) geschriff darff R  $\cdot$ dannen] \textit{om.} R \textbf{29} die] [d*]: den V W Q R \textbf{30} zergât] vergat R \newline
\end{minipage}
\end{table}
\end{document}
