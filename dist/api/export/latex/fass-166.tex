\documentclass[8pt,a4paper,notitlepage]{article}
\usepackage{fullpage}
\usepackage{ulem}
\usepackage{xltxtra}
\usepackage{datetime}
\renewcommand{\dateseparator}{.}
\dmyyyydate
\usepackage{fancyhdr}
\usepackage{ifthen}
\pagestyle{fancy}
\fancyhf{}
\renewcommand{\headrulewidth}{0pt}
\fancyfoot[L]{\ifthenelse{\value{page}=1}{\today, \currenttime{} Uhr}{}}
\begin{document}
\begin{table}[ht]
\begin{minipage}[t]{0.5\linewidth}
\small
\begin{center}*D
\end{center}
\begin{tabular}{rl}
\textbf{166} & dô bat in vlîzeclîche\\ 
 & Gurnemanz, der \textbf{triwen} rîche,\\ 
 & daz er \textbf{vaste} æze\\ 
 & unt \textbf{der} müede \textbf{sîn} vergæze.\\ 
5 & \begin{large}M\end{large}an huop den tisch, dô des \textbf{wart} zît.\\ 
 & "ich wæne, daz ir müede sît",\\ 
 & \textbf{sus} sprach der wirt. "\textbf{wæret} ir iht vruo?"\\ 
 & "got weiz, mîn muoter \textbf{slief} \textbf{duo}.\\ 
 & \textbf{diu} kan sô vil niht wachen."\\ 
10 & der wirt begunde lachen.\\ 
 & er vuorten an die slâfstat.\\ 
 & \textbf{der wirt} in \textbf{sich} ûz sloufen bat.\\ 
 & ungern erz tet, doch muost ez sîn.\\ 
 & ein declachen härmîn\\ 
15 & wart geleit über \textbf{sînen blôzen} lîp.\\ 
 & sô werde vruht gebar nie wîp.\\ 
 & Grôz müede \textbf{und} slâf in lêrte,\\ 
 & daz er sich \textbf{selten} kêrte\\ 
 & an die anderen sîten.\\ 
20 & sus kunder tages \textbf{erbîten}.\\ 
 & Dô gebôt der vürste mære,\\ 
 & daz ein bat bereite wære\\ 
 & \textbf{reht umbe den mitten morgens} tac\\ 
 & ze ende an \textbf{den teppich}, \textbf{der dâ} lac.\\ 
25 & daz muose des morgens alsô sîn.\\ 
 & man warf \textbf{dâ} rôsen oben \textbf{în}.\\ 
 & swie wênic \textbf{man umb in} dâ rief,\\ 
 & der gast erwachte, \textbf{der} dâ slief.\\ 
 & der junge, werde, süeze man\\ 
30 & gienc sitzen in die \textit{kuofen sân}.\\ 
\end{tabular}
\scriptsize
\line(1,0){75} \newline
D \newline
\line(1,0){75} \newline
\textbf{5} \textit{Initiale} D  \textbf{17} \textit{Majuskel} D  \textbf{21} \textit{Majuskel} D  \newline
\line(1,0){75} \newline
\textbf{30} kuofen sân] \textit{om.} D \newline
\end{minipage}
\hspace{0.5cm}
\begin{minipage}[t]{0.5\linewidth}
\small
\begin{center}*m
\end{center}
\begin{tabular}{rl}
 & dô bat in vlîzeclîche\\ 
 & Gurnemanz, der rîche,\\ 
 & daz er \textbf{vaster} æze\\ 
 & und \textbf{sîner} müede vergæze.\\ 
5 & \begin{large}M\end{large}an huop den tisch, dô des \textbf{wart} zît.\\ 
 & "ich wæne, daz ir müede sît",\\ 
 & sprach der wirt. "\textbf{wârt} ir iht vruo?"\\ 
 & "go\textit{t} weiz, mîn muoter \textbf{sl\textit{ie}fe} \textbf{nû}.\\ 
 & \textbf{diu} kan sô vil niht wachen."\\ 
10 & der wirt begunde lachen.\\ 
 & er vuorte in an die slâfstat.\\ 
 & \textbf{der \textit{wirt}} in \textbf{sich} ûz slo\textit{uf}en bat.\\ 
 & ungerne er daz tet, doch muos ez sîn.\\ 
 & ein declachen hermîn\\ 
15 & wart geleit über \textbf{des tumben} \textit{l}îp.\\ 
 & sô werde vruht g\textit{e}b\textit{a}r nie wîp.\\ 
 & grôz müede slâf in lêrte,\\ 
 & daz er sich \textbf{selten} kêrte\\ 
 & an die anderen sîten.\\ 
20 & sus kunde er tages \textbf{erbîten}.\\ 
 & dô gebôt der vürste mære,\\ 
 & daz ein bat bereit wære\\ 
 & \textbf{des morgens umb den mitte\textit{n}} tac\\ 
 & zuo ende an \textbf{dem teppich}, \textbf{der d\textit{â}} lac.\\ 
25 & daz muos des morgens alsô sîn.\\ 
 & \textit{m}an warf \textbf{dâ} rôsen oben \textbf{în}.\\ 
 & wie wênic \textbf{man umb in} dâ rief,\\ 
 & der gast erwachete, \textbf{der} d\textit{â} slief.\\ 
 & der junge, werde, süeze man\\ 
30 & gienc sitzen in die ku\textit{of}en sân.\\ 
\end{tabular}
\scriptsize
\line(1,0){75} \newline
m n o Fr69 \newline
\line(1,0){75} \newline
\textbf{5} \textit{Initiale} m o Fr69   $\cdot$ \textit{Capitulumzeichen} n  \newline
\line(1,0){75} \newline
\textbf{1} bat] hat o \textbf{2} Gurnemanz] Gurnemancz m o Gurmantz n \textbf{5} des] das n o \textit{om.} Fr69 \textbf{6} wæne] wende o \textbf{8} got] Go m  $\cdot$ sliefe] sleiffe m slieff n o \textbf{9} diu kan] Dar kam o \textbf{11} vuorte] fúrt o \textbf{12} wirt] \textit{om.} m  $\cdot$ sloufen] sloͯchen m sloffen n (o) \textbf{13} er daz] er es n o (Fr69) \textbf{15} des] das o  $\cdot$ lîp] wip m \textbf{16} gebar] gab er m \textbf{17} slâf in] in sloffen n an slaffen o \textbf{19} anderen] ander n o \textbf{20} erbîten] erbeiten o \textbf{23} umb] úff o  $\cdot$ mitten] mittem m \textbf{24} der] er o  $\cdot$ dâ] do m n o \textbf{26} man] Wan m \textbf{27} dâ] do n o \textbf{28} erwachete] der erwachete n der wachte o  $\cdot$ dâ] do m n o \textbf{30} kuofen] kussen m schiff n schuͦffe o \newline
\end{minipage}
\end{table}
\newpage
\begin{table}[ht]
\begin{minipage}[t]{0.5\linewidth}
\small
\begin{center}*G
\end{center}
\begin{tabular}{rl}
 & dô bat in vlîziclîche\\ 
 & Gurnomanz, der \textbf{triwen} rîche,\\ 
 & daz er \textbf{vaste} æze\\ 
 & unt \textbf{der} müede \textbf{sîn} vergæze.\\ 
5 & man huop den tisch, dô des \textbf{was} zît.\\ 
 & "ich wæne, daz ir müede sît",\\ 
 & sprach der wirt. "\textbf{wæret} ir iht vruo?"\\ 
 & "got weiz, mîn muoter \textbf{sliefe} \textbf{nû}.\\ 
 & \textbf{si} kan sô vil niht wachen."\\ 
10 & der wirt begunde lachen.\\ 
 & er vuort in an die slâfstat.\\ 
 & \textbf{der wirt} in \textbf{sich} ûz sloufen bat.\\ 
 & ungernerz tet, doch muosez sîn.\\ 
 & ein declachen hermîn\\ 
15 & wart geleit über \textbf{sînen blôzen} lîp.\\ 
 & sô werde vruht gebar nie wîp.\\ 
 & grôz müede \textbf{unde} slâf in lêrte,\\ 
 & daz er sich \textbf{selten} kêrte\\ 
 & \textbf{umbe} an die anderen sîten.\\ 
20 & sus kunder tages \textbf{bîten}.\\ 
 & dô gebôt der vürste mære,\\ 
 & daz ein bat bereit wære\\ 
 & \textbf{reht umbe den mitten morgens} tac\\ 
 & zende an \textbf{dem tepche}, \textbf{dâ er dâ} lac.\\ 
25 & daz muose des morgens alsô sîn.\\ 
 & man warf rôsen oben \textbf{drîn}.\\ 
 & swie wênic \textbf{umbe in man} dâ rief,\\ 
 & der gast erwachte, \textbf{der} dâ slief.\\ 
 & der junge, werde, süeze man\\ 
30 & gienc sitzen in die kuofen sân.\\ 
\end{tabular}
\scriptsize
\line(1,0){75} \newline
G I O L M Q R Z Fr17 \newline
\line(1,0){75} \newline
\textbf{5} \textit{Initiale} I O R Z Fr17  \textbf{7} \textit{Initiale} L  \textbf{17} \textit{Initiale} M  \newline
\line(1,0){75} \newline
\textbf{1} dô] Da M Z \textbf{2} Gurnomanz] curnomanz G Gurnemanz I (O) M Gvrnomantz L (Q) Gurnamancz R Gvrnemantz Z Gvrnamanz Fr17  $\cdot$ triwen rîche] triwen [rechen]: reichen Q Riche R \textbf{3} vaste] waste Fr17 \textbf{4} der müede sîn] siner muͯde L \textbf{5} man] ÷an O  $\cdot$ dô des] des Z do do d Fr17 \textbf{7} sprach der wirt] DEr wirt sprach L  $\cdot$ wæret] wart I O Fr17 vart Q  $\cdot$ ir] \textit{om.} Z  $\cdot$ iht] ich R \textbf{8} sliefe] slaffet I (Fr17) \textbf{9} si] sin I Die L \textbf{10} lachen] zu lachen Q \textbf{12} sich ûz sloufen] sich ab ziehen O vsz schloffen R \textbf{13} ungernerz] swie vngern erz I Vngerne er daz L  $\cdot$ doch] do L \textbf{15} blôzen] \textit{om.} I Fr17 \textbf{16} vruht] vvrht Fr17  $\cdot$ gebar nie] gewan nye eyn Q \textbf{17} grôz] [÷Ro*]: ÷Roz M  $\cdot$ unde slâf in] Jn schlaffen R \textbf{19} umbe] vbe Fr17 \textbf{20} tages] des tages I \textbf{21} dô] Da M Z \textbf{22} bat bereit] bat gereite M ba reitet Z \textbf{23} umbe den] an dem I vmb des Q  $\cdot$ mitten morgens] mittem morgens I mitten morgen L (M) morges mitten Q mitten morges zit R \textbf{24} tepche] bette I  $\cdot$ dâ er dâ] da er I O L R do er Q \textbf{25} daz] Das des R \textbf{26} rôsen] der rosen L R Z da rosin M  $\cdot$ drîn] in L (M) (R) Z \textbf{27} swie] wie I (L) (M) (Q) (R)  $\cdot$ umbe in man dâ] man vmb in da I (O) (L) (R) man da vmme yn M vmb in do Q man vmb in Z \textbf{28} erwachte] erwachet L (R)  $\cdot$ der dâ] da er I R der do O (Q) \textbf{29} junge werde] werde junge M werde R \textbf{30} in die kuofen] indaz bade O \newline
\end{minipage}
\hspace{0.5cm}
\begin{minipage}[t]{0.5\linewidth}
\small
\begin{center}*T
\end{center}
\begin{tabular}{rl}
 & Dô bat in vlîzec\textit{lîc}he\\ 
 & Gurnemanz, der \textbf{triuwen} rîche,\\ 
 & daz er \textbf{vaste} æze\\ 
 & unde \textbf{der} müede \textbf{sîn} vergæze.\\ 
5 & man huop den tisch, dô des \textbf{wart} zît.\\ 
 & "Ich wæne, daz ir müede sît",\\ 
 & sprach der wirt. "\textbf{wârt} ir \textbf{ûf} iht vruo?"\\ 
 & "Goteweiz, mîn muoter \textbf{sliefe} \textbf{nuo}.\\ 
 & \textbf{si} kan sô vil niht wachen."\\ 
10 & der wirt begunde lachen.\\ 
 & er vuortin an die slâfstat,\\ 
 & \textbf{dâ er} in ûz sloufen bat.\\ 
 & ungernerz tet, doch muosez sîn.\\ 
 & ein declachen härmîn\\ 
15 & wart geleit über \textbf{sînen blôzen} lîp.\\ 
 & sô werde vruht gebar nie wîp.\\ 
 & grôz müede \textbf{unde} slâf in lêrte,\\ 
 & daz er sich \textbf{wênic} kêrte\\ 
 & \textbf{umbe} an die anderen sîten.\\ 
20 & su\textit{s} kunder \textbf{des} tages \textbf{bîten}.\\ 
 & \begin{large}D\end{large}ô gebôt der vürste mære,\\ 
 & daz ein bat bereit wære\\ 
 & \textbf{rehte umbe den mitten morgens} tac\\ 
 & zende an \textbf{dem teppich}, \textbf{dâ er} lac.\\ 
25 & daz muose des morgens alsô sîn.\\ 
 & man warf \textbf{im} rôsen oben \textbf{drîn}.\\ 
 & swie wênic \textbf{man umbe in} dâ rief,\\ 
 & der gast erwachete, \textbf{dâr} dâ slief.\\ 
 & der junge, werde, süeze man\\ 
30 & gie sitzen in die kuofen sân.\\ 
\end{tabular}
\scriptsize
\line(1,0){75} \newline
T U V W \newline
\line(1,0){75} \newline
\textbf{1} \textit{Majuskel} T  \textbf{6} \textit{Majuskel} T  \textbf{8} \textit{Majuskel} T  \textbf{21} \textit{Initiale} T U V W  \newline
\line(1,0){75} \newline
\textbf{1} vlîzeclîche] vlizeche T \textbf{2} Gurnemanz] Guͦrnemanz U Gurnemantz W  $\cdot$ triuwen rîche] erentreiche W \textbf{4} vergæze] vergaz U \textbf{5} des] iz U \textbf{7} \textit{nach 166.7:} Ia ich herre ich enwaiß wo zuo / Ich sag eúch das fúr warhait / Daz sich mein muoter hat slafen gelait W   $\cdot$ wârt ir ûf iht vruo] wern ir icht fro W \textbf{8} \textit{Vers 166.8 fehlt} W  \textbf{10} lachen] des lachen W \textbf{12} dâ] Do U (V) (W) \textbf{17} lêrte] das lerte W \textbf{18} wênic] selten W \textbf{20} sus] sv T  $\cdot$ des] \textit{om.} W \textbf{21} gebôt] bot U \textbf{22} bereit] bereitet V \textbf{23} Vmbe des morgens tag W  $\cdot$ morgens] morgen U \textbf{24} dâ er] do er do U do er V W \textbf{26} im] do W  $\cdot$ oben drîn] obenin V obnan ein W \textbf{27} swie] Wie U W  $\cdot$ dâ] do U V W \textbf{28} dâr dâ] da er do U do er V do er do W \textbf{30} die kuofen] daz bat U V die batstanben W \newline
\end{minipage}
\end{table}
\end{document}
