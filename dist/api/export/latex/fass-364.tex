\documentclass[8pt,a4paper,notitlepage]{article}
\usepackage{fullpage}
\usepackage{ulem}
\usepackage{xltxtra}
\usepackage{datetime}
\renewcommand{\dateseparator}{.}
\dmyyyydate
\usepackage{fancyhdr}
\usepackage{ifthen}
\pagestyle{fancy}
\fancyhf{}
\renewcommand{\headrulewidth}{0pt}
\fancyfoot[L]{\ifthenelse{\value{page}=1}{\today, \currenttime{} Uhr}{}}
\begin{document}
\begin{table}[ht]
\begin{minipage}[t]{0.5\linewidth}
\small
\begin{center}*D
\end{center}
\begin{tabular}{rl}
\textbf{364} & \textbf{\begin{large}K\end{large}ünnet} ir danne ritters vuore spehen,\\ 
 & ir müezet im \textbf{rehter} dinge jehen.\\ 
 & sîn lîp \textbf{gein valsche nie wart} balt.\\ 
 & swer im dâr über \textbf{tuot} gewalt,\\ 
5 & \textbf{wærez} mîn vater oder \textbf{mîn} kint,\\ 
 & alle, die gein im in \textbf{zorne} sint,\\ 
 & mîne mage oder mîn bruoder,\\ 
 & die m\textit{üe}sen diu strîtes ruoder\\ 
 & gein mir ziehen. ich wil in \textbf{wern},\\ 
10 & vor \textbf{unrehten strîten} \textbf{nern},\\ 
 & swâ ich, hêrre, vor iwern hulden mac.\\ 
 & ûz schildes ambet in einen sac\\ 
 & wolt ich mich ê \textbf{ziehen},\\ 
 & sô verre ûz arde vliehen,\\ 
15 & dâ mich niemen erkande,\\ 
 & ê daz ir iwer schande,\\ 
 & hêrre, an im begienget.\\ 
 & \textbf{guotlîche} ir enpfienget\\ 
 & \textbf{billîcher} \textbf{alle}, die her sint komen\\ 
20 & unt iwern kumber hânt vernomen,\\ 
 & denne daz ir si wellet rouben.\\ 
 & des sult ir \textbf{iuch} gelouben."\\ 
 & Der vürste sprach: "\textbf{nû} lâz mich \textbf{in} \textbf{gesehen},\\ 
 & dâ mac niht arges \textbf{ûz} geschehen."\\ 
25 & er \textbf{reit}, dâ er Gawanen sach.\\ 
 & zwei ougen unt ein herze jach,\\ 
 & di\textit{u} Lyppaut mit \textbf{im} brâhte dar,\\ 
 & daz der gast wære wol gevar\\ 
 & unt rehte manlîche site\\ 
30 & sînen gebærden \textbf{wonten} mite.\\ 
\end{tabular}
\scriptsize
\line(1,0){75} \newline
D Fr3 Fr4 \newline
\line(1,0){75} \newline
\textbf{1} \textit{Initiale} D  \textbf{23} \textit{Initiale} Fr3 Fr4   $\cdot$ \textit{Majuskel} D  \newline
\line(1,0){75} \newline
\textbf{1} Künnet ir] Kunder Fr3  $\cdot$ vuore] svn Fr3 \textbf{3} gein] zv Fr3 (Fr4) \textbf{8} müesen] mvͦsen D (Fr3) (Fr4)  $\cdot$ diu] des Fr3 \textbf{10} vor] Vnd vor Fr3 (Fr4) \textbf{11} hêrre] \textit{om.} Fr3 \textbf{14} sô] Vnd so Fr3 \textbf{16} daz ir] dan er Fr3 dan ir Fr4 \textbf{19} alle] als Fr3 \textbf{23} sprach] \textit{om.} Fr4  $\cdot$ nû] \textit{om.} Fr3  $\cdot$ gesehen] sehen Fr3 \textbf{24} mac] ne mac Fr4 \textbf{25} Gawanen] gawan Fr3 gawanin Fr4 \textbf{26} ein] \textit{om.} Fr4 \textbf{27} diu] di D  $\cdot$ Lyppaut] Lyppaot D lippaot Fr3 lippaoth Fr4  $\cdot$ mit im] \textit{om.} Fr3 \textbf{30} gebærden wonten] geberen wonete Fr3 \newline
\end{minipage}
\hspace{0.5cm}
\begin{minipage}[t]{0.5\linewidth}
\small
\begin{center}*m
\end{center}
\begin{tabular}{rl}
 & \textbf{künnet} ir denne ritters vuore spehen,\\ 
 & ir müezet im \textbf{rehter} dinge jehen.\\ 
 & sîn lîp \textbf{gegen valsche nie wart} balt.\\ 
 & \textit{we}r ime dâr über \textbf{tuot} gewalt,\\ 
5 & \textbf{wær ez} mîn vater oder \textbf{mîn} kint,\\ 
 & alle, die gegen ime in \textbf{zorne} sint,\\ 
 & mîne mage oder mîn bruoder,\\ 
 & die müesen diu strîtes ruoder\\ 
 & gegen mir ziehen. ich wil in \textbf{wern},\\ 
10 & vor \textbf{unrehten strîten} \textbf{nern},\\ 
 & wâ ich, hêrre, vor iuwern hulden mac.\\ 
 & \dag und\dag  schiltes ambet in einen sac\\ 
 & wol\textit{t} ich mich ê \textbf{geziehen},\\ 
 & sô verre ûz arde vliehen,\\ 
15 & d\textit{â} mich niemen erkande,\\ 
 & ê daz ir iuwere schande,\\ 
 & hêrre, an im begienget.\\ 
 & \textbf{guotlîch} i\textit{r} enpfienget\\ 
 & \textbf{billîcher} \textbf{alle}, die her sint komen\\ 
20 & und iuwern kumber hânt vernomen,\\ 
 & danne daz ir si wellet rouben.\\ 
 & des sult ir \textbf{mir} gelouben."\\ 
 & der vürste sprach: "lâz mich \textbf{in} \textbf{sehen},\\ 
 & dâ mac niht arges \textbf{von} geschehen."\\ 
25 & er \textbf{reit}, d\textit{â} er Gawanen sach.\\ 
 & zwei ougen und ein herze jach,\\ 
 & diu Lippo\textit{u}t mit \textbf{ime} brâhte dar,\\ 
 & daz der gast wær wol gevar\\ 
 & und rehte manlîche site\\ 
30 & sînen gebærden \textbf{wonten} mite.\\ 
\end{tabular}
\scriptsize
\line(1,0){75} \newline
m n o \newline
\line(1,0){75} \newline
\newline
\line(1,0){75} \newline
\textbf{1} spehen] [se*]: sehen o \textbf{2} im] jme denne n \textbf{4} wer] [*]: Vor m \textbf{8} die] Sú n (o)  $\cdot$ diu] \textit{om.} n o \textbf{10} unrehten strîten] vnrechter strite o \textbf{13} wolt] Wol m \textbf{15} dâ] Do m n o \textbf{17} begienget] begingen n \textbf{18} ir] ich m  $\cdot$ enpfienget] enpfingen n \textbf{19} alle die] alde o \textbf{21} si] sú sú n  $\cdot$ wellet] wellen o \textbf{24} von geschehen] vs beschehen n (o) \textbf{25} dâ] do m n o  $\cdot$ Gawanen] gawannen o \textbf{26} jach] gach m \textbf{27} Lippout] lippoat m lippaot n lipaot o \textbf{30} wonten] wonte n o \newline
\end{minipage}
\end{table}
\newpage
\begin{table}[ht]
\begin{minipage}[t]{0.5\linewidth}
\small
\begin{center}*G
\end{center}
\begin{tabular}{rl}
 & \textbf{künnet} ir \textit{danne} rîters vuore spehen,\\ 
 & ir müezt im \textbf{rehter} dinge jehen.\\ 
 & sîn lîp \textbf{wart nie gein valsche} balt.\\ 
 & swer im dâr über \textbf{tæte} gewalt,\\ 
5 & \textbf{ez wære} mîn vater oder \textbf{mîniu} kint,\\ 
 & alle, die gein im in \textit{\textbf{zorn}} sint,\\ 
 & mîne mage ode mîne bruoder,\\ 
 & die m\textit{üe}sen diu strîtes ruoder\\ 
 & gein mir ziehen. ich wil in \textbf{\textit{w}eren},\\ 
10 & vo\textit{r} \textbf{unrehten strîten} \textbf{\textit{n}eren},\\ 
 & swâ ich, hêrre, vor iweren hulden mac.\\ 
 & ûz schiltes ambet in einen sac\\ 
 & wolt ich mich ê \textbf{ziehen},\\ 
 & sô verre ûz arde vliehen,\\ 
15 & dâ mich niemen erkande,\\ 
 & ê daz ir iwer schande,\\ 
 & hêrre, an im begienget.\\ 
 & \textbf{billîche} ir enpfienget\\ 
 & \textbf{guotlîche} \textbf{alle}, die her sint komen\\ 
20 & unde iweren kumber habent vernomen,\\ 
 & dane daz ir \textit{si} welt rouben.\\ 
 & des sult ir \textbf{iuch} gelouben."\\ 
 & der vürste sprach: "\textbf{nû} lâ mich \textbf{sehen},\\ 
 & dâ\textbf{ne} mac niht arges \textbf{zuo} geschehen."\\ 
25 & er \textbf{vuort in}, dâ er Gawanen sach.\\ 
 & zwei ougen unde ein herze jach,\\ 
 & diu Libaut mit \textbf{im} brâhte dar,\\ 
 & daz der gast wære wolgevar\\ 
 & unt \textbf{daz} rehte manlîche site\\ 
30 & sînen gebærden \textbf{wonte} mite.\\ 
\end{tabular}
\scriptsize
\line(1,0){75} \newline
G I O L M Q R Z Fr21 Fr22 Fr38 \newline
\line(1,0){75} \newline
\textbf{3} \textit{Initiale} I O L Z   $\cdot$ \textit{Capitulumzeichen} R  \textbf{23} \textit{Initiale} I  \newline
\line(1,0){75} \newline
\textbf{1} künnet] Kondet M  $\cdot$ danne] \textit{om.} G  $\cdot$ spehen] rehte spehen I sprechen Q \textbf{2} dinge] vuͤr I (O) \textbf{3} sîn] ÷in O  $\cdot$ gein valsche] gefalsche Q von falsche R [kein]: gein falsche Z \textbf{4} swer] Wer L M Q R \textbf{5} wære] si Z  $\cdot$ oder] \textit{om.} R  $\cdot$ mîniu] min I R \textbf{6} Alle die in zorne gen Jm sind R  $\cdot$ gein] mit I  $\cdot$ in] mit I  $\cdot$ zorn] haze G  $\cdot$ sint] sin L \textbf{7} mage] magde Q  $\cdot$ ode] vnd L \textit{om.} R \textbf{8} müesen] moͮsen G (M) (Q) (Z) (Fr21)  $\cdot$ diu] des R \textit{om.} Z \textbf{9} wil] \textit{om.} M  $\cdot$ weren] neren G \textbf{10} vor] von G  $\cdot$ unrehten] vnrehtem I (M) (Q)  $\cdot$ strîten] strite I (Q) cristen Z  $\cdot$ neren] weren G [wern]: nern I erneren Q \textbf{11} swâ] Wa L M (Q) R  $\cdot$ vor] in I  $\cdot$ hulden] schulden R \textbf{12} einen] eȳ M einem R  $\cdot$ sac] [sach]: satch G \textbf{13} wolt] Wil R  $\cdot$ ê] îe I \textbf{15} dâ] Daz L (R) Do Q \textbf{16} ê] Er M  $\cdot$ ir] [ich]: ir O \textbf{18} billîche] guͤtlich I (O) (L) (M) (Z) (Fr21) billicher Q (R)  $\cdot$ ir] ir in I O R \textbf{19} guotlîche] vnd I Billicher vnde O Billicher L M Z Fr21  $\cdot$ her] \textit{om.} Q  $\cdot$ sint] \textit{om.} M sein Q (Z) \textbf{20} iweren] ein I  $\cdot$ kumber] kom ern R \textbf{21} si] vns G  $\cdot$ welt] woͯltent R \textbf{22} iuch] wol M \textbf{23} nû] \textit{om.} M Q R  $\cdot$ mich] mich in L Q (R) Z myn M  $\cdot$ sehen] gesechen R sehn in Fr22 \textbf{24} dâne] Da O (Q) R  $\cdot$ arges] anders L vbels Z  $\cdot$ zuo] vz O (Q) (R) Z Fr21 >vsz< L \textit{om.} M  $\cdot$ geschehen] beschehen R \textbf{25} vuort] fvͤrt O  $\cdot$ dâ] do Q  $\cdot$ er Gawanen] er Gawan I irgawan M her Gawanen R \textbf{26} ougen] ouge M  $\cdot$ ein] \textit{om.} O  $\cdot$ jach] er sach L \textbf{27} Libaut] Lybavt O (Z) libavt L libayt M lybaut Q Libant R libovt Fr21 \textbf{29} manlîche] mýnnecliche L \textbf{30} sînen gebærden] Seinē geberde Q  $\cdot$ wonte] wonten I L (Z) \newline
\end{minipage}
\hspace{0.5cm}
\begin{minipage}[t]{0.5\linewidth}
\small
\begin{center}*T
\end{center}
\begin{tabular}{rl}
 & \textbf{welt} ir danne rîters vuore spehen,\\ 
 & ir müezet im \textbf{guoter} dinge jehen.\\ 
 & sîn lîp \textbf{wart nie gegen valsche} balt.\\ 
 & swer im dâr über \textbf{tæte} gewalt,\\ 
5 & \textbf{ez wære} mîn vater oder \textbf{mîn} kint,\\ 
 & alle, die gegen im in \textbf{strîte} sint,\\ 
 & mîne mage oder mîne bruoder,\\ 
 & die müesen di\textit{u} strîtes ruoder\\ 
 & gegen mir ziehen. ich wil in \textbf{nern},\\ 
10 & vor \textbf{unrehtem strîte} \textbf{wern},\\ 
 & swâ ich, hêrre, vor iuwern hulden mac.\\ 
 & ûz schiltes ambet in einen sac\\ 
 & wolt ich mich ê \textbf{ziehen},\\ 
 & sô verre ûz arde vliehen,\\ 
15 & dâ mich nieman erkande,\\ 
 & ê daz i\textit{r} iuwer schande,\\ 
 & hêrre, an im begienget.\\ 
 & \textbf{güetlîcher} ir enpfienget,\\ 
 & \textbf{billîcher} \textbf{alse}, die her sint komen\\ 
20 & unde iuwern kumber hânt vernomen,\\ 
 & danne daz ir si welt rouben.\\ 
 & des sult ir \textbf{mir} gelouben."\\ 
 & \begin{large}D\end{large}er vürste sprach: "\textbf{nû} lâz mich \textbf{in} \textbf{sehen},\\ 
 & dâ mac niht arges \textbf{an} geschehen."\\ 
25 & er \textbf{vuortin}, dâ er Gawanen sach.\\ 
 & zwei ougen unde ein herze jach,\\ 
 & di\textit{u} Lybaut mit brâhte dar,\\ 
 & daz der gast wære wol gevar\\ 
 & unde \textbf{daz} rehte manlîche site\\ 
30 & sînen gebærden \textbf{wonte} mite.\\ 
\end{tabular}
\scriptsize
\line(1,0){75} \newline
T V W \newline
\line(1,0){75} \newline
\textbf{3} \textit{Initiale} W  \textbf{23} \textit{Initiale} T  \newline
\line(1,0){75} \newline
\textbf{1} danne] \textit{om.} W  $\cdot$ spehen] sehen W \textbf{4} tæte] [*]: tuͦt V \textbf{6} strîte] zorne V \textbf{7} oder] alde T vnd W \textbf{8} diu] die T \textit{om.} W \textbf{9} nern] wern V (W) \textbf{10} Vor vnrehten [*]: striten nern V  $\cdot$ Vor vnrechten gewalt neren W \textbf{11} swâ] Wo W \textbf{12} ûz] Von W \textbf{13} ê ziehen] e geziehen V ziehen W \textbf{14} arde] der erde W \textbf{15} dâ] Daz V Do W \textbf{16} ir iuwer] ich iuwer T [*]: ir uwere V \textbf{18} Guͤtlichen ir in billich entpfiengent W \textbf{19} billîcher alse] Billicher alle V Also W \textbf{20} kumber] schaden W \textbf{23} Der vürste] Er W \textbf{24} an] auß W \textbf{25} dâ] do V W  $\cdot$ Gawanen] Gawan V (W) \textbf{27} diu] die T  $\cdot$ Lybaut] lẏbaut V lybout W  $\cdot$ mit brâhte] mit im brahte V bracht mit im W \textbf{30} sînen] Seinem W  $\cdot$ wonte] volgent W \newline
\end{minipage}
\end{table}
\end{document}
