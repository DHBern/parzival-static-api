\documentclass[8pt,a4paper,notitlepage]{article}
\usepackage{fullpage}
\usepackage{ulem}
\usepackage{xltxtra}
\usepackage{datetime}
\renewcommand{\dateseparator}{.}
\dmyyyydate
\usepackage{fancyhdr}
\usepackage{ifthen}
\pagestyle{fancy}
\fancyhf{}
\renewcommand{\headrulewidth}{0pt}
\fancyfoot[L]{\ifthenelse{\value{page}=1}{\today, \currenttime{} Uhr}{}}
\begin{document}
\begin{table}[ht]
\begin{minipage}[t]{0.5\linewidth}
\small
\begin{center}*D
\end{center}
\begin{tabular}{rl}
\textbf{621} & \begin{large}D\end{large}ô er si verre komen sach,\\ 
 & hin zer herzoginne er sprach:\\ 
 & "kumt \textbf{jenez} volc gein uns ze wer?"\\ 
 & si sprach: "ez ist Clinschors her,\\ 
5 & die iwer kûme hânt erbiten.\\ 
 & mit vreude \textbf{si koment} \textbf{nû} geriten\\ 
 & unt wellent iuch enpfâhen.\\ 
 & daz \textbf{darf} iu niht versmâhen,\\ 
 & sît ez diu vreude in gebôt."\\ 
10 & Nû was ouch Plippalinot\\ 
 & mit sîner clâren tohter fier\\ 
 & komen in einem ussier.\\ 
 & verre ûf \textbf{den} plân si gein im gienc,\\ 
 & diu magt \textbf{in mit vreude} enpfienc.\\ 
15 & Gawan bôt \textbf{ir} sînen gruoz;\\ 
 & si kust im stegreif und vuoz\\ 
 & \textbf{unt} enpfieng ouch \textbf{die} herzogîn.\\ 
 & si \textbf{nam} in bî dem zoume sîn\\ 
 & \textbf{und} bat erbeizen \textbf{den} man.\\ 
20 & \textbf{diu vrouwe} und Gawan\\ 
 & giengen an des schiffes ort.\\ 
 & ein teppech unt ein kulter dort\\ 
 & lâgen; an der selben stete\\ 
 & diu herzogîn \textbf{durch sîne} bete\\ 
25 & zuo Gawane nider saz.\\ 
 & des verjen tohter niht vergaz,\\ 
 & si entwâpente in, sus \textbf{hœre} ich sagen.\\ 
 & ir mantel \textbf{hete si} dar \textbf{getragen},\\ 
 & der des nahtes \textbf{ob} im lac,\\ 
30 & dô er ir herberge pflac.\\ 
\end{tabular}
\scriptsize
\line(1,0){75} \newline
D Z Fr68 \newline
\line(1,0){75} \newline
\textbf{1} \textit{Initiale} D Z Fr68  \textbf{10} \textit{Majuskel} D  \newline
\line(1,0){75} \newline
\textbf{1} Dô] Da Z \textbf{3} uns] mir Z \textbf{4} Clinschors] Clinscors D klingezores Z clinsdiores Fr68 \textbf{5} iwer kûme hânt] ewers kumens habn Z \textbf{6} Mit frevden komen sie (kument si nu Fr68 ) geriten Z (Fr68) \textbf{7} wellent] wellet Fr68 \textbf{8} darf] endarf Z \textbf{9} ez diu vreude in] inz die freude Z i::: div frowe Fr68 \textbf{10} Plippalinot] plipalinot Z \textbf{12} einem] einen Fr68 \textbf{13} di maget verre gein im gienc Fr68 \textbf{14} vf den plan da sin entfienc Fr68  $\cdot$ in mit vreude] mit frevden in Z \textbf{16} im] in Fr68 \textbf{18} dem] den Fr68 \textbf{19} und] Sie Z  $\cdot$ den] disen Z \textbf{20} diu vrouwe] Orgeluse Z  $\cdot$ und] vnde min her Fr68 \textbf{22} kulter] kolte Fr68 \textbf{25} Gawane] gawan Z \textbf{27} entwâpente in] entwappent in Z nen:wapendin Fr68  $\cdot$ hœre] hort Z \textbf{28} getragen] zv getragen Z \textbf{30} dô] Da Z \newline
\end{minipage}
\hspace{0.5cm}
\begin{minipage}[t]{0.5\linewidth}
\small
\begin{center}*m
\end{center}
\begin{tabular}{rl}
 & dô er s\textit{i} verre komen sach,\\ 
 & hin zuor herzoginne er sprach:\\ 
 & "komt \textbf{jenez} volc gegen uns zuo wer?"\\ 
 & si sprach: "ez ist Clinsor\textit{s} her,\\ 
5 & die iuwer kûme hânt erbiten.\\ 
 & mit vröuden \textbf{komen\textit{t} si} \textit{\textbf{nû}} geri\textit{t}en\\ 
 & und wellent iuch enp\textit{fâ}hen.\\ 
 & daz \textbf{en}\textbf{d\textit{a}r\textit{f}} iu niht versmâhen,\\ 
 & sît ez diu vröude in gebôt."\\ 
10 & nû was ouch Plippalinot\\ 
 & mit sîner clâren tohter fier\\ 
 & komen in eine\textit{m} \textit{u}ssier.\\ 
 & verre ûf \textbf{dem} plân si gegen im gienc,\\ 
 & diu maget \textbf{mit vröuden in} enpfienc.\\ 
15 & Gawan bôt \textbf{ir} sînen gruoz;\\ 
 & si kuste im stegrei\textit{f} und vuoz\\ 
 & \textbf{und} enpfienc ouch \textbf{die} herzogîn.\\ 
 & \textit{s}i \textbf{nam} in bî dem zoume sîn\\ 
 & \textbf{und} bat erbeizen \textbf{den} man.\\ 
20 & \textbf{diu vrouwe} und Gawan\\ 
 & giengen a\textit{n d}es schiffes ort.\\ 
 & ein teppich und ein kulter dort\\ 
 & lâgen; an der selben stete\\ 
 & diu herzogîn \textbf{sunder} bete\\ 
25 & zuo Gawan nider saz.\\ 
 & des verigen tohte\textit{r} \textit{n}i\textit{ht} vergaz,\\ 
 & si entwâpent in, sus \textbf{hôrte} ich sagen.\\ 
 & ir mantel \textbf{het si} dar \textbf{getragen},\\ 
 & der des nahtes \textbf{ob} im lac,\\ 
30 & dô er ir herberge pflac.\\ 
\end{tabular}
\scriptsize
\line(1,0){75} \newline
m n o \newline
\line(1,0){75} \newline
\newline
\line(1,0){75} \newline
\textbf{1} si] sich m \textbf{2} er] >er< o \textbf{4} Clinsors] clinsor m o  $\cdot$ her] hert n \textbf{6} koment] komen m n kamen o  $\cdot$ nû] ẏm m (n) (o)  $\cdot$ geriten] gerigen m \textbf{7} enpfâhen] enpflohen m \textbf{8} endarf] endorft m \textbf{11} mit] [Min]: Mit n \textbf{12} einem ussier] einen issier m eȳ uͯssier o \textbf{16} kuste] kúst o  $\cdot$ stegreif] stegreiffe m \textbf{18} si] Sin suͯ m \textbf{21} an des] an den des m \textbf{22} ein] \textit{om.} n \textbf{26} des] De: o  $\cdot$ tohter niht] tohtten in m \newline
\end{minipage}
\end{table}
\newpage
\begin{table}[ht]
\begin{minipage}[t]{0.5\linewidth}
\small
\begin{center}*G
\end{center}
\begin{tabular}{rl}
 & dô er si verre komen sach,\\ 
 & \textit{hin}z der herzoginne er sprach:\\ 
 & "kumet \textbf{jenez} volc gein uns ze wer?"\\ 
 & \begin{large}S\end{large}i sprach: "ez ist Clinsor\textit{s} her,\\ 
5 & die iwer kûme hânt erbiten.\\ 
 & mit vröuden \textbf{koment si} geriten\\ 
 & unde wellent iuch enpfâhen.\\ 
 & daz \textbf{en}\textbf{darf} iu niht versmâhen,\\ 
 & sît ez diu vröude in gebôt."\\ 
10 & nû was ouch Pliplalinot\\ 
 & mit sîner clâren tohter \textit{fier}\\ 
 & \textit{k}omen \textit{in} einem ur\textit{ss}ier.\\ 
 & verre ûf \textbf{den} plân si gein im gienc,\\ 
 & diu maget \textbf{\textit{mit} \textit{vröuden in}} enpfienc.\\ 
15 & Gawan bôt \textbf{ir} sînen gruoz;\\ 
 & si kust im \textbf{den} stegereif unde \textbf{den} vuoz.\\ 
 & \textbf{in} enpfienc ouch \textbf{diu} herzogîn.\\ 
 & si \textbf{nâmen} i\textit{n} \textit{b}î dem zoume sîn.\\ 
 & \textbf{si} bat erbeizen \textbf{disen} man.\\ 
20 & \textbf{Orgeluse} unde Gawan\\ 
 & giengen an des scheffes ort.\\ 
 & ein teppich unde ein kulter dort\\ 
 & lâgen; an der selben stet\\ 
 & diu herzogîn \textbf{durch sîn} bet\\ 
25 & zuo Gawane nider saz.\\ 
 & des verjen tohter niht vergaz,\\ 
 & s\textit{i} entwâpent in, sus \textbf{hôrt} ich sagen.\\ 
 & ir mandel \textbf{hiez man ir} dar \textbf{tragen},\\ 
 & der des nahtes \textbf{ob} im lac,\\ 
30 & dô er ir herberge pflac.\\ 
\end{tabular}
\scriptsize
\line(1,0){75} \newline
G I L M Z \newline
\line(1,0){75} \newline
\textbf{1} \textit{Initiale} L Z  \textbf{4} \textit{Initiale} G  \textbf{13} \textit{Initiale} I  \newline
\line(1,0){75} \newline
\textbf{1} dô] Da M Z \textbf{2} hinz] ze G \textbf{3} uns] mir Z \textbf{4} Clinsors] chlinshor G Clinisorsz L klingezores Z \textbf{5} iwer kûme] ewers kumens Z \textbf{6} si] sýe nv L \textbf{9} ez diu vröude in] inz die freude Z \textbf{10} Pliplalinot] pliapalinot I plipalinot L M Z \textbf{11} clâren] \textit{om.} I  $\cdot$ fier] \textit{om.} G shiere I \textbf{12} fier chomen vf einem [v*]: vrfier G  $\cdot$ einem] ein L (M) \textbf{13} den] dem L dē M \textbf{14} mit vröuden in] in mit froͮden G \textbf{15} ir] yn M \textbf{16} im den] im I L (M) Z  $\cdot$ unde den] vnd L M Z \textbf{17} in] Sie L Vnd Z \textbf{18} nâmen] nam L (M) (Z)  $\cdot$ in bî] in oͮch  bi G \textbf{19} bat] baten I \textbf{20} Orgeluse] Orguluse I Orgelýse L \textbf{22} ein kulter] kuͯlter L \textbf{25} Gawane] Gawan I (Z) \textbf{27} si] sin G  $\cdot$ entwâpent] entwapinde M  $\cdot$ sus hôrt ich] sus hoͤr ich I hort L \textbf{28} hiez man ir] hiez man I het sie Z  $\cdot$ tragen] zv getragen Z \textbf{30} dô] Da M Z \newline
\end{minipage}
\hspace{0.5cm}
\begin{minipage}[t]{0.5\linewidth}
\small
\begin{center}*T
\end{center}
\begin{tabular}{rl}
 & dô er si verre komen sach,\\ 
 & hin zuo der herzogîn er \textbf{dô} sprach:\\ 
 & "kumet \textbf{ein} volc gein uns zuo wer?"\\ 
 & si sprach: "ez ist Clynsors her,\\ 
5 & die iuwer kûme hânt erbiten.\\ 
 & mit vreuden \textbf{koment si} \textbf{nû} geriten\\ 
 & und wollent iuch entvâhen.\\ 
 & daz \textbf{en}\textbf{sol} iu niht versmâhen,\\ 
 & sît ez diu vreude in gebôt."\\ 
10 & nû was ouch Plipalinot\\ 
 & mit sîner clâren tohter fier\\ 
 & k\textit{o}men in eime ussier.\\ 
 & verre ûf \textbf{jenen} plân si gein im gienc,\\ 
 & diu maget \textbf{mit vreuden in} entvienc.\\ 
15 & Gawan bôt \textbf{in} sînen gruoz;\\ 
 & si kust im stegereif und vuoz\\ 
 & \textbf{und} entvienc ouch \textbf{die} herzogîn.\\ 
 & si \textbf{nam} in bî dem zoume sîn.\\ 
 & \textbf{si} bat erbeizen \textbf{disen} man.\\ 
20 & \textbf{Orgeluse} und Gawan\\ 
 & giengen an des schiffes ort.\\ 
 & ein teppich und ein kulter dort\\ 
 & lâgen; an der selben stet\\ 
 & diu herzogîn \textbf{durch sîne} bet\\ 
25 & zuo Gawane nider saz.\\ 
 & des vergen tohter niht vergaz,\\ 
 & si entwâpent in, sus \textbf{hôrt} ich sagen.\\ 
 & ir mantel \textbf{hâte si} dar \textbf{getragen},\\ 
 & der des nahtes \textbf{über} im lac,\\ 
30 & dô er ir herberge pflac.\\ 
\end{tabular}
\scriptsize
\line(1,0){75} \newline
U V W Q R Fr39 \newline
\line(1,0){75} \newline
\textbf{1} \textit{Initiale} W Fr39   $\cdot$ \textit{Capitulumzeichen} R  \newline
\line(1,0){75} \newline
\textbf{2} dô] \textit{om.} V W \textbf{3} ein] gens V (W) (Q) (R) (Fr39) \textbf{4} Clynsors] clinsors V klynshors W clinszhor Q Clinshors R Fr39 \textbf{5} die] Div Fr39  $\cdot$ erbiten] [erb*]: erbitten V erbieten Q \textbf{6} koment] kumpt Q  $\cdot$ si nû] sv́ V siniv Fr39 \textbf{8} ensol] endarff W (Q) R (Fr39) \textbf{10} Plipalinot] plypalmat U plypalinot W \textbf{11} mit] Mir W  $\cdot$ tohter] toͤhter V \textbf{12} komen] Quamen U  $\cdot$ eime] ein W Q R Fr39 \textbf{13} jenen] den W Q Fr39 dem R \textbf{14} diu] Mit Q  $\cdot$ entvienc] geuieng W \textbf{15} Gawan] Gabon Q Gawin R  $\cdot$ in] ir V W R Fr39 \textbf{17} die] div Fr39 \textbf{18} bî] mit Fr39  $\cdot$ sîn] hin R \textbf{19} disen] [d*]: den V \textbf{20} Orgeluse] Orgelusse Q Orguluse R  $\cdot$ Gawan] herr gawan W \textbf{25} Gawane] Gawin R \textbf{27} sus] als Q \textbf{29} des] eines W  $\cdot$ über] ob V W Q R Fr39 \textbf{30} pflac] pfag Q \newline
\end{minipage}
\end{table}
\end{document}
