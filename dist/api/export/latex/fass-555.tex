\documentclass[8pt,a4paper,notitlepage]{article}
\usepackage{fullpage}
\usepackage{ulem}
\usepackage{xltxtra}
\usepackage{datetime}
\renewcommand{\dateseparator}{.}
\dmyyyydate
\usepackage{fancyhdr}
\usepackage{ifthen}
\pagestyle{fancy}
\fancyhf{}
\renewcommand{\headrulewidth}{0pt}
\fancyfoot[L]{\ifthenelse{\value{page}=1}{\today, \currenttime{} Uhr}{}}
\begin{document}
\begin{table}[ht]
\begin{minipage}[t]{0.5\linewidth}
\small
\begin{center}*D
\end{center}
\begin{tabular}{rl}
\textbf{555} & \begin{large}D\end{large}urch iwer güete, wer \textbf{die} sîn."\\ 
 & dô \textbf{erschrac} daz juncvröuwelîn\\ 
 & \textbf{unt sprach}: "hêrre, nû vrâgt \textbf{es} niht.\\ 
 & ich bin, dius \textbf{nimmer iu} vergiht.\\ 
5 & ich\textbf{n} kan iu niht von \textbf{in} gesagen;\\ 
 & ob ichz \textbf{halt} weiz, ich \textbf{sol}z verdagen.\\ 
 & lâtz iu \textbf{von mir} niht swære\\ 
 & \textbf{sîn} und vrâget anderre mære.\\ 
 & daz rât ich, welt ir volgen mir."\\ 
10 & Gawan sprach aber wider zir,\\ 
 & Mit vrâge \textbf{er gienc} dem mære nâch\\ 
 & umb al die vrouwen, die er dâ sach\\ 
 & \textbf{sitzende} ûf dem palas.\\ 
 & diu magt \textbf{wol} \textbf{sô} getriwe was,\\ 
15 & daz si von herzen weinde\\ 
 & unt grôze klage erscheinde.\\ 
 & Dennoch was ez harte vruo;\\ 
 & innen des gie ir vater zuo.\\ 
 & der liez ez âne zürnen gar,\\ 
20 & ob diu maget wol gevar\\ 
 & ihtes \textbf{dâ} wære betwungen\\ 
 & \textbf{unt} ob dâ \textbf{was} gerungen.\\ 
 & dem gebârte si gelîche,\\ 
 & diu magt zühte rîche,\\ 
25 & wande si dem bette nâhe saz.\\ 
 & daz liez ir vater âne haz.\\ 
 & \textbf{Dô sprach er}: "\textbf{tohter}, \textbf{weinet} niht.\\ 
 & swaz in schimpfe \textbf{alsus} geschiht,\\ 
 & ob daz von êrste bringet zorn,\\ 
30 & der ist schiere dâ nâch verkorn."\\ 
\end{tabular}
\scriptsize
\line(1,0){75} \newline
D \newline
\line(1,0){75} \newline
\textbf{1} \textit{Initiale} D  \textbf{11} \textit{Majuskel} D  \textbf{17} \textit{Majuskel} D  \textbf{27} \textit{Majuskel} D  \newline
\line(1,0){75} \newline
\newline
\end{minipage}
\hspace{0.5cm}
\begin{minipage}[t]{0.5\linewidth}
\small
\begin{center}*m
\end{center}
\begin{tabular}{rl}
 & durch iuwer güete, wer \textbf{d\textit{ie}} sîn."\\ 
 & dô \textbf{erschrac} daz juncvröuwelîn.\\ 
 & \textbf{si sprach}: "hêrre, nû vrâg\textit{et} \textbf{es} niht;\\ 
 & ich bin, diu es \textbf{niemer iu} vergiht.\\ 
5 & ich kan iu niht von \textbf{im} gesagen;\\ 
 & ob ich \textit{ez} \textbf{joch} weiz, ich \textbf{solte} ez ver\textit{d}agen.\\ 
 & lâtz iu \textbf{von mir} \textit{n}i\textit{ht} swære\\ 
 & \dag~\dag\ und vrâget ander mære.\\ 
 & daz râte ich, wellet ir volgen mir."\\ 
10 & Gawan sprach aber wider zuo ir,\\ 
 & mit vrâge \textbf{er gienc} dem mære nâch\\ 
 & umb alle die vrouwe\textit{n}, die er d\textit{â} sach\\ 
 & \textbf{sitzen} ûf dem palas.\\ 
 & diu maget \textbf{sô} getriuwe was,\\ 
15 & daz si von herzen weinde\\ 
 & und grôze klage erscheinde.\\ 
 & dannoch was ez hart\textit{e} vruo;\\ 
 & innen des gienc ir vater zuo.\\ 
 & der liez ez âne z\textit{ü}r\textit{n}en gar,\\ 
20 & ob diu maget wol gevar\\ 
 & ihtes \textbf{d\textit{â}} wær betwungen\\ 
 & \textbf{und} ob d\textit{â} \textbf{was} gerungen.\\ 
 & dem gebârte si glîch,\\ 
 & diu maget zühte rîch,\\ 
25 & wan si dem bette nâhe saz.\\ 
 & \dag dô\dag  liez ir vater âne haz.\\ 
 & \textbf{er sprach}: "\textbf{tohter}, \textbf{nû} \textbf{weine} niht.\\ 
 & waz in schimpf \textbf{alsus} geschiht,\\ 
 & ob daz von êrste bringet zorn,\\ 
30 & der ist schier dâ nâch verkorn."\\ 
\end{tabular}
\scriptsize
\line(1,0){75} \newline
m n o \newline
\line(1,0){75} \newline
\textbf{10} \textit{Capitulumzeichen} n  \newline
\line(1,0){75} \newline
\textbf{1} wer die] wer do m wo d: o \textbf{2} erschrac] erscharck m erschracke n erstag o \textbf{3} vrâget] fragtte m (o) \textbf{6} ich ez] ich m  $\cdot$ joch] \textit{om.} n  $\cdot$ verdagen] vertragen m \textbf{7} von] [e]: von o  $\cdot$ niht] in m \textbf{12} vrouwen] frouwe m  $\cdot$ dâ] do m n o \textbf{17} ez] g es o  $\cdot$ harte] hartten m \textbf{18} zuo] [e]: zuͦ o \textbf{19} liez] liesse n  $\cdot$ zürnen] zilren m \textbf{21} dâ] do m n o  $\cdot$ betwungen] getwungen n (o) \textbf{22} dâ] do m n o \textbf{24} zühte] zuhtten m (n) (o) \textbf{26} liez] liesse n \textbf{28} schimpf alsus] schipff also o \textbf{29} \textit{Die Verse 555.29-30 fehlen} n  \newline
\end{minipage}
\end{table}
\newpage
\begin{table}[ht]
\begin{minipage}[t]{0.5\linewidth}
\small
\begin{center}*G
\end{center}
\begin{tabular}{rl}
 & \begin{large}D\end{large}urch iuwer güete, wer \textbf{die} sîn."\\ 
 & dô \textbf{erschrac} daz juncvröuwelîn.\\ 
 & \textbf{si sprach}: "hêrre, nû vrâget \textbf{es} niht;\\ 
 & ich bin, dius \textbf{nimmer iu} vergiht.\\ 
5 & ich \textbf{en}kan iu niht von \textbf{in} gesagen;\\ 
 & ob ich ez \textbf{halt} weiz, ich \textbf{sol} ez verdagen.\\ 
 & lât ez iu niht \textbf{sîn} swære\\ 
 & unde vrâget anderre mære.\\ 
 & daz rât ich, welt ir volgen mir."\\ 
10 & Gawan sprach aber wider zir,\\ 
 & mit vrâge \textbf{er gienc} dem mære nâch\\ 
 & umbe al die vrouwen, die er dâ sach\\ 
 & \textbf{sitzende} ûf dem palas.\\ 
 & diu maget \textbf{wol} \textbf{sô} getriuwe was,\\ 
15 & daz si von herzen weinde\\ 
 & unde grôze klage erscheinde.\\ 
 & dannoch was ez harte vruo;\\ 
 & innen des gienc ir vater zuo.\\ 
 & der liez ez âne zürnen gar,\\ 
20 & ob diu maget wol gevar\\ 
 & ihtes \textbf{dâ} wære betwungen\\ 
 & \textbf{unde} ob dâ \textbf{was} gerungen.\\ 
 & dem gebârte si gelîche,\\ 
 & diu maget zühte rîche,\\ 
25 & wande si dem bette nâhen saz.\\ 
 & daz liez ir vater âne haz.\\ 
 & \textbf{dô sprach er}: "\textbf{tohter}, \textbf{weinet} niht.\\ 
 & swaz in schimpfe \textbf{alsus} geschiht,\\ 
 & \textit{ob} daz von êrste bringet zorn,\\ 
30 & der ist schier\textit{e d}â nâch verkorn."\\ 
\end{tabular}
\scriptsize
\line(1,0){75} \newline
G I L M Z Fr23 Fr62 \newline
\line(1,0){75} \newline
\textbf{1} \textit{Initiale} G L Z Fr23  \textbf{3} \textit{Initiale} Fr62  \textbf{13} \textit{Initiale} I  \textbf{27} \textit{Initiale} I  \newline
\line(1,0){75} \newline
\textbf{1} die] dise I \textbf{2} dô] Da M  $\cdot$ juncvröuwelîn] frauwelin L \textbf{3} hêrre] \textit{om.} M  $\cdot$ es] sin I Z \textbf{4} dius] osz dy M der Fr62  $\cdot$ iu] \textit{om.} M \textbf{5} ich enkan iu] ich chan ev I ouh kan ih Fr62 \textbf{6} halt] ioh Fr62  $\cdot$ verdagen] verclagen Fr62 \textbf{7} iu] uͯch von mir L (M) (Z) (Fr62)  $\cdot$ sîn] \textit{om.} L M Fr62  $\cdot$ swære] sweren M \textbf{8} anderre] andere Fr62 \textbf{9} welt] wollz M werlt Fr23 \textbf{10} aber] \textit{om.} M Fr62  $\cdot$ wider] \textit{om.} I Fr23 \textbf{11} vnd gienc mit frage dem mere nah Fr62  $\cdot$ vrâge] \textit{om.} L  $\cdot$ dem mære] den meren L \textbf{12} al] \textit{om.} Fr62  $\cdot$ er] \textit{om.} L \textbf{13} sitzende] Sitzen I L \textbf{14} wol sô] so wol Fr23 so Fr62 \textbf{16} unde] vil I \textbf{17} dannoch] Danch Fr23  $\cdot$ vruo] fro Fr62 \textbf{18} innen] Jnners Fr23 \textbf{19} ez] \textit{om.} L  $\cdot$ zürnen] zorn I (M) (Fr23) \textbf{20} diu] der Fr23 \textbf{21} dâ] \textit{om.} I Z  $\cdot$ betwungen] Getwungen I \textbf{22} was] vns Fr23 \textbf{23} gebârte] gebart I Fr23  $\cdot$ si gelîche] sigeliche Fr62 \textbf{26} daz] dar Fr62 \textbf{27} dô] Da M  $\cdot$ er] ir M  $\cdot$ weinet] wenit L \textit{om.} M \textbf{28} swaz] Waz L (M)  $\cdot$ alsus] sus Fr62  $\cdot$ geschiht] gegiht Fr23 \textbf{29} ob] Nu G  $\cdot$ bringet] bringz M \textbf{30} ist] ist doch I ir ist M  $\cdot$ schiere] schiere doh G  $\cdot$ verkorn] [verlorn]: verkorn I verlorn Fr23 \newline
\end{minipage}
\hspace{0.5cm}
\begin{minipage}[t]{0.5\linewidth}
\small
\begin{center}*T
\end{center}
\begin{tabular}{rl}
 & durch iuwer güete, wer \textbf{si} sîn."\\ 
 & Dô \textbf{sprach} daz juncvröuwelîn:\\ 
 & "hêrre, nû\textbf{ne} vrâget niht.\\ 
 & ich bin, dius \textbf{iu niemer} vergiht.\\ 
5 & ich \textbf{en}kan iu niht von \textbf{in} gesagen;\\ 
 & ob ichz \textbf{halt} weiz, ich \textbf{sol}z verdagen.\\ 
 & Lâtz iu niht \textbf{sîn} swære\\ 
 & unde vrâget anderre mære.\\ 
 & daz rât ich, welt ir volgen mir."\\ 
10 & Gawan sprach aber wider zir,\\ 
 & mit vrâge \textbf{gienc er} dem mære nâch\\ 
 & umbe al die vrouwen, die er dâ sach\\ 
 & \textbf{sitzen} ûf dem palas.\\ 
 & diu maget \textbf{wol} getriuwe was,\\ 
15 & daz si von herzen weinde\\ 
 & unde grôze klage erscheinde.\\ 
 & \textit{\begin{large}D\end{large}}annoch was ez harte vruo;\\ 
 & indes gienc ir vater zuo.\\ 
 & der liez ez âne zürnen gar,\\ 
20 & ob diu maget wol gevar\\ 
 & ihtes wære betwungen\\ 
 & \textbf{oder} ob dâ \textbf{wære} gerungen.\\ 
 & dem gebârte si glîche,\\ 
 & diu maget zühte rîche,\\ 
25 & wande si dem bette nâhe saz.\\ 
 & daz liez ir vater âne haz.\\ 
 & \textbf{Dô sprach er}: "\textbf{vrouwe}, \textbf{weinet} niht.\\ 
 & swaz in schimpfe \textbf{sus} geschiht,\\ 
 & ob daz von êrst bringet zorn,\\ 
30 & der ist schiere dâ nâch verkorn."\\ 
\end{tabular}
\scriptsize
\line(1,0){75} \newline
T U V W O Q R Fr39 \newline
\line(1,0){75} \newline
\textbf{1} \textit{Initiale} O Fr39   $\cdot$ \textit{Capitulumzeichen} R  \textbf{2} \textit{Initiale} W   $\cdot$ \textit{Majuskel} T  \textbf{7} \textit{Majuskel} T  \textbf{17} \textit{Initiale} T V  \textbf{27} \textit{Initiale} W   $\cdot$ \textit{Majuskel} T  \newline
\line(1,0){75} \newline
\textbf{1} \textit{Die Verse 553.1-599.30 fehlen} U   $\cdot$ durch] ÷vrch O \textbf{3} nûne vrâget] nun fragent W nv enfragte O numme fragen R  $\cdot$ niht] mich nicht W mich R \textbf{4} bin] binsz Q  $\cdot$ dius iu] die úch V div des O di di es euch Q div e: Fr39  $\cdot$ niemer] niht V niemer iv Fr39 \textbf{5} enkan] kan W (Q) R Fr39  $\cdot$ niht] \textit{om.} R Fr39  $\cdot$ in] [*]: in V euch Q In nit R (Fr39)  $\cdot$ gesagen] sagen Q \textbf{6} ob] Vnd ob W  $\cdot$ ichz] \textit{om.} R  $\cdot$ halt] wol V \textit{om.} W  $\cdot$ weiz] west W (O)  $\cdot$ ich solz] ich solts W (O) ich weis Jch R ich s::s Fr39 \textbf{7} Lant ez úch niht [vo*]: von mir sin swere V  $\cdot$ Lat eúchs nicht sein zuͦ schwere W  $\cdot$ Lat ev iz niht wesen swere O  $\cdot$ swære] were Q \textbf{8} Vnd fraget mich andere mere W  $\cdot$ Vnd fragt andrú mere R \textbf{9} ir] irs Q Fr39 \textbf{10} Gawan] Gawin R  $\cdot$ wider] \textit{om.} W R \textbf{11} vrâge] fleisse W  $\cdot$ gienc er] gen ich O er gieng R e: gienc Fr39  $\cdot$ dem mære] den mærn O \textbf{12} vrouwen] frowe R  $\cdot$ er] ich O  $\cdot$ dâ] do V W Fr39 \textit{om.} R \textbf{13} sitzen] Sitzende W O (R) Fr39 \textbf{14} wol] wol so V O Q R Fr39 also W \textbf{16} unde] Vnd auch W \textbf{17} Dannoch] ÷Annoch T Dannacht R \textbf{18} indes] [Jnnes]: Jnndes O Jnne R  $\cdot$ ir] der R \textbf{19} zürnen] zorn O (R) \textbf{21} \textit{Die Verse 555.21-30 fehlen} O  \textbf{22} dâ] do V W Q Fr39  $\cdot$ wære] was W R Fr39 was wer Q \textbf{23} si] sy gantz W \textbf{24} zühte rîche] zv́hten riche V (Q) (R) zúchtenreiche W \textbf{26} haz] allen haß W \textbf{27} Er sprach tohter weinent niht V  $\cdot$ vrouwe] tochter W (Q) (R) (Fr39) \textbf{28} swaz] Was W Q (R)  $\cdot$ sus] alsus Q R Fr39 \textbf{29} von êrst] zuͦm ersten W zum erst Q \textbf{30} der ist] Das ist vil W Das ist R \newline
\end{minipage}
\end{table}
\end{document}
