\documentclass[8pt,a4paper,notitlepage]{article}
\usepackage{fullpage}
\usepackage{ulem}
\usepackage{xltxtra}
\usepackage{datetime}
\renewcommand{\dateseparator}{.}
\dmyyyydate
\usepackage{fancyhdr}
\usepackage{ifthen}
\pagestyle{fancy}
\fancyhf{}
\renewcommand{\headrulewidth}{0pt}
\fancyfoot[L]{\ifthenelse{\value{page}=1}{\today, \currenttime{} Uhr}{}}
\begin{document}
\begin{table}[ht]
\begin{minipage}[t]{0.5\linewidth}
\small
\begin{center}*D
\end{center}
\begin{tabular}{rl}
\textbf{373} & "\begin{large}T\end{large}ohter, swes dîn wille gert,\\ 
 & hân ichz, des bistû gewert.\\ 
 & \textbf{ôwol} der vruht, diu an dir lac!\\ 
 & dîn geburt was der sælden tac."\\ 
5 & "Vater, \textbf{sô wil} ich dirz sagen,\\ 
 & heinlîche mînen kumber klagen.\\ 
 & nâch dînen gnâden dar zuo sprich."\\ 
 & er bat si heben vür sich.\\ 
 & Si sprach: "war kœme denne mîn gespil?"\\ 
10 & dô hielt \textbf{der ritter bî im} vil,\\ 
 & die striten, wer si solde nemen,\\ 
 & des moht ieslîchen wol gezemen,\\ 
 & iedoch bôt man si einem dar.\\ 
 & Clauditte was \textbf{ouch} wol gevar.\\ 
15 & Al rîtende sprach \textbf{ir vater} zir:\\ 
 & "Obilot, nû sage mir\\ 
 & ein teil von dîner nœte."\\ 
 & "\textbf{dâ hân ich} kleinœte\\ 
 & dem vremden ritter gelobt.\\ 
20 & ich wæne, mîn \textbf{sin} \textbf{hât} \textbf{getobt}.\\ 
 & hân ich im niht ze gebene,\\ 
 & wa\textit{z} toug \textbf{ich} denne ze lebene,\\ 
 & sît er mir dienst hât geboten?\\ 
 & \textbf{sô} muoz ich schemelîche roten,\\ 
25 & ob ich im niht ze geben hân.\\ 
 & nie meide wart sô liep ein man."\\ 
 & Dô sprach er: "tohter, wart an mich,\\ 
 & ich sol des wol bereiten dich.\\ 
 & sît dû dienstes \textbf{von im} gerst,\\ 
30 & ich gib dir, daz dû in gewerst,\\ 
\end{tabular}
\scriptsize
\line(1,0){75} \newline
D \newline
\line(1,0){75} \newline
\textbf{1} \textit{Initiale} D  \textbf{5} \textit{Majuskel} D  \textbf{9} \textit{Majuskel} D  \textbf{15} \textit{Majuskel} D  \textbf{27} \textit{Majuskel} D  \newline
\line(1,0){75} \newline
\textbf{14} Clauditte] Clavdîte D \textbf{16} Obilot] Obylot D \textbf{22} waz] was D \textbf{29} dienstes] diens D \newline
\end{minipage}
\hspace{0.5cm}
\begin{minipage}[t]{0.5\linewidth}
\small
\begin{center}*m
\end{center}
\begin{tabular}{rl}
 & "tohter, wes dîn wille gert,\\ 
 & hân ichz, des bistû gewert.\\ 
 & \textbf{ouch} \textbf{wol} der vrühte, diu an dir lac!\\ 
 & dîn geburt was der sælden tac."\\ 
5 & "vater, \textbf{sô wil} ich dirz sagen,\\ 
 & heimelîche mî\textit{n}en kumber klagen.\\ 
 & nâch dînen gnâden dar zuo spr\textit{i}ch."\\ 
 & er bat si heben vür sich.\\ 
 & si sprach: "war kœme danne mîn gespil?"\\ 
10 & d\textit{ô} hielt \textbf{der ritter bî ime} vil,\\ 
 & die striten, wer si solte nemen,\\ 
 & des m\textit{o}hte ieglîchen wol gezemen,\\ 
 & iedoch bôt man si einem dar.\\ 
 & Clauditte was \textbf{ouch} wol gevar.\\ 
15 & alrîtende sprach \textbf{ir vater} zuo ir:\\ 
 & "Obilot, nû sage mir\\ 
 & ein teil von dîner nœte."\\ 
 & "\textbf{dâ hân ich} kleinœte\\ 
 & dem vremde\textit{n} ritter gel\textit{o}b\textit{e}t.\\ 
20 & ich wæne, mîn \textbf{sin}  \textbf{getobet}.\\ 
 & hân ich ime niht ze gebene,\\ 
 & waz tou\textit{c} \textbf{ich} danne ze lebene,\\ 
 & sît er mir dienest hât geboten?\\ 
 & \textbf{sô} muoz ich schamelîch roten,\\ 
25 & ob ich ime niht ze gebene hân.\\ 
 & \textit{ni}e megde wart sô liep ein man."\\ 
 & dô sprach er: "tohter, warte an mich,\\ 
 & ich \textit{sol des} wol bereiten dich.\\ 
 & sît dû dienstes \textbf{von ime} gerst,\\ 
30 & ich gibe dir, daz dû in \textbf{wol} gewerst,\\ 
\end{tabular}
\scriptsize
\line(1,0){75} \newline
m n o \newline
\line(1,0){75} \newline
\newline
\line(1,0){75} \newline
\textbf{3} diu] der die o \textbf{5} ich dirz] ichs dir n \textbf{6} mînen] minnen m  $\cdot$ kumber] komen o \textbf{7} nâch] Do noch n  $\cdot$ sprich] sprach m \textbf{9} war kœme] warke o \textbf{10} dô] Die m  $\cdot$ bî] do bẏ o \textbf{11} solte] solten o \textbf{12} mohte] moͯhte m solte n \textbf{13} bôt] bat o  $\cdot$ einem] einen n (o) \textbf{14} Clauditte] [Claudie*]: Claudite n Claudite o \textbf{15} ir] in o \textbf{18} dâ] Do n o \textbf{19} vremden] frundem m  $\cdot$ gelobet] gelebot m gloubet n \textbf{20} mîn sin] mine sinne hant n (o) \textbf{22} touc] touwe m \textbf{23} sît] Sit dasz o \textbf{25} gebene] gebe o \textbf{26} nie] Me m \textbf{28} sol des] des sol m \textbf{29} von] vom o \newline
\end{minipage}
\end{table}
\newpage
\begin{table}[ht]
\begin{minipage}[t]{0.5\linewidth}
\small
\begin{center}*G
\end{center}
\begin{tabular}{rl}
 & "tohter, swes dîn wille gert,\\ 
 & hân ich ez, des bistû gewert.\\ 
 & \textbf{ôwol} der vruht, diu an dir lac!\\ 
 & dîn geburt was der sælden tac."\\ 
5 & "vater, \textbf{sô wil} ich dirz sagen,\\ 
 & heinlîche mînen kumber klagen.\\ 
 & nâch dînen genâden dar \textit{zuo} sprich."\\ 
 & er bat si heben vür sich.\\ 
 & si sprach: "war kœme dane mîn gespil?"\\ 
10 & dô hielt \textbf{dâ} \textbf{bî im rîter} vil,\\ 
 & die striten, wer \textit{si} solte nemen,\\ 
 & des moht ieslîchen wol gezemen,\\ 
 & iedoch bôt man si einem dar.\\ 
 & Claudite was \textbf{ouch} wol gevar.\\ 
15 & alrîtende sprach \textbf{der vürste} zir:\\ 
 & "Obilot, nû sage mir\\ 
 & ein teil von dîner nœte."\\ 
 & "\textbf{dâ hân ich} kleinœte\\ 
 & dem vrömden rîter gelobet.\\ 
20 & ich wæne, mîn \textbf{zuht} \textbf{sî} \textbf{vertobet}.\\ 
 & hân ich im niht ze gebenne,\\ 
 & waz touc \textbf{ich} danne ze lebenne,\\ 
 & \begin{large}S\end{large}ît er mir dienst hât geboten?\\ 
 & \textbf{nû} muoz ich schamelîchen roten,\\ 
25 & obe ich im niht ze gebene hân.\\ 
 & nie magede wart sô liep ein man."\\ 
 & dô sprach er: "tohter, warte an mich,\\ 
 & ich sol \textit{d}es wol bereiten dich.\\ 
 & sît \textit{dû} dienstes \textbf{an in} gerst,\\ 
30 & ich gibe dir, daz dû in gewerst,\\ 
\end{tabular}
\scriptsize
\line(1,0){75} \newline
G I O L M Q R Z Fr24 Fr38 \newline
\line(1,0){75} \newline
\textbf{5} \textit{Initiale} O L Fr38   $\cdot$ \textit{Capitulumzeichen} R  \textbf{21} \textit{Initiale} I  \textbf{23} \textit{Initiale} G  \textbf{27} \textit{Initiale} M Fr24  \newline
\line(1,0){75} \newline
\textbf{1} \textit{Die Verse 370.13-412.12 fehlen} Q   $\cdot$ swes] wez L (M) (R) (Z) \textbf{3} ôwol] Wol O  $\cdot$ der] dir M Fr24 \textbf{5} vater] ÷ater O  $\cdot$ ich dirz] ichz dir M \textbf{7} zuo] nach G  $\cdot$ sprich] sprach L \textbf{8} [der riter hat gewert mich]: er bat si heben vur sich G  $\cdot$ sich] sich ach L \textbf{9} war kœme] kom Z  $\cdot$ dane] \textit{om.} R \textbf{10} dô] Da O M R Z  $\cdot$ hielt] hielttent R  $\cdot$ im] nuͯ L \textit{om.} M \textbf{11} die] Sie L  $\cdot$ si] die G \textbf{12} Das moch ir ietlichem wol gezemen R \textbf{13} iedoch] doch I Je do R  $\cdot$ man si] mans O L Z Fr24 Fr38  $\cdot$ einem] einen M Z \textbf{14} Claudite] Glauditte I Clavdit O Clauͯditte L Clauditte M Z Clavdît Fr24 Clavditte Fr38 \textbf{15} vürste] vatter R \textbf{16} Obilot] Obylot O Z Fr24 Fr38 Oblett R \textbf{18} dâ] Do R \textbf{19} vrömden] fromdem I (O) fremdē M (Fr38)  $\cdot$ gelobet] gelop I \textbf{20} Jch wenn ich hab getobt R  $\cdot$ zuht sî vertobet] sin hat getobt O L (M) (Fr24) (Fr38) sin hab getobt Z \textbf{21} gebenne] gebent R \textbf{22} Was toyc den myn leben M  $\cdot$ ich] mir Z \textbf{25} ze gebene] zebenne L \textbf{26} Ein meide ward nie so lieb einem man R  $\cdot$ magede] myde M  $\cdot$ wart sô liep] so liep wart Z \textbf{27} dô sprach er] Er sprach O L R (Fr38) \textbf{28} sol des] soles G sol R  $\cdot$ bereiten] beraten I bewarren R [bet]: beraten Z \textbf{29} dû] \textit{om.} G M  $\cdot$ dienstes] dienst I  $\cdot$ an in] von im O L (M) (R) Z Fr24 (Fr38) \newline
\end{minipage}
\hspace{0.5cm}
\begin{minipage}[t]{0.5\linewidth}
\small
\begin{center}*T
\end{center}
\begin{tabular}{rl}
 & "Tohter, swes dîn wille gert,\\ 
 & hân ichz, des bistû gewert.\\ 
 & \textbf{wol} der vruht, d\textit{iu} an dir lac!\\ 
 & dîn geburt was der sælden tac."\\ 
5 & "Vater, \textbf{sol} ich dirz \textbf{danne} sagen\\ 
 & \textbf{unde} heimlîche mînen kumber klagen.\\ 
 & nâch dînen gnâden dar zuo sprich."\\ 
 & er bat si heben vür sich.\\ 
 & Si sprach: "war kœme danne mîn gespil?"\\ 
10 & Dô hielt \textbf{bî im rîter} vil,\\ 
 & die striten, wer si sulte nemen,\\ 
 & des mohte iegeslîchen wol gezemen,\\ 
 & iedoch bôt man si einem dar.\\ 
 & Claudite was wol gevar.\\ 
15 & alrîtende sprach \textbf{der vürste} zir:\\ 
 & "Obylot, nû sage mir\\ 
 & ein teil von dîner nœte."\\ 
 & "\textbf{Ich hân} kleinœte\\ 
 & dem vremden rîter gelobet.\\ 
20 & ich wæne, mîn \textbf{sin} \textbf{hât} \textbf{getobet}.\\ 
 & \textbf{unde} hân ich im niht ze gebenne,\\ 
 & waz touc \textbf{mir} danne ze lebenne,\\ 
 & sît er mir dienst hât geboten?\\ 
 & \textbf{sô} muoz ich schemelîche roten,\\ 
25 & ob ich im niht ze gebenne hân.\\ 
 & nie megde wart sô liep ein man."\\ 
 & Dô sprach er: "tohter, warte an mich,\\ 
 & ich sol des wol bereiten dich,\\ 
 & \multicolumn{1}{l}{ - - - }\\ 
30 & \multicolumn{1}{l}{ - - - }\\ 
\end{tabular}
\scriptsize
\line(1,0){75} \newline
T V W \newline
\line(1,0){75} \newline
\textbf{1} \textit{Initiale} W   $\cdot$ \textit{Majuskel} T  \textbf{5} \textit{Majuskel} T  \textbf{9} \textit{Majuskel} T  \textbf{10} \textit{Majuskel} T  \textbf{18} \textit{Majuskel} T  \textbf{27} \textit{Majuskel} T  \newline
\line(1,0){75} \newline
\textbf{1} swes] wes W \textbf{2} ichz] ich W \textbf{3} wol] [*]: O wol V O wol W  $\cdot$ diu] der T  $\cdot$ lac] gelag W \textbf{5} sol] so wil V  $\cdot$ danne] \textit{om.} V rechte W \textbf{6} unde] \textit{om.} W \textbf{7} dînen] deiner W \textbf{10} bî im rîter] do bi im ritter V do ritter bei im W \textbf{12} des] Das W  $\cdot$ mohte] [*]: moͤhte V moͤchte W \textbf{15} alrîtende] Al reiten W \textbf{16} Obylot] Obilot V \textbf{21} unde hân] Han ich W  $\cdot$ ich] \textit{om.} V \textbf{22} mir] [i*]: ich V \textbf{23} er] ir W  $\cdot$ hât] habt W \textbf{24} sô] nun W \textbf{25} im niht] \textit{om.} W \textbf{26} megde] kinde W \textbf{29} \textit{Die Verse 373.29-30 fehlen} T W   $\cdot$ Sit dv dienstes von im gerst V \textbf{30} Jch gibe dir daz dv in [*]: wol gewerst V \newline
\end{minipage}
\end{table}
\end{document}
