\documentclass[8pt,a4paper,notitlepage]{article}
\usepackage{fullpage}
\usepackage{ulem}
\usepackage{xltxtra}
\usepackage{datetime}
\renewcommand{\dateseparator}{.}
\dmyyyydate
\usepackage{fancyhdr}
\usepackage{ifthen}
\pagestyle{fancy}
\fancyhf{}
\renewcommand{\headrulewidth}{0pt}
\fancyfoot[L]{\ifthenelse{\value{page}=1}{\today, \currenttime{} Uhr}{}}
\begin{document}
\begin{table}[ht]
\begin{minipage}[t]{0.5\linewidth}
\small
\begin{center}*D
\end{center}
\begin{tabular}{rl}
\textbf{711} & \textbf{\begin{large}A\end{large}rnive} z\textbf{eime} junchêrrelîn\\ 
 & \textbf{sprach}: "\textbf{nû} sage dem sune mîn,\\ 
 & daz er mich balde \textbf{spreche}\\ 
 & unt daz aleine zeche."\\ 
5 & der knabe \textbf{Artusen brâhte}.\\ 
 & \textbf{Arnive} des gedâhte,\\ 
 & si wolde\textbf{z} in lâzen hœren,\\ 
 & ob er m\textit{ö}hte \textbf{zerstœren},\\ 
 & \textbf{nâch wem} der clâren Itonje\\ 
10 & \textbf{was} sô herzenlîche wê.\\ 
 & Des \textbf{künec} Gramoflanzes kint\\ 
 & \textbf{nâch Artuse} komen sint.\\ 
 & die erbeizten ûf dem velde.\\ 
 & vor dem kleinem gezelde\\ 
15 & einer Benen sitzen sach\\ 
 & bî der, diu zArtuse sprach:\\ 
 & "giht des diu herzogîn vür prîs,\\ 
 & ob \textbf{mîn bruoder mir} mîn âmîs\\ 
 & sleht durch ir lôsen rât,\\ 
20 & des \textbf{m\textit{ö}ht er} jehen vür missetât.\\ 
 & waz hât der künec im getân?\\ 
 & \textbf{er solt} in mîn geniezen lân.\\ 
 & Treit mîn bruoder sinne,\\ 
 & er weiz unser zweier minne\\ 
25 & sô lûter âne \textbf{truopheit};\\ 
 & pfligt er triwe, ez wirt im leit.\\ 
 & sol \textbf{mir sîn} hant erwerben\\ 
 & nâch dem künege ein sûrez sterben?\\ 
 & \begin{large}H\end{large}êrre, daz sî iu geklagt",\\ 
30 & sprach zArtuse diu süeze magt.\\ 
\end{tabular}
\scriptsize
\line(1,0){75} \newline
D \newline
\line(1,0){75} \newline
\textbf{1} \textit{Initiale} D  \textbf{11} \textit{Majuskel} D  \textbf{23} \textit{Majuskel} D  \textbf{29} \textit{Initiale} D  \newline
\line(1,0){75} \newline
\textbf{8} möhte] mohte D \textbf{9} Itonje] Jtoniê D \textbf{11} Gramoflanzes] Gramoflanzs D \textbf{20} möht] moht D \newline
\end{minipage}
\hspace{0.5cm}
\begin{minipage}[t]{0.5\linewidth}
\small
\begin{center}*m
\end{center}
\begin{tabular}{rl}
 & \textbf{Arune} zuo \textbf{einem} junchêrrelîn\\ 
 & \textbf{sprach}: "\textbf{nû} sage dem sun mîn,\\ 
 & daz er mich balde \textbf{spreche}\\ 
 & und daz alein \dag reche\dag ."\\ 
5 & der knappe \textbf{Artusen brâhte}.\\ 
 & \textbf{Arune} des gedâhte,\\ 
 & si wolte \textbf{ez} in lâzen hœren,\\ 
 & ob er\textbf{z} m\textit{ö}hte \textbf{erstœren},\\ 
 & \textbf{nâch wem} der clâren Ithonie\\ 
10 & \textbf{was} sô herzenlîchen wê.\\ 
 & \hspace*{-.7em}\big| \textbf{innen des ouch} komen sint\\ 
 & \hspace*{-.7em}\big| des \textbf{künic} Gramolantzes kint.\\ 
 & die erbeizten ûf dem velde.\\ 
 & vo\textit{r} dem kleinen gezelde\\ 
15 & einer Benen sitzen sach\\ 
 & bî der, diu  Artuse sprach:\\ 
 & "\textit{g}iht des diu herzogîn vür prîs,\\ 
 & ob \textbf{mir mîn bruoder} mîn âmîs\\ 
 & sleht durch ir lôsen rât,\\ 
20 & des \textbf{m\textit{ö}hte e\textit{r}} jehen vür missetât.\\ 
 & waz het der künic im getân?\\ 
 & \textbf{er solte} in mîn geniezen lân.\\ 
 & treit mîn bruoder sinne,\\ 
 & er weiz u\textit{ns}er zweier minne\\ 
25 & sô lûter âne \textbf{truopheit};\\ 
 & pfliget e\textit{r} triuwe, ez wirt im leit.\\ 
 & sol \textbf{mîn} hant erwerben\\ 
 & nâch dem künige ein sûrez sterben?\\ 
 & hêrre, daz sî iu geklaget",\\ 
30 & sprach \textit{zuo} Artuse diu süeze maget.\\ 
\end{tabular}
\scriptsize
\line(1,0){75} \newline
m n o \newline
\line(1,0){75} \newline
\newline
\line(1,0){75} \newline
\textbf{1} Arune] Arniwe n Arnive o \textbf{6} Arune] Arniwe n Arnive o  $\cdot$ gedâhte] bedachte n \textbf{8} möhte] mohtte m (o)  $\cdot$ erstœren] zerstoͯren n (o) \textbf{9} Ithonie] jthonie m jtonie n (o) \textbf{10} herzenlîchen] hertzeclichen n (o) \textbf{11} Gramolantzes] gramolatzes o \textbf{13} velde] folde o \textbf{14} vor] Von m \textbf{17} giht] Siht m Git o \textbf{19} sleht] Flecht o \textbf{20} möhte er] mohtte es m mochte er o \textbf{21} waz] Wes n \textbf{24} unser] [vs]: vsser m \textbf{26} er] es m  $\cdot$ triuwe] \textit{om.} o  $\cdot$ wirt] wurnt n nuͯ es wirt o \textbf{27} mîn] min sin n (o) \textbf{28} ein sûrez] an sines o \textbf{30} zuo] sẏ m \newline
\end{minipage}
\end{table}
\newpage
\begin{table}[ht]
\begin{minipage}[t]{0.5\linewidth}
\small
\begin{center}*G
\end{center}
\begin{tabular}{rl}
 & \textbf{\begin{large}A\end{large}rnive} \textbf{sprach} z\textbf{einem} junchêrrelîn:\\ 
 & "sage dem \textbf{lieben} sune mîn,\\ 
 & daz er mich balde \textbf{spreche}\\ 
 & unde daz aleine zeche."\\ 
5 & der knappe \textbf{ze Artuse gâhte}.\\ 
 & \textbf{Arnive} des gedâhte,\\ 
 & si wolde in lâzen hœren,\\ 
 & ob er m\textit{ö}hte \textbf{zerstœren},\\ 
 & \textbf{daz} der clâren Itonie\\ 
10 & \textbf{tet} sô herzenlîche wê.\\ 
 & des \textbf{künec} Gramoflanzes kint\\ 
 & \textbf{nâch Artuse} komen sint.\\ 
 & die erbeizten ûf dem velde.\\ 
 & vor dem kleinen gezelde\\ 
15 & einer Benen sitzen sach\\ 
 & bî der, diu ze Artuse sprach:\\ 
 & "giht des diu herzogîn vür brîs,\\ 
 & ob \textbf{mîn bruoder} mîn âmîs\\ 
 & sleht durch ir lôsen rât,\\ 
20 & des \textbf{m\textit{ö}ht ir} jehen vür missetât.\\ 
 & \hspace*{-.7em}\big| \textbf{ir sült} in mîn geniezen lân.\\ 
 & \hspace*{-.7em}\big| waz hât der künec im getân?\\ 
 & treit mîn bruoder sinne,\\ 
 & er weiz unser zweier minne\\ 
25 & sô lûter âne \textbf{valscheit};\\ 
 & pfliget er triwe, ez wirt im leit.\\ 
 & sol \textbf{mir sîn} hant erwerben\\ 
 & nâch dem künege ein sûrez sterben?\\ 
 & hêrre, daz sî iu geklaget",\\ 
30 & sprach ze Artuse diu süeze maget.\\ 
\end{tabular}
\scriptsize
\line(1,0){75} \newline
G I L M Z Fr18 Fr22 \newline
\line(1,0){75} \newline
\textbf{1} \textit{Initiale} G I L Z Fr18  \textbf{21} \textit{Initiale} I  \textbf{23} \textit{Initiale} M  \newline
\line(1,0){75} \newline
\textbf{1} Arnive] ARniue I ARnẏue Fr18  $\cdot$ sprach] \textit{om.} L M Z Fr18 \textbf{2} sage] nu sage I Sprach nv L (M) (L) (Z) (Fr18)  $\cdot$ lieben] \textit{om.} I L M Z Fr18 \textbf{3} spreche] gespreche I M Z (Fr18) sprache L \textbf{4} zeche] gezeche Z \textbf{5} Artuse] artus Z \textbf{6} Arnive] Arniua I ARniue Fr18  $\cdot$ des] do L \textbf{8} möhte] mohte G (I) (L) (M) Z (Fr18) \textbf{9} daz] Nach wem L (M) Z Fr18  $\cdot$ Itonie] Jconie Z ẏto::: Fr18 \textbf{10} tet] Waz L (M) (Z) (Fr18)  $\cdot$ herzenlîche] hertzeclichen L (M) \textbf{11} künec] chunges I  $\cdot$ Gramoflanzes] Gramorflanzes M gramoflantzes Z \textbf{12} Artuse] artus Z \textbf{14} dem kleinen] einem chleinem I  $\cdot$ gezelde] gezerde Fr18 \textbf{16} der] einer I  $\cdot$ diu ze] zcu M die Z  $\cdot$ Artuse] artus Z \textbf{17} vür] zv Z \textbf{19} lôsen] bosen I \textbf{20} möht] moht G I (M) Z Fr18 mochte L \textbf{22} ir sült] Er sol L Er sold Z \textbf{23} sinne] Getriwe sinne I \textbf{24} weiz] weiz wol I  $\cdot$ zweier] beider I [zwerer]: zweier L \textbf{26} pfliget] hat I (Fr18)  $\cdot$ er] ir M \textbf{28} sûrez] \textit{om.} L Fr18 swers M (Z) \textbf{30} Artuse] Artus I (Z)  $\cdot$ diu süeze] [ein]: diu shonev I die L die klare Z \newline
\end{minipage}
\hspace{0.5cm}
\begin{minipage}[t]{0.5\linewidth}
\small
\begin{center}*T
\end{center}
\begin{tabular}{rl}
 & \textbf{Arnyve} zuo \textbf{dem} junchêrrelîn\\ 
 & \textbf{sprach}: "\textbf{nû} sage dem sune mîn,\\ 
 & daz er mich balde \textbf{gespreche}\\ 
 & und daz aleine zeche."\\ 
5 & der knappe \textbf{zuo Artuse gâhete}.\\ 
 & \textbf{Arnyve} des gedâhte,\\ 
 & s\textit{i} wolt in lâzen hœren,\\ 
 & ob er möhte \textbf{\textit{ze}rstœren},\\ 
 & \textbf{nâch wem} der clâren Itonie\\ 
10 & \textbf{wære} sô herzeclîche wê.\\ 
 & des \textbf{küneges} Gramoflanzes kint\\ 
 & \textbf{nâch Artuse} komen sint.\\ 
 & die erbeizten ûf dem velde.\\ 
 & vor dem kleinen gezelde\\ 
15 & einer Benen sitzen sach\\ 
 & bî der, diu zuo Artuse sprach:\\ 
 & "giht des \textit{diu} herzogîn vür prîs,\\ 
 & o\textit{b} \textbf{mîn bruoder} mîn âmîs\\ 
 & sleht durch ir lôsen rât,\\ 
20 & des \textbf{möht er} jehen vür missetât.\\ 
 & \hspace*{-.7em}\big| \textbf{er solte} in mîn geniezen lân.\\ 
 & \hspace*{-.7em}\big| waz hât der künec im getân?\\ 
 & treit mîn bruoder sinne,\\ 
 & er weiz unser zweier minne\\ 
25 & sô lûter âne \textbf{valscheit};\\ 
 & pfliget er triuwe, \textit{ez wir}t im leit.\\ 
 & sol \textbf{mir sîn} hant erwerben\\ 
 & nâch dem künege ein sû\textit{r}e\textit{z} sterben?\\ 
 & hêrre, daz sî iu geklaget",\\ 
30 & sprach zuo Artuse diu süeze maget.\\ 
\end{tabular}
\scriptsize
\line(1,0){75} \newline
U V W Q R \newline
\line(1,0){75} \newline
\newline
\line(1,0){75} \newline
\textbf{1} Arnyve] Arniue V Q Arnyue W R  $\cdot$ zuo dem] zvͦ eime V \textit{om.} Q zu den zwein R \textbf{2} sage] sagt R \textbf{3} mich balde] balde mich V  $\cdot$ gespreche] bespreche W \textbf{4} zeche] reche W gezeche Q speche R \textbf{5} gâhete] [*]: gahte V \textbf{6} [Arniue*]: Arniuen die des gedachte V  $\cdot$ Arnyve] Arnyue W R Arniue Q \textbf{7} si] So U \textbf{8} ob er] [*]: Ob erz V  $\cdot$ möhte] mochte U Q  $\cdot$ zerstœren] zuͦ der storen U \textbf{9} Itonie] Jtonie U ẏtonie V ytonie W Q Jtonye R \textbf{10} wære] Was W (Q) R  $\cdot$ herzeclîche] hertzelichen W (Q) \textbf{11} Gramoflanzes] gramaflanzes V gramoflantzes W Q Gramoflanczes R \textbf{12} [Jr *men]: nach artuse komen sint V  $\cdot$ Artuse] artus Q (R)  $\cdot$ sint] die kind R \textbf{14} kleinen] kleinem Q \textbf{16} [B*]: Bi der die zvͦ artuse sprach V  $\cdot$ Artuse] Artusen R \textbf{17} des] das W R  $\cdot$ diu] \textit{om.} U \textbf{18} ob mîn] Oder min U [Ob min*]: Ob mir min V \textbf{20} möht] moͯch R \textbf{21} hât] het V (R) \textbf{23} sinne] rechte sinne W \textbf{26} ez wirt] ist U \textbf{28} dem] \textit{om.} Q  $\cdot$ sûrez] suͦze U \textbf{29} sî] \textit{om.} U \textbf{30} süeze] reine W \newline
\end{minipage}
\end{table}
\end{document}
