\documentclass[8pt,a4paper,notitlepage]{article}
\usepackage{fullpage}
\usepackage{ulem}
\usepackage{xltxtra}
\usepackage{datetime}
\renewcommand{\dateseparator}{.}
\dmyyyydate
\usepackage{fancyhdr}
\usepackage{ifthen}
\pagestyle{fancy}
\fancyhf{}
\renewcommand{\headrulewidth}{0pt}
\fancyfoot[L]{\ifthenelse{\value{page}=1}{\today, \currenttime{} Uhr}{}}
\begin{document}
\begin{table}[ht]
\begin{minipage}[t]{0.5\linewidth}
\small
\begin{center}*D
\end{center}
\begin{tabular}{rl}
\textbf{413} & mîn vrouwe Antikonie,\\ 
 & vor \textbf{valscheit} diu vrîe,\\ 
 & \begin{large}D\end{large}ort al weinende bî \textbf{im} stêt.\\ 
 & ob iu daz niht ze herzen gêt,\\ 
5 & sît iuch bêde ein muoter truoc,\\ 
 & sô \textbf{gedenket}, hêrre, ob ir sît kluoc,\\ 
 & ir sandet in der meide her.\\ 
 & wære niemen sînes geleites wer,\\ 
 & er solte \textbf{ie}doch durch si genesen."\\ 
10 & der künec liez einen vride wesen,\\ 
 & unz er sich baz bespræche,\\ 
 & wie er sînen vater \textbf{ræche}.\\ 
 & Unschuldec was hêr Gawan.\\ 
 & ez hete ein ander hant getân,\\ 
15 & wande der stolze Ehcunat\\ 
 & eine lanzen durch in lêrte pfat,\\ 
 & dô er Jofreiden fiz Idœl\\ 
 & vuorte gegen Barbigœl,\\ 
 & den er bî Gawane vienc.\\ 
20 & durch \textbf{den disiu} nôt \textbf{ergienc}.\\ 
 & Dô der vride wart getân,\\ 
 & daz volc huop sich von strîte sân,\\ 
 & manneglîch ze\textbf{n} herbergen sîn.\\ 
 & Antikonie, diu künegîn,\\ 
25 & ir vetern sun vaste umbevienc.\\ 
 & manec kus an sînen munt ergienc,\\ 
 & daz er Gawanen het ernert\\ 
 & \textbf{und} sich selben untât erwert.\\ 
 & si sprach: "dû bist mînes vetern sun.\\ 
30 & dû \textbf{kundest} durch niemen missetuon."\\ 
\end{tabular}
\scriptsize
\line(1,0){75} \newline
D \newline
\line(1,0){75} \newline
\textbf{3} \textit{Initiale} D  \textbf{13} \textit{Majuskel} D  \textbf{21} \textit{Majuskel} D  \newline
\line(1,0){75} \newline
\textbf{1} Antikonie] Antikonîe D \textbf{17} Jofreiden] Jofreyden D  $\cdot$ Idœl] ydoͤl D \textbf{18} Barbigœl] Barbigoͤl D \textbf{24} Antikonie] Antykonie D \textbf{29} vetern] [veter]: vetern D \newline
\end{minipage}
\hspace{0.5cm}
\begin{minipage}[t]{0.5\linewidth}
\small
\begin{center}*m
\end{center}
\begin{tabular}{rl}
 & mîn vrouwe Anticonie,\\ 
 & vor \textbf{valsch} diu vrîe,\\ 
 & dort al weinend bî \textbf{im} stât.\\ 
 & ob iu daz niht ze herzen gât,\\ 
5 & sît iuch beide ein muoter truoc,\\ 
 & sô \textbf{gedenket}, hêrre, ob ir sît kluoc,\\ 
 & ir santet i\textit{n} der megde her.\\ 
 & wær niemen sînes geleites wer,\\ 
 & er solte \textbf{ie}doch durch si genesen."\\ 
10 & der künic liez einen vride wesen,\\ 
 & unz er sich baz bespræche,\\ 
 & wie er sînen vater \textbf{ræche}.\\ 
 & unschuldic was hêr Gawan.\\ 
 & ez hette ein ander hant getân,\\ 
15 & want der stolze Ehkunacht\\ 
 & eine lan\textit{z}en durch in lêrte pfat,\\ 
 & dô er Jofreiden fir Idol\\ 
 & vuorte gegen Parbigol,\\ 
 & den er bî Gawane vienc.\\ 
20 & durch \textbf{den dis\textit{iu}} nôt \textbf{ergienc}.\\ 
 & dô der vride wart getân,\\ 
 & d\textit{az} volc huop sich von strîte sân,\\ 
 & mengelîch ze\textbf{n} herbergen sîn.\\ 
 & Anticonie, diu künigîn,\\ 
25 & ir vetern sun vaste umbevienc.\\ 
 & manic kus an sînen munt ergienc,\\ 
 & daz er Gawanen hette ernert\\ 
 & \textbf{und} sich selben untât erwert.\\ 
 & si sprach: "dû bist mînes vetern sun.\\ 
30 & dû \textbf{en}\textbf{kundest} durch niemen missetuon."\\ 
\end{tabular}
\scriptsize
\line(1,0){75} \newline
m n o \newline
\line(1,0){75} \newline
\newline
\line(1,0){75} \newline
\textbf{1} Anticonie] antitonie o \textbf{6} sô] Des n o \textbf{7} santet] santen n (o)  $\cdot$ in] ym m \textbf{10} einen] in n \textit{om.} o \textbf{12} sînen] sin o  $\cdot$ ræche] gereche n o \textbf{13} Gawan] gawann o \textbf{15} want] Wenne n Do an o  $\cdot$ stolze] stoltz n  $\cdot$ Ehkunacht] echkumacht o \textbf{16} eine] Ein n o  $\cdot$ lanzen] lanren m lantze n  $\cdot$ pfat] pfacht n \textbf{17} Jofreiden] jotfriden n o  $\cdot$ fir Idol] firidol n o \textbf{20} den] den den n  $\cdot$ disiu] dise m n o \textbf{22} daz] Do m \textbf{23} mengelîch] Menlich n o  $\cdot$ zen herbergen] zuͯ der herberge n (o) \textbf{24} Anticonie] Antitonie o \textbf{27} Gawanen] gawane n o \textbf{28} selben] selb n o  $\cdot$ untât] anstat o \newline
\end{minipage}
\end{table}
\newpage
\begin{table}[ht]
\begin{minipage}[t]{0.5\linewidth}
\small
\begin{center}*G
\end{center}
\begin{tabular}{rl}
 & mîn vrouwe Antikonie,\\ 
 & vor \textbf{valscheit} diu vrîe,\\ 
 & dort alweinde bî \textbf{iu} stêt.\\ 
 & obe iu daz niht ze herzen gêt,\\ 
5 & sît iuch bêde ein muoter truoc,\\ 
 & sô \textbf{denket}, hêrre, obe ir sît kluoc.\\ 
 & ir sandet in der magede her.\\ 
 & wære niemen sînes geleites wer,\\ 
 & er solt \textbf{ie}doch dur si genesen."\\ 
10 & der künec liez einen vride wesen,\\ 
 & unze er sich baz bespræche,\\ 
 & wier sînen vater \textbf{ræche}.\\ 
 & unschuldec was hêr Gawan.\\ 
 & ez hete ein ander hant getân,\\ 
15 & wan der stolze Ehkunat\\ 
 & eine lanzen durch in lêrte pfat,\\ 
 & dô er Jofreiden fiz Idol\\ 
 & vuorte gegen Barbigol,\\ 
 & den er bî Gawane vienc.\\ 
20 & dur \textbf{den disiu} nôt \textbf{ergienc}.\\ 
 & dô der vride wart getân,\\ 
 & daz volc huop sich von strîte sân,\\ 
 & manegelîch ze herbergen sîn.\\ 
 & Antikonie, diu künegîn,\\ 
25 & ir veteren sun vaste umbevienc.\\ 
 & manec kus an sînen munt ergienc,\\ 
 & daz er Gawanen het ernert,\\ 
 & sich selben untât erwert.\\ 
 & si sprach: "dû bist mînes vetern sun.\\ 
30 & dû\textbf{ne} \textbf{kanst} dur niemen missetuon."\\ 
\end{tabular}
\scriptsize
\line(1,0){75} \newline
G I O L M Q R Z \newline
\line(1,0){75} \newline
\textbf{3} \textit{Initiale} O L M Z   $\cdot$ \textit{Capitulumzeichen} R  \textbf{5} \textit{Initiale} I  \textbf{19} \textit{Initiale} I  \newline
\line(1,0){75} \newline
\textbf{1} mîn] Mir M  $\cdot$ Antikonie] anticonie I [antiko*]: antikonŷe O Amkonie L antichonie M anrigonie Q anteconye R \textbf{2} valscheit] valsheit gar I valsche M \textbf{3} dort] ÷ort O Dor Q Die doͯrt R  $\cdot$ bî iu] bi in I \textit{om.} O bẏ im L (M) (Q) (R) (Z) \textbf{4} iu daz] das euch Q \textbf{5} bêde] beidu R \textbf{6} denket] gedenchet O L (M) (Q) (R) (Z)  $\cdot$ hêrre] \textit{om.} O  $\cdot$ ir] \textit{om.} L \textbf{7} sandet] sendet Q (R) \textbf{8} wære niemen sînes] Ob er [niemes]: niemens O Wer nymant sein Q  $\cdot$ wer] mer Q \textbf{9} solt] soͯlt R  $\cdot$ iedoch] doch I  $\cdot$ genesen] [genissen]: genessen Q \textbf{10} liez] \textit{om.} L \textbf{11} baz] vol R  $\cdot$ bespræche] bespreche I Q Z gespreche M verspreche R \textbf{12} wier] Vnd wy er Q  $\cdot$ sînen] siner I  $\cdot$ ræche] reche I L M Q R Z \textbf{13} hêr] er M Q  $\cdot$ Gawan] Gawin R \textbf{14} hant] man I Z \textbf{15} Ehkunat] ekunat G hekunat I Ahkvnat O ekúnat Q Ehkunant R eckunat Z \textbf{16} eine] Ein O L Q R Z  $\cdot$ lanzen durch in lêrte] langen durch Jn leren R lantzen lerte durch in Z  $\cdot$ pfat] pfant R \textbf{17} dô] Da Z  $\cdot$ Jofreiden] lofreiden G (M) iofride I Jofrieden L Jofriden R Z  $\cdot$ Idol] ydol O \textbf{18} Barbigol] barbidol I (L) \textbf{19} er] \textit{om.} Q [ir]: er Z  $\cdot$ Gawane] Gawan I (Z) Gawanen O Gaweinen R \textbf{20} nôt] tiost L \textbf{21} \textit{Die Verse 413.21-26 fehlen} M   $\cdot$ dô] Da Z \textbf{22} huop sich] gie I hvͦbe sich O  $\cdot$ sân] dan L R \textbf{23} manegelîch] Jegelich Z  $\cdot$ ze] zeder I (R) (Z) zen O (L) (Q)  $\cdot$ herbergen] herberge I (R) (Z) \textbf{24} Antikonie] Anticonia I Antykonye O Antykonie R \textbf{25} ir] Jrs R  $\cdot$ veteren sun] veter svn O veters sun R \textbf{26} sînen] sinem O sind R \textbf{27} daz] Der M  $\cdot$ er] der R  $\cdot$ Gawanen] Gawan I O [ga*]: gawanen Q Gawinen R \textbf{28} sich] vnd sich I (O) (L) (M) (Q) (R) (Z)  $\cdot$ selben] selben het I selber R \textbf{29} si] Dy M  $\cdot$ bist] binst Q  $\cdot$ sun] [zvn]: svn Z \textbf{30} dûne] du I (O) (R)  $\cdot$ kanst] chvndest O (L) (M) (Q) (R) (Z) \newline
\end{minipage}
\hspace{0.5cm}
\begin{minipage}[t]{0.5\linewidth}
\small
\begin{center}*T
\end{center}
\begin{tabular}{rl}
 & mîn vrouwe Antickonie,\\ 
 & vor \textbf{valsche} diu vrîe,\\ 
 & dort alweinde bî \textbf{im} stât.\\ 
 & ob iu daz niht ze herzen gât,\\ 
5 & sît iu beide ein muoter truoc,\\ 
 & sô \textbf{gedenket}, hêrre, ob ir sît kluoc,\\ 
 & ir santet in der megede her.\\ 
 & wære nieman sînes geleites wer,\\ 
 & er solte doch durch si genesen."\\ 
10 & Der künec liez einen vride wesen,\\ 
 & unz er sich baz bespræche,\\ 
 & wier sînen vater \textbf{geræche}.\\ 
 & \begin{large}U\end{large}nschuldic was hêr Gawan.\\ 
 & ez hete ein ander hant getân,\\ 
15 & wand der stolze Ehcunat\\ 
 & ein lanze durch in lêrte pfat,\\ 
 & dô er Joffriden fis Idol\\ 
 & vuorte gegen Barbigol,\\ 
 & den er bî Gawane vienc.\\ 
20 & durch \textbf{dise} nôt \textbf{er gienc}.\\ 
 & Dô der vride wart getân,\\ 
 & daz volc huop sich von strîte sân,\\ 
 & maneglîch ze\textbf{n} herbergen sîn.\\ 
 & Antickonie, diu künegîn,\\ 
25 & ir vetern sun vaste umbevienc.\\ 
 & manec kus an sînen munt ergienc,\\ 
 & daz er Gawanen hete ernert\\ 
 & \textbf{und} sich selben untât erwert.\\ 
 & si sprach: "dû bist mînes vetern suon.\\ 
30 & dû\textbf{ne} \textbf{kundes\textit{t}} durch nieman missetuon."\\ 
\end{tabular}
\scriptsize
\line(1,0){75} \newline
T U V W \newline
\line(1,0){75} \newline
\textbf{3} \textit{Initiale} V W  \textbf{13} \textit{Initiale} T U  \textbf{21} \textit{Majuskel} T  \newline
\line(1,0){75} \newline
\textbf{1} Antickonie] Antikonie U (V) (W) \textbf{4} ob] Oder U \textbf{8} geleites] geleiter W \textbf{9} doch] iedoch V \textbf{10} vride] vriden U \textbf{11} unz] Mit U \textbf{12} geræche] reche U V W \textbf{14} hant] man W \textbf{15} wand] Wan U V (W)  $\cdot$ Ehcunat] Eckuͦnat U elẏkvnat V ekunat W \textbf{17} Joffriden fis Idol] Joffriden fisidol T Jofreiden fisidol U iofriden fisidol V W \textbf{18} Barbigol] Barbygol T \textbf{20} durch] Durch den U V W \textbf{22} volc] \textit{om.} U \textbf{23} zen] zvͦr V (W)  $\cdot$ herbergen] herberge W \textbf{24} Antickonie] Anticonie U Antikonie V W \textbf{25} ir] Jrs U V (W) \textbf{27} daz] Do W \textbf{28} erwert] ernert U \textbf{30} dûne] Du W \newline
\end{minipage}
\end{table}
\end{document}
