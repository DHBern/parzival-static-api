\documentclass[8pt,a4paper,notitlepage]{article}
\usepackage{fullpage}
\usepackage{ulem}
\usepackage{xltxtra}
\usepackage{datetime}
\renewcommand{\dateseparator}{.}
\dmyyyydate
\usepackage{fancyhdr}
\usepackage{ifthen}
\pagestyle{fancy}
\fancyhf{}
\renewcommand{\headrulewidth}{0pt}
\fancyfoot[L]{\ifthenelse{\value{page}=1}{\today, \currenttime{} Uhr}{}}
\begin{document}
\begin{table}[ht]
\begin{minipage}[t]{0.5\linewidth}
\small
\begin{center}*D
\end{center}
\begin{tabular}{rl}
\textbf{227} & unt hiez mich zuo \textbf{z}iu rîten în."\\ 
 & "\textbf{hêrre}, \textbf{ir sult} willekomen sîn,\\ 
 & sît \textbf{ez} der vischære verjach,\\ 
 & man biutet iu êre unt gemach\\ 
5 & durch in, der iuch sande wider",\\ 
 & sprach der knappe unt lie die brücke nider.\\ 
 & In die burc der küene reit\\ 
 & ûf einen hof wît unt breit.\\ 
 & \textbf{durch} schimpf \textbf{er} niht zertret was.\\ 
10 & dâ stuont al kurz \textbf{grüene} gras.\\ 
 & dâ was \textbf{bûhurdieren} vermiten,\\ 
 & mit banieren selten überriten\\ 
 & \textbf{alsô} der anger z\textbf{Abenberc}.\\ 
 & selten vrœlîchiu werc\\ 
15 & \textbf{was dâ gevrümt} ze \textbf{langer} stunt:\\ 
 & in was \textbf{wol} \textbf{herzen jâmer} kunt.\\ 
 & \textbf{Wênec er des} gein in engalt.\\ 
 & in enpfiengen ritter jung unt alt.\\ 
 & \textbf{vil kleiner} junchêrrelîn\\ 
20 & sprungen gein dem zoume sîn.\\ 
 & ieslîchez vürz ander greif,\\ 
 & \textbf{si habten sînen} stegreif.\\ 
 & Sus muoser von dem orse stên.\\ 
 & in bâten ritter vürbaz gên;\\ 
25 & \textbf{die} vuorten in an sîn gemach.\\ 
 & \textbf{harte} schiere daz geschach,\\ 
 & daz er mit zuht entwâpent wart.\\ 
 & dô si den jungen âne bart\\ 
 & gesâhen alsus minneclîch,\\ 
30 & si jâhen, er wære sælden rîch.\\ 
\end{tabular}
\scriptsize
\line(1,0){75} \newline
D \newline
\line(1,0){75} \newline
\textbf{7} \textit{Majuskel} D  \textbf{17} \textit{Majuskel} D  \textbf{23} \textit{Majuskel} D  \newline
\line(1,0){75} \newline
\textbf{13} zAbenberc] zAbenberch D \newline
\end{minipage}
\hspace{0.5cm}
\begin{minipage}[t]{0.5\linewidth}
\small
\begin{center}*m
\end{center}
\begin{tabular}{rl}
 & und hiez mich zuo i\textit{u} rîten în."\\ 
 & "\textbf{hêrre}, \textbf{ir sullet} wilkomen sîn,\\ 
 & sît \textbf{es} der vischære verjach,\\ 
 & man biutet iu êre und gemach\\ 
5 & durch in, der \textit{iuch} sande wider",\\ 
 & sprach der knabe und liez die brücke nider.\\ 
 & in die burc der küene reit\\ 
 & ûf eine\textit{n} hof wît und breit.\\ 
 & \textbf{durch} schimpf \textbf{er} niht zertret was.\\ 
10 & d\textit{â} stuont al kurz \textbf{grüene} gras.\\ 
 & dâ was \textbf{bûh\textit{u}rdieren} vermiten,\\ 
 & mit banieren selten überriten\\ 
 & \textbf{alsô} der anger ze \textbf{Abenberc}.\\ 
 & selten vrœlîchiu werc\\ 
15 & \textbf{was dâ gevrumt} ze \textbf{langer} stunt:\\ 
 & in was \textbf{wol} \textbf{he\textit{rz}ejâmer} kunt.\\ 
 & \textbf{wênic er des} gegen \textit{in} engalt.\\ 
 & in enpfiengen ritter junc und alt.\\ 
 & \textbf{vil kleiner} junchêrrelîn\\ 
20 & spr\textit{u}ngen gegen dem zoume sîn.\\ 
 & ieglîche\textit{z} vür daz ander greif\\ 
 & \textbf{zuo zoum und an} stegreif.\\ 
 & sus muos er von dem rosse stên.\\ 
 & in bâten ritter vürbaz gên;\\ 
25 & \textbf{die} vuorten in an sîn gemach.\\ 
 & \textbf{harte} schiere daz geschach,\\ 
 & daz er mit zühte entwâpent wart.\\ 
 & dô si den jungen âne bart\\ 
 & gesâhen alsô minneclîch,\\ 
30 & si jâhen, er wær sælden rîch.\\ 
\end{tabular}
\scriptsize
\line(1,0){75} \newline
m n o Fr69 \newline
\line(1,0){75} \newline
\newline
\line(1,0){75} \newline
\textbf{1} hiez] hiesse n  $\cdot$ iu] in m \textbf{5} iuch] \textit{om.} m n o \textbf{8} einen] einem m (o) \textbf{9} zertret] zuͯ trege n \textbf{10} dâ] Do m n o \textbf{11} dâ] Du n Do o  $\cdot$ bûhurdieren] buhardier en m (n) buwerharderen o \textbf{13} Abenberc] abenberg m oben berg n obenberg o \textbf{14} vrœlîchiu] frolich o \textbf{15} dâ] do n o \textbf{16} wol] \textit{om.} n  $\cdot$ herzejâmer] heisse iamer m (n) (o) \textbf{17} des] es o  $\cdot$ in] \textit{om.} m \textbf{20} sprungen] Springen m \textbf{21} ieglîchez] Yeglicher m (n) \textbf{22} zuo] An n o Fr69 \textbf{23} muos] muͯste n o \textbf{24} in] Jna n (o) \newline
\end{minipage}
\end{table}
\newpage
\begin{table}[ht]
\begin{minipage}[t]{0.5\linewidth}
\small
\begin{center}*G
\end{center}
\begin{tabular}{rl}
 & unde hiez mich zuo iu rîten în."\\ 
 & "\textbf{hêrre}, \textbf{ir sult} willekomen sîn,\\ 
 & sît \textbf{es} der vischære verjach,\\ 
 & man biut iu êre unde gemach\\ 
5 & \begin{large}D\end{large}urch in, der iuch sande wider",\\ 
 & sprach der knappe unde lie die brücke nider.\\ 
 & in die burc der küene reit\\ 
 & ûf einen hof wît unde breit.\\ 
 & \textbf{mit} schimpfe\textbf{r} niht zertretet was.\\ 
10 & dâ stuont al kurz \textbf{grüene} gras.\\ 
 & dâ was \textbf{bûhurt gar} vermiten,\\ 
 & mit banieren selten übergeriten\\ 
 & \textbf{sô} der anger \textbf{dâ} ze \textbf{Babenberc}.\\ 
 & selten vrœlîchiu werc\\ 
15 & \textbf{was dâ gevrumt} ze \textbf{maniger} stunt:\\ 
 & in was \textbf{wol} \textbf{herzen jâmer} kunt.\\ 
 & \textbf{wênic er des} gein in engalt.\\ 
 & in enpfiengen rîter junc unde alt.\\ 
 & \textbf{vil kleiner} junchêrrelîn\\ 
20 & sprungen gein dem zoume sîn.\\ 
 & ieslîchez vür daz ander greif,\\ 
 & \textbf{si ha\textit{b}ten sînen} stegreif.\\ 
 & sus muoser von dem orse stên.\\ 
 & in bâten rîter vürbaz gên.\\ 
25 & \textbf{si} vuorten in an sîn gemach.\\ 
 & \textbf{harte} schier daz geschach,\\ 
 & daz er mit zuht entwâpent wart.\\ 
 & dô si den jungen âne bart\\ 
 & gesâhen alsô minniclîch,\\ 
30 & si jâhen, er wære sælden rîch.\\ 
\end{tabular}
\scriptsize
\line(1,0){75} \newline
G I O L M Q R Z Fr21 Fr54 \newline
\line(1,0){75} \newline
\textbf{3} \textit{Initiale} Q  \textbf{5} \textit{Initiale} G  \textbf{7} \textit{Initiale} I O L M Fr21  \textbf{9} \textit{Capitulumzeichen} R  \textbf{23} \textit{Initiale} I  \newline
\line(1,0){75} \newline
\textbf{1} mich] \textit{om.} O  $\cdot$ iu] zeuch Q \textbf{3} es] ez I des O \textbf{4} biut] erbivtet O (Q) (Fr21) \textbf{5} in] den L ym M \textbf{6} der knappe] er L (M) Fr21  $\cdot$ brücke] prukken I (M) \textbf{7} in] ÷n O  $\cdot$ burc] [bruk]: burc I brucke Q  $\cdot$ küene] konigk Q \textbf{9} mit] Duͯrch L  $\cdot$ zertretet] zcutreten M zu iret Q \textbf{10} dâ] Do Q  $\cdot$ al] alsz M (R) all do Q \textbf{11} dâ] Do Q  $\cdot$ bûhurt gar] bvhvrdiern O (L) (M) (Q) (Z) (Fr21) buhurdierencz R \textbf{12} mit] vnd mit I  $\cdot$ übergeriten] vberriten I (L) (M) (Q) (R) (Z) (Fr21) \textbf{13} sô] Als O (L) (M) Q R Fr21 Aso Z  $\cdot$ dâ] \textit{om.} O L M Q R Z Fr21  $\cdot$ Babenberc] babenberch G ambenberch O Abenberch L Z abinberc M aben berck Q abenberg R abenberc Fr21 \textbf{15} dâ] do Q  $\cdot$ ze maniger] zelanger O (L) (M) (Z) \textbf{16} wol] wol da M \textit{om.} Z  $\cdot$ herzen] hertze L (R) (Fr21) \textbf{17} gein in] Gein im I an ym M gein Fr21 \textbf{18} enpfiengen] einphiengen G entpfinck Q enpfiegen Fr21 \textbf{19} kleiner] cleinev I  $\cdot$ junchêrrelîn] jungelinc M \textbf{20} zoume] :::n Fr54 \textbf{21} ander greif] and griff R \textbf{22} habten] halten G hildin M  $\cdot$ sînen] im den L Fr54 \textbf{24} in bâten rîter] Div riter in bate::: Fr54 \textbf{25} si] Di Fr54  $\cdot$ in] \textit{om.} R  $\cdot$ sîn] sinen O Z \textbf{26} ::: vil s::: Fr54 \textbf{27} daz] Da Fr21  $\cdot$ er] \textit{om.} R  $\cdot$ mit zuht] mit zuhten I (R) (Fr21) vil schone L  $\cdot$ entwâpent] entphangin M \textbf{28} dô] Da M Z \textbf{29} gesâhen] Sahn O (Q) (Fr21)  $\cdot$ alsô] \textit{om.} I alsvs O (L) (M) (R) (Z) Fr21 (Fr54) \textbf{30} jâhen] dachten M  $\cdot$ sælden] freuden I \newline
\end{minipage}
\hspace{0.5cm}
\begin{minipage}[t]{0.5\linewidth}
\small
\begin{center}*T
\end{center}
\begin{tabular}{rl}
 & unde hiez mich zuo \textbf{z}iu rîten în."\\ 
 & "\textbf{Sô sult ir} willekome sîn,\\ 
 & sît \textbf{ez} der vischære verjach,\\ 
 & man biut iu êre unde gemach\\ 
5 & durch i\textit{n}, der iuch sante wider",\\ 
 & sprach der knappe unde lie \textit{die} brücke nider.\\ 
 & In die burc der küene reit\\ 
 & ûf einen hof wît unde breit.\\ 
10 & \hspace*{-.7em}\big| dâ stuont al kurz \textbf{kleine} gras,\\ 
 & \hspace*{-.7em}\big| \textbf{durch} schimpf \textbf{ez} niht zertretet was.\\ 
 & dâ was \textbf{bûh\textit{u}rdieren} vermiten,\\ 
 & mit banieren selten überriten\\ 
 & \textbf{als} der anger ze \textbf{Abenberc}.\\ 
 & selten vrœlîch\textit{iu} werc\\ 
15 & \textbf{vrumten si} ze \textbf{langer} stunt:\\ 
 & in was \textbf{herzeriuwe} kunt.\\ 
 & \textbf{des er doch wênic} gegen in engalt.\\ 
 & in enpfiengen rîter junc unde alt\\ 
 & \textbf{unde clâre} junchêrrelîn,\\ 
20 & \textbf{die} sprungen gegen dem zoume sîn.\\ 
 & ieslîche\textit{z} vür daz ander greif\\ 
 & \textbf{unde habeten im den} stegreif.\\ 
 & Sus muoser von dem orse st\textit{ê}n.\\ 
 & in bâten \textbf{die} rîter vürbaz gên\\ 
25 & \textbf{unde} vuorten in an sîn gemach.\\ 
 & \textbf{dar nâch vil} schiere daz geschach,\\ 
 & daz er mit zuht entwâpent wart.\\ 
 & dô si den jungen âne bart\\ 
 & gesâhen alsô minneclîch,\\ 
30 & si jâhen, er wære sældenrîch.\\ 
\end{tabular}
\scriptsize
\line(1,0){75} \newline
T U V W \newline
\line(1,0){75} \newline
\textbf{2} \textit{Majuskel} T  \textbf{7} \textit{Initiale} W   $\cdot$ \textit{Majuskel} T  \textbf{23} \textit{Majuskel} T  \newline
\line(1,0){75} \newline
\textbf{1} zuo ziu rîten] riten zuͦ vch U riten zvͦ v́ch hin V zuͦ im reiten W \textbf{2} Sô sult ir] [*]: herre ir sollent V \textbf{5} in] îv T  $\cdot$ iuch] îv T \textbf{6} der knappe] er W  $\cdot$ die] \textit{om.} T \textbf{7} burc] bruck W \textbf{10} dâ] Do U V W  $\cdot$ kurz] [kr*]: kruͦt U  $\cdot$ kleine] cleine vnd U [*]: gruͤne V klames W \textbf{9} zertretet] zuͦ treten U \textbf{11} dâ] Do U V W  $\cdot$ bûhurdieren] bvherdieren T bvhieren V \textbf{12} banieren] banien U  $\cdot$ überriten] v́ber geritten V \textbf{13} Abenberc] babenberg V abenberck W \textbf{14} vrœlîchiu] vroliche T \textbf{15} vrumten si] [*]: Waz do gevrumt V \textbf{16} was] was wol U (V) W  $\cdot$ herzeriuwe] hertzen reúwe W \textbf{17} des] Dez V  $\cdot$ in] im W \textbf{18} rîter junc] iung ritter W \textbf{19} clâre] cleine U (V) (W) \textbf{21} ieslîchez] [ieslicher]: iesliches T Jeclicher U  $\cdot$ daz] den U  $\cdot$ ander] andern U \textbf{22} [*]: An zvͦm vnde an stegereif V \textbf{23} stên] stêin T \textbf{27} zuht] zuchten U (W) \textbf{29} alsô] alsus W \newline
\end{minipage}
\end{table}
\end{document}
