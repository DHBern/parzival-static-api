\documentclass[8pt,a4paper,notitlepage]{article}
\usepackage{fullpage}
\usepackage{ulem}
\usepackage{xltxtra}
\usepackage{datetime}
\renewcommand{\dateseparator}{.}
\dmyyyydate
\usepackage{fancyhdr}
\usepackage{ifthen}
\pagestyle{fancy}
\fancyhf{}
\renewcommand{\headrulewidth}{0pt}
\fancyfoot[L]{\ifthenelse{\value{page}=1}{\today, \currenttime{} Uhr}{}}
\begin{document}
\begin{table}[ht]
\begin{minipage}[t]{0.5\linewidth}
\small
\begin{center}*D
\end{center}
\begin{tabular}{rl}
\textbf{404} & diu dicke von\textbf{me} \textbf{Heitstein}\\ 
 & über al die marke \textbf{schein}.\\ 
 & wol im, derz heinlîche an ir\\ 
 & \textbf{solde prüeven}, \textbf{des} geloubet mir,\\ 
5 & der \textbf{vant} kurzwîle dâ\\ 
 & bezzer denne anderswâ.\\ 
 & Ich mac \textbf{des} von vrouwen jehen,\\ 
 & als mir diu ougen kunnen spehen.\\ 
 & swar ich \textbf{rede kêre} ze guote,\\ 
10 & diu bedarf wol zühte huote.\\ 
 & Nû \textbf{hœre} dise âventiure\\ 
 & der getriwe unt der gehiure.\\ 
 & ich enruoche umbe die ungetriwen.\\ 
 & mit dürkelen \textbf{riwen}\\ 
15 & hânt si alle ir sælekeit verlorn.\\ 
 & des \textbf{muoz} ir \textbf{sêle} lîden zorn.\\ 
 & Ûf den hof \textbf{dort} vür den palas reit\\ 
 & Gawan gein der gesellecheit,\\ 
 & als in der künec sande,\\ 
20 & der sich \textbf{selben} an im \textbf{schande}.\\ 
 & ein ritter, der in brâhte dar,\\ 
 & in vuorte, dâ saz wol gevar\\ 
 & Antikonie, diu künegîn.\\ 
 & sol wîplîch êre sîn gewin,\\ 
25 & des koufes hete si vil gepflegen\\ 
 & unt alles valsches sich bewegen.\\ 
 & dâ mite ir kiusche prîs erwarp.\\ 
 & owê, daz sô vruo erstarp\\ 
 & von Veldeke der wîse man!\\ 
30 & der kunde si baz gelobt hân.\\ 
\end{tabular}
\scriptsize
\line(1,0){75} \newline
D \newline
\line(1,0){75} \newline
\textbf{7} \textit{Majuskel} D  \textbf{11} \textit{Majuskel} D  \textbf{17} \textit{Majuskel} D  \newline
\line(1,0){75} \newline
\textbf{1} Heitstein] heit stein D \textbf{23} Antikonie] Antikonŷe D \textbf{29} Veldeke] Veldekke D \newline
\end{minipage}
\hspace{0.5cm}
\begin{minipage}[t]{0.5\linewidth}
\small
\begin{center}*m
\end{center}
\begin{tabular}{rl}
 & diu dicke von\textbf{\textit{m}e} \textbf{Hertstein}\\ 
 & über alle die marke \textbf{schein}.\\ 
 & wol im, der ez heinlîche an ir\\ 
 & \textbf{dô brüef\textit{e}t}, \textbf{des} geloubet mir,\\ 
5 & der \textbf{vindet} kurzewîle dâ\\ 
 & \textit{be}zzer danne anderswâ.\\ 
 & ich mac \textbf{wol des} von vrouwen jehen,\\ 
 & als mir diu ougen kunnen spehen.\\ 
 & war ich \textbf{rede kêre} ze guote,\\ 
10 & diu bedarf wol zühte huote.\\ 
 & \begin{large}N\end{large}û \textbf{hœre} dise âventiure\\ 
 & der getriuwe und der gehiure.\\ 
 & ich enruoche umb die ungetriuwen.\\ 
 & mit d\textit{ür}kelen \textbf{triuwen}\\ 
15 & hânt si alle ir sælekeit verlorn.\\ 
 & des \textbf{muoze} ir \textbf{sælde} lîden zorn.\\ 
 & ûf den hof vür den palas reit\\ 
 & Gawan gegen der \textit{ge}sellicheit,\\ 
 & als in der künic sante,\\ 
20 & der sich \textbf{sît} an ime \textbf{schante}.\\ 
 & ein ritter, der in brâhte dar,\\ 
 & in vuorte, dâ saz wolgevar\\ 
 & Anticonie, diu künigîn.\\ 
 & sol wîplîch êre sîn gewin,\\ 
25 & des koufes het si vil \textit{ge}pflegen\\ 
 & und alles valsches sich bewegen.\\ 
 & dâ mite ir kiusche prîs erwarp.\\ 
 & ouwê, daz s\textit{ô} v\textit{ruo} erstarp\\ 
 & von Veldecke der wîse man!\\ 
30 & der kunde si baz gelobet hân.\\ 
\end{tabular}
\scriptsize
\line(1,0){75} \newline
m n o \newline
\line(1,0){75} \newline
\textbf{11} \textit{Initiale} m o   $\cdot$ \textit{Capitulumzeichen} n  \newline
\line(1,0){75} \newline
\textbf{1} vonme Hertstein] vonne hert stein m \textbf{4} dô brüefet] Do brieffent m Sol pruͯffen n (o)  $\cdot$ des] dasz o  $\cdot$ geloubet] [pruͯfent]: gleubent o \textbf{5} dâ] do n \textbf{6} bezzer] Vsser m \textbf{9} war] Was n \textbf{10} bedarf] bedaff o \textbf{14} dürkelen] truckelen m  $\cdot$ triuwen] ruwen n o \textbf{15} alle] al o \textbf{16} des] Das n o  $\cdot$ muoze] muͯse m muͯsz n muͦs o  $\cdot$ sælde] sele n o \textbf{17} Vff dem hoff vor dem pallas reit o \textbf{18} gesellicheit] sellikeit m \textbf{22} dâ] so do n do o \textbf{23} Anticonie] Antikonie m \textbf{25} gepflegen] pflegen m \textbf{28} sô vruo] sẏ fuͯr m \textbf{29} Veldecke] feldeg n feldig o \textbf{30} kunde] kuͯnde o \newline
\end{minipage}
\end{table}
\newpage
\begin{table}[ht]
\begin{minipage}[t]{0.5\linewidth}
\small
\begin{center}*G
\end{center}
\begin{tabular}{rl}
 & diu dicke vo\textbf{me} \textbf{Heitstein}\\ 
 & über al die marke \textbf{liehte} \textbf{schein}.\\ 
 & \begin{large}W\end{large}ol im, derz heinlîche an ir\\ 
 & \textbf{sol prüeven}, \textbf{daz} geloubet mir,\\ 
5 & der \textbf{vindet} kurzewîle dâ\\ 
 & bezze\textit{r} \textit{d}anne anderswâ.\\ 
 & ich mac \textbf{des wol} von vrouwen jehen,\\ 
 & als mir diu ougen kunnen spehen.\\ 
 & swar ich \textbf{rede kêre} ze guote,\\ 
10 & diu bedarf wol zühte huote.\\ 
 & nû \textbf{hœret} dise âventiure\\ 
 & der getriwe unde der gehiure.\\ 
 & ich enruoche umbe die ungetriwen.\\ 
 & mit dürkelen \textbf{riwen}\\ 
15 & hânt si alle ir sælicheit verlorn.\\ 
 & des \textbf{muoz} ir \textbf{sêle} lîden zorn.\\ 
 & ûf den hof \textbf{dort} vür den palas reit\\ 
 & Gawan gein der gesellicheit,\\ 
 & als in der künic sande,\\ 
20 & der sich \textbf{selben} an im \textbf{schande}.\\ 
 & ein rîter, der in brâhte dar,\\ 
 & in vuorte, dâ saz \textbf{diu} wol gevar\\ 
 & Antikonie, diu künigîn.\\ 
 & sol wîplîch êre sîn gewin,\\ 
25 & des koufes het si vil gepflegen\\ 
 & unde alles valsches sich bewegen.\\ 
 & dâ mit ir kiusche prîs erwarp.\\ 
 & owê, da\textit{z} \textit{s}ô vruo erstarp\\ 
 & von Veldeke der wîse man!\\ 
30 & der kunde si baz gelobet hân.\\ 
\end{tabular}
\scriptsize
\line(1,0){75} \newline
G I O L M Q R Z Fr22 \newline
\line(1,0){75} \newline
\textbf{1} \textit{Initiale} I L Z   $\cdot$ \textit{Capitulumzeichen} R  \textbf{3} \textit{Initiale} G M  \textbf{11} \textit{Initiale} R  \textbf{13} \textit{Initiale} I  \newline
\line(1,0){75} \newline
\textbf{1} \textit{Die Verse 370.13-412.12 fehlen} Q   $\cdot$ dicke] diche was O  $\cdot$ vome] von den L  $\cdot$ Heitstein] haitstain I aitsteine O beitstein L heit steyn M heit steine Z \textbf{2} liehte schein] liehter shain I er scheine O erschein L (M) R scheine Z \textbf{3} im] in G \textit{om.} R  $\cdot$ derz] der Z \textbf{4} sol] So R  $\cdot$ geloubet] globent R \textbf{6} [becere]: bezere vil dane anderswa G \textbf{7} des] das R  $\cdot$ wol] \textit{om.} Z  $\cdot$ jehen] ie M \textbf{8} spehen] spe M \textbf{9} swar] Wa L (M) (R)  $\cdot$ ich] sich L ir Z  $\cdot$ rede kêre] die rede cher I ker red R \textbf{11} dise] die I \textbf{13} ich enruoche] Ichn ruchte I  $\cdot$ die] \textit{om.} M \textbf{14} dürkelen riwen] tunkeln rúwe R \textbf{15} hânt si] hab si I Habents L (Z) Hatten sie M  $\cdot$ sælicheit] gesellecheit O wirdekeit L (Z) \textbf{16} muoz] muͤz I  $\cdot$ lîden] [sin]: liden O \textbf{17} dort vür] fvr O (L) (M) (R) (Fr22) \textbf{18} Gein der Gawan der gein der gesellecheit O \textbf{19} der] der der I \textbf{20} selben] selber R \textbf{22} vuorte] brahte I  $\cdot$ dâ] [daz]: da G do R  $\cdot$ diu] \textit{om.} I O L Z \textbf{23} Antikonie] Antigonia I Antygonie O Antichonie L M \textbf{24} sol] So O  $\cdot$ sîn] sie M \textbf{25} \textit{Versdopplung} Z   $\cdot$ vil] wol R \textbf{28} daz sô] daz ie so G das sy R  $\cdot$ erstarp] starp L [irwarp]: irstarp M \textbf{29} Veldeke] veldechin I veldich L veldecke M (Z) \textbf{30} si] \textit{om.} L \newline
\end{minipage}
\hspace{0.5cm}
\begin{minipage}[t]{0.5\linewidth}
\small
\begin{center}*T
\end{center}
\begin{tabular}{rl}
 & diu dicke von \textbf{Heitstein}\\ 
 & über aldie marke \textbf{erschein}.\\ 
 & wol im, derz heinlîche an ir\\ 
 & \textbf{sol prüeven}, \textbf{daz} gelo\textit{u}bet mir,\\ 
5 & der \textbf{vindet} kurzewîle dâ\\ 
 & bezzer danne anderswâ.\\ 
 & ich mac \textbf{des wol} von vrouwen jehen,\\ 
 & alse mir diu ougen kunnen spehen.\\ 
 & swar ich \textbf{kêre mîn rede} ze guote,\\ 
10 & diu bedarf wol zühte huote.\\ 
 & \begin{large}N\end{large}û \textbf{hœret} dise âventiure\\ 
 & der getriuwe unde der gehiure.\\ 
 & in ruoche umbe die ungetriuwen.\\ 
 & mit dürkeln \textbf{riuwen}\\ 
15 & hânt si alle ir sælekeit verlorn.\\ 
 & des \textbf{muoz} ir \textbf{sêle} lîden zorn.\\ 
 & Ûf den hof vür den palas reit\\ 
 & Gawan gegen der gesellecheit,\\ 
 & als in der künec \textbf{dar} sante,\\ 
20 & der sich \textbf{selben} an im \textbf{geschante}.\\ 
 & ein rîter, der in brâhte dar,\\ 
 & in vuorte, dâ saz wol gevar\\ 
 & Antickonie, diu künegîn.\\ 
 & sol wîplîch êre sîn gewin,\\ 
25 & des koufes hete si vil gepflegen\\ 
 & unde alles valsches sich bewegen.\\ 
 & dâ mite ir kiusche prîs erwarp.\\ 
 & Owê, daz sô vruo erstarp\\ 
 & von Veldecke der wîse man!\\ 
30 & der kunde si baz gelobet hân.\\ 
\end{tabular}
\scriptsize
\line(1,0){75} \newline
T U V W \newline
\line(1,0){75} \newline
\textbf{9} \textit{Initiale} V  \textbf{11} \textit{Initiale} T U  \textbf{17} \textit{Majuskel} T  \textbf{28} \textit{Majuskel} T  \newline
\line(1,0){75} \newline
\textbf{1} von] vomme V (W)  $\cdot$ Heitstein] hertstein V W \textbf{4} geloubet] gelobet T \textbf{5} dâ] do W \textbf{7} des wol] wol das W \textbf{9} swar] War U W  $\cdot$ kêre mîn rede] rede kere W \textbf{10} wol] von W \textbf{14} dürkeln] dvnkeln V \textbf{15} hânt] Hat W \textbf{17} Ûf] Vor U \textbf{18} gesellecheit] selekeit U \textbf{19} in] im U  $\cdot$ dar] \textit{om.} W \textbf{20} sich selben] sich selbe U sit V sich selber W  $\cdot$ geschante] schante U (W) sich geschante V \textbf{22} dâ] do V W \textbf{23} Antickonie] Antikonie T U Antykonie V Antilonie W \textbf{24} sol] So U \newline
\end{minipage}
\end{table}
\end{document}
