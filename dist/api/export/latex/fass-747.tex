\documentclass[8pt,a4paper,notitlepage]{article}
\usepackage{fullpage}
\usepackage{ulem}
\usepackage{xltxtra}
\usepackage{datetime}
\renewcommand{\dateseparator}{.}
\dmyyyydate
\usepackage{fancyhdr}
\usepackage{ifthen}
\pagestyle{fancy}
\fancyhf{}
\renewcommand{\headrulewidth}{0pt}
\fancyfoot[L]{\ifthenelse{\value{page}=1}{\today, \currenttime{} Uhr}{}}
\begin{document}
\begin{table}[ht]
\begin{minipage}[t]{0.5\linewidth}
\small
\begin{center}*D
\end{center}
\begin{tabular}{rl}
\textbf{747} & \begin{large}D\end{large}ô sprach der heidenische man:\\ 
 & "dînes strîtes ich wênec angest hân.\\ 
 & stüende ich \textbf{gar blôz}, sît ich hân swert,\\ 
 & dû wærst doch schumpfentiure gewert,\\ 
5 & sît dîn swert \textbf{zerbrosten} ist.\\ 
 & al dîn werlîcher list\\ 
 & mac dich vor tôde niht bewarn,\\ 
 & i\textbf{ne} \textbf{welle} dich anders gerne sparn.\\ 
 & Ê dû \textbf{begundest} ringen,\\ 
10 & mîn swert lieze ich \textbf{klingen}\\ 
 & beidiu durch \textbf{îser} unt durch vel."\\ 
 & der heiden starc und snel\\ 
 & tet \textbf{manlîchen} site schîn:\\ 
 & "diz swert sol unser \textbf{deweders} sîn."\\ 
15 & ez warf der küene degen balt\\ 
 & verre von im in den walt.\\ 
 & er sprach: "sol \textbf{nû hie} strît ergên,\\ 
 & \textbf{dâ} muoz glîchiu schanze \textbf{stên}."\\ 
 & \textbf{Dô} sprach \textbf{der rîche} Feirefîz:\\ 
20 & "helt, durch dîner zühte vlîz,\\ 
 & sît dû bruoder megest hân,\\ 
 & sô sage \textbf{mir}, wie ist er getân?\\ 
 & tuo mir sîn antlütze erkant,\\ 
 & wie dir sîn varwe sî genant."\\ 
25 & Dô sprach Herzeloyden kint:\\ 
 & "als ein geschriben permint,\\ 
 & swarz \textbf{und} blanc \textbf{her} unt dâ,\\ 
 & \textbf{sus} nante mir in Eckuba."\\ 
 & \begin{large}D\end{large}er heiden sprach: "\textbf{der} bin ich."\\ 
30 & si bêde \textbf{wênec dô} sûmten sich,\\ 
\end{tabular}
\scriptsize
\line(1,0){75} \newline
D \newline
\line(1,0){75} \newline
\textbf{1} \textit{Initiale} D  \textbf{9} \textit{Majuskel} D  \textbf{19} \textit{Majuskel} D  \textbf{25} \textit{Majuskel} D  \textbf{29} \textit{Initiale} D  \newline
\line(1,0){75} \newline
\newline
\end{minipage}
\hspace{0.5cm}
\begin{minipage}[t]{0.5\linewidth}
\small
\begin{center}*m
\end{center}
\begin{tabular}{rl}
 & dô sprach der heidensch man:\\ 
 & "dînes strîtes ich wênic angest hân.\\ 
 & stüende ich \textbf{gar blôz}, sît ich hân swert,\\ 
 & d\textit{û} wærest doch schumpfentiure gewert,\\ 
5 & sît dîn swert \textbf{zerbrochen} ist.\\ 
 & al dîn werlîcher list\\ 
 & mac dich vor tôde niht bewarn,\\ 
 & ich \textbf{wil} dich anders gern sparn.\\ 
 & ê dû \textbf{beginnest} ringen,\\ 
10 & mîn swert liez ich \textbf{klingen}\\ 
 & beidiu durch \textbf{îsen} und durch vel."\\ 
 & der heiden starc und snel\\ 
 & tet \textbf{manlî\textit{ch}} \textit{s}it\textit{e} schîn:\\ 
 & "diz swert sol unser \textbf{ieweders} sîn."\\ 
15 & ez warf der küene degen balt\\ 
 & verre von im \textit{in} d\textit{en wa}lt.\\ 
 & er sprach: "sol \textbf{mîn} strît ergân,\\ 
 & \textbf{d\textit{â}} muoz glîchiu schanz \textbf{stân}."\\ 
 & \textbf{aber} sprach \textbf{dô} Ferefiz:\\ 
20 & "helt, durch dîner zühte vlîz,\\ 
 & sît dû bruoder mogest hân,\\ 
 & sô sage \textbf{mir}, wie ist e\textit{r} getân?\\ 
 & tu\textit{o} mir \textit{s}în antlitz erkant,\\ 
 & wie dir sîn varwe sî genant."\\ 
25 & dô sprach \textbf{der} Hercz\textit{e}loiden kint:\\ 
 & "als ein geschriben birmint,\\ 
 & swarz \textbf{und} blanc \textbf{hie} und dâ,\\ 
 & \textbf{sus} nante mir in E\textit{c}uba."\\ 
 & der heiden sprach: "\textbf{der} bin ich."\\ 
30 & si beide \textbf{wênic} sûmten sich,\\ 
\end{tabular}
\scriptsize
\line(1,0){75} \newline
m n o V V' Fr69 \newline
\line(1,0){75} \newline
\textbf{1} \textit{Initiale} V V'  \newline
\line(1,0){75} \newline
\textbf{2} dînes] [*]: dins V Vwers V' \textbf{3} stüende] Stunt V' \textbf{4} dû] Do m o \textbf{5} zerbrochen] gebrochen n V V' \textbf{6} al dîn] Aller V'  $\cdot$ werlîcher] werlich o \textbf{8} wil] enwelle V V'  $\cdot$ sparn] sprann o \textbf{9} beginnest] beguͯndest o (V) (V') \textbf{12} heiden] heidinne V'  $\cdot$ starc] starcke n \textbf{13} tet] Din n  $\cdot$ manlîch] [m*]: manlist m  $\cdot$ site] strit m sitt \sout{ma} o sitten V (V') \textbf{14} ieweders] entweders n (o) deweders V V' \textbf{16} in den walt] vff das felt m \textbf{17} mîn] me n nv hie V nv V' \textbf{18} dâ] Do m n o V V'  $\cdot$ muoz] muͯst o \textbf{19} Ferefiz] ferefis m n fe:e fis o ferevis V fereuis V' \textbf{22} er] es m  $\cdot$ getân] gewann o \textbf{23} tuo] Duͯt m n (o)  $\cdot$ sîn] min m  $\cdot$ antlitz] anczlit o  $\cdot$ erkant] bekant V' \textbf{24} dir sîn] dir din sin n sin V' \textbf{25} dô] [So]: Do V'  $\cdot$ Herczeloiden] hertzogloiden m hertzeloẏden n herczeleiden o herzelauden V (V') \textbf{27} hie] [d]: hie V'  $\cdot$ dâ] do n [d*]: da V \textbf{28} Ecuba] eruba m ecubo n ecuͯ:a o eckuba V V' \textbf{29} heiden] heidin V' \textbf{30} sûmten] svndent V \newline
\end{minipage}
\end{table}
\newpage
\begin{table}[ht]
\begin{minipage}[t]{0.5\linewidth}
\small
\begin{center}*G
\end{center}
\begin{tabular}{rl}
 & \begin{large}D\end{large}ô sprach der heidenische man:\\ 
 & "dînes strîtes ich wênic angest hân.\\ 
 & stüende ich \textbf{alblôz}, sît ich hân swert,\\ 
 & dû wærst doch schumpfentiure gewert,\\ 
5 & sît dîn swert \textbf{zerbrochen} ist.\\ 
 & al dîn werlîcher list\\ 
 & mac dich vor tôde niht bewarn,\\ 
 & ich\textbf{ne} \textbf{welle} dich anders gerne sparn.\\ 
 & ê dû \textbf{begundest} ringen,\\ 
10 & mîn swert lieze ich \textbf{dringen}\\ 
 & beidiu durch \textbf{îsen} unde durch vel."\\ 
 & der heiden starc unde snel\\ 
 & tet \textbf{manlîche} site schîn:\\ 
 & "ditze swert sol unser \textbf{deweders} sîn."\\ 
15 & ez warf der küene degen balt\\ 
 & verre von im in den walt.\\ 
 & er sprach: "sol \textbf{hie} strît ergên,\\ 
 & \textbf{der} muoz gelîche schanze \textbf{stên}."\\ 
 & \textbf{dô} sprach \textbf{der rîche} Feirafiz:\\ 
20 & "helt, durch dîner zühte vlîz,\\ 
 & sît dû bruoder mügest hân,\\ 
 & sô sage, wie ist er getân?\\ 
 & tuo mir sîn antlütze erkant,\\ 
 & wie dir sîn varwe sî genant."\\ 
25 & dô sprach Herzeloiden kint:\\ 
 & "als ein geschriben bermint,\\ 
 & swarz, blanc \textbf{her} unde dâ,\\ 
 & \textbf{alsô} nande mirn Ekuba."\\ 
 & der heiden sprach: "\textbf{daz} bin ich."\\ 
30 & si bêde \textbf{dô wênic} sûmden sich,\\ 
\end{tabular}
\scriptsize
\line(1,0){75} \newline
G I L M Z Fr48 Fr50 \newline
\line(1,0){75} \newline
\textbf{1} \textit{Initiale} G Z Fr48 Fr50  \textbf{17} \textit{Initiale} I  \newline
\line(1,0){75} \newline
\textbf{1} Dô] da M  $\cdot$ heidenische] heidenischer M \textbf{2} wênic] luͯtzel L \textbf{3} stüende] stunt M (Z) (Fr48) (Fr50)  $\cdot$ alblôz] gar bloz L (M) Z Fr48 Fr50  $\cdot$ swert] ein swert I \textbf{4} doch] \textit{om.} Fr50  $\cdot$ schumpfentiure] entschvmphentvre Fr50 \textbf{7} vor] von I \textbf{8} ichne] Jch L M  $\cdot$ welle] wil M \textbf{9} begundest] begýnnest L (M) [begvnnest]: begvndest  Fr50 \textbf{10} mîn swert] mit swerte I (M)  $\cdot$ lieze] das liesz M (Z) (Fr48) (Fr50) \textbf{11} beidiu] \textit{om.} Fr48  $\cdot$ îsen] yser Fr50 \textbf{12} der] Den L  $\cdot$ heiden starc] starch heiden Fr50 \textbf{13} manlîche site] manlich sin Fr50 \textbf{14} ditze] Daz L (M) (Fr50)  $\cdot$ unser] vnsz M vnder Z  $\cdot$ deweders] beider L iclichers M \textbf{16} von im] hin dan I  $\cdot$ den] dem Z \textbf{17} hie strît] nu strit hie I (L) hie strit Nu M (Z) hie nv strit Fr50 \textbf{18} der] Da M Daz Z (Fr50)  $\cdot$ gelîche] an Gelicher I zuͯ glicher L  $\cdot$ stên] Gesten I \textbf{19} dô] Mer L Da M  $\cdot$ rîche] \textit{om.} Z  $\cdot$ Feirafiz] feirefiz G Z Ferefiz L feirefisz M feyrafiz Fr50 \textbf{20} vlîz] pris I \textbf{21} mügest] sulst I \textbf{22} sô] \textit{om.} L \textbf{25} dô] Da L M  $\cdot$ Herzeloiden] herzeloyden G herzenlauden I hertzelovden L der herczloyden M herzelovden Z \textbf{27} swarz] Warz M \textbf{28} Ekuba] eckuba G (Z) eccuba I Ecuͯba L \textbf{29} heiden] heide M \textbf{30} dô wênic] wenig L da wenic M wenic da Z \newline
\end{minipage}
\hspace{0.5cm}
\begin{minipage}[t]{0.5\linewidth}
\small
\begin{center}*T
\end{center}
\begin{tabular}{rl}
 & dô sprach der heidensche man:\\ 
 & "dînes strîtes ich wênic angest hân.\\ 
 & stüende ich \textbf{hie blôz}, sît ich hân \textbf{ein} swert,\\ 
 & dû wærest doch schumpfentiure gewert,\\ 
5 & sît dîn swert \textbf{zerbrochen} ist.\\ 
 & al dîn werlîcher list\\ 
 & mac dich vor \textbf{dem} tôde niht bewarn,\\ 
 & ich \textbf{en}\textbf{welle} dich anders gerne sparn.\\ 
 & ê dû \textbf{begundest} ringen,\\ 
10 & mîn swert, \textbf{daz} liez ich \textbf{dringen}\\ 
 & beidiu durch \textbf{îsen} und durch vel."\\ 
 & der heiden starc und snel\\ 
 & tet \textbf{menschlîche} site schîn:\\ 
 & "diz swert sol unser \textbf{beider} sîn."\\ 
15 & ez warf der küene degen balt\\ 
 & verre von im in den walt.\\ 
 & er sprach: "sol \textbf{hie nû} strît ergân,\\ 
 & \textbf{daz} muoz gelîche schanze \textbf{hân}."\\ 
 & \textbf{dô} sprach \textbf{\textit{der} rîche} Ferefis:\\ 
20 & "helt, durch dîner zühte vlîz,\\ 
 & sît dû bruoder mügest hân,\\ 
 & sô sage, wie ist er getân?\\ 
 & tuo mir sîn antlitze erkant,\\ 
 & wie di\textit{r s}în varwe sî genant."\\ 
25 & \begin{large}D\end{large}ô sprach Herzeloyde kint:\\ 
 & "als ein geschriben permint,\\ 
 & swarz, blanc \textbf{her} und dâ,\\ 
 & \textbf{alsô} nante mir in Eckuba."\\ 
 & der heiden sprach: "\textbf{daz} bin ich."\\ 
30 & si beide \textbf{dô wênic} sû\textit{m}eten sich,\\ 
\end{tabular}
\scriptsize
\line(1,0){75} \newline
U W Q R \newline
\line(1,0){75} \newline
\textbf{1} \textit{Initiale} Q R  \textbf{25} \textit{Initiale} W  \newline
\line(1,0){75} \newline
\textbf{3} hie] \textit{om.} W gar Q R  $\cdot$ ein] \textit{om.} W Q R \textbf{4} doch] \textit{om.} W \textbf{6} werlîcher] werliche Q \textbf{7} vor dem] von W vor Q R \textbf{8} enwelle] woͤlle W (Q) (R) \textbf{9} begundest] begúnest Q \textbf{13} menschlîche] mannlich W (Q) (R)  $\cdot$ site] sinen R \textbf{14} diz] Das Q R  $\cdot$ sol] so Q  $\cdot$ unser] vnsz R  $\cdot$ beider] deweders W (R) oweders Q \textbf{15} ez] warff W Er R \textbf{17} hie nû strît] nun strit hie R \textbf{18} daz] Do W  $\cdot$ hân] stan W (Q) (R) \textbf{19} der] \textit{om.} U  $\cdot$ Ferefis] ferafiß W ferrefisz Q feirefis R \textbf{20} helt] Herr R  $\cdot$ zühte] zuchtten R \textbf{22} sage] sage an W \textbf{23} \textit{Versdoppelung (mit Anteil aus 747.24):} Tuͦ mir sin antlút erkant / Wir dir sige sin antlút erkant R   $\cdot$ tuo] Thuͦt W \textbf{24} wie] Och wie R  $\cdot$ sîn] sin sin U  $\cdot$ sî] sein Q  $\cdot$ genant] bekant R \textbf{25} Herzeloyde] herzeleide U hertzeloyden W hertzeloude Q herczelaudes R \textbf{26} als ein] Alsam W  $\cdot$ geschriben] beschriben Q \textbf{27} swarz blanc] Wart blanck Q  $\cdot$ dâ] dan R \textbf{28} Eckuba] ekuba W Q \textbf{29} daz] der R \textbf{30} beide dô wênic] wenig do beide W beide sich R  $\cdot$ sûmeten] suͦneten U \newline
\end{minipage}
\end{table}
\end{document}
