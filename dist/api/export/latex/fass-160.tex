\documentclass[8pt,a4paper,notitlepage]{article}
\usepackage{fullpage}
\usepackage{ulem}
\usepackage{xltxtra}
\usepackage{datetime}
\renewcommand{\dateseparator}{.}
\dmyyyydate
\usepackage{fancyhdr}
\usepackage{ifthen}
\pagestyle{fancy}
\fancyhf{}
\renewcommand{\headrulewidth}{0pt}
\fancyfoot[L]{\ifthenelse{\value{page}=1}{\today, \currenttime{} Uhr}{}}
\begin{document}
\begin{table}[ht]
\begin{minipage}[t]{0.5\linewidth}
\small
\begin{center}*D
\end{center}
\begin{tabular}{rl}
\textbf{160} & \begin{large}V\end{large}rou Ginover, diu künegin,\\ 
 & \textbf{sprach jæmerlîcher worte sin}:\\ 
 & "owê \textbf{und} heiâ hei!\\ 
 & Artuses werdecheit enzwei\\ 
5 & sol brechen noch diz wunder,\\ 
 & der ob der tavelrunder\\ 
 & den hœhsten prîs solde tragen,\\ 
 & daz \textbf{der} \textbf{von} Nantes lît erslagen.\\ 
 & sînes erbeteils er gerte,\\ 
10 & dâ \textbf{man} in sterbens werte.\\ 
 & er was doch messenîe alhie,\\ 
 & alsô daz dechein ôre nie\\ 
 & deheine sîn untât vernam.\\ 
 & er was vor wildem valsche zam.\\ 
15 & der was vil gar von im geschaben.\\ 
 & nû muoz ich alze vruo begraben\\ 
 & ein slôz ob dem prîse.\\ 
 & sîn herze an \textbf{zühten} wîse,\\ 
 & obem slôze ein hantveste,\\ 
20 & riet im benamen daz beste,\\ 
 & \textbf{Swâ} man nâch wîbes minne\\ 
 & mit ellenthaftem sinne\\ 
 & solt erzeigen mannes triwe.\\ 
 & ein berendiu vruht al niwe\\ 
25 & ist \textbf{trûrens} ûf diu wîp gesæt.\\ 
 & ûz dîner wunden jâmer wæt.\\ 
 & \textbf{dir} was doch wol sô rôt dîn hâr,\\ 
 & daz dîn bluot die bluomen clâr\\ 
 & niht \textbf{rœter dorfte} machen.\\ 
30 & dû \textbf{swendest} wîplîch lachen."\\ 
\end{tabular}
\scriptsize
\line(1,0){75} \newline
D \newline
\line(1,0){75} \newline
\textbf{1} \textit{Initiale} D  \textbf{21} \textit{Majuskel} D  \newline
\line(1,0){75} \newline
\textbf{4} Artuses] Artvss D \newline
\end{minipage}
\hspace{0.5cm}
\begin{minipage}[t]{0.5\linewidth}
\small
\begin{center}*m
\end{center}
\begin{tabular}{rl}
 & \begin{large}V\end{large}rouwe Ginover, diu künigîn,\\ 
 & \textbf{tet jâmerlîche klage schîn}.\\ 
 & \textbf{si sprach}: "owê \textbf{und} heiâ hei!\\ 
 & Artuses werdicheit enzwei\\ 
5 & sol brechen noch diz wunder,\\ 
 & der ob der tavelrunder\\ 
 & den hœhesten prîs solte tragen,\\ 
 & daz \textit{\textbf{er}} \textbf{vor} Na\textit{n}tes lît erslagen.\\ 
 & sînes erbeteils er gerte,\\ 
10 & dô \textbf{man} in sterbens werte.\\ 
 & er was doch massenîe alhie,\\ 
 & alsô daz kein ôre nie\\ 
 & dekein sîn untât vernam.\\ 
 & er was vor wildem valsche zam.\\ 
15 & der was vil gar von ime geschaben.\\ 
 & nû muoz ich al ze vruo begraben\\ 
 & ein slôz ob dem prîse.\\ 
 & sîn herze an \textbf{zühten} wîse,\\ 
 & obe\textit{m} slôze ein hantveste,\\ 
20 & riet ime benamen daz beste,\\ 
 & \textbf{sît} man nâch wîbes minne\\ 
 & mit ellenthaftem sinne\\ 
 & solte erzöugen mannes triuwe.\\ 
 & ein berndiu vruht al niuwe\\ 
25 & ist \textbf{t\textit{r}ûrens} ûf diu wîp gesæt.\\ 
 & ûz dîner wunden jâmer wæt.\\ 
 & \textbf{dâr} was doch wol sô rôt dîn hâr,\\ 
 & daz dîn bluot die bluomen clâr\\ 
 & niht \textbf{râtet tôtvar} machen.\\ 
30 & dû \textbf{wendest} wîplîc\textit{h} lachen."\\ 
\end{tabular}
\scriptsize
\line(1,0){75} \newline
m n o \newline
\line(1,0){75} \newline
\textbf{1} \textit{Initiale} m  \newline
\line(1,0){75} \newline
\textbf{1} \textit{Die Verse 158.18-160.3 fehlen (Blattverlust)} o   $\cdot$ Ginover] ginouer m ginofer n \textbf{2} jâmerlîche] jemeclich n \textbf{3} Suͯ sprach owe owe oweẏ n \textbf{4} Artuses] Artusus o \textbf{5} diz] des n o \textbf{6} der] dirre n \textbf{8} er] da m der o  $\cdot$ vor Nantes] vor nates m vorgenantes o \textbf{9} gerte] gert n o \textbf{10} werte] wert n o \textbf{12} kein] keins o \textbf{13} dekein] Do kein n \textbf{14} wildem valsche] wilden falschen n o \textbf{15} der] Des o  $\cdot$ von] vom n  $\cdot$ geschaben] beschaben n o \textbf{16} muoz] mynns o  $\cdot$ al ze] also n \textbf{17} slôz] slos das o \textbf{19} obem] Ob ein m Obe eime n (o)  $\cdot$ hantveste] hantfesse o \textbf{20} riet] Det n o \textbf{23} erzöugen] erzeigen n \textbf{24} \textit{Versdoppelung (mit Anteil aus Vers 160.25):} Ein bernde fruͯnt alle nuwe / Jst bernde fruͯht alnúwe o  \textbf{25} trûrens] turrens m \textbf{27} wol sô] so wol n \textbf{29} râtet] roͯtet n \textbf{30} wendest] wondest n o  $\cdot$ wîplîch] wiplichen m \newline
\end{minipage}
\end{table}
\newpage
\begin{table}[ht]
\begin{minipage}[t]{0.5\linewidth}
\small
\begin{center}*G
\end{center}
\begin{tabular}{rl}
 & vrô Schinover, diu künigîn,\\ 
 & \textbf{sprach jæmerlîcher worte sin}:\\ 
 & "owê \textbf{und} heiâ hei!\\ 
 & Artuses werdicheit enzwei\\ 
5 & sol brechen noch diz wunder,\\ 
 & der obe der tavelrunder\\ 
 & den hœhesten brîs solte tragen,\\ 
 & daz \textbf{der} \textbf{vor} Nantis lît erslagen.\\ 
 & sînes erbeteiles er gerte,\\ 
10 & dô \textbf{man} in sterbens werte.\\ 
 & er was doch massenîe al hie,\\ 
 & alsô daz dehein ôre nie\\ 
 & dehein sîn untât vernam.\\ 
 & er was vor wilde\textit{m} valsche zam.\\ 
15 & der was vil gar von im geschaben.\\ 
 & nû muoz ich alze vruo begraben\\ 
 & ein slôz obe dem brîse\\ 
 & \textbf{unt} sîn herze an \textbf{zühten} wîse,\\ 
 & obem slôze ein hantveste,\\ 
20 & riet im benamen daz beste,\\ 
 & \textbf{swâ} man nâch wîbes minne\\ 
 & mit ellenthaftem sinne\\ 
 & solt erzeigen mannes triwe.\\ 
 & ein ber\textit{n}diu vruht al niwe\\ 
25 & ist \textbf{trûrens} ûf diu wîp gesæt.\\ 
 & \begin{large}Û\end{large}z dîner wunden jâmer wæt.\\ 
 & \textbf{dir} was doch wol sô rôt dîn hâr,\\ 
 & daz dîn bluot die bluomen klâr\\ 
 & niht \textbf{rœter dorfte} machen.\\ 
30 & dû \textbf{swendest} wîplîch lachen."\\ 
\end{tabular}
\scriptsize
\line(1,0){75} \newline
G I O L M Q R Z Fr17 Fr36 Fr47 \newline
\line(1,0){75} \newline
\textbf{1} \textit{Initiale} L R Z  \textbf{9} \textit{Initiale} I  \textbf{26} \textit{Initiale} G  \textbf{27} \textit{Initiale} M  \newline
\line(1,0){75} \newline
\textbf{1} Schinover] Ginofer I Ginover O Gýnovire L ginovern M gynoúer Q Signower R gynouer Z \textbf{2} jæmerlîcher] mit iamer daz I yemmerlich Q (R)  $\cdot$ sin] schin O sein Q \textbf{3} owê] A:::we O Owy M (R)  $\cdot$ heiâ] eia I (Q) heya heya R  $\cdot$ hei] ey Q \textbf{4} Artuses] Artuͯses L Artus Q R \textbf{5} sol] Sob R  $\cdot$ brechen noch] noch brechen O (Q) brechen nach Z brechen Fr47  $\cdot$ diz] daz I  $\cdot$ wunder] wuder M \textbf{6} der obe] Dar obe M \textbf{7} hœhesten] hoͯchste R \textbf{8} daz] Dast R  $\cdot$ vor] von O M  $\cdot$ Nantis] Nantes L Natis R  $\cdot$ lît] hat O \textbf{9} sînes] [Mins]: Sins I  $\cdot$ gerte] gert Fr47 \textbf{10} dô] Da M Z  $\cdot$ man] nam O  $\cdot$ in sterbens] sterbes Q Jn strittes R  $\cdot$ werte] wert Fr47 \textbf{11} doch] avch O (L) (Q) (Fr47)  $\cdot$ massenîe] mæsfeneie O  $\cdot$ al] \textit{om.} I O M Q R Fr47 \textbf{12} dehein] icheine M  $\cdot$ ôre] rosz L \textbf{13} sîn] \textit{om.} I  $\cdot$ vernam] von im vernan I nie vernam Z \textbf{14} vor] von O M (Q)  $\cdot$ wildem] wilden G \textbf{15} vil] \textit{om.} R \textbf{16} ich] \textit{om.} Q R Fr47  $\cdot$ vruo] frye M \textbf{17} slôz] bluͦm R  $\cdot$ obe] uff M \textbf{18} unt] \textit{om.} Z  $\cdot$ an] nach I  $\cdot$ zühten] zuͯchte L \textbf{19} obem] Obe O (L) (M)  $\cdot$ slôze] der truͯwe L (Fr47) \textbf{20} riet] Rieht L \textbf{21} swâ] Wo L (M) (Q) (R) \textbf{22} ellenthaftem sinne] ellinthaffte [mynne]: synne M elleinhaften sinde Q ellenthafftten sinne R \textbf{23} solt] Sal M \textbf{24} berndiu] berdiv G berrende R \textbf{25} trûrens] trawres Q  $\cdot$ gesæt] gesagt I O Fr36 \textbf{26} dîner] dinen I L R  $\cdot$ jâmer] Jame R  $\cdot$ wæt] wagt I O Fr36 \textbf{27} dir] Die M \textbf{28} klâr] gar I Fr36 \textbf{29} niht] Noch Q  $\cdot$ rœter] liehter I (Fr17)  $\cdot$ dorfte] dorfften M (R) (Fr17) \textbf{30} swendest] wendest Q  $\cdot$ wîplîch] wibes I L M wiplich ez O \newline
\end{minipage}
\hspace{0.5cm}
\begin{minipage}[t]{0.5\linewidth}
\small
\begin{center}*T
\end{center}
\begin{tabular}{rl}
 & \begin{large}V\end{large}rou Gynover, diu künegîn,\\ 
 & \textbf{sprach jæmerlîcher worte schîn}:\\ 
 & "ouwê! heiâ hei!\\ 
 & Artuses werdecheit enzwei\\ 
5 & sol brechen noch diz wunder,\\ 
 & der ob der tavelrunder\\ 
 & den hœhesten prîs solte tragen,\\ 
 & daz \textbf{der} \textbf{von} Nantes lît erslagen.\\ 
 & sînes erbeteiles er gerte,\\ 
10 & dô \textbf{er} in sterbens werte.\\ 
 & er was doch massenîe alhie,\\ 
 & alsô daz dehein ôre nie\\ 
 & dehein sîn untât vernam.\\ 
 & er was vor wildem valsche zam.\\ 
15 & der was vil gar von im geschaben.\\ 
 & nû muoz ich alze vruo begraben\\ 
 & ein slôz obem prîse\\ 
 & \textbf{und} sîn herze an \textbf{zühte} wîse,\\ 
 & obem slôze ein hantveste.\\ 
20 & \textbf{er} riet im benamen daz beste,\\ 
 & \textbf{swâ} man nâch wîbes minne\\ 
 & mit ellenthaftem sinne\\ 
 & solte erzeigen mannes triuwe.\\ 
 & ein berndiu vruht al niuwe\\ 
25 & ist \textbf{trûren} ûf diu wîp gesæt.\\ 
 & ûz dîner wunden jâmer wæt.\\ 
 & \textbf{dir} was doch wol sô rôt dîn hâr,\\ 
 & daz dîn bluot die bluomen clâr\\ 
 & niht \textbf{rœter dorften} machen.\\ 
30 & dû \textbf{swendest} wîplîch lachen."\\ 
\end{tabular}
\scriptsize
\line(1,0){75} \newline
T U V W \newline
\line(1,0){75} \newline
\textbf{1} \textit{Initiale} T U V  \newline
\line(1,0){75} \newline
\textbf{1} \textit{Die Verse 159.13-160.30 fehlen} W   $\cdot$ Gynover] Schinover U [*]: ginover V \textbf{2} jæmerlîcher] iamerliche V \textbf{3} ouwê] Owe vnd U V \textbf{5} brechen] brechten U \textbf{6} ob] of U \textbf{8} von] vor U V \textbf{10} er] [*]: man V \textbf{11} massenîe] der massenie U [d*]: der massenie V \textbf{15} der] Vnd U [Vnd]: Der V \textbf{16} vruo] vro U \textbf{17} obem] of dem U \textbf{18} und] [*]: was V  $\cdot$ zühte] zv́hten V \textbf{19} obem] Of eim U \textbf{20} er riet] [*iet]: Riet V \textbf{21} swâ] Wa U \textbf{23} erzeigen] v́rzoͤigen V \textbf{25} trûren] trurens V  $\cdot$ gesæt] gesaget V \textbf{26} wæt] [*]: waget V \textbf{29} dorften] dorfte V \newline
\end{minipage}
\end{table}
\end{document}
