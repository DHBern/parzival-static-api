\documentclass[8pt,a4paper,notitlepage]{article}
\usepackage{fullpage}
\usepackage{ulem}
\usepackage{xltxtra}
\usepackage{datetime}
\renewcommand{\dateseparator}{.}
\dmyyyydate
\usepackage{fancyhdr}
\usepackage{ifthen}
\pagestyle{fancy}
\fancyhf{}
\renewcommand{\headrulewidth}{0pt}
\fancyfoot[L]{\ifthenelse{\value{page}=1}{\today, \currenttime{} Uhr}{}}
\begin{document}
\begin{table}[ht]
\begin{minipage}[t]{0.5\linewidth}
\small
\begin{center}*D
\end{center}
\begin{tabular}{rl}
70.9 & die doch der \textbf{hôhe} gerten \textbf{niht},\\ 
10 & \textbf{des} der küneginne zil vergiht,\\ 
 & ir lîbes \textbf{unt} \textbf{ir lande}.\\ 
 & si gerten \textbf{anderre pfande}.\\ 
 & \begin{large}N\end{large}û was ouch Gahmuretes lîp\\ 
 & in harnasche, dâ sîn wîp\\ 
15 & \textbf{wart einer suone bî} gemant,\\ 
 & daz ir von Schotten Vridebrant\\ 
 & ze \textbf{gebe} sande vür ir schaden.\\ 
 & mit strîte heter si \textbf{verladen}.\\ 
 & ûf erde niht sô guotes was.\\ 
20 & dô \textbf{schouwet er} den adamas.\\ 
 & daz was ein helm, dâr ûf man bant\\ 
 & einen anker, dâ man inne vant\\ 
 & verwieret edel gesteine,\\ 
 & grôz, niht ze kleine.\\ 
25 & \textbf{daz} was iedoch ein swærer last.\\ 
 & gezimieret \textbf{wart} der gast.\\ 
 & wie sîn schilt ge\textit{h}êrt sî?\\ 
 & \textbf{mit} golde von Arabi\\ 
 & ein \textbf{tiweriu} \textbf{buckel} \textbf{drûf} geslagen,\\ 
30 & swære, die er \textbf{muose} tragen.\\ 
71.1 & diu gap von \textbf{rœte} \textbf{al solhez} brehen,\\ 
 & daz man sich \textbf{drinne mohte} ersehen,\\ 
 & ein zobelîn anker drunde.\\ 
 & mir selben ich wol gunde,\\ 
5 & des er hete an den lîp gegert,\\ 
71.6 & wand \textbf{ez} was maneger marke wert.\\ 
69.29 & Nû was ouch \textbf{der künec von Francrîche} tôt.\\ 
30 & des wîp in \textbf{dicke} in grôze nôt\\ 
\end{tabular}
\scriptsize
\line(1,0){75} \newline
D Fr33 \newline
\line(1,0){75} \newline
\textbf{13} \textit{Initiale} D  \textbf{27} \textit{Initiale} Fr33  \textbf{69.29} \textit{Majuskel} D  \newline
\line(1,0){75} \newline
\textbf{12} anderre] ander Fr33 \textbf{13} Gahmuretes] Gahmvretes D Gamuretes Fr33 \textbf{16} Schotten] Scotten D \textbf{17} gebe] gelte Fr33 \textbf{19} erde] erden Fr33 \textbf{20} schouwet] schouwete Fr33 \textbf{23} verwieret] verwirket Fr33 \textbf{27} gehêrt] gebert D \textbf{28} Arabi] Arabî D \textbf{71.4} selben] selbem Fr33 \textbf{71.6} wand] \textit{om.} Fr33 \textbf{69.29} künec von] roi de Fr33  $\cdot$ Francrîche] Vranchrihe D franze Fr33 \textbf{69.30} in] an Fr33 \newline
\end{minipage}
\hspace{0.5cm}
\begin{minipage}[t]{0.5\linewidth}
\small
\begin{center}*m
\end{center}
\begin{tabular}{rl}
70.9 & die doch der \textbf{hôhe} gerten \textbf{iht}.\\ 
10 & der küniginne zil vergiht\\ 
 & ir lîbes \textbf{und} \textbf{ir lande}.\\ 
 & \multicolumn{1}{l}{ - - - }\\ 
 & \begin{large}N\end{large}û was ouch Gahmuretes lîp\\ 
 & in har\textit{n}asche, dâ sîn wîp\\ 
15 & \textbf{wart einer suone bî} ge\textit{m}ant,\\ 
 & daz ir von Schotten Fridebrant\\ 
 & ze \textbf{gebe} sante vür ir schaden.\\ 
 & mit strîte het er si \textbf{verladen}.\\ 
 & ûf erde niht sô guotes was.\\ 
20 & dô \textbf{schouwete er} den adamas.\\ 
 & daz was ein helm, dâr ûfe \textit{man} bant\\ 
 & einen anker, dâ man inne vant\\ 
 & verwieret edel gesteine,\\ 
 & grôz, niht ze kleine.\\ 
25 & \textbf{daz} was iedoch ein swærer last.\\ 
 & gezimieret \textbf{was} der gast.\\ 
 & wie sîn schilt gehêret sî?\\ 
 & \textbf{mit} golde von Arabi\\ 
 & ein \textbf{tiuriu} \textbf{buckel} \textbf{dâr ûf} geslagen,\\ 
30 & swære, die er \textbf{muose} tragen.\\ 
71.1 & diu gap von \textbf{golde} \textbf{al solichen} brehen,\\ 
 & daz man sich \textbf{drinne moht} ersehen,\\ 
 & ein zobelîn anker drunde.\\ 
 & mir selben ich wol gunde,\\ 
5 & des er hete an den lîp gegert,\\ 
71.6 & wanne \textbf{ez} was maniger marke wert.\\ 
69.29 & \begin{large}N\end{large}û was ouch \textbf{rois de Franze} tôt.\\ 
30 & des wîp in \textbf{dicke} in grôze nôt\\ 
\end{tabular}
\scriptsize
\line(1,0){75} \newline
m n o \newline
\line(1,0){75} \newline
\textbf{13} \textit{Initiale} m n  \textbf{69.29} \textit{Initiale} m   $\cdot$ \textit{Capitulumzeichen} n  \newline
\line(1,0){75} \newline
\textbf{9} iht] nicht n (o) \textbf{12} \textit{Vers 70.12 fehlt} m   $\cdot$ Gar (Hie o ) one alle schande n (o) \textbf{13} Gahmuretes] gahmurettes m gamires n gamuretes o  $\cdot$ lîp] wip n \textbf{14} harnasche] harschiasche m haristasta n haristaste o  $\cdot$ dâ] do n o  $\cdot$ wîp] lip n \textbf{15} \textit{Versfolge 70.16-15} n   $\cdot$ suone] lune m (n) o  $\cdot$ bî gemant] bigewant m (n) (o) \textbf{16} Schotten] schoten n schotte o  $\cdot$ Fridebrant] vride brant m friedebrant o \textbf{17} gebe] geben o \textbf{18} verladen] v́ber laden n (o) \textbf{19} erde] erden n o  $\cdot$ sô] zu o \textbf{20} schouwete] [scho]: schuͦwet n schowet o  $\cdot$ adamas] adamast m o [palas]: adamas n \textbf{21} man] \textit{om.} m \textbf{22} inne] in n (o) \textbf{23} verwieret] Ver weret o \textbf{26} was] wart n o \textbf{27} gehêret] gehoͯret n \textbf{28} Arabi] araby n arabẏ o \textbf{30} muose] muͯsse m muͯste n (o) \textbf{71.1} al solichen] also liechtes n \textit{om.} o \textbf{71.2} moht] moͯchte n muͯste o  $\cdot$ ersehen] sehen o \textbf{71.3} anker] ancke n \textbf{71.4} selben] selber n selbes o \textbf{71.5} des] Das n o \textbf{69.29} rois] wis o  $\cdot$ de Franze] defrancze m de frantz n o \newline
\end{minipage}
\end{table}
\newpage
\begin{table}[ht]
\begin{minipage}[t]{0.5\linewidth}
\small
\begin{center}*G
\end{center}
\begin{tabular}{rl}
70.9 & die doch der \textbf{hôhe} gerten \textbf{niht},\\ 
10 & \textbf{alse} der künigîn zil vergiht,\\ 
 & ir lîbes \textbf{noch} \textbf{ir lande}.\\ 
 & si gerten \textbf{andere pfande}.\\ 
 & nû was ouch Gahmuretes lîp\\ 
 & in harnasch, dâ sîn wîp\\ 
15 & \textbf{wart einer suone bî} gemant,\\ 
 & daz ir von Schotten Fridebrant\\ 
 & ze \textbf{gelte} sande vür ir schaden.\\ 
 & mit strîte het er si \textbf{überladen}.\\ 
 & ûf erde niht sô guotes was.\\ 
20 & dô \textbf{schouwet er} den adamas.\\ 
 & daz was ein helm, dâr ûf man bant\\ 
 & einen anker, dâ man inne vant\\ 
 & verwieret edel gesteine,\\ 
 & grôz, niht ze kleine.\\ 
25 & \textbf{daz} was iedoch ein swærer last.\\ 
 & gezimiert \textbf{wart} der gast.\\ 
 & wie sîn schilt gehêret sî?\\ 
 & \textbf{ûz} golde von Arabi\\ 
 & ein \textbf{rîchiu} \textbf{buckel} \textbf{drûf} geslagen,\\ 
30 & swære, die er \textbf{muose} tragen.\\ 
71.1 & diu gab von \textbf{rœte} \textbf{alsolhez} brehen,\\ 
 & daz man sich \textbf{drinne het} ersehen,\\ 
 & ein zobelîn anker drunde.\\ 
 & mir selbem ich wol gunde,\\ 
5 & des er het an den lîp gegert,\\ 
71.6 & wan \textbf{ez} was maniger marke wert.\\ 
69.29 & nû was ouch \textbf{roy de Franze} tôt.\\ 
30 & des wîp in \textbf{dicke} in grôze nôt\\ 
\end{tabular}
\scriptsize
\line(1,0){75} \newline
G I O L M Q R Z Fr21 \newline
\line(1,0){75} \newline
\textbf{9} \textit{Initiale} O  \textbf{13} \textit{Initiale} I M  \textbf{19} \textit{Initiale} Q  \textbf{69.29} \textit{Initiale} I L R Z Fr21  \newline
\line(1,0){75} \newline
\textbf{9} die] Iie O  $\cdot$ hôhe] hvͦbe Fr21 \textbf{10} alse] der I Des O L (M) R Fr21  $\cdot$ künigîn] kᵫnginnen R \textbf{12} \textit{Vers 70.12 fehlt} R   $\cdot$ andere] anderr I (O) \textbf{13} ouch] \textit{om.} I M  $\cdot$ Gahmuretes] Gahmurets G Gamvretes O Gahmuͯretes L gamuretis M gamuretes Q Z Gahmurtes R Gamoretes Fr21 \textbf{14} dâ] so da M do Q \textbf{15} einer] eine R  $\cdot$ gemant] genant Q \textbf{16} ir] er R  $\cdot$ Schotten] schoten G O shotten I schottin M  $\cdot$ Fridebrant] vridbrant I fridebant R \textbf{18} het] hat R  $\cdot$ überladen] verladen O (M) (Q) Z (Fr21) beladen R \textbf{19} \textit{Versfolge 70.20-19} L   $\cdot$ erde] ir erdin M \textbf{20} dô] Da M Z  $\cdot$ schouwet] schouwete M \textbf{21} daz was ein] Vnd einen O  $\cdot$ bant] [vant]: bant O \textbf{22} vant] erkand R \textbf{23} verwieret] Verwirret L (Q) Fr21 Vor wirket M \textbf{24} niht ze] vnde L vnde nicht zcu M (Q) \textbf{25} ein] sin O  $\cdot$ swærer] grozzer I \textbf{26} gezimiert] Gamuert Q  $\cdot$ wart] was O (L) Fr21 \textbf{27} gehêret] gezieret L \textbf{28} ûz] von I Mit O L M Q R Z Fr21  $\cdot$ von] vz I  $\cdot$ Arabi] Arabý L (R) arabey Q \textbf{29} rîchiu] richer L Riche R  $\cdot$ buckel] bvckeln Z  $\cdot$ drûf geslagen] durch schlagen Q \textbf{30} muose] muͤste I \textbf{71.1} diu] Der L  $\cdot$ rœte] rot O  $\cdot$ alsolhez] ein solhen I ein solhez O (R) solches L (Q) (Z) \textbf{71.2} ersehen] besehen M erschen Q \textbf{71.3} zobelîn anker] zcobilic engkir M zobel anker Fr21 \textbf{71.4} selbem] seben I selben O L (M) Z Fr21 selber Q R \textbf{71.5} des] Das R  $\cdot$ het] had M  $\cdot$ den] sinen I dem R \textbf{69.29} de] der I O Fr21 zu Q  $\cdot$ Franze] vranze I frantze L Z \textbf{69.30} in grôze] groze O \newline
\end{minipage}
\hspace{0.5cm}
\begin{minipage}[t]{0.5\linewidth}
\small
\begin{center}*T (U)
\end{center}
\begin{tabular}{rl}
70.9 & die doch der \textbf{habe} gerten \textbf{niht},\\ 
10 & \textbf{des} d\textit{er} küniginne zil vergiht,\\ 
 & ir lîbes \textbf{noch} \textbf{ir landes}.\\ 
 & si gerten \textbf{anderes pfandes}.\\ 
 & nû was ouch Gahmuretes lîp\\ 
 & in harnasche, d\textit{â} sîn \textbf{werdez} wîp\\ 
15 & \textbf{suone mite wart} ge\textit{m}ant,\\ 
 & daz ir von Schotten Fridebrant\\ 
 & zuo \textbf{gelte} sante vür ir schaden.\\ 
 & mit strîte het er si \textbf{beladen}.\\ 
 & ûf erde niht sô guotes was.\\ 
20 & dô \textbf{schouweten si} den adamas.\\ 
 & daz was ein helm, dâr ûf man bant\\ 
 & einen enker, dâ man i\textit{nn}e vant\\ 
 & verw\textit{i}eret edel gesteine,\\ 
 & grôz \textbf{und} niht zuo kleine.\\ 
 & \hspace*{-.7em}\big| \textbf{wol} gezimiert \textbf{wart} der \textbf{werde} gast.\\ 
25 & \hspace*{-.7em}\big| \textbf{diz} was iedoch ein swærer last.\\ 
 & wie sîn schilt gehêret sî?\\ 
 & \textbf{von} \textbf{guotem} golde von Arabi\\ 
 & \textbf{was} ein \textbf{buckeler} \textbf{durch} geslagen,\\ 
30 & swære, die e\textit{r} \textbf{wolte} tragen.\\ 
71.1 & diu gap von \textbf{rœte} \textbf{alsoliche} brehen,\\ 
 & daz man sich \textbf{mohte dâr in} ersehen,\\ 
 & ein zobelîn anker drunde.\\ 
 & mir selbe ich wol gunde,\\ 
5 & des er hete an den lîp gegert,\\ 
71.6 & wan \textbf{daz} was maneger marke wert.\\ 
69.29 & \begin{large}N\end{large}û was ouch \textbf{ro\textit{y}s de Franze} tôt.\\ 
30 & des wîp in \textbf{ofte} in grôze nôt\\ 
\end{tabular}
\scriptsize
\line(1,0){75} \newline
U V W T \newline
\line(1,0){75} \newline
\textbf{13} \textit{Majuskel} T  \textbf{69.29} \textit{Initiale} U V W T  \newline
\line(1,0){75} \newline
\textbf{9} der habe] der [*]: hoͤhe V habe T \textbf{10} der] die U \textbf{11} landes] lande T \textbf{12} anderes pfandes] anderre pfande T \textbf{13} Gahmuretes] Gahmuͦretes U Gamurettes V gamuretes W \textbf{14} dâ sîn werdez wîp] do sin werdez wip U do sin [*]: wip V do sein schwartzes weib W da sin wip T \textbf{15} [*]: wart einer suͦne bigemant V  $\cdot$ wart einer svͦne mit gemant T  $\cdot$ gemant] genant U (W) \textbf{16} Schotten] Schoten T  $\cdot$ Fridebrant] fridebrand W Fridebant T \textbf{17} sante] satzte W \textbf{18} het] hat T  $\cdot$ beladen] verladen W (T) \textbf{19} erde] erden V \textbf{20} schouweten si] schowete [*]: er V schôuweter T \textbf{22} dâ man inne] da man ime U dar in man W \textbf{23} verwieret] Verweret U Gewirket V Verwircket W  $\cdot$ gesteine] steine V \textbf{24} Zegrôze noch ze cleine T  $\cdot$ zuo] \textit{om.} W \textbf{26} \textit{Versfolge 70.25-26} T   $\cdot$ werde] selbe V  $\cdot$ gast] man W \textbf{25} \textit{nach 70.25:} Suß stuͦnde der werde gast / Das im nichtes enbrast W   $\cdot$ Als ich eúch gesaget han W  $\cdot$ diz] daz T  $\cdot$ swærer] swere V  $\cdot$ last] [gast]: last T \textbf{27} gehêret] gezieret W \textbf{28} von guotem golde] Mit guͦtem god W mit golde T  $\cdot$ Arabi] araby W Arabŷ T \textbf{29} was] \textit{om.} V T  $\cdot$ buckeler] [*]: riche buckel V buckel W riche bvckel T  $\cdot$ durch] [*]: druf V dar auff W (T) \textbf{30} Wer die ere solte tragen W  $\cdot$ die er] diere U  $\cdot$ wolte] muͤse V (T) solte W \textbf{71.1} diu] daz T  $\cdot$ rœte] rôtiv T  $\cdot$ alsoliche] alsoliches V (W) (T) \textbf{71.2} mohte dâr in] moͤhte drinne V darynn moͤcht W drinne mohte T \textbf{71.3} drunde] als er wol kunde W \textbf{71.4} Stuͦnd do bei mir selben ich wol gunde W  $\cdot$ selbe] selben V selbem T \textbf{71.6} daz] es W \textbf{69.29} roys] Ros U \textbf{69.30} ofte in grôze] [*]: dicke in grosze V dicke brahte in T \newline
\end{minipage}
\end{table}
\end{document}
