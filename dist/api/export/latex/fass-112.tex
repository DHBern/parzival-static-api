\documentclass[8pt,a4paper,notitlepage]{article}
\usepackage{fullpage}
\usepackage{ulem}
\usepackage{xltxtra}
\usepackage{datetime}
\renewcommand{\dateseparator}{.}
\dmyyyydate
\usepackage{fancyhdr}
\usepackage{ifthen}
\pagestyle{fancy}
\fancyhf{}
\renewcommand{\headrulewidth}{0pt}
\fancyfoot[L]{\ifthenelse{\value{page}=1}{\today, \currenttime{} Uhr}{}}
\begin{document}
\begin{table}[ht]
\begin{minipage}[t]{0.5\linewidth}
\small
\begin{center}*D
\end{center}
\begin{tabular}{rl}
\textbf{112} & bestatten sper unt \textbf{ouch} daz bluot\\ 
 & ze münster, sô man \textbf{die} tôten tuot.\\ 
 & in Gahmuretes lande\\ 
 & \multicolumn{1}{l}{ - - - }\\ 
 & \multicolumn{1}{l}{ - - - }\\ 
 & \multicolumn{1}{l}{ - - - }\\ 
 & man jâmer dô bekande.\\ 
5 & \textbf{\begin{large}D\end{large}annen} über den vierzehenden tac\\ 
 & diu vrouwe eines \textbf{kindelînes} gelac,\\ 
 & eines sunes, der sölher \textbf{lide} was,\\ 
 & daz si vil kûme \textbf{dran} genas.\\ 
 & Hie ist der âventiure wurf gespilt\\ 
10 & unt ir \textbf{begin ist} gezilt,\\ 
 & wand er ist alrêst geborn,\\ 
 & dem diz mære \textbf{wart} erkorn.\\ 
 & sînes vater vreude unt \textbf{des} nôt,\\ 
 & bêde sîn leben unt \textbf{sîn} tôt,\\ 
15 & des habt ir \textbf{wol} ein teil vernomen.\\ 
 & nû wizzet, wâ von iu sî komen\\ 
 & dises mæres sachwalte\\ 
 & unt wie man den behalte.\\ 
 & man barg in \textbf{vor} ritterschaft,\\ 
20 & \textbf{ê} er \textbf{kœme} an sîner witze kraft.\\ 
20 & \multicolumn{1}{l}{ - - - }\\ 
20 & \multicolumn{1}{l}{ - - - }\\ 
20 & \multicolumn{1}{l}{ - - - }\\ 
20 & \multicolumn{1}{l}{ - - - }\\ 
20 & \multicolumn{1}{l}{ - - - }\\ 
20 & \multicolumn{1}{l}{ - - - }\\ 
 & Dô diu küneginne sich versan\\ 
 & unt ir kindelîn \textbf{wider} zir gewan,\\ 
 & \textbf{si unt ander} vrouwen\\ 
 & \textbf{begunden in allenthalben} schouwen,\\ 
 & \multicolumn{1}{l}{ - - - }\\ 
 & \multicolumn{1}{l}{ - - - }\\ 
 & \multicolumn{1}{l}{ - - - }\\ 
 & \multicolumn{1}{l}{ - - - }\\ 
25 & zwischen den beinen sîn visellîn.\\ 
 & \textbf{er muose} vil getriutet sîn,\\ 
 & \textbf{dô er hete} \textbf{manlîchiu} lit.\\ 
 & er wart mit swerten sît ein smit;\\ 
 & vil viwers er von \textbf{helmen} sluoc.\\ 
30 & sîn herze menlîch ellen truoc.\\ 
\end{tabular}
\scriptsize
\line(1,0){75} \newline
D Fr33 \newline
\line(1,0){75} \newline
\textbf{5} \textit{Initiale} D  \textbf{9} \textit{Majuskel} D  \textbf{21} \textit{Großinitiale} Fr33   $\cdot$ \textit{Majuskel} D  \newline
\line(1,0){75} \newline
\textbf{2} ze] Zem Fr33 \textbf{3} Gahmuretes] Gahmvretes D Gamuretes Fr33 \textbf{6} gelac] lac Fr33 \textbf{7} was] [plac]: was Fr33 \textbf{8} vil kûme dran] is kume Fr33 \textbf{10} begin] bogen Fr33 \textbf{14} bêde] \textit{om.} Fr33 \textbf{19} Den barc man von ritterschaft Fr33 \textbf{21} Dô] Ho Fr33 \textbf{22} wider] \textit{om.} Fr33 \newline
\end{minipage}
\hspace{0.5cm}
\begin{minipage}[t]{0.5\linewidth}
\small
\begin{center}*m
\end{center}
\begin{tabular}{rl}
 & bestatten sper und \textbf{ouch} daz bluot\\ 
 & ze münste\textit{r}, sô man tôten tuot.\\ 
 & in Gahmuretes lande\\ 
 & \multicolumn{1}{l}{ - - - }\\ 
 & \multicolumn{1}{l}{ - - - }\\ 
 & \multicolumn{1}{l}{ - - - }\\ 
 & man jâmer dô bekande.\\ 
5 & \textbf{dan} über den vierzehenden tac\\ 
 & diu vrouwe ein\textit{es} \textbf{kindelîn\textit{es}} ge\textit{l}a\textit{c},\\ 
 & eines sunes, der solicher \textbf{glide} was,\\ 
 & daz si vil kûme \textbf{sîn} genas.\\ 
 & hie ist der âventiure wurf gespilt\\ 
10 & und ir \textbf{begin ist} gezilt,\\ 
 & wan er ist aller êrst geborn,\\ 
 & dem diz mære \textbf{wart} erkorn.\\ 
 & sînes vater vröude und \textbf{die} nôt,\\ 
 & beide sîn leben und \textbf{sînen} tôt,\\ 
15 & des habet ir \textbf{wol} ein teil vernomen.\\ 
 & nû wizzet, wâ von i\textit{u} sî komen\\ 
 & dises mære\textit{s} sachwalte\\ 
 & und wie man den behalte.\\ 
 & man barc in \textbf{von} ritterschaft,\\ 
20 & \textbf{ê} er \textbf{wære komen} an sîner witze kraft.\\ 
20 & \multicolumn{1}{l}{ - - - }\\ 
20 & \multicolumn{1}{l}{ - - - }\\ 
20 & \multicolumn{1}{l}{ - - - }\\ 
20 & \multicolumn{1}{l}{ - - - }\\ 
20 & \multicolumn{1}{l}{ - - - }\\ 
20 & \multicolumn{1}{l}{ - - - }\\ 
 & dô diu künigîn sich versan\\ 
 & und ir kindelîn \textbf{wider} zuo ir gewan,\\ 
 & \textbf{si und andere} vrouwen\\ 
 & \textbf{begunden in allenthalben} schouwen,\\ 
 & \multicolumn{1}{l}{ - - - }\\ 
 & \multicolumn{1}{l}{ - - - }\\ 
 & \multicolumn{1}{l}{ - - - }\\ 
 & \multicolumn{1}{l}{ - - - }\\ 
25 & zwischen den beinen sîn visellîn.\\ 
 & \textbf{er muose} vil getriutet sîn,\\ 
 & \textbf{daz er hete} \textbf{manlîchiu} lit.\\ 
 & er wart mit swerten sît ein smit;\\ 
 & vil viures er von \textbf{helmen} sluoc.\\ 
30 & sîn herze manlîch ellen truoc.\\ 
\end{tabular}
\scriptsize
\line(1,0){75} \newline
m n o \newline
\line(1,0){75} \newline
\textbf{1} \textit{Illustration mit Überschrift:} Also gamiret (gamuͯret o  ) dot was vnd man in zuͦ dem (\textit{om.} o  ) múnster begruͦp vnd sin frouwe grosz leit hette vmb in n (o)   $\cdot$ \textit{Initiale} n o  \newline
\line(1,0){75} \newline
\textbf{2} ze] Zuͯ dem n  $\cdot$ münster] muͯnsten m  $\cdot$ sô] dar n  $\cdot$ tuot] doͯt o \textbf{3} Gahmuretes] gahmurettes m gamiretes n gamuͯretes o \textbf{4} man jâmer dô] Den jomer man n (o) \textbf{5} vierzehenden] viertzehen n xiiij o \textbf{6} eines kindelînes] ein kindelin m eins kindes o  $\cdot$ gelac] genas m \textbf{7} glide] lyde n (o) \textbf{8} sîn] sins o \textbf{11} aller êrst] aleerst o \textbf{12} diz] dise n \textbf{13} die] des n o \textbf{14} sînen] sin n o \textbf{15} habet] hap o \textbf{16} iu] ich m \textbf{17} dises] Disz n (o)  $\cdot$ mæres sachwalte] mere sache walte m meres [riches]: sache walten o \textbf{20} ê] Er o  $\cdot$ wære komen] keme n o  $\cdot$ sîner] \textit{om.} n o \textbf{21} dô] \textit{om.} o \textbf{22} wider] \textit{om.} n o \textbf{26} muose] muͯste n o \textbf{28} smit] [lit]: smit smit n \newline
\end{minipage}
\end{table}
\newpage
\begin{table}[ht]
\begin{minipage}[t]{0.5\linewidth}
\small
\begin{center}*G
\end{center}
\begin{tabular}{rl}
 & bestatten sper und \textit{daz} bluot\\ 
 & ze \textit{\textbf{dem}} münster, sô man tôten tuot.\\ 
 & in Gahmuretes lande\\ 
 & \multicolumn{1}{l}{ - - - }\\ 
 & \multicolumn{1}{l}{ - - - }\\ 
 & \multicolumn{1}{l}{ - - - }\\ 
 & man jâmer dô bekande.\\ 
5 & \textbf{dar nâch} über den vierzehenden tac\\ 
 & diu vrouwe eines \textbf{kindes} gelac,\\ 
 & eines sunes, der solher \textbf{lide} was,\\ 
 & daz si vil kûme \textbf{dran} genas.\\ 
 & hie ist der âventiure wurf gespilt\\ 
10 & unde ir \textbf{beginnens} gezilt,\\ 
 & \begin{large}W\end{large}an er ist alrêrst geboren,\\ 
 & dem diz mære \textbf{wart} erkoren.\\ 
 & sînes vater vröude und \textbf{des} nôt,\\ 
 & beidiu sîn leben und \textbf{sînen} tôt,\\ 
15 & des habt ir \textbf{ê} ein teil vernomen.\\ 
 & nû wizzet, wâ von iu sî komen\\ 
 & dises mæres sachwalte\\ 
 & unde wie man den behalte.\\ 
 & man barg in \textbf{von} rîterschaft,\\ 
20 & \textbf{ê} er \textbf{k\textit{œ}me} an sîner witze kraft.\\ 
20 & \multicolumn{1}{l}{ - - - }\\ 
20 & \multicolumn{1}{l}{ - - - }\\ 
20 & \multicolumn{1}{l}{ - - - }\\ 
20 & \multicolumn{1}{l}{ - - - }\\ 
20 & \multicolumn{1}{l}{ - - - }\\ 
20 & \multicolumn{1}{l}{ - - - }\\ 
 & dô diu künigîn sich versan\\ 
 & unde ir kindelîn zuo ir gewan,\\ 
 & \textbf{si unde ander} vrouwen\\ 
 & \textbf{\textit{begunden} in \textit{allenthalben}} schouwen,\\ 
 & \multicolumn{1}{l}{ - - - }\\ 
 & \multicolumn{1}{l}{ - - - }\\ 
 & \multicolumn{1}{l}{ - - - }\\ 
 & \multicolumn{1}{l}{ - - - }\\ 
25 & zwischen den beinen sîn visellîn.\\ 
 & \textbf{dô muoser} vil ge\textit{t}ri\textit{ut}et sîn,\\ 
 & \textbf{dô er hete} \textbf{manlîchiu} lit.\\ 
 & er wart mit swerten sît ein smit;\\ 
 & vil viures er von \textbf{helme} sluoc.\\ 
30 & sîn herze manlîch ellen truoc.\\ 
\end{tabular}
\scriptsize
\line(1,0){75} \newline
G I O L M Q R Z \newline
\line(1,0){75} \newline
\textbf{1} \textit{Initiale} O M  \textbf{5} \textit{Initiale} L Q R Z  \textbf{9} \textit{Überschrift:} her parcifal L   $\cdot$ \textit{Initiale} I L  \textbf{11} \textit{Initiale} G  \textbf{27} \textit{Initiale} I  \newline
\line(1,0){75} \newline
\textbf{1} bestatten] ÷Lestaten O Gestaten Q Bestattentent R  $\cdot$ daz] \textit{om.} G avch daz O (L) (M) (Q) (Z) \textbf{2} dem] \textit{om.} G  $\cdot$ sô] als O (L) (M) Q R Z  $\cdot$ tôten] die toten O (M) (Z) den besten Q den totten R \textbf{3} Gahmuretes] gahmurets G Gamvretes O Gahmuͯretes L gamuretis M gamúredes Q gamuretes Z \textbf{4} man jâmer] den iamer manger I  $\cdot$ dô] da I M R Z  $\cdot$ bekande] erkande M Z \textbf{5} über den] andem I uber R  $\cdot$ vierzehenden] vierzehendem I virtzehende Q vierczechen R \textbf{6} eines] ein Z  $\cdot$ kindes] chindelins O (R) (Z)  $\cdot$ gelac] gelage I \textbf{7} eines sunes] Einen sun R  $\cdot$ lide] lute M \textbf{8} \textit{nach 112.8:} Nv ist ein degen vsserkorn / Parcifal al hie geborn / Der git mit siner stuͦre / Alrest schone Auentuͯre / Explizit Gahmvͯret / Jncipit parcifal L   $\cdot$ vil] \textit{om.} L  $\cdot$ kûme] kom Q R \textbf{9} âventiure] avetvre I  $\cdot$ wurf] [wrf]: worf O werff M \textbf{10} ir beginnens] beginnes O (Q) ir begynnes M ir begen ist Z \textbf{12} diz] \sout{ist} dize O \textbf{13} vröude] freudo R  $\cdot$ des nôt] not I [tot]: not O seyn nott Q \textbf{15} des] Dasz Q (R)  $\cdot$ ir ê] ir I ir hie L ie M \textbf{17} dises] Dizse O Des L  $\cdot$ sachwalte] shawalte I sache balte Z \textbf{18} den] in I \textbf{19} barg] brach Q  $\cdot$ von] vor I L M R ovch vor Z \textbf{20} ê] ê daz I  $\cdot$ kœme] chome G kom L (Q) (Z)  $\cdot$ an sîner] zesiner I (Z)  $\cdot$ witze] \textit{om.} Z \textbf{21} dô] Da M Z \textbf{22} ir kindelîn] si ir chint I  $\cdot$ zuo ir] zuͤ sich I wider zir O (L) (M) (Q) (R) (Z)  $\cdot$ gewan] nam Q \textbf{23} vrouwen] ir frowen I \textbf{24} begunden in allenthalben] alenthalben sin begunden G begunden allenthalben I  $\cdot$ schouwen] schewen R \textbf{25} zwischen den] vnder den I Zwischen sinen L En czwuschin den M Zwuschem Q Zwischen Z  $\cdot$ beinen] beẏne Q  $\cdot$ sîn] vnd L  $\cdot$ visellîn] zusilin R \textit{om.} Z \textbf{26} dô] Daz O L (M) (Q) (R) o\textit{m. } Z  $\cdot$ muoser] mvͦse O (L) (M) (Q) (R) Er must Z  $\cdot$ getriutet] gebriset G \textbf{27} dô er] Daz er I Der O R Er L M Q Da er Z  $\cdot$ manlîchiu] manliche R \textbf{28} er] Vnd Q  $\cdot$ swerten] werden M \textbf{29} viures] [vurs]: viurs I  $\cdot$ von] vom Q  $\cdot$ helme] helmen I Z \textbf{30} ellen] eren Q \newline
\end{minipage}
\hspace{0.5cm}
\begin{minipage}[t]{0.5\linewidth}
\small
\begin{center}*T (U)
\end{center}
\begin{tabular}{rl}
 & \textbf{und} bestatten \textbf{daz} sper und daz bluot\\ 
 & zuo \textbf{dem} münster, sô man tôten tuot\\ 
 & in Gahmuretes lande.\\ 
 & "owê, schade und schande,\\ 
 & den wir nû genomen hân",\\ 
 & sô sprâchen Gahmuretes man.\\ 
 & \multicolumn{1}{l}{ - - - }\\ 
5 & \textbf{dar nâch} über den vierzehenden tac\\ 
 & diu vrouwe eines \textbf{kindes} gelac,\\ 
 & eines sunes, der solher \textbf{lide} was,\\ 
 & daz si vil kûme genas.\\ 
 & hie ist der âventiure wurf gespilt\\ 
10 & und ir \textbf{beginnen} gezilt,\\ 
 & wan er ist aller êrst geborn,\\ 
 & deme disiu mære \textbf{ist} erkorn.\\ 
 & sînes vater vreude und \textbf{des} nôt,\\ 
 & beide sîn leben und \textbf{sîn} tôt,\\ 
15 & des habet ir ein teil vernomen.\\ 
 & nû wizzet, wâ von iu sî komen\\ 
 & di\textit{s} mære\textit{s} sachwalte\\ 
 & und wie man den behalte.\\ 
 & man barc in \textbf{vor} rîterschaft,\\ 
20 & \textbf{unz} er \textbf{kam} an sîner witze kraft,\\ 
20 & wan ez vorhte diu künegîn,\\ 
20 & ob ir vil liebez kindelîn\\ 
20 & ritters werc gesæhe,\\ 
20 & daz ir zuo dem sune geschæhe,\\ 
20 & als sîme vater Gahmurete.\\ 
20 & dâ von sin gehalten tete\\ 
 & \multicolumn{1}{l}{ - - - }\\ 
 & \multicolumn{1}{l}{ - - - }\\ 
 & \textbf{niht wan bî} vrouwen.\\ 
 & \textbf{si muozen dicke} schouwen,\\ 
 & wan er was ein rîcher bêâfiz.\\ 
 & ich wæne, got sînen vlîz\\ 
 & mit kunst an in kêrte,\\ 
 & dô er in leben lêrte.\\ 
25 & \multicolumn{1}{l}{ - - - }\\ 
 & \multicolumn{1}{l}{ - - - }\\ 
 & \textbf{got gap im} \textbf{starkiu, schœniu} lit.\\ 
 & er wart mit swerten sît ein smit;\\ 
 & vil viures er von \textbf{helmen} sluoc.\\ 
30 & sîn herze manlîch ellen truoc.\\ 
\end{tabular}
\scriptsize
\line(1,0){75} \newline
U V W T \newline
\line(1,0){75} \newline
\textbf{5} \textit{Überschrift:} Hye ward partzifal gamurettes sun geboren diser auentúr herre W   $\cdot$ \textit{Initiale} W T  \textbf{9} \textit{Majuskel} T  \textbf{11} \textit{Überschrift:} Hie ist kúnig Gamuretes buͦch vs der Parcifals vatter was So hebet hie an der prologus von Parcifal der vs welschem zuͦ túschem ist gemaht Vnde vohet hie sine kintheit an V   $\cdot$ \textit{Initiale} V  \textbf{13} \textit{Überschrift:} Der prologus si hin geleit nv hoͤrent Parcifals kintheit Dar noch sin manheit hohen pris erwarp in maniger hande wis Als ir har nach beuinden wol dis buͦch es v́ch vnderwisen sol V   $\cdot$ \textit{Initiale} V   $\cdot$ \textit{Majuskel} T  \textbf{14} \textit{Majuskel} T  \textbf{21} \textit{Majuskel} T  \textbf{23} \textit{Majuskel} T  \newline
\line(1,0){75} \newline
\textbf{1} bestatent sper vnd blvͦt T  $\cdot$ bestatten] besteten W \textbf{2} sô] als W sam T  $\cdot$ tôten] den toten V (W) die tôten T \textbf{3} Gahmuretes] Gahmuͦretes U Gamurettes V gamuretes W \textbf{3} \textit{Die Verse 112.3¹-3#'3 fehlen} T  \textbf{3} Gahmuretes] Gamuretes V (W) \textbf{4} \textit{Vers 112.4 fehlt} U V W   $\cdot$ grôz iamer man bekande T \textbf{5} ÷Arnach úber viertzehen tag W  $\cdot$ vierzehenden] vierzigesten T \textbf{6} kindes] svnes T  $\cdot$ gelac] lag V \textbf{7} eines sunes] \textit{om.} T  $\cdot$ lide] glide W (T) \textbf{8} si] si sin V \textbf{9} ist] dist W \textbf{10} beginnen] beginnen ist W beginnens T \textbf{12} \textit{nach 112.12: Einschub 112.12\textasciicircum1-12\textasciicircum4\textasciicircum9\textasciicircum6 (Prologus)} V   $\cdot$ disiu] dis V W (T) \textbf{13} vreude] froͤden W \textit{om.} T  $\cdot$ des] der W \textbf{14} beide] Beidiv T  $\cdot$ sîn tôt] sin [not]: dot U der tot W \textbf{16} sî] ist T \textbf{17} dis mæres] Dise mere U  $\cdot$ sachwalte] sach als ich eúch zalte W \textbf{18} den] sy W in T \textbf{20} unz] Bit U ê T  $\cdot$ kam] kême T \textbf{20} \textit{Die Verse 112.20¹-20#'6 fehlen} T   $\cdot$ vorhte] vorcht W \textbf{20} ritters werc] Ritter were W \textbf{20} Gahmurete] Gahmuͦrete U Gamuret V gemuret W \textbf{21} \textit{Die Verse 112.21-22 fehlen} U V W   $\cdot$ Do div kvnegin sich versân T \textbf{22} vnd wider ir kint zvͦ zir gewan T \textbf{23} Si vnd ander ir vrôuwen T \textbf{24} begvnden allentalben scouwen T  $\cdot$ muozen] muͤste [*]: in V muͤste in W \textbf{24} \textit{Die Verse 112.24¹-24#'4 fehlen} T   $\cdot$ rîcher] rechter W \textbf{24} in leben] in [lerte]: leben U sein leben W \textbf{25} \textit{Die Verse 112.25-26 fehlen} U V W   $\cdot$ zw:::elin \textit{(Vers nachträglich weitgehend radiert)} T \textbf{26} er mvese vil getrvtet sin T \textbf{27} do heter manlichiv lit T  $\cdot$ schœniu lit] schoͤnet glit W \textbf{29} von] v̂z T \newline
\end{minipage}
\end{table}
\end{document}
