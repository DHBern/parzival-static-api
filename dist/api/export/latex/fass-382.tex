\documentclass[8pt,a4paper,notitlepage]{article}
\usepackage{fullpage}
\usepackage{ulem}
\usepackage{xltxtra}
\usepackage{datetime}
\renewcommand{\dateseparator}{.}
\dmyyyydate
\usepackage{fancyhdr}
\usepackage{ifthen}
\pagestyle{fancy}
\fancyhf{}
\renewcommand{\headrulewidth}{0pt}
\fancyfoot[L]{\ifthenelse{\value{page}=1}{\today, \currenttime{} Uhr}{}}
\begin{document}
\begin{table}[ht]
\begin{minipage}[t]{0.5\linewidth}
\small
\begin{center}*D
\end{center}
\begin{tabular}{rl}
\textbf{382} & \textbf{\begin{large}D\end{large}er grâve von der Muntane}\\ 
 & vuor gegen Gawane.\\ 
 & dâ wart \textbf{ein rîchiu} tjost getân,\\ 
 & daz der starke Laheduman\\ 
5 & hinderm orse ûfem acker lac.\\ 
 & dar nâch er sicherheit pflac,\\ 
 & der \textbf{stolze} degen wert erkant.\\ 
 & diu ergienc in Gawans hant.\\ 
 & Dô streit der herzoge Astor\\ 
10 & \textbf{den zingeln} aller næhste vor.\\ 
 & dâ ergienc manec \textbf{hurteclîcher} strît.\\ 
 & \textbf{dicke} 'Nantes' wart geschrît,\\ 
 & Artuses herzeichen.\\ 
 & die herten, niht die weichen,\\ 
15 & \textbf{was dâ} manec ellender Berteneis\\ 
 & unt \textbf{die} soldiere von \textbf{Destrigleis}\\ 
 & ûz Erekes lande,\\ 
 & der tât man dâ bekande.\\ 
 & \textbf{Ir} pflac \textbf{der herzoge von} Lanverunz.\\ 
20 & ouch mohte Poydiconjunz\\ 
 & die Bertenoise \textbf{hân} ledec \textbf{lân}.\\ 
 & sô wart ez dâ von in getân.\\ 
 & si \textbf{wâren} Artuse\\ 
 & zer Muntane Kluse\\ 
25 & ab gevangen, dâ man strîten sach.\\ 
 & in eime sturme daz geschach.\\ 
 & si schrîten 'Nantes' nâch ir siten\\ 
 & \textbf{hie oder} swâ si \textbf{strîtes biten}.\\ 
 & \textbf{daz} was ir krîe unt ir art.\\ 
30 & etslîcher truoc vil \textbf{grâwen} bart.\\ 
\end{tabular}
\scriptsize
\line(1,0){75} \newline
D \newline
\line(1,0){75} \newline
\textbf{1} \textit{Initiale} D  \textbf{9} \textit{Majuskel} D  \textbf{19} \textit{Majuskel} D  \newline
\line(1,0){75} \newline
\textbf{13} Artuses] Artvss D \textbf{15} Berteneis] bertenêis D \textbf{16} Destrigleis] Destriglêis D \textbf{20} Poydiconjunz] Poydiconivnz D \textbf{24} zer Mvntane chlvse D \newline
\end{minipage}
\hspace{0.5cm}
\begin{minipage}[t]{0.5\linewidth}
\small
\begin{center}*m
\end{center}
\begin{tabular}{rl}
 & \textbf{lacons de Funt\textit{a}ne}\\ 
 & vuor gegen Gawane.\\ 
 & d\textit{â} wart \textbf{sô rîch ein} just getân,\\ 
 & daz der starke L\textit{a}h\textit{e}duman\\ 
5 & hinder dem rosse ûf dem acker lac.\\ 
 & dar nâch er sicherheite pflac,\\ 
 & der \textbf{stolze} degen wert erkant.\\ 
 & diu ergienc in Gawanes hant.\\ 
 & dô streit der herzoge Astor\\ 
10 & \textbf{den zingelen} aller nâhest vor.\\ 
 & d\textit{â} ergienc manic \textbf{herteclîcher} strît.\\ 
 & \textbf{dicke} \textbf{dô} 'Nantes' wart geschrît,\\ 
 & Ar\textit{tuses} herzeichen.\\ 
 & die herten, niht die weichen,\\ 
15 & \textbf{was dâ} manic ellender Britune\textit{i}s\\ 
 & und \textbf{die} soldiere von \textbf{D\textit{e}strigal\textit{e}is}\\ 
 & ûz Er\textit{e}ckes lande,\\ 
 & der tât man d\textit{â} bekande.\\ 
 & \textbf{ir} pflac \dag die\dag  \textbf{de} Lave\textit{r}unz.\\ 
20 & ouch m\textit{o}hte Poidic\textit{o}niunz\\ 
 & die Britunoise \textbf{hân} \textit{l}e\textit{d}ic \textbf{lân}.\\ 
 & sô wart ez dâ von in getân.\\ 
 & si \textbf{wâren} Artuse\\ 
 & zer Mu\textit{n}tane Cluse\\ 
25 & ab gevangen, d\textit{â} man strîten sach.\\ 
 & in einem sturme daz geschach.\\ 
 & si schrîten 'N\textit{an}tes' nâch ir siten\\ 
 & \textbf{hie oder} wâ si \textbf{strîtes biten}.\\ 
 & \textbf{daz} was ir krîe und ir art.\\ 
30 & etslîcher truoc vil \textbf{hôhen} bart.\\ 
\end{tabular}
\scriptsize
\line(1,0){75} \newline
m n o \newline
\line(1,0){75} \newline
\newline
\line(1,0){75} \newline
\textbf{1} Lacons] Lacontz n Laconcz o  $\cdot$ Funtane] funitone m fantane o \textbf{3} dâ] Do m n \textbf{4} starke] herre der starcke n  $\cdot$ Laheduman] lehaduman m leheduman n lehedumman o \textbf{6} er sicherheite] sicherheit er n (o) \textbf{7} wert] wit n o \textbf{8} hant] hat o \textbf{9} streit] steit o  $\cdot$ herzoge] herczoger o \textbf{10} zingelen] zingel n o  $\cdot$ vor] :or o \textbf{11} dâ] Do m n o  $\cdot$ herteclîcher] [s]: hurteclicher n hurteklich o \textbf{12} Nantes] nantez o  $\cdot$ geschrît] gesit n \textbf{13} Artuses] Ar m \textbf{14} herten] herczen o \textbf{15} was dâ] Do was n Was do o  $\cdot$ Brituneis] brittunes m pritoneisz n britaneis o \textbf{16} Destrigaleis] dstrigalois m descrigoleisz n desrigaleis o \textbf{17} Ereckes] erckes m \textbf{18} tât] tag n o  $\cdot$ dâ] do m n o \textbf{19} ir] Er n o  $\cdot$ die] \textit{om.} o  $\cdot$ de Laverunz] de laueramvnz m lauerantz n delauerancz o \textbf{20} mohte] moͯhte m  $\cdot$ Poidiconiunz] poidicunivnz m poidicomantz n poidicaniuncz o \textbf{21} Britunoise] brittunoise m britoneise n britaneise o  $\cdot$ hân ledic] handelig m in ledig n o \textbf{22} ez dâ von in] es do von in n [in]: auch da von in o \textbf{23} Artuse] artuͯse o \textbf{24} Zer mutane cluse m  $\cdot$ Zuͯ der montanẏe cluse n  $\cdot$ Zer montanie cluse o \textbf{25} dâ] do m n o  $\cdot$ strîten] strites o \textbf{26} sturme] turnie o \textbf{27} Nantes] notes m \textbf{28} strîtes] striten o \textbf{29} ir krîe und] ie kriege oder o \textbf{30} hôhen] growen n (o) \newline
\end{minipage}
\end{table}
\newpage
\begin{table}[ht]
\begin{minipage}[t]{0.5\linewidth}
\small
\begin{center}*G
\end{center}
\begin{tabular}{rl}
 & \textbf{lechkuns Emontane}\\ 
 & vuor gein Gawane.\\ 
 & dâ wart \textbf{ein rîchiu} tjost getân,\\ 
 & daz der starke Lachdoman\\ 
5 & hinderm orse ûf dem acker lac.\\ 
 & dar nâch er sicherheite pflac,\\ 
 & \begin{large}D\end{large}er \textbf{starke} degen wert erkant.\\ 
 & diu ergie in Gawans hant.\\ 
 & dô streit der herzoge Astor\\ 
10 & \textbf{dem zingel} aller næhest vor.\\ 
 & dâ ergienc manic \textbf{\textit{her}t\textit{e}lîcher} strît.\\ 
 & \textbf{vil} \textbf{dicke} 'Nantis' wart geschrît,\\ 
 & Artuses herzeichen.\\ 
 & die herten, niht die weichen,\\ 
15 & \textbf{was dâ} manic ellender Britaneis\\ 
 & unde soldiere von \textbf{Destrigeis}\\ 
 & ûz Erekes lande,\\ 
 & der tât man dâ bekande.\\ 
 & \textbf{der} pflac \textbf{duc de} Lanvarunz.\\ 
20 & ouch mohte Poydeconiunz\\ 
 & die Britaneis \textbf{alle} ledic \textbf{lân}.\\ 
 & sô wart ez dâ von in getân.\\ 
 & si \textbf{wurden} Artuse\\ 
 & zer Montanie Kluse\\ 
25 & abe gevangen, dâ man strîten sach.\\ 
 & in einem sturme daz geschach.\\ 
 & si schrîten 'Nantis' nâch ir siten\\ 
 & \textbf{d\textit{â} unde} swâ si \textbf{sider striten}.\\ 
 & \textbf{ez} was ir krîe unde ir art.\\ 
30 & etslîcher truoc vil \textbf{grâwen} bart.\\ 
\end{tabular}
\scriptsize
\line(1,0){75} \newline
G I O L M Q R Z Fr21 Fr41 \newline
\line(1,0){75} \newline
\textbf{1} \textit{Initiale} I O L Z Fr21   $\cdot$ \textit{Capitulumzeichen} R  \textbf{7} \textit{Initiale} G  \textbf{13} \textit{Initiale} I  \newline
\line(1,0){75} \newline
\textbf{1} \textit{Die Verse 370.13-412.12 fehlen} Q   $\cdot$ :::aive uon :::en::: Fr41  $\cdot$ lechkuns] ÷ch kons O  $\cdot$ Emontane] emuntage G emontange I de Montane M Ehmontange R demontange Z \textbf{2} vuor] Kam R  $\cdot$ Gawane] Gawange I Z \textbf{3} rîchiu tjost] richer stich R \textbf{4} starke] riche I  $\cdot$ Lachdoman] lahdoman G (O) (L) (Z) lohdeman I lahodoman Fr21 Lahedum::: Fr41 \textbf{5} hinder dem orse lac vf dem acker I \textbf{6} er] \textit{om.} M \textbf{7} starke degen] starche riter I stolze degn Fr41 \textbf{8} Gawans] Gawenes I Gawanes O gawanis M Gawan Fr21 \textbf{9} dô] Da O M Z  $\cdot$ Astor] Custor R \textbf{10} dem zingel] den zingel I Den zingelin O Den zingeln L (M) (R) Z Fr21 (Fr41) \textbf{11} hertelîcher] riterlicher G herter O (R) (Fr21) herte L \textbf{12} Nantis] nandes I Nantes O L Z Fr41 Nantys R  $\cdot$ wart] war I \textbf{13} Artuses] Artuͯses L Artusesz M Artus R Z (Fr21) artuss Fr41 \textbf{14} herten] heten Z  $\cdot$ niht] vnd Fr41 \textbf{15} ellender] ellenthafter I \textit{om.} Fr41  $\cdot$ Britaneis] pritonoys I briteneis O M Brittoneisz L britannys R brituneis Z Britaneis Fr21 Berte::: Fr41 \textbf{16} unde] Vnd die Z  $\cdot$ von] \textit{om.} M  $\cdot$ Destrigeis] destrigoys I destrigeisz L destriges Z destr:geis Fr21 \textbf{17} Erekes] erchel I Mrehtes O Erachez L erkes M Fr21 Erekers R ereckes Z \textbf{18} der] da I  $\cdot$ tât] tet I M getat L  $\cdot$ dâ] der I do R wol Z  $\cdot$ bekande] erkande L (M) R \textbf{19} der phlac duclanuamurs I  $\cdot$ Der pflach dvch der Lanvaronz O  $\cdot$ Der der hertzoge von Lvnvarnz L  $\cdot$ Der pflag der herczog von Lonvarvncz R  $\cdot$ Der pflac duc von Lanverunz Z  $\cdot$ ::: pflac dvc lanva:::vnz Fr21  $\cdot$ ::: Fr41 \textbf{20} Poydeconiunz] poydecomunz I Poydekomvnz O Poý de Conivnz L poide kvnivnz M poydekonivncz R poydekonivntz Z poy de kvnivnz Fr21 \textbf{21} die] Zuͯ L  $\cdot$ Britaneis] [pritane*]: pritaneys G pritonoys I britaneyse O Brittaneise L brytonyse R brituneis Z ::: Fr21  $\cdot$ alle] han Z \textbf{22} dâ] do O R  $\cdot$ von] vor R \textbf{23} Artuse] Artuͯse L \textbf{24} zer] Zv Z  $\cdot$ Montanie Kluse] montanie chluse G (O) [montan*]: montanien chluse I Montanie cluͯse L montane cluse M montange Cluse R montanie cluse Z Montanîe clvse Fr21 \textbf{25} dâ] do R  $\cdot$ sach] shac I \textbf{26} in einem] im I  $\cdot$ sturme] strume G strite L \textbf{27} schrîten] sriren I rieffin M schrient R  $\cdot$ Nantis] nantes G I (O) (L) (Z) Natys R \textbf{28} dâ] do G (R)  $\cdot$ unde] vnder ein ander I oder O L (M) R Z (Fr21)  $\cdot$ swâ] wa L M R Z  $\cdot$ sider striten] sider [strten]: striten G [riten]: striten I strites biten O (L) (R) Z Fr21 stritens biten M \textbf{30} vil] \textit{om.} I \newline
\end{minipage}
\hspace{0.5cm}
\begin{minipage}[t]{0.5\linewidth}
\small
\begin{center}*T
\end{center}
\begin{tabular}{rl}
 & \textbf{Lechuns Emuntane}\\ 
 & vuor gegen Gawane.\\ 
 & dâ wart \textbf{ein rîchiu} tjost getân,\\ 
 & daz der starke Lachdoman\\ 
5 & hinderm orse ûf dem acker lac.\\ 
 & dar nâch er sicherheite pflac,\\ 
 & der \textbf{starke} degen wert erkant.\\ 
 & diu ergienc in Gawanes hant.\\ 
 & \begin{large}D\end{large}ô streit der herzoge Astor\\ 
10 & \textbf{den zingeln} aller næhest vor.\\ 
 & dâ ergienc manec \textbf{hurteclîcher} strît.\\ 
 & \textbf{vil} 'Nantes' wart geschrît,\\ 
 & Artuses herzeichen.\\ 
 & die herten, niht die weichen,\\ 
15 & \textbf{dâ was} manec ellender Brituneiz\\ 
 & unde soldiere von \textbf{Astrigeiz}\\ 
 & ûz Ereckes lande,\\ 
 & der tât man dâ bekande.\\ 
 & \textbf{der} pflac \textbf{duc von} Lunveruns.\\ 
20 & ouch mohte Poydekuniuns\\ 
 & die Brituneise \textbf{hân} ledic \textbf{verlân}.\\ 
 & sô wart ez dâ von in getân.\\ 
 & si \textbf{wurden} Artuse\\ 
 & zer Montanien Cluse\\ 
25 & abe gevangen, dâ man strîten sach.\\ 
 & in einem sturme daz geschach.\\ 
 & si schrîten 'Nantes' nâch ir siten\\ 
 & \textbf{dâ oder} \textbf{anders}, swâ si \textbf{striten}.\\ 
 & \textbf{daz} was ir krîe unde ir art.\\ 
30 & etslîcher truoc vil \textbf{grâwen} bart.\\ 
\end{tabular}
\scriptsize
\line(1,0){75} \newline
T V W \newline
\line(1,0){75} \newline
\textbf{1} \textit{Majuskel} T  \textbf{9} \textit{Initiale} T V  \newline
\line(1,0){75} \newline
\textbf{1} Lechuns] Der graue von V LEhkoͤns W  $\cdot$ Emuntane] vitane V chinontange W \textbf{2} Gawane] gawange W \textbf{3} dâ] Do V W \textbf{4} Lachdoman] [Lachtoman]: Lachdoman T [*]: lahedvman V lohdoman W \textbf{9} Astor] kastor W \textbf{11} dâ] Do V W  $\cdot$ hurteclîcher] herter W \textbf{12} vil] Vil dike V (W)  $\cdot$ Nantes] nantis W \textbf{13} Artuses] Artus W \textbf{15} dâ] [*]: Do V Do W  $\cdot$ ellender] ellenthaffter W  $\cdot$ Brituneiz] [*]: brituneis V brituneis W \textbf{16} Astrigeiz] [*]: destrigaleis V destrigreis W \textbf{17} ûz] Von V \textbf{18} dâ] do V \textit{om.} W \textbf{19} duc von] herzoge de V duck de W  $\cdot$ Lunveruns] lvuernuns V lanuaruns W \textbf{20} Poydekuniuns] [poẏdekv*]: poẏdekvmuns V poydekomuns W \textbf{21} hân ledic verlân] all ledig lan W \textbf{22} dâ von in] von in [*]: da V do von in W \textbf{23} wurden] [w*]: warent V \textbf{24} Zer mvntanie clvse V  $\cdot$ Zuͦ montange kluse W \textbf{25} dâ] do V W \textbf{26} in] An V \textbf{27} schrîten] schrvwen V schryen W  $\cdot$ Nantes] nantis W \textbf{28} Do oder wo sy bitten W  $\cdot$ dâ] [*]: Hie V \textbf{29} daz] Auch W \textbf{30} vil] wol W \newline
\end{minipage}
\end{table}
\end{document}
