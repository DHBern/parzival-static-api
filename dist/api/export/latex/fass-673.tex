\documentclass[8pt,a4paper,notitlepage]{article}
\usepackage{fullpage}
\usepackage{ulem}
\usepackage{xltxtra}
\usepackage{datetime}
\renewcommand{\dateseparator}{.}
\dmyyyydate
\usepackage{fancyhdr}
\usepackage{ifthen}
\pagestyle{fancy}
\fancyhf{}
\renewcommand{\headrulewidth}{0pt}
\fancyfoot[L]{\ifthenelse{\value{page}=1}{\today, \currenttime{} Uhr}{}}
\begin{document}
\begin{table}[ht]
\begin{minipage}[t]{0.5\linewidth}
\small
\begin{center}*D
\end{center}
\begin{tabular}{rl}
\textbf{673} & \textit{\begin{large}I\end{large}}r m\textit{ö}ht\textbf{z} einer witwen wol tuon."\\ 
 & Artus sprach: "dîner muomen sun,\\ 
 & Gaherjeten, si dort hât\\ 
 & unt Garelen, der rîters tât\\ 
5 & in manegem poynder worhte.\\ 
 & mir wart der unervorhte\\ 
 & \textbf{an mîner sîten genomen}.\\ 
 & ein \textbf{unser} poynder was sô \textbf{komen}\\ 
 & mit hurte unz an ir barbigân.\\ 
10 & \textbf{hurtâ}, wie ez dâ wart getân\\ 
 & von dem werden Melianze von Liz!\\ 
 & under \textbf{eine} baniere wîz\\ 
 & ist er hin ûf gevangen.\\ 
 & \textbf{diu} baniere hât enpfangen\\ 
15 & von zobele ein \textbf{swarze} strâle\\ 
 & mit herzen bluotes mâle,\\ 
 & \textbf{nâch} mannes kumber gevar.\\ 
 & ›\textbf{Lirivoyn}' rief al diu schar,\\ 
 & die \textbf{under der} durch \textbf{strîten} riten;\\ 
20 & die hânt den \textbf{prîs} hin ûf erstriten.\\ 
 & mir ist ouch mîn neve Jofreit\\ 
 & hin ûf gevangen, deist mir leit.\\ 
 & diu nâchhuote was gestern mîn;\\ 
 & dâ von gedêch mir dirre pîn."\\ 
25 & Der künec sînes schaden vil verjach.\\ 
 & diu herzogîn mit zühten sprach:\\ 
 & "hêrre, ich sage iuch des lasters buoz.\\ 
 & ir \textbf{en}het mîn decheinen gruoz,\\ 
 & ir mugt mir schaden hân getân,\\ 
30 & den ich doch ungedienet hân.\\ 
\end{tabular}
\scriptsize
\line(1,0){75} \newline
D Fr8 \newline
\line(1,0){75} \newline
\textbf{1} \textit{Initiale} D  \textbf{25} \textit{Initiale} Fr8   $\cdot$ \textit{Majuskel} D  \newline
\line(1,0){75} \newline
\textbf{1} Ir möhtz] ÷R mohtz D Jr mochtz Fr8 \textbf{2} Artus] Arthus Fr8 \textbf{3} Gaherjeten] Gaherîeten D Gaharierten Fr8 \textbf{4} Garelen] Garêlen D \textbf{9} unz] \textit{om.} Fr8  $\cdot$ barbigân] barbegan D \textbf{11} Melianze] Melẏanze Fr8  $\cdot$ Liz] Lîz D Lẏz Fr8 \textbf{12} eine] einer Fr8 \textbf{15} swarze] \textit{om.} Fr8 \textbf{18} Lirivoyn] Lẏrivoẏn Fr8 \textbf{19} Die von der burch striten riten Fr8 \textbf{21} Jofreit] ẏofreit Fr8 \textbf{28} enhet mîn] bietet mir Fr8 \newline
\end{minipage}
\hspace{0.5cm}
\begin{minipage}[t]{0.5\linewidth}
\small
\begin{center}*m
\end{center}
\begin{tabular}{rl}
 & ir m\textit{ö}ht \textbf{da\textit{z}} \textit{e}iner witwe\textit{n w}ol tuon."\\ 
 & Artus sprach: "dîner muomen sun,\\ 
 & Gaherieten, si dort hât\\ 
 & und Garellen, der ritters tât\\ 
5 & in manigem poinder worhte.\\ 
 & mir wart der unervorhte\\ 
 & \textbf{an mîner sîten genomen}.\\ 
 & ein \textbf{unser} ponder was sô \textbf{komen}\\ 
 & mit hurte unz a\textit{n i}r barbigân.\\ 
10 & \textbf{hurtâ}, wie ez d\textit{â} wart getân\\ 
 & von dem werden Melianzen von Liz!\\ 
 & under \textbf{einer} banier wîz\\ 
 & ist er hin ûf ge\textit{v}angen.\\ 
 & \textbf{diu} banier het enpfangen\\ 
15 & vo\textit{n} zobele ein \textbf{swarze} strâl\\ 
 & mit herzen bluotes mâl,\\ 
 & \textbf{mit} mannes kumber gevar.\\ 
 & ›\textbf{Lirwo\textit{in}}' rief alliu diu schar,\\ 
 & die \textbf{under der} durch \textbf{strîten} riten;\\ 
20 & die hânt de\textit{n} \textbf{prîs} hin ûf erstriten.\\ 
 & mir ist ouch mîn neve Jof\textit{rei}t\\ 
 & hin ûf gevangen, daz ist mir leit.\\ 
 & diu nâchhuote was gestern mîn;\\ 
 & dâ von ged\textit{êch} mir diser pîn."\\ 
25 & der küni\textit{c} sînes schaden vil verjach.\\ 
 & diu herzogîn mit zühten sprach:\\ 
 & "hêrre, ich sage iuch des lasters buoz.\\ 
 & ir hetet mîn dekeinen gruoz,\\ 
 & ir moget mir schaden hân getân,\\ 
30 & den ich doch ungedienet hân.\\ 
\end{tabular}
\scriptsize
\line(1,0){75} \newline
m n o Fr69 \newline
\line(1,0){75} \newline
\newline
\line(1,0){75} \newline
\textbf{1} Jr moht das wol einer wittwen suͯn wol tuͯn m  $\cdot$ möht daz] moͯchten das n mocht das o \textbf{3} Gaherieten] Gaherietten m [Gahi]: Gaherieten n gaherten o Gaher:ten Fr69 \textbf{4} der] [den]: der Fr69 \textbf{5} in] An o  $\cdot$ worhte] forchte o \textbf{6} \textit{Vers 673.6 fehlt} o  \textbf{9} unz an ir] vns an den ir m  $\cdot$ barbigân] barbigon m \textbf{10} hurtâ] Hurte n hvrt Fr69  $\cdot$ dâ] do m o dort n \textbf{11} Melianzen] Meliantzen m meliantz n meliancz o melian: Fr69  $\cdot$ von Liz] von lis m do lies o \textbf{13} gevangen] gegangen m n o \textbf{14} diu] Die m n o  $\cdot$ het] hette n hat Fr69 \textbf{15} von] Vo m \textbf{17} gevar] [gemal]: gefar o \textbf{18} Lirwoin] Lirwom m Liriwom o \textbf{20} den] des m \textbf{21} Jofreit] jofert m jofrit n \textbf{24} gedêch] gedaht m gedechte n (o) \textbf{25} künic] kunnige m \textbf{28} hetet] hetten o  $\cdot$ dekeinen] do keinen n \newline
\end{minipage}
\end{table}
\newpage
\begin{table}[ht]
\begin{minipage}[t]{0.5\linewidth}
\small
\begin{center}*G
\end{center}
\begin{tabular}{rl}
 & \textit{\begin{large}I\end{large}}r m\textit{ö}h\textit{t} \textit{\textbf{ez}} einer witwen wol tuon."\\ 
 & Artus sprach: "dîner muomen sun,\\ 
 & Gahareten, si dort hât\\ 
 & unde Gareln, der rîters tât\\ 
5 & in manige\textit{m} poynder worhte.\\ 
 & mir wart der unervorhte\\ 
 & \textit{\textbf{gevangen, sus ist ez komen}}.\\ 
 & ein \textbf{langer} poynder was sô \textbf{genomen}\\ 
 & mit hurte unze an ir barbigân.\\ 
10 & \textbf{nutâ}, wie ez dâ wart getân\\ 
 & von dem werden Melianze von Liz!\\ 
 & under \textbf{einer} baniere wîz\\ 
 & ist er hin ûf gevangen.\\ 
 & \textbf{die} banier het enpfangen\\ 
15 & von zobel ein \textbf{swarziu} strâle\\ 
 & mit herzebluotes mâle,\\ 
 & \textbf{nâch} mannes kumber gevar.\\ 
 & ›\textbf{Logroys}' rief al diu schar,\\ 
 & die \textbf{drunder} durch \textbf{strîten} riten;\\ 
20 & die hânt den \textbf{strît} hin ûf erstriten.\\ 
 & mir ist ouch mîn neve Jofreit\\ 
 & hin ûf gevangen, daz ist mir leit.\\ 
 & diu nâchhuote was gester mîn;\\ 
 & dâ von gedêch mir dirre pîn."\\ 
25 & der künic sînes schaden vil verjach.\\ 
 & diu herzogîn mit zühten sprach:\\ 
 & "hêrre, ich sage iuch des lasters buoz.\\ 
 & ir\textbf{n} het mîn deheinen gruoz,\\ 
 & ir muget mir schaden hân getân,\\ 
30 & den ich doch ungedienet hân.\\ 
\end{tabular}
\scriptsize
\line(1,0){75} \newline
G I L M Z Fr61 \newline
\line(1,0){75} \newline
\textbf{1} \textit{Initiale} G I L Z  \textbf{17} \textit{Initiale} I  \textbf{25} \textit{Initiale} Fr61  \newline
\line(1,0){75} \newline
\textbf{1} Ir] Er G  $\cdot$ möht ez] mohte G mochtet L M Fr61 mohtez Z \textbf{2} Artus] Artuͯs L Artaus Fr61 \textbf{3} Gahareten] Gahereten I L Gaheren M Gaheieten Z Gahireten Fr61 \textbf{4} Gareln] Garebi I karlin L Garelen Fr61 \textbf{5} manigem] manigen G  $\cdot$ poynder] poýnde L (M) \textbf{6} mir wart der] mit tat er Fr61  $\cdot$ unervorhte] vnerworchte M \textbf{7} \textit{Vers 673.7 fehlt} G   $\cdot$ An miner siten (seiten wart Fr61 ) genomen L (M) Z (Fr61) \textbf{8} Ein vnser poýnder waz so komen L (M) (Z) (Fr61)  $\cdot$ sô] \textit{om.} I \textbf{9} hurte unze] husz M  $\cdot$ ir] die L \textbf{10} nutâ] warta I Huͯrta L (Z) Nura M Avoẏ Fr61 \textbf{11} dem] den Z  $\cdot$ Melianze] melianz I Meliantzen Z Meliantz Fr61  $\cdot$ Liz] lýz L lisz M [liez]: :eiz Fr61 \textbf{12} einer] eine Z \textbf{13} ist] wart I \textbf{14} het] het er I hat Z  $\cdot$ enpfangen] bevangen L Fr61 \textbf{15} swarziu] swarzie L swartzen Fr61 \textbf{16} mit herzebluotes] von herzen pluͦtes I \textbf{17} gevar] [var]: gevar L \textbf{18} Logroys] Logrois G Lyravoin L Liranzcin M Lyrivoin Z Lẏravoyn Fr61 \textbf{19} drunder] [drun*]: drunder I \textbf{20} strît] pris Z  $\cdot$ hin ûf] \textit{om.} I  $\cdot$ erstriten] er stiten L \textbf{21} ouch] \textit{om.} Fr61  $\cdot$ Jofreit] Iofreit G (M) Lofreit I Johfræit Fr61 \textbf{24} dirre] disen Fr61 \textbf{25} sînes schaden vil] sines vil L vil seins schadens Fr61 \textbf{26} herzogîn] [chungin m]: herzogin I [hertz*]: hertzogin L \textbf{27} des] \textit{om.} Fr61 \textbf{28} irn] Jr L M (Fr61)  $\cdot$ gruoz] [buͤz]: [suͤz]: Gruͤz I \textbf{30} ungedienet] verdienet I niht gedient L (Fr61) vnuordienet M \newline
\end{minipage}
\hspace{0.5cm}
\begin{minipage}[t]{0.5\linewidth}
\small
\begin{center}*T
\end{center}
\begin{tabular}{rl}
 & i\textit{r} m\textit{ö}ht\textbf{z} einer witwen wol tuon."\\ 
 & Artus sprach: "dîner muomen sun,\\ 
 & Gaherieten, si dort hât\\ 
 & und Gareln, der ritters tât\\ 
5 & in manegem poynder worhte.\\ 
 & mir wart der unervorhte\\ 
 & \textbf{an mîner sîten genomen}.\\ 
 & ein \textbf{un\textit{s}er} poynder was sô \textbf{komen}\\ 
 & mit hurte unz an ir barbigân.\\ 
10 & \textbf{hurte}, wie ez d\textit{â} wart getân\\ 
 & von dem werden Melyanze von Liz!\\ 
 & under \textbf{eine\textit{r}} banier wîz\\ 
 & ist er hin ûf gevangen.\\ 
 & \textit{\textbf{die} banier hât enpfangen}\\ 
15 & von zobel ein \textbf{swarziu} strâle\\ 
 & mit herzebluotes mâle,\\ 
 & \textbf{nâch} manne\textit{s} kumbe\textit{r} gevar.\\ 
 & ›\textbf{Lyravoyn}' rief al diu schar,\\ 
 & die \textbf{drunder} durch \textbf{strîte} riten;\\ 
20 & die habent den \textbf{strît} hin ûf erstriten.\\ 
 & mir ist ouch mîn neve Jofreit\\ 
 & hin ûf gevangen, daz ist mir leit.\\ 
 & diu nâchhuote was gester mîn;\\ 
 & dâ von gedêc\textit{h} mir disiu pîn."\\ 
25 & der künic sînes schaden vil verjach.\\ 
 & diu herzogîn mit zühten sprach:\\ 
 & "hêrre, ich sag iuch des lasters buoz.\\ 
 & ir \textbf{en}het mîn keinen gruoz,\\ 
 & ir mogt mir schaden hân getân,\\ 
30 & den ich doch ungedienet hân.\\ 
\end{tabular}
\scriptsize
\line(1,0){75} \newline
Q R W V \newline
\line(1,0){75} \newline
\textbf{1} \textit{Initiale} Q  \textbf{25} \textit{Initiale} V  \newline
\line(1,0){75} \newline
\textbf{1} Jr [mohten* *wen]: mohtentz einer witwen wol tvn V  $\cdot$ ir] Jch Q  $\cdot$ möhtz] mochts Q  $\cdot$ witwen] wirtin W \textbf{3} Gaherieten] Kacheriten R [G*]: Gaherietten V \textbf{4} Gareln] Kareln R karellen W [karel*]: karellen V \textbf{5} manegem] menger R \textbf{6} \textit{Vers 673.6 fehlt} R  \textbf{7} sîten] site V \textbf{8} unser] vnter Q vnsz R  $\cdot$ sô] also W \textbf{10} hurte] Hurta R W (V)  $\cdot$ dâ] do Q W V \textbf{11} werden] werde R  $\cdot$ Melyanze] melianze Q (R) V melyanz W  $\cdot$ Liz] lisz Q lis W \textbf{12} einer] einen Q \textbf{13} hin] hie R \textbf{14} \textit{Vers 673.14 fehlt} Q  \textbf{15} swarziu] schwarcze R \textbf{17} Nach [mann*]: mannes kvmber gevar V  $\cdot$ mannes kumber] manne kummers Q \textbf{18} Lyravoyn] Liravoyn Q Lyrauoyn R W Jora ioͮ V \textbf{19} strîte] striten V \textbf{20} strît] pris R (W) V \textbf{21} Jofreit] Joferit R iofreit W V \textbf{22} mir leit] min [nit]: nid R \textbf{24} gedêch] gedecht Q gedenket V  $\cdot$ disiu] dirre R \textbf{25} vil] nit R \textbf{28} enhet] hetten R \newline
\end{minipage}
\end{table}
\end{document}
