\documentclass[8pt,a4paper,notitlepage]{article}
\usepackage{fullpage}
\usepackage{ulem}
\usepackage{xltxtra}
\usepackage{datetime}
\renewcommand{\dateseparator}{.}
\dmyyyydate
\usepackage{fancyhdr}
\usepackage{ifthen}
\pagestyle{fancy}
\fancyhf{}
\renewcommand{\headrulewidth}{0pt}
\fancyfoot[L]{\ifthenelse{\value{page}=1}{\today, \currenttime{} Uhr}{}}
\begin{document}
\begin{table}[ht]
\begin{minipage}[t]{0.5\linewidth}
\small
\begin{center}*D
\end{center}
\begin{tabular}{rl}
\textbf{554} & \begin{large}D\end{large}iu magt ir dienstes niht vergaz;\\ 
 & \textbf{vür d\textit{az} bette ûfn teppech} saz\\ 
 & diu clâre juncvrouwe.\\ 
 & bî mir ich selten schouwe,\\ 
5 & daz mir âbents oder vruo\\ 
 & sölch âventiure \textbf{slîche} zuo.\\ 
 & \textbf{Bî einer wîle} Gawan erwachete;\\ 
 & er sach \textbf{an si} und lachete\\ 
 & \textbf{unt sprach}: "got \textbf{halde} iuch, \textbf{vröuwelîn},\\ 
10 & daz ir durch den willen mîn\\ 
 & iwern slâf sus brechet\\ 
 & unt an iu selber rechet,\\ 
 & \textbf{des} ich niht hân gedienet \textbf{gar}."\\ 
 & Dô sprach \textit{diu} magt wol gevar:\\ 
15 & "\textbf{iwers dienstes} \textbf{wil} ich \textbf{enbern};\\ 
 & ich \textbf{ensol} niwan \textbf{hulde} gern.\\ 
 & hêrre, gebiet über mich;\\ 
 & swaz ir gebiet, daz \textbf{leist} ich.\\ 
 & al die mit mînem vater sint,\\ 
20 & beidiu mîn muoter unt ir kint\\ 
 & sulen iuch \textbf{ze hêrren immer} hân;\\ 
 & \textbf{sô liebe habt ir} uns getân."\\ 
 & Er sprach: "sît ir iht lange komen?\\ 
 & het ich iwer kunft ê vernomen,\\ 
25 & daz wære mir liep durch vrâgen,\\ 
 & wolt iuch des niht betrâgen,\\ 
 & daz ir mirz \textbf{geruochet} sagen:\\ 
 & ich hân in \textbf{disen} zwein tagen\\ 
 & vil vrouwen ob \textbf{mir} gesehen;\\ 
30 & von den sult ir mir verjehen\\ 
\end{tabular}
\scriptsize
\line(1,0){75} \newline
D Fr7 \newline
\line(1,0){75} \newline
\textbf{1} \textit{Initiale} D Fr7  \textbf{7} \textit{Majuskel} D  \textbf{14} \textit{Majuskel} D  \textbf{23} \textit{Majuskel} D  \newline
\line(1,0){75} \newline
\textbf{1} dienstes] diens D \textbf{2} daz] des D \textbf{14} diu] \textit{om.} D \textbf{15} dienstes] diens D \newline
\end{minipage}
\hspace{0.5cm}
\begin{minipage}[t]{0.5\linewidth}
\small
\begin{center}*m
\end{center}
\begin{tabular}{rl}
 & diu magt ir dienstes niht vergaz;\\ 
 & \textbf{vür daz bette ûf den teppic\textit{h}} \textit{s}az\\ 
 & diu clâre juncvrouwe.\\ 
 & bî mir i\textit{ch} \textit{selten} schouwe,\\ 
5 & daz mir âbents oder vruo\\ 
 & solich âventiur \textbf{sleich} zuo.\\ 
 & \textbf{bî einer wîle} Gawan erwachte;\\ 
 & er sach \textbf{an s\textit{i}} und lachte\\ 
 & \textbf{und sprach}: "got \textbf{halt} iuch, \textbf{vröuwelîn},\\ 
10 & daz ir durch den willen mîn\\ 
 & iuwern slâf sus brechet\\ 
 & und an iu selber rechet,\\ 
 & \textbf{daz} ich niht hân gediene\textit{t} \textbf{\textit{d}ar}."\\ 
 & dô sprach diu maget wol gevar:\\ 
15 & "\textbf{iuwer dienst} \textbf{wil} ich \textbf{mêrn};\\ 
 & ich \textbf{ensol} niht wan \textbf{hulde} gern.\\ 
 & hêrre, gebietet über mich;\\ 
 & waz ir gebietet, daz \textbf{leiste} ich.\\ 
 & aldie mit mînem vater sint,\\ 
20 & beidiu mîn muoter und ir kint\\ 
 & soln iuch \textbf{iemer zuo hêrren} hân;\\ 
 & \textbf{ir habt sô wol zuo} uns getân."\\ 
 & er sprach: "sît ir iht lange komen?\\ 
 & het ich iuwer \textit{kunft} ê vernomen,\\ 
25 & daz wær mir liep durch vrâgen,\\ 
 & wolt iuch des niht betrâgen,\\ 
 & daz ir mirz \textbf{geruochet} sagen:\\ 
 & ich hân in zwein tagen\\ 
 & vil vrouwen ob \textbf{mir} gesehen;\\ 
30 & von den sullet ir mir verjehen\\ 
\end{tabular}
\scriptsize
\line(1,0){75} \newline
m n o \newline
\line(1,0){75} \newline
\newline
\line(1,0){75} \newline
\textbf{2} \textit{Versdoppelung (\textasciicircum2o); Lesarten des vorausgehenden Verses mit \textasciicircum1o bezeichnet} o   $\cdot$ Vor den deppich sas \textsuperscript{1}\hspace{-1.3mm} o  $\cdot$ teppich saz] teppich sach vnd sas m teppich sú sasz n \textbf{4} ich selten] iht m \textbf{5} vruo] fro o \textbf{8} si] sich m \textbf{9} und] [O]: Vnd n  $\cdot$ vröuwelîn] frowelich o \textbf{13} gedienet dar] gedienet dan vnd dar m \textbf{15} mêrn] enbern n o \textbf{21} soln iuch iemer] Solt úch iemen o \textbf{22} sô] zuͦ o  $\cdot$ zuo] an n  $\cdot$ uns] [vnd]: vns m \textbf{24} kunft] \textit{om.} m \textbf{27} ir mirz] irs mir n  $\cdot$ geruochet] gerúchen o \textbf{28} zwein] disen zwein n (o) \newline
\end{minipage}
\end{table}
\newpage
\begin{table}[ht]
\begin{minipage}[t]{0.5\linewidth}
\small
\begin{center}*G
\end{center}
\begin{tabular}{rl}
 & \begin{large}D\end{large}iu maget ir dienstes niht vergaz;\\ 
 & \textbf{vürz bette ûffen tepich} \textbf{si} sa\textit{z},\\ 
 & diu clâre juncvrouwe.\\ 
 & bî mir ich selten schouwe,\\ 
5 & daz mir âbendes oder vruo\\ 
 & solch âventiure \textbf{slîche} zuo.\\ 
 & \textbf{bî einer wîle} Gawan erwachet;\\ 
 & er sach \textbf{an s\textit{i}} unde \textit{l}achet\\ 
 & \textbf{unde sprach}: "got \textbf{halde} iuch, \textbf{vröuwelîn},\\ 
10 & daz ir durch den willen mîn\\ 
 & iuwern slâf sus brechet\\ 
 & unde an iu selber rechet,\\ 
 & \textbf{des} ich niht hân gedienet \textbf{gar}."\\ 
 & dô sprach diu maget wolgevar:\\ 
15 & "\textbf{iuwer\textit{s} dienstes} \textbf{wil} ich \textbf{enbern};\\ 
 & ich \textbf{ensol} niht wan \textbf{hulde} gern.\\ 
 & hêrre, gebiete\textit{t} über mich;\\ 
 & swaz ir gebiete\textit{t}, daz \textbf{leiste} ich.\\ 
 & al die mit mînem vater sint,\\ 
20 & beidiu mîn muoter unde ir kint\\ 
 & suln iuch \textbf{ze hêrren immer} hân;\\ 
 & \textbf{sô liebe habet ir} uns getân."\\ 
 & er sprach: "sît ir iht lange komen?\\ 
 & het ich iuwer kunft ê vernomen,\\ 
25 & daz wære mir lieb durch vrâgen,\\ 
 & wolt iuch des niht betrâgen,\\ 
 & daz ir mirz \textbf{geruochet} sagen:\\ 
 & ich hân in \textbf{disen} zwein tagen\\ 
 & vil vrouwen obe \textbf{mir} gesehen;\\ 
30 & von den sult ir mir verjehen\\ 
\end{tabular}
\scriptsize
\line(1,0){75} \newline
G I L M Z Fr23 Fr62 \newline
\line(1,0){75} \newline
\textbf{1} \textit{Initiale} G L Z Fr23 Fr62  \textbf{3} \textit{Initiale} I  \textbf{23} \textit{Initiale} I  \newline
\line(1,0){75} \newline
\textbf{1} dienstes] dienst I (Fr23) \textbf{2} vürz bette] vur den Gast I  $\cdot$ ûffen] uff eyn M  $\cdot$ si] \textit{om.} Fr62  $\cdot$ saz] sach G [gesach]: gesaz I \textbf{4} ich selten] icht selten M selten ich Z \textbf{5} âbendes] spate Fr62 \textbf{6} slîche] slýchen L (M) sleich Fr23 \textbf{7} Gawan sint schiere erwachete Fr62 \textbf{8} si] sich G  $\cdot$ lachet] erlachet G lachete I (L) (M) (Z) Fr62 \textbf{9} unde] Er M Fr23 \textbf{11} sus] alsus M durch mih Fr23 \textbf{12} selber] selbe I selbin M (Fr23) \textbf{14} dô] Da M \textbf{15} iuwers] Iuwer G  $\cdot$ dienstes] dienst Fr23  $\cdot$ wil] wold Fr23 \textbf{16} ensol] sol I Fr23 Fr62  $\cdot$ niht wan] nevr Z  $\cdot$ hulde] ewer hulde I (L) hulden Fr62 \textbf{17} gebietet] gebiette G \textbf{18} swaz] Waz L (M)  $\cdot$ gebietet] gebiete G welt I  $\cdot$ leiste] tuͤn I \textbf{19} al] vnd alle I  $\cdot$ mînem] minen I Fr62 \textbf{20} beidiu] \textit{om.} I  $\cdot$ ir] \textit{om.} M \textbf{21} hêrren] eren Fr23  $\cdot$ immer] \textit{om.} I \textbf{22} liebe] lip Fr23 \textbf{23} iht] \textit{om.} I (Fr23) \textbf{24} iuwer kunft] iwer chraft Fr23 uch Fr62  $\cdot$ ê] e hie Fr62 \textbf{25} wære] war Fr23  $\cdot$ lieb] \textit{om.} Z \textbf{27} mirz] mir Fr23  $\cdot$ geruochet] geruͤchtet I (L) (M) (Fr62) \textbf{28} ich] Jn Z  $\cdot$ hân] bin Fr23 \textbf{29} obe] abe M boben Fr62  $\cdot$ mir] mir da Fr23 \textbf{30} mir] \textit{om.} L \newline
\end{minipage}
\hspace{0.5cm}
\begin{minipage}[t]{0.5\linewidth}
\small
\begin{center}*T
\end{center}
\begin{tabular}{rl}
 & \textit{\begin{large}D\end{large}}iu maget ir dienstes niht vergaz;\\ 
 & \textbf{ûf den teppich vür daz bette} \textbf{si} saz,\\ 
 & diu clâre juncvrouwe.\\ 
 & bî mir ich selten schouwe,\\ 
5 & daz mir âbendes oder vruo\\ 
 & sölch âventiure \textbf{slîche} zuo.\\ 
 & \textbf{Vil schiere} Gawan erwachete;\\ 
 & er sach \textbf{si an} unde lachete:\\ 
 & "got \textbf{grüeze} iuch, \textbf{juncvröuwelîn},\\ 
10 & daz ir durch den willen mîn\\ 
 & iuwern slâf sus brechet\\ 
 & unde an iu selben rechet,\\ 
 & \textbf{daz} ich niht hân gedienet \textbf{gar}."\\ 
 & Dô sprach diu maget wol gevar:\\ 
15 & "\textbf{iuwers dienstes} \textbf{sol} ich \textbf{enbern},\\ 
 & \textbf{wan} ich \textbf{wil} niuwan \textbf{hulden} gern.\\ 
 & hêrre, gebiet über mich;\\ 
 & swaz ir gebiet, daz \textbf{tuon} ich.\\ 
 & alle die mit mînem vater sint,\\ 
20 & beidiu mîn muoter unde ir kint\\ 
 & suln iuch \textbf{ze hêrren iemer} hân;\\ 
 & \textbf{sô liebe habt ir} uns getân."\\ 
 & Er sprach: "sît \textit{ir} iht lange komen?\\ 
 & het ich iuwer kunft ê vernomen,\\ 
25 & daz wære mir liep durch vrâgen,\\ 
 & woltiuch des niht betrâgen,\\ 
 & daz ir mirz \textbf{geruochtet} sagen:\\ 
 & ich hân in \textbf{disen} zwein tagen\\ 
 & vil vrouwen ob \textbf{uns} gesehen;\\ 
30 & von den sult ir mir verjehen\\ 
\end{tabular}
\scriptsize
\line(1,0){75} \newline
T U V W O Q R Fr39 \newline
\line(1,0){75} \newline
\textbf{1} \textit{Initiale} T V O Q Fr39   $\cdot$ \textit{Capitulumzeichen} R  \textbf{7} \textit{Majuskel} T  \textbf{14} \textit{Majuskel} T  \textbf{23} \textit{Majuskel} T  \newline
\line(1,0){75} \newline
\textbf{1} \textit{Die Verse 553.1-599.30 fehlen} U   $\cdot$ Diu] ÷iv T O \textbf{2} ûf den] vffens V  $\cdot$ si] \textit{om.} V W O Q R Fr39 \textbf{4} schouwe] schawen Q \textbf{6} \textit{teilweise Textverlust 554.6-16 (Blatt teils abgeschnitten)} O  \textbf{7} Gawan] Gaw::: O Gawin R \textbf{8} si an] an sy W (O) (Q) R (Fr39) \textbf{9} Vnde sprach got halte úch vrowelin V  $\cdot$ iuch] iv T dich Fr39 \textbf{10} ir] ich Fr39 \textbf{11} sus] also W (Q)  $\cdot$ brechet] brechen R \textbf{12} unde] Wann W  $\cdot$ selben] selber W R  $\cdot$ rechet] rechen R \textbf{14} wol] liecht W \textbf{16} \textit{korrigierende Versdoppelung (Anteil aus Vers 554.15 im ersten Vers wird im zweiten Vers eliminiert):} Jch wil úwer huld enbern / Jch wil uwer hulde gern R   $\cdot$ wan] \textit{om.} V W O Q Fr39  $\cdot$ wil] sol W  $\cdot$ niuwan] alleine eúwer W  $\cdot$ hulden] hulde V W (O) Q (Fr39) \textbf{17} gebiet] nun gebietent W \textbf{18} swaz] Was W Q R  $\cdot$ gebiet] geheissent V \textbf{20} Beide [*]: muͦter vnde kint V  $\cdot$ muoter] vater Q  $\cdot$ ir] \textit{om.} Q \textbf{21} suln] Sol W  $\cdot$ iuch] iv T \textbf{22} sô] Soclhe W  $\cdot$ uns] an vnß W \textbf{23} ir] \textit{om.} T  $\cdot$ lange] langes Q \textbf{24} ê] \textit{om.} O \textbf{25} daz] Dar Q \textbf{26} woltiuch] woltiv T \textbf{27} ir] irs R  $\cdot$ geruochtet] geruͦchent V (W) (Q) (R) (Fr39) gervͦchte O \textbf{28} zwein] treyen Q \textbf{29} uns] vnz alhie V vns hie R Fr39 \newline
\end{minipage}
\end{table}
\end{document}
