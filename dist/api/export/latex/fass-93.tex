\documentclass[8pt,a4paper,notitlepage]{article}
\usepackage{fullpage}
\usepackage{ulem}
\usepackage{xltxtra}
\usepackage{datetime}
\renewcommand{\dateseparator}{.}
\dmyyyydate
\usepackage{fancyhdr}
\usepackage{ifthen}
\pagestyle{fancy}
\fancyhf{}
\renewcommand{\headrulewidth}{0pt}
\fancyfoot[L]{\ifthenelse{\value{page}=1}{\today, \currenttime{} Uhr}{}}
\begin{document}
\begin{table}[ht]
\begin{minipage}[t]{0.5\linewidth}
\small
\begin{center}*D
\end{center}
\begin{tabular}{rl}
\textbf{93} & dô sprach der künec Hardiz:\\ 
 & "nû kêrt an manheit iwern vlîz.\\ 
 & ob ir manheit kunnet tragen,\\ 
 & sô sult ir leit ze mâzen klagen."\\ 
5 & \textbf{Sîn kumber leider was} ze grôz.\\ 
 & ein güsse im von den ougen vlôz.\\ 
 & er schuof den rittern ir gemach\\ 
 & unt gienc, dâ er sîne kamern sach,\\ 
 & ein \textbf{kleine} gezelt von samît.\\ 
10 & die naht er dolte jâmers zît.\\ 
 & \textbf{\begin{large}A\end{large}ls der ander} tac erschein,\\ 
 & si wurden \textbf{alle des} enein,\\ 
 & die innern unt daz ûzer her,\\ 
 & \textbf{swer} dâ mit strîteclîcher wer\\ 
15 & \textbf{wære}, junc oder alt\\ 
 & oder blœde oder balt,\\ 
 & \textbf{die}\textbf{ne} \textbf{solten} tjustieren niht.\\ 
 & dô schein der mitte morgen lieht.\\ 
 & si wâren mit strîte \textbf{sô} \textbf{verriben}\\ 
20 & unt ors mit sporn \textbf{alsô} \textbf{vertriben},\\ 
 & daz die \textbf{vrechen} ritterschaft\\ 
 & \textbf{ie dennoch twanc} der müede kraft.\\ 
 & Diu künegîn reit dô selbe\\ 
 & nâch \textbf{den} werden \textbf{hin} ze velde\\ 
25 & unt \textbf{brâht die} mit ir in die stat.\\ 
 & die besten si dort inne bat,\\ 
 & daz si zer Leoplane riten.\\ 
 & dô\textbf{ne} wart ir bete niht vermiten.\\ 
 & Si \textbf{kômen}, dâ man messe sanc\\ 
30 & dem trûrigen künege von Zazamanc.\\ 
\end{tabular}
\scriptsize
\line(1,0){75} \newline
D \newline
\line(1,0){75} \newline
\textbf{5} \textit{Majuskel} D  \textbf{11} \textit{Initiale} D  \textbf{23} \textit{Majuskel} D  \textbf{29} \textit{Majuskel} D  \newline
\line(1,0){75} \newline
\textbf{1} Hardiz] Hardŷz D \textbf{30} Zazamanc] Zazamanch D \newline
\end{minipage}
\hspace{0.5cm}
\begin{minipage}[t]{0.5\linewidth}
\small
\begin{center}*m
\end{center}
\begin{tabular}{rl}
 & dô sprach der künec Hardiz:\\ 
 & "nû k\textit{ê}ret an manheit \textit{iuwe}ren vlîz.\\ 
 & ob ir manheit kunnet tragen,\\ 
 & sô sollet ir leit ze mâze klagen."\\ 
5 & \textbf{sîn kumber leider was} zuo grôz.\\ 
 & ein güsse ime von den ougen vlôz.\\ 
 & er schuof den rittern ir gemach\\ 
 & und gienc, d\textit{â} er sîne kameren sach,\\ 
 & ein \textbf{kleine} gezelt von samît.\\ 
10 & die naht er dolte jâmers zît.\\ 
 & \textbf{\begin{large}A\end{large}ls der ander} tac erschein,\\ 
 & si wurden \textbf{alle des} in \textit{e}in,\\ 
 & die inren und daz ûzer her,\\ 
 & \textbf{wer} dâ mit strîteclîcher wer\\ 
15 & \textbf{wære}, junc oder alt\\ 
 & oder blœde oder balt,\\ 
 & \textbf{die} \textbf{en}\textbf{solten} justieren niht.\\ 
 & dô schein der mitte morgen lieht.\\ 
 & si wâren mit strîte \textbf{wol} \textbf{verriben}\\ 
20 & und \textbf{diu} ros mit sporn \textbf{als} \textbf{über\textit{trib}en},\\ 
 & d\textit{az} die \textbf{vrechen} ritterschaft\\ 
 & \textbf{ie dannoch twanc} der müede kraft.\\ 
 & diu künigîn reit dô selbe\\ 
 & nâch \textbf{den} werden \textbf{hin} ze velde\\ 
25 & und \textbf{brâhte si} mit ir in die stat.\\ 
 & die besten si dort inn\textit{e} \textit{b}at,\\ 
 & daz si zer Lewe plane riten.\\ 
 & dô \textbf{en}wart ir bete niht vermiten.\\ 
 & si \textbf{kâmen}, d\textit{â} man mess\textit{e} \textit{s}anc\\ 
30 & dem trûrigen künic von Zazamanc,\\ 
\end{tabular}
\scriptsize
\line(1,0){75} \newline
m n o \newline
\line(1,0){75} \newline
\textbf{11} \textit{Illustration mit Überschrift:} Also die herren mit grosser macht von der burg enweg ritten n (o)   $\cdot$ \textit{Initiale} m n o  \newline
\line(1,0){75} \newline
\textbf{1} Hardiz] hardis m n o \textbf{2} kêret] koͯrent m  $\cdot$ iuweren] iren m n (o) \textbf{3} kunnet] kuͯndent o \textbf{4} sô] Sie o  $\cdot$ mâze] mossen n o  $\cdot$ klagen] [tragen]: klagen m \textbf{5} zuo] so n o \textbf{6} güsse] gosz n gruͦs o  $\cdot$ den] dem o \textbf{8} dâ] do m n o  $\cdot$ sîne] sin o  $\cdot$ kameren] komer n kamerer o \textbf{10} jâmers zît] komers vnd jomers sit n \textbf{12} wurden] wuͯrden o  $\cdot$ in ein] inin m \textbf{13} daz] [vsz]: das n \textbf{14} dâ] do n o \textbf{16} oder] \textit{om.} n o \textbf{18} mitte] mitten n o  $\cdot$ lieht] [niht]: liht o \textbf{19} wol] so n o  $\cdot$ verriben] vertriben o \textbf{20} als] \textit{om.} n o  $\cdot$ übertriben] uber laden m \textbf{21} daz] Da m \textbf{26} besten] beste o  $\cdot$ inne bat] inne was vnd bat m \textbf{27} zer] [zerrittenet]: [zerri*tenet]: zerr m  $\cdot$ Lewe plane] lewep lane m planẏe n (o) \textbf{28} enwart] wart n o \textbf{29} dâ] do m n o  $\cdot$ messe sanc] messe vant vnd sang m \textbf{30} Zazamanc] zazamang m n (o) \newline
\end{minipage}
\end{table}
\newpage
\begin{table}[ht]
\begin{minipage}[t]{0.5\linewidth}
\small
\begin{center}*G
\end{center}
\begin{tabular}{rl}
 & dô sprach der künic Hardiz:\\ 
 & "nû kêret an manheit iweren vlîz.\\ 
 & obe ir manheit kunnet tragen,\\ 
 & sô sult ir leit ze mâze klagen."\\ 
5 & \textbf{dô was sîn kumber} \textbf{al}ze grôz.\\ 
 & ein güsse im von den ougen vlôz.\\ 
 & er schuof den rîteren ir gemach\\ 
 & unde gienc, dâ er sîne kamer sach,\\ 
 & ein \textbf{wênic} gezelt von samît.\\ 
10 & die naht er dolte jâmers zît.\\ 
 & \textbf{des morgens, dô der} tac erschein,\\ 
 & si wurden \textbf{algelîche} enein,\\ 
 & die inneren und daz ûzer her,\\ 
 & \textbf{daz} dâ \textbf{was} mit strîticlîcher wer,\\ 
15 & \textbf{si w\textit{æ}ren} junc oder alt\\ 
 & oder blœde oder balt,\\ 
 & \textbf{si}\textbf{ne} \textbf{solden} tjostieren niht.\\ 
 & dô schein der mitter morgen lieht.\\ 
 & si wâren mit strîte \textbf{alsô} \textbf{verriben}\\ 
20 & unt \textbf{diu} ors mit sporen \textbf{sô} \textbf{vertriben},\\ 
 & daz \textbf{al} die \textbf{vrechen} rîterschaft\\ 
 & \textbf{twanc iedoch} der müede kraft.\\ 
 & diu künigîn reit dô selbe\\ 
 & nâch \textbf{dem} werden ze velde\\ 
25 & unde \textbf{vuorte si} mit ir in die stat.\\ 
 & \begin{large}D\end{large}ie besten si dort inne bat,\\ 
 & daz si zer Lewen plange riten.\\ 
 & dâ wart ir bete niht vermiten.\\ 
 & si \textbf{vuoren}, dâ man messe sanc\\ 
30 & dem trûrigen künige von Zazamanc.\\ 
\end{tabular}
\scriptsize
\line(1,0){75} \newline
G I O L M Q R Z Fr21 \newline
\line(1,0){75} \newline
\textbf{1} \textit{Initiale} O  \textbf{11} \textit{Initiale} I L R Z Fr21  \textbf{26} \textit{Initiale} G  \newline
\line(1,0){75} \newline
\textbf{1} dô] ÷o O Da M  $\cdot$ Hardiz] hardis R \textbf{2} vlîz] Hiz Fr21 \textbf{3} manheit] mit manheit L (M) \textbf{4} leit] or leit M  $\cdot$ ze mâze] zerehte I mit zmaze O zu mere R zv mazzen Z \textbf{5} dô] Da M Z  $\cdot$ alze] also R \textbf{6} güsse] gússin R  $\cdot$ vlôz] [ran]: floz M do flosz Q \textbf{7} ir] irn Z \textbf{8} unde] Er Fr21  $\cdot$ dâ] do O Q  $\cdot$ kamer sach] kamariͤe shach I \textbf{9} ein] Jn Q  $\cdot$ wênic] wenigz O (Q) [clein]: cleine  R \textbf{10} naht] [nach]: nacht R  $\cdot$ dolte] dolt I Fr21 dolt in O dochte M dolt er Q  $\cdot$ zît] sit I \textbf{11} dô] da M Z  $\cdot$ erschein] her schein Q \textbf{12} algelîche] alle gelich O (Q) (R) (Z) (Fr21) \textbf{14} dâ] do Q \textit{om.} R  $\cdot$ strîticlîcher] strites O \textbf{15} wæren] waren G L Fr21 \textbf{16} blœde] broͤde I [blvge]: blvege O \textbf{17} sine] Si O Fr21  $\cdot$ tjostieren] tuͯrnieren L (R) trostiren Q \textbf{18} dô] Da M Z  $\cdot$ schein] irschein M  $\cdot$ der] des Q  $\cdot$ mitter] werde O mitten L Z [frawe]: frwe Q fruͦ R lieht Fr21  $\cdot$ morgen] morgens Q \textbf{19} alsô] alsus I \textit{om.} R so Fr21  $\cdot$ verriben] bleben M vertriben R \textbf{20} mit sporen sô] alsus mit sporn I mit sporn O mit sporn also L M (Q) Z also mit spern R  $\cdot$ vertriben] getriben R \textbf{21} vrechen] werde I freche R \textbf{22} twanc iedoch] Je dannoch twanch O (L) (Fr21) Dannoch twanc M (Q) (R) Z  $\cdot$ der] die R  $\cdot$ müede] minne O Fr21 \textbf{23} reit dô selbe] reit do sebe I zuͯ velde reit L reit da selbe M R Z \textbf{24} dem] den I O L M Q Z Fr21  $\cdot$ werden] werden hin I (Z) helden L  $\cdot$ ze velde] vil gemeit L \textbf{25} vuorte si] furt si I (O) furt Q fuͦrttes R [fvr*t]: fvrt Fr21  $\cdot$ in die] \textit{om.} Z  $\cdot$ stat] [tat]: stat Fr21 \textbf{26} dort] da Z \textbf{27} zer] zuͤ ir I  $\cdot$ Lewen plange] lewe planie G lewen plangen I lewe planîe O Lew planie L lewe plane M leuplange Q leit plange R leo plane Z liͤwe planie Fr21 \textbf{28} dâ] Danne M (Z) Do Q (R)  $\cdot$ ir] diu I \textbf{29} dâ] do Q \textbf{30} dem] den I  $\cdot$ trûrigen] truren M trurrigē R trvrigem Fr21  $\cdot$ künige] \textit{om.} I  $\cdot$ von] zu Q  $\cdot$ Zazamanc] zazamanch G O L zazamant Q zesmanc R \newline
\end{minipage}
\hspace{0.5cm}
\begin{minipage}[t]{0.5\linewidth}
\small
\begin{center}*T (U)
\end{center}
\begin{tabular}{rl}
 & dô sprach der künec Hardiz:\\ 
 & "nû kêret an manheit iuwern vlîz,\\ 
 & \textbf{und} ob ir manheit kunne\textit{t} tragen,\\ 
 & sô solt ir leit ze mâze klagen."\\ 
5 & \textbf{dô was sîn kumber} \textbf{al} zuo grôz.\\ 
 & ein güsse im von den ougen vlôz.\\ 
 & er schuof den rittern ir gemach\\ 
 & und gienc, dâ er sîne kamer sach,\\ 
 & ein \textbf{wênic} gezelt von samît.\\ 
10 & die nah\textit{t} er dolte jâmers zît.\\ 
 & \textbf{\begin{large}D\end{large}es morgens, dô der} tac erschein,\\ 
 & si wurden \textbf{des alle} in ein,\\ 
 & die innern und daz ûzer her,\\ 
 & \textbf{daz} dâ \textit{m}it strîtlîcher wer\\ 
15 & \multicolumn{1}{l}{ - - - }\\ 
 & \multicolumn{1}{l}{ - - - }\\ 
 & \textbf{si} \textbf{wolten} jostieren niht.\\ 
 & dô schein der mitten morgen lieht.\\ 
 & si wâren mit strîte \textbf{als} \textbf{vertriben}\\ 
20 & und \textbf{d\textit{iu}} \textit{ors} mit sporn \textbf{alsô} \textbf{verriben},\\ 
 & daz \textbf{al} die \textbf{werde} ritterschaft\\ 
 & \textbf{twanc dannoch} d\textit{er} müede kraft.\\ 
 & diu künegîn reit dô selbe\\ 
 & nâch \textbf{den} werden \textbf{hin} zuo vel\textit{d}e\\ 
25 & und \textbf{vuorte si} mit ir in die stat.\\ 
 & die besten si dort inne bat,\\ 
 & daz si zuo der Lewe planie riten.\\ 
 & dô wart ir bete niht vermiten.\\ 
 & si \textbf{vuoren}, d\textit{â} man messe sanc\\ 
30 & dem trûrigen künige von Zazamanc.\\ 
\end{tabular}
\scriptsize
\line(1,0){75} \newline
U V W T \newline
\line(1,0){75} \newline
\textbf{1} \textit{Initiale} T  \textbf{7} \textit{Majuskel} T  \textbf{11} \textit{Initiale} U V W   $\cdot$ \textit{Majuskel} T  \textbf{16} \textit{Majuskel} T  \textbf{22} \textit{Majuskel} T  \textbf{23} \textit{Initiale} W  \textbf{29} \textit{Majuskel} T  \newline
\line(1,0){75} \newline
\textbf{1} Hardiz] hardyz U hardis V W \textbf{3} und] \textit{om.} V T  $\cdot$ kunnet] kunne U kúnnen W \textbf{4} mâze] mossen V \textbf{5} al zuo] also V W \textbf{8} dâ] do W  $\cdot$ sîne kamer] sein kamerere W \textbf{9} ein wênic] Ein klein V (T) In einem W \textbf{10} naht] nach U  $\cdot$ er dolte] die dolt er W dolt er T \textbf{11} Des] EIns W \textbf{12} wurden] wunden T  $\cdot$ des alle] alle gleich W (T) \textbf{14} dâ] do W  $\cdot$ mit] nit U \textbf{15} \textit{Die Verse 93.15-16 fehlen} U   $\cdot$ Waren iung oder alt V  $\cdot$ Nieman were fúrbas W  $\cdot$ si weren ivnc oder alde T \textbf{16} beide bloͤde oder balt V  $\cdot$ Denne der den ancker maß W  $\cdot$ Bloͤde oder balde T \textbf{17} si wolten] Sy wolten auch W sine solten T \textbf{18} daz wart mit triuwen gar verpfliht T  $\cdot$ mitten] mitte V \textbf{19} als] so W T  $\cdot$ vertriben] [ver*]: verriben V vberladen W \textbf{20} Vnd hetten von muͤde grossen schaden W  $\cdot$ diu ors] dorch sie U  $\cdot$ alsô] \textit{om.} T  $\cdot$ verriben] [ver*]: vertriben V \textbf{21} da von si horten riterscaft T  $\cdot$ werde] werden W \textbf{22} Twanc dannoch die muͦde craft U · Twang der vnmassen muͤde krafft W · Mit grozer samenvnge craft T \textbf{23} reit dô selbe] do selbe reit W ir reise tete T \textbf{24} Zuͦ velde nicht lenger sy enbeit W  $\cdot$ Ze velde hin Gahmvrete T  $\cdot$ den] [dem]: den V  $\cdot$ velde] velbe U \textbf{25} vuorte si] wuͦrte in T  $\cdot$ mit ir] wider W \textbf{26} si] er T \textbf{27} zuo der] auff die W  $\cdot$ Lewe planie] lev planie U leu planie V plane W leuwe planie T \textbf{28} dô] da T \textbf{29} dâ] do U V W  $\cdot$ messe] ein messe W \textbf{30} Zazamanc] zazamang V W \newline
\end{minipage}
\end{table}
\end{document}
