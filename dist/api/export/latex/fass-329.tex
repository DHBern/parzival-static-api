\documentclass[8pt,a4paper,notitlepage]{article}
\usepackage{fullpage}
\usepackage{ulem}
\usepackage{xltxtra}
\usepackage{datetime}
\renewcommand{\dateseparator}{.}
\dmyyyydate
\usepackage{fancyhdr}
\usepackage{ifthen}
\pagestyle{fancy}
\fancyhf{}
\renewcommand{\headrulewidth}{0pt}
\fancyfoot[L]{\ifthenelse{\value{page}=1}{\today, \currenttime{} Uhr}{}}
\begin{document}
\begin{table}[ht]
\begin{minipage}[t]{0.5\linewidth}
\small
\begin{center}*D
\end{center}
\begin{tabular}{rl}
\textbf{329} & \begin{large}S\end{large}wie vremdez mir \textbf{hie} wære,\\ 
 & ich kom ouch her durch mære\\ 
 & unt zerkennen âventiure.\\ 
 & nû lît diu hœhste stiure\\ 
5 & an iu, \textbf{des al getouftiu} diet\\ 
 & mit prîse \textbf{sich} von laster schiet.\\ 
 & Sol guot gebærde \textbf{iuch} helfen \textbf{iht}\\ 
 & unt daz man iu mit wârheit giht\\ 
 & liehter varwe unt manlîcher site?\\ 
10 & kraft \textbf{mit} jugende vert dâ mite."\\ 
 & Diu rîche, wîse heidenîn\\ 
 & het an künste den gewin,\\ 
 & daz si wol reite franzeis.\\ 
 & \textbf{dô} antwurte ir der Waleis,\\ 
15 & solch was sîn rede wider sie:\\ 
 & "got lône iu, vrouwe, daz ir hie\\ 
 & \textbf{mir} gebt sô güetlîchen trôst.\\ 
 & i\textbf{ne} bin doch trûrens \textbf{niht erlôst}\\ 
 & \textbf{unt} wil iuch des bescheiden.\\ 
20 & i\textbf{ne} mag \textbf{es} sô niht geleiden,\\ 
 & als ez mir leide kündet,\\ 
 & daz sich nû maneger sündet\\ 
 & \textbf{an mir}, der niht weiz \textbf{mîner} klage,\\ 
 & unt \textbf{ich} dâ bî sîn spotten trage.\\ 
25 & I\textbf{ne} wil deheiner vreude \textbf{jehen},\\ 
 & i\textbf{ne} \textbf{müeze} \textbf{alrêst} den Grâl gesehen,\\ 
 & diu wîle \textbf{sî} kurz oder lanc.\\ 
 & mich jaget des endes mîn gedanc;\\ 
 & dâ von gescheide ich nimmer\\ 
30 & mînes lebens immer.\\ 
\end{tabular}
\scriptsize
\line(1,0){75} \newline
D \newline
\line(1,0){75} \newline
\textbf{1} \textit{Initiale} D  \textbf{7} \textit{Majuskel} D  \textbf{11} \textit{Majuskel} D  \textbf{25} \textit{Majuskel} D  \newline
\line(1,0){75} \newline
\newline
\end{minipage}
\hspace{0.5cm}
\begin{minipage}[t]{0.5\linewidth}
\small
\begin{center}*m
\end{center}
\begin{tabular}{rl}
 & wie vrömede ez mir wære,\\ 
 & ich kam ouch her durch mære\\ 
 & und ze erkennen âventiure.\\ 
 & nû lît diu hœheste stiure\\ 
5 & an iu, \textbf{des alliu getouftiu} diet\\ 
 & mit prîse \textbf{sich} von laster schiet.\\ 
 & sol guot gebærde \textbf{iuch} helfen \textbf{iht}\\ 
 & und daz man iu mit wârheit giht\\ 
 & liehter varwe und manlîcher \textit{site}?\\ 
10 & kraft \textbf{mit} jugende vert dâ mite."\\ 
 & diu rîche, wîse heidenîn\\ 
 & hete an künste den gewin,\\ 
 & daz si wol rette franz\textit{e}is.\\ 
 & \textbf{dô} antwurte ir der Waleis,\\ 
15 & solich was sîn rede wider sie:\\ 
 & "got lône iu, vrouwe, daz ir hie\\ 
 & \textbf{mir} gebt sô güetlîchen trôst.\\ 
 & ich bin doch trûrens \textbf{niht erlôst},\\ 
 & \textbf{ic\textit{h}} \textit{w}il iuch des bescheiden.\\ 
20 & ich mac \textbf{es} sô niht geleiden,\\ 
 & als ez mir leide kündet,\\ 
 & daz sich nû maniger sündet\\ 
 & \textbf{an mir}, der niht weiz \textbf{mîner} klage,\\ 
 & und \textbf{ich} dâ bî sîn spotten trage.\\ 
25 & ich wil dekeiner vröude \textbf{jehen},\\ 
 & \textit{i}\textbf{\textit{n}e} \textbf{müeze} \textbf{aller êrst} den Grâl gesehen,\\ 
 & diu wîle, \textbf{wart si} kurz oder lanc.\\ 
 & mich jaget des endes mîn gedanc;\\ 
 & dâ von gescheide ich niemer\\ 
30 & mînes lebenes iemer.\\ 
\end{tabular}
\scriptsize
\line(1,0){75} \newline
m n o \newline
\line(1,0){75} \newline
\newline
\line(1,0){75} \newline
\textbf{1} wære] hie were n o \textbf{2} her] herre o \textbf{4} diu] de o \textbf{5} des alliu getouftiu diet] das alle getauͯfft det o \textbf{6} mit] Von o \textbf{9} und] \textit{om.} n o  $\cdot$ site] \textit{om.} m \textbf{10} jugende] tugenden n tugende o \textbf{12} hete] Hat n  $\cdot$ künste] kúnsten n (o) \textbf{13} franzeis] franczois m frantzois n franczosis o \textbf{15} solich] Sol ich o \textbf{19} ich wil] Jch vnd wil m Jch enwil o \textbf{20} sô] wol o \textbf{21} leide] beide o \textbf{23} mîner] mẏne o \textbf{25} dekeiner] do heiner n  $\cdot$ vröude] [frouit]: frouiden o \textbf{26} ine] Me m Jch n (o)  $\cdot$ gesehen] sehen n o \textbf{27} wart si] sy n sie o  $\cdot$ kurz] kurcze m \textbf{30} lebenes] lebens n o \newline
\end{minipage}
\end{table}
\newpage
\begin{table}[ht]
\begin{minipage}[t]{0.5\linewidth}
\small
\begin{center}*G
\end{center}
\begin{tabular}{rl}
 & swie vrömdez mir \textbf{hie} wære,\\ 
 & ich kom ouch her durch mære\\ 
 & unt zerkennene âventiure.\\ 
 & nû lît diu hœheste stiure\\ 
5 & an iu, \textbf{der getouften} diet:\\ 
 & mit prîse \textbf{ich} von laster schiet.\\ 
 & sol guot gebærde \textbf{iuch} helfen \textbf{niht}\\ 
 & unt daz man iu mit wârheit giht\\ 
 & liehter varwe unde manlîcher site?\\ 
10 & kraft \textbf{mit} jugende vert dâ mite."\\ 
 & diu \textit{r}î\textit{ch}e, \textit{w}î\textit{s}e heidenîn\\ 
 & het an kunst den gewin,\\ 
 & daz si wol redete franz\textit{e}is.\\ 
 & \textbf{dô} antwurte ir der Waleis,\\ 
15 & solch was sîn rede wider sie:\\ 
 & "got lône iu, vrouwe, daz ir hie\\ 
 & \textbf{mir} gebet sô güetlîchen trôst.\\ 
 & ich bin doch trûrenes \textbf{unerlôst}\\ 
 & \textbf{unde} wil iuch des bescheiden.\\ 
20 & ich\textbf{ne} mag \textbf{ez} sô niht geleiden,\\ 
 & als ez mir leide kündet,\\ 
 & daz sich nû maniger sündet\\ 
 & \textbf{an mir}, der niht weiz \textbf{mîner} klage,\\ 
 & unde \textbf{ich} dâ bî sîn spoten trage.\\ 
25 & ich wil neheiner vröuden \textbf{pflegen},\\ 
 & ich \textbf{muoz} \textbf{alrêrst} den Grâl gesehen,\\ 
 & diu wîle \textbf{sî} kurz oder lanc.\\ 
 & mich jaget des endes mîn gedanc;\\ 
 & dâ von gescheide ich nimmer\\ 
30 & mînes lebens immer.\\ 
\end{tabular}
\scriptsize
\line(1,0){75} \newline
G I O L M Q R Z Fr21 Fr27 \newline
\line(1,0){75} \newline
\textbf{13} \textit{Initiale} O  \textbf{14} \textit{Capitulumzeichen} R  \textbf{15} \textit{Initiale} I  \textbf{25} \textit{Initiale} Z  \textbf{29} \textit{Initiale} L Fr21  \newline
\line(1,0){75} \newline
\textbf{1} swie] Wie L (M) (Q) R  $\cdot$ mir hie] hie mir O mir Q \textbf{2} kom] \textit{om.} Z \textbf{3} zerkennene] zerchenne hie I erkennend R \textbf{5} An iv daz ist gar g:::t Fr21  $\cdot$ der] deist gar O (M) (R) der ist L der gar Q daz ist alle Z  $\cdot$ getouften] getaufte I (L) (M) (Q) (Z) getavstev O die tofftte R \textbf{6} ich] uͯch L sich Q R Z \textbf{7} gebærde] gebære O (L) (M) (Z)  $\cdot$ iuch] nun R  $\cdot$ niht] iht I (Q) \textbf{9} liehter] Lýchte L Lichter Q  $\cdot$ varwe] frawe Q  $\cdot$ manlîcher] manlich I \textbf{10} jugende] duͯgende L \textbf{11} rîche wîse] wise riche G riche wisen Z \textbf{12} kunst] kuͯsch L \textbf{13} daz] ÷az O  $\cdot$ redete] redet O \textit{om.} Z  $\cdot$ franzeis] franzoys G fronzois I [franzoyis]: franzeyis O Frantzeisz L franciosis M franzois Q Fr27 Z :::zois Fr21 \textbf{14} dô] Da M Z Des R  $\cdot$ Waleis] waleys O waleisz L \textbf{15} solch] Selic M \textbf{17} mir] Mit Q \textbf{18} bin] en byn M (Fr21)  $\cdot$ unerlôst] niht erlost I O L Z (Fr21) (Fr27) nicht er los M \textbf{19} iuch des] ev daz I úchs R \textbf{20} ichne] ich I (O) (R)  $\cdot$ sô niht] niht so O  $\cdot$ geleiden] erleiden Q (Fr27) \textbf{21} ez] er Q R \textbf{22} nû] nv leider O \textit{om.} Z \textbf{23} niht weiz] nit weist R weiz \sout{niht} Z  $\cdot$ mîner] mine I (L) \textbf{24} ich dâ bî] da bý L do bey ich Q  $\cdot$ sîn] \textit{om.} R  $\cdot$ spoten] spotte Q (R)  $\cdot$ trage] tragen Q \textbf{25} wil] enwil L (M) Z (Fr21)  $\cdot$ neheiner] deheine Fr21  $\cdot$ vröuden] \textit{om.} I O frewde Q (Fr27)  $\cdot$ pflegen] iehen L Q (R) Z phlen M \textbf{26} muoz] wil I enmuͯsze L (Z)  $\cdot$ gesehen] ersehen I sehen Z \textbf{27} oder] alde Fr27 \textbf{29} gescheide] gevde O geschiede R scheide Fr21 \textbf{30} mînes lebens] mines libes I Mein leben Q  $\cdot$ immer] iamer I \newline
\end{minipage}
\hspace{0.5cm}
\begin{minipage}[t]{0.5\linewidth}
\small
\begin{center}*T
\end{center}
\begin{tabular}{rl}
 & Swie vremde\textit{z} mir wære,\\ 
 & ich kom ouch her durch mære\\ 
 & unde zerkennen âventiure.\\ 
 & nû lît di\textit{u} hœheste stiure\\ 
5 & an iu, \textbf{deist gar getouftiu} diet.\\ 
 & mit prîse \textbf{ich} \textbf{iuch} von laster schiet.\\ 
 & sol guot gebærde helfen \textbf{iht}\\ 
 & unde daz man iu mit wârheit giht\\ 
 & liehter varwe unde manlîcher site?\\ 
10 & kraft \textbf{bî} j\textit{ug}e\textit{n}t vert dâ mite."\\ 
 & Diu rîche, wîse heidenîn\\ 
 & het an kunst den gewin,\\ 
 & daz si wol redete franz\textit{e}is.\\ 
 & \textbf{Sus} antwürt ir der Waleis,\\ 
15 & sölh was sîn rede wider sie:\\ 
 & "got lône iu, vrouwe, daz ir \textbf{mir} hie\\ 
 & gebt sô güetlîchen trôst.\\ 
 & ich bin doch trûrens \textbf{unerlôst}\\ 
 & \textbf{unde} wil iuch des bescheiden.\\ 
20 & i\textbf{ne} mag \textbf{ez} sô niht geleiden,\\ 
 & als ez mir leide kündet,\\ 
 & daz sich nû maneger sündet,\\ 
 & der niht weiz \textbf{mîne} klage\\ 
 & unde dâ bî sîn spoten trage.\\ 
25 & i\textbf{ne} wil deheiner vröude \textbf{jehen},\\ 
 & i\textbf{ne} \textbf{müeze} den Grâl \textbf{ê} gesehen,\\ 
 & di\textit{u} wîle \textbf{sî} kurz oder lanc.\\ 
 & mich jaget des endes mîn gedanc;\\ 
 & dâ von gescheid ich niemer\\ 
30 & mînes lebens iemer.\\ 
\end{tabular}
\scriptsize
\line(1,0){75} \newline
T U V W \newline
\line(1,0){75} \newline
\textbf{1} \textit{Majuskel} T  \textbf{11} \textit{Majuskel} T  \textbf{14} \textit{Majuskel} T  \textbf{21} \textit{Initiale} V  \newline
\line(1,0){75} \newline
\textbf{1} Swie] Wie U V W  $\cdot$ vremdez] vremdes T  $\cdot$ mir] mir [*]: hie V mir hie W \textbf{3} zerkennen] zuͦ kennen U [kennende]: zerkennende V \textbf{4} diu] die T \textbf{5} An vch U  $\cdot$ deist] der V  $\cdot$ getouftiu] die getauffte W \textbf{6} \textit{Vers 329.6 fehlt} U   $\cdot$ ich iuch] ich iv T sich V ich W \textbf{7} gebærde] geberde [*]: v́ch V \textbf{9} liehter] Liechte W \textbf{10} bî] mit W  $\cdot$ jugent] ivnget T \textbf{11} heidenîn] heidemin U \textbf{12} het] hat W \textbf{13} redete] sprichet W  $\cdot$ franzeis] franzoys T U franzeẏs V frantzoeis W \textbf{14} antwürt] antwurte W  $\cdot$ ir] \textit{om.} W  $\cdot$ Waleis] [wall*s]: walleẏs V \textbf{15} sölh] Selic U \textbf{16} mir] \textit{om.} U V W \textbf{17} gebt] [*]: Mir gebent V Mir gebent W  $\cdot$ güetlîchen] treúwelichen W \textbf{18} doch] noch V  $\cdot$ unerlôst] nit erlost W \textbf{20} ine] Ich W  $\cdot$ sô] \textit{om.} W \textbf{21} ez mir leide] mirs lait W \textbf{22} sich] ich U \textbf{23} der] [*]: An mir der V  $\cdot$ niht] nit rechte U (W)  $\cdot$ mîne] [*]: miner V \textbf{24} bî] [*]: bi ich V  $\cdot$ spoten trage] spot dragen U \textbf{25} ine] [*]: Jch V Ich W \textbf{26} [*]: Jch muͤse alrest den gral gesehen V  $\cdot$ ine] Jch U (W)  $\cdot$ müeze] muͦz U  $\cdot$ den Grâl ê] alrerst den gral U (W) \textbf{27} diu] die T  $\cdot$ wîle] zeit W \textbf{29} niemer] nymmer mer W \newline
\end{minipage}
\end{table}
\end{document}
