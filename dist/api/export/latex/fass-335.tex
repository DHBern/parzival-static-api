\documentclass[8pt,a4paper,notitlepage]{article}
\usepackage{fullpage}
\usepackage{ulem}
\usepackage{xltxtra}
\usepackage{datetime}
\renewcommand{\dateseparator}{.}
\dmyyyydate
\usepackage{fancyhdr}
\usepackage{ifthen}
\pagestyle{fancy}
\fancyhf{}
\renewcommand{\headrulewidth}{0pt}
\fancyfoot[L]{\ifthenelse{\value{page}=1}{\today, \currenttime{} Uhr}{}}
\begin{document}
\begin{table}[ht]
\begin{minipage}[t]{0.5\linewidth}
\small
\begin{center}*D
\end{center}
\begin{tabular}{rl}
\textbf{335} & \textbf{\begin{large}D\end{large}ô} bereite \textbf{ouch} sich hêr Gawan\\ 
 & als ein kampfbære man\\ 
 & hin vür den künec von Ascalun.\\ 
 & des trûrte manec Bertun\\ 
5 & \textbf{unt} manec wîb und magt.\\ 
 & herzenlîche wart geklagt\\ 
 & von in \textbf{sîn} strîtes reise.\\ 
 & der werdecheit ein weise\\ 
 & wart nû diu tavelrunder.\\ 
10 & Gawan maz besunder,\\ 
 & wâ mit er m\textit{ö}hte \textbf{wol} gesigen.\\ 
 & alt, \textbf{herte} schilde wol gedigen\\ 
 & - ern ruochte, wie si wâren gevar -,\\ 
 & \textbf{die} brâhten koufliute dar\\ 
15 & \textbf{ûf} ir \textbf{soumen}, doch niht veile.\\ 
 & der wurden im drî ze teile.\\ 
 & \textbf{Dô} erwarp der \textbf{mære} strîtes helt\\ 
 & siben ors \textbf{ze kampfe} erwelt.\\ 
 & \textbf{ze} sînen vriwenden er dô \textbf{nam}\\ 
20 & zwelf \textbf{schärpfiu} sper von Angram,\\ 
 & starke rœrîne schefte drîn\\ 
 & von Orastegentesin\\ 
 & ûz einem heidenischem muor.\\ 
 & Gawan nam urloub und vuor\\ 
25 & mit unverzagter manheit.\\ 
 & Artus was im vil bereit,\\ 
 & er gab im rîcher \textbf{koste} solt:\\ 
 & lieht gesteine \textbf{unt} rôtez golt\\ 
 & unt \textbf{silbers} manegen sterlinc.\\ 
30 & gein sorgen wielzen sîniu dinc.\\ 
\end{tabular}
\scriptsize
\line(1,0){75} \newline
D \newline
\line(1,0){75} \newline
\textbf{1} \textit{Initiale} D  \textbf{17} \textit{Majuskel} D  \newline
\line(1,0){75} \newline
\textbf{3} Ascalun] Ascalv̂n D \textbf{4} Bertun] bertv̂n D \textbf{11} möhte] mohte D \textbf{22} Orastegentesin] Orastegentesîn D \newline
\end{minipage}
\hspace{0.5cm}
\begin{minipage}[t]{0.5\linewidth}
\small
\begin{center}*m
\end{center}
\begin{tabular}{rl}
 & \textbf{dô} ber\textit{ei}t sich hêr Gawan\\ 
 & als ein kampfbære man\\ 
 & hin vür den künic von Ascalun.\\ 
 & des trûrete manic Britû\textit{n}\\ 
5 & \textbf{und} manic wîp und maget.\\ 
 & herzenlîch wart geklaget\\ 
 & von in \textbf{sîn} strîtes reise.\\ 
 & der werdecheit ein weise\\ 
 & wart nû diu tavelrunder.\\ 
10 & Gawan maz besunder,\\ 
 & wâ mite er m\textit{ö}ht gesigen.\\ 
 & alte, \textbf{herte} schilte wol gedigen\\ 
 & - er enruochte, wie si wâren gevar -,\\ 
 & \textbf{die} brâhte\textit{n} koufliute dar\\ 
15 & \textbf{in} ir \textbf{schiffe}, doch niht veile.\\ 
 & der wurden ime drîe ze teile.\\ 
 & \textbf{dô} erwarp der \textbf{wâre} strîtes helt\\ 
 & siben ros \textbf{ze kampf} er\textit{w}elt.\\ 
 & \textbf{ze} sînen vriunden er dô \textbf{nam}\\ 
20 & zwelf \textbf{scharfiu} sper von Agra\textit{m},\\ 
 & starke rœrîne schefte drîn\\ 
 & von Aras\textit{t}egentesin\\ 
 & ûz einem heidenischen muor.\\ 
 & Gawan nam urloup und vuor\\ 
25 & mit unverzagete\textit{r} manheit.\\ 
 & Artus was im vil bereit,\\ 
 & er gap im rîcher \textbf{koste} solt:\\ 
 & lieht gesteine \textbf{und} rôtez golt\\ 
 & und \textbf{silbers} manigen sterlinc.\\ 
30 & gegen sorgen wielzen sîniu dinc.\\ 
\end{tabular}
\scriptsize
\line(1,0){75} \newline
m n o \newline
\line(1,0){75} \newline
\newline
\line(1,0){75} \newline
\textbf{1} bereit] beriet m  $\cdot$ Gawan] gewan o \textbf{3} Ascalun] ascelun n ascaluͯn o \textbf{4} des] Das o  $\cdot$ Britun] brittuͯm m britẏm o \textbf{7} strîtes] strite o \textbf{11} möht] moht m moch mochte o \textbf{14} brâhten] brahte m (o) \textbf{15} schiffe] schiffen n o  $\cdot$ veile] [geẏl]: feẏl o \textbf{18} erwelt] er velt m \textbf{20} scharfiu] starck n o  $\cdot$ sper] spor o  $\cdot$ Agram] agran m angnam n angram o \textbf{21} rœrîne] roͯren n \textbf{22} Von arasce gente sin m  $\cdot$ Von oraste gente sin n o \textbf{23} einem] einer n (o) \textbf{25} unverzageter] vnuerzagettem m \textbf{29} silbers manigen] manigen silbers n manigen silbern o \textbf{30} wielzen sîniu] wieltze sin n \newline
\end{minipage}
\end{table}
\newpage
\begin{table}[ht]
\begin{minipage}[t]{0.5\linewidth}
\small
\begin{center}*G
\end{center}
\begin{tabular}{rl}
 & \textbf{nû} bereite \textbf{ouch} sich hêr Gawan\\ 
 & als ein kampfbære man\\ 
 & hin vür den künic von Aschalun.\\ 
 & des trûrte manic Britun,\\ 
5 & manic wîp unde maget.\\ 
 & herzenlîchen wart geklaget\\ 
 & von in \textbf{sînes} strîtes reise.\\ 
 & der werdicheit ein weise\\ 
 & wart nû diu tavelrunder.\\ 
10 & Gawan maz besunder,\\ 
 & wâ mit er m\textit{ö}hte \textbf{wol} gesigen.\\ 
 & alte, \textbf{herte} schilte wol gedigen\\ 
 & - er enruohte, wie si wâren gevar -,\\ 
 & \textbf{si} brâhten koufliute dar\\ 
15 & \textbf{ûf} ir \textbf{soumen}, doch niht veile.\\ 
 & der wurden im drî ze teile.\\ 
 & \textbf{ouch} erwarp der \textbf{wâre} strîtes helt\\ 
 & siben ors \textbf{gein strîte} \textbf{wol} erwelt.\\ 
 & \textbf{ze} sînen vriunden er dô \textbf{nam}\\ 
20 & zwelf \textbf{scharpfiu} sper von Angram,\\ 
 & starc rœrîne schefte drîn\\ 
 & von Orastegentesin\\ 
 & ûz einem heidenschen muor.\\ 
 & Gawan nam urloup unde vuor\\ 
25 & mit unverzagter manheit.\\ 
 & Artus was im vil bereit,\\ 
 & er gap im rîcher \textbf{\textit{kost}e} solt:\\ 
 & lieht gesteine, rôtez golt\\ 
 & unde \textbf{silbers} manigen sterlinc.\\ 
30 & gein sorgen wielzen sîniu dinc.\\ 
\end{tabular}
\scriptsize
\line(1,0){75} \newline
G I O L M Q R Z Fr21 Fr27 Fr39 \newline
\line(1,0){75} \newline
\textbf{1} \textit{Initiale} I L R Z Fr21 Fr39  \textbf{13} \textit{Initiale} O  \textbf{17} \textit{Initiale} I  \newline
\line(1,0){75} \newline
\textbf{1} ouch sich] sich auch I (M) (R)  $\cdot$ hêr Gawan] irgawan M \textbf{2} kampfbære] kamphare L kampflere Q kampffechter R \textbf{3} den] der L Z Fr39  $\cdot$ von] \textit{om.} R  $\cdot$ Aschalun] ascalun I (M) (R) (Z) (Fr39) Aschalvͦn O Fr21 Ascalon L ascaluͯn Q \textbf{4} trûrte] truret I (O) (Fr21)  $\cdot$ Britun] pritun I Brittvn L Fr39 :ritvn Fr21 b::: Fr27 \textbf{5} manic] Vnde manich O (L) (M) (Q) (R) (Z) (Fr21) (Fr27) (Fr39) \textbf{8} weise] weisei Z \textbf{9} diu] der M (Fr27) \textbf{10} besunder] bisundirn M \textbf{11} wâ] Swa Fr27  $\cdot$ möhte wol] mohte wol G (O) (L) (M) Z (Fr21) (Fr27) Fr39 wol mocht Q \textbf{12} alte] Hie O Alse M \textit{om.} Fr21  $\cdot$ gedigen] gedingen Q \textbf{13} er enruohte] ÷r rvͦcht O Er ruhte Z Ern rvht Fr21  $\cdot$ wâren] weren R \textbf{14} brâhten] brachte Q braht in Fr27 \textbf{15} ûf] Vs Fr39 \textbf{16} wurden] werden Q  $\cdot$ im] en M \textbf{17} ouch] Do Q R  $\cdot$ erwarp] erward R  $\cdot$ wâre] selbe I mere Q R  $\cdot$ strîtes] striten M \textbf{18} gein strîte wol] zvͤ dem champhe I gein champf O (L) (M) (Q) (Fr39) ein knappen R zv kampfe Z \textbf{19} vriunden] frewden Q  $\cdot$ dô] da M Z \textbf{20} scharpfiu] \textit{om.} I starcke L (M) (Z) starchir Fr39  $\cdot$ Angram] angaram I agram Q (R) \textbf{21} starc] Arke Q  $\cdot$ schefte] schafftte R \textbf{22} Orastegentesin] orest gentesin O (L) M Fr39 orastegente sin Q aragaste gentesin R Oreaste gentesin Z \textbf{23} einem heidenschen] einer haidnischer I einem heidenischem O (L) Z (Fr39) eyme heidischeme M einer heidenschen R \textbf{24} nam urloup] vrlovp nam O \textbf{26} Artus] Artuͯs L \textbf{27} im] in R  $\cdot$ koste] gabe G \textbf{28} lieht] Lýcht L (M) (Q) Edel R  $\cdot$ rôtez] vnde rotez O \textbf{29} unde silbers] Von Silber R \textbf{30} wielzen] wisten I wieszen L wilten Q wuͦchsen R \newline
\end{minipage}
\hspace{0.5cm}
\begin{minipage}[t]{0.5\linewidth}
\small
\begin{center}*T
\end{center}
\begin{tabular}{rl}
 & \textbf{\begin{large}N\end{large}û} bereit \textbf{ouch} sich hêr Gawan\\ 
 & Als ein kampfbære man\\ 
 & hin vür den künec von Ascalun.\\ 
 & des trûrete manec Britun\\ 
5 & \textbf{unde} manec wîp unde maget.\\ 
 & herzeclîche \textbf{ez} wart geklaget\\ 
 & von in \textbf{sîn} strîtes reise.\\ 
 & der werdecheite ein weise\\ 
 & wart nû di\textit{u} tavelrunder.\\ 
10 & Gawan maz besunder,\\ 
 & wâ mit er m\textit{ö}hte \textbf{wol} gesigen.\\ 
 & alte schilte wol gedigen\\ 
 & - ern ruohte, wie si wâren gevar -,\\ 
 & \textbf{die} brâhten koufliute dar\\ 
15 & \textbf{ûf} ir \textbf{soumen}, doch niht veile.\\ 
 & der wurden im drîe ze teile.\\ 
 & \textbf{ouch} erwarp der strîtes helt\\ 
 & siben ors \textbf{gegen kampfe} \textbf{ûz} erwelt.\\ 
 & \textbf{von} sînen vriunden er dô \textbf{gewan}\\ 
20 & Zwelf \textbf{starkiu} sper von Angran,\\ 
 & starke rœrîne schefte drîn\\ 
 & von Orestegentesin\\ 
 & ûz einem heidenschem muor.\\ 
 & Gawan nam urloup unde vuor\\ 
25 & mit unverzaget\textit{er} manheit.\\ 
 & Artus was im vil bereit,\\ 
 & er gab im rîcher \textbf{gabe} solt:\\ 
 & lieht gesteine, rôtez golt\\ 
 & unde \textbf{silber}, manege\textit{n} sterlinc.\\ 
30 & gegen sorgen wielzen sîniu dinc.\\ 
\end{tabular}
\scriptsize
\line(1,0){75} \newline
T U V W \newline
\line(1,0){75} \newline
\textbf{1} \textit{Initiale} T U W  \textbf{2} \textit{Majuskel} T  \textbf{20} \textit{Majuskel} T  \textbf{21} \textit{Initiale} V  \newline
\line(1,0){75} \newline
\textbf{1} bereit ouch sich] bereite sich U V bereite sich auch W  $\cdot$ hêr Gawan] gaban W \textbf{3} Ascalun] [A*]: Ascalvn T Aschalun U astalun W \textbf{4} Britun] Brituͦn U brittvn V \textbf{5} \textit{Vers 335.5 fehlt} U   $\cdot$ wîp] fraw W \textbf{6} ez wart] ward W \textbf{7} in] im W  $\cdot$ sîn] sins V (W) \textbf{8} weise] fraise W \textbf{9} diu] die T \textbf{11} mit er möhte wol] mit er mohte wol T (U) [mitt*]: mitte er moͤhte wol V mit er volle moͤcht W \textbf{12} schilte wol] scilte [wol*]: wol T herte schilte wol U V W \textbf{15} ûf ir soumen] [J*]: Jn ir schiffen V \textbf{17} der] der ware U V W \textbf{20} starkiu] [*]: scharpfe V scharpffe W  $\cdot$ Angran] agran U [angr*]: angran V \textbf{21} rœrîne] roͤren W \textbf{22} Orestegentesin] Oristegende sin U Oregestegentesin V orestegente sin W \textbf{23} heidenschem] heidenschen V \textbf{25} unverzageter] vnverzaget T \textbf{27} im rîcher gabe] richer gabe im U im reicher varwe W \textbf{28} lieht] Edeles W  $\cdot$ gesteine] [gestein*]: gesteine V \textbf{29} silber] silbers U V W  $\cdot$ manegen] manegem T \newline
\end{minipage}
\end{table}
\end{document}
