\documentclass[8pt,a4paper,notitlepage]{article}
\usepackage{fullpage}
\usepackage{ulem}
\usepackage{xltxtra}
\usepackage{datetime}
\renewcommand{\dateseparator}{.}
\dmyyyydate
\usepackage{fancyhdr}
\usepackage{ifthen}
\pagestyle{fancy}
\fancyhf{}
\renewcommand{\headrulewidth}{0pt}
\fancyfoot[L]{\ifthenelse{\value{page}=1}{\today, \currenttime{} Uhr}{}}
\begin{document}
\begin{table}[ht]
\begin{minipage}[t]{0.5\linewidth}
\small
\begin{center}*D
\end{center}
\begin{tabular}{rl}
\textbf{761} & wolt ich \textbf{d}âventiure \textbf{vürbaz} \textbf{lân}.\\ 
 & Dô enbôt mîn hêr Gawan\\ 
 & ze hove Artuse mære,\\ 
 & wer dâ komen wære.\\ 
5 & der rîche heiden wære dâ,\\ 
 & den diu heidenîn Eckuba\\ 
 & sô prîste bî dem Plimizœl.\\ 
 & Jofreit fiz Ydœl\\ 
 & Artuse \textbf{daz} mære sagete,\\ 
10 & \textbf{des} er vreude vil bejagete.\\ 
 & Jofreit bat in ezzen vruo\\ 
 & unt clârlîche grîfen zuo\\ 
 & mit \textbf{rîtern} und mit \textbf{vrouwen} schar\\ 
 & \textbf{unt} höfschlîche \textbf{komen} dar,\\ 
15 & daz siz sô \textbf{ane geviengen}\\ 
 & \textbf{unt} werdeclîche enpfiengen\\ 
 & des stolzen Gahmuretes kint.\\ 
 & "Swaz hie werder liute sint,\\ 
 & die bringe ich", sprach der Bertenois.\\ 
20 & Jofreit sprach: "er ist sô kurtois,\\ 
 & ir muget in alle gerne sehen,\\ 
 & wan ir \textbf{sult} wunder an im spehen.\\ 
 & er \textit{v}ert ûz grôzer rîcheit.\\ 
 & \textbf{sîniu wâpenlîchiu} kleit\\ 
25 & niemen vergelten möhte.\\ 
 & deheiner hant daz t\textit{ö}hte.\\ 
 & Lœver, Bertane, Engellant,\\ 
 & von Paris unz an Wizsant,\\ 
 & der d\textit{ar} gein leite al die terre,\\ 
30 & ez wære\textbf{m} gelte verre."\\ 
\end{tabular}
\scriptsize
\line(1,0){75} \newline
D \newline
\line(1,0){75} \newline
\textbf{2} \textit{Majuskel} D  \textbf{18} \textit{Majuskel} D  \newline
\line(1,0){75} \newline
\textbf{3} Artuse] Artvͦse D \textbf{7} Plimizœl] Primizoͤl D \textbf{8} Ydœl] ydoͤl D \textbf{9} Artuse] Artvͦse D \textbf{17} Gahmuretes] Gahmvrets D \textbf{19} Bertenois] Bertenoys D \textbf{23} vert] wert D \textbf{26} töhte] tohte D \textbf{27} Lœver] Loͤver D \textbf{28} Wizsant] Wîzsant D \textbf{29} dar gein] drigein D \newline
\end{minipage}
\hspace{0.5cm}
\begin{minipage}[t]{0.5\linewidth}
\small
\begin{center}*m
\end{center}
\begin{tabular}{rl}
 & wolt ich \textbf{die} âventiur \textbf{hie} \textbf{lân}.\\ 
 & dô enbôt mîn hêr Gawan\\ 
 & zuo hove Artuse mære,\\ 
 & wer d\textit{â} komen wære.\\ 
5 & der rîche heiden wær dâ,\\ 
 & den diu \textit{heidenîn} Ecuba\\ 
 & sô prîste bî dem Plimizol.\\ 
 & \textit{Jo}fr\textit{e}it fiz Idol\\ 
 & Artuse \textbf{daz} mære sagete,\\ 
10 & \textbf{des} er vröude vil bejagete.\\ 
 & Jofr\textit{e}it bat in ezzen vruo\\ 
 & und clærlîch grîfen zuo,\\ 
 & \hspace*{-.7em}\big| \textbf{daz er} hübschlîch \textbf{kæme} dar,\\ 
 & \hspace*{-.7em}\big| mit \textbf{ritter} und mit \textbf{vröuden} schar,\\ 
15 & \textbf{und} daz s\textit{i ez} sô \textbf{an geviengen},\\ 
 & \textbf{daz si} wirdeclîch enpfiengen\\ 
 & des stolzen Gahmuretes kint.\\ 
 & "waz hie werder liute sint,\\ 
 & die bring ich", sprach der Britun\textit{o}is.\\ 
20 & Jofr\textit{e}it sprach: "er ist sô kurtois,\\ 
 & ir moget in alle \textit{gerne} sehen,\\ 
 & wan ir \textbf{müget} wunder an im spehen.\\ 
 & er vert ûz grôzer rîcheit.\\ 
 & \textbf{sîn wâp\textit{en}lîchez} kleit\\ 
25 & niemen vergelten m\textit{ö}hte.\\ 
 & dekeiner hant daz t\textit{ö}hte.\\ 
 & Lo\textit{v}er, Britanie, Engelant,\\ 
 & von Paris unz an Wisant,\\ 
 & der dar gegen leit alle die terre,\\ 
30 & \textit{ez wære \textbf{deme} gelte} verre."\\ 
\end{tabular}
\scriptsize
\line(1,0){75} \newline
m n o V V' W \newline
\line(1,0){75} \newline
\newline
\line(1,0){75} \newline
\textbf{1} Kostliche gezierde wart do bereit V'  $\cdot$ die] dise V \textbf{2} Do enpot [*]: her gawin als man seit V'  $\cdot$ enbôt] enhort o  $\cdot$ hêr] herre her n \textbf{3} Artuse] artúse o  $\cdot$ mære] die mere V' \textbf{4} dâ] do m n o V V' \textbf{5} dâ] do n \textbf{6} heidenîn] \textit{om.} m heiden n o  $\cdot$ Ecuba] ecubo n eckuba V V' \textbf{7} Plimizol] plimzol n o \textbf{8} Jofreit] Got frid m Gotfrit n Gotfrid o Joffrit V Joffrid V'  $\cdot$ fiz Idol] fis idol m vis idol n fisz idol o fis ydol V V' \textbf{10} bejagete] beiage V' \textbf{11} \textit{Die Verse 761.11-12 fehlen} V'   $\cdot$ Jofreit] Jofrit m n o Joffrit V \textbf{14} daz er] Vnd V' \textbf{13} ritter] rittern V (V')  $\cdot$ mit vröuden] froͧwen n (o) (V) (V')  $\cdot$ schar] [scharn]: schar V' \textbf{15} und] \textit{om.} V'  $\cdot$ si ez] sich m  $\cdot$ sô] \textit{om.} W  $\cdot$ geviengen] fingen o \textbf{16} daz si] vnd V' \textbf{17} des] Das o  $\cdot$ stolzen] guͦten W  $\cdot$ Gahmuretes] gamuretes n W gahnmuretes o Gamerettes V gamereten V' \textbf{18} waz] Swaz V  $\cdot$ liute] frawen W \textbf{19} der] do V'  $\cdot$ Britunois] Prittunis m britoneisz o brittunos V brittinvs V' helt W \textbf{20} Jofreit] Jofrit m n o Joffrit V Joffrid V' Iofrit W  $\cdot$ sô kurtois] [zuͯ*]: zuͯ curtois n so teúr gezelt W \textbf{21} in alle] ir o  $\cdot$ gerne] \textit{om.} m \textbf{22} müget] múgen n soͤllent V (V') \textbf{24} sîn] Sine V V' Seins W  $\cdot$ wâpenlîchez] wapliches m woppencliches n (o) wopenliche V (V') \textbf{25} möhte] mohtte m (o) (V') \textbf{26} dekeiner] Do keiner n  $\cdot$ töhte] dohtte m (n) (o) (V') \textbf{27} \textit{Die Verse 761.27-30 fehlen} V'   $\cdot$ Lover] Loner m n o Loͤuer W  $\cdot$ Britanie] brittanie m V pritanẏe n  $\cdot$ Engelant] vnd engellant n engellant o W engenlant V \textbf{28} Paris] parisz o pariß W  $\cdot$ Wisant] wissant n o V wysant W \textbf{29} \textit{Die Verse 761.29-30 fehlen} W   $\cdot$ leit] leite V \textbf{30} \textit{Vers 761.30 fehlt} o   $\cdot$ Jn ferre lant fere m  $\cdot$ Wie dem do werre n \newline
\end{minipage}
\end{table}
\newpage
\begin{table}[ht]
\begin{minipage}[t]{0.5\linewidth}
\small
\begin{center}*G
\end{center}
\begin{tabular}{rl}
 & \begin{large}W\end{large}olde ich \textbf{dise} âventiure \textbf{vürbaz} \textbf{hân}.\\ 
 & dô enbôt mîn hêr Gawan\\ 
 & ze hove Artuse mære,\\ 
 & wer dâ komen wære.\\ 
5 & der rîche heiden wære dâ,\\ 
 & den diu heideninne Ekuba\\ 
 & sô brîste bî dem Blimzol.\\ 
 & Jofreit fis Idol\\ 
 & Artus \textbf{dô} mære sagte,\\ 
10 & \textbf{des} er vröude vil bejagte.\\ 
 & Jofreit bat in ezzen vruo\\ 
 & unde clârlîchen grîfen zuo\\ 
 & mit \textbf{rîtern} unde mit \textbf{vrouwen} schar\\ 
 & \textbf{unde} höfschlîche \textbf{komen} dar,\\ 
15 & daz siz sô \textbf{an geviengen}\\ 
 & \textbf{unde} werdeclîche enpfiengen\\ 
 & des stolzen Gahmureten kint.\\ 
 & "swaz hie werder liute sint,\\ 
 & die bringe ich", sprach der Britaneis.\\ 
20 & Jofreit sprach: "er ist sô kurteis,\\ 
 & ir müget in alle gerne sehen,\\ 
 & wan ir \textbf{müget} wunder an im spehen.\\ 
 & er vert ûz grôzer rîcheit.\\ 
 & \textbf{sîniu wâpenlîchiu} kleit\\ 
25 & niemen vergelten m\textit{ö}hte.\\ 
 & deheiner hant daz t\textit{ö}hte.\\ 
 & Lover, Britanie, Engellant,\\ 
 & von Paris unze an Wizsant,\\ 
 & der dar gein leit alle die terre,\\ 
30 & ez wære \textbf{jenem} gelte verre."\\ 
\end{tabular}
\scriptsize
\line(1,0){75} \newline
G I L M Z Fr45 Fr48 Fr70 \newline
\line(1,0){75} \newline
\textbf{1} \textit{Initiale} G I L Z Fr48  \textbf{23} \textit{Initiale} I  \newline
\line(1,0){75} \newline
\textbf{1} Wolde] SOlt Fr48 (Fr70)  $\cdot$ hân] lan Z Fr48 \textbf{2} dô] Da M Z  $\cdot$ hêr Gawan] ergawan M \textbf{3} Artuse] artus I Z (Fr48) \textbf{4} dâ] do Fr48 \textbf{6} den diu] de de Fr70  $\cdot$ Ekuba] Ekv̂ba G eccuba I Ecuͯba L ekuͯba M Eckvba Z (Fr48) \textbf{7} dem] den Fr70  $\cdot$ Blimzol] blimizol I plimszol L plimizcol M plimizol Z (Fr70) :::limizol Fr45 plimitzol Fr48 \textbf{8} Jofreit] Jofrit I Jofreýt L Yofreit Fr70  $\cdot$ fis Idol] Fiscidol I fýzedol L fyus ydol Fr70 \textbf{9} Artus] artuse I (L) (Fr70) Art::: Fr45  $\cdot$ dô] \textit{om.} I da M Z die Fr45 (Fr70) \textbf{10} des] Das M  $\cdot$ er] hi Fr70  $\cdot$ vröude] frovden L \textbf{11} Jofreit] Jofreid Fr45 Jofroit Fr70 \textbf{12} \textit{Versfolge 761.14-13} Fr70  \textbf{14} höfschlîche] hofelichen M hemeliche Fr70  $\cdot$ komen] queme Fr70 \textbf{15} geviengen] viengen I (L) (M) (Fr45) (Fr70) \textbf{16} enpfiengen] [Geviengen]: enphiengen I \textbf{17} Gahmureten] Gahmuretes I (L) gamuretis M gamureten Z Gamoretis Fr45 gamurettes Fr70 \textbf{18} swaz] Waz L (M) (Fr45)  $\cdot$ werder] stolzer I (Z) \textbf{19} die] der Fr70  $\cdot$ der] \textit{om.} M  $\cdot$ Britaneis] pritoys I Brittanoys L britvneis Z britteneẏs Fr45 Brittoneis Fr70 \textbf{20} Jofreit] Iofreit G Jofreid Fr45  $\cdot$ sô] \textit{om.} L \textbf{21} müget] svlt Z \textbf{22} wan] \textit{om.} I  $\cdot$ müget] svlt L (M) (Fr70) \textbf{24} wâpenlîchiu] wapan L wafenlichen Fr45 \textbf{25} vergelten] vorgeld M  $\cdot$ möhte] mohte G I (L) (M) (Fr45) (Fr70) \textbf{26} töhte] tohte G (I) (L) (M) (Fr45) (Fr70) \textbf{27} Lover] [*over]: lover G Jouer I Leover L Louer Fr45  $\cdot$ Britanie] britânie G pritanie I brittanie Fr45 Brittanya Fr70  $\cdot$ Engellant] vnd engenlant I engillant M engelant Fr45 (Fr70) \textbf{28} unze an] vnd von Fr45 byz an Fr70  $\cdot$ Wizsant] wiz lant I wilsant M wizzen sant Z wizsant Fr45 Brabant Fr70 \textbf{29} der dar gein leit] der da Geuailte inne I Vnd der gein leite L Der den geyn leite M der da gein leit Z (Fr70) Der den gein leit Fr45  $\cdot$ alle die] an de Fr45 \textbf{30} jenem] deme M (Fr45) (Fr70) \newline
\end{minipage}
\hspace{0.5cm}
\begin{minipage}[t]{0.5\linewidth}
\small
\begin{center}*T
\end{center}
\begin{tabular}{rl}
 & wolt ich \textbf{dise} âventiure \textbf{vürbaz} \textbf{hân}.\\ 
 & dô enbôt mîn hêr Gawan\\ 
 & zuo hove Artuse mære,\\ 
 & wer dâ komen wære.\\ 
5 & \textit{\begin{large}D\end{large}}er rîche heiden wære d\textit{â},\\ 
 & den diu heide\textit{n}în Eckuba\\ 
 & sô prîste bî dem Plymizol.\\ 
 & Jofreit fis Idol\\ 
 & Artuse \textbf{die} mære sagete,\\ 
10 & \textbf{daz} er vreuden vil bejagete.\\ 
 & Jofreit bat in ezzen vruo\\ 
 & und clârlîchen grîfen zuo\\ 
 & mit \textbf{rîtern} und mit \textbf{vrouwen} schar\\ 
 & \textbf{und} höveschlîchen \textbf{komen} dar,\\ 
15 & daz si ez sô \textbf{ane viengen}\\ 
 & \textbf{und} wirdeclîche enpfiengen\\ 
 & des stolzen Gahmuretes kint.\\ 
 & "waz hie werder liute sint,\\ 
 & die bringe ich", sprach der Britonois.\\ 
20 & Jofreit sprach: "er ist sô kurtois,\\ 
 & ir muget in alle gerne sehen,\\ 
 & wan ir \textbf{sult} wunder an im spehen.\\ 
 & er vert ûz grôzer rîcheit.\\ 
 & \textbf{sîn wâpenlîch} kleit\\ 
25 & nieman vergelten m\textit{ö}hte.\\ 
 & dekeiner hant daz t\textit{ö}hte.\\ 
 & Lover, Britanie \textbf{und} Engellant,\\ 
 & von Paris unz an Wizsant,\\ 
 & der dar geine leget al die terre,\\ 
30 & ez wære \textbf{dem} gelte verre."\\ 
\end{tabular}
\scriptsize
\line(1,0){75} \newline
U W Q R \newline
\line(1,0){75} \newline
\textbf{5} \textit{Initiale} U W  \newline
\line(1,0){75} \newline
\textbf{1} hân] lan Q \textbf{2} enbôt] enbot Im R \textbf{3} Artuse] kúnig artus W Artus R \textbf{4} dâ] do W Q \textbf{5} Der] Oer U  $\cdot$ dâ] do U Q \textbf{6} heidenîn] heiden nin U  $\cdot$ Eckuba] eckuͦba U Ekuba R \textbf{7} sô prîste] Zu preyse Q  $\cdot$ Plymizol] plimizol U Q R \textbf{8} Jofreit] Iofrid W Jofrit R  $\cdot$ fis Idol] fifidol U fißydol W fisidol Q R \textbf{9} Artuse] Artus W Q R \textbf{10} daz er vreuden] Daran er froͤden W Der e frewde Q Der ir froͯd R  $\cdot$ bejagete] heiagte W \textbf{11} Jofreit] Iofrid W Jofrit R \textbf{14} höveschlîchen] hofflichen R \textbf{17} Gahmuretes] Gahmuͦretes U gamúretes Q \textbf{19} Britonois] briteneis Q \textbf{20} Jofreit] Jofrit R \textbf{23} ûz] vser R \textbf{24} sîn wâpenlîch] Seine weppenliche Q Sine wauppenkliche R \textbf{25} möhte] mochte U Q \textbf{26} hant] hot Q  $\cdot$ töhte] dochte U (Q) \textbf{27} Lover] Louer Q R  $\cdot$ Britanie] Britanye U britange Q  $\cdot$ und] \textit{om.} Q R \textbf{28} unz an] mit an U (R) vnd von Q  $\cdot$ Wizsant] wisant Q \textbf{29} der dar] Dar der R  $\cdot$ leget] legte Q lege R  $\cdot$ al] an Q \newline
\end{minipage}
\end{table}
\end{document}
