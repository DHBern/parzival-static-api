\documentclass[8pt,a4paper,notitlepage]{article}
\usepackage{fullpage}
\usepackage{ulem}
\usepackage{xltxtra}
\usepackage{datetime}
\renewcommand{\dateseparator}{.}
\dmyyyydate
\usepackage{fancyhdr}
\usepackage{ifthen}
\pagestyle{fancy}
\fancyhf{}
\renewcommand{\headrulewidth}{0pt}
\fancyfoot[L]{\ifthenelse{\value{page}=1}{\today, \currenttime{} Uhr}{}}
\begin{document}
\begin{table}[ht]
\begin{minipage}[t]{0.5\linewidth}
\small
\begin{center}*D
\end{center}
\begin{tabular}{rl}
\textbf{421} & \begin{large}D\end{large}er lantgrâve ellens rîche\\ 
 & sprach: "ir redet dem gelîche,\\ 
 & als maneger weiz an iu vür wâr\\ 
 & iwer zît \textbf{unt} iwer jâr.\\ 
5 & ir râtet mir, \textbf{dar} ich wolt \textbf{iedoch},\\ 
 & \textbf{unt} sprechet, ir tætet, \textbf{als riet ein} koch\\ 
 & den küenen Niblungen,\\ 
 & die sich unbetwungen\\ 
 & ûz huoben, dâ man an in rach,\\ 
10 & daz Sivride dâ vor geschach.\\ 
 & mich muoz hêr Gawan slahen tôt\\ 
 & oder ich gelêre in râche nôt."\\ 
 & "Des volge ich", sprach Liddamus,\\ 
 & "wan swaz sîn œheim Artus\\ 
15 & hât unt die von India,\\ 
 & der mirz hie gæbe, als siz hânt dâ,\\ 
 & \textbf{der} mirz ledeclîche bræhte,\\ 
 & ich liezez, ê \textbf{daz} ich væhte.\\ 
 & Nû \textbf{behaldet} prîs, des man iu giht.\\ 
20 & Segramors enbin ich niht,\\ 
 & den man durch vehten binden muoz.\\ 
 & ich \textbf{erwirbe} sus wol \textbf{küneges} gruoz.\\ 
 & Sibeche nie swert erzôch,\\ 
 & er was ie \textbf{bî den}, dâ man vlôch.\\ 
25 & doch muose man \textbf{in} vlêhen.\\ 
 & grôze gebe unt starkiu lêhen\\ 
 & enpfieng er \textbf{von} Ermenriche genuoc.\\ 
 & nie swert er doch durch helm gesluoc.\\ 
 & mir wirt verschert nimmer vel\\ 
30 & durch iuch, hêr Kyngrimursel.\\ 
\end{tabular}
\scriptsize
\line(1,0){75} \newline
D Fr1 Fr5 \newline
\line(1,0){75} \newline
\textbf{1} \textit{Initiale} D Fr5  \textbf{13} \textit{Capitulumzeichen} Fr5   $\cdot$ \textit{Majuskel} D  \textbf{19} \textit{Majuskel} D  \newline
\line(1,0){75} \newline
\textbf{2} gelîche] vil giliche Fr5 \textbf{5} dar ich wolt] daz wolt ich Fr5 \textbf{6} tætet] tetinz Fr5 \textbf{7} den] Dez Fr5  $\cdot$ Niblungen] Nibelvngen Fr1 nibilungin Fr5 \textbf{9} ûz] dar Fr1 \textbf{10} Sivride] Sîvride D Fr1 in vride Fr5  $\cdot$ vor] von Fr5 \textbf{11} mich muoz hêr Gawan] her Gawan mvͦz mich Fr1 mich muͦez her Gauwan Fr5 \textbf{13} Liddamus] Lẏddamvs Fr1 \textbf{15} India] Jndia D Fr1 Fr5 \textbf{16} mirz] mirs Fr5  $\cdot$ hie] \textit{om.} Fr1 \textbf{17} der] vnt Fr1 \textbf{18} liezez] liez ez Fr5  $\cdot$ daz] danne Fr5 \textbf{20} Segramors] Seygremors Fr5 \textbf{23} Sibeche] Sîbche D Sẏbeche Fr1 Sibiche Fr5  $\cdot$ erzôch] gizoch Fr5 \textbf{25} doch muose] Do mvͦes Fr5 \textbf{27} Ermenriche] Ermeriche D Ermenrîche Fr1 ermirrich Fr5 \textbf{28} gesluoc] sluͦec Fr5 \textbf{29} verschert] virsert Fr5 \textbf{30} Kyngrimursel] Kẏngrimvrsel Fr1 \newline
\end{minipage}
\hspace{0.5cm}
\begin{minipage}[t]{0.5\linewidth}
\small
\begin{center}*m
\end{center}
\begin{tabular}{rl}
 & der la\textit{nt}grâve ellens rîche\\ 
 & sprach: "ir redet dem gelîche,\\ 
 & als maniger weiz an iu vür wâr\\ 
 & iuwer zît \textbf{und} iuwer jâr.\\ 
5 & ir râtet mir, \textbf{dar} ich wolte \textbf{iedoch},\\ 
 & \textbf{und} sprechet, ir tætet, \textbf{als r\textit{ie}t ein} koch\\ 
 & den küenen Nibelungen,\\ 
 & die sich unbetwungen\\ 
 & \textit{û}z huoben, d\textit{â} man an in rach,\\ 
10 & daz Syfride dâ vor geschach.\\ 
 & mich muoz hêr Gawan slahen tôt\\ 
 & oder ich gelêre in râche nôt."\\ 
 & "des volge ich", sprach Liddamus,\\ 
 & "wanne waz sî\textit{n} \textit{œ}heim Artus\\ 
15 & hât und \textit{d}ie von India,\\ 
 & der mirz hie gæbe, als  hânt dâ,\\ 
 & \textbf{und} mirz ledeclîch bræhte,\\ 
 & ich lieze ez, ê \textbf{danne} ich væhte.\\ 
 & nû \textbf{haltet} prîs, d\textit{e}s man iu giht.\\ 
20 & Segramors enbin ich niht,\\ 
 & den man durch vehten binden muoz.\\ 
 & ich \textbf{erwürbe} sus wol \textbf{küniges} gruoz.\\ 
 & Sibec\textit{h}e nie swert erzôch,\\ 
 & er was ie, d\textit{â} man \textbf{dô} vlôch.\\ 
25 & doch muose man \textbf{im} vlêhen.\\ 
 & grôze gâbe und starkiu lêhen\\ 
 & enpfienc er \textbf{von} Ermenrich genuoc.\\ 
 & nie swert er doch durch helm gesluoc.\\ 
 & mir wirt vers\textit{ch}ertet niemer vel\\ 
30 & durch iuch, hêr Kingri\textit{m}ursel.\\ 
\end{tabular}
\scriptsize
\line(1,0){75} \newline
m n o \newline
\line(1,0){75} \newline
\newline
\line(1,0){75} \newline
\textbf{1} lantgrâve] latmgraffe m \textbf{3} als] An n \textbf{5} dar] das n \textbf{6} sprechet] sprechen n  $\cdot$ tætet] dientent o  $\cdot$ riet] reit m \textbf{7} Nibelungen] nebelingen n nebelungen o \textbf{8} die] Jch o \textbf{9} ûz] Bs m  $\cdot$ huoben] heben o  $\cdot$ dâ] do m n o  $\cdot$ an] \textit{om.} n \textbf{10} Syfride] sẏ fride m sẏfrit n sifrit o \textbf{11} \textit{Versfolge 421.12-11} n   $\cdot$ hêr Gawan] hergawan m \textbf{13} des] Das n (o)  $\cdot$ Liddamus] lidamus o \textbf{14} sîn œheim] sin sun vnd oͯhem m  $\cdot$ Artus] artusz o \textbf{15} die] hie m  $\cdot$ India] jndia m n \textbf{17} ledeclîch] lideclich n (o) \textbf{18} lieze] liesz o \textbf{19} haltet] behaltet n hab o  $\cdot$ des] das m den o \textbf{20} Segramors] Segramuͯrs o \textbf{21} binden] [bin]: binden o \textbf{22} erwürbe] erwirbe n o \textbf{23} Sibeche] Sibehte m Sebeche o \textbf{24} er] E o  $\cdot$ dâ] do m n  $\cdot$ dô] \textit{om.} n o \textbf{25} muose] musse m muͯsz n  $\cdot$ im] eyn o  $\cdot$ vlêhen] [fliehen]: flehen o \textbf{27} Ermenrich] er menrich m o \textbf{28} gesluoc] sluͯg n \textbf{29} verschertet] versertet m \textbf{30} Kingrimursel] kingrinvrsel m kingrumúrsel n konigrummesel o \newline
\end{minipage}
\end{table}
\newpage
\begin{table}[ht]
\begin{minipage}[t]{0.5\linewidth}
\small
\begin{center}*G
\end{center}
\begin{tabular}{rl}
 & der lantgrâve ellens rîche\\ 
 & sprach: "ir reit dem gelîche,\\ 
 & als maniger weiz an iu vür wâr\\ 
 & iwer zît \textbf{unde} iwer jâr.\\ 
5 & ir râtet mir, \textbf{dar} ich wolt \textbf{doch},\\ 
 & \textbf{unt} sprechet, ir tætet, \textbf{als riet ein} koch\\ 
 & den küenen Nibelungen,\\ 
 & die sich unbetwungen\\ 
 & ûz huoben, dâ man an in rach,\\ 
10 & daz Sifride dâ vor geschach.\\ 
 & mich muoz hêr Gawan slahen tôt\\ 
 & oder ich gelêre in râche nôt."\\ 
 & "des volge ich", sprach Lidamus,\\ 
 & "wan swaz sîn œheim Artus\\ 
15 & h\textit{â}t unde die von India,\\ 
 & der mirz hie gæbe, als siz hânt dâ,\\ 
 & \textbf{der} mirz lediclîche br\textit{æ}hte,\\ 
 & ich liezez, ê ich væhte.\\ 
 & nû \textbf{behaltet} brîs, des man iu giht.\\ 
20 & Segremors enbin ich niht,\\ 
 & den man durch vehten binden muoz.\\ 
 & ich \textbf{erwirbe} sus wol \textbf{küniges} gruoz.\\ 
 & Sibeche nie swert erzôch,\\ 
 & er was ie \textbf{gerne}, dâ man vlôch.\\ 
25 & doch muose man \textbf{in} vlêhen.\\ 
 & grôz gâbe unde starkiu lêhen\\ 
 & \begin{large}E\end{large}npfie er \textbf{von} Ermenrich genuoc.\\ 
 & nie swert er doch durch helm gesluoc.\\ 
 & mir wirt verschert nimer vel\\ 
30 & durch iuch, hêr Kingrimursel.\\ 
\end{tabular}
\scriptsize
\line(1,0){75} \newline
G I O L M Q R Z \newline
\line(1,0){75} \newline
\textbf{1} \textit{Initiale} I O L R Z  \textbf{13} \textit{Initiale} I  \textbf{27} \textit{Initiale} G  \textbf{29} \textit{Initiale} I  \newline
\line(1,0){75} \newline
\textbf{1} der] ÷er O Der konnick M  $\cdot$ ellens rîche] erentreiche Q \textbf{3} als] An Q  $\cdot$ iu] auch Q \textbf{5} dar] daz L  $\cdot$ ich wolt] wolte L wolt ich R  $\cdot$ doch] ioch R \textbf{6} unt sprechet] Nv sprechen L  $\cdot$ tætet] ratet M Riettent R  $\cdot$ riet] tet R \textbf{7} den] Dem Q  $\cdot$ Nibelungen] Nybelvngen O nýbelvngen L Nebelungen M nyblúngen Q Nẏblungen R nebulungen Z \textbf{8} sich] ich sihe I \textbf{9} huoben] hawen I  $\cdot$ dâ] do L Q R  $\cdot$ an in rach] an yn [sach]: rach M in an sach Q \textbf{10} Sifride] sifrit I Sýfride L siffride M sie frid Q Syfriden R Syfride Z \textbf{11} hêr Gawan] ergawan M her Gawin R \textbf{12} oder] olde G  $\cdot$ gelêre in] gelerne I lerre In R \textbf{13} Lidamus] Liddamus O (L) (Q) Z litdamus M Liddanus R \textbf{14} wan] Vnd L  $\cdot$ swaz] waz L (M) (Q) (R)  $\cdot$ sîn] si O  $\cdot$ Artus] Artuͯs L [litdamus]: artus M \textbf{15} hât] hant G  $\cdot$ die] der I  $\cdot$ India] Jndia O L Q R Z \textbf{16} gæbe] geben R  $\cdot$ siz] isz M \textbf{17} \textit{Versfolge 421.18-17} O L Q R   $\cdot$ der] vnd I  $\cdot$ lediclîche] ledechlchichen O lechliche Q  $\cdot$ bræhte] brahte G \textbf{18} ê] E daz L ehir danne M (Q) (Z) \textbf{19} behaltet] behabt O Q (R)  $\cdot$ iu] \textit{om.} O \textbf{20} enbin] des pin O bin L Q R den bin M \textbf{21} vehten] rechten Q \textbf{22} sus] es Q  $\cdot$ küniges] [kundes]: kunges Q \textbf{23} Sibeche] [sin becche]: si becche I Sybche O Sibich L Z Sie beche M Sibche Q Sibch R  $\cdot$ erzôch] gezoch I L (R) \textbf{24} gerne] \textit{om.} I R bei den O (L) (M) (Q) (Z)  $\cdot$ dâ] die Q do R \textbf{25} doch] Do Q  $\cdot$ muose] muͤz I  $\cdot$ man] en man M  $\cdot$ in] ým L da M \textbf{26} grôz gâbe] Groziv geb O  $\cdot$ starkiu] starchen I \textit{om.} M  $\cdot$ lêhen] [flehen]: lehen Z \textbf{27} Enpfie] Enpfienge O  $\cdot$ er] \textit{om.} M  $\cdot$ Ermenrich] Ermriche O Ermentriche L ermenriche M (Q) emrichen R Ermerich Z \textbf{28} doch durch] durc den I dur R  $\cdot$ helm] swerter Q  $\cdot$ gesluoc] sluͤc I (M) (Q) (R) (Z) \textbf{29} mir] Min Q  $\cdot$ wirt] enwirt L  $\cdot$ verschert] versert R (Z)  $\cdot$ vel] vil M \textbf{30} Kingrimursel] Kyngrimvrsel O (M) (Q) kungrumursel R \newline
\end{minipage}
\hspace{0.5cm}
\begin{minipage}[t]{0.5\linewidth}
\small
\begin{center}*T
\end{center}
\begin{tabular}{rl}
 & \begin{large}D\end{large}er lantgrâve ellens rîche\\ 
 & sprach: "ir redet dem glîche,\\ 
 & als maneger weiz an iu vür wâr\\ 
 & iuwere zît, iuwer jâr.\\ 
5 & ir râtet mir, \textbf{als} ich wolte \textbf{doch},\\ 
 & \textbf{ir} sprechet, ir tætet \textbf{alsein} koch\\ 
 & den küenen Nybelungen,\\ 
 & die sich unbetwungen\\ 
 & ûz huoben, dâ man an in rach,\\ 
10 & daz Sifride dâ vor geschach.\\ 
 & mich muoz hêr Gawan slahen tôt\\ 
 & oder ich gelêre in râche nôt."\\ 
 & "Des volgich \textbf{im}", sprach Lyddamus,\\ 
 & "wan swaz sîn œheim Artus\\ 
15 & hât unde die von India,\\ 
 & der mirz hie gæbe, als siz hânt dâ,\\ 
 & \hspace*{-.7em}\big| ich liezez, ê \textbf{dann}ich v\textit{æ}hte,\\ 
 & \hspace*{-.7em}\big| \dag \textbf{der} mirz ledeclîchen bræhte.\dag \\ 
 & nû \textbf{behabt} prîs, des man iu giht.\\ 
20 & Segremors enbin ich niht,\\ 
 & den man durch vehten binden muoz.\\ 
 & ich \textbf{erwürbe} sus wol \textbf{künege} gruoz.\\ 
 & Sibeche nie swert erzôch,\\ 
 & er was ie \textbf{bî den}, \textit{dâ} man vlôch.\\ 
25 & doch muose man \textbf{im} vlêhen.\\ 
 & grôz gebe unde stark\textit{iu} lêhen\\ 
 & enpfienc er \textbf{mit} Ermeneriche genuoc.\\ 
 & nie swert er doch durch helm gesluoc.\\ 
 & Mir wirt verschert niemer vel\\ 
30 & durch iuch, hêr Kyngrimursel.\\ 
\end{tabular}
\scriptsize
\line(1,0){75} \newline
T U V W \newline
\line(1,0){75} \newline
\textbf{1} \textit{Initiale} T U V W  \textbf{13} \textit{Majuskel} T  \textbf{29} \textit{Majuskel} T  \newline
\line(1,0){75} \newline
\textbf{4} zît] zit vnde V (W) \textbf{5} als] dar U V W \textbf{6} ir sprechet] Vnd sprechet U (V) (W)  $\cdot$ tætet] [*det]: dedet U ratet W  $\cdot$ alsein] als riet ein V W \textbf{7} Nybelungen] nebeluͦngen U nibelvngen V (W) \textbf{9} huoben] huͦhen W  $\cdot$ dâ] do W \textbf{10} Sifride] sy fride W \textbf{13} im] \textit{om.} V W  $\cdot$ Lyddamus] litdamus V lidamus W \textbf{14} swaz] waz U (W) \textbf{15} India] Jndia T Judia U \textbf{18} dannich] dannoch U dann das ich W  $\cdot$ væhte] vehte T (U) (V) (W) \textbf{17} der] [*]: vnde V \textbf{19} >Nv behabent pris [*]: dez men v́ch giht< V  $\cdot$ behabt] behaldet U habt W \textbf{20} enbin] bin U V W \textbf{21} vehten] vuͦchte U \textbf{22} künege] kv́niges V (W) \textbf{23} Sibeche] Sibiche V Sybeche W  $\cdot$ erzôch] gezoch V W \textbf{24} bî den dâ] biden T bi den do U (W) [*]: do bi  V \textbf{25} muose] mvese T mvͤste V  $\cdot$ im] [*]: im V \textbf{26} starkiu] starcke T \textbf{27} mit] von V W  $\cdot$ Ermeneriche] Ermenriche U [ermenrich]: ermenriche V einem reiche W \textbf{28} doch] \textit{om.} W  $\cdot$ gesluoc] sluͦc U (W) \textbf{29} verschert] verseret U verschertet V W \textbf{30} iuch] îv T  $\cdot$ Kyngrimursel] kyngrimorsel U kẏngrimvrsel V kingrimursel W \newline
\end{minipage}
\end{table}
\end{document}
