\documentclass[8pt,a4paper,notitlepage]{article}
\usepackage{fullpage}
\usepackage{ulem}
\usepackage{xltxtra}
\usepackage{datetime}
\renewcommand{\dateseparator}{.}
\dmyyyydate
\usepackage{fancyhdr}
\usepackage{ifthen}
\pagestyle{fancy}
\fancyhf{}
\renewcommand{\headrulewidth}{0pt}
\fancyfoot[L]{\ifthenelse{\value{page}=1}{\today, \currenttime{} Uhr}{}}
\begin{document}
\begin{table}[ht]
\begin{minipage}[t]{0.5\linewidth}
\small
\begin{center}*D
\end{center}
\begin{tabular}{rl}
\textbf{564} & umbe mich. swaz ich veiles hân,\\ 
 & daz ist iu \textbf{gar dan} undertân.\\ 
 & \begin{large}V\end{large}art vürbaz, lâts walten got.\\ 
 & hât iuch Plippalinot,\\ 
5 & der verge, her gewîset?\\ 
 & manec vrouwe prîset\\ 
 & iwer komen in ditze lant,\\ 
 & ob si \textbf{hie} erlœset iwer hant.\\ 
 & welt ir nâch \textbf{âventiuren} gên,\\ 
10 & sô lât \textbf{daz} ors al stille stên;\\ 
 & \textbf{des hüete} ich, welt irz an \textbf{mich} lân."\\ 
 & Dô sprach mîn hêr Gawan:\\ 
 & "wærez in iwern mâzen,\\ 
 & ich woltz iu gerne lâzen.\\ 
15 & nû entsitze ich iwer rîcheit;\\ 
 & sô rîchen marschalc ez erleit\\ 
 & nie, sît ich dar ûf gesaz."\\ 
 & Der krâmære sprach ân \textbf{allen} haz:\\ 
 & "\textbf{hêrre}, ich \textbf{selbe} und \textbf{al} mîn habe\\ 
20 & - waz \textbf{m\textit{ö}hte} ich mêr \textbf{nû} \textbf{sprechen} drabe? -\\ 
 & ist iwer, sult ir hie genesen.\\ 
 & wes \textbf{m\textit{ö}ht ich} billîcher wesen?"\\ 
 & Gawanen sîn ellen lêrte,\\ 
 & ze vuoz \textit{er} vürbaz kêrte\\ 
25 & manlîche und unverzagt.\\ 
 & als ich iu ê hân gesagt,\\ 
 & \textbf{er vant} der bürge wîte,\\ 
 & daz ieslîch ir sîte\\ 
 & stuont mit bûwenlîcher wer.\\ 
30 & \textbf{vür} \textbf{allen} \textbf{sturm} niht ein ber\\ 
\end{tabular}
\scriptsize
\line(1,0){75} \newline
D \newline
\line(1,0){75} \newline
\textbf{3} \textit{Initiale} D  \textbf{12} \textit{Majuskel} D  \textbf{18} \textit{Majuskel} D  \newline
\line(1,0){75} \newline
\textbf{20} möhte] mohte D \textbf{22} möht] moht D \textbf{24} er] \textit{om.} D \newline
\end{minipage}
\hspace{0.5cm}
\begin{minipage}[t]{0.5\linewidth}
\small
\begin{center}*m
\end{center}
\begin{tabular}{rl}
 & umb mic\textit{h}. \textit{w}az ich veiles hân,\\ 
 & daz \textit{ist} iu \textbf{gar den} undertân.\\ 
 & vart vür\textit{b}az, lât es walten got.\\ 
 & het iuch Pl\textit{i}pp\textit{a}l\textit{in}ot,\\ 
5 & der vere, her gewîset?\\ 
 & manic vrouwe prîset\\ 
 & iuwer komen in diz lant,\\ 
 & ob si erlœset iuwer hant.\\ 
 & welt ir nâch \textbf{âventiur} gân,\\ 
10 & sô lât \textbf{daz} ros al stille stân;\\ 
 & \textbf{des hüete} ich, wel\textit{t} \textit{irz} an \textbf{mir} lân."\\ 
 & dô sprach mîn hêr Gawan:\\ 
 & "wær ez in iuwer\textit{n} mâzen,\\ 
 & ich wolt ez iu gerne lâzen.\\ 
15 & nû entsitze ich iuwer rîcheit;\\ 
 & sô rîchen marschalc ez erleit\\ 
 & nie, sît ich dar ûf gesaz."\\ 
 & der krâmer sprach âne \textbf{allen} haz:\\ 
 & "\textbf{hêrre}, ich \textbf{selbe} und \textbf{al} mîn habe\\ 
20 & - waz \textbf{m\textit{ö}ht} ich mêr \textbf{nû} \textbf{sprechen} drabe? -\\ 
 & \dag daz zwâr\dag , solt ir hie genesen.\\ 
 & wes \textbf{solt ir} billîcher wesen?"\\ 
 & Gawanen sîn ellen lêrte,\\ 
 & zuo vuoz er vürbaz kêrte\\ 
25 & manlîch und unverzaget.\\ 
 & als ich iu ê hân gesaget,\\ 
 & \textbf{vant er} der bürge wîte,\\ 
 & daz ieglîch ir sîte\\ 
 & stuont mit bûwelîcher wer.\\ 
30 & \textbf{vor} \textbf{allem} \textbf{sturm} niht ein ber\\ 
\end{tabular}
\scriptsize
\line(1,0){75} \newline
m n o \newline
\line(1,0){75} \newline
\newline
\line(1,0){75} \newline
\textbf{1} \textit{Die Verse 562.7-564.18 fehlen} o   $\cdot$ mich waz] mich ich was m \textbf{2} ist] \textit{om.} m \textbf{3} vürbaz] fuͯrdas m \textbf{4} Plippalinot] plimplamot m plimamot n \textbf{10} al] alle n \textbf{11} ich] \textit{om.} n  $\cdot$ welt irz] woltte m  $\cdot$ mir] mich n \textbf{12} hêr] herre her n \textbf{13} iuwern] uwer m \textbf{20} möht] moht m (o)  $\cdot$ nû] ẏm o \textbf{21} daz zwâr] Das ist wor n Dast war o \textbf{22} solt] sol o \newline
\end{minipage}
\end{table}
\newpage
\begin{table}[ht]
\begin{minipage}[t]{0.5\linewidth}
\small
\begin{center}*G
\end{center}
\begin{tabular}{rl}
 & \begin{large}U\end{large}mb mich. swaz ich veiles hân,\\ 
 & daz ist iu \textbf{gar danne} undertân.\\ 
 & vart vürbaz, lât es walten got.\\ 
 & hât \textit{iuch} Pliplalinot,\\ 
5 & der verje, her gewîset?\\ 
 & manic vrouwe brîset\\ 
 & iuwer komen in diz lant,\\ 
 & ob s\textit{i} \textit{e}rlœset iuwer hant.\\ 
 & welt ir nâch \textbf{âventiure} gên,\\ 
10 & sô lât \textbf{diz} ors al stille stên;\\ 
 & \textbf{des hüete} ich, welt irz an \textbf{mich} lân."\\ 
 & dô sprach mîn hêrre Gawan:\\ 
 & "wærez in iuwern mâzen,\\ 
 & ich woldez \textit{iu gerne} lâzen.\\ 
15 & n\textit{û} entsitze ich iuwer rîcheit;\\ 
 & sô rîchen marschalc ez erleit\\ 
 & nie, sît ich drûf gesaz."\\ 
 & der krâmære sprach ân \textbf{allen} haz:\\ 
 & "\textbf{hêrre}, ich \textbf{selbe} unde \textbf{alle} mîn habe\\ 
20 & - waz \textbf{m\textit{ö}ht} ich mêre \textbf{brechen} drabe? -\\ 
 & ist iuwer, sult ir hie genesen.\\ 
 & wes \textbf{\textit{möh}t ich} billîcher wesen?"\\ 
 & Gawan sîn ellen lêrte,\\ 
 & ze vuoze er vürbaz kêrte\\ 
25 & manlîche unde unverzaget.\\ 
 & als ich iu ê hân gesaget,\\ 
 & \textbf{er vant} der bürge wîte,\\ 
 & daz ieslîch ir sîte\\ 
 & stuont mit bûwelîcher wer.\\ 
30 & \textbf{vür} \textbf{allen} \textbf{sturm} niht ein ber\\ 
\end{tabular}
\scriptsize
\line(1,0){75} \newline
G I L M Z \newline
\line(1,0){75} \newline
\textbf{1} \textit{Initiale} G L Z  \textbf{11} \textit{Initiale} I  \textbf{25} \textit{Überschrift:} Hie get her gawan in die bvrch zv marvale nach der auentevre Z  \textbf{26} \textit{Initiale} Z  \newline
\line(1,0){75} \newline
\textbf{1} swaz] waz L (M) \textbf{2} iu] uwir M  $\cdot$ gar danne] dann gar I \textbf{3} es] sin I Z \textbf{4} iuch] \textit{om.} G  $\cdot$ Pliplalinot] [pli*]: Pliplalinot G [pliatpalnot]: pliatpalinot I Plipalinot L (M) Z \textbf{6} brîset] gepriset M \textbf{7} diz] disze L \textbf{8} si erlœset] si hie erloset G \textbf{9} âventiure] auenturen I \textbf{10} diz] das M  $\cdot$ al] \textit{om.} Z \textbf{11} ich] ich ev I \textbf{12} dô] Da M  $\cdot$ hêrre Gawan] ergawan M \textbf{13} iuwern mâzen] uwir masze M \textbf{14} iu gerne] gerne iv G \textbf{15} nû] Nvne G \textbf{16} erleit] ny erleit M \textbf{17} nie] \textit{om.} M  $\cdot$ drûf] erst dar uff M  $\cdot$ gesaz] saz I \textbf{18} allen] \textit{om.} M \textbf{19} hêrre] \textit{om.} Z  $\cdot$ selbe] selben M  $\cdot$ alle] al Z \textbf{20} möht] moht G L (M) Z  $\cdot$ mêre] nv mer L mer nu Z  $\cdot$ brechen] gebrechen I sprechen L (M) (Z) \textbf{21} ist] Ez ist Z \textbf{22} möht] solt G moht I L (M) \textbf{23} Gawan] Gawanen L  $\cdot$ sîn] sine M \textbf{24} ze vuoze] zefuzen I \textbf{26} iu] \textit{om.} M \textbf{28} daz iegeslicher site I \textbf{29} stuͤnt vur buwenliche wer I \textbf{30} allen sturm] alle stuͯrme L allisz stuͯrmen M  $\cdot$ ber] her I [her]: ber L \newline
\end{minipage}
\hspace{0.5cm}
\begin{minipage}[t]{0.5\linewidth}
\small
\begin{center}*T
\end{center}
\begin{tabular}{rl}
 & umbe mich. swaz ich veiles hân,\\ 
 & daz ist iu \textbf{danne gar} undertân.\\ 
 & vart vürbaz, lâts walten got.\\ 
 & hât iuch Plypalinot,\\ 
5 & der verje, her gewîset?\\ 
 & manec vrouwe prîset\\ 
 & iuwer komen in diz lant,\\ 
 & ob si \textbf{hie} erlœset iuwer hant.\\ 
 & welt ir nâch \textbf{âventiure} gên,\\ 
10 & sô lât \textbf{diz} ors alstille stên;\\ 
 & \textbf{daz behalt} ich, welt irz an \textbf{mich} lân."\\ 
 & Dô sprach mîn hêr Gawan:\\ 
 & "wær ez in iuwern mâzen,\\ 
 & ich wolt ez iu gerne lâzen.\\ 
15 & nû entsitz ich iuwer rîcheit;\\ 
 & sô rîchen marschalc ez erleit\\ 
 & nie, sît ich drûf gesaz."\\ 
 & Der krâmer sprach âne haz:\\ 
 & "ich unde \textbf{alle} mîne habe\\ 
20 & - waz \textbf{mac} ich mê \textbf{nû} \textbf{sprechen} drabe? -,\\ 
 & \textbf{dês}t iuwer, sult ir hie genesen.\\ 
 & wes \textbf{solt ich} billîcher wesen?"\\ 
 & \textit{\begin{large}G\end{large}}awan sîn ellen lêrte,\\ 
 & ze vuoz er vürbaz kêrte\\ 
25 & manlîch unde unverzaget.\\ 
 & als ich iu ê hân gesaget,\\ 
 & \textbf{er vant} der bürge wîte,\\ 
 & daz ieslîch ir sîte\\ 
 & stuont mit bûwenlîcher wer.\\ 
30 & \textbf{vor} \textbf{allen} \textbf{stürmen} niht ein ber\\ 
\end{tabular}
\scriptsize
\line(1,0){75} \newline
T U V W Q R Fr25 Fr39 \newline
\line(1,0){75} \newline
\textbf{1} \textit{Initiale} Fr25 Fr39   $\cdot$ \textit{Capitulumzeichen} R  \textbf{12} \textit{Majuskel} T  \textbf{18} \textit{Majuskel} T  \textbf{23} \textit{Initiale} T V W  \newline
\line(1,0){75} \newline
\textbf{1} \textit{Die Verse 553.1-599.30 fehlen} U   $\cdot$ swaz] was W Q R \textbf{3} lâts] lat daz R lat des Fr39 \textbf{4} iuch] îv T  $\cdot$ Plypalinot] plẏpaminot V plipalinot W R Fr25 Fr39 pliualinot Q \textbf{7} diz] das Q R \textbf{8} hie erlœset iuwer] erloͤset [*]: hie úwer V hîe lvͦset iwer Fr25 \textbf{10} diz] das R  $\cdot$ alstille] stille V \textbf{13} iuwern mâzen] úwer masze R \textbf{14} ez iu] eúch W euch es Q \textbf{15} nû] entsicz R \textbf{16} ez] ez nie Fr25 \textbf{17} gesaz] sas R \textbf{18} âne] an allen W \textbf{20} ich] úch R  $\cdot$ mê nû sprechen] nun sprechen mer Q (Fr25) mer sprechen R me:::v gesprechen Fr39 \textbf{21} dêst] Die ist Q  $\cdot$ hie] hie nv V \textbf{22} solt] moht V (Q) Fr25 Fr39 moͤcht W (R)  $\cdot$ ich] ez V \textbf{23} Gawan] ÷awan T Gawin R Gawanen Fr25 \textbf{26} als] As V  $\cdot$ ê] \textit{om.} Fr25 \textbf{27} der bürge] die bvrch Fr25 \textbf{29} mit] Jn R  $\cdot$ bûwenlîcher] bvlicher Fr25 bu:::cher Fr39 \textbf{30} allen] alle Q R  $\cdot$ stürmen] sturm W (Q) (R) (Fr25) Fr39  $\cdot$ ein ber] enber W (R) [en]: eyn ber Q \newline
\end{minipage}
\end{table}
\end{document}
