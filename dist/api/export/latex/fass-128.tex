\documentclass[8pt,a4paper,notitlepage]{article}
\usepackage{fullpage}
\usepackage{ulem}
\usepackage{xltxtra}
\usepackage{datetime}
\renewcommand{\dateseparator}{.}
\dmyyyydate
\usepackage{fancyhdr}
\usepackage{ifthen}
\pagestyle{fancy}
\fancyhf{}
\renewcommand{\headrulewidth}{0pt}
\fancyfoot[L]{\ifthenelse{\value{page}=1}{\today, \currenttime{} Uhr}{}}
\begin{document}
\begin{table}[ht]
\begin{minipage}[t]{0.5\linewidth}
\small
\begin{center}*D
\end{center}
\begin{tabular}{rl}
\textbf{128} & daz gît gelücke und hôhen muot,\\ 
 & ob si kiusche ist unt guot.\\ 
 & Dû solt \textbf{ouch} wizzen, sun mîn,\\ 
 & der \textbf{stolze}, \textbf{küene} Læhelin\\ 
5 & dînen vürsten ab ervaht zwei lant,\\ 
 & \textbf{die} solten dienen dîner hant,\\ 
 & Wâleis und Norgals.\\ 
 & ein dîn vürste, Turkentals,\\ 
 & den tôt von sîner hende enpfienc.\\ 
10 & dîn volc er sluoc unt vienc."\\ 
 & "\textbf{diz} rich ich, muoter, \textbf{ruochte}\textbf{s} got.\\ 
 & in verwundet noch mîn gabylôt."\\ 
 & \textbf{Des} morgens, dô der tac erschein,\\ 
 & der knappe \textbf{balde wart} \textbf{ein},\\ 
15 & im was gegen Artuse gâch.\\ 
 & \textbf{vrou Herzeloyde} \textbf{in kuste unt} lief im nâch.\\ 
 & der \textbf{werlde riwe} al geschach.\\ 
 & dô si ir \textbf{sun} niht \textbf{langer} sach\\ 
 & - der \textbf{reit enwec}, wem ist deste baz? -,\\ 
20 & dâ viel diu vrouwe valsches laz\\ 
 & ûf die erde, al dâ si jâmer sneit,\\ 
 & sô daz \textbf{si} ein sterben niht vermeit.\\ 
 & ir vil getriulîcher tôt\\ 
 & der vrouwen wert die hellenôt.\\ 
25 & Ôwol \textbf{si}, daz si muoter wart!\\ 
 & sus vuor di\textit{e} lônes bernde vart\\ 
 & ein wurzel der güete\\ 
 & \textbf{unt} ein st\textit{a}m der diemüete.\\ 
 & owê, daz wir nû niht \textbf{en}hân\\ 
30 & ir sippe \textbf{unz} an den \textbf{elften} spân!\\ 
\end{tabular}
\scriptsize
\line(1,0){75} \newline
D Fr13 \newline
\line(1,0){75} \newline
\textbf{3} \textit{Majuskel} D  \textbf{13} \textit{Initiale} Fr13   $\cdot$ \textit{Majuskel} D  \textbf{25} \textit{Majuskel} D  \newline
\line(1,0){75} \newline
\textbf{7} Norgals] Norgâls D :::ls Fr13 \textbf{8} Turkentals] Tvrkentâls D torkentals Fr13 \textbf{11} rich] reche Fr13  $\cdot$ ruochtes] geruchtz Fr13 \textbf{14} ein] inein Fr13 \textbf{16} \textit{Versfolge 128.17-16} Fr13   $\cdot$ vrou Herzeloyde] Sin muter Fr13 \textbf{17} al] alda Fr13 \textbf{18} ir sun] in Fr13 \textbf{19} Du wart ir verst danne baz Fr13 \textbf{20} dâ] du Fr13 \textbf{26} die] div D \textbf{28} stam] stem D \newline
\end{minipage}
\hspace{0.5cm}
\begin{minipage}[t]{0.5\linewidth}
\small
\begin{center}*m
\end{center}
\begin{tabular}{rl}
 & daz gît gelücke und hôhen muot,\\ 
 & ob si kiusche ist und guot.\\ 
 & dû solt \textbf{ouch} wizzen, sun mîn,\\ 
 & der \textbf{stolze}, \textbf{schœne} Lehelin\\ 
5 & dînen vürsten abe ervaht zwei lant,\\ 
 & \textbf{die} solten dienen dîner hant,\\ 
 & Wâleis und Norgals.\\ 
 & ein dîn vürste, Tarkentals,\\ 
 & den tôt von sîner hende enpfienc.\\ 
10 & dîn volc er sluoc und vienc."\\ 
 & "\textbf{daz} riche ich, muoter, \textbf{ruoche}\textbf{s} got.\\ 
 & in verwundet noch mîn gabilô\textit{t}."\\ 
 & \textbf{\begin{large}D\end{large}es} morgens, dô der tac e\textit{r}schein,\\ 
 & der knappe \textbf{balde wart} \textbf{enein},\\ 
15 & im was gegen Artuse gâch.\\ 
 & \textbf{diu vrouwe} \textbf{in kuste und} lief im nâch.\\ 
 & der \textbf{werde\textit{n}} al\textbf{dâ} geschach,\\ 
 & dô \textit{si} ir \textbf{sun} niht \textbf{langer} sach\\ 
 & - der \textbf{rîtet enwec}, wem ist deste baz? -,\\ 
20 & dâ viel diu vrouwe valsches laz\\ 
 & ûf die erde, aldâ si jâmer sneit,\\ 
 & sô daz \textbf{si} ein sterben niht vermeit.\\ 
 & ir vil getriuweclîcher tôt\\ 
 & d\textit{er} vrouwen wert die hellenôt.\\ 
25 & ôwol, daz si \textbf{ie} muoter wart!\\ 
 & sus vuor die lônes b\textit{e}ren\textit{d}e vart\\ 
 & ein wurzel der güete\\ 
 & \textbf{und} ein stam der diemüete.\\ 
 & owê, daz wir nû niht hân\\ 
30 & ir sippe, \textbf{und} an den \textbf{zwelften} spân\\ 
\end{tabular}
\scriptsize
\line(1,0){75} \newline
m n o \newline
\line(1,0){75} \newline
\textbf{13} \textit{Illustration mit Überschrift:} Also der knappe von siner muͦtter vnd von den sinen hin weg schiet vnd zuͦ einer gar schoͯnen frouwen kam vnd in gar minneclichen enpfing n   $\cdot$ \textit{Initiale} m n o  \newline
\line(1,0){75} \newline
\textbf{1} gît] \textit{om.} n \textbf{4} schœne] kuͯne n (o) \textbf{5} dînen] Der stolze n \textbf{7} Wâleis] Warleisz n Waleise o \textbf{8} Tarkentals] turkentals n o \textbf{9} \textit{Versfolge 128.10-9} m  \textbf{11} ruoches] ruͯchet es n (o) \textbf{12} gabilôt] gabilon m \textbf{13} erschein] ein schein m \textbf{17} werden] [*]: werde m  $\cdot$ aldâ geschach] do beschach n o \textbf{18} si] \textit{om.} m \textbf{20} dâ] Do n o \textbf{21} aldâ] do n o \textbf{23} getriuweclîcher] getruwer n (o) \textbf{24} der] Die m  $\cdot$ die] der o \textbf{25} ôwol] Nuͯ wol n (o) \textbf{26} berende] brendie m \textbf{29} nû] in n o \textbf{30} und] bitze n (o)  $\cdot$ spân] man n [ban]: span o \newline
\end{minipage}
\end{table}
\newpage
\begin{table}[ht]
\begin{minipage}[t]{0.5\linewidth}
\small
\begin{center}*G
\end{center}
\begin{tabular}{rl}
 & daz gît gelücke und hôhen muot,\\ 
 & op si kiusche ist und guot.\\ 
 & dû solt \textbf{ouch} wizzen, sun mîn,\\ 
 & \textbf{daz} der \textbf{stolze} Lehelin\\ 
5 & dînen vürsten abe ervaht zwei lant.\\ 
 & \textbf{diu} solten dienen dîner hant,\\ 
 & Wâleis und Nurgals.\\ 
 & ein dîn vürste, Turkentals,\\ 
 & den tôt von sîner hende enpfienc.\\ 
10 & dîn volc er sluoc und vienc."\\ 
 & "\textbf{\begin{large}D\end{large}az} riche ich, muoter, \textbf{wil}\textbf{z} got.\\ 
 & in verwundet noch mîn gabilôt."\\ 
 & \textbf{des} morgens, dô der tac erschein,\\ 
 & der knappe \textbf{wart vil balde} \textbf{enein},\\ 
15 & im was gein Artuse gâch.\\ 
 & \textbf{diu küniginne} lief im nâch.\\ 
 & der \textbf{werlde riwe} al \textbf{hie} geschach.\\ 
 & dô si ir \textbf{sun} niht \textbf{mêre} sach\\ 
 & - der \textbf{vert von ir}, wem ist deste baz? -,\\ 
20 & dô viel diu vrouwe valsches laz\\ 
 & \textit{ûf} die erde, al dâ si jâmer sneit,\\ 
 & sô daz \textbf{si} ein sterben niht vermeit.\\ 
 & ir vil getriulîcher tôt\\ 
 & der vrouwen wert die hellenôt.\\ 
25 & ôwol \textbf{si}, daz si \textbf{ie} muoter wart!\\ 
 & sus vuor die lônes bernden vart\\ 
 & ein wurzel der güete,\\ 
 & ein stam der diemüete.\\ 
 & owê, daz wir nû niht \textbf{en}hân\\ 
30 & ir sippe \textbf{unze} an den \textbf{einliften} spân!\\ 
\end{tabular}
\scriptsize
\line(1,0){75} \newline
G I O L M Q R Z Fr35 \newline
\line(1,0){75} \newline
\textbf{3} \textit{Initiale} M  \textbf{7} \textit{Initiale} I  \textbf{11} \textit{Initiale} G  \textbf{13} \textit{Initiale} L R Z  \newline
\line(1,0){75} \newline
\textbf{1} gît] Git dir I (O) (R) \textbf{3} dû] Do M  $\cdot$ ouch] \textit{om.} I \textbf{4} daz] \textit{om.} O L M Q R Z  $\cdot$ stolze] stolz chvͦn O (L) (M) (Q) (Z) stolcz kúng R  $\cdot$ Lehelin] Lechelin R \textbf{6} solten] sollin M  $\cdot$ dienen] deinen Q \textbf{7} Wâleis] Walaiz L Walleis M Waleys Q  $\cdot$ Nurgals] norgals I Nvͦrgals O [*]: Norgals L nűrgals Q \textbf{8} vürste] fᵫrsten R  $\cdot$ Turkentals] tuͦrkedals I Tuͯrkantals L tyrkentals M tvrchentals Z \textbf{9} tôt] tut Q \textbf{10} er] \textit{om.} M  $\cdot$ sluoc] erschluͯch L fieng R  $\cdot$ vienc] erschluͦg R \textbf{11} wilz] ruͤch sin I rvͦcht sin O Ruͯches es L ruchtz Q gervhtez Z \textbf{12} verwundet] vorwindet M \textbf{13} dô] dv O da M Z \textbf{14} wart vil balde] balde wart I  $\cdot$ enein] ein ein I mein Q R \textbf{15} was] wart O M  $\cdot$ Artuse] Artuͯse L artus M artusen Z \textbf{16} diu küniginne] herzelaude I Frov herzenlavde (herczeloude M herzelovde Z ) in chvste vnd O (M) (Z) Die vrowe kuͯste in vnd L Herzeloúde (Herczelaude R ) in kuste vnd Q (R)  $\cdot$ im] my M \textbf{17} riwe] truͯwe L (R)  $\cdot$ al] \textit{om.} O L M Q R Z Fr35 \textbf{18} dô] Da M Z  $\cdot$ ir sun] ir svns O (M) (Q) (Z) (Fr35)  $\cdot$ niht mêre] niht langer L nicht en M nymmer Q  $\cdot$ sach] Gesach I \textbf{19} vert] reit R  $\cdot$ ir] \textit{om.} L  $\cdot$ deste] des Z \textbf{20} dô] Da M Z  $\cdot$ vrouwe] vrohe L \textbf{21} ûf] an G  $\cdot$ erde] erden I M R  $\cdot$ al] \textit{om.} I O M Q R Fr35 \textbf{22} si] \textit{om.} Q  $\cdot$ ein sterben] da ein sterben I sterben da O iamers sterben Z \textbf{24} vrouwen wert] frowen werte I wert frawen Q  $\cdot$ die] \textit{om.} I der L (R) \textbf{25} ôwol] Wol L O wol das R  $\cdot$ si ie] si O ye Q (Fr35) \textbf{26} sus] Ausz Q  $\cdot$ die lônes bernden] diu lonsbernde I dy lonis bernde M (Q) der sun sin R \textbf{27} wurzel] wiszel Q \textbf{28} ein] Vnd ein Z  $\cdot$ stam] stein L \textbf{29} owê] Awe I O  $\cdot$ enhân] han O L (M) Q (R) Z Fr35 \textbf{30} unze] bisz Q  $\cdot$ einliften] ein lesten O ein liesten L czwelfftin M \newline
\end{minipage}
\hspace{0.5cm}
\begin{minipage}[t]{0.5\linewidth}
\small
\begin{center}*T (U)
\end{center}
\begin{tabular}{rl}
 & daz gît \textbf{dir} glücke und hôhen muot,\\ 
 & ob si kiusche ist und guot.\\ 
 & dû solt wizzen, sun mîn,\\ 
 & \textbf{daz} der \textbf{küene} Lehelin\\ 
5 & dînen vürsten ab ervaht zwei lant,\\ 
 & \textbf{die} solten dienen dîner hant,\\ 
 & Wâleis und Norgals.\\ 
 & ein dîn vürste, Turkentals,\\ 
 & den tôt von sîner hant enpfienc.\\ 
10 & dîn volc er sluoc und vienc."\\ 
 & "\textbf{daz} rich ich, muoter, \textbf{ruochet} \textbf{ez} got.\\ 
 & in verwundet noch mîn gabilôt."\\ 
 & \textbf{\begin{large}E\end{large}ines} morgen\textit{s}, dô der tac erschein,\\ 
 & der knappe \textbf{wart vil balde} \textbf{enein},\\ 
15 & im was gein Artuse gâch.\\ 
 & \textbf{vrou Herzeloyde} \textbf{in kust und} lief im nâch.\\ 
 & der \textbf{werlde riuwe} al \textbf{hie} geschach.\\ 
 & dô si ir \textbf{suns} niht \textbf{mêre} sach\\ 
 & - der \textbf{vert von ir}, wem ist deste baz? -,\\ 
20 & dô viel diu vrouwe valsches laz\\ 
 & ûf die erde, al dâ si jâmer sneit,\\ 
 & sô daz \textit{e}in sterben niht vermeit.\\ 
 & ir vil getriuwelîcher tôt\\ 
 & der vrouwen w\textit{e}rt die hellenôt.\\ 
25 & ôwol \textbf{ir}, daz si muoter wart!\\ 
 & sus vuor die lônes bernde vart\\ 
 & ein wurzel der güete\\ 
 & \textbf{und} ein stam der \textit{di}emüete.\\ 
 & owê, daz wir nû niht hân\\ 
30 & ir sippe \textbf{unz} an den \textbf{einliften} spân!\\ 
\end{tabular}
\scriptsize
\line(1,0){75} \newline
U V W T \newline
\line(1,0){75} \newline
\textbf{3} \textit{Majuskel} T  \textbf{11} \textit{Majuskel} T  \textbf{13} \textit{Initiale} U V W T  \textbf{20} \textit{Majuskel} T  \textbf{24} \textit{Majuskel} T  \textbf{29} \textit{Majuskel} T  \newline
\line(1,0){75} \newline
\textbf{1} dir] \textit{om.} W \textbf{2} kiusche ist] reine ist W ist kvsce T \textbf{3} wizzen] oͮch wússen V (T) \textbf{4} daz] \textit{om.} W T  $\cdot$ küene] stoltze kúnig W stolze kvͤne T  $\cdot$ Lehelin] loͤhelin V lehelein W \textbf{5} dînen vürsten] Deinem vatter W \textbf{7} Wâleis] Waleise U Walleis V Waleisc T  $\cdot$ Norgals] norgalß W \textbf{8} ein dîn vürste] diner vursten einer T  $\cdot$ Turkentals] turckentals W \textbf{10} dîn] Dem W \textbf{11} ruochet ez] vnd ruͦcht es W \textbf{13} Eines] Des T  $\cdot$ morgens] morgen U \textbf{14} vil] gar W \textbf{15} Artuse] artus V \textbf{16} \textit{Die Verse 128.16-17 fehlen} T   $\cdot$ Herzeloyde] Herzeleide U herzelaude V hertzeloyde W  $\cdot$ in kust] in kvste V kusten W \textbf{17} geschach] beschach W \textbf{18} sin mvͦter in niemer me gesach T  $\cdot$ ir suns] irn sun W  $\cdot$ sach] ensach V \textbf{19} der vert von ir] er vert hin T  $\cdot$ deste] nun dester W \textbf{22} ein] in U sv́ V (W) \textbf{24} wert] wirt U  $\cdot$ die hellenôt] der hellen not W \textbf{25} ôwol ir] O wol V O wol sy W wol ir T  $\cdot$ si] si ie V (W) T \textbf{26} die] div T \textbf{28} und ein stam] Ein stain W  $\cdot$ diemüete] gemuͦte U \textbf{29} owê] We V  $\cdot$ hân] enhan W \textbf{30} unz] bit U  $\cdot$ den] ir T \newline
\end{minipage}
\end{table}
\end{document}
