\documentclass[8pt,a4paper,notitlepage]{article}
\usepackage{fullpage}
\usepackage{ulem}
\usepackage{xltxtra}
\usepackage{datetime}
\renewcommand{\dateseparator}{.}
\dmyyyydate
\usepackage{fancyhdr}
\usepackage{ifthen}
\pagestyle{fancy}
\fancyhf{}
\renewcommand{\headrulewidth}{0pt}
\fancyfoot[L]{\ifthenelse{\value{page}=1}{\today, \currenttime{} Uhr}{}}
\begin{document}
\begin{table}[ht]
\begin{minipage}[t]{0.5\linewidth}
\small
\begin{center}*D
\end{center}
\begin{tabular}{rl}
\textbf{533} & \begin{large}L\end{large}ât nâher gên, hêr minnen druc.\\ 
 & ir tuot der \textbf{vreude} \textbf{al}solhen zuc,\\ 
 & daz sich dürkelt vreuden stat\\ 
 & unt bant sich der \textbf{riwen} pfat.\\ 
5 & sus \textbf{breitet} sich der riwen slâ.\\ 
 & \textbf{gienge} ir reise anderswâ\\ 
 & danne in des herzen hôhen muot,\\ 
 & daz d\textit{iu}hte mich gein vreuden guot.\\ 
 & Ist minne unvuoge balt,\\ 
10 & dar zuo dunket si mich zalt;\\ 
 & oder giht sis ûf ir kintheit,\\ 
 & \textbf{swem} si vüeget herzeleit?\\ 
 & unvuoge gan ich baz ir jugent,\\ 
 & denne \textbf{daz} si \textbf{ir} alter bræche tugent.\\ 
15 & Vil dinges ist von ir geschehen.\\ 
 & wederhalp sol ich des jehen?\\ 
 & wil si mit jungen ræten\\ 
 & ir alten site unstæten,\\ 
 & sô wirt si schiere an \textbf{prîse} laz.\\ 
20 & man sol \textbf{sis} \textbf{underscheiden} baz.\\ 
 & Lûter minne ich prîse\\ 
 & unt alle, die sint wîse,\\ 
 & ez sî wîp oder man,\\ 
 & von den ich\textbf{s} ganze volge hân.\\ 
25 & swâ liep gein liebe erhüebe\\ 
 & lûter âne trüebe,\\ 
 & \textbf{denne} \textbf{wederz} des verdrüzze,\\ 
 & \textbf{daz} minne ir herze slüzze\\ 
 & mit minne, \textbf{von} der wanc ie vlôch -\\ 
30 & diu minne ist ob den andern hôch.\\ 
\end{tabular}
\scriptsize
\line(1,0){75} \newline
D Fr7 Fr31 \newline
\line(1,0){75} \newline
\textbf{1} \textit{Initiale} D Fr7  \textbf{9} \textit{Majuskel} D  \textbf{15} \textit{Majuskel} D  \textbf{21} \textit{Majuskel} D  \newline
\line(1,0){75} \newline
\textbf{1} minnen] minne Fr7 \textbf{4} riwen] riͮwe Fr31 \textbf{6} gienge] gienc Fr7 Gein Fr31 \textbf{8} diuhte] dvhte D (Fr7) Fr31 \textbf{9} unvuoge] ir vngefuge Fr7 ir vnvuͦge Fr31 \textbf{11} sis] si ez Fr31 \textbf{12} swem] Dem Fr31  $\cdot$ herzeleit] hercen lait Fr7 \textbf{13} unvuoge] Vngfuͤge Fr7  $\cdot$ ir] der Fr31 \textbf{14} :::enne si dem alter bræche s::: Fr31  $\cdot$ bræche] breche Fr7 \textbf{17} mit jungen ræten] mir ivngen raiten Fr7 \textbf{20} Wan sol sie ez bewisen baz Fr31 \textbf{27} Der enweders :::z verdrvzze Fr31 \textbf{29} mit] Min Fr31  $\cdot$ wanc] wanke Fr31 \textbf{30} ist] \textit{om.} Fr31  $\cdot$ hôch] zoch Fr31 \newline
\end{minipage}
\hspace{0.5cm}
\begin{minipage}[t]{0.5\linewidth}
\small
\begin{center}*m
\end{center}
\begin{tabular}{rl}
 & \begin{large}L\end{large}ât nâher gân, hêr minnen druc.\\ 
 & ir tuot der  \textbf{al}solichen zuc,\\ 
 & daz sich dürke\textit{l}t vröuden stat\\ 
 & und ba\textit{n}t sich der \textbf{riuwen} pfat.\\ 
5 & sus \textbf{bereitet} sich der riuwe slâ.\\ 
 & \textbf{giengen} ir reise anderswâ\\ 
 & dan in des herzen hôhe\textit{n} muot,\\ 
 & daz d\textit{iu}hte mich gegen vröuden guot.\\ 
 & ist minne \textbf{ir} ungevüege balt,\\ 
10 & dar zuo dunket si mich zuo alt;\\ 
 & oder giht si es ûf ir kintheit,\\ 
 & \textbf{wem} si vüeget herzeleit?\\ 
 & ungevüege gan ich baz ir jugent,\\ 
 & dan \textbf{daz} si alter bræch \textbf{ir} tugent.\\ 
15 & vil dinges ist von ir geschehen.\\ 
 & wederhalp sol ich des jehen?\\ 
 & wil si mit jungen ræten\\ 
 & ir alten sit unstæten,\\ 
 & sô wirt s\textit{i} schier an \textbf{prîse} laz.\\ 
20 & man sol \textbf{sus} \textbf{underscheiden} baz:\\ 
 & lûter minne ich prîse\\ 
 & und alle, die sint wîse,\\ 
 & ez sî wîp oder man,\\ 
 & von den ich ganze volge hân.\\ 
25 & wâ liep gegen liebe erhüebe\\ 
 & lûter âne trüebe,\\ 
 & \textbf{der} \textbf{enwederz} des verdrüzze,\\ 
 & \textbf{daz} minne ir herz slüzze\\ 
 & mit minnen, \textbf{von} der wanc ie vlôch -\\ 
30 & diu minne ist ob den andern hôch.\\ 
\end{tabular}
\scriptsize
\line(1,0){75} \newline
m n o \newline
\line(1,0){75} \newline
\textbf{1} \textit{Initiale} m   $\cdot$ \textit{Capitulumzeichen} n  \newline
\line(1,0){75} \newline
\textbf{1} hêr minnen druc] her mynen truͦg n [herminen]: hereminen truͦg o \textbf{2} tuot der] truͦg o  $\cdot$ alsolichen] also sollichen n \textbf{3} dürkelt] durcket m \textbf{4} bant] batt m \textbf{5} riuwe] ruwen n (o) \textbf{6} reise] reiser o \textbf{7} hôhen] [hohr]: hoher m \textbf{8} diuhte] duhtte m (n) (o)  $\cdot$ vröuden] freuide o \textbf{9} ir ungevüege] [*]: vngefuge o \textbf{10} si] \textit{om.} o \textbf{13} ungevüege] Vnfuͯge n Jnfúge o \textbf{14} dan] [Das]: Dan o  $\cdot$ alter] alte o \textbf{16} des] das n dis o \textbf{18} sit] sint o \textbf{19} si] so m \textbf{23} sî] sigent n \textbf{27} verdrüzze] vertros o \textbf{28} slüzze] slos o \newline
\end{minipage}
\end{table}
\newpage
\begin{table}[ht]
\begin{minipage}[t]{0.5\linewidth}
\small
\begin{center}*G
\end{center}
\begin{tabular}{rl}
 & \begin{large}L\end{large}ât nâher gên, hêr minnen druc.\\ 
 & ir tuot der \textbf{vröuden} \textbf{al}solhen zuc,\\ 
 & daz sich dürkelt \textbf{der} vröuden stat\\ 
 & unt bant sich der \textbf{triuwen} pfat.\\ 
5 & sus \textbf{breitet} sich der riuwen slâ.\\ 
 & \textbf{g\textit{i}en\textit{g}e} ir reise anderswâ\\ 
 & danne in des herzen hôhen muot,\\ 
 & daz d\textit{iu}hte mich gein vröuden guot.\\ 
 & ist minne \textbf{ir} ungevüege balt,\\ 
10 & dar zuo dunket si mich ze alt;\\ 
 & ode giht sis ûf ir kintheit,\\ 
 & \textbf{swem} si vüeget herzeleit?\\ 
 & ungevüege gan ich baz ir jugent,\\ 
 & danne \textbf{daz} si \textbf{ir} alter bræche tugent.\\ 
15 & vi\textit{l d}inges ist von ir geschehen.\\ 
 & wederhalp sol ich de\textit{s j}ehen?\\ 
 & wil si mit jungen ræten\\ 
 & ir alten site unstæten,\\ 
 & sô wirt si schiere an \textbf{brîse} laz.\\ 
20 & man sol \textbf{sis} \textbf{underscheiden} baz.\\ 
 & lûter minne ich prîse\\ 
 & unde alle, die sint wîse,\\ 
 & e\textit{z} sî wîp ode man,\\ 
 & von den ich\textbf{s} ganze volge hân.\\ 
25 & swâ lieb g\textit{ei}n liebe erhüebe\\ 
 & lûter âne trüebe,\\ 
 & \textbf{der} \textbf{enwederz} des verdrüzze,\\ 
 & \textbf{daz} minne ir herze slüzze\\ 
 & mit minnen, \textbf{von} der wanc i\textit{e} vlôc\textit{h} -\\ 
30 & diu minne ist obe den andern hôch.\\ 
\end{tabular}
\scriptsize
\line(1,0){75} \newline
G I L M Z Fr19 \newline
\line(1,0){75} \newline
\textbf{1} \textit{Initiale} G I L Z  \textbf{15} \textit{Initiale} I  \newline
\line(1,0){75} \newline
\textbf{2} ir] Je M  $\cdot$ vröuden] freude I (M) (Z) [*]: mẏnne  L  $\cdot$ alsolhen] solhen I \textbf{3} der] \textit{om.} L Z uwir M \textbf{4} bant] beneget I  $\cdot$ triuwen] ruͯwen L (M) (Z) \textbf{5} breitet] bereit L  $\cdot$ riuwen] triwen I  $\cdot$ slâ] [stat]: sla G \textbf{6} gienge] gene G \textbf{8} diuhte] duhte G (I) (M) \textbf{9} minne ir] ir mynne M  $\cdot$ ungevüege] vnfuge M (Z) \textbf{10} si] \textit{om.} M \textbf{11} giht] git L M \textbf{12} swem] Wem L (M)  $\cdot$ herzeleit] leit I \textbf{13} ungevüege] Vnfuͯge L (M) (Z) \textbf{14} ir] in I im L  $\cdot$ tugent] ir tuͯgent L \textbf{15} vil dinges] vil uil dinges G \textbf{16} des jehen] des nu iehen G Sprechin M \textbf{17} ræten] rechen M \textbf{18} ir alten site] Jr alte sýte L Jrn aldin seten M \textbf{23} ez] Es G \textbf{24} von] An M  $\cdot$ den] dem I des M  $\cdot$ ichs] ich L Fr19 \textbf{25} swâ] Wo L (M)  $\cdot$ gein] gan G \textbf{26} âne] vnde nicht M \textbf{27} \textit{Versfolge 533.28-27} Fr19   $\cdot$ enwederz] den dwedez I deweders L keyner M  $\cdot$ des] der M Fr19 \textbf{28} ir] in Z  $\cdot$ herze slüzze] herzen shuzze I hertze fluͯsze L \textbf{29} mit] Min Z  $\cdot$ minnen] mynne M  $\cdot$ der wanc] den wanken M  $\cdot$ ie vlôch] ih floc G \textbf{30} minne] \textit{om.} I  $\cdot$ obe] uff M  $\cdot$ hôch] vil hoch Fr19 \newline
\end{minipage}
\hspace{0.5cm}
\begin{minipage}[t]{0.5\linewidth}
\small
\begin{center}*T
\end{center}
\begin{tabular}{rl}
 & Lât nâher gên, hêr minnen druc.\\ 
 & ir tuot der \textbf{minne} sölhen zuc,\\ 
 & daz sich dürkelt vröuden stat\\ 
 & unde bant sich der \textbf{riuwen} pfat.\\ 
5 & sus \textbf{breitet} sich der riuwen slâ.\\ 
 & \textbf{gienge} ir reise anderswâ\\ 
 & danne in des herzen hôhen muot,\\ 
 & daz d\textit{iu}hte mich gegen vröuden guot.\\ 
10 & \hspace*{-.7em}\big| dar zuo dunket si mich ze alt,\\ 
 & \hspace*{-.7em}\big| ist minne \textbf{ir} unvuoge balt;\\ 
 & oder giht si\textit{s} ûf ir kintheit,\\ 
 & \textbf{dem} si vüeget herzeleit?\\ 
 & unvuoge gan ich baz ir jugent,\\ 
 & danne si \textbf{dem} alter bræche \textbf{ir} tugent.\\ 
15 & vil dinges ist von ir geschehen.\\ 
 & wederhalp sol ich des jehen?\\ 
 & wil si mit jungen ræten\\ 
 & ir alte site unstæten,\\ 
 & sô wirt si schiere an \textbf{witzen} laz.\\ 
20 & man sol \textbf{sis} \textbf{underwîsen} baz.\\ 
 & Lûter minne ich prîse\\ 
 & unde alle, die sint wîse,\\ 
 & ez sî wîp oder man,\\ 
 & von den ich\textbf{s} ganze volge hân.\\ 
25 & swâ liep gegen liebe erhüebe\\ 
 & lûter âne trüebe,\\ 
 & \textbf{dâ} \textbf{deweder\textit{z}} des verdrüzze,\\ 
 & \textbf{ob} minne ir herze slüzze\\ 
 & mit minne, \textbf{die} der wanc ie vlôch -\\ 
30 & d\textit{iu} minne ist ob den andern hôch.\\ 
\end{tabular}
\scriptsize
\line(1,0){75} \newline
T U V W O Q R Fr40 \newline
\line(1,0){75} \newline
\textbf{1} \textit{Majuskel} T  \textbf{7} \textit{Initiale} O Fr40  \textbf{21} \textit{Majuskel} T  \newline
\line(1,0){75} \newline
\textbf{1} hêr minnen druc] her minnen trug V wer minne truͦg R \textbf{2} ir] Jn R  $\cdot$ der] die R  $\cdot$ minne] minnen U [*]: minne V frevden O (Q) vreude Fr40  $\cdot$ sölhen] soliche U  $\cdot$ zuc] fuͦg R \textbf{3} sich dürkelt] sich durkel U sich [tvnkelt]: tvrkelt V enget sich der O sy tunkelt R  $\cdot$ vröuden] der vroͤiden V eúwer W \textbf{4} bant] meret O bind R  $\cdot$ riuwen] [*rv́wen]: rv́wen V froͯden R \textbf{5} sus] Als Q  $\cdot$ breitet] bereit Q (R) (Fr40)  $\cdot$ riuwen] rúwe R \textbf{6} ir] [*]: uwer V uwer R \textbf{7} danne] ÷anne O  $\cdot$ des] das W \textbf{8} diuhte] dvhte T (U) V (W) (O) (Q) (R) \textbf{10} \textit{Versfolge 533.9-10} W O Q R Fr40  \textbf{9} unvuoge] vngefvͤge V (O) \textbf{11} giht] git U  $\cdot$ sis] siz T \textbf{12} vüeget] wegt O \textbf{13} unvuoge] Vngevuge U (O)  $\cdot$ ich] ichs R  $\cdot$ ir] der O \textbf{14} si] das sy W (Q) (R) (Fr40)  $\cdot$ bræche] breche T O (Fr40) neme W brechte Q  $\cdot$ ir] \textit{om.} W Q R Fr40 \textbf{15} \textit{Die Verse 533.15-20 fehlen} O   $\cdot$ geschehen] beschechen R \textbf{16} wederhalp] Beiderhalp U Widerhaben W twederhalp Fr40  $\cdot$ sol] so Q  $\cdot$ ich des] ich dez V ichs R \textbf{17} mit] mit irn W  $\cdot$ jungen ræten] iungem [raten]: ræten Fr40 \textbf{18} alte] alten U V (Q) R Fr40  $\cdot$ site] siten U (W) Q Fr40 sind R \textbf{19} wirt] wurd R  $\cdot$ witzen] preise W Q Fr40 froͯden R \textbf{20} sis] es Q (R) irz Fr40  $\cdot$ underwîsen] vnderscheiden U V (W) Q R (Fr40)  $\cdot$ baz] daz U \textbf{22} die] die do W  $\cdot$ sint] sine O \textbf{24} den] [dem]: den V dem W O  $\cdot$ ichs] ich R \textbf{25} swâ] Wa U (W) R So Q  $\cdot$ liep] leib W (Q)  $\cdot$ liebe] leyp Q lib::: Fr40 \textbf{27} dâ dewederz] da de weders T Do sie beider U [Di:weders]: Do ieweders V Do deweders W Da twederz O Da entweders Q Das deweders R  $\cdot$ des] dez V \textit{om.} R  $\cdot$ verdrüzze] verdruͦzzen U \textbf{29} \textit{Versfolge 533.30-29} V   $\cdot$ die] ie O  $\cdot$ wanc] wane W  $\cdot$ ie] do O \textbf{30} diu] die T \newline
\end{minipage}
\end{table}
\end{document}
