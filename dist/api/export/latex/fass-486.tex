\documentclass[8pt,a4paper,notitlepage]{article}
\usepackage{fullpage}
\usepackage{ulem}
\usepackage{xltxtra}
\usepackage{datetime}
\renewcommand{\dateseparator}{.}
\dmyyyydate
\usepackage{fancyhdr}
\usepackage{ifthen}
\pagestyle{fancy}
\fancyhf{}
\renewcommand{\headrulewidth}{0pt}
\fancyfoot[L]{\ifthenelse{\value{page}=1}{\today, \currenttime{} Uhr}{}}
\begin{document}
\begin{table}[ht]
\begin{minipage}[t]{0.5\linewidth}
\small
\begin{center}*D
\end{center}
\begin{tabular}{rl}
\textbf{486} & \begin{large}D\end{large}ie zwêne gesellen niht verdrôz,\\ 
 & si giengen, dâ der brunne vlôz;\\ 
 & si wuoschen würze unt ir krût.\\ 
 & \textbf{ir} munt wart selten lachens lût.\\ 
5 & ieweder sîne hende\\ 
 & twuoc. an einem gebende\\ 
 & truoc Parzival \textbf{îwîn loup}\\ 
 & vür\textit{z} ors. \textbf{ûf} ir ramschoup\\ 
 & giengen si wider zuo de\textit{n k}oln.\\ 
10 & man dorfte in niht mêr spîse holn;\\ 
 & dâne was gesoten \textbf{noch} gebrâten\\ 
 & \textbf{unt} ir küchen unberâten.\\ 
 & Parzival mit sinne\\ 
 & durch \textbf{die} getriwe minne,\\ 
15 & \textbf{die er} \textbf{gein} sînem wirte truoc,\\ 
 & \textbf{in dûhte}, er hete baz genuoc,\\ 
 & \textbf{denne} \textbf{dô} sîn pflac Gurnemanz\\ 
 & \textbf{unt} dô \textbf{sô} \textbf{maneger} vrouwen \textbf{varwe} glanz\\ 
 & ze Munsalvæsche vür in gienc,\\ 
20 & dâ er wirtschaft vome Grâle enpfienc.\\ 
 & der wirt mit triwen wîse\\ 
 & sprach: "neve, disiu spîse\\ 
 & sol dir niht versmâhen.\\ 
 & dû\textbf{ne} \textbf{vündest} in allen gâhen\\ 
25 & deheinen wirt, der dir günde baz\\ 
 & guoter wirtschaft âne haz."\\ 
 & Parzival sprach: "hêrre,\\ 
 & \textbf{der} gotes gruoz mir verre,\\ 
 & ob mich ie baz gezæme,\\ 
30 & \textbf{swes} ich \textbf{von} wirte næme."\\ 
\end{tabular}
\scriptsize
\line(1,0){75} \newline
D \newline
\line(1,0){75} \newline
\textbf{1} \textit{Initiale} D  \newline
\line(1,0){75} \newline
\textbf{7} Parzival] Parcifal D \textbf{8} vürz] fvrs D \textbf{9} den koln] den ir choln D \textbf{13} Parzival] Parcifal D \textbf{19} Munsalvæsche] Mvnsælvæsche D \textbf{27} Parzival] Parcifal D \newline
\end{minipage}
\hspace{0.5cm}
\begin{minipage}[t]{0.5\linewidth}
\small
\begin{center}*m
\end{center}
\begin{tabular}{rl}
 & \begin{large}D\end{large}ie zwên gesellen niht verdrôz,\\ 
 & si giengen, d\textit{â} der brunne vlôz;\\ 
 & si wuosc\textit{h}en würz und ir krût.\\ 
 & \textbf{ir} munt wart selten lachens lût.\\ 
5 & ietwe\textit{de}r sîn hende\\ 
 & \dag truoc\dag . an einem gebende\\ 
 & truoc Parcifal \textbf{îwîn loup}\\ 
 & vür daz ros. \textbf{vür} ir ramschoup\\ 
 & giengen si wider zuo den koln.\\ 
10 & man dorft in niht mê spîse holn;\\ 
 & dân was gesoten \textbf{und} gebrâten.\\ 
 & ir küchen \textbf{was} unb\textit{e}râten.\\ 
 & \hspace*{-.7em}\big| durch \textbf{die} getriuwe minne,\\ 
 & \hspace*{-.7em}\big| \textbf{die} Parcifal mit sinne\\ 
15 & \textbf{zuo} sînem wirte truoc,\\ 
 & \textbf{dûht in}, er het baz genuoc,\\ 
 & \textbf{wan} \textbf{dô} sîn pflac Gurnemanz\\ 
 & \textbf{und} dô \textbf{sô} \textbf{maniger} vrouwen \textbf{varwe} glanz\\ 
 & zuo Muntsalvasche vür in gienc,\\ 
20 & dô er wirtschaft von dem Grâl enpfienc.\\ 
 & der wirt mit triuwe\textit{n} \textit{w}îse\\ 
 & sprach: "neve, disiu spîse\\ 
 & sol dir niht versmâhen.\\ 
 & dû \textbf{vindest} in allen gâhen\\ 
25 & keinen wirt, der dir g\textit{ünn}e \textit{b}az\\ 
 & guoter wirtschaft âne haz."\\ 
 & Parcifal sprach: "hêrre,\\ 
 & \textbf{der} gotes gruoz mir verre,\\ 
 & ob mich ie baz gezæme,\\ 
30 & \textbf{wes} ich \textbf{vom} wirte næme."\\ 
\end{tabular}
\scriptsize
\line(1,0){75} \newline
m n o \newline
\line(1,0){75} \newline
\textbf{1} \textit{Initiale} m   $\cdot$ \textit{Capitulumzeichen} n  \newline
\line(1,0){75} \newline
\textbf{2} dâ] do m n o \textbf{3} wuoschen] wuschsen m [wuche]: wuchsen o \textbf{5} ietweder] Jettwer m \textbf{9} den] dem o \textbf{10} dorft] bedorffte n \textbf{11} und] noch n o \textbf{12} unberâten] vnbratten m \textbf{15} sînem] [e*]: eyme o \textbf{16} dûht] Dúcht o \textbf{17} Gurnemanz] gurnemancz m gurnemantz n o \textbf{18} dô] da o  $\cdot$ maniger] manig n  $\cdot$ varwe] \textit{om.} n o \textbf{19} Muntsalvasche] [munt sal*]: munt saluasce m muntsaluasce n (o) \textbf{21} triuwen wîse] truwen sprach wise m truwe wise o \textbf{23} sol dir] Solt ir o \textbf{25} günne baz] gomie was m \textbf{27} Parcifal] Sparcifal o \newline
\end{minipage}
\end{table}
\newpage
\begin{table}[ht]
\begin{minipage}[t]{0.5\linewidth}
\small
\begin{center}*G
\end{center}
\begin{tabular}{rl}
 & \begin{large}D\end{large}ie zwêne gesellen niht verdrôz,\\ 
 & si giengen, dâ der brunne vlôz;\\ 
 & si wuoschen würz unde ir krût.\\ 
 & \textbf{ir} munt wart selten lachens lût.\\ 
5 & ietweder sîne hende\\ 
 & twuoc. an einem gebende\\ 
 & truoc Parzival \textbf{wînloup}\\ 
 & vürz ors. \textbf{ûf} ir ramschoup\\ 
 & giengen si wider zuo den kolen.\\ 
10 & man dorfte in ni\textit{ht} m\textit{ê} spîse holen;\\ 
 & dâne was gesoten \textbf{noch} gebrâten\\ 
 & \textbf{unt} ir küchen unberâten.\\ 
 & Parzival mit sinne\\ 
 & durch getriuwe minne,\\ 
15 & \textbf{die er} \textbf{gein} sînem wirte truoc,\\ 
 & \textbf{in dûhte}, er hete \textit{b}az genuoc,\\ 
 & \textbf{danne} \textbf{dô} sîn pflac Gurnomanz,\\ 
 & dô \textbf{sô} \textbf{maniger} vrouwen \textbf{varwe} glanz\\ 
 & ze Muntsalvatsche vür in gienc,\\ 
20 & dô er wirtschaft vonem Grâle enpfienc.\\ 
 & der wirt mit triuwen wîse\\ 
 & sprach: "neve, disiu spîse\\ 
 & sol dir niht versmâhen.\\ 
 & dû\textbf{ne} \textbf{vindest} in allen gâhen\\ 
25 & deheine\textit{n} wirt, der dir \textit{günde baz}\\ 
 & guoter wirtschaft âne haz."\\ 
 & Parzival sprach: "hêrre,\\ 
 & \textbf{der} gotes gruoz mir verre,\\ 
 & op mich \textit{i}e baz gezæme,\\ 
30 & \textbf{swaz} ich \textbf{von} wirte næme."\\ 
\end{tabular}
\scriptsize
\line(1,0){75} \newline
G I O L M Z \newline
\line(1,0){75} \newline
\textbf{1} \textit{Initiale} G I O L Z  \textbf{17} \textit{Initiale} I  \newline
\line(1,0){75} \newline
\textbf{1} Die] ÷îe O  $\cdot$ niht] \textit{om.} I \textbf{3} würz] wurzen I (O)  $\cdot$ ir] \textit{om.} I O L \textbf{4} ir] Sin O L \textbf{5} ietweder] Jr y wilchir M \textbf{6} twuoc] [Tewch]: Tewuch G Truc M  $\cdot$ einem] ein I sinem O L (M) \textbf{7} Parzival] parziual G parzifal I L M Barcifal O parcifal Z  $\cdot$ wînloup] ywin lavp O (L) (M) (Z) \textbf{8} vürz] vur sin I  $\cdot$ ûf] \textit{om.} I  $\cdot$ ramschoup] reym scoyp M \textbf{9} den] ir Z \textbf{10} niht mê] nimer G \textbf{11} dâne] Da O \textbf{12} küchen] chuchel I \textbf{13} Parzival] Parziual G parzifal I (L) (M) Parcifal O Z \textbf{14} getriuwe] der truwin M die getrewen Z \textbf{16} baz] [heiz]: haz G \textbf{17} dô] da M Z  $\cdot$ Gurnomanz] Gurnemanz I (O) (M) (Z) Gvrnomantz L \textbf{18} dô] vnde do O (L) (Z) Vnde da M  $\cdot$ maniger vrouwen varwe] manich frowen O manich frowe L maniger frowen Z \textbf{19} ze Muntsalvatsche] zemvntsaluatsche G zemuntsaluasche I Zemvntsalvatsche O Zv montsalvatsche Z  $\cdot$ vür in] vur in in I gein im L \textbf{24} vindest] fvndest O (L) (M) (Z)  $\cdot$ allen] al I \textbf{25} deheine wirt der dir baz gunde G \textbf{27} Parzival] Parziual G Parzifal I L M Parcifal O Z \textbf{28} der] \textit{om.} O L M  $\cdot$ mir] sý mir L \textbf{29} mich] mir I L  $\cdot$ ie] hie G ie spise I [ee]: ie O \textbf{30} swaz] swa I Swes O Wez L (M)  $\cdot$ ich] ich sie ie I \newline
\end{minipage}
\hspace{0.5cm}
\begin{minipage}[t]{0.5\linewidth}
\small
\begin{center}*T
\end{center}
\begin{tabular}{rl}
 & \begin{large}D\end{large}ie zwêne gesellen niht verdrôz,\\ 
 & si giengen, dâ der brunne vlôz;\\ 
 & si wuoschen würze unde ir krût.\\ 
 & \textbf{sîn} munt wart selten lachens lût.\\ 
5 & ietweder sîne hende\\ 
 & twuoc. an einem gebende\\ 
 & truoc Parcifal \textbf{îwîn loup}\\ 
 & vür daz ors. \textbf{ûf} ir ramschoup\\ 
 & giengen si wider zuo den koln.\\ 
10 & man dorftin niht mê spîse holn;\\ 
 & dâne was gesoten \textbf{noch} gebrâten\\ 
 & \textbf{unde} ir küchen unberâten.\\ 
 & Parcifal \textbf{sprach} mit sinne\\ 
 & durch \textbf{die} getriuwe minne,\\ 
15 & \textbf{dier} \textbf{gegen} sînem wirte truoc,\\ 
 & \textbf{in dûhte}, er hete baz genuoc,\\ 
 & \textbf{danne} \textbf{sô} sîn pflac Gurnemanz\\ 
 & \textbf{unde} dô \textbf{manec} vrouwe glanz\\ 
 & ze Munsalvasche vür in gienc,\\ 
20 & dôr wirtschaft vonme Grâle enpfienc.\\ 
 & Der wirt mit triuwen wîse\\ 
 & sprach: "neve, dis\textit{iu} spîse\\ 
 & sol dir niht versmâhen.\\ 
 & dû\textbf{ne} \textbf{vündest} in allen gâhen\\ 
25 & deheinen wirt, der dir günde baz\\ 
 & guoter wirtschaft âne haz."\\ 
 & Parcifal sprach: "hêrre,\\ 
 & gotes gruoz mir verre,\\ 
 & ob mich ie baz gezæme,\\ 
30 & \textbf{swes} ich \textbf{von} wirte næme."\\ 
\end{tabular}
\scriptsize
\line(1,0){75} \newline
T U V W Q R Fr40 \newline
\line(1,0){75} \newline
\textbf{1} \textit{Initiale} T R Fr40  \textbf{21} \textit{Initiale} W   $\cdot$ \textit{Majuskel} T  \newline
\line(1,0){75} \newline
\textbf{1} \textit{Die Verse 453.1-502.30 fehlen} U   $\cdot$ \textit{Versfolge 486.2-1} W  \textbf{2} dâ] do V W Q \textbf{3} würze] die wúrtzel W wurczen R  $\cdot$ ir krût] daz kraut W crawt Q jsaut Fr40 \textbf{4} sîn] [*]: Jr V  $\cdot$ selten] seldes Q  $\cdot$ lachens] lachen Q \textbf{5} sîne] seiner Fr40  $\cdot$ hende] hende twͦg V \textbf{6} twuoc] \textit{om.} V trug Fr40  $\cdot$ gebende] gebvnde trvͦg V \textbf{7} truoc] \textit{om.} V  $\cdot$ Parcifal] Parzifal V (Fr40) partzifal W Q parczifal R  $\cdot$ îwîn] ywidins V núwe R \textbf{8} ir] ein W  $\cdot$ ramschoup] rein schaub W \textbf{10} dorftin] bedorfftt in R \textbf{11} dâne was] [D*]: Da enwaz V Do enwas W (Q) (R) [do]: da was  Fr40  $\cdot$ gesoten] gosoten Fr40  $\cdot$ noch] [*]: noch V vnd R \textbf{12} küchen] keúschen W \textbf{13} Parcifal] Parzifal V (Fr40) Partzifal W Q Parczifal R  $\cdot$ sprach] \textit{om.} W Q R Fr40 \textbf{14} die] \textit{om.} W Q R Fr40 \textbf{17} sô] do W Q R Fr40  $\cdot$ Gurnemanz] gurnemantz W gurnomantz Q garnumancz R gvrnamanz Fr40 \textbf{18} manec] so manig V W (Q) (R) Fr40  $\cdot$ vrouwe] vrowen V \textbf{19} Munsalvasche] Mvnsalvasce T mvntsalvasche V montsaluatschs W muntsaluasche Q Munsaluashe R (Fr40) \textbf{20} dôr] da er Fr40  $\cdot$ vonme] vorme V (Fr40) \textbf{21} triuwen] treúwe W \textbf{22} disiu] dise T R die Q (Fr40) \textbf{23} sol dir] Sol dich V Soltu von mir R \textbf{24} dûne vündest] Du enuindest W Won du findest R  $\cdot$ allen] alle W  $\cdot$ gâhen] gaheen Q \textbf{25} deheinen] Dehin R \textbf{27} Parcifal] Parzifal V (Fr40) Partzifal W Q Parczifal R \textbf{28} verre] were R \textbf{30} swes] Wes W R Was Q  $\cdot$ von] vom Q \newline
\end{minipage}
\end{table}
\end{document}
