\documentclass[8pt,a4paper,notitlepage]{article}
\usepackage{fullpage}
\usepackage{ulem}
\usepackage{xltxtra}
\usepackage{datetime}
\renewcommand{\dateseparator}{.}
\dmyyyydate
\usepackage{fancyhdr}
\usepackage{ifthen}
\pagestyle{fancy}
\fancyhf{}
\renewcommand{\headrulewidth}{0pt}
\fancyfoot[L]{\ifthenelse{\value{page}=1}{\today, \currenttime{} Uhr}{}}
\begin{document}
\begin{table}[ht]
\begin{minipage}[t]{0.5\linewidth}
\small
\begin{center}*D
\end{center}
\begin{tabular}{rl}
\textbf{3} & sô daz si \textbf{niht} geriwe\\ 
 & ir kiusche und ir triwe.\\ 
 & Vor gote ich guoten wîben bite,\\ 
 & \textbf{daz} \textbf{in} \textbf{rehtiu} mâze \textbf{volge} mite.\\ 
5 & scham ist ein slôz ob allen siten.\\ 
 & ich \textbf{en}darf in \textbf{niht mêr} heiles biten.\\ 
 & \textbf{diu} valsche \textbf{erwirbet} valschen prîs.\\ 
 & wie stæte ist ein dünnez îs,\\ 
 & daz ougestheize sunnen hât?\\ 
10 & \textbf{ir} lop \textbf{vil} balde alsus zergât.\\ 
 & \textbf{manec} wîbes \textbf{schœne an lobe} ist breit.\\ 
 & ist \textbf{dâ} daz herze conterfeit,\\ 
 & \textbf{die} lob ich, als ich solde\\ 
 & \textbf{daz} safer \textbf{im} golde.\\ 
15 & \textbf{ich enhân} \textbf{daz} niht vür \textbf{lîhti\textit{u}} \textit{d}in\textit{c},\\ 
 & swer in den kranken messinc\\ 
 & verwürket \textbf{edeln} rubîn\\ 
 & und al die \textbf{âventiure} sîn.\\ 
 & \textbf{dem} glîche ich \textbf{rehte} wîbes muot.\\ 
20 & diu ir wîpheit rehte tuot,\\ 
 & \textbf{dâ}\textbf{ne} sol ich varwe prüeven \textit{n}iht\\ 
 & noch herzen dach, \textbf{daz} man sih\textit{t}.\\ 
 & ist si inrehalp \textbf{des} \textbf{brust} bew\textit{art},\\ 
 & sô ist \textbf{werder} prîs dâ niht v\textit{e}rsch\textit{art}.\\ 
25 & \begin{Large}S\end{Large}olt ich wîp un\textit{de man}\\ 
 & zerehte prüeven, als ich \textit{kan},\\ 
 & dâ vüere e\textit{i}n l\textit{an}g\textit{ez \textbf{mære} mite}.\\ 
 & nû h\textit{œr}et d\textit{ir}re âventiure site.\\ 
 & diu lât iuch wizzen beide\\ 
30 & von li\textit{e}be und von lei\textit{de}.\\ 
\end{tabular}
\scriptsize
\line(1,0){75} \newline
D \newline
\line(1,0){75} \newline
\textbf{3} \textit{Versal} D  \textbf{25} \textit{Großinitiale} D  \newline
\line(1,0){75} \newline
\textbf{1} geriwe] geriwe \textit{nachträglich korrigiert zu:} gerîwe D \textbf{2} triwe] triwe \textit{nachträglich korrigiert zu:} trîwe D \textbf{15} lîhtiu dinc] lihti::in: \textit{nachträglich korrigiert zu:} lihtiv dinc D \textbf{16} messinc] messinc \textit{nachträglich korrigiert zu:} meͦssinc D \textbf{17} verwürket] verwurchet \textit{nachträglich korrigiert zu:} verwürchet D \textbf{21} niht] :iht \textit{nachträglich korrigiert zu:} niht D \textbf{22} siht] sih: \textit{nachträglich korrigiert zu:} siht D \textbf{23} bewart] bew::: \textit{nachträglich korrigiert zu:} bewart D \textbf{24} verschart] v:rsch::: \textit{nachträglich korrigiert zu:} versch::: D \textbf{25} unde man] un::: D \textbf{26} kan] ::: D \textbf{27} da fv̂re e:n l::g::: D \textbf{28} hœret dirre] h::et d::re D \textbf{29} beide] beide \textit{nachträglich korrigiert zu:} heide D \textbf{30} liebe] libe D  $\cdot$ leide] lei:: \textit{nachträglich korrigiert zu:} leide D \newline
\end{minipage}
\hspace{0.5cm}
\begin{minipage}[t]{0.5\linewidth}
\small
\begin{center}*m
\end{center}
\begin{tabular}{rl}
 & sô daz \textit{si} \textbf{niht} geriuwe\\ 
 & ir kiusche und ir triuwe.\\ 
 & vor got ich guoten wîben bite,\\ 
 & \textbf{die} \textbf{in} \textbf{rehter} mâze \textbf{volgent} mite.\\ 
5 & scham ist ein slôz ob allen siten.\\ 
 & ich darf in \textbf{nimer} heiles biten.\\ 
 & \textbf{diu} valsche \textbf{erwirbet} valschen brîs.\\ 
 & wie stæte ist ein dü\textit{nn}e\textit{z} îs,\\ 
 & \dag diu angest herze\dag  sunne hât?\\ 
10 & \textbf{ir} lop \textbf{vil} balde alsus zergât,\\ 
 & \textbf{wenne} wîbes \textbf{schœne an lobe} ist breit.\\ 
 & ist \textbf{dâ} daz herze conterfeit,\\ 
 & \textbf{die} lob ich, alsô ich solde\\ 
 & \textbf{daz} safer \textbf{in} golde.\\ 
15 & \textbf{ine hân} \textbf{daz} niht vür \textbf{lîhter} din\textit{c},\\ 
 & wer in den kranken messinc\\ 
 & verwürket \textbf{edel} rubîn\\ 
 & und alle die \textbf{âventiure} sîn.\\ 
 & \textbf{den} glîch ich \textbf{rehten} wîbes muot.\\ 
20 & diu ir wîpheit rehte tuot,\\ 
 & \textbf{denne} sol ich varwe prüeven niht\\ 
 & noch \textbf{ir} herzen dach, \textbf{dô} man \textbf{dâ} siht.\\ 
 & ist s\textit{i} \textit{in}nerhalp \textbf{der} \textbf{brüste} bewart,\\ 
 & sô ist \textbf{worden} prîs d\textit{â} niht verschart.\\ 
25 & \multicolumn{1}{l}{ - - - }\\ 
 & \multicolumn{1}{l}{ - - - }\\ 
 & \multicolumn{1}{l}{ - - - }\\ 
 & \multicolumn{1}{l}{ - - - }\\ 
 & \multicolumn{1}{l}{ - - - }\\ 
30 & \multicolumn{1}{l}{ - - - }\\ 
\end{tabular}
\scriptsize
\line(1,0){75} \newline
m n o W \newline
\line(1,0){75} \newline
\newline
\line(1,0){75} \newline
\textbf{1} si niht] nit \textit{nachträglich korrigiert zu:} es sy nit m sie mit o  $\cdot$ geriuwe] getruwe n o \textbf{2} ir kiusche] Jie kúsche o  $\cdot$ triuwe] ruwe n \textbf{4} volgent] volge n \textbf{5} allen] allem n \textbf{6} nimer] mẏner o (W) \textbf{7} erwirbet] erwiebet o \textbf{8} stæte] stette \textit{nachträglich korrigiert zu:} vnstette m steter o  $\cdot$ dünnez] dumer \textit{nachträglich korrigiert zu: } dunnrer m dúnner n dornner o \textbf{9} diu] [De]: Dẏe n  $\cdot$ herze] hertzen W  $\cdot$ sunne] sinne n o W \textbf{10} alsus] \textit{om.} o suß W  $\cdot$ zergât] vergat n [vergar]: vergat o \textbf{11} an lobe ist breit] an lobe ist bereit n (o) ist ein lob brai W \textbf{12} dâ] do n o W \textbf{14} safer] saffier m n o saffir W  $\cdot$ in] in dem W \textbf{15} ine] Jch n o (W)  $\cdot$ lîhter] liehter o  $\cdot$ dinc] dinge m \textbf{18} die] \textit{om.} W  $\cdot$ sîn] sint o drin W \textbf{19} den] Die n o W  $\cdot$ rehten] rechter n (o) W  $\cdot$ muot] nuͯt o gút W \textbf{20} rehte] rehter o nit rechte W \textbf{21} prüeven] preisen W \textbf{22} ir] ÿres m (n) (o) (W)  $\cdot$ dach] [tag]: tach n  $\cdot$ dâ] \textit{om.} n o W \textbf{23} si innerhalp] sumnerthalp m summerhalp n (o) sunderhalb W  $\cdot$ brüste] prúfe W  $\cdot$ bewart] gewart o \textbf{24} dâ] do m n o \textit{om.} W  $\cdot$ verschart] furzart o verstart W \textbf{25} \textit{Die Verse 3.25-4.8 fehlen} m n o W  \newline
\end{minipage}
\end{table}
\newpage
\begin{table}[ht]
\begin{minipage}[t]{0.5\linewidth}
\small
\begin{center}*G
\end{center}
\begin{tabular}{rl}
 & sô daz si \textbf{iht} geriwe\\ 
 & ir kiusche und ir triwe.\\ 
 & vor gote ich guoten wîben bite,\\ 
 & \textbf{daz} \textbf{in} \textbf{rehtiu} mâze \textbf{volge} mite.\\ 
5 & scham ist ein slôz obe allen siten.\\ 
 & ich \textbf{en}darf in \textbf{niht mê} heiles biten.\\ 
 & \textbf{\begin{large}D\end{large}iu} valsche \textbf{wirbet} valschen brîs.\\ 
 & wie stæte ist ein dünnez îs,\\ 
 & daz ougestheize sunnen hât?\\ 
10 & \textbf{ir} lop \textbf{vil} balde alsus zergât.\\ 
 & \textbf{maniges} wîbes \textbf{lop an schœne} ist breit.\\ 
 & ist \textbf{dâ} daz herze conterfeit,\\ 
 & \textbf{die} lobe ich, als ich solde\\ 
 & \textbf{daz} safer \textbf{in dem} golde.\\ 
15 & \textbf{ouch hân ich} niht vür \textbf{ringiu} dinc,\\ 
 & swer in den kranken messinc\\ 
 & verwürket \textbf{edelen} rubîn\\ 
 & unde alle die \textbf{natûre} sîn.\\ 
 & \textbf{dem} gelîche ich \textbf{rehten} wîbes muot.\\ 
20 & diu ir wîpheit rehte tuot,\\ 
 & \textbf{dâ}\textbf{ne} sol ich varwe brüeven niht\\ 
 & noch \textbf{ir} herzen dach, \textbf{daz} man \textbf{dâ} siht.\\ 
 & ist si innerhalp \textbf{dâr} bewart,\\ 
 & sô ist \textbf{werder} brîs dâ niht verschart.\\ 
25 & solt ich \textbf{nû} wîp und man\\ 
 & ze rehte brüeven, als ich kan,\\ 
 & dâ vüere ein langez \textbf{mære} mite.\\ 
 & nû hœret dirre âventiure site.\\ 
 & diu lât iuch wizzen beide\\ 
30 & von liebe und \textbf{ouch} von leide.\\ 
\end{tabular}
\scriptsize
\line(1,0){75} \newline
G O L M Q Z Fr58 \newline
\line(1,0){75} \newline
\textbf{1} \textit{Initiale} O M Q  \textbf{7} \textit{Initiale} G  \textbf{25} \textit{Initiale} Z   $\cdot$ \textit{Capitulumzeichen} L  \newline
\line(1,0){75} \newline
\textbf{1} sô] ÷o O Do Q  $\cdot$ si] es Q \textbf{3} wîben] weybe Q \textbf{4} volge] volget L \textbf{5} scham] Schawm Q  $\cdot$ ein] \textit{om.} O  $\cdot$ obe] uff M \textbf{6} endarf] darf O (L) bedarff Q  $\cdot$ niht mê] mynner M \textbf{7} valsche] valschen Z  $\cdot$ wirbet] erwirbet O werbent Z \textbf{8} ist ein dünnez] vnd wie frvmez Z (Fr58) \textbf{9} daz] Der L M Z Fr58  $\cdot$ ougestheize] angst heysse Q  $\cdot$ sunnen] sunne O (Q) \textbf{10} zergât] irgat M vergat Q \textbf{11} maniges] Mannigk M  $\cdot$ wîbes] libes Q  $\cdot$ lop an schœne ist] schone ane lob ist O (M) (Z) schone vil lobez ist L schone ist an lobe Q scho::: Fr58  $\cdot$ breit] bereit L M \textbf{12} daz herze] des hertzen Z daz herczen Fr58 \textbf{13} lobe ich als ich] lobe als ich Q loben ich Z loben ich solde Fr58 \textbf{14} daz] Der M Den Q  $\cdot$ safer] sapheir O saffir L asfer Q saphir Z  $\cdot$ dem] \textit{om.} Z \textbf{15} ouch hân ich] Jch (en Q ) han dasz Q (Z)  $\cdot$ vür ringiu] vorlichte Q fvr wehe Z  $\cdot$ dinc] dien M \textbf{16} swer] Wer L Q  $\cdot$ in den] indem O \textbf{17} verwürket] Verwirffet L  $\cdot$ edelen] den edlen Z \textbf{18} alle] aller O  $\cdot$ natûre] av:::ver O aventuͯre L (M) (Q) (Z) \textbf{19} dem] Den M \textbf{20} ir] in L \textbf{21} An der warbe ich varbe niht L  $\cdot$ dâne] Da von O (M) Da Z  $\cdot$ varwe] farben Q \textbf{22} noch] [Noch]: Nach Z  $\cdot$ ir] \textit{om.} L irs Q  $\cdot$ dâ] >da< G do O Q \textit{om.} M \textbf{23} si] \textit{om.} L  $\cdot$ dâr] der G der prvst O (L) (M) (Q) (Z) \textbf{24} sô] Schone M  $\cdot$ werder] tiwer O dewidder M  $\cdot$ dâ] do Q  $\cdot$ verschart] [schart]: verschart G gespart L \textbf{25} solt] Sal M Wold Z \textbf{27} vüere] vuͦr O (M) (Z) \textbf{28} nû] Avch O \textbf{29} diu] Der M \textbf{30} ouch] \textit{om.} O L M Q Z \newline
\end{minipage}
\hspace{0.5cm}
\begin{minipage}[t]{0.5\linewidth}
\small
\begin{center}*T
\end{center}
\begin{tabular}{rl}
 & sô daz si\textit{\textbf{s}} \textbf{iht} geriuwe,\\ 
 & ir kiusche und ir triuwe.\\ 
 & Vor gote ich guoten wîben b\textit{i}te,\\ 
 & \textbf{sô} \textbf{daz} \textbf{ir} \textbf{rehte} mâze \textbf{volge} mite.\\ 
5 & Scham ist ein slôz ob allen siten.\\ 
 & i\textbf{ne} darf in \textbf{niht mêr} heiles biten.\\ 
 & \textbf{Der} valsche \textbf{erwirbet} valschen prîs.\\ 
 & wie stæte ist ein dünnez îs,\\ 
 & daz ougestheize sunne hât?\\ 
10 & \textbf{sîn} lop balde alsus zergât.\\ 
 & \textbf{\begin{large}M\end{large}aneges} wîbes \textbf{schœne an lobe} ist breit.\\ 
 & ist \textbf{aber} daz herze kunterfeit,\\ 
 & \textbf{daz} lob ich, als ich solde\\ 
 & \textbf{der} safer \textbf{in dem} golde.\\ 
15 & \textbf{Ouch enhab ich}\textbf{z} niht vür \textbf{ein geringez} dinc,\\ 
 & swer in den kranken messinc\\ 
 & verwirket \textbf{edele} rubîn\\ 
 & und aldi\textit{e} \textbf{âventiure} sîn.\\ 
 & \textbf{dem} glîch ich \textbf{rehten} wîbes muot.\\ 
20 & diu ir wîpheit rehte tuot,\\ 
 & \textbf{dâ}\textbf{ne} sol ich varw\textit{e} prüeven niht\\ 
 & noch \textbf{ir} herzen dach, \textbf{daz} man \textbf{dâ} siht.\\ 
 & ist si innertalp \textbf{der} \textbf{brust} bewart,\\ 
 & sô ist \textbf{werder} prîs dâ niht verschart.\\ 
25 & \begin{large}S\end{large}olt ich \textbf{nû} wîp und man\\ 
 & ze rehte prüeven, als ich kan,\\ 
 & dâ vüere ein langez \textbf{ende} mite.\\ 
 & Nû hœret dirre âventiure site.\\ 
 & diu lât iuch wizzen beide\\ 
30 & von liebe und \textbf{ouch} von leide.\\ 
\end{tabular}
\scriptsize
\line(1,0){75} \newline
T U V Fr32 \newline
\line(1,0){75} \newline
\textbf{3} \textit{Majuskel} T  \textbf{5} \textit{Versal} Fr32   $\cdot$ \textit{Majuskel} T  \textbf{7} \textit{Majuskel} T  \textbf{11} \textit{Initiale} T U Fr32  \textbf{15} \textit{Majuskel} T  \textbf{25} \textit{Initiale} T U V Fr32  \textbf{28} \textit{Majuskel} T  \newline
\line(1,0){75} \newline
\textbf{1} sis] siz T si Fr32 \textbf{3} bite] b:te T \textbf{4} sô daz ir] Daz ir U Daz in V (Fr32) \textbf{5} ein] \textit{om.} Fr32  $\cdot$ allen] [alle*]: alle U \textbf{6} ine] Man U (V)  $\cdot$ in niht mêr] ir niemer Fr32 \textbf{7} Der] [D*]: Dv́ V div Fr32 \textbf{8} ein dünnez] [ei*nnes]: ein dv́mnes V \textbf{9} daz] Der U Daz in V  $\cdot$ ougestheize] augestheize \textit{nachträglich korrigiert zu:} augesthenne U oͮgeste heisse V \textbf{10} sîn] Jr U V (Fr32)  $\cdot$ balde] vil balde V  $\cdot$ alsus] also V \textbf{11} Maneges wîbes] Manec wip Fr32  $\cdot$ breit] bereit U \textbf{12} aber] do V \textit{om.} Fr32  $\cdot$ kunterfeit] ein kuͦnterfei U \textbf{13} daz] Die U V (Fr32) \textbf{14} der safer] der safir T (U) (Fr32) Glas sapfir V \textbf{15} [*]: Jch enhan daz nv́t fúr lihte ding V  $\cdot$ enhab ichz] enhat Fr32  $\cdot$ ein geringez] ringiv Fr32 \textbf{16} swer] Wer U \textbf{17} edele] edelen V \textbf{18} aldie] aldiv T \textbf{19} rehten] rehte V \textbf{21} dâne sol] Dane sol sol U Do ensol V  $\cdot$ varwe] varw: T \textbf{22} herzen] [*]: herze V  $\cdot$ dâ] \textit{om.} U V \textbf{23} innertalp] innertahip U  $\cdot$ bewart] wol bewart Fr32 \textbf{24} werder] ir werder V  $\cdot$ dâ] do V \textbf{25} nû] \textit{om.} V \textbf{26} pruͤfen als ich zerehte kan V \textbf{27} dâ vüere] Do fuͦrte ich V  $\cdot$ ende] mere V \textbf{29} iuch] îv T (Fr32) \textbf{30} ouch] \textit{om.} Fr32 \newline
\end{minipage}
\end{table}
\end{document}
