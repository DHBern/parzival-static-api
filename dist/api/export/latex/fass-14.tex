\documentclass[8pt,a4paper,notitlepage]{article}
\usepackage{fullpage}
\usepackage{ulem}
\usepackage{xltxtra}
\usepackage{datetime}
\renewcommand{\dateseparator}{.}
\dmyyyydate
\usepackage{fancyhdr}
\usepackage{ifthen}
\pagestyle{fancy}
\fancyhf{}
\renewcommand{\headrulewidth}{0pt}
\fancyfoot[L]{\ifthenelse{\value{page}=1}{\today, \currenttime{} Uhr}{}}
\begin{document}
\begin{table}[ht]
\begin{minipage}[t]{0.5\linewidth}
\small
\begin{center}*D
\end{center}
\begin{tabular}{rl}
\textbf{14} & der bâruc in vür sünde\\ 
 & gît wandels urkünde.\\ 
 & \begin{large}Z\end{large}wêne \textbf{bruoder} von Babilon,\\ 
 & \textbf{Pompeius} und Ipomidon,\\ 
5 & den nam der bâruc Ninnive.\\ 
 & daz was \textbf{al ir} vorderen ê.\\ 
 & \textbf{si} tâten wer mit \textbf{krefte} schîn.\\ 
 & dar kom der \textbf{junge} Anschevin.\\ 
 & \textbf{dem} wart der bâruc vil holt.\\ 
10 & \textbf{jâ} nam nâch dienste al dâ den solt\\ 
 & Gahmuret, der werde man.\\ 
 & \textbf{nû} \textbf{erloubet im}, daz er \textbf{müeze} hân\\ 
 & ander wâpen, denne \textbf{im} Gandin\\ 
 & dâ vor gap, der vater sîn.\\ 
15 & der hêrre pflac mit gernden siten\\ 
 & ûf sîne \textbf{covertiure} gesniten\\ 
 & enker lieht hermîn.\\ 
 & dâ nâch muose ouch daz ander sîn,\\ 
 & \textbf{ûf sîme} schilte und an der wât.\\ 
20 & noch grüener denne ein smarât\\ 
 & was geprüevet sîn gereite gar\\ 
 & und nâch dem achmardî \textbf{var}.\\ 
 & daz \textbf{ist} ein sîdîn lachen.\\ 
 & dar \textbf{zuo} hiez er \textbf{im} machen\\ 
25 & \textbf{wâpenroc} und kursît\\ 
 & - \textbf{ez ist} bezzer denne \textbf{der} samît -,\\ 
 & hermîn anker drûf genæt,\\ 
 & \textbf{guldîniu} seil \textbf{dran} gedræt.\\ 
 & \textbf{sîne ankere} \textbf{heten} niht bekort\\ 
30 & ganzes landes noch landes ort.\\ 
\end{tabular}
\scriptsize
\line(1,0){75} \newline
D \newline
\line(1,0){75} \newline
\textbf{3} \textit{Initiale} D  \newline
\line(1,0){75} \newline
\textbf{5} Ninnive] ninivé D \textbf{8} Anschevin] Anscivin D \textbf{11} Gahmuret] Gahmvͦreth D \textbf{13} Gandin] Gaudîn D \newline
\end{minipage}
\hspace{0.5cm}
\begin{minipage}[t]{0.5\linewidth}
\small
\begin{center}*m
\end{center}
\begin{tabular}{rl}
 & der bâruc in vür sünde\\ 
 & gît wandels urkünde.\\ 
 & zwêne \textbf{gebruoder} von Babilon,\\ 
 & \textbf{Pompeius} und Ypomedon,\\ 
5 & den nam der bâruc Ni\textit{niv}e.\\ 
 & daz was \textbf{aller} vorderen ê.\\ 
 & \textbf{die} tâten were mit \textbf{kreften} schîn.\\ 
 & dar kam der A\textit{n}schevin.\\ 
 & \textbf{dem} wart der bâruc vil holt.\\ 
10 & \textbf{der} nam nâch dienste aldâ den solt,\\ 
 & Gahmuret. der werde man\\ 
 & \textbf{ime erloubt}, daz er \textbf{mües} hân\\ 
 & ander wâpen, dann \textbf{in} Gand\textit{in}\\ 
 & dâ vor gap, der vater sîn.\\ 
15 & der hêrre pflac mit ger\textit{n}den siten\\ 
 & ûf sîne \textbf{âventiure} gesniten\\ 
 & anker lieht he\textit{r}mîn.\\ 
 & dar nâch muose ouch daz ander sîn,\\ 
 & \textbf{ûfme} schilte und an der w\textit{â}t.\\ 
20 & noch grüener denne ein smar\textit{â}t\\ 
 & was gebrüefet sîn ger\textit{ei}te gar\\ 
 & und nâch dem achmardî \textbf{gevar}.\\ 
 & daz \textbf{ist} ein sîdîn lachen.\\ 
 & dâr \textbf{ûz} hiez er \textbf{im} machen\\ 
25 & \textbf{wâpenröcke} und kursît\\ 
 & - \textbf{ez ist} bezzer denne \textbf{daz} samît -,\\ 
 & hermî\textit{n} anker drûf genæt,\\ 
 & \textbf{guldîn\textit{iu}} seil \textbf{dâr an} gedræt.\\ 
 & \textbf{sîn anker} \textbf{hete} niht bek\textit{o}rt\\ 
30 & ganzes landes noch landes \textit{o}rt.\\ 
\end{tabular}
\scriptsize
\line(1,0){75} \newline
m n o \newline
\line(1,0){75} \newline
\newline
\line(1,0){75} \newline
\textbf{1} sünde] suͯnden o \textbf{2} gît] Sit o \textbf{3} gebruoder] bruͯder n \textbf{4} Pompeius] [Pomprius]: pompeius m Ponpeius n Ponpeiusz o  $\cdot$ Ypomedon] ÿpomedon m ipomidon n pomidon o \textbf{5} bâruc Ninive] baruͦck nÿme m barugknynne n baruͯg knẏ nne o \textbf{8} der] der junge n (o)  $\cdot$ Anschevin] ausceuin \textit{nachträglich korrigiert zu:} ansceuin m auscenin n ansceim o \textbf{11} Gahmuret] Gamiret n Gamuͯret o  $\cdot$ werde] wede o \textbf{12} daz] nuͦ das n (o)  $\cdot$ mües] muͦsz n \textbf{13} dann in] dannen m n o  $\cdot$ Gandin] gandan fin \textit{nachträglich korrigiert zu:} gandin m gandin sin n gandi sin o \textbf{14} der] dem n \textbf{15} gernden] geruͯden m gerendem n \textbf{16} âventiure] vffenture \textit{nachträglich korrigiert zu:} offenture m \textbf{17} hermîn] herre mÿn m (n) (o) \textbf{19} ûfme] Vff me \textit{nachträglich korrigiert zu:} Vff dem m  $\cdot$ und] vnd so n  $\cdot$ wât] want m n o \textbf{20} smarât] smarent m smarant n (o) \textbf{21} gereite] geriette m gerete n o \textbf{22} achmardî] aht mardi o \textbf{24} hiez] hiesse n \textbf{25} wâpenröcke] Woppen recke n Woppen rock o \textbf{26} denne] wenne n \textbf{27} hermîn] Her mim m Her min n (o)  $\cdot$ drûf] druft m  $\cdot$ genæt] [genent]: genaͤt m genant o \textbf{28} guldîniu] Guldinen m  $\cdot$ gedræt] getrant o \textbf{29} bekort] bekert m \textbf{30} ort] ert m \newline
\end{minipage}
\end{table}
\newpage
\begin{table}[ht]
\begin{minipage}[t]{0.5\linewidth}
\small
\begin{center}*G
\end{center}
\begin{tabular}{rl}
 & der bâruc in vür sünde\\ 
 & gît wandeles urkünde.\\ 
 & zwêne \textbf{bruoder} von Babilon,\\ 
 & \textbf{Ponpeirus} und Ipomidon,\\ 
5 & den nam der bâruc Ninve.\\ 
 & daz was \textbf{al ir} vorderen ê.\\ 
 & \textbf{si} tâten wer mit \textbf{kreften} schîn.\\ 
 & dar kom der \textbf{junge} Antschevin.\\ 
 & \textbf{ime} wart der bâruc vil holt.\\ 
10 & \textbf{er} nam nâch dienste al dâ den solt,\\ 
 & \begin{large}G\end{large}ahmuret, der werde man.\\ 
 & \textbf{erloubet im}, daz er \textbf{m\textit{üe}ze} hân\\ 
 & ander wâpen, dane \textbf{im} Gandin\\ 
 & dâ vor gap, der vater sîn.\\ 
15 & der hêrre pflac mit gernden siten\\ 
 & ûf sîne \textbf{covertiure} gesniten\\ 
 & anker lieht hermîn.\\ 
 & dar nâch muose ouch daz ander sîn,\\ 
 & \textbf{ûf dem} schilt und an der wât.\\ 
20 & noch grüener danne ein smarât\\ 
 & was geprüevet sîn gereite gar,\\ 
 & nâch dem achmardî \textbf{gevar}.\\ 
 & daz \textbf{ist} ein sîdîn lachen.\\ 
 & dâr \textbf{ûz} hiez er machen\\ 
25 & \textbf{wâpenroc} und kursît\\ 
 & - \textbf{deist} bezzer danne samît -,\\ 
 & härmîn anker drûf genæt,\\ 
 & \textbf{guldîniu} seil \textbf{drûf} gedræt.\\ 
 & \textbf{sîn anker} \textbf{heten} niht bekort\\ 
30 & ganzes landes noch landes ort.\\ 
\end{tabular}
\scriptsize
\line(1,0){75} \newline
G O L M Q R W Z Fr29 Fr32 Fr36 \newline
\line(1,0){75} \newline
\textbf{1} \textit{Initiale} O M  \textbf{5} \textit{Versal} Fr32  \textbf{11} \textit{Initiale} G L R Z Fr32 Fr36  \textbf{15} \textit{Initiale} W  \newline
\line(1,0){75} \newline
\textbf{1} der] ÷er O  $\cdot$ in] \textit{om.} W im Fr36  $\cdot$ sünde] ir sunde Q \textbf{2} gît] Nicht M mit Fr36 \textbf{3} Babilon] babylon O :::ba::: Fr29 Babẏlon Fr32 \textbf{4} Ponpeirus] Pompeius O L R W (Z) (Fr32) Pomperys M Pomperrus Q ponpeirvs Fr36  $\cdot$ Ipomidon] Jpomidon O ýpomidon L ichpomidon M Fr36 jhpomidon Q (Fr32) yhpomidon R ypomidon W Z \textbf{5} den] \textit{om.} W  $\cdot$ nam] man M  $\cdot$ bâruc] brauk \textit{nachträglich korrigiert zu:} baruk Q  $\cdot$ Ninve] ninive O Z (Fr32) Fr36 Nýnýve L mynne M mi nne Q in Nyniue R :::nive Fr29 \textbf{6} al] aller L W Z alles Q (R) alle M Fr36  $\cdot$ vorderen] orden L \textbf{7} tâten] ta O \textbf{8} junge] ivgne Fr36  $\cdot$ Antschevin] anschevin G O (R) Anshewin L anschevyn M ansheűin Q antscheuin W anshevin Z Fr32 :::shevin Fr29 \textbf{9} ime] Dem Q Z  $\cdot$ der] die M  $\cdot$ bâruc] braugk \textit{nachträglich korrigiert zu:} baruͯgk Q \textbf{10} er] Ja R (Fr32)  $\cdot$ nam] [*an]: man Fr32  $\cdot$ nâch] an M mit R  $\cdot$ al dâ] da O L M Z do W \textbf{11} Gahmuret] Gamuret O M W Z Gahmvret L Fr36 Gamúret Q Gamvͦret Fr32  $\cdot$ der werde] der er welte Q \textbf{12} erloubet] Nv erlovbt O (L) (Q) (R) (W) Z (Fr32)  $\cdot$ müeze] muͯst L  $\cdot$ hân] [lan]: han O \textbf{13} dane im] im den Jm R  $\cdot$ Gandin] gandîn O gaudin W gadin Fr32 \textbf{14} dâ vor gap] Do im gab vor R  $\cdot$ sîn] sin sin Fr32 \textbf{15} hêrre] helt Fr32  $\cdot$ pflac] hot Q  $\cdot$ gernden] guͦten O gerúden Q \textbf{16} ûf] vfen Fr32  $\cdot$ sîne] siner O seinen Q  $\cdot$ covertiure] awetewren Q dekin R \textbf{17} lieht] lýcht L (M) (Q) \textbf{18} muose] musz Q (R) (W) \textbf{19} dem] den O M sinē Q sinen R sinem Z Fr32  $\cdot$ an] \textit{om.} Z \textbf{20} danne] wanne M  $\cdot$ ein] \textit{om.} W  $\cdot$ smarât] schmark R \textbf{21} sîn] sine O \textbf{22} nâch] Vnd nach O L (M) Q R W Z (Fr32)  $\cdot$ gevar] var W \textbf{23} sîdîn lachen] sydelachen R sidin lilachen Z \textbf{24} ûz] vber O vff L vmb Q  $\cdot$ machen] im machen O (M) Q R W Z Fr32 \textbf{26} danne] wanne M  $\cdot$ samît] der samit O L M Z din samit Fr32 \textbf{27} härmîn] Armenie Q  $\cdot$ drûf] dran Z \textbf{28} guldîniu] guldine Fr32  $\cdot$ drûf] dar an L (Q) (R) W Z Fr32 da M  $\cdot$ gedræt] [getret]: getreit R \textbf{29} sîn] Ein Q  $\cdot$ heten] \textit{om.} L hette W \textbf{30} ganzes landes] Gantze lant L  $\cdot$ ort] hort Q \newline
\end{minipage}
\hspace{0.5cm}
\begin{minipage}[t]{0.5\linewidth}
\small
\begin{center}*T
\end{center}
\begin{tabular}{rl}
 & der bâruc in vür \textbf{ir} sünde\\ 
 & gît wandels urkünde.\\ 
 & Zwêne \textbf{bruodere} von Babylon,\\ 
 & \textbf{Pompeius} und Ihpomidon,\\ 
5 & den nam der bâruc Ninive.\\ 
 & daz was \textbf{al ir} vorderen ê.\\ 
 & \textbf{si} tâten wer mit \textbf{kreften} schîn.\\ 
 & dar kam der \textbf{junge} Anschevin.\\ 
 & \textbf{im} wart der bâruc vil holt.\\ 
10 & \textbf{jâ} nam nâch dienste aldâ den solt\\ 
 & Gahmuret, der werde man.\\ 
 & \textbf{nû} \textbf{erloubet im}, daz er \textbf{müeze} hân\\ 
 & ander wâpen, danne \textbf{im} Gandin\\ 
 & dâ vor gap, der vater sîn.\\ 
15 & Der hêrre pflac mit gernden siten\\ 
 & ûf sîne \textbf{covertiure} gesniten\\ 
 & anker liehte härmîn.\\ 
 & dar nâch muose ouch daz ander sîn,\\ 
 & \textbf{ûf dem} schilte und an der wât.\\ 
20 & noch grüener danne ein smarât\\ 
 & was geprüevet sîn gereite gar\\ 
 & und nâch dem achmardî \textbf{gevar}.\\ 
 & daz \textbf{was} ein sîdîn lachen.\\ 
 & dâr \textbf{ûz} hiez er \textbf{im} machen\\ 
25 & \textbf{wâpenroc} und kursît\\ 
 & - \textbf{daz ist} bezzer danne samît -,\\ 
 & härmîn anker drûf genæt\\ 
 & \textbf{und} \textbf{sîdîniu} seil \textbf{dâr an} gedræt.\\ 
 & \textbf{sîn anker} \textbf{heten} niht bekort\\ 
30 & ganzes landes noch landes ort.\\ 
\end{tabular}
\scriptsize
\line(1,0){75} \newline
T U V \newline
\line(1,0){75} \newline
\textbf{3} \textit{Majuskel} T  \textbf{15} \textit{Majuskel} T  \newline
\line(1,0){75} \newline
\textbf{1} in vür ir] [in]: er vuͦr ir U \textbf{3} Babylon] Babylôn T Babilon V \textbf{4} Pompeius] Pompeis U Pomperius V  $\cdot$ Ihpomidon] Jchpomidôn T Jpomidon U ẏpomidon V \textbf{5} nam] man U  $\cdot$ Ninive] ninivê T Niniue U V \textbf{6} al] aller V  $\cdot$ vorderen] vorder U \textbf{7} si] Sis U \textbf{8} Anschevin] Anscheuin V \textbf{10} jâ] \textit{om.} U do V \textbf{11} Gahmuret] gahmvret T Gahmuͦret U Gamuret V \textbf{13} Gandin] guͦndin U \textbf{16} gesniten] waren gesniden U (V) \textbf{18} muose] mvese T muͤst V \textbf{21} sîn gereite] [*]: sin gereite V \textbf{23} was] ist U V \textbf{27} härmîn] Hermine U V \textbf{28} und sîdîniu] Guldine U (V) \textbf{29} heten] hete U \newline
\end{minipage}
\end{table}
\end{document}
