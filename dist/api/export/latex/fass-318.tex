\documentclass[8pt,a4paper,notitlepage]{article}
\usepackage{fullpage}
\usepackage{ulem}
\usepackage{xltxtra}
\usepackage{datetime}
\renewcommand{\dateseparator}{.}
\dmyyyydate
\usepackage{fancyhdr}
\usepackage{ifthen}
\pagestyle{fancy}
\fancyhf{}
\renewcommand{\headrulewidth}{0pt}
\fancyfoot[L]{\ifthenelse{\value{page}=1}{\today, \currenttime{} Uhr}{}}
\begin{document}
\begin{table}[ht]
\begin{minipage}[t]{0.5\linewidth}
\small
\begin{center}*D
\end{center}
\begin{tabular}{rl}
\textbf{318} & \begin{large}N\end{large}û ist iwer prîs ze valsche komen.\\ 
 & owê, daz ie wart vernomen\\ 
 & von mir, daz Herzeloyden barn\\ 
 & an \textbf{prîse} \textbf{hât sus} missevarn!"\\ 
5 & Cundrie was selbe \textbf{sorgens} pfant.\\ 
 & al weinende si die hende want,\\ 
 & daz manec zaher den andern sluoc.\\ 
 & grôz jâmer si ûz ir ougen truoc.\\ 
 & die magt lêrt ir triwe\\ 
10 & \textbf{wol klagen ir} herzen riwe.\\ 
 & Wider vür den \textbf{wirt} si kêrte,\\ 
 & ir mære si dâ \textbf{gemêrte}.\\ 
 & Si sprach: "ist hie kein ritter wert,\\ 
 & des ellen \textbf{prîses} \textbf{hât} gegert\\ 
15 & unt dar zuo hôher minne?\\ 
 & ich weiz vier küneginne\\ 
 & unt vier hundert juncvrouwen,\\ 
 & die man gerne m\textit{ö}hte schouwen.\\ 
 & \textbf{ze} Schastel Marvale die sint.\\ 
20 & al âventiure ist ein wint,\\ 
 & wan die man dâ bezalen mac,\\ 
 & \textbf{hôher} minne \textbf{wert} \textbf{bejac}.\\ 
 & al hab ich der reise pîn,\\ 
 & ich wil \textbf{doch} hînte drûffe sîn."\\ 
25 & Diu magt trûrec, niht gemeit,\\ 
 & ân urloup \textbf{dannen} reit.\\ 
 & al weinende si dicke wider sach.\\ 
 & nû hœret, wie si ze jungest sprach:\\ 
 & "\textbf{Ay}, Munsalvæsche, jâmers zil,\\ 
30 & \textbf{wê}, daz dich niemen trœsten wil!"\\ 
\end{tabular}
\scriptsize
\line(1,0){75} \newline
D \newline
\line(1,0){75} \newline
\textbf{1} \textit{Initiale} D  \textbf{11} \textit{Majuskel} D  \textbf{13} \textit{Majuskel} D  \textbf{25} \textit{Majuskel} D  \textbf{29} \textit{Majuskel} D  \newline
\line(1,0){75} \newline
\textbf{5} Cundrie] Cvndrîe D \textbf{18} möhte] mohte D \textbf{19} Schastel] Scastel D \textbf{29} Munsalvæsche] Mvnsalvæsce D \newline
\end{minipage}
\hspace{0.5cm}
\begin{minipage}[t]{0.5\linewidth}
\small
\begin{center}*m
\end{center}
\begin{tabular}{rl}
 & nû ist iuwer prîs ze valsche komen.\\ 
 & ouwê, daz ie wart vernomen\\ 
 & von mir, daz Herczeloiden barn\\ 
 & an \textbf{prîse} \textbf{hât sus} missevarn!"\\ 
5 & Condrie was selbe \textbf{sorgen} pfant.\\ 
 & alweinende si die hende want,\\ 
 & daz manic zaher den andern sluoc.\\ 
 & grôz jâmer si ûz ir ougen truoc.\\ 
 & die maget lêrte ir triuwe\\ 
10 & \textbf{wol \textit{klag}en ir} \textit{herz}en riuwe.\\ 
 & wider vür den \textbf{wirt} si kêrte,\\ 
 & ir mære si dâ \textbf{gemêrte}.\\ 
 & si sprach: "ist hie kein ritter wert,\\ 
 & des ellen \textbf{prîses} \textbf{hât} gegert\\ 
15 & und dar zuo hôher minne?\\ 
 & ich weiz vier küniginne\\ 
 & und vier hundert juncvrouwen,\\ 
 & die man gerne m\textit{ö}hte schouwen.\\ 
 & \textbf{ze} Schah\textit{t}el Ma\textit{rve}ile die sint.\\ 
20 & alliu âventiure ist ein wint,\\ 
 & wan die man d\textit{â} bezal\textit{en} mac,\\ 
 & \textbf{hôher} minne \textbf{wert} \textbf{gejac}.\\ 
 & al habe ich der reise pîn,\\ 
 & ich wil \textbf{noch} hînaht drûffe sîn."\\ 
25 & diu magt trûric, niht gemeit,\\ 
 & âne urloup \textbf{vonme ringe} reit.\\ 
 & alweinende si dicke wider sach.\\ 
 & nû hœret, wie s\textit{i} \textit{z}e jungest sprach:\\ 
 & "\textbf{ai}, M\textit{u}n\textit{t}salvasche, jâmers zil,\\ 
30 & \textbf{wê}, daz dich niemen trœsten wil!"\\ 
\end{tabular}
\scriptsize
\line(1,0){75} \newline
m n o \newline
\line(1,0){75} \newline
\newline
\line(1,0){75} \newline
\textbf{3} Herczeloiden] hertzeleiden n herczeleide o \textbf{4} hât] hette n \textbf{5} Condrie] Condri n o  $\cdot$ selbe] selbes n  $\cdot$ sorgen pfant] sorgepfant o \textbf{10} klagen ir herzen] herczen ir clagen m clagen irn herczen o  $\cdot$ riuwe] truwe n \textbf{12} dâ] do n o \textbf{14} gegert] begert n o \textbf{18} möhte] moht m (o) \textbf{19} Schahtel Marveile] schahsel [man]: mauniaile m schathal maruail n schalten maruale o \textbf{21} dâ] do m n o  $\cdot$ bezalen] bezalt m \textbf{22} wert] wart o  $\cdot$ gejac] beẏag n (o) \textbf{23} reise] reisen o \textbf{24} hînaht] hinnach o \textbf{28} si ze jungest] sich [j]: ze jungest m sú n (o) \textbf{29} Muntsalvasche] Mrinsaluasce m monscaluasce n [mont]: monsceluasce o \textbf{30} wê] Wie o \newline
\end{minipage}
\end{table}
\newpage
\begin{table}[ht]
\begin{minipage}[t]{0.5\linewidth}
\small
\begin{center}*G
\end{center}
\begin{tabular}{rl}
 & nû ist iuwer brîs ze valsche komen.\\ 
 & owê, daz ie wart vernomen\\ 
 & von mir, daz Herzeloide barn\\ 
 & an \textbf{triuwen} \textbf{hât sus} missevarn!"\\ 
5 & \multicolumn{1}{l}{ - - - }\\ 
 & \multicolumn{1}{l}{ - - - }\\ 
 & \multicolumn{1}{l}{ - - - }\\ 
 & \multicolumn{1}{l}{ - - - }\\ 
 & die maget lêrte ir triuwe\\ 
10 & \textbf{al klagende} herzeriuwe.\\ 
 & wider vür den \textbf{künic} si kêrte,\\ 
 & ir mære si dâ \textbf{gemêrte}.\\ 
 & si sprach: "ist hie dehein rîter wert,\\ 
 & des ellen \textbf{prîses} \textbf{habe} gegert\\ 
15 & unt dar zuo hôher minne?\\ 
 & ich weiz vier küniginne\\ 
 & unt vier hundert juncvrouwen,\\ 
 & die man gerne m\textit{ö}hte schouwen.\\ 
 & \textbf{ûf} Tschastel Marveile \textit{d}i\textit{e} sint.\\ 
20 & al âventiure ist ein wint,\\ 
 & wan die man dâ bezalen mac,\\ 
 & \textbf{werder} minne \textbf{hôch} \textbf{bejac}.\\ 
 & al habe ich \textbf{dar} der reise pîn,\\ 
 & ich wil \textbf{noch} hînt dar ûffe sîn."\\ 
25 & diu maget trûric, niht gemeit,\\ 
 & âne urloup \textbf{vome ringe} reit.\\ 
 & al weinende si dicke wider sach.\\ 
 & nû hœret, wie si ze jungest sprach:\\ 
 & "\textbf{â}, Muntsalvatsche, jâmers zil,\\ 
30 & \textbf{wê}, daz dich niemen trœsten wil!"\\ 
\end{tabular}
\scriptsize
\line(1,0){75} \newline
G I O L M Q R Z Fr22 Fr39 Fr40 \newline
\line(1,0){75} \newline
\textbf{9} \textit{Initiale} O L  \textbf{11} \textit{Capitulumzeichen} R  \textbf{19} \textit{Initiale} Z  \textbf{23} \textit{Initiale} I  \textbf{25} \textit{Initiale} Q R Fr39 Fr40  \newline
\line(1,0){75} \newline
\textbf{1} \textit{Die Verse 316.7-318.8 fehlen} L  \textbf{3} Von myr das hercze loydenbarn M  $\cdot$ daz] \textit{om.} Q  $\cdot$ Herzeloide] herzeloyde G herzenlauden I herzelavden O herzenlouden Q herczelauden R hertzelovde Z herzelauden Fr39 herzelouden Fr40 \textbf{4} an triuwen] Nu preise Q An pris R (Fr39) (Fr40)  $\cdot$ hât sus] sunst hot Q (R) (Fr39) (Fr40) \textbf{5} \textit{Die Verse 318.5-8 fehlen} G I O L M Q R Z Fr39 Fr40  \textbf{9} die] ÷ie O  $\cdot$ lêrte] lert I (O) (L) (Q) (R) (Z) (Fr39) (Fr40)  $\cdot$ ir triuwe] mit trewen Q mit trewe Fr40 \textbf{10} al klagende] Wol clagen ir Z  $\cdot$ herzeriuwe] herzen riwe I (Z) hertzen ruͯwen L \textbf{11} künic] wirt Z  $\cdot$ kêrte] kerten Q \textbf{12} dâ] do L Q R Fr39  $\cdot$ gemêrte] gemerten Q \textbf{13} ist hie] hie ist L ist Q  $\cdot$ wert] so wert I hy wert Q \textbf{14} des ellen] der ellens I (Fr39) Des allis M Des eren Q  $\cdot$ prîses] prise R  $\cdot$ habe] had Fr40 \textbf{18} möhte] mohte G I O (L) (M) (Q) Z Fr39 Fr40 \textbf{19} Tschastel marveile] tschaterMarfeile G shatermalveile I tschahtel mær val O tshahtel Marveil L schachtel marveile M tschahtel marueile Q schahtel marveile R tschahtel marveile Z Fr39 tschahteil marveile Fr40  $\cdot$ die] si G \textbf{20} al] Vil L \textbf{21} die man] man die Q  $\cdot$ dâ] do Q Fr39 \textbf{22} werder] Hoher Z  $\cdot$ minne] \textit{om.} O  $\cdot$ hôch] wert Z \textbf{23} Swie ich dar habe der raise pin I  $\cdot$ al] \textit{om.} O An L Allein Q R (Fr39) (Fr40)  $\cdot$ dar] \textit{om.} O L Q R Fr39 Fr40 \textbf{24} noch] doch O L M Q Z Fr22 Fr39 Fr40  $\cdot$ hînt] hute M \textbf{27} weinende] wende O  $\cdot$ dicke wider] hin wider O hinder sich R \textbf{28} ze jungest] zum iungsten Q \textbf{29} â] Auch I Ey O (L) M Z (Fr22) (Fr39) (Fr40) Ein Q Bẏ R  $\cdot$ Muntsalvatsche] muntshaluasch I mvntschalvatschi O Muntsal vatsche M muntsalfache Q Munschaluashe R montsalvatsche Z munsaluashe Fr39 munsalvashe Fr40 \textbf{30} wê] owe I  $\cdot$ dich niemen] man dich nicht M \newline
\end{minipage}
\hspace{0.5cm}
\begin{minipage}[t]{0.5\linewidth}
\small
\begin{center}*T
\end{center}
\begin{tabular}{rl}
 & nû ist iuwer prîs ze valsche komen.\\ 
 & ouwê, daz ie wart vernomen\\ 
 & von mir, daz Herzeloyden barn\\ 
 & an \textbf{triuwen} \textbf{sus hât} missevarn!"\\ 
5 & \multicolumn{1}{l}{ - - - }\\ 
 & \multicolumn{1}{l}{ - - - }\\ 
 & \multicolumn{1}{l}{ - - - }\\ 
 & \multicolumn{1}{l}{ - - - }\\ 
 & \begin{large}D\end{large}ie maget lêrte ir triuwe\\ 
10 & \textbf{al klagende} herzeriuwe.\\ 
 & wider vür den \textbf{künec} si kêrte,\\ 
 & ir mære si dâ \textbf{mêrte}.\\ 
 & si sprach: "ist hie dehein rîter wert,\\ 
 & des ellen \textbf{prîs} \textbf{habe} gert\\ 
15 & unde dar zuo hôher minne?\\ 
 & ich weiz vier küneginne\\ 
 & unde vier hundert juncvrouwen,\\ 
 & die man gerne m\textit{ö}hte schouwen.\\ 
 & \textbf{ûf} Tschahtel Marvele die sint.\\ 
20 & all\textit{iu} âventiure ist ein wint,\\ 
 & wan die man dâ bezalen mac,\\ 
 & \textbf{werder} minne \textbf{hôch} \textbf{bejac}.\\ 
 & al hab \textit{i}ch \textbf{dar} der reise pîn,\\ 
 & ich wil \textbf{noch} hînt drûffe sîn."\\ 
25 & Diu maget trûric, niht gemeit,\\ 
 & âne urloup \textbf{vonme ringe} reit.\\ 
 & al weinende si dicke wider sach.\\ 
 & nû hœret, wie si ze jungest sprach:\\ 
 & "\textbf{ei}, Munsalvasche, jâmers zil,\\ 
30 & \textbf{ouwê}, daz dich nieman trœsten wil!"\\ 
\end{tabular}
\scriptsize
\line(1,0){75} \newline
T U V W \newline
\line(1,0){75} \newline
\textbf{9} \textit{Initiale} T U W  \textbf{25} \textit{Majuskel} T  \textbf{29} \textit{Initiale} V  \newline
\line(1,0){75} \newline
\textbf{3} Herzeloyden] herzelauden V hertzeloyden W \textbf{4} triuwen] prise V  $\cdot$ hât] \textit{om.} U \textbf{5} \textit{Die Verse 318.5-8 fehlen} T U W   $\cdot$ Kvndrie was selbe sorgen phant V \textbf{6} Alweinende sv́ die hende want V \textbf{7} Das manig zaher den andern slvͦg V \textbf{8} Gros iomer sú vz ir oͮgen truͦg V \textbf{10} al] Alle U  $\cdot$ herzeriuwe] herzen rvwe V \textbf{12} mære] klage W  $\cdot$ dâ] do V fúrbas W  $\cdot$ mêrte] gemerte U V werte W \textbf{13} ist hie] hie ist W \textbf{14} des ellen] Der ellens W  $\cdot$ habe] hat W \textbf{15} hôher] hohe W \textbf{16} vier] wie U \textbf{17} juncvrouwen] frawen W \textbf{18} möhte] mohte T (U) \textbf{19} Tschahtel Marvele] tschatel [mar*]: marvele T Schatelmarveile U (V) kastelle de marfeile W  $\cdot$ die] sy W \textbf{20} alliu] alle T (U) (V) (W)  $\cdot$ ist] sint U V (W) \textbf{21} man dâ] [*]: men do V \textbf{22} Die úberhebet ir aller schlag W \textbf{23} al] Alle U [A*]: Allein V Aldar W  $\cdot$ hab ich] [habich]: habch T hant ich U  $\cdot$ dar] do U \textit{om.} V W \textbf{24} hînt] \textit{om.} U \textbf{25} trûric] traurens W \textbf{27} dicke wider] vmb W \textbf{28} ze] zuͦm W \textbf{29} ei] Ein U Hey W  $\cdot$ Munsalvasche] mvnsalvasce T muͦntsalvatsche U mvntsalvasche V montsaluatz W \textbf{30} ouwê] Wie U \textit{om.} V We W  $\cdot$ trœsten] rechen W \newline
\end{minipage}
\end{table}
\end{document}
