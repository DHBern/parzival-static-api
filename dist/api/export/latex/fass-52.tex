\documentclass[8pt,a4paper,notitlepage]{article}
\usepackage{fullpage}
\usepackage{ulem}
\usepackage{xltxtra}
\usepackage{datetime}
\renewcommand{\dateseparator}{.}
\dmyyyydate
\usepackage{fancyhdr}
\usepackage{ifthen}
\pagestyle{fancy}
\fancyhf{}
\renewcommand{\headrulewidth}{0pt}
\fancyfoot[L]{\ifthenelse{\value{page}=1}{\today, \currenttime{} Uhr}{}}
\begin{document}
\begin{table}[ht]
\begin{minipage}[t]{0.5\linewidth}
\small
\begin{center}*D
\end{center}
\begin{tabular}{rl}
\textbf{52} & Gahmurete, ir hêrren.\\ 
 & die selben wâren die êrren.\\ 
 & \multicolumn{1}{l}{ - - - }\\ 
 & \multicolumn{1}{l}{ - - - }\\ 
5 & \multicolumn{1}{l}{ - - - }\\ 
 & \multicolumn{1}{l}{ - - - }\\ 
 & \multicolumn{1}{l}{ - - - }\\ 
 & \multicolumn{1}{l}{ - - - }\\ 
 & \textbf{Dô} hete Prothizilas,\\ 
10 & der von arde ein vürste was,\\ 
 & \textbf{lâzen} ein herzentuom.\\ 
 & daz lêch er dem, der manegen ruom\\ 
 & mit sîner hant bejagete.\\ 
 & \textbf{gein} strîte er nie verzagete:\\ 
15 & Lachfilirost schahtelakunt\\ 
 & nam ez \textbf{mit} vanen sân zestunt.\\ 
 & \begin{large}V\end{large}on Azagouc die vürsten hêr\\ 
 & nâmen den Schotten Hiuteger\\ 
 & unt Gaschieren den \textbf{Orman}.\\ 
20 & si giengen vür ir hêrren \textbf{sân}.\\ 
 & der liez si ledic \textbf{umb ir} bete.\\ 
 & des \textbf{danketen} si \textbf{dô} Gahmurete.\\ 
 & Hiuteger de\textit{n} Schotten\\ 
 & si bâten sunder spotten:\\ 
25 & "lât mîme hêrren \textbf{daz} gezelt\\ 
 & hie \textbf{umbe} âventiure gelt.\\ 
 & ez zucket uns Isenhartes leben,\\ 
 & daz Vridebrande wart gegeben\\ 
 & diu \textbf{zierde} unsers landes.\\ 
30 & sîn vreude \textbf{diu} stuont pfandes.\\ 
\end{tabular}
\scriptsize
\line(1,0){75} \newline
D Fr9 \newline
\line(1,0){75} \newline
\textbf{9} \textit{Majuskel} D  \textbf{17} \textit{Initiale} D Fr9  \newline
\line(1,0){75} \newline
\textbf{1} Gahmurete] Gahmvrete D Gamvrete Fr9 \textbf{3} \textit{Die Verse 52.3-8 fehlen} D Fr9  \textbf{9} Prothizilas] Protyzilas D protẏzila: Fr9 \textbf{11} lâzen] Gelazen Fr9 \textbf{12} lêch] liez Fr9 \textbf{15} Lachfilirost] Lahfilirost D Lach fille rost Fr9 \textbf{17} Azagouc] Azagoͮch D azagou: Fr9 \textbf{18} Schotten] scotten D  $\cdot$ Hiuteger] Hivteger D \textbf{19} Gaschieren] gatzẏeren Fr9  $\cdot$ Oriman] Orman D \textbf{21} ledic] liedich Fr9 \textbf{22} dô] \textit{om.} Fr9  $\cdot$ Gahmurete] Gahmvrete D gamvrete Fr9 \textbf{23} Hiuteger] Hvteger D Hutegern Fr9  $\cdot$ den] der D  $\cdot$ Schotten] Scotten D (Fr9) \textbf{25} lât] Laz Fr9 \textbf{26} âventiure] aventivren Fr9 \textbf{27} zucket] zuget Fr9  $\cdot$ Isenhartes] Jsenhartes D ysenhartes Fr9 \textbf{28} Vridebrande] Fridebrande D \textbf{29} diu] Her was Fr9 \newline
\end{minipage}
\hspace{0.5cm}
\begin{minipage}[t]{0.5\linewidth}
\small
\begin{center}*m
\end{center}
\begin{tabular}{rl}
 & Gahmuret, i\textit{r} hêrren.\\ 
 & die selben wâren die êrren.\\ 
 & \multicolumn{1}{l}{ - - - }\\ 
 & \multicolumn{1}{l}{ - - - }\\ 
5 & \multicolumn{1}{l}{ - - - }\\ 
 & \multicolumn{1}{l}{ - - - }\\ 
 & \multicolumn{1}{l}{ - - - }\\ 
 & \multicolumn{1}{l}{ - - - }\\ 
 & \textbf{\textit{\begin{large}D\end{large}}ô} het Protizilas,\\ 
10 & der von art ein vürste was,\\ 
 & \textbf{lâzen} ein herzentuom.\\ 
 & daz lêch er dem, der manigen ruom\\ 
 & mit sîner hant bejagete.\\ 
 & \textbf{gegen} strîte er nie verzagete:\\ 
15 & Lachfill\textit{i}rost schahtelakunt\\ 
 & nam \dag erz\dag  \textbf{die} v\textit{a}nen sân zestunt.\\ 
 & von Az\textit{a}gou\textit{c} die vürsten hêr\\ 
 & nâmen den Schotten H\textit{u}teger\\ 
 & und Gaschieren den \textbf{\textit{O}r\textit{i}m\textit{an}}.\\ 
20 & si giengen vür ir hêrren \textbf{sân}.\\ 
 & der liez si ledic \textbf{umb ir} bet\textit{e}.\\ 
 & des \textbf{danketen} si \textbf{d\textit{ô}} Gahmuret\textit{e}.\\ 
 & Hutegeren den Schotten\\ 
 & si bâten sunder spotten:\\ 
25 & "lât mînem hêrren \textbf{daz} gezelt\\ 
 & hie \textbf{umb} âventiure gelt.\\ 
 & ez zuckete uns Ysenhartes lebe\textit{n},\\ 
 & daz Fridebrand\textit{e} wart gegeben\\ 
 & diu \textbf{zierde} unsers landes.\\ 
30 & sîn vr\textit{öu}de \textbf{diu} stuont pfandes.\\ 
\end{tabular}
\scriptsize
\line(1,0){75} \newline
m n o \newline
\line(1,0){75} \newline
\textbf{9} \textit{Initiale} m   $\cdot$ \textit{Capitulumzeichen} n  \newline
\line(1,0){75} \newline
\textbf{1} Gahmuret] Gamiret n Gamuͯret o  $\cdot$ ir] ire \textit{nachträglich korrigiert zu:} irem m \textbf{3} \textit{Die Verse 52.3-8 fehlen} m n o  \textbf{9} Dô] So \textit{(Initialbuchstabe }d \textit{vorgeschrieben)} m \textbf{12} lêch] [let]: lech m leht o \textbf{13} bejagete] [beiagent]: beiagette m \textbf{15} Lachfillirost] lahfiltrost m Lahfillarost n Lahsillarost o \textbf{16} erz] er n  $\cdot$ vanen sâ] vnnensa m fanen do o \textbf{17} Azagouc] azigout m azogunt n azaguͯnt o \textbf{18} den] der n  $\cdot$ Huteger] hitteger m húttiger n huͯttiger o \textbf{19} Oriman] erme \textit{nachtäglich korrigiert zu:} ein man m ernman n ariman o \textbf{21} ledic] leidig n  $\cdot$ bete] betten m \textbf{22} des] Das o  $\cdot$ dô] da m  $\cdot$ Gahmurete] gahmuretten m gamirette n gamuͯrete o \textbf{23} Hutegeren] [*]: Huttegeren m Húttegern n Húttigern o  $\cdot$ Schotten] scotten o \textbf{27} Ysenhartes] ÿsenhartes m ẏsenhartes n isenhartes o  $\cdot$ leben] lebens m \textbf{28} Fridebrande] fridebrandes m fribrande n fr:de brande o \textbf{30} vröude] fruͯnde m \newline
\end{minipage}
\end{table}
\newpage
\begin{table}[ht]
\begin{minipage}[t]{0.5\linewidth}
\small
\begin{center}*G
\end{center}
\begin{tabular}{rl}
 & Gahmuret, ir hêrren.\\ 
 & die selben wâren die êrren.\\ 
 & \textit{n}âher drungen die von Zazamanc\\ 
 & mit grôzer vuore, niht ze kranc,\\ 
5 & \textbf{unde} enpfiengen, als ir \textbf{hêrre} hiez,\\ 
 & von im ir lant und des geniez,\\ 
 & als \textbf{iegelîchen} ane gezôch.\\ 
 & diu armuot ir hêrren vlôch.\\ 
 & \textbf{dô} hete Prozitalas,\\ 
10 & der von arde ein vürste was,\\ 
 & \textbf{lâzen} ein herzentuom.\\ 
 & daz lêch er dem, der manigen ruom\\ 
 & mit sîner hant bejagete.\\ 
 & \textbf{an} strît er nie verzagete:\\ 
15 & Lafilirost schahtelakunt\\ 
 & nam ez \textbf{mit} vanen sân zestunt.\\ 
 & von Azagouc die vürsten hêr\\ 
 & nâmen den Schotten Huteger\\ 
 & unde Gatschieren den \textbf{Norman}.\\ 
20 & \textit{si} giengen vür ir hêrren \textbf{stân}.\\ 
 & der lie si ledec \textbf{durch sîne} bet.\\ 
 & des \textbf{dankten} si \textbf{dô} Gahmuret.\\ 
 & Hutegeren den Schotten\\ 
 & si bâten sunder spotten:\\ 
25 & "lât mînem hêrren \textbf{diz} gezelt\\ 
 & hie \textbf{umbe} âventiure gelt.\\ 
 & ez zuct uns Ysenhartes leben,\\ 
 & daz Fridebrande wart gegeben\\ 
 & diu \textbf{gezierde} unsers landes.\\ 
30 & sîn vröude stuont \textbf{dô} pfandes.\\ 
\end{tabular}
\scriptsize
\line(1,0){75} \newline
G I O L M Q R Z \newline
\line(1,0){75} \newline
\textbf{1} \textit{Initiale} O M  \textbf{17} \textit{Initiale} I  \textbf{27} \textit{Initiale} Q Z   $\cdot$ \textit{Capitulumzeichen} L  \newline
\line(1,0){75} \newline
\textbf{1} \textit{Die Verse 48.21-54.6 fehlen} R   $\cdot$ Gahmuret] Gahmureten I ÷Amvret O Gahmuͯret L GAmuͯret M Gamvret Q Gamurete Z  $\cdot$ ir] irr Q \textbf{2} die êrren] der êren I ierren Z \textbf{3} \textit{Die Verse 52.3-8 folgen auf 53.14} Z   $\cdot$ nâher] dar naher G Nach M Jn aber Z  $\cdot$ drungen] erdrungen I  $\cdot$ von] vo I  $\cdot$ Zazamanc] zazamanch G L zazamat Q \textbf{4} kranc] dranc Z \textbf{5} unde] Si O (L) (M) (Q) (Z)  $\cdot$ als] als als O on also M  $\cdot$ hêrre] frowe O L (M) (Q) Z \textbf{7} als] Alsie M  $\cdot$ iegelîchen] itzlichem Q \textbf{8} hêrren] herre Z \textbf{9} dô] Nv O L (M) (Q) Z  $\cdot$ Prozitalas] [protizalas]: Protizalas I Portizalas O L (Z) protisalas M prothizalas Q \textbf{11} lâzen] Verlaszen L Gelaszin M Zazen Q \textbf{12} lêch] \textit{om.} M [let]: leth Q  $\cdot$ der] den M Z \textbf{13} sîner hant] ritterschaft L \textbf{14} er] \textit{om.} L M \textbf{15} Lafilirost] lafiz rios G lafiz roy I Lafillirost O Z Lafilluost Q  $\cdot$ schahtelakunt] schatecunt O Thatelakuͯnt L (Q) \textbf{16} vanen] im Q  $\cdot$ sân] da I \textbf{17} Azagouc] azagoͮch G azagauc I azagavch O Azagouch L azagouck Q atzagovc Z \textbf{18} den shotten namen huͤtiger I  $\cdot$ den] die O  $\cdot$ Schotten] schoten G O  $\cdot$ Huteger] hvͦteger O huͯtteger L hvtteger Z \textbf{19} Gatschieren] Gantschiern I Gatschiern O Z Gatschieiren L gatschier M gatschiren Q  $\cdot$ Norman] ornier man M \textbf{20} si] vnde G \textbf{21} Der sie ledichet vmbe ir bet L  $\cdot$ der] Dy M  $\cdot$ lie] lib \textit{nachträglich korrigiert zu:} lisz Q  $\cdot$ durch sîne bete] dvrch ir bet O vmme or gebet M vmb erbet Q (Z) \textbf{22} dankten] [danchen]: danchetn I dankin M (Z)  $\cdot$ dô] \textit{om.} M da Z  $\cdot$ Gahmuret] Gamvret O Gahmuͯret L gamuͯret M gamúret Q gamurete Z \textbf{23} Hutegeren] huͤtigern I Hvͦtegern O Huͯttegeren L Hutegern M Huttiger Q Hvtteger Z  $\cdot$ den] dem M Z  $\cdot$ Schotten] schoten G shotten I schottin M \textbf{24} bâten] baten in O  $\cdot$ spotten] schotten \textit{nachträglich korrigiert zu:} spotten Q \textbf{25} mînem] minen O vnsern L meinē Q  $\cdot$ diz] das M (Q) \textbf{27} ez] Er M  $\cdot$ Ysenhartes] ẏsenhartes G Jsenhartes L Jsenhartis M eysenharts Q isenhartes Z \textbf{28} daz] Da L Do es Q  $\cdot$ Fridebrande] vridbrande I Frýdebrande L fridebrant Q  $\cdot$ gegeben] geben Q \textbf{29} gezierde] zierde L \textbf{30} stuont dô] div stvnt O (L) (Z) stunt M (Q) \newline
\end{minipage}
\hspace{0.5cm}
\begin{minipage}[t]{0.5\linewidth}
\small
\begin{center}*T (U)
\end{center}
\begin{tabular}{rl}
 & Gahmuret, ir hêrren.\\ 
 & die selben wâren die êrren.\\ 
 & nâher drungen die \textit{von} Zazamanc\\ 
 & mit grôzer vuore, niht ze kranc.\\ 
5 & \textbf{si} entviengen, als ir \textbf{vrouwe} hiez,\\ 
 & von im ir lant und des geniez,\\ 
 & als \textbf{iegelîchem} an gezôch.\\ 
 & diu armuot ir hêrren vlôch.\\ 
 & \textbf{nû} hete Protizalas,\\ 
10 & der von art ein vürste was,\\ 
 & \textbf{verlâzen} ein herzogentuom.\\ 
 & daz lêch er dem, der manegen ruom\\ 
 & mit sîner hant bejagete.\\ 
 & \textbf{an} strîte er nie verzagete:\\ 
15 & Lafilliros\textit{t} \textit{s}chahtelakunt\\ 
 & nam ez \textbf{mit} vanen sân zestunt.\\ 
 & von Azagouc die vürsten hêr\\ 
 & nâmen \textit{den} Schotten Huteger\\ 
 & und Gatschier den \textbf{Norman}.\\ 
20 & si giengen vür ir hêrren \textbf{stân}.\\ 
 & der liez si ledic \textbf{umb ir} bet.\\ 
 & des \textbf{ge\textit{n}âdete\textit{n}} si Gahmuret.\\ 
 & Hutegern den Schotten\\ 
 & si bâten sunder spotten:\\ 
25 & "lât mîne\textit{m} hêrren \textbf{diz} gezelt\\ 
 & hie \textbf{ûf} âventiure gelt.\\ 
 & ez zuht uns Isenhartes leben,\\ 
 & daz Fridebrant wart gegeben\\ 
 & diu \textbf{zier\textit{de}} unsers landes.\\ 
30 & sîn vreude stuont pfandes.\\ 
\end{tabular}
\scriptsize
\line(1,0){75} \newline
U V W T \newline
\line(1,0){75} \newline
\textbf{3} \textit{Majuskel} T  \textbf{9} \textit{Majuskel} T  \textbf{17} \textit{Initiale} T  \textbf{25} \textit{Majuskel} T  \textbf{27} \textit{Initiale} W  \newline
\line(1,0){75} \newline
\textbf{1} Gahmuret] Gahmuͦret U Gamuret V W Gahmvrete T  $\cdot$ ir] dem W \textbf{2} Die selben mit eren W \textbf{3} nâher] [No*]: Noher V Nach in T  $\cdot$ von] \textit{om.} U  $\cdot$ Zazamanc] zazamang V W \textbf{5} ir] sy ir W \textbf{7} iegelîchem] es ieglichen W ieglichen T \textbf{8} vlôch] harte floch W \textbf{9} nû] Do T  $\cdot$ hete] hette auch W  $\cdot$ Protizalas] [Prothi*salas]: Prothizsalas V protyzalas W [Protil*]: Protisalas T \textbf{10} vürste] her fúrste V \textbf{11} verlâzen] verlâzen gar T \textbf{15} Lafillirost] Lac fyllirost U [La*]: La fili rost V Lac filli roys W Lafilly ros T  $\cdot$ schahtelakunt] de scathelakunt U [*]: schahtelakvnt V de kastel kund W \textbf{16} ez] er W  $\cdot$ sân] so V W \textbf{17} Azagouc] azaguͦc U azagoug V azagoc W Azagôvc T  $\cdot$ die vürsten] der fúrste W \textbf{18} nâmen den] Namen U Nan von W  $\cdot$ Schotten] schoten T  $\cdot$ Huteger] Huͦteger U Hútiger V (W) hivtegêr T \textbf{19} Gatschier] Gasciern T  $\cdot$ Norman] Normân T \textbf{20} si] vnd T \textbf{22} genâdeten] [ge*]: gedadete U danketen T  $\cdot$ si] si do T  $\cdot$ Gahmuret] Gahmuͦret U Gamurette V (W) Gahmvrete T \textbf{23} Hutegern] Huͦtegern U Hútigern V W  $\cdot$ Schotten] Schoten T \textbf{25} mînem] minen U \textbf{26} ûf] vmb T \textbf{27} ez] ER W  $\cdot$ Isenhartes] Jsenhartes U V T ysenhartes W  $\cdot$ leben] loben W \textbf{28} daz] [d*]: do ez V Das sy W  $\cdot$ Fridebrant] fridebrande V (T) fridebrand W \textbf{29} zierde] zieret U [*zierde]: gezierde V gezirde W (T) \textbf{30} des stvnt sin vrôude pfandes T  $\cdot$ stuont] die stunt V (W) \newline
\end{minipage}
\end{table}
\end{document}
