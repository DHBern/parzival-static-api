\documentclass[8pt,a4paper,notitlepage]{article}
\usepackage{fullpage}
\usepackage{ulem}
\usepackage{xltxtra}
\usepackage{datetime}
\renewcommand{\dateseparator}{.}
\dmyyyydate
\usepackage{fancyhdr}
\usepackage{ifthen}
\pagestyle{fancy}
\fancyhf{}
\renewcommand{\headrulewidth}{0pt}
\fancyfoot[L]{\ifthenelse{\value{page}=1}{\today, \currenttime{} Uhr}{}}
\begin{document}
\begin{table}[ht]
\begin{minipage}[t]{0.5\linewidth}
\small
\begin{center}*D
\end{center}
\begin{tabular}{rl}
\textbf{230} & \begin{large}I\end{large}e vier gesellen sundersiz.\\ 
 & dâ zwischen was ein underviz,\\ 
 & dâr \textbf{vor} ein teppech sinewel.\\ 
 & fillu roy Frimutel\\ 
5 & mohte wol \textbf{geleisten} daz.\\ 
 & eines dinges man dâ niht vergaz:\\ 
 & si\textbf{ne} hete niht betûret,\\ 
 & mit marmel was gemûret\\ 
 & drî vierecke viwerram.\\ 
10 & dâr ûffe was des viwers nam,\\ 
 & holz \textbf{hiez} 'lign âlôê'.\\ 
 & sô \textbf{grôziu} viwer sît noch ê\\ 
 & sach niemen hie ze Wildenberc.\\ 
 & jenez wâren kostelîchiu werc.\\ 
15 & Der wirt sich selben \textbf{setzen} bat\\ 
 & gein der \textbf{mitteln} \textbf{viwerstat}\\ 
 & \textbf{ûf} \textbf{ein} spanbette.\\ 
 & \textbf{ez} was worden wette\\ 
 & zwischen im unt der vröude;\\ 
20 & er lebte niht wan töude.\\ 
 & In den palas kom gegangen,\\ 
 & der dâ wart wol enpfangen,\\ 
 & Parzival, der \textbf{lieht gevar},\\ 
 & von im, der in sante dar.\\ 
25 & \textbf{der} \textbf{liez in} \textbf{dâ} niht langer stên.\\ 
 & \textbf{in bat der wirt} nâher gên\\ 
 & \textbf{unt sitzen}: "zuo mir dâ her an.\\ 
 & sazte ich iuch verre \textbf{dort} hin dan,\\ 
 & daz wære \textbf{iu} alze gastlîch",\\ 
30 & \textbf{sus sprach der wirt} jâmers rîch.\\ 
\end{tabular}
\scriptsize
\line(1,0){75} \newline
D \newline
\line(1,0){75} \newline
\textbf{1} \textit{Initiale} D  \textbf{15} \textit{Majuskel} D  \textbf{21} \textit{Majuskel} D  \newline
\line(1,0){75} \newline
\textbf{13} Wildenberc] Wildenberch D \newline
\end{minipage}
\hspace{0.5cm}
\begin{minipage}[t]{0.5\linewidth}
\small
\begin{center}*m
\end{center}
\begin{tabular}{rl}
 & i\textit{e} vier gesellen sundersiz.\\ 
 & d\textit{â} enzwischen was ein underviz,\\ 
 & dâr \textbf{vor} ein teppich sinewel.\\ 
 & fili rois Frimutel\\ 
5 & mohte wol \textbf{gelusten} daz.\\ 
 & eines dinges man d\textit{â} niht vergaz:\\ 
 & si hete niht betûret,\\ 
 & mit marmel was gemûret\\ 
 & drî vierecke viurrame.\\ 
10 & dâr ûf was des viures name,\\ 
 & holz \textbf{hiez} 'lingnum âlôê'.\\ 
 & sô \textbf{grôz} \dag viurer\dag  sît noch ê\\ 
 & sach niemen hie ze Wildenberc.\\ 
 & jenez wâre\textit{n} kostlîchiu \textit{w}e\textit{r}c.\\ 
15 & \begin{large}D\end{large}er wirt sich selben \textbf{setzen} bat\\ 
 & gegen der \textbf{mittelen} \textbf{vürsten stat}\\ 
 & \textbf{ûf} \textbf{einem} spanbette.\\ 
 & \textbf{es} was worden wette\\ 
 & zwischen ime und der vröude;\\ 
20 & er lebete niht wen töude.\\ 
 & in den palas kam gegangen,\\ 
 & der d\textit{â} wart wol enpfangen,\\ 
 & Parcifal, der \textbf{wolgevar},\\ 
 & von ime, \dag von\dag  sante dar.\\ 
25 & \textbf{er} \textbf{beleip} niht langer stên.\\ 
 & \dag dis\dag  \textbf{bat der wirt} \textbf{dar} nâher gên.\\ 
 & \textbf{er sprach}: "\textbf{sitzet} zuo mir d\textit{â} her an.\\ 
 & sazt ich iuch verre hin dan,\\ 
 & daz wære \textbf{iu} al ze gastlîch."\\ 
30 & \textbf{der wirt was} jâmers rîch\\ 
\end{tabular}
\scriptsize
\line(1,0){75} \newline
m n o Fr69 \newline
\line(1,0){75} \newline
\textbf{15} \textit{Initiale} m o Fr69   $\cdot$ \textit{Capitulumzeichen} n  \newline
\line(1,0){75} \newline
\textbf{1} ie] Jr m Jo n o  $\cdot$ sundersiz] sunder siczen o \textbf{2} dâ] Das m Do n o  $\cdot$ enzwischen] enwisch das o  $\cdot$ underviz] ander vitz o \textbf{4} Frimutel] frimvttell m frimuͯtel o \textbf{5} mohte] Moͯchte n \textbf{6} dâ] do m n o \textbf{8} marmel] marner n o \textbf{9} vierecke] [vieecge]: vierecge Fr69  $\cdot$ viurrame] mit rame n \textbf{10} des viures name] des fúr name n dasz furname o \textbf{11} âlôê] alowe o \textbf{12} grôz] grosse n o \textbf{13} ze Wildenberc] ze wilden berg m (n) (o) zewildenberg Fr69 \textbf{14} wâren] ware m  $\cdot$ kostlîchiu] koͯstlich n (o)  $\cdot$ werc] beig m \textbf{15} selben] selber n o  $\cdot$ setzen] siczen o \textbf{18} worden] worder o \textbf{19} vröude] fronde n o \textbf{20} töude] tonde n (o) \textbf{22} dâ] do m n o  $\cdot$ wart wol] wol wart n \textbf{24} von] do n \textit{om.} o \textbf{26} dis] Des n (o)  $\cdot$ der] her o  $\cdot$ dar] her n o \textbf{27} sprach] [spracher]: sprache o  $\cdot$ sitzet] sitzen n (o)  $\cdot$ dâ] do m n o  $\cdot$ an] [im]: an m [in]: an o \textbf{28} sazt] [Sc]: Siczt o  $\cdot$ verre] ferre dort n o  $\cdot$ dan] an o \textbf{29} al] alle n o \newline
\end{minipage}
\end{table}
\newpage
\begin{table}[ht]
\begin{minipage}[t]{0.5\linewidth}
\small
\begin{center}*G
\end{center}
\begin{tabular}{rl}
 & \textit{ie vier gesellen sundersiz.}\\ 
 & \textit{dâ} \textit{enzwischen} \textit{was ein underviz,}\\ 
 & dâ \textbf{vor} ein teppich sinwel.\\ 
 & filiroys Frimutel\\ 
5 & maht wol \textbf{geleisten} daz.\\ 
 & eines dinges man dâ niht vergaz:\\ 
 & si\textbf{ne} hete niht betûret,\\ 
 & mit marmel was gemûret\\ 
 & drî vierecke viurram.\\ 
10 & dâr ûffe was des viures nam,\\ 
 & holz \textbf{hiez} 'lignâlôê'.\\ 
 & sô \textbf{grôziu} viur sît noch ê\\ 
 & sach niemen hie ze Wildenberc.\\ 
 & jenez wâren kosticlîchiu werc.\\ 
15 & der wirt sich selbe \textbf{sitzen} bat\\ 
 & gein der \textbf{mitteren} \textbf{viurstat}\\ 
 & \textbf{an} \textbf{ein} spanbette.\\ 
 & \textbf{es} was worden wette\\ 
 & zwischen im unde der vröude;\\ 
20 & er lebte niht wan töude.\\ 
 & in den palas kom gegangen,\\ 
 & der dâ wart wol enpfangen,\\ 
 & Parzival, der \textbf{lieht gevar},\\ 
 & von im, der in sande dar.\\ 
 & \hspace*{-.7em}\big| \textbf{in bat der wirt} nâher gên.\\ 
25 & \hspace*{-.7em}\big| \textbf{er} \textbf{lie in} \textbf{dâ} niht lenger stên:\\ 
 & "\textbf{sitzet} zuo mir \textit{dâ} her an.\\ 
 & satzte ich iuch verre \textbf{dort} hin dan,\\ 
 & daz wære alze gastlîch",\\ 
30 & \textbf{sô sprach der wirt} jâmers rîch.\\ 
\end{tabular}
\scriptsize
\line(1,0){75} \newline
G I O L M Q R Z Fr21 \newline
\line(1,0){75} \newline
\textbf{1} \textit{Großinitiale} Z   $\cdot$ \textit{Initiale} O L Fr21  \textbf{7} \textit{Initiale} I  \textbf{13} \textit{Initiale} R  \textbf{23} \textit{Initiale} I  \newline
\line(1,0){75} \newline
\textbf{1} \textit{Die Verse 230.1-2 fehlen} G   $\cdot$ ie] ÷e O Die M Hye Q  $\cdot$ sundersiz] sunder wicz vnd sicz R \textbf{2} enzwischen] entwischent L zwúschen R  $\cdot$ underviz] sunder sitz Q widerwicz R vnder witz Fr21 \textbf{3} ein] \textit{om.} I \textbf{4} Frimutel] vrumuntel I Frýmvntel L fruͯmutel M frimitel Q \textbf{5} maht] er moht I Der moht O (L) (M) (Q) (R) Z \textbf{6} dinges] dingen O  $\cdot$ dâ] do Q \textbf{7} sine hete] Si het O (L) Sie enhatten M Sine enhetten R  $\cdot$ betûret] betrubit M betrúret Q betrut R \textbf{9} drî] Dreyer Q  $\cdot$ vierecke] wiur ekke I vierekige R  $\cdot$ viurram] Niwer Ram I fier ram Q \textbf{10} was] \textit{om.} R  $\cdot$ nam] ram Z \textbf{11} hiez] \textit{om.} I heiszet L  $\cdot$ lignâlôê] lingnum aloe I (O) (L) (M) (Fr21) ligaloe Q \textbf{12} grôziu] Groz I (M) (Q) (R) \textbf{13} niemen hie] man nie L  $\cdot$ Wildenberc] wilden berch G wildeberc I wildeberch O wildenberg L (R) wiltberc M wilden berck Q wildenberch Z \textbf{14} jenez] [ien*]: ienz G Es R  $\cdot$ kosticlîchiu] kosteclich I hoffliche Q \textbf{15} selbe] selben O L (M) Q Z Fr21 selber R  $\cdot$ sitzen] bitten R \textbf{16} der] \textit{om.} Z  $\cdot$ mitteren] mitteln L mittel M mitten R  $\cdot$ viurstat] hertelen hert stat L \textbf{18} wette] [wete]: wette G mitte wette R \textbf{19} im] on M  $\cdot$ der vröude] vrowende L \textbf{20} er] ern I (O) (Fr21)  $\cdot$ lebte] lept I (O) (Q) (R) (Z) (Fr21)  $\cdot$ töude] gevde Z \textbf{21} \textit{Versdoppelung 231.25, 231.27-232.2 und 230.21 (²O) nach 230.21; Lesarten der vorausgehenden Verse mit ¹O bezeichnet} O   $\cdot$ in] vf G  $\cdot$ den palas] dem R \textbf{22} der] \textit{om.} O  $\cdot$ dâ wart] wart I Da wart da O wart da L M Fr21 do wart Q (R) \textbf{23} Parzival] Parzifal I M Parcifal O L Z Fr21 Partzifal Q Parczifal R  $\cdot$ lieht] lichte M (Q) \textbf{24} in] in da Z \textbf{26} gên] [gitan]: gan M \textbf{25} er] ern I (M) (Z) (Fr21) Es Q  $\cdot$ dâ] do Q \textbf{27} dâ] hie G do Q R \textbf{28} \textit{Vers 230.28 fehlt} R   $\cdot$ satzte] seze I (Q) (Fr21)  $\cdot$ iuch] \textit{om.} I  $\cdot$ verre dort] dort verre Q \textbf{29} alze] ev al ze I (Q) (R) (Z) (Fr21) iv gar ze O allen zcu M  $\cdot$ gastlîch] schwach R \textbf{30} sô] \textit{om.} I L Do O Da M  $\cdot$ jâmers rîch] Jamersach R \newline
\end{minipage}
\hspace{0.5cm}
\begin{minipage}[t]{0.5\linewidth}
\small
\begin{center}*T
\end{center}
\begin{tabular}{rl}
 & ie vier gesellen \textbf{ein} sundersiz.\\ 
 & dâ enzwischen was ein under\textit{v}iz,\\ 
 & dâr \textbf{ûf} ein teppich sinewel.\\ 
 & Fil li roys Frimutel\\ 
5 & mohte wol \textbf{geleisten} daz.\\ 
 & eines dinges man dâ niht vergaz:\\ 
 & si\textbf{ne} hete \textbf{des} niht betûret,\\ 
 & mit marmel was gemûret\\ 
 & drîe vierecke vîurrame.\\ 
10 & dâr ûf was des viures name,\\ 
 & holz \textbf{heizet} 'lignâlôê'.\\ 
 & sô \textbf{grôziu} viur sît noch ê\\ 
 & sach niemen hie ze Wildenberc.\\ 
 & jene\textit{z} wâren kosteclîch\textit{iu} werc.\\ 
15 & \begin{large}D\end{large}er wirt sich selbe \textbf{setzen} bat\\ 
 & gegen der \textbf{mitteln} \textbf{hertstat}\\ 
 & \textbf{ûf} \textbf{ein} spanbette.\\ 
 & \textbf{ez} was worden wette\\ 
 & Zwischen im unde der vröude;\\ 
20 & er\textbf{n} lebete niht wan töude.\\ 
 & In daz palas kom gegangen,\\ 
 & der dâ wart wol enpfangen\\ 
 & \hspace*{-.7em}\big| von im, der in sante dar,\\ 
 & \hspace*{-.7em}\big| Parcifal, der \textbf{lieht gevar}.\\ 
25 & \textbf{er}\textbf{n} \textbf{liez in} niht langer stân.\\ 
 & \textbf{der wirt bat in} nâher gân:\\ 
 & "\textbf{sitzet} zuo mir dâ her an.\\ 
 & sat ich iuch verre \textbf{dort} hin dan,\\ 
 & daz wære al ze gastlîch",\\ 
30 & \textbf{sprach der wirt} jâmers rîch.\\ 
\end{tabular}
\scriptsize
\line(1,0){75} \newline
T U V W \newline
\line(1,0){75} \newline
\textbf{4} \textit{Majuskel} T  \textbf{15} \textit{Initiale} T U V  \textbf{19} \textit{Majuskel} T  \textbf{21} \textit{Majuskel} T  \newline
\line(1,0){75} \newline
\textbf{1} ein] \textit{om.} U V W  $\cdot$ sundersiz] sunder W \textbf{2} dâ] Do U W  $\cdot$ enzwischen] tuschen U  $\cdot$ underviz] [d*]: vnderwitz T vnder sitz U [vnder*]: vnderwitz V sitz vnder W \textbf{4} Frimutel] Frimuͦtel U frimvntel V \textbf{5} mohte] Moͤchte W \textbf{6} dâ] do V W \textbf{7} sine hete] Sie in heten U Sv́ [enhette*]: enhette V Sy hette W  $\cdot$ des] dez V \textbf{9} vierecke] vier ecket W \textbf{11} holz] Ein holtz W  $\cdot$ lignâlôê] [lign*loe]: lignum aloe V lignum aloe W \textbf{13} sach] Sach do W  $\cdot$ hie] \textit{om.} W  $\cdot$ Wildenberc] wilden berc U wildesberg V wildenberck W \textbf{14} jenez] iens T (W) Lenz U  $\cdot$ kosteclîchiu] costecliche T die kuͦstliche U \textbf{15} selbe] selber U selben V W  $\cdot$ setzen] sitzen U [*zen]: sezzen V \textbf{16} mitteln] mittelsten W  $\cdot$ hertstat] [*]: vúrstat V \textbf{17} spanbette] [V*]: rein spanbette V \textbf{20} wan töude] den mit [*]: tovde V \textbf{21} daz] den U V W \textbf{22} dâ] do U V W \textbf{24} in] in vor W \textbf{23} Parcifal] Parzifal T U V Partzifal W \textbf{25} ern liez in] Er lies in do W \textbf{26} nâher] [*]: dar naher V \textbf{27} \textit{Versfolge 230.28-27} U   $\cdot$ sitzet zuo mir] [*]: Er sprach sittzent V  $\cdot$ dâ] do U \textit{om.} W \textbf{28} iuch] îv T \textbf{29} daz] Do U  $\cdot$ al ze] [*]: v́ch alze V  $\cdot$ gastlîch] gaistlich W \textbf{30} sprach] [*]: So sprach V \newline
\end{minipage}
\end{table}
\end{document}
