\documentclass[8pt,a4paper,notitlepage]{article}
\usepackage{fullpage}
\usepackage{ulem}
\usepackage{xltxtra}
\usepackage{datetime}
\renewcommand{\dateseparator}{.}
\dmyyyydate
\usepackage{fancyhdr}
\usepackage{ifthen}
\pagestyle{fancy}
\fancyhf{}
\renewcommand{\headrulewidth}{0pt}
\fancyfoot[L]{\ifthenelse{\value{page}=1}{\today, \currenttime{} Uhr}{}}
\begin{document}
\begin{table}[ht]
\begin{minipage}[t]{0.5\linewidth}
\small
\begin{center}*D
\end{center}
\begin{tabular}{rl}
\textbf{218} & Dô sprach er: "vrouwe, sît ir daz,\\ 
 & der ich sol dienen âne haz?\\ 
 & ein teil \textbf{michs twinget} nôt.\\ 
 & \textbf{sîn} dienst \textbf{iu enbôt} der ritter rôt.\\ 
5 & \textbf{der} wil \textbf{vil} \textbf{ganze} pflihte hân,\\ 
 & swaz iu \textbf{ze laster} ist getân.\\ 
 & Ouch bit er\textbf{z} Artuse klagen.\\ 
 & ich wæne, ir sît durch in geslagen.\\ 
 & vrouwe, ich bringe iu sicherheit.\\ 
10 & \textbf{sus} gebôt, der mit mir \textbf{dâ} streit.\\ 
 & \textbf{nû} leist ich\textbf{z} gerne, swenn ir welt.\\ 
 & mîn lîp gein tôde was verselt."\\ 
 & Vrou Cunneware de Lalant\\ 
 & greif an die \textbf{geîserten} hant.\\ 
15 & al dâ vrou Ginover saz,\\ 
 & diu âne den künec mit ir az.\\ 
 & Keie \textbf{ouch} vor dem tische stuont,\\ 
 & al dâ im \textbf{diz mære wart} kunt.\\ 
 & \textbf{der} widersaz \textbf{im} ein teil.\\ 
20 & des wart vrou Cunneware geil.\\ 
 & \textbf{Dô} sprach \textit{er}: "vrouwe, dirre man,\\ 
 & swaz der \textbf{hât gein iu} getân,\\ 
 & des ist \textbf{er} vast underzogen.\\ 
 & \textbf{doch} wæne ich \textbf{des}, er \textbf{ist ûf gelogen}.\\ 
25 & Ich tetz durch hovelîchen site\\ 
 & unt wolt iuch hân \textbf{gebezzert} mite.\\ 
 & dâr umbe \textbf{hân ich} iweren haz.\\ 
 & \textbf{iedoch} wil ich iu râten daz:\\ 
 & \textbf{heizet} entwâpen disen gevangen,\\ 
30 & \textbf{in} mac hie stêns erlangen."\\ 
\end{tabular}
\scriptsize
\line(1,0){75} \newline
D \newline
\line(1,0){75} \newline
\textbf{1} \textit{Majuskel} D  \textbf{7} \textit{Majuskel} D  \textbf{13} \textit{Majuskel} D  \textbf{21} \textit{Majuskel} D  \textbf{25} \textit{Majuskel} D  \newline
\line(1,0){75} \newline
\textbf{21} er] \textit{om.} D \newline
\end{minipage}
\hspace{0.5cm}
\begin{minipage}[t]{0.5\linewidth}
\small
\begin{center}*m
\end{center}
\begin{tabular}{rl}
 & dô sprach er: "vrouwe, sît ir daz,\\ 
 & der \textit{i}ch sol dienen âne haz?\\ 
 & ein teil \textbf{mich es twinget} nôt.\\ 
 & \textbf{sînen} dienst \textbf{enbiutet iu} der ritter rôt.\\ 
5 & \textbf{der} wil \textbf{vil} \textbf{grôze} pflihte hân,\\ 
 & waz iu \textbf{ze laster} ist getân.\\ 
 & ouch bittet er Artusen klagen.\\ 
 & ich wæne, ir sît durch in geslagen.\\ 
 & vrouwe, ich bringe iu sicherheit,\\ 
10 & \textbf{alsô} \textbf{mir} gebôt, der mit mir streit.\\ 
 & \textbf{nû} leist ich\textbf{z} gerne, wenne \textit{i}r welt.\\ 
 & mîn lîp gegen tôde was verselt."\\ 
 & \textit{\begin{large}V\end{large}}rouwe Cu\textit{nn}eware de Lalant\\ 
 & greif an die \textbf{geîserten} hant.\\ 
15 & aldâ vrouwe Ginovere saz,\\ 
 & diu âne d\textit{en} künic mit ir az.\\ 
 & K\textit{e}ie \textbf{ouch} vor dem tisch\textit{e} \textit{s}tuont,\\ 
 & aldâ ime \textbf{war\textit{t} \textit{d}iz mære} kunt.\\ 
 & \textbf{der} widersaz \textbf{es ime} ein teil.\\ 
20 & des wart vrouwe Cunnewar\textit{e} geil.\\ 
 & \textbf{doch} sprach er: "vrouwe, dirre man,\\ 
 & waz der \textbf{hât gegen iu} getân,\\ 
 & des ist \textbf{der} vaste underzogen.\\ 
 & \textbf{dô} wæne ich, er \textbf{ist angelogen}.\\ 
25 & ich tete ez durch hovelîche site\\ 
 & und wolt iuch hân \textbf{gebezzert} mite.\\ 
 & dâr umbe \textbf{hân ich} iuweren haz.\\ 
 & \textbf{iedoch} wil ich iu râten daz:\\ 
 & \textbf{heizen} entwâpen disen gevang\textit{en},\\ 
30 & \textbf{in} mac hie stêns erlangen."\\ 
\end{tabular}
\scriptsize
\line(1,0){75} \newline
m n o Fr69 \newline
\line(1,0){75} \newline
\textbf{13} \textit{Initiale} m o   $\cdot$ \textit{Capitulumzeichen} n  \newline
\line(1,0){75} \newline
\textbf{2} ich] uch m \textbf{3} nôt] rot o \textbf{4} iu] rich o \textbf{5} grôze] gantze n (o) \textbf{7} Artusen] artussen m artusez o \textbf{11} ir] mir m \textbf{12} verselt] gezelt n o \textbf{13} Vrouwe] Arouewe m ÷Rowe o  $\cdot$ Cunneware] Cumeware m coniuera n Convern o  $\cdot$ de] dez o \textbf{14} geîserten] geserten n o \textbf{15} vrouwe] frowet o  $\cdot$ Ginovere] ginofere n ginefore o \textbf{16} den] die m t den o  $\cdot$ künic] [koment]: konig o \textbf{17} Keie] Kaie m Keẏe n o  $\cdot$ dem] disem n  $\cdot$ tische stuont] tische sas vnd stuͯnt m \textbf{18} wart diz] wart die dis m wart dise n wart des o \textbf{20} vrouwe] [tan]: [stan]: frouwe m  $\cdot$ Cunneware] cunewaree m conuwera n Conne waren o \textbf{21} doch] Do n \textbf{23} der] er n o  $\cdot$ underzogen] vnder zoigen o \textbf{24} dô] Doch n o  $\cdot$ er] es n o  $\cdot$ angelogen] vngelogen n o \textbf{25} site] [strit]: sitte o \textbf{26} iuch] v́chz Fr69 \textbf{27} iuweren] iren m n eren o \textbf{29} heizen] Heissent o  $\cdot$ entwâpen] entwoppent o  $\cdot$ gevangen] gevang m \textbf{30} in] Jnne o \newline
\end{minipage}
\end{table}
\newpage
\begin{table}[ht]
\begin{minipage}[t]{0.5\linewidth}
\small
\begin{center}*G
\end{center}
\begin{tabular}{rl}
 & dô sprach er: "vrouwe, sît ir daz,\\ 
 & der ich sol dienen âne haz?\\ 
 & ein teil \textbf{twinget mich \textit{sîn}} nôt.\\ 
 & dienst \textbf{iu enbiut} der rîter rôt\\ 
5 & \textbf{unde} wil \textbf{ganze} pflihte hân,\\ 
 & swaz iu \textbf{ze laster} ist getân.\\ 
 & ouch bit er\textbf{z} Artuse klagen.\\ 
 & ich wæne, ir sît durch in geslagen.\\ 
 & vrouwe, ich bringiu sicherheit.\\ 
10 & \textbf{sus} gebôt, der mit mir streit.\\ 
 & \textbf{nû} leist ich gerne, sw\textit{en} ir welt.\\ 
 & mîn lîp gein tôde was verselt."\\ 
 & vrou Kuneware de Lalant\\ 
 & greif an die \textbf{gesêrten} hant.\\ 
15 & al dâ vrô Schinover saz,\\ 
 & diu âne den künic mit ir az.\\ 
 & Kay vor dem tische stuont,\\ 
 & al dâ im \textbf{wart diz mære} kunt.\\ 
 & \textbf{er} widersaz \textbf{i\textit{m}} ein teil.\\ 
20 & des wart vrô Kuneware geil.\\ 
 & \textbf{doch} sprach er: "vrouwe, dirre man,\\ 
 & swaz der \textbf{gein iu hât} getân,\\ 
 & des ist \textbf{er} vaste underzogen.\\ 
 & \textbf{doch} wæne ich \textbf{des}, er\textbf{st ûf gelogen}.\\ 
25 & ich tetz durch höviclîche site\\ 
 & unde wolt iuch hân \textbf{gezogen} \textbf{dâr} mite.\\ 
 & \textit{d}âr umbe \textbf{hân ich} iweren haz.\\ 
 & \textbf{\textit{ied}och} wil ich iu râten daz:\\ 
 & \textbf{\textit{h}eizet} entwâpenen disen gevangen,\\ 
30 & \textbf{\textit{de}n} mac hie stên\textit{s} erlangen."\\ 
\end{tabular}
\scriptsize
\line(1,0){75} \newline
G I O L M Q R Z Fr21 Fr40 \newline
\line(1,0){75} \newline
\textbf{1} \textit{Initiale} I R  \textbf{7} \textit{Initiale} L Fr21  \textbf{9} \textit{Initiale} O Q Fr40  \textbf{17} \textit{Initiale} I M Z  \newline
\line(1,0){75} \newline
\textbf{1} dô] Da O M Z \textbf{2} ich] \textit{om.} M  $\cdot$ dienen] dien I \textbf{3} twinget] twiget Fr21  $\cdot$ mich sîn] miches G mich des L iz mich M \textbf{4} dienst] Sein dinst Q (R) (Fr40) Sinen dienst Z  $\cdot$ iu enbiut] enbiut ev I iv enbot O (Q) (R) \textbf{5} wil] wil vil L M Q R Z Fr21 Fr40 \textbf{6} swaz] Waz L (Q) (R)  $\cdot$ ist] sey Q s::: Fr40 \textbf{7} bit] hiez I O L (Q) (R) Fr21 Fr40 bet M  $\cdot$ erz] er >daz< L  $\cdot$ Artuse] Artvsen O (Q) (R) (Z) (Fr21) Artuͯsen L artu::: Fr40  $\cdot$ klagen] sagen L \textbf{8} sît durch in] duͯrch in sit L \textbf{9} vrouwe] ÷rowe O \textit{om.} L  $\cdot$ bringiu] bringe uͯch mýne L brenge M \textbf{10} sus] daz I \textit{om.} Q  $\cdot$ gebôt] enbot iv O Gebot mir Q bot iv Fr21  $\cdot$ mit mir] mir mit R  $\cdot$ streit] reyt Q \textbf{11} ich] ichz O (M) (Q) (R) Z (Fr21) Fr40  $\cdot$ swen] swaz G [swen]: swe O wenne L (M) (Q) (R) Z \textbf{12} lîp] leypt Q  $\cdot$ gein] [ze*]: zedem I  $\cdot$ verselt] geselt O L Q (Fr21) \textbf{13} Kuneware] kunware I (O) M Q Cuͦnaware R kvnneware Z kvnwar Fr21 cuneware Fr40  $\cdot$ de] von R  $\cdot$ Lalant] labant R \textbf{14} an] in I  $\cdot$ gesêrten] Gesergeten I sichernde O verserten Q geiserten Z \textbf{15} vrô Schinover] diu kunginne I frov Gynover O (Z) Genovire L fro ginover M fraw gynoúer Q frow Gynouer R vrow ginove::: Fr40 \textbf{16} diu] Vnd L \textbf{17} Kay] Kai G Key O (Z) (Fr40) Kaý L Keye M Keẏ R Fr21  $\cdot$ vor] auch vor Q (R) Fr40  $\cdot$ dem] den M \textbf{18} al dâ im] Do mit Q da im Fr40  $\cdot$ diz] daz I Fr21 \textbf{19} er] Es Q Der Z  $\cdot$ im] imz G ez im O Fr21 Fr40 es L \textbf{20} Kuneware] kunware I (O) M Q Cunware R kunneware Z kvnwar Fr21 cu::: Fr40 \textbf{21} doch] do I (O) (L) (Q) Fr40 Da Z  $\cdot$ er] keý L \textbf{22} swaz] Waz L (Q) (R)  $\cdot$ der] er O Fr21  $\cdot$ gein iu hât] hat gein iv O (L) (M) (R) (Z) Fr21 (Fr40) \textbf{24} doch wæne ich] Jch wæn O (L) (Q) Fr21 (Fr40) Wene ich R  $\cdot$ des] \textit{om.} I O L Q R Fr21 Fr40  $\cdot$ erst] ir sit I  $\cdot$ ûf] an O L Q R Fr21 Fr40 \textbf{25} höviclîche] [bvbschlichen]: hvbschlichen O hofelichen Z (Fr21)  $\cdot$ site] siten Q list sit Fr21 \textbf{26} wolt] wol Q  $\cdot$ hân] da han M  $\cdot$ gezogen] gebezzert O (L) (M) (Q) (R) Z (Fr21)  $\cdot$ dâr] \textit{om.} O L M R Z \textbf{27} dâr] :::ar G  $\cdot$ hân ich] ich han O L  $\cdot$ haz] [zorn]: haz O \textbf{28} iedoch] :::och G  $\cdot$ daz] baz I (M) (R) Z \textbf{29} heizet] :::eizet G  $\cdot$ entwâpenen] entwasten O \textbf{30} den] :::n G Jn O L (M) Q R Z Fr21 (Fr40)  $\cdot$ hie] \textit{om.} I hie wol R  $\cdot$ stêns] stende G stanens L ston R  $\cdot$ erlangen] belangen O (L) (Q) R Fr21 b::: Fr40 \newline
\end{minipage}
\hspace{0.5cm}
\begin{minipage}[t]{0.5\linewidth}
\small
\begin{center}*T
\end{center}
\begin{tabular}{rl}
 & Dô sprach er: "vrouwe, sît ir daz,\\ 
 & der ich sol dienen âne haz?\\ 
 & ein teil \textbf{betwinget mich des} nôt.\\ 
 & dienst \textbf{enbiut iu} der rîter rôt.\\ 
5 & \textbf{der} wil \textbf{vi\textit{l}} \textbf{\textit{g}anze} pflihte hân,\\ 
 & swaz iu \textbf{lasters} ist getân.\\ 
 & ouch bittet er\textbf{z} Artusen klagen.\\ 
 & ich wæne, ir sît durch in geslagen.\\ 
 & Vrouwe, ich bringiu sicherheit,\\ 
10 & \textbf{alsez} gebôt, der mit mir streit.\\ 
 & \textbf{daz} leistich gerne, swenn ir welt.\\ 
 & mîn lîp gegen tôde was verselt."\\ 
 & \begin{large}V\end{large}rou Cunneware de Lalant\\ 
 & greif an die \textbf{geîserten} hant.\\ 
15 & Aldâ vrô Gynover saz,\\ 
 & diu âne den künec mit ir az.\\ 
 & Key \textbf{ouch} vor dem tische stuont,\\ 
 & aldâ im \textbf{wart diz mære} kunt.\\ 
 & \textbf{er} widersaz \textbf{ez} ein teil.\\ 
20 & des wart vrou Cunneware geil.\\ 
 & \textbf{dô} sprach er: "vrouwe, dirre man,\\ 
 & swaz der \textbf{hât gegen iu} getân,\\ 
 & des ist \textbf{er} vaste underzogen.\\ 
 & \textbf{doch} wænich \textbf{vaste}, er \textbf{sî belogen}.\\ 
25 & ich tet ez durch höveschlîchen site\\ 
 & unde wolte iuch hân \textbf{gebezzert} mite.\\ 
 & dâr umbe \textbf{hâtich} iuwern haz.\\ 
 & \textbf{doch} wil ich iu râten daz:\\ 
 & \textbf{heizet} entwâpen disen gevangen,\\ 
30 & \textbf{in} mac hie stêns erlangen."\\ 
\end{tabular}
\scriptsize
\line(1,0){75} \newline
T U V W \newline
\line(1,0){75} \newline
\textbf{1} \textit{Majuskel} T  \textbf{9} \textit{Majuskel} T  \textbf{13} \textit{Initiale} T U W  \textbf{15} \textit{Majuskel} T  \textbf{17} \textit{Majuskel} T  \newline
\line(1,0){75} \newline
\textbf{3} mich des] es mich W \textbf{4} dienst] [*]: Sinen dienst V \textbf{5} vil ganze] vil vilganze T \textbf{6} swaz] Waz U (W)  $\cdot$ lasters] [l*]: zelaster V \textbf{7} \textit{Versfolge 218.8-7} W   $\cdot$ bittet erz] butet er iz U [*]: hies ers V bitte ers W  $\cdot$ Artusen] artuse W \textbf{10} alsez] Alse [*]: mir V Alsus W  $\cdot$ gebôt] gebot er W \textbf{11} swenn] wan U wem W \textbf{12} gegen tôde was] waz gegen dem tode V  $\cdot$ verselt] geuelt W \textbf{13} Cunneware] Cvmuware U Kvnneware V kunnewar W  $\cdot$ Lalant] laland W \textbf{14} geîserten] geiserte U gesereten V \textbf{15} vrô] vor W  $\cdot$ Gynover] Gẏnover V schinouer W \textbf{17} Key] [Kei*]: Keige V \textbf{18} Als die truͦchssessen thuͦnd W  $\cdot$ diz] dise U \textbf{19} er widersaz ez] Er wider sazen U [*]: Er widersas [*]: ez im V Er entsaß das mere doch W \textbf{20} vrou] \textit{om.} W  $\cdot$ Cunneware] Kvnneware T Cuͦmeware U kunnewar W  $\cdot$ geil] in ir gail W \textbf{21} dô] Doch U V \textbf{22} swaz] Waz U (W)  $\cdot$ hât gegen iu] gen uch hat W \textbf{24} vaste er sî] er ist vaste U er ist V wol er ist W  $\cdot$ belogen] [*]: an gelogen V angelogen W \textbf{25} höveschlîchen] hoveliche V hoͤuische W \textbf{27} hâtich] han ich V (W) \textbf{28} daz] bas W \textbf{30} stêns] ston W  $\cdot$ erlangen] belangen W \newline
\end{minipage}
\end{table}
\end{document}
