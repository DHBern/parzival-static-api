\documentclass[8pt,a4paper,notitlepage]{article}
\usepackage{fullpage}
\usepackage{ulem}
\usepackage{xltxtra}
\usepackage{datetime}
\renewcommand{\dateseparator}{.}
\dmyyyydate
\usepackage{fancyhdr}
\usepackage{ifthen}
\pagestyle{fancy}
\fancyhf{}
\renewcommand{\headrulewidth}{0pt}
\fancyfoot[L]{\ifthenelse{\value{page}=1}{\today, \currenttime{} Uhr}{}}
\begin{document}
\begin{table}[ht]
\begin{minipage}[t]{0.5\linewidth}
\small
\begin{center}*D
\end{center}
\begin{tabular}{rl}
\textbf{577} & \begin{large}D\end{large}er sig ist iwer hiute.\\ 
 & nû trœstet uns armen liute,\\ 
 & ob iweren wunden sî alsô,\\ 
 & daz wir mit iu \textbf{wesen} vrô."\\ 
5 & Er sprach: "sæhet ir mich gerne leben,\\ 
 & sô sult ir mir helfe geben."\\ 
 & des bat er die \textbf{vrouwen}:\\ 
 & "lât \textbf{mîne} wunden schouwen\\ 
 & \textbf{etswen}, der dâ kunne mite.\\ 
10 & sol ich begên noch strîtes site,\\ 
 & sô bindet \textbf{mînen helm ûf} unt gêt ir hin;\\ 
 & den lîp ich gern wernde bin."\\ 
 & Si jâhen: "ir sît nû strîtes vrî,\\ 
 & hêrre, lât uns iu wesen bî.\\ 
15 & wan einiu sol gewinnen\\ 
 & an vier küneginnen\\ 
 & daz botenbrôt, ir lebt noch.\\ 
 & man sol iu bereiten \textbf{ouch}\\ 
 & gemach unt erzenîe clâr\\ 
20 & unt \textbf{wol} mit triwen nemen war,\\ 
 & mit salben sô gehiure,\\ 
 & diu vür die quaschiure\\ 
 & unt vür die wunden ein genist\\ 
 & mit senfte helfeclîchen ist."\\ 
25 & Der meide einiu dannen spranc\\ 
 & sô balde, daz si ninder hanc.\\ 
 & \textbf{diu} brâhte ze hove mære,\\ 
 & daz er bî lebene wære\\ 
 & "unt alsô \textbf{lebelîche},\\ 
30 & daz er uns vreuden rîche\\ 
\end{tabular}
\scriptsize
\line(1,0){75} \newline
D Fr7 Fr59 \newline
\line(1,0){75} \newline
\textbf{1} \textit{Initiale} D  \textbf{5} \textit{Majuskel} D  \textbf{13} \textit{Majuskel} D  \textbf{25} \textit{Majuskel} D  \newline
\line(1,0){75} \newline
\textbf{11} mînen] mir den Fr7 \textbf{14} bî] ::t Fr7 \textbf{15} einiu] vnser einiv Fr7 \textbf{17} botenbrôt ir] bettenbrot daz Fr7 \textbf{19} erzenîe] arcedei::: Fr7 \textbf{24} senfte] senften Fr7 \textbf{27} diu] Si Fr7 \newline
\end{minipage}
\hspace{0.5cm}
\begin{minipage}[t]{0.5\linewidth}
\small
\begin{center}*m
\end{center}
\begin{tabular}{rl}
 & der sic ist iuwer hiute.\\ 
 & nû trœstet uns armen liute,\\ 
 & ob iuwern wunden sî alsô,\\ 
 & daz wir mit iu \textbf{werden} vrô."\\ 
5 & er sprach: "sæhet ir mich gern leben,\\ 
 & sô solt ir mir helfe geben."\\ 
 & des \textit{b}at er die \textbf{juncvrouwen}:\\ 
 & "lât \textbf{die} wunden schouwen\\ 
 & \textbf{etw\textit{e}n}, der dâ k\textit{ünn}e mite.\\ 
10 & so\textit{l} ich begân n\textit{o}ch strîtes site,\\ 
 & sô bindet \textbf{mir} \textbf{\textit{den} helm} und gât ir hin;\\ 
 & den lîp ich gerne wernde bin."\\ 
 & si jâhen: "ir sît nû strîtes vrî,\\ 
 & hêrre, lât uns iu wesen \textit{b}î.\\ 
15 & wan einiu sol gew\textit{i}nnen\\ 
 & an vier küniginnen\\ 
 & daz boten brôt, ir lebt noch.\\ 
 & man sol iu bereiten \textbf{doch}\\ 
 & gemach und arz\textit{e}n\textit{îe} clâr\\ 
20 & und mit triuwen nemen war,\\ 
 & mit salben sô gehiur,\\ 
 & diu vür die quaschiur\\ 
 & und vür die wunden ein genist\\ 
 & mit senfte helfeclîchen ist."\\ 
25 & \begin{large}D\end{large}er megde einiu dannen spranc\\ 
 & sô balde, daz si nindert hanc,\\ 
 & \textbf{und} brâhte zuo hove mære,\\ 
 & daz er bî l\textit{e}ben wære\\ 
 & "und alsô \textbf{lebelîch},\\ 
30 & daz er uns vröuden rîch\\ 
\end{tabular}
\scriptsize
\line(1,0){75} \newline
m n o \newline
\line(1,0){75} \newline
\textbf{25} \textit{Illustration mit Überschrift:} Also her gawan zuͯ bette lag vnd gar sere wunt was vnd die kv́niginn in besohent (besehent o  ) n (o)   $\cdot$ \textit{Großinitiale} n   $\cdot$ \textit{Initiale} m o  \newline
\line(1,0){75} \newline
\textbf{4} werden] wesen n o \textbf{7} bat] lat m \textbf{9} etwen] Ettwan m Etwenn o  $\cdot$ künne] kome m \textbf{10} sol ich] So ich m Solt ir o  $\cdot$ noch] nach m \textbf{11} den] \textit{om.} m \textbf{12} wernde] wernden o \textbf{13} si] So o \textbf{14} iu] aúch o  $\cdot$ bî] frẏ m \textbf{15} gewinnen] gewonnen m \textbf{17} Das bottent brot ir leben noch n \textbf{19} arzenîe] arczein m \textbf{20} mit] wol mit n o \textbf{24} senfte] senftten m (n) (o)  $\cdot$ ist] list o \textbf{26} nindert] nider o \textbf{28} leben] lelben m \newline
\end{minipage}
\end{table}
\newpage
\begin{table}[ht]
\begin{minipage}[t]{0.5\linewidth}
\small
\begin{center}*G
\end{center}
\begin{tabular}{rl}
 & \begin{large}D\end{large}er sic ist iuwer hiute.\\ 
 & nû trœstet uns arme liute,\\ 
 & ob iuwern wunden sî alsô,\\ 
 & daz wir mit iu \textbf{wesen} vrô."\\ 
5 & er sprach: "sæhet ir mich gerne leben,\\ 
 & sô sult ir mir helfe geben."\\ 
 & des bat er die \textbf{vrouwen}:\\ 
 & "lât \textbf{mîne} wunde\textit{n} schouwen\\ 
 & \textbf{eteswenne}, der dâ kunne mite.\\ 
10 & sol ich begên noch strîtes site,\\ 
 & sô bint \textbf{mînen helm ûf} unde gêt ir hin;\\ 
 & den lîp ich gerne werende bin."\\ 
 & si jâhen: "ir sît nû strîtes vrî,\\ 
 & hêrre, lât uns iu wesen bî.\\ 
15 & wan einiu sol gewinnen\\ 
 & an vier küneginnen\\ 
 & daz boten brôt, ir lebet noch.\\ 
 & man sol iu bereiten \textbf{ouch}\\ 
 & gemach unde erzenîe clâr\\ 
20 & unde \textbf{wol} mit triuwen nemen war,\\ 
 & mit salben sô gehiure,\\ 
 & diu vür die quatschiure\\ 
 & unde vür die wunden ein genist\\ 
 & mit senfte helfeclîchen ist."\\ 
25 & der meide einiu dannen spranc\\ 
 & sô balde, daz si ninder hanc.\\ 
 & \textbf{diu} brâhte ze hove mære,\\ 
 & daz er bî lebene wære\\ 
 & "unde alsô \textbf{lebelîche},\\ 
30 & daz er uns vröuden rîche\\ 
\end{tabular}
\scriptsize
\line(1,0){75} \newline
G I L M Z \newline
\line(1,0){75} \newline
\textbf{1} \textit{Initiale} G I L Z  \textbf{21} \textit{Initiale} I M  \newline
\line(1,0){75} \newline
\textbf{2} trœstet] trosten Z  $\cdot$ arme] armen I (M) Z \textbf{3} iuwern] uwer L  $\cdot$ sî] sin L \textbf{5} sæhet] sehet I sit M \textbf{7} vrouwen] juncfrouwen M \textbf{8} wunden] wunde G \textbf{10} noch] nach Z \textbf{11} mînen] mir den I (Z)  $\cdot$ ûf] \textit{om.} L  $\cdot$ ir] \textit{om.} L \textbf{13} jâhen] sprechin M  $\cdot$ nû] \textit{om.} I \textbf{14} uns iu] evch vns Z \textbf{15} einiu] eyne dy M \textbf{17} daz boten brôt] da betenbrot I \textbf{18} ouch] dach M (Z) \textbf{19} erzenîe] erzenien I arzedie M \textbf{20} wol] auch I  $\cdot$ nemen] \textit{om.} Z \textbf{22} quatschiure] quaskure I \textbf{24} ist] list I M Z \textbf{26} ninder] nirgen M \textbf{27} mære] niwiu mere I \textbf{29} lebelîche] lobeliche L M \newline
\end{minipage}
\hspace{0.5cm}
\begin{minipage}[t]{0.5\linewidth}
\small
\begin{center}*T
\end{center}
\begin{tabular}{rl}
 & Der sic ist iuwer hiute.\\ 
 & nû trœstet uns armen liute,\\ 
 & ob iuweren wunden sî alsô,\\ 
 & daz \textit{wir} mit iu \textbf{wesen} vrô."\\ 
5 & er sprach: "sæht ir mich gerne leben,\\ 
 & sô solt ir mir helfe geben."\\ 
 & des bat er die \textbf{vrouwen}:\\ 
 & "lât \textbf{mîne} wunden schouwen\\ 
 & \textbf{etwene}, der dâ künne mite.\\ 
10 & sol ich begên noch strîtes site,\\ 
 & sô bindet \textbf{ûf mînen helm} und gêt \textit{i}r \textit{h}in;\\ 
 & den lîp ich gerne wernde bin."\\ 
 & si jâhen: "ir sît \textit{nû} strîtes vrî,\\ 
 & hêrre, lât uns iu wesen bî.\\ 
15 & wan einiu sol gewinnen\\ 
 & an vier küniginnen\\ 
 & daz boten brôt, ir lebt noch.\\ 
 & man sol iu bereiten \textbf{doch}\\ 
 & gemach und erzenîe klâr\\ 
20 & und \textbf{wol} mit triuwen nemen war,\\ 
 & mit salben sô gehiure,\\ 
 & diu vür die quatschiure\\ 
 & und vü\textit{r d}ie wunden ein genist\\ 
 & mit senfte helflîchen ist."\\ 
25 & der meide einiu danne spranc\\ 
 & sô balde, daz si nindert hanc.\\ 
 & \textbf{si} brâhte zuo hove mære,\\ 
 & daz er bî leben wære\\ 
 & "und alsô \textbf{lobelîche},\\ 
30 & daz er uns vreude rîche\\ 
\end{tabular}
\scriptsize
\line(1,0){75} \newline
Q R W V U \newline
\line(1,0){75} \newline
\textbf{1} \textit{Initiale} Q   $\cdot$ \textit{Capitulumzeichen} R  \textbf{13} \textit{Initiale} V  \textbf{25} \textit{Initiale} W  \newline
\line(1,0){75} \newline
\textbf{1} \textit{Die Verse 553.1-599.30 fehlen} U  \textbf{2} trœstet] troͯsten R  $\cdot$ armen] arme W V \textbf{3} iuweren] v́wer V  $\cdot$ sî] sin V \textbf{4} wir] mit Q  $\cdot$ wesen] werden W \textbf{8} mîne] úwer R \textbf{9} etwene] Iemant W Etswen V  $\cdot$ dâ künne] do kúnne W [*]: do koͤnne V \textbf{10} sol] So R [S*]: Sol V \textbf{11} bindet] bindet mir W (V)  $\cdot$ mînen helm] \textit{om.} W  $\cdot$ ir hin] herein Q [*]: ir hin V \textbf{12} wernde] weren W \textbf{13} jâhen] iahet W  $\cdot$ nû] in Q [*n *]: nv V \textbf{15} einiu sol] eine so R \textbf{17} boten brôt] bottenbrot V W \textbf{18} doch] och W \textbf{19} erzenîe klâr] essen klare W \textbf{23} vür] fur in Q \textbf{24} [M*]: Mit senfte helfeclichen [list]: ist V  $\cdot$ senfte] senfftten R (W)  $\cdot$ helflîchen] haimlichen W \textbf{25} einiu] eine R \textbf{27} si] Dú R  $\cdot$ mære] die mere W \textbf{28} Das der ritter lebendig were R  $\cdot$ er] gawan W \textbf{29} lobelîche] lebeleiche W (V) \textbf{30} uns vreude rîche] vns froͯden Riche R vnd froͤdenreiche W [*]: vnz froͤiden riche V \newline
\end{minipage}
\end{table}
\end{document}
