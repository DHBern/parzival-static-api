\documentclass[8pt,a4paper,notitlepage]{article}
\usepackage{fullpage}
\usepackage{ulem}
\usepackage{xltxtra}
\usepackage{datetime}
\renewcommand{\dateseparator}{.}
\dmyyyydate
\usepackage{fancyhdr}
\usepackage{ifthen}
\pagestyle{fancy}
\fancyhf{}
\renewcommand{\headrulewidth}{0pt}
\fancyfoot[L]{\ifthenelse{\value{page}=1}{\today, \currenttime{} Uhr}{}}
\begin{document}
\begin{table}[ht]
\begin{minipage}[t]{0.5\linewidth}
\small
\begin{center}*D
\end{center}
\begin{tabular}{rl}
\textbf{544} & \begin{large}V\end{large}on dem wazzer ûfez lant.\\ 
 & er gie unt truog \textbf{ûf} sîner hant\\ 
 & ein mûzersprinzelîn al grâ.\\ 
 & ez was sîn reht lêhen dâ,\\ 
5 & swer tjustierte ûf dem plân,\\ 
 & daz er daz ors solte hân\\ 
 & jenes, der dâ \textbf{læge};\\ 
 & unt \textbf{disem}, der siges pflæge,\\ 
 & des hende solt er nîgen\\ 
10 & \textbf{und} sînen prîs niht verswîgen.\\ 
 & Sus zinsete man im \textbf{blüemîn} velt.\\ 
 & \textbf{daz} was sîn \textbf{beste} huoben gelt,\\ 
 & oder ob sîn mûzersprinzelîn\\ 
 & eine galandern lêrte pîn.\\ 
15 & von anders nihtiu gienc sîn pfluoc,\\ 
 & daz dûht in urbor genuoc.\\ 
 & Er was \textbf{geborn} von rîters art,\\ 
 & \textbf{mit guoten zühten} wol bewart.\\ 
 & \textbf{er gienc} zuo Gawane.\\ 
20 & den zins von dem plâne,\\ 
 & \textbf{den} \textbf{iesch} er \textbf{zühteclîche}.\\ 
 & Gawan, der ellens rîche,\\ 
 & sprach: "hêrre, i\textbf{ne} wart nie koufman.\\ 
 & ir megt mich zolles wol erlân."\\ 
25 & Des schiffes \textbf{hêrre} wider sprach:\\ 
 & "hêrre, sô manec vrouwe sach,\\ 
 & daz iu der prîs ist hie geschehen.\\ 
 & ir sult \textbf{mir} mînes rehtes jehen."\\ 
 & "hêrre, tuot mir reht bekant."\\ 
30 & "ze rehter tjost hât iwer hant\\ 
\end{tabular}
\scriptsize
\line(1,0){75} \newline
D Fr7 \newline
\line(1,0){75} \newline
\textbf{1} \textit{Initiale} D  \textbf{11} \textit{Majuskel} D  \textbf{17} \textit{Majuskel} D  \textbf{25} \textit{Majuskel} D  \newline
\line(1,0){75} \newline
\newline
\end{minipage}
\hspace{0.5cm}
\begin{minipage}[t]{0.5\linewidth}
\small
\begin{center}*m
\end{center}
\begin{tabular}{rl}
 & von dem wazzer ûf daz lant.\\ 
 & er gienc und truoc \textbf{an} sîner hant\\ 
 & ein mûzersprinzelîn al grâ.\\ 
 & ez was sîn reht \textit{l}êhen dâ,\\ 
5 & wer justierte ûf dem plân,\\ 
 & daz er daz ros solte hân\\ 
 & jenes, der d\textit{â} \textbf{gelæge};\\ 
 & und \textbf{dem}, der \textbf{d\textit{â}} siges pflæge,\\ 
 & des hende solte er nîgen\\ 
10 & \textbf{und} sînen prîs niht verswîgen.\\ 
 & sus zinset man i\textit{m} \textbf{bluomen} velt.\\ 
 & \textbf{daz} was sîn \textbf{bestez} huoben gelt,\\ 
 & oder ob sîn mûzersprinzelîn\\ 
 & \dag âne\dag  galander lêrte pîn.\\ 
15 & von anders niht gie sîn pfluoc,\\ 
 & daz dûht in urbor genuoc.\\ 
 & er was \textbf{erborn} von ritters art.\\ 
 & \textbf{mit guoten zühten} wol bewart\\ 
 & \textbf{gienc er} zuo Gawan.\\ 
20 & den zins von \textit{dem} plân\\ 
 & \textbf{vordert} er \textbf{zühteclîch}.\\ 
 & Gawan, der ellens rîch,\\ 
 & spr\textit{a}ch: "hêrre, ich wart nie koufman.\\ 
 & ir müg\textit{t m}ich zolles wol erlân."\\ 
25 & des schiffes \textbf{hêrre} wider sprach:\\ 
 & "hêrre, sô manic vrouwe sach,\\ 
 & daz iu der prîs ist hie geschehen.\\ 
 & ir solt \textbf{mir} mînes rehtes jehen."\\ 
 & "hêrre, tuot mir reht bekant."\\ 
30 & "zuo rehter just het iuwer hant\\ 
\end{tabular}
\scriptsize
\line(1,0){75} \newline
m n o \newline
\line(1,0){75} \newline
\newline
\line(1,0){75} \newline
\textbf{2} an] vff n (o) \textbf{4} lêhen] hehen m  $\cdot$ dâ] do n \textbf{6} daz er] Do er n \textbf{7} dâ] do m n o \textbf{8} dâ] do m \textit{om.} n o \textbf{9} solte er nîgen] soltú ernigen o \textbf{11} im] in m o \textbf{14} galander] galadern n (o) \textbf{16} daz] Do o \textbf{17} was] wart n \textbf{20} dem] \textit{om.} m \textbf{23} sprach] Sprech m \textbf{24} mügt mich] muͯgt múgt mich m \textbf{28} rehtes] rechten n o \newline
\end{minipage}
\end{table}
\newpage
\begin{table}[ht]
\begin{minipage}[t]{0.5\linewidth}
\small
\begin{center}*G
\end{center}
\begin{tabular}{rl}
 & \begin{large}V\end{large}on dem wazzer ûfez lant.\\ 
 & er gienc unde truoc \textbf{ûf} sîner hant\\ 
 & ein mûz\textit{er}sprinzelîn al grâ.\\ 
 & ez was sîn reht lêhen dâ,\\ 
5 & swer tjostierte ûf dem plân,\\ 
 & daz er daz ors solde hân\\ 
 & jenes, der dâ \textbf{læge};\\ 
 & unde \textbf{dises}, der siges pflæge,\\ 
 & des hende solde er nîgen,\\ 
10 & sînen brîs niht verswîgen.\\ 
 & sus zinst man im \textbf{bluomen} velt.\\ 
 & \textbf{daz} was sîn \textbf{beste} huoben gelt,\\ 
 & ode obe sîn mûz\textit{er}sprinzelîn\\ 
 & ein galander lêrte pîn.\\ 
15 & von anders niht gienc sîn pfluoc,\\ 
 & daz dûhte in urbor genuoc.\\ 
 & \textit{e}r was \textbf{geborn} von rîters art,\\ 
 & \textbf{an guoter zuht} wol bewart.\\ 
 & \textbf{er gienc} zuo Gawane.\\ 
20 & den zins von dem plâne,\\ 
 & \textbf{den} \textbf{iesch} er \textbf{zühteclîche}.\\ 
 & Gawan, der ellens rîche,\\ 
 & sprach: "hêrre, ich\textbf{ne} wart nie koufman.\\ 
 & ir muget mich zolles wol erlân."\\ 
25 & des schiffes \textbf{hêrre} wider sprach:\\ 
 & "hêrre, sô manic vrouwe sach,\\ 
 & daz iu der brîs \textit{ist hie} geschehen.\\ 
 & ir sult \textbf{mir} mînes rehtes jehen."\\ 
 & "hêrre, tuot mir reht bekant."\\ 
30 & "ze rehter tjost hât iuwer hant\\ 
\end{tabular}
\scriptsize
\line(1,0){75} \newline
G I L M Z \newline
\line(1,0){75} \newline
\textbf{1} \textit{Initiale} G L Z  \textbf{15} \textit{Initiale} I  \textbf{25} \textit{Initiale} I  \newline
\line(1,0){75} \newline
\textbf{3} mûzersprinzelîn] mûzsprinzelin G (I) \textbf{4} reht] rechtes L rechte M \textbf{5} swer] Wer L M  $\cdot$ tjostierte] tiostieret L tiostiere M \textbf{6} hân] lan Z \textbf{8} dises] diseme L (M) (Z)  $\cdot$ der] der da I (Z) dersz L \textbf{10} sînen] sins I Vnd sinen L (Z) Vff Sinen M  $\cdot$ brîs] prises I \textbf{11} im] im sin I nuͯ daz L  $\cdot$ bluomen] bloͮmin G \textbf{12} beste] bester I  $\cdot$ huoben] huͯbe L (Z) \textbf{13} ode obe] Vnd ob L Ab her M  $\cdot$ mûzersprinzelîn] mvzsprinzelin G (I) \textbf{14} ein] Einen L  $\cdot$ pîn] vahin L \textbf{16} urbor] vbrigz I \textbf{17} er was] Er was Er was G \textbf{18} an guoter zuht] An guͦtir [An]: zuht G an Guͤtcher zuhte I Mit guͯten zuͯchten L (M) (Z) \textbf{19} Gawane] Gawan I \textbf{20} den zins] Vnd iesch L Des zins Z \textbf{21} den iesch er] Den zinsz L \textbf{23} hêrre] \textit{om.} I  $\cdot$ ichne] ich I Z  $\cdot$ wart] enwart L \textbf{27} ist hie] hie ist G \textbf{28} rehtes] rechtens L rechten M  $\cdot$ jehen] Geben I \textbf{29} tuot] tu Z  $\cdot$ reht] vwer reht L  $\cdot$ bekant] erchant I \newline
\end{minipage}
\hspace{0.5cm}
\begin{minipage}[t]{0.5\linewidth}
\small
\begin{center}*T
\end{center}
\begin{tabular}{rl}
 & von dem wazzer ûfez lant.\\ 
 & er gienc unde truoc \textbf{ûf} sîner hant\\ 
 & ein mûzersprinz\textit{e}lîn al grâ.\\ 
 & ez was sîn reht lêhen dâ,\\ 
5 & swer tjostierte ûf dem plân,\\ 
 & daz er daz ors solte hân\\ 
 & Jenes, der dâ \textbf{læge};\\ 
 & unde der \textbf{des} siges pflæge,\\ 
 & des hende solt er nîgen\\ 
10 & \textbf{unde} sînen prîs niht verswîgen.\\ 
 & sus zinset man im \textbf{bluomen} velt.\\ 
 & \textbf{ez} was sîn \textbf{beste} huoben gelt,\\ 
 & oder ob sîn mûzersprinz\textit{e}lîn\\ 
 & eine galander lêrte pîn.\\ 
15 & von anders niht\textit{iu} gie sîn pfluoc,\\ 
 & daz dûht in urbor genuoc.\\ 
 & er was \textbf{geborn} von rîters art,\\ 
 & \textbf{mit guoten zühten} wol bewart.\\ 
 & \textbf{\textit{\begin{large}E\end{large}}r gie} zuo Gawane.\\ 
20 & den zins von dem plâne\\ 
 & \textbf{iesch} er \textbf{gezogenlîche}.\\ 
 & Gawan, der ellens rîche,\\ 
 & sprach: "hêrre, i\textbf{ne} wart nie koufman.\\ 
 & ir muget mich zolles wol erlân."\\ 
25 & des schiffes \textbf{meister} wider sprach:\\ 
 & "hêrre, sô manec vrouwe \textbf{ez} sach,\\ 
 & daz iu der prîs ist hie geschehen.\\ 
 & ir sult mînes rehtes jehen."\\ 
 & "hêrre, tuot mir reht bekant."\\ 
30 & "ze rehter tjost hât iuwer hant\\ 
\end{tabular}
\scriptsize
\line(1,0){75} \newline
T U V W O Q R Fr40 \newline
\line(1,0){75} \newline
\textbf{1} \textit{Initiale} O Fr40  \textbf{7} \textit{Majuskel} T  \textbf{19} \textit{Initiale} T U V  \textbf{25} \textit{Initiale} R  \newline
\line(1,0){75} \newline
\textbf{1} von] ÷on O  $\cdot$ ûfez] vntz auff das W \textbf{3} ein] Er Q  $\cdot$ mûzersprinzelîn] mvzer sprinzerlin T musz springtelin Q muͯsnerlin spewer R \textbf{4} reht] rehte O  $\cdot$ lêhen] leben Fr40 \textbf{5} swer] Wer U W Q R  $\cdot$ tjostierte] streit R \textbf{6} solte] soͤlte V \textbf{7} Jenes] Eines U  $\cdot$ dâ] do U W dag Q \textbf{8} der] \textit{om.} U er Q  $\cdot$ siges] siches Q \textbf{11} im] in W  $\cdot$ bluomen] bluͦmin U daz O \textbf{12} beste huoben gelt] bestes huͤben gelt W bester hvͦbe gelt O best huͦbgelt R \textbf{13} mûzersprinzelîn] mvzersprinzerlin T músser sperwerlin R mouzer sprintellein Fr40 \textbf{14} eine] Einen O \textbf{15} von] Wan O  $\cdot$ anders] ander W  $\cdot$ nihtiu] nihte T (R) \textbf{16} urbor] vrbos W gar R \textbf{19} Er] ÷r T  $\cdot$ Gawane] Gaweine R \textbf{20} dem] \textit{om.} U \textbf{21} \textit{Versfolge 544.22-21} U   $\cdot$ iesch] Hieß W  $\cdot$ gezogenlîche] zúchtigleiche W (O) (Q) (R) (Fr40) \textbf{22} Gawan] Gawain R  $\cdot$ der ellens] den eren Q \textbf{23} hêrre] \textit{om.} O  $\cdot$ ine] ich W O Q (R) Fr40 \textbf{25} meister] herre W Q Fr40 \textbf{26} ez] \textit{om.} U V W O Q R Fr40 \textbf{27} ist hie] hy ist Fr40 \textbf{28} mînes] mir mines O (Q) (R) (Fr40)  $\cdot$ rehtes] herze Q Rechtten R  $\cdot$ jehen] mir iehen V \textbf{29} bekant] erkant W (O) Q Fr40 \textbf{30} rehter tjost] rechttem strit R \newline
\end{minipage}
\end{table}
\end{document}
