\documentclass[8pt,a4paper,notitlepage]{article}
\usepackage{fullpage}
\usepackage{ulem}
\usepackage{xltxtra}
\usepackage{datetime}
\renewcommand{\dateseparator}{.}
\dmyyyydate
\usepackage{fancyhdr}
\usepackage{ifthen}
\pagestyle{fancy}
\fancyhf{}
\renewcommand{\headrulewidth}{0pt}
\fancyfoot[L]{\ifthenelse{\value{page}=1}{\today, \currenttime{} Uhr}{}}
\begin{document}
\begin{table}[ht]
\begin{minipage}[t]{0.5\linewidth}
\small
\begin{center}*D
\end{center}
\begin{tabular}{rl}
\textbf{422} & \begin{large}D\end{large}es hân ich mich gein iu bedâht."\\ 
 & dô sprach der künec Vergulaht:\\ 
 & "swîget iwerer wehselmære!\\ 
 & ez ist \textbf{mir} von iu bêden swære,\\ 
5 & daz ir der \textbf{worte} sît sô vrî.\\ 
 & ich bin iu al ze nâhen bî\\ 
 & ze sus getânem \textbf{gebrehte}.\\ 
 & ez stêt \textbf{mir noch iu} niht rehte."\\ 
 & \textbf{\begin{large}D\end{large}iz} was ûf dem palas,\\ 
10 & \textbf{al} dâ sîn swester komen was.\\ 
 & bî \textbf{ir} stuont hêr Gawan\\ 
 & unt \textbf{manec ander} werder man.\\ 
 & der künec sprach zer swester sîn:\\ 
 & "nû nim den gesellen dîn\\ 
15 & unt \textbf{ouch} den lantgrâven zuo dir.\\ 
 & die mir guotes gunnen, \textbf{die} gên mit mir,\\ 
 & unt \textbf{râtet} mirz wægeste, \textbf{waz} ich tuo."\\ 
 & si sprach: "dâ lege dîne \textbf{triwe} zuo."\\ 
 & Nû \textbf{gêt} der künec an sînen rât.\\ 
20 & diu küneginne genomen hât\\ 
 & ir vetern sun unt ir gast.\\ 
 & d\textit{az} dritte was der sorgen last.\\ 
 & ân alle missewende\\ 
 & nam si Gawanen mit \textbf{ir} hende\\ 
25 & unt vuorten, dâ si \textbf{wolte} wesen.\\ 
 & si sprach zim: "wæret ir niht genesen,\\ 
 & \textbf{des} heten \textbf{schaden} elliu lant."\\ 
 & an der küneginne hant\\ 
 & gienc des werden Lotes sun.\\ 
30 & er mohtez \textbf{ouch dô} vil gerne tuon.\\ 
\end{tabular}
\scriptsize
\line(1,0){75} \newline
D Fr1 Fr5 \newline
\line(1,0){75} \newline
\textbf{1} \textit{Initiale} D Fr5  \textbf{3} \textit{Initiale} Fr1  \textbf{9} \textit{Capitulumzeichen} Fr5   $\cdot$ \textit{Majuskel} D  \textbf{19} \textit{Capitulumzeichen} Fr5   $\cdot$ \textit{Majuskel} D  \newline
\line(1,0){75} \newline
\textbf{2} Vergulaht] Vergvlaht D \textbf{11} Gawan] Gauwan Fr5 \textbf{12} manec ander] anders manech Fr1 \textbf{14} nim] nim dv Fr1 \textbf{15} ouch] \textit{om.} Fr1 \textbf{16} mir] oͮch mir Fr5  $\cdot$ gên] gengin Fr5 \textbf{21} ir vetern sun] den lantgraven Fr1 \textbf{22} daz dritte] des dritte D Des froͮde Fr5 \textbf{24} Gawanen] Gawann D Gauwan Fr5  $\cdot$ mit ir] bi der Fr1 \textbf{26} zim] vnt Fr1 \textbf{29} Lotes] Lôts Fr1 lotis Fr5 \textbf{30} er mohtez ouch] daz moht er Fr1 \newline
\end{minipage}
\hspace{0.5cm}
\begin{minipage}[t]{0.5\linewidth}
\small
\begin{center}*m
\end{center}
\begin{tabular}{rl}
 & des hân ich mich gegen iu bed\textit{âht}."\\ 
 & dô sprach der künic Vergulaht:\\ 
 & "swîget iuwer wehselmære!\\ 
 & ez ist von iu beiden swære,\\ 
5 & daz ir der \textbf{worte} sît sô vrî.\\ 
 & ich bin iu al ze nâhen bî\\ 
 & ze sus getâne\textit{m} \textbf{gebreh\textit{t}e}.\\ 
 & ez stât \textbf{mir noch iu} niht rehte."\\ 
 & \textbf{daz} was ûf dem palas,\\ 
10 & \textbf{al}dâ sîn swester komen was.\\ 
 & bî \textbf{ir} stuont hêr Gawan\\ 
 & und \textbf{manic ander} \textit{werder} man.\\ 
 & der künic sprach zuo der swester sîn:\\ 
 & "nû nim den gesellen dîn\\ 
15 & und \textbf{ouch} den lantgrâven zuo dir.\\ 
 & die mir guote\textit{s g}unnen, gên mit mir,\\ 
 & und \textbf{râtet} mir daz wægeste, \textbf{waz} ich tuo."\\ 
 & si sprach: "dâ lege dîne \textbf{triuwe} zuo."\\ 
 & \begin{large}N\end{large}û \textbf{gât} der künic an sînen rât.\\ 
20 & diu künigîn genomen hât\\ 
 & ir veteren sun und ir gast.\\ 
 & daz drite was der sorgen last.\\ 
 & âne alle missewende\\ 
 & nam si Gawanen mit \textbf{der} hende\\ 
25 & und vuorte in, dâ si \textbf{solte} wesen.\\ 
 & si sprach zuo ime: "wæret ir niht genesen,\\ 
 & \textbf{es} heten \textbf{schaden} alliu lant."\\ 
 & an der küniginne hant\\ 
 & gienc des werden Lotes sun.\\ 
30 & er mohte ez vil gerne tuon.\\ 
\end{tabular}
\scriptsize
\line(1,0){75} \newline
m n o \newline
\line(1,0){75} \newline
\textbf{19} \textit{Initiale} m n o  \newline
\line(1,0){75} \newline
\textbf{1} bedâht] bed m gedacht n \textbf{2} Vergulaht] vergulacht n vergulat o \textbf{5} worte] wuͯrt o \textbf{7} getânem gebrehte] getonen gebreche m \textbf{11} Gawan] gawann o \textbf{12} ander werder man] ander [w]: man m \textbf{16} guotes gunnen] guttes ellen gvnnen m  $\cdot$ gên] got n go o \textbf{17} daz wægeste] hie n o \textbf{18} lege dîne triuwe] lige din getruwe n \textbf{21} ir veteren] Jrs vettern n Jr vittern o \textbf{24} Gawanen] gawannen m \textbf{25} dâ] do n o \textbf{27} es] Des n o  $\cdot$ schaden] scheiden o \textbf{28} küniginne] kv́nniginnen n (o) \textbf{29} des] dasz o  $\cdot$ Lotes] lots m luchs n lúchtes o \textbf{30} mohte] moͯchte n  $\cdot$ ez] es do n o \newline
\end{minipage}
\end{table}
\newpage
\begin{table}[ht]
\begin{minipage}[t]{0.5\linewidth}
\small
\begin{center}*G
\end{center}
\begin{tabular}{rl}
 & des hân ich mich gein iu bedâht."\\ 
 & dô sprach der künic Vergulaht:\\ 
 & "swîget iwere wehselmære!\\ 
 & ez ist \textbf{mir} von iu beiden swære,\\ 
5 & daz ir der \textbf{worte} sît sô vrî.\\ 
 & ich bin iu alze nâhen bî\\ 
 & ze sus getânem \textbf{\textit{ge}br\textit{e}hte}.\\ 
 & ez stêt \textbf{iu noch mir} niht rehte."\\ 
 & \textbf{diz} was ûf dem palas,\\ 
10 & dâ sîn swester komen was.\\ 
 & bî \textbf{der} stuont hêr Gawan\\ 
 & unde \textbf{anders manic} werder man.\\ 
 & der künic sprach zer swester sîn:\\ 
 & "nû nim den gesellen dîn\\ 
15 & unt den lantgrâven zuo dir.\\ 
 & die mir guotes gunnen, \textbf{die} gên mit mir,\\ 
 & unde \textbf{rât} mir daz wægeste, \textbf{daz} ich tuo."\\ 
 & si sprach: "dâ lege dîne \textbf{triwe} zuo."\\ 
 & nû \textbf{\textit{g}ê\textit{t}} der künic an sînen rât.\\ 
20 & diu künigîn genomen hât\\ 
 & ir veteren sun unde ir gast.\\ 
 & daz drite was der sorgen last.\\ 
 & âne \textit{alle} missewende\\ 
 & \textit{nam si} Gawanen mit \textbf{ir} hende\\ 
25 & unt vuort in, dâ si \textbf{wolt} wesen.\\ 
 & si sprach zim: "wæret ir niht genesen,\\ 
 & \textbf{des} heten \textbf{schaden} elliu lant."\\ 
 & an der küniginne hant\\ 
 & gienc des werden Lotes sun.\\ 
30 & er mohtz \textbf{ouch dô} vil gerne tuon.\\ 
\end{tabular}
\scriptsize
\line(1,0){75} \newline
G I O L M Q R Z \newline
\line(1,0){75} \newline
\textbf{1} \textit{Initiale} O L Z   $\cdot$ \textit{Capitulumzeichen} R  \textbf{3} \textit{Initiale} M  \textbf{15} \textit{Initiale} I  \textbf{18} \textit{Capitulumzeichen} R  \textbf{19} \textit{Initiale} R  \newline
\line(1,0){75} \newline
\textbf{1} des] ÷es O  $\cdot$ ich] \textit{om.} Z  $\cdot$ bedâht] verdaht L gedacht Q \textbf{2} dô] Da M  $\cdot$ Vergulaht] virgulaht I vergvlaht O (L) Z vergulacht M Q R \textbf{4} iu] \textit{om.} L  $\cdot$ beiden] \textit{om.} Q \textbf{5} worte] mere R  $\cdot$ sô] \textit{om.} I \textbf{6} ich bin iu] Jr sit mir Z \textbf{7} getânem] getanen Z  $\cdot$ gebrehte] brahte G \textbf{8} ez] Ez en M (Q)  $\cdot$ iu noch mir] mir noch ev I (O) (M) (Q) (R) Z mir doch L  $\cdot$ rehte] reachte L \textbf{10} dâ] Al da O (L) (M) (R) (Q) Z \textbf{11} der] ir O L M Q R Z  $\cdot$ Gawan] Gawin R \textbf{12} Vnd menger ander Rittere fin R  $\cdot$ anders manic werder] ander manc werder I manich ander wærder O (M) (Q) (Z) manig wert ander L \textbf{15} unt] vnde ovch O (L) (M) (Q) (R) (Z)  $\cdot$ zuo] denn zu R \textbf{16} gên] gien I gond R  $\cdot$ mit] zuͤ I (Z) \textbf{17} rât] raten I O Q (R)  $\cdot$ wægeste] beste M  $\cdot$ daz ich] waz ich I O L (M) (R) Z \textbf{18} lege] ker I  $\cdot$ zuo] darzcu M \textbf{19} gêt] gie G \textbf{21} ir veteren] Jr vetir M Jren vetern Q Jrs vettern R  $\cdot$ ir gast] sinen gast Z \textbf{23} alle] \textit{om.} G  $\cdot$ missewende] wisse wende Q \textbf{24} nam si] si nam G  $\cdot$ Gawanen] [gawan]: gawanen G Gawan I O Z Gawinen R  $\cdot$ mit ir] bi der I (Z) mit siner R \textbf{25} dâ] do Q  $\cdot$ wolt] wolten R \textbf{26} zim] \textit{om.} I \textbf{27} des] De R  $\cdot$ schaden] schande R \textbf{28} küniginne] konnige M kunginnen R \textbf{29} des werden] der werde I (L) \textbf{30} dô] da O L M \textit{om.} Z  $\cdot$ vil] \textit{om.} R \newline
\end{minipage}
\hspace{0.5cm}
\begin{minipage}[t]{0.5\linewidth}
\small
\begin{center}*T
\end{center}
\begin{tabular}{rl}
 & des hân ich mich gegen iu bedâht."\\ 
 & Dô sprach der künec Vergulaht:\\ 
 & "swîget iuwerre wehselmære!\\ 
 & ez ist \textbf{mir} von iu beiden swære,\\ 
5 & daz ir der \textbf{mære} sît sô vrî.\\ 
 & ich bin iu alze nâhe bî\\ 
 & ze sus getânem \textbf{rehte}.\\ 
 & e\textit{z}\textbf{n} stêt \textbf{mir noch iu} niht rehte."\\ 
 & \textbf{\begin{large}D\end{large}iz} was ûf dem palas,\\ 
10 & \textbf{al} dar sîn swester komen was.\\ 
 & bî \textbf{ir} stuont hêr Gawan\\ 
 & unde \textbf{ander manec} wert man.\\ 
 & Der künec sprach zer swester sîn:\\ 
 & "nû nim den gesellen dîn\\ 
15 & unde den lantgrâven ze dir.\\ 
 & die mir guotes gunnen, \textbf{die} gân mit mir\\ 
 & unde \textbf{râten} mirz wægeste, \textbf{waz} ich tuo."\\ 
 & Si sprach: "dâ lege \textit{d}îne \textbf{sinne} zuo."\\ 
 & Nû \textbf{gie} der künec an sînen rât.\\ 
20 & diu künegîn genomen hât\\ 
 & ir vetern sun unde ir gast.\\ 
 & daz drite was der sorgen last.\\ 
 & âne alle missewende\\ 
 & nam si Gawanen mit \textbf{ir} hende\\ 
25 & unde vuortin, dâ si \textbf{wolte} wesen.\\ 
 & si sprach \textbf{hin} zim: "wæret ir niht genesen,\\ 
 & \textbf{des} heten \textbf{schande} alliu lant."\\ 
 & An der küneginne hant\\ 
 & gie des werden Lotes suon.\\ 
30 & er mohtez \textbf{ouch dô} vil gerne tuon.\\ 
\end{tabular}
\scriptsize
\line(1,0){75} \newline
T U V W \newline
\line(1,0){75} \newline
\textbf{2} \textit{Majuskel} T  \textbf{9} \textit{Initiale} T U  \textbf{13} \textit{Majuskel} T   $\cdot$ \textit{Initiale} W  \textbf{18} \textit{Majuskel} T  \textbf{19} \textit{Majuskel} T  \textbf{28} \textit{Majuskel} T  \newline
\line(1,0){75} \newline
\textbf{2} künec] \textit{om.} U  $\cdot$ Vergulaht] vergvlaht T vergulacht U W \textbf{6} [J*]: Jch bin v́ch alzenahe bi V \textbf{7} getânem] getame U  $\cdot$ rehte] gebrechte U (V) W \textbf{8} ezn] esn T Es W  $\cdot$ rehte] zuͦ rechte W \textbf{11} ir] der V \textbf{12} ander manec wert] manec ander werder U (W) manig werder ander V \textbf{15} den] \textit{om.} U oͮch den V (W) \textbf{16} die gân] [*]: die gen V gan W \textbf{17} râten] ratet W  $\cdot$ ich] \textit{om.} U \textbf{18} dîne] sine T dem W  $\cdot$ sinne] trv́we V treúwer W \textbf{19} gie] get U V W \textbf{21} ir vetern] Jrs vettern V (W) \textbf{24} Gawanen] gawan W  $\cdot$ mit ir] [*]: mit ir V \textbf{25} dâ] do U V W  $\cdot$ wolte] [*olte]: solte V \textbf{26} hin] \textit{om.} V W \textbf{27} heten] \textit{om.} W  $\cdot$ schande] schaden U W \textbf{29} werden] kv́niges V  $\cdot$ Lotes] lottes W \textbf{30} ouch] \textit{om.} W  $\cdot$ dô] da V \newline
\end{minipage}
\end{table}
\end{document}
