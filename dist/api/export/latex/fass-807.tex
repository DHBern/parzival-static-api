\documentclass[8pt,a4paper,notitlepage]{article}
\usepackage{fullpage}
\usepackage{ulem}
\usepackage{xltxtra}
\usepackage{datetime}
\renewcommand{\dateseparator}{.}
\dmyyyydate
\usepackage{fancyhdr}
\usepackage{ifthen}
\pagestyle{fancy}
\fancyhf{}
\renewcommand{\headrulewidth}{0pt}
\fancyfoot[L]{\ifthenelse{\value{page}=1}{\today, \currenttime{} Uhr}{}}
\begin{document}
\begin{table}[ht]
\begin{minipage}[t]{0.5\linewidth}
\small
\begin{center}*D
\end{center}
\begin{tabular}{rl}
\textbf{807} & \begin{large}F\end{large}eirefiz \textbf{si vuorte} mit der hant,\\ 
 & \textbf{dô} si des wirtes muomen vant,\\ 
 & Repansen de Schoye, stên.\\ 
 & \textbf{dâ muose küssens vil} ergên.\\ 
5 & Dar zuo ir munt was \textbf{ê} sô rôt.\\ 
 & der leit von küssen nû die nôt,\\ 
 & daz ez mich müet unt ist mir leit,\\ 
 & daz ich niht hân \textbf{solhe} arbeit\\ 
 & vür si. wand si kom müediu zin.\\ 
10 & juncvrouwen vuorten ir vrouwen hin.\\ 
 & Die rîter \textbf{in} dem palas\\ 
 & beliben, der wol gekerzet was,\\ 
 & die harte liehte brunnen.\\ 
 & dô wart mit \textbf{zuht} begunnen\\ 
15 & \textbf{gereitschaft} gein dem Grâle.\\ 
 & den truog man zallem mâle\\ 
 & der diet niht durch schouwen vür,\\ 
 & niht wan \textbf{ze} hôchgezîte kür.\\ 
 & Durch daz si trôstes wânden,\\ 
20 & \textbf{dô} si sich vreuden ânden\\ 
 & des âbents umbe daz bluotige sper,\\ 
 & dô wart der Grâl durch helfe ger\\ 
 & vür getragen an der selben zît.\\ 
 & Parcifal si liez in \textbf{sorgen} sît.\\ 
25 & mit vreude er \textbf{wirt nû} vür getragen.\\ 
 & \textbf{ir} \textbf{sorge} ist \textbf{under} gar geslagen.\\ 
 & Dô diu künegîn ir reisegewant\\ 
 & ab gezôch unt \textbf{sich} gebant,\\ 
 & si kom, als \textbf{ez ir} wol gezam.\\ 
30 & Feirefiz \textbf{an} einer tür si nam.\\ 
\end{tabular}
\scriptsize
\line(1,0){75} \newline
D \newline
\line(1,0){75} \newline
\textbf{1} \textit{Initiale} D  \textbf{5} \textit{Majuskel} D  \textbf{11} \textit{Majuskel} D  \textbf{19} \textit{Majuskel} D  \textbf{27} \textit{Majuskel} D  \newline
\line(1,0){75} \newline
\textbf{3} Repansen de Schoye] Repansen de scoye D \newline
\end{minipage}
\hspace{0.5cm}
\begin{minipage}[t]{0.5\linewidth}
\small
\begin{center}*m
\end{center}
\begin{tabular}{rl}
 & Ferefiz \textbf{vuorte si} mit der hant,\\ 
 & \textbf{daz} si des wirtes muomen vant,\\ 
 & Repanse de schoyen, stên.\\ 
 & \textbf{d\textit{â} muoste küssens vil} ergên.\\ 
5 & dar zuo ir munt was \textbf{ê} sô rôt.\\ 
 & der leit von küssen nû die nôt,\\ 
 & daz ez mich müejet und ist mir leit,\\ 
 & daz ich niht hab \textbf{solich} arbeit\\ 
 & vür si. wan si kam müediu zuo in.\\ 
10 & \textbf{die} juncvrowen vuorten ir vrowen hin.\\ 
 & die ritter \textbf{in} dem palas\\ 
 & beliben, der wol gekerzet was,\\ 
 & die harte liehte brunnen.\\ 
 & dô wart mit \textbf{zuht} begunnen\\ 
15 & \textbf{bereitschaft} gege\textit{n} dem Grâl.\\ 
 & den truoc man zuo allem mâl\\ 
 & der di\textit{et} niht durch schouwen vür,\\ 
 & niht wan \textbf{zer} hôchzîte kür.\\ 
 & durch daz si trôstes wânden,\\ 
20 & \textbf{dô} si sich vröuden ânden\\ 
 & des âbendes umb daz bluotige sper,\\ 
 & dô wart der Grâl durch helfe ger\\ 
 & vür getragen an der selben zît.\\ 
 & Parcifal si liez in \textbf{sorgen} sît.\\ 
25 & mit vröude er \textbf{nû wirt} vür getragen.\\ 
 & \textbf{ir} \textbf{sorge} ist gar \textbf{hin dan} geslagen.\\ 
 & \begin{large}D\end{large}ô d\textit{iu} künigîn ir reiseg\textit{e}want\\ 
 & ab gezôch und \textbf{sich} gebant,\\ 
 & si kam, als \textbf{irz} wol gezam.\\ 
30 & Ferefiz \textbf{an} einer tür si nam.\\ 
\end{tabular}
\scriptsize
\line(1,0){75} \newline
m n o V V' W \newline
\line(1,0){75} \newline
\textbf{27} \textit{Initiale} m n V V' W  \newline
\line(1,0){75} \newline
\textbf{1} Ferefiz] Ferefis m o V Fferrefis n Ferevis V' Ferafis W  $\cdot$ vuorte si] sú fúrte n (o) (V') (W) sv́ [fvͦrten]: fvͦrte  V \textbf{2} daz] Do n o V V' W  $\cdot$ muomen] mvͦme V (W) \textbf{3} Sie wart enpfangen do gar wol V'  $\cdot$ Repanse de schoyen] Repanse descoien m Repansen descoyen n Repansen [die]: de scoien o [R*panscen]: Repanscen deschoyen V Vrepans de tschoye W \textbf{4} Als man frouwen uon rechte sol V'  $\cdot$ dâ] Do m n o V W  $\cdot$ muoste] mvͤste V  $\cdot$ küssens] kuͯssen n (o) \textbf{5} \textit{Die Verse 807.5-10 fehlen} V'   $\cdot$ ê] vil o \textit{om.} V ie W \textbf{7} und] das n \textbf{8} hab] \textit{om.} V \textbf{9} vür si wan] [F*de]: Fúr sv́ han wande V  $\cdot$ zuo] fúr n \textbf{12} gekerzet] bekertzet n (V) (V') (W) \textbf{15} gegen] gegem m \textbf{16} den] Der n  $\cdot$ truoc] entruͦc V  $\cdot$ zuo allem] [zeallem]: zeallen V \textbf{17} der diet niht] Der diechtte niht m (n) Den luten allen V' \textbf{18} wan] dan V'  $\cdot$ zer hôchzîte] zvͦ hochgeziten V zu hochziten V' \textbf{19} wânden] funden V' \textbf{20} An den heiligen stunden V' \textbf{21} des] Das n  $\cdot$ daz] des n \textbf{23} vür] Her fur V'  $\cdot$ an] in W \textbf{24} \textit{statt 807.24-26:} Jr sorge waz do hin gelit V'   $\cdot$ Parcifal] Parzefal V Partzifal W  $\cdot$ liez] liesse n \textbf{25} vröude] froͤiden V  $\cdot$ er nû wirt] ward er fúr W \textbf{27} diu] der m  $\cdot$ reisegewant] reise gawant m \textbf{29} irz] ez ir V ir V' W \textbf{30} Ferevisen an einer tur sie nam \textit{(Fortsetzung in 808.11)} V'  $\cdot$ Ferefiz] Ferefis m Ferrefis n Ferefisen V Ferafis W \newline
\end{minipage}
\end{table}
\newpage
\begin{table}[ht]
\begin{minipage}[t]{0.5\linewidth}
\small
\begin{center}*G
\end{center}
\begin{tabular}{rl}
 & \begin{large}F\end{large}eirafiz \textbf{si vuort} mit der hant,\\ 
 & \textbf{dâ} si des wirtes muomen vant,\\ 
 & Urrepanse de schoyen, stên.\\ 
 & \textbf{vil küssens muose dâ} ergên.\\ 
5 & dar zuo ir munt was sô rôt.\\ 
 & der leit von küssen nû die nôt,\\ 
 & daz ez mich müet unde ist mir leit,\\ 
 & daz ich niht hân \textbf{die} arbeit\\ 
 & vür si. wan si kom müediu zin.\\ 
10 & juncvrouwen vuorten ir vrouwen \textit{hin}.\\ 
 & die rîter \textbf{ûf} de\textit{m} palas\\ 
 & beliben, der wol ge\textit{k}er\textit{z}et was,\\ 
 & di\textit{e} \textit{h}arte lieht brunnen.\\ 
 & dâ wart mit \textbf{zühten} begunnen\\ 
15 & \textbf{bereitschaf\textit{t}} \textit{g}einm Grâl.\\ 
 & den truoc man zallem mâle\\ 
 & der diet niht durch schouwen vür,\\ 
 & niwan \textbf{durch} hôchzîte kür.\\ 
 & durch daz si trôstes wânden,\\ 
20 & \textbf{dô} si sich vröuden ânden\\ 
 & des âbendes umbe daz bluot\textit{ic} sper,\\ 
 & dâ wart der Grâl durch helfe ger\\ 
 & vür getragen an der selben zît.\\ 
 & Parzival si lie in \textbf{ruowen} sît.\\ 
25 & mit vröude er \textbf{wart nû} vür getragen.\\ 
 & \textbf{ir} \textbf{riwe} ist \textbf{under} gar geslagen.\\ 
 & dô diu küniginne ir reisegewant\\ 
 & abe gezôch unde \textbf{ir} gebant,\\ 
 & si kom, als \textbf{ez ir} wol gezam.\\ 
30 & Feirafiz \textbf{in} einer tür si nam.\\ 
\end{tabular}
\scriptsize
\line(1,0){75} \newline
G I L Z \newline
\line(1,0){75} \newline
\textbf{1} \textit{Initiale} G I L  \textbf{19} \textit{Initiale} I  \newline
\line(1,0){75} \newline
\textbf{1} \textit{Die Verse 806.1-807.24 fehlen} Z   $\cdot$ Feirafiz] Ferefis L \textbf{3} Urrepanse de schoyen] vrrepansede scoyen G vrrepanse de schoien I Vrrepansa de schoie L \textbf{4} küssens] chussen I \textbf{10} hin] \textit{om.} G \textbf{11} dem] den G \textbf{12} gekerzet] geziereit G \textbf{13} die harte] die cherzen harte G  $\cdot$ lieht] lýchte L \textbf{14} zühten] zuhte I \textbf{15} bereitschaft geinm] bereitschaft de geinm G Gereitschaft gein dem L \textbf{16} man] \textit{om.} L \textbf{17} der] Die L  $\cdot$ niht] nvwan L \textbf{18} hôchzîte] hochgezite I hochgeziten L \textbf{19} \textit{Die Verse 807.19-24 fehlen} L  \textbf{21} bluotic] ploͮte G \textbf{22} dâ] do I \textbf{24} Parzival] [parzifal]: Parzifal I  $\cdot$ ruowen] riwe I \textbf{25} vröude] freuden I (Z)  $\cdot$ er wart nû] wart er I er wirt nv L er wirt Z \textbf{27} dô] Da Z \textbf{28} ir] sich Z \textbf{30} Feirafiz] Ferefis L Feirefiz Z  $\cdot$ in einer tür si] sý an einer tvr L an einer tvͤr sie Z \newline
\end{minipage}
\hspace{0.5cm}
\begin{minipage}[t]{0.5\linewidth}
\small
\begin{center}*T
\end{center}
\begin{tabular}{rl}
 & Ferefis \textbf{vuorte si} mit der hant,\\ 
 & \textbf{d\textit{â}} si des wirtes muomen vant,\\ 
 & Repansen de joien, stên.\\ 
 & \textbf{vil küssens muose d\textit{â}} ergên.\\ 
5 & dar zuo ir munt was sô rôt.\\ 
 & der leit von küssen nû die nôt,\\ 
 & daz ez mich müet und ist mir leit,\\ 
 & daz ich niht hân \textbf{die} arbeit\\ 
 & vür si. wan si kam müediu zuo in.\\ 
10 & juncvrouwen vuorten ir vrouwen hin.\\ 
 & die rîter \textbf{ûf} dem palas\\ 
 & bliben, der wol gekerzet was,\\ 
 & die harte liehte brunnen.\\ 
 & dô wart mit \textbf{zuht} begunnen\\ 
15 & \textbf{gereitschaft} gein dem Grâle.\\ 
 & den truoc man zuo al\textit{lem} mâle\\ 
 & der diet niht durch schouwen vür,\\ 
 & niht wan \textbf{durch} hôchgezîte kür.\\ 
 & durch daz si trôstes wânden,\\ 
20 & \textbf{daz} \textit{si} sich vreuden ânden\\ 
 & des âbendes umb daz bluotic sper,\\ 
 & dô wart der Grâl durch helfe ger\\ 
 & vür getragen an der selben zît.\\ 
 & Parcifal \textit{si} liez in \textbf{sorgen} sît.\\ 
25 & mit vreuden er \textbf{wart nû} vür getragen.\\ 
 & \textbf{riuwe} ist \textbf{under} gar geslagen.\\ 
 & dô diu küneginne ir reisegewant\\ 
 & abe gezôch und \textbf{ir} gebant,\\ 
 & si kam, als \textbf{ez ir} wol gezam.\\ 
30 & Ferefis \textbf{an} einer tür si nam.\\ 
\end{tabular}
\scriptsize
\line(1,0){75} \newline
U Q R \newline
\line(1,0){75} \newline
\newline
\line(1,0){75} \newline
\textbf{1} Ferefis] feirefisz Q Fierfis R  $\cdot$ vuorte si] sie furte Q (R)  $\cdot$ mit] by R \textbf{2} dâ] Do U Q R \textbf{3} Repansen de joien] Repansen de ioien U Repansen deschoyen Q Rapensen dioninen R \textbf{4} dâ] do U Q R \textbf{8} ich niht] [icht]: ich nicht Q \textbf{9} kam müediu zuo in] kam múde zu ym Q kan muͯde sin R \textbf{12} gekerzet] geczieret R \textbf{13} Mit liechttern die helle brunnen R \textbf{14} zuht] zúchtten R  $\cdot$ begunnen] begrunen Q \textbf{16} allem] alme U \textbf{18} hôchgezîte] hochtzite Q (R) \textbf{20} si] \textit{om.} U \textbf{23} selben] selbe R \textbf{24} Parcifal] Parzifal U Partzifal Q Parczifal R  $\cdot$ si] \textit{om.} U \textbf{25} vreuden] frewde Q  $\cdot$ wart] wirt Q (R) \textbf{26} riuwe] Jr rew͑e Q (R)  $\cdot$ under gar] vnd gar Q gar vnder R  $\cdot$ geslagen] erslagen Q \textbf{28} gezôch] zeczoch R  $\cdot$ gebant] [gewand]: geband R \textbf{30} Ferefis] feizefisz Q Feirefis R \newline
\end{minipage}
\end{table}
\end{document}
