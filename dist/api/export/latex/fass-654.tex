\documentclass[8pt,a4paper,notitlepage]{article}
\usepackage{fullpage}
\usepackage{ulem}
\usepackage{xltxtra}
\usepackage{datetime}
\renewcommand{\dateseparator}{.}
\dmyyyydate
\usepackage{fancyhdr}
\usepackage{ifthen}
\pagestyle{fancy}
\fancyhf{}
\renewcommand{\headrulewidth}{0pt}
\fancyfoot[L]{\ifthenelse{\value{page}=1}{\today, \currenttime{} Uhr}{}}
\begin{document}
\begin{table}[ht]
\begin{minipage}[t]{0.5\linewidth}
\small
\begin{center}*D
\end{center}
\begin{tabular}{rl}
\textbf{654} & \textbf{\begin{large}S\end{large}i} enbietent iu dienst unt ir komen.\\ 
 & iwer botschaft wart von in vernomen\\ 
 & alsô werdeclîche,\\ 
 & \textbf{daz} arme und rîche\\ 
5 & sich vreuten, wand ich \textbf{tet in} kunt,\\ 
 & daz ir noch wæret wol gesunt.\\ 
 & Ich \textbf{vant} dâ hers ein wunder.\\ 
 & ouch wart diu tavelrunder\\ 
 & besetzet durch iwer botschaft.\\ 
10 & ob rîters prîs gewan ie kraft,\\ 
 & ich meine an werdecheite,\\ 
 & die \textbf{lenge unt ouch die breite}\\ 
 & treit iwer prîs die krône\\ 
 & ob anderen prîsen schône."\\ 
15 & Er seit im ouch, wie daz geschach,\\ 
 & daz er die küneginne \textbf{sprach},\\ 
 & unt waz im diu mit triwen riet.\\ 
 & er sagete im ouch von al der diet,\\ 
 & von rîtern unt von vrouwen,\\ 
20 & daz er \textbf{si} m\textit{ö}hte schouwen\\ 
 & ze Joflanze vo\textit{r} der zît,\\ 
 & ê würde sînes kampfes strît.\\ 
 & Gawans \textbf{sorge gar verswant},\\ 
 & \textbf{niht wan vreude} er \textbf{ime herzen vant}.\\ 
25 & \multicolumn{1}{l}{ - - - }\\ 
 & \multicolumn{1}{l}{ - - - }\\ 
 & al sîner sor\textit{g}e er \textbf{gar} vergaz;\\ 
 & er gienc \textbf{hin} wider und saz\\ 
 & unt was mit vreuden dâ ze hûs,\\ 
30 & unze daz der künec Artus\\ 
\end{tabular}
\scriptsize
\line(1,0){75} \newline
D \newline
\line(1,0){75} \newline
\textbf{1} \textit{Initiale} D  \textbf{7} \textit{Majuskel} D  \textbf{15} \textit{Majuskel} D  \newline
\line(1,0){75} \newline
\textbf{20} möhte] mohte D \textbf{21} vor] von D \textbf{25} \textit{Die Verse 654.25-26 fehlen} D  \textbf{27} sorge] sorde D \newline
\end{minipage}
\hspace{0.5cm}
\begin{minipage}[t]{0.5\linewidth}
\small
\begin{center}*m
\end{center}
\begin{tabular}{rl}
 & \textbf{die} enbiete\textit{n}t iu dienst und ir komen.\\ 
 & iuwer botschaft wart von in vernomen\\ 
 & alsô wirdeclîch,\\ 
 & \textbf{daz} arm und rîch\\ 
5 & sich vröweten, wen ich \textbf{in t\textit{e}t} kunt,\\ 
 & daz ir noch wæret wol gesunt.\\ 
 & ich \textbf{vant} d\textit{â} hers ein wunder.\\ 
 & ouch wart diu ta\textit{v}elrunder\\ 
 & besetzet durch iuwer botschaft.\\ 
10 & ob ritters prîs gewan ie kraft,\\ 
 & ich meine an wirdicheite,\\ 
 & die \textbf{lenge und ouch die breite}\\ 
 & treit iuwer prîs die krône\\ 
 & ob ander\textit{en} prîsen schône."\\ 
15 & er seit im ouch, wie daz geschach,\\ 
 & daz er die künigîn \textbf{sprach},\\ 
 & und waz i\textit{m d}iu mit triuwen riet.\\ 
 & er sagt im ouch von alder diet,\\ 
 & von rittern und von vrouwen,\\ 
20 & daz er \textbf{si} m\textit{ö}hte schouwen\\ 
 & zuo Joflanze vor der zît,\\ 
 & ê würde sînes kampfes strît.\\ 
 & Gawans \textbf{sorge gar verswant},\\ 
 & \textbf{niht \textit{wa}n vröude} er \textbf{im herzen vant}.\\ 
25 & \multicolumn{1}{l}{ - - - }\\ 
 & \multicolumn{1}{l}{ - - - }\\ 
 & alsîner sorge er \textbf{gar} vergaz;\\ 
 & er gienc \textbf{hin} wider und saz\\ 
 & und was mit vröuden dâ ze hûs,\\ 
30 & unz daz der künic Artus\\ 
\end{tabular}
\scriptsize
\line(1,0){75} \newline
m n o \newline
\line(1,0){75} \newline
\newline
\line(1,0){75} \newline
\textbf{1} enbietent] enbiettet m enbieten n \textbf{3} alsô] Also gar n \textbf{4} arm] ouch arm n \textbf{5} tet] tot m \textbf{6} wæret] worent o \textbf{7} dâ] do m n o \textbf{8} tavelrunder] taufelrunder m \textbf{10} gewan ie] gawan ir o \textbf{11} an] die n \textbf{14} anderen] ander m [anden]: anderen n \textbf{17} im diu] ẏm ẏe die m \textbf{18} sagt] sagete n  $\cdot$ alder] al:er o \textbf{20} möhte] mohtte m (o) \textbf{21} Joflanze] joflantz m n Joflancz o \textbf{24} wan] von m \textbf{25} \textit{Die Verse 654.25-26 fehlen} m n o  \textbf{29} dâ] do n o  $\cdot$ hûs] has o \textbf{30} Artus] artuͯs o \newline
\end{minipage}
\end{table}
\newpage
\begin{table}[ht]
\begin{minipage}[t]{0.5\linewidth}
\small
\begin{center}*G
\end{center}
\begin{tabular}{rl}
 & \textbf{\begin{large}S\end{large}i} enbietent iu dienst und ir komen.\\ 
 & iwer botschaft wart von in vernomen\\ 
 & als\textit{ô w}erdeclîche:\\ 
 & \textbf{der} arme unde \textbf{der} rîche\\ 
5 & sich vröuten, wan ich \textbf{tet in} kunt,\\ 
 & daz ir noch wæret wol gesunt.\\ 
 & ich \textbf{sach} dâ hers ein wunder.\\ 
 & ouch wart diu tavelrunder\\ 
 & besetzet durch iwer botschaft.\\ 
10 & obe rîters brîs gewan ie kraft,\\ 
 & ich meine an \textbf{langer} werdecheit,\\ 
 & die \textbf{sint iu alle dâ bereit}."\\ 
 & \multicolumn{1}{l}{ - - - }\\ 
 & \multicolumn{1}{l}{ - - - }\\ 
15 & er sagte im ouch, wie daz geschach,\\ 
 & daz er die küniginne \textbf{gesprach},\\ 
 & unde waz im diu mit triwen riet.\\ 
 & er sagte im ouch von al der diet,\\ 
 & von \textbf{den} rîtern unde von \textbf{den} vrouwen,\\ 
20 & daz er \textbf{die} m\textit{ö}hte schouwen\\ 
 & ze Tschofflanze vor der zît,\\ 
 & ê würde sînes kampfes strît.\\ 
 & Gawan \textbf{ûz sorgen in vröude trat};\\ 
 & \textbf{den knappen} er\textbf{z verswîgen bat}.\\ 
25 & \multicolumn{1}{l}{ - - - }\\ 
 & \multicolumn{1}{l}{ - - - }\\ 
 & al sîner sorgen er \textbf{dâ} vergaz;\\ 
 & er gienc \textbf{hin} wider unde saz\\ 
 & unde was mit vröuden dâ ze hûs,\\ 
30 & unze daz der künic Artus\\ 
\end{tabular}
\scriptsize
\line(1,0){75} \newline
G I L M Z \newline
\line(1,0){75} \newline
\textbf{1} \textit{Initiale} G L Z  \newline
\line(1,0){75} \newline
\textbf{1} Si] Dy M  $\cdot$ enbietent] enbuͯtet L  $\cdot$ dienst] ir dienst Z \textbf{2} botschaft] bot botschaft L \textbf{3} Also gar werdechliche G  $\cdot$ also werdechlichen I \textbf{4} Daz armen vnde richen I  $\cdot$ Daz arme vnd riche L (M) (Z) \textbf{5} vröuten] vreuwent I  $\cdot$ tet in] in tet Z \textbf{6} noch] \textit{om.} I  $\cdot$ wol gesunt] [noch]: gesunt M \textbf{7} hers] [her]: hêr I \textbf{9} besetzet] Besatz L \textbf{11} langer] [a]: langer G \textit{om.} Z \textbf{12} Die lenge vnd ouch die breite Z  $\cdot$ alle] allev I \textbf{13} \textit{Die Verse 654.13-14 fehlen} G I L M   $\cdot$ Treit ewer pris die krone Z \textbf{14} Ob anderm prise schone Z \textbf{15} sagte] seit I (L) (Z) \textbf{16} gesprach] sprach L \textbf{18} sagte] sagt I L Z \textbf{19} Von rittern vnd von vrowen L \textbf{20} möhte] mohte G (L) (M) Z \textbf{21} Tschofflanze] soffanze I Tschoflanze L schoflancze M tschofflantze Z \textbf{23} \textit{statt 654.23-24:} Gawans sorge gar verswant / Niht wan frevde er in dem hertzen vant / Gawan vz sorgen in frevde trat / Den knappen erz verswigen bat Z   $\cdot$ sorgen] sorge M \textbf{25} \textit{Die Verse 654.25-26 fehlen} G I L M  \textbf{27} sorgen] sorge Z  $\cdot$ dâ] \textit{om.} I L \textbf{30} unze] Vnd Z  $\cdot$ Artus] Artuͯs L \newline
\end{minipage}
\hspace{0.5cm}
\begin{minipage}[t]{0.5\linewidth}
\small
\begin{center}*T
\end{center}
\begin{tabular}{rl}
 & \textbf{si} enbieten\textit{t} iu dienst und ir komen.\\ 
 & iuwer botschaft wart von i\textit{n} vernomen\\ 
 & alsô werdeclîche,\\ 
 & \textbf{daz} arme und rîche\\ 
5 & sich vreuwten, wan ich \textbf{tet in} kunt,\\ 
 & daz ir noch wæret \textit{w}o\textit{l} gesunt.\\ 
 & ich \textbf{\textit{v}ant} d\textit{â} hers ein wunder.\\ 
 & ouch wart diu tave\textit{l}runder\\ 
 & besetzet durch iuwer botschaft.\\ 
10 & ob ritte\textit{r}s prîs gewan ie kraft,\\ 
 & ich mein an \textbf{langer} werdecheit,\\ 
 & die \textbf{sîn iu alle d\textit{â} bereit}."\\ 
 & \multicolumn{1}{l}{ - - - }\\ 
 & \multicolumn{1}{l}{ - - - }\\ 
15 & er sagt im ouch, wie daz geschach,\\ 
 & daz er die künigîn \textbf{gesprach},\\ 
 & und waz im diu mit triuwen riet.\\ 
 & er sagt im ouch von alder diet,\\ 
 & von rittern und von vrouwen,\\ 
20 & daz er \textbf{die} m\textit{ö}hte schouwen\\ 
 & zuo Tschoflanze vor der zît,\\ 
 & ê würde sînes kampfes strît.\\ 
 & Gawans \textbf{sorge gar verswant},\\ 
 & \textbf{niht wan vreude} er \textbf{in dem herzen vant}.\\ 
25 & \multicolumn{1}{l}{ - - - }\\ 
 & \multicolumn{1}{l}{ - - - }\\ 
 & al sîner sorge er \textbf{dô} vergaz;\\ 
 & er gienc wider und saz\\ 
 & und wa\textit{s m}it vreuden d\textit{â} zuo hûs,\\ 
30 & unz daz der künic Artus\\ 
\end{tabular}
\scriptsize
\line(1,0){75} \newline
Q R W V Fr40 \newline
\line(1,0){75} \newline
\textbf{1} \textit{Initiale} R Fr40  \newline
\line(1,0){75} \newline
\textbf{1} enbietent] entbiten Q  $\cdot$ ir] ir schnelles R \textbf{2} von in] von im Q Jnen gancz R \textbf{4} daz] Do W \textbf{6} wæret] weren R  $\cdot$ wol] noch Q \textbf{7} vant] wand Q  $\cdot$ dâ] do Q W V \textbf{8} tavelrunder] tauefrunder Q \textbf{10} ritters] rittes Q [ritter*]: ritters V \textbf{11} langer] [*]: langer V \textbf{12} Die [*]: sint v́ch alle do bereit V  $\cdot$ iu] io R  $\cdot$ alle dâ] alle do Q W da alle R \textbf{13} \textit{Die Verse 654.13-14 fehlen} Q R W V Fr40  \textbf{15} er] [O*]: Er V  $\cdot$ sagt] sagte W  $\cdot$ geschach] beschach V \textbf{16} gesprach] besprach W \textbf{17} im diu] die im W \textbf{18} sagt] sagte W  $\cdot$ alder] aller W \textbf{20} die] [s*]: sv́ V  $\cdot$ möhte] mochte Q R (Fr40) \textbf{21} Tschoflanze] schoflancze R tschoflantze W schoflanze V tschofflanz Fr40  $\cdot$ vor] in W \textbf{23} Gawans] Gawins R  $\cdot$ gar] gantz W \textbf{25} \textit{Die Verse 654.25-26 fehlen} Q R W V Fr40  \textbf{27} sorge] not W sorgen V (Fr40)  $\cdot$ dô] gar R \textit{om.} W V \textbf{28} wider] hin wider R (W) V (Fr40) \textbf{29} was mit] was do mit Q  $\cdot$ dâ] do Q W \textbf{30} Artus] artuß W \newline
\end{minipage}
\end{table}
\end{document}
