\documentclass[8pt,a4paper,notitlepage]{article}
\usepackage{fullpage}
\usepackage{ulem}
\usepackage{xltxtra}
\usepackage{datetime}
\renewcommand{\dateseparator}{.}
\dmyyyydate
\usepackage{fancyhdr}
\usepackage{ifthen}
\pagestyle{fancy}
\fancyhf{}
\renewcommand{\headrulewidth}{0pt}
\fancyfoot[L]{\ifthenelse{\value{page}=1}{\today, \currenttime{} Uhr}{}}
\begin{document}
\begin{table}[ht]
\begin{minipage}[t]{0.5\linewidth}
\small
\begin{center}*D
\end{center}
\begin{tabular}{rl}
\textbf{392} & \begin{large}D\end{large}ô sprach der \textbf{junge} Meljanz:\\ 
 & "iwer \textbf{zuht} \textbf{was} \textbf{ie} \textbf{sô} ganz,\\ 
 & die wîle daz ich wonte hie,\\ 
 & daz iwer rât mich nie verlie.\\ 
5 & het ich iu baz gevolget dô,\\ 
 & sô sæhe man mich hiute vrô.\\ 
 & Nû helfet mir, grâve Scherules,\\ 
 & wande ich iu wol getrûwe des,\\ 
 & umbe mînen hêrren, der mich hie hât\\ 
10 & - \textbf{sô hœrent} wol \textbf{bêde} iwern rât -,\\ 
 & \textbf{unt} Lyppaut, der ander vater mîn,\\ 
 & \textbf{der} tuo sîne zuht nû \textbf{gein} mir schîn.\\ 
 & \textbf{sîner hulde het ich} niht verlorn,\\ 
 & wolde\textbf{s} sîn tohter hân \textbf{enborn}.\\ 
15 & diu prüevete gein mir tôren schimpf.\\ 
 & daz was unvrouwelîch gelimpf."\\ 
 & Dô sprach der werde Gawan:\\ 
 & "hie wirt ein suone getân,\\ 
 & die \textbf{nimmer} scheidet wan der tôt."\\ 
20 & dô kâmen die der ritter rôt\\ 
 & hin ûz hete gevangen\\ 
 & \textbf{ûf} vür den künec gegangen.\\ 
 & \textbf{die} \textbf{sageten}, wie ez dâ wære komen.\\ 
 & dô Gawan hete vernomen\\ 
25 & sîniu wâpen, der mit \textbf{in} dâ streit\\ 
 & unt \textbf{wem} si gâben sicherheit,\\ 
 & unt dô si im \textbf{sagten} umben Grâl,\\ 
 & dô dâht er des, \textbf{daz} Parzival\\ 
 & dises mæres wære ein urhap.\\ 
30 & sîn nîgen er gein himel gap,\\ 
\end{tabular}
\scriptsize
\line(1,0){75} \newline
D \newline
\line(1,0){75} \newline
\textbf{1} \textit{Initiale} D  \textbf{7} \textit{Majuskel} D  \textbf{17} \textit{Majuskel} D  \newline
\line(1,0){75} \newline
\textbf{1} Meljanz] Melianz D \textbf{7} Scherules] Scervles D \textbf{11} Lyppaut] Lyppaot D \newline
\end{minipage}
\hspace{0.5cm}
\begin{minipage}[t]{0.5\linewidth}
\small
\begin{center}*m
\end{center}
\begin{tabular}{rl}
 & dô sprach der \textbf{künic} Mel\textit{i}anz:\\ 
 & "iuwer \textbf{zuht} \textbf{was} \textbf{ie} ganz,\\ 
 & die wîle daz ich wonte hie,\\ 
 & daz iuwer rât mich nie verlie.\\ 
5 & hete ich iu baz gevolget dô,\\ 
 & sô sæhe man mich hiute vrô.\\ 
 & nû helfet mir, grâve Scherules,\\ 
 & wand ich iu wol getrûwe des,\\ 
 & umb mînen hêrren, der mich hie hât\\ 
10 & - \textbf{si hœrent} wol \textbf{beide} iuwern rât -,\\ 
 & \textbf{und} Lippo\textit{u}t, der ander vater mîn,\\ 
 & \textbf{der} tuo sîne zuht nû \textbf{gegen} mir schîn.\\ 
 & \textbf{sîner hulde het ich} niht verlorn,\\ 
 & wolt \textbf{es} sîn tohter hân \textbf{enborn}.\\ 
15 & diu brüefete gegen mir tôren schimpf.\\ 
 & daz was unvrowelîch glimpf."\\ 
 & dô sprach der werde Gawan:\\ 
 & "hie wirt ein suone getân,\\ 
 & die \textbf{niemen} scheidet wan der tôt."\\ 
20 & dô kômen \textit{d}ie der ritter rôt\\ 
 & hin ûz hete gevangen\\ 
 & \textbf{ûf} vür den künic gegangen.\\ 
 & \textbf{die} \textbf{sageten}, wie ez dâ wære komen.\\ 
 & dô Gawan hete vernomen\\ 
25 & sîniu wâpen, der mit \textbf{in} d\textit{â} streit\\ 
 & und \textbf{wem} si gâben sicherheit,\\ 
 & und dô si im \textbf{sageten} umben Grâl,\\ 
 & dô dâht er des, Parcifal\\ 
 & dises mæres wær ein urhap.\\ 
30 & sîn nîgen er gegen himele gap,\\ 
\end{tabular}
\scriptsize
\line(1,0){75} \newline
m n o \newline
\line(1,0){75} \newline
\newline
\line(1,0){75} \newline
\textbf{1} Melianz] melancz m meliantze n meliancze o \textbf{2} ganz] so gantz n (o) \textbf{4} daz] \textit{om.} n o \textbf{5} iu] \textit{om.} n o \textbf{6} sô] Sa o  $\cdot$ vrô] frowe o \textbf{7} helfet] helffe n  $\cdot$ Scherules] scerules m n [sceru*]: scerules o \textbf{8} wol] \textit{om.} n \textbf{9} umb] Vnd vmb n \textbf{11} Lippout] lippoat m lippaot n lippoot o \textbf{13} sîner] Sin n o  $\cdot$ niht verlorn] verl:rn o \textbf{14} wolt] volt o  $\cdot$ enborn] verborn n (o) \textbf{15} diu] Sú n \textbf{16} unvrowelîch] vngefrouwelich n  $\cdot$ glimpf] [gewin]: glimpf m \textbf{19} wan] denne n (o) \textbf{20} kômen] koment m (n) (o)  $\cdot$ die] ẏe m \textbf{23} die] Wie o  $\cdot$ dâ] do n o \textbf{25} der] der der n  $\cdot$ dâ] do m o \textit{om.} n \textbf{27} dô] \textit{om.} n o  $\cdot$ sageten] sagte o \textbf{28} dâht] gedochte n (o) \textbf{29} dises] Vrhab dis o \newline
\end{minipage}
\end{table}
\newpage
\begin{table}[ht]
\begin{minipage}[t]{0.5\linewidth}
\small
\begin{center}*G
\end{center}
\begin{tabular}{rl}
 & dô sprach der \textbf{junge} Melianz:\\ 
 & "iwer \textbf{triwe}, \textbf{diu} \textbf{ist} \textbf{sô} ganz,\\ 
 & die wîle daz ich wonte hie,\\ 
 & daz iwer rât mich nie verlie.\\ 
5 & het ich iu baz gevolget dô,\\ 
 & sô sæhe man mich hiute vrô.\\ 
 & \begin{large}N\end{large}û helfet mir, grâve Tscherules,\\ 
 & wan ich iu wol getrûwe des,\\ 
 & umbe mînen hêrren, der mich hie hât.\\ 
10 & \textbf{si vernement} wol \textbf{bêde} iwern rât.\\ 
 & Libaut, der ander vater mîn,\\ 
 & tuo sîne zuht nû \textbf{an} mi\textit{r} \textit{s}chîn.\\ 
 & \textbf{ichne hete sîner hulde} niht verloren,\\ 
 & wolte sîn tohter haben \textbf{enboren}.\\ 
15 & diu prüevete gein mir tôren schimpf.\\ 
 & daz was unvroulîch gelimpf."\\ 
 & dô sprach der werde Gawan:\\ 
 & "hie wirt ein suone getân,\\ 
 & die \textbf{niemen} scheidet wan der tôt."\\ 
20 & dô kômen die der rîter rôt\\ 
 & hin ûz hete gevangen\\ 
 & vür den künic gegangen\\ 
 & \textbf{unde} \textbf{sagten}, wiez dâ wære komen.\\ 
 & dô Gawan hete vernomen\\ 
25 & sîniu wâpen, der mit \textbf{in} dâ streit\\ 
 & unde \textbf{wem} si gâben sicherheit,\\ 
 & unde \textit{dô} si im \textbf{sagten} umben Grâl,\\ 
 & dô dâhter des, \textbf{wie} Parzival\\ 
 & dises mæres wære ein urhap.\\ 
30 & sîn nîgen er gein himele gap,\\ 
\end{tabular}
\scriptsize
\line(1,0){75} \newline
G I O L M Q R Z Fr28 \newline
\line(1,0){75} \newline
\textbf{1} \textit{Initiale} I O L M R Z  \textbf{7} \textit{Initiale} G  \textbf{17} \textit{Initiale} I   $\cdot$ \textit{Capitulumzeichen} R  \newline
\line(1,0){75} \newline
\textbf{1} \textit{Die Verse 370.13-412.12 fehlen} Q   $\cdot$ dô] ÷o O Da M  $\cdot$ Melianz] meliancz R meliantz Z \textbf{2} triwe] zuht Z  $\cdot$ diu ist sô] ist so O ist M die waz so R was ie so Z die ist Fr28 \textbf{4} iwer rât mich] mich ewer rat I úwer rat R \textbf{5} iu] uchs R  $\cdot$ dô] da M \textbf{7} Tscherules] schurles I Tschervles O Tsheruͯles L scerules M scherules R [tc]: tscherules Fr28 \textbf{8} getrûwe] getrúwen R \textbf{10} bêde] beidu R \textbf{11} Libaut] Lybavt O Z Lýbat L Libayt M Lybant R Lẏbaut Fr28  $\cdot$ ander] ander ander L \textbf{12} tuo] Der tv Z  $\cdot$ nû an mir schîn] nv an mir >nv< schin G an mir shin I (O) (L) an mir Nu schyn M \textbf{13} ichne] ich I (O) (R)  $\cdot$ sîner] sine R  $\cdot$ hulde] hulder L \textbf{14} sîn] es sine R  $\cdot$ enboren] verborn I (O) (L) (M) (Z) (Fr28) \textbf{15} diu] Dv L  $\cdot$ prüevete] bruͤuet I (O) (M) (Z) (Fr28) \textbf{16} unvroulîch] ein [vnfrevlich]: vnfrevelich I \textbf{17} dô] Da M \textbf{19} wan] dan Z \textbf{20} dô] Da M Z \textbf{21} hete] hette gete L \textbf{22} oͮch quam der chvͦnig Fr28  $\cdot$ vür] Ovch fvͤr O (L) (M) (R) Vf fvr Z \textbf{23} sagten] seit I \textit{om.} O  $\cdot$ wære] waz R \textbf{24} dô] Da O M  $\cdot$ Gawan] ::: Fr28  $\cdot$ hete] hat R \textbf{25} sîniu] Sine R  $\cdot$ in] im L Fr28 \textit{om.} M \textbf{26} gâben] gabin da M \textbf{27} dô] \textit{om.} G da M Z  $\cdot$ im] \textit{om.} R \textbf{28} dô] Da M  $\cdot$ dâhter] ducht er R  $\cdot$ des wie] wie I wis daz R des daz Z  $\cdot$ Parzival] parzifal I M Barceval O parcifal L Z barczifal R parzeval Fr28 \textbf{29} dises mæres] des mers I Dieser mere L \textbf{30} nîgen er] [nigel]: nigener I  $\cdot$ gein] gegein R chegn den Fr28 \newline
\end{minipage}
\hspace{0.5cm}
\begin{minipage}[t]{0.5\linewidth}
\small
\begin{center}*T
\end{center}
\begin{tabular}{rl}
 & \begin{large}D\end{large}ô sprach der \textbf{junge} Melyanz:\\ 
 & "iuwer \textbf{triuwe} \textbf{ist} \textbf{sô} ganz,\\ 
 & die wîle daz ich wonte hie,\\ 
 & daz iuwer rât mich nie verlie.\\ 
5 & hetich iu baz gevolget dô,\\ 
 & sô sæhe man mich hiute vrô.\\ 
 & Nû helfet mir, grâve Tscherules,\\ 
 & wandich iu wol getriuwe des,\\ 
 & umbe mînen hêrren, der mich hie hât.\\ 
10 & \textbf{si vernem\textit{en}t} wol iuwern rât.\\ 
 & Lybaut, der ander vater mîn,\\ 
 & tuo sîne zuht nû \textbf{an} mir schîn.\\ 
 & \textbf{ich enhete sîner hulde} niht verlorn\\ 
 & \textbf{unde} wolte\textbf{z} sîn tohter hân \textbf{verborn}.\\ 
15 & diu prüevete gegen mir tôren schimpf.\\ 
 & daz was unvrôlîch gelimpf."\\ 
 & Dô sprach der werde Gawan:\\ 
 & "hie wirt ein suone getân,\\ 
 & die \textbf{niemen} scheidet wan der tôt."\\ 
20 & Dô kômen die der rîter rôt\\ 
 & hin ûz hete gevangen\\ 
 & vür den künec gegangen\\ 
 & \textbf{unde} \textbf{vrâgeten}, wiez dâ wære komen.\\ 
 & Dô Gawan hete vernomen\\ 
25 & sîn\textit{iu} wâpen, der mit \textbf{im} dâ streit\\ 
 & unde \textbf{dem} si gâben sicherheit,\\ 
 & unde dô sim \textbf{gesageten} umbe den Grâl,\\ 
 & dô dâhter des, \textbf{wie} Parcifal\\ 
 & dises mæres wære ein urhap.\\ 
30 & sîn \textit{nî}gen er gegen himele gap,\\ 
\end{tabular}
\scriptsize
\line(1,0){75} \newline
T V W \newline
\line(1,0){75} \newline
\textbf{1} \textit{Initiale} T W  \textbf{7} \textit{Majuskel} T  \textbf{11} \textit{Majuskel} T  \textbf{17} \textit{Majuskel} T  \textbf{20} \textit{Majuskel} T  \textbf{24} \textit{Majuskel} T  \newline
\line(1,0){75} \newline
\textbf{1} junge] kv́nig V  $\cdot$ Melyanz] melianz V meliantz W \textbf{2} triuwe] zvht V  $\cdot$ ist] waz ie V was W \textbf{4} verlie] erlie W \textbf{7} Tscherules] Tscerules T scherules V \textbf{10} vernement] vernemt T hoͤrent V  $\cdot$ wol] wol beide V (W) \textbf{11} Lybaut] Vnde lẏppaut V Lybout W \textbf{12} tuo] Der tuͦ V  $\cdot$ nû an mir] gegen mir nv V an mir W \textbf{13} enhete sîner] hette sine V enhet sein W \textbf{14} unde] \textit{om.} W  $\cdot$ verborn] enboren W \textbf{16} unvrôlîch] vnfrowelich V vnfroͤlicher W \textbf{22} vür] Auch fúr W \textbf{23} unde vrâgeten] Die sageten V Vnd sagten W  $\cdot$ dâ] do V W \textbf{25} sîniu] sine T  $\cdot$ im dâ] in do V (W) \textbf{26} dem] wem V W \textbf{27} sim gesageten] sú im sageten V sy saiten W \textbf{28} Parcifal] parzifal V partzifal W \textbf{29} dises] Diz V \textbf{30} nîgen] ingen T  $\cdot$ himele] dem himel W \newline
\end{minipage}
\end{table}
\end{document}
