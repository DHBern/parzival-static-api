\documentclass[8pt,a4paper,notitlepage]{article}
\usepackage{fullpage}
\usepackage{ulem}
\usepackage{xltxtra}
\usepackage{datetime}
\renewcommand{\dateseparator}{.}
\dmyyyydate
\usepackage{fancyhdr}
\usepackage{ifthen}
\pagestyle{fancy}
\fancyhf{}
\renewcommand{\headrulewidth}{0pt}
\fancyfoot[L]{\ifthenelse{\value{page}=1}{\today, \currenttime{} Uhr}{}}
\begin{document}
\begin{table}[ht]
\begin{minipage}[t]{0.5\linewidth}
\small
\begin{center}*D
\end{center}
\begin{tabular}{rl}
\textbf{571} & \begin{large}D\end{large}ô \textbf{hôrt} er ein \textbf{gebrummen},\\ 
 & als der \textbf{wol} zweinzec trummen\\ 
 & slüege \textbf{hie} ze tanze.\\ 
 & sîn vester muot der ganze,\\ 
5 & den diu wâre zagheit\\ 
 & nie \textbf{verscherte} noch versneit,\\ 
 & dâhte: "waz sol mir geschehen?\\ 
 & ich m\textit{ö}hte nû wol kumbers jehen.\\ 
 & wil sich mîn kumber mêrn,\\ 
10 & ze wer sol ich mich kêrn."\\ 
 & Nû sach er gein\textbf{s} gebûwers tür.\\ 
 & ein \textbf{starker} lewe spranc \textbf{dar} vür;\\ 
 & der was als ein ors \textbf{sô} hôch.\\ 
 & Gawan, der ie ungerne vlôch,\\ 
15 & den schilt mit den riemen nam.\\ 
 & er tet, als ez der wer gezam:\\ 
 & er spranc ûf \textbf{den} estrîch.\\ 
 & \textbf{durch hunger was} vreislîch\\ 
 & dirre starke lewe grôz,\\ 
20 & des er doch wênec dâ genôz.\\ 
 & Mit zorne lief er an den man,\\ 
 & ze wer stuont hêr Gawan.\\ 
 & er hete im den schilt nâch genomen:\\ 
 & sîn êrster grif was alsô komen\\ 
25 & durch den schilt mit \textbf{al} den klân.\\ 
 & von \textbf{tiere} \textbf{ist selten ê} getân\\ 
 & \textbf{sîn} grif durch solhe herte.\\ 
 & Gawan sich \textbf{zuckes} werte:\\ 
 & ein bein hin ab er im swanc.\\ 
30 & der lewe ûf drîen vüezen spranc;\\ 
\end{tabular}
\scriptsize
\line(1,0){75} \newline
D \newline
\line(1,0){75} \newline
\textbf{1} \textit{Initiale} D  \textbf{11} \textit{Majuskel} D  \textbf{21} \textit{Majuskel} D  \newline
\line(1,0){75} \newline
\textbf{8} möhte] mohte D \newline
\end{minipage}
\hspace{0.5cm}
\begin{minipage}[t]{0.5\linewidth}
\small
\begin{center}*m
\end{center}
\begin{tabular}{rl}
 & dô \textbf{hœret} er ein \textbf{brummen},\\ 
 & als der \textbf{wol} zweinzic trummen\\ 
 & slüege \textbf{d\textit{â}} \textit{zuo} tanze.\\ 
 & sîn vester muot der ganze,\\ 
5 & den diu wâre zagheit\\ 
 & nie \textbf{versêrt} noch versneit,\\ 
 & \textbf{er} dâht: "waz sol mir geschehen?\\ 
 & ich m\textit{ö}hte \textit{nû} wol kumbers jehen.\\ 
 & wil sich mîn kumber mêren,\\ 
10 & zuo wer sol ich mich kêren."\\ 
 & nû sach er gegen gebûres tür.\\ 
 & ein \textbf{starker} lewe spranc \textbf{her} vür;\\ 
 & der was als ein ros \textbf{sô} hôch.\\ 
 & Gawan, der ie \dag gerne\dag  vlôch,\\ 
15 & den schilt \textbf{er} mit den riemen nam.\\ 
 & er tet, als ez der wer gezam:\\ 
 & er spranc ûf \textbf{dem} estrîch.\\ 
 & \textbf{dô was durch hunger} vreislîch\\ 
 & diser starke lewe grôz,\\ 
20 & des er doch wênic d\textit{â} genôz.\\ 
 & mit \textit{zorne} lief er an den man,\\ 
 & zuo wer stuont hêr Gawan.\\ 
 & er het im den schilt nâhe genomen:\\ 
 & sîn êrster grif was alsô komen\\ 
25 & durch den schilt mit \textbf{al} den klân.\\ 
 & von \textbf{tier} \textbf{ie selten wart} getân\\ 
 & \textbf{ein} grif durch solich herte.\\ 
 & Gawan sich \textbf{zuckens} werte:\\ 
 & ein bein hin ab er im swanc.\\ 
30 & der lewe ûf drîn vüezen spranc;\\ 
\end{tabular}
\scriptsize
\line(1,0){75} \newline
m n o \newline
\line(1,0){75} \newline
\newline
\line(1,0){75} \newline
\textbf{1} hœret] horte n (o)  $\cdot$ ein brummen] einen bronnen n (o) \textbf{3} dâ zuo] do m o do zuͯ n \textbf{4} vester] pfeffer o \textbf{8} möhte] mohtte m (o)  $\cdot$ nû] ẏme m (n) o \textbf{11} sach] gesach n \textbf{13} der] De: o \textbf{17} estrîch] ersterich n \textbf{20} dâ] do m n o \textbf{21} zorne] \textit{om.} m  $\cdot$ lief] lieffe n \textbf{23} im] im do o \textbf{28} zuckens] zuckes n (o) \textbf{30} vüezen] bein n \newline
\end{minipage}
\end{table}
\newpage
\begin{table}[ht]
\begin{minipage}[t]{0.5\linewidth}
\small
\begin{center}*G
\end{center}
\begin{tabular}{rl}
 & \begin{large}D\end{large}ô \textbf{hôrt} er ein \textbf{gebrummen},\\ 
 & als der \textbf{wol} zweinzec trummen\\ 
 & slüege \textbf{hie} ze tanze.\\ 
 & sîn vester muot der ganze,\\ 
5 & den diu wâre zageheit\\ 
 & nie \textbf{verscherte} noch versneit,\\ 
 & dâhte: "waz sol mir geschehen?\\ 
 & ich m\textit{ö}hte nû wol kumbers jehen.\\ 
 & wil sich mîn kumber mêren,\\ 
10 & ze wer sol ich mich kêren."\\ 
 & nû sach er gein\textbf{es} gebûres tür.\\ 
 & ein \textbf{grôzer} lewe spranc \textbf{her} vür;\\ 
 & der was als ein ors \textbf{als} hôch.\\ 
 & Gawan, der ie ungerne vlôch,\\ 
15 & den schilt \textbf{er} mit den riemen nam.\\ 
 & er tet, als ez der wer gezam:\\ 
 & er spranc ûf \textbf{den} estrîch.\\ 
 & \textbf{durch hunger was} vreislîch\\ 
 & dirre starke lewe grôz,\\ 
20 & des er doch wênic dâ genôz.\\ 
 & mit zorne lief er an den man,\\ 
 & ze wer stuont hêr Gawan.\\ 
 & er het im den schilt nâch genomen:\\ 
 & sîn êrster grif was alsô komen\\ 
25 & durch den schilt mit den klân.\\ 
 & von \textbf{tiere} \textbf{ist selten ê} getân\\ 
 & \textbf{sîn} grif durch solhe herte.\\ 
 & Gawan sich \textbf{zuckes} werte:\\ 
 & ein bein hin abe \textit{er} im swanc.\\ 
30 & der lewe ûf drîn vüezen spranc;\\ 
\end{tabular}
\scriptsize
\line(1,0){75} \newline
G I L M Z Fr23 \newline
\line(1,0){75} \newline
\textbf{1} \textit{Initiale} G L Z Fr23  \textbf{13} \textit{Initiale} I  \newline
\line(1,0){75} \newline
\textbf{1} Dô] Da M (Z)  $\cdot$ hôrt] gehorit M (Z) (Fr23)  $\cdot$ ein gebrummen] einen brunnen Fr23 \textbf{2} trummen] drungen M (Fr23) \textbf{3} slüege] slug Fr23  $\cdot$ hie] \textit{om.} L \textbf{4} vester] [swester]: wester Fr23 \textbf{6} nie] Hie Fr23  $\cdot$ verscherte] verseret Z (Fr23) \textbf{7} dâhte] er daht I  $\cdot$ mir] mir noch I \textbf{8} möhte] mohte G (I) L (M) (Z)  $\cdot$ nû] uͯch L ni Fr23 \textbf{9} mîn] \textit{om.} Z \textbf{10} ze] Gein Z \textbf{11} geines] iensz L  $\cdot$ gebûres] gebuͯren L  $\cdot$ tür] tor M \textbf{12} grôzer] groͤz I starcher L (M) (Z)  $\cdot$ her] dar M \textbf{13} als hôch] hoch I so hoh L (M) (Z) \textbf{15} den riemen] dem riemen I remen L deme ryme M \textbf{16} ez der wer] der wer L wer der M  $\cdot$ gezam] zam Fr23 \textbf{23} im] in Fr23 \textbf{24} êrster] erste M \textbf{25} den klân] alden klan M al den klan Z (Fr23) \textbf{26} tiere] tiern I  $\cdot$ ê] \textit{om.} I \textbf{28} zuckes] zuchhens I \textbf{29} ein] Sin Fr23  $\cdot$ hin abe er im] hin abe si im G er im abe I er im hin abe Fr23 \textbf{30} drîn] den Fr23 \newline
\end{minipage}
\hspace{0.5cm}
\begin{minipage}[t]{0.5\linewidth}
\small
\begin{center}*T
\end{center}
\begin{tabular}{rl}
 & Dô \textbf{gehôrt} er ein \textbf{gebrummen},\\ 
 & alse der zwênzic trummen\\ 
 & slüege \textbf{hie} ze tanze.\\ 
 & sîn vester muot der ganze,\\ 
5 & den diu wâre zageheit\\ 
 & nie \textbf{versêrte} noch versneit,\\ 
 & dâhte: "waz sol mir geschehen?\\ 
 & ich m\textit{ö}hte nû wol kumbers jehen.\\ 
 & wil sich mîn kumber mêren,\\ 
10 & ze wer sol ich mich kêren."\\ 
 & Nû sach er gegen \textbf{jenes} gebûres tür.\\ 
 & ein \textbf{starker} lewe spranc \textbf{her} vür;\\ 
 & der was als ein ors \textbf{sô} hôch.\\ 
 & Gawan, der ie ungerne vlôch,\\ 
15 & den schilt \textbf{er} mit den riemen nam.\\ 
 & er tet, als ez der wer gezam:\\ 
 & er spranc ûf \textbf{den} estrîch.\\ 
 & \textbf{durch hunger was} vreislîch\\ 
 & dirre starke lewe grôz,\\ 
20 & des er doch wênic dâ genôz.\\ 
 & mit zorne lief er an den man,\\ 
 & ze wer stuont hêr Gawan.\\ 
 & er het im de\textit{n} \textit{s}chilt nâch genomen:\\ 
 & sîn êrster grif was alsô komen\\ 
25 & durch den schilt mit \textbf{al}den klân.\\ 
 & von \textbf{tieren} \textbf{ist selten ê} getân\\ 
 & \textbf{ein} grif durch solhe her\textit{te}.\\ 
 & Gawan sich \textbf{zuckes} werte:\\ 
 & ein bein hin abe er im swanc.\\ 
30 & der lewe ûf drîn vüezen spranc;\\ 
\end{tabular}
\scriptsize
\line(1,0){75} \newline
T U V W Q R Fr39 \newline
\line(1,0){75} \newline
\textbf{1} \textit{Capitulumzeichen} R   $\cdot$ \textit{Majuskel} T  \textbf{11} \textit{Majuskel} T  \textbf{21} \textit{Initiale} W  \newline
\line(1,0){75} \newline
\textbf{1} \textit{Die Verse 553.1-599.30 fehlen} U   $\cdot$ gehôrt] hort V W Q R \textbf{2} der] der wol V W Q R Fr39 \textbf{3} slüege] Schliege R \textbf{4} der] \textit{om.} R \textbf{5} den] Dann Q \textbf{7} dâhte] Da daht er V  $\cdot$ sol] ist W \textbf{8} möhte] mohte T V (Q)  $\cdot$ nû] noch Q \textbf{10} wer] wem Q \textbf{11} jenes] des Q R \textbf{12} Einen starken loͤwen springen her fv́r V \textbf{13} sô] \textit{om.} Q \textbf{14} Gawan] Gawin R \textbf{15} den schilt] Dem schilt W  $\cdot$ den riemen] dem Riemen R \textbf{16} ez] er Q des Fr39 \textbf{17} er] Der Q  $\cdot$ den] \textit{om.} Q \textbf{18} was] waz er V  $\cdot$ vreislîch] freistlich R \textbf{19} starke] starcker Q kranke R \textbf{20} \textit{Vers 571.20 ist am Rand nachgetragen und später radiert:} Dez er do wening doch :enos V   $\cdot$ dâ] [*]: do V do W R \textit{om.} Q \textbf{22} Gawan] gewan R \textbf{23} den schilt] dem scilt scilt T  $\cdot$ nâch] [*]: noch V \textbf{25} mit] \textit{om.} R  $\cdot$ alden] allen W \textbf{26} tieren] [*]: tiere V tiere W (Q) (R) Fr39  $\cdot$ getân] [*]: getan V vernom R \textbf{27} ein] [*]: Sin V Sein W Q (R) (Fr39)  $\cdot$ herte] her:: T \textbf{28} Gawan] Gawin R  $\cdot$ zuckes] [*]: zuckes V zukens R \textbf{29} [*]: Ein bein hin abe er im swang V \textbf{30} drîn vüezen] [*]: drien fuͤssen V sein fussze Q \newline
\end{minipage}
\end{table}
\end{document}
