\documentclass[8pt,a4paper,notitlepage]{article}
\usepackage{fullpage}
\usepackage{ulem}
\usepackage{xltxtra}
\usepackage{datetime}
\renewcommand{\dateseparator}{.}
\dmyyyydate
\usepackage{fancyhdr}
\usepackage{ifthen}
\pagestyle{fancy}
\fancyhf{}
\renewcommand{\headrulewidth}{0pt}
\fancyfoot[L]{\ifthenelse{\value{page}=1}{\today, \currenttime{} Uhr}{}}
\begin{document}
\begin{table}[ht]
\begin{minipage}[t]{0.5\linewidth}
\small
\begin{center}*D
\end{center}
\begin{tabular}{rl}
\textbf{5} & \textbf{swer} ie dâ pflac der lande,\\ 
 & der gebôt wol âne schande\\ 
 & - \textbf{daz} ist ein wârheit sunder wân -,\\ 
 & daz der \textbf{aldeste} bruoder solde hân\\ 
5 & sînes vater ganzen erbeteil.\\ 
 & daz was \textbf{der} \textbf{jungern} \textbf{unheil},\\ 
 & daz \textbf{in} der tôt die pflihte brach,\\ 
 & \textbf{als} \textbf{in} vater leben verjach.\\ 
 & dâ vor was ez gemeine.\\ 
10 & sus hât ez der alter eine.\\ 
 & \textbf{daz} schuof iedoch ein wîse man,\\ 
 & daz alter guot solde hân.\\ 
 & jugent hât vil werdecheit,\\ 
 & \textbf{daz} alter siuften unde leit.\\ 
15 & ez \textbf{en}wart nie niht als unvruot\\ 
 & \textbf{sô} alter und armuot.\\ 
 & künige, grâven, herzogen\\ 
 & - daz sag ich iu vür ungelogen -,\\ 
 & daz die \textbf{dâ} \textbf{huobe} enterbet sint\\ 
20 & unz an daz \textbf{elter} kint,\\ 
 & \textbf{daz} \textbf{ist} ein vremdiu zeche.\\ 
 & der kiusche und der vreche\\ 
 & Gahmuret, der wîgant,\\ 
 & verlôs sus bürge und lant,\\ 
25 & dâ sîn vater schône\\ 
 & truoc sceptrum und \textbf{die} krône\\ 
 & mit grôzer küneclîcher kraft,\\ 
 & unz er lac tôt an rîterschaft.\\ 
 & \begin{large}D\end{large}ô klagete man in sêre.\\ 
30 & die \textbf{ganzen} triwe und êre\\ 
\end{tabular}
\scriptsize
\line(1,0){75} \newline
D \newline
\line(1,0){75} \newline
\textbf{23} \textit{Versal} D  \textbf{29} \textit{Initiale} D  \newline
\line(1,0){75} \newline
\textbf{23} Gahmuret] Gagmuret D \newline
\end{minipage}
\hspace{0.5cm}
\begin{minipage}[t]{0.5\linewidth}
\small
\begin{center}*m
\end{center}
\begin{tabular}{rl}
 & \textbf{wer} ie d\textit{â} pflac der lande,\\ 
 & der gebôt wol âne schande\\ 
 & - \textbf{daz} ist ein wârheit sunder wân -,\\ 
 & daz der \textbf{alteste} bruoder solte hân\\ 
5 & sînes vater ganzen erbeteil.\\ 
 & daz was \textbf{der} \textbf{jungen} \textbf{michel heil},\\ 
 & daz \textbf{in} der tôt die pflihte brach,\\ 
 & \textbf{als} \textbf{in} \textbf{ir} vater leben verjach.\\ 
 & dâ \textit{vor dô} was e\textit{z} gemeine.\\ 
10 & sus hât ez der elter ein\textit{e}.\\ 
 & \textbf{daz} schuof iedoch ein wîser man,\\ 
 & daz alter guot solt hân.\\ 
 & jugent het vil wirdicheit,\\ 
 & \textbf{dô} alter siufzen und leit.\\ 
15 & ez wart nie \dag man alsô vruot\dag \\ 
 & \textbf{sô} alter und armuot.\\ 
 & künige, grâven \textbf{und} herzogen\\ 
 & - daz s\textit{ag ich iu vür un}gelogen -,\\ 
 & \textit{daz die \textbf{jungen} enterbet sint}\\ 
20 & \textit{unz an daz \textbf{eltest} kint,}\\ 
 & \textit{\textbf{daz} \textbf{was} ein vremdiu zeche.}\\ 
 & \textit{der kiusche und der vreche}\\ 
 & \textit{Gahmuret, der wîgant,}\\ 
 & \textit{verlôs sus bürge und lant,}\\ 
25 & \textit{dâ sîn vater schône}\\ 
 & \textit{truoc zepter und krône}\\ 
 & mit grôze\textit{r} küniclîcher \textit{kraft},\\ 
 & unz er lac tôt ane ritterschaft.\\ 
 & dô klagete man in sêre.\\ 
30 & die \textbf{ganzen} triuwe und êre\\ 
\end{tabular}
\scriptsize
\line(1,0){75} \newline
m n o W \newline
\line(1,0){75} \newline
\newline
\line(1,0){75} \newline
\textbf{1} wer ie] Wie er W  $\cdot$ dâ] do m n o W \textbf{4} solte] sol W \textbf{5} ganzen] gancze o \textbf{7} pflihte] pflig o \textbf{8} als] Das n  $\cdot$ ir] der n o  $\cdot$ vater] engel n \textbf{9} dâ vor dô] Da do vor m Do do vor n Dan do vor W  $\cdot$ ez] er m n o \textbf{10} hât] hette n  $\cdot$ eine] eÿner m \textbf{12} solt] sol W \textbf{14} leit] lont o \textbf{15} alsô] so n o W  $\cdot$ vruot] kruͦt n fruͦht o \textbf{16} \textit{nach 5.16:} In her iaget zuͦ seinen zil / So ist seiner witze nit zuͦ vil W  \textbf{18} sag ich iu] sÿe nit sind m (n) sie sint nit o  $\cdot$ ungelogen] [vegelogen]: vergelogen m [vrl*]: vrgelogen n vrgelogen o \textbf{19} \textit{Die Verse 5.19-26 fehlen} m n o  \textbf{23} Gahmuret] Gamuret W \textbf{27} \textit{Versdoppelung 5.27 (²m ²n ²o) nach 5.28; Lesarten der vorausgehenden Verse mit ¹m ¹n ¹o bezeichnet:} Mit grossen kuniglicher krafft \textsuperscript{2}\hspace{-1.3mm} m  · Mit grosser konniglicher macht \textsuperscript{2}\hspace{-1.3mm} n  (\textsuperscript{2}\hspace{-1.3mm} o   $\cdot$ \sout{Mit grossener kuniglicher} \textsuperscript{1}\hspace{-1.3mm} m  $\cdot$ grôzer] gosser \textsuperscript{1}\hspace{-1.3mm} o \textbf{30} ganzen] gancze m (n) o (W) \newline
\end{minipage}
\end{table}
\newpage
\begin{table}[ht]
\begin{minipage}[t]{0.5\linewidth}
\small
\begin{center}*G
\end{center}
\begin{tabular}{rl}
 & \textbf{der} ie dâ pflac der lande,\\ 
 & der gebôt wol âne schande\\ 
 & - \textbf{diz} ist ein wârheit sunder wân -,\\ 
 & daz der \textbf{elter} bruoder solte hân\\ 
5 & sînes vater ganzen erbeteil.\\ 
 & daz was \textbf{der} \textbf{jungeren} \textbf{u\textit{n}heil},\\ 
 & daz \textbf{in} der tôt die pf\textit{liht}e brach,\\ 
 & \textbf{der} \textbf{in} \textbf{ir} vater leben verjach.\\ 
 & dâ vor was ez gemeine.\\ 
10 & sus hât ez der alter eine.\\ 
 & \textbf{ez} schuof iedoch ein wîse man,\\ 
 & daz alter guot solte hân.\\ 
 & jugent hât vil werdecheit,\\ 
 & \textbf{daz} alter siuften und leit.\\ 
15 & ez \textbf{en}wart nie niht als unvruot\\ 
 & \textbf{sô} alter und armuot.\\ 
 & künige, grâven, herzogen\\ 
 & - daz sage ich iu vür ungelogen -,\\ 
 & daz die \textbf{dâ} \textbf{huobe} enterbet sint\\ 
20 & unze an daz \textbf{elteste} kint,\\ 
 & \textbf{\begin{large}D\end{large}az} \textbf{ist} ein vremdiu zeche.\\ 
 & der kiusche und der vreche\\ 
 & Gahmuret, der wîgant,\\ 
 & verlôs sus bürge und lant,\\ 
25 & dâ sîn vater schône\\ 
 & truoc zepter und krône\\ 
 & mit grôzer küniclîcher kra\textit{ft},\\ 
 & unzer lac tôt an rîterschaft.\\ 
 & dô klagte man in sêre.\\ 
30 & die \textbf{grôzen} triwe und êre\\ 
\end{tabular}
\scriptsize
\line(1,0){75} \newline
G O L M Q Z Fr32 Fr58 \newline
\line(1,0){75} \newline
\textbf{1} \textit{Initiale} O  \textbf{11} \textit{Versal} Fr32  \textbf{21} \textit{Initiale} G  \textbf{23} \textit{Initiale} L  \textbf{29} \textit{Initiale} L Q Z Fr32  \newline
\line(1,0){75} \newline
\textbf{1} der ie] ÷wer ie O Der E L So wer y M (Z) Wer Q swer Fr32 Wer ie Fr58  $\cdot$ dâ] \textit{om.} O L M do Q  $\cdot$ pflac] pflagt Q \textbf{2} gebôt] gebort Q \textbf{3} diz] Daz Z Fr58 \textbf{4} elter] elten L elste Q (Fr32) edeler  Fr58  $\cdot$ solte] sol L \textbf{5} ganzen] ganczes Fr58 \textbf{6} der] des M  $\cdot$ jungeren] ivngen O ivnger Z (Fr58)  $\cdot$ unheil] v:heil G gar vnheil Fr58 \textbf{7} tôt] lob M  $\cdot$ die pflihte] die ph:::e G div pflihte Fr32 \textbf{8} der] Als Z Fr58  $\cdot$ ir] ires Q  $\cdot$ leben] lebent O bleben L  $\cdot$ verjach] irfach M \textbf{9} ez] or M \textit{om.} Fr32 \textbf{10} hât ez] hatte osz M  $\cdot$ alter] elterrn Q \textit{om.} Fr32 \textbf{11} ez] Dasz Q (Z) (Fr32) (Fr58)  $\cdot$ iedoch] doch Q \textbf{12} guot] gute Q \textbf{14} daz] Dar Q  $\cdot$ siuften] sevfen O  $\cdot$ leit] hertzeleit Z (Fr58) \textbf{15} ez enwart] Es wart L (Z) (Fr58) E zn wart n\textit{achträglich korrigiert zu: }DEzn wart Q  $\cdot$ als] so O L M Q Z Fr32 Fr58 \textbf{16} sô] Also M (Q) (Fr58) \textbf{17} künige] Kvnig Z (Fr58)  $\cdot$ herzogen] vnde herczogen M (Q) (Z) (Fr58) \textbf{18} sage] saget Fr58  $\cdot$ vür] vorwar M  $\cdot$ ungelogen] vngen \textit{nachträglich korrigiert zu:} vngelogen Q \textbf{19} dâ huobe] von hoffe M da hube \textit{nachträglich korrigiert zu:} da hude Q bi huͦbe Fr32  $\cdot$ enterbet] enterbent L Q \textbf{20} unze an] Bisz uff M \textbf{21} Daz] Dizze O (M)  $\cdot$ zeche] [keche]: czeche M \textbf{22} vreche] freiche Q [fre*he]: freche Z \textbf{23} Gahmuret] gahmvret G Gahmoret O Gahmuͯret L Gachmuͯret M Gamúret Q Gamuret Z ganvͦret Fr32 \textbf{24} verlôs sus] Das die Q  $\cdot$ lant] die land Q \textbf{25} dâ] Do O Q \textbf{26} krône] cronen M \textbf{27} küniclîcher] kuntlicher Z tvgentlicher Fr32  $\cdot$ kraft] chra:: G \textbf{28} unzer] Vntz dasz er Q  $\cdot$ an] an an M \textbf{29} dô] Da O Z  $\cdot$ klagte] chlagt O (Q) (Z) \textbf{30} die grôzen] Die ganzen O (M) (Z) (Fr32) (Fr58) Gantze L Die grosse Q \newline
\end{minipage}
\hspace{0.5cm}
\begin{minipage}[t]{0.5\linewidth}
\small
\begin{center}*T
\end{center}
\begin{tabular}{rl}
 & \textbf{swer} ie dâ \textit{p}flac der lande,\\ 
 & der gebôt wol âne schande\\ 
 & - \textbf{Diz} ist ein wârheit sunder wân -,\\ 
 & daz der \textbf{elteste} bruoder solte hân\\ 
5 & sînes vater ganzez erbeteil.\\ 
 & daz was \textbf{des} \textbf{jungern} \textbf{unheil},\\ 
 & daz \textbf{im} der tôt die pflihte brach,\\ 
 & \textbf{daz} \textbf{im} \textbf{irs} vater leben verjach.\\ 
 & dâ vor was ez gemeine.\\ 
10 & sus hât ez der elter eine.\\ 
 & \textbf{Ez} schuof iedoch ein wîser man,\\ 
 & \textbf{daz} daz alter guot solte hân.\\ 
 & Jugent hât vil werdecheit,\\ 
 & \textbf{daz} alter siufzen und leit.\\ 
15 & Ez \textbf{en}wart nie niht als unvruot\\ 
 & \textbf{als} alter und armuot.\\ 
 & Künege, grâven, herzogen\\ 
 & - daz sagich iu vür ungelogen -,\\ 
 & daz die \textbf{der} \textbf{huobe} enterbet sint\\ 
20 & unz an daz \textbf{elteste} kint,\\ 
 & \textbf{diz} \textbf{ist} ein vremed\textit{iu} zeche.\\ 
 & der kiusche und der vreche\\ 
 & Gahmuret, der wîgant,\\ 
 & verlôs sus bürge und lant,\\ 
25 & dâ sîn vater schône\\ 
 & truoc scepter und krône\\ 
 & mit grôzer küneclîcher kraft,\\ 
 & unz er lac tôt an rîterschaft.\\ 
 & \begin{large}D\end{large}ô klagete man in sêre.\\ 
30 & die \textbf{g\textit{a}nzen} triuwe und êre\\ 
\end{tabular}
\scriptsize
\line(1,0){75} \newline
T U V \newline
\line(1,0){75} \newline
\textbf{3} \textit{Majuskel} T  \textbf{11} \textit{Majuskel} T  \textbf{13} \textit{Majuskel} T  \textbf{15} \textit{Majuskel} T  \textbf{17} \textit{Majuskel} T  \textbf{29} \textit{Initiale} T U V  \newline
\line(1,0){75} \newline
\textbf{1} swer] Wer U  $\cdot$ dâ] do U V  $\cdot$ pflac] :flac T \textbf{3} Diz] Daz V \textbf{4} elteste] edelste U elter V \textbf{6} des jungern] der iuͦngern U der iungen V \textbf{7} im] in U [in]: im V  $\cdot$ die] \textit{om.} U \textbf{8} daz im irs] Der in ir U Dez in ir V  $\cdot$ leben] lebende V  $\cdot$ verjach] versach U iach V \textbf{10} hât ez] hetez U \textbf{11} schuof] geschuͦf V \textbf{14} daz] \textit{om.} V \textbf{15} Ez enwart] Ezuͦ wart U  $\cdot$ als] so V \textbf{16} als alter] So alter ist V \textbf{19} huobe] huben U (V) \textbf{21} vremediu] vremede T \textbf{22} kiusche] \textit{om.} U \textbf{23} Gahmuret] Gahmvret T Gahmvreth U Gamuret V \textbf{25} dâ] Do V \textbf{28} lac tôt] tot lag V \textbf{30} ganzen] g:nzen T ganze U \newline
\end{minipage}
\end{table}
\end{document}
