\documentclass[8pt,a4paper,notitlepage]{article}
\usepackage{fullpage}
\usepackage{ulem}
\usepackage{xltxtra}
\usepackage{datetime}
\renewcommand{\dateseparator}{.}
\dmyyyydate
\usepackage{fancyhdr}
\usepackage{ifthen}
\pagestyle{fancy}
\fancyhf{}
\renewcommand{\headrulewidth}{0pt}
\fancyfoot[L]{\ifthenelse{\value{page}=1}{\today, \currenttime{} Uhr}{}}
\begin{document}
\begin{table}[ht]
\begin{minipage}[t]{0.5\linewidth}
\small
\begin{center}*D
\end{center}
\begin{tabular}{rl}
\textbf{403} & "\textbf{\textit{\begin{large}I\end{large}}ch sihe iuch} gern, als tuon ich sie.\\ 
 & doch hânt \textbf{mich} \textbf{grôze} vrouwen ie\\ 
 & ir werden handelunge erlân."\\ 
 & sus sprach \textbf{der stolze} Gawan.\\ 
5 & Der künec sande einen ritter dar\\ 
 & unt enbôt der meide, daz si sîn war\\ 
 & sô næme, daz langiu wîle\\ 
 & in diuhte ein \textbf{kurziu île}.\\ 
 & Gawan vuor, dar der künec gebôt.\\ 
10 & welt ir noch, swîg ich grôzer nôt.\\ 
 & nein, ich wilz \textbf{iu} vürbaz sagen.\\ 
 & strâze unt \textbf{ein} pfert \textbf{begunde} tragen\\ 
 & Gawanen gein der porte\\ 
 & an des palases orte.\\ 
15 & swer bûwes ie begunde,\\ 
 & baz denne ich sprechen kunde\\ 
 & von dises bûwes veste.\\ 
 & dâ lag ein burc, diu beste,\\ 
 & diu ie genant wart \textbf{ertstift}.\\ 
20 & unmâzen wît \textbf{was} ir begrift.\\ 
 & Der bürge lop sule wir \textbf{hie} lân,\\ 
 & wande ich iu vil ze sagen hân\\ 
 & von des küneges swester, einer magt.\\ 
 & hie ist von bûwe vil gesagt.\\ 
25 & die prüeve ich rehte, als ich sol.\\ 
 & was si schœne, daz stuont ir wol,\\ 
 & unt hete \textbf{si} dar zuo rehten muot,\\ 
 & daz was \textbf{gein werdecheit ir} guot,\\ 
 & \begin{large}S\end{large}ô daz ir site unt ir sin\\ 
30 & was gelîch der marcgrævîn,\\ 
\end{tabular}
\scriptsize
\line(1,0){75} \newline
D \newline
\line(1,0){75} \newline
\textbf{1} \textit{Initiale} D  \textbf{5} \textit{Majuskel} D  \textbf{21} \textit{Majuskel} D  \textbf{29} \textit{Initiale} D  \newline
\line(1,0){75} \newline
\textbf{1} Ich] ÷ch \textit{nachträglich korrigiert zu:} Ich D \textbf{13} Gawanen] Gawann D \newline
\end{minipage}
\hspace{0.5cm}
\begin{minipage}[t]{0.5\linewidth}
\small
\begin{center}*m
\end{center}
\begin{tabular}{rl}
 & "\textbf{ich s\textit{i}he iuch} gerne, als tuon ich \textit{s}ie.\\ 
 & doch hânt \textbf{mîn} \textbf{grôze} vrouwen ie\\ 
 & ir werden handelunge erlân."\\ 
 & sus sprach \textbf{de\textit{r} werde} Gawan.\\ 
5 & der künic sante einen ritter dar\\ 
 & und enbôt der megde, daz si sîn war\\ 
 & sô næme, daz langiu wîle\\ 
 & \textit{i}n d\textit{i}uhte ein \textbf{kurziu île}.\\ 
 & Gawan v\textit{uo}r, dar der künic gebôt.\\ 
10 & welt ir noch, swîg ich grôzer nôt.\\ 
 & nein, ich wilz \textbf{iu} vürbaz sagen.\\ 
 & strâze und \textbf{ein} pfert \textbf{begun\textit{d}en} tragen\\ 
 & Gawan gege\textit{n} der porte\\ 
 & an des palases orte.\\ 
15 & wer bûwes ie begu\textit{n}de,\\ 
 & baz dan ich sprechen k\textit{u}nde\\ 
 & von dises bûwes veste.\\ 
 & d\textit{â} lac ein burc, diu beste,\\ 
 & diu ie genant wart \textbf{ertstift}.\\ 
20 & u\textit{n}mâzen wît \textbf{was} ir b\textit{e}grif\textit{t}.\\ 
 & \begin{large}D\end{large}er bürge lop sullen wir \textbf{hie} lân,\\ 
 & wand ich iu vil ze sagene hân\\ 
 & von des küniges swester, einer maget.\\ 
 & hie ist von bûwe vil gesaget.\\ 
25 & die brüefe ich reht, als ich sol.\\ 
 & was si schœne, daz stuont ir wol,\\ 
 & und hete dar zuo rehten muot,\\ 
 & daz was \textbf{gegen werdicheit ir} guot,\\ 
 & sô daz ir site und ir sin\\ 
30 & was gelîch der marcgrævîn,\\ 
\end{tabular}
\scriptsize
\line(1,0){75} \newline
m n o \newline
\line(1,0){75} \newline
\textbf{21} \textit{Initiale} m n  \newline
\line(1,0){75} \newline
\textbf{1} sihe] sehe m  $\cdot$ sie] hie m \textbf{2} doch] Ouch n (o)  $\cdot$ hânt] bant o  $\cdot$ mîn grôze] mich grossen n (o) \textbf{4} der] de m \textbf{8} in] Jch in m  $\cdot$ diuhte] duhte m (n) \textbf{9} vuor] var m \textbf{10} swîg ich] ich swige n swige o \textbf{12} und] \textit{om.} n o  $\cdot$ begunden] begungen m begunde o \textbf{13} gegen] gege m \textbf{15} bûwes] buwens o  $\cdot$ begunde] begungde m \textbf{16} baz] Has o  $\cdot$ kunde] kinde m kuͯnde o \textbf{18} dâ] Do m n o \textbf{19} genant] \textit{om.} n o  $\cdot$ ertstift] er stifft m gestifft n (o) \textbf{20} unmâzen] Vnd massen m  $\cdot$ begrift] bgriff m \textbf{21} Der bürge] Die bure o \textbf{23} einer] ein n (o) \textbf{25} als ich] \textit{om.} n \textbf{29} site] [sin]: sitte o \newline
\end{minipage}
\end{table}
\newpage
\begin{table}[ht]
\begin{minipage}[t]{0.5\linewidth}
\small
\begin{center}*G
\end{center}
\begin{tabular}{rl}
 & "\textbf{ich sihe iuch} gerne, als tuon ich sie.\\ 
 & doch hânt \textbf{mich} \textbf{grôze} vrouwen ie\\ 
 & ir werden handelunge erlân."\\ 
 & sus sprach \textbf{mîn hêr} Gawan.\\ 
5 & der künic sande einen rîter dar\\ 
 & unde enbôt der meide, daz si sîn war\\ 
 & sô næme, daz langiu wîle\\ 
 & in d\textit{i}uhte ein \textbf{kurziu île}.\\ 
 & \textit{Gawan} vuor dar, \textbf{als} der künic gebôt.\\ 
10 & welt ir noch, swîge ich grôzer nôt.\\ 
 & nein, ich wilz \textbf{iu} vürbaz sagen.\\ 
 & strâze unde \textbf{ein} pfert \textbf{begunden} tragen\\ 
 & Gawanen gein der porte\\ 
 & an des palases orte.\\ 
15 & swer bûwes ie begunde,\\ 
 & baz dane ich sprechen kunde\\ 
 & von dises bûwes veste.\\ 
 & dâ lac ein burc, diu beste,\\ 
 & diu ie genant wart \textbf{ertstift}.\\ 
20 & u\textit{n}mâzen wît \textbf{was} ir begrift.\\ 
 & der bürge lop sulen wir \textbf{nû} lân,\\ 
 & wan ich iu vil ze sagene hân\\ 
 & von des küniges swester, einer maget.\\ 
 & hie ist von bûwe vil gesaget.\\ 
25 & die prüeve ich reht, als ich sol.\\ 
 & was si schœne, daz stuont ir wol,\\ 
 & \textit{und} hete \textbf{si} dar zuo rehten muot,\\ 
 & daz was \textbf{gein werdicheit ir} guot,\\ 
 & sô daz ir site unde ir sin\\ 
30 & was gelîch der margrævîn,\\ 
\end{tabular}
\scriptsize
\line(1,0){75} \newline
G I O L M Q R Z \newline
\line(1,0){75} \newline
\textbf{1} \textit{Initiale} I O L  \textbf{15} \textit{Initiale} I  \newline
\line(1,0){75} \newline
\textbf{1} \textit{Die Verse 370.13-412.12 fehlen} Q   $\cdot$ ich] ÷ch O  $\cdot$ tuon ich] tuͤt I tvͦn O tuͦt och R \textbf{2} hânt] hat R  $\cdot$ grôze] groser R \textbf{4} mîn hêr] der stolze I O (R) der werde L M Z  $\cdot$ Gawan] man O \textbf{5} sande] sant mit Jm R \textbf{6} enbôt] erbot L  $\cdot$ der meide] den meiden M  $\cdot$ si] sin O \textbf{7} langiu] lange I R \textbf{8} diuhte] duhte G (I) (O) (L) (M) (R)  $\cdot$ ein] \textit{om.} L  $\cdot$ kurziu île] churzewile I (O) kurcze ile R \textbf{9} Gawan] er G  $\cdot$ als] da I \textit{om.} O L M R Z \textbf{10} noch] nv O  $\cdot$ swîge ich] swigen I \textbf{11} ich] ich ich I  $\cdot$ wilz iu] wil sev I wil ivz O wils ir R \textbf{12} ein pfert] pfat L  $\cdot$ begunden] begunde I (L) (R) \textbf{13} Gawanen] Gawan I (M) R  $\cdot$ porte] porten M \textbf{15} swer] Wer L M R  $\cdot$ bûwes] buwens I (M) \textbf{16} baz dane ich] Das dannoch M \textbf{17} dises] \textit{om.} O  $\cdot$ bûwes] buwens M \textbf{18} diu] \textit{om.} I \textbf{19} ie genant] genant L (M) egenant Z  $\cdot$ ertstift] ir stift I er stiffte M erstifft R (Z) \textbf{20} unmâzen] vmazen G Vnwiszen L  $\cdot$ was] \textit{om.} I  $\cdot$ begrift] vmbgriff R \textbf{21} sulen] \textit{om.} O  $\cdot$ nû] hie O L M \textit{om.} R Z  $\cdot$ lân] lasen ston R \textbf{26} stuont] stvͦnde O \textbf{27} und] \textit{om.} G  $\cdot$ rehten] stetten R \textbf{28} gein] an I  $\cdot$ ir] \textit{om.} M R \textbf{29} ir sin] sin L \textbf{30} was gelîch] gelich was I \newline
\end{minipage}
\hspace{0.5cm}
\begin{minipage}[t]{0.5\linewidth}
\small
\begin{center}*T
\end{center}
\begin{tabular}{rl}
 & "\textbf{si siht mich} gerne, als tuon ich sie.\\ 
 & doch hânt \textbf{mich} \textbf{grœzer} vrouwen ie\\ 
 & ir werder handelunge erlân."\\ 
 & sus sprach \textbf{der werde} Gawan.\\ 
5 & \begin{large}D\end{large}er künec sante einen rîter dar\\ 
 & unde enbôt der megede, \textit{daz si sîn w}ar\\ 
 & sô næme, daz lang\textit{iu} wîle\\ 
 & in d\textit{i}uhte ein \textbf{kurzewîle}.\\ 
 & Gawan vuor, dar der künec gebôt.\\ 
10 & \textit{wolt ir noch, swîge ich grôzer nôt.}\\ 
 & Nein, ich wilz vürbaz sagen.\\ 
 & strâze unde pfert \textbf{begunden} tragen\\ 
 & Gawanen gegen der porte\\ 
 & an des palases orte.\\ 
15 & Swer bûwes ie begunde,\\ 
 & baz dann ich sprechen kunde\\ 
 & von disses bûwes veste.\\ 
 & dâ lac ein burc, diu beste,\\ 
 & di\textit{u} ie genant wart \textbf{ein stift}.\\ 
20 & unmâze wît \textbf{wart} ir begrift.\\ 
 & der bürge lop suln wir \textbf{hie} lân,\\ 
 & wandich iu vil ze sagenne hân\\ 
 & von des küneges swester, einer maget.\\ 
 & hie ist von bûwe vil gesaget.\\ 
25 & die prüeve ich rehte, als ich sol.\\ 
 & \textit{w}as si schœne, daz stuont ir wol,\\ 
 & unde hâte \textbf{si} dar zuo rehten muot,\\ 
 & daz was \textbf{ir gegen werdecheite} guot,\\ 
 & sô daz ir site unde ir sin\\ 
30 & was glîch der margrævîn,\\ 
\end{tabular}
\scriptsize
\line(1,0){75} \newline
T U V W \newline
\line(1,0){75} \newline
\textbf{5} \textit{Initiale} T U V  \textbf{11} \textit{Majuskel} T  \textbf{15} \textit{Majuskel} T  \newline
\line(1,0){75} \newline
\textbf{1} si siht mich] Jch sihe úch V (W) \textbf{2} mich grœzer] min grosse V mich grosse W \textbf{3} werder] werde U werden W \textbf{4} werde] stolze U V (W) \textbf{6} daz si sîn war] wol gevar T daz [sin]: sú sin war V \textbf{7} daz] die U  $\cdot$ langiu] lange T \textbf{8} diuhte] duhte T (U) (V) (W)  $\cdot$ kurzewîle] kv́rze ile V (W) \textbf{10} \textit{Vers 403.10 fehlt (Zeile ausgespart)} T   $\cdot$ noch swîge ich] noch ich swige V nun schweig ich W \textbf{11} vürbaz] úch fúrbas V (W) \textbf{12} pfert] ein pfert V W  $\cdot$ tragen] in tragen V \textbf{13} Gawanen] Gawan U \textbf{15} Swer bûwes] Wer buͦwens U Wer bauwes W \textbf{16} dann ich] dannoch W \textbf{18} dâ] Do V W \textbf{19} diu] die T  $\cdot$ ein stift] ertstift V erstifft W \textbf{20} wart] waz V \textbf{23} küneges] \textit{om.} W \textbf{26} was] d was T \textbf{27} hâte] hette V \textbf{28} ir gegen werdecheite] gen wirdikait ir W \newline
\end{minipage}
\end{table}
\end{document}
