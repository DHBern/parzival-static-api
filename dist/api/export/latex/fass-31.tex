\documentclass[8pt,a4paper,notitlepage]{article}
\usepackage{fullpage}
\usepackage{ulem}
\usepackage{xltxtra}
\usepackage{datetime}
\renewcommand{\dateseparator}{.}
\dmyyyydate
\usepackage{fancyhdr}
\usepackage{ifthen}
\pagestyle{fancy}
\fancyhf{}
\renewcommand{\headrulewidth}{0pt}
\fancyfoot[L]{\ifthenelse{\value{page}=1}{\today, \currenttime{} Uhr}{}}
\begin{document}
\begin{table}[ht]
\begin{minipage}[t]{0.5\linewidth}
\small
\begin{center}*D
\end{center}
\begin{tabular}{rl}
\textbf{31} & unser vanen sint \textbf{erkant},\\ 
 & daz zwêne vinger ûz der hant\\ 
 & \textbf{biutet} gein dem eide,\\ 
 & ir\textbf{ne} \textbf{geschehe} nie sô leide,\\ 
5 & wan sît \textbf{daz} Isenhart lac tôt.\\ 
 & \textbf{mîner} vrouwen \textbf{vrumt er} \textbf{herzen nôt}.\\ 
 & \textbf{sus} stêt diu künegîn gemâl,\\ 
 & vrou Belakane, sunder twâl\\ 
 & in \textbf{einen} blanken samît,\\ 
10 & gesniten \textbf{von} swarzer varwe, sît\\ 
 & daz wir diu wâpen \textbf{kuren} an in.\\ 
 & ir triwe an jâmer hât gewin.\\ 
 & die steckent ob den porten hôch.\\ 
 & vor die andern ähte uns \textbf{suochet} noch\\ 
15 & des \textbf{stolzen} Vridebrandes her,\\ 
 & die getouften von über mer.\\ 
 & ieslîcher porte ein vürste pfliget,\\ 
 & der sich strîtes \textbf{ûz} bewiget\\ 
 & mit sîner baniere.\\ 
20 & Wir haben Gaschiere\\ 
 & gevangen einen grâven ab.\\ 
 & der biute\textit{t} uns vil grôze hab.\\ 
 & der ist Kayletes swester sun.\\ 
 & swaz uns der nû mac getuon,\\ 
25 & daz muoz ie dirre gelten.\\ 
 & sölch gelücke kumt uns selten.\\ 
 & \textit{\begin{large}G\end{large}}rüenes angers \textbf{lützel}, sandes\\ 
 & wol drîzec \textbf{poinder} landes\\ 
 & ist zir gezelten vome graben.\\ 
30 & dâ wirt vil manec tjost erhaben."\\ 
\end{tabular}
\scriptsize
\line(1,0){75} \newline
D \newline
\line(1,0){75} \newline
\textbf{20} \textit{Majuskel} D  \textbf{27} \textit{Initiale} D  \newline
\line(1,0){75} \newline
\textbf{5} Isenhart] Jsenhart D \textbf{15} Vridebrandes] Fridebrandes D \textbf{20} Gaschiere] Gascîere D \textbf{22} biutet] bivten D \textbf{27} Grüenes] ÷rvͤnes D \newline
\end{minipage}
\hspace{0.5cm}
\begin{minipage}[t]{0.5\linewidth}
\small
\begin{center}*m
\end{center}
\begin{tabular}{rl}
 & unser \textit{v}ane\textit{n} sint \textbf{erkant},\\ 
 & daz zwêne vinger ûz der hant\\ 
 & \textbf{biutet} gegen dem eide,\\ 
 & ir \textbf{geschehe} nie sô leide,\\ 
5 & \dag sô\dag  \textbf{daz} Ysenhart lac tôt.\\ 
 & \textbf{mîner} vrouwen \textbf{vromter} \textbf{herzenôt}.\\ 
 & \textbf{sus} stât diu küniginne gemâl,\\ 
 & vrowe Belakane, sunder twâ\textit{l}\\ 
 & in \textbf{einen} blanken samît,\\ 
10 & gesniten \textbf{von} swarzer varwe, sît\\ 
 & daz \textit{wir} diu wâpe\textit{n} \textbf{k\textit{enn}en} an in.\\ 
 & ir triuwe an jâmer hât gewin.\\ 
 & die steckent ob den porten hôch.\\ 
 & vür die andern ahte uns \textbf{suoch\textit{e}t} noch\\ 
15 & des \textbf{stolzen} Fridebrandes her,\\ 
 & \textit{d}ie getouften von über mer.\\ 
 & ietlîcher porte ein v\textit{ürst}e pfliget,\\ 
 & der sich strîtes \textbf{ûz} bewiget\\ 
 & mit sîner baniere.\\ 
20 & wir haben G\textit{a}schiere\\ 
 & gevangen einen grâven abe.\\ 
 & der biutet uns vil grôze habe.\\ 
 & der ist Kailetes swestersun.\\ 
 & waz uns der nû mac getuon,\\ 
25 & daz muoz \textbf{uns} ie dirre gelten.\\ 
 & solich glück kumet uns selten.\\ 
 & grüenes angers \textbf{lützel}, sandes\\ 
 & wol drîzic \textbf{poinders} landes\\ 
 & ist \dag ir gezelte umbgraben\dag .\\ 
30 & d\textit{â} wirt vil manic just erhaben."\\ 
\end{tabular}
\scriptsize
\line(1,0){75} \newline
m n o W \newline
\line(1,0){75} \newline
\newline
\line(1,0){75} \newline
\textbf{1} vanen] wanner \textit{nachträglich korrigiert zu:} Banner m \textbf{4} ir] Jre m \textbf{5} Ysenhart] ÿsenhart m ẏsenhart n o \textbf{6} vromter] fromtt es W \textbf{7} diu küniginne] der konig o  $\cdot$ gemâl] gemal \textit{nachträglich korrigiert zu:} gemalt m \textbf{8} Belakane] belaken n belakon o belakanen W  $\cdot$ twâl] twang \textit{nachträglich korrigiert zu:} qwall m \textbf{9} einen] einem n (o) (W) \textbf{10} von] vor o \textbf{11} wir] \textit{om.} m  $\cdot$ wâpen] wappena m  $\cdot$ kennen] kumen m \textbf{14} andern] ander o  $\cdot$ uns] vs o  $\cdot$ suochet] suͯcheit m \textbf{16} die] Sie \textit{nachträglich korrigiert zu:} Die m  $\cdot$ getouften] getauͯfftan o \textbf{17} porte] porten W  $\cdot$ vürste] fride m \textbf{18} ûz bewiget] gar erwiget n o W \textbf{19} sîner] seinem W \textbf{20} Gaschiere] gar schiere m gar schier n geschier o gatschier W \textbf{22} uns] \textit{om.} W \textbf{23} der] Er n o W  $\cdot$ Kailetes] kailittes m kaẏlites n kalitas o gayletes W \textbf{24} mac] mage o \textbf{25} uns] \textit{om.} W \textbf{27} angers] ackers W \textbf{30} dâ] Do m n o W \newline
\end{minipage}
\end{table}
\newpage
\begin{table}[ht]
\begin{minipage}[t]{0.5\linewidth}
\small
\begin{center}*G
\end{center}
\begin{tabular}{rl}
 & unser vanen sint \textbf{bekant},\\ 
 & daz zwêne vinger ûz der hant\\ 
 & \textbf{bietent} gein dem eide,\\ 
 & ir\textbf{n} \textbf{geschæhe} nie sô leide,\\ 
5 & wan sît \textbf{daz} Ysenhart lac tôt.\\ 
 & \textbf{mîner} vrouwen \textbf{vuoget er} \textbf{herzenôt}.\\ 
 & \textbf{sô} stêt diu künigîn gemâl,\\ 
 & vrou Belacane, sunder twâl\\ 
 & in \textbf{einem} blanken samît,\\ 
10 & gesniten \textbf{von} swarzer varwe, sît\\ 
 & \textit{daz} wir diu wâpen \textbf{kuren} an in.\\ 
 & ir triwe an jâmer hât gewin.\\ 
 & die steckent obe den borten hôch.\\ 
 & vür die anderen ahte uns \textbf{suochet} noch\\ 
15 & des \textbf{küenen} Fridebrandes her,\\ 
 & die getouften von über mer.\\ 
 & ieslîcher borte ein vürste pfliget,\\ 
 & der sich strîtes \textbf{ûz} bewiget\\ 
 & mit sîner baniere.\\ 
20 & wir haben Gatschiere\\ 
 & gevangen einen grâven abe.\\ 
 & der biut uns vil grôze habe.\\ 
 & der ist Kailetes swester sun.\\ 
 & swaz uns der nû mac getuon,\\ 
25 & daz muoz ie dirre gelten.\\ 
 & solch gelücke kumt uns selten.\\ 
 & grüenes angers \textbf{wênic}, sandes\\ 
 & wol drîzic \textbf{ponder} landes\\ 
 & ist zir gezelten vome graben.\\ 
30 & dâ wirt vil manic tjost erhaben."\\ 
\end{tabular}
\scriptsize
\line(1,0){75} \newline
G O L M Q R Z Fr29 Fr32 \newline
\line(1,0){75} \newline
\textbf{1} \textit{Initiale} O M Fr29  \textbf{13} \textit{Versal} Fr32  \textbf{27} \textit{Initiale} L Q R Z  \newline
\line(1,0){75} \newline
\textbf{1} unser] ÷nser O  $\cdot$ vanen] frawen Q  $\cdot$ bekant] erchant Fr29 \textbf{3} bietent] Beduͯdet L Peutet Q (Z) (Fr29) (Fr32) Schwert R  $\cdot$ dem] in den R \textbf{4} irn] Jr L (M) Q Z  $\cdot$ geschæhe] geschach O L M geschechen R \textbf{5} daz] \textit{om.} R Z  $\cdot$ Ysenhart] ẏsenhart G Jsenhart L R Fr32 eyszenhart Q isenhart Z \textbf{6} mîner vrouwen] Mier O Min frowe L  $\cdot$ vuoget] frvmt O L (Q) (Z) (Fr29) frunt M (R) (Fr32)  $\cdot$ er] [ez]: er G ir L (M) (R) [er]: jr  Fr32  $\cdot$ herzenôt] herzen not O (Q) herte not Z \textbf{7} sô] sunst Q (R) (Fr32)  $\cdot$ gemâl] gein in gemal L zu male Q \textbf{8} Die fraw leyt sunder quale Q  $\cdot$ Belacane] belachane G Bolacan O Belecane L \textbf{9} blanken] blachem O blanchem L \textbf{10} von] mit Fr32  $\cdot$ varwe] \textit{om.} Z  $\cdot$ sît] wit Q \textbf{11} daz] sit G  $\cdot$ kuren] schowen R \textbf{12} an] on R han Fr29 \textbf{13} steckent obe] stechin vff M  $\cdot$ den] dy M dem Q  $\cdot$ borten] phortin M (Q) andern Fr32  $\cdot$ hôch] [boch]: hoch Q \textbf{14} die] den Q  $\cdot$ ahte] achten Q  $\cdot$ uns] vnde R (Z)  $\cdot$ suochet] svͦchent O (R) (Z) Fr29 (Fr32) suͯch L \textbf{15} küenen] kvniges L  $\cdot$ Fridebrandes] frýdebandes L Fridebrandis M fridebranes Q fridebandes R Fridbrandes Fr29 \textbf{16} von] har R \textit{om.} Fr32 \textbf{17} borte] \textit{om.} M pforten Q \textbf{18} der] Des Q  $\cdot$ ûz] dor ausz Q gar R \textbf{19} sîner] seinen Q \textbf{20} Gatschiere] katschiere G Gathschiere O gatschire Q G:::ere Fr29 gatschîere Fr32 \textbf{21} einen] mit einem Fr32 \textbf{22} biut] bedeudet Q  $\cdot$ grôze] grosszer Q \textbf{23} Kailetes] kaylettes O keyletes L kayletes R keiletes Z kaẏletes Fr32 \textbf{24} swaz] Waz L (M) (Q) (R)  $\cdot$ der nû mac] der mach O (R) (Fr32) mac der M der schade mag Q der mve mac Z \textbf{25} ie] \textit{om.} Q \textbf{26} kumt] kvnt L \textbf{27} sandes] sanges L \textbf{28} ponder] púnire Q \textbf{29} zir] zcu den M von ir Q zu der R  $\cdot$ gezelten] gezelt O viendin M  $\cdot$ vome] von den O zu dem Q \textbf{30} dâ] Do Q  $\cdot$ tjost] tost M \newline
\end{minipage}
\hspace{0.5cm}
\begin{minipage}[t]{0.5\linewidth}
\small
\begin{center}*T
\end{center}
\begin{tabular}{rl}
 & unser vanen sint \textbf{bekant},\\ 
 & daz zwêne vinger ûz der hant\\ 
 & \textbf{biutet} gegen dem eide,\\ 
 & ir\textbf{n} \textbf{geschehe} nie sô leide,\\ 
5 & wan sît Isenhart lac tôt.\\ 
 & \textbf{der} vrouwen \textbf{vrumt ir} \textbf{herze nôt}.\\ 
 & \textbf{sus} stêt diu künegîn gemâl,\\ 
 & vrou Belacane, sunder twâl\\ 
 & in \textbf{einem} blanken samît,\\ 
10 & gesniten \textbf{nâch} swarzer varwe, sît\\ 
 & daz wir diu wâpen \textbf{kurn} ane in.\\ 
 & ir triuwe an jâmere hât gewin.\\ 
 & die steckent ob den porten hôch.\\ 
 & vür die andern ahte uns \textbf{suochent} noch\\ 
15 & des \textbf{küenen} Fridebrandes her,\\ 
 & die getouften von über mer.\\ 
 & ieslîcher porte ein vürste pfliget,\\ 
 & der sich strîtes \textbf{ouch} bewiget\\ 
 & mit sîner baniere.\\ 
20 & wir haben Gatschiere\\ 
 & gevangen einen grâven abe.\\ 
 & der biutet uns vil grôze habe.\\ 
 & der ist Kayletes swester sun.\\ 
 & swaz uns der nû mac getuon,\\ 
25 & daz muoz ie dirre gelten.\\ 
 & sölch glücke kumt uns selten.\\ 
 & Grüenes angers \textbf{lützel}, sandes\\ 
 & wol drîzic \textbf{poynder} landes\\ 
 & ist zir gezelten vonme graben.\\ 
30 & dâ wirt vil manec tjost erhaben."\\ 
\end{tabular}
\scriptsize
\line(1,0){75} \newline
T U V \newline
\line(1,0){75} \newline
\textbf{27} \textit{Majuskel} T  \newline
\line(1,0){75} \newline
\textbf{3} biutet] Min vreuwe buͦtet U (V) \textbf{5} Isenhart] Jsenhart T U [*hart]: Jsinhart V  $\cdot$ lac] lac do U lag da V \textbf{6} Jr herze ir vruͦnt vil manige not U [*]: Miner vroͮwen frv́mete er not V \textbf{8} vrou] Vor U  $\cdot$ Belacane] Belekane V  $\cdot$ twâl] wal U \textbf{10} varwe] varwen V \textbf{11} diu] druͦ U \textbf{14} die] den U  $\cdot$ ahte] ahtewe T ahten U V  $\cdot$ suochent] suͦchet U V  $\cdot$ noch] auch U \textbf{15} Fridebrandes] fridebrans U \textbf{17} porte] porten U V \textbf{18} sich strîtes] strites sich U \textbf{20} Gatschiere] gatsciere T gar schiere U \textbf{23} Kayletes] keyles U kaẏletes V \textbf{24} swaz] Waz U \textbf{30} dâ] Do U (V) \newline
\end{minipage}
\end{table}
\end{document}
