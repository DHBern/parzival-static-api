\documentclass[8pt,a4paper,notitlepage]{article}
\usepackage{fullpage}
\usepackage{ulem}
\usepackage{xltxtra}
\usepackage{datetime}
\renewcommand{\dateseparator}{.}
\dmyyyydate
\usepackage{fancyhdr}
\usepackage{ifthen}
\pagestyle{fancy}
\fancyhf{}
\renewcommand{\headrulewidth}{0pt}
\fancyfoot[L]{\ifthenelse{\value{page}=1}{\today, \currenttime{} Uhr}{}}
\begin{document}
\begin{table}[ht]
\begin{minipage}[t]{0.5\linewidth}
\small
\begin{center}*D
\end{center}
\begin{tabular}{rl}
\textbf{399} & \begin{large}N\end{large}û hœret von \textbf{âventiuren} sagen\\ 
 & \textbf{unt} helfet mir dâr under klagen\\ 
 & Gawans grôzen kumber.\\ 
 & mîn wîser unt mîn tumber,\\ 
5 & die tuonz durch ir gesellecheit\\ 
 & unt \textbf{lâzen} in mit mir \textbf{sîn} leit.\\ 
 & owê, nû solt ich swîgen!\\ 
 & nein, lât vürbaz sîgen,\\ 
 & \textbf{der} etswenne gelücke neic\\ 
10 & unt nû gein ungemache seic.\\ 
 & Disiu burc was gehêrt sô,\\ 
 & daz Eneas Kartago\\ 
 & nie sô hêrrenlîche vant,\\ 
 & dâ vroun Didon tôt was minnen pfant.\\ 
15 & waz si palase pflæge,\\ 
 & \textbf{unt} wie vil dâ türne læge?\\ 
 & ir hete Acraton genuoc,\\ 
 & diu âne Babylonie \textbf{ie} truoc\\ 
 & ame griffe die \textbf{grœsten} wîte\\ 
20 & nâch heiden worte strîte.\\ 
 & si was alumbe wol sô hôch\\ 
 & \textbf{unt} \textbf{dâ} si gein dem mer \textbf{gezôch},\\ 
 & \textbf{decheinen sturm si widersaz}\\ 
 & noch grôzen, unge\textit{v}üegen haz.\\ 
25 & dâr vor lac raste breit ein plân,\\ 
 & dar über reit hêr Gawan.\\ 
 & \textbf{vünf} hundert ritter oder mêr,\\ 
 & ob den allen was einer hêr,\\ 
 & die kômen im dâ widerriten\\ 
30 & in liehten kleidern wol gesniten.\\ 
\end{tabular}
\scriptsize
\line(1,0){75} \newline
D \newline
\line(1,0){75} \newline
\textbf{1} \textit{Initiale} D  \textbf{11} \textit{Majuskel} D  \newline
\line(1,0){75} \newline
\textbf{14} Didon] Tydon D \textbf{24} ungevüegen] vngefivͤgen D \newline
\end{minipage}
\hspace{0.5cm}
\begin{minipage}[t]{0.5\linewidth}
\small
\begin{center}*m
\end{center}
\begin{tabular}{rl}
 & \begin{large}N\end{large}û hœre\textit{t} von \textbf{âventiure} sagen\\ 
 & \textbf{und} helfet mir dâr under klagen\\ 
 & Gawanes grôzen kumber.\\ 
 & mîn wîser und mîn tumber,\\ 
5 & die \dag muos\dag  durch ir gesellecheit\\ 
 & und \textbf{lâzent} in mit mir \textbf{wesen} leit.\\ 
 & ouwê, nû solt ich swîgen!\\ 
 & nein, lât vürbaz sîgen,\\ 
 & \textbf{der} etwenne glücke neic\\ 
10 & und nun gegen ungemache seic.\\ 
 & disiu burc was gehêret sô,\\ 
 & daz Eneas Karth\textit{a}g\textit{o}\\ 
 & \textit{ni}e sô hêrrenlîch vant,\\ 
 & d\textit{â} vrouwen Dydon tôt was \textbf{in} min\textit{n}e\textit{n} pfant.\\ 
15 & waz si palase pflæge,\\ 
 & \textbf{und} wie vil dâ türne læge?\\ 
 & ir hete A\textit{cr}aton genuoc,\\ 
 & diu âne Ba\textit{b}ilonie \textbf{ie} truoc\\ 
 & a\textit{m}e griffe die \textbf{grœsten} wîte\\ 
20 & nâch heiden worte strîte.\\ 
 & si was al umbe wol sô hôch\\ 
 & \textbf{unz} \textbf{d\textit{â}} si gegen dem mer \textbf{gezôch},\\ 
 & \textbf{daz si enkeinem sturm dô entsaz}\\ 
 & \textit{noch grôzen, ungevüegen haz.}\\ 
25 & dâr vor lac raste breit ein plân,\\ 
 & dar über reit hêr Gawan.\\ 
 & \textbf{wol} hundert ritter oder mêr,\\ 
 & ob den allen was einer hêr,\\ 
 & die kômen ime d\textit{â} widerriten\\ 
30 & in liehten kleidern wol gesniten.\\ 
\end{tabular}
\scriptsize
\line(1,0){75} \newline
m n o \newline
\line(1,0){75} \newline
\textbf{1} \textit{Capitulumzeichen} n   $\cdot$ \textit{Initiale} m  \newline
\line(1,0){75} \newline
\textbf{1} hœret] hoͯren m \textbf{3} Gawanes] Gawans n o \textbf{6} lâzent] lossen n (o) \textbf{9} neic] [leit]: neig m \textbf{11} gehêret] gehoret o \textbf{12} Karthago] karthoge m \textbf{13} nie] Me m \textbf{14} dâ] Do m n o  $\cdot$ vrouwen] frouwe m n (o)  $\cdot$ Dydon] dẏdon m dido n die don o  $\cdot$ minnen] minem m \textbf{15} palase pflæge] phallas pflegen o \textbf{16} und] \textit{om.} n o  $\cdot$ dâ] do n o  $\cdot$ türne læge] turneẏ legen o \textbf{17} Acraton] Attaton m atratuͯn o \textbf{18} Babilonie] bapilonie m \textbf{19} ame] Anne m  $\cdot$ griffe] begriff n o  $\cdot$ grœsten] groste n \textbf{21} al umbe] allẏm o  $\cdot$ sô] also n \textbf{22} dâ] do m n o \textbf{23} enkeinem] gegen keinem n o  $\cdot$ dô] \textit{om.} n o \textbf{24} \textit{Vers 399.24 fehlt} m  \textbf{28} allen] aller o  $\cdot$ einer] ein n \textbf{29} kômen] komment o  $\cdot$ dâ] do m n o  $\cdot$ widerriten] >wider< ritten o \newline
\end{minipage}
\end{table}
\newpage
\begin{table}[ht]
\begin{minipage}[t]{0.5\linewidth}
\small
\begin{center}*G
\end{center}
\begin{tabular}{rl}
 & nû hœret von \textbf{âventiure} sagen\\ 
 & \textbf{unde} helfet mir dâr under klagen\\ 
 & Gawans grôzen kumber.\\ 
 & mîn wîser unde mîn tumber,\\ 
5 & die tuonz durch ir gesellicheit\\ 
 & unde \textbf{lâzen} in mit mir \textbf{sîn} leit.\\ 
 & owê, nû solt ich swîgen!\\ 
 & nein, lât vürbaz sîgen,\\ 
 & \textbf{dâr} etswenne gelücke neic\\ 
10 & unde nû gein ungemache seic.\\ 
 & disiu burc was gehêrt sô,\\ 
 & daz Eneas Kartigo\\ 
 & nie sô hêrlîchen vant,\\ 
 & dâ vroun Didon tôt was minnen pfant.\\ 
15 & waz si palase pflæge,\\ 
 & wie vil dâ türne læge?\\ 
 & ir hete Acraton genuoc,\\ 
 & diu âne Babilone \textbf{ie} truoc\\ 
 & \textit{\begin{large}A\end{large}}nme griffe die \textbf{hœhesten} wîte\\ 
20 & nâch heidene worte strîte.\\ 
 & si was alumbe wol sô hôch\\ 
 & \textbf{unt} \textbf{dâ} si gein dem mer \textbf{gezôch},\\ 
 & \textbf{deheinen sturm si widersaz}\\ 
 & noch grôzen, ungevüegen haz.\\ 
25 & dâr vor lac raste breit ein plân,\\ 
 & dar über reit hêr Gawan.\\ 
 & \textbf{vünf} hundert rîter oder mêr,\\ 
 & obe den allen was einer hêr,\\ 
 & die kômen im dâ widerriten\\ 
30 & in liehten kleidern wol gesniten.\\ 
\end{tabular}
\scriptsize
\line(1,0){75} \newline
G I O L M Q R Z \newline
\line(1,0){75} \newline
\textbf{1} \textit{Initiale} I O L M R Z  \textbf{19} \textit{Initiale} G I  \newline
\line(1,0){75} \newline
\textbf{1} \textit{Die Verse 370.13-412.12 fehlen} Q   $\cdot$ nû] ÷v O \textbf{2} dâr under] das wunder M (R) \textbf{3} Gawans] gawanes G [Gawan]: Gawans L \textbf{4} Mit wiser vnde mit tvmmer M \textbf{5} ir] \textit{om.} M  $\cdot$ gesellicheit] geselheit R \textbf{6} lâzen in] lazenz in I laszin M  $\cdot$ sîn] wesen I Z \textbf{7} solt] sol Z  $\cdot$ swîgen] swisen Z \textbf{8} lât] nu lat I laszt M  $\cdot$ sîgen] seigen I \textbf{9} dâr] dem I (M) Der O L R Z \textbf{11} gehêrt] gehoͯret R \textbf{12} Kartigo] kartago I (O) (L) M R Z \textbf{13} hêrlîchen] herliche O L (M) \textbf{14} dâ] Do O L R  $\cdot$ vroun] \textit{om.} L frouwe M (R)  $\cdot$ Didon] Tidone I Dydon O (R) didonen L  $\cdot$ minnen] minne I (L) (M) \textbf{16} wie] vnd wie I (O) (M) (R) (Z) \textbf{17} hete] hat M  $\cdot$ Acraton] agraton I Acraten L acoron M \textbf{18} diu âne] die ein I  $\cdot$ Babilone] babilonie G I (O) (L) M babylonie Z  $\cdot$ ie truoc] nie getruͤc I \textbf{19} Anme] Manme G  $\cdot$ griffe] griffen O (M)  $\cdot$ hœhesten] grosten O L (M) Z \textbf{20} nâch] an I  $\cdot$ worte] voran R \textbf{21} alumbe] alle vmbe L  $\cdot$ hôch] hohe O \textbf{22} dâ] do R  $\cdot$ gezôch] zoch L R Z \textbf{24} noch] vnd I  $\cdot$ grôzen] grossem R keinen Z  $\cdot$ ungevüegen] vngefuͦgem R \textbf{25} raste] vaste L eyner raste M  $\cdot$ breit] wit R \textbf{28} einer] ein O R \textbf{29} die] Sie L  $\cdot$ kômen] bechomen I  $\cdot$ im dâ] im alle I da im L Im do R \textbf{30} in] mit I  $\cdot$ liehten] lichten L M  $\cdot$ wol] [we*]: well R \newline
\end{minipage}
\hspace{0.5cm}
\begin{minipage}[t]{0.5\linewidth}
\small
\begin{center}*T
\end{center}
\begin{tabular}{rl}
 & \begin{large}N\end{large}û hœret von \textbf{âventiure} sagen.\\ 
 & \textbf{nû} helfet mir dâr under klagen\\ 
 & Gawans grôzen kumber.\\ 
 & mîn wîser unde mîn tumber,\\ 
5 & die tuonz durch ir gesellecheit\\ 
 & unde \textbf{lâzen} in mit mir \textbf{wesen} leit.\\ 
 & Owê, nû soltich swîgen!\\ 
 & Nein, lât vürbaz sîgen,\\ 
 & \textbf{dâr} etswenne glücke neic\\ 
10 & unde nû gegen ungemache seic.\\ 
 & Dis\textit{iu} burc was gehêret sô,\\ 
 & daz Eneas Karthago\\ 
 & nie sô hêrlîche vant,\\ 
 & dâ v\textit{r}ôn Dydons tôt was \textbf{in} minnen pfant.\\ 
15 & waz si palase pflæge,\\ 
 & wie vil dâ türne læge?\\ 
 & ir hete Acraton genuoc,\\ 
 & di\textit{u} âne Babylone truoc\\ 
 & anme griffe die \textbf{grœste} wîte\\ 
20 & \textit{nâch heiden worte strîte}.\\ 
 & si was alumbe wol sô hôch\\ 
 & \textbf{unde} \textbf{daz} si gegen dem mer \textbf{sich} \textbf{zôch},\\ 
 & \textbf{deheinen sturm si widersaz}\\ 
 & noch grôzen, ungevüegen haz.\\ 
25 & dâ vor lac raste breit ein plân,\\ 
 & dar über reit hêr Gawan.\\ 
 & \textbf{vünf} hundert rîter oder mêr,\\ 
 & ob den allen was einer hêr,\\ 
 & die kômen im dâ widerriten\\ 
30 & in liehten kleidern wol gesniten.\\ 
\end{tabular}
\scriptsize
\line(1,0){75} \newline
T U V W \newline
\line(1,0){75} \newline
\textbf{1} \textit{Initiale} T U W  \textbf{7} \textit{Majuskel} T  \textbf{8} \textit{Majuskel} T  \textbf{11} \textit{Majuskel} T  \newline
\line(1,0){75} \newline
\textbf{2} nû] Vnde V W \textbf{3} Gawans] Gawanes V  $\cdot$ grôzen] grosser W \textbf{5} tuonz] muͦz U  $\cdot$ ir] mine U [i*]: ire V seine W \textbf{6} in] eúch W  $\cdot$ wesen] sein W \textbf{8} Nein] Mein W \textbf{9} dâr] Der U W Dem V \textbf{11} Disiu] Dise T \textbf{12} Karthago] kathago U kartago W \textbf{13} hêrlîche] herlichen W \textbf{14} dâ vrôn] da von T (U) Do fraw W  $\cdot$ Dydons] dydon V W  $\cdot$ in minnen] minnen U (V) innen W \textbf{15} palase] palast W \textbf{16} wie] Vnde wie V (W)  $\cdot$ dâ türne] do torne U túrne do V W \textbf{17} Acraton] Acroton T \textbf{18} diu] die T  $\cdot$ Babylone] Babylon U babilonie V babilone W  $\cdot$ truoc] ie truͦc U (V) ie getruͦg W \textbf{19} grœste] groͤsten V hoͤchsten W \textbf{20} \textit{Vers 399.20 fehlt (Zeile ausgespart)} T U  \textbf{22} unde daz] Vntze do V  $\cdot$ sich zôch] gezoch V W \textbf{23} Das sv́ nie keinen sturm do entsas V \textbf{24} ungevüegen] vngewuͦffen W \textbf{25} raste] vaste W \textbf{29} dâ] do V W \newline
\end{minipage}
\end{table}
\end{document}
