\documentclass[8pt,a4paper,notitlepage]{article}
\usepackage{fullpage}
\usepackage{ulem}
\usepackage{xltxtra}
\usepackage{datetime}
\renewcommand{\dateseparator}{.}
\dmyyyydate
\usepackage{fancyhdr}
\usepackage{ifthen}
\pagestyle{fancy}
\fancyhf{}
\renewcommand{\headrulewidth}{0pt}
\fancyfoot[L]{\ifthenelse{\value{page}=1}{\today, \currenttime{} Uhr}{}}
\begin{document}
\begin{table}[ht]
\begin{minipage}[t]{0.5\linewidth}
\small
\begin{center}*D
\end{center}
\begin{tabular}{rl}
\textbf{40} & - daz \textbf{sag} ich \textbf{iu} ûf die triwe mîn -,\\ 
 & bestêt \textbf{ir den} Anschevin,\\ 
 & \textit{\begin{large}D\end{large}}er mîne sicherheit dort hât.\\ 
 & \textbf{ir sult} merken mînen rât\\ 
5 & \textbf{unt} dar zuo, \textbf{hêrre}, mîne bete.\\ 
 & ich hân geheizen Gahmurete,\\ 
 & daz ich iuch alle wende.\\ 
 & daz lobt ich sîner hende.\\ 
 & durch mich lât iwer streben sîn.\\ 
10 & er tuot iu kraft an strîte schîn."\\ 
 & Dô sprach der \textbf{künec} Kaylet:\\ 
 & "ist \textbf{ez} mîn neve Gahmuret\\ 
 & filuroy Gandin,\\ 
 & mit dem lâz ich mîn strîten sîn.\\ 
15 & lât mir\textbf{n} zoum!" "ine lâzez niht,\\ 
 & ê \textbf{daz} mîn ouge alrêste \textbf{ersiht}\\ 
 & iwer blôzez houbet.\\ 
 & daz mîne ist mir betoubet."\\ 
 & den helm \textbf{er im her ab} \textbf{dô} bant.\\ 
20 & Gahmuret \textbf{mêr} strîtes vant.\\ 
 & ez was wol mitter morgen dô.\\ 
 & die von der stat des wâren vrô,\\ 
 & die dise tjost ersâhen.\\ 
 & \textbf{si} begunden alle gâhen\\ 
25 & an ir \textbf{werlîchen} letze.\\ 
 & er was vor in ein netze:\\ 
 & swaz drunder kom, daz was \textbf{beslagen}.\\ 
 & ein ander ors, sus \textbf{hœre} ich sagen,\\ 
 & dâr ûf saz der werde.\\ 
30 & daz vlouc unt ruorte die erde\\ 
\end{tabular}
\scriptsize
\line(1,0){75} \newline
D \newline
\line(1,0){75} \newline
\textbf{3} \textit{Initiale} D  \textbf{11} \textit{Majuskel} D  \newline
\line(1,0){75} \newline
\textbf{2} Anschevin] Ascevin D \textbf{3} der] ÷eR D \textbf{6} Gahmurete] Gahmvrete D \textbf{12} Gahmuret] Gahmvret D \textbf{20} Gahmuret] Gahmvret D \newline
\end{minipage}
\hspace{0.5cm}
\begin{minipage}[t]{0.5\linewidth}
\small
\begin{center}*m
\end{center}
\begin{tabular}{rl}
 & - daz \textbf{sage} ich ûf die triuwe mîn -,\\ 
 & bestêt \textbf{i\textit{uch} der} A\textit{n}schevin,\\ 
 & der mîne sicherheit dort hât.\\ 
 & \textbf{ir sollet} merken mînen rât\\ 
5 & \textbf{und} dar zuo, \textbf{hêrre}, mîne bete.\\ 
 & ich hân geheizen Gahmurete,\\ 
 & daz ich iuch alle wende.\\ 
 & daz lobet ich sîner hende.\\ 
 & durch mich lât iuwer streben sîn.\\ 
10 & er tuot iu kraft an strîte schîn."\\ 
 & \begin{large}D\end{large}ô sprach der \textbf{küene} Kailet:\\ 
 & "ist \textbf{daz} mîn neve Gahmuret\\ 
 & fili ro\textit{y}s Gandin,\\ 
 & mit dem lâz ich mîn \textit{s}trîten sîn.\\ 
15 & lât mir \textbf{mîn} zoum!" "ine lâz ez niht,\\ 
 & ê \textbf{daz} mîn ouge aller êrst \textbf{ersiht}\\ 
 & i\textit{u}wer blôzez houbet.\\ 
 & daz mîne ist mir betoubet."\\ 
 & den helm \textbf{her abe er im} \textbf{dô} bant.\\ 
20 & Gahmuret \textbf{nimer} strîtes vant.\\ 
 & ez was wol mitter morgen dô.\\ 
 & die von der stat des wâren vrô,\\ 
 & die dise just ersâhen.\\ 
 & \textbf{si} b\textit{e}gunden alle gâhen\\ 
25 & an ir \textbf{werlîchen} letze.\\ 
 & er was vo\textit{r} i\textit{n} ein netze:\\ 
 & waz drunder kam, daz was \textbf{beslagen}.\\ 
 & ein ander ros, sus \textbf{hôrt} ich sagen,\\ 
 & dâr ûf saz der werde.\\ 
30 & daz vl\textit{ou}c und r\textit{uor}te die erde\\ 
\end{tabular}
\scriptsize
\line(1,0){75} \newline
m n o W \newline
\line(1,0){75} \newline
\textbf{11} \textit{Initiale} m n o W  \newline
\line(1,0){75} \newline
\textbf{1} ich] ich úch n \textbf{2} iuch] in m n o  $\cdot$ Anschevin] ausceuin \textit{nachträglich korrigiert zu:} ansceuin m auscevin n ansce vin o antscheuin W \textbf{3} dort] dor n \textbf{6} Gahmurete] gahmurette m gamirette n gamuͯret o gamuret W \textbf{8} lobet] gelobet n  $\cdot$ ich] icb W \textbf{9} streben] sterben W \textbf{11} Kailet] gailet n gaylet o W \textbf{12} daz] din das o  $\cdot$ Gahmuret] gemiret n gamuret o W \textbf{13} roys] ros m (n) o  $\cdot$ Gandin] gaudin W \textbf{14} strîten] :tritten \textit{nachträglich korrigiert zu:} stritten m streit W \textbf{15} mir mîn zoum] uweren zorn n (o) (W)  $\cdot$ ine] ich n o W \textbf{16} mîn] \textit{om.} o \textbf{17} iuwer] Jn wer m o \textbf{18} betoubet] beroubt n (o) (W) \textbf{19} im dô bant] do bann o \textbf{20} Gahmuret] Gamiret n Gemuͯret o Gamuret W  $\cdot$ nimer] nit me n o W  $\cdot$ vant] do vant W \textbf{21} mitter] mitten n \textbf{23} ersâhen] ersagen o \textbf{24} begunden] bengunden m begunde n  $\cdot$ gâhen] johen n jagen o nahen W \textbf{26} vor in] von ir m von in n o  $\cdot$ netze] besetze W \textbf{27} waz] Das n o Der W  $\cdot$ daz] der W \textbf{30} vlouc] floch m n sloch o  $\cdot$ ruorte] ruͯwette m \newline
\end{minipage}
\end{table}
\newpage
\begin{table}[ht]
\begin{minipage}[t]{0.5\linewidth}
\small
\begin{center}*G
\end{center}
\begin{tabular}{rl}
 & - daz \textbf{nim} ich û\textit{f} die triwe mîn -,\\ 
 & bestêt \textbf{ir den} Antschevin,\\ 
 & der mîne sicherheit dort hât.\\ 
 & \textbf{nû sult ir} merken mînen rât.\\ 
5 & dar zuo \textbf{hœret} mîne bet.\\ 
 & ich hân geheizen Gahmuret,\\ 
 & daz ich iuch alle wende.\\ 
 & daz lobt ich sîner hende.\\ 
 & durch mich lât iwer st\textit{r}eben sîn.\\ 
10 & er tuot iu kraft an strîte schîn."\\ 
 & dô sprach der \textbf{künic} Kailet:\\ 
 & "ist \textbf{daz} mîn neve Gahmuret\\ 
 & filli rois Gandin,\\ 
 & mit dem lâze ich mîn strîten sîn.\\ 
15 & lât mir \textbf{den} zoum!" "ine lâzez \textbf{iu} niht,\\ 
 & ê mîn ouge alrêrst \textbf{gesiht}\\ 
 & iwer blôzez houbet.\\ 
 & daz mîne ist mir betoubet."\\ 
 & den helm \textbf{er im abe} bant.\\ 
20 & Gahmuret \textbf{nimê} strîtes vant.\\ 
 & ez was wol mitter morgen dô.\\ 
 & die von der stat des wâren vrô,\\ 
 & die dise tjost ersâhen.\\ 
 & \textbf{die} begunden alle gâhen\\ 
25 & an ir \textbf{gewærlîche} letze.\\ 
 & er was vor in ein netze:\\ 
 & swaz drunder kom, daz was \textbf{beslagen}.\\ 
 & ein ander ors, sus \textbf{hôrte} ich sagen,\\ 
 & \begin{large}D\end{large}âr ûf saz der werde.\\ 
30 & daz vlouc und ruorte die erde\\ 
\end{tabular}
\scriptsize
\line(1,0){75} \newline
G O L M Q R Z Fr21 \newline
\line(1,0){75} \newline
\textbf{1} \textit{Initiale} O  \textbf{11} \textit{Initiale} L M Q R Z Fr21  \textbf{29} \textit{Initiale} G  \newline
\line(1,0){75} \newline
\textbf{1} daz] ÷az O  $\cdot$ nim] sag O (L) (M) Q R Z Fr21  $\cdot$ ûf] vs G iv avf O (L) (M) (Q) (Z) \textbf{2} bestêt] Vnd bestet O L (M) R Z (Fr21) Vnd bester Q  $\cdot$ Antschevin] anschvin O Anshewin L aschvyn M ansheúin Q anschevin R (Fr21) anshevin Z \textbf{4} nû sult ir] Jr svlt Z \textbf{5} dar] Vnd dar O L (M) (Q) R Z (Fr21)  $\cdot$ hœret] herre O L (M) Q R Z Fr21  $\cdot$ bet] bete G \textbf{6} ich hân] Mich had M  $\cdot$ Gahmuret] Gahmvret G Gamvret O Gahmuͯret L gamuret M Z gamúert Q Gahmoret Fr21 \textbf{8} lobt] gelobete M \textbf{9} iwer] uwern M  $\cdot$ streben] stereben G sterben O striten M (Q) \textbf{10} kraft an strîte] krafft an strites M fast streiten an Q \textbf{11} Kailet] kaylet O M Q R Fr21 kaýlet L gailet Z \textbf{12} daz] dis R  $\cdot$ Gahmuret] Gahmvret G Gamvret O Gahmuͯret L gamurat M gaműert Q gamuret Z Gahmoret Fr21 \textbf{13} Gandin] candin G Gaúdin Q \textbf{14} Nu lot er strayten sein Q \textbf{15} lât] La Z  $\cdot$ mir] mich L  $\cdot$ den] den den Z  $\cdot$ ine] ich O L Jch onsz M  $\cdot$ lâzez] laz O (L) (Fr21) laszin M  $\cdot$ iu] iv sin O \textit{om.} Q R Z \textbf{16} ê] E daz L  $\cdot$ ouge] augen Q (R)  $\cdot$ alrêrst] vor L alles Q  $\cdot$ gesiht] ersiht O (Q) (R) Z Fr21 irsien M \textbf{17} iwer] Wer Q  $\cdot$ blôzez] blosze M \textbf{18} betoubet] ertobet R \textbf{19} abe bant] her abe do bant O L da abe bant M do abe hand R her abe bant Z her do bant Fr21 \textbf{20} Gahmuret] Gahmvret G L Gamvret O Gamurat M Gamúret Q Gamuret Z Gahmoret Fr21  $\cdot$ nimê] niht mer O (L) Fr21 mir M mer Q R Z \textbf{21} wol] \textit{om.} L  $\cdot$ mitter] mitten M  $\cdot$ dô] da M \textbf{22} des wâren] da waren O worin des M dy worren Q \textbf{23} dise] sine L M  $\cdot$ tjost] dingk Q \textbf{24} gâhen] iehen Q \textbf{25} gewærlîche] werlich O (L) (Q) (R) Fr21 werlichin M (Z) \textbf{27} swaz] Waz L (M) (Q) (R) Z  $\cdot$ daz] \textit{om.} R  $\cdot$ beslagen] erslagen L (Q) (Fr21) \textbf{28} sus] \textit{om.} M R dasz Q fvs Z  $\cdot$ hôrte] hor Z \textbf{30} vlouc] sluͯch L  $\cdot$ erde] erden M \newline
\end{minipage}
\hspace{0.5cm}
\begin{minipage}[t]{0.5\linewidth}
\small
\begin{center}*T (U)
\end{center}
\begin{tabular}{rl}
 & - daz \textbf{sag} ich \textbf{iu} ûf die triuwe mîn -,\\ 
 & bestât \textbf{ir den} Anschevin,\\ 
 & der mîne sicherheit dort hât.\\ 
 & \textbf{nû sult ir} merken mînen rât\\ 
5 & \textbf{und} dar zuo, \textbf{hêrre}, mîne bete.\\ 
 & ich hân geheizen Gahmurete,\\ 
 & daz ich iuch alle wende.\\ 
 & daz lobete ich sîner hende.\\ 
 & durch mich lât iuwer streben sîn.\\ 
10 & er tuot iu kraft an strîte schîn."\\ 
 & \begin{large}D\end{large}ô sprach der \textbf{künec} Kaylet:\\ 
 & "ist \textbf{diz} mîn neve Gahmuret\\ 
 & fillyroys Gandin,\\ 
 & mit dem lâz ich mîn strîten sîn.\\ 
15 & lât mir \textbf{den} zoum!" "ichne lâz ez niht,\\ 
 & ê mîn ouge aller êrst \textbf{gesiht}\\ 
 & iuwer blôzez houbet.\\ 
 & daz mîn ist mir betoubet."\\ 
 & den helm \textbf{er im dô abe} bant.\\ 
20 & Gahmuret \textbf{niht mê} strîtes vant.\\ 
 & ez was wol mitte morgen dô.\\ 
 & d\textit{ie} von d\textit{er} stat des wâren vrô,\\ 
 & die dise jost ersâhen.\\ 
 & \textbf{die} begunden alle gâhen\\ 
25 & an ir \textbf{werlîche} letze.\\ 
 & er was vor in ein netze:\\ 
 & waz dar under kam, daz was \textbf{erslagen}.\\ 
 & ein ander ors, sus \textbf{hôrt} ich sagen,\\ 
 & dâr ûf saz der werde.\\ 
30 & daz vlouc und ruorte die erde\\ 
\end{tabular}
\scriptsize
\line(1,0){75} \newline
U V T \newline
\line(1,0){75} \newline
\textbf{11} \textit{Initiale} U V T  \textbf{15} \textit{Majuskel} T  \textbf{19} \textit{Majuskel} T  \textbf{28} \textit{Majuskel} T  \newline
\line(1,0){75} \newline
\textbf{1} iu] \textit{om.} T \textbf{2} bestât] vnde bestât T  $\cdot$ Anschevin] Anschauin V Anscevin T \textbf{6} Gahmurete] Gamurette V \textbf{11} Kaylet] kylet U kaẏlet V \textbf{12} diz] daz V  $\cdot$ Gahmuret] Gahmuͦret U Gamuret V \textbf{13} Gandin] Gaudin U \textbf{15} ez] sin T \textbf{16} e [*]: das min oͮge baz ersiht V  $\cdot$ gesiht] siht T \textbf{19} dô] \textit{om.} T \textbf{20} Gahmuret] Gahmuͦret U Gamuret V \textbf{21} mitte] mitten V miter T \textbf{22} die von der] Do von die U die in der V  $\cdot$ des] die T \textbf{25} werlîche] wêrlichen T \textbf{27} waz] Swas V (T)  $\cdot$ erslagen] beslagen V (T) \newline
\end{minipage}
\end{table}
\end{document}
