\documentclass[8pt,a4paper,notitlepage]{article}
\usepackage{fullpage}
\usepackage{ulem}
\usepackage{xltxtra}
\usepackage{datetime}
\renewcommand{\dateseparator}{.}
\dmyyyydate
\usepackage{fancyhdr}
\usepackage{ifthen}
\pagestyle{fancy}
\fancyhf{}
\renewcommand{\headrulewidth}{0pt}
\fancyfoot[L]{\ifthenelse{\value{page}=1}{\today, \currenttime{} Uhr}{}}
\begin{document}
\begin{table}[ht]
\begin{minipage}[t]{0.5\linewidth}
\small
\begin{center}*D
\end{center}
\begin{tabular}{rl}
\textbf{596} & \textbf{gein} sînem getriwen wirte,\\ 
 & der in \textbf{vil} wênec irte\\ 
 & \textbf{alles}, \textbf{des} sîn wille gerte.\\ 
 & eines spers er in gewerte,\\ 
5 & daz was starc und unbeschaben.\\ 
 & er het ir manegez ûf erhaben\\ 
 & \textbf{dort} anderthalben ûf sînem plân.\\ 
 & Dô bat \textbf{in} mîn hêr Gawan\\ 
 & \textbf{überverte} schiere.\\ 
10 & in einem ussiere\\ 
 & vuort ern über \textbf{an}z lant,\\ 
 & dâ er den Turkoten vant,\\ 
 & wert unt hôch gemuot.\\ 
 & \textbf{er} was \textbf{vor} schanden \textbf{alsô} behuot,\\ 
15 & daz missewende an im verswant.\\ 
 & sîn prîs was \textbf{sô hôch} erkant,\\ 
 & swer gein im tjustierens pflac,\\ 
 & daz \textbf{der} hinderm orse \textbf{lac}\\ 
 & von sîner tjoste valle.\\ 
20 & sus \textbf{het} er si alle,\\ 
 & die gein im \textbf{ie} durch \textbf{prîs} geriten,\\ 
 & mit tjustieren überstriten.\\ 
 & ouch tet sich ûz der degen wert,\\ 
 & daz er mit spern \textbf{sunder} swert\\ 
25 & hôhen prîs wolte \textbf{erben}\\ 
 & oder sînen prîs verderben.\\ 
 & swer den prîs bezalte,\\ 
 & daz er in mit \textbf{tjoste} valte,\\ 
 & \textbf{dâ würder} \textbf{âne wer} \textbf{gesehen},\\ 
30 & dem wolt er sicherheit \textbf{verjehen}.\\ 
\end{tabular}
\scriptsize
\line(1,0){75} \newline
D Z Fr7 \newline
\line(1,0){75} \newline
\textbf{3} \textit{Initiale} Z  \textbf{8} \textit{Majuskel} D  \newline
\line(1,0){75} \newline
\textbf{1} gein] Zv Z \textbf{2} vil] des Z \textbf{3} alles des] Swes Z \textbf{7} dort] \textit{om.} Z \textbf{8} Dô] Da Z \textbf{12} Turkoten] Tvrkoiten Z \textbf{14} er] Der Z \textbf{16} was sô hôch] da fvͤr Z \textbf{18} lac] gelac Z \textbf{21} ie durch prîs] durch pris ie Z \textbf{24} sunder] ane Z \textbf{25} erben] erwerben Z \textbf{27} bezalte] an im bezalte Z \textbf{30} verjehen] iehen Z \newline
\end{minipage}
\hspace{0.5cm}
\begin{minipage}[t]{0.5\linewidth}
\small
\begin{center}*m
\end{center}
\begin{tabular}{rl}
 & \textbf{gegen} sînem getriuwen wirte,\\ 
 & der in \textbf{vil} wênic irte,\\ 
 & \textbf{al} \textbf{des} sîn wille gerte.\\ 
 & eines spers er in gewerte,\\ 
5 & daz was starc und unbeschaben.\\ 
 & er het ir manigez ûf erhaben\\ 
 & \textbf{dort} anderhalp ûf sînem plân.\\ 
 & dô bat mîn hêr Gawan\\ 
 & \textbf{überverte} schier.\\ 
10 & in einem ussier\\ 
 & vuort er in über \textbf{in} daz lant,\\ 
 & d\textit{â} er den Turkoi\textit{t}en vant,\\ 
 & wert und hôch gemuot.\\ 
 & \textbf{er} was \textbf{von} schanden \textbf{sô} behuot,\\ 
15 & daz missewende an im verswant.\\ 
 & sîn prîs was \textbf{sô hôhe} erkant,\\ 
 & wer gegen im justierens pflac,\\ 
 & daz \textbf{der} hinder dem ros \textbf{gelac}\\ 
 & von sîner juste valle.\\ 
20 & sus \textbf{het} er si alle,\\ 
 & die gegen im \textbf{ie} durch \textbf{prîs} geriten,\\ 
 & mit justieren überstriten.\\ 
 & ouch tete sich ûz der degen wert,\\ 
 & daz er mit sp\textit{e}rn \textbf{sunder} swert\\ 
25 & hôhen prîs wolte \textbf{erben}\\ 
 & oder sînen prîs verderben.\\ 
 & wer \textbf{aber} den prîs bezalt,\\ 
 & daz ern mit \textbf{juste} valt,\\ 
 & \textbf{d\textit{â} würde er} \textbf{âne wer} \textbf{gesehen},\\ 
30 & dem wolt er sicherheit \textbf{jehen}.\\ 
\end{tabular}
\scriptsize
\line(1,0){75} \newline
m n o \newline
\line(1,0){75} \newline
\newline
\line(1,0){75} \newline
\textbf{1} sînem] sinen o \textbf{2} wênic] venig o  $\cdot$ irte] irt n \textbf{3} wille] gewille o \textbf{8} hêr] herre her n \textbf{11} in daz] an das n o \textbf{12} dâ] Do m n o  $\cdot$ Turkoiten] turcoien m (n) tortoien o \textbf{16} sîn] So o \textbf{24} er] \textit{om.} n  $\cdot$ spern] sporn m \textbf{29} dâ] Do m n o \textbf{30} jehen] verjehen n (o) \newline
\end{minipage}
\end{table}
\newpage
\begin{table}[ht]
\begin{minipage}[t]{0.5\linewidth}
\small
\begin{center}*G
\end{center}
\begin{tabular}{rl}
 & \textbf{zuo} sînem getriuwen wirte,\\ 
 & der in \textbf{des} wênic irte,\\ 
 & \textbf{swes} sîn wille gerte.\\ 
 & eines spers er in gewerte,\\ 
5 & daz was starc unde unbeschaben.\\ 
 & er het ir manigez ûf erhaben\\ 
 & anderhalp ûf \textit{sîn}e\textit{m} plân.\\ 
 & dô bat \textit{\textbf{in}} mîn hêr Gawan\\ 
 & \textbf{übervarn} schiere.\\ 
10 & in einem \textit{urs}siere\\ 
 & vuort er in über \textbf{an} daz lant,\\ 
 & dâ er den Turkoiten vant,\\ 
 & \textit{w}ert unde hôc\textit{h}gemuot.\\ 
 & \textbf{der} was \textbf{vor} schanden \textbf{sô} behuot,\\ 
15 & daz missewende an im verswant.\\ 
 & sîn prîs \textit{was} \textbf{dâ vür} erkant,\\ 
 & swer gein im tjostierens pflac,\\ 
 & daz \textbf{er} hinder dem orse \textbf{gelac}\\ 
 & von sîner tjoste valle.\\ 
20 & sus \textbf{überreit} ers alle,\\ 
 & die gein im \textbf{ie} durch \textbf{strît} geriten,\\ 
 & mit tjostieren überstriten.\\ 
 & ouch tet sich ûz der degen wert,\\ 
 & daz er mit spern \textbf{âne} swert\\ 
25 & hôhen prîs wolt \textbf{erwerben}\\ 
 & oder sînen prîs \textbf{lân} verderben.\\ 
 & swer den prîs \textbf{an im} bezalte,\\ 
 & daz er in mit \textbf{tjostierne} valte,\\ 
 & \textbf{würde er dâ} \textbf{sigelôs} \textbf{ersehen},\\ 
30 & dem wolt er sicherheit \textbf{jehen}.\\ 
\end{tabular}
\scriptsize
\line(1,0){75} \newline
G I L M Z \newline
\line(1,0){75} \newline
\textbf{3} \textit{Initiale} L Z  \textbf{17} \textit{Initiale} I  \newline
\line(1,0){75} \newline
\textbf{1} getriuwen] getriwem I \textbf{2} wênic] niht L \textbf{3} swes] Wez L (M) \textbf{4} in gewerte] an in gerte L \textbf{5} unbeschaben] beslagen L vmmb scabin M \textbf{7} sînem] den G \textbf{8} dô] Da M Z  $\cdot$ in] \textit{om.} G  $\cdot$ hêr Gawan] ergawan M \textbf{9} übervarn] Vber verte L (M) Z \textbf{10} urssiere] vesiere G vrshiere I vsiere M \textbf{12} dâ] do I  $\cdot$ Turkoiten] turkoẏten G Turkoyden I turkoyten M \textbf{13} Vert vnde hohe gemuͦt G \textbf{14} vor] von I  $\cdot$ sô] also Z \textbf{16} was] \textit{om.} G Z  $\cdot$ erkant] bechant I (L) \textbf{17} swer] Wer L M  $\cdot$ tjostierens] tiosters M \textbf{18} er] der L Z  $\cdot$ gelac] belac M \textbf{19} sîner] syneme M \textbf{20} überreit] het L (M) Z \textbf{21} ie durch strît] ie durch strite G mit strite I durch pris ie Z  $\cdot$ geriten] riten I \textbf{22} überstriten] vor die vber striten I \textbf{26} lân] \textit{om.} L M Z \textbf{27} swer] Wer L M  $\cdot$ den] dem I \textit{om.} L \textbf{28} tjostierne] tioste L M Z \textbf{29} Da wurde er ane were gesehen Z  $\cdot$ ersehen] gesehen I \textbf{30} jehen] veriehen I \newline
\end{minipage}
\hspace{0.5cm}
\begin{minipage}[t]{0.5\linewidth}
\small
\begin{center}*T
\end{center}
\begin{tabular}{rl}
 & \textbf{zuo} sînem getriuwen wirte,\\ 
 & der in \textbf{des} wênic irte,\\ 
 & \textbf{wes} sîn wille gerte.\\ 
 & eines spers er in gewerte,\\ 
5 & daz was starc und unbeschaben.\\ 
 & er het ir manegez ûf erhaben\\ 
 & anderhalp ûf sînem plân.\\ 
 & dô bat \textbf{in} mîn hêr Gawan\\ 
 & \textbf{überverte} schiere.\\ 
10 & in einem ussiere\\ 
 & vuort er in über \textbf{an} daz lant,\\ 
 & d\textit{â} er den Turkoyten vant,\\ 
 & wert und hôch gemuo\textit{t}.\\ 
 & \textbf{der} was \textbf{vor} schanden \textbf{alsô} behuo\textit{t},\\ 
15 & daz missewende an im verswant.\\ 
 & sîn prîs was \textbf{dâ vür} erkant,\\ 
 & wer gên im tjostierens pflac,\\ 
 & daz \textbf{der} hinder\textit{m} ros \textbf{gelac}\\ 
 & von sîner tjost valle.\\ 
20 & sus \textbf{het} er si alle,\\ 
 & die gên im durch \textbf{strît} geriten,\\ 
 & mit tjostieren überstriten.\\ 
 & ouch tet sich ûz der degen wert,\\ 
 & daz er mit spern \textbf{âne} swert\\ 
25 & hôhen prîs wolte \textbf{erben}\\ 
 & oder sînen prîs verderben.\\ 
 & wer den prîs \textbf{an im} bezalte,\\ 
 & daz \textit{er} in mit \textbf{tjoste} valte,\\ 
 & \textbf{würd er d\textit{â}} \textbf{sigelôs} \textbf{ersehen},\\ 
30 & dem wolt er sicherheit \textbf{jehen}.\\ 
\end{tabular}
\scriptsize
\line(1,0){75} \newline
Q R W V U \newline
\line(1,0){75} \newline
\textbf{13} \textit{Überschrift:} >Wie Gawan den turkoyten nider stach vnde v́ber den Sabins daz ris brach< (am Spaltenende nachgetragen; ersetzt eine zwischenspaltig eingetragene, später radierte Rubrik, die zwischen die Verse 596.12 und 13 eingeschoben werden sollte) V  \newline
\line(1,0){75} \newline
\textbf{1} \textit{Die Verse 553.1-599.30 fehlen} U  \textbf{3} wes] Swes V \textbf{5} starc und] starb R \textbf{7} anderhalp] Dort anderhalb V  $\cdot$ sînem] sinen R \textbf{9} überverte] V́ber fuͦrte R \textbf{12} dâ] Do Q W V  $\cdot$ Turkoyten] turkoiten Q (V) \textbf{13} gemuot] gemute Q \textbf{14} schanden] schande W  $\cdot$ alsô] wol R so wol W  $\cdot$ behuot] behute Q \textbf{17} wer] Swer V  $\cdot$ tjostierens] strittes R \textbf{18} hinderm] hinders Q  $\cdot$ gelac] lag W \textbf{19} sîner tjost] sinem stiche R \textbf{20} sus] Als Q  $\cdot$ het] hat W \textbf{21} durch strît] [*]: ie durch pris V  $\cdot$ geriten] Ritten R \textbf{24} spern] sporn R (W) \textbf{25} erben] erwerben W V \textbf{26} sînen] sin R [*]: sinen V \textbf{27} wer] Swer V \textbf{28} er] \textit{om.} Q  $\cdot$ tjoste] strit R \textbf{29} dâ] do Q R W V  $\cdot$ ersehen] gesechen R \newline
\end{minipage}
\end{table}
\end{document}
