\documentclass[8pt,a4paper,notitlepage]{article}
\usepackage{fullpage}
\usepackage{ulem}
\usepackage{xltxtra}
\usepackage{datetime}
\renewcommand{\dateseparator}{.}
\dmyyyydate
\usepackage{fancyhdr}
\usepackage{ifthen}
\pagestyle{fancy}
\fancyhf{}
\renewcommand{\headrulewidth}{0pt}
\fancyfoot[L]{\ifthenelse{\value{page}=1}{\today, \currenttime{} Uhr}{}}
\begin{document}
\begin{table}[ht]
\begin{minipage}[t]{0.5\linewidth}
\small
\begin{center}*D
\end{center}
\begin{tabular}{rl}
\textbf{559} & \textbf{\begin{large}I\end{large}u} sînen prîs \textbf{hie} lâzen hât,\\ 
 & der manege rîterlîche \textbf{tât}\\ 
 & gevrümt hât, der süeze?\\ 
 & von rehte ich in \textbf{alsus} grüeze.\\ 
5 & mit ellen ist sîn rîterschaft.\\ 
 & sô manege tugent diu gotes kraft\\ 
 & in mannes herze nie gestiez,\\ 
 & ân Ithern von Gaheviez.\\ 
 & der Ithern vor Nantes sluoc,\\ 
10 & mîn schif in gestern übertruoc.\\ 
 & er hât mir vünf ors gegeben\\ 
 & - got in mit sælden lâze leben -,\\ 
 & di\textit{u} \textbf{herzogen} und \textbf{künege} riten.\\ 
 & swaz er hât \textbf{ab} \textbf{in} \textbf{erstriten},\\ 
15 & daz \textbf{wirt} ze Pelrapeire gesagt;\\ 
 & ir sicherheit \textbf{hât er} bejagt.\\ 
 & sîn schilt treit maneger tjoste mâl.\\ 
 & er reit \textbf{hie} \textbf{vorschen} umben Grâl."\\ 
 & Gawan sprach: "war ist \textbf{er} komen?\\ 
20 & saget \textbf{mir}, wirt, hât er vernomen,\\ 
 & dô er sô nâhe was hie bî,\\ 
 & \textbf{waz} \textbf{disiu} âventiure sî?"\\ 
 & "Hêrre, er\textbf{n} hât\textbf{s} niht ervarn.\\ 
 & ich kunde mich des wol bewarn,\\ 
25 & daz ich \textbf{es} im zuo gewüege:\\ 
 & \textbf{unvuoge} ich danne trüege.\\ 
 & het ir \textbf{selbe} vrâgens niht \textbf{erdâht},\\ 
 & \textbf{nimmer} wæret irs innen brâht\\ 
 & von mir, waz hie mæres ist,\\ 
30 & mit vorhten \textbf{scharpf} \textbf{ein} strenger list.\\ 
\end{tabular}
\scriptsize
\line(1,0){75} \newline
D \newline
\line(1,0){75} \newline
\textbf{1} \textit{Initiale} D  \textbf{23} \textit{Majuskel} D  \newline
\line(1,0){75} \newline
\textbf{8} Ithern] Jthern D \textbf{9} Ithern] Jthern D  $\cdot$ Nantes] Nates D \textbf{13} diu] di D \newline
\end{minipage}
\hspace{0.5cm}
\begin{minipage}[t]{0.5\linewidth}
\small
\begin{center}*m
\end{center}
\begin{tabular}{rl}
 & \textbf{iu} sînen prîs gelâzen hât,\\ 
 & der manige ritterlîch \textbf{getât}\\ 
 & gevr\textit{o}mt het, der süeze?\\ 
 & von reht ich in \textbf{alsus} grüeze.\\ 
5 & mit ellen ist sîn ritterschaft.\\ 
 & sô manige tugent diu gotes kraft\\ 
 & in mannes herz nie gestiez,\\ 
 & âne Ithern von Gaheviez.\\ 
 & der I\textit{t}hern vo\textit{r} Nantes sluoc,\\ 
10 & mîn schif in gestern übertruoc.\\ 
 & er het mir vünf ros gegeben\\ 
 & - got in mit sæl\textit{d}e\textit{n} lâz\textit{e} leben -,\\ 
 & diu \textbf{herzogen} und \textbf{künige} riten.\\ 
 & waz er het \textbf{ab} \textbf{in} \textbf{erstriten},\\ 
15 & daz \textbf{wart} zuo Pelraperie gesaget;\\ 
 & ir sicherheit \textbf{het er} bejaget.\\ 
 & sîn schilt treit maniger juste mâl.\\ 
 & er reit \textbf{hie} \textbf{vorschende} umb den Grâl."\\ 
 & Gawan sprach: "war ist \textbf{er} komen?\\ 
20 & sagt \textbf{mir}, wirt, h\textit{â}t \textit{e}r vernomen,\\ 
 & dô er sô nâhe was hie bî,\\ 
 & \textbf{daz} \textbf{disiu} âventiur sî?"\\ 
 & "hêrre, er het \textbf{es} niht ervarn.\\ 
 & ich kunde mich des wol bewarn,\\ 
25 & daz ich\textbf{s} im zuo gewüege:\\ 
 & \textbf{ungevüege} ich \textit{d}a\textit{n} trüege.\\ 
 & het ir \textbf{selbe} vrâgens niht \textbf{erdâht},\\ 
 & \textbf{nieme\textit{r}} wæret irs innen brâht\\ 
 & von mir, waz hie mæres ist,\\ 
30 & mit vorhten \dag starp\dag  \textbf{ein} strenger list.\\ 
\end{tabular}
\scriptsize
\line(1,0){75} \newline
m n o \newline
\line(1,0){75} \newline
\newline
\line(1,0){75} \newline
\textbf{3} gevromt] Gefremt m o \textbf{5} mit] Min n \textbf{8} Ithern] [jchern]: jthern m jetern o  $\cdot$ Gaheviez] gahevies m n o \textbf{9} Ithern] ichern m itern n jthern o  $\cdot$ vor] von m n o  $\cdot$ Nantes] nanttes m \textbf{11} ros] \textit{om.} o \textbf{12} sælden] selbe m selde o  $\cdot$ lâze] lassen m \textbf{13} herzogen] herczoge m hertzogin n (o)  $\cdot$ künige] konigin o \textbf{14} het] hette n \textbf{15} daz] Des o  $\cdot$ Pelraperie] pelrapeir n pelrapier o \textbf{16} ir] Jch o \textbf{18} vorschende] ferstende o \textbf{20} hât er] habt ir m  $\cdot$ vernomen] wernomen o \textbf{25} im] vmmb o  $\cdot$ gewüege] gefuͯge n fuͦge o \textbf{26} \textit{Vers 559.26 fehlt} n   $\cdot$ dan] kam m \textbf{27} selbe] selbes n selbens o  $\cdot$ vrâgens] frage o \textbf{28} niemer] Niemen m Miemer o  $\cdot$ irs] irs jrs o \newline
\end{minipage}
\end{table}
\newpage
\begin{table}[ht]
\begin{minipage}[t]{0.5\linewidth}
\small
\begin{center}*G
\end{center}
\begin{tabular}{rl}
 & \textbf{\begin{large}N\end{large}û} sînen brîs \textbf{hie} lâzen hât,\\ 
 & der manige rîterlîche \textbf{tât}\\ 
 & gevrümet hât, der süeze?\\ 
 & von rehte ich in \textbf{alsus} grüeze.\\ 
5 & mit ellen ist sîn rîterschaft.\\ 
 & sô manige tugent diu gotes kraft\\ 
 & in mannes herze nie gestiez,\\ 
 & âne Itheren von Kahaviez.\\ 
 & der Ithern vo\textit{r} Nantis sluoc,\\ 
10 & mîn schif in gester übertruoc.\\ 
 & er hât mir vünf ors gegeben\\ 
 & - got in mit sælden lâze leben -,\\ 
 & di\textit{u} \textbf{herzogen} unde \textbf{künige} riten.\\ 
 & swaz er hât \textbf{abe} \textbf{den} \textbf{gestriten},\\ 
15 & daz \textbf{wirt} ze Pelrapeire gesaget;\\ 
 & ir sicherheit \textbf{er hât} bejaget.\\ 
 & sîn schilt treit maniger tjoste mâl.\\ 
 & er reit \textbf{hie} \textbf{vorschende} umbe \textit{d}en Grâl."\\ 
 & Gawan sprach: "war ist \textbf{er} komen?\\ 
20 & saget \textbf{mir}, wirt, hât er vernomen,\\ 
 & dô er sô nâhen was hie bî,\\ 
 & \textbf{waz} \textbf{disiu} âventiure sî?"\\ 
 & "hêrre, er\textbf{n} hât \textbf{es} niht ervaren.\\ 
 & ich kunde mich des wol bewaren,\\ 
25 & daz ich \textbf{es} im zuo gewüege:\\ 
 & \textbf{ungevüege} ich danne trüege.\\ 
 & het ir \textbf{selbe} vrâgens niht \textbf{erdâht},\\ 
 & \textbf{nimmer} wært irs innen brâht\\ 
 & von mir, waz hie mæres ist,\\ 
30 & mit vorhten \textbf{scharpf} \textbf{ein} strenger list.\\ 
\end{tabular}
\scriptsize
\line(1,0){75} \newline
G I L M Z Fr23 \newline
\line(1,0){75} \newline
\textbf{1} \textit{Initiale} G Z Fr23  \textbf{19} \textit{Initiale} I  \newline
\line(1,0){75} \newline
\textbf{1} Nû] Vch L M (Z) (Fr23)  $\cdot$ hie] \textit{om.} Fr23  $\cdot$ lâzen] gilaszin M \textbf{4} in] \textit{om.} L \textbf{8} Itheren] Jtherin G Jthern I (M) Jhtern L Jchern Z  $\cdot$ Kahaviez] Gahauiez I Kaheviez L kahaviesz M Caheviez Z \textbf{9} der] Den Z  $\cdot$ Ithern] Jtern I Jhtern L Jthern M parcifal Z  $\cdot$ vor Nantis] von nantis G (M) vor nantes I (L) (Z) \textbf{11} \textit{Versfolge 559.12-11} L  \textbf{13} diu] Die G (I)  $\cdot$ herzogen unde künige riten] chunge herzogen vnd riter I herzcogin vnde konnigin riten M \textbf{14} swaz] Waz L (M)  $\cdot$ hât abe den] ab den hat I hat ab in L Z yn hat abe M  $\cdot$ gestriten] erstriten I L M \textbf{15} ze Pelrapeire] [zepeilp]: zepeilrapeire G zepairrapeir I \textbf{16} sicherheit] sichert L  $\cdot$ er hât] hat er M Z \textbf{17} maniger tjoste] manec Tioste I mangen tiost L \textbf{18} vorschende] vorschen L (M) (Z)  $\cdot$ den Grâl] engral G \textbf{21} dô] Da M \textbf{23} ern] er I  $\cdot$ es] si I \textbf{25} es im] ims I  $\cdot$ gewüege] gefuge M \textbf{26} ungevüege] Vnfuͯge L (M) (Z)  $\cdot$ ich] \textit{om.} I \textbf{27} selbe vrâgens niht] niht selbe vrage L selben vragens nicht M  $\cdot$ erdâht] gidacht M \textbf{28} wært] war L  $\cdot$ irs] ir sin I isz M \textbf{29} mæres] mêr I (L) \textbf{30} vorhten] vorshen I  $\cdot$ scharpf ein strenger] starchen strengen L \newline
\end{minipage}
\hspace{0.5cm}
\begin{minipage}[t]{0.5\linewidth}
\small
\begin{center}*T
\end{center}
\begin{tabular}{rl}
 & \textbf{iu} sînen prîs \textbf{hie} gelâzen hât,\\ 
 & der manege rîterlîche \textbf{tât}\\ 
 & gevrümt hât, der süeze?\\ 
 & von rehte ich in \textbf{sus} grüeze.\\ 
5 & mit ellen ist sîn rîterschaft.\\ 
 & sô manege tugent diu gotes kraft\\ 
 & in mannes herze nie gestiez,\\ 
 & âne Ithern von Kaheviez.\\ 
 & der Ithern vo\textit{r} Nantes sluoc,\\ 
10 & mîn schif in gester übertruoc.\\ 
 & er hât mir vünf ors gegeben\\ 
 & - got in mit sælden lâze leben -,\\ 
 & di\textit{u} \textbf{künege} unde \textbf{herzogen} riten.\\ 
 & swaz er hât \textbf{an} \textbf{in} \textbf{erstriten},\\ 
15 & daz \textbf{wirt} ze Peilrapere gesaget;\\ 
 & ir sicherheit \textbf{hât er} bejaget.\\ 
 & sîn schilt treit maneger tjoste mâl.\\ 
 & er reit \textbf{ie} \textbf{vorschen} umbe den Grâl."\\ 
 & \textit{\begin{large}G\end{large}}awan sprach: "war ist \textbf{der} komen?\\ 
20 & saget, \textbf{hêrre} wirt, hât \textit{er} vernomen,\\ 
 & dô er sô nâhe was hie bî,\\ 
 & \textbf{waz} \textbf{dirre} âventiure sî?"\\ 
 & "Hêrre, er hât \textbf{ir} niht ervarn.\\ 
 & ich kunde mich des wol bewarn,\\ 
25 & daz ich\textbf{z} im zuo gewüege:\\ 
 & \textbf{unvuoge} ich danne trüege.\\ 
 & het ir vrâgens niht \textbf{gedâht},\\ 
 & \textbf{niener} wæret irs innen brâht\\ 
 & von mir, swaz hie mæres ist,\\ 
30 & mit vorhten \textbf{starker}, strenger list.\\ 
\end{tabular}
\scriptsize
\line(1,0){75} \newline
T U V W Q R Fr25 Fr39 Fr40 \newline
\line(1,0){75} \newline
\textbf{17} \textit{Initiale} V Fr25 Fr39 Fr40   $\cdot$ \textit{Capitulumzeichen} R  \textbf{19} \textit{Initiale} T W  \textbf{23} \textit{Majuskel} T  \newline
\line(1,0){75} \newline
\textbf{1} \textit{Die Verse 553.1-599.30 fehlen} U   $\cdot$ iu sînen prîs] [*]: V́ch sinen pris V  $\cdot$ gelâzen] lasszen Q (Fr25) (Fr40) \textbf{4} sus] alsvs V (W) (Q) (R) Fr25 (Fr40) als::: Fr39 \textbf{5} ellen] eren Q allem R ellen: al Fr25 \textbf{6} manege] mage W mache Q mengú R  $\cdot$ diu] in Q \textbf{7} herze] hertzen W \textbf{8} Ithern] Jthern T ẏtern V ythern W [ichern]: ithern Q Jchtern R ẏhtern Fr39  $\cdot$ Kaheviez] kahevies V kahafies W Kachevies R kahaviez Fr25 kahewiez Fr40 \textbf{9} \textit{Versfolge 555.10-9} Fr25   $\cdot$ Ithern] Jthern T ytern V Jchtern R Jtern Fr25 ẏhtern Fr39  $\cdot$ vor] von T V W  $\cdot$ Nantes] nates T natis Q R Nantis Fr25 Fr40 kantis Fr39 \textbf{10} schif] hilff R \textbf{12} in mit sælden lâze] laze in mit selden V in mit frewde losz Q (Fr40) laus in mit erren R \textbf{13} diu] die T \textbf{14} swaz] Was W Q R  $\cdot$ an] ob W ab Q R Fr25 Fr39 Fr40 \textbf{15} wirt] ward W (Q) (Fr25) (Fr40)  $\cdot$ ze Peilrapere] ze Peilraper T zuͦ pelrapeir W (Fr39) zu pelrapeire Q (Fr40) zu palrapeire R zepeilrapiere Fr25 \textbf{16} hât er] er hat W R Fr25 Fr39 \textbf{17} treit] trvͦg V  $\cdot$ maneger] der W  $\cdot$ tjoste] strit R \textbf{18} ie] hie V W Q R Fr25 Fr39 Fr40  $\cdot$ vorschen] vorschende V Fr25 \textbf{19} Gawan] ÷Awan T Gawin R  $\cdot$ war ist der] war ist er V W (Q) R Fr40 war is: er Fr39 \textbf{20} hêrre] mir V  $\cdot$ er vernomen] vernomen T [*]: er vernomen V \textbf{21} dô] Da V Das W \textbf{22} dirre] dise V W (Q) Fr40 disu R (Fr25) \textbf{23} er hât] ern hat V (W) (Q) Fr40  $\cdot$ ir] ez V (Q) (R) (Fr25) Fr40 er W \textbf{25} ichz] ichs ich Q  $\cdot$ gewüege] gefuͦge W (R) \textbf{26} danne] dennoch V \textbf{27} gedâht] bedocht Q \textbf{28} niener] Niemer V (W) (Q) (R) (Fr25) (Fr40) \textbf{29} von] Vor Q  $\cdot$ swaz] was V W Q (R) (Fr25) (Fr40)  $\cdot$ mæres] mere Q (Fr25) (Fr40) \textbf{30} starker] stark ein V (W) (Q) R Fr40 scharf ein Fr25 :::ein Fr39 \newline
\end{minipage}
\end{table}
\end{document}
