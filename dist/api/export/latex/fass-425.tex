\documentclass[8pt,a4paper,notitlepage]{article}
\usepackage{fullpage}
\usepackage{ulem}
\usepackage{xltxtra}
\usepackage{datetime}
\renewcommand{\dateseparator}{.}
\dmyyyydate
\usepackage{fancyhdr}
\usepackage{ifthen}
\pagestyle{fancy}
\fancyhf{}
\renewcommand{\headrulewidth}{0pt}
\fancyfoot[L]{\ifthenelse{\value{page}=1}{\today, \currenttime{} Uhr}{}}
\begin{document}
\begin{table}[ht]
\begin{minipage}[t]{0.5\linewidth}
\small
\begin{center}*D
\end{center}
\begin{tabular}{rl}
\textbf{425} & \begin{large}E\end{large}r ist manheit unt ellens hêr.\\ 
 & der helt gebôt mir dennoch mêr,\\ 
 & daz ich ân \textbf{arge} liste\\ 
 & inre jâres vriste,\\ 
5 & ob ich\textbf{s Grâles erwürbe} niht,\\ 
 & daz ich ir k\textit{œ}me, der man giht\\ 
 & der krône ze Pelrapeire.\\ 
 & ir vater hiez Tampenteire.\\ 
 & swenne \textbf{si mîn ouge} an \textbf{sæhe},\\ 
10 & daz ich \textbf{sicherheit ir} \textbf{jæhe}.\\ 
 & Er enbôt ir, \textbf{ob} si dæhte an in,\\ 
 & daz wære an vreuden sîn gewin,\\ 
 & unt er wære\textbf{z}, der si lôste ê\\ 
 & von dem künege Clamide."\\ 
15 & Dô si die rede \textbf{erhôrten} sus,\\ 
 & dô sprach aber Liddamus:\\ 
 & "mit dirre hêrren urloube ich nuo\\ 
 & spriche. ouch râten si dar zuo.\\ 
 & swes iuch dort twanc \textbf{der eine} man,\\ 
20 & des sî hie pfant hêr Gawan.\\ 
 & der vederslagt ûf \textbf{iweren} k\textit{lo}ben.\\ 
 & bitet in iu vor uns allen loben,\\ 
 & daz er iu den Grâl gewinne.\\ 
 & lât in mit guoter minne\\ 
25 & von iu hinnen rîten\\ 
 & unt nâch dem Grâle strîten.\\ 
 & die scham \textbf{wir alle m\textit{üe}sen} \textbf{klagen},\\ 
 & würder \textbf{in iwerem} hûs erslagen.\\ 
 & \textbf{nû} vergebt im sîne schulde\\ 
30 & durch iwerre swester hulde.\\ 
\end{tabular}
\scriptsize
\line(1,0){75} \newline
D Fr1 Fr5 \newline
\line(1,0){75} \newline
\textbf{1} \textit{Initiale} D Fr5  \textbf{11} \textit{Majuskel} D  \textbf{15} \textit{Initiale} Fr1   $\cdot$ \textit{Capitulumzeichen} Fr5   $\cdot$ \textit{Majuskel} D  \newline
\line(1,0){75} \newline
\textbf{1} Er] Ez Fr5 \textbf{4} inre] Minir Fr5 \textbf{5} ichs] ich Fr5 \textbf{6} ich] \textit{om.} Fr5  $\cdot$ kœme] chome D (Fr5) \textbf{7} ze] [*i]: bi Fr5  $\cdot$ Pelrapeire] Pelrapeẏre Fr1 peilrapeire Fr5 \textbf{8} Tampenteire] Tampenteẏre Fr1 tampeteire Fr5 \textbf{10} sicherheit ir] ir sichirheit Fr5 \textbf{11} dæhte] gidæhte Fr5 \textbf{12} wære] wrde Fr1 \textbf{13} vnt er hete si erloͤset e Fr1 \textbf{14} Clamide] Chlammide D Chlamide Fr1 \textbf{15} erhôrten] gehorten Fr1 \textbf{16} aber] \textit{om.} Fr5  $\cdot$ Liddamus] Lyddamvs Fr1 \textbf{17} Mit dierre vrlop ir sprich nv Fr5 \textbf{18} spriche] \textit{om.} Fr5  $\cdot$ dar] dir Fr5 \textbf{20} Gawan] Gauwan Fr5 \textbf{21} iweren] iwerm Fr1  $\cdot$ kloben] cholbn D \textbf{22} iu] \textit{om.} Fr5 \textbf{23} gewinne] giwunne Fr5 \textbf{25} iu] vns Fr1 \textbf{27} daz laster mohte wir niht verchlagn Fr1  $\cdot$ alle müesen] alle mvͦsen D mvͦesin alle Fr5 \textbf{29} nû] vnt Fr1 \newline
\end{minipage}
\hspace{0.5cm}
\begin{minipage}[t]{0.5\linewidth}
\small
\begin{center}*m
\end{center}
\begin{tabular}{rl}
 & er ist manheit und ellens hêr.\\ 
 & der helt gebôt mir dennoch mêr,\\ 
 & daz ich âne liste\\ 
 & inre jâres vriste,\\ 
5 & ob ich \textbf{des Grâles erwürb\textit{e}} niht,\\ 
 & daz \textit{ich} ir k\textit{œ}me, der man giht\\ 
 & der krône ze Pelraperie.\\ 
 & ir vater hie Tampenterie.\\ 
 & wenne \textbf{mîn ouge si} an \textbf{gesæhe},\\ 
10 & daz ich \textbf{ir sicherheit} \textbf{verjæhe}.\\ 
 & er enbôt ir, \textbf{daz} si d\textit{æ}hte an in,\\ 
 & daz wære an vrœden sîn gewin,\\ 
 & und er wære, der si lôste ê\\ 
 & von dem künige Clamide."\\ 
15 & \begin{large}D\end{large}ô si die rede \textbf{erhôrten} sus,\\ 
 & dô sprach aber Liddamus:\\ 
 & "mit dirre hêrren urloube ich nû\\ 
 & sprich. ouch râten \textbf{ime} si dar zuo.\\ 
 & wes iuch dort twanc \textbf{jener} man,\\ 
20 & des sî hie pfant hêr G\textit{a}wan.\\ 
 & der vederslag\textit{t} ûf \textbf{iuweren} kloben.\\ 
 & bittet in iu vor uns allen loben,\\ 
 & daz er iu den Grâl gewinne,\\ 
 & \textbf{und} lât in mit guoter minne\\ 
25 & von iu hinnan rîten\\ 
 & und nâch dem Grâle strîten.\\ 
 & die schame \textbf{müesen wir alle} \textbf{klagen},\\ 
 & würde er \textbf{im} hûse erslagen,\\ 
 & \textbf{und} vergebet ime sîne schulde\\ 
30 & durch iuwerre swester hulde.\\ 
\end{tabular}
\scriptsize
\line(1,0){75} \newline
m n o \newline
\line(1,0){75} \newline
\textbf{15} \textit{Initiale} m   $\cdot$ \textit{Capitulumzeichen} n  \newline
\line(1,0){75} \newline
\textbf{3} ich] ist n \textbf{4} inre] Jn des n o \textbf{5} erwürbe] erwúrber m \textbf{6} ich ir kœme] ẏr kome m ich ir kome n ich kumme o  $\cdot$ der man] dar man n dar an o \textbf{7} ze] von n  $\cdot$ Pelraperie] pelrapeir n pelrapier o \textbf{8} Tampenterie] tampanteir n tampantue o \textbf{9} gesæhe] sehe o \textbf{11} ir] \textit{om.} o  $\cdot$ dæhte] dohte m gedecht n \textbf{12} vrœden] friden o \textbf{15} erhôrten] erhorte o \textbf{16} Liddamus] liddamuͯs o \textbf{18} ime] \textit{om.} n o \textbf{19} iuch] auch o \textbf{20} des] Das o  $\cdot$ Gawan] gewan m o \textbf{21} vederslagt] veder slag m vedern slaht o  $\cdot$ iuweren] yren m (o) irem n \textbf{23} Grâl] grole n \textbf{25} \textit{Versdoppelung (²o); Lesarten des vorausgehenden Verses mit ¹o bezeichnet} o   $\cdot$ von iu] Vnd auch \textsuperscript{2}\hspace{-1.3mm} o \textbf{27} müesen] mussen m muͯssent n (o) \textbf{28} im] in uwerem n (o) \textbf{30} iuwerre] ire m \newline
\end{minipage}
\end{table}
\newpage
\begin{table}[ht]
\begin{minipage}[t]{0.5\linewidth}
\small
\begin{center}*G
\end{center}
\begin{tabular}{rl}
 & er ist manheit unde ellens hêr.\\ 
 & der helt gebôt mir dannoch mêr,\\ 
 & daz ich âne \textbf{arge} liste\\ 
 & inner jâres vriste,\\ 
5 & obe ich \textbf{erwürbe des Grâles} niht,\\ 
 & daz ich ir k\textit{œ}me, der man \textbf{dâ} giht\\ 
 & der krône ze Pelrapeire.\\ 
 & ir vater hiez Tampunteire.\\ 
 & swenne \textbf{si mîn ouge} an \textbf{sæhe},\\ 
10 & daz ich \textbf{sicherheit ir} \textbf{jæhe}.\\ 
 & \begin{large}E\end{large}r enbôt ir, \textbf{ob} si d\textit{æ}hte an in,\\ 
 & daz wære an vröuden sîn gewin,\\ 
 & \textit{und} er w\textit{æ}r\textbf{z}, der si lôste ê\\ 
 & von dem künige Clamide."\\ 
15 & dô si die rede \textbf{hôrten} sus,\\ 
 & dô sprach aber Lidamus:\\ 
 & "mit dirre hêrren urloube ich nû\\ 
 & spriche. ouch râten si dar zuo.\\ 
 & swes iuch dort twanc \textbf{der eine} man,\\ 
20 & des sî hie pfant hêr Gawan.\\ 
 & der vederslaget ûf \textbf{iweren} kloben.\\ 
 & bit in iu vor uns allen loben,\\ 
 & daz er iu den Grâl gewinne.\\ 
 & lât in mit guoter minne\\ 
25 & von iu hinnen rîten\\ 
 & unde nâch dem Grâle strîten.\\ 
 & die scham \textbf{wir alle m\textit{üe}sen} \textbf{klagen},\\ 
 & würder \textbf{in iwerem} hûse erslagen.\\ 
 & \textbf{nû} vergebet im sîne schulde\\ 
30 & durch iwerre swester hulde.\\ 
\end{tabular}
\scriptsize
\line(1,0){75} \newline
G I O L M Q R Z Fr21 \newline
\line(1,0){75} \newline
\textbf{1} \textit{Initiale} I O L Q Z Fr21   $\cdot$ \textit{Capitulumzeichen} R  \textbf{11} \textit{Initiale} G  \textbf{15} \textit{Initiale} I  \newline
\line(1,0){75} \newline
\textbf{1} er] ÷r O Es Q  $\cdot$ manheit] manlich I  $\cdot$ ellens] ellensriche I eren Q ellent R  $\cdot$ hêr] [riche]: her Fr21 \textbf{2} der helt] Er R \textbf{3} arge] argen I R \textbf{4} inner] Jn eynes M Jn der Z \textbf{5} Ob ich des grals er wirbe niht O  $\cdot$ Ob ichs Grals erwuͯrbe niht L (M) (Q) (R) (Z) (Fr21) \textbf{6} ich ir] ich dar M ich Q zu der R  $\cdot$ kœme] chome G I (O) (L) (M) (Q) (R) (Fr21)  $\cdot$ der man dâ] der man O (M) (Q) (R) Z (Fr21) dar man L  $\cdot$ giht] spricht M \textbf{7} der] Die R  $\cdot$ ze Pelrapeire] zepailrapeir I [zePelrapere]: zePelrapeire O zcu pelrapere M zu pelrapeyre Q zu pelarapiere R zepelrapeir Fr21 \textbf{8} hiez] haizt der chunc I  $\cdot$ Tampunteire] Tanpantaier I tampvteire O Tampuntiere R tampvnteir Fr21 \textbf{9} Wanne sie myn ougen an sehen M (Q) (R)  $\cdot$ swenne] Wenne L  $\cdot$ si] sin I \textbf{10} sicherheit ir] ir sicherhait I (R) fiantze ir L sicherheit Q  $\cdot$ jæhe] eriehe Q \textbf{11} Er enbôt] Erein bot L Er erbot R Der enbot Fr21  $\cdot$ ob si] obs Fr21  $\cdot$ dæhte] dahte G gedaht L gedechet Q \textbf{13} und er] er G vnde O  $\cdot$ wærz] warz G (L) were I R  $\cdot$ lôste] erloste R \textbf{14} Clamide] chlamide I Glamide O \textbf{15} dô] Da M  $\cdot$ die] dr O  $\cdot$ hôrten] er horten O (L) (M) (Q) (Z) (Fr21)  $\cdot$ sus] alsus R \textbf{16} dô] Da M  $\cdot$ Lidamus] Lyddamvs O Liddamuͯs L litdamuͯs M liddanus Q Liddiganius R liddamus Z (Fr21) \textbf{17} nû] daz tuͦ R \textbf{18} spriche] \textit{om.} R  $\cdot$ râten] [kautent]: Rautent R \textbf{19} swes] Wes O (L) M Q R  $\cdot$ iuch] \textit{om.} O  $\cdot$ dort twanc] zwang doͯrt R  $\cdot$ eine] enig R \textbf{20} hie pfant] phant hie I  $\cdot$ hêr Gawan] ergawan M Gawin der an R \textbf{21} der] des I  $\cdot$ vederslaget] vedirslachte M widerlege R  $\cdot$ iweren] ivrem O (L) (M) (Z)  $\cdot$ kloben] clobe R \textbf{22} in] ir L ich Q  $\cdot$ iu] \textit{om.} R  $\cdot$ allen] alle M allen daz Fr21 \textbf{23} er iu] ir uͯch L er nun R euch Q \textbf{24} guoter minne] guͦtten sinnen R \textbf{27} scham] shande I (R)  $\cdot$ wir] mir Q  $\cdot$ alle] \textit{om.} O  $\cdot$ müesen] moͮsen G (I) (O) (L) (M) (Q) (R) (Z) (Fr21)  $\cdot$ klagen] tragen L clage M \textbf{30} iwerre] ewren Q \newline
\end{minipage}
\hspace{0.5cm}
\begin{minipage}[t]{0.5\linewidth}
\small
\begin{center}*T
\end{center}
\begin{tabular}{rl}
 & Er ist manheit unde ellens hêr.\\ 
 & der helt gebôt mir dannoch mêr,\\ 
 & daz ich âne \textbf{valsche} liste\\ 
 & inre jâres vriste,\\ 
5 & ob ich\textbf{s Grâles erwürbe} niht,\\ 
 & daz ich ir k\textit{œ}me, der man giht\\ 
 & der krône ze Peilrapere.\\ 
 & ir vater hiez Tampuntere.\\ 
 & swenne \textbf{\textit{si} mîn ouge} an \textbf{sæhe},\\ 
10 & daz ich \textbf{sicherheit ir} \textbf{jæhe}.\\ 
 & er enbôt ir, \textbf{ob} si dæhte an in,\\ 
 & daz wære an vröuden sîn gewin,\\ 
 & unde er wære\textbf{z}, der si lôste ê\\ 
 & von dem künege Clamide."\\ 
15 & \begin{large}D\end{large}ô si die rede \textbf{hôrten} sus,\\ 
 & dô sprach aber Lyddamus:\\ 
 & "mit dirre hêrren urloube ich nû\\ 
 & spriche. ouch râten si dar zuo.\\ 
 & swes iuch dort twanc \textbf{der eine} man,\\ 
20 & des sî hie pfant hêr Gawan.\\ 
 & der vederslaget ûf \textbf{iuwerm} kloben.\\ 
 & bitet  iu vor \dag in\dag  allen loben,\\ 
 & daz er iu den \textit{Grâl} gewinne.\\ 
 & lât in mit guoter minne\\ 
25 & von iu hinnen rîten\\ 
 & unde nâch dem Grâle strîten.\\ 
 & die schame \textbf{wir alle müesen} \textbf{tragen},\\ 
 & würder \textbf{in iuwerm} hûse erslagen.\\ 
 & \textbf{nû} vergebt im sîne schulde\\ 
30 & durch iuwer swester hulde.\\ 
\end{tabular}
\scriptsize
\line(1,0){75} \newline
T U V W \newline
\line(1,0){75} \newline
\textbf{1} \textit{Majuskel} T  \textbf{15} \textit{Initiale} T U W  \newline
\line(1,0){75} \newline
\textbf{3} valsche] \textit{om.} U arge V W \textbf{4} inre] Jn U Jn dez V \textbf{6} kœme] come T (U) (V) gebe W  $\cdot$ der] [*]: der V \textbf{7} Peilrapere] peylrapere V pelrapeir W \textbf{8} Tampuntere] Tampvntêre T Tampuͦntere U tamputere V tampenteir W \textbf{9} swenne si] swenne in T Wan sie U Wenn sy in in W \textbf{11} ob] [*]: daz V \textbf{13} wærez] were U V (W)  $\cdot$ lôste] erlost W \textbf{14} Clamide] [Tamide]: Clamide T klamide W \textbf{15} hôrten] gehorten U V erhorten W \textbf{16} Lyddamus] Lyddamuͦs U [lidtam*]: lidtamus V lydamuß W \textbf{17} dirre] disen U diser W  $\cdot$ urloube] erleibe U  $\cdot$ ich] [*]: ich V \textit{om.} W \textbf{19} swes] Wes U W  $\cdot$ iuch] îv T  $\cdot$ der eine] [*]: ienre V der einig W \textbf{21} vederslaget] [*]: vederslaget V vederslag W  $\cdot$ iuwerm] [*]: uwerem V \textbf{22} Bietet vns vor vns allen loben U  $\cdot$ Bittent in v́ch vor vns allen loben V  $\cdot$ Bit in vor vnß allen loben W \textbf{23} Grâl] \textit{om.} T \textbf{24} lât in] [*ant]: vnde lant V \textbf{27} tragen] [*]: clagen V klagen W \textbf{29} nû] [*]: Nv V \newline
\end{minipage}
\end{table}
\end{document}
