\documentclass[8pt,a4paper,notitlepage]{article}
\usepackage{fullpage}
\usepackage{ulem}
\usepackage{xltxtra}
\usepackage{datetime}
\renewcommand{\dateseparator}{.}
\dmyyyydate
\usepackage{fancyhdr}
\usepackage{ifthen}
\pagestyle{fancy}
\fancyhf{}
\renewcommand{\headrulewidth}{0pt}
\fancyfoot[L]{\ifthenelse{\value{page}=1}{\today, \currenttime{} Uhr}{}}
\begin{document}
\begin{table}[ht]
\begin{minipage}[t]{0.5\linewidth}
\small
\begin{center}*D
\end{center}
\begin{tabular}{rl}
\textbf{121} & \textbf{lûte rief der knappe} sân:\\ 
 & "hilf got, dû maht wol helfe hân!"\\ 
 & der \textbf{vorder} zornes sich bewac,\\ 
 & dô der knappe ime pfade lac.\\ 
5 & "dirre tœrsche Wâleise\\ 
 & unsich wendet \textbf{gâher} reise."\\ 
 & \textbf{einen} prîs, den wir Beier tragen,\\ 
 & muoz ich von Wâleisen sagen:\\ 
 & die sint tœrscher denne \textbf{Beiersch} her\\ 
10 & unt doch bî manlîcher wer.\\ 
 & swer in den zwein landen wirt,\\ 
 & gevuoge ein wunder an im birt.\\ 
 & \begin{large}D\end{large}ô kom geleischieret\\ 
 & unt wol gezimieret\\ 
15 & ein ritter, dem was harte gâch.\\ 
 & er \textbf{reit} in striteclîchen nâch,\\ 
 & die verre \textbf{wâren von im} komen.\\ 
 & zwêne ritter heten im genomen\\ 
 & eine \textbf{juncvrouwen} in sîme lande.\\ 
20 & den helt \textbf{ez dûhte} schande.\\ 
 & in müete der juncvrouwen leit,\\ 
 & diu jæmerlîche vor \textbf{in} reit.\\ 
 & dise drî wâren sîne man.\\ 
 & \textbf{er} reit ein schœne kastelân.\\ 
25 & sînes schildes was vil wênic ganz.\\ 
 & er hiez Karnahkarnanz\\ 
 & Lah cons Ulterlec.\\ 
 & \textbf{er} sprach: "wer irret uns den wec?"\\ 
 & sus \textbf{vuor} er zuome knappen sân;\\ 
30 & den dûht er als ein got getân.\\ 
\end{tabular}
\scriptsize
\line(1,0){75} \newline
D \newline
\line(1,0){75} \newline
\textbf{13} \textit{Initiale} D  \textbf{27} \textit{Majuskel} D  \newline
\line(1,0){75} \newline
\textbf{9} Beiersch] beiersc D \textbf{26} Karnahkarnanz] [karnahkar*anz]: karnahkarnanz D \textbf{27} Ulterlec] vlterlech D \newline
\end{minipage}
\hspace{0.5cm}
\begin{minipage}[t]{0.5\linewidth}
\small
\begin{center}*m
\end{center}
\begin{tabular}{rl}
 & \textbf{und begunde dâ mite ruofen} sân:\\ 
 & "hilf got, dû maht wol helfe hân!"\\ 
 & der \textbf{vorder} zornes sich bewac,\\ 
 & dô der knappe in dem pfade lac.\\ 
5 & \textbf{er sprach}: "dirre tœrsche Wâleise\\ 
 & un\textit{s} wendet \textbf{gæhe} reise."\\ 
 & \textbf{einen} prîs, den wir Peiere tragen,\\ 
 & muoz ich von Wâleisen sagen:\\ 
 & die sint \dag toͯrftter\dag  dane \textbf{Beiers} her\\ 
10 & und doch bî manlîcher wer.\\ 
 & wer in den zweigen landen wirt,\\ 
 & gevuoge ein wunder an ime birt.\\ 
 & \begin{large}D\end{large}ô kam geleischieret\\ 
 & und wol gezimieret\\ 
15 & ein ritter, dem was h\textit{a}rte gâch.\\ 
 & er \textbf{reit} in strîteclîchen nâch,\\ 
 & d\textit{ie} verre \textbf{von im w\textit{â}ren} komen.\\ 
 & z\textit{w}ên ritter heten im genomen\\ 
 & eine \textbf{juncvrouwen} in sînem lande.\\ 
20 & den helt \textbf{ez dûhte} schande.\\ 
 & in müegete der juncvrouwen leit,\\ 
 & diu jâmerlîch vor \textbf{ime} reit.\\ 
 & dise drîe wâren sîne man.\\ 
 & \textbf{ein} reit ein schœne kastelân.\\ 
25 & sînes schiltes was vil wênic ganz.\\ 
 & er hiez Karnachkarnanz\\ 
 & la cons Ulterlec.\\ 
 & \textbf{er} sprach: "wer irret uns den wec?"\\ 
 & sus \textbf{reit} er zuo de\textit{m} knappen sân;\\ 
30 & den dûht er als ein got getân.\\ 
\end{tabular}
\scriptsize
\line(1,0){75} \newline
m n o \newline
\line(1,0){75} \newline
\textbf{13} \textit{Initiale} m   $\cdot$ \textit{Capitulumzeichen} n  \newline
\line(1,0){75} \newline
\textbf{3} zornes] zorn o \textbf{6} uns] Vnd m  $\cdot$ gæhe] gegen o \textbf{7} wir] mir m  $\cdot$ Peiere] peẏer n [beig*]: beiger o \textbf{8} Wâleisen] walleissen m \textbf{9} toͯrftter] torstiger n o  $\cdot$ Beiers] peẏers n \textbf{12} \textit{Vers 121.12 fehlt} n  \textbf{13} geleischieret] geloischieret n golaschieret o \textbf{15} harte] herte m \textbf{17} die] Der m  $\cdot$ wâren] werren m \textbf{18} zwên] Zen m \textbf{19} juncvrouwen] jungfrouwe n (o) \textbf{21} müegete] muͯget n (o) \textbf{24} schœne] schonen n (o)  $\cdot$ kastelân] cappelan o \textbf{26} Karnachkarnanz] karnach karnancz m kanoch kurnantz n kornach kornancz o \textbf{27} Ulterlec] ulter leg m v́lter leg n vlter log o \textbf{29} dem] den m \newline
\end{minipage}
\end{table}
\newpage
\begin{table}[ht]
\begin{minipage}[t]{0.5\linewidth}
\small
\begin{center}*G
\end{center}
\begin{tabular}{rl}
 & \textbf{vil lûte rief der knappe} sân:\\ 
 & "hilf got, dû maht wol helfe hân!"\\ 
 & der \textbf{vordere} zornes sich bewac,\\ 
 & dô der knappe in dem pfade lac.\\ 
5 & "dirre tœrsche Wâleise\\ 
 & uns wendet \textbf{gæhe} reise."\\ 
 & \textbf{den} prîs, den wir Beiger tragen,\\ 
 & muoz ich von Wâleisen sagen:\\ 
 & die sint tœrscher danne \textbf{Beige\textit{r}sch} her\\ 
10 & unt doch bî manlîcher wer.\\ 
 & swer in den zwein landen wirt,\\ 
 & gevuoge ein wunder an im birt.\\ 
 & dô kom geleisiert\\ 
 & unde wol gezimiert\\ 
15 & ein rîter, dem was harte gâch.\\ 
 & er \textbf{reit} in strîticlîchen nâch,\\ 
 & die verre \textbf{v\textit{on} i\textit{m} wâren} komen.\\ 
 & zwêne rîter heten im genomen\\ 
 & eine \textbf{vrouwen} in sînem lande.\\ 
20 & den helt \textbf{ez dûhte} schande.\\ 
 & in muote der juncvrouwen leit,\\ 
 & diu jæme\textit{r}lîchen vor \textbf{im} reit.\\ 
 & dise drî wâren sîne man.\\ 
 & \textbf{er} reit ein schœne kastelân.\\ 
25 & sînes schilte\textit{s} was vil wênic ganz.\\ 
 & er hiez Karnakarnanz\\ 
 & lech cuns Ultrec.\\ 
 & \textbf{er} sprach: "wer irret uns den wec?"\\ 
 & sus \textbf{vuor} er zuo dem knappen sân;\\ 
30 & den dûhter als \textit{ein} got getân.\\ 
\end{tabular}
\scriptsize
\line(1,0){75} \newline
G I O L M Q R Z Fr36 \newline
\line(1,0){75} \newline
\textbf{1} \textit{Initiale} O  \textbf{13} \textit{Initiale} I L R Z Fr36  \textbf{23} \textit{Initiale} M  \newline
\line(1,0){75} \newline
\textbf{1} vil] ÷il O \textbf{2} dû maht wol] mochstu M \textbf{3} der vordere] Der vor des M Do wider Q  $\cdot$ bewac] bewanck Q \textbf{4} dô] Da Z \textbf{5} Wâleise] waleisze L Q [wel]: waleise R \textbf{6} uns] Vnd R  $\cdot$ gæhe] Gaher I (L) (M) \textbf{7} Beiger] baiere I paier O beyer L beger M beier Q Z hie R \textbf{8} muoz ich] mac ich wol I Muste ich M  $\cdot$ von] von dem L  $\cdot$ Wâleisen] waleise O waleisin M wallaisen R \textbf{9} die] Da M Sie Q  $\cdot$ Beigersch] beigesch G baierisch I pairsch O beyersz L beiers M beiersh Q birscher R beirisch Z \textbf{11} swer] Wer L M Q R  $\cdot$ den] \textit{om.} R  $\cdot$ wirt] wert M \textbf{12} ein] \textit{om.} O an Q  $\cdot$ an im] drane I \textbf{13} dô] Da Z  $\cdot$ geleisiert] geloisiert G gen leisieret R \textbf{15} harte gâch] vngemach Fr36 \textbf{16} er] Der Fr36 \textbf{17} verre] vier I  $\cdot$ von im wâren] fur in waren G vor im waren L waren von im Z Fr36 \textbf{18} im] in O  $\cdot$ genomen] benomen O (Q) \textbf{20} ez dûhte] ducht es R \textbf{21} muote] muͤt I (O) (Z) (Fr36) muͦtten R \textbf{22} jæmerlîchen] iamelichen G \textbf{24} er] Einer Q \textbf{25} sînes schiltes] sines schilte G Sines schiltes dez L Sin schilt Fr36  $\cdot$ vil] \textit{om.} I L R  $\cdot$ wênic] lvzel O (L) (M) (Q) \textbf{26} Karnakarnanz] karnagar nanz I [kana]: kanah karnanz L karnacarcaz M karnahkarnancz Q karnah karnancz R karnach karnantz Z \textbf{27} lech cuns] lech chunc I Ze chvͦns O Zetvns L  $\cdot$ Ultrec] vterlech O vlter lec L (R) vterlich M vrterlec Q vlterlech Z \textbf{28} er] \textit{om.} I Der O (R) \textbf{29} \textit{Vers 121.29 fehlt} M   $\cdot$ sus] Ausz Q  $\cdot$ dem] den R \textbf{30} den dûhter] Er duͯhte in L Den duchte M Den do uch er Q  $\cdot$ als] alsam M  $\cdot$ ein] \textit{om.} G \newline
\end{minipage}
\hspace{0.5cm}
\begin{minipage}[t]{0.5\linewidth}
\small
\begin{center}*T (U)
\end{center}
\begin{tabular}{rl}
 & \textbf{lûte rief der knabe} sân:\\ 
 & "hilf got, dû maht wol helfe hân!"\\ 
 & der \textbf{vorderste} zornes sich bewac,\\ 
 & dô der knabe in dem pfade lac.\\ 
5 & "dirre tœrischer Wâleise\\ 
 & uns wendet \textbf{gâher} reise."\\ 
 & \textbf{den} prîs, den wir Beier tragen,\\ 
 & muoz ich von Wâleise\textit{n} sagen:\\ 
 & die sint tœrischer dan \textbf{Beiersch} her\\ 
10 & und doch bî manlîcher wer.\\ 
 & wer in den zwein landen wirt,\\ 
 & gevuoge ein wunder an im birt.\\ 
 & \begin{large}D\end{large}ô kam geleisieret\\ 
 & und wol gezimieret\\ 
15 & ein ritter, dem was harte gâch.\\ 
 & er \textbf{vuor} in strîtlîche nâch,\\ 
 & die verre \textbf{vor in wâren} komen.\\ 
 & zwêne rîter heten im genomen\\ 
 & eine \textbf{vrouwen} in sîme lande.\\ 
20 & den helt \textbf{dûht ez} schande.\\ 
 & in müete der juncvrouwen leit,\\ 
 & diu jæmerlîche vo\textit{r} \textbf{im} reit.\\ 
 & dise drî wâren sîne man.\\ 
 & \textbf{er} reit ein schœne kastelân.\\ 
25 & sînes schiltes was vil wênic ganz.\\ 
 & er hiez Garnagarnanz\\ 
 & lech cuns Urtelec.\\ 
 & \textbf{der} sprach: "wer irret uns den wec?"\\ 
 & sus \textbf{vuor} er zuo dem knaben sân;\\ 
30 & den dûht er als ein got getân.\\ 
\end{tabular}
\scriptsize
\line(1,0){75} \newline
U V W T \newline
\line(1,0){75} \newline
\textbf{1} \textit{Majuskel} T  \textbf{3} \textit{Initiale} T  \textbf{5} \textit{Majuskel} T  \textbf{7} \textit{Majuskel} T  \textbf{13} \textit{Initiale} U V W   $\cdot$ \textit{Majuskel} T  \textbf{23} \textit{Majuskel} T  \textbf{29} \textit{Majuskel} T  \newline
\line(1,0){75} \newline
\textbf{1} lûte] Vil lvte V (T) \textbf{3} vorderste] [*]: vorder V ritter W vorder T \textbf{4} in] svs in T \textbf{5} tœrischer] toͤrsche V (W) (T)  $\cdot$ Wâleise] walleise V walêise T \textbf{6} gâher] gahe V (W) \textbf{7} Beier] [b*]: beiger V bayer W bêier T \textbf{8} muoz ich] Ich muͦß nun W  $\cdot$ von] von den V  $\cdot$ Wâleisen] waleise U walleisen V \textbf{9} Beiersch] beiers V bayersch W beiersc T \textbf{11} wer] Swer V (T) \textbf{12} gevuoge] Genuͦg W \textbf{13} geleisieret] gelesieret V galaysieret W [gelaiseieret]: gelaisieret T \textbf{15} harte] vaste W \textbf{16} er] Der W  $\cdot$ vuor] reit T \textbf{17} vor in wâren] von im waren W wâren von im T \textbf{18} im] in V \textbf{19} vrouwen] vreuͦwe U \textbf{20} dûht ez] dvht ez were ein V es dauchte W (T) \textbf{22} vor] von U \textbf{26} Garnagarnanz] garnagarnantz V gurnagarnantz W \textbf{27} Urtelec] vrteleg V freilet W vlter lech T \textbf{28} der] [Er]: Der V \textbf{29} er] der V  $\cdot$ zuo] nach T  $\cdot$ dem] den W \newline
\end{minipage}
\end{table}
\end{document}
