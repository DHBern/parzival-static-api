\documentclass[8pt,a4paper,notitlepage]{article}
\usepackage{fullpage}
\usepackage{ulem}
\usepackage{xltxtra}
\usepackage{datetime}
\renewcommand{\dateseparator}{.}
\dmyyyydate
\usepackage{fancyhdr}
\usepackage{ifthen}
\pagestyle{fancy}
\fancyhf{}
\renewcommand{\headrulewidth}{0pt}
\fancyfoot[L]{\ifthenelse{\value{page}=1}{\today, \currenttime{} Uhr}{}}
\begin{document}
\begin{table}[ht]
\begin{minipage}[t]{0.5\linewidth}
\small
\begin{center}*D
\end{center}
\begin{tabular}{rl}
\textbf{738} & \begin{large}M\end{large}în kunst mir des niht witze gît,\\ 
 & daz ich \textbf{gesage} disen strît\\ 
 & bescheidenlîch, als er ergienc.\\ 
 & ieweders \textbf{ouge blic} enpfienc,\\ 
5 & daz er den andern komen sach.\\ 
 & \textbf{sweder\textit{s}} herze \textbf{dâr umbe} vreuden jach,\\ 
 & dâ stuont ein \textbf{trûren} nâhe bî.\\ 
 & \textbf{die lûtern truopheite vrî},\\ 
 & \textbf{ieweder} des andern herze truoc;\\ 
10 & ir vremde was heinlîch genuoc.\\ 
 & Nû\textbf{ne} mag ich \textbf{disen heiden}\\ 
 & \textbf{von dem} getouften \textbf{niht gescheiden},\\ 
 & si\textbf{ne} \textbf{wellen} haz erzeigen.\\ 
 & daz solt in vreude neigen,\\ 
15 & die sint erkant vür guotiu wîp.\\ 
 & ieweder durch vriwendinne lîp\\ 
 & sîn verch gein \textbf{der hert\textit{e}} bôt.\\ 
 & gelücke \textbf{scheidez} âne \textbf{tôt}.\\ 
 & Den lewen sîn muoter tôt gebirt,\\ 
20 & von sînes vater galme er \textbf{lebendec} wirt.\\ 
 & dise \textbf{zwêne wâren ûz} krache \textbf{erborn},\\ 
 & Von maneger tjost \textbf{nâch} prîse erkorn.\\ 
 & Si kunden ouch mit tjoste,\\ 
 & mit sper \textbf{zerender} koste.\\ 
25 & \textbf{Leischierende} si die zoume\\ 
 & kurzten unt tâten goume,\\ 
 & swenne si \textbf{punierten},\\ 
 & daz si \textbf{niht} failierten.\\ 
 & si pflâgen\textbf{s} \textbf{unvergezzen}.\\ 
30 & dâ wart vaste gesezzen\\ 
\end{tabular}
\scriptsize
\line(1,0){75} \newline
D \newline
\line(1,0){75} \newline
\textbf{1} \textit{Initiale} D  \textbf{11} \textit{Majuskel} D  \textbf{19} \textit{Majuskel} D  \textbf{22} \textit{Majuskel} D  \textbf{23} \textit{Majuskel} D  \textbf{25} \textit{Majuskel} D  \newline
\line(1,0){75} \newline
\textbf{6} sweders] swederz D \textbf{17} herte] herten D \newline
\end{minipage}
\hspace{0.5cm}
\begin{minipage}[t]{0.5\linewidth}
\small
\begin{center}*m
\end{center}
\begin{tabular}{rl}
 & mîn kunst mir des niht \textit{witz}e gît,\\ 
 & daz ich \textbf{gesage} disen strît\\ 
 & bescheidenlîch, als er ergienc.\\ 
 & ietweders \textbf{ougenblic} enpfienc,\\ 
5 & daz er den andern komen sach.\\ 
 & \textbf{ietweders} herz \textbf{dâr umbe} vröude jach,\\ 
 & d\textit{â} stuont ein \textbf{trûren} nâhe bî.\\ 
 & \textbf{\textit{sw}ie vrö\textit{me}de ietweder\textit{m s}î},\\ 
 & \textbf{daz er} des andern herz truoc,\\ 
10 & ir vremde was heimlîch \textbf{doch} genuoc.\\ 
 & \begin{Large}N\end{Large}û mac ich \textbf{niht gescheiden}\\ 
 & \textbf{den} getouften \textbf{von dem heiden},\\ 
 & si \textbf{wellent} haz erzeigen.\\ 
 & daz solt in vröude neigen,\\ 
15 & die sint erkant vür guot\textit{iu} wîp,\\ 
 & \textbf{wan} ietweder durch vriundîn lîp\\ 
 & sîn verch gegen \textbf{der h\textit{e}rte} bôt.\\ 
 & \textit{g}lück \textbf{scheidet daz} âne \textbf{nôt}.\\ 
 & den lewen sîn muoter \textit{tôt} gebirt,\\ 
20 & von sînes vater galm er \textbf{lebendic} wirt.\\ 
 & dise \textbf{zwê\textit{ne} ûz} krach \textbf{sint} \textbf{erborn},\\ 
 & von maniger just \textbf{nâch} prîs erkorn.\\ 
 & si kunden ouch mit joste,\\ 
 & mit sper \textbf{zierende\textit{r}} koste.\\ 
25 & \textbf{leisierende} si die zoume\\ 
 & kur\textit{zt}en und tâten goume,\\ 
 & wan si \textbf{punierten},\\ 
 & daz si \textit{\textbf{niht}} failierten.\\ 
 & si pflâgen\textbf{s} \textbf{unvermezzen}.\\ 
30 & d\textit{â} wart vaste gesezzen\\ 
\end{tabular}
\scriptsize
\line(1,0){75} \newline
m n o V V' \newline
\line(1,0){75} \newline
\textbf{11} \textit{Illustration mit Überschrift:} Also parcifal vnd der heiden mit einander stritten m (n)   $\cdot$ \textit{Großinitiale} n   $\cdot$ \textit{Initiale} m V  \newline
\line(1,0){75} \newline
\textbf{1} \textit{Die Verse 737.15-738.4 fehlen} V'   $\cdot$ witze] mẏnne m \textbf{4} \textit{Die Verse 738.4-5 fehlen} o  \textbf{5} daz er] Jr ietlich V' \textbf{6} Jr keiner ouch da nit ensprach V'  $\cdot$ ietweders] Sweders V \textbf{7} \textit{Die Verse 738.7-12 fehlen} V'   $\cdot$ dâ] Do m n o V \textbf{8} Die froͯde yetwederm by vnd sẏ m  $\cdot$ swie vrömede] Die freẏde n (o) \textbf{9} herz] herczen o \textbf{11} \textit{Die Verse 738.11-739.3 fehlen} o   $\cdot$ mac] enmag V  $\cdot$ gescheiden] bescheiden n \textbf{13} \textit{statt 738.13-14:} Sie beide ir sper do neigeten / Vnd einander do erzeigetenn V'  \textbf{15} \textit{Die Verse 738.15-739.2 fehlen} V'   $\cdot$ guotiu] guͯten m n \textbf{16} ietweder] iewederre V \textbf{17} herte] hurte m n \textbf{18} glück] Slug m  $\cdot$ scheidet daz] scheide es V  $\cdot$ nôt] [*]: den tot V \textbf{19} tôt gebirt] do gebot gebirt m \textbf{20} lebendic] lebende V \textbf{21} zwêne] zwe m \textbf{22} nâch] vs n [*]: vs V \textbf{24} sper zierender] sperziernde m sperzerender V \textbf{25} leisierende] [Lassier*]: Lassiernten V \textbf{26} kurzten] Kurtzen m n \textbf{27} wan] Swenne V  $\cdot$ punierten] pumierten n \textbf{28} niht] \textit{om.} m \textbf{29} unvermezzen] vnuergessen V \textbf{30} dâ] Do m n V \newline
\end{minipage}
\end{table}
\newpage
\begin{table}[ht]
\begin{minipage}[t]{0.5\linewidth}
\small
\begin{center}*G
\end{center}
\begin{tabular}{rl}
 & \begin{large}M\end{large}în kunst mir des niht witze gît,\\ 
 & daz ich \textbf{sage} disen strît\\ 
 & bescheidenlîch, als er ergienc.\\ 
 & ietweders \textbf{ougenblic} enpfienc,\\ 
5 & daz er den andern komen sach.\\ 
 & \textbf{sweders} herze vröude jach,\\ 
 & dâ stuont ein \textbf{trôst} nâhen bî.\\ 
 & \textbf{die lûtern tumpheit vrî},\\ 
 & \textbf{ietweder} des andern herze truoc;\\ 
10 & ir vremde was heinlîch genuoc.\\ 
 & nû\textbf{ne} mac ich \textbf{disen heiden}\\ 
 & \textbf{von dem} getouften \textbf{niht gescheiden},\\ 
 & si\textbf{ne} \textbf{wellen} haz erzeigen.\\ 
 & daz solt in vröude neigen,\\ 
15 & die sint erkant vür guotiu wîp.\\ 
 & ietweder durch vriundinne lîp\\ 
 & sîn verch gein \textbf{der herte} bôt.\\ 
 & gelücke \textbf{scheide si} ân \textbf{den} \textbf{tôt}.\\ 
 & den lewen sîn muoter tôt gebirt,\\ 
20 & von sînes vater galme er \textbf{lebende} wirt.\\ 
 & dise \textbf{wâren ûz} krache \textbf{erborn},\\ 
 & von maniger tjost \textbf{ûz} brîse erkorn.\\ 
 & si kunden ouch mit tjoste.\\ 
 & mit sper \textbf{ze\textit{r}ender} koste\\ 
25 & \textbf{leisierten} si die zoume,\\ 
 & kurzten \textbf{si} unde tâten goume,\\ 
 & swenne si \textbf{punierten},\\ 
 & daz si \textbf{iht} failierten.\\ 
 & si pflâgen \textbf{unvergezzen},\\ 
30 & dâ wart vaste gesezzen\\ 
\end{tabular}
\scriptsize
\line(1,0){75} \newline
G I L M Z Fr24 \newline
\line(1,0){75} \newline
\textbf{1} \textit{Initiale} G L Z Fr24  \textbf{3} \textit{Initiale} I  \textbf{19} \textit{Initiale} I  \newline
\line(1,0){75} \newline
\textbf{1} des] \textit{om.} Z \textbf{2} sage] gesage L M Z Fr24 \textbf{3} bescheidenlîch] Beschemliche L  $\cdot$ ergienc] gienc M \textbf{4} ietweders] Jetweder L (M)  $\cdot$ ougenblic] augen bliche I ovge bliche L ovge blick Fr24 \textbf{6} sweders] Beders L Widers M Jetweders Z  $\cdot$ vröude] darvmme vroide M (Z) \textbf{7} trôst] truren M (Z) \textbf{8} die] der ie I  $\cdot$ tumpheit] trvpheit Z \textbf{9} ietweder] Jclichir M \textbf{10} heinlîch] eyn heymmelich M \textbf{11} mac] mage Fr24  $\cdot$ disen heiden] niht gescheiden L \textbf{12} niht gescheiden] disen heiden L \textbf{14} solt] sol I  $\cdot$ neigen] \sout{geben} neigen L \textbf{15} die] Si M \textbf{16} ietweder] Jclichir M \textbf{17} herte] herre I \textbf{18} scheide] schiet L  $\cdot$ den tôt] not I \textbf{20} sînes] siner I des L  $\cdot$ lebende] lebendic I Z (Fr24) lebin M \textbf{21} dise] Dise zcwene M (Z)  $\cdot$ krache] hoher art I  $\cdot$ erborn] Geborn I (M) \textbf{22} ûz] zuͤ I \textbf{24} sper] spern I L  $\cdot$ zerender] ze ender G \textbf{25} leisierten] Zcu yserten M Lesiernde Z \textbf{26} si] \textit{om.} Z \textbf{27} swenne] Wenne L (M) \textbf{28} iht] niht L (M) Z Fr24 \textbf{29} si pflâgen] des phlagen si I \newline
\end{minipage}
\hspace{0.5cm}
\begin{minipage}[t]{0.5\linewidth}
\small
\begin{center}*T
\end{center}
\begin{tabular}{rl}
 & \begin{large}M\end{large}îniu kunst mir des niht witze gît,\\ 
 & daz ich \textbf{gesage} disen strît\\ 
 & bescheidenlîche, als er ergienc.\\ 
 & ietweder\textit{s} \textbf{\textit{o}ugenbli\textit{c}} enpfienc,\\ 
5 & daz er den andern komen sach.\\ 
 & \textbf{\textit{iet}weder\textit{s}} herze \textbf{dâr umbe} vreude jach,\\ 
 & \textit{dâ stuont ein \textbf{trûren} nâhe bî}.\\ 
 & \textbf{der lûtern tumpheit vrî},\\ 
 & \textbf{ietweder} des andern herze truoc;\\ 
10 & ir vremede was heimelîche genuoc.\\ 
 & nû\textbf{ne} mac ich \textbf{disen heiden}\\ 
 & \textbf{von dem} getouften \textbf{niht gescheiden},\\ 
 & si \textbf{en}\textbf{wellen} haz erzeigen.\\ 
 & daz solde in vreuden neigen,\\ 
15 & die sint erkant vür guotiu wîp.\\ 
 & ietweder durch vriundinne lîp\\ 
 & sîn verch gein \textbf{dem herzen} bôt.\\ 
 & gelücke \textbf{scheide si} âne \textbf{den} \textbf{tôt}.\\ 
 & den lewen sîn muoter tôt gebirt,\\ 
20 & von sînes vater galm er \textbf{lebendic} wirt.\\ 
 & dise \textbf{zwêne wâren zuo} krache \textbf{geborn},\\ 
 & von maneger jost \textbf{ûz} prîse erkorn.\\ 
 & si kunden ouch mit joste.\\ 
 & mit spere \textbf{zernder} koste\\ 
25 & \textbf{le\textit{i}siereten} si die zoume,\\ 
 & kurzeten \textbf{si} und tâten goume,\\ 
 & wanne si \textbf{gepungiereten},\\ 
 & daz si \textbf{niht} falliereten.\\ 
 & si pflâgen \textbf{unvergezzen},\\ 
30 & d\textit{â} wart vaste gesezzen\\ 
\end{tabular}
\scriptsize
\line(1,0){75} \newline
U W Q R \newline
\line(1,0){75} \newline
\textbf{1} \textit{Initiale} U W R  \newline
\line(1,0){75} \newline
\textbf{3} ergienc] der ginck Q \textbf{4} Jequeder site sie augen blicke intfinc U \textbf{6} Ietweders] Weder daz U Weders Q Deweders R  $\cdot$ jach] des iach R \textbf{7} \textit{Vers 738.7 fehlt} U   $\cdot$ dâ] Do W Q  $\cdot$ ein] en R \textbf{8} der] Die Q R  $\cdot$ lûtern] luttrem R \textbf{9} herze truoc] herczen by im truͦg R \textbf{11} nûne] Nene Q  $\cdot$ mac] enmag R \textbf{13} enwellen] woͤllen W (Q) (R)  $\cdot$ haz] [b*z]: haz U \textbf{14} daz] Die W  $\cdot$ vreuden] froͤde W (Q) (R) \textbf{15} vür] durch Q \textbf{17} verch] werck Q [*]: Roch R  $\cdot$ dem herzen] der herte W Q R \textbf{18} scheide si] schaide ich W sy schoide R \textbf{20} vater] varers Q  $\cdot$ galm] galm vnd gescheẏ R  $\cdot$ lebendic] lebend W \textbf{21} zuo] auß W vser R  $\cdot$ geborn] erborn W gancz erborn R \textbf{22} maneger jost] mengen strit R \textbf{23} \textit{Versfolge 738.24-23} U   $\cdot$ joste] tiosten Q ir tioste R \textbf{25} leisiereten] Lersiereten U Leifirten Q \textbf{26} tâten] tatens W \textbf{27} gepungiereten] pungierten W (Q) (R) \textbf{30} dâ] Daz U Do W Q \newline
\end{minipage}
\end{table}
\end{document}
