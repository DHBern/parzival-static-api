\documentclass[8pt,a4paper,notitlepage]{article}
\usepackage{fullpage}
\usepackage{ulem}
\usepackage{xltxtra}
\usepackage{datetime}
\renewcommand{\dateseparator}{.}
\dmyyyydate
\usepackage{fancyhdr}
\usepackage{ifthen}
\pagestyle{fancy}
\fancyhf{}
\renewcommand{\headrulewidth}{0pt}
\fancyfoot[L]{\ifthenelse{\value{page}=1}{\today, \currenttime{} Uhr}{}}
\begin{document}
\begin{table}[ht]
\begin{minipage}[t]{0.5\linewidth}
\small
\begin{center}*D
\end{center}
\begin{tabular}{rl}
\textbf{47} & in küssen unt vâhen zir.\\ 
 & er sprach: "nû \textbf{ging} \textbf{ouch} her ze mir."\\ 
 & \textbf{der wirt} in \textbf{kuste selbe} dô.\\ 
 & si wâren \textbf{ze sehen ein ander} vrô.\\ 
5 & \textit{\begin{large}G\end{large}}ahmuret sprach aber sân:\\ 
 & "owê, \textbf{junc} süezer man!\\ 
 & waz \textbf{solte} her dîn kranker lîp?\\ 
 & sag an, gebôt dir \textbf{daz} ein wîp?"\\ 
 & \multicolumn{1}{l}{ - - - }\\ 
 & \multicolumn{1}{l}{ - - - }\\ 
 & "\textbf{die} gebietent \textbf{wênic, hêrre}, mir.\\ 
10 & mich hât mîn \textbf{veter} Gaschier\\ 
 & her brâht. er weiz wol selbe wie.\\ 
 & ich hân im tûsent rîter hie\\ 
 & \textbf{unt} stên im dienestlîche bî.\\ 
 & \textbf{ze} \textbf{Rœms} \textbf{in} Normandi\\ 
15 & kom ich zer samnunge.\\ 
 & ich \textbf{brâht} im helde junge.\\ 
 & \textbf{Ich} vuor von Schampane durch in.\\ 
 & nû wil kunst unt sin\\ 
 & \textbf{der} \textbf{schade} an in kêren,\\ 
20 & ir enwelt iuch \textbf{selben} êren.\\ 
 & gebietet ir, sô lât in mîn\\ 
 & geniezen, \textbf{semften} \textbf{sînen} pîn."\\ 
 & "Den rât nim \textbf{dû} \textbf{vil} gar zuo dir.\\ 
 & var dû unt \textit{m}î\textit{n} hêr Gaschier\\ 
25 & unt bringet \textbf{mir} Kayleten her."\\ 
 & dô wur\textit{b}en si des heldes ger.\\ 
 & si brâhten in \textbf{durch sîne} bete.\\ 
 & dô wart \textbf{ouch er} von Gahmurete\\ 
 & minneclîche enpfangen\\ 
30 & \textbf{unt dicke} umbevangen\\ 
\end{tabular}
\scriptsize
\line(1,0){75} \newline
D Fr14 \newline
\line(1,0){75} \newline
\textbf{5} \textit{Initiale} D Fr14  \textbf{17} \textit{Majuskel} D  \textbf{23} \textit{Majuskel} D  \newline
\line(1,0){75} \newline
\textbf{5} Gahmuret] ÷ahmvret D Ga:::mvret Fr14 \textbf{10} Gaschier] Gascier D Fr14 \textbf{14} Rœms] Roͤms D Fr14  $\cdot$ Normandi] Normandî D \textbf{16} im] dem Fr14 \textbf{17} Schampane] Scampane D Scampanie Fr14 \textbf{20} selben] selbe Fr14 \textbf{22} semften] senftet Fr14 \textbf{24} mîn] nim D  $\cdot$ Gaschier] Gascier D Fr14 \textbf{25} Kayleten] Kaileten Fr14 \textbf{26} wurben] wrden D \textbf{28} Gahmurete] Gahmvrete D Fr14 \newline
\end{minipage}
\hspace{0.5cm}
\begin{minipage}[t]{0.5\linewidth}
\small
\begin{center}*m
\end{center}
\begin{tabular}{rl}
 & in küssen und vâhen ze ir.\\ 
 & er sprach: "nû \textbf{ganc} \textbf{ouch} her zuo mir."\\ 
 & \textbf{der wirt} in \textbf{selber kuste} dô.\\ 
 & si wâren \textbf{ze se\textit{h}ende ein \textit{a}nder} vrô.\\ 
5 & \begin{large}G\end{large}ahmuret sprach aber sân:\\ 
 & "owê, \dag juncvrouwen\dag  man!\\ 
 & waz \textbf{solt} \textit{h}er dîn kranker lîp?\\ 
 & sag an, gebôt dir \textbf{daz} ein wîp?"\\ 
 & \multicolumn{1}{l}{ - - - }\\ 
 & \multicolumn{1}{l}{ - - - }\\ 
 & "\textbf{die} gebieten\textit{t} \textbf{wênic, hêrre}, mi\textit{r}.\\ 
10 & mich hât mîn \textbf{v\textit{e}ter} G\textit{a}schier\\ 
 & her brâht. er wei\textit{z} wol selber wie.\\ 
 & ich hân im tûsent ritter hie\\ 
 & \textbf{und} stân i\textit{m}e dienstlîchen bî.\\ 
 & \textbf{ze} \textbf{Rumes} \textbf{in} Normandi\\ 
15 & kam ich zer samenunge.\\ 
 & ich \textbf{brâhte} im helde junge.\\ 
 & \textbf{ich} vuor von Schampanie durch in.\\ 
 & nû wil kunst und sin\\ 
 & \textbf{der} \textbf{schade} an in kêren,\\ 
20 & ir enwelt iuch \textbf{selben} \textit{ê}ren.\\ 
 & gebietet ir\textbf{s}, sô lât in mîn\\ 
 & genie\textit{z}e\textit{n}, \textbf{senftet} \textbf{\textit{m}în\textit{e}} pîn."\\ 
 & "den rât nim \textbf{dû} gar zuo dir.\\ 
 & var dû und mîn hêrre Gaschier\\ 
25 & und bringet \textbf{mir} Kaileten her."\\ 
 & dô wurben si des heldes ger.\\ 
 & si brâhten in \textbf{durch sîne} bete.\\ 
 & dô wart \textbf{er ouch} von Gahmurete\\ 
 & minneclîchen enpfangen\\ 
30 & \textbf{und dicke} umbevangen\\ 
\end{tabular}
\scriptsize
\line(1,0){75} \newline
m n o \newline
\line(1,0){75} \newline
\textbf{5} \textit{Initiale} m   $\cdot$ \textit{Capitulumzeichen} n  \newline
\line(1,0){75} \newline
\textbf{3} dô] da o \textbf{4} sehende] sehehende m  $\cdot$ ein ander] eineinder m \textbf{5} Gahmuret] Gamiret n Gamuret o \textbf{6} juncvrouwen] jung swartzer n (o) \textbf{7} her] er \textit{nachträglich korrigiert zu:} dir m \textbf{9} die] Das n  $\cdot$ gebietent] gebietten m (o)  $\cdot$ mir] min m \textbf{10} veter] vatter m  $\cdot$ Gaschier] gar schier m gaschúr n [ga*]: gaschier o \textbf{11} weiz] weist m \textbf{13} stân] [stand]: stann m (o)  $\cdot$ ime] immre m \textbf{14} Rumes] romes n o  $\cdot$ in] von n  $\cdot$ Normandi] normandy n \textbf{17} Schampanie] Schamppanie m schampponẏ n schampanẏ o \textbf{20} selben] [selbes]: selben m selber o  $\cdot$ êren] keren m \textbf{22} geniezen] Genieffet m  $\cdot$ senftet] senffterent n (o)  $\cdot$ mîne] sinem m \textbf{24} mîn] nẏm o  $\cdot$ Gaschier] Gascier m gascúr n gascir o \textbf{25} bringet] bring n  $\cdot$ Kaileten] kailletten m kaẏleten n kaliten o \textbf{26} heldes] heiles n \textbf{28} Gahmurete] Gahmuͯrete m gamirette n gamuͯreten o \newline
\end{minipage}
\end{table}
\newpage
\begin{table}[ht]
\begin{minipage}[t]{0.5\linewidth}
\small
\begin{center}*G
\end{center}
\begin{tabular}{rl}
 & in küssen und vâhen zir.\\ 
 & er sprach: "nû \textbf{geng} \textbf{ouch} her ze mir."\\ 
 & \textbf{der wirt} in \textbf{kuste selbe} dô.\\ 
 & si wâren \textbf{ze sehene ein ander} vrô.\\ 
5 & Gahmuret sprach aber sân:\\ 
 & "owê, \textbf{junge} süezer man!\\ 
 & waz \textbf{solte} her dîn kranker lîp?\\ 
 & sage an, gebôt dir \textbf{daz} ein wîp?"\\ 
 & \multicolumn{1}{l}{ - - - }\\ 
 & \multicolumn{1}{l}{ - - - }\\ 
 & "\textbf{die} gebietent, \textbf{hêrre, wênic} mir.\\ 
10 & mich hât mîn \textbf{veter} Gatschier\\ 
 & her brâht. er weiz wol selbe wie.\\ 
 & ich hân im tûsent rîter hie\\ 
 & \textbf{unt} stên im dienstlîchen bî.\\ 
 & \textbf{ze} \textbf{Roumes} \textbf{in} Normandi\\ 
15 & kom ich z\textit{e}r samnunge.\\ 
 & ich \textbf{brâht} im helde junge.\\ 
 & \textbf{ich} vuor von Schampanie durch in.\\ 
 & nû wil kunst und sin\\ 
 & \textbf{ir} \textbf{schaden} an in kêren,\\ 
20 & irn welt iuch \textbf{selben} êren.\\ 
 & gebiet ir, sô lât in mîn\\ 
 & geniezen, \textbf{senftet} \textbf{sînen} pîn."\\ 
 & \textbf{er sprach}: "den rât nim gar ze dir.\\ 
 & var dû und mîn hêr Gatschier\\ 
25 & unde bringet Kaileten her."\\ 
 & dô wurben si des heldes ger.\\ 
 & si brâhten in \textbf{\textit{nâch} sîne\textit{r}} bet.\\ 
 & dô wart \textbf{ouch er} von Gahmuret\\ 
30 & \hspace*{-.7em}\big| \textbf{\textit{vil} dick} umbevangen\\ 
 & \hspace*{-.7em}\big| \textbf{unde} minniclîche enpfangen\\ 
\end{tabular}
\scriptsize
\line(1,0){75} \newline
G I O L M Q R Z Fr21 \newline
\line(1,0){75} \newline
\textbf{1} \textit{Initiale} O  \textbf{27} \textit{Initiale} Q R  \newline
\line(1,0){75} \newline
\textbf{1} \textit{Die Verse 44.7-51.12 fehlen} Z   $\cdot$ in] ÷n O  $\cdot$ zir] in zuͤ ir I \textbf{2} ouch] \textit{om.} I  $\cdot$ her] \textit{om.} L \textbf{3} in kuste selbe] in selbe kuste I in kvste sebe O kuste in selbe L on kuste selben M in kuste selber Q in kvst selb Fr21  $\cdot$ dô] da M \textbf{5} Gahmuret] Gamvret O Gahmuͯret L Gamurat M Gamuert Q Gahmoret Fr21 \textbf{6} owê] Awe I O Q  $\cdot$ junge süezer] iunger suzzer I (L) (Q) (R) svͦzer ivnger O iunge susze M (Fr21) \textbf{7} solte] wolt I [sol]: solte L soltu R  $\cdot$ her] or M \textbf{8} an] \textit{om.} M \textbf{9} herre die gebietent wenc mir I (L) \textbf{10} mîn] mit R  $\cdot$ Gatschier] gatschir M Q (R) \textbf{11} her] Er M  $\cdot$ er] ich L  $\cdot$ selbe] \textit{om.} L Q R \textbf{12} ich] Vnd L  $\cdot$ im] nv L \textbf{13} unt stên] die stent I \textbf{14} Roumes] roͮmes G Rovme I rvͦm O Ronich L ronis M Q (Fr21) Koͯnis R  $\cdot$ Normandi] normandý L normendẏ R \textbf{15} kom ich] Komen wir M  $\cdot$ zer] zir G ze O (L) \textbf{16} brâht] braht ich O fuͯrte L  $\cdot$ helde] [here]: held I ritter M ein helde Q \textbf{17} ich] ch I Vnd L  $\cdot$ vuor von] fuͯr vntze L durch R  $\cdot$ Schampanie] shanpange I tscampanie O schampine M schanpfange Q schampange R tsampanie Fr21  $\cdot$ in] hien M \textbf{18} wil] wil [beẏ]: beid Q  $\cdot$ sin] gewyn M \textbf{19} ir schaden] Den schaden O L Der schade M Q R  $\cdot$ kêren] kere M \textbf{20} irn] Jr L Q  $\cdot$ selben] selb I (R) selben danne L \textbf{21} ir] [s]: ir O irs Q \textbf{22} senftet] senft im O (M) Senften R  $\cdot$ sînen] syne M (Q) (R) \textbf{23} gar] gal M  $\cdot$ ze] ziv I \textbf{24} var] War M  $\cdot$ mîn] \textit{om.} I  $\cdot$ Gatschier] Gatschir O L (Q) R gatischir M \textbf{25} bringet] bring I bringet mir O L (M) (R) bringe mir Q  $\cdot$ Kaileten] Gahilet I kayleten O (Q) kaýleten L kaileiten M kaylatten R \textbf{26} wurben] [wurden]: wurben Q \textbf{27} si] Vnd L  $\cdot$ nâch sîner] dur sine G so noch siner O \textbf{28} ouch er] er avch O ouch [Gah]: er L  $\cdot$ von] vnd R  $\cdot$ Gahmuret] hahmuret I Gamvret O Gahmuͯret L gamuͯret M gaműert Q \textbf{30} \textit{Versfolge 47.29-30} O L M Q R   $\cdot$ vil dick] ditch G Vnd dich O (L) (M) (Q) (R) \textbf{29} unde] Vil O L M Q R  $\cdot$ minniclîche] mynneche L myndiglich Q \newline
\end{minipage}
\hspace{0.5cm}
\begin{minipage}[t]{0.5\linewidth}
\small
\begin{center}*T (U)
\end{center}
\begin{tabular}{rl}
 & i\textit{n} küssen und vâhen z\textit{ir}.\\ 
 & er sprach: "nû \textbf{gêt} \textbf{ir} her zuo mir."\\ 
 & \textbf{diu wirt\textit{î}n} in \textbf{selbe kuste} dô.\\ 
 & si wâren \textbf{ein ander zuo sehen} vrô.\\ 
5 & Gahmuret sprach aber sân:\\ 
 & "owê, \textbf{junc} süezer man!\\ 
 & waz \textbf{wolte} her dîn kranker lîp?\\ 
 & sage an, gebôt dir \textbf{diz} ein wîp?"\\ 
 & er sprach: "nein, ez ist niht,\\ 
 & waz sô anders mir geschiht.\\ 
 & \textbf{si} gebietent, \textbf{hêrre, wênic} mir.\\ 
10 & mich hât mîn \textbf{hêrre} Gatschier\\ 
 & her brâht. er weiz wol selbe wie.\\ 
 & ich hân im tûsent ritter hie,\\ 
 & \textbf{die} stân im dienstlîchen bî.\\ 
 & \textbf{zer} \textbf{Romischer} Normandi\\ 
15 & kom ich zuo der samenunge.\\ 
 & ich \textbf{vuorte} im helde junge\\ 
 & \textbf{und} vuo\textit{r} von S\textit{cham}pan\textit{ie} durch in.\\ 
 & nû wil kunst und sin\\ 
 & \textbf{den} \textbf{schaden} an in kêren,\\ 
20 & ir \textit{en}welt \textit{iuch} \textbf{selbe\textit{r}} \textit{ê}ren.\\ 
 & gebietet ir, sô l\textit{â}t in mîn\\ 
 & geniezen, \textbf{sanfter} \textbf{sîn in} pîn."\\ 
 & \textbf{er sprach}: "den rât nim gar zuo dir.\\ 
 & var dû und mîn hêrre Gatschier\\ 
25 & und bringet \textbf{mir} Kayleten her."\\ 
 & dô wurben si des heldes ger.\\ 
 & si brâhten in \textbf{nâch sîner} bete.\\ 
 & dô wart \textbf{\textit{er} ouch} von Gahmurete\\ 
 & \textbf{vil} minniclîche enpfangen,\\ 
30 & \textbf{mit armen} umbevangen\\ 
\end{tabular}
\scriptsize
\line(1,0){75} \newline
U V W T \newline
\line(1,0){75} \newline
\textbf{3} \textit{Majuskel} T  \textbf{5} \textit{Majuskel} T  \textbf{9} \textit{Majuskel} T  \textbf{23} \textit{Majuskel} T  \textbf{27} \textit{Initiale} W  \textbf{28} \textit{Majuskel} T  \newline
\line(1,0){75} \newline
\textbf{1} in] Jr U  $\cdot$ zir] zart U \textbf{2} gêt ir] gang auch W (T) \textbf{3} diu wirtîn] Die wirten U Der wurt V (W) Der wunde man T  $\cdot$ selbe] selber V W \textit{om.} T \textbf{4} ein ander zuo sehen] enander ze sehende V (W) zesehene ein ander T \textbf{5} Gahmuret] Gahmuͦret U Gamuret V W \textbf{7} wolte] solte T  $\cdot$ kranker] ivnger T \textbf{8} diz] daz V (W) (T) \textbf{8} \textit{Die Verse 47.8¹-8² fehlen} T   $\cdot$ nein ez ist] neines V (W) \textbf{8} waz] swas V \textbf{9} si] So W Die T  $\cdot$ wênic] noch wenig W \textbf{10} hêrre] vetter W neve T  $\cdot$ Gatschier] Gatschir V (W) Gascier T \textbf{11} selbe] recht W \textbf{13} die] vnd T  $\cdot$ stân] stond W \textbf{14} zer] zuͦ V W (T)  $\cdot$ Romischer] Romiscer U Rvmes V roͤmischer W Rome T  $\cdot$ Normandi] Normandy U [*]: in Normandi V normandei W in Normandj T \textbf{16} vuorte im] [*]: brohte im V brahtim T \textbf{17} und] ich T  $\cdot$ vuor] vuͦre U  $\cdot$ von] dvrh T  $\cdot$ Schampanie] spangen U (V) scampanie W \textbf{18} kunst] kost W \textbf{19} den] [D*]: Der V  $\cdot$ in] eúch W \textbf{20} Ir wellent eúch enteren W  $\cdot$ enwelt iuch selber] wolnt selber von U enwellen v́ch selber V en welt îv selben T \textbf{21} ir] irs V  $\cdot$ sô lât] solnt U \textbf{22} sanfter sîn in] vnde senftern sinen V senftent seine W senftet sinen T \textbf{24} Gatschier] Gatschir V (W) Gaschir T \textbf{25} bringet] bringen W  $\cdot$ Kayleten] kyleten U kaẏleten V kyeleten W \textbf{28} er ouch] auͦch U auch er W (T)  $\cdot$ Gahmurete] Gahmuͦrete U Gamurete V gamurette W Gahmvrete T \textbf{29} vil] \textit{om.} W \textbf{30} mit armen] vnde dicke T \newline
\end{minipage}
\end{table}
\end{document}
