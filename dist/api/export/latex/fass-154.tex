\documentclass[8pt,a4paper,notitlepage]{article}
\usepackage{fullpage}
\usepackage{ulem}
\usepackage{xltxtra}
\usepackage{datetime}
\renewcommand{\dateseparator}{.}
\dmyyyydate
\usepackage{fancyhdr}
\usepackage{ifthen}
\pagestyle{fancy}
\fancyhf{}
\renewcommand{\headrulewidth}{0pt}
\fancyfoot[L]{\ifthenelse{\value{page}=1}{\today, \currenttime{} Uhr}{}}
\begin{document}
\begin{table}[ht]
\begin{minipage}[t]{0.5\linewidth}
\small
\begin{center}*D
\end{center}
\begin{tabular}{rl}
\textbf{154} & daz dû den wîn vergüzze,\\ 
 & \textbf{unvuoge} dich verdrüzze.\\ 
 & ir decheinen lustet strîtes.\\ 
 & gip mir, dâ \textbf{dû ûffe} rîtes,\\ 
5 & \textbf{unt} dar zuo al dîn harnas.\\ 
 & daz enpfieng ich ûf dem palas,\\ 
 & dâr inne ich ritter werden muoz.\\ 
 & widersagt sî dir mîn gruoz,\\ 
 & ob dû mirz ungerne gîst.\\ 
10 & wer mich, ob dû bî witzen sîst."\\ 
 & Der künec von Kukumerlant\\ 
 & sprach: "hât Artuses hant\\ 
 & dir mîn harnasch gegeben,\\ 
 & daz tæter ouch mîn leben,\\ 
15 & m\textit{ö}htestû \textbf{mirz} \textbf{an} gewinnen.\\ 
 & sus kan er vriwende minnen.\\ 
 & was \textbf{er dir aber ê} iht holt,\\ 
 & \textbf{dîn dienst gedient sô schiere} den solt."\\ 
 & "Ich getar wol \textbf{dienen}, swaz ich sol;\\ 
20 & ouch hât er mich gewert \textbf{vil} wol.\\ 
 & gip her unt lâz dîn lantreht.\\ 
 & i\textbf{ne} wil niht langer sîn ein kneht,\\ 
 & ich sol schildes ambet hân."\\ 
 & er greif im nâch dem zoume sân:\\ 
25 & "dû maht wol \textbf{wesen} Læhelin,\\ 
 & von dem mir klaget diu muoter mîn."\\ 
 & \textit{\begin{large}D\end{large}}er rîter umbe kêrt den schaft\\ 
 & unt stach den knappen sô mit kraft,\\ 
 & daz er unt sîn pferdelîn\\ 
30 & \textbf{muosen vallende} ûf \textbf{die} bluomen \textbf{sîn}.\\ 
\end{tabular}
\scriptsize
\line(1,0){75} \newline
D \newline
\line(1,0){75} \newline
\textbf{11} \textit{Majuskel} D  \textbf{19} \textit{Majuskel} D  \textbf{27} \textit{Initiale} D  \newline
\line(1,0){75} \newline
\textbf{11} Kukumerlant] kvchvmerlant D \textbf{12} Artuses] Artvs D \textbf{15} möhtestû] mohtestv D \textbf{27} Der] ÷er D \newline
\end{minipage}
\hspace{0.5cm}
\begin{minipage}[t]{0.5\linewidth}
\small
\begin{center}*m
\end{center}
\begin{tabular}{rl}
 & daz dû den wîn vergüzze,\\ 
 & \textbf{ungevuoge} dich verdrüzze.\\ 
 & ir dekeinen lustet strîtes.\\ 
 & gip mir, dâ \textbf{dû ûffe} rîtes,\\ 
5 & \textbf{und} dar zuo al dîn harnas.\\ 
 & daz enpfien\textit{c} ich ûf dem palas,\\ 
 & dâr inne ich ritter werden muoz.\\ 
 & widersagt sî dir mîn gruoz,\\ 
 & ob dû mirz ung\textit{e}r\textit{n}e gîst.\\ 
10 & wer mich, ob dû bî witzen sîst."\\ 
 & \begin{large}D\end{large}er künic von Kukumerlant\\ 
 & sprach: "hât Artuses hant\\ 
 & dir mîn harna\textit{s}ch gegeben,\\ 
 & daz tæt er \textbf{lîhte} ouch mîn leben,\\ 
15 & möhtest dû \textbf{mir ez} \textbf{ab} gewinnen.\\ 
 & sus kan er vr\textit{iun}de minnen.\\ 
 & was \textbf{aber er dir ê} ih\textit{t} holt,\\ 
 & \textbf{dû hâst sô schiere gedienet} den solt."\\ 
 & "ich getar wol \textbf{dienen}, waz ich sol;\\ 
20 & ouch hât er mich gewert wol.\\ 
 & gip her und lâ dîn lantreht.\\ 
 & ich wil niht langer sîn ein kneht,\\ 
 & ich sol schiltes ambet hân."\\ 
 & er greif ime nâch dem zoume sân:\\ 
25 & "dû maht wol \textbf{wesen} Lehelin,\\ 
 & von dem mi\textit{r} klagete diu muoter mîn."\\ 
 & der ritter umbe kêrte den schaft\\ 
 & und stach den knaben sô mit kraft,\\ 
 & daz er und sîn pferdelîn\\ 
30 & \textbf{muosen vallen} ûf \textbf{die} bluomen \textbf{sîn}.\\ 
\end{tabular}
\scriptsize
\line(1,0){75} \newline
m n o \newline
\line(1,0){75} \newline
\textbf{11} \textit{Initiale} m   $\cdot$ \textit{Capitulumzeichen} n  \textbf{16} \textit{Capitulumzeichen} n  \newline
\line(1,0){75} \newline
\textbf{2} ungevuoge dich] Vnfúge dich dich n \textbf{3} dekeinen] do keinen n  $\cdot$ lustet strîtes] luͯsten stritens o \textbf{5} al] allen n alle o  $\cdot$ dîn] dinen n \textbf{6} enpfienc] entpfinge m \textbf{8} mîn] din o \textbf{9} ungerne] vngeverde m \textbf{10} bî witzen] witzig n (o) \textbf{11} Kukumerlant] kúkumer lant m koncumerlant n kúcúmer lant o \textbf{12} hât Artuses] artuses hat n \textbf{13} harnasch] harnach m  $\cdot$ gegeben] geben n o \textbf{15} möhtest dû] Mochtest o  $\cdot$ ab] an n o \textbf{16} vriunde] froͯde m \textbf{17} dir] die o  $\cdot$ iht] ich m \textbf{19} dienen] getienen n \textbf{25} Lehelin] [lehen]: lehelin n \textbf{26} mir] min m  $\cdot$ klagete] claget n \textbf{29} daz] Vnd o \textbf{30} muosen] Muͯstent n (o)  $\cdot$ sîn] hin n o \newline
\end{minipage}
\end{table}
\newpage
\begin{table}[ht]
\begin{minipage}[t]{0.5\linewidth}
\small
\begin{center}*G
\end{center}
\begin{tabular}{rl}
 & daz dû den wîn vergüzze,\\ 
 & \textbf{ungevuoge} dich verdrüzze.\\ 
 & ir neheinen lustet strîtes.\\ 
 & gib mir, dâ \textbf{ûffe dû} rîtes,\\ 
5 & dar zuo al dîn harnas.\\ 
 & daz enpfienc ich ûf dem palas,\\ 
 & dâr inne ich rîter werden muoz.\\ 
 & widersaget sî dir mîn gruoz,\\ 
 & obe dû mirz ungerne gîst.\\ 
10 & wer mich, obe dû bî witzen sîst."\\ 
 & der künic von Kukumerlant\\ 
 & sprach: "hât Artuses hant\\ 
 & dir mîn harnasch gegeben,\\ 
 & \textbf{dêswâr}, daz tæter ouch mîn leben,\\ 
15 & m\textit{ö}htestû \textbf{mirz} \textbf{an} gewinnen.\\ 
 & sus kan er vriunde minnen.\\ 
 & was \textbf{er dir aber ê} iht holt,\\ 
 & \textbf{dîn dienst gedient sô schiere} den solt."\\ 
 & "ich getar wol \textbf{dienen}, swaz ich sol;\\ 
20 & ouch hât er mich gewert \textbf{vil} wol.\\ 
 & gip her und lâ dîn lantreht.\\ 
 & ich wil niht langer sîn ein kneht,\\ 
 & \begin{large}I\end{large}ch sol schiltes ambet hân."\\ 
 & er greif im nâch dem zoume sân:\\ 
25 & "dû maht wol \textbf{sîn} Lehelin,\\ 
 & von dem mir klaget diu muoter mîn."\\ 
 & der rîter umbe kêrte den schaft\\ 
 & unde stach den knappen sô mit kraft,\\ 
 & daz er und sîn pferdelîn\\ 
30 & \textbf{muosen vallende} ûf \textbf{die} bluomen \textbf{sîn}.\\ 
\end{tabular}
\scriptsize
\line(1,0){75} \newline
G I O L M Q R Z \newline
\line(1,0){75} \newline
\textbf{11} \textit{Initiale} I  \textbf{23} \textit{Initiale} G  \textbf{27} \textit{Illustration mit Überschrift:} Hie strit parzifal mit dem Rotten Ritter Vnd feltt Jn nider zu tod vnd leit sine wappen an vnd sicz vff sin Ros also wunder vnd lett den harnach v́ber sin narren kleid R   $\cdot$ \textit{Initiale} I O L R Z  \newline
\line(1,0){75} \newline
\textbf{1} den] \textit{om.} Q Z \textbf{2} ungevuoge] Vnfuͯge L (Q) (R) (Z) \textbf{3} ir neheinen] Jnkeinen Q Jr keinet Z  $\cdot$ lustet] luste M \textbf{4} gib] Gipt Q  $\cdot$ dâ ûffe dû] da du vf I (L) (M) (Q) daz dv O \textbf{5} dar] Das L  $\cdot$ al dîn] haldin M als dein Q \textbf{6} enpfienc] empfach R \textbf{8} widersaget sî] Wider sag sig R  $\cdot$ dir mîn] mir dein Q \textbf{9} mirz] mir Q  $\cdot$ ungerne] niht gerne L \textbf{10} wer] Gewer I (O) (M) (Q) (R) (Z)  $\cdot$ bî witzen] wizzich I (O) (Q) by wicze M In wiczen R  $\cdot$ sîst] bist L R \textbf{11} der] \textit{om.} O  $\cdot$ Kukumerlant] kukunberlant I cvcumerlant O kvcuͯmerlant L kukuͯmer lant M kakúmerlant Q kvnkvmenlant Z \textbf{12} hât] hat dich R  $\cdot$ Artuses] artus Q arttus R  $\cdot$ hant] har gesant R \textbf{13} dir] Vnd dir R  $\cdot$ gegeben] geben Q \textbf{14} dêswâr] Deswaz M Entswar Q Zwar Z  $\cdot$ tæter] hot er Q tuͦt er R  $\cdot$ ouch] vf O (L) (Q) umb R  $\cdot$ mîn] din L R \textbf{15} möhtestû] mohtestv G (I) (O) (L) (M) (Z) Gethon mocht er Q \textbf{16} sus] Vs Q  $\cdot$ er vriunde minnen] ir froude Mÿnne M \textbf{17} er dir aber] aber er dir I (O) (L) (R) her abir dir M (Q)  $\cdot$ ê] y M  $\cdot$ iht] ich L \textbf{18} din dienst hat so shier gedient solt I  $\cdot$ gedient] gediende L getent M  $\cdot$ sô] \textit{om.} R  $\cdot$ schiere] schrier R \textbf{19} getar] tar M  $\cdot$ dienen] gedienen Q  $\cdot$ swaz] waz L M (Q) (R) Z \textbf{20} vil] \textit{om.} M \textbf{22} wil] en wel M (Q) (R) (Z)  $\cdot$ ein] din I Z \textbf{24} greif] greých L greff R \textbf{25} wol] \textit{om.} M  $\cdot$ sîn] wesen O L (M) (Q) R Z  $\cdot$ Lehelin] Lechelin R \textbf{26} mir] dro R  $\cdot$ klaget] clagte L \textbf{27} der] ÷er O  $\cdot$ umbe kêrte] kerte vmb L \textbf{28} knappen] chappen I  $\cdot$ sô] do R \textbf{30} muosen] Mvͦse O  $\cdot$ vallende] ir val I vallen O L (M) (Q) (R) Z  $\cdot$ ûf] an I in L  $\cdot$ die bluomen] der erde I der blvͦmen O (L) (M) (R) (Z)  $\cdot$ sîn] schin O L (M) R Z hin Q \newline
\end{minipage}
\hspace{0.5cm}
\begin{minipage}[t]{0.5\linewidth}
\small
\begin{center}*T (U)
\end{center}
\begin{tabular}{rl}
 & daz dû den wîn vergüzze,\\ 
 & \textbf{u\textit{n}vuoge} dich verdrüzze.\\ 
 & ir dekeinen lustet strîtes.\\ 
 & gip mir, dâ \textbf{dûffe} rîtes,\\ 
5 & dar zuo al dîn harnas.\\ 
 & daz enpfienc ich ûf dem palas,\\ 
 & dâ inne ich rîter werden muoz.\\ 
 & widersaget sî dir mîn gruoz,\\ 
 & ob dû mir ez ungerne gîst.\\ 
10 & wer mich \textbf{ez}, ob dû bî witzen sîst."\\ 
 & der künec von Kukumerlant\\ 
 & sprach: "hât Artuses hant\\ 
 & dir mîn harnasch gegeben,\\ 
 & \textbf{dêswâr}, daz tæt er ouch mîn leben,\\ 
15 & m\textit{ö}htes dû\textbf{\textit{z} mir} \textbf{abe} gewinnen.\\ 
 & sus kan er vriunt \textit{m}innen.\\ 
 & was \textbf{aber er dir} iht holt,\\ 
 & \textbf{dîn dienst erholt sô schiere} den solt."\\ 
 & "ich getar wol \textbf{gedienen}, waz ich sol;\\ 
20 & ouch hât er mich gewert wol.\\ 
 & gip her und lâ dîn lantreht.\\ 
 & ich wil niht langer sîn ein kneht,\\ 
 & ich sol schiltes ambet hân."\\ 
 & er greif im nâch dem zoume sân:\\ 
25 & "dû maht wol \textbf{wesen} Lehelin,\\ 
 & von dem mir klagete diu muo\textit{t}e\textit{r} mîn."\\ 
 & der rîter umbe kêrte den schaft\\ 
 & und stach den knappen sô mit kraft,\\ 
 & daz er und sîn pferdelîn\\ 
30 & \textbf{vielen} ûf \textbf{der} bluomen \textbf{schîn}.\\ 
\end{tabular}
\scriptsize
\line(1,0){75} \newline
U V W T \newline
\line(1,0){75} \newline
\textbf{11} \textit{Überschrift:} Hie strait partzifal mit ytham von kayfies vnd erschos in W   $\cdot$ \textit{Platz für Illustration ausgespart} W   $\cdot$ \textit{Initiale} W   $\cdot$ \textit{Majuskel} T  \textbf{19} \textit{Initiale} T  \textbf{21} \textit{Majuskel} T  \textbf{27} \textit{Majuskel} T  \newline
\line(1,0){75} \newline
\textbf{2} unvuoge] Vnd vuͦge U Vngefuͤge W  $\cdot$ verdrüzze] er drússe W \textbf{3} strîtes] strttes V \textbf{4} dâ dûffe] [*]: har daz dv do V do auff du W \textbf{5} dar zuo] din ors vnd T  $\cdot$ al dîn] allen dinen V (W) dinen T \textbf{6} enpfienc ich] ich enpfing W \textbf{9} mir ez] mirs V T mirn W \textbf{10} wer] [*]: Gewer V gewer T  $\cdot$ mich ez] [*]: mich es V mich W T  $\cdot$ bî witzen] witzig W wîse T  $\cdot$ sîst] bist V \textbf{11} Kukumerlant] kuͦkuͦmerlant U kukumber land W \textbf{12} hât] hat dir W \textbf{13} dir] \textit{om.} W \textbf{14} daz tæt er ouch] destoͤt er W \textbf{15} möhtes dûz mir] Mocht iz duͦze mir U Ja moͤhtest dv [*]: mirs V Moͤchtestu mir es W mohtez dvz mir T \textbf{16} er] ir U  $\cdot$ minnen] gewinnen U [*]: minnen V \textbf{17} aber] \textit{om.} T  $\cdot$ iht] e v́t V (W) (T) \textbf{18} dîn dienst erholt] din dinst gediende (gedienet W ) V (W) dv hast gedient T  $\cdot$ sô] \textit{om.} W  $\cdot$ den] \textit{om.} V \textbf{19} ich] ÷ch T  $\cdot$ wol] \textit{om.} W T  $\cdot$ waz] swaz V (T) \textbf{22} wil] enwil V  $\cdot$ kneht] knechc W \textbf{23} \textit{nach 154.23:} vnd als vnsanfte niht me draben / alsich nv lange han getân T  \textbf{25} Lehelin] lehalin W \textbf{26} von dem mir klagete] Vber den mit klagt W  $\cdot$ muoter] muͦme U  $\cdot$ mîn] mir T \textbf{27} umbe kêrte den] kerte vmb seinen W [missekerte*]: missekerte den T \textbf{29} pferdelîn] orselin T \newline
\end{minipage}
\end{table}
\end{document}
