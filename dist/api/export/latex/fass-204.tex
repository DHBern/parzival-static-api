\documentclass[8pt,a4paper,notitlepage]{article}
\usepackage{fullpage}
\usepackage{ulem}
\usepackage{xltxtra}
\usepackage{datetime}
\renewcommand{\dateseparator}{.}
\dmyyyydate
\usepackage{fancyhdr}
\usepackage{ifthen}
\pagestyle{fancy}
\fancyhf{}
\renewcommand{\headrulewidth}{0pt}
\fancyfoot[L]{\ifthenelse{\value{page}=1}{\today, \currenttime{} Uhr}{}}
\begin{document}
\begin{table}[ht]
\begin{minipage}[t]{0.5\linewidth}
\small
\begin{center}*D
\end{center}
\begin{tabular}{rl}
\textbf{204} & diu küneginne  habe \textbf{besant}\\ 
 & Ithern von Kukumerlant.\\ 
 & des wâpen kom zer tjoste vür\\ 
 & unt wart getragen nâch prîses kür."\\ 
5 & der künec sprach zem knappen sân:\\ 
 & "Condwiramurs wil mich hân\\ 
 & unt ich ir lîp unt ir lant.\\ 
 & Kingrun, \textbf{mîn scheneschalt},\\ 
 & \textbf{mir} mit wârheit enbôt,\\ 
10 & si gæben die stat \textbf{durch} hungers nôt\\ 
 & unt daz diu küneginne\\ 
 & mir büte ir \textbf{werden} minne."\\ 
 & Der knappe \textbf{erwarp} dâ \textbf{niht wan} haz.\\ 
 & der künec mit her reit vürbaz.\\ 
15 & im kom ein ritter widervarn,\\ 
 & der ouch daz ors niht kunde sparn.\\ 
 & der seit \textbf{diu} selben mære.\\ 
 & Clamide wart swære\\ 
 & vreude unt rîterlîcher sin.\\ 
20 & \textbf{ez} dûhte in grôz ungewin.\\ 
 & des küneges \textbf{man}, \textbf{ein vürste}, sprach:\\ 
 & "Kingrunen \textbf{dâ} niemen sach\\ 
 & strîten vür unser manheit.\\ 
 & \textbf{niwan} vür sich einen \textbf{er} \textbf{dâ} streit.\\ 
25 & nû lât in sîn ze tôde erslagen.\\ 
 & sulen \textbf{durch} daz zwei her verzagen,\\ 
 & diz unt jenez vor der stat?"\\ 
 & sînen hêrren er trûren lâzen bat.\\ 
 & "wir sulenz \textbf{noch} baz versuochen.\\ 
30 & wellent si wer geruochen,\\ 
\end{tabular}
\scriptsize
\line(1,0){75} \newline
D \newline
\line(1,0){75} \newline
\textbf{13} \textit{Majuskel} D  \newline
\line(1,0){75} \newline
\textbf{2} Ithern] Jthern D  $\cdot$ Kukumerlant] Chvchvmerlant D \textbf{6} Condwiramurs] Condwir amvrs D \textbf{8} Kingrun] kingrvͦn D \textbf{18} Clamide] Chlamide D \newline
\end{minipage}
\hspace{0.5cm}
\begin{minipage}[t]{0.5\linewidth}
\small
\begin{center}*m
\end{center}
\begin{tabular}{rl}
 & diu künigîn habe \textbf{besant}\\ 
 & Ither\textit{n} von Kukumerlant.\\ 
 & des wâpen kam zer juste vür\\ 
 & und wart getragen nâch prîses kür."\\ 
5 & der künic sprach zem knappen sân:\\ 
 & "Condwieramurs wil mich hân\\ 
 & und ich ir lîp und ir lant.\\ 
 & Kingr\textit{u}n, \textbf{mîn schinschant},\\ 
 & \textbf{mir} mit wâ\textit{r}heit enbôt,\\ 
10 & si g\textit{æ}ben die stat \textbf{durch} hungers nôt\\ 
 & und daz diu küniginne\\ 
 & mir büt\textit{e} ir \textbf{werde} minne."\\ 
 & der knappe \textbf{erwarp} d\textit{â} \textbf{niuwen} haz.\\ 
 & der künic mit her reit vürbaz.\\ 
15 & im kam ein ritter widervarn,\\ 
 & der ouch daz ros niht kunde sparn.\\ 
 & der seite \textbf{die} selben mære.\\ 
 & Cla\textit{m}ide wart swære\\ 
 & vröude und ritterlîcher sin.\\ 
20 & \textbf{ez} dûht in grôz ungewin.\\ 
 & \begin{large}D\end{large}es küniges \textbf{vürsten einer} sprach:\\ 
 & "Kingr\textit{un}en niemen sach\\ 
 & strîte\textit{n} \textit{v}ür unser manheit.\\ 
 & \textbf{\textit{n}iwan} vür sich eine\textit{n} \textbf{er} \textbf{d\textit{â}} streit.\\ 
25 & nû lât in sîn ze tôde erslagen.\\ 
 & sullen \textbf{durch} daz zwei her verzagen,\\ 
 & diz und jenez vor der stat?"\\ 
 & sînen hêrren er trûren lâzen bat.\\ 
 & "wir sullen ez baz versuochen.\\ 
30 & wellent si wer geruochen,\\ 
\end{tabular}
\scriptsize
\line(1,0){75} \newline
m n o Fr69 \newline
\line(1,0){75} \newline
\textbf{21} \textit{Initiale} m n o  \newline
\line(1,0){75} \newline
\textbf{1} besant] gesant o \textbf{2} Ithern] Jther m Jthern n o Jethern Fr69  $\cdot$ Kukumerlant] cucumer lant m kontumur lant n cocumerlant o gugumerlant Fr69 \textbf{4} getragen] getrages n \textbf{6} Condwieramurs] Conduwier armes n Condiwier amers o Condewier amurs Fr69 \textbf{7} ich] ich vnd o \textbf{8} Kingrun] Kingrin m Kingruͯn n Konigrẏm o \textbf{9} wârheit] woheit m  $\cdot$ enbôt] enbat o \textbf{10} gæben] gaben m gohent n (o) \textbf{12} büte] buͯttet m bitte n  $\cdot$ werde] werden o \textbf{13} dâ] do m n o \textbf{14} vürbaz] do fúrbasz n \textbf{17} selben] selbe n o \textbf{18} Clamide] Clanide m \textbf{22} Kingrunen] Kingrimen m \textbf{23} strîten vür] Stritten vnd fuͯr m \textbf{24} niwan vür] Miwan fur m Fúr n o  $\cdot$ einen] einem m  $\cdot$ er] der o  $\cdot$ dâ] do m n o \newline
\end{minipage}
\end{table}
\newpage
\begin{table}[ht]
\begin{minipage}[t]{0.5\linewidth}
\small
\begin{center}*G
\end{center}
\begin{tabular}{rl}
 & diu küniginne habe \textbf{gesant}\\ 
 & Itheren von Kukumerlant.\\ 
 & des wâpen kom zer tjoste vür\\ 
 & unde wart getragen \textit{nâch} brîses kür."\\ 
5 & der künic sprach zem knappen sân:\\ 
 & "Condwiramurs wil mich hân\\ 
 & \begin{large}U\end{large}nde ich ir lîp unt ir lant.\\ 
 & Kingrun, \textbf{mîn sinschalt},\\ 
 & \textbf{mir} mit wârheit enbôt,\\ 
10 & si g\textit{æ}ben die stat \textbf{von} hungers nôt\\ 
 & unde daz diu küniginne\\ 
 & mir büte \textbf{vaste} ir minne."\\ 
 & der knappe \textbf{vant} dâ \textbf{niht wan} haz.\\ 
 & der künic mit her reit vürbaz.\\ 
15 & im kom ein rîter widervarn,\\ 
 & der ouch daz ors niht kunde sparn.\\ 
 & der sagte \textbf{diu} selben mære.\\ 
 & Clamide wart swære\\ 
 & vröude unde rîterlîcher sin.\\ 
20 & \textbf{ez} dûht in grôz ungewin.\\ 
 & des küniges \textbf{man}, \textbf{ein vürste}, sprach:\\ 
 & "Kingrunen niemen sach\\ 
 & strîten vür unser manheit.\\ 
 & \textbf{niht wan} vür sich einen \textbf{er} streit.\\ 
25 & nû lât in sîn ze tôde erslagen.\\ 
 & sulen \textbf{durch} daz zwei her verzagen,\\ 
 & diz unde jenez vor der stat?"\\ 
 & sînen hêrren er trûren lâzen bat.\\ 
 & "wir sulenz \textbf{noch} baz versuochen.\\ 
30 & \textbf{unde} wellent si wer geruochen,\\ 
\end{tabular}
\scriptsize
\line(1,0){75} \newline
G I O L M Q R Z Fr21 \newline
\line(1,0){75} \newline
\textbf{5} \textit{Initiale} I O L M Fr21  \textbf{7} \textit{Initiale} G  \textbf{9} \textit{Initiale} R  \textbf{13} \textit{Initiale} Q   $\cdot$ \textit{Capitulumzeichen} L  \textbf{15} \textit{Initiale} Z  \textbf{21} \textit{Initiale} I  \newline
\line(1,0){75} \newline
\textbf{1} habe] dar hab R  $\cdot$ gesant] besant Z \textbf{2} Itheren] ytern I Ythern O Jhtern L R Jch ern M Jthern Q Fr21 Jchern Z  $\cdot$ Kukumerlant] kvcumerlant O (Q) (Fr21) kvcuͯmerlant L kvnkvmerlant Z \textbf{3} zer] zuͯ L (R) \textbf{4} nâch] mit G nohc O  $\cdot$ brîses] prise Z \textbf{5} der] ÷er O  $\cdot$ zem] >zuͯ< den L zcu der M  $\cdot$ sân] \textit{om.} I \textbf{6} Condwiramurs] Gondwiramurs I kvndwiramvrs O Condwir amvrs L Kondwir amuͯrs M Kundwiramúrs Q Kundwiamuͦrs R Kvndewiramvrs Z Kvndwir amvrs Fr21  $\cdot$ wil] div wil O Fr21 \textbf{7} ich] och R \textbf{8} Kingrun] Kyngrvn O M (R) Kýngrvn L Kyngrún Q  $\cdot$ mîn sinschalt] min schiniscalt I mîn schenechant O wil ez wesen phant L myn sinetschalt M mein seneckant Q min sinetschant R min smetschalant Z min seneschant Fr21 \textbf{9} Wan er mit warheit mir enbot L \textbf{10} gæben] gaben G  $\cdot$ von] dvrch O (L) (M) (R) (Z) (Fr21) mit Q  $\cdot$ hungers] hvnger L \textbf{11} daz] \textit{om.} L \textbf{12} büte vaste] vaste bot I bv̂t O (Q) (R) (Z) enbot L bot M Fr21  $\cdot$ minne] werde minne O (L) (M) Q R Z (Fr21) \textbf{13} vant dâ] do I er warp da O (L) (M) (Z) (Fr21) warbp do Q gewan R  $\cdot$ niht wan haz] im widersaz I \textbf{14} mit her reit] reit mit hêr O mit dem her reyt Q \textbf{15} im] Jn M \textbf{16} daz] \textit{om.} Fr21 \textbf{17} sagte] seit I (O) (Q) R (Z) (Fr21) \textbf{18} Clamide] Glamide O  $\cdot$ swære] svare Fr21 \textbf{19} rîterlîcher] ritterlichen L (Q)  $\cdot$ sin] schyn M (R) \textbf{20} ez dûht in] in duhte I Es ducht R  $\cdot$ grôz] ein grosser Q (R)  $\cdot$ ungewin] [gewin]: vngewin Q \textbf{21} küniges] kvnig L [kvnige]: kvniges Fr21  $\cdot$ ein] Jm R \textbf{22} Kingrunen] chingruͤn I Kyngrvnen O (R) Kýngrvnen L Kyngruͯnen M Kingrúnen Q \textbf{23} vür] von L \textbf{24} niht wan] \textit{om.} L  $\cdot$ einen] selben O ein Z  $\cdot$ streit] do streit O Q da streit L M Z Fr21 da reit R \textbf{25} in] \textit{om.} I  $\cdot$ sîn] \textit{om.} Q \textbf{26} durch daz] dar vmbe L  $\cdot$ zwei her] czweyn herren M \textbf{27} diz] Dir M  $\cdot$ jenez] iener M \textbf{28} er trûren lâzen] er zurn lazen I trurren laszen er do R \textbf{29} wir] Wirs Q  $\cdot$ sulenz] suln I (R) \textbf{30} unde] \textit{om.} R  $\cdot$ wer] uwer M (Q) \newline
\end{minipage}
\hspace{0.5cm}
\begin{minipage}[t]{0.5\linewidth}
\small
\begin{center}*T
\end{center}
\begin{tabular}{rl}
 & diu künegîn \textbf{in} habe \textbf{besant}:\\ 
 & Ithern von Cukumberlant.\\ 
 & des wâpen kom zer tjost vür\\ 
 & unde wart getragen nâch prîses kür."\\ 
5 & Der künec sprach zeme knappen sân:\\ 
 & "Kundewiramurs wil mich hân\\ 
 & unde ich ir lîp unde ir lant.\\ 
 & Kyngrun \textbf{wil es wesen pfant},\\ 
 & \textbf{wander} mit wârheit \textbf{mir} enbôt,\\ 
10 & si gæben die stat \textbf{durch} hungers nôt\\ 
 & unde daz diu küneginne\\ 
 & mir büte ir \textbf{werde} mi\textit{nn}e."\\ 
 & \begin{large}D\end{large}er knappe \textbf{erwarp} dâ \textbf{niht wan} haz.\\ 
 & der künec mit her reit vürbaz.\\ 
15 & im kom ein rîter widervarn,\\ 
 & der ouch daz ors niht kunde sparn.\\ 
 & der sagete \textbf{die} selben mære.\\ 
 & Clamide wart swære\\ 
 & vröude unde rîterlîcher sin.\\ 
20 & \textbf{daz} dûhtin grôz ungewin.\\ 
 & Des küneges \textbf{man}, \textbf{ein vürste}, sprach:\\ 
 & "Kyngrunen niemen sach\\ 
 & strîten vür unser manheit.\\ 
 & vür sich einen \textbf{der} \textbf{dâ} streit.\\ 
25 & nû lât in sîn ze tôde erslagen.\\ 
 & s\textit{ul}n \textbf{umbe} daz zwei her verzagen,\\ 
 & diz unde jenez vor der stat?"\\ 
 & sînen hêrren er trûren lâzen bat.\\ 
 & "wir suln ez baz versuochen.\\ 
30 & wellent si wer geruochen,\\ 
\end{tabular}
\scriptsize
\line(1,0){75} \newline
T U V W \newline
\line(1,0){75} \newline
\textbf{5} \textit{Majuskel} T  \textbf{13} \textit{Initiale} T U W  \textbf{21} \textit{Majuskel} T  \newline
\line(1,0){75} \newline
\textbf{1} in habe besant] [*]: dar hatte bekant V hab in dar gesand W \textbf{2} Ithern] Jthern T U Ẏtern V Ythern W  $\cdot$ Cukumberlant] Cuͦkuͦmerlant U kvcumerlant V kukumber land W \textbf{3} kom] [*]: kam V kan W \textbf{6} Kundewiramurs] Kuͦndewiramuͦrs U [Kvndewiramu*]: Kvndewiramurs V Gundwiramurs W  $\cdot$ hân] doch han W \textbf{7} ich] \textit{om.} W \textbf{8} Kyngrun] Kyngruͦn U Kingrun V W \textbf{9} wander] Wann er W \textbf{10} si gæben] [S*geben]: Sv́ ergeben V Sy gaben W \textbf{12} büte] buͦt U  $\cdot$ minne] mimme T \textbf{13} dâ] do V \textit{om.} W \textbf{14} her] dem her W \textbf{17} selben] selbe U V \textbf{18} swære] hart schwere W \textbf{19} sin] [*]: sin V \textbf{20} grôz ungewin] grossen gewin W \textbf{22} Kyngrunen] Kyngruͦnen U Kẏngrunen V Kingrun W  $\cdot$ niemen] niemam U hie nieman V \textbf{24} der dâ] er do U V W \textbf{25} sîn ze tôde] zuͦ tode sein W \textbf{26} suln] slvn T \textbf{27} jenez] iehens U \textbf{29} baz] noch baz V \textbf{30} wer] vns U \newline
\end{minipage}
\end{table}
\end{document}
