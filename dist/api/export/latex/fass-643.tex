\documentclass[8pt,a4paper,notitlepage]{article}
\usepackage{fullpage}
\usepackage{ulem}
\usepackage{xltxtra}
\usepackage{datetime}
\renewcommand{\dateseparator}{.}
\dmyyyydate
\usepackage{fancyhdr}
\usepackage{ifthen}
\pagestyle{fancy}
\fancyhf{}
\renewcommand{\headrulewidth}{0pt}
\fancyfoot[L]{\ifthenelse{\value{page}=1}{\today, \currenttime{} Uhr}{}}
\begin{document}
\begin{table}[ht]
\begin{minipage}[t]{0.5\linewidth}
\small
\begin{center}*D
\end{center}
\begin{tabular}{rl}
\textbf{643} & \textit{\begin{large}K\end{large}}unnen si zwei nû minne steln,\\ 
 & daz mag ich unsanfte heln.\\ 
 & ich sage vil lîhte, \textbf{daz} \textbf{dâ} geschach,\\ 
 & wan daz man \textbf{dem} \textbf{unvuoge ie} jach,\\ 
5 & der verholniu mære \textbf{ie} machte breit.\\ 
 & Ez ist \textbf{ouch} noch den \textbf{höfschen} leit,\\ 
 & ouch unsæliget er sich dâr mite.\\ 
 & zuht \textbf{sî des} slôz ob minne site.\\ 
 & Nû vuogete diu strenge minne\\ 
10 & unt diu clâre herzoginne,\\ 
 & daz Gawans \textbf{vreude} \textbf{was} verzert:\\ 
 & er wære immer \textbf{unerwert}\\ 
 & sunder âmîen.\\ 
 & die philosophîen\\ 
15 & unt alle, die ie gesâzen,\\ 
 & dâ si starke liste mâzen,\\ 
 & Kancor unt \textbf{Thebit}\\ 
 & unt Trebuchet, der smit,\\ 
 & der Frimutels swert ergruop,\\ 
20 & dâ von sich starkez wunder huop,\\ 
 & Dar zuo \textbf{al} der arzâte kunst,\\ 
 & ob si im trüegen guote gunst\\ 
 & mit temperîe \textbf{ûz} würze kraft,\\ 
 & âne wîplîch geselleschaft\\ 
25 & sô müeser sîne \textbf{scherpfe} nôt\\ 
 & hân brâht unz an \textbf{den sûren} tôt.\\ 
 & Ich wil \textbf{iu}z mære machen kurz:\\ 
 & er vant die rehten hirzwurz,\\ 
 & diu im half, daz er genas,\\ 
30 & sô daz im arges \textbf{niht} enwas;\\ 
\end{tabular}
\scriptsize
\line(1,0){75} \newline
D Z Fr1 \newline
\line(1,0){75} \newline
\textbf{1} \textit{Initiale} D Z Fr1  \textbf{6} \textit{Majuskel} D  \textbf{9} \textit{Majuskel} D   $\cdot$ \textit{Versal} Fr1  \textbf{21} \textit{Majuskel} D  \textbf{27} \textit{Majuskel} D  \newline
\line(1,0){75} \newline
\textbf{1} Kunnen] Bvnnen D \textbf{3} vil lîhte daz dâ] ev liht waz daz Z \textbf{4} man dem unvuoge ie] dem die vnfuge Z \textbf{5} ie machte] machet Z machte Fr1 \textbf{8} sî des] sich Z  $\cdot$ minne] mime Fr1 \textbf{12} unerwert] vnernert Z Fr1 \textbf{17} Kancor] Chanchor D (Fr1) Chancor Z  $\cdot$ Thebit] Thebît D Tebit Z \textbf{18} Trebuchet] Trebucher Z Trebichet Fr1 \textbf{22} trüegen] trvgen Z  $\cdot$ guote gunst] [gvͦ*]: gvnst Fr1 \textbf{26} den] sinen Z \textbf{30} sô] \textit{om.} Z \newline
\end{minipage}
\hspace{0.5cm}
\begin{minipage}[t]{0.5\linewidth}
\small
\begin{center}*m
\end{center}
\begin{tabular}{rl}
 & k\textit{ünn}en si zwei nû minne steln,\\ 
 & daz mac ich unsanfte heln.\\ 
 & ich sage vil lîht, \textbf{waz} \textbf{d\textit{â}} geschach,\\ 
 & wan daz man \textbf{ungevüege ie} jach,\\ 
5 & der verholniu mære maht\textit{e} breit.\\ 
 & ez ist \textbf{ouch} noch den \textbf{hœhesten} leit,\\ 
 & ouch unsæliget er sich dâ mite.\\ 
 & zuht \textbf{ist daz} slôz ob minnen site.\\ 
 & \begin{large}N\end{large}û vuoget diu strenge minne\\ 
10 & und diu clâre herzoginne,\\ 
 & daz Gawans \textbf{wer} \textbf{was} verzert:\\ 
 & er wær iemer \textbf{unernert}\\ 
 & sunder âmîen.\\ 
 & die philo\textit{so}p\textit{h}îen\\ 
15 & und alle, die ie gesâzen,\\ 
 & d\textit{â} si starke liste mâzen,\\ 
 & Kanchor und \textbf{Thebit}\\ 
 & und Trebuchet, der smit,\\ 
 & der \dag Frimutel\dag  swert ergruop,\\ 
20 & dâ von sich starkez wunder huop,\\ 
 & dar zuo \textbf{aller} der arzet kunst,\\ 
 & \textit{ob} si im trüegen guote gunst\\ 
 & mit temper\textit{îe} \textbf{ûz} würze kraft,\\ 
 & âne wîplîch geselleschaft\\ 
25 & sô mües er sîn \textbf{scharpfe} nôt\\ 
 & hân brâht unz an \textbf{sînen} tôt.\\ 
 & ich wil daz mære mache\textit{n k}urz:\\ 
 & er vant die rehten hirzwurz,\\ 
 & diu im half, daz er genas,\\ 
30 & sô daz im arges \textbf{niht} enwas;\\ 
\end{tabular}
\scriptsize
\line(1,0){75} \newline
m n o \newline
\line(1,0){75} \newline
\textbf{9} \textit{Initiale} m n  \newline
\line(1,0){75} \newline
\textbf{1} Künnen] Komen m \textbf{3} lîht] liecht o  $\cdot$ dâ] do m n o \textbf{4} ungevüege] vnfúge n dem vngefuge o \textbf{5} mahte] mahtten m \textbf{6} hœhesten] honeschen n hoͯfesten o \textbf{7} unsæliget] vngeseliget o \textbf{8} minnen] mẏnem o \textbf{11} was verzert] verzaget n \textbf{14} philosophîen] pfilopsien m [pfilosophoph*]: pfilosophophyen n \textbf{16} dâ] Do m n o \textbf{19} der] Die n  $\cdot$ Frimutel] frimuttel m \textbf{22} ob] Vnd m  $\cdot$ trüegen] truͦgent o \textbf{23} temperîe] temper m temperi n tempori o \textbf{25} sô] Sús o  $\cdot$ mües] mus m muͯsz n o \textbf{27} wil] wil úch n (o)  $\cdot$ machen kurz] machen kunt vnd kurtz m machen [kunt]: kurtz n \textbf{28} hirzwurz] hitze wurtz n \textbf{30} arges] argas n \newline
\end{minipage}
\end{table}
\newpage
\begin{table}[ht]
\begin{minipage}[t]{0.5\linewidth}
\small
\begin{center}*G
\end{center}
\begin{tabular}{rl}
 & \begin{large}K\end{large}unnen si zwei nû minne steln,\\ 
 & daz mag ich unsanfte heln.\\ 
 & ich sage vil lîhte, \textbf{waz} \textbf{dâ} geschach,\\ 
 & wan daz man \textbf{dem} \textbf{ie ungevüege} jach,\\ 
5 & der verholn\textit{iu} mær machet breit.\\ 
 & ez ist noch den \textbf{hovescharn} leit,\\ 
 & ouch unsæliget er sich dâr mite.\\ 
 & \textit{zuht} \textit{\textbf{ist ein}} slôz op minne site.\\ 
 & nû vuoget diu strenge minne\\ 
10 & unde diu clâre herzoginne,\\ 
 & daz Gawans \textbf{sorge} \textbf{wart} verzert:\\ 
 & er wær immer \textbf{unernert}\\ 
 & sunder âmîen.\\ 
 & die philosophîen\\ 
15 & unde alle, die ie gesâzen,\\ 
 & dâ si starke liste mâzen,\\ 
 & Charncor unde \textbf{Bebit}\\ 
 & unde Trebuchet, der smit,\\ 
 & der Frimutels swert ergruop,\\ 
20 & dâ von sich starke\textit{z} wunder huop,\\ 
 & dar zuo \textbf{al} der arzet kunst,\\ 
 & ob si im trüegen guote gunst\\ 
 & mit temper\textit{î}e \textbf{unde mit} würze kraft,\\ 
 & âne wîplîche geselleschaft\\ 
25 & sô m\textit{üe}se er sîn\textit{e} \textbf{swære} nôt\\ 
 & hân brâht unze an \textbf{sînen} tôt.\\ 
 & ich wil \textbf{iu} daz mære machen kurz:\\ 
 & er vant die rehten hirzwurz,\\ 
 & diu im half, daz er genas,\\ 
30 & sô daz im arges \textbf{nie}ne was;\\ 
\end{tabular}
\scriptsize
\line(1,0){75} \newline
G I L M Z Fr18 \newline
\line(1,0){75} \newline
\textbf{1} \textit{Initiale} G L Z Fr18  \textbf{9} \textit{Initiale} I  \newline
\line(1,0){75} \newline
\textbf{1} Kunnen] Kunne M  $\cdot$ nû minne] mynnen M \textbf{2} mag] chan I \textbf{3} sage] sage uͯch L (M) (Z)  $\cdot$ vil] \textit{om.} L Z  $\cdot$ waz] daz L  $\cdot$ dâ] daz Z \textbf{4} man] \textit{om.} Z  $\cdot$ ie ungevüege] vnfuge ýe L (Fr18) die vnfuge Z \textbf{5} der] Die Fr18  $\cdot$ verholniu mær] uirholne mær G (Z) verholnie mere L vorholne M  $\cdot$ machet] machent Fr18 \textbf{6} noch] ouch nach Z  $\cdot$ den] \textit{om.} L  $\cdot$ hovescharn] houischærn G hoͤfsheren I (M) hvbschen Z :::schbæren Fr18 \textbf{7} dâr mite] \textit{om.} I mit L \textbf{8} zuht ist ein] \textit{om.} G sý L (M) (Fr18) Zvht sich Z  $\cdot$ minne] minnen I \textbf{10} clâre] clage L \textbf{11} Gawans] Gawanz L :::ans Fr18  $\cdot$ sorge] frovde L (M) (Z) (Fr18)  $\cdot$ wart] was I \textbf{12} unernert] vnerwert L \textbf{13} sunder âmîen] [Sundirnamien]: Sundirn amien M \textbf{15} ie] \textit{om.} L \textbf{16} dâ] Daz L  $\cdot$ liste] minne I \textbf{17} Charncor] Granchor I Crancor L Krankor M Chancor Z :::r Fr18  $\cdot$ Bebit] Tebit L Z debit M :ebiet Fr18 \textbf{18} Trebuchet] Trebuch I trebuchit M Trebucher Z ::ebvchet Fr18  $\cdot$ smit] ::r smit Fr18 \textbf{19} \textit{Versdoppelung 643.19-21 vor 643.19 auf vorhergehender, sonst unbeschriebener Seite, radiert:} Der fr:::t:::ls ::: e:gruͦp / :a::: uon sich ::: erhoͧp / dar zuͦ al ::: G   $\cdot$ der] Des M  $\cdot$ Frimutels] frimuldes G frimundels I frymutels M ::: Fr18 \textbf{20} starkez] starches G \textbf{21} al der] aller I  $\cdot$ arzet] [az]: arzet G \textbf{22} trüegen] trugen I  $\cdot$ guote] gute guͤte I \textbf{23} temperîe] tempre G  $\cdot$ unde mit] vz Z  $\cdot$ würze] wurzen I \textbf{25} müese er] muͦse er G (L) muste M  $\cdot$ sîne swære] siner swære G sine scharphe L M (Z) \textbf{26} sînen] den I sinen suren M (Z) \textbf{27} kurz] [churze]: churz G \textbf{30} sô] \textit{om.} I Z  $\cdot$ arges] [a*]: arges G  $\cdot$ niene] niht ein L \newline
\end{minipage}
\hspace{0.5cm}
\begin{minipage}[t]{0.5\linewidth}
\small
\begin{center}*T
\end{center}
\begin{tabular}{rl}
 & k\textit{unn}en si zwei nû minne stelen,\\ 
 & daz mac ich unsanfte helen.\\ 
 & ich sage vil lîhte, \textbf{waz} geschach,\\ 
 & wan daz man \textbf{dem} \textbf{ie unvuoge} jach,\\ 
5 & der verholniu mære machet breit.\\ 
 & ez ist noch den \textbf{höveschsten} leit,\\ 
 & ouch unsæliget er sich dâ mite.\\ 
 & zuht \textbf{sî des} slôz ob minne site.\\ 
 & nû vuogte diu strenge minne\\ 
10 & und diu clâre herzoginne,\\ 
 & daz Gawans \textbf{vreude} \textbf{was} verzert:\\ 
 & er wære immer \textbf{unernert}\\ 
 & sunder âmîen.\\ 
 & die philo\textit{so}phîen\\ 
15 & und alle, die ie gesâzen,\\ 
 & d\textit{â} si starke liste mâzen,\\ 
 & Kanchor und \textbf{Thepi\textit{t}}\\ 
 & und Trebuket, der smit,\\ 
 & der Frimutels swert ergruop,\\ 
20 & dâ von sich starkez wunder huop,\\ 
 & dâ zuo der arzet kunst,\\ 
 & ob si im trüegen guote gunst\\ 
 & mit temperîe \textbf{ûz} würze kraft,\\ 
 & âne wîplîch geselleschaft\\ 
25 & sô müest er sîne \textbf{scharpfe} nôt\\ 
 & hân brâht unz an \textbf{den sûren} tôt.\\ 
 & ich wil \textbf{iu} daz mære machen kurz:\\ 
 & er vant die rehten hirzwurz,\\ 
 & diu im half, daz er\textbf{s} genas,\\ 
30 & sô daz im arges \textbf{nie} enwas;\\ 
\end{tabular}
\scriptsize
\line(1,0){75} \newline
Q R W V \newline
\line(1,0){75} \newline
\textbf{1} \textit{Initiale} Q V   $\cdot$ \textit{Capitulumzeichen} R  \newline
\line(1,0){75} \newline
\textbf{1} kunnen] Komen Q  $\cdot$ minne] minnen V  $\cdot$ stelen] [stillen]: stellen Q steln vnd heln R \textbf{2} ich] \textit{om.} R  $\cdot$ helen] vnmeln R \textbf{3} \textit{Versfolge 643.4-5-3} Q   $\cdot$ geschach] ich gesach R do geschach W V \textbf{4} unvuoge] vngefvͦge V \textbf{5} verholniu] verholne R \textbf{6} den] dem R  $\cdot$ höveschsten] hoffenschem R hoͤuischen W (V) \textbf{8} sî] [*]: si V  $\cdot$ des] das W  $\cdot$ minne] minnen R (V) \textbf{9} vuogte] fuͦrte W \textbf{11} Gawans] Gawins R hern Gawans V  $\cdot$ vreude] [froͯw*]: froͯde R trvren V  $\cdot$ was] ward R \textbf{13} âmîen] sine amien V \textbf{14} die] Die meister von V  $\cdot$ philosophîen] philopfien Q \textbf{16} dâ] Do Q R W Daz V \textbf{17} Kanchor] Chanchor Q Tankor V  $\cdot$ Thepit] thepic Q Thebit R (W) tebit V \textbf{18} Trebuket] tuchbuchet Q Trebuͦkett R trebuchet W \textbf{19} Frimutels] frimútels Q frimitels W frimuntels V \textbf{21} dâ] Das W  $\cdot$ der] aller der R aller W V \textbf{22} guote] guͦtten R (V) \textbf{23} temperîe ûz würze] temperieren wurczen R \textbf{25} müest] muͦß W \textbf{26} sûren] suren grimen R \textbf{27} iu] úchs R \textbf{28} rehten] rechte Q  $\cdot$ hirzwurz] huswurcz R hertz wurtz W \textbf{29} ers] er R W V \textbf{30} nie] nit nie R nymmer W niht V  $\cdot$ enwas] was R W \newline
\end{minipage}
\end{table}
\end{document}
