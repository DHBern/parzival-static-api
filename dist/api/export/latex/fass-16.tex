\documentclass[8pt,a4paper,notitlepage]{article}
\usepackage{fullpage}
\usepackage{ulem}
\usepackage{xltxtra}
\usepackage{datetime}
\renewcommand{\dateseparator}{.}
\dmyyyydate
\usepackage{fancyhdr}
\usepackage{ifthen}
\pagestyle{fancy}
\fancyhf{}
\renewcommand{\headrulewidth}{0pt}
\fancyfoot[L]{\ifthenelse{\value{page}=1}{\today, \currenttime{} Uhr}{}}
\begin{document}
\begin{table}[ht]
\begin{minipage}[t]{0.5\linewidth}
\small
\begin{center}*D
\end{center}
\begin{tabular}{rl}
\textbf{16} & sîn ellen \textbf{strebte} sunder wanc.\\ 
 & \textbf{von} dan vuor \textbf{ein} gein Zazamanc\\ 
 & in daz künecrîche.\\ 
 & die klageten \textbf{al gelîche}\\ 
5 & Isenharten, der den lîp\\ 
 & \textbf{in dieneste} verlôs umbe ein wîp.\\ 
 & de\textit{s} twang in Belakane,\\ 
 & diu süeze, valsches âne.\\ 
 & daz si im ir minne \textbf{nie} \textbf{gebôt},\\ 
10 & des lag er nâch ir minne tôt.\\ 
 & \begin{large}D\end{large}en râchen sîne mâge\\ 
 & offenlîche unde an der lâge.\\ 
 & die vrouwen twungen si mit her.\\ 
 & diu was mit ellenthafter wer,\\ 
15 & dô Gahmuret kam in \textbf{ir} lant,\\ 
 & daz von \textit{Sch}otten Vridebrant\\ 
 & mit \textbf{schiffes} her verbrande,\\ 
 & ê \textbf{daz} er dannen wande.\\ 
 & nû hœret, wie unser rîter var!\\ 
20 & daz mer warf in mit sturme dar,\\ 
 & sô daz er kûme \textbf{iedoch} genas.\\ 
 & gein der küneginne palas\\ 
 & kom er gesigelt in die habe.\\ 
 & dâ wart \textbf{\textit{e}r} vil \textbf{beschouwet} abe.\\ 
25 & dô sach er ûz anz velt.\\ 
 & dâ was geslagen manec \textbf{zelt}\\ 
 & al umbe die stat \textbf{wan} \textbf{gein dem} mer.\\ 
 & dâ \textbf{lâgen zwei kreftigiu} her.\\ 
 & dô h\textit{ie}z er \textit{v}râgen \textbf{der} mære,\\ 
30 & wes diu burc wære.\\ 
\end{tabular}
\scriptsize
\line(1,0){75} \newline
D Fr9 \newline
\line(1,0){75} \newline
\textbf{11} \textit{Initiale} D  \textbf{15} \textit{Initiale} Fr9  \newline
\line(1,0){75} \newline
\textbf{2} ein] her Fr9  $\cdot$ Zazamanc] zazamanch D \textbf{5} Isenharten] Jsenharten D Ẏsenharten Fr9 \textbf{6} in dieneste] Zvͦ eẏner ziost Fr9 \textbf{7} des] dez D  $\cdot$ Belakane] [belane]: belacane D belekane Fr9 \textbf{8} diu] [dv]: div D \textbf{12} lâge] [lagt]: lage D \textbf{15} Gahmuret] gamvret Fr9 \textbf{16} Schotten] chsotten D scotten Fr9 \textbf{21} iedoch] \textit{om.} Fr9 \textbf{24} er] her D  $\cdot$ beschouwet] gescouwet Fr9 \textbf{25} dô sach er] [*]: do saher D \textbf{29} dô] Da Fr9  $\cdot$ hiez] heiz D  $\cdot$ vrâgen] wragen D \newline
\end{minipage}
\hspace{0.5cm}
\begin{minipage}[t]{0.5\linewidth}
\small
\begin{center}*m
\end{center}
\begin{tabular}{rl}
 & sîn ellen \textbf{strebete} sunder wanc.\\ 
 & \textbf{von} dannen vuor \textbf{er} gegen Zazamanc\\ 
 & in daz künicrîche.\\ 
 & die klag\textit{e}te\textit{n} \textbf{alle glîche}\\ 
5 & Ysenharten, der den lîp\\ 
 & \textbf{in dienste} vlôs umb ei\textit{n} wîp.\\ 
 & des twanc in Belakane,\\ 
 & diu süeze, valsches âne.\\ 
 & daz si ime ir minne \textbf{nie} \textbf{gebôt},\\ 
10 & des lac er nâch ir minne tôt.\\ 
 & den râchen sîne mâge\\ 
 & offenlîche und an der lâge.\\ 
 & die vrowen twungen si mit her.\\ 
 & diu was mit ellenthaf\textit{t}er wer,\\ 
15 & \begin{large}D\end{large}ô Gahmuret kam in \textbf{daz} lant,\\ 
 & daz von Schotten Fridebrant\\ 
 & mit \textbf{schiffes} her verbrante,\\ 
 & ê \textbf{daz} er dannen wante.\\ 
 & nû hœret, wie unser ritter var!\\ 
20 & daz mer warf in mit sturme dar,\\ 
 & sô daz er kûme \textbf{iedoch} genas.\\ 
 & gegen der küniginn\textit{e} palas\\ 
 & kam er gesigelt in die habe.\\ 
 & d\textit{â} wart \textbf{er} vil \textbf{geschouwet} abe.\\ 
25 & dô sach er ûz an daz velt.\\ 
 & d\textit{â} was geslagen manic \textbf{gezelt}\\ 
 & al umb die stat \textbf{d\textit{â}} \textbf{gegen dem} mere.\\ 
 & d\textit{â} \textbf{lâgen zwei kreftig\textit{iu}} here.\\ 
 & dô hiez er vrâgen \textbf{der} mære,\\ 
30 & wes diu burc wære.\\ 
\end{tabular}
\scriptsize
\line(1,0){75} \newline
m n o \newline
\line(1,0){75} \newline
\textbf{15} \textit{Initiale} m   $\cdot$ \textit{Capitulumzeichen} n  \newline
\line(1,0){75} \newline
\textbf{1} strebete] [stebte]: strebete o \textbf{2} gegen] gan n  $\cdot$ Zazamanc] zazamanck m zazamang n o \textbf{4} klageten] klagentend m  $\cdot$ alle glîche] alglich o \textbf{5} Ysenharten] Jsenharten m (n) (o) \textbf{6} in] Jn den n  $\cdot$ vlôs] flosz \textit{nachträglich korrigiert zu:} verlosz m  $\cdot$ ein] eÿm m \textbf{7} in] an o  $\cdot$ Belakane] belacane m belbetane n belbekane o \textbf{8} diu süeze] Das flisse o \textbf{9} ir] \textit{om.} o \textbf{10} ir] ie o  $\cdot$ minne] [mye]: mynne n \textbf{12} lâge] kage o \textbf{13} twungen] twingen n twingent o \textbf{14} ellenthafter] ellenthaffer m \textbf{15} Dô] Da o  $\cdot$ Gahmuret] gamiret n gamuͯret o \textbf{16} Fridebrant] [fridebant]: fridebrant o \textbf{17} schiffes] scifftes o \textbf{18} er dannen] [er*]: erwanden o \textbf{21} iedoch] ÿe [noch]: doch m \textbf{22} gegen] Geben o  $\cdot$ küniginne] kunnginnen m konige o \textbf{23} kam] Harte kam n \textbf{24} dâ] Do m n o \textbf{26} dâ] Do m n Das o  $\cdot$ was] sach er n \textbf{27} al umb] Vlvmmb o  $\cdot$ dâ] do m \textit{om.} n o \textbf{28} dâ] Do m n o  $\cdot$ kreftigiu] krefttiger m \textbf{30} burc] kneht o \newline
\end{minipage}
\end{table}
\newpage
\begin{table}[ht]
\begin{minipage}[t]{0.5\linewidth}
\small
\begin{center}*G
\end{center}
\begin{tabular}{rl}
 & sîn ellen \textbf{wære} sunder wanc.\\ 
 & dannen vuor \textbf{er} gein Zazamanc\\ 
 & in daz künicrîche.\\ 
 & die klageten \textbf{algelîche}\\ 
5 & Ysenharten, der den lîp\\ 
 & \textbf{in dienste} verlôs umbe ein wîp.\\ 
 & des twanc in Belacane,\\ 
 & \begin{large}D\end{large}iu süeze, valsches âne.\\ 
 & daz si im \textbf{niht} ir minne \textbf{bôt},\\ 
10 & des lag er nâch ir minnen tôt.\\ 
 & den râchen sîne mâge\\ 
 & offenlîche und an der lâge.\\ 
 & die vrouwen twungen si mit her.\\ 
 & diu was mit ellenthafter wer,\\ 
15 & dô Gahmuret kom in \textbf{ir} lant,\\ 
 & daz \textbf{ir} von Schotten Fridebrant\\ 
 & mit \textbf{schiffes} her verbrande,\\ 
 & ê \textbf{daz} er dannen wande.\\ 
 & nû hœret, wie unser rîter var!\\ 
20 & daz mer warf in mit sturme dar,\\ 
 & sô daz er kûme \textbf{iedoch} genas.\\ 
 & gein der küniginne palas\\ 
 & kom er gesigelt in die habe.\\ 
 & dâ wart \textbf{er} vil \textbf{geschouwet} abe.\\ 
25 & dô sach er ûz an daz velt.\\ 
 & dâ was geslagen manic \textbf{gezelt}\\ 
 & al umbe die stat \textbf{unze} \textbf{an daz} mer.\\ 
 & dâ \textbf{lac ein kreftigez} her.\\ 
 & dô hiez er vrâgen \textbf{der} mære,\\ 
30 & wes diu burc wære,\\ 
\end{tabular}
\scriptsize
\line(1,0){75} \newline
G O L M Q R W Z Fr29 Fr32 Fr36 \newline
\line(1,0){75} \newline
\textbf{1} \textit{Initiale} O M  \textbf{8} \textit{Initiale} G  \textbf{9} \textit{Versal} Fr32  \textbf{13} \textit{Initiale} O  \textbf{15} \textit{Initiale} L Q R Z Fr32  \textbf{19} \textit{Initiale} W  \textbf{21} \textit{Versal} Fr32  \newline
\line(1,0){75} \newline
\textbf{1} sîn] ÷in O Min M Sunder Q  $\cdot$ ellen] elle M ellen \textit{nachträglich korrigiert zu:} eren Q  $\cdot$ wære] strebt O (M) Z strepte L (Q) (R) (Fr29) (Fr32) strebe W  $\cdot$ sunder] sunden M \textbf{2} dannen] Van dan O (L) (Q) (R) (W) (Z) (Fr29) (Fr32) Noch danne M  $\cdot$ gein] zuͯ L  $\cdot$ Zazamanc] zazamanch G O L zazamamanc Q zasamant R zazamang W \textbf{4} die] Do W  $\cdot$ algelîche] in alle [ge*]: geliche O alle gliche M (Q) (R) (W) (Z) \textbf{5} Ysenharten] Jsnharten O Jsenharten L Z Jsenhartin M Eysenharten Q Jsenhartten R Ysenhart W :::ha::: Fr29 ẏsenharten Fr32 \textbf{6} in] An M  $\cdot$ umbe] im \textit{nachträglich korrigiert zu:} vm Q \textbf{7} twanc] betwanc M  $\cdot$ Belacane] Belacâne O Belecane L belatanck Q pellicane W delacane Z Be:::ane Fr29 \textbf{8} süeze] fraw W  $\cdot$ valsches] valsche Q  $\cdot$ âne] ane k* \textit{nachträglich korrigiert zu:} wanck Q eine R \textbf{9} daz] Do W  $\cdot$ si] \textit{om.} L R Fr32  $\cdot$ niht ir minne] ir minne ie O (L) (M) Z ir mynne nye Q (R) (W)  $\cdot$ bôt] gebot O L M Q R W Z (Fr32) \textbf{10} nâch] on R  $\cdot$ minnen] minne O (L) M (Q) R W Z Fr32 \textbf{11} râchen] Rechen R  $\cdot$ mâge] magen M \textbf{12} und] vnder M  $\cdot$ an der] ane L (W) eynander M  $\cdot$ lâge] tage O lagen M \textbf{13} die] ÷ie O  $\cdot$ twungen] \textit{om.} M  $\cdot$ mit] mir Fr32 \textbf{14} ellenthafter] allenhafftter R \textbf{15} dô] Da Z  $\cdot$ Gahmuret] Gamvret O Gahmuͯret L gamuret M (R) W Z gamuert Q gamvͦret Fr32 \textbf{16} daz] Do W  $\cdot$ Schotten] schoten G O schottin M scôten Fr32 :chotten Fr36  $\cdot$ Fridebrant] Frýdebrant L fridebant R v::: Fr36 \textbf{17} schiffes her] grozem her O (L) (Fr36) groszin her M grossem schiffe her Q grosem schiffes her R (Fr32) stiffes her Z \textbf{19} unser] my sie M \textbf{20} warf in] Jn warff R \textbf{21} kûme] kam Q  $\cdot$ iedoch] da L do W \textit{om.} Fr32 \textbf{22} küniginne] konnige M \textbf{23} in] an W \textbf{24} dâ] Do Q R W \textbf{25} dô] Da O L M Z  $\cdot$ er] \textit{om.} Fr32  $\cdot$ velt] [lant]: velt O \textbf{26} dâ] Do Q W  $\cdot$ gezelt] gezalt Q zelt R \textbf{27} unze an daz] bisz andas M wan bey dem Q vnd gen dem R wan gegn dem Fr32 \textbf{28} dâ lac] lagen Q Da lagent R (Fr29) (Fr32) Do lag W  $\cdot$ ein kreftigez] zwey krefftige Q (R) (Fr29) (Fr32) \textbf{29} \textit{Vers 16.29 fehlt} R   $\cdot$ dô] Da O M Z  $\cdot$ der] \textit{om.} L Q Fr32 \newline
\end{minipage}
\hspace{0.5cm}
\begin{minipage}[t]{0.5\linewidth}
\small
\begin{center}*T
\end{center}
\begin{tabular}{rl}
 & sîn ellen \textbf{strebete} sunder wanc.\\ 
 & \textbf{von} dannen vuor \textbf{er} gegen Zazamanc\\ 
 & in daz künecrîche.\\ 
 & die klageten \textbf{alglîche}\\ 
5 & Isenharten, der den lîp\\ 
 & verlôs umb ein wîp.\\ 
 & des twanc in Belacane,\\ 
 & diu süeze, valsches âne.\\ 
 & daz sim ir minne \textbf{nie} \textbf{gebôt},\\ 
10 & des lag er nâch ir minne tôt.\\ 
 & den râchen sîne mâge\\ 
 & offenlîche und an der lâge.\\ 
 & die vrouwen twungen si mit her.\\ 
 & diu was mit ellenthafter wer,\\ 
15 & Dô Gahmuret kom in \textbf{ir} lant,\\ 
 & daz \textbf{ir} von Schotten Fridebrant\\ 
 & mit \textbf{grôzem} her verbrande,\\ 
 & ê er \textbf{von} dannen wande.\\ 
 & \begin{large}N\end{large}û hœret, wie unser rîter var!\\ 
20 & daz mer warf in mit sturme dar,\\ 
 & sô daz er kûme genas.\\ 
 & gegen der küneginne palas\\ 
 & kom er gesigelt in die habe.\\ 
 & dâ wart  vil \textbf{geschouwet} abe.\\ 
25 & dô sach er ûz an daz velt.\\ 
 & dâ was geslagen manec \textbf{gezelt}\\ 
 & alumbe die stat \textbf{unz} \textbf{an daz} mer.\\ 
 & dâ \textbf{lac ein kreftigez} her.\\ 
 & Dô hiez er vrâgen mære,\\ 
30 & wes diu burc wære,\\ 
\end{tabular}
\scriptsize
\line(1,0){75} \newline
T U V \newline
\line(1,0){75} \newline
\textbf{15} \textit{Initiale} U V   $\cdot$ \textit{Majuskel} T  \textbf{19} \textit{Initiale} T  \textbf{29} \textit{Majuskel} T  \newline
\line(1,0){75} \newline
\textbf{1} ellen] ellende U \textbf{2} von dannen vuor er] Er vuͦr von dan U (V)  $\cdot$ Zazamanc] zazamang V \textbf{5} Isenharten] ysenharten T Jsenbarten U Jsenharten V \textbf{6} verlôs] Jn dinste verlos U (V) \textbf{7} twanc] betwanc U  $\cdot$ Belacane] Belacan ê U Belecane V \textbf{10} minne] minnen U \textbf{12} und] \textit{om.} U V \textbf{14} ellenthafter] ellentschafter U \textbf{15} Gahmuret] Gahmvret T Gamuret V \textbf{17} her] here er U \textbf{23} kom] Kuͦmet U \textbf{24} dâ] do V  $\cdot$ vil geschouwet] er vil beschowet U V \textbf{26} dâ] do V \textbf{27} unz] vͦz U \textbf{28} dâ] do V  $\cdot$ lac ein kreftigez] [lag*]: lagent zwei creftige V \textbf{29} mære] der mere V \textbf{30} Waz buͦrge daz were U \newline
\end{minipage}
\end{table}
\end{document}
