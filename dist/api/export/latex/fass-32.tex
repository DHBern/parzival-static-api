\documentclass[8pt,a4paper,notitlepage]{article}
\usepackage{fullpage}
\usepackage{ulem}
\usepackage{xltxtra}
\usepackage{datetime}
\renewcommand{\dateseparator}{.}
\dmyyyydate
\usepackage{fancyhdr}
\usepackage{ifthen}
\pagestyle{fancy}
\fancyhf{}
\renewcommand{\headrulewidth}{0pt}
\fancyfoot[L]{\ifthenelse{\value{page}=1}{\today, \currenttime{} Uhr}{}}
\begin{document}
\begin{table}[ht]
\begin{minipage}[t]{0.5\linewidth}
\small
\begin{center}*D
\end{center}
\begin{tabular}{rl}
\textbf{32} & disiu mære \textbf{sagt} im gar sîn wirt:\\ 
 & "ein ritter nimmer daz verbirt,\\ 
 & ern kom durch \textbf{tjostieren} vür.\\ 
 & ob der sîn dienest dort verlür\\ 
5 & an ir, diu in sante her?\\ 
 & waz hülfe \textbf{in} denne sîn vrechiu ger?\\ 
 & daz ist der stolze Hiuteger.\\ 
 & von dem mag ich wol sprechen mêr,\\ 
 & sît wir hie sîn besezzen,\\ 
10 & daz der helt vermezzen\\ 
 & \textbf{ie} \textbf{s}morgens vil bereite was\\ 
 & \textbf{vor} der porte \textbf{gein} dem palas.\\ 
 & ouch \textbf{ist} von dem küenen man\\ 
 & kleinœtes vil gevuoret dan,\\ 
15 & daz er durch unser schilte stach,\\ 
 & des man vür \textbf{grôze} koste jach,\\ 
 & \textbf{sô ez} die krîgierre brâchen drab.\\ 
 & er valt uns manegen rîter ab.\\ 
 & er læt sich gerne schouwen.\\ 
20 & in lobent ouch unser vrouwen.\\ 
 & swen wîp lobent, der wirt \textbf{erkant}.\\ 
 & \textbf{er} hât den prîs \textbf{ze} sîner hant\\ 
 & unt sînes herzen wunne."\\ 
 & \textbf{dô} hete diu müede sunne\\ 
25 & ir liehten blic \textbf{hin} zir gelesen.\\ 
 & \textbf{des} bankens muose ein ende wesen.\\ 
 & der gast mit sîme wirte reit.\\ 
 & \textbf{er} vant sîn ezzen al bereit.\\ 
 & \textit{ic}h muoz iu von ir spîse sagen,\\ 
30 & diu wart mit zühten vür getragen.\\ 
\end{tabular}
\scriptsize
\line(1,0){75} \newline
D \newline
\line(1,0){75} \newline
\newline
\line(1,0){75} \newline
\textbf{7} Hiuteger] Hvteger D \textbf{9} sîn] sin \sout{hie} D \textbf{29} ich] oh D \newline
\end{minipage}
\hspace{0.5cm}
\begin{minipage}[t]{0.5\linewidth}
\small
\begin{center}*m
\end{center}
\begin{tabular}{rl}
 & disiu mære \textbf{sagete} ime gar sîn wirt:\\ 
 & "ein ritter nieme\textit{r} \textit{d}az verbirt,\\ 
 & er enkome durch \textbf{justieren} vür.\\ 
 & ob der sînen dienst dort verlür\\ 
5 & an ir, diu in sante her?\\ 
 & \dag wer hie sîn danne sîner frechttiger\dag \\ 
 & daz ist der stolze \textit{Hu}teger.\\ 
 & von dem mac ich wol sprechen mêr,\\ 
 & sît wir hie sîn bese\textit{zz}en,\\ 
10 & daz der helt verme\textit{zz}en\\ 
 & \textbf{ie} morgens vil bereite was\\ 
 & \textbf{vor} der porte \textbf{gegen} dem palas.\\ 
 & ouch \textbf{ist} vo\textit{n} dem küenen man\\ 
 & kleinœtes vil gevuoret dan,\\ 
15 & daz er durch unser schilte stach,\\ 
 & des man v\textit{ür} \textbf{grôze} \textit{k}oste jach,\\ 
 & \textbf{sô ez} die \textit{kroijie}re brâchen drabe.\\ 
 & er valte uns manigen ritter abe.\\ 
 & er lât sich gerne schouwen.\\ 
20 & in lobent ouch unser vrouwen.\\ 
 & wen wîp lobent, der wirt \textbf{erkant}.\\ 
 & \textbf{er} hât den prîs \textbf{in} sîner hant\\ 
 & und sînes herzen wunne."\\ 
 & \textbf{dô} hete diu müede sunne\\ 
25 & ir liehten blic \textbf{hin} zuo ir gelesen.\\ 
 & \textbf{des} banikens muose ein ende wesen.\\ 
 & der gast mit sînem wirte reit,\\ 
 & \textbf{der} vant sîn ezzen al bereit.\\ 
 & \textit{\begin{large}I\end{large}}ch muoz iu von ir sp\textit{î}se sagen,\\ 
30 & diu wart mit zühten vür getragen.\\ 
\end{tabular}
\scriptsize
\line(1,0){75} \newline
m n o W \newline
\line(1,0){75} \newline
\textbf{1} \textit{Initiale} W  \textbf{29} \textit{Überschrift:} Also gamiret vnd sin volck gar herlich enpfangen wurden von der kv́nnigin vnd zuͯ disch sossen vnd in gar herliche kost fúr wart getragen n   $\cdot$ \textit{Großinitiale} n   $\cdot$ \textit{Initiale} m  \newline
\line(1,0){75} \newline
\textbf{1} sagete] [seiget]: saget n saht o (W) \textbf{2} niemer daz] niemen bas m \textbf{3} enkome] [enkaͯme]: enkome m kumme n (o) W \textbf{4} verlür] verlor o \textbf{6} Wer hie siner frecheit beger n o (W)  $\cdot$ sîn] sin \textit{nachträglich korrigiert zu:} szin m  $\cdot$ sîner] siner \textit{nachträglich korrigiert zu:} sin der m \textbf{7} \textit{Versfolge 32.8-7} W   $\cdot$ Huteger] mitteger m nẏtiger n nyttiger o hútiger W \textbf{8} mac] so mag n \textbf{9} besezzen] bese:en \textit{nachträglich korrigiert zu:} beseszen m \textbf{10} vermezzen] ver me:en \textit{nachträglich korrigiert zu:} ver meszen m \textbf{12} gegen] fúr n (o) W  $\cdot$ dem] den W \textbf{13} von] vor m o \textbf{14} dan] an o \textbf{15} er] der o \textbf{16} des] Das W  $\cdot$ vür] vil m  $\cdot$ koste] scoste m \textbf{17} kroijiere] schirre \textit{nachträglich korrigiert zu:} schich m schiere n (o) \textbf{18} valte] vellet W \textbf{19} sich] sich sich o \textbf{21} wîp] die weib W \textbf{22} in] zuͦ n o W \textbf{24} sunne] sinne o \textbf{25} hin] \textit{om.} n o W \textbf{26} banikens] banicken n (o) (W)  $\cdot$ muose] muͦsz n (o) (W) \textbf{27} sînem] [sẏnen g]: sẏnen wuͯrte o \textbf{28} der] Er n o W  $\cdot$ al] alle n \textbf{29} Ich] ÷Cch m [Ch]: JCh o  $\cdot$ ir] \textit{om.} W  $\cdot$ spîse] spsse \textit{nachträglich korrigiert zu:} spisse m \textbf{30} zühten vür] zucht fúr in W \newline
\end{minipage}
\end{table}
\newpage
\begin{table}[ht]
\begin{minipage}[t]{0.5\linewidth}
\small
\begin{center}*G
\end{center}
\begin{tabular}{rl}
 & disiu mære \textbf{seit} im gar sîn wirt:\\ 
 & "ein rîter nimer daz verbirt,\\ 
 & er enkom \textbf{hie} durch \textbf{tjoste} vür.\\ 
 & op der sîn dienst dort verlür\\ 
5 & an ir, diu in sande her?\\ 
 & waz hülfe \textbf{in} dane sîn vrechiu ger?\\ 
 & daz ist der stolze Huteger.\\ 
 & von dem mag ich wol sprechen mêr,\\ 
 & sît wir hie sîn besezzen,\\ 
10 & daz \textbf{ie} der helt vermezzen\\ 
 & \textbf{des} morgens vil bereit was\\ 
 & \textbf{gein} der porte \textbf{vor} dem palas.\\ 
 & ouch \textbf{wart} von dem küenen man\\ 
 & kleinœdes vil gevuoret dan,\\ 
15 & daz er durch unser schilte stach,\\ 
 & des man vür \textbf{grôze} koste jach,\\ 
 & \textbf{swene ez} die kroijierære brâchen drabe.\\ 
 & er valt uns manigen rîter abe.\\ 
 & er lât sich gerne schouwen.\\ 
20 & in lobent ouch unser vrouwen.\\ 
 & swen wîp lobent, der wirt \textbf{bekant},\\ 
 & \textbf{der} hât den brîs \textbf{ze} sîner hant\\ 
 & unde sînes herzen wunne."\\ 
 & \textbf{nû} hete diu müede sunne\\ 
25 & ir liehten blic \textbf{hin}ze ir gelesen.\\ 
 & \textbf{des} bankenes muose ein ende wesen.\\ 
 & der gast mit sînem wirte reit.\\ 
 & \textbf{er} vant sîn ezzen albereit.\\ 
 & ich muoz iu von ir spîse sagen,\\ 
30 & diu wart mit zühten vür getragen.\\ 
\end{tabular}
\scriptsize
\line(1,0){75} \newline
G O L M Q R Z Fr32 \newline
\line(1,0){75} \newline
\textbf{1} \textit{Initiale} M  \textbf{11} \textit{Versal} Fr32  \textbf{27} \textit{Initiale} O  \textbf{29} \textit{Initiale} L Q Z Fr32  \newline
\line(1,0){75} \newline
\textbf{1} gar] \textit{om.} L M \textbf{2} ein] Der M  $\cdot$ nimer daz] niemer da L das nummer M (Q) \textbf{3} enkom] kome L (M) (R)  $\cdot$ hie] \textit{om.} O L M Q R Z  $\cdot$ tjoste] tyostiern O (L) (M) (Q) (R) (Z) \textbf{4} der] er L  $\cdot$ sîn] sinen L Z  $\cdot$ dort] ich Q  $\cdot$ verlür] vorlor M \textbf{5} diu] die da R \textbf{6} hülfe] hilff Q  $\cdot$ sîn] \textit{om.} O M  $\cdot$ vrechiu] freche R \textbf{7} \textit{Versfolge 32.8-7} Q   $\cdot$ Huteger] Hvͤteger O hvttegere L Nuteger M huitiger R Hivteger Fr32 \textbf{8} dem] den M dē Q  $\cdot$ mag] [manch]: mach Q \textbf{9} hie] \textit{om.} Q \textbf{10} ie] \textit{om.} L Z \textbf{11} des] Je des Z  $\cdot$ vil] ye Q  $\cdot$ was] \textit{om.} O \textbf{12} gein] Vor O L M Q R Z (Fr32)  $\cdot$ der porte] den phortin M der pforten Q (Z)  $\cdot$ vor] gein O L M (Q) (R) Z (Fr32)  $\cdot$ palas] palas reit O \textbf{13} wart] ist Z  $\cdot$ dem] den R  $\cdot$ küenen] kúne Q \textbf{14} dan] [dar]: dan R \textbf{16} des] Das R  $\cdot$ vür] vil M  $\cdot$ grôze] groziv O \textbf{17} swene] So O L R Z Do M Q (Fr32)  $\cdot$ kroijierære] grogiere O kaphare L kirre Z  $\cdot$ brâchen] brachtent R  $\cdot$ drabe] abe O \textbf{18} er] [Es]: Er Q  $\cdot$ rîter] rittern M \textbf{21} swen] Wen L (Q) (R)  $\cdot$ wîp lobent] lobent wip Z  $\cdot$ bekant] erkant L \textbf{22} der] Er O L M Q R Z (Fr32)  $\cdot$ ze] in Q \textbf{24} nû] Da Z  $\cdot$ sunne] sunde Q \textbf{25} ir] Ern M (Q)  $\cdot$ liehten] lýchten L (M) (Q)  $\cdot$ blic] bliche Z  $\cdot$ gelesen] gelaszen R \textbf{26} \textit{Die Verse 32.27-28 fehlen} R   $\cdot$ bankenes] banichen O (Z) buches M dachs Q banches Fr32 \textbf{27} der] ÷er O \textbf{28} er] Der Z  $\cdot$ albereit] alle bereit M \textbf{29} muoz] wil O L Q R Fr32  $\cdot$ spîse] spile Z \textbf{30} getragen] guragin M \newline
\end{minipage}
\hspace{0.5cm}
\begin{minipage}[t]{0.5\linewidth}
\small
\begin{center}*T
\end{center}
\begin{tabular}{rl}
 & Dis\textit{iu} mære \textbf{seit} im gar sîn wirt:\\ 
 & "ein rîter niemer daz verbirt,\\ 
 & ern kome durch \textbf{tjostieren} vür.\\ 
 & ob der sînen dienst dort verlür\\ 
5 & an ir, diu in \textbf{dâ} sante her?\\ 
 & waz hülfe danne sîn vrech\textit{iu} ger?\\ 
 & daz ist der stolze Huteger.\\ 
 & von dem mag ich wol sprechen mêr,\\ 
 & sît wir hie sîn besezzen,\\ 
10 & daz \textbf{ie} der helt vermezzen\\ 
 & \textbf{des} morgens vil bereit was\\ 
 & \textbf{vor} der porten \textbf{gegen} dem palas.\\ 
 & ouch \textbf{wart} von dem küenen man\\ 
 & kleinœdes vil gevuoret dan,\\ 
15 & daz er durch unser schilte stach,\\ 
 & des man vür \textbf{rîche} koste jach,\\ 
 & \textbf{swaz} die crâierer brâchen drabe.\\ 
 & er valte uns manegen rîter abe.\\ 
 & er lât sich gerne schouwen.\\ 
20 & in lobent ouch unser vrouwen.\\ 
 & swen wîp lobent, der wirt \textbf{erkant},\\ 
 & \textbf{der} hât den prîs \textbf{ze} sîner hant\\ 
 & und sînes herzen wunne."\\ 
 & \textbf{Nû} hete diu müede sunne\\ 
25 & ir liehten blic zir gelesen.\\ 
 & \textbf{ir} banekens muose ein ende wesen.\\ 
 & \begin{large}D\end{large}er gast mit sînem wirte reit,\\ 
 & \textbf{der} vant sîn ezzen al bereit.\\ 
 & Ich muoz iu von ir spîse sagen,\\ 
30 & diu wart mit zühten vür getragen.\\ 
\end{tabular}
\scriptsize
\line(1,0){75} \newline
T U V \newline
\line(1,0){75} \newline
\textbf{1} \textit{Majuskel} T  \textbf{24} \textit{Majuskel} T  \textbf{27} \textit{Initiale} T  \textbf{29} \textit{Initiale} U V   $\cdot$ \textit{Majuskel} T  \newline
\line(1,0){75} \newline
\textbf{1} Disiu] Dise T \textbf{2} daz] do U \textbf{3} ern] Er U V \textbf{4} sîn] sinen V \textbf{5} dâ] do U V \textbf{6} hülfe] helfe U  $\cdot$ vrechiu] vreche T \textbf{7} Huteger] Huͦteger U Hútinger V \textbf{12} der] [den]: der V \textbf{14} dan] von dan U V \textbf{17} swaz] Waz U [S*s]: So es V  $\cdot$ crâierer brâchen] craiere brache U [kroiere]: kroierer brachent V \textbf{19} sich gerne] gerne sie U \textbf{21} swen] Wen U \textbf{26} ir] Des U (V) \textbf{28} al] al do U \newline
\end{minipage}
\end{table}
\end{document}
