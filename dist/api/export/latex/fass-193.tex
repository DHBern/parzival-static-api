\documentclass[8pt,a4paper,notitlepage]{article}
\usepackage{fullpage}
\usepackage{ulem}
\usepackage{xltxtra}
\usepackage{datetime}
\renewcommand{\dateseparator}{.}
\dmyyyydate
\usepackage{fancyhdr}
\usepackage{ifthen}
\pagestyle{fancy}
\fancyhf{}
\renewcommand{\headrulewidth}{0pt}
\fancyfoot[L]{\ifthenelse{\value{page}=1}{\today, \currenttime{} Uhr}{}}
\begin{document}
\begin{table}[ht]
\begin{minipage}[t]{0.5\linewidth}
\small
\begin{center}*D
\end{center}
\begin{tabular}{rl}
\textbf{193} & Ûffen teppech kniete si vür in.\\ 
 & si heten beidiu kranken sin,\\ 
 & er unt diu küneginne,\\ 
 & an \textbf{bî ligender} minne.\\ 
5 & \textbf{Hie} wart alsus geworben:\\ 
 & an vreuden verdorben\\ 
 & was diu magt. des twanc si schem.\\ 
 & ob er si hin an iht nem?\\ 
 & \textbf{leider} des enkan er niht.\\ 
10 & âne kunst ez doch geschiht\\ 
 & \begin{large}M\end{large}it einem \textbf{alsô} \textbf{bewandem} vride,\\ 
 & daz si diu süenebæren lide\\ 
 & \textbf{niht} \textbf{zuo ein ander} brâhten.\\ 
 & wênec si des gedâhten.\\ 
15 & der meide jâmer was sô grôz,\\ 
 & vil zeher von ir ougen vlôz\\ 
 & ûf den jungen Parzival.\\ 
 & der \textbf{erhôrte} ir \textbf{weinens} \textbf{sölhen} schal,\\ 
 & daz er si \textbf{wachende} an \textbf{gesach}.\\ 
20 & leide unt liebe im dran geschach.\\ 
 & Ûf rihte sich der junge man.\\ 
 & zer küneginne sprach er sân:\\ 
 & "vrouwe, bin ich iwer spot?\\ 
 & ir soldet knien \textbf{alsus} \textbf{vür} got.\\ 
25 & geruochet sitzen zuo mir her",\\ 
 & daz \textbf{was} \textbf{sîn} bete unt \textbf{sîn} ger,\\ 
 & "oder leit iuch hie, \textbf{aldâ ich} lac.\\ 
 & lât mich belîben, swâ ich mac."\\ 
 & Si sprach: "welt ir iuch êren,\\ 
30 & sölhe mâze gein \textbf{mir} kêren,\\ 
\end{tabular}
\scriptsize
\line(1,0){75} \newline
D \newline
\line(1,0){75} \newline
\textbf{1} \textit{Majuskel} D  \textbf{5} \textit{Majuskel} D  \textbf{11} \textit{Initiale} D  \textbf{21} \textit{Majuskel} D  \textbf{29} \textit{Majuskel} D  \newline
\line(1,0){75} \newline
\newline
\end{minipage}
\hspace{0.5cm}
\begin{minipage}[t]{0.5\linewidth}
\small
\begin{center}*m
\end{center}
\begin{tabular}{rl}
 & ûf den teppich \textit{k}n\textit{ie}wete si vür in.\\ 
 & si heten beidiu kranken sin,\\ 
 & er und diu küniginne,\\ 
 & an \textbf{beligen der} minne.\\ 
5 & \textbf{hie} wart alsus geworben:\\ 
 & an vröuden verdorben\\ 
 & was diu maget. des twanc si scheme.\\ 
 & ob er si hi\textit{n} \textit{a}n iht neme?\\ 
 & \textbf{leider} des enkan er niht.\\ 
10 & âne kunst ez doch geschiht\\ 
 & mit einem \textbf{alsô} \textbf{bewanden} vride,\\ 
 & daz si d\textit{iu} süenebære\textit{n} \textit{l}ide\\ 
 & \textbf{niht} \textbf{ze ei\textit{n} ander} brâhten.\\ 
 & wênic si des gedâhten.\\ 
15 & \begin{large}D\end{large}er megede jâmer was sô grôz,\\ 
 & vil zehere von ir ougen vlôz\\ 
 & ûf den jungen Parcifal.\\ 
 & der \textbf{erhôrte} ir \textbf{weinens} \textbf{solhen} schal,\\ 
 & daz er si \textbf{wachende} an \textbf{gesach}.\\ 
20 & leide und liebe ime dran geschach.\\ 
 & ûf rihte sich der junge man.\\ 
 & zer küniginne sprach er sân:\\ 
 & "vrouwe, bin ich iuwer spot?\\ 
 & ir soltet kniewen \textbf{alsus} \textbf{vür} got.\\ 
25 & geruochet sitzen zuo mir her,\\ 
 & daz \textbf{ist} \textbf{mîn} bete und \textbf{mîn} ger,\\ 
 & oder leget iuch hie, \textbf{d\textit{â} ich dô} lac.\\ 
 & lât mich belîben, w\textit{â} ich mac."\\ 
 & si sprach: "welt ir iuch êren,\\ 
30 & solich m\textit{â}ze gegen \textbf{mir} kêren,\\ 
\end{tabular}
\scriptsize
\line(1,0){75} \newline
m n o Fr69 \newline
\line(1,0){75} \newline
\textbf{15} \textit{Initiale} m o Fr69   $\cdot$ \textit{Capitulumzeichen} n  \newline
\line(1,0){75} \newline
\textbf{1} kniewete] kunverwette m knuwet n o \textbf{4} beligen der] by ligender n (o) \textbf{8} hin an] hin in an m  $\cdot$ iht] sich n o \textbf{10} geschiht] beschicht n verdirpt o \textbf{11} bewanden] bewunden n o \textbf{12} diu] des m [die]: dire o  $\cdot$ süenebæren lide] súneberen d lide m sune bernde lide n sunebern ligen o \textbf{13} niht] Mich o  $\cdot$ ein ander] [eine*]: eineander m \textbf{14} gedâhten] gedacten Fr69 \textbf{16} von] was so [*]: von n \textbf{18} erhôrte] hort n o  $\cdot$ weinens] weinen n o \textbf{19} er] \textit{om.} o \textbf{20} leide und liebe] Liep vnd leit n o  $\cdot$ dran] do n o  $\cdot$ geschach] beschach n o \textbf{21} rihte] richt n o \textbf{24} soltet] sollent o  $\cdot$ alsus] also n (o)  $\cdot$ vür] vor o \textbf{25} her] \textit{om.} n \textbf{27} leget] loget o  $\cdot$ dâ] do m n o  $\cdot$ dô] \textit{om.} n o \textbf{28} belîben] bliben bliben n  $\cdot$ wâ] wog m \textbf{30} mâze] messe m \newline
\end{minipage}
\end{table}
\newpage
\begin{table}[ht]
\begin{minipage}[t]{0.5\linewidth}
\small
\begin{center}*G
\end{center}
\begin{tabular}{rl}
 & ûf den tepe\textit{ch} \textit{knie}te si vür in.\\ 
 & si heten bêdiu kranken sin,\\ 
 & \begin{large}E\end{large}r unde diu küniginne,\\ 
 & an \textbf{bî ligen\textit{der}} minne.\\ 
5 & \textbf{hie} wart alsus geworben:\\ 
 & an vröuden verd\textit{o}rben\\ 
 & was diu maget. des twanc si sch\textit{em}e.\\ 
 & ober si hin an iht neme?\\ 
 & \textbf{leider} des enkan er niht.\\ 
10 & âne kunst ez doch geschiht\\ 
 & mit einem \textbf{sô} \textbf{benanten} vride,\\ 
 & daz si diu süenebæren lide\\ 
 & \textbf{ninder} \textbf{zein ander} brâhten.\\ 
 & \textbf{wie} wênic si des gedâhten!\\ 
15 & der magede jâmer was sô grôz,\\ 
 & vil zahere von ir ougen vlôz\\ 
 & ûf den ju\textit{n}gen Parzival.\\ 
 & der \textbf{hôrte} ir \textbf{weinen}, \textbf{solhen} schal,\\ 
 & daz er si \textbf{lachende} an \textbf{sach}.\\ 
20 & leit unde liep im dran geschach.\\ 
 & ûf rihte sich der junge man.\\ 
 & zer küniginne sprach er sân:\\ 
 & "vrouwe, bin ich iwer spot?\\ 
 & ir solt knien \textbf{sus} \textbf{vor} got.\\ 
25 & geruochet sitzen zuo mir her",\\ 
 & daz \textit{\textbf{was}} \textbf{sîn} bet unde \textbf{ouch} \textbf{sîn} ger,\\ 
 & "oder leit iuch hie, \textbf{al dâ ich} lac.\\ 
 & lât mich belîben, swâ ich mac."\\ 
 & si sprach: "wolt ir iuch êren,\\ 
30 & solhe mâze gein \textbf{mir} kêren,\\ 
\end{tabular}
\scriptsize
\line(1,0){75} \newline
G I O L M Q R Z \newline
\line(1,0){75} \newline
\textbf{3} \textit{Initiale} G  \textbf{5} \textit{Initiale} I O L  \textbf{15} \textit{Capitulumzeichen} L  \textbf{19} \textit{Initiale} Z  \textbf{29} \textit{Initiale} I M Q  \newline
\line(1,0){75} \newline
\textbf{1} tepech] tepe:: G  $\cdot$ kniete] :::te G chniet I O (Q) (Z)  $\cdot$ si] si nider O \textbf{4} ligender] ligen::: G ligender libe vnd M \textbf{5} hie] ÷ie O Die L \textbf{6} verdorben] verd:rben G \textbf{7} scheme] sch::e G scham L R scheyn M (Q) \textbf{8} si] siv O  $\cdot$ iht neme] icht nein M mich nem Q \textbf{11} sô] also M Z  $\cdot$ benanten] bewarrem O gewanten L bewanten M (R) bewantem Q Z \textbf{12} süenebæren] sunnaberin M sunderbaren R  $\cdot$ lide] site M \textbf{13} ninder] nih I (O) (L) (M) (Q) (R) (Z)  $\cdot$ zein ander] zesamen O (R) zu sampte Q \textbf{14} gedâhten] gedachte Q \textbf{16} zahere] zerer Q  $\cdot$ von] vsz R \textbf{17} jungen] ivgen G shoͤnen I  $\cdot$ Parzival] parzifal I Parcifal O (L) (Z) partzifal Q parczifal R \textbf{18} weinen] wainens I (O) (L) (Q) weynes M (Z) \textbf{19} lachende] wachende L M Z on lachent R  $\cdot$ sach] gesach I Z \textbf{20} Ditze het er fvr einen (ein Q R ) vngemach O (Q) (R)  $\cdot$ leit unde liep] Liep vnd lait L \textbf{21} junge] iunge suͤzzer I werde O \textbf{22} er] ir M \textbf{23} bin ich] ich bin O \textbf{24} solt] soldet I (O) (Q) (Z) soͯltten R  $\cdot$ knien sus] keinen fvs L keinen sust R  $\cdot$ vor] vuͤr I (O) (Q) (R) (Z) \textbf{25} sitzen zuo mir] sitzet zu mir Q zu mir siczen R \textbf{26} was] \textit{om.} G M  $\cdot$ bet] red R  $\cdot$ ouch] \textit{om.} L Q R  $\cdot$ ger] beger Q \textbf{27} lac] [h*]: hac I \textbf{28} swâ] alda O wo L Q (R) \textbf{29} iuch] mich I O \textbf{30} gein] zu R \newline
\end{minipage}
\hspace{0.5cm}
\begin{minipage}[t]{0.5\linewidth}
\small
\begin{center}*T
\end{center}
\begin{tabular}{rl}
 & ûffen teppich kniete si vür in.\\ 
 & si heten beid\textit{iu} kranken sin,\\ 
 & er unde di\textit{u} küneginne.\\ 
 & an \textbf{bî ligender} minne\\ 
5 & wart alsus geworben:\\ 
 & an vröuden verdorben\\ 
 & \textit{w}a\textit{s} diu maget. des twanc si scheme.\\ 
 & ob er si hin an \textbf{sich} iht neme?\\ 
 & \textbf{Nein}, des enkan er niht!\\ 
10 & âne kunst ez doch geschiht\\ 
 & mit eime \textbf{sô} \textbf{gewande\textit{n}} vride,\\ 
 & daz si di\textit{u} süenebæren lide\\ 
 & \textbf{niender} \textbf{zesamene} brâhten.\\ 
 & wênec si des gedâhten.\\ 
15 & \begin{large}D\end{large}er megde jâmer was sô grôz,\\ 
 & \textbf{daz} vil zehere von ir ougen vlôz\\ 
 & ûf den jungen Parcifal.\\ 
 & der \textbf{erhôrte} ir \textbf{weinens} schal,\\ 
 & daz er si \textbf{wachende} ane \textbf{sach}.\\ 
20 & leit unde liep im dran geschach.\\ 
 & ûf rihte sich der junge man.\\ 
 & zer küneginne sprach er sân:\\ 
 & "vrouwe, bin ich iuwer spot?\\ 
 & ir soltet knien \textbf{sus} \textbf{vür} got.\\ 
25 & geruochet sitzen zuo mir her",\\ 
 & daz \textbf{was} \textbf{sîn} bete unde \textbf{ouch} \textbf{sîn} ger,\\ 
 & "oder leget iuch hie, \textbf{dâ ich} lac.\\ 
 & lât mich blîben, swâ ich mac."\\ 
 & Si sprach: "welt ir iuch êren,\\ 
30 & solhe mâze gegen \textbf{iu} kêren,\\ 
\end{tabular}
\scriptsize
\line(1,0){75} \newline
T U V W \newline
\line(1,0){75} \newline
\textbf{9} \textit{Majuskel} T  \textbf{15} \textit{Initiale} T V W  \textbf{29} \textit{Majuskel} T  \newline
\line(1,0){75} \newline
\textbf{1} ûffen] Vffenz V \textbf{2} beidiu] beide T bade U \textbf{3} diu] die T \textbf{4} bî ligender] rechter werder suͤsser W \textbf{5} wart alsus] Hie wart alsvs V Alsus ward W \textbf{6} verdorben] gar verdorben W \textbf{7} was] swaz T \textbf{8} iht] [n*]: iht V \textit{om.} W \textbf{9} Owe nein des kan er nicht W \textbf{10} geschiht] beschiht V \textbf{11} gewanden] gewante T gewantem W \textbf{12} diu] die T \textbf{13} niender] Nider U  $\cdot$ zesamene] zuͦ einander W \textbf{14} wênec] Vil wening V \textbf{16} zehere] zehern U trehene V \textbf{17} Parcifal] parzifal V partzifal W \textbf{18} der] Da V  $\cdot$ ir weinens] ir weines U ir weinens solichen V irs wainens soͤlchen W \textbf{19} Das er entwachte vnd an sy sach W  $\cdot$ wachende] [*achende]: lachende V \textbf{20} Do er lag an seines schlafes gemach W \textbf{21} rihte] rachte U \textbf{25} sitzen zuo mir] zuͦ mir sitzen W \textbf{26} ouch] \textit{om.} W \textbf{27} iuch] îv T  $\cdot$ hie dâ] her al da U hie do V her do W  $\cdot$ lac] [*]: do lag V nun lag W \textbf{28} swâ] al da U wo W \textbf{29} iuch] îv T [*]: v́ch V eúwern W \textbf{30} iu] ir U mir V W \newline
\end{minipage}
\end{table}
\end{document}
