\documentclass[8pt,a4paper,notitlepage]{article}
\usepackage{fullpage}
\usepackage{ulem}
\usepackage{xltxtra}
\usepackage{datetime}
\renewcommand{\dateseparator}{.}
\dmyyyydate
\usepackage{fancyhdr}
\usepackage{ifthen}
\pagestyle{fancy}
\fancyhf{}
\renewcommand{\headrulewidth}{0pt}
\fancyfoot[L]{\ifthenelse{\value{page}=1}{\today, \currenttime{} Uhr}{}}
\begin{document}
\begin{table}[ht]
\begin{minipage}[t]{0.5\linewidth}
\small
\begin{center}*D
\end{center}
\begin{tabular}{rl}
\textbf{212} & unt vedern würfe in den wint.\\ 
 & dennoch was Gahmuretes kint\\ 
 & ninder müede an decheinem lide.\\ 
 & dô wânde Clamide, \textbf{daz} der vride\\ 
5 & wære \textbf{gebrochen} \textbf{ûz} der stat.\\ 
 & sînen kampfgenôz er bat,\\ 
 & daz er sich selben êrte\\ 
 & unt \textbf{mangen würfe} werte.\\ 
 & ez giengen ûf in slege grôz.\\ 
10 & die wâren \textbf{wol} \textbf{mangen} \textbf{steines} genôz.\\ 
 & Sus antwurte im des landes wirt:\\ 
 & "ich wæne, dich \textbf{mangen} wurf verbirt.\\ 
 & wan dâ vür ist mîn triwe pfant.\\ 
 & hetest êt vride von mîner hant,\\ 
15 & dir \textbf{enbræche} manegen swenkel\\ 
 & brust, houbet noch den schenkel."\\ 
 & Clamide dranc müede zuo.\\ 
 & diu was im dennoch \textbf{gar} ze vruo.\\ 
 & \begin{large}S\end{large}ig gewunnen, sig verlorn\\ 
20 & wart sunder dâ mit strîte erkorn.\\ 
 & doch wart der künec Clamide\\ 
 & \textbf{an} schumpfentiwer \textbf{beschouwet} ê\\ 
 & \textbf{mit} eime niderzucke.\\ 
 & von Parzivales drucke\\ 
25 & \textbf{bluot wæte} ûz ôren unt \textbf{ûz} \textbf{der} nasen.\\ 
 & daz machete rôt den grüenen wasen.\\ 
 & er \textbf{enblôzte} imz houbet schiere\\ 
 & \textbf{von} helme unt von herseniere.\\ 
 & gein \textbf{slage} saz der betwungen lîp.\\ 
30 & der sigehafte sprach: "mîn wîp\\ 
\end{tabular}
\scriptsize
\line(1,0){75} \newline
D \newline
\line(1,0){75} \newline
\textbf{11} \textit{Majuskel} D  \textbf{19} \textit{Initiale} D  \newline
\line(1,0){75} \newline
\textbf{2} Gahmuretes] Gahmvretes D \textbf{4} Clamide] Chlamide D \textbf{17} Clamide] Chlamide D \textbf{21} Clamide] Chlamide D \newline
\end{minipage}
\hspace{0.5cm}
\begin{minipage}[t]{0.5\linewidth}
\small
\begin{center}*m
\end{center}
\begin{tabular}{rl}
 & und vederen würfe in den wint.\\ 
 & dannoch was Gahmuretes kint\\ 
 & niender müede an keinem lide.\\ 
 & dô w\textit{â}nde Clamide, \textbf{daz} der vride\\ 
5 & wære \textbf{ge\textit{b}roch\textit{en}} \textbf{ûz} der stat.\\ 
 & sînen kampfgenôz er bat,\\ 
 & daz er sich selben êrte\\ 
 & und \textbf{manigen wurf} werte.\\ 
 & ez giengen ûf in slege grôz.\\ 
10 & die wâren \textbf{wol} \textbf{ma\textit{n}gen} \textbf{stein} genôz.\\ 
 & sus antwurt ime des landes wirt:\\ 
 & "ich wæne, dich \textbf{mangen} wurf verbirt.\\ 
 & wanne dâ vür ist mîn triuwe pfant.\\ 
 & hetest oht vride von mîner hant,\\ 
15 & dir \textbf{enbræchen} manigen swenkel\\ 
 & brust, houbet noch den schenkel."\\ 
 & Clamide dranc müede zuo.\\ 
 & diu was im dannoch \textbf{al} ze vruo.\\ 
 & si\textit{ge} gewunnen, sige verlorn\\ 
20 & wart sunder d\textit{â} mite strîte erkorn.\\ 
 & doch wart der künic Clamide\\ 
 & \textbf{an} sch\textit{um}pfentiure \textbf{beschouwet} ê\\ 
 & \textbf{mit} einem niderzucke.\\ 
 & von Parcifales drucke\\ 
25 & \textbf{wæte ime bluot} ûz \textit{ôr}en und \textit{nas}en.\\ 
 & daz maht rô\textit{t} den grüenen wasen.\\ 
 & er \textbf{blôzete} ime daz houbet schiere\\ 
 & \textbf{von} helme und von hersniere.\\ 
 & gegen \textbf{slage} saz der betwungene lîp.\\ 
30 & der sigehafte sprach: "mîn wîp\\ 
\end{tabular}
\scriptsize
\line(1,0){75} \newline
m n o Fr69 \newline
\line(1,0){75} \newline
\newline
\line(1,0){75} \newline
\textbf{1} vederen] veder Fr69  $\cdot$ wint] win n \textbf{2} Gahmuretes] gahmurettes m gahmúretes n gamarutez o \textbf{3} niender] Niergent n  $\cdot$ keinem] sinem n (o) \textbf{4} wânde] wunde m \textbf{5} wære gebrochen] Were gesproch m Gebrochen were Fr69 \textbf{6} bat] do bat n o \textbf{9} slege] der slege n \textbf{10} mangen] manigen m der n o \textbf{12} mangen] manigen n o \textbf{14} hetest oht] Hettest du n (o) \textbf{15} dir] Die o  $\cdot$ enbræchen] enbreche n \textbf{18} diu was] Die wasz die wasz o \textbf{19} Sú gewunnen sú verluren n  $\cdot$ sige gewunnen] Sigewunen m Sẏge gewinnen o \textbf{20} dâ] do m n o  $\cdot$ erkorn] erkuren n \textbf{22} schumpfentiure] scenfentvre m sconfertuͯr o \textbf{24} Parcifales] parcifals n \textbf{25} wæte] Wuͯte n (o)  $\cdot$ ôren und nasen] nasen vnd oren m \textbf{26} rôt] ros m  $\cdot$ grüenen] [guͯnen]: gruͯnen o \textbf{27} blôzete] enbloͯsset n (o) (Fr69)  $\cdot$ ime daz] [in]: imz Fr69  $\cdot$ schiere] schúr n \textbf{28} hersniere] hersemir n horsemir o \textbf{29} der] daz Fr69 \newline
\end{minipage}
\end{table}
\newpage
\begin{table}[ht]
\begin{minipage}[t]{0.5\linewidth}
\small
\begin{center}*G
\end{center}
\begin{tabular}{rl}
 & unde vedere würfe in den wint.\\ 
 & dannoch was Gahmuretes kint\\ 
 & ninder müede an deheinem lide.\\ 
 & dô wânde Clamide, der vride\\ 
5 & wære \textbf{zerbrochen} \textbf{ûz} der stat.\\ 
 & sînen kampfgenôz er bat,\\ 
 & daz er sich selben êrte\\ 
 & unde \textbf{mangen wurf} werte.\\ 
 & ez giengen ûf in \textit{sleg}e grôz.\\ 
10 & die wâren \textbf{mangen} \textbf{steine} genôz.\\ 
 & sus antwurt im des landes wirt:\\ 
 & "ich wæne, dich \textbf{manic} wurf verbirt.\\ 
 & wan dâ vür ist mîn triwe pfant.\\ 
 & hetest êt vride von mîner hant,\\ 
15 & dir \textbf{enbræche} manigen swenkel\\ 
 & brust, houbt noch den schenkel."\\ 
 & Clamide dranc müede zuo.\\ 
 & diu was im dannoch \textbf{gar} ze vruo.\\ 
 & \begin{large}S\end{large}ic gewunnen, sic verloren\\ 
20 & wart sunder dâ mit strîte erkoren.\\ 
 & doch wart der küni\textit{c} Clamide\\ 
 & \textbf{en} schumpfentiure \textbf{geschouwet} ê\\ 
 & \textbf{von} einem niderzucke.\\ 
 & von Parzivals drucke\\ 
25 & \textbf{bluot wæte} ûz ôren unde \textbf{ûz} nasen.\\ 
 & daz machte rôt den grüenen wasen.\\ 
 & er \textbf{enblôzt} im daz houbt schier\\ 
 & \textbf{von} helme unde von harsenier.\\ 
 & gein \textbf{slege} saz der betwungene lîp.\\ 
30 & der sigehafte sprach: "mîn wîp\\ 
\end{tabular}
\scriptsize
\line(1,0){75} \newline
G I O L M Q R Z \newline
\line(1,0){75} \newline
\textbf{3} \textit{Initiale} I  \textbf{19} \textit{Initiale} G  \textbf{21} \textit{Initiale} I Z  \textbf{29} \textit{Initiale} R  \newline
\line(1,0){75} \newline
\textbf{1} vedere] vedren I (R) \textbf{2} Gahmuretes] Gahmurets G Gamvretes O Gahmuͯretes L gamuretis M gamúretes Q gahmurtes R gamuretes Z \textbf{3} ninder] Nirgen M Nieman R Hinder Z  $\cdot$ deheinem lide] dehainen liden I keyme lete M \textbf{4} dô] Da M Z  $\cdot$ Clamide] Glamide O  $\cdot$ der] das der Q (Z) \textbf{5} ûz] vff L \textbf{6} kampfgenôz] champhgenozen I (R)  $\cdot$ er] er do Q \textbf{7} selben] selber L Q R (Z) \textbf{8} mangen wurf] mangem wurfe I mangem werfen O manigen worff M (Q) (R) (Z) \textbf{9} slege] wurfe G slegen L \textbf{10} wâren] wern I  $\cdot$ mangen] maniges O (Q) (R) o\textit{m. } L mannigens M wol maniges Z  $\cdot$ steine] slegen I steins O (M) Z in angesteines L streytes Q stein R \textbf{11} im] \textit{om.} R  $\cdot$ wirt] wit M \textbf{12} wæne] warn Q  $\cdot$ manic] mangen I (M) (Q)  $\cdot$ verbirt] verbit M \textbf{13} wan] Wa R  $\cdot$ dâ vür ist] do ist fur Q \textbf{14} hetest êt] hetstet G het ir ein I Hettest eht duͯ L Hettis du M (Q) (Z) \textbf{15} enbræche] brachte M brechen R  $\cdot$ manigen] mange L maniger Z \textbf{16} brust houbt] Hovpt bruͯst L  $\cdot$ noch] nach O  $\cdot$ den] \textit{om.} L die R  $\cdot$ schenkel] [enchel]: schenchel G senchel O swenchel L \textbf{17} Clamide] klamide I Glamide O Clamite Q \textbf{18} diu] Das R  $\cdot$ gar] al I  $\cdot$ vruo] \textit{om.} Q \textbf{19} Sic] Sie Q  $\cdot$ sic] sie Q \textbf{20} dâ] do Q  $\cdot$ erkoren] [verkorn]: erkorn O verlorn R \textbf{21} Man sach den kvnig Clamiden L  $\cdot$ doch] Do R  $\cdot$ künic] chune G  $\cdot$ Clamide] Glamide O \textbf{22} en schumpfentiure] entschunfentvre G in shunphenture I Tschvmpfentivre L Din Schinphentur M An [schúmpe]: schúmpfentewr Q An v́berwunden R An tschumpfentevr Z  $\cdot$ geschouwet ê] liden L \textbf{23} von] Mit Q R  $\cdot$ niderzucke] nider [buͤche]: buche I \textbf{24} Parzivals] parzifals I Parcifales O (L) (Z) parzifales M partzifals Q parczifals R \textbf{25} wæte] wiel I wæt O fur Q wuͦt R weten Z  $\cdot$ ôren] munden R  $\cdot$ ûz nasen] vz der nasen I Z nasin M (R) \textbf{26} machte] macht O (R) (Z)  $\cdot$ grüenen] \textit{om.} R \textbf{27} enblôzt] enblocz R \textbf{28} von] Vom Q R  $\cdot$ helme] helmen M  $\cdot$ unde] \textit{om.} L  $\cdot$ harsenier] harnisire M halspere R \textbf{29} gein] Ein R  $\cdot$ slege] slag O (L) (M) (R) (Z) slagen Q  $\cdot$ der] \textit{om.} I  $\cdot$ betwungene] betungen R \newline
\end{minipage}
\hspace{0.5cm}
\begin{minipage}[t]{0.5\linewidth}
\small
\begin{center}*T
\end{center}
\begin{tabular}{rl}
 & unde vedern würfe in den wint.\\ 
 & Dannoch was Gahmuretes kint\\ 
 & niender müede an deheinem lide.\\ 
 & Dô wânde Clamide, der vride\\ 
5 & wære \textbf{gebrochen} \textbf{von} der stat.\\ 
 & sînen kampfgenôz er bat,\\ 
 & daz er sich selben êrte\\ 
 & unde \textbf{manegem wurfe} werte.\\ 
 & ez giengen ûf in slege grôz.\\ 
10 & die wâren \textbf{wol} \textbf{manege\textit{m}} \textbf{steine} genôz.\\ 
 & Sus antwurtim des landes wirt:\\ 
 & "ich wæne, dich \textbf{mane\textit{c}} wurf verbirt.\\ 
 & wan dâ vür ist mîn triuwe pfant.\\ 
 & hetest eht vride von mîner hant,\\ 
15 & dir \textbf{enbræche} manegen swenkel\\ 
 & brust, houbet noch den schenkel."\\ 
 & \begin{large}C\end{large}lamide dranc müede zuo.\\ 
 & diu was im dannoch \textbf{al}ze vruo.\\ 
 & Sige gewunnen, sige verlorn\\ 
20 & wart sunder dâ mit strîte erkorn.\\ 
 & \multicolumn{1}{l}{ - - - }\\ 
 & \multicolumn{1}{l}{ - - - }\\ 
 & \textbf{mit} einem niderzucke\\ 
 & von Parcifals drucke\\ 
25 & \textbf{wæt im bluot} ûz ôren unde \textbf{ûz} nasen.\\ 
 & daz mahte rôt den grüenen wasen.\\ 
 & er \textbf{entwâpent} im daz houbet schiere\\ 
 & \textbf{vom} helme unde von harscheniere.\\ 
 & gegen \textbf{slage} saz der betwungene lîp.\\ 
30 & Der sigehafte sprach: "mîn wîp\\ 
\end{tabular}
\scriptsize
\line(1,0){75} \newline
T U V W \newline
\line(1,0){75} \newline
\textbf{2} \textit{Majuskel} T  \textbf{4} \textit{Majuskel} T  \textbf{11} \textit{Majuskel} T  \textbf{17} \textit{Initiale} T U W  \textbf{19} \textit{Majuskel} T  \textbf{30} \textit{Majuskel} T  \newline
\line(1,0){75} \newline
\textbf{1} würfe] wuͦrfen U \textbf{2} Gahmuretes] Gahmvretes T Gahmuͦretes U gamuretes V gamurettes W \textbf{3} niender] Was nit U Niergent V  $\cdot$ lide] gelede U (W) \textbf{4} Dô] Da V  $\cdot$ Clamide] klamide W \textbf{6} bat] do bat W \textbf{7} selben] selber U W \textbf{8} Vnd neúwen eine auff in kerte W  $\cdot$ manegem wurfe] manegen wurf U (V) \textbf{9} ez] Sie U \textbf{10} manegem] manegen T maniger V mangen W \textbf{11} wirt] [w*t]: wirt V \textbf{12} manec] manegen T [manigen]: manig V  $\cdot$ verbirt] verwirt W \textbf{13} Wo do fúr mein treúwe ist pfand W  $\cdot$ ist] \textit{om.} U \textbf{14} hetest eht vride] Hes duͦ vreden U Hettest du eht fride V \textbf{15} enbræche manegen] zuͦ breche manec U enbreche manger V breche mangel W \textbf{17} Clamide] KLamide W \textbf{18} diu] Des W  $\cdot$ dannoch alze] noch gar zuͦ W \textbf{20} dâ] do V W  $\cdot$ erkorn] do ercorn U \textbf{21} Doch wart der kv́nig clamide V \textbf{22} [Ent schvmph*]: An schvmphentúre geschowet e V \textbf{23} niderzucke] nider zuͦcken U \textbf{24} Parcifals] Parzifals T parzifales V partzifals W \textbf{25} wæt im bluot] Bluͦt wete im W  $\cdot$ unde] \textit{om.} U \textbf{27} entwâpent] intwapente U  $\cdot$ im] \textit{om.} W \textbf{28} vom] Von U V W \newline
\end{minipage}
\end{table}
\end{document}
