\documentclass[8pt,a4paper,notitlepage]{article}
\usepackage{fullpage}
\usepackage{ulem}
\usepackage{xltxtra}
\usepackage{datetime}
\renewcommand{\dateseparator}{.}
\dmyyyydate
\usepackage{fancyhdr}
\usepackage{ifthen}
\pagestyle{fancy}
\fancyhf{}
\renewcommand{\headrulewidth}{0pt}
\fancyfoot[L]{\ifthenelse{\value{page}=1}{\today, \currenttime{} Uhr}{}}
\begin{document}
\begin{table}[ht]
\begin{minipage}[t]{0.5\linewidth}
\small
\begin{center}*D
\end{center}
\begin{tabular}{rl}
\textbf{135} & mîn \textbf{tjoste} in hinderz ors \textbf{verswanc},\\ 
 & daz in der satel \textbf{ninder} dranc.\\ 
 & ich hân dicke prîs bezalt\\ 
 & unt manegen ritter ab gevalt.\\ 
5 & des \textbf{en}moht ich nû geniezen niht.\\ 
 & \textbf{ein} hôhez laster mir des giht.\\ 
 & \begin{large}S\end{large}i hazzent mich besunder,\\ 
 & die \textbf{von} der tavelrunder,\\ 
 & der ich ähte nider stach,\\ 
10 & \textbf{dâ} ez manec \textbf{wert} \textbf{juncvrouwe} sach,\\ 
 & umbe den sperwære ze Kanadic.\\ 
 & ich behielt iu prîs unt mir den sic.\\ 
 & daz sâhet ir unt Artus,\\ 
 & der \textbf{mîne swester hât} ze hûs,\\ 
15 & die süezen Cunnewaren.\\ 
 & \textbf{ir} munt kan niht gebâren\\ 
 & mit lachen, ê si den \textbf{gesiht},\\ 
 & dem man des \textbf{hœhsten} prîses giht.\\ 
 & wan kœme mir \textbf{doch} der selbe man,\\ 
20 & sô würde ein \textbf{strîten} hie getân\\ 
 & als hiute morgen, dô ich streit\\ 
 & unt eime vürsten vrumte leit,\\ 
 & der mir sîn tjustieren bôt.\\ 
 & von mîner tjoste lag er tôt.\\ 
25 & Ich \textbf{en}wil iu niht von zorne sagen,\\ 
 & daz maneger hât sîn wîp geslagen\\ 
 & umb \textbf{ir} krenker schulde.\\ 
 & het ich dienest oder hulde,\\ 
 & daz ich iu solte bieten,\\ 
30 & ir m\textit{üe}st iuch mangels nieten.\\ 
\end{tabular}
\scriptsize
\line(1,0){75} \newline
D \newline
\line(1,0){75} \newline
\textbf{7} \textit{Initiale} D  \textbf{25} \textit{Majuskel} D  \newline
\line(1,0){75} \newline
\textbf{11} Kanadic] kanedich D \textbf{13} Artus] Artv̂s D \textbf{30} müest] mvͦst D \newline
\end{minipage}
\hspace{0.5cm}
\begin{minipage}[t]{0.5\linewidth}
\small
\begin{center}*m
\end{center}
\begin{tabular}{rl}
 & mîn \textbf{hant} in hinderz ros \textbf{verswanc},\\ 
 & daz in der satel \textbf{nider} dranc.\\ 
 & ich hân dicke prîs bezalt\\ 
 & und manigen ritter ab gevalt.\\ 
5 & des \textbf{en}m\textit{o}hte ich nû geniezen niht.\\ 
 & \textbf{mîn} hôhez laster mir des giht.\\ 
 & si hazzent mich besunder,\\ 
 & die \textbf{von} der tavelrunder,\\ 
 & der ich ehte nider stach,\\ 
10 & \textbf{daz} ez manige \textbf{juncvrouw\textit{e}} sach,\\ 
 & umbe den sperwer ze Kan\textit{e}dic.\\ 
 & ich behielt iu \textbf{den} prîs und mir den sic.\\ 
 & daz sâhet \textit{i}r und Artus.\\ 
 & der \textbf{hât mîne swester} ze hûs,\\ 
15 & die süezen Cunnewaren.\\ 
 & \textbf{ir} munt kan niht gebâren\\ 
 & mit lachene, ê si den \textbf{gesiht},\\ 
 & dem man des \textbf{hœhesten} prîses giht.\\ 
 & wanne kæm mir \textbf{doch} der selbe man,\\ 
20 & sô würde ein \textbf{strît} hie getân\\ 
 & als hiute morgen, dô ich streit\\ 
 & und einem vürsten vrumete leit,\\ 
 & der mir sîn justieren bôt.\\ 
 & von mîner juste lac er tôt.\\ 
25 & ich \textbf{en}wil iu niht von zorne sagen,\\ 
 & daz maniger hât sîn wîp geslagen\\ 
 & umb \textbf{ir} krenker schulde.\\ 
 & hete ich dienest oder hulde,\\ 
 & daz ich iu solte bieten,\\ 
30 & ir müest iuch mangels nieten.\\ 
\end{tabular}
\scriptsize
\line(1,0){75} \newline
m n o \newline
\line(1,0){75} \newline
\newline
\line(1,0){75} \newline
\textbf{1} hinderz] hinder n o  $\cdot$ verswanc] geswang n o \textbf{5} enmohte] enmoͯchte m mag n o \textbf{6} des] das n o \textbf{7} hazzent] hasset n \textbf{10} juncvrouwe] jungfrouwen m \textbf{11} ze Kanedic] zekandig m zuͦ canedig n zu kanadig o \textbf{12} iu den prîs und] den pris o \textbf{13} sâhet] sohen n sagent o  $\cdot$ ir] mir m  $\cdot$ Artus] artuͯs o \textbf{14} Den (Dem o ) min swester hat zuͦ husz n (o) \textbf{15} süezen] suͯsse n  $\cdot$ Cunnewaren] kunne woren m konne woren n konnen woren o \textbf{20} hie] vil o \textbf{22} einem] einen n \textbf{25} enwil] wil n o \newline
\end{minipage}
\end{table}
\newpage
\begin{table}[ht]
\begin{minipage}[t]{0.5\linewidth}
\small
\begin{center}*G
\end{center}
\begin{tabular}{rl}
 & \textit{mîn \textbf{tjost} in hinder daz ors \textbf{swanc},}\\ 
 & \textit{daz in der satel \textbf{ninder} dranc.}\\ 
 & \textit{ich hân dicke brîs bezalt}\\ 
 & \textit{und manigen rîter ab gevalt.}\\ 
5 & \textit{des \textbf{en}moht ich nû geniezen niht.}\\ 
 & \textit{\textbf{ein} hôhez laster mir des giht.}\\ 
 & si hazzent mich besunder,\\ 
 & die \textbf{von} der tavelrunder,\\ 
 & der ich ahte nider stach,\\ 
10 & \textbf{dâ}z manic \textbf{wert} \textbf{vrouwe} sach,\\ 
 & umbe den sparwære ze Kanadic.\\ 
 & ich behielt iu brîs und mir den sic.\\ 
 & daz sâhet ir und Artus,\\ 
 & der \textbf{mîne swester hât} ze hûs,\\ 
15 & die süezen Kunewaren.\\ 
 & \textbf{\textit{de}r} munt kan niht gebâren\\ 
 & mit lachene, ê si den \textbf{gesiht},\\ 
 & dem man des \textbf{hœhesten} brîses g\textit{i}ht.\\ 
 & wan kœme mir \textbf{nû} der selbe man,\\ 
20 & sô würde ein \textbf{strîten} hie getân\\ 
 & als hiute morgen, dô ich streit\\ 
 & unde einem vürsten vrumte leit,\\ 
 & der mir sîn tjostieren bôt.\\ 
 & von mîner tjoste lag er tôt.\\ 
25 & ich wil iu niht von zorne sagen,\\ 
 & \textit{d}a\textit{z} maniger hât sîn wîp geslagen\\ 
 & umbe \textbf{michel} krenker schulde.\\ 
 & het ich dienst oder hulde,\\ 
 & daz ich iu solte bieten,\\ 
30 & ir müeset \textit{i}uch mangels nieten.\\ 
\end{tabular}
\scriptsize
\line(1,0){75} \newline
G I O L M Q R Z \newline
\line(1,0){75} \newline
\textbf{5} \textit{Initiale} Q  \textbf{7} \textit{Initiale} I O L R Z  \textbf{25} \textit{Initiale} I  \newline
\line(1,0){75} \newline
\textbf{1} \textit{Die Verse 134.27-135.6 fehlen} G   $\cdot$ in] \textit{om.} Z  $\cdot$ swanc] ver swanch O (L) (M) (R) (Z) erswanck Q \textbf{2} ninder] nider M Z der nider Q \textbf{3} hân] bin R  $\cdot$ bezalt] gezalt Q \textbf{5} enmoht] mochte M  $\cdot$ nû] \textit{om.} O Q \textbf{6} ein] E R  $\cdot$ des] \textit{om.} L das R  $\cdot$ giht] vergiht O geschiht L (Q) \textbf{7} si hazzent] ÷ihazzent I (O) \textbf{8} tavelrunder] tauelrunden Q tauelrunde R \textbf{10} Da iz werde manninc juncfrouwe zach M  $\cdot$ dâz] Daz es L (R)  $\cdot$ wert] \textit{om.} O  $\cdot$ vrouwe] ivnchfrawe O (L) (R) (Z)  $\cdot$ sach] an sach O \textbf{11} ze] \textit{om.} R  $\cdot$ Kanadic] chanadich G Ganadic I kanadich O kanedich L lanadick Q kandadik R \textbf{12} brîs] den bris I \textit{om.} L  $\cdot$ und] \textit{om.} M R \textbf{13} sâhet] sagt Q  $\cdot$ und] vnd der kvnig L \textbf{14} der] Des Q  $\cdot$ mîne] miner R \textbf{15} Kunewaren] [*]: Guͤniwarn I Gvnwaren O (M) Cvnewaren L Contrwaren Q kúnnewauren R Cvmiewaren Z \textbf{16} der] ir G  $\cdot$ kan niht] niht chan O (L) (Q) kan úch R  $\cdot$ gebâren] gabren R \textbf{17} lachene] lachent R  $\cdot$ gesiht] gesyhte L geschicht Q \textbf{18} des hœhesten] des besten O (M) den hoͯste R  $\cdot$ brîses] prise R  $\cdot$ giht] [ig*]: iehet G gýchte L \textbf{19} kœme] kam Q  $\cdot$ der] dy M  $\cdot$ selbe] selbig Q \textbf{20} strîten] strît O (Q) (R) \textbf{21} hiute morgen] hiuͤten morgen I (L) (Z)  $\cdot$ dô] da M Z \textbf{22} unde] vnder I  $\cdot$ vürsten] fᵫrste R \textbf{24} von] Vnd Z \textbf{25} wil] enwil L (M) R Z  $\cdot$ niht von zorne] von zcorne nicht M nit me zorne R \textbf{26} daz] wan G  $\cdot$ hât sîn wîp] sin wip hat I  $\cdot$ geslagen] erslagen L (R) \textbf{27} \textit{Versfolge 135.28-27} O   $\cdot$ michel krenker] cleiner I michels chrencher O krencker L michels kleiner Q \textbf{29} ich] \textit{om.} O  $\cdot$ solte] sode Q  $\cdot$ bieten] biten L M (Q) \textbf{30} müeset] must M  $\cdot$ iuch] sivch G ioch R  $\cdot$ mangels] manniges M (Q) \newline
\end{minipage}
\hspace{0.5cm}
\begin{minipage}[t]{0.5\linewidth}
\small
\begin{center}*T (U)
\end{center}
\begin{tabular}{rl}
 & mîn \textbf{tjost} in hinder\textit{z} ors \textbf{swanc},\\ 
 & daz in der satel \textbf{nider} dranc.\\ 
 & ich hân dicke prîs bezalt\\ 
 & und manegen rîter abe gevalt.\\ 
5 & des moht ich nû geniezen niht.\\ 
 & \textbf{ein} hôhez laster mir des giht.\\ 
 & \begin{large}S\end{large}i hazzent mich besunder,\\ 
 & die \textbf{ob} der tavelrunder,\\ 
 & de\textit{r} ich aht\textit{e} nider stach,\\ 
10 & \textbf{daz} ez manige \textbf{juncvrouwe} sach,\\ 
 & umb den sperwer zuo Kanadic.\\ 
 & ich behielt iu prîs und mir den sic.\\ 
 & daz sâhet ir und Artus,\\ 
 & der \textbf{mîne swester hât} zuo hûs,\\ 
15 & die süezen Cunnewaren.\\ 
 & \textbf{ir} munt kan niht gebâren\\ 
 & mit lachene, ê si den \textbf{siht},\\ 
 & dem man des \textbf{besten} prîses giht.\\ 
 & wan kœme mir der selbe man,\\ 
20 & sô würde ein \textbf{strîten} hie getân\\ 
 & als hiute morgen, dô ich streit\\ 
 & und eime vürsten vrumte leit,\\ 
 & der mir sîn tjostieren bôt.\\ 
 & von mîner tjost lac er tôt.\\ 
25 & ich \textbf{en}wil iu niht von zorne sagen,\\ 
 & daz maneger hât sîn wîp geslagen\\ 
 & umb \textbf{michel} krenker schulde.\\ 
 & het ich dienst oder hulde,\\ 
 & daz ich iu solte bieten,\\ 
30 & ir müeset iuch mangels nieten.\\ 
\end{tabular}
\scriptsize
\line(1,0){75} \newline
U V W T \newline
\line(1,0){75} \newline
\textbf{7} \textit{Initiale} U V W T  \textbf{25} \textit{Majuskel} T  \newline
\line(1,0){75} \newline
\textbf{1} hinderz] hinder U \textbf{2} nider] niender T \textbf{4} rîter] degen T \textbf{5} moht] moͤht V (W)  $\cdot$ nû] eúch W \textbf{6} des] das W  $\cdot$ giht] [geschi*]: geschiht V \textbf{7} Si hazzent] Dje hassent V SY hassen W \textbf{8} ob der] von W von der T \textbf{9} der ich ahte] [Der]: Den ich ahteworbe U Der etwie vil ich V Der ich do ethwe W daz ich ir ehtwe T \textbf{10} ez] \textit{om.} W  $\cdot$ juncvrouwe] wert [*]: frowe V werd iunckfrawe W vrouwe T \textbf{11} Kanadic] Canadic U [*]: kanadig V benedic W \textbf{12} prîs] den pris T \textbf{15} süezen] suͤsse W  $\cdot$ Cunnewaren] Kuͦnewaren U kvnnewaren V kunnenbarn W kvnnewâren T \textbf{16} ir] der T \textbf{17} siht] gesicht W \textbf{18} dem] den T  $\cdot$ besten] hoͤchsten W \textbf{19} mir] mir nun W \textbf{20} strîten hie] strit alhie V \textbf{21} morgen] am morgen W \textbf{23} sîn] \textit{om.} V T \textbf{25} enwil] wil V W T \textbf{29} Ich sol eúch zuͦ dienst nit me bieten W \textbf{30} müeset] muͤssen W  $\cdot$ mangels] mangel W \newline
\end{minipage}
\end{table}
\end{document}
