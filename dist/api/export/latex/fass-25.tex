\documentclass[8pt,a4paper,notitlepage]{article}
\usepackage{fullpage}
\usepackage{ulem}
\usepackage{xltxtra}
\usepackage{datetime}
\renewcommand{\dateseparator}{.}
\dmyyyydate
\usepackage{fancyhdr}
\usepackage{ifthen}
\pagestyle{fancy}
\fancyhf{}
\renewcommand{\headrulewidth}{0pt}
\fancyfoot[L]{\ifthenelse{\value{page}=1}{\today, \currenttime{} Uhr}{}}
\begin{document}
\begin{table}[ht]
\begin{minipage}[t]{0.5\linewidth}
\small
\begin{center}*D
\end{center}
\begin{tabular}{rl}
\textbf{25} & wir solden vîende wênic sparn,\\ 
 & sît Vridebrant ist \textbf{hin} gevarn.\\ 
 & der lœset dort sîn eigen lant.\\ 
 & ein künec, \textbf{heizet} Hernant,\\ 
5 & den er durch \textbf{Herlinde} \textbf{sluoc},\\ 
 & des mâge tuont im \textbf{leit} genuoc.\\ 
 & si\textbf{ne} wellent sich es niht mâzen.\\ 
 & er hât hie helde \textbf{lâzen}:\\ 
 & den herzogen Hiuteger,\\ 
10 & des \textbf{rîtertât} uns \textbf{manegiu sêr}\\ 
 & \textbf{vrumt}, unt sîn geselleschaft.\\ 
 & ir strît hât kunst unt kraft.\\ 
 & sô hât hie manegen soldier\\ 
 & von Normandie Gaschier,\\ 
15 & der wîse degen hêre.\\ 
 & \textbf{noch} hât hie rîter mêre\\ 
 & Kaylet von Hoskurast,\\ 
 & manegen zornigen gast.\\ 
 & die \textbf{bræhten} alle in ditze lant\\ 
20 & \textbf{der} Schotten künec Vridebrant\\ 
 & unt sîner genôze viere\\ 
 & mit manegem soldiere.\\ 
 & \textbf{\textit{\begin{large}W\end{large}}esterhalp dor\textit{t}} an dem mer,\\ 
 & dâ lît Isenhartes her\\ 
25 & mit \textbf{vliezenden} ougen.\\ 
 & offenlîch noch tougen\\ 
 & gesach si \textit{n}immer \textbf{mêr} dehein man,\\ 
 & sine m\textit{üe}sen jâmers \textbf{wunder} hân.\\ 
 & ir \textbf{herzen regen die güsse warp},\\ 
30 & sît an der tjost ir hêrre starp."\\ 
\end{tabular}
\scriptsize
\line(1,0){75} \newline
D Fr14 \newline
\line(1,0){75} \newline
\textbf{23} \textit{Initiale} D Fr14  \newline
\line(1,0){75} \newline
\textbf{9} Hiuteger] Hv̂teger D \textbf{14} Gaschier] Gascier D Gascir Fr14 \textbf{15} der] so Fr14 \textbf{17} Kaylet] kailet Fr14  $\cdot$ Hoskurast] Hoskvrast D hoscvrast Fr14 \textbf{19} bræhten] brachter Fr14 \textbf{20} Schotten] Scoten D (Fr14)  $\cdot$ Vridebrant] Vridbrant Fr14 \textbf{21} sîner] sine Fr14 \textbf{23} Westerhalp] ÷esterhalp D  $\cdot$ dort] dor D \textbf{24} Isenhartes] Jsenhartes D Jsenharts Fr14 \textbf{27} nimmer] mimmer D \textbf{28} müesen] mvͦsen D (Fr14) \newline
\end{minipage}
\hspace{0.5cm}
\begin{minipage}[t]{0.5\linewidth}
\small
\begin{center}*m
\end{center}
\begin{tabular}{rl}
 & wir solten vîende wênic sparn,\\ 
 & sît Fridebrant ist \textbf{hin} gevarn.\\ 
 & der lœset dort sîn eigen lan\textit{t}.\\ 
 & ein künic, \textbf{heizet} Hernant,\\ 
5 & den er durch \textbf{Herelinde} \textbf{sl\textit{uoc}},\\ 
 & des mâge tuont ime \textbf{lîdens} genuoc.\\ 
 & si wellent sichs niht mâzen.\\ 
 & er hât hie helde \textbf{lâzen}:\\ 
 & den herzogen H\textit{u}t\textit{e}ger,\\ 
10 & - des \textbf{ritter tet} uns \textbf{meni\textit{ge swer}} -,\\ 
 & \textbf{vriunt} und sîn geselleschaft.\\ 
 & ir strît het kunst und kraft.\\ 
 & sô hât hie manige\textit{n} soldier\\ 
 & von Normandie Gaschier,\\ 
15 & der wîs\textit{e} degen hêre.\\ 
 & \textbf{noch} hât hie ritter mêre\\ 
 & Kailet \textit{v}on Hos\textit{cu}r\textit{a}st,\\ 
 & manigen zornigen gast.\\ 
 & die \textbf{brâhten} alle in diz lant\\ 
20 & \textbf{dirre} Schotten künic Fridebrant\\ 
 & und sîner genô\textit{z}e viere\\ 
 & mit manigem soldiere.\\ 
 & \textbf{wes\textit{t}erhalp dort} an dem mer,\\ 
 & dâ lît Ysenhartes her\\ 
25 & mit \textbf{vlie\textit{z}enden} ougen.\\ 
 & offenlîchen noch tougen\\ 
 & gesach si nimmer \textbf{mêr} kein man,\\ 
 & sine müesen jâmers \textbf{wunder} hân.\\ 
 & i\textit{r} \textbf{her\textit{ze sô gar verdarp}},\\ 
30 & sît an der just ir hêrre starp."\\ 
\end{tabular}
\scriptsize
\line(1,0){75} \newline
m n o \newline
\line(1,0){75} \newline
\newline
\line(1,0){75} \newline
\textbf{2} gevarn] gerarn \textit{nachträglich korrigiert zu:} gevarn m \textbf{3} lant] lang \textit{nachträglich korrigiert zu:} lant m \textbf{4} Hernant] hernant \textit{nachträglich korrigiert zu:} herrnant m heis hern ant o \textbf{5} er] ir o  $\cdot$ Herelinde] herelingen n herre linde o  $\cdot$ sluoc] slagt \textit{nachträglich korrigiert zu:} sluͯg m \textbf{6} lîdens] leides n o \textbf{7} sichs] sieches o \textbf{8} helde] hellende o  $\cdot$ lâzen] lassen \textit{nachträglich korrigiert zu:} gelassen m \textbf{9} Huteger] hittiger m n o \textbf{10} des] * \textit{nachträglich korrigiert zu:} Der o  $\cdot$ menige] menicher m  $\cdot$ swer] \textit{om.} m \textbf{13} manigen] maniger m manigem n \textbf{14} Gaschier] Gascier m cascier n kascier o \textbf{15} wîse] wisen m \textbf{17} Kailet] Kaylet n Kahilet o  $\cdot$ von] non m n o  $\cdot$ Hoscurast] hostnarst \textit{nachträglich korrigiert zu:} horstnast m \textbf{20} künic] koͯnnige n (o)  $\cdot$ Fridebrant] vridebrant m \textbf{21} genôze] genore \textit{nachträglich korrigiert zu:} genosze m \textbf{23} westerhalp] Westnerhalp m \textbf{24} dâ] Do n o  $\cdot$ Ysenhartes] jsenhartes m isenhartes n ẏsenhartes o \textbf{25} vliezenden] flierenden m n o \textbf{26} tougen] dauwen o \textbf{27} mêr] \textit{om.} n o  $\cdot$ kein] keinen n \textbf{28} sine] Sú n (o) \textbf{29} Jn herczen m \newline
\end{minipage}
\end{table}
\newpage
\begin{table}[ht]
\begin{minipage}[t]{0.5\linewidth}
\small
\begin{center}*G
\end{center}
\begin{tabular}{rl}
 & wir solten vînde wênic sparen,\\ 
 & sît Fridebrant ist \textbf{hin} gevaren.\\ 
 & der lœset dort sîn eigen lant.\\ 
 & ein künic, \textbf{der hiez} Hernant,\\ 
5 & den er durch \textbf{Herlinde} \textbf{sluoc},\\ 
 & des mâge tuont im \textbf{leit} genuoc.\\ 
 & si\textbf{ne} wellent si\textit{ch}s niht mâzen.\\ 
 & er hât hie helde \textbf{lâzen}:\\ 
 & den herzogen Huteger,\\ 
10 & des \textbf{rîters tât} uns \textbf{manic sêr}\\ 
 & \textbf{vrumet}, und sîn geselleschaft.\\ 
 & ir strît hât kunst und kraft.\\ 
 & sô hât hie manigen soldier\\ 
 & von Normandie Gatschier,\\ 
15 & der wîse degen hêre.\\ 
 & \textbf{ouch} hât hie rîter mêre\\ 
 & Kailet von Hoscurast,\\ 
 & \textbf{vil} manigen zornigen gast.\\ 
 & die \textbf{brâht} alle in diz lant\\ 
20 & \textbf{der} Schotten künic Fridebrant\\ 
 & unde sîner genôze viere\\ 
 & mit manigem soldiere.\\ 
 & \textbf{\begin{large}D\end{large}ort westerthalp} an dem mer,\\ 
 & dâ lît Ysenhartes her\\ 
25 & mit \textbf{vliezende\textit{n}} ougen.\\ 
 & offenlîch noch tougen\\ 
 & gesach si nimer \textbf{mê} dehein man,\\ 
 & sine m\textit{üe}sen jâmers \textbf{wunder} hân.\\ 
 & ir \textbf{herzen regen in güsse warp},\\ 
30 & sît an der tjost ir hêrre starp."\\ 
\end{tabular}
\scriptsize
\line(1,0){75} \newline
G O L M Q R W Z Fr29 Fr32 Fr71 \newline
\line(1,0){75} \newline
\textbf{1} \textit{Initiale} O  \textbf{3} \textit{Versal} Fr32  \textbf{13} \textit{Initiale} Fr71  \textbf{17} \textit{Versal} Fr32  \textbf{23} \textit{Initiale} G L Q W Fr32  \newline
\line(1,0){75} \newline
\textbf{1} wir] ÷ir O  $\cdot$ vînde] die feinde Q (R) \textbf{2} sît] Mit Q  $\cdot$ Fridebrant] vridebrant Fr32 vri::: Fr71  $\cdot$ hin] her hin Q \textbf{3} der] Er L \textbf{4} der hiez] heisset R W der heizet Fr32  $\cdot$ Hernant] Nermant O herfant M herrant Q \textbf{5} durch] noch W  $\cdot$ Herlinde] herlinden O Z (Fr32) \textbf{6} des mâge tuont] Das machte Q Des mage tuͦt R  $\cdot$ leit] leydes Q (W) \textbf{7} sine] Sie M (R)  $\cdot$ sichs] sis G sich sin R \textbf{8} er] [Her]: er O  $\cdot$ hie] die Z  $\cdot$ lâzen] gelaszen L (M) (R) (W) (Fr32) \textbf{9} Huteger] Hvͦtger O huͯttiger L Nutiger M hertiger Q húttiger R hútiger W hutteger Z Hivteger Fr32 \textbf{10} rîters tât] ritter tvnt L (W) ritter teten M mannes ritters t*nt \textit{nachträglich korrigiert zu:} mannes ritter tat Q rat tet R  $\cdot$ manic] mannigen M (Q) menger R  $\cdot$ sêr] swer L \textbf{11} Vnd do zuͦ sein geselschafft W  $\cdot$ vrumet] Frymt Q  $\cdot$ geselleschaft] geneleschaft Z \textbf{13} sô hât hie] Auch hat er W \textbf{14} von] Vor Q  $\cdot$ Normandie] Ormendie L (W) thormandie M Normadie Q R :::mandîe Fr29  $\cdot$ Gatschier] gatschir M R :::chîer Fr29 gatschîer Fr32 \textbf{15} \textit{Versfolge 25.16-15} M   $\cdot$ hêre] [mere]: here M herre R \textbf{16} ouch] Noch L Q R W (Fr32)  $\cdot$ hie] er Q \textbf{17} Kailet] Kaylet O M R Fr29 Kaýlet L Kaẏlet Fr32 Kleinot Q Gaiolet W Geilet Z Cheilet Fr71  $\cdot$ von] vnd Z  $\cdot$ Hoscurast] hoschvͦrast O hoskvrast L hoschurast M hoch scutat Q hotschurast W hoschvrast Z hoscvrast Fr29 \textbf{19} brâht] brachte er L broche Q  $\cdot$ in diz] indaz O (R) in dise Q \textbf{20} Schotten] schoten G schotte::: Fr71  $\cdot$ Fridebrant] Frýdebrant L fridebrand R vridebrant Fr32 \textbf{21} sîner] sein Q (Z) (Fr71)  $\cdot$ genôze] genossen R W Z \textbf{23} Dort] Doͯrst R  $\cdot$ westerthalp] westert L  $\cdot$ an] bie M \textbf{24} dâ] Do Q  $\cdot$ Ysenhartes] ẏsenhartes G Fr32 isenhartes O ýsenhartes L ysenhartis M eysenbartes Q Jsenharttes R  $\cdot$ her] herre M (R) \textbf{25} vliezenden] [flieiendem]: fliezendem G vlieszendē M \textbf{26} noch] vnd Z \textbf{27} gesach] Geschag Q  $\cdot$ nimer mê] nie mere sit L nimmer Z  $\cdot$ dehein] do heim Q \textbf{28} sine] Sy R (Z) (Fr71)  $\cdot$ müesen] moͮsen G (O) (R) (Fr32) (Fr71) muste Z \textbf{29} ir] Jn Z  $\cdot$ herzen] herze O  $\cdot$ regen in] in regen O regen die Q (R) (Fr32)  $\cdot$ güsse] guß W  $\cdot$ warp] erwarp L (W) gab M \textbf{30} sît] Sider R  $\cdot$ der tjost] der \textit{nachträglich korrigiert zu:} desz Q dem strit R  $\cdot$ starp] erstarp L \newline
\end{minipage}
\hspace{0.5cm}
\begin{minipage}[t]{0.5\linewidth}
\small
\begin{center}*T
\end{center}
\begin{tabular}{rl}
 & wir solten vîende wênic sparn,\\ 
 & sît Fridebrant ist \textbf{hinnen} gevarn.\\ 
 & der lœset dort sîn eigen lant.\\ 
 & ein künec, \textbf{der hiez} Hernant,\\ 
5 & den er durch \textbf{Berlinde} \textbf{ersluoc},\\ 
 & des mâge tuont im \textbf{leit} genuoc.\\ 
 & si\textbf{ne} wellent sichs niht mâzen.\\ 
 & er hât hie helde \textbf{verlâzen}:\\ 
 & de\textit{n} herzogen Hiuteger,\\ 
10 & des \textbf{rîter hânt} uns \textbf{manegiu sêr}\\ 
 & \textbf{gevrumt}, und sîn geselleschaft.\\ 
 & ir strît hât kunst und kraft.\\ 
 & Sô hât hie manegen soldier\\ 
 & von Normandie Gatschier,\\ 
15 & der wîse degen hêr.\\ 
 & \textbf{ouch} hât hie rîter mêr\\ 
 & Kaylet von Hoscurast,\\ 
 & \textbf{vil} manegen zornigen gast.\\ 
 & die \textbf{brâhte} alle in diz lant\\ 
20 & \textbf{der} Schotten künec Fridebrant\\ 
 & und sîner genôze viere\\ 
 & mit manegem soldiere.\\ 
 & \textbf{\begin{large}D\end{large}ort westerhalp} an dem mer,\\ 
 & dâ lît Isenhartes her\\ 
25 & mit \textbf{riezenden} ougen.\\ 
 & offenlîche noch tougen\\ 
 & gesach si niemer \textbf{sît} kein man,\\ 
 & sine müesen jâmers \textbf{vil} hân.\\ 
 & ir \textbf{herzen regen in güsse erwarp},\\ 
30 & sît an der tjost ir hêrre starp."\\ 
\end{tabular}
\scriptsize
\line(1,0){75} \newline
T U V \newline
\line(1,0){75} \newline
\textbf{13} \textit{Majuskel} T  \textbf{23} \textit{Initiale} T  \newline
\line(1,0){75} \newline
\textbf{1} sparn] spran U \textbf{2} hinnen] hin U V \textbf{4} hiez] heisset V  $\cdot$ Hernant] [Her*]: Hernant V \textbf{5} Berlinde] Berlindin U berlinden V  $\cdot$ ersluoc] sluͦc U (V) \textbf{6} im] vns U \textbf{7} sine] Sie U \textbf{8} verlâzen] gelazen U (V) \textbf{9} den] dem T U  $\cdot$ Hiuteger] Hivteger T Huͦteger U Huttiger V \textbf{10} hânt] duͦnt U (V)  $\cdot$ manegiu] manigin U manig V \textbf{11} gevrumt und] Geruͦmet vnd U [*]: Vnde dar zvͦ V \textbf{13} hie] sie U \textbf{14} Gatschier] Gatscier T \textbf{15} wîse] [wisen]: wise T \textbf{17} Kaylet] Kylet U Kaẏlet V  $\cdot$ Hoscurast] [Hoccvrast]: Hoscvrast T [*ast]: hohschurast V \textbf{20} der] von V  $\cdot$ Schotten] schoten T schotte U  $\cdot$ Fridebrant] Fridebant T \textbf{21} genôze] genozen U \textbf{24} dâ] Do U (V)  $\cdot$ Isenhartes] jsenhartes T (V) ysenhartes U \textbf{25} riezenden] vliezenden U (V) \textbf{26} noch] vnd U (V) \textbf{27} sît] me U \textit{om.} V \textbf{28} sine] Sie U [si]: sú V \textbf{29} in] ein U V \textbf{30} hêrre] herze U [her*]: herre V \newline
\end{minipage}
\end{table}
\end{document}
