\documentclass[8pt,a4paper,notitlepage]{article}
\usepackage{fullpage}
\usepackage{ulem}
\usepackage{xltxtra}
\usepackage{datetime}
\renewcommand{\dateseparator}{.}
\dmyyyydate
\usepackage{fancyhdr}
\usepackage{ifthen}
\pagestyle{fancy}
\fancyhf{}
\renewcommand{\headrulewidth}{0pt}
\fancyfoot[L]{\ifthenelse{\value{page}=1}{\today, \currenttime{} Uhr}{}}
\begin{document}
\begin{table}[ht]
\begin{minipage}[t]{0.5\linewidth}
\small
\begin{center}*D
\end{center}
\begin{tabular}{rl}
\textbf{374} & \textit{\begin{large}O\end{large}}b dich halt dîn muoter lieze.\\ 
 & got gebe, daz ich\textbf{s} genieze.\\ 
 & \textbf{owî}, \textbf{er} stolz werder man,\\ 
 & waz ich gedingen gein im hân!\\ 
5 & nie wort ich \textbf{dennoch} zim gesprach,\\ 
 & in mîme slâfe ich in hînte sach."\\ 
 & Lyppaut \textbf{gie vür die} herzogîn\\ 
 & unt Obilot, diu tohter sîn.\\ 
 & \textbf{dô sprach er}: "vrouwe, stiwert uns zwei.\\ 
10 & mîn herze nâch vreuden schrei,\\ 
 & dô mich got dirre magt beriet\\ 
 & unt mich von ungemüete schiet."\\ 
 & Diu alte herzogîn sprach sân:\\ 
 & "waz welt ir mînes guotes hân?"\\ 
15 & "vrouwe, sît \textbf{irs} uns bereit,\\ 
 & Obilot wil bezzer kleit.\\ 
 & si dunket sich\textbf{s} \textbf{mit} wirde wert,\\ 
 & sît sô werder man ir minne gert\\ 
 & \textbf{unt} er \textbf{ir biutet} dienstes vil\\ 
20 & unt ouch ir kleinœde wil."\\ 
 & Dô sprach der meide muoter:\\ 
 & "er süezer man vil guoter!\\ 
 & ich wæne, ir meinet den vremden gast.\\ 
 & sîn \textbf{blic} ist reht \textbf{ein} meien glast."\\ 
25 & dô hiez \textbf{tragen dar} diu wîse\\ 
 & samît von Ethnise.\\ 
 & \textbf{unversniten} wât \textbf{truoc man} dâ mite,\\ 
 & pfelle von Thabronite\\ 
 & ûzem lande ze Tribalibot.\\ 
30 & \textbf{an} Kaukasas daz golt ist rôt,\\ 
\end{tabular}
\scriptsize
\line(1,0){75} \newline
D \newline
\line(1,0){75} \newline
\textbf{1} \textit{Initiale} D  \textbf{13} \textit{Majuskel} D  \textbf{21} \textit{Majuskel} D  \newline
\line(1,0){75} \newline
\textbf{1} Ob] ÷b D \textbf{7} Lyppaut] Lẏppaot D \textbf{8} Obilot] Obylot D \textbf{16} Obilot] Obylot D \textbf{26} Ethnise] Ethnîse D \textbf{30} Kaukasas] koͮkesas D \newline
\end{minipage}
\hspace{0.5cm}
\begin{minipage}[t]{0.5\linewidth}
\small
\begin{center}*m
\end{center}
\begin{tabular}{rl}
 & ob dich halt dîn muoter lieze.\\ 
 & got gebe, daz ich\textbf{s} genieze.\\ 
 & \textbf{ouwê}, \textbf{er} stolzer, werder man,\\ 
 & waz ich gedingen gegen ime hân!\\ 
5 & \textit{ni}e wort ich \textbf{danne} zuo ime gesp\textit{rach},\\ 
 & in mînem slâfe ich in hînaht sach."\\ 
 & Lippo\textit{u}t \textbf{gienc vür die} herzogîn\\ 
 & und Obil\textit{o}t, diu tohter sîn.\\ 
 & \textbf{dô sprach er}: "vrouwe, s\textit{ti}u\textit{r}et uns zwei.\\ 
10 & mîn herze nâch vröuden schrei,\\ 
 & dô mich got dirre magt beriet\\ 
 & und mich von ungemüete schiet."\\ 
 & diu alte herzogîn sprach sân:\\ 
 & "waz wellet ir mînes guotes hân?"\\ 
15 & "vrouwe, sît \textbf{irs} uns bereit,\\ 
 & Obilot wil bezzer kleit.\\ 
 & si dunket sich \textbf{niht} wirde wert,\\ 
 & sît sô werder man ir minne gert\\ 
 & \textbf{und} er \textbf{ir biutet} dienstes vil\\ 
20 & und ouch ir kleinœte wil."\\ 
 & dô sprach der megde muoter:\\ 
 & "er süezer man vil guoter!\\ 
 & ich wæne, ir meinet den vrömden gast.\\ 
 & sîn \textbf{blic} ist reht \textbf{ein} meien glast."\\ 
25 & dô hiez \textbf{tragen dar} diu wîse\\ 
 & samît von \textit{Eth}nise.\\ 
 & \textbf{unversnitener} wât \textbf{truoc man} dâ mite,\\ 
 & pfelle von Tabronite\\ 
 & ûzem lande ze Tribalibot.\\ 
30 & \textbf{an} K\textit{a}uk\textit{a}sas daz golt ist rôt,\\ 
\end{tabular}
\scriptsize
\line(1,0){75} \newline
m n o \newline
\line(1,0){75} \newline
\newline
\line(1,0){75} \newline
\textbf{1} lieze] liesz o \textbf{3} ouwê] Obe n (o) \textbf{4} gedingen] gedinge n o \textbf{5} nie] Me m n o  $\cdot$ danne] dannoch n o  $\cdot$ gesprach] gespart m sprach n o \textbf{7} Lippout] Lippoat m Lippaot n o \textbf{8} Obilot] obilet m abilot o  $\cdot$ diu] doe o \textbf{9} stiuret] strunent m \textbf{11} beriet] bereit o \textbf{16} Obilot] Obilat o \textbf{17} wert] [wet]: wert m \textbf{18} sô] \textit{om.} o  $\cdot$ ir] ie o \textbf{19} dienstes] dinest o \textbf{20} ouch] uch o \textbf{22} er] \textit{om.} n o \textbf{25} tragen] tragen tragen o \textbf{26} von] vnd n o  $\cdot$ Ethnise] anise m etnẏse n etnise o \textbf{27} unversnitener] Vnversnitten n (o) \textbf{28} Tabronite] thabronit n thabornitte o \textbf{29} Tribalibot] tribolabot o \textbf{30} Kaukasas] koukesas m kancasas n cancasas o \newline
\end{minipage}
\end{table}
\newpage
\begin{table}[ht]
\begin{minipage}[t]{0.5\linewidth}
\small
\begin{center}*G
\end{center}
\begin{tabular}{rl}
 & obe dich halt dîn muoter lieze.\\ 
 & got gebe, daz ich \textbf{es} genieze.\\ 
 & \textbf{owê}, \textbf{der} stolze, werde man,\\ 
 & waz ich gedingen gein im hân!\\ 
5 & \textit{nie wort ich \textbf{dannoch} zuo im gesprach,}\\ 
 & \textit{in mînem slâfe ich in hînt sach."}\\ 
 & Libaut \textbf{reit zer} herzogîn\\ 
 & unde Obilot, diu tohter sîn.\\ 
 & \textbf{er sprach}: "vrouwe, stiurt uns zwei.\\ 
10 & mîn herze nâch vröiden schrei,\\ 
 & dô mich got dirre maget beriet\\ 
 & unde mich von ungemüete schiet."\\ 
 & diu alte herzogîn sprach sân:\\ 
 & "waz welt ir mînes guotes hân?"\\ 
15 & "vrouwe, sît \textbf{irs} uns bereit,\\ 
 & Obilot wil bezzer kleit.\\ 
 & si dunket sich \textbf{es} \textbf{mit} wirde wert,\\ 
 & sît sô wer\textit{der} man ir minne gert\\ 
 & \textbf{und} er \textbf{ir biutet} dienstes vil\\ 
20 & unde ouch ir kleinœde wil."\\ 
 & dô sprach der magede muoter:\\ 
 & "er süezer man vil guoter!\\ 
 & ich wæne, ir meinet den vrömden gast.\\ 
 & sîn \textbf{varwe} ist reht \textbf{ein} meien glast."\\ 
25 & dô hiez \textbf{dar tragen} diu wîse\\ 
 & samît von Ethnise.\\ 
 & \textbf{unversniten} wât \textbf{truoc man} dâ mit,\\ 
 & pfelle von Tabrunit\\ 
 & ûzem lande ze Tribalibot.\\ 
30 & \textbf{in} Kausakas daz golt ist rôt,\\ 
\end{tabular}
\scriptsize
\line(1,0){75} \newline
G I O L M Q R Z Fr24 Fr38 \newline
\line(1,0){75} \newline
\textbf{1} \textit{Initiale} Fr38  \textbf{7} \textit{Initiale} I O L R Z Fr38  \textbf{23} \textit{Initiale} I  \newline
\line(1,0){75} \newline
\textbf{1} \textit{Die Verse 370.13-412.12 fehlen} Q   $\cdot$ Ob [ich]: dich ovch dinen mvͦter lieze Fr38  $\cdot$ dich halt] ioch halt dich R \textbf{2} ich es] ich sin R \textbf{3} owê] Owi L (M) (Fr24) Fr38  $\cdot$ der] er O L R Z Fr24 Fr38 ir M  $\cdot$ stolze] stoltzer L (M) (R) Z (Fr38)  $\cdot$ werde] werder I (O) (L) (M) (R) (Z) (Fr24) (Fr38) \textbf{4} Waz ich gen im gediennen kan R  $\cdot$ gedingen] gedinge M  $\cdot$ gein] von I \textbf{5} \textit{Die Verse 374.5-6 fehlen} G   $\cdot$ gesprach] [sprach]: gesprach L sprach M Fr38 \textbf{6} mînem] mimen R  $\cdot$ slâfe] schlas R  $\cdot$ hînt] hute M \textbf{7} Libaut] ÷ybavt O LJbavt L Libort M Lybant R Lybavt Z Fr38  $\cdot$ reit] qvam Z \textbf{8} Obilot] Obylot O Fr38 oblet R [Obilon]: Obilot Z \textbf{9} er sprach] do sprach er I (O) (L) (R) (Z) (Fr38)  $\cdot$ stiurt] stevr Z \textbf{11} dô] Da Z  $\cdot$ mich got dirre] got mir disú R mich got von dirre Fr38 \textbf{12} ungemüete] vngemache O L R Fr38 \textbf{13} alte] altú R \textit{om.} Z  $\cdot$ sprach] do sprach Z \textbf{15} irs uns] ir sin vns I Z irs mys M ir des R \textbf{16} Obilot] Obylot O Fr38 Obilet L Oblet R  $\cdot$ bezzer] bezriu I \textbf{17} sich es] sis I Z mich L \textbf{18} werder] wert G \textbf{19} er] \textit{om.} M  $\cdot$ dienstes] dienst I \textbf{22} er] Jr Fr38 \textbf{23} wæne] mein R  $\cdot$ meinet] nemet M \textbf{24} varwe] minn I blich O L (M) (R) (Z) (Fr38)  $\cdot$ reht] alse M rechtter R \textbf{25} dô] Da M Z  $\cdot$ dar tragen] tragen dar O Z Fr38 \textbf{26} Ethnise] entyse G I etnise O (Z) Arnise L ot nise M ethinisie R Etnyse Fr38 \textbf{27} unversniten] Vnversnitene O Vnuerschrotten R  $\cdot$ man] \textit{om.} R \textbf{28} Tabrunit] taprunit G tanprunit I tabvrnit O taprvnite M Thaburnit R \textbf{29} ûzem] vz einem I Vsserm R  $\cdot$ ze] von I \textit{om.} R  $\cdot$ Tribalibot] tribabilot M trẏbalibot R Tribalybot Fr38 \textbf{30} in Kausakas] inGaugushasch I Van kavkasas O An Cavcasas L Ancaucusas M An Kankasas R An kaukasas Z An kavkasas Fr38  $\cdot$ ist] \textit{om.} O \newline
\end{minipage}
\hspace{0.5cm}
\begin{minipage}[t]{0.5\linewidth}
\small
\begin{center}*T
\end{center}
\begin{tabular}{rl}
 & ob dich halt dîn muoter lieze.\\ 
 & got gebe, daz ich \textbf{dîn} genieze.\\ 
 & \textbf{ouwê}, stolz werder man,\\ 
 & waz ich gedinge gegen im hân!\\ 
5 & nie wort ich \textbf{dannoch} zim gesprach,\\ 
 & in mînem slâfe ich in hînt sach."\\ 
 & \begin{large}L\end{large}ybaut \textbf{reit zer} herzogîn\\ 
 & unde Obylot, diu tohter sîn.\\ 
 & \textbf{dô sprach er}: "vrouwe, stiuret uns zwei.\\ 
10 & mîn herze nâch vröuden schrei,\\ 
 & dô mich got dirre megde beriet\\ 
 & unde mich von ungemüete schiet."\\ 
 & Diu alte herzogîn sprach sân:\\ 
 & "waz welt ir mînes guotes hân?"\\ 
15 & "vrouwe, sît uns bereit,\\ 
 & Obylot wil bezzer kleit.\\ 
 & si dunket sich\textbf{s} \textbf{mit} wirde wert,\\ 
 & sît sô werder man ir minne gert,\\ 
 & \textbf{wan} er \textbf{biutet ir} dienstes vil\\ 
20 & unde ouch ir kleinœte wil."\\ 
 & Dô sprach der megde muoter:\\ 
 & "er süezer man vil guoter!\\ 
 & ich wæne, ir meinet den vremden gast.\\ 
 & sîn \textbf{blic} ist rehte \textbf{eines} meien glast."\\ 
25 & Dô hiez \textbf{tragen dar} diu wîse\\ 
 & samît von Etnise.\\ 
 & \textbf{unversniten} wât \textbf{man truoc} dâr mite,\\ 
 & pfelle von Tabrunite\\ 
 & ûz dem lande ze Tribalibot.\\ 
30 & \textbf{ane} Coucasas daz golt ist rôt,\\ 
\end{tabular}
\scriptsize
\line(1,0){75} \newline
T V W \newline
\line(1,0){75} \newline
\textbf{7} \textit{Initiale} T W  \textbf{13} \textit{Majuskel} T  \textbf{21} \textit{Majuskel} T  \textbf{25} \textit{Majuskel} T  \newline
\line(1,0){75} \newline
\textbf{1} halt] ioch V W \textbf{2} dîn] \textit{om.} W \textbf{3} stolz] [*]: er stoltz V er vil stoltzer W \textbf{5} ich dannoch zim] mein mund im W \textbf{7} Lybaut] Libaut V LYbout W  $\cdot$ reit zer] [*]: gieng fúr die V reit hin zuͦ der W \textbf{8} Obylot] obilot V \textbf{10} nâch] vor V \textbf{13} alte] \textit{om.} W  $\cdot$ sân] ie san W \textbf{15} sît] sint [*]: irs V \textbf{16} Obylot] [O*]: Obilot V Obilot W \textbf{17} sichs] sy sei es W \textbf{19} [*]: Vnde er ir bútet dienstes vil V Wann er ir beútet dienste vil W \textbf{23} vremden] werden W \textbf{24} eines meien] als ein W \textbf{26} Etnise] enise W \textbf{27} unversniten] Verschnitten W  $\cdot$ man truoc] trvͦg men V (W) \textbf{28} Tabrunite] Thabrvnite T [*]: Tabronitte V tabronit W \textbf{29} ze] \textit{om.} W  $\cdot$ Tribalibot] Tribalybot T \textbf{30} ane] In W  $\cdot$ Coucasas] [koͮkesa*]: koͮkesas V kankasas W \newline
\end{minipage}
\end{table}
\end{document}
