\documentclass[8pt,a4paper,notitlepage]{article}
\usepackage{fullpage}
\usepackage{ulem}
\usepackage{xltxtra}
\usepackage{datetime}
\renewcommand{\dateseparator}{.}
\dmyyyydate
\usepackage{fancyhdr}
\usepackage{ifthen}
\pagestyle{fancy}
\fancyhf{}
\renewcommand{\headrulewidth}{0pt}
\fancyfoot[L]{\ifthenelse{\value{page}=1}{\today, \currenttime{} Uhr}{}}
\begin{document}
\begin{table}[ht]
\begin{minipage}[t]{0.5\linewidth}
\small
\begin{center}*D
\end{center}
\begin{tabular}{rl}
\textbf{582} & dise, die und aber jene.\\ 
 & er \textbf{was êt} in der alten sene\\ 
 & nâch Orgelusen, der clâren,\\ 
 & wande im in sînen jâren\\ 
5 & \textbf{kein} wîp sô nâhe \textbf{nie} gegienc,\\ 
 & etswâ dâ er minne enpfienc\\ 
 & oder dâ im minne was versagt.\\ 
 & Dô sprach der helt unverzagt\\ 
 & zuo sîner meisterinne,\\ 
10 & der alten küneginne:\\ 
 & "vrouwe, ez krenket \textbf{mir} mîne zuht\\ 
 & - ir meget mirs jehen vür ungenuht -,\\ 
 & sulen \textbf{dise} vrouwen vor mir stên.\\ 
 & gebiet in, daz si sitzen gên,\\ 
15 & oder heizet si mit mir ezzen."\\ 
 & "Alhie wirt niht gesezzen\\ 
 & von ir \textbf{enkeiner} unz an mich.\\ 
 & hêrre, si m\textit{ö}hten schamen sich,\\ 
 & solten si iu niht dienen vil,\\ 
20 & wande ir sît unser vreuden zil.\\ 
 & doch, hêrre, swaz ir gebiet in,\\ 
 & daz \textbf{sulen si} leisten, hab \textbf{wir} sin."\\ 
 & Die edelen mit der hôhen art\\ 
 & wâren ir zühte des bewart,\\ 
25 & wande siz mit willen tâten.\\ 
 & ir süezen münde in bâten\\ 
 & dâ \textbf{stêns}, unz er geæze,\\ 
 & daz ir enkeiniu sæze.\\ 
 & dô daz geschach, si giengen wider.\\ 
30 & Gawan sich leite slâfen nider.\\ 
\end{tabular}
\scriptsize
\line(1,0){75} \newline
D Fr7 \newline
\line(1,0){75} \newline
\textbf{1} \textit{Initiale} Fr7  \textbf{8} \textit{Majuskel} D  \textbf{16} \textit{Majuskel} D  \textbf{23} \textit{Majuskel} D  \newline
\line(1,0){75} \newline
\textbf{1} dise die] Nise hie Fr7 \textbf{6} dâ] do Fr7 \textbf{7} versagt] versagete Fr7 \textbf{8} unverzagt] vnverzagete Fr7 \textbf{11} ez krenket mir] ir crenket Fr7 \textbf{17} unz] wan Fr7 \textbf{18} möhten] mohten D Fr7 \textbf{20} ir] \textit{om.} Fr7 \textbf{21} gebiet] gebieten Fr7 \textbf{22} hab wir] han si Fr7 \textbf{25} siz] sie Fr7 \textbf{26} süezen] sueze Fr7 \textbf{27} Da stende vntz daz er geeze Fr7 \textbf{30} slâfen] vnsanfte Fr7 \newline
\end{minipage}
\hspace{0.5cm}
\begin{minipage}[t]{0.5\linewidth}
\small
\begin{center}*m
\end{center}
\begin{tabular}{rl}
 & dise, die und ab jene.\\ 
 & er \textbf{wachet} in der alten sene\\ 
 & nâch Urgeluse, der clâren,\\ 
 & want im in sînen jâren\\ 
5 & \textbf{nie} wîp sô nâhe gegienc,\\ 
 & etwâ d\textit{â} er minne enpfienc\\ 
 & oder d\textit{â} im minne was versaget.\\ 
 & dô sprach der helt unverzaget\\ 
 & zuo sîner meisterîn,\\ 
10 & der alten künigîn:\\ 
 & "vrouwe, ez krenket mîn zuht\\ 
 & - ir müget mirs jehen vür un\textit{g}enuht -,\\ 
 & sullen \textbf{die} vrouwen vor mir stân.\\ 
 & gebiet \textit{i}n, daz si sitzen gân,\\ 
15 & oder heizet si mit mir ezzen."\\ 
 & "alhie wirt niht gesezzen\\ 
 & von ir \textbf{\textit{d}ekeiner} unz an mich.\\ 
 & hêrre, si m\textit{ö}hten schamen sich,\\ 
 & solte\textit{ns} iu niht dienen vil,\\ 
20 & wan ir sît unser vröuden zil.\\ 
 & doch, hêrre, waz ir gebietet in,\\ 
 & daz \textbf{solten wir} leisten, hâ\textit{n} \textbf{\textit{w}ir} sin."\\ 
 & die edeln mit der hôhe\textit{n} art\\ 
 & wâren ir zühte des bewart,\\ 
25 & wand si ez mit willen tâten.\\ 
 & ir süezen münde in bâten\\ 
 & d\textit{â} \textbf{stênde}, unz er geæze,\\ 
 & daz ir \textbf{ê} keiniu sæze.\\ 
 & dô daz geschach, si giengen wider.\\ 
30 & Gawan sich leit slâfen \textit{n}ider.\\ 
\end{tabular}
\scriptsize
\line(1,0){75} \newline
m n o \newline
\line(1,0){75} \newline
\newline
\line(1,0){75} \newline
\textbf{1} jene] ynne o \textbf{2} wachet] wachsset n [wach*]: wachset o \textbf{3} Urgeluse] vrgeluͯse o \textbf{6} etwâ] Etwan o  $\cdot$ dâ] do m n o \textbf{7} dâ] do m n o  $\cdot$ im] in o \textbf{12} ungenuht] vnenuht m \textbf{13} die] dise n o  $\cdot$ stân] [stat]: stan o \textbf{14} in] ein m \textbf{17} dekeiner] die keiner m do keiner n \textbf{18} möhten] mohtten m \textbf{19} soltens] Solttencz m \textbf{21} gebietet] gebieten n \textbf{22} solten wir] súllent sú n (o)  $\cdot$ hân wir] hant ir m \textbf{23} hôhen] hoher m \textbf{26} süezen] suͯsse n \textbf{27} dâ] Do m n o \textbf{28} ê keiniu] do keine n dekein o \textbf{30} leit] leite n o  $\cdot$ nider] mẏder m \newline
\end{minipage}
\end{table}
\newpage
\begin{table}[ht]
\begin{minipage}[t]{0.5\linewidth}
\small
\begin{center}*G
\end{center}
\begin{tabular}{rl}
 & dise, die unde aber jene.\\ 
 & er \textbf{was êt} in der alten sene\\ 
 & nâch Orgelusen, der clâren,\\ 
 & wande im in sînen jâren\\ 
5 & \textbf{nie} wîb sô nâhen \textbf{nie} gegienc,\\ 
 & etswâ dâ er minne enpfienc\\ 
 & ode dâ im minne was versaget.\\ 
 & dô sprach der helt unverzaget\\ 
 & ze sîner meisterinne,\\ 
10 & der alten küneginne:\\ 
 & "vrouwe, ez krenket \textbf{mir} mîn zuht\\ 
 & - ir muget mirs jehen vür ungenuht -,\\ 
 & suln \textbf{dise} vrouwen vor mir stên.\\ 
 & gebiet in, daz si sitzen gên,\\ 
15 & oder heizet si mit mir ezzen."\\ 
 & "al hie wirt niht gesezzen\\ 
 & von ir \textbf{deheiner} unze an mich.\\ 
 & hêrre, si m\textit{ö}hten schamen sich,\\ 
 & solden si iu niht dienen vil,\\ 
20 & wande ir sît unserre vröuden zil.\\ 
 & doch, hêrre, swaz ir gebiet in,\\ 
 & daz \textbf{suln wir} leisten, habe \textbf{wir} sin."\\ 
 & die edeln mit der hôhen art\\ 
 & wâren ir zühte des bewart,\\ 
25 & wande siz mit wille\textit{n} tâten.\\ 
 & ir süezen münde in bâten\\ 
 & dâ \textbf{stênes}, unze er geæze,\\ 
 & daz ir deheiniu sæze.\\ 
 & dô daz geschach, si giengen wider.\\ 
30 & Gawan sich leite slâfen nider.\\ 
\end{tabular}
\scriptsize
\line(1,0){75} \newline
G I L M Z Fr19 \newline
\line(1,0){75} \newline
\textbf{1} \textit{Initiale} L Z Fr19  \textbf{9} \textit{Initiale} I  \newline
\line(1,0){75} \newline
\textbf{3} Orgelusen] Orgulusen I Orgalisen L orgelosin M Orgilvsen Fr19 \textbf{5} Dehein wip nie so nahen gie L  $\cdot$ nie wîb] dehain wip I (Fr19)  $\cdot$ nie gegienc] Gienc I \textbf{6} er] er doch L  $\cdot$ minne] libe M  $\cdot$ enpfienc] pfie L \textbf{7} minne] libe M \textbf{8} dô] Da M \textbf{9} ze] Ee I \textbf{10} der] De L \textbf{11} mir] \textit{om.} L \textbf{12} mirs] mir sin I \textbf{13} dise] [di*]: dise G \textbf{14} in daz] sý da L \textbf{15} oder] olde G \textbf{16} gesezzen] [vergeszen]: geseszen L vorgesszin M (Fr19) \textbf{17} unze] sunz I \textbf{18} möhten] mohten G I (L) (M) Z Fr19 \textbf{20} unserre] vnsere G unser I L (M) (Fr19)  $\cdot$ zil] spil I \textbf{21} swaz] waz L (M) \textbf{22} suln wir] svln sie L (M) (Z) (Fr19)  $\cdot$ habe wir] habent sý L \textbf{23} der] ir Z  $\cdot$ hôhen] hohesten L \textbf{24} des] so M \textbf{25} siz] sie L  $\cdot$ willen] wille G \textbf{26} süezen] suͤzze Z (Fr19) \textbf{27} dâ] Do L  $\cdot$ stênes] stensh I \textbf{28} daz] Dar L \textbf{29} dô] Da M \textbf{30} leite slâfen] slafin leite M leite vnsanfe Z \newline
\end{minipage}
\hspace{0.5cm}
\begin{minipage}[t]{0.5\linewidth}
\small
\begin{center}*T
\end{center}
\begin{tabular}{rl}
 & Dise, die und aber jene.\\ 
 & er \textbf{was noch} in der alten sene\\ 
 & nâch Orgelusen, der klâren,\\ 
 & wan im in sînen jâren\\ 
5 & \textbf{kein} wîp sô nâhen \textbf{nie} gegienc,\\ 
 & etswan dô er minne enpfienc\\ 
 & oder dô im minne was versagt.\\ 
 & dô sprach der helt unverzagt\\ 
 & zuo sîner meisterinne,\\ 
10 & der alten küniginne:\\ 
 & "vrou, ez krenket mîne zuht\\ 
 & - ir mogt mirs jehen vür ungenuht -,\\ 
 & sollen \textbf{dise} vrouwen vor mir stên.\\ 
 & gebiet in, daz si sitzen gên,\\ 
15 & oder heizet si mit mir ezzen."\\ 
 & "alhie wirt niht \textit{ges}ezzen\\ 
 & vo\textit{n} ir \textbf{einer} unz an mich.\\ 
 & hêrre, si m\textit{ö}hten schamen sich,\\ 
 & soldens iu niht dienen vil,\\ 
20 & wan ir sît unser vreuden zil.\\ 
 & doch, hêrre, waz ir gebietet in,\\ 
 & daz \textbf{sollen si} leisten, haben \textbf{si} sin."\\ 
 & die edelen mit der hôhen art\\ 
 & wâren ir zuht des bewart,\\ 
25 & wa\textit{n} siz mit willen tâten.\\ 
 & ir süezen münde in bâten\\ 
 & dâ \textbf{stênes}, unz er geæze,\\ 
 & daz ir keiniu sæze.\\ 
 & dô daz geschach, si giengen wider.\\ 
30 & Gawan sich legte slâfen nider.\\ 
\end{tabular}
\scriptsize
\line(1,0){75} \newline
Q R W V U \newline
\line(1,0){75} \newline
\textbf{1} \textit{Initiale} Q   $\cdot$ \textit{Capitulumzeichen} R  \newline
\line(1,0){75} \newline
\textbf{1} \textit{Die Verse 553.1-599.30 fehlen} U  \textbf{2} noch] recht R echt W (V) \textbf{3} Orgelusen] orgulusen R \textbf{5} gegienc] ergieng W \textbf{6} dô] da V \textbf{7} dô] da R V \textbf{11} mîne] mir meine W \textbf{12} mirs] mir R  $\cdot$ ungenuht] vnzucht R \textbf{15} ezzen] esse W \textbf{16} gesezzen] vergessen Q \textbf{17} von] Vor Q \textbf{18} möhten] mochten Q  $\cdot$ schamen] schawen W \textbf{21} waz] swaz V  $\cdot$ gebietet] gebietten R \textbf{22} si leisten] wir laisten W  $\cdot$ si] wir W [*]: wir V \textbf{24} ir zuht] mit zúchten W \textbf{25} wan] Wans Q \textbf{26} süezen] susse Q (R) \textbf{27} dâ] Do W Des R \textbf{28} keiniu] deheine R \textbf{30} Gawan] Gawin R  $\cdot$ sich legte] leite sich V \newline
\end{minipage}
\end{table}
\end{document}
