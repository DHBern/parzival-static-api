\documentclass[8pt,a4paper,notitlepage]{article}
\usepackage{fullpage}
\usepackage{ulem}
\usepackage{xltxtra}
\usepackage{datetime}
\renewcommand{\dateseparator}{.}
\dmyyyydate
\usepackage{fancyhdr}
\usepackage{ifthen}
\pagestyle{fancy}
\fancyhf{}
\renewcommand{\headrulewidth}{0pt}
\fancyfoot[L]{\ifthenelse{\value{page}=1}{\today, \currenttime{} Uhr}{}}
\begin{document}
\begin{table}[ht]
\begin{minipage}[t]{0.5\linewidth}
\small
\begin{center}*D
\end{center}
\begin{tabular}{rl}
\textbf{356} & \begin{large}\textbf{Ê}\end{large} daz wir uns von zinnen wern\\ 
 & Meljanzes bêden hern.\\ 
 & ez sint \textbf{doch allez meistec} kint,\\ 
 & die mit dem künege \textbf{dâ} komen sint.\\ 
5 & \textbf{dâ erwerbe wir} \textbf{vil} lîhte ein pfant,\\ 
 & dâ von ie grôzer zorn verswant.\\ 
 & der künec ist lîhte \textbf{alsô} gemuot,\\ 
 & \textbf{swenne}r \textbf{hie} ritterschaft getuot,\\ 
 & er \textbf{sol} uns \textbf{nôt} erlâzen\\ 
10 & unt al sîn zürnen mâzen.\\ 
 & veltstrîtes \textbf{sol} uns \textbf{doch} baz gezemen,\\ 
 & \textbf{den} \textbf{daz} si uns ûz der mûre nemen.\\ 
 & wir solten wol gedingen,\\ 
 & dort in ir \textbf{snüeren} ringen,\\ 
15 & wan Poydiconjunzes kraft,\\ 
 & der vüert die herten ritterschaft.\\ 
 & \textbf{dâ ist} unser \textbf{grœster} vreise,\\ 
 & die gevangen Berteneise.\\ 
 & der pfligt der herzoge Astor.\\ 
20 & den siht man \textbf{hie} \textbf{gegen} strîte vor.\\ 
 & dâ ist ouch sîn sun Melyacanz.\\ 
 & het \textbf{den erzogen} Gurnamanz,\\ 
 & sô wære sîn prîs gehœhet gar,\\ 
 & \textbf{doch siht man in} in strîtes schar.\\ 
25 & Dâ engegen ist uns grôz helfe komen."\\ 
 & ir habt ir râten wol vernomen.\\ 
 & der vürste tet, als man im riet:\\ 
 & die mûre \textbf{er} ûzen porten \textbf{schiet}.\\ 
 & die burgære ellens unbetrogen\\ 
30 & begunden ûz ze velde zogen:\\ 
\end{tabular}
\scriptsize
\line(1,0){75} \newline
D \newline
\line(1,0){75} \newline
\textbf{1} \textit{Initiale} D  \textbf{25} \textit{Majuskel} D  \newline
\line(1,0){75} \newline
\textbf{2} Meljanzes] Melianzes D \textbf{15} Poydiconjunzes] Poydiconivnzs D \textbf{20} gegen strîte] gegenstrite D \newline
\end{minipage}
\hspace{0.5cm}
\begin{minipage}[t]{0.5\linewidth}
\small
\begin{center}*m
\end{center}
\begin{tabular}{rl}
 & \textbf{ê} daz \textit{w}ir uns von zinnen wern\\ 
 & Mel\textit{i}anzes beiden hern,\\ 
 & \textbf{wand} ez sint \textbf{d\textit{az} mêrteil alliu} kint,\\ 
 & die mit dem künige komen sint,\\ 
5 & \textbf{daz wir erwerben} lîht\textit{e} \textit{e}in pfant,\\ 
 & d\textit{â} von i\textit{e} grôze\textit{r} zorn verswant.\\ 
 & der künic ist lîhte \textbf{s\textit{ô}} \textit{g}emuot,\\ 
 & \textbf{wenne} er \textbf{hie} ritterschaft g\textit{et}uot,\\ 
 & er \textbf{sol} uns \textbf{nôt} erlâzen\\ 
10 & und allez sîn zürne\textit{n m}âzen.\\ 
 & veltstrîtes \textbf{sol} uns \textbf{doch} baz gezemen,\\ 
 & \textbf{danne} \textbf{daz} si uns ûz der mûren nemen.\\ 
 & wir solten wol gedingen,\\ 
 & dort in ir \textbf{snüeren} ringen,\\ 
15 & wanne Poidic\textit{o}niunz\textit{es} kraft,\\ 
 & der \textit{v}üert die herten ritterschaft.\\ 
 & \textbf{dâ ist} unser \textbf{grôziu} vreise,\\ 
 & die gevangenen Brituneise.\\ 
 & der pfligt der herzoge Astor.\\ 
20 & den siht man \textbf{hie} \textbf{gegen} strîte vor.\\ 
 & dâ ist ouch sîn s\textit{u}n Meliaganz.\\ 
 & hete \textbf{den erzogen} Gurnemanz,\\ 
 & sô wære sîn prîs gehœhet gar,\\ 
 & \textbf{doch siht man in} in strîtes schar.\\ 
25 & dâ gegen ist uns \textbf{doch} grôziu helfe komen."\\ 
 & ir habt ir râten wol vernomen.\\ 
 & \begin{large}D\end{large}er vürste tet, als man im riet:\\ 
 & die mûre \textbf{er} ûzen porten \textbf{schiet}.\\ 
 & die burgære ellens unbetrogen\\ 
30 & begunden ûz ze velde zogen.\\ 
\end{tabular}
\scriptsize
\line(1,0){75} \newline
m n o \newline
\line(1,0){75} \newline
\textbf{27} \textit{Initiale} m n  \newline
\line(1,0){75} \newline
\textbf{1} \textit{Versfolge 356.11-26 (¹m), 355.17-22 (¹m) (Bl. 228v), Versdoppelung 355.17-22 (²m), dann 355.23-356.10 (Bl. 229r), Versdoppelung 356.11-14 (²m), dann 356.27-357.14 (Bl. 229v)} m   $\cdot$ wir] ir m  $\cdot$ zinnen] zuͦmen o \textbf{2} Melianzes] Melanzes m Meliantzes n Melianczes o  $\cdot$ beiden] beide n o \textbf{3} daz] den m  $\cdot$ alliu] \textit{om.} n o \textbf{5} lîhte ein] lihte pin ein m vil licht eyn o \textbf{6} dâ] Do m n o  $\cdot$ ie grôzer] ir grosse m \textbf{7} sô gemuot] so genuͯg vnd gemuͯt m \textbf{8} \textit{Versdoppelung nach 356.6 (mit Anteil aus Vers 356.7):} Der konig hie ritterschaft getuͦt o   $\cdot$ getuot] guͯt m \textbf{10} allez] alle o  $\cdot$ zürnen mâzen] [zunen]: zurnen lassen vnd massen m \textbf{12} danne] Wenne n (o)  $\cdot$ ûz] vff n (o) \textbf{14} ir snüeren] der snuͯre n (o) \textbf{15} Poidiconiunzes] poidicuniunz m poidicomintz n paidocomuͯncz o \textbf{16} vüert] snuret m  $\cdot$ herten] herte n o \textbf{17} dâ] Do n o \textbf{18} Brituneise] brittuneise m britaneise o \textbf{20} vor] fúr o \textbf{21} dâ] Do n o  $\cdot$ sîn] myn o  $\cdot$ sun] sÿn m  $\cdot$ Meliaganz] meliacancz m melracantz n melracancz o \textbf{22} erzogen] [h*]: ierczogen o  $\cdot$ Gurnemanz] Gurnemancz m (o) gurnemantz n \textbf{28} ûzen] úsz dem o \textbf{30} velde] helffe n \newline
\end{minipage}
\end{table}
\newpage
\begin{table}[ht]
\begin{minipage}[t]{0.5\linewidth}
\small
\begin{center}*G
\end{center}
\begin{tabular}{rl}
 & daz wir uns von zinnen weren\\ 
 & Melianzes beiden heren.\\ 
 & ez sint \textbf{doch allez meiste} kint,\\ 
 & die mit dem künige \textbf{dâ} komen sint.\\ 
5 & \textbf{dâ erwerben wir} \textbf{vil} lîhte \textit{ein} pfant,\\ 
 & dâ von ie grôzer zorn verswant.\\ 
 & der künic ist lîhte \textbf{sô} gemuot,\\ 
 & \textbf{sô} er \textbf{hie} rîterschaft getuot,\\ 
 & er \textbf{sol} uns \textbf{nôt} erlâzen\\ 
10 & unde al sîn zürnen mâzen.\\ 
 & veltstrîtes \textbf{sol} uns baz gezemen,\\ 
 & \textbf{dane} si uns ûz der \textit{mûre} nemen.\\ 
 & wir solten wol gedingen,\\ 
 & dort in ir \textbf{snüeren} ringen,\\ 
15 & wan Poydekoniunzes kraft,\\ 
 & der vüert die herten rîterschaft.\\ 
 & \textbf{deist} unser \textbf{grœzestiu} vreise:\\ 
 & die gevangen Britaneise.\\ 
 & der pfliget der herzoge Astor.\\ 
20 & den siht man \textbf{dâ} \textbf{in} strîte vor.\\ 
 & dâ ist ouch sîn sun Meliahganz.\\ 
 & hete \textbf{den erzogen} Gurnomanz,\\ 
 & sô wære sîn prîs gehœhet gar,\\ 
 & \textbf{doch siht man in} in strîtes schar.\\ 
25 & dâ engegene ist uns grôz helfe komen."\\ 
 & ir habet ir râten wol vernomen.\\ 
 & der vürste tet, als man im riet:\\ 
 & die mûre \textit{\textbf{er}} ûz den borten \textbf{schriet}.\\ 
 & die burgære ellens unbetrogen\\ 
30 & begunden ûz ze velde \textit{zog}en:\\ 
\end{tabular}
\scriptsize
\line(1,0){75} \newline
G I O L M Q R Z Fr39 \newline
\line(1,0){75} \newline
\textbf{1} \textit{Initiale} O L Fr39  \textbf{3} \textit{Initiale} M   $\cdot$ \textit{Capitulumzeichen} R  \textbf{5} \textit{Initiale} Q  \textbf{13} \textit{Initiale} I  \textbf{27} \textit{Initiale} I  \newline
\line(1,0){75} \newline
\textbf{1} daz] ÷ daz O E Daz L (Q) (R) [Daz]: E daz  Z E Fr39  $\cdot$ von zinnen] vorn zennen L  $\cdot$ weren] erwern L (Fr39) \textbf{2} Melianzes] Melyanzes Q  $\cdot$ beiden] boden L \textbf{3} doch] \textit{om.} Z  $\cdot$ allez] alle O aller L Fr39  $\cdot$ meiste] maistel I meistig R \textbf{4} die] Die do Q  $\cdot$ dâ] \textit{om.} I O M Q Z do Fr39 \textbf{5} wir werben lihte da ein phant I  $\cdot$ dâ] Do Q Fr39  $\cdot$ erwerben] irwerbe M (Q) (Z) (Fr39)  $\cdot$ wir] mir Q  $\cdot$ vil] \textit{om.} R  $\cdot$ ein] \textit{om.} G \textbf{6} von] vor Q  $\cdot$ ie] ir I M \textbf{7} sô] also L M Q R Z Fr39 \textbf{8} sô] swenn I (O) (Z) (Fr39) Wenne L (M) (Q) (R)  $\cdot$ getuot] thud M (R) \textbf{10} al] allen I alle O Z  $\cdot$ sîn zürnen] sinem zorn I \textbf{11} veltstrîtes] velt strit I (L) (Fr39) Velt striten M Welt streites Q  $\cdot$ sol] solde O mocht Q R  $\cdot$ uns] vnz doch L (M) (Z) (Fr39) vns docht Q \textbf{12} dane] Danne daz O L (M) (Q) (R) Z (Fr39)  $\cdot$ si] \textit{om.} Q Z  $\cdot$ ûz] vser R  $\cdot$ mûre] \textit{om.} G mvren L (M) \textbf{13} solten] suln M \textbf{14} in] mit L  $\cdot$ snüeren] snuͤre I (M) (R) \textbf{15} Poydekoniunzes] poidekonivnzes G poydecomunzes I Poydekvmvnzes O Poý de Konivnzes L poide kuͯniunzes M poydekomvn͑z Q poẏdekomvnres R :oy de konyvnzes Fr39 \textbf{16} der] \textit{om.} I  $\cdot$ vüert] fuͯrte L (Fr39)  $\cdot$ herten] grozen I herte Q \textbf{17} deist] Do ist Q  $\cdot$ grœzestiu] grozev I grosser Q groͯste R \textbf{18} Britaneise] pritoneise I briteneise O Q Brittaneise L britoneise R brituneise Z \textbf{19} der] dy M  $\cdot$ herzoge] hetzoge Q  $\cdot$ Astor] astar M castor Q R \textbf{20} siht] sich Q sist Fr39  $\cdot$ dâ] do Q Fr39 \textbf{21} dâ] Do Q Fr39  $\cdot$ Meliahganz] Meliaganz I Melyacanz O Meliahkanz L Z Fr39 Meliachkanz M meliahkantz Q Meliahkancz R \textbf{22} hete] Het >er< O Het er L (Fr39)  $\cdot$ erzogen] herzogen O (L) (M) (R) Fr39  $\cdot$ Gurnomanz] kurnomanz G Gurnemanz I (M) (Z) (Fr39) [*vrnæmanz]: Gvrnæmanz  O Gvrnomantz L gyrnomanz Q Gurnomancz R \textbf{24} doch] Da O  $\cdot$ in in] in O (L) M (R) (Fr39) \textbf{25} uns] \textit{om.} I  $\cdot$ grôz helfe] gruͦz Fr39 \textbf{26} râten] ir rat I \textbf{27} im] in hiez O \textbf{28} die] Div O  $\cdot$ mûre] Muren M (R)  $\cdot$ er] \textit{om.} G  $\cdot$ borten] phorten M (Q) toren R  $\cdot$ schriet] schiet O L (M) (Q) R Z Fr39 \textbf{29} ellens] inellens O elles Q \textbf{30} begunden] begunde Q  $\cdot$ ze velde] gein velde I gefelde Q  $\cdot$ zogen] chomen G \newline
\end{minipage}
\hspace{0.5cm}
\begin{minipage}[t]{0.5\linewidth}
\small
\begin{center}*T
\end{center}
\begin{tabular}{rl}
 & \textbf{baz}, \textbf{ê} daz wir uns von zinnen wern\\ 
 & Melyanzes beiden hern.\\ 
 & \hspace*{-.7em}\big| der künec ist lîhte \textbf{sô} gemuot,\\ 
 & \hspace*{-.7em}\big| \textbf{swenne}r \textbf{die} rîterschaft getuot,\\ 
 & \hspace*{-.7em}\big| er \textbf{solte} uns \textbf{nœte} erlâzen\\ 
10 & \hspace*{-.7em}\big| unde alsîn zürnen mâzen.\\ 
 & \hspace*{-.7em}\big| ez sint \textbf{ouch almeistic} kint,\\ 
 & \hspace*{-.7em}\big| die mit dem künege komen sint.\\ 
5 & \hspace*{-.7em}\big| \textbf{dâ erwerbe wir} \textbf{vil} lîhte ein pfant,\\ 
 & \hspace*{-.7em}\big| dâ von i\textit{e} grôzer zorn verswant.\\ 
 & veltstrîtes \textbf{mac} uns \textbf{doch} baz gezemen,\\ 
 & \textbf{ê} \textbf{daz} si uns ûz der mûre nemen.\\ 
 & wir solten wol gedingen,\\ 
 & dort in ir \textbf{snüer} ringen,\\ 
15 & wan Poydekuniunzes kraft,\\ 
 & der vüeret die herten rîterschaft.\\ 
 & \textbf{daz ist} unser \textbf{grœst\textit{iu}} vreise:\\ 
 & die gevangen Brituneise.\\ 
 & der pfliget der herzoge Astor.\\ 
20 & den siht man \textbf{in} strîte vor.\\ 
 & dâ ist ouch sîn sun Melyahganz.\\ 
 & het \textbf{er den herzogen} Gurnemanz,\\ 
 & sô wære sîn prîs gehœhet gar.\\ 
 & \textbf{er ist ouch dâ} in strîtes schar.\\ 
25 & dâ engegene ist uns grôz\textit{iu} helfe komen."\\ 
 & ir habt ir râten wol vernomen.\\ 
 & \begin{large}D\end{large}er vürste tet, alse man im riet:\\ 
 & die mûren \textbf{man} ûz den porten \textbf{schiet}.\\ 
 & die burgære ellens unbetrogen\\ 
30 & begunden ûz ze velde zogen.\\ 
\end{tabular}
\scriptsize
\line(1,0){75} \newline
T V W \newline
\line(1,0){75} \newline
\textbf{7} \textit{Initiale} W  \textbf{27} \textit{Initiale} T  \newline
\line(1,0){75} \newline
\textbf{1} baz] \textit{om.} V W \textbf{2} Melyanzes] Melianzes V Melians W \textbf{7} lîhte] lechte W \textbf{8} swenner] Wenn er W  $\cdot$ die] hie V \textbf{9} solte] sol V \textbf{3} almeistic] almechtig W \textbf{4} mit dem] \textit{om.} W \textbf{5} dâ erwerbe wir] Daz wir erwerben V An den erwerbe wir W  $\cdot$ vil] \textit{om.} V \textbf{6} von ie] von ir T vor in W \textbf{11} veltstrîtes] Zuͦ veld streit W  $\cdot$ doch] \textit{om.} W \textbf{12} ê daz si] Danne daz V Ee sy W  $\cdot$ mûre] muren V \textbf{15} wan] Von W  $\cdot$ Poydekuniunzes] poydekvmvnzes V poyde gumunzes W \textbf{17} grœstiu] groste T groͤster W \textbf{18} Brituneise] brittvneise V brituneyse W \textbf{20} in strîte] hie gegen strite V in streiten W \textbf{21} dâ] Do V W  $\cdot$ Melyahganz] meliaganz V meliagantz W \textbf{22} het er den herzogen] Den hat erzogen W  $\cdot$ Gurnemanz] gurnemantz W \textbf{24} er ist ouch dâ] Doch siht man in V Er ist auch do W \textbf{25} engegene] gegen W  $\cdot$ grôziu] groze T \textbf{26} wol] \textit{om.} W \textbf{28} mûren man] muren er V maure er W \newline
\end{minipage}
\end{table}
\end{document}
