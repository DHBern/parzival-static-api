\documentclass[8pt,a4paper,notitlepage]{article}
\usepackage{fullpage}
\usepackage{ulem}
\usepackage{xltxtra}
\usepackage{datetime}
\renewcommand{\dateseparator}{.}
\dmyyyydate
\usepackage{fancyhdr}
\usepackage{ifthen}
\pagestyle{fancy}
\fancyhf{}
\renewcommand{\headrulewidth}{0pt}
\fancyfoot[L]{\ifthenelse{\value{page}=1}{\today, \currenttime{} Uhr}{}}
\begin{document}
\begin{table}[ht]
\begin{minipage}[t]{0.5\linewidth}
\small
\begin{center}*D
\end{center}
\begin{tabular}{rl}
\textbf{454} & \textbf{\textit{\begin{large}E\end{large}}r} was ein \textit{heiden} vaterhalp,\\ 
 & Flegetanis, der an ein kalp\\ 
 & bette, als \textbf{ob} ez wære \textbf{sîn} got.\\ 
 & wie mac der \textbf{tievel} \textbf{sölhen} spot\\ 
5 & gevüegen an sô wîser diet,\\ 
 & daz si niht scheidet ode schiet\\ 
 & dâ von, der treit die hœhsten hant\\ 
 & unt dem elliu wunder sint bekant?\\ 
 & Flegetanis der heiden\\ 
10 & kunde \textbf{uns} wol bescheiden\\ 
 & ieslîches sternen hinganc\\ 
 & unt sîner \textbf{künfte} widerwanc,\\ 
 & wie lange ieslîcher \textbf{umbe} gêt,\\ 
 & ê er wider an sîn zil gestêt.\\ 
15 & mit der sternen umbereise vart\\ 
 & ist gep\textit{r}üefe\textit{t} \textbf{aller} \textbf{menschlîcher} art.\\ 
 & Flegetanis der heiden sach,\\ 
 & dâ von er \textbf{blûweclîche} sprach,\\ 
 & inme gestirne mit sînen ougen\\ 
20 & verholnebæriu tougen.\\ 
 & er jach, ez \textbf{hiez} ein dinc der Grâl.\\ 
 & des namen las er sunder twâl\\ 
 & inme gestirne, wie \textbf{der} hiez.\\ 
 & "ein schar in ûf der erden liez,\\ 
25 & diu vuor ûf über die sterne hôch,\\ 
 & ob die ir unschult wider zôch,\\ 
 & sît muoz sîn pflegen getouftiu vruht\\ 
 & mit alsô kiuschlîcher zuht.\\ 
 & diu menscheit ist immer wert,\\ 
30 & \textbf{der} zuo dem Grâle wirt \textbf{gegert}."\\ 
\end{tabular}
\scriptsize
\line(1,0){75} \newline
D \newline
\line(1,0){75} \newline
\textbf{1} \textit{Initiale} D  \newline
\line(1,0){75} \newline
\textbf{1} Er] ÷r D  $\cdot$ heiden] \textit{om.} D \textbf{16} geprüefet] gepvͤfel D \newline
\end{minipage}
\hspace{0.5cm}
\begin{minipage}[t]{0.5\linewidth}
\small
\begin{center}*m
\end{center}
\begin{tabular}{rl}
 & \textbf{er} was ein heiden vaterhalp,\\ 
 & \textit{F}legetanis, der an ein kalp\\ 
 & bette, als \textbf{ob} ez wær \textbf{ein} got.\\ 
 & wie mac der \textbf{tiuvel} \textbf{solichen} spot\\ 
5 & gevüegen a\textit{n} \textit{s}ô wîser diet,\\ 
 & daz si niht scheidet oder schiet\\ 
 & dâ von, der treit die hœhesten hant\\ 
 & und de\textit{m} alliu wunder sint bekant?\\ 
 & Flegetanis der heiden\\ 
10 & kunde wol bescheiden\\ 
 & ieglîches sternen hinganc\\ 
 & und sîner \textbf{künste} widerwanc,\\ 
 & wie lange ieglîcher \textbf{umbe} gât,\\ 
 & ê er wider an sîn zil gestât.\\ 
15 & mit der sternen umbereise vart\\ 
 & ist gebrüefet \textbf{ieglîches} \textbf{menschen} art.\\ 
 & Flegetanis der heiden sach,\\ 
 & dâ von er \textbf{blœdeclîchen} sprach,\\ 
 & in dem gestirn mit sînen ougen\\ 
20 & verholnbæriu tougen.\\ 
 & er jach, ez \textbf{hiez} ein dinc der Grâl.\\ 
 & des name\textit{n l}as er sunder twâl\\ 
 & in dem gestirn, wie \textbf{er} hiez.\\ 
 & "ein schar in ûf der erden liez,\\ 
25 & diu vuor ûf über die sternen hôch,\\ 
 & ob die ir unschuld wider zôch,\\ 
 & sît muoz sîn pflegen getouftiu vruht\\ 
 & mit alsô kiuschlîcher zuht,\\ 
 & \textbf{daz} diu menscheit ist iemer wert,\\ 
30 & \textbf{der} zuo dem Grâl wirt \textbf{begert}."\\ 
\end{tabular}
\scriptsize
\line(1,0){75} \newline
m n o \newline
\line(1,0){75} \newline
\newline
\line(1,0){75} \newline
\textbf{1} \textit{Versfolge 454.2-1} n  \textbf{2} Flegetanis] Pflegetanis m \textbf{3} ein] sin n o \textbf{4} wie] [Wa]: Wie m \textbf{5} an sô] an do so m  $\cdot$ wîser] wisen o \textbf{7} hœhesten] hoheste m n o \textbf{8} dem] den m  $\cdot$ wunder] wunden o \textbf{9} Flegetanis] Fletanis o \textbf{14} ê] [Er]: E m Er o \textbf{15} sternen] sterne o  $\cdot$ umbereise] vmb [*]: reisse m vmb kreisz n \textbf{16} ieglîches] [iglichen]: igliches o \textbf{17} heiden] meister n \textbf{19} \textit{Versdoppelung 454.19-21 (²o) nach 454.22; Lesarten des vorausgehenden Verses mit ¹o bezeichnet} o   $\cdot$ gestirn] gostirm \textsuperscript{1}\hspace{-1.3mm} o \textbf{20} verholnbæriu] Werholn berge o \textbf{21} der] ein n \textbf{22} namen las] namen iach vnd las m \textbf{28} kiuschlîcher] kusclichen o \newline
\end{minipage}
\end{table}
\newpage
\begin{table}[ht]
\begin{minipage}[t]{0.5\linewidth}
\small
\begin{center}*G
\end{center}
\begin{tabular}{rl}
 & \textbf{\begin{large}E\end{large}z} was ein heiden vaterhalp,\\ 
 & Fleigetanis, der an ein kalp\\ 
 & bette, als ez wære \textbf{sîn} got.\\ 
 & wie mac der \textbf{tievel} \textbf{sînen} spot\\ 
5 & gevüegen an sô wîser diet,\\ 
 & daz si niht scheidet oder schiet\\ 
 & dâ von, der treit die hœhesten hant\\ 
 & unt dem elliu wunder sint bekant?\\ 
 & Fleigetanis der heiden\\ 
10 & kunde \textbf{uns} wol bescheiden\\ 
 & iegelîches sternes hinganc\\ 
 & unt sîne\textit{r} \textbf{künste} widerwanc,\\ 
 & wie lange iegeslîcher \textbf{umbe} gêt,\\ 
 & ê er wider an sîn zil gestêt.\\ 
15 & mit der sternen umbereise vart\\ 
 & ist geprüevet \textbf{aller} \textbf{menschen} art.\\ 
 & Fleigetanis der heiden sach,\\ 
 & dâ von er \textbf{blûclîchen} sprach,\\ 
 & i\textit{n} de\textit{m} gestirne mit sînen ougen\\ 
20 & verholnbæriu tougen.\\ 
 & er jach, ez \textbf{wære} ein dinc, der Grâl.\\ 
 & des namen las er sunder twâl\\ 
 & inme gestirne, wie \textbf{der} hiez.\\ 
 & "ein schar in ûf der erden liez,\\ 
25 & diu vuor ûf über die sternen hôch,\\ 
 & op die ir unschult wider zôch,\\ 
 & sît muoz sîn pflegen getouftiu vruht\\ 
 & mit alsô kiuschlîcher zuht.\\ 
 & diu menscheit ist immer wert,\\ 
30 & \textbf{der} zuo dem Grâl wirt \textbf{gegert}."\\ 
\end{tabular}
\scriptsize
\line(1,0){75} \newline
G I O L M Z \newline
\line(1,0){75} \newline
\textbf{1} \textit{Initiale} G I O L M Z  \textbf{17} \textit{Initiale} I  \newline
\line(1,0){75} \newline
\textbf{1} Ez] ÷r O ER L (Z) Dr M \textbf{2} Fleigetanis] flegitamus I Flegetanis O L M Z  $\cdot$ der] dar M \textbf{3} als] als ob O (M) Z  $\cdot$ sîn] \textit{om.} L eyn M \textbf{4} sînen] solhen O (L) \textbf{5} gevüegen] Gefvͤget Z  $\cdot$ wîser] werder L \textbf{7} der] er M \textbf{8} sint] sin Z \textbf{9} Fleigetanis] flegitamus I Flegetanis O L M Z \textbf{11} sternes] sternen L (M) Z  $\cdot$ hinganc] yn gangk M \textbf{12} sîner] sinen G  $\cdot$ künste] [kuste]: kunste I kunffte M  $\cdot$ widerwanc] widerswanc I widervanc Z \textbf{13} lange] \textit{om.} I  $\cdot$ iegeslîcher] angeslich er I \textbf{14} ê] Er M \textbf{15} sternen] stern I O (M) sterne L  $\cdot$ umbereise] vmmb M \textbf{16} aller] alle M  $\cdot$ menschen] menslicher O (L) (M) (Z) \textbf{17} Fleigetanis] [Fleigetanisis]: Fleigetanis G Flegitamus I Flegetanis O L M Z \textbf{18} er] \textit{om.} L  $\cdot$ blûclîchen] tugistlichen M \textbf{19} in dem] Im den G Andem O  $\cdot$ gestirne] [gestirie]: gestirne G \textbf{21} jach] sprach M  $\cdot$ wære] hiez Z  $\cdot$ dinc] \textit{om.} M \textbf{22} las] [laz]: erlaz L \textbf{23} inme] an dem I  $\cdot$ der] er Z \textbf{24} ein] Er Z  $\cdot$ erden] erde L \textbf{25} ûf] \textit{om.} O  $\cdot$ über] vur I  $\cdot$ sternen] stern I Z sterne O (M) \textbf{26} ir] \textit{om.} O \textbf{27} muoz] must I (O) (M) \textbf{28} kiuschlîcher] kuͯscher L \textbf{29} menscheit] [mennischet]: mennischeit G \textbf{30} wirt] ist I \newline
\end{minipage}
\hspace{0.5cm}
\begin{minipage}[t]{0.5\linewidth}
\small
\begin{center}*T
\end{center}
\begin{tabular}{rl}
 & \textbf{er} was ein heiden vaterhalp,\\ 
 & Flegetanis, der an ein kalp\\ 
 & bette, alse \textbf{ob}z wære \textbf{sîn} got.\\ 
 & wie mac der \textbf{vâlant} \textbf{sînen} spot\\ 
5 & gevüegen an sô wîser diet,\\ 
 & daz si niht scheidet oder schiet\\ 
 & dâ von, der treget die hœheste hant\\ 
 & unde dem alliu wunder sint bekant?\\ 
 & Flegetanis der heiden\\ 
10 & kunde \textbf{uns} wol bescheiden\\ 
 & etslîches sternen hineganc\\ 
 & unde sîner \textbf{künfte} widerwanc,\\ 
 & wie lange etslîcher \textbf{hine} gêt,\\ 
 & ê er wider an sîn zil gestêt.\\ 
15 & mit der sterne umbereise vart\\ 
 & ist geprüevet \textbf{aller} \textbf{menschlîch} art.\\ 
 & Flegetanis der heiden sach,\\ 
 & dâ von e\textit{r} \textbf{blûclîchen} sprach,\\ 
 & in dem gestirne mit sînen ougen\\ 
20 & verholnbæriu tougen.\\ 
 & er jach, ez \textbf{wære} ein dinc, der Grâl.\\ 
 & des namen las er sunder twâl\\ 
 & in dem gestirne, wie \textbf{der} hiez.\\ 
 & "ein schar in ûf der erden liez,\\ 
25 & di\textit{u} vuor ûf über die sterren hôch,\\ 
 & ob die ir unschult wider zôch,\\ 
 & sît muoz sîn pflegen getouftiu vruht\\ 
 & mit alsô kiuschlîcher zuht.\\ 
 & diu menscheit ist iemer wert,\\ 
30 & \textbf{diu} zem Grâle wirt \textbf{gewert}."\\ 
\end{tabular}
\scriptsize
\line(1,0){75} \newline
T U V W Q R \newline
\line(1,0){75} \newline
\textbf{1} \textit{Initiale} Q   $\cdot$ \textit{Capitulumzeichen} R  \newline
\line(1,0){75} \newline
\textbf{1} \textit{Die Verse 453.1-502.30 fehlen} U  \textbf{2} Flegetanis] Flegetonis V Elegetanis W Flegentans Q \textbf{3} obz wære] wer es W \textbf{4} vâlant] tv́vel V (W) (Q) (R)  $\cdot$ sînen] solchen W (Q) (R) \textbf{5} gevüegen] [*gen]: gevuegen T  $\cdot$ wîser] weisen Q \textbf{6} si] sich Q \textbf{7} treget] do treit V treg W  $\cdot$ hœheste] hohesten V (W) (Q) (R) \textbf{8} sint] sein Q \textbf{9} Flegetanis] Flegentanis Q \textbf{11} etslîches] [*]: JEcliches V  $\cdot$ sternen] sternes Q \textbf{12} künfte] [kv́n*e]: kv́nfte V kúnste W (R) \textbf{13} hine] vmbe V W Q R \textbf{15} sterne] sternen V R  $\cdot$ umbereise] vmbkraiße W  $\cdot$ vart] ward >fart< R \textbf{16} aller menschlîch art] [*]: iecliches menschen art V aller menschen art R \textbf{17} Flegetanis] Flegatanis R \textbf{18} er] ê T  $\cdot$ blûclîchen] blv́deklichen V buchlichen Q \textbf{20} verholnbæriu] [Verholneber*]: Verholneberne V Verholúbaͯre R \textbf{21} jach] sprach W  $\cdot$ ez] er Q \textbf{22} namen] namme V \textbf{23} der] er V das W \textbf{25} diu] die T  $\cdot$ ûf] \textit{om.} W  $\cdot$ sterren] sterne W Q \textbf{27} muoz] muͦst R  $\cdot$ getouftiu] getrv́we V \textbf{28} kiuschlîcher] kv́scher V kúnstlicher R \textbf{29} diu menscheit] [*]: Daz die moͤnscheit V \textbf{30} diu zem] Der zvͦ V (W) (Q) (R)  $\cdot$ gewert] gegert V W Q R \newline
\end{minipage}
\end{table}
\end{document}
