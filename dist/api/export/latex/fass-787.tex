\documentclass[8pt,a4paper,notitlepage]{article}
\usepackage{fullpage}
\usepackage{ulem}
\usepackage{xltxtra}
\usepackage{datetime}
\renewcommand{\dateseparator}{.}
\dmyyyydate
\usepackage{fancyhdr}
\usepackage{ifthen}
\pagestyle{fancy}
\fancyhf{}
\renewcommand{\headrulewidth}{0pt}
\fancyfoot[L]{\ifthenelse{\value{page}=1}{\today, \currenttime{} Uhr}{}}
\begin{document}
\begin{table}[ht]
\begin{minipage}[t]{0.5\linewidth}
\small
\begin{center}*D
\end{center}
\begin{tabular}{rl}
\textbf{787} & \begin{Large}A\end{Large}nfortas unt die sîne\\ 
 & \textbf{noch} \textbf{vor} jâmer dolten pîne.\\ 
 & ir triwe liez in in der nôt.\\ 
 & dicke er warb \textbf{umbe si} den tôt.\\ 
5 & der wære ouch \textbf{schiere an im} \textbf{geschehen},\\ 
 & wan daz si in dicke liezen sehen\\ 
 & den Grâl unts Grâles kraft.\\ 
 & er sprach zuo sîner rîterschaft:\\ 
 & "ich weiz wol, pflæget ir triwe,\\ 
10 & sô erbarmet iuch mîn riwe.\\ 
 & wie lange sol diz an mir wern?\\ 
 & welt ir iu selben rehtes gern,\\ 
 & sô müezet ir \textbf{gelten mich} vor gote.\\ 
 & ich stuont \textbf{ie} gern ziwerm gebote,\\ 
15 & sît ich von êrste wâpen truoc.\\ 
 & ich hân engolten des genuoc,\\ 
 & ob mir ie unprîs geschach\\ 
 & unt ob daz iwer keiner sach.\\ 
 & Sît ir \textbf{vor} untriwen bewart,\\ 
20 & sô lœset mich \textbf{durch} des \textbf{helmes} art\\ 
 & unt durch des \textbf{schildes} orden!\\ 
 & ir sît \textbf{dicke} innen worden,\\ 
 & ob ez iu niht versmâhte,\\ 
 & daz ich diu beidiu brâhte\\ 
25 & unverzagt ûf rîterlîchiu werc.\\ 
 & ich hân tal und berc\\ 
 & mit maneger tjost überzilt\\ 
 & unt mit dem swerte alsô gespilt,\\ 
 & daz es die vîende an \textbf{mir} verdrôz,\\ 
30 & \textbf{swie} \textbf{wênec} ich des gein iu genôz.\\ 
\end{tabular}
\scriptsize
\line(1,0){75} \newline
D \newline
\line(1,0){75} \newline
\textbf{1} \textit{Großinitiale} D  \textbf{19} \textit{Majuskel} D  \newline
\line(1,0){75} \newline
\newline
\end{minipage}
\hspace{0.5cm}
\begin{minipage}[t]{0.5\linewidth}
\small
\begin{center}*m
\end{center}
\begin{tabular}{rl}
 & Anfortas und die sîne\\ 
 & \textbf{vor} jâmer dolte\textit{n} pîne.\\ 
 & ir \textit{tr}i\textit{uwe} liez in in der nôt.\\ 
 & dicke er warp \textbf{umb si} den tôt.\\ 
5 & der wær ouch \textbf{schier an im} \textbf{beschehen},\\ 
 & wan daz si in dicke liez\textit{en} sehen\\ 
 & den Grâl und des Grâles kraft.\\ 
 & er sprach zuo sîner ritterschaft:\\ 
 & "ich weiz wol, pflægt ir triuwe,\\ 
10 & sô erbarmet iuch mîn riuwe.\\ 
 & wie lange sol diz an mir \textit{w}ern?\\ 
 & welt ir iu selber rehtes gern,\\ 
 & sô müezet ir \textbf{gelten mich} vor got.\\ 
 & ich stuont \textbf{ê} gern zuo iuwerm gebot,\\ 
15 & sît ich von êrste wâpen truoc.\\ 
 & ich \textit{hân} engolten des genuoc,\\ 
 & ob mir ie unprîs geschach\\ 
 & und ob daz iuwer k\textit{ein}er sach.\\ 
 & sît ir \textbf{vor} untriuwe bewart,\\ 
20 & sô lœset mich \textbf{durch} des \textbf{helmes} art\\ 
 & und durch des \textbf{schiltes} orden!\\ 
 & ir sît \textbf{dicke} innen worden,\\ 
 & ob ez iu niht versmâhte,\\ 
 & daz ich diu beidiu brâhte\\ 
25 & unverzagt ûf ritterlîchiu werc.\\ 
 & ich hân tal und ber\textit{c}\\ 
 & mit maniger just überzilt\\ 
 & und mit dem swert alsô gespilt,\\ 
 & daz es die vîende an \textbf{mir} verdrôz,\\ 
30 & \textbf{wênic} ich des gegen iu genôz.\\ 
\end{tabular}
\scriptsize
\line(1,0){75} \newline
m n o V V' W Fr6 \newline
\line(1,0){75} \newline
\textbf{1} \textit{Freier Platz für Überschrift} Fr6   $\cdot$ \textit{Initiale} V V' Fr6  \newline
\line(1,0){75} \newline
\textbf{1} Anfortas] Anfortes V' \textbf{2} vor] Noch vor V (Fr6) Von W  $\cdot$ dolten] doltte m \textbf{3} ir] Sein W  $\cdot$ triuwe] pin m n o (W)  $\cdot$ liez] liesse n \textbf{4} er warp umb si] vmb sy er warb W \textbf{5} der] Das W  $\cdot$ beschehen] geschehen n o (V') Fr6 \textbf{6} liezen] lies m o liesse n \textbf{7} kraft] grosse krafft n \textbf{8} [*]: So enphing er abir craft vnd macht V' \textbf{9} \textit{Die Verse 787.9-788.20 fehlen} V'  \textbf{11} wern] vern m \textbf{12} iu selber] úch selbes n (W) sels uͯch o iv selbe Fr6  $\cdot$ rehtes] rechten o \textbf{13} vor] fúr o \textbf{14} ê] ie V (Fr6) \textbf{16} hân] \textit{om.} m \textbf{17} geschach] beschach W \textbf{18} keiner] komer m (n) (o) (W)  $\cdot$ sach] iach W \textbf{20} lœset] lassent W \textbf{21} schiltes] hiltes n \textbf{24} diu beidiu] die baiden W \textbf{25} unverzagt] Vnuerzagte o  $\cdot$ werc] werge m \textbf{26} berc] berge m \textbf{27} maniger] manigeme V \textbf{30} wênic] Swie wenig V (Fr6) \newline
\end{minipage}
\end{table}
\newpage
\begin{table}[ht]
\begin{minipage}[t]{0.5\linewidth}
\small
\begin{center}*G
\end{center}
\begin{tabular}{rl}
 & \begin{large}A\end{large}nfortas unde die sîne\\ 
 & \textbf{von} jâmer dolten pîne.\\ 
 & ir triwe liez in in der nôt.\\ 
 & \textbf{vil} dicke er warp \textbf{datze in} den tôt.\\ 
5 & d\textit{er} wære ouch \textbf{schiere an i\textit{m}} \textbf{geschehen},\\ 
 & wan daz sin dicke liezen sehen\\ 
 & den Grâl unde des Grâles kraft.\\ 
 & er sprach ze sîner rîterschaft:\\ 
 & "ich weiz wol, pflæget ir triwe,\\ 
10 & sô erbarmet iuch mîn riwe.\\ 
 & wie lange sol ditze ane mir wern?\\ 
 & welt ir iu selben rehtes gern,\\ 
 & sô müezt ir \textbf{gelten mich} vor got.\\ 
 & ich stuont \textbf{ie} gerne ze iurem gebot,\\ 
15 & sît ich von êrste wâpen truoc.\\ 
 & ich hân engolten des genuoc,\\ 
 & op mir ie unbrîs geschach\\ 
 & unde op daz iwer deheiner sach.\\ 
 & sît ir \textbf{von} untriwen bewart,\\ 
20 & sô lœset mich \textbf{durch} des \textbf{himels} art\\ 
 & unde durch des \textbf{himels} orden!\\ 
 & ir sît \textbf{wol} innen worden,\\ 
 & ob ez iu niht versmâhte,\\ 
 & daz ich diu bêde brâhte\\ 
25 & unverzaget ûf rîterlîchiu werc.\\ 
 & ich hân tal unde berc\\ 
 & mit maniger tjost überzilt\\ 
 & unde mit dem swerte alsô gespilt,\\ 
 & daz es die vînde an \textbf{mir} verdrôz,\\ 
30 & \textbf{swie} \textbf{kleine} ich des gein iu genôz.\\ 
\end{tabular}
\scriptsize
\line(1,0){75} \newline
G I L M Z \newline
\line(1,0){75} \newline
\textbf{1} \textit{Initiale} G L M Z  \textbf{21} \textit{Initiale} I  \newline
\line(1,0){75} \newline
\textbf{2} von] Nach Z \textbf{3} in in] in Z \textbf{4} er warp] warp er Z  $\cdot$ datze in] dazz in I im Z \textbf{5} der] daz G  $\cdot$ im] in G \textbf{6} sin dicke] si diche in I  $\cdot$ liezen] liesz in M \textbf{12} selben] selber L selbe Z \textbf{13} ir gelten mich] ir mich Gelten I (Z) gelten mich L (M) \textbf{14} ze] nach Z \textbf{15} \textit{Die Verse 787.15-16 fehlen} L   $\cdot$ sît] des I \textbf{16} ich] Sie Z \textbf{17} ie unbrîs] vnpris ýe L \textbf{19} von untriwen] vor vntruͯwe L (M) (Z) \textbf{20} lœset mich] solt nv L laszt mich M  $\cdot$ himels] helmes Z \textbf{21} himels] schildes Z \textbf{22} wol] des wol I diche L (M) (Z) \textbf{23} versmâhte] versmaht Z \textbf{24} bêde] bet Z \textbf{25} rîterlîchiu] rilichev I \textbf{26} hân] beidev I \textbf{28} dem swerte] den swerten Z  $\cdot$ gespilt] [Gezilt]: Gespilt I \textbf{29} es] er Z \textbf{30} swie] Wie L (M) \newline
\end{minipage}
\hspace{0.5cm}
\begin{minipage}[t]{0.5\linewidth}
\small
\begin{center}*T
\end{center}
\begin{tabular}{rl}
 & \begin{Large}A\end{Large}nfortas und die sîne\\ 
 & \textbf{von} jâmer dolten pîne.\\ 
 & ir triuwe liez in in der nôt.\\ 
 & \textbf{vil} dicke er warp \textbf{durc\textit{h} i\textit{n}} den tôt.\\ 
5 & der wære ouch \textbf{an im schiere} \textbf{geschehen},\\ 
 & wan daz si in dicke liezen sehen\\ 
 & den Grâl und des Grâles kraft.\\ 
 & er sprach zuo sîner rîterschaft:\\ 
 & "ich weiz wol, pflæget ir triuwe,\\ 
10 & sô erbarmet iuch mîn riuwe.\\ 
 & wie lange sol diz an mir wern?\\ 
 & wolt ir iu selber rehtes gern,\\ 
 & sô müezet ir \textbf{mich gelten} vor gote.\\ 
 & ich stuont \textbf{ie} gerne zuo iuwerme gebote,\\ 
15 & sît ich von êrst wâpen truoc.\\ 
 & ich hân engolten des genuoc,\\ 
 & ob mir ie unprîs geschach\\ 
 & und ob daz iuwer dekeiner sach.\\ 
 & sît ir \textbf{vor} untriuwe bewart,\\ 
20 & sô lœset mich \textbf{von} des \textbf{helmes} art\\ 
 & und durch des \textbf{schiltes} orden!\\ 
 & ir sît \textbf{dicke} innen worden,\\ 
 & ob ez iu niht versmâhte,\\ 
 & daz ich diu beide brâhte\\ 
25 & unverzaget ûf rîterlîchiu werc.\\ 
 & ich hân tal und berc\\ 
 & mit maneger jost überzilt\\ 
 & und mit dem swerte alsô gespilt,\\ 
 & daz es die vînde an \textbf{mich} verdrôz,\\ 
30 & \textbf{wie} \textbf{kleine} ich des gein iu genôz.\\ 
\end{tabular}
\scriptsize
\line(1,0){75} \newline
U Q R \newline
\line(1,0){75} \newline
\textbf{1} \textit{Großinitiale} U R  \newline
\line(1,0){75} \newline
\textbf{1} \textit{Die Verse 784.9-789.18 fehlen} Q   $\cdot$ sîne] sinen R \textbf{2} dolten] littencz R \textbf{4} durch in] duͦrch in in U do zim R \textbf{5} an im schiere] schier an Im R \textbf{9} pflæget] vnd pflegt R \textbf{10} erbarmet] erbarmete R \textbf{11} diz] das R \textbf{12} wolt] Welte R  $\cdot$ selber] selbes R \textbf{17} geschach] beschach R \textbf{19} vor] von R \textbf{20} von] durch R \textbf{22} worden] werden R \textbf{25} rîterlîchiu] Ritterliche R \textbf{28} gespilt] geczilt R \textbf{29} mich] mir R \newline
\end{minipage}
\end{table}
\end{document}
