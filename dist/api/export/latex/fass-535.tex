\documentclass[8pt,a4paper,notitlepage]{article}
\usepackage{fullpage}
\usepackage{ulem}
\usepackage{xltxtra}
\usepackage{datetime}
\renewcommand{\dateseparator}{.}
\dmyyyydate
\usepackage{fancyhdr}
\usepackage{ifthen}
\pagestyle{fancy}
\fancyhf{}
\renewcommand{\headrulewidth}{0pt}
\fancyfoot[L]{\ifthenelse{\value{page}=1}{\today, \currenttime{} Uhr}{}}
\begin{document}
\begin{table}[ht]
\begin{minipage}[t]{0.5\linewidth}
\small
\begin{center}*D
\end{center}
\begin{tabular}{rl}
\textbf{535} & \begin{large}V\end{large}on passâschen ungeverte grôz\\ 
 & gienc an \textbf{ein} wazzer, daz dâ vlôz,\\ 
 & \textbf{schifræhe}, snel und breit,\\ 
 & dâ engein er unt diu vrouwe reit.\\ 
5 & An dem urvar ein anger lac,\\ 
 & dâr \textbf{ûfe} man vil tjoste pflac.\\ 
 & überz wazzer \textbf{stuont} \textbf{daz} kastel.\\ 
 & Gawan, der degen snel,\\ 
 & sach einen rîter nâch im varn,\\ 
10 & der schilt noch sper niht kunde sparn.\\ 
 & Orgeluse, diu rîche,\\ 
 & sprach hôchvertlîche:\\ 
 & "ob mir\textbf{z} iwer munt vergiht,\\ 
 & sô brich ich mîner triwe niht.\\ 
15 & ich \textbf{het}\textbf{s} iu ê sô \textbf{vil} gesagt,\\ 
 & daz ir vil lasters hie bejagt.\\ 
 & nû wert iuch, ob ir kunnet wern,\\ 
 & iuch \textbf{en}mac anders niht ernern.\\ 
 & der dort kumt, iuch sol sîn hant\\ 
20 & sô vellen, ob iu ist zertrant\\ 
 & inder iwer niderkleit.\\ 
 & daz \textbf{lât} iu durch die vrouwen leit\\ 
 & \textbf{sîn}, die ob iu sitzent und sehent.\\ 
 & waz, ob die iwer laster spehent?"\\ 
25 & Des schiffes meister über her\\ 
 & kom durch Orgelusen ger.\\ 
 & vonme lande inz schif si kêrte,\\ 
 & daz Gawanen trûren lêrte.\\ 
 & \begin{large}D\end{large}iu rîche, wol geborne\\ 
30 & sprach wider ûz mit zorne:\\ 
\end{tabular}
\scriptsize
\line(1,0){75} \newline
D Fr7 Fr31 \newline
\line(1,0){75} \newline
\textbf{1} \textit{Initiale} D Fr7  \textbf{5} \textit{Majuskel} D  \textbf{25} \textit{Majuskel} D  \textbf{29} \textit{Initiale} D  \newline
\line(1,0){75} \newline
\textbf{1} passâschen ungeverte] passanen vngenerte Fr7 \textbf{3} schifræhe] Schefrich Fr7 \textbf{11} Orgeluse] Orgelv̂se D \textbf{15} ich hets iu ê] :::ans iv doch Fr31 \textbf{16} vil lasters hie] hie lasters vil Fr31 \textbf{18} enmac] enkan Fr31 \textbf{22} iu] mir Fr31 \textbf{24} die] sie Fr31 \textbf{25} Des schiffes meister] :::schiffmaister Fr31  $\cdot$ über] vbe Fr31 \textbf{26} Orgelusen] Orgelv̂sen D orgelysin Fr31 \newline
\end{minipage}
\hspace{0.5cm}
\begin{minipage}[t]{0.5\linewidth}
\small
\begin{center}*m
\end{center}
\begin{tabular}{rl}
 & von pas\textit{s}âschen ungeverte grôz\\ 
 & gienc an \textbf{ein} wazzer, daz d\textit{â} vlôz,\\ 
 & \textbf{schifreht}, sne\textit{l} und breit,\\ 
 & dâ gegen er und diu vrouwe reit.\\ 
5 & an dem urvar ein anger lac,\\ 
 & dâr \textbf{ûf} man vil juste pflac.\\ 
 & über daz wazzer \textbf{stuont} \textbf{daz} kastel.\\ 
 & Gawan, der degen snel,\\ 
 & sach einen ritter nâch im varn,\\ 
10 & der schilt noch sper niht kund\textit{e} sparn.\\ 
 & Urgeluse, diu rîch,\\ 
 & sprach hôchverteclîch:\\ 
 & "ob mir\textbf{s} iuwer munt vergiht,\\ 
 & sô brich ich mîner \textit{tri}u\textit{w}e niht.\\ 
15 & ich \textbf{het} \textbf{es} iu ê sô \textbf{wol} ges\textit{ag}et,\\ 
 & daz ir vil lasters hie bejaget.\\ 
 & nû weret iuch, ob ir k\textit{unn}e\textit{t} wern,\\ 
 & iuch mac anders niht ernern.\\ 
 & der dort komt, iuch sol sîn hant\\ 
20 & sô vellen, ob \dag si\dag  iu ist zertrant\\ 
 & iender iuwer niderkleit.\\ 
 & daz \textbf{lât} iu durch die vrouwe\textit{n} leit\\ 
 & \dag~\dag\ , die ob iu sitzent und sehent.\\ 
 & waz, o\textit{b} die iuwer laster spehent?"\\ 
25 & des schiffes meister über her\\ 
 & kam durch Urgelusen ger.\\ 
 & von dem lande in daz schif s\textit{i} kêrte,\\ 
 & daz Gawan trûren lêrte.\\ 
 & diu rîch, wol geborne\\ 
30 & sprach wider ûz mit zorne:\\ 
\end{tabular}
\scriptsize
\line(1,0){75} \newline
m n o \newline
\line(1,0){75} \newline
\newline
\line(1,0){75} \newline
\textbf{1} passâschen] pascoscen m (o) pascasien n \textbf{2} wazzer] fenster n  $\cdot$ dâ] do m n \textbf{3} snel] snelle m n \textbf{8} Gawan] Gegen Gawan o \textbf{10} kunde] kunden m \textbf{11} Urgeluse] Vrelusze o \textbf{12} hôchverteclîch] hochfferchtecliche n \textbf{14} triuwe] juste m \textbf{15} wol] vil n o  $\cdot$ gesaget] geset m \textbf{16} ir] er n \textbf{17} weret] weren m  $\cdot$ iuch] nú o  $\cdot$ kunnet] komen m \textbf{19} sîn] min n \textbf{21} iender] Jergent n \textbf{22} iu] \textit{om.} o  $\cdot$ vrouwen] frouwe m \textbf{23} sehent] jehent n \textbf{24} ob] oh m \textbf{26} Urgelusen] orilusen o \textbf{27} si] sich m  $\cdot$ kêrte] kert o \textbf{28} Gawan] gawanen n o \newline
\end{minipage}
\end{table}
\newpage
\begin{table}[ht]
\begin{minipage}[t]{0.5\linewidth}
\small
\begin{center}*G
\end{center}
\begin{tabular}{rl}
 & \begin{large}V\end{large}on passâschen ungeverte grôz\\ 
 & gienc an \textbf{ein} wazzer, daz dâ vlôz,\\ 
 & \textbf{schifræhe}, snel unde breit,\\ 
 & dâ engegen er unde diu vrouwe reit.\\ 
5 & an dem urvar ein anger lac,\\ 
 & dâr \textbf{ûffe} man vil tjoste pflac.\\ 
 & überz wazzer \textbf{stuont} \textbf{daz} kastel.\\ 
 & Gawan, der degen snel,\\ 
 & sach einen rîter nâch im varn,\\ 
10 & der schilt noch sper niht kunde sparn.\\ 
 & Orgeluse, diu rîche,\\ 
 & sprac\textit{h} \textit{h}ôchverticlîche:\\ 
 & "op mir\textbf{s} iuwer munt vergiht,\\ 
 & sô briche ich mîner triuwe niht.\\ 
15 & ich \textbf{hete}\textbf{s} iu ê sô \textbf{vil} gesaget,\\ 
 & daz ir vil lasters hie bejaget.\\ 
 & nû wert iuch, obe ir kunnet wern,\\ 
 & iuch \textbf{en}mac anders niht ernern.\\ 
 & der dort kumet, iuch sol sîn hant\\ 
20 & sô vellen, ob iu ist zertrant\\ 
 & iener iuwer niderkleit.\\ 
 & daz \textbf{sî} iu durch die vrouwen leit,\\ 
 & die obe iu sitzent und sehent.\\ 
 & waz, op die iuwer laster spehent?"\\ 
25 & des schiffes meister über her\\ 
 & kom durch Orgelusen ger.\\ 
 & vome lande inz schif si kêrte,\\ 
 & daz Gawanen trûren lêrte.\\ 
 & diu rîche, wol geborne\\ 
30 & sprach wider ûz mit zorne:\\ 
\end{tabular}
\scriptsize
\line(1,0){75} \newline
G I L M Z Fr19 \newline
\line(1,0){75} \newline
\textbf{1} \textit{Initiale} G I L M Z Fr19  \textbf{19} \textit{Initiale} I  \newline
\line(1,0){75} \newline
\textbf{2} an] \textit{om.} M  $\cdot$ ein] einem I  $\cdot$ dâ] \textit{om.} Z \textbf{3} schifræhe] Schýf riche L (Z) Schiff bruche M  $\cdot$ breit] grosz breit L \textbf{4} vrouwe] vrouwen M \textbf{5} urvar] vnvare M  $\cdot$ anger] vnger M \textbf{7} wazzer] \textit{om.} I \textbf{10} noch] vnd L (Z)  $\cdot$ niht] \textit{om.} I \textbf{11} Orgeluse] Orguluse I Orgelýse L Orgiloise M Orgilvse Fr19 \textbf{12} sprach >si< hôchverteclîche G \textbf{13} mirs] mir I \textbf{14} sô briche] son brich I (M)  $\cdot$ mîner] mýne L  $\cdot$ triuwe] triwen I (Z) (Fr19) \textbf{15} hetes] het I  $\cdot$ iu ê] y M \textbf{16} vil] so vil M \textbf{17} ir] ir evch I ich uch M \textbf{18} ernern] erwern I \textbf{20} zertrant] entrant M \textbf{21} iener] Jrgin M Jenden Z \textbf{22} sî] wirt L lat M Z Fr19  $\cdot$ leit] sin leit Z \textbf{23} obe] uff M  $\cdot$ sehent] ez sehent Z \textbf{24} op] uff M  $\cdot$ die] ir Z  $\cdot$ iuwer] ev Z \textbf{26} Orgelusen] Orgulusen I Orgelýsen L orgiloisin M orgilusen Fr19 \textbf{27} vome] von I \textbf{28} daz] Da Z  $\cdot$ Gawanen] Gawan I (M) (Z) \textbf{30} wider] her wider I \newline
\end{minipage}
\hspace{0.5cm}
\begin{minipage}[t]{0.5\linewidth}
\small
\begin{center}*T
\end{center}
\begin{tabular}{rl}
 & Von passâschen \textbf{ein} ungeverte grôz\\ 
 & gienc an \textbf{daz} wazzer, daz dâ vlôz,\\ 
 & \textbf{schifrechic}, snel unde breit,\\ 
 & dar gegen er unde diu vrouwe reit.\\ 
5 & An dem urvar ein anger lac,\\ 
 & dâ man vil tjoste \textbf{ûffe} pflac.\\ 
 & überz wazzer \textbf{lac} \textbf{ein} kastel.\\ 
 & Gawan, der degen snel,\\ 
 & sach einen rîter nâch im varn,\\ 
10 & der schilt noch sper niht kunde sparn.\\ 
 & \textit{\begin{large}O\end{large}}rgeluse, diu rîche,\\ 
 & sprach hôchverteclîche:\\ 
 & "ob mir\textbf{s} iuwer munt vergiht,\\ 
 & sô\textbf{ne} brich ich mîner triuwe niht.\\ 
15 & ich \textbf{hân}\textbf{z} iu ê sô \textbf{wol} gesaget,\\ 
 & daz ir vil lasters hie bejaget.\\ 
 & nû wert iuch, ob ir \textbf{iuch} kunnet wern,\\ 
 & iuch mac anders niht ernern.\\ 
 & der dort kumet, iuch sol sîn hant\\ 
20 & sô vellen, ob iu ist zertrant\\ 
 & iender iuwer niderkleit.\\ 
 & daz \textbf{sî} iu durch die vrouwen leit,\\ 
 & die ob iu sitzent unde sehent.\\ 
 & waz, ob die iuwer laster spehent?"\\ 
25 & Des schiffes meister über her\\ 
 & kom durch Orgelusen ger.\\ 
 & vonme lande in daz schif si kêrte,\\ 
 & daz Gawanen trûren lêrte.\\ 
 & Diu rîche, wol geborne\\ 
30 & sprach wider ûz mit zorne:\\ 
\end{tabular}
\scriptsize
\line(1,0){75} \newline
T U V W O Q R \newline
\line(1,0){75} \newline
\textbf{1} \textit{Initiale} W   $\cdot$ \textit{Majuskel} T  \textbf{5} \textit{Majuskel} T  \textbf{11} \textit{Initiale} T U V  \textbf{25} \textit{Initiale} W   $\cdot$ \textit{Majuskel} T  \textbf{29} \textit{Majuskel} T  \newline
\line(1,0){75} \newline
\textbf{1} passâschen] Passas O  $\cdot$ ein] \textit{om.} O Q R \textbf{2} gienc] Dinck Q  $\cdot$ daz] ein V W O Q R  $\cdot$ dâ] do U W Q \textbf{3} schifrechic] Schifrichec U Schifrehe V (O) Schiffreich W (Q) (R)  $\cdot$ snel] bald Q \textbf{4} gegen] in gein U (V) (O) (Q) (R) \textbf{5} urvar] fúruar W vnfar Q v́berfar R  $\cdot$ anger] ander V W  $\cdot$ lac] hag W \textbf{6} vil tjoste] tyost vil V \textbf{8} Gawan] Gawin R \textbf{10} noch] vnd W  $\cdot$ sper] spern R \textbf{11} Orgeluse] ÷Rgelvse T Orelusze Q Orguluse R \textbf{13} munt] mut Q \textbf{14} sône brich] So in brechte U So enbir V So briche O (R)  $\cdot$ mîner] mine R  $\cdot$ triuwe] truͦwen U (O) (Q) \textbf{15} hânz] han O  $\cdot$ sô] \textit{om.} R  $\cdot$ wol] vil U V W O Q R \textbf{16} lasters] laster Q  $\cdot$ hie] alhie Q \textbf{17} wert iuch] wert îv T wert O  $\cdot$ ob ir] vbir U koͤnnent ir V  $\cdot$ iuch kunnet] iv kvnnet T v́ch V konnet Q kunen R \textbf{18} iuch] iv T  $\cdot$ mac] enkan V enmag W (O) R  $\cdot$ niht] nichts W \textbf{19} kumet] komp R  $\cdot$ iuch] îv T \textbf{20} iu] \textit{om.} O  $\cdot$ zertrant] [*]: zerdrant V \textbf{21} iender] Jergent V  $\cdot$ niderkleit] in der cleit Q wirdikeit R \textbf{22} iu] ivch T  $\cdot$ die] \textit{om.} U sie Q \textbf{24} waz] Weis R \textbf{25} über] [h*]: vber T úwer W \textbf{26} Orgelusen] orlugens U orgulusen R \textbf{27} daz] \textit{om.} Q \textbf{28} Gawanen] gawan W gawann Q Gawin R \textbf{29} rîche] reich vnd W \textbf{30} ûz] [*vz]: vz U her ausz Q \newline
\end{minipage}
\end{table}
\end{document}
