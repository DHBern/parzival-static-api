\documentclass[8pt,a4paper,notitlepage]{article}
\usepackage{fullpage}
\usepackage{ulem}
\usepackage{xltxtra}
\usepackage{datetime}
\renewcommand{\dateseparator}{.}
\dmyyyydate
\usepackage{fancyhdr}
\usepackage{ifthen}
\pagestyle{fancy}
\fancyhf{}
\renewcommand{\headrulewidth}{0pt}
\fancyfoot[L]{\ifthenelse{\value{page}=1}{\today, \currenttime{} Uhr}{}}
\begin{document}
\begin{table}[ht]
\begin{minipage}[t]{0.5\linewidth}
\small
\begin{center}*D
\end{center}
\begin{tabular}{rl}
\textbf{426} & \begin{large}E\end{large}r hât hie erliten grôze nôt\\ 
 & und \textbf{er} muoz nû kêren in den tôt.\\ 
 & swaz erden \textbf{hât umbeslagen}z mer,\\ 
 & \textbf{dâ}ne \textbf{gelac} nie hûs sô wol ze wer\\ 
5 & als Munsalvæsche, swâ diu stêt.\\ 
 & von strîte rûher wec dar gêt.\\ 
 & bî sîme gemache in \textbf{hînte} lât.\\ 
 & morgen sag man im den rât."\\ 
 & Des volgeten al \textbf{die} râtgeben.\\ 
10 & sus behielt \textbf{hêr} Gawan \textbf{dâ} sîn leben.\\ 
 & man pflac des heldes unverzagt\\ 
 & des nahtes \textbf{al} \textbf{dâ}, wart mir gesagt,\\ 
 & daz harte guot \textbf{was} sîn gemach.\\ 
 & dô man den \textbf{mitten} morgen sach\\ 
15 & unt \textbf{dô} man messe \textbf{gesanc},\\ 
 & ûf \textbf{dem palase} \textbf{was} grôz gedranc\\ 
 & von bovel unt von werder diet.\\ 
 & der künec tet, als man im riet.\\ 
 & er hiez Gawanen bringen.\\ 
20 & den wolter nihtes twingen,\\ 
 & \textbf{wan} als ir selbe hât gehôrt.\\ 
 & nû seht, wâ in brâhte dort\\ 
 & Antikonie, diu wolgevar.\\ 
 & ir vetern sun kom mit ir dar\\ 
25 & unt ander genuoge des küneges man.\\ 
 & diu küneginne vuorte Gawanen\\ 
 & vür den künec an ir hende.\\ 
 & ein schapel was ir gebende.\\ 
 & \begin{large}I\end{large}r munt den bluomen nam \textbf{ir} prîs.\\ 
30 & ûf dem schapele \textbf{decheinen} \textbf{gewîs}\\ 
\end{tabular}
\scriptsize
\line(1,0){75} \newline
D Fr1 Fr5 Fr68 \newline
\line(1,0){75} \newline
\textbf{1} \textit{Initiale} D Fr5  \textbf{9} \textit{Capitulumzeichen} Fr5   $\cdot$ \textit{Majuskel} D  \textbf{11} \textit{Initiale} Fr1  \textbf{22} \textit{Versal} Fr1  \textbf{29} \textit{Initiale} D  \newline
\line(1,0){75} \newline
\textbf{2} er] \textit{om.} Fr1 \textbf{3} hât] het Fr5 \textbf{4} dâne] sone Fr1 \textbf{5} Munsalvæsche] Mvnsalvæsce D Fr1 mvntsaluahsch Fr5 \textbf{6} strîte] stein Fr5 \textbf{7} bî] an Fr1 \textbf{8} sag] so sage Fr1 \textbf{9} volgeten] gevolgeten Fr1  $\cdot$ al] alle Fr1 im al Fr5 \textbf{10} Gawan] Gauwan Fr5 \textbf{11} man] Wan Fr5 \textbf{12} al] \textit{om.} Fr1 \textbf{15} dô] \textit{om.} Fr1 \textbf{16} was] wart Fr1  $\cdot$ grôz] michil Fr5 \textbf{17} von] [vol]: von Fr5 \textbf{19} Gawanen] Gauwan Fr5 \textbf{21} wan] wen Fr68 \textbf{22} wâ] wer Fr5 \textbf{23} Antikonie] Antẏconîe Fr1 Antichonie Fr5 anthýconie Fr68  $\cdot$ diu] di Fr68 \textbf{25} ander genuoge] andir [k]: gnvͦege Fr5 andere gnuge Fr68 \textbf{26} diu] di Fr68  $\cdot$ Gawanen] Gauwan Fr5 \textbf{29} den bluomen nam] nam den blvͦmen Fr1  $\cdot$ ir] den Fr68 \textbf{30} dem] den Fr68  $\cdot$ decheinen] in alle Fr5 dicheine Fr68  $\cdot$ gewîs] wîs Fr1 (Fr5) (Fr68) \newline
\end{minipage}
\hspace{0.5cm}
\begin{minipage}[t]{0.5\linewidth}
\small
\begin{center}*m
\end{center}
\begin{tabular}{rl}
 & er hât hie erliten grôze nôt\\ 
 & und muoz nû kêren in den tôt.\\ 
 & waz erden \textbf{umbslagen hât} daz mere,\\ 
 & \textbf{d\textit{â}} en\textbf{gelac} nie hûs sô wol ze were\\ 
5 & als Munsaluasce, wâ diu stât.\\ 
 & von strîte rûher wec dar gât.\\ 
 & bî sînem gemache in \textbf{hînaht} lât.\\ 
 & morgen sage man ime den rât."\\ 
 & des volgeten alle  râtg\textit{eb}en.\\ 
10 & sus behielt \textbf{hê\textit{r}} Gawan \textbf{d\textit{â}} sîn leben.\\ 
 & \begin{large}M\end{large}an pflac des heldes unverzaget\\ 
 & des nahtes, wart mir gesaget,\\ 
 & daz harte guot \textbf{was} sîn gemach.\\ 
 & dô man den \textbf{mitten} morgen sach\\ 
15 & und \textbf{dô} man messe \textbf{sanc},\\ 
 & ûf \textbf{dem palase} \textbf{was} grôz ge\textit{dra}nc\\ 
 & von povel und von werder tiet.\\ 
 & der künic t\textit{e}t, als man ime riet.\\ 
 & er hiez Gawanen bringen.\\ 
20 & den wolt er nihte\textit{s} \textit{t}wi\textit{n}gen,\\ 
 & als ir selbe habt gehôrt.\\ 
 & nû sehet, wâ in brâhte dort\\ 
 & Anticonie, diu wolgevar.\\ 
 & ir vetere\textit{n} sun kam mit ir dar\\ 
25 & und andere genuoge des küniges man.\\ 
 & diu künigîn vuorte Gawan\\ 
 & vür den künic an ir hende.\\ 
 & ein schapel was ir gebende.\\ 
 & ir munt den pluomen nam \textbf{den} prîs.\\ 
30 & ûf dem schapel \textbf{dekein} \textbf{wîs}\\ 
\end{tabular}
\scriptsize
\line(1,0){75} \newline
m n o \newline
\line(1,0){75} \newline
\textbf{11} \textit{Initiale} m   $\cdot$ \textit{Capitulumzeichen} n  \newline
\line(1,0){75} \newline
\textbf{1} hât hie erliten] hette erlitten hie n (o) \textbf{3} erden] erde n er o \textbf{4} dâ] Do m n o  $\cdot$ engelac] gelag n o \textbf{5} Munsaluasce] múntsaluasc n muntsaluͯasce o \textbf{8} man] nam o \textbf{9} des] Das o  $\cdot$ alle] alle die n o  $\cdot$ râtgeben] rat gegeben m \textbf{10} hêr] he m  $\cdot$ dâ] do m \textit{om.} n o \textbf{12} nahtes] nachtes do n (o) \textbf{13} harte] \sout{n*} harte m  $\cdot$ guot] guͯte n \textbf{15} sanc] gesang n o \textbf{16} was] \textit{om.} n  $\cdot$ gedranc] gestarng m \textbf{17} povel] pondel n ponel o \textbf{18} tet] tot m \textbf{20} nihtes] nihte m  $\cdot$ twingen] swigen m \textbf{21} als] Wenne also n Wan als o  $\cdot$ selbe] selb n o \textbf{23} Anticonie] Antitonie o \textbf{24} ir] Jrs n  $\cdot$ veteren] veterer m  $\cdot$ ir] jme n \textbf{25} andere] ander n o  $\cdot$ genuoge] genuͯg n o \textbf{30} dekein] do keine n \newline
\end{minipage}
\end{table}
\newpage
\begin{table}[ht]
\begin{minipage}[t]{0.5\linewidth}
\small
\begin{center}*G
\end{center}
\begin{tabular}{rl}
 & er hât hie erliten grôze nôt\\ 
 & unde muoz nû kêren in den tôt.\\ 
 & swaz erden \textbf{hât umbeslagen} daz mer,\\ 
 & \textbf{sô}ne \textbf{gestuont} nie hûs sô wol ze wer\\ 
5 & als Muntsalfatsche, swâ diu stêt.\\ 
 & von strîte rûher wec dar gêt.\\ 
 & bî sînem gemach in \textbf{hînt} \textbf{hie} lât.\\ 
 & morgen sage man im den rât."\\ 
 & des volgten al \textbf{die} râtgeben.\\ 
10 & sus behielt \textbf{hêr} Gawan \textbf{dâ} sîn leben.\\ 
 & man pflac des heldes unverzaget\\ 
 & des nahts \textbf{dâ}, wart mir gesaget,\\ 
 & daz harte guot \textbf{was} sîn gemach.\\ 
 & dô man den \textbf{mitteren} morgen sach\\ 
15 & unt man messe \textbf{gesanc},\\ 
 & ûf \textbf{dem palase} \textbf{wart} grôz gedranc\\ 
 & von povel und von werder diet.\\ 
 & der künec tet, als man im riet.\\ 
 & er hiez Gawanen bringen.\\ 
20 & den wolter nihtes dwingen,\\ 
 & \textbf{wan} als ir selbe habet gehôrt.\\ 
 & nû seht, wâ in brâhte dort\\ 
 & Antikonie, diu wolgevar.\\ 
 & ir vetern sun kom mit ir dar\\ 
25 & unde ander genuoge des küneges man.\\ 
 & diu künegîn vuorte Gawan\\ 
 & vür den künec an ir hende.\\ 
 & ein tschappel was ir gebende.\\ 
 & ir munt den bluo\textit{m}en nam \textbf{den} brîs.\\ 
30 & ûf dem tschappele \textbf{deheine} \textbf{wîs}\\ 
\end{tabular}
\scriptsize
\line(1,0){75} \newline
G I O L M Q R Z Fr21 \newline
\line(1,0){75} \newline
\textbf{1} \textit{Initiale} I L Q Z Fr21  \textbf{7} \textit{Capitulumzeichen} R  \textbf{11} \textit{Initiale} R  \textbf{17} \textit{Initiale} I  \newline
\line(1,0){75} \newline
\textbf{1} hie erliten] erlitten hie R  $\cdot$ grôze] groz O (R) \textbf{2} kêren] riten I \textbf{3} swaz] Waz L (M) (Q) (R)  $\cdot$ erden] erde O L  $\cdot$ hât umbeslagen] vmbeslagen hat I (Q) (R) hat vmbes langez Fr21  $\cdot$ daz mer] [damer]: daz mer I mer Fr21 \textbf{4} sône] Da O Z Fr21 So L R Do Q  $\cdot$ gestuont] lac I Z gelach O L (M) (Q) (R) (Fr21) \textbf{5} Muntsalfatsche] muntsalvatsche G (L) (Fr21) muntshaluasce I [Mvnsalvatsche]: Mvntsalvatsche O muntsalvasche Q Munsauasche R montsalvatsche Z  $\cdot$ swâ] wa L M R nú Q so Fr21  $\cdot$ diu] si Q \textbf{6} wec] wer M \textbf{7} bî sînem] Bisinen Fr21  $\cdot$ hie] \textit{om.} I O L Q R Z Fr21 \textbf{8} sage man] saget I sagen wir L \textbf{9} al] im al I alle L M Q R Z Fr21  $\cdot$ die] \textit{om.} R \textbf{10} sus] Es Q  $\cdot$ hêr Gawan] ergawan M gaban Q Gawin R  $\cdot$ dâ sîn] daz I Fr21 do sin L (Q) sin M R \textbf{11} heldes] helden R \textbf{12} dâ] alda M Z do Q \textbf{13} harte] gar Z  $\cdot$ guot] gvͦte O  $\cdot$ was] ward R \textbf{14} dô] Da O M Z  $\cdot$ mitteren] mitten I O L (M) (Q) R Z Fr21  $\cdot$ morgen sach] [tac]: morgen geschach Fr21 \textbf{15} unt] Vnd da O (M) (Z) Vnd do L (Q) (R) Fr21  $\cdot$ gesanc] sanc M \textbf{16} dem] den L  $\cdot$ wart] was I O (M) Q (R) Z Fr21 \textbf{17} von] Vo O R  $\cdot$ und von] vnd R \textbf{19} Gawanen] Gawan I Gawainen R \textbf{21} wan] Wanne L (Q)  $\cdot$ selbe] selbin M selb R \textit{om.} Z  $\cdot$ habet] hat M hab R \textbf{22} nû seht wâ] Nu secht nu Q Nun sechent R Sehet wa Z  $\cdot$ in] im O Fr21 \textbf{23} Antikonie] Anticonia I Antykonîe O Antigonie Fr21  $\cdot$ wolgevar] \sout{kvnigin} wol gevar Fr21 \textbf{24} ir] Jrs Z  $\cdot$ vetern] vetir M (R)  $\cdot$ sun] \textit{om.} R  $\cdot$ ir] [im]: ir G yme M (Q) \textbf{25} genuoge] gnûc Q (R) (Z) \textbf{26} künegîn] [konige]: konigin Q [kúngvn]: kúngin R  $\cdot$ vuorte] nam her I  $\cdot$ Gawan] Gawain R \textbf{27} vür] vnd furt in vur I  $\cdot$ ir] der Z \textbf{28} ir] er R \textbf{29} munt den] munde die I  $\cdot$ bluomen] bloͮen G  $\cdot$ nam] namen I  $\cdot$ den brîs] ir prisz M \textbf{30} dem tschappele] dem haubet I den tschape Q  $\cdot$ deheine] dehain I (R) deheinen O (Q) Fr21 keyne M keinen Z \newline
\end{minipage}
\hspace{0.5cm}
\begin{minipage}[t]{0.5\linewidth}
\small
\begin{center}*T
\end{center}
\begin{tabular}{rl}
 & er hât hie erliten grôze nôt\\ 
 & und muoz nû kêren in den tôt.\\ 
 & swaz erden \textbf{umbeslagen hât} daz mer,\\ 
 & \textbf{sô}ne \textbf{gelac} nie hûs sô wol ze wer\\ 
5 & alse Munsalvasche, swâ diu stât.\\ 
 & von strîte rûher wec dar gât.\\ 
 & bî sînem gemache in \textbf{niht} lât.\\ 
 & morgen sage man im den rât."\\ 
 & \begin{large}D\end{large}es volgeten alle \textbf{die} râtgeben.\\ 
10 & sus behielt Gawan sîn leben.\\ 
 & man pflac des heldes unverzaget\\ 
 & des nahtes \textbf{dâ}, wart mir gesaget,\\ 
 & daz harte guot \textbf{wart} sîn gemach.\\ 
 & dô man den \textbf{driten} morgen sach\\ 
15 & und \textbf{dô} man messe \textbf{gesanc},\\ 
 & ûf \textbf{den palas} \textbf{wart} grôz gedranc\\ 
 & von bovele und von werder diet.\\ 
 & Der künec tet, als man im riet.\\ 
 & er hiez Gawanen bringen.\\ 
20 & den wolter nihtes twingen,\\ 
 & \textbf{wan} alsir selbe habt gehôrt.\\ 
 & Nû seht, wâ in brâhte dort\\ 
 & Antickonie, diu wolgevar.\\ 
 & ir vetern sun kom mit ir dar\\ 
25 & und andere genuoge des küneges man.\\ 
 & diu künegîn vuorte Gawan\\ 
 & vür den künec an ir hende.\\ 
 & ein schapel was ir gebende.\\ 
 & ir munt den bluomen nam \textbf{den} prîs.\\ 
30 & ûf dem schapel \textbf{deheine} \textbf{wîs}\\ 
\end{tabular}
\scriptsize
\line(1,0){75} \newline
T U V W \newline
\line(1,0){75} \newline
\textbf{9} \textit{Initiale} T U V  \textbf{18} \textit{Majuskel} T  \textbf{22} \textit{Majuskel} T  \newline
\line(1,0){75} \newline
\textbf{1} er] Fr W \textbf{3} swaz] Waz U (W)  $\cdot$ umbeslagen hât] hat vmbeschlagen W \textbf{4} sône] Do V So W \textbf{5} Munsalvasche] mvnsalvasce T muͦntsalvatsche U mvntschavasche V montsaluatsch W  $\cdot$ swâ] wa U (W) \textbf{6} wec] wer U \textbf{7} in niht lât] er hint lac U in [hin*]: hinaht lat V in heint lat W \textbf{8} morgen] Morges W  $\cdot$ im] in W \textbf{10} sîn] do sin U V (W) \textbf{12} dâ] do U V W \textbf{13} wart] was W \textbf{14} driten] [*]: mitten V mitten W \textbf{16} den] dem U V W  $\cdot$ wart] waz V (W) \textbf{20} nihtes] nit U \textbf{21} selbe] selber W \textbf{23} Antickonie] Antikonie U W Antykonie V \textbf{24} ir] [Jr]: Jrs U Irs W \textbf{25} andere] ander U V W  $\cdot$ genuoge] gnuͦg W \textbf{27} vür] Eúr W  $\cdot$ ir hende] irn henden U \textbf{29} nam] man U  $\cdot$ den prîs] irn preiß W \textbf{30} deheine] dekeine U keine W \newline
\end{minipage}
\end{table}
\end{document}
