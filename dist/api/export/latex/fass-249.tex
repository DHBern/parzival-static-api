\documentclass[8pt,a4paper,notitlepage]{article}
\usepackage{fullpage}
\usepackage{ulem}
\usepackage{xltxtra}
\usepackage{datetime}
\renewcommand{\dateseparator}{.}
\dmyyyydate
\usepackage{fancyhdr}
\usepackage{ifthen}
\pagestyle{fancy}
\fancyhf{}
\renewcommand{\headrulewidth}{0pt}
\fancyfoot[L]{\ifthenelse{\value{page}=1}{\today, \currenttime{} Uhr}{}}
\begin{document}
\begin{table}[ht]
\begin{minipage}[t]{0.5\linewidth}
\small
\begin{center}*D
\end{center}
\begin{tabular}{rl}
\textbf{249} & \textbf{\begin{large}D\end{large}er valscheite} widersaz\\ 
 & \textbf{kêrte} ûf der huofslege kraz.\\ 
 & \textbf{sîn scheiden} dan, daz riwet mich.\\ 
 & \textbf{alrêst nû} âventiwert ez sich.\\ 
5 & dô begunde krenken sich ir spor.\\ 
 & sich schieden, die dâ riten vor.\\ 
 & ir slâ wart smal, diu ê was breit.\\ 
 & \textbf{er} verlôs \textbf{si} gar, daz was im leit.\\ 
 & \textbf{mære} vriesch \textbf{dô} der junge man,\\ 
10 & dâ von er \textbf{herzenôt} gewan.\\ 
 & \textbf{Dô erhôrte} der \textbf{degen} ellens rîch\\ 
 & einer vrouwen stimme jæmerlîch.\\ 
 & \textbf{ez} was dennoch von touwe naz.\\ 
 & vor im ûf einer linden saz\\ 
15 & ein magt, der vuogte ir triwe nôt.\\ 
 & ein gebalsemt ritter tôt\\ 
 & \textbf{lent} ir \textbf{zwischen den} armen.\\ 
 & \textbf{swenz} niht wolt erbarmen,\\ 
 & der si \textbf{sô} sitzen sæhe,\\ 
20 & untriwen ich im jæhe.\\ 
 & Sîn \textbf{ors} \textbf{dô gein ir} wante,\\ 
 & der wênic si \textbf{bekante}.\\ 
 & si was doch sîner muomen kint.\\ 
 & al irdisch triwe was ein wint,\\ 
25 & wan die \textbf{man} an ir lîbe sach.\\ 
 & Parzival si gruozte und sprach:\\ 
 & "vrouwe, mir ist \textbf{vil} leit\\ 
 & iwer senlîchiu arbeit.\\ 
 & bedurft ir mînes dienstes iht,\\ 
30 & in iwerem dienste man mich siht."\\ 
\end{tabular}
\scriptsize
\line(1,0){75} \newline
D \newline
\line(1,0){75} \newline
\textbf{1} \textit{Großinitiale} D  \textbf{11} \textit{Majuskel} D  \textbf{21} \textit{Majuskel} D  \newline
\line(1,0){75} \newline
\textbf{29} dienstes] diens D \newline
\end{minipage}
\hspace{0.5cm}
\begin{minipage}[t]{0.5\linewidth}
\small
\begin{center}*m
\end{center}
\begin{tabular}{rl}
 & \textbf{der valsch\textit{eit}e} widersaz\\ 
 & \textbf{kêrte} ûf der huofslege kraz.\\ 
 & \textbf{si schieden} dannen, daz riuwet mich.\\ 
 & \textbf{allerêrst nû} âvent\textit{iurt} ez \textit{s}ich.\\ 
5 & dô begunde krenken sich ir spor.\\ 
 & sich schieden, die dâ riten vor.\\ 
 & ir slâ wart smal, diu ê was breit.\\ 
 & \textbf{er} verlôs \textbf{si} gar, daz \textit{was} im leit.\\ 
 & \textbf{\begin{large}N\end{large}û} vr\textit{ie}sch der junge, \textbf{süeze} man\\ 
10 & \textbf{mære}, dâ von er \textbf{nôt} gewan.\\ 
 & \textbf{dô erhôrte} der \textbf{degen} ellens rîch\\ 
 & einer vrouwen stimme jâmerlîch.\\ 
 & \textbf{er} was dan\textit{noch} von touwe naz.\\ 
 & vor ime ûf einer linde\textit{n} saz\\ 
15 & ein maget, de\textit{r} vuogte ir triuwe nôt.\\ 
 & ein gebalsamt ritter tôt\\ 
 & \textbf{lent} ir \textbf{zwischen den} armen.\\ 
 & \textbf{wen ez} niht wolte erbarmen,\\ 
 & der \textit{si} \textbf{sô} sitzen sæhe,\\ 
20 & untriuwen ich ime jæhe.\\ 
 & sîn \textbf{ros} \textbf{gegen ir dô} wante,\\ 
 & der wênic si \textbf{bekante}.\\ 
 & si was doch sîner muomen kint.\\ 
 & alliu irdis\textit{ch}iu triuwe was ein wint,\\ 
25 & wan die \textbf{man} an ir lîb\textit{e} sach.\\ 
 & Parcifal si gruozte und sprach:\\ 
 & "vrouwe, mir ist \textbf{sêre} leit\\ 
 & iuwer senlîchiu arbeit.\\ 
 & bedurfet ir mînes dienstes \textit{i}ht,\\ 
30 & i\textit{n} iuwerem dienste man mich siht."\\ 
\end{tabular}
\scriptsize
\line(1,0){75} \newline
m n o Fr69 \newline
\line(1,0){75} \newline
\textbf{9} \textit{Illustration mit Überschrift:} Wie parcifal Sigunen vff einer linden vant m  Also parcifal frouwe sigunen vff einer linden fant sitzen n (o)   $\cdot$ \textit{Initiale} m n o  \newline
\line(1,0){75} \newline
\textbf{1} valscheite] valsche m falscheite sie o valchheite Fr69 \textbf{2} huofslege] huͯpslag n (o)  $\cdot$ kraz] dratz n (o) \textbf{3} schieden] scheident o  $\cdot$ dannen] \textit{om.} n \textbf{4} nû] \textit{om.} n o  $\cdot$ âventiurt] auentes m offentúrte n auentuͯr o  $\cdot$ ez] er o  $\cdot$ sich] mich m \textbf{5} sich ir] schier n o  $\cdot$ spor] por o \textbf{6} schieden] schieden denn n  $\cdot$ dâ] do n o  $\cdot$ vor] fuͯr o \textbf{7} ir] Die o \textbf{8} was] \textit{om.} m \textbf{9} vriesch] freisch m o \textbf{11} ellens rîch] ellenrich n (o) \textbf{13} dannoch] dane m \textbf{14} linden] lindes m \textbf{15} der] den m  $\cdot$ triuwe] tauwe o \textbf{16} gebalsamt] gebalsmat o \textbf{18} ez] \textit{om.} o \textbf{19} si] \textit{om.} m \textbf{20} untriuwen] Vntrúwe Fr69 \textbf{22} si] sú do n  $\cdot$ bekante] erkante Fr69 \textbf{23} doch] oͮch Fr69 \textbf{24} alliu irdischiu] Alle ir disse m Oder dise n o Al Jrdensch Fr69 \textbf{25} lîbe] liben m \textbf{26} Parcifal] Parcipfal n \textbf{27} ist] [ir]: ist n \textbf{29} bedurfet] Bedurff o  $\cdot$ iht] niht m \textbf{30} in iuwerem] Jr jrem m Jn irem n (o) \newline
\end{minipage}
\end{table}
\newpage
\begin{table}[ht]
\begin{minipage}[t]{0.5\linewidth}
\small
\begin{center}*G
\end{center}
\begin{tabular}{rl}
 & \textbf{sich huop der v\textit{e}lsche} widersaz\\ 
 & \textbf{vaste} ûf der huofslege kraz.\\ 
 & \textbf{sîn scheiden} dan, daz riwet mich.\\ 
 & \textbf{alrêrst nû} âventiurt ez sich.\\ 
5 & dô begunde krenken sich ir spor.\\ 
 & sich schieden, die dâ riten vor.\\ 
 & ir slâ wart smal, diu ê was breit.\\ 
 & \textbf{er} verlôs \textbf{si} gar, daz was im leit.\\ 
 & \textbf{mære} vriesch der junge man,\\ 
10 & dâ von er \textbf{herzenôt} gewan.\\ 
 & \textbf{ez vernam} der \textbf{helt} ellens rîche\\ 
 & einer vrouwen stimme jæmerlîche.\\ 
 & \textbf{ez} was dannoch von touwe naz.\\ 
 & vor im ûf einer linden saz\\ 
15 & ein maget, der vuoget ir triwe nôt.\\ 
 & ein gebalsemet rîter tôt\\ 
 & \textbf{lent} ir \textbf{zwischen den} armen.\\ 
 & \textbf{den ez} niht wolt erbarmen,\\ 
 & der si \textbf{\textit{s}ô} sitzen sæhe,\\ 
20 & untriwen ic\textit{h} \textit{i}m jæhe.\\ 
 & sîn \textbf{ors} \textbf{dô gein ir} wande,\\ 
 & der wênic si \textbf{bekande}.\\ 
 & si was doch sîner muomen kint.\\ 
 & al irdesch triwe was ein wint,\\ 
25 & wan die \textbf{man} an ir lîbe sach.\\ 
 & Parzival si gruozte unde sprach:\\ 
 & "\textbf{nû wizzet}, vrouwe, mir ist leit\\ 
 & iwer senelîchiu arebeit.\\ 
 & \textit{b}e\textit{durf}t ir mînes dienstes iht,\\ 
30 & in iwerem dienste man mich siht."\\ 
\end{tabular}
\scriptsize
\line(1,0){75} \newline
G I O L M Q R Z Fr21 Fr23 Fr36 Fr40 Fr51 \newline
\line(1,0){75} \newline
\textbf{1} \textit{Initiale} I  \textbf{5} \textit{Überschrift:} Aventiwer wie Parzifal bedwanch Orillus sunder twal vnd frowen Iescutten hulde gewan I   $\cdot$ \textit{Initiale} I M  \textbf{9} \textit{Initiale} L Q R Z Fr21 Fr36  \textbf{13} \textit{Initiale} O  \textbf{27} \textit{Initiale} I  \newline
\line(1,0){75} \newline
\textbf{1} sich] ÷cͮh I  $\cdot$ der] des O (L) Fr21  $\cdot$ velsche] valsche G I (Q) (R) valsches O (L) M (Z) Fr21 \textbf{2} vaste] \textit{om.} O L M Q Fr21  $\cdot$ der huofslege] der [uff]: huffslege M den huͦffsclag R \textbf{3} scheiden dan] dan shaiden I  $\cdot$ riwet] muͤt I \textbf{4} nû] \textit{om.} R \textbf{5} dô] Da Z  $\cdot$ ir] sin I daz O der R \textbf{6} sich schieden] si shieden sich I Si schieden Fr21  $\cdot$ dâ] do Q \textbf{7} ir slâ] diu vart I  $\cdot$ wart] was R  $\cdot$ ê] \textit{om.} I \textbf{8} er verlôs si] Die verlosz er L \textbf{9} vriesch] freýsch L vernam R gefriesch Z  $\cdot$ der] do der O L (Q) (Fr21) (Fr36) da der M Z \textbf{10} herzenôt] herze leit I (L) (M) (Fr21) herzenlæit O (Z) (Fr36) hertzen not Q \textbf{11} ez] Da O M Z Do L Q R Fr21 (Fr36)  $\cdot$ rîche] reichen Q \textbf{12} jæmerlîche] diu was iemerlich I \textbf{13} ez] ÷z O  $\cdot$ dannoch] den acht R \textbf{14} ûf] vnder L (R)  $\cdot$ linden] linde Q \textbf{15} der] dy M  $\cdot$ vuoget] fuͯgte L (M) (Q) (R) (Z) \textbf{16} ein gebalsemet] einen Gebalsmeten I  $\cdot$ tôt] rot L \textbf{17} lent] leint I (O) (Q) (Fr21) Fr36 Lende L Z Leite M lag R  $\cdot$ den] ir Z \textit{om.} Fr21 \textbf{18} \textit{Vers 249.18 fehlt} Q   $\cdot$ den] dem I (O)  $\cdot$ ez] daz L \textbf{19} der] Die Fr51  $\cdot$ si sô] si also G so ritter Fr36  $\cdot$ sitzen] siczende R \textbf{20} \textit{nach 249.20:} Sin ros gieng ander wide / Da weinnet sy ir leide R   $\cdot$ untriwen] Vntrúw R (Fr51) ::triwe Fr36  $\cdot$ ich im] iches im G ich mich M ich dem R ich ir Fr21  $\cdot$ jæhe] yehen Q \textbf{21} dô gein] Gein I er do gein O L er gen R da gein Z her zvͦ Fr51 \textbf{22} der wênic] Der vil wenich O Wie wenig er R Die sie weinich Fr51  $\cdot$ bekande] erkande R Z (Fr51) \textbf{23} muomen] moẏen Fr51 \textbf{24} al irdesch] alle irdische I Al irdigen O Als ir disse Q  $\cdot$ triwe] triwen O trowe R \textbf{25} wan] Mer Fr51 \textbf{26} Parzival] [parzifal]: Parzifal I Parcifal O L Z Fr21 Parzifal M (Fr40) Partzifal Q Parczifal R Parzẏual Fr51  $\cdot$ si gruozte] si gruzt I (O) (Fr21) grvzte sie Z \textbf{27} Vrowe mir ist harde leyt Fr51 \textbf{28} senelîchiu] grozer Fr51 \textbf{29} bedurft] [beroͮcht]: geroͮcht G vnd bedurft I Getorbe Fr51  $\cdot$ dienstes] diens I dienist Fr36 \textbf{30} in iwerem] An ivwen Fr51  $\cdot$ mich] mir Fr51 \newline
\end{minipage}
\hspace{0.5cm}
\begin{minipage}[t]{0.5\linewidth}
\small
\begin{center}*T
\end{center}
\begin{tabular}{rl}
 & \textbf{Der valscheite} widersaz\\ 
 & \textbf{kêrte} ûf der huofslege kraz.\\ 
 & \textbf{sîn scheiden} dan, daz riuwet mich.\\ 
 & \textbf{Nû êrst} âventiuret \textit{ez} sich.\\ 
5 & dô begunde krenken sich ir spor.\\ 
 & sich schieden, die dâ riten vor.\\ 
 & ir slâ wart smal, diu ê was breit.\\ 
 & \textbf{die} verlôs \textbf{er} gar, daz was im leit.\\ 
 & \textbf{\begin{large}M\end{large}ære} vriesch der junge man,\\ 
10 & dâ von er \textbf{herzenôt} gewan.\\ 
 & \textbf{dô erhôrte} der \textbf{helt} ellens rîch\\ 
 & einer vrouwen stimme jæmerlîch.\\ 
 & \textbf{ez} was dannoch von touwe naz.\\ 
 & vor im ûf einer linden saz\\ 
15 & ein maget, der vuocte ir triuwe nôt.\\ 
 & ein gebalsemeter rîter tôt\\ 
 & \textbf{lac an} ir armen.\\ 
 & \textbf{den si} niht wolte erbarmen,\\ 
 & der si \textbf{alsô} sitzen sæhe,\\ 
20 & untriuwen ich im jæhe.\\ 
 & sîn \textbf{ôre} \textbf{er gegen ir} wande,\\ 
 & der wênic si \textbf{erkande}.\\ 
 & si was doch sîner muomen kint.\\ 
 & all\textit{iu} irdensch\textit{iu} triuwe was ein wint,\\ 
25 & wan die\textbf{r} an ir lîbe sach.\\ 
 & Parcifal si gruozte unde sprach.\\ 
 & \textbf{er sprach}: "vrouwe, mir ist leit\\ 
 & iuwer senelîch\textit{iu} arbeit.\\ 
 & bedurfet ir mînes dienstes iht,\\ 
30 & in iuwerm dienste man mich siht."\\ 
\end{tabular}
\scriptsize
\line(1,0){75} \newline
T U V W \newline
\line(1,0){75} \newline
\textbf{1} \textit{Initiale} U W   $\cdot$ \textit{Majuskel} T  \textbf{4} \textit{Majuskel} T  \textbf{9} \textit{Initiale} T  \textbf{21} \textit{Überschrift:} Hie kam parzifal zvͦm anderen male zvͦ sinre nv́ftelen sigvnen V   $\cdot$ \textit{Initiale} V  \newline
\line(1,0){75} \newline
\textbf{2} kraz] tratz W \textbf{4} Nû êrst] Allererst nun W  $\cdot$ ez] \textit{om.} T \textbf{5} ir] sin U [*]: ir V \textbf{6} schieden] scheiden U  $\cdot$ dâ] do U W \textbf{7} diu] die T  $\cdot$ ê] [*]: e V \textbf{9} Mære vriesch] Mere vreischen U Nv vernam V Ein mere vriesch W \textbf{10} [*]: mere do von er not gewan V  $\cdot$ herzenôt] hertzelait W \textbf{11} ellens] al zuͦ U \textbf{17} Lainte ir vnder den armen W \textbf{18} den si] [D*]: Swen ez V Wenn das W \textbf{20} untriuwen] Vntrv́we V \textbf{21} ôre] ors V (W)  $\cdot$ wande] wende U \textbf{22} wênic] wenit U  $\cdot$ erkande] erkende U \textbf{24} alliu irdenschiu] alle irdensce T \textbf{25} wan dier an] [W*]: Wande die man an V Wann do er an W  $\cdot$ lîbe] liebe W \textbf{26} Parcifal] Parzifal V Partzifal W  $\cdot$ si] \textit{om.} U \textbf{27} er sprach] Er [sagete*]: sagete U Er sagete V Vil selig W \textbf{28} senelîchiu] seneliche T semeliche U \newline
\end{minipage}
\end{table}
\end{document}
