\documentclass[8pt,a4paper,notitlepage]{article}
\usepackage{fullpage}
\usepackage{ulem}
\usepackage{xltxtra}
\usepackage{datetime}
\renewcommand{\dateseparator}{.}
\dmyyyydate
\usepackage{fancyhdr}
\usepackage{ifthen}
\pagestyle{fancy}
\fancyhf{}
\renewcommand{\headrulewidth}{0pt}
\fancyfoot[L]{\ifthenelse{\value{page}=1}{\today, \currenttime{} Uhr}{}}
\begin{document}
\begin{table}[ht]
\begin{minipage}[t]{0.5\linewidth}
\small
\begin{center}*D
\end{center}
\begin{tabular}{rl}
\textbf{344} & \textbf{\begin{large}E\end{large}r} ist Poydiconjunzes sun\\ 
 & \textbf{unt} wil \textbf{ouch} rîterschaft hie tuon.\\ 
 & der pfligt der ellens rîche\\ 
 & dicke unverzagetlîche.\\ 
5 & waz \textbf{hilfet} \textbf{sîn} manlîcher site?\\ 
 & ein swînmuoter, lief \textbf{ir} mite\\ 
 & ir verhelîn, \textbf{di\textit{u}} wert ouch sie.\\ 
 & i\textbf{ne} \textbf{gehôrte} man \textbf{geprîsen} nie,\\ 
 & was sîn ellen âne vuoge.\\ 
10 & des volgent \textbf{mir} genuoge.\\ 
 & Hêrre, \textbf{noch} hœret ein wunder,\\ 
 & \textbf{lât iu daz} sagen besunder:\\ 
 & grôz her \textbf{nâch iu dâ} vüeret,\\ 
 & den sîn unvuoge rüeret,\\ 
15 & der künec Meljanz von Liz.\\ 
 & \textbf{hôchvartlîchen} zornes vlîz\\ 
 & hât er gevrumt ân nôt.\\ 
 & unrehtiu minne im daz gebôt."\\ 
 & Der knappe in sîner zuht verjach:\\ 
20 & "hêrre, ich sagez iu, wand ich ez sach:\\ 
 & des \textbf{künec} Meljanzes vater,\\ 
 & \textbf{in} tôdes leger vür sich bat er\\ 
 & die vürsten sînes landes.\\ 
 & \textbf{unerlôst} pfandes\\ 
25 & stuont sîn ellenthaftez leben.\\ 
 & daz muose sich dem tôde ergeben.\\ 
 & in der selben riwe\\ 
 & bevalch er ûf ir triwe\\ 
 & Meljanzen, den clâren,\\ 
30 & allen \textbf{den}, die dâ wâren.\\ 
\end{tabular}
\scriptsize
\line(1,0){75} \newline
D \newline
\line(1,0){75} \newline
\textbf{1} \textit{Initiale} D  \textbf{11} \textit{Majuskel} D  \textbf{19} \textit{Majuskel} D  \newline
\line(1,0){75} \newline
\textbf{1} Poydiconjunzes] Poydiconivnz D \textbf{7} diu] die D \textbf{15} Meljanz] Melianz D  $\cdot$ Liz] Lîz D \textbf{21} Meljanzes] Melianzes D \textbf{29} Meljanzen] Melianzen D \newline
\end{minipage}
\hspace{0.5cm}
\begin{minipage}[t]{0.5\linewidth}
\small
\begin{center}*m
\end{center}
\begin{tabular}{rl}
 & \textbf{er} ist Poidico\textit{ni}u\textit{n}zes sun\\ 
 & \textbf{und} wil ritterschaft hie tuon.\\ 
 & der pfliget der ellens rîche\\ 
 & dicke unverzagelîche.\\ 
5 & waz \textbf{hilfet} \textbf{sîn} manlîcher site?\\ 
 & ein swînmuoter, lief \textbf{ir} mite\\ 
 & ir verhelîn, \textbf{daz} werte ouch sie.\\ 
 & \textit{i}\textbf{\textit{n}e} \textbf{gehôrte} man \textbf{prîsen} nie,\\ 
 & was sîn ellen \textit{â}n vuoge.\\ 
10 & des volgent \textbf{mir} genuoge.\\ 
 & \begin{large}H\end{large}êrre, \textbf{noch} hœret ein wunder,\\ 
 & \textbf{lât iu daz} sagen besunder:\\ 
 & grôz her \textbf{nâch uns d\textit{â}} vüeret,\\ 
 & den sîn unvuoge rüeret,\\ 
15 & der künic Melia\textit{n}z von Liz.\\ 
 & \textbf{hôchverteclîchen} zornes vlîz\\ 
 & hât er gevromet âne nôt.\\ 
 & unrehtiu minne ime daz gebôt."\\ 
 & der knappe in sîner zuht verjach:\\ 
20 & "hêrre, ich sage ez iu, wand ich ez sach:\\ 
 & des \textbf{künic} Melianzes vater,\\ 
 & \textbf{in} tôdes leger vür sich bat er\\ 
 & die vürsten sînes landes,\\ 
 & \textbf{wand} \textbf{unerlœsetes} pfandes\\ 
25 & stuont sîn ellenthaftez leben.\\ 
 & daz muose sich dem tôde ergeben.\\ 
 & in der selben riuwe\\ 
 & b\textit{e}valch er ûf ir triuwe\\ 
 & Melianzen, den clâren,\\ 
30 & allen, die dâ wâren.\\ 
\end{tabular}
\scriptsize
\line(1,0){75} \newline
m n o \newline
\line(1,0){75} \newline
\textbf{11} \textit{Initiale} m   $\cdot$ \textit{Capitulumzeichen} n  \newline
\line(1,0){75} \newline
\textbf{1} Poidiconiunzes] poidi compvmcz m poidiconiuntz n poidicomens o \textbf{2} wil] wil ouch n wil uch o \textbf{3} pfliget] pfligent o  $\cdot$ der ellens] ellens n o \textbf{7} ouch] uch o \textbf{8} ine] Me m Jch n o \textbf{9} ân vuoge] vnfuͯge m \textbf{12} iu] uch uch o \textbf{13} dâ] do m n o \textbf{14} unvuoge] vngefúge o \textbf{15} Melianz] meliacz m meliantz n meliancz o  $\cdot$ Liz] licz m litz n \textbf{16} vlîz] witz n (o) \textbf{17} nôt] bat o \textbf{20} wand] dan o \textbf{21} des] Das n o  $\cdot$ Melianzes] meliantzes n melianczes o \textbf{26} muose] muͯsse m muͯsz n \textbf{28} bevalch er] Bualch er m Befalsch er ir o \textbf{29} Melianzen] Melianczen m Meliantzen n Meleanczen o \textbf{30} allen] Allen den o  $\cdot$ dâ] do n o \newline
\end{minipage}
\end{table}
\newpage
\begin{table}[ht]
\begin{minipage}[t]{0.5\linewidth}
\small
\begin{center}*G
\end{center}
\begin{tabular}{rl}
 & \textbf{der} ist Poydeconiunzes sun;\\ 
 & \textbf{der} wil \textbf{ouch} rîterschaft \textit{hie} tuon.\\ 
 & der pfliget der ellens rîche\\ 
 & dicke unverzagetlîche.\\ 
5 & waz \textbf{touc} \textbf{sîn} manlîcher site?\\ 
 & ein swînemuoter, lief \textbf{im} mite\\ 
 & ir verhelîn, \textbf{diu} werte ouch sie.\\ 
 & ich\textbf{ne} \textbf{hôrte} man \textbf{gebrîsen} nie,\\ 
 & was sîn ellen âne vuoge.\\ 
10 & des volgent \textbf{noch} genuoge.\\ 
 & hêrre, \textbf{nû} hœret ein wunder,\\ 
 & \textbf{lât iu daz} sagen besunder:\\ 
 & grôz her \textbf{dâ nâch iu} vüeret,\\ 
 & den sîn ungevüege rüeret,\\ 
15 & der künic Melianz von Liz.\\ 
 & \textbf{hôchverticlîchen} zornes vlîz\\ 
 & \begin{large}H\end{large}ât er gevrumet \textbf{gar} ân nôt.\\ 
 & unrehtiu minne im daz gebôt."\\ 
 & der knappe in sîner zuht verjach:\\ 
20 & "hêrre, ich sagez iu, wan ichz sach:\\ 
 & des \textbf{künic} Melianzes vater,\\ 
 & \textbf{an}\textbf{s} tôdes leger vür sich bat er\\ 
 & die vürsten sînes landes.\\ 
 & \textbf{unerlôset} pfandes\\ 
25 & stuont sîn ellenthaftez leben.\\ 
 & daz muose sich dem tôde ergeben.\\ 
 & in der selben riwe\\ 
 & bevalch er ûf ir triwe\\ 
 & Melianzen, den clâren,\\ 
30 & allen \textbf{den}, die dâ wâren.\\ 
\end{tabular}
\scriptsize
\line(1,0){75} \newline
G I O L M Q R Z Fr22 Fr40 \newline
\line(1,0){75} \newline
\textbf{1} \textit{Initiale} I O L M Q Z Fr40   $\cdot$ \textit{Capitulumzeichen} R  \textbf{17} \textit{Initiale} G  \newline
\line(1,0){75} \newline
\textbf{1} der] Ez I M ÷r O ER L (Q) (R) (Z) (Fr40)  $\cdot$ ist] \textit{om.} O  $\cdot$ Poydeconiunzes] poideconivnzes G boyde conuinzes I poydiekomvnzes O Poý de Conivnzes L poide kuͯniunzes M poydekvmvnzes Q poidekonivrres R poidekonivnzes Z Poẏde konivnzis Fr22 poydekoniunzez Fr40  $\cdot$ sun] schwester sun R \textbf{2} der] er I vnde O (L) M (Q) (R) (Z) (Fr40)  $\cdot$ rîterschaft] ritterschat Q  $\cdot$ hie] da G \textbf{3} der] des I (L)  $\cdot$ ellens] eren Q \textbf{4} dicke] \textit{om.} I  $\cdot$ unverzagetlîche] vnde verzægeliche O \textbf{5} touc] tuͤt I (R) \textbf{6} Eins schwines muͦtte luff er mitte R  $\cdot$ ein swînemuoter] eins swines muͤter I Eyn swynen Muter M  $\cdot$ im] \textit{om.} L ir Z Fr40 \textbf{7} verhelîn] vierkiln M verhelnen R  $\cdot$ diu] \textit{om.} M das Q (Fr40)  $\cdot$ werte] wert O Q R Fr22 Fr40 virt L  $\cdot$ ouch] \textit{om.} I \textbf{8} ichne] me I Jch O Q R  $\cdot$ hôrte] gehort I (L) R  $\cdot$ gebrîsen] geprisens I \textbf{9} was] \textit{om.} L Das Q  $\cdot$ sîn] sie I  $\cdot$ ellen] eren Q  $\cdot$ âne vuoge] hat vnfuͯge L \textbf{10} volgent] volgeten I volget R  $\cdot$ noch] auch noch I (L) (M) Q (R) (Fr22) (Fr40) mir Z \textbf{11} hêrre] Horte Q \textbf{12} besunder] bisundern M \textbf{13} dâ nâch iu] nach iv da O (R) Fr40 nach auch do Q \textbf{14} ungevüege] vnfuͯge L (Q) (R) (Z) (Fr40) \textbf{15} Melianz] Milianz O melians Q  $\cdot$ Liz] lisz M Q licz R \textbf{16} hôchverticlîchen] suftechlichen I Hochwertiglichen Q \textbf{17} Hât] Het L (Q)  $\cdot$ gar] \textit{om.} R \textbf{18} unrehtiu] vnreht iv G Vnrechtte R  $\cdot$ im] \textit{om.} Z  $\cdot$ gebôt] bott Q \textbf{19} in sîner] an sine I in grosser R \textbf{20} sagez iu] sagev I (R)  $\cdot$ ichz] ich R Z Fr40 \textbf{21} des] von des I  $\cdot$ künic] chunges I (O) (L) (Q) (R) (Fr40)  $\cdot$ Melianzes] Milianzes O Melianz M Malianzes R \textbf{22} andes tode lege er vur sich bat er I  $\cdot$ ans] Jn O L M Q R (Z) (Fr40)  $\cdot$ leger] ger Q  $\cdot$ sich] in L (M) \textbf{25} ellenthaftez] erenthafftes Q \textbf{26} muose] muͤst I  $\cdot$ sich dem tôde] dem tod sich Q \textbf{27} selben] selbe Q R \textbf{28} bevalch] Bevalche O befalle Q Enpfalch Z  $\cdot$ er] \textit{om.} Z  $\cdot$ ir] sin L di Fr40 \textbf{29} Melianzen] Meliansen Q Melianzin Fr22 Melianden Fr40  $\cdot$ den] dem R \textbf{30} den die] die I den R  $\cdot$ dâ] do Q \newline
\end{minipage}
\hspace{0.5cm}
\begin{minipage}[t]{0.5\linewidth}
\small
\begin{center}*T
\end{center}
\begin{tabular}{rl}
 & \textbf{Er} ist Poydekuniunzes suon\\ 
 & \textbf{unde} wil \textbf{ouch} rîterschaft hie tuon.\\ 
 & der pfliget der ellens rîche\\ 
 & dicke unverzagelîche.\\ 
5 & waz \textbf{touc} \textbf{im} manlîcher site?\\ 
 & ein swînmuoter, lief \textbf{im} mite\\ 
 & ir verhelîn, \textbf{di\textit{u}} werte ou\textit{ch} sie.\\ 
 & ich \textbf{hôrte} \textbf{den} man \textbf{gesprechen} nie,\\ 
 & was sîn ellen âne vuoge.\\ 
10 & des volgent \textbf{noch} genuoge.\\ 
 & Hêrre, \textbf{nû} hœret ein wunder,\\ 
 & \textbf{daz lât iu} sagen besunder:\\ 
 & grôz her \textbf{dâ nâch iu} vüeret,\\ 
 & den sîn unvuoge rüeret,\\ 
15 & Der künec Melyanz von Liz.\\ 
 & \textbf{hôchverteclîche} zornes vlîz\\ 
 & hât er gevrumt \textbf{gar} âne nôt.\\ 
 & unreht\textit{iu} minne im daz gebôt."\\ 
 & der knappe in sîner zuht verjach:\\ 
20 & "hêrre, ich sagez iu, wan ichz sach:\\ 
 & Des \textbf{küneges} Melyanzes vater,\\ 
 & \textbf{an} tôdes leger vür sich bat er\\ 
 & die vürsten sînes landes.\\ 
 & \textbf{unerlôset} pfandes\\ 
25 & stuont sîn ellenthaftez leben.\\ 
 & daz muose sich dem tôde ergeben.\\ 
 & in der selbe\textit{n} riuwe\\ 
 & bevalch er ûf ir triuwe\\ 
 & Melyanzen, den clâren,\\ 
30 & allen \textbf{den}, die dâ wâren.\\ 
\end{tabular}
\scriptsize
\line(1,0){75} \newline
T V W \newline
\line(1,0){75} \newline
\textbf{1} \textit{Initiale} W   $\cdot$ \textit{Majuskel} T  \textbf{6} \textit{Initiale} V  \textbf{11} \textit{Majuskel} T  \textbf{15} \textit{Majuskel} T  \textbf{21} \textit{Majuskel} T  \newline
\line(1,0){75} \newline
\textbf{1} Er] DAs W  $\cdot$ Poydekuniunzes] [poy*]: podekvnivnzes T [po*dekvmunzes]: poydekvmvnzes V poyde gumunzes W  $\cdot$ suon] [s*]: swester svn V \textbf{2} hie] \textit{om.} W \textbf{3} der pfliget] [D*]: Der pfliget V Das pfliget W \textbf{5} touc] thuͦt W  $\cdot$ im] sin V (W) \textbf{7} ir] Die W  $\cdot$ diu] die T das V \textit{om.} W  $\cdot$ ouch] ov T \textbf{8} Jnen gehorte man geprisen nie V \textbf{10} noch] mir V auch W \textbf{12} daz lât iu] Lant v́ch [de]: daz V Lat eúch W \textbf{13} dâ nâch iu] do noch [v*]: vch do V do hernach W \textbf{15} Melyanz] melianz V melians W  $\cdot$ Liz] lŷz T lis V W \textbf{17} gar] \textit{om.} V \textbf{18} unrehtiu] vnrehte T \textbf{19} verjach] beiach W \textbf{20} sagez iu] sag eúchs W  $\cdot$ ichz] ich V \textbf{21} küneges] kv́nig V  $\cdot$ Melyanzes] melianzes V W \textbf{22} tôdes leger] tode schlege W \textbf{24} unerlôset] Wande vnerlostes V Vnerloͤstes W \textbf{26} muose] muͤste V  $\cdot$ ergeben] geben W \textbf{27} selben] selber T \textbf{29} Melyanzen] Melianzen V Meliansen W \textbf{30} dâ] do V noch do W \newline
\end{minipage}
\end{table}
\end{document}
