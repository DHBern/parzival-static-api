\documentclass[8pt,a4paper,notitlepage]{article}
\usepackage{fullpage}
\usepackage{ulem}
\usepackage{xltxtra}
\usepackage{datetime}
\renewcommand{\dateseparator}{.}
\dmyyyydate
\usepackage{fancyhdr}
\usepackage{ifthen}
\pagestyle{fancy}
\fancyhf{}
\renewcommand{\headrulewidth}{0pt}
\fancyfoot[L]{\ifthenelse{\value{page}=1}{\today, \currenttime{} Uhr}{}}
\begin{document}
\begin{table}[ht]
\begin{minipage}[t]{0.5\linewidth}
\small
\begin{center}*D
\end{center}
\begin{tabular}{rl}
\textbf{540} & \textbf{\begin{large}Û\end{large}f liez er doch} den wîgant\\ 
 & âne \textbf{gesicherte} hant.\\ 
 & ietweder ûf die bluomen saz.\\ 
 & Gawan sînes kumbers niht vergaz,\\ 
5 & daz sîn pfert was sô kranc.\\ 
 & den wîsen lêrte sîn gedanc,\\ 
 & daz er daz ors mit sporn rite,\\ 
 & unz er versuochte sînen site.\\ 
 & daz was gewâpent wol vür strît;\\ 
10 & pfellel unde samît\\ 
 & was sîn ander covertiur.\\ 
 & sît erz erwarp mit âventiur,\\ 
 & \textbf{durch waz} solt erz \textbf{nû} rîten niht,\\ 
 & sît ez ze rîten im geschiht?\\ 
15 & Er saz drûf, \textbf{dô vuor ez sô},\\ 
 & sîner wîten sprünge \textbf{er was} \textbf{al} vrô.\\ 
 & dô sprach er: "bistû\textbf{z}, Gringuljete,\\ 
 & daz \textbf{Urjans} mit valscher bete\\ 
 & - er weiz wol wie - an mir erwarp?\\ 
20 & dâ von iedoch sîn prîs verdarp\\ 
 & wer hât dich \textbf{sus} gewâpent sider?\\ 
 & ob dûz bist, got hât dich wider\\ 
 & mir schône gesendet,\\ 
 & der dicke kumber wendet."\\ 
25 & Er erbeizte drab, ein \textbf{marc} er vant:\\ 
 & des Grâles wâpen was gebrant,\\ 
 & ein turteltûbe, an \textbf{sînen} buoc.\\ 
 & Læhelin ze\textbf{r} tjoste sluoc\\ 
 & drûffe den von Prienlascors.\\ 
30 & Orilus wart diz ors.\\ 
\end{tabular}
\scriptsize
\line(1,0){75} \newline
D Fr31 \newline
\line(1,0){75} \newline
\textbf{1} \textit{Initiale} D  \textbf{15} \textit{Majuskel} D  \textbf{25} \textit{Majuskel} D  \newline
\line(1,0){75} \newline
\textbf{1} Do liez er vf den wigant Fr31 \textbf{4} Gawan] Ga::n Fr31 \textbf{5} daz] Da:: Fr31 \textbf{8} sînen] sine Fr31 \textbf{9} wol vür strît] so fur s:::te Fr31 \textbf{12} erz] ers Fr31  $\cdot$ âventiur] iuentiur Fr31 \textbf{13} erz] ers Fr31 \textbf{17} Bist du ez sprach er do Gringulet Fr31  $\cdot$ Gringuljete] Gringvliet D \textbf{18} Urjans] Vrians D Fr31 \textbf{21} sus] \textit{om.} Fr31 \textbf{26} was] ez waz Fr31 \textbf{29} drûffe] dvr Fr31  $\cdot$ Prienlascors] p::enles::: Fr31 \newline
\end{minipage}
\hspace{0.5cm}
\begin{minipage}[t]{0.5\linewidth}
\small
\begin{center}*m
\end{center}
\begin{tabular}{rl}
 & \textbf{ûf liez er doch} den wîgant\\ 
 & âne \textbf{sicherheit} hant.\\ 
 & ietweder ûf die bluomen saz.\\ 
 & Gawan sînes kumber\textit{s} niht vergaz,\\ 
5 & daz sîn pfert was sô kranc.\\ 
 & den wîsen lêrte sîn gedanc,\\ 
 & daz er daz ros mit sporn rite.\\ 
 & \multicolumn{1}{l}{ - - - }\\ 
 & daz was gewâpen\textit{t} wol vür strî\textit{t};\\ 
10 & pfelle und samît\\ 
 & was sîn ander c\textit{o}vertiur.\\ 
 & sît erz erwarp mit âventiur,\\ 
 & \textbf{durch waz} solt erz \textbf{nû} rîten niht,\\ 
 & sît ez zuo rîten im geschiht?\\ 
15 & er saz dar ûf, \textbf{dô vuor ez sô},\\ 
 & sîner wîten sprünge \textbf{was \textit{er}} vrô.\\ 
 & dô sprach er: "bistû Gringulete,\\ 
 & daz \textbf{Frians} mit valscher bete\\ 
 & - er weiz wol wi\textit{e} - \textit{a}n mir erwarp?\\ 
20 & dâ von iedoch sîn prîs verdarp\\ 
 & wer het dich \textbf{sus} gewâpent sider?\\ 
 & ob dû ez \textit{bist}, \textit{got} het dich wider\\ 
 & mir schône gesendet,\\ 
 & der dick\textit{e} \textit{k}umber wendet."\\ 
25 & er erbeizte dar ab, ein \dag ros\dag  er \dag rant\dag .\\ 
 & des Grâles wâpen was gebrant,\\ 
 & ein turteltûbe, an \textbf{sînem} buoc.\\ 
 & Lehelin zuo juste sluoc\\ 
 & dar ûf den von Prienlasco\textit{r}s.\\ 
30 & Oriluse wart diz ors.\\ 
\end{tabular}
\scriptsize
\line(1,0){75} \newline
m n o \newline
\line(1,0){75} \newline
\newline
\line(1,0){75} \newline
\textbf{1} liez] liesse n \textbf{2} sicherheit] [sicheit]: sicherheit o \textbf{3} saz] [sach]: sasz n \textbf{4} kumbers] kumer m \textbf{8} \textit{Vers 540.8 fehlt} m n o  \textbf{9} gewâpent] gewoppen m verwoppent n  $\cdot$ strît] stritte m (n) (o) \textbf{10} \textit{Vers 540.10 fehlt} n  \textbf{11} covertiur] conuertur m (n) conútúr o \textbf{13} nû] \textit{om.} n \textbf{15} vuor] fuͦre n \textbf{16} er] \textit{om.} m \textbf{17} Gringulete] gringulette m \textbf{19} wie an] wie er an m [we]: wie an n wo an o \textbf{21} gewâpent] gewoppens n \textbf{22} bist got] got bist m \textbf{24} dicke kumber] dicke du kumber m decke kommer o \textbf{25} er rant] errant n o \textbf{27} sînem] sinen n o  $\cdot$ buoc] buͯge o \textbf{29} den von] von den o  $\cdot$ Prienlascors] prienlascoros m prienlanscros n \newline
\end{minipage}
\end{table}
\newpage
\begin{table}[ht]
\begin{minipage}[t]{0.5\linewidth}
\small
\begin{center}*G
\end{center}
\begin{tabular}{rl}
 & \textbf{\begin{large}Û\end{large}f liez er doch} den wîgant\\ 
 & âne \textbf{gesicherte} hant.\\ 
 & ietweder ûf die bluomen saz.\\ 
 & Gawan sînes kumbers niht vergaz,\\ 
5 & daz sîn pfert was sô kranc.\\ 
 & den wîsen lêrte sîn gedanc,\\ 
 & daz er daz ors mit sporn rite,\\ 
 & \textit{unz er versuochte sîne site.}\\ 
 & daz was gewâpent wol vür strît;\\ 
10 & pfelle unde samît\\ 
 & was sîn ander covertiure.\\ 
 & sît erz erwarp mit âventiure,\\ 
 & \textbf{durch waz} solt erz \textbf{nû} rîten niht,\\ 
 & sît ez ze rîten im geschiht?\\ 
15 & er saz drûf, \textbf{dô vuor ez sô},\\ 
 & sîner wîten sprünge \textbf{er was} \textbf{al}vrô.\\ 
 & dô sprach er: "bistû\textbf{z}, Gringuliet,\\ 
 & daz \textbf{Vrians} mit valscher bet\\ 
 & - er weiz wol wie - an mir erwarp?\\ 
20 & dâ von iedoch sîn brîs verdarp\\ 
 & wer hât dich \textbf{sus} gewâpent sider?\\ 
 & ob dûz bist, got hât dich wider\\ 
 & mir schône gesendet,\\ 
 & der dicke kumber wendet."\\ 
25 & er erbeizte drabe, ein \textbf{marc} er vant:\\ 
 & des Grâles wâpen was gebrant,\\ 
 & ein turteltûbe, an \textbf{sînen} buoc.\\ 
 & Lehelin ze\textbf{r} tjoste sluoc\\ 
 & drûfe den von Prienlacors.\\ 
30 & Orillus wart ditze ors.\\ 
\end{tabular}
\scriptsize
\line(1,0){75} \newline
G I L M Z Fr19 \newline
\line(1,0){75} \newline
\textbf{1} \textit{Initiale} G L Z Fr19  \textbf{13} \textit{Initiale} I M  \newline
\line(1,0){75} \newline
\textbf{1} doch] do L da M \textit{om.} Z \textbf{2} gesicherte] sicherheite L \textbf{3} ietweder] Jetwederer L Jr ichlichir M \textbf{4} Gawan] :::an Fr19  $\cdot$ sînes] \textit{om.} M \textbf{7} daz] diz I  $\cdot$ rite] reit L \textbf{8} \textit{Vers 540.8 fehlt} G   $\cdot$ unz] Vsz M  $\cdot$ versuochte] versuchet Z  $\cdot$ site] siten M \textbf{13} nû] \textit{om.} M \textbf{14} ez] erz L  $\cdot$ ze rîten im] im zeriten I \textbf{15} dô] nv L (M) da Z \textbf{16} sîner] sinen I  $\cdot$ er was] waz er L  $\cdot$ alvrô] fro I L (M) \textbf{17} dô sprach er] er sprach I  $\cdot$ Gringuliet] Gingruliet I Gringuͯliet L gringulet M \textbf{18} Vrians] vrianz I vriansz L vriancz M  $\cdot$ bet] bete G \textbf{19} mir] yme M \textbf{20} iedoch] doch I \textbf{21} sus] \textit{om.} I \textbf{23} gesendet] her gesendet M \textbf{25} erbeizte] enbeiszte L beiszte M  $\cdot$ marc] mal M \textbf{28} zer] ze I (L) \textbf{29} Prienlacors] prienlatsors G prienlatscors I Prienlaiors L prienlatsors M prienlatsioͤrs Z \textbf{30} Orillus] Orilus I Z Oriluͯs L Oriluse M  $\cdot$ ditze] das M  $\cdot$ ors] rorsz L \newline
\end{minipage}
\hspace{0.5cm}
\begin{minipage}[t]{0.5\linewidth}
\small
\begin{center}*T
\end{center}
\begin{tabular}{rl}
 & \textbf{doch liez er ûf} den wîgant\\ 
 & âne \textbf{gesicherte} hant.\\ 
 & ietweder ûf die bluomen saz.\\ 
 & Gawan sînes kumbers niht vergaz,\\ 
5 & daz sîn pfert was sô kranc.\\ 
 & den wîsen lêrte sîn gedanc,\\ 
 & daz er daz ors mit sporn rite,\\ 
 & unz er versuochte sînen site.\\ 
 & daz was gewâpent wol vür strît;\\ 
10 & pfellôr unde samît\\ 
 & was sîn ander covertiure.\\ 
 & sît e\textit{r}z e\textit{r}warp mit âventiure,\\ 
 & \textbf{war umbe} solterz rîten niht,\\ 
 & sît ez ze rîtenne im geschiht?\\ 
15 & er saz drûf, \textbf{ez vuor alsô},\\ 
 & sîner wîten sprünge \textbf{er was} \textbf{al}vrô.\\ 
 & dô sprach er: "bistû\textbf{z}, Krynguliet,\\ 
 & daz \textbf{Vryanz} mit valscher bet\\ 
 & - er weiz wol wie - an mir erwarp?\\ 
20 & dâ von iedoch sîn prîs verdarp\\ 
 & wer hât dich gewâpent sider?\\ 
 & ob dûz bist, got hât dich wider\\ 
 & mir schône gesendet,\\ 
 & der dicke kumber wendet."\\ 
25 & er erbeizete drabe, ein \textbf{mâl} er vant:\\ 
 & des Grâles wâpen was gebrant,\\ 
 & ein turteltûbe, an \textbf{sînen} buoc.\\ 
 & Lehelin ze\textbf{r} tjost sluoc\\ 
 & drûffe den von Prienlaiors.\\ 
30 & Oriluse wart diz ors.\\ 
\end{tabular}
\scriptsize
\line(1,0){75} \newline
T U V W O Q R Fr40 \newline
\line(1,0){75} \newline
\textbf{1} \textit{Initiale} O Q Fr40  \textbf{17} \textit{Initiale} W  \newline
\line(1,0){75} \newline
\textbf{1} doch] ÷a O  $\cdot$ liez] lieze O \textbf{2} âne] Eine R  $\cdot$ gesicherte] sicherheite U \textbf{3} ietweder] Beider site U  $\cdot$ die] den O \textbf{4} Gawan] Gawin R \textbf{6} lêrte] lert Fr40 \textbf{8} unz] Mit U  $\cdot$ versuochte] versucht Q  $\cdot$ sînen] sine U V O (Fr40) \textbf{12} erz erwarp] ez [er warp]: erzwarp T ers erward R \textbf{15} saz] sasz es Q  $\cdot$ ez vuor alsô] do vuͦrez so U (V) (W) (O) (Q) (R) (Fr40) \textbf{16} wîten] weide Q  $\cdot$ er was alvrô] er waz vro V (Fr40) was er fro O \textbf{17} bistûz] bistu W  $\cdot$ Krynguliet] kringalet V kringuliet W (O) (Fr40) kinguleit Q kringulet R \textbf{18} Vryanz] Fryans T vrians U V W O R Fr40 freians Q \textbf{19} wie] wie ers V (Q) wie er R  $\cdot$ mir] mich U V W \textbf{21} dich] dich svs V (W) (R) svs O \textbf{22} dûz bist] du heist Q  $\cdot$ wider] her wider R \textbf{23} gesendet] her gesendet Q \textbf{25} erbeizete] erbalzte W [erbencz]: erbeycz R  $\cdot$ ein mâl] ein marck W (O) (Q) Fr40 \textit{om.} R \textbf{27} sînen] sinē V (Q) seinem W \textbf{28} Lehelin] Lehelyn T Lehalein W Lechlin R lehelein Fr40 \textbf{29} Prienlaiors] prienlascors V prienlashors O prienloiors R \textbf{30} Oriluse] Orilus W (O) Q R (Fr40)  $\cdot$ diz] daz O \newline
\end{minipage}
\end{table}
\end{document}
