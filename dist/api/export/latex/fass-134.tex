\documentclass[8pt,a4paper,notitlepage]{article}
\usepackage{fullpage}
\usepackage{ulem}
\usepackage{xltxtra}
\usepackage{datetime}
\renewcommand{\dateseparator}{.}
\dmyyyydate
\usepackage{fancyhdr}
\usepackage{ifthen}
\pagestyle{fancy}
\fancyhf{}
\renewcommand{\headrulewidth}{0pt}
\fancyfoot[L]{\ifthenelse{\value{page}=1}{\today, \currenttime{} Uhr}{}}
\begin{document}
\begin{table}[ht]
\begin{minipage}[t]{0.5\linewidth}
\small
\begin{center}*D
\end{center}
\begin{tabular}{rl}
\textbf{134} & ir \textbf{en}welt iuch einer site schamen:\\ 
 & ir liezet küneginne namen\\ 
 & unt \textbf{hiezet} durch mich ein herzogîn.\\ 
 & der \textbf{kouf} gît mir ungewin.\\ 
5 & \textbf{\textit{S}în} manheit ist \textbf{doch} sô quec,\\ 
 & daz iwer bruoder Erec,\\ 
 & mîn swâger, fillii roy Lac,\\ 
 & iuch wol \textbf{dâr umbe} hazzen mac.\\ 
 & mich erkennet \textbf{iedoch} der wîse\\ 
10 & an sô \textbf{bewantem} prîse,\\ 
 & der ninder mac entêret sîn,\\ 
 & wan daz er mich vor Prurin\\ 
 & mit sîner tjost valte.\\ 
 & an im ich sît bezalte\\ 
15 & hôhen prîs vor Karnant:\\ 
 & ze rehter tjost \textbf{stach} in mîn hant\\ 
 & hinderz ors durch fîanze.\\ 
 & durch sînen schilt mî\textit{n} lanze\\ 
 & \textbf{iwer} kleinœde brâhte.\\ 
20 & vil wênic ich dô gedâhte\\ 
 & iwerer minne einem anderm trûte,\\ 
 & mîn vrouwe Jeschute.\\ 
 & vrouwe, ir sult gelouben des,\\ 
 & daz der stolze Galoes,\\ 
25 & fillii roy Gandin,\\ 
 & \textbf{tôt lac} von der tjoste mîn.\\ 
 & Ir hielt ouch dâ nâhen bî,\\ 
 & dâ Plihopliheri\\ 
 & gein mir durch tjustieren reit\\ 
30 & unt mich sîn strîten niht vermeit.\\ 
\end{tabular}
\scriptsize
\line(1,0){75} \newline
D \newline
\line(1,0){75} \newline
\textbf{5} \textit{Initiale} D  \textbf{27} \textit{Majuskel} D  \newline
\line(1,0){75} \newline
\textbf{5} Sîn] ÷in D \textbf{6} Erec] Erech D \textbf{7} Lac] lach D \textbf{18} mîn] mit D \textbf{22} Jeschute] Jescvͦte D \textbf{24} Galoes] Gâloes D \textbf{25} Gandin] Gândin D \textbf{28} Plihopliheri] Plihopliherî D \newline
\end{minipage}
\hspace{0.5cm}
\begin{minipage}[t]{0.5\linewidth}
\small
\begin{center}*m
\end{center}
\begin{tabular}{rl}
 & ir wellet iuch einer site schamen:\\ 
 & ir liezet küniginne namen\\ 
 & und \textbf{heizet} durch mich ein herzogîn.\\ 
 & der \textbf{wechsel} gît mir ungewin.\\ 
5 & \textbf{sîn} manheit ist \textbf{iedoch} sô quec,\\ 
 & daz iuwer bruoder Erec,\\ 
 & mîn swâger, filli roi Lac,\\ 
 & iuch wol \textbf{dâr umbe} hazzen mac.\\ 
 & mich erkennet \textbf{iedoch} der wîse\\ 
10 & an sô \textbf{bewantem} prîse,\\ 
 & der niender mac entêret sîn,\\ 
 & wand daz er mich vor Prur\textit{i}n\\ 
 & mit sîner juste valte.\\ 
 & an ime ich sît bezalte\\ 
15 & hôhen prîs vor Karnant:\\ 
 & ze rehter just \textbf{stach} in mî\textit{n h}ant\\ 
 & hinder daz ros durch fîanze.\\ 
 & durch sînen schilt mîn lanze\\ 
 & \textbf{iuwer} kleinœte brâhte.\\ 
20 & vil wênic ich dô gedâhte\\ 
 & iuwerre minne eine\textit{m} anderen trûte,\\ 
 & mîn vrouwe Jeschute.\\ 
 & vrouwe, ir sullet glouben des,\\ 
 & daz der stolze Galoes,\\ 
25 & filli rois Gandin,\\ 
 & \textbf{tôt lac} von der juste mîn.\\ 
 & ir hieltet ouch dâ nâh\textit{e} bî,\\ 
 & dô \textbf{der stolze} Blihabliori\\ 
 & gegen mir durch justieren reit\\ 
30 & und mich sîn strîten niht vermeit.\\ 
\end{tabular}
\scriptsize
\line(1,0){75} \newline
m n o \newline
\line(1,0){75} \newline
\newline
\line(1,0){75} \newline
\textbf{2} liezet küniginne] nessent konige o \textbf{4} der] Vnd o \textbf{5} iedoch] doch n \textbf{6} Erec] erreck m ereg n o \textbf{7} filli] silli n fille o  $\cdot$ roi Lac] roi lag m roẏlag n roilack o \textbf{9} erkennet] erkenne o \textbf{10} bewantem] bewanten n \textbf{11} niender] niergen n \textbf{12} Prurin] prurn m parúrin o \textbf{13} sîner juste] sinen witzen n \textbf{14} ich] icht o \textbf{15} vor] von n \textbf{16} rehter] hoher rechter o  $\cdot$ mîn hant] min lip vnd hant m \textbf{18} mîn] mit o \textbf{19} brâhte] so brochte n \textbf{21} iuwerre] Jre m Jr n o  $\cdot$ einem] einen m n o \textbf{22} Jeschute] jescute m jescúte n jescuͯte o \textbf{24} Galoes] gaoles o \textbf{26} von] vor n \textbf{27} hieltet] hielten n o  $\cdot$ dâ nâhe] da nahẏ m nohe do n do nohe o \textbf{28} Blihabliori] blihoby n bliobẏ o \textbf{29} durch] duͯch o \newline
\end{minipage}
\end{table}
\newpage
\begin{table}[ht]
\begin{minipage}[t]{0.5\linewidth}
\small
\begin{center}*G
\end{center}
\begin{tabular}{rl}
 & ir\textbf{ne} welt iuch einer site schamen:\\ 
 & ir liezet küniginne namen\\ 
 & unde \textbf{heizet} durch mich ein herzogîn.\\ 
 & der \textbf{kouf} gît mir ungewin.\\ 
5 & \textbf{\begin{large}S\end{large}în} manheit, \textbf{diu} ist \textbf{wol} sô kec,\\ 
 & daz iwer bruoder Erec,\\ 
 & mîn \textit{s}w\textit{â}ge\textit{r}, filiroys Lac,\\ 
 & iuch wol \textbf{dâr umbe} hazzen mac.\\ 
 & mich erkennet \textbf{ouch} der wîse\\ 
10 & an sô \textbf{gewandem} brîse,\\ 
 & der ninder mag entêret sîn,\\ 
 & wan daz er mich vor Prurin\\ 
 & mit sîner tjoste valte.\\ 
 & an im ich sît bezalte\\ 
15 & \textbf{vil} hôhen prîs vor Karnant:\\ 
 & ze rehter tjost \textbf{stach} in mîn hant\\ 
 & hinderz ors durch fîanze.\\ 
 & durch sînen schilt mîn lanze\\ 
 & \textbf{iwer} kleinœde brâhte.\\ 
20 & vil wênic ich dô gedâhte\\ 
 & iwerre minne einem anderen trûte,\\ 
 & mîn vrouwe Jeschute.\\ 
 & vrouwe, ir sult gelouben des,\\ 
 & daz der stolze Galoes,\\ 
25 & filliroys Gandin,\\ 
 & \textbf{tôt lac} von der tjoste mîn.\\ 
 & \textit{ir hieltet ouch dâ} nâhe \textit{bî,}\\ 
 & \textit{dô Plioblecheri}\\ 
 & \textit{gein mir durch tjostieren reit}\\ 
30 & \textit{unde mich sîn strît}en \textit{niht vermeit.}\\ 
\end{tabular}
\scriptsize
\line(1,0){75} \newline
G I O L M Q R Z \newline
\line(1,0){75} \newline
\textbf{5} \textit{Initiale} G L R Z  \textbf{7} \textit{Initiale} M  \textbf{17} \textit{Initiale} I  \newline
\line(1,0){75} \newline
\textbf{1} irne] Jr R  $\cdot$ einer] uwe R \textbf{3} \textit{Vers 134.3 fehlt} R   $\cdot$ heizet] hiezet O (M)  $\cdot$ ein] \textit{om.} I M \textbf{4} kouf] kovs L  $\cdot$ gît] der git O L \textbf{5} Sîn] MJn L (Z)  $\cdot$ diu] \textit{om.} I L M Q R Z  $\cdot$ wol] doch O L (M) Q R Z  $\cdot$ sô] \textit{om.} O \textbf{6} Erec] erek G erech O Ereck L (Q) \textbf{7} \textit{Versfolge 134.8-7} G   $\cdot$ swâger] geswige G  $\cdot$ filiroys] filydeRoi I fil der Roy O  $\cdot$ Lac] lach G O L lack Q \textbf{8} iuch wol dâr umbe] evch drumbe wol I \textbf{9} ouch] wol L \textbf{11} der] [Der]: Dar L  $\cdot$ ninder] nerigen M  $\cdot$ entêret] en rerit M \textbf{12} mich] \sout{mich} O  $\cdot$ Prurin] pruͦrin G R Brvrin O prutin Q \textbf{14} an] Am Q  $\cdot$ sît] sich O \textit{om.} M \textbf{15} Karnant] charnant I \textbf{16} ze] mit I  $\cdot$ mîn] \textit{om.} L \textbf{18} \textit{Vers 134.18 fehlt} Q   $\cdot$ sînen] mynen M  $\cdot$ mîn] mit I L \textbf{19} iwer] do ich eweriu I Jtweder Q  $\cdot$ kleinœde] clainende R \textbf{20} vil wênic] Vil Lvzel O (L) (Q) Wie luczil M Vil clainende wenig R  $\cdot$ dô] \textit{om.} I R da M Z  $\cdot$ gedâhte] dachte M \textbf{21} minne] múmen Q  $\cdot$ einem anderen] eimandrem I (O)  $\cdot$ trûte] trug Q \textbf{22} Jeschute] ieschute G I [L*]: Jeschvte O Jescute L M R Jescut Q iescute Z \textbf{23} des] das R \textbf{24} Galoes] Gaoles L \textbf{25} filliroys] Fil lo Roys O  $\cdot$ Gandin] candin I \textbf{26} tôt lac] Lach dot L \textbf{27} \textit{Die Verse 134.27-135.6 fehlen} G   $\cdot$ ir hieltet] Er hielt L (Q)  $\cdot$ dâ nâhe bî] dabi I do nahen bey Q \textbf{28} dô] Da min herre O Da R Z  $\cdot$ Plioblecheri] plihori O plýopheri L plioplihori M philopfieri Q phiophilihon R [phopheheri]: pliopheheri Z \textbf{29} durch tjostieren] tiostierrent R \textbf{30} mich] ich Z  $\cdot$ strîten] strit I \newline
\end{minipage}
\hspace{0.5cm}
\begin{minipage}[t]{0.5\linewidth}
\small
\begin{center}*T (U)
\end{center}
\begin{tabular}{rl}
 & ir welt iuch \textbf{dan} eines siten schamen:\\ 
 & ir liezet \textbf{einer} künegîn namen\\ 
 & und \textbf{hiezet} durch mich ein herzogîn.\\ 
 & der \textbf{kouf} gît mir ungewin.\\ 
5 & \textbf{mîne} manheit ist \textbf{doch wol} sô quec,\\ 
 & daz iuwer bruoder Erec,\\ 
 & mîn swâger, filliroy Lac,\\ 
 & \textbf{er} iuch wol hazzen mac.\\ 
 & mich erkennet \textbf{ouch} der wîse\\ 
10 & an sô \textbf{gewantem} prîse,\\ 
 & der ni\textit{en}der mac entêre\textit{t} sîn,\\ 
 & wan daz er mich vor Prurin\\ 
 & mit sîner tjost valte.\\ 
 & an im ich sît bezalte\\ 
15 & \textbf{vil} hôhen prîs vor Karnant:\\ 
 & zuo rehter tjost \textbf{valt}in mîn hant\\ 
 & hinder\textit{z} ors durch fîanze.\\ 
 & \textit{durch} sînen schilt mîn lanze\\ 
 & \textbf{iuch zuo} kleinœde brâhte.\\ 
20 & vil wênic ich dô gedâhte\\ 
 & iuwer minne eime andern trûte,\\ 
 & mîn vrou Jeschute.\\ 
 & vrouwe, ir solt gelouben des,\\ 
 & daz der stolze Galoes,\\ 
25 & fillirois Gandin,\\ 
 & \textbf{lac tôt} von der tjost mîn.\\ 
 & ir hieltet ouch d\textit{â} nâhe bî,\\ 
 & dô \textbf{hêr} Plyopliori\\ 
 & gein mir \textit{durch} tjostieren reit\\ 
30 & und mich sîn strîten niht vermeit.\\ 
\end{tabular}
\scriptsize
\line(1,0){75} \newline
U V W T \newline
\line(1,0){75} \newline
\textbf{5} \textit{Initiale} W   $\cdot$ \textit{Majuskel} T  \newline
\line(1,0){75} \newline
\textbf{1} welt] enwellent V (W) (T)  $\cdot$ eines siten] eins dings W der rede T \textbf{2} einer] \textit{om.} W T \textbf{3} hiezet durch mich] haissent durch mich W heizet T  $\cdot$ ein] \textit{om.} W \textbf{4} kouf] koͮf der V wehsel T  $\cdot$ ungewin] doch vngewin W \textbf{5} ist] die ist V  $\cdot$ wol] \textit{om.} V T \textbf{6} Erec] ereg V \textbf{7} Filliroy] dez kv́niges svn V  $\cdot$ Lac] lag V lâc T \textbf{8} er iuch wol] V́ch wol darvmbe V (W) îv drvmbe wol T  $\cdot$ hazzen] haissen W \textbf{9} erkennet] erkante W bekennet T  $\cdot$ der] wol der W \textbf{10} gewantem] gewanten V \textbf{11} niender] nider U niergent V  $\cdot$ entêret] enteren U \textbf{12} Prurin] Pruͦrin U \textbf{15} Karnant] Garnant T \textbf{16} valtin] stach in W (T) \textbf{17} hinderz] hinder U \textbf{18} durch sînen] [Siner]: Sinen U \textbf{19} iuch zuo] Eúch ein W îuwer T \textbf{20} vil] wie T \textbf{22} vrou] liebe frauw W  $\cdot$ Jeschute] Jescuͦte U Jescute V (T) iestute W \textbf{23} Ir súllen an zweifel wissen des W  $\cdot$ vrouwe ir solt] ir svlt mir T \textbf{25} Fillirois] Dez kv́niges svn V  $\cdot$ Gandin] Gaudin U (W) \textbf{26} lac tôt] Dort lag W tôt lac T \textbf{27} dâ] do U V W \textbf{28} dô hêr] Do der stoltze W da T  $\cdot$ Plyopliori] Bliobliori U V pliopleheri W Plyopliorj T \textbf{29} durch] \textit{om.} U \textbf{30} mich sîn strîten] mich sines strites V in [mi*]: min strîten T  $\cdot$ vermeit] verm T \newline
\end{minipage}
\end{table}
\end{document}
