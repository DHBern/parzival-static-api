\documentclass[8pt,a4paper,notitlepage]{article}
\usepackage{fullpage}
\usepackage{ulem}
\usepackage{xltxtra}
\usepackage{datetime}
\renewcommand{\dateseparator}{.}
\dmyyyydate
\usepackage{fancyhdr}
\usepackage{ifthen}
\pagestyle{fancy}
\fancyhf{}
\renewcommand{\headrulewidth}{0pt}
\fancyfoot[L]{\ifthenelse{\value{page}=1}{\today, \currenttime{} Uhr}{}}
\begin{document}
\begin{table}[ht]
\begin{minipage}[t]{0.5\linewidth}
\small
\begin{center}*D
\end{center}
\begin{tabular}{rl}
\textbf{281} & \begin{large}Û\end{large}f gerihtiu sper wir müezen sehen.\\ 
 & welt ir danne vür ein ander schehen\\ 
 & alsô vreche rüden, \textbf{den} meisters hant\\ 
 & ab \textbf{stroufet} \textbf{ir} bant,\\ 
5 & dar zuo \textbf{trag} ich niht willen.\\ 
 & ich sol den schal gestillen.\\ 
 & \textbf{ich hilf iu}, swâ \textbf{ez} \textbf{niht rât mac} sîn;\\ 
 & \textbf{des} wartet an daz ellen mîn."\\ 
 & Dise gelübde habt ir wol vernomen.\\ 
10 & welt ir nû hœren, war sî komen\\ 
 & Parzival der Waleis?\\ 
 & von snêwe was ein niwe leis\\ 
 & des nahtes vast ûf in gesnît.\\ 
 & ez \textbf{en}was iedoch niht snêwes zît,\\ 
15 & Ist ez, als ich\textbf{z} vernomen hân.\\ 
 & Artus, der \textbf{meienbære} man,\\ 
 & swaz man \textbf{ie von dem} gesprach,\\ 
 & z\textbf{einen} pfingsten daz geschach\\ 
 & oder in des meien bluomen zît.\\ 
20 & waz man im süezes luftes gît!\\ 
 & \textbf{Diz} mære ist \textbf{hie vaste} undersniten,\\ 
 & ez parriert sich mit snêwes siten.\\ 
 & \textbf{sîne} valkenære von Karidœl\\ 
 & \textbf{riten} sâbents \textbf{zuo} dem Plimizœl\\ 
25 & \textbf{durch} beizen, dâ si schaden kuren:\\ 
 & ir besten valken si verluren.\\ 
 & der \textbf{gâhete} von in balde\\ 
 & unt stuont die naht ze walde.\\ 
 & von überkrüpfe daz geschach,\\ 
30 & daz im was von dem luoder gâch.\\ 
\end{tabular}
\scriptsize
\line(1,0){75} \newline
D \newline
\line(1,0){75} \newline
\textbf{1} \textit{Initiale} D  \textbf{9} \textit{Majuskel} D  \textbf{15} \textit{Majuskel} D  \textbf{21} \textit{Majuskel} D  \newline
\line(1,0){75} \newline
\textbf{23} Karidœl] karidol D \textbf{24} Plimizœl] Plimizol D \newline
\end{minipage}
\hspace{0.5cm}
\begin{minipage}[t]{0.5\linewidth}
\small
\begin{center}*m
\end{center}
\begin{tabular}{rl}
 & ûf gerihtiu sper wir müezen sehen.\\ 
 & welt ir denne vür ein ander schehen\\ 
 & als vreche ruden, \textbf{den} meisters hant\\ 
 & abe \textbf{stroufet} \textbf{diu} bant,\\ 
5 & dar zuo \textbf{hân} ich niht willen.\\ 
 & ich sol den schal gestillen.\\ 
 & \hspace*{-.7em}\big| \textbf{doch} wartet an daz ellen mîn,\\ 
 & \hspace*{-.7em}\big| \textbf{daz ich iu helfe}, wâ \textbf{ez} \textbf{sol} sîn."\\ 
 & \begin{large}D\end{large}ise gelübde habt ir wol vernomen.\\ 
10 & wellet ir nû hœren, war sî komen\\ 
 & Parcifal der Waleis?\\ 
 & von snêwe was ein \textit{n}iuwe leis\\ 
 & des nahtes vaste ûf in gesnît.\\ 
 & ez was iedoch niht snêwes zît,\\ 
15 & ist ez, als ich\textbf{z} vernomen hân.\\ 
 & Artus, der \textbf{meigenbære} man,\\ 
 & waz man \textbf{v\textit{o}n dem ie} ges\textit{pr}ach,\\ 
 & ze \textbf{einer} pfingsten daz geschach\\ 
 & oder in des meigen bluomen zît.\\ 
20 & waz man ime süezes luftes gît!\\ 
 & \textbf{daz} mære ist \textbf{hie vast} undersniten,\\ 
 & ez parrieret sich mi\textit{t} snêwes siten.\\ 
 & \textbf{sîne} valkenære von K\textit{a}ridol\\ 
 & \textbf{riten} sâbents \textbf{ze}m Plimizol\\ 
25 & \textbf{durc\textit{h}} \textit{b}eizen, d\textit{â} si schaden kurn:\\ 
 & ir besten valken si verlurn.\\ 
 & der \textbf{ergâhete} von in balde\\ 
 & und stuont die naht ze\textbf{m} walde.\\ 
 & von überkrü\textit{p}fe daz geschach,\\ 
30 & daz ime was von dem luoder gâch.\\ 
\end{tabular}
\scriptsize
\line(1,0){75} \newline
m n o \newline
\line(1,0){75} \newline
\textbf{9} \textit{Initiale} m   $\cdot$ \textit{Capitulumzeichen} n  \newline
\line(1,0){75} \newline
\textbf{2} schehen] spehen n o \textbf{3} \textit{Versfolge 281.4-3} n   $\cdot$ ruden] enden n \textbf{4} stroufet] trauffte o \textbf{6} gestillen] bestellen o \textbf{10} sî] sú n sie o \textbf{11} Waleis] waleisz o \textbf{12} niuwe leis] ruweleis m n [rubeleis]: rabeleis o \textbf{13} des] Das o \textbf{14} ez was iedoch] >ez< Was edoch o \textbf{16} meigenbære] meẏen beren n (o) \textbf{17} Was man ven dem ye geschach m  $\cdot$ Was man ye von dem gesprach n  $\cdot$ Was mann e von deme gesprach o \textbf{19} des] das o \textbf{21} mære] mir o  $\cdot$ vast undersniten] >fast< vnder snitten o \textbf{22} mit] mich m \textbf{23} Karidol] koridol m \textbf{24} riten sâbents] Rittentz obentz m Rittent des oben n Rittent dez abens o  $\cdot$ zem Plimizol] zuͯ einem plimzol n (o) \textbf{25} durch beizen] Durch h beissen m  $\cdot$ dâ] do m n o \textbf{27} ergâhete] gohete n (o)  $\cdot$ in] \textit{om.} o \textbf{28} zem] zuͯ n (o) \textbf{29} überkrüpfe] v̂ber crvffe m vwer krupff o \newline
\end{minipage}
\end{table}
\newpage
\begin{table}[ht]
\begin{minipage}[t]{0.5\linewidth}
\small
\begin{center}*G
\end{center}
\begin{tabular}{rl}
 & ûf gerihtiu sper wir müezen sehen.\\ 
 & welt ir danne \textit{vü}r ein ander schehen\\ 
 & als vreche rüden \textbf{in} meisters hant,\\ 
 & \textbf{sôn} abe \textbf{gezucket wirt} \textbf{ir} bant,\\ 
5 & dar zuo \textbf{hân} ich niht willen.\\ 
 & ich sol den schal gestillen.\\ 
 & \textbf{ich hilfiu}, swâ\textbf{s} \textbf{niht rât mac} sîn;\\ 
 & \textbf{des} wartet an daz ellen mîn."\\ 
 & dise gelübede habet ir wol vernomen.\\ 
10 & welt ir nû hœren, war sî komen\\ 
 & Parzival der Waleis?\\ 
 & von snêwe was ein niwiu leis\\ 
 & des nahtes vaste ûf in gesnît.\\ 
 & ez was iedoch niht snêwes zît,\\ 
15 & ist \textit{ez}, als ich vernomen hân.\\ 
 & Artus, der \textbf{meigenbære} man,\\ 
 & swaz man \textbf{ie von dem} gesprach,\\ 
 & z\textbf{einen} pfingesten daz geschach\\ 
 & oder in des meigen bluomen zît.\\ 
20 & waz man im süezes luftes gît!\\ 
 & \textbf{daz} mære ist \textbf{hie vaste} undersniten,\\ 
 & ez parriert sich mit snêwes siten.\\ 
 & \textbf{sîne} valkenære \textit{von} Karidol\\ 
 & \textbf{\textit{rit}en} des âbendes \textbf{bî} dem Blimzol\\ 
25 & \textbf{durch} beizen, dâ si schaden kuren:\\ 
 & ir besten valken si verluren.\\ 
 & \begin{large}D\end{large}er \textbf{gâhte} von in balde\\ 
 & unde stuont die naht ze walde.\\ 
 & von überkrüpfe daz geschach,\\ 
30 & daz im was von dem luodere gâch.\\ 
\end{tabular}
\scriptsize
\line(1,0){75} \newline
G I O L M Q R Z Fr30 Fr60 \newline
\line(1,0){75} \newline
\textbf{1} \textit{Initiale} O L R Z  \textbf{21} \textit{Initiale} I  \textbf{23} \textit{Capitulumzeichen} L  \textbf{27} \textit{Initiale} G  \newline
\line(1,0){75} \newline
\textbf{1} ûf] ÷f O  $\cdot$ gerihtiu] gerichiv Fr30 \textbf{2} danne] da O  $\cdot$ vür] wider G den fúr R  $\cdot$ schehen] spen M sechen R \textbf{3} vreche] vrechen Q (R) verh Fr30  $\cdot$ rüden] ruten M  $\cdot$ in] \textit{om.} L M Q R Z Fr30  $\cdot$ meisters] meister Z  $\cdot$ hant] hund R \textbf{4} sôn] \textit{om.} I O L M Q R Z Fr30  $\cdot$ gezucket wirt] zuchtan I zvchende O zuͯchent L (M) (R) (Z) zihent Q  $\cdot$ ir] iriu I ir iv O \textbf{6} gestillen] stillen I R Fr30 \textbf{7} hilfiu] helfe M  $\cdot$ swâs] swa ez I (Z) swa sin O waz L wo iz M (Q) (R)  $\cdot$ rât mac] rat sol I mach rat O (M) \textbf{8} des wartet] deswar tet I Des siet M Das wartet Q des wart Fr30  $\cdot$ an daz ellen] andy elle M \textbf{9} dise] diz I (Fr30) Disiv O Dy Q (R) \textbf{10} hœren] horet L  $\cdot$ war] was R  $\cdot$ komen] kamen L seyn kom Q \textbf{11} Parzival] parzifal I (L) (M) Parcifal O Z Fr30 Partzfal Q Parczifal R  $\cdot$ der] de Q  $\cdot$ Waleis] waleẏs Fr30 \textbf{12} ein niwiu leis] eniweleis I nv niwe lêis O \textbf{14} ez] ezn I (O) (R) (Fr30) Vnd L  $\cdot$ iedoch] doch I  $\cdot$ snêwes] snewens I M (Q) schewes R \textbf{15} ist ez] ist G ez ist I  $\cdot$ ich] ich ez L (Q) (R) (Z) (Fr30) \textbf{16} Artus] Artvse O  $\cdot$ meigenbære] mægebere O mynnebare L manber M muͦgebere R \textbf{17} swaz] Waz L (M) (Q) (R) Z  $\cdot$ dem] im O (M) Q (R) Z Fr60 \textit{om.} L \textbf{18} zeinen] Zcu eȳ M (Q) Ze einem R (Fr30)  $\cdot$ daz] es Q  $\cdot$ geschach] gesach Fr60 \textbf{19} meigen bluomenzît] bluͤnden maien zit I \textbf{20} im] in I Nu M  $\cdot$ süezes] Guͤtes I \textbf{21} daz] Disz L  $\cdot$ ist] ich M  $\cdot$ hie] \textit{om.} O Z  $\cdot$ undersniten] vndersnite Z \textbf{22} parriert] parteret M  $\cdot$ sich] \textit{om.} Z  $\cdot$ snêwes] senuwen I suches M schewes R senewes Fr60 \textbf{23} sîne valkenære] sin valchner I (Q) (R) (Z) Zwene valckenar L  $\cdot$ von] ze G  $\cdot$ Karidol] charidol G kaiedol M kardiol Z \textbf{24} riten des âbendes] waren des abendes G Ritens habenden O (Fr60) Rittent des aubent R  $\cdot$ bî dem] vorm O (M) Fr60 zv dem Z  $\cdot$ Blimzol] blimizol I brimizol O (Fr60) plimizol L Q Z Blimizcol M bimizol R \textbf{25} beizen] heczin M  $\cdot$ dâ si] sie da L des sie M do sie Q (R) \textbf{27} gâhte] iagete M (Q)  $\cdot$ in] yme M (Q) \textbf{28} die naht] zenaht O  $\cdot$ walde] felle Q \textbf{29} geschach] beschach R \textbf{30} was] \textit{om.} Q Fr30  $\cdot$ dem luodere] den levten O luͯder L dem :::en Fr60  $\cdot$ gâch] wasz gach Q \newline
\end{minipage}
\hspace{0.5cm}
\begin{minipage}[t]{0.5\linewidth}
\small
\begin{center}*T
\end{center}
\begin{tabular}{rl}
 & ûf gerihtiu sper wir müezen sehen.\\ 
 & welt ir danne vür ein ander schehen,\\ 
 & als vreche rüden \textbf{ûz} meisters hant\\ 
 & abe \textbf{zuckent} \textbf{ir} bant,\\ 
5 & dar zuo \textbf{hab}ich niht willen.\\ 
 & ich sol den schal gestillen.\\ 
 & \textbf{ich hilfiu}, swâ\textbf{z} \textbf{niht rât mac} sîn;\\ 
 & \textbf{des} wartet an daz ellen mîn."\\ 
 & \begin{large}D\end{large}ise gelübede habt ir wol vernomen.\\ 
10 & welt ir nû hœren, war sî komen\\ 
 & Parcifal der Waleis?\\ 
 & von snê was ein niuwe leis\\ 
 & des nahtes vaste ûf in gesnît.\\ 
 & ez was iedoch niht snêwes zît,\\ 
15 & ist ez, als ich\textbf{z} vernomen hân.\\ 
 & Artus, der \textbf{niuwebære} man,\\ 
 & swaz man \textbf{ie von dem} gesprach,\\ 
 & Z\textbf{einen} pfingesten daz geschach\\ 
 & oder in des meien bluomen zît.\\ 
20 & waz man im süezes luftes gît!\\ 
 & \textbf{diz} mære ist \textbf{vaste hie} undersniten,\\ 
 & ez parrieret sich mit snêwes siten.\\ 
 & \textbf{Ein} valkenære von Karidol\\ 
 & \textbf{reit} des âbendes \textbf{bî} dem Plymizol\\ 
25 & beizen, \textbf{al} dâ si schaden kurn:\\ 
 & ir besten valken si verlurn.\\ 
 & der \textbf{gâhte} von in balde\\ 
 & unde stuont die naht ze walde.\\ 
 & von überkrüpfe daz geschach,\\ 
30 & daz im was vonme luoder gâch.\\ 
\end{tabular}
\scriptsize
\line(1,0){75} \newline
T U V W \newline
\line(1,0){75} \newline
\textbf{9} \textit{Überschrift:} Hie kvmet parzifal zvͦm anderen male zvͦ kv́nig artvs hof do er bi dem grale waz gesin >vnde bi anfortas< V   $\cdot$ \textit{Initiale} T U V W  \textbf{18} \textit{Majuskel} T  \textbf{23} \textit{Majuskel} T  \newline
\line(1,0){75} \newline
\textbf{1} gerihtiu] gerechte U \textbf{2} ir] \textit{om.} W  $\cdot$ schehen] spehen V W \textbf{3} ûz] [*]: den V \textbf{4} zuckent] zucken U zúchten W  $\cdot$ ir] ire V \textbf{5} habich] habete ich U  $\cdot$ niht] [*]: niht V mit W \textbf{7} hilfiu] helf vch U  $\cdot$ swâz] waz U (W) swa V  $\cdot$ niht rât mac] [*]: ez sol V ich mag daz muͦß W \textbf{8} des wartet] [D*]: Dez wartent V Das wartent W  $\cdot$ ellen] elle U \textbf{9} Dise] DIe W \textbf{10} war] wer W \textbf{11} Parcifal] Parzifal T V Partzifal W  $\cdot$ Waleis] walleis V \textbf{13} in] \textit{om.} W  $\cdot$ gesnît] snit U \textbf{14} was] \textit{om.} W  $\cdot$ niht] \textit{om.} U \textbf{15} ichz] ich W \textbf{16} niuwebære] [*]: meigenbere V lobebere W \textbf{17} swaz] Waz U (W)  $\cdot$ man ie] [ie doch]: doch ie U ie W \textbf{18} Zeinen pfingesten] [Zeî*]: Zeînen [pfin*]: pfinkesten T \textbf{19} oder] Recht W \textbf{20} waz] Wande V  $\cdot$ man] \textit{om.} W \textbf{21} diz] Dise U  $\cdot$ vaste hie] hie vaste U (W)  $\cdot$ undersniten] vnderstritten W \textbf{23} Ein] [Sir*]: Sine V Die W \textbf{24} reit] [Rit*]: Rittent V Ritten W  $\cdot$ des] eins W  $\cdot$ Plymizol] plimizol V W \textbf{25} beizen] Dvrch beizen V (W)  $\cdot$ al] \textit{om.} U V W \textbf{28} stuont] bestuͦnt U (W) [*]: stuͦnt  V \textbf{29} überkrüpfe] v́ber kroͤphen V \textbf{30} was] do waz W  $\cdot$ luoder] luͦden U \newline
\end{minipage}
\end{table}
\end{document}
