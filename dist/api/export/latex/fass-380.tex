\documentclass[8pt,a4paper,notitlepage]{article}
\usepackage{fullpage}
\usepackage{ulem}
\usepackage{xltxtra}
\usepackage{datetime}
\renewcommand{\dateseparator}{.}
\dmyyyydate
\usepackage{fancyhdr}
\usepackage{ifthen}
\pagestyle{fancy}
\fancyhf{}
\renewcommand{\headrulewidth}{0pt}
\fancyfoot[L]{\ifthenelse{\value{page}=1}{\today, \currenttime{} Uhr}{}}
\begin{document}
\begin{table}[ht]
\begin{minipage}[t]{0.5\linewidth}
\small
\begin{center}*D
\end{center}
\begin{tabular}{rl}
\textbf{380} & \textbf{\begin{large}D\end{large}ô} \textbf{ersach} mîn hêr Gawan,\\ 
 & daz gevlohten was der plân,\\ 
 & die vriwent in der vîende schar.\\ 
 & er huob ouch \textbf{sich} \textbf{mit poynder} dar.\\ 
5 & müelîche \textbf{sîn was} ze warten.\\ 
 & diu ors \textbf{doch} wênec sparten\\ 
 & Scherules unt die sîne.\\ 
 & \textbf{Gawan si brâhte} in pîne.\\ 
 & waz er dâ ritter nider stach\\ 
10 & \textbf{unt} waz er starker sper zerbrach,\\ 
 & der \textbf{werden} tavelrunde bote,\\ 
 & het er die kraft niht von gote,\\ 
 & sô wære dâ prîs vür in gegert.\\ 
 & dâ wart erklenget manec swert.\\ 
15 & im wâren alein beidiu her,\\ 
 & gein den was sîn hant ze wer,\\ 
 & die von Liz unt die von Gors.\\ 
 & von bêder sît er maneg ors\\ 
 & gezogen brâhte schiere\\ 
20 & zuo sînes wirtes baniere.\\ 
 & \textbf{er} vrâgte, obs iemen wolte dâ.\\ 
 & \textbf{der} was \textbf{dâ vil}, die sprâchen jâ.\\ 
 & si wurden \textbf{al} gelîche\\ 
 & sîner geselleschefte rîche.\\ 
25 & \textbf{Dô} kom ein ritter her gevarn,\\ 
 & der ouch diu sper niht kunde sparn.\\ 
 & \textbf{der burcgrâve von} Beaveis\\ 
 & unt Gawan, der kurteis,\\ 
 & kômen an ein ander,\\ 
30 & daz der junge Lisavander\\ 
\end{tabular}
\scriptsize
\line(1,0){75} \newline
D \newline
\line(1,0){75} \newline
\textbf{1} \textit{Initiale} D  \textbf{25} \textit{Majuskel} D  \newline
\line(1,0){75} \newline
\textbf{7} Scherules] Scervles D \textbf{17} Liz] Lŷz D \textbf{30} Lisavander] Lysavandr D \newline
\end{minipage}
\hspace{0.5cm}
\begin{minipage}[t]{0.5\linewidth}
\small
\begin{center}*m
\end{center}
\begin{tabular}{rl}
 & \textbf{\begin{large}D\end{large}ô} \textbf{ersach} mîn hêr G\textit{a}wan,\\ 
 & daz gevlohten was der plân,\\ 
 & die vriunt in der vîende schar.\\ 
 & er huop ouch \textbf{sich} \textbf{mit poinder} dar.\\ 
5 & \dag in welich wîs\dag  \textbf{sîn} ze warten.\\ 
 & diu ros \textbf{doch} wênic sparten\\ 
 & Sch\textit{e}r\textit{ul}es und \textbf{alle} die sîne.\\ 
 & \textbf{die brâhte Gawan} in pîne.\\ 
 & waz er d\textit{â} ritter nider stach\\ 
10 & \textbf{und} waz er starker sper zerbrach,\\ 
 & der \textbf{werden} tavelrunder bote,\\ 
 & hete er die kraft niht von gote,\\ 
 & sô wære dâ prîs vür in gegert.\\ 
 & d\textit{â} wart erklenget manic swert.\\ 
15 & im wâre\textit{n} alein beidiu her,\\ 
 & gegen den was sîn hant ze wer,\\ 
 & die von Liz und die von Gros.\\ 
 & von beider sîte er manic ros\\ 
 & gezogen brâhte schiere\\ 
20 & zuo sînes wirtes baniere.\\ 
 & \textbf{er} vrâgete, ob si \textit{iem}en wolte d\textit{â}.\\ 
 & \textbf{der} was \textbf{dâ vil}, \textit{die s}pr\textit{â}chen \textit{jâ}.\\ 
 & si wurden \textbf{alle} gelîche\\ 
 & sîner geselleschaft rîche.\\ 
25 & \textbf{dô} kam ein ritter her gevarn,\\ 
 & der ouch diu s\textit{p}er niht kunde sparn.\\ 
 & \textbf{laschahtelis de} B\textit{e}a\textit{v}ois\\ 
 & und Gawan, de\textit{r} kurtois,\\ 
 & kômen an ein ander,\\ 
30 & daz der junge Lisava\textit{n}der\\ 
\end{tabular}
\scriptsize
\line(1,0){75} \newline
m n o \newline
\line(1,0){75} \newline
\textbf{1} \textit{Initiale} m n o  \newline
\line(1,0){75} \newline
\textbf{1} ersach] er sach m n o  $\cdot$ mîn hêr] mẏnen herren o  $\cdot$ Gawan] gewan m \textbf{2} gevlohten] geflockten o \textbf{3} die vriunt] Der vint n  $\cdot$ vîende] vigenden n \textbf{7} Scherules] Schirmes m n o \textbf{9} dâ] do m n o \textbf{13} wære] er o  $\cdot$ dâ] do n o  $\cdot$ gegert] begert n o \textbf{14} dâ] Do m n o  $\cdot$ erklenget] er clinget n \textbf{15} wâren] warem m  $\cdot$ her] [heren]: her o \textbf{16} ze wer] sower o \textbf{17} Liz] lisz n liesz o  $\cdot$ von] vor n  $\cdot$ Gros] grosz n gors o \textbf{18} manic] magnig o \textbf{19} brâhte] brocke o \textbf{21} iemen] einen m ẏeman sú n  $\cdot$ dâ] dar m do n o \textbf{22} was dâ] [do]: was do n wasz do o  $\cdot$ die sprâchen jâ] gesprochen dar m \textbf{23} wurden] wurde o  $\cdot$ gelîche] gliche \sout{siner} o \textbf{26} sper] sprer m \textbf{27} laschahtelis] Laschathelún n Laschahtelún o  $\cdot$ Beavois] branois m n beanois o \textbf{28} der] de m n o  $\cdot$ kurtois] turtois n tortonis o \textbf{30} Lisavander] lisavanfander m lisarander n o \newline
\end{minipage}
\end{table}
\newpage
\begin{table}[ht]
\begin{minipage}[t]{0.5\linewidth}
\small
\begin{center}*G
\end{center}
\begin{tabular}{rl}
 & \textbf{nû} \textbf{sach} mîn hêr Gawan,\\ 
 & daz gevlohten was der plân,\\ 
 & die vriunde \textit{i}n der vînde schar.\\ 
 & er huop ouch \textbf{sich} \textbf{des endes} dar.\\ 
5 & müelîch \textbf{was sîn} ze warten.\\ 
 & diu ors \textbf{dô} wênic sparten\\ 
 & Tscherules unde die sîne.\\ 
 & \textbf{Gawan si brâht} in pîne.\\ 
 & \begin{large}W\end{large}az er dâ rîter nider stach,\\ 
10 & waz er \textbf{dâ} starker sper zerbrach,\\ 
 & der \textbf{werde} tavelrunde bote,\\ 
 & het er die kraft niht von gote,\\ 
 & sô wære dâ prîs vür in gegert.\\ 
 & dâ wart erklenget manic swert.\\ 
15 & im wâren \textit{aleine} beidiu her,\\ 
 & gein den was sîn hant ze wer,\\ 
 & die von Liz unde die von Gors.\\ 
 & von beider sîte er mani\textit{c} ors\\ 
 & gezogen brâhte schiere\\ 
20 & zuo sînes wirtes baniere\\ 
 & \textbf{unde} vrâgte, op si iemen wolte dâ.\\ 
 & \textbf{ir} was \textbf{genuoc}, die sprâchen jâ.\\ 
 & si wurden \textbf{al}gelîche\\ 
 & sîner geselleschefte rîche.\\ 
25 & \textbf{nû} kom ein rîter her gevaren,\\ 
 & der ouch diu sper niht kunde sparen.\\ 
 & \textbf{der burcgrâve von} Beavoys\\ 
 & unde Gawan, der kurtoys,\\ 
 & \textbf{die} kômen an ein ander,\\ 
30 & daz der junge Lisavander\\ 
\end{tabular}
\scriptsize
\line(1,0){75} \newline
G I O L M Q R Z Fr21 \newline
\line(1,0){75} \newline
\textbf{1} \textit{Initiale} I O L M Z Fr21   $\cdot$ \textit{Capitulumzeichen} R  \textbf{9} \textit{Initiale} G  \textbf{15} \textit{Initiale} I  \newline
\line(1,0){75} \newline
\textbf{1} \textit{Die Verse 370.13-412.12 fehlen} Q   $\cdot$ nû] ÷v O Hv M  $\cdot$ mîn] \textit{om.} R \textbf{2} gevlohten] vlochtin M \textbf{3} in der] vnder G \textbf{4} ouch sich] sich ouch M ich och R  $\cdot$ des endes] mit poynder O (L) (Z) (Fr21) mit poynde M mit strite R \textbf{5} was sîn] si was O sin waz L (M) (R) (Z) (Fr21) \textbf{6} dô] da M Z er R  $\cdot$ sparten] gesparten O L (M) sparte R \textbf{7} Tscherules] Scurles I Tschervles O Fr21 Thervles L Scerules M Therules R  $\cdot$ die] alle die I (M) R (Z) (Fr21)  $\cdot$ sîne] sinen R \textbf{8} pîne] pinen R \textbf{9} dâ] do O R  $\cdot$ stach] [sprach]: :prach Z \textbf{10} waz] vnde waz O (L) (M) (R) (Z) (Fr21)  $\cdot$ dâ] \textit{om.} O L R Z Fr21  $\cdot$ zerbrach] zv stach Z \textbf{11} der] die I  $\cdot$ werde] werden L Z  $\cdot$ tavelrunde] Tauelrunder I (O) (M) (R) (Z) (Fr21) \textbf{12} die] der Fr21  $\cdot$ niht von] [vor]: von M \textbf{13} dâ] do R  $\cdot$ prîs] prises L R \textbf{14} dâ] Do R  $\cdot$ erklenget] erzogn Fr21 \textbf{15} im] Jn Fr21  $\cdot$ aleine] gelic G allu R \textbf{17} Liz] lis I lýz L lisz M Loys R liez Z  $\cdot$ Gors] Goͤrse O Gorsz L gorcz M Goͤrss Z \textbf{18} von beider sîte] [*]: zebeider siten I Von beiden sitten R  $\cdot$ er] \textit{om.} Z  $\cdot$ manic] mangen G \textbf{19} gezogen] Er gezogen Z  $\cdot$ brâhte] brahter I \textbf{21} vrâgte] fragt I O Fr21 \textbf{25} her] dar I \textbf{26} sper] spern R  $\cdot$ kunde] \textit{om.} Z \textbf{27} \textit{Versfolge 380.28-27} I   $\cdot$ der] vnde der I  $\cdot$ von] \textit{om.} I  $\cdot$ Beavoys] beauoys G beafoys I Beavoẏs O Beaveis L beareis R Bearoys Fr21 \textbf{29} die] \textit{om.} Z  $\cdot$ ein] an I \textbf{30} Lisavander] lisauander G lisamander I Liswander O Lýsavander L lisavandir M Lysavander R Z ::: Fr21 \newline
\end{minipage}
\hspace{0.5cm}
\begin{minipage}[t]{0.5\linewidth}
\small
\begin{center}*T
\end{center}
\begin{tabular}{rl}
 & \textbf{Dô} \textbf{sach} mîn hêr Gawan,\\ 
 & daz gevlohten was der plân,\\ 
 & die vriunt in der vîende schar.\\ 
 & er huop ouch \textbf{mit poynder} dar.\\ 
5 & müelîch \textbf{sîn was} ze warten.\\ 
 & diu ors \textbf{si} wênic sparten,\\ 
 & Tscherules unde \textbf{al} die sîne.\\ 
 & \textit{\textbf{Gawan si brâhte}} \textit{in pîne.}\\ 
 & waz er dâ rîter nider stach\\ 
10 & \textbf{unde} waz er starker sper zerbrach,\\ 
 & der \textbf{werden} tavelrunder bote,\\ 
 & het er die kraft niht von gote,\\ 
 & sô wære dâ prîs vür in gegert.\\ 
 & dâ wart erklenget manec swert.\\ 
15 & im wâren alein beidiu her,\\ 
 & gegen den was sîn hant ze wer,\\ 
 & die von Lyz unde die von Gors.\\ 
 & von beider sît \textit{er} manec ors\\ 
 & gezogen brâhte schiere\\ 
20 & ze sînes wirtes baniere\\ 
 & \textbf{unde} vrâgete \textbf{si}, ob si ieman wolte dâ.\\ 
 & \textbf{ir} was \textbf{genuoc}, die sprâchen jâ.\\ 
 & si wurden \textbf{al}gelîche\\ 
 & sîner geselleschefte rîche.\\ 
25 & \textbf{\begin{large}D\end{large}\textit{ô}} kom ein rîter her gevarn,\\ 
 & der ouch diu sper niht kunde sparn.\\ 
 & \textbf{der burcgrâve von} Beavoys\\ 
 & unde Gawan, der kurtoys,\\ 
 & \textbf{die} kômen an ein ander,\\ 
30 & daz der junge Lysavander\\ 
\end{tabular}
\scriptsize
\line(1,0){75} \newline
T V W \newline
\line(1,0){75} \newline
\textbf{1} \textit{Initiale} W   $\cdot$ \textit{Majuskel} T  \textbf{25} \textit{Initiale} T  \newline
\line(1,0){75} \newline
\textbf{1} Dô] NVn W \textbf{4} ouch] oͮch sich V (W)  $\cdot$ poynder] poyndier T \textbf{6} si] do W \textbf{7} Tscherules] Schervles V Descherules W \textbf{8} \textit{Vers 380.8 fehlt (Zeile ausgespart)} T  \textbf{9} dâ] do V W \textbf{10} er] er do W \textbf{11} werden] werde W \textbf{13} dâ prîs] do pris V do peises W \textbf{14} dâ] Do V W \textbf{17} Lyz] Lŷz T lis V W  $\cdot$ Gors] gorß W \textbf{18} er] \textit{om.} T \textbf{21} si ob] ob sv́ V (W)  $\cdot$ dâ] do V \textbf{22} ir was] Ja V \textbf{25} Dô] Dv T Nv V (W) \textbf{26} diu] der V \textbf{27} Beavoys] Beavôys T Beavoẏs V beaueis W \textbf{30} Lysavander] lẏsavander V lysauander W \newline
\end{minipage}
\end{table}
\end{document}
