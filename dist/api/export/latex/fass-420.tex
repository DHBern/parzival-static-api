\documentclass[8pt,a4paper,notitlepage]{article}
\usepackage{fullpage}
\usepackage{ulem}
\usepackage{xltxtra}
\usepackage{datetime}
\renewcommand{\dateseparator}{.}
\dmyyyydate
\usepackage{fancyhdr}
\usepackage{ifthen}
\pagestyle{fancy}
\fancyhf{}
\renewcommand{\headrulewidth}{0pt}
\fancyfoot[L]{\ifthenelse{\value{page}=1}{\today, \currenttime{} Uhr}{}}
\begin{document}
\begin{table}[ht]
\begin{minipage}[t]{0.5\linewidth}
\small
\begin{center}*D
\end{center}
\begin{tabular}{rl}
\textbf{420} & \begin{large}D\end{large}â rechet\textbf{z}. ich entet im niht.\\ 
 & ich wæne, mirs ouch \textbf{iemen} giht.\\ 
 & Iwern vetern sol ich wol verklagen.\\ 
 & sîn sun die krône nâch im \textbf{sol} tragen.\\ 
5 & der ist mir \textbf{ze hêrren hôch} genuoc.\\ 
 & diu künegîn Flurdamurs in truoc.\\ 
 & sîn vater \textbf{was} Kingrisin,\\ 
 & sîn an der künec Gandin.\\ 
 & ich wil iuch \textbf{baz} bescheiden des:\\ 
10 & Gahmuret unt Galoes\\ 
 & sîne œheime wâren.\\ 
 & i\textbf{ne} \textbf{wolte} sîn \textbf{gerne} vâren,\\ 
 & ich mohte mit \textbf{êren} von sîner hant\\ 
 & mit vanen enpfâhen \textbf{mîn} lant.\\ 
15 & Swer vehten welle, der tuo \textbf{daz}.\\ 
 & \textbf{bin ich} gein dem strîte laz,\\ 
 & \textbf{ich vreische iedoch} \textbf{diu mære} wol.\\ 
 & swer prîs \textbf{ime} strîte \textbf{hol},\\ 
 & des danken im diu stolzen wîp.\\ 
20 & ich wil durch niemen mînen lîp\\ 
 & verleiten in ze scharpfen pîn.\\ 
 & waz Wolfhartes \textbf{solt} ich sîn?\\ 
 & mir ist in den strît der wec vergraben,\\ 
 & gegen vehten \textbf{diu gir} verhabt.\\ 
25 & wurdet ir mir\textbf{s} nimmer holt,\\ 
 & ich tæte ê als Rumolt,\\ 
 & \textbf{der} \textbf{künec} Gunther riet,\\ 
 & dô er von Wormeze gein den Hiunen schiet:\\ 
 & er bat \textbf{in} lange sniten bæn\\ 
30 & \textbf{und} in \textbf{sîme} kezzel umbe dræn."\\ 
\end{tabular}
\scriptsize
\line(1,0){75} \newline
D Fr5 \newline
\line(1,0){75} \newline
\textbf{1} \textit{Initiale} D Fr5  \textbf{15} \textit{Majuskel} D  \newline
\line(1,0){75} \newline
\textbf{1} entet] tet Fr5 \textbf{2} iemen] nieman Fr5 \textbf{4} die krône nâch im sol tragen] sol die krone nah im tragin Fr5 \textbf{5} ze] zieime Fr5 \textbf{6} Flurdamurs] flurdarmurs Fr5 \textbf{7} Kingrisin] kyngrisin Fr5 \textbf{8} sîn] Vnd der Fr5 \textbf{10} Gahmuret] Gahmvret D Gahmureth Fr5 \textbf{12} sîn gerne] gerne sin Fr5 \textbf{13} mit êren] mit eren wol Fr5 \textbf{15} tuo] tuͦ ouch Fr5 \textbf{20} wil] ni wil Fr5 \textbf{21} in] hin Fr5 \textbf{22} Wolfhartes] wolfhartis Fr5  $\cdot$ solt] sol Fr5 \textbf{23} in den] inme Fr5  $\cdot$ vergraben] virgrabit Fr5 \textbf{26} Rumolt] Rvͦmolt D kimolt Fr5 \textbf{27} künec] dem kunige Fr5  $\cdot$ Gunther] givnther Fr5 \textbf{28} Wormeze] wormis Fr5 \textbf{30} in sîme] indem Fr5 \newline
\end{minipage}
\hspace{0.5cm}
\begin{minipage}[t]{0.5\linewidth}
\small
\begin{center}*m
\end{center}
\begin{tabular}{rl}
 & dâ rech\textit{et} \textbf{daz}. ich entet ime niht.\\ 
 & ich wæne, mirs ouch \textbf{ieman} giht.\\ 
 & \textit{iu}weren vetern sol ich wol verklagen.\\ 
 & sîn sun die krône nâch ime \textbf{sol} tragen.\\ 
5 & der ist mir \textbf{hôch ze hêrren} genuoc.\\ 
 & diu künigîn Flurdanurs \textit{in} truoc.\\ 
 & sîn vater \textbf{der hiez} Kingrisin\\ 
 & \textbf{und} sîn ane der künic Gandin.\\ 
 & ich wil iuch \textbf{baz} bescheiden des:\\ 
10 & Gahmuret und Galoes\\ 
 & sîne ôheim wâren.\\ 
 & \textit{ich} \textbf{wolte} sîn \textbf{gerne} vâren,\\ 
 & ich mohte mit \textbf{êre} von sîner hant\\ 
 & mit vanen enpfâhen \textbf{sîn} lant.\\ 
15 & wer \textit{v}ehten welle, der tuo \textbf{ez}.\\ 
 & \textbf{ich bin} gegen  strîte laz,\\ 
 & \textbf{iedoch vreische ich} \textbf{die \textit{mær}e} wol.\\ 
 & wer prîs \textbf{in dem} strîte \textbf{hol},\\ 
 & des danken ime diu stolzen wîp.\\ 
20 & ich wil durch niemen mînen lîp\\ 
 & verleiten in ze scharfen pîn.\\ 
 & waz Wolfhartes \textbf{sol} ich sîn?\\ 
 & mir ist in den strît der wec vergrabet,\\ 
 & gegen vehten \textbf{diu gir} verhabet.\\ 
25 & wurdet ir mir\textbf{s} niemer holt,\\ 
 & ich tæte ê als Rumolt,\\ 
 & \textbf{der} \textbf{künic} Gunthere riet,\\ 
 & dô er von Urmesse gegen den Hunden schiet:\\ 
 & er bat \textbf{in} lange \dag sutten bî\dag \\ 
30 & \textbf{und} i\textit{n} \textbf{\textit{s}înem} kezzele umb dræn."\\ 
\end{tabular}
\scriptsize
\line(1,0){75} \newline
m n o \newline
\line(1,0){75} \newline
\newline
\line(1,0){75} \newline
\textbf{1} dâ] Do n o  $\cdot$ rechet] rehte m \textbf{3} iuweren] Weren m \textbf{6} Flurdanurs] flandarmurs n flurdammurs o  $\cdot$ in] \textit{om.} m \textbf{8} Gandin] gaudin n \textbf{9} bescheiden] scheiden o \textbf{10} Gahmuret] Gamuret n Gamuͯret o \textbf{11} sîne] Sin o \textbf{12} ich] Nie m \textbf{13} ich] Vnd n o  $\cdot$ mohte] \textit{om.} n wolte o  $\cdot$ êre] eren n o  $\cdot$ sîner] sinen o \textbf{14} sîn] min n (o) \textbf{15} vehten welle] reht enwelle m wehten welle o  $\cdot$ ez] das n (o) \textbf{16} strîte] dem strite n o \textbf{17} vreische] frisch o  $\cdot$ mære] wile m \textbf{19} danken] danckent n \textbf{21} ze scharfen] so harte n so scharffe o \textbf{22} Wolfhartes] wolff hartes n o  $\cdot$ sol] solt n o \textbf{24} verhabet] verhat m vergahet o \textbf{25} mirs] mẏnns o \textbf{26} Rumolt] ruͯmolt o \textbf{27} Gunthere] gúnther n gunther o \textbf{28} Urmesse] wurms n wormsz o  $\cdot$ Hunden] huͯnden m húnen n huͯnen o \textbf{29} sutten] sotten o  $\cdot$ bî] bern n o \textbf{30} in] jn den m  $\cdot$ dræn] dern n o \newline
\end{minipage}
\end{table}
\newpage
\begin{table}[ht]
\begin{minipage}[t]{0.5\linewidth}
\small
\begin{center}*G
\end{center}
\begin{tabular}{rl}
 & dâ recht \textbf{ez}. ich entet im niht.\\ 
 & ich wæne, mirs och \textbf{iemen} giht.\\ 
 & iweren veteren sol ich wol verklagen.\\ 
 & sîn sun die krône nâch im \textbf{sol} tragen.\\ 
5 & derst mir \textbf{ze hêrren hôch} genuoc.\\ 
 & diu künegîn Flurdamurs in truoc.\\ 
 & sîn vater \textbf{was} Kingrisin,\\ 
 & sîn ane der künec Gandin.\\ 
 & ich wil iuch \textbf{gar} bescheiden des:\\ 
10 & Gahmuret und Galoes\\ 
 & sîne ôheime wâren.\\ 
 & ich \textbf{en}\textbf{welle} sîn \textbf{anders} vâren,\\ 
 & ich mahte mit \textbf{êren} von sîner hant\\ 
 & mit vanen enpfâhen \textbf{mîn} lant.\\ 
15 & \begin{large}S\end{large}wer vehten welle, der tuo \textbf{daz}.\\ 
 & \textbf{bin ich} gein dem strîte laz,\\ 
 & \textbf{ich vreische iedoch} \textbf{diu mære} wol.\\ 
 & swer brîs \textbf{ime} strîte \textbf{erhol},\\ 
 & des danken im diu stolzen wîp.\\ 
20 & ich wil dur niemen mînen lîp\\ 
 & verleiten in ze scharpfen pîn.\\ 
 & waz Wolfhartes \textbf{solt} ich sîn?\\ 
 & mirst in den strît der wec vergrabet,\\ 
 & gein vehtene \textbf{diu gir} verhabet.\\ 
25 & wurdet ir mir\textbf{s} nimer holt,\\ 
 & ich tæte ê, als Rumolt\\ 
 & \textbf{dem} \textbf{künege} Gunther riet,\\ 
 & dô er von Wormeze gein den Hûnen schiet:\\ 
 & er bat \textbf{im} lange sniten bæn,\\ 
30 & in \textbf{einem} kezzel umbe dræn."\\ 
\end{tabular}
\scriptsize
\line(1,0){75} \newline
G I O L M Q R Z \newline
\line(1,0){75} \newline
\textbf{1} \textit{Initiale} O L Z  \textbf{7} \textit{Initiale} I  \textbf{15} \textit{Initiale} G  \newline
\line(1,0){75} \newline
\textbf{1} DA rechent ichs mit uͯch niht L  $\cdot$ dâ] ÷a O  $\cdot$ recht ez] rechet irz I  $\cdot$ entet] tet I Q (R) \textbf{2} wæne] wens R  $\cdot$ mirs och] auch mir des I och mirs R  $\cdot$ iemen] niemen O (Q) (R)  $\cdot$ giht] gleich Q \textbf{3} ich] \textit{om.} O  $\cdot$ verklagen] erklagen Q \textbf{5} mir] \textit{om.} R  $\cdot$ hôch] ovch O mir R \textbf{6} Flurdamurs] furdamus I flurdammursz M flurdamúrs Q  $\cdot$ in] \textit{om.} L in da Q \textbf{7} Kingrisin] chingrigrisin I camͮgrisin O kingrisen Q kẏngrisin R \textbf{8} sîn] Vnd sin O L (M) (Q) (R) (Z)  $\cdot$ künec] chvnne O  $\cdot$ Gandin] candin I Gandyn O gaudin Q Gadin R \textbf{9} gar] al I baz O L (M) (Q) Z \textit{om.} R \textbf{10} Gahmuret] Gamvret O L Gamuret M Q Z  $\cdot$ Galoes] galoesz M \textbf{11} sîne] sin I (O) (L) (Q) R Z  $\cdot$ wâren] was en Q \textbf{12} enwelle] well I wolde O (L) (R) Z en wolde M (Q)  $\cdot$ anders] den gerne I gerne O L M Q R Z \textbf{13} mahte mit êren] [mochte eren]:  mochte mit eren O \textbf{14} mit] Meinen Q  $\cdot$ mîn] minev I [s]: miniv O vnd Q \textbf{15} Swer] Wer L M Q R \textbf{16} dem] \textit{om.} L \textbf{17} iedoch] doch Q R \textbf{18} swer] swer mit I Wer L M Q R  $\cdot$ ime] \textit{om.} I  $\cdot$ erhol] da behol I hol O L M Q R Z \textbf{20} wil] enwil L (M)  $\cdot$ mînen] myn M \textbf{21} ze] so I R den M  $\cdot$ scharpfen] scharpffe R \textbf{22} Wolfhartes] wolfharts G wolfartes I wolffartesz M hoffwartes R wolfhartz Z \textbf{23} in den] Gein I indem O  $\cdot$ strît] striten M  $\cdot$ vergrabet] verhabt I \textbf{24} vehtene] vechten vechten L  $\cdot$ diu gir] gar I dú ir R  $\cdot$ verhabet] vergrapt I \textbf{25} wurdet] Wirdet L  $\cdot$ mirs] mir sin I mir Q \textit{om.} R \textbf{26} Rumolt] ruͦmolt I (O) Ruͯmolt L ru͑molt M rúmolt Q \textbf{27} dem] der I (O) (L) (M) (Z) Der me Q (R)  $\cdot$ Gunther] Gunthern I Gvnter L guther Q gúntherr R Gvͤnther Z  $\cdot$ riet] [*]: riet Q \textbf{28} dô] Da M Z  $\cdot$ von Wormeze gein] gen wurmse von R  $\cdot$ Wormeze] wormze I [wrmze]: wormze O wormesze M wurms Q wormz Z  $\cdot$ den] \textit{om.} O  $\cdot$ Hûnen] hvͦnen O hvͯnen L hvn M hounen Q húnen R \textbf{29} im] in I O L (M) Q Z  $\cdot$ sniten] siten Q  $\cdot$ bæn] bin M \textbf{30} in] vnd in I (O) (L) (M) (Q) (R) (Z)  $\cdot$ in einem] in sinem O L (M) im R in sinen Z  $\cdot$ umbe] \textit{om.} M \newline
\end{minipage}
\hspace{0.5cm}
\begin{minipage}[t]{0.5\linewidth}
\small
\begin{center}*T
\end{center}
\begin{tabular}{rl}
 & dâ rechet\textbf{z}. ine tet im niht.\\ 
 & ich wæne, mirs ouch \textbf{niemen} giht.\\ 
 & iuwern vetern sol ich wol verklagen.\\ 
 & sîn sun die krône nâch im \textbf{solte} tragen.\\ 
5 & der ist mir \textbf{ze hêrren hôch} genuoc.\\ 
 & diu künegîn Flordamurs in truoc.\\ 
 & sîn vater \textbf{was} Kyngrisin\\ 
 & \textbf{und} sîn ane der künec Gandin.\\ 
 & ich wil iu \textbf{baz} bescheiden des:\\ 
10 & Gahmuret und Galoes\\ 
 & sîne ôheime wâren.\\ 
 & ich \textbf{en}\textbf{wolte} sîn \textbf{gerne} vâren,\\ 
 & ich mohte mit \textbf{êren} von sîner hant\\ 
 & mit vanen enpfâhen \textbf{mîn} lant.\\ 
15 & swer vehten welle, der tuo \textbf{daz}.\\ 
 & \textbf{bin ich} gegen dem strîte laz,\\ 
 & \textbf{ich vreische doch} \textbf{die mære} wol.\\ 
 & swer prîs \textbf{in} strîte \textbf{hol},\\ 
 & des danken im di\textit{u} stolzen wîp.\\ 
20 & ich\textbf{n} wil durch niemen mînen lîp\\ 
 & verleiten in ze scharpfen pîn.\\ 
 & waz Wolfhartes \textbf{solt} ich sîn?\\ 
 & mirst in den strît der wec vergrabet,\\ 
 & gegen vehtene \textbf{dir} verhabt.\\ 
25 & wurdet ir mir niemer holt,\\ 
 & ich tæte ê als Rumolt,\\ 
 & \textbf{der} \textbf{dem} \textbf{künege} Gunter riet,\\ 
 & dô er von Wurmeze gegen den Hûnen schiet:\\ 
 & er bat \textbf{in} lange sniten bæn\\ 
30 & \textbf{und} in \textbf{sînem} kezzel umbe dræn."\\ 
\end{tabular}
\scriptsize
\line(1,0){75} \newline
T U V W \newline
\line(1,0){75} \newline
\newline
\line(1,0){75} \newline
\textbf{1} dâ] Do W \textbf{2} mirs ouch] oͮch mirs V  $\cdot$ niemen giht] beschicht W \textbf{3} vetern] vetter W \textbf{4} solte] sol U V W \textbf{6} Flordamurs] Flordamuͦrs U flurdamurs V fluͦdamurß W  $\cdot$ in] \textit{om.} W \textbf{7} was] der [*]: hiez V  $\cdot$ Kyngrisin] kingrusin U kẏngrisin V kingrisin W \textbf{8} Gandin] gaudin U W \textbf{10} Gahmuret] Gahmvret T Gahmuͦret U Gamvret V Gamuret W \textbf{12} sîn] sin denne V \textbf{13} mohte] moͤhte V (W) \textbf{15} swer] Wer U W  $\cdot$ tuo] tvͤge V \textbf{16} bin ich] Jch bin V \textbf{17} ich vreische doch] Jedoch [*]: friesche ich V \textbf{18} swer] Wer U W  $\cdot$ in] imme V im W \textbf{19} diu] die T \textbf{20} ichn] Ich W \textbf{21} in] im U  $\cdot$ ze] so V \textit{om.} W \textbf{22} Wolfhartes] wolffhartes W \textbf{23} Mirst] Mir W  $\cdot$ vergrabet] vergraben V \textbf{24} dir] ie gir U die gir V W  $\cdot$ verhabt] verhaben V \textbf{25} mir] [*]: mirz V \textbf{26} Rumolt] Ruͦmolt U reinolt V \textbf{27} Gunter] Guͦnther U gvͦnther V gúnther W \textbf{28} Wurmeze] wormeze U V wormße W  $\cdot$ Hûnen] huͦnen U hv́nen V (W) \textbf{29} bæn] [*]: ben V \newline
\end{minipage}
\end{table}
\end{document}
