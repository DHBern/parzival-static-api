\documentclass[8pt,a4paper,notitlepage]{article}
\usepackage{fullpage}
\usepackage{ulem}
\usepackage{xltxtra}
\usepackage{datetime}
\renewcommand{\dateseparator}{.}
\dmyyyydate
\usepackage{fancyhdr}
\usepackage{ifthen}
\pagestyle{fancy}
\fancyhf{}
\renewcommand{\headrulewidth}{0pt}
\fancyfoot[L]{\ifthenelse{\value{page}=1}{\today, \currenttime{} Uhr}{}}
\begin{document}
\begin{table}[ht]
\begin{minipage}[t]{0.5\linewidth}
\small
\begin{center}*D
\end{center}
\begin{tabular}{rl}
\textbf{360} & \textbf{\begin{large}P\end{large}oydiconjunzes zorn} \textbf{was} ganz\\ 
 & \textbf{ûf} sînen neven Meljanz.\\ 
 & \textbf{doch} brâht der \textbf{werde, junge} man\\ 
 & \textbf{vil tjoste} durch sînen schilt her dan.\\ 
5 & daz \textbf{en}\textbf{dorfte} \textbf{sîn} niwer prîs \textbf{niht} klagen.\\ 
 & Nû hœret von Obien sagen:\\ 
 & diu bôt ir hazzes genuoc\\ 
 & Gawane, dern âne schulde truoc.\\ 
 & si wolde im werben schande:\\ 
10 & einen garzûn si sande\\ 
 & hin ze Gawane, dâ \textbf{der} saz.\\ 
 & \textbf{si sprach}: "\textbf{nû} vrâge in vürbaz,\\ 
 & ob diu ors veile sîn\\ 
 & \textbf{unt} ob in sînen soumschrîn\\ 
15 & lige inder werdez krâmgewant.\\ 
 & wir vrouwen koufenz \textbf{al} zehant."\\ 
 & Der garzûn kom gegangen.\\ 
 & mit \textbf{zorn} er wart enpfangen.\\ 
 & Gawans ougen blicke\\ 
20 & \textbf{in lêrten} herzen schricke.\\ 
 & der garzûn sô verzagete,\\ 
 & daz er\textbf{n} vrâgete \textbf{noch} ensagete\\ 
 & \textbf{al} daz in sîn vrouwe \textbf{werben} hiez.\\ 
 & Gawan die rede \textbf{ouch} niht liez,\\ 
25 & er sprach: "vart hin, ir ribalt!\\ 
 & mûlslege al ungezalt\\ 
 & sult ir \textbf{hie vil} enpfâhen,\\ 
 & welt ir mir vürbaz nâhen."\\ 
 & der garzûn dannen lief \textbf{oder} gie.\\ 
30 & nû hœret, wie ez Obie an vie:\\ 
\end{tabular}
\scriptsize
\line(1,0){75} \newline
D \newline
\line(1,0){75} \newline
\textbf{1} \textit{Initiale} D  \textbf{6} \textit{Majuskel} D  \textbf{17} \textit{Majuskel} D  \newline
\line(1,0){75} \newline
\textbf{1} Poydiconjunzes] Poydiconivnzs D \textbf{2} Meljanz] Melianz D \textbf{6} Obien] Obyen D \newline
\end{minipage}
\hspace{0.5cm}
\begin{minipage}[t]{0.5\linewidth}
\small
\begin{center}*m
\end{center}
\begin{tabular}{rl}
 & \textbf{\textit{\begin{large}P\end{large}}oid\textit{i}coniunz} \textbf{was} ganz\\ 
 & \textbf{wider} sînen neven Mel\textit{i}anz.\\ 
 & \textbf{doch} brâhte der \textbf{werde, junge} man\\ 
 & \textbf{vil juste} durch sînen schilt \textit{h}er dan.\\ 
5 & daz \textbf{en}\textbf{dorfte} \textbf{sîn} niwer prîs \textbf{niht} klagen.\\ 
 & nû hœret von Obi\textit{e}n sagen:\\ 
 & diu bôt ir hazzes genuoc\\ 
 & Gawane, \dag der\dag  âne schulde truoc.\\ 
 & si wolte ime werben schande:\\ 
10 & einen garzûn si sande\\ 
 & hin zuo Gawane, d\textit{â} \textbf{der} saz.\\ 
 & \textbf{si sprach}: "\textbf{nû} vrâgen vürbaz,\\ 
 & ob \textbf{ime} diu ros veile sîn\\ 
 & \textbf{und} ob in sînen soumschrîn\\ 
15 & lige iender werd\textit{e}z krâmgewant.\\ 
 & wir vrouwen koufenz \textbf{alle} zehant."\\ 
 & der garzûn kam gegangen.\\ 
 & mit \textbf{zorn} er \textit{w}a\textit{r}t enpfangen.\\ 
 & Gawans ougen blicke\\ 
20 & \textbf{in lêrten} herzen schricke.\\ 
 & der garzûn sô verzagete,\\ 
 & daz er \textbf{en}vrâget \textbf{und} \textit{ens}agete\\ 
 & \textbf{allez} daz in sîn vrouwe \textbf{sagen} hiez.\\ 
 & Gawan die rede \textbf{gar} \textbf{ouch} niht \textbf{e\textit{n}}liez,\\ 
25 & er sprach: "vart hin, ir ribalt!\\ 
 & mûlslege al un\textit{g}ezalt\\ 
 & sullet ir \textbf{hie vil} enpfâhen,\\ 
 & welt ir mir vürbaz nâhen."\\ 
 & der garzûn dannen lief \textbf{oder} gienc.\\ 
30 & nû hœret, wie ez Obie ane vienc:\\ 
\end{tabular}
\scriptsize
\line(1,0){75} \newline
m n o \newline
\line(1,0){75} \newline
\textbf{1} \textit{Initiale} m   $\cdot$ \textit{Capitulumzeichen} n  \newline
\line(1,0){75} \newline
\textbf{1} Poidiconiunz] COidoconivnz m Coidicomintz n Coidicamincz o \textbf{2} Melianz] meleanz m meliantz n meliancz o \textbf{4} her dan] derdan m \textbf{5} endorfte] endoͯrffte n  $\cdot$ niwer] hoher n hohe o \textbf{6} Obien] obian m obẏen n abẏen o \textbf{8} Gawane] Gawan n o \textbf{10} Ein ganczen sie fande o \textbf{11} Gawane] gawan n o  $\cdot$ dâ] do m n o  $\cdot$ der] er n \textbf{12} vrâgen] frogent n (o) \textbf{13} sîn] sint o \textbf{14} sînen soumschrîn] sinem schonem schrin n sẏme saum schrin o \textbf{15} iender] ẏergent n inwert o  $\cdot$ werdez] werders m \textbf{16} alle zehant] alzuͯ hant n (o) \textbf{17} garzûn] garczẏm o \textbf{18} er wart] ervant m \textbf{20} in] Jnne o \textbf{21} garzûn] garczẏm o  $\cdot$ verzagete] verzaget n \textbf{22} und] noch n o  $\cdot$ ensagete] verzagette m ensaget n \textbf{23} allez] Also n \textbf{24} gar] \textit{om.} n o  $\cdot$ enliez] erlies m \textbf{26} al] \textit{om.} n o  $\cdot$ ungezalt] vnbezalt m \textbf{27} hie] hie hie o \textbf{29} garzûn] garczẏm o  $\cdot$ lief] liesz o \textbf{30} Obie] obye n oͯbie o \newline
\end{minipage}
\end{table}
\newpage
\begin{table}[ht]
\begin{minipage}[t]{0.5\linewidth}
\small
\begin{center}*G
\end{center}
\begin{tabular}{rl}
 & \textbf{Poydekoniunzes zorn} \textbf{was} ganz\\ 
 & \textbf{ûf} sînen neven Melianz.\\ 
 & \textbf{doch} brâhte der \textbf{junge, werde} man\\ 
 & \textbf{manige tjost} durch sînen schilt her dan.\\ 
5 & daz \textbf{en}\textbf{darf} \textbf{sîn} niwer brîs \textbf{niht} klagen.\\ 
 & nû hœret \textbf{ouch} von Obien sagen:\\ 
 & diu bôt ir hazzes genuoc\\ 
 & Gawane, dern âne schulde truoc.\\ 
 & si wolt im werben schande:\\ 
10 & einen garzûn si sande\\ 
 & hin ze Gawane, dâ \textbf{er} saz.\\ 
 & "\textbf{sage im} \textbf{unde} vrâge in vürbaz,\\ 
 & \begin{large}O\end{large}be diu ors veile sîn\\ 
 & \textbf{o\textit{der}} obe in sînen soumschrîn\\ 
15 & lige inder werdez krâmgewant.\\ 
 & wir vrouwen koufenz \textbf{al}zehant."\\ 
 & der garzûn kom gegangen.\\ 
 & mit \textbf{hazze} er wart enpfangen.\\ 
 & Gawanes ougen blicke\\ 
20 & \textbf{in lêrten} herzen schricke.\\ 
 & der garzûn sô verzagte,\\ 
 & daz er vrâgte \textbf{noch} ensagte,\\ 
 & daz in sîn vrouwe \textbf{werben} hiez.\\ 
 & Gawan die rede \textbf{doch} niht \textbf{en}liez,\\ 
25 & er sprach: "vart hin, ir ribalt!\\ 
 & mûlslege al ungezalt\\ 
 & sult ir \textbf{von mir} enpfâhen,\\ 
 & welt ir mir vürbaz nâhen."\\ 
 & der garzûn dan lief \textbf{unde} gienc.\\ 
30 & nû hœret, wiez Obie an vienc:\\ 
\end{tabular}
\scriptsize
\line(1,0){75} \newline
G I O L M Q R Z \newline
\line(1,0){75} \newline
\textbf{1} \textit{Initiale} I O L Z   $\cdot$ \textit{Capitulumzeichen} R  \textbf{6} \textit{Capitulumzeichen} R  \textbf{13} \textit{Initiale} G I  \newline
\line(1,0){75} \newline
\textbf{1} Poydekoniunzes] poidekonivnzes G (Z) Poydecomunz I ÷Oydekvmvnzes O POý de Conivnz L Poidekvnivnz M poydekomvirczes R  $\cdot$ was] wart O L M Q (R) \textbf{2} sînen] syme M sinē Q  $\cdot$ Melianz] Melyanz O Meliancz R Meliantz Z \textbf{3} junge werde] iunge werder I ivnge O werde ivnge L (M) (Q) (R) (Z) \textbf{4} manige] Manchen Q (R)  $\cdot$ sînen] sine R  $\cdot$ schilt] \textit{om.} Q  $\cdot$ her] hin R \textbf{5} daz] des I  $\cdot$ endarf] darf O (R)  $\cdot$ sîn] sich R \textbf{6} nû] Vnd Q  $\cdot$ hœret] horen L  $\cdot$ Obien] obyen O obie M obnen R \textbf{7} ir] irn R \textit{om.} Z  $\cdot$ hazzes] lassen Q \textbf{8} Gawane] Gawan I O L Q Z  $\cdot$ dern] [der]: daz O der Q  $\cdot$ truoc] da truͤc I \textbf{9} wolt] wolten Q  $\cdot$ schande] schanden Q \textbf{11} ze] vsz L  $\cdot$ Gawane] Gawan I O L (M) (Q) (Z) her Gawan R  $\cdot$ dâ] do O Q R  $\cdot$ er] er da I der O L Q \textbf{12} sage] Sagt O  $\cdot$ vrâge] fragt O fragte L \textbf{13} sîn] sint M (R) \textbf{14} oder] olde G  $\cdot$ sînen] sinem O L (M) sinē Q  $\cdot$ soumschrîn] somschirin sind R \textbf{15} lige] Lege M  $\cdot$ inder] irgen M nidert Q  $\cdot$ werdez krâmgewant] krams gewand R \textbf{16} koufenz] koufftenz M (R)  $\cdot$ alzehant] als zehand R \textbf{18} hazze] zorn Z  $\cdot$ er wart] ward er R \textbf{19} Gawanes] Gawans I O L M Q R Z  $\cdot$ ougen] auge Q \textbf{20} herzen] herze O (M) (R) herte Q \textbf{21} sô] was so O \textbf{22} er] ern I Q  $\cdot$ vrâgte] frage R  $\cdot$ ensagte] sagte Q \textbf{23} in] \textit{om.} I \textbf{24} die rede doch] doch die rede O der rede doch L Z  $\cdot$ enliez] liez I L (Q) (R) \textbf{25} ir] her Z  $\cdot$ ribalt] irrebalt L rabalt M \textbf{26} al] alvmb I \textit{om.} L Z  $\cdot$ ungezalt] bezalt I vnverzalt O \textbf{27} sult ir] Soltu R  $\cdot$ von mir] von mir vil I vil von mir O L (M) Z \textbf{28} welt ir] Wiltu R \textbf{29} lief unde] vil drate I \textbf{30} wiez] wie O  $\cdot$ Obie] Obye O (Q) (R)  $\cdot$ vienc] geuienc I (O) (Z) \newline
\end{minipage}
\hspace{0.5cm}
\begin{minipage}[t]{0.5\linewidth}
\small
\begin{center}*T
\end{center}
\begin{tabular}{rl}
 & \textbf{\begin{large}P\end{large}oydekuniunzes zorn} \textbf{wart} ganz\\ 
 & \textbf{ûf} sînen neven Melyanz.\\ 
 & \textbf{ouch} brâhte der \textbf{junge, werde} man\\ 
 & \textbf{manege tjost} durch sînen schilt her dan,\\ 
5 & daz \textbf{ez} \textbf{darf} niuwer prîs  klagen.\\ 
 & Nû hœret \textbf{ouch} von Obyen sagen:\\ 
 & diu bôt ir hazzes genuoc\\ 
 & Gawane, dern âne schulde truoc.\\ 
 & Si wolt im werben schande:\\ 
10 & Einen garzûn si sande\\ 
 & hin ze Gawane, dâ \textbf{er} saz.\\ 
 & "\textbf{sagim} \textbf{unde} vrâgin vürbaz,\\ 
 & ob diu ors veile sîn\\ 
 & \textbf{oder} obin sînen soumschrîn\\ 
15 & lige iender werdez krâmgewant.\\ 
 & wir vrouwen koufenz \textbf{al}zehant."\\ 
 & Der garzûn kom gegangen.\\ 
 & mit \textbf{hazze} er wart enpfangen.\\ 
 & Gawanes ougen blic\\ 
20 & \textbf{lêrten in} herzen schric.\\ 
 & Der garzûn sô verzagete,\\ 
 & daz er\textbf{n} vrâgete \textbf{noch} ensagete,\\ 
 & daz in sîn vrouwe \textbf{werben} hiez.\\ 
 & Gawan die rede \textbf{doch} niht \textbf{en}liez,\\ 
25 & er sprach: "vart hin, ir ribalt!\\ 
 & mûlslege alungezalt\\ 
 & sult ir \textbf{von mir} enpfâhen,\\ 
 & welt ir mir vürbaz nâhen."\\ 
 & Der garzûn dan lief \textbf{unde} gienc.\\ 
30 & nû hœret, wiez Obye ane vienc:\\ 
\end{tabular}
\scriptsize
\line(1,0){75} \newline
T V W \newline
\line(1,0){75} \newline
\textbf{1} \textit{Initiale} T W  \textbf{6} \textit{Majuskel} T  \textbf{9} \textit{Majuskel} T  \textbf{10} \textit{Majuskel} T  \textbf{17} \textit{Majuskel} T  \textbf{21} \textit{Majuskel} T  \textbf{29} \textit{Majuskel} T  \newline
\line(1,0){75} \newline
\textbf{1} Poydekuniunzes] [*]: Poẏdicomvnz V Poide guniunsenes W  $\cdot$ wart] waz V wit W \textbf{2} ûf] [*]: Wider V  $\cdot$ Melyanz] melianz V W \textbf{4} manege] Manigen V \textbf{5} Daz endorfte [si*]: sin nvwer pris niht clagen V  $\cdot$ Das darf sein núwer preis nit klagen W \textbf{6} Obyen] obẏen V \textbf{7} hazzes] hassens W \textbf{8} Gawane] Gawan W \textbf{9} werben] werden W \textbf{11} Gawane] gawan W  $\cdot$ dâ] do V W \textbf{12} vürbaz] bas W \textbf{13} ob] [O*]: Ob im V \textbf{14} oder] alde T \textbf{15} werdez] werder W \textbf{16} koufenz] koͮftens V \textbf{20} lêrten in] In lerten W \textbf{24} die rede] \textit{om.} W  $\cdot$ doch] oͮch V iedoch W \textbf{25} hin] hin weg W \textbf{27} sult] Svllen V  $\cdot$ enpfâhen] [*]: hie vil enphahen V \textbf{29} dan lief] lief dannan V \textbf{30} Obye] obẏe V \newline
\end{minipage}
\end{table}
\end{document}
