\documentclass[8pt,a4paper,notitlepage]{article}
\usepackage{fullpage}
\usepackage{ulem}
\usepackage{xltxtra}
\usepackage{datetime}
\renewcommand{\dateseparator}{.}
\dmyyyydate
\usepackage{fancyhdr}
\usepackage{ifthen}
\pagestyle{fancy}
\fancyhf{}
\renewcommand{\headrulewidth}{0pt}
\fancyfoot[L]{\ifthenelse{\value{page}=1}{\today, \currenttime{} Uhr}{}}
\begin{document}
\begin{table}[ht]
\begin{minipage}[t]{0.5\linewidth}
\small
\begin{center}*D
\end{center}
\begin{tabular}{rl}
\textbf{806} & der knabe \textbf{sîn wolde} küssen niht.\\ 
 & \textbf{werden kinden man noch vorhte giht}.\\ 
 & \begin{large}D\end{large}es lachete der heiden.\\ 
 & dô begunden si sich scheiden\\ 
5 & ûf dem hove \textbf{unt} dô diu künegîn\\ 
 & erbeizet was. \textbf{in kom gewin}\\ 
 & an ir mit \textbf{vreuden} künfte al dar.\\ 
 & man vuorte si, dâ \textbf{werdiu} schar\\ 
 & von \textbf{maneger} clâren vrouwen was.\\ 
10 & Feirefiz unt Anfortas\\ 
 & mit zühten stuonden bêde\\ 
 & bî \textbf{der} vrouwen an \textbf{der grêde}.\\ 
 & Repanse de schoye\\ 
 & \textbf{unt} von \textbf{Gruonlant} Garschiloye,\\ 
15 & Florie \textbf{von} \textbf{Lunel},\\ 
 & \textbf{liehtiu} ougen \textbf{und} \textbf{clâriu} vel\\ 
 & \textbf{die} truogen \textbf{unt} \textbf{magetuomlîchen} prîs.\\ 
 & Dâ stuont ouch \textbf{swankel} als ein rîs,\\ 
 & der schœne unt güete niht gebrach\\ 
20 & unt der man \textbf{im} ze tohter jach,\\ 
 & von Ryl Jernise;\\ 
 & diu maget hiez Ampflise.\\ 
 & von Tenabroc, \textbf{ist} mir gesagt,\\ 
 & stuont dâ Clarinschanze, ein \textbf{süeziu} magt,\\ 
25 & \textbf{liehter varwe} gar unverkrenket,\\ 
 & als ein âmeize gelenket.\\ 
 & Feirefiz gein der wirtîn trat.\\ 
 & diu künegîn \textbf{den sich} küssen bat.\\ 
 & si kuste ouch Anfortasen dô\\ 
30 & unt was sîner \textbf{urlœsunge} vrô.\\ 
\end{tabular}
\scriptsize
\line(1,0){75} \newline
D \newline
\line(1,0){75} \newline
\textbf{3} \textit{Initiale} D  \textbf{18} \textit{Majuskel} D  \newline
\line(1,0){75} \newline
\textbf{13} Repanse de schoye] Repanse de scoye D \textbf{14} Gruonlant] Grvͦnlant D  $\cdot$ Garschiloye] Garsciloye D \textbf{15} Florie] Florîe D \textbf{21} Jernise] Jernîse D \textbf{23} Tenabroc] Tenabroch D \textbf{24} Clarinschanze] Clarinscanze D \newline
\end{minipage}
\hspace{0.5cm}
\begin{minipage}[t]{0.5\linewidth}
\small
\begin{center}*m
\end{center}
\begin{tabular}{rl}
 & der knabe \textbf{sîn wolte} küssen niht,\\ 
 & \textbf{von vorht, als kinden noch beschiht}.\\ 
 & des lachete der heiden.\\ 
 & dô begunden si sich scheiden\\ 
5 & ûf dem hof \textbf{und} \textit{dô} diu künigîn\\ 
 & erbeizet was. \textbf{in kam gew\textit{i}n}\\ 
 & an ir mit \textbf{vröuden} kun\textit{f}t al dar.\\ 
 & man vuorte si, d\textit{â} \textbf{werdiu} schar\\ 
 & von \textbf{maniger} clâren vrôwen was.\\ 
10 & Ferefiz und Anfortas\\ 
 & mit zühten stuonden beide\\ 
 & bî \textbf{der} vrowen an \textbf{dem gereite}.\\ 
 & \textbf{d\textit{â} stuont} Repanse de schoye\\ 
 & \textbf{und} von \textbf{Gr\textit{un}lant} Gar\textit{sch}iloye,\\ 
15 & Florie \textbf{von} \textbf{Lunel}.\\ 
 & \textbf{l\textit{i}ehtiu} ougen, \textbf{clâriu} vel\\ 
 & \textbf{si} truogen \textbf{und} \textbf{magetlîchen} prîs.\\ 
 & d\textit{â} stuont ouch \textbf{swankel} als ein rîs,\\ 
 & der schœnde und güete niht gebrach\\ 
20 & und der man zuo tohter jach\\ 
 & von Rile Gernise;\\ 
 & diu maget hiez Ampflise.\\ 
 & von Tenebroc, \textbf{ist} mir gesaget,\\ 
 & stuont d\textit{â} Clarinschantz, ein maget,\\ 
25 & \textbf{liehter varwe} gar unverkrenket,\\ 
 & als ein âmeize gelenket.\\ 
 & Ferefiz gegen der wirtîn trat.\\ 
 & diu künigîn \textbf{sich den} küssen bat.\\ 
 & si kuste ouch Anfortassen dô\\ 
30 & und was sîner \textbf{geniste} vrô.\\ 
\end{tabular}
\scriptsize
\line(1,0){75} \newline
m n o V V' W \newline
\line(1,0){75} \newline
\textbf{27} \textit{Initiale} V W  \newline
\line(1,0){75} \newline
\textbf{1} \textit{Die Verse 805.28-806.30 fehlen} V'  \textbf{2} vorht] [vorhten]: vorhte V  $\cdot$ beschiht] geschicht o (V) W \textbf{5} und dô diu] vnd die m n o [*ie]: do die V do die W \textbf{6} in] ym o  $\cdot$ gewin] gewein m \textbf{7} mit] mir mit o mir W  $\cdot$ vröuden] froͤide V  $\cdot$ kunft] kunst m [kv́n*]: kv́nfte V \textbf{8} dâ] do m n o V W \textbf{9} clâren] kluͦgen W  $\cdot$ was] sas o \textbf{10} \textit{nach 806.10:} Vnde die tavelrunder schar / Alle gemeinliche gar V   $\cdot$ Ferefiz] Ferefis m o Ferrefis n Artus ferefis V Ferafis W \textbf{11} beide] sv́ alle do V \textbf{12} der] [de*]: den V den W  $\cdot$ an dem gereite] ander grete ho V an der grede W \textbf{13} dâ] Do m n o V W  $\cdot$ Repanse de schoye] repanse descoie m repanse de scoye n repanse de scoie o repanse deschoye V vrepans de tschoye W \textbf{14} Grunlant] gringulant m grinulant n grinailant o  $\cdot$ Garschiloye] gar filoie m (o) garfiloye n (W) \textbf{15} Florie] Florige V  $\cdot$ von] vnd o (W)  $\cdot$ Lunel] lymel W \textbf{16} liehtiu] Lechtte m (o) \textbf{17} si] Die V  $\cdot$ und] \textit{om.} W  $\cdot$ magetlîchen] magetvͦmlichen V \textbf{18} dâ] Do m n o V W  $\cdot$ ouch swankel] schwanckeln W \textbf{19} schœnde] schone o (V) (W)  $\cdot$ und] noch W  $\cdot$ niht] nie W  $\cdot$ gebrach] gebracht o \textbf{20} man] man imme V \textbf{21} Rile] Ryle V \textbf{22} Ampflise] anflise V anfolise W \textbf{23} von] \textit{om.} o  $\cdot$ Tenebroc] tenebrog m n (o) Tenabroke V tenebrock W \textbf{24} dâ] do m n o V W  $\cdot$ Clarinschantz] clorin schantz n clarinschacz o [clarinf*]: clarinsanse V klarissante W  $\cdot$ maget] svͤsze maget V \textbf{26} gelenket] gelenken V \textbf{27} Ferefiz] Ferefis m o V Ferre vis n FErafis W \textbf{28} sich] \textit{om.} n \textbf{29} Anfortassen] anforttassen m anfortasen n anfortass:: o artusen vnde anfortassen V anfortas W  $\cdot$ dô] so o \textbf{30} geniste] genistv́ V gesunthait W \newline
\end{minipage}
\end{table}
\newpage
\begin{table}[ht]
\begin{minipage}[t]{0.5\linewidth}
\small
\begin{center}*G
\end{center}
\begin{tabular}{rl}
 & der knabe \textbf{sîn wolde} küssen niht.\\ 
 & \textbf{werden kinden man noch vorhte giht}.\\ 
 & des lacht der heiden.\\ 
 & dô begunden si sich scheiden\\ 
5 & ûf dem hof, dâ diu künigîn\\ 
 & erbeizt was, \textbf{unde giengen în}\\ 
 & an ir mit \textbf{werder} kunft al dar.\\ 
 & man vuort si, dâ \textbf{werdiu} schar\\ 
 & von \textbf{maniger} klâren vrouwen was.\\ 
10 & Feirafiz unde Anfortas\\ 
 & mit zühten stuonden bêde\\ 
 & bî \textbf{den} vrouwen an \textbf{der grêde}.\\ 
 & Urrepanse de schoye,\\ 
 & von \textbf{Gruonlanden} Karziloyde,\\ 
15 & Flori \textbf{unde} \textbf{Ionel},\\ 
 & \textbf{clâriu} ougen \textbf{unde} \textbf{lie\textit{h}tiu} vel\\ 
 & \textbf{die} truogen \textbf{magetlîchen} prîs.\\ 
 & dâ stuont ouch \textbf{swankel} als ein rîs,\\ 
 & der schœn unde güete niht gebrach\\ 
20 & unde der man \textbf{im} ze tohter jach,\\ 
 & von Rile Kernise;\\ 
 & diu maget hiez Amflise.\\ 
 & von Tenebroch, \textbf{ist} mir gesaget,\\ 
 & stuont dâ Clarissanze, ein \textbf{süeziu} maget,\\ 
25 & \textbf{an ir schœne} gar unverkrenket,\\ 
 & als ein âmeize gelenket.\\ 
 & Feirafiz gên der wirtinne trat.\\ 
 & diu küneginne \textbf{sich den} küssen bat.\\ 
 & si kust ouch Anfortasen dô\\ 
30 & unde was sîner \textbf{urlœsunge} vrô.\\ 
\end{tabular}
\scriptsize
\line(1,0){75} \newline
G I L Z \newline
\line(1,0){75} \newline
\textbf{11} \textit{Initiale} I  \newline
\line(1,0){75} \newline
\textbf{1} \textit{Die Verse 806.1-807.24 fehlen} Z  \textbf{2} Duͦrch die geteilten angesiht L \textbf{3} lacht] lachte I (L) \textbf{5} dâ] do L \textbf{6} în] dar in L \textbf{7} werder] frovden L \textbf{10} Feirafiz] veirafiz G Ferefiz L  $\cdot$ Anfortas] Amfortas L \textbf{11} bêde] si bêde I \textbf{13} Urrepanse de schoye] vrrenpanse de scoyte G vrrepanse de shoye I Vrrepansa de [sho*e]: shoie L \textbf{14} Gruonlanden] groͮnlanden G Grunlanden I grvnlande L  $\cdot$ Karziloyde] karsciloye I Gragiloie L \textbf{15} Flori] Florie L  $\cdot$ Ionel] jonel L \textbf{16} clâriu] Lýchte L  $\cdot$ liehtiu] lietiv G clare L \textbf{17} die] \textit{om.} I  $\cdot$ magetlîchen] vnd magetvmlichen L \textbf{21} Rile] rise L  $\cdot$ Kernise] scernise G schernise I (L) \textbf{22} Amflise] anphise I \textbf{23} \textit{Die Verse 806.23-24 fehlen} L   $\cdot$ Tenebroch] tenbroͮch G Tenabruc I \textbf{24} Clarissanze] klarissanze I \textbf{25} gar] \textit{om.} L \textbf{27} Feirafiz] feiraviz G Ferefiz L \textbf{29} kust] kusten I  $\cdot$ Anfortasen] Anfortassen I Amfortassen L \textbf{30} urlœsunge] losvnge L \newline
\end{minipage}
\hspace{0.5cm}
\begin{minipage}[t]{0.5\linewidth}
\small
\begin{center}*T
\end{center}
\begin{tabular}{rl}
 & der k\textit{na}ppe \textbf{wolte sîn} küssen niht.\\ 
 & \textbf{werden kinden man noch vorhte giht}.\\ 
 & des lachete der heiden.\\ 
 & dô begunden si sich scheiden\\ 
5 & û\textit{f} dem hove, dâ diu künegîn\\ 
 & erbeizet was. \textbf{im kam gewin}\\ 
 & an ir mit \textbf{vreuden} künfte al dar.\\ 
 & man vuorte si, dâ \textbf{manegiu} schar\\ 
 & von \textbf{manegen} clâren vrouwen was.\\ 
10 & Ferefis und Anfortas\\ 
 & mit zühten stuonden bêde\\ 
 & bî \textbf{den} vrouwen an \textbf{der grêde}.\\ 
 & Repanse de joie,\\ 
 & von \textbf{Gruonlant} Garschiloie,\\ 
15 & Florie \textbf{und} \textbf{Zunel},\\ 
 & \textbf{liehtiu} ougen \textbf{und} \textbf{clârez} vel\\ 
 & \textbf{die} truogen \textbf{und} \textbf{magetuomlîchen} prîs.\\ 
 & d\textit{â} stuont ouch \textbf{swanc} als ein rîs,\\ 
 & der schœne und güete niht gebrach\\ 
20 & und der man \textbf{im} zuo tohter jach,\\ 
 & von Rile Schernise;\\ 
 & diu maget hiez Anflise.\\ 
 & von Tenebroc, \textbf{wart} mir gesaget,\\ 
 & s\textit{t}uont dâ Clarissanze, ein \textbf{süeziu} maget,\\ 
25 & \textbf{an ir schœne} gar unverkrenket,\\ 
 & als ein âmeize gelenket.\\ 
 & Ferefis gein der wirtîn trat.\\ 
 & diu küneginne \textbf{sich den} küssen bat.\\ 
 & \begin{large}S\end{large}i kust ouch Anfortassen dô\\ 
30 & und was sîner \textbf{erlœsunge} vrô.\\ 
\end{tabular}
\scriptsize
\line(1,0){75} \newline
U Q R \newline
\line(1,0){75} \newline
\textbf{1} \textit{Initiale} R  \textbf{29} \textit{Initiale} U  \newline
\line(1,0){75} \newline
\textbf{1} knappe] kanppe U  $\cdot$ wolte sîn küssen] sein kussen wolt Q sin wolte kússen R \textbf{3} lachete] lachet R \textbf{5} ûf] Vz U  $\cdot$ dâ] do Q R \textbf{6} erbeizet] Erbeiczes R  $\cdot$ im] in Q R \textbf{8} dâ] do Q  $\cdot$ manegiu] werde Q \textbf{9} manegen] menger R \textbf{10} Ferefis] feirefisz Q Feirefis R \textbf{12} den] der Q  $\cdot$ der] ir R \textbf{13} Repanse de joie] Repanze de Joye U Repanse detschoye Q Repanse deioie R \textbf{14} Gruonlant] Gruͦnelant U grűnalt Q Gruͦnland R  $\cdot$ Garschiloie] Garsiloye U karshiloye Q Garshilioe R \textbf{15} Zunel] zuͦnel U júnel Q \textbf{16} liehtiu] Lichte Q  $\cdot$ clârez] clare Q R \textbf{17} die] Sy R  $\cdot$ magetuomlîchen] manlichen R \textbf{18} dâ] Do U Q  $\cdot$ swanc] swanckel Q (R)  $\cdot$ als] sam Q \textbf{20} zuo] zur Q \textbf{21} Rile] kyle Q R  $\cdot$ Schernise] [*ermise]: bermise U gernise Q R \textbf{22} Anflise] anflize Q anflyse R \textbf{24} stuont] Suͦnt U  $\cdot$ Clarissanze] Clarisancze R  $\cdot$ süeziu] suͯsze R \textbf{25} gar] [dar]: gar R  $\cdot$ unverkrenket] vn verklenket Q \textbf{27} Ferefis] feyrefisz Q Feriefis R \textbf{28} sich] in R \textbf{30} sîner] \textit{om.} R \newline
\end{minipage}
\end{table}
\end{document}
