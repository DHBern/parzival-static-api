\documentclass[8pt,a4paper,notitlepage]{article}
\usepackage{fullpage}
\usepackage{ulem}
\usepackage{xltxtra}
\usepackage{datetime}
\renewcommand{\dateseparator}{.}
\dmyyyydate
\usepackage{fancyhdr}
\usepackage{ifthen}
\pagestyle{fancy}
\fancyhf{}
\renewcommand{\headrulewidth}{0pt}
\fancyfoot[L]{\ifthenelse{\value{page}=1}{\today, \currenttime{} Uhr}{}}
\begin{document}
\begin{table}[ht]
\begin{minipage}[t]{0.5\linewidth}
\small
\begin{center}*D
\end{center}
\begin{tabular}{rl}
\textbf{361} & \textbf{\begin{large}E\end{large}inen} junchêrren si \textbf{sprechen} bat\\ 
 & den burcgrâven von der stat.\\ 
 & der was geheizen Scherules.\\ 
 & si sprach: "dû solt in bitten des,\\ 
5 & daz erz durch mînen willen tuo\\ 
 & unt manlîche grîfe zuo.\\ 
 & under\textbf{n œlboumen} \textbf{bîme} graben\\ 
 & stên\textit{t} siben ors, diu sol er haben,\\ 
 & unt ander rîcheite vil.\\ 
10 & ein koufman uns hie \textbf{triegen} wil.\\ 
 & bitte in, daz er \textbf{daz} wende.\\ 
 & ich getrûwe des sîner hende,\\ 
 & si \textbf{neme}z unvergolten.\\ 
 & ouch hât erz unbescholten."\\ 
15 & Der knappe \textbf{hin nider} sagete\\ 
 & al daz sîn vrouwe klagete.\\ 
 & "ich sol vor \textbf{triegen} uns bewarn",\\ 
 & sprach Scherules. "ich wil dar varn."\\ 
 & Er reit hin \textbf{ûf}, dâ Gawan saz,\\ 
20 & der \textbf{selten ellens} ie vergaz.\\ 
 & an dem \textbf{er vant} krancheite vlust,\\ 
 & lieht antlütze \textbf{unt} \textbf{hôhe} brust\\ 
 & unt einen ritter wol gevar.\\ 
 & Scherules in prüevete gar,\\ 
25 & sîn arme unt ieweder hant\\ 
 & unt swaz \textbf{geschickede er} dâ vant.\\ 
 & Dô sprach er: "hêrre, ir sît ein gast.\\ 
 & guoter witze uns \textbf{gar} gebrast,\\ 
 & sît ir niht herberge hât.\\ 
30 & \textbf{nû} prüevetz \textbf{uns} vür missetât.\\ 
\end{tabular}
\scriptsize
\line(1,0){75} \newline
D Fr4 \newline
\line(1,0){75} \newline
\textbf{1} \textit{Initiale} D  \textbf{15} \textit{Initiale} Fr4   $\cdot$ \textit{Majuskel} D  \textbf{19} \textit{Majuskel} D  \textbf{27} \textit{Majuskel} D  \newline
\line(1,0){75} \newline
\textbf{3} Scherules] Scervles D \textbf{8} stênt] sten D \textbf{18} Scherules] Scervles D tserules Fr4 \textbf{24} Scherules] Scervles D tserules Fr4 \textbf{27} hêrre] \textit{om.} Fr4 \textbf{28} guoter] herre gutir Fr4 \newline
\end{minipage}
\hspace{0.5cm}
\begin{minipage}[t]{0.5\linewidth}
\small
\begin{center}*m
\end{center}
\begin{tabular}{rl}
 & \textbf{\textit{\begin{large}E\end{large}}inen} junchêrren si \textbf{sprechen} bat\\ 
 & den burcgrâven von der stat.\\ 
 & der was geheizen Scherules.\\ 
 & s\textit{i} sprach: "dû solt in bitten des,\\ 
5 & daz er ez durch mînen willen tuo\\ 
 & und manlîche grîfe zuo.\\ 
 & under\textbf{n \textit{œ}lboumen} \textbf{bîme} graben\\ 
 & st\textit{â}nt siben ros, diu sol er \textit{h}aben,\\ 
 & und ander rîcheite vil.\\ 
10 & ein koufman uns hie \textbf{kriegen} wil.\\ 
 & bit in, daz er \textbf{daz} wende.\\ 
 & ich getrûwe des sîner hende,\\ 
 & si \textbf{neme} ez \textbf{ime} unvergolten.\\ 
 & ouch hât er ez unbescholten."\\ 
15 & der knappe \textbf{hin wider} sagete\\ 
 & al daz sîn vrouwe klagete.\\ 
 & "ich sol vo\textit{r} \textbf{\textit{k}riegen} uns bewarn",\\ 
 & sprach Scherules. "ich wil dar varn."\\ 
 & er reit hin \textbf{ûf}, dâ Gawan saz,\\ 
20 & der \textbf{ellen\textit{s} \textit{s}elten} ie vergaz.\\ 
 & an dem \textbf{vant er} krancheit vlust,\\ 
 & lieht an\textit{t}litze, \textbf{hôhe} brust\\ 
 & und einen ritter wol gevar.\\ 
 & Scher\textit{ule}s in brüefete gar,\\ 
25 & sîne arme und ietwedere hant\\ 
 & und waz \textbf{geschickede er} d\textit{â} vant.\\ 
 & dô sprach er: "hêrre, ir sît ein gast.\\ 
 & guoter witze uns \textbf{gar} gebrast,\\ 
 & sît ir niht herberge hât.\\ 
30 & \textbf{man} brüefet ez \textbf{uns} vür missetât.\\ 
\end{tabular}
\scriptsize
\line(1,0){75} \newline
m n o \newline
\line(1,0){75} \newline
\textbf{1} \textit{Initiale} m o   $\cdot$ \textit{Capitulumzeichen} n  \newline
\line(1,0){75} \newline
\textbf{1} Einen] Eeinen m  $\cdot$ sprechen] sprachen o \textbf{2} den] Der o \textbf{3} Scherules] scerules m scernles n scerúles o \textbf{4} si] Sit m \textbf{6} zuo] do zuͯ n (o) \textbf{7} œlboumen] eleboumen m \textbf{8} stânt] Stunt m (o)  $\cdot$ haben] schaben m \textbf{12} des] dasz o \textbf{16} al] \textit{om.} n o \textbf{17} vor kriegen] vor vns kriegen m \textbf{18} Scherules] scerules m screngeles n screngels o \textbf{19} dâ] do n o \textbf{20} ellens selten] ellens riche seltten m  $\cdot$ ie] nie o \textbf{21} krancheit] krancke n o \textbf{22} antlitze] antzlicze m antlitz vnd n anczlit vnd o \textbf{24} Scherules] Scerleus m Sterulus n Screnluͯs o \textbf{26} dâ] do m n o \textbf{29} niht] \textit{om.} n \newline
\end{minipage}
\end{table}
\newpage
\begin{table}[ht]
\begin{minipage}[t]{0.5\linewidth}
\small
\begin{center}*G
\end{center}
\begin{tabular}{rl}
 & \textbf{einen} junchêrren si \textbf{sprechen} bat\\ 
 & den burcgrâven von der stat.\\ 
 & der was geheizen Tscherules.\\ 
 & si sprach: "dû solt in biten des,\\ 
5 & daz erz durch mînen willen tuo\\ 
 & unde manlîche grîfe zuo.\\ 
 & under\textbf{n œlboumen} \textbf{ame} graben\\ 
 & stênt siben ors, diu sol er haben,\\ 
 & unde ander rîcheite vil.\\ 
10 & ein koufman uns hie \textbf{triegen} wil.\\ 
 & bit in, daz er\textbf{z} wende.\\ 
 & ich getrûwe des sîner hende,\\ 
 & si \textbf{nem} ez unvergolten.\\ 
 & ouch hât erz unbescholten."\\ 
15 & der knappe \textbf{hin nider} sagte\\ 
 & al daz sîn vrouwe klagte.\\ 
 & "ich sol vor \textbf{triegen} uns bewaren",\\ 
 & sprach Tscherules. "ich wil dar varen."\\ 
 & er reit hin \textbf{ûf}, dâ Gawan saz,\\ 
20 & der \textbf{selten ellens} ie vergaz.\\ 
 & an dem \textbf{er vant} krancheit vlust,\\ 
 & lieht antlütze \textbf{unde} \textbf{hôhe} brust\\ 
 & unde einen rîter wolgevar.\\ 
 & Tscherules in bruovte gar,\\ 
25 & sîne arme unde ietwedere hant\\ 
 & unde swaz \textbf{geschickede er} dâ vant.\\ 
 & dô sprach er: "hêrre, ir sît ein gast.\\ 
 & guoter witze uns \textbf{gar} gebrast,\\ 
 & sît ir niht herberge hât.\\ 
30 & \textbf{nû} prüevet ez \textbf{uns} vür missetât.\\ 
\end{tabular}
\scriptsize
\line(1,0){75} \newline
G I O L M Q R Z Fr38 \newline
\line(1,0){75} \newline
\textbf{1} \textit{Initiale} I O L Q R Z  \textbf{17} \textit{Initiale} I  \textbf{22} \textit{Initiale} Fr38  \newline
\line(1,0){75} \newline
\textbf{1} einen] ÷inen O \textbf{3} Tscherules] Scrules I Tschivles O tschervles L scherules M (R) \textbf{5} erz] er I  $\cdot$ tuo] tun Q \textbf{6} manlîche] weckerliche Q (R) \textbf{7} undern œlboumen] vnder dem [olbaum]: olbaam I Vndirme olbovme M Vnderm oͯlbomen R  $\cdot$ ame] bi dem O (L) (M) (Q) (Z) bin R \textbf{8} diu] di O \textbf{10} hie triegen] betrige M \textbf{11} erz] er daz O L (M) (Q) (R) Z  $\cdot$ wende] wenden Q \textbf{12} des] daz L (Q) wol R \textbf{13} nem ez] nemens I (O) (M) (R) [meine*]: meinet  L  $\cdot$ unvergolten] vergolten Q \textbf{14} erz unbescholten] er vns besholten I (Z) \textbf{15} nider] wider Q R  $\cdot$ sagte] gahte O \textbf{16} sîn] \textit{om.} L \textbf{17} vor] von I  $\cdot$ uns] vns wol Z \textbf{18} Tscherules] Scrules I Tschervles O tshervles L scherules M R (Z)  $\cdot$ dar] da hin I \textbf{19} er] Der O L M Q R Z  $\cdot$ dâ] do R  $\cdot$ Gawan] her Gwan R \textbf{20} ellens] ellen Z  $\cdot$ ie] nie R \textbf{21} an] Jn O \textbf{22} lieht] leht I Lycht L (Q) (R) (Fr38)  $\cdot$ unde] \textit{om.} I \textbf{24} Tscherules] Scrules I Tschervles O Fr38 Tsheruͯles L Scerulus M Scherules R \textbf{25} ietwedere] ixliche M \textbf{26} swaz] waz L (M) (Q) (R)  $\cdot$ dâ] do Q R \textbf{27} er] \textit{om.} I der O \newline
\end{minipage}
\hspace{0.5cm}
\begin{minipage}[t]{0.5\linewidth}
\small
\begin{center}*T
\end{center}
\begin{tabular}{rl}
 & \textbf{\begin{large}M\end{large}înen} junchêrren si \textbf{gesprechen} bat\\ 
 & den burcgrâven von der stat.\\ 
 & der was geheizen Tscherules.\\ 
 & si sprach: "dû solt in bitten des,\\ 
5 & daz erz durch mînen willen tuo\\ 
 & unde manlîche grîfe zuo.\\ 
 & under\textbf{m œlboume} \textbf{bîme} graben\\ 
 & stânt siben ors, diu sol er haben,\\ 
 & unde anderre rîcheite vil.\\ 
10 & ein koufman uns hie \textbf{triegen} wil.\\ 
 & bit in, daz er\textbf{z} wende.\\ 
 & ich getriuwe des sîner hende,\\ 
 & si \textbf{neme\textit{n}}z unvergolten.\\ 
 & ouch het erz unbescholten."\\ 
15 & Der knappe \textbf{hinder} sagete\\ 
 & al daz sîn vrouwe klagete.\\ 
 & "Ich sol vor \textbf{triegen} uns bewarn",\\ 
 & sprach Tscherules. "ich wil dar varn."\\ 
 & \begin{large}E\end{large}r reit hin \textbf{ûz}, dâ Gawan saz,\\ 
20 & der \textbf{selten ellens} ie vergaz.\\ 
 & an dem \textbf{er vant} krancheite vlust,\\ 
 & lieh\textit{t} antlitze \textbf{unde} \textbf{starke} brust\\ 
 & unde einen rîter wol gevar.\\ 
 & Tscherules in prüevete gar,\\ 
25 & sîne arme unde ietweder hant\\ 
 & unde swaz \textbf{er geschickede} dâ vant.\\ 
 & Dô sprach er: "hêrre, ir sît ein gast.\\ 
 & guoter witze uns gebrast,\\ 
 & sît ir niht herberge hât.\\ 
30 & \textbf{nû} prüevetz \textbf{niht} vür missetât.\\ 
\end{tabular}
\scriptsize
\line(1,0){75} \newline
T V W \newline
\line(1,0){75} \newline
\textbf{1} \textit{Initiale} T  \textbf{15} \textit{Majuskel} T  \textbf{17} \textit{Majuskel} T  \textbf{19} \textit{Initiale} T V  \textbf{27} \textit{Majuskel} T  \newline
\line(1,0){75} \newline
\textbf{1} Mînen] Einen V (W)  $\cdot$ gesprechen] sich besprechen W \textbf{3} Tscherules] Tscervles T schervles V scherules W \textbf{7} underm œlboume] Vndern oleyboͮmen V \textbf{9} unde] Vnd auch W \textbf{10} triegen] betrigen W \textbf{11} erz] er in W \textbf{12} des] des wol W \textbf{13} nemenz] nementz T nemens [*]: im V \textbf{15} hinder] [*]: hin nider V hin nider W \textbf{18} Tscherules] Tscervles T Schervles V scherules W \textbf{19} ûz dâ] vf do V (W) \textbf{21} dem] der W  $\cdot$ krancheite] selten krancke W \textbf{22} lieht] liehte T  $\cdot$ starke] \textit{om.} W \textbf{24} Tscherules] Tscervles T Schervles V De scherules W  $\cdot$ prüevete] gepruͤfete W \textbf{25} ietweder] ie deweder W \textbf{26} swaz] was W  $\cdot$ dâ] do V W \textbf{27} hêrre] \textit{om.} W  $\cdot$ sît] seit hie W \textbf{28} uns] vnß allen W \textbf{29} herberge] herbergen V \textbf{30} nû prüevetz niht] [N*]: Nv pruͤventz vnz niht V Nit pruͤfent es vns W \newline
\end{minipage}
\end{table}
\end{document}
