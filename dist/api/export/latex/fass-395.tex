\documentclass[8pt,a4paper,notitlepage]{article}
\usepackage{fullpage}
\usepackage{ulem}
\usepackage{xltxtra}
\usepackage{datetime}
\renewcommand{\dateseparator}{.}
\dmyyyydate
\usepackage{fancyhdr}
\usepackage{ifthen}
\pagestyle{fancy}
\fancyhf{}
\renewcommand{\headrulewidth}{0pt}
\fancyfoot[L]{\ifthenelse{\value{page}=1}{\today, \currenttime{} Uhr}{}}
\begin{document}
\begin{table}[ht]
\begin{minipage}[t]{0.5\linewidth}
\small
\begin{center}*D
\end{center}
\begin{tabular}{rl}
\textbf{395} & \textbf{\textit{\begin{large}L\end{large}}yppaut}, sîn wîp, sîniu kint.\\ 
 & \textbf{ûf} giengen die dâ komen sint.\\ 
 & \textbf{der wirt} gein sîme hêrren spranc.\\ 
 & ûf \textbf{dem} palase \textbf{was} grôz gedranc,\\ 
5 & dâ er \textbf{den vîent} unt \textbf{die vriwende} enpfienc.\\ 
 & \textbf{Melyanz} bî \textbf{Gawane} gienc.\\ 
 & "kunde\textbf{z} iu niht versmâhen,\\ 
 & mit kusse \textbf{iuch} \textbf{wolde} \textbf{enpfâhen}\\ 
 & iwer altiu vriwendîn,\\ 
10 & ich meine mîn wîp, \textbf{die} herzogîn."\\ 
 & Melyanz antwurte dem wirte sân:\\ 
 & "ich wil \textbf{gern} ir kus mit \textbf{gruoze} hân,\\ 
 & zweier vrouwen, \textbf{die} ich hie sihe,\\ 
 & der dritten ich niht suone gihe."\\ 
15 & des weinten die eltern dô.\\ 
 & Obilot was \textbf{vil} vrô.\\ 
 & Der künec mit kusse enpfangen wart\\ 
 & unt zwêne ander künege âne bart,\\ 
 & \textbf{als tet} der herzoge Marangliez.\\ 
20 & Gawanen \textbf{man kusses} \textbf{ouch} niht erliez\\ 
 & \textbf{unt daz er næme} sîne vrouwen dar.\\ 
 & er dructe daz kint wol gevar\\ 
 & als \textbf{eine tocken} an sîne brust.\\ 
 & des twang in vriwentlîch gelust.\\ 
25 & \textbf{hin} ze Melyanze \textbf{er} sprach:\\ 
 & "iwer hant mir sicherheite \textbf{jach}.\\ 
 & der sît nû ledec unt gebt si her.\\ 
 & aller mîner vreuden wer\\ 
 & sitzet an dem arme mîn.\\ 
30 & \textbf{ir} gevangen sult ir sîn."\\ 
\end{tabular}
\scriptsize
\line(1,0){75} \newline
D \newline
\line(1,0){75} \newline
\textbf{1} \textit{Initiale} D  \textbf{17} \textit{Majuskel} D  \newline
\line(1,0){75} \newline
\textbf{1} Lyppaut] ÷yppaot \textit{nachträglich korrigiert zu:} Lyppaot D \textbf{19} Marangliez] Maranglŷez D \textbf{20} Gawanen] Gawann D \newline
\end{minipage}
\hspace{0.5cm}
\begin{minipage}[t]{0.5\linewidth}
\small
\begin{center}*m
\end{center}
\begin{tabular}{rl}
 & \textbf{Lippo\textit{u}t}, sî\textit{n} wîp \textbf{und} sîniu kint.\\ 
 & \textbf{ûf} giengen \textit{die} d\textit{â} komen sint.\\ 
 & \textbf{der wirt} gege\textit{n s}înem hêrren spranc.\\ 
 & ûf \textbf{dem} palas \textbf{was} grô\textit{z} gedranc,\\ 
5 & d\textit{â} er \textbf{den vîent} und \textbf{den vriunt} enpfienc.\\ 
 & \textbf{Mel\textit{i}anz} bî \textbf{Gawan} gienc.\\ 
 & "kond\textit{e}\textbf{z} iu niht versmâhen,\\ 
 & mit kusse \textbf{iuch} \textbf{wolte} \textbf{enpfâhen},\\ 
 & \textbf{hêrre}, iuwer altiu vriundîn,\\ 
10 & ich meine \textit{mîn} wîp, \textbf{die} herzogîn."\\ 
 & Mel\textit{i}anz antwurt\textit{e d}em wirte sân:\\ 
 & "ich wil \textbf{gerne} ir kus mit \textbf{grüezen} hân,\\ 
 & zweier vrouwen, \textbf{die} ich hie sihe,\\ 
 & der dritten ich niht su\textit{on}e gihe."\\ 
15 & des weineten die alteren dô.\\ 
 & Obilot was \textbf{vaste} vrô.\\ 
 & der künic mit kusse enpfangen wart\\ 
 & und zwêne andere künige âne bart,\\ 
 & \textbf{als tet} der herzoge Marangliez.\\ 
20 & Gawan \textbf{man kusses} \textbf{ouch} niht erliez\\ 
 & \textbf{und daz er næme} sîne vrouwen dar.\\ 
 & er druhte daz kint wol gevar\\ 
 & als \textbf{eine tocken} an sîne brust.\\ 
 & des twanc in vriuntlîch gelust.\\ 
25 & \textbf{hin} ze Melianze \textbf{er} sprach:\\ 
 & "iuwer hant mir sicherheit \textbf{jach}.\\ 
 & der sît nû ledic und gebet si her.\\ 
 & aller mîner vröuden wer\\ 
 & sitz\textit{e}t an dem arme mîn.\\ 
30 & \textbf{ir} gevangen sullet ir sîn."\\ 
\end{tabular}
\scriptsize
\line(1,0){75} \newline
m n o \newline
\line(1,0){75} \newline
\newline
\line(1,0){75} \newline
\textbf{1} Lippout] Lippoat m Lippaot n o  $\cdot$ sîn] sine m  $\cdot$ sîniu] sin n o \textbf{2} die dâ] do m die do n \textbf{3} gegen sînem] gegen dem sinem m \textbf{4} was] \textit{om.} n o  $\cdot$ grôz] grosse m \textbf{5} dâ] Do m n o  $\cdot$ den vriunt] frint n (o) \textbf{6} Melianz] Meleancz m Meliantz n Meliancz o  $\cdot$ Gawan] Gawane n o \textbf{7} kondez] Kondencz m (n) \textbf{8} kusse] kúsch n (o)  $\cdot$ enpfâhen] enpohen o \textbf{9} vriundîn] fruͯnden o \textbf{10} mîn] \textit{om.} m \textbf{11} Melianz] Meleancz m meliantz n Meliancz o  $\cdot$ antwurte dem] antwurte vnd dem m \textbf{12} kus] kúsch n (o)  $\cdot$ grüezen] grússe n (o) \textbf{14} dritten ich] dirte ich hie o  $\cdot$ suone] sunte m \textbf{17} kusse] kúsch n (o) \textbf{18} andere] vnd ir o \textbf{19} der] \textit{om.} o  $\cdot$ Marangliez] maranglies m o marangliesz n \textbf{20} ouch] \textit{om.} n o \textbf{25} Melianze] meliancze m meliantz n meliancz o \textbf{29} sitzet] Siczent m \newline
\end{minipage}
\end{table}
\newpage
\begin{table}[ht]
\begin{minipage}[t]{0.5\linewidth}
\small
\begin{center}*G
\end{center}
\begin{tabular}{rl}
 & \textbf{\begin{large}D\end{large}er wirt}, sîn wîp \textbf{unde} sîniu kint.\\ 
 & \textbf{vür} giengen die dâ komen sint.\\ 
 & \textbf{Libaut} gein sînem hêrren spranc.\\ 
 & ûf \textbf{dem} palase \textbf{wart} grôz gedranc,\\ 
5 & dâ er \textbf{die vînde} und\textit{e} \textbf{\textit{v}riunde} enpfienc.\\ 
 & \textbf{Gawan} bî \textbf{Melianze} gienc.\\ 
 & "kunde \textbf{ez} iu niht versmâhen,\\ 
 & mit kusse \textbf{iuch} \textbf{wolt} \textbf{enpfâhen}\\ 
 & iwer altiu vriundîn,\\ 
10 & ich meine mîn wîp, \textbf{die} herzogîn."\\ 
 & Melianz antwurte \textit{de}m \textit{wirte} sân:\\ 
 & "ich wil ir kus mit \textbf{gruoze} hân,\\ 
 & zweier vrouwen, \textbf{die} ich hie sihe,\\ 
 & der driten ich niht suone gihe."\\ 
15 & des weinden die elteren \textbf{bêde} dô.\\ 
 & Obilot was \textbf{vaste} vr\textit{ô}.\\ 
 & der künic mit kusse enpfangen wart\\ 
 & unt zwêne ander künige âne bart,\\ 
 & \textbf{als tet} der herzoge Marangliez.\\ 
20 & Gawan \textbf{man kusses} \textbf{ê} niht erliez\\ 
 & \textbf{unde er næme} sîne vrouwen dar.\\ 
 & er dructe daz kint wolgevar\\ 
 & als \textbf{ein tockelîn} an sîne brust.\\ 
 & des twanc in vriuntlîch gelust.\\ 
25 & \textbf{Gawan} ze Melianze sprach:\\ 
 & "iwer hant mir sicherheit \textbf{verjach}.\\ 
 & der sît nû ledec unde gebet si her.\\ 
 & aller mîner vröuden wer\\ 
 & sitzet an dem arme mîn.\\ 
30 & \textbf{der} gevangen sult ir sîn."\\ 
\end{tabular}
\scriptsize
\line(1,0){75} \newline
G I O L M Q R Z Fr28 \newline
\line(1,0){75} \newline
\textbf{1} \textit{Initiale} G  \textbf{3} \textit{Initiale} I O L Z  \textbf{17} \textit{Initiale} I  \newline
\line(1,0){75} \newline
\textbf{1} \textit{Die Verse 370.13-412.12 fehlen} Q   $\cdot$ sîn] Sine M sy R  $\cdot$ unde] \textit{om.} M R Fr28  $\cdot$ sîniu] sine I R \textbf{2} \textit{Vers 395.2 fehlt} R   $\cdot$ giengen] Gein den I  $\cdot$ dâ] da nu I \textbf{3} Libaut] ÷ybavt O Lybauͯt L Lybant R Lybavt Z \textbf{4} ûf] Vsze M  $\cdot$ wart] was so I  $\cdot$ grôz] \textit{om.} M \textbf{5} dâ] daz I  $\cdot$ er die] er I L M Fr28 man R  $\cdot$ vînde unde vriunde] vinde vnde die frivnde G frunt vnd viand Fr28 \textbf{6} Meliantz bi gawan gienc Z  $\cdot$ Melianze] Melianzen I Melyanzen O Meleanze L Meliancze R :::ze Fr28 \textbf{8} kusse] kussen I  $\cdot$ enpfâhen] verschmachen R \textbf{9} altiu] alleú R \textbf{10} die] div O \textbf{11} Melianz] Melyanz O Mielianz L Meliancz R Meliantz Z  $\cdot$ antwurte] antwurt I (O) (R) (Z)  $\cdot$ dem wirte] im G \textbf{12} ir] \textit{om.} O  $\cdot$ hân] enpfan Z \textbf{13} zweier] Vwir M  $\cdot$ sihe] sihen M \textbf{14} niht suone] sin nih I  $\cdot$ gihe] engihe I gihen M verieche R \textbf{15} elteren bêde] elter bede O alte beidu R  $\cdot$ dô] da M \textbf{16} Obilot] Obylot O Oblet R obẏlot Fr28  $\cdot$ was vaste] vaste was R  $\cdot$ vrô] froͮ G \textbf{17} der] Den R  $\cdot$ kusse] kussen M \textbf{18} ander] \textit{om.} Fr28 \textbf{19} Marangliez] Maranglies I (M) R Moranglyez O Marangleiz L \textbf{20} Gawan] Gawanen O (Fr28)  $\cdot$ man kusses] man chussens I kvsses man L (Fr28) man och kusses R  $\cdot$ ê] \textit{om.} O L M R Z Fr28  $\cdot$ erliez] enliez I (M) \textbf{21} er] das er R (Z) ern Fr28  $\cdot$ næme] nant im auch I næme ovch O (L) (Z) (Fr28) nam ouch M \textbf{22} dructe] druct I \textbf{23} tockelîn] tochen O (L) (M) (R) (Z) (Fr28) \textbf{25} ze Melianze] zemelianze G ze Melyanzen O ze Melianczen R zv meliantze Z \textbf{26} hant] \textit{om.} O \textbf{27} der] des I (O)  $\cdot$ unde] \textit{om.} O  $\cdot$ gebet] geb I \textbf{28} wer] [ger]: wer I \textbf{29} sitzet] Siczent R \textbf{30} Jr svlt ir gevangen sin Z  $\cdot$ ir] ir aber I \newline
\end{minipage}
\hspace{0.5cm}
\begin{minipage}[t]{0.5\linewidth}
\small
\begin{center}*T
\end{center}
\begin{tabular}{rl}
 & \textbf{der wirt}, sîn wîp, sîniu kint.\\ 
 & \textbf{vür} giengen die dâ komen sint.\\ 
 & \textbf{Lybaut} \textbf{ûf} gegen sînem hêrren spranc.\\ 
 & ûf \textbf{den} palas \textbf{wart} grôz gedranc,\\ 
5 & dâ er \textbf{vîent} unde \textbf{vriunt} enpfienc.\\ 
 & \textbf{Gawan} bî \textbf{Melyanze} gienc.\\ 
 & "Kunde\textbf{r} iu niht versmâhen,\\ 
 & mit kusse \textbf{iu} \textbf{solte} \textbf{nâhen}\\ 
 & iuwer alt\textit{iu} vriundîn,\\ 
10 & ich meine mîn wîp, \textbf{diu} herzogîn."\\ 
 & \textit{Melyanz antwurte dem wirte sân:}\\ 
 & "ich wil ir kus mit \textbf{gruoze} hân,\\ 
 & zweier vrouwen, \textbf{der} ich hie sihe,\\ 
 & der dritten ich niht suone gihe."\\ 
15 & Des weinden die eltern \textbf{beide} dô.\\ 
 & Obylot was \textbf{vaste} vrô.\\ 
 & Der künec mit kusse enpfangen wart\\ 
 & unde zwêne andere künege âne bart\\ 
 & \textbf{unde} der herzoge Marangliez.\\ 
20 & Gawanen \textbf{küssens man} niht erliez.\\ 
 & \textbf{er nam ouch} sîne vrouwen dar.\\ 
 & er dructe daz kint wol gevar\\ 
 & als \textbf{eine tocken} an sîne brust.\\ 
 & des twanc in vriuntlîch gelust.\\ 
25 & \textbf{\begin{large}G\end{large}awan} ze Melyanze sprach:\\ 
 & "iuwer hant mir sicherheit \textbf{verjach}.\\ 
 & der sît nû ledic unde gebet si her.\\ 
 & aller mîner vröuden wer\\ 
 & sitzet an dem arme mîn.\\ 
30 & \textbf{der} gevangene sult ir \textbf{nû} sîn."\\ 
\end{tabular}
\scriptsize
\line(1,0){75} \newline
T V W \newline
\line(1,0){75} \newline
\textbf{3} \textit{Initiale} W  \textbf{6} \textit{Majuskel} T  \textbf{7} \textit{Majuskel} T  \textbf{11} \textit{Majuskel} T  \textbf{15} \textit{Majuskel} T  \textbf{17} \textit{Majuskel} T  \textbf{25} \textit{Initiale} T  \newline
\line(1,0){75} \newline
\textbf{1} wîp] wip vnde V (W) \textbf{2} dâ] do V W \textbf{3} Lybaut] Lippaut V LYbout W  $\cdot$ ûf] \textit{om.} V W \textbf{4} ûf den] Vffem V Auff dē W  $\cdot$ wart] was V (W) \textbf{5} dâ er vîent] Do er vient V Do veinde W \textbf{6} Melyanze] melianze V W \textbf{7} Kunder] Kvnde ez V (W) \textbf{8} solte nâhen] wolte enpfahen V (W) \textbf{9} iuwer] Herre uwer V  $\cdot$ altiu] alte T \textbf{11} Do sprach der werde Gawan T  $\cdot$ Melyanz] Melianz V W \textbf{12} wil] wil gerne V \textbf{13} der] die V W \textbf{14} gihe] [*ihe]: engihe V \textbf{15} eltern] alten W \textbf{16} Obylot] Obelot T Obilot V Abilot W \textbf{19} unde] Alse tet V (W)  $\cdot$ Marangliez] Maranglîez T maranglies W \textbf{20} Gawanen] Gawan V  $\cdot$ küssens man] men kvssens V kusses man W \textbf{21} er nam ouch] Vnde daz er neme V (W) \textbf{25} Gawan] KAwan T  $\cdot$ Melyanze] melianze V melianz W \textbf{28} vröuden] frawen W \textbf{30} nû] \textit{om.} W \newline
\end{minipage}
\end{table}
\end{document}
