\documentclass[8pt,a4paper,notitlepage]{article}
\usepackage{fullpage}
\usepackage{ulem}
\usepackage{xltxtra}
\usepackage{datetime}
\renewcommand{\dateseparator}{.}
\dmyyyydate
\usepackage{fancyhdr}
\usepackage{ifthen}
\pagestyle{fancy}
\fancyhf{}
\renewcommand{\headrulewidth}{0pt}
\fancyfoot[L]{\ifthenelse{\value{page}=1}{\today, \currenttime{} Uhr}{}}
\begin{document}
\begin{table}[ht]
\begin{minipage}[t]{0.5\linewidth}
\small
\begin{center}*D
\end{center}
\begin{tabular}{rl}
\textbf{723} & \begin{large}N\end{large}û was \textbf{ez} ouch ime her \textbf{sô} komen,\\ 
 & Artus hete \textbf{al} \textbf{dâ} genomen\\ 
 & vride von der herzogîn.\\ 
 & der was ergetzens gewin\\ 
5 & komen nâch Cidegaste,\\ 
 & den si \textbf{ê klagete} \textbf{sô} vaste.\\ 
 & ir zorn was nâch verdecket:\\ 
 & si hete \textbf{gewecket}\\ 
 & von Gawane \textbf{etslîch} umbevanc;\\ 
10 & dâ von ir \textbf{zorn} was \textbf{sô} kranc.\\ 
 & Artus der Bertenoys\\ 
 & nam die \textbf{clâren} vrouwen kurtois,\\ 
 & beide \textbf{meide} und wîp,\\ 
 & die truogen \textbf{vlæteclîchen} lîp.\\ 
15 & er hete der werden hundert\\ 
 & \textbf{in} ein gezelt gesundert.\\ 
 & Niht lieber m\textit{ö}ht \textbf{ir} sîn geschehen,\\ 
 & \textbf{wan} daz si den künec \textbf{solde} sehen,\\ 
 & \textbf{Itonjen}, diu ouch dâ saz.\\ 
20 & stæter \textbf{vreude} \textbf{si} niht vergaz,\\ 
 & \textbf{doch} kôs man an ir ougenschîn,\\ 
 & daz si \textbf{diu} minne lêrte pîn.\\ 
 & dâ saz manec rîter lieht gemâl,\\ 
 & \textbf{doch} truoc der werde Parzival\\ 
25 & den prîs vor ander clârheit.\\ 
 & Gramoflanz an die snüere reit.\\ 
 & dô vuorte der \textbf{künec} unervorht,\\ 
 & in \textbf{Gampfassasche} geworht,\\ 
 & einen pfelle \textbf{mit} golde vesten;\\ 
30 & der begunde \textbf{verre} glesten.\\ 
\end{tabular}
\scriptsize
\line(1,0){75} \newline
D \newline
\line(1,0){75} \newline
\textbf{1} \textit{Initiale} D  \textbf{17} \textit{Majuskel} D  \newline
\line(1,0){75} \newline
\textbf{11} Bertenoys] Bertenoẏs D \textbf{17} möht] moht D \textbf{19} Itonjen] Jtonien D \textbf{24} Parzival] Parcifal D \textbf{28} Gampfassasche] Ganpfassasce D \newline
\end{minipage}
\hspace{0.5cm}
\begin{minipage}[t]{0.5\linewidth}
\small
\begin{center}*m
\end{center}
\begin{tabular}{rl}
 & \begin{large}N\end{large}û was ouch in dem her komen,\\ 
 & \textbf{daz} Artus het \textit{ge}nomen\\ 
 & vride von der herzogîn.\\ 
 & der was ergetzens gewin\\ 
5 & komen nâch Zidegast,\\ 
 & den si \textbf{klagte ê} \textbf{sô} vast.\\ 
 & ir zorn was nâch verdecket,\\ 
 & \textbf{wan} si hette \textbf{erwecket}\\ 
 & von Gawan \textbf{etlîchen} umb\textit{e}v\textit{a}nc;\\ 
10 & d\textit{â} von ir \textbf{zorn} was \textbf{worden} kranc.\\ 
 & Artus der Brittunois\\ 
 & nam die \textbf{clâre\textit{n}} vrowen \dag artois\dag ,\\ 
 & beide \textbf{megde} und wîp,\\ 
 & die truogen \textbf{vlæteclîchen} lîp.\\ 
15 & er het \textit{der} werden hundert\\ 
 & \textbf{in} ein gezelt gesundert.\\ 
 & niht lieber m\textit{ö}hte sîn geschehen,\\ 
 & \textbf{dan} daz si den künic \textbf{solten} sehen.\\ 
 & \textbf{Ithonie}, diu ouch d\textit{â} saz,\\ 
20 & stæter \textbf{vröude} \textbf{si} niht vergaz,\\ 
 & \textbf{doch} kôs man an ir ougenschîn,\\ 
 & daz si minne lêrte pîn.\\ 
 & d\textit{â} saz manic ritter lieht gemâl,\\ 
 & \textbf{doch} truoc der werde Parcifal\\ 
25 & den prîs vor \textit{ander} clârheit.\\ 
 & Gramolantz an die snüer reit.\\ 
 & dô vuorte der \textbf{künic} unervorht,\\ 
 & in \textbf{Ganfassasc} geworht,\\ 
 & einen pfelle \textbf{mit} golde vesten;\\ 
30 & der begunde \textbf{vaste} glesten.\\ 
\end{tabular}
\scriptsize
\line(1,0){75} \newline
m n o Fr69 \newline
\line(1,0){75} \newline
\textbf{1} \textit{Initiale} m n  \newline
\line(1,0){75} \newline
\textbf{2} genomen] vernomen m \textbf{4} ergetzens] erczeczens o \textbf{8} wan] Wenne n \textbf{9} umbevanc] vmb fing m \textbf{10} dâ] Do m n o \textbf{11} Artus] Artuͯs o  $\cdot$ Brittunois] britunois n britonús o \textbf{12} clâren] clare m \textbf{14} vlæteclîchen] flechteclichen n \textbf{15} der] \textit{om.} m \textbf{17} möhte] mohtte m (o) \textbf{19} Ithonie] Jthonie m n Jtonie o  $\cdot$ dâ] do m n o \textbf{21} kôs] kose n  $\cdot$ ir] iren n \textbf{22} minne] die minne n (o) \textbf{23} dâ] Do m n o \textbf{24} doch] da Fr69 \textbf{25} ander] \textit{om.} m \textbf{26} Gramolantz] Gramolancz o \textbf{28} Ganfassasc] ganfassat o \textbf{30} vaste] ferre n o \newline
\end{minipage}
\end{table}
\newpage
\begin{table}[ht]
\begin{minipage}[t]{0.5\linewidth}
\small
\begin{center}*G
\end{center}
\begin{tabular}{rl}
 & nû was \textbf{ez} ouch in dem her \textbf{sô} komen,\\ 
 & \begin{large}A\end{large}rtus hete \textbf{dâ} genomen\\ 
 & \textbf{einen} vride von der herzogîn.\\ 
 & der was ergetzens gewin\\ 
5 & komen nâch Zidegaste,\\ 
 & den si \textbf{ê klagte} vaste.\\ 
 & ir zorn was nâch verdeckt,\\ 
 & \textbf{wan} si het \textbf{erwecket}\\ 
 & von Gawane \textbf{manec} umbevanc;\\ 
10 & dâ von ir \textbf{zürnen} was \textbf{sô} kranc.\\ 
 & Artus der Britaneis\\ 
 & nam die vrouwen kurteis,\\ 
 & beide \textbf{maget} unde wîp,\\ 
 & die truogen \textbf{vlætigen} lîp.\\ 
15 & er het der werden hundert\\ 
 & \textbf{under} ein gezelt gesundert.\\ 
 & niht lieber m\textit{ö}hte \textbf{ir} sîn geschehen,\\ 
 & daz si den künic \textbf{solde} sehen,\\ 
 & \textbf{Itonie}, diu ouch dâ saz\\ 
20 & \textbf{unde} stæter \textbf{vröude} niht vergaz.\\ 
 & \textbf{dô} kôs man an ir ougenschîn,\\ 
 & daz si \textbf{diu} minne lêrte pîn.\\ 
 & dâ saz manec rîter lieht gemâl.\\ 
 & \textbf{dô} truoc der werde Parcival\\ 
25 & den brîs vor ander clârheit.\\ 
 & Gramoflanz an die snüere reit.\\ 
 & dô v\textit{u}o\textit{r}te der \textbf{degen} unervorht,\\ 
 & in \textbf{Tschofflanze} geworht,\\ 
 & einen pfelle \textbf{von} golde vesten;\\ 
30 & der begunde \textbf{verre} glesten.\\ 
\end{tabular}
\scriptsize
\line(1,0){75} \newline
G I L M Z Fr20 Fr45 \newline
\line(1,0){75} \newline
\textbf{1} \textit{Initiale} I L M Z Fr20  \textbf{2} \textit{Initiale} G  \newline
\line(1,0){75} \newline
\textbf{1} nû] ÷v Fr20  $\cdot$ ouch] \textit{om.} M Fr45  $\cdot$ in dem] an die Fr45  $\cdot$ sô] \textit{om.} Z \textbf{3} vride] \textit{om.} M  $\cdot$ herzogîn] herzgin Fr20 \textbf{5} Zidegaste] Cidegaste G (Z) (Fr20) zitegaste I [*]: Citegaste L zcitegaste M Cytigaste Fr45 \textbf{6} ê] \textit{om.} I y M  $\cdot$ klagte] clagete so L (M) clagt so Z \textbf{7} verdeckt] [er wechet]: er verdechet I \textbf{9} Gawane] Gawan I L (Z) Fr45  $\cdot$ manec] etslich L (Z) Fr45 ettislicher M \textbf{10} zürnen] zorn I \textbf{11} der] der chunic Fr20  $\cdot$ Britaneis] britanoeis G pritonoys I Brittanoýs L britunois Z britteneẏs Fr45 \textbf{13} \textit{Versfolge 723.14-13} M Fr45   $\cdot$ beide] beidiu I  $\cdot$ maget] megde Z (Fr45) \textbf{16} an hohem prise uz gesundert Fr20 \textbf{17} möhte] mohte G (I) (L) (M) (Z) (Fr20) machte Fr45 \textbf{18} daz si den] Daz sẏ der L wen dazs en Fr45 \textbf{19} \textit{Verse 723.19-23 kontrahiert zu:} Jtonie div och das [sz]: saz so lieht gemal Fr20   $\cdot$ Itonie] Jtonîe G Jconie Z Jtonie Fr45 \textbf{20} \textit{Die Verse 723.20-22 fehlen} Fr20   $\cdot$ niht] ny M \textbf{21} dô] Da M Z Fr45 \textbf{22} minne lêrte] lertan I \textbf{23} dâ] Do L  $\cdot$ lieht] licht L M \textbf{24} dô] Da M Z nv Fr20  $\cdot$ Parcival] parcifal G Z Fr20 parzifal I L M persciual Fr45 \textbf{25} vor] fvr Z Fr20 \textbf{26} Gramoflanz] Gramorflanz M Gramoflantz Z \textbf{27} dô] Da M Z  $\cdot$ vuorte] forhte G cherte Fr20 \textbf{28} in Tschofflanze] Jntschoffanze G inshoffanze I Jn Tschoflanze L Jn kanfaz saszte M Jn kanfassashe Z intschofanz Fr20 in kamfassatsche Fr45 \textbf{29} von] mit L M Fr45  $\cdot$ vesten] veste M \newline
\end{minipage}
\hspace{0.5cm}
\begin{minipage}[t]{0.5\linewidth}
\small
\begin{center}*T
\end{center}
\begin{tabular}{rl}
 & \begin{large}N\end{large}û was \textbf{ez} ouch in dem here \textbf{sô} komen,\\ 
 & Artus hete \textbf{dâ} genomen\\ 
 & \textbf{einen} vriden von der herzogîn.\\ 
 & der \textit{was} ergetzens gewin\\ 
5 & komen nâch Cydegaste,\\ 
 & den si \textbf{klagete} \textbf{sô} vaste.\\ 
 & ir zorn was nâch verdecket,\\ 
 & \textbf{wan} si hete \textbf{erwecket}\\ 
 & von Gawane \textbf{etlîchen} umbevanc;\\ 
10 & d\textit{â} von ir \textbf{zürnen} was \textbf{sô} kranc.\\ 
 & \begin{large}A\end{large}rtus der Brituneis\\ 
 & nam die \textbf{clâren} vrouwen kurteis,\\ 
 & beide \textbf{maget} und wîp,\\ 
 & die truogen \textbf{vlætigen} lîp.\\ 
15 & er hete der werden hundert\\ 
 & \textbf{under} ein gezelte gesundert.\\ 
 & niht lieber m\textit{ö}ht \textbf{ir} sîn geschehen,\\ 
 & daz si den künec \textbf{solte} sehen,\\ 
 & \textbf{Itonie}, diu ouch dâ saz\\ 
20 & \textbf{und} stæter \textbf{vreuden} niht vergaz.\\ 
 & \textbf{dô} kôs man an ir ougenschîn,\\ 
 & daz si \textbf{diu} minne lêrte pîn.\\ 
 & dâ saz manec rîter lieht gemâl,\\ 
 & \textbf{doch} truoc der werde Parcifal\\ 
25 & den prîs vor ander clârheit.\\ 
 & Gramoflanz an die snüere reit.\\ 
 & dô vuorte der \textbf{künec} unervorht,\\ 
 & in \textbf{Kanfassaie} geworht,\\ 
 & eine\textit{n} pfelle \textbf{mit} golde vesten;\\ 
30 & der begunde \textbf{verre} glesten.\\ 
\end{tabular}
\scriptsize
\line(1,0){75} \newline
U V W Q R \newline
\line(1,0){75} \newline
\textbf{1} \textit{Initiale} U R  \textbf{11} \textit{Initiale} U V W  \newline
\line(1,0){75} \newline
\textbf{1} ez] \textit{om.} W \textbf{2} Artus] Kúnig artus W  $\cdot$ dâ] do V W Q R \textbf{3} vriden] fride V Q \textbf{4} was] \textit{om.} U  $\cdot$ ergetzens] ergeczent R \textbf{5} Cydegaste] Gydegaste V cytegaste W Cidegaste Q R \textbf{6} si] sv́ e V (W) (Q) (R)  $\cdot$ klagete] klagt W  $\cdot$ sô] \textit{om.} Q \textbf{9} Gawane] gawan W Gawin R  $\cdot$ etlîchen] ettelich V (W) (Q) (R) \textbf{10} dâ] Do U V W Q  $\cdot$ zürnen] zwrne Q \textbf{11} Artus] ÷Vnig artus W  $\cdot$ Brituneis] Brituͦneis U britoneise Q Britonois R \textbf{13} maget] [*]: megede V \textbf{14} vlætigen] [fleti*]: fletigen V \textbf{15} hete] hetre W \textbf{16} gezelte gesundert] zelt besunder R \textbf{17} möht] mocht U Q R \textbf{18} sehen] schawen Q \textbf{19} Itonie] Jtonie U Q R Jconie V Ytonie W  $\cdot$ dâ] do V W Q \textbf{20} stæter vreuden] stete frewde Q \textbf{22} \textit{Versdoppelung nach 722.3} R  \textbf{23} dâ] Do V W Q  $\cdot$ lieht gemâl] licht gemal Q \textbf{24} doch] Do W  $\cdot$ Parcifal] Parzifal U parzefal V partzifal W Q parczifal R \textbf{25} den] Dan W  $\cdot$ vor] fúr W  $\cdot$ ander] der Q \textbf{26} Gramoflanz] Gramaflancz V Gramoflantz W Q Gramoflancz R \textbf{27} künec] degen Q (R) \textbf{28} Kanfassaie] kankasas W kanfasaie Q \textbf{29} einen] Einez U [E*]: Einen V Eine Q  $\cdot$ pfelle mit] pfellen von R \newline
\end{minipage}
\end{table}
\end{document}
