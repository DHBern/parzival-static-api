\documentclass[8pt,a4paper,notitlepage]{article}
\usepackage{fullpage}
\usepackage{ulem}
\usepackage{xltxtra}
\usepackage{datetime}
\renewcommand{\dateseparator}{.}
\dmyyyydate
\usepackage{fancyhdr}
\usepackage{ifthen}
\pagestyle{fancy}
\fancyhf{}
\renewcommand{\headrulewidth}{0pt}
\fancyfoot[L]{\ifthenelse{\value{page}=1}{\today, \currenttime{} Uhr}{}}
\begin{document}
\begin{table}[ht]
\begin{minipage}[t]{0.5\linewidth}
\small
\begin{center}*D
\end{center}
\begin{tabular}{rl}
\textbf{231} & \textbf{\textit{\begin{large}D\end{large}}er wirt} het durch siecheit\\ 
 & \textbf{grôziu} viwer unt \textbf{an im} warmiu kleit.\\ 
 & wît \textbf{unt} lanc zobelîn,\\ 
 & \textbf{sus} muose \textbf{ûzen} unt innen sîn\\ 
5 & \textbf{der} bellîz unt \textbf{der} mantel drobe.\\ 
 & der swecheste balc \textbf{wære} wol ze lobe,\\ 
 & der was doch swarz unt grâ.\\ 
 & des \textbf{selben} was ein hûbe \textbf{al}dâ\\ 
 & ûf sîme houbte, zwivalt\\ 
10 & von zobele, den man tiure galt.\\ 
 & sinwel \textbf{arabesch ein} borte\\ 
 & oben drûf \textbf{gehôrte},\\ 
 & \textbf{mitten dran} ein knopfelîn,\\ 
 & ein durchliuhtic rubîn.\\ 
15 & dâ \textbf{saz} manec ritter kluoc,\\ 
 & dâ man jâmer vür si truoc.\\ 
 & ein knappe spranc zer tür \textbf{dar} în,\\ 
 & der truog eine glevîn.\\ 
 & \textbf{der} site was ze trûren guot:\\ 
20 & an der snîden \textbf{huop} sich bluot\\ 
 & unt lief den schaft unz \textbf{ûf} die hant,\\ 
 & \textbf{dârz} \textbf{in} dem ermel \textbf{widerwant}.\\ 
 & Dâ wart geweinet unt geschrît\\ 
 & \textbf{ûf} dem palase wît.\\ 
25 & daz \textbf{volc} von drîzec landen\\ 
 & \textbf{m\textit{ö}htez} \textbf{den ougen niht} enblanden.\\ 
 & er truoc si in sînen henden\\ 
 & \textbf{alumbe zen} vier wenden\\ 
 & \textbf{unz} aber wider zuo der tür.\\ 
30 & der knappe spranc hin ûz \textbf{dar vür}.\\ 
\end{tabular}
\scriptsize
\line(1,0){75} \newline
D \newline
\line(1,0){75} \newline
\textbf{1} \textit{Initiale} D  \textbf{23} \textit{Majuskel} D  \newline
\line(1,0){75} \newline
\textbf{1} Der] ÷er \textit{nachträglich korrigiert zu:} Der D \textbf{26} möhtez] mohtez D \newline
\end{minipage}
\hspace{0.5cm}
\begin{minipage}[t]{0.5\linewidth}
\small
\begin{center}*m
\end{center}
\begin{tabular}{rl}
 & \textbf{und} hete durch \textbf{sîne} siecheit\\ 
 & \textbf{grôziu} viure und warmiu kleit.\\ 
 & wît \textbf{und} lanc zobelîn,\\ 
 & \textbf{sus} muose \textbf{ûzen} und inne sîn\\ 
5 & \textbf{ein} belz und \textbf{ein} mantel drobe.\\ 
 & der swecheste balc \textbf{wær} wol ze lobe,\\ 
 & der was doch swarz und grâ.\\ 
 & des was ein hûbe d\textit{â}\\ 
 & ûf sînem houbete, zwivalt\\ 
10 & von zobele, den man tiure galt.\\ 
 & sinewel \textbf{arabisch ein} borte\\ 
 & oben dar ûf \textbf{gehôrte},\\ 
 & \textbf{mitten dran} ein knopfelîn,\\ 
 & ein durchliuhtic rubîn.\\ 
15 & d\textit{â} \textbf{saz} manic ritter kluoc,\\ 
 & d\textit{â} man jâmer \textit{vür} si truoc.\\ 
 & ein knappe spranc zer tür \textbf{her} în,\\ 
 & der truoc ein glevîn.\\ 
 & \textbf{der} site was ze trûren guot:\\ 
20 & an der snîden \textbf{huop} sich bluot\\ 
 & und lief de\textit{n} schaft unz \textbf{an} die hant,\\ 
 & \textbf{dârz} \textbf{in} dem ermel \textbf{widerwant}.\\ 
 & dâ wart gew\textit{e}inet und geschrît\\ 
 & \textbf{ûf} dem palase wî\textit{t}.\\ 
25 & daz \textbf{volc} von drîzic landen\\ 
 & \textbf{m\textit{ö}ht ez} \textbf{den ougen niht} enblanden.\\ 
 & er truoc si in sînen henden\\ 
 & \textbf{al umbe zen} vier wenden\\ 
 & \textbf{unz} aber wider zuo der tür.\\ 
30 & der knappe spranc hin ûz \textbf{dar vür}.\\ 
\end{tabular}
\scriptsize
\line(1,0){75} \newline
m n o Fr69 \newline
\line(1,0){75} \newline
\textbf{17} \textit{Initiale} Fr69  \newline
\line(1,0){75} \newline
\textbf{1} siecheit] sicherheit o \textbf{4} muose] muͯsse m muͯste n (o) \textbf{8} des] Des selben n (o)  $\cdot$ dâ] do m n \textbf{10} tiure] dort o \textbf{11} arabisch] arabasc o \textbf{12} gehôrte] gehort n \textbf{15} dâ] Do m o So n \textbf{16} dâ] Do m n  $\cdot$ vür] \textit{om.} m \textbf{17} tür her în] torren in o \textbf{21} den] dem m o \textbf{22} dârz] Das n o \textbf{23} dâ] Do o  $\cdot$ geweinet] gewinet m \textbf{24} wît] wip m \textbf{26} möht] Moht m (n) o \textbf{29} wider] \textit{om.} o \newline
\end{minipage}
\end{table}
\newpage
\begin{table}[ht]
\begin{minipage}[t]{0.5\linewidth}
\small
\begin{center}*G
\end{center}
\begin{tabular}{rl}
 & \textbf{der wirt} hete durch siecheit\\ 
 & \textbf{grôziu} viur unde warmiu kleit.\\ 
 & wît, lanc zobelîn,\\ 
 & \textbf{sus} muose \textbf{ûzen} unde innen sîn\\ 
5 & \textbf{ein} belz unde \textbf{ein} mandel drobe.\\ 
 & der swecheste balc \textbf{was} wol ze lobe,\\ 
 & der was doch swarz unde grâ.\\ 
 & des \textbf{selben} was ein hûbe dâ\\ 
 & ûf sînem houbte, zwivalt\\ 
10 & von zobele, den man tiure galt.\\ 
 & sinewel \textbf{arabesch ein} borte\\ 
 & oben drûf \textbf{gehôrte};\\ 
 & \textbf{dâr an was} ein k\textit{n}opfelîn,\\ 
 & ein durchliuhtic rubîn.\\ 
15 & dâ \textbf{saz} manic rîter kluoc,\\ 
 & dâ man jâmer vür si truoc.\\ 
 & ein knappe spranc zer tür \textbf{her} în,\\ 
 & der truoc eine glevîn.\\ 
 & \textbf{der} site was ze trûren guot:\\ 
20 & an der snîden \textbf{huop} sich bluot\\ 
 & unde lief den schaft unze \textbf{an} die hant,\\ 
 & \textbf{da\textit{z}} \textbf{\textit{a}n} dem ermel \textbf{widerwant}.\\ 
 & dâ wart geweinet unde geschrît\\ 
 & \textbf{in} dem palase wît.\\ 
25 & daz \textbf{volc} von drîzic landen\\ 
 & \textbf{m\textit{ö}htez} \textbf{den ougen niht} enblanden.\\ 
 & \begin{large}E\end{large}r truoc si in sînen henden\\ 
 & \textbf{Z\textit{e} allen} vier wenden\\ 
 & \textbf{unze} aber wider \textbf{hin} zer tür.\\ 
30 & der knappe spranc hin ûz \textbf{dar vür}.\\ 
\end{tabular}
\scriptsize
\line(1,0){75} \newline
G I O L M Q R Z Fr21 \newline
\line(1,0){75} \newline
\textbf{1} \textit{Initiale} O L Q Z Fr21  \textbf{15} \textit{Initiale} I  \textbf{17} \textit{Initiale} R  \textbf{27} \textit{Initiale} G  \newline
\line(1,0){75} \newline
\textbf{1} der wirt] er I ÷er wirt O  $\cdot$ hete] het an im I sprach O der het R  $\cdot$ siecheit] sicherheit I (O) R \textbf{2} grôziu] Grosz M Q (Z) (Fr21) Gozú R  $\cdot$ unde] an ym M vnd an im Q (Z) (Fr21) vnd an In R  $\cdot$ warmiu] warm M  $\cdot$ kleit] [leit]: chleit I \textbf{3} wît] Wit vnde O (L) (M) (Q) (R) (Z) (Fr21) \textbf{4} muose] mústens Q  $\cdot$ ûzen] vͤz I \textbf{5} ein belz] ein belliz wit I Einen peltz Q  $\cdot$ ein mandel] der mantel L (Z) einen mantel Q \textbf{6} swecheste] \textit{om.} M  $\cdot$ ze lobe] zobele L \textbf{8} selben] selbe M  $\cdot$ dâ] do Q \textbf{9} zwivalt] zwifal Fr21 \textbf{10} von] Vom R \textbf{11} arabesch] arabensch G arabisk I arabisch O (L) M Q Z arebesch R arabẏsch Fr21  $\cdot$ borte] pforte Q porte R Fr21 \textbf{12} oben] Obnan R \textbf{13} dâr an] En mitten dran L Mitten tran Q (R) (Z)  $\cdot$ was] \textit{om.} R Z  $\cdot$ knopfelîn] chophelin G knevflin Z \textbf{14} durchliuhtic] durcluhter I (R) dvrch lvtich O (Fr21) \textbf{15} dâ] Dasz Q \textbf{16} dâ] den I Do Q R Fr21 \textbf{17} her] dar O L (Q) Fr21 \textit{om.} R \textbf{18} glevîn] glesin R \textbf{20} snîden] snide I (O) (L) (Q) (Fr21) schinden R  $\cdot$ bluot] daz bluͦt R \textbf{21} unde lief] Vn lisz Q  $\cdot$ den] dem R  $\cdot$ unze] bisz Q  $\cdot$ an] vf O (M) (Q) Z Fr21 \textbf{22} daz an] daze im an G Daz ez an L (M) (R) Z  $\cdot$ dem ermel] den ermeln M \textbf{23} dâ] Do Q R \textbf{24} in] Vf Z \textbf{25} \textit{Versdoppelung 231.25, 231.27-232.2 und 230.21 (²O) nach 230.21; Lesarten der vorausgehenden Verse mit ¹O bezeichnet} O  \textbf{26} möhtez] mohtez G (I) (L) (M) (Q) (R) (Z) Moht O \textbf{27} si] \textit{om.} O ez L (Q) (Z) \textbf{28} Ze] zen G  $\cdot$ wenden] enden Z \textbf{29} unze] Vnz er M  $\cdot$ hin zer] andie I fvr di \textsuperscript{2}\hspace{-1.3mm} O zuͯ der L \textbf{30} hin ûz] hinzuͤ I \newline
\end{minipage}
\hspace{0.5cm}
\begin{minipage}[t]{0.5\linewidth}
\small
\begin{center}*T
\end{center}
\begin{tabular}{rl}
 & \textbf{Der wirt} hete durch siecheit\\ 
 & \textbf{grôz} viur unde \textbf{an im} warmiu kleit.\\ 
 & wît \textbf{unde} lanc zobelîn,\\ 
 & \textbf{als} muose \textbf{ûz} unde inne sîn\\ 
5 & \textbf{der} bellez unde \textbf{der} mantel drobe.\\ 
 & der swechste balc \textbf{wære} wol ze lobe,\\ 
 & der was doch swarz unde grâ.\\ 
 & des \textbf{selben} was ein hûbe dâ\\ 
 & ûf sînem houbete, zwivalt\\ 
10 & von zobele, den man tiure galt.\\ 
 & sinewel \textbf{ein arabesch} borte\\ 
 & oben drûf \textbf{hôrte},\\ 
 & \textbf{enmitten dran} ein knopfelîn,\\ 
 & ein durchliuhtic rubîn.\\ 
15 & dâ \textbf{was} manec rîter kluoc,\\ 
 & dâ man jâmer vür si truoc.\\ 
 & Ein knappe spranc zer tür \textbf{dar} în,\\ 
 & der truoc eine glevîn.\\ 
 & \textbf{des} site was ze trûrene guot:\\ 
20 & \textbf{oben} an der snîden \textbf{erhuop} sich bluot\\ 
 & unde lief den schaft \textbf{nider} unz \textbf{ûf} die hant,\\ 
 & \textbf{daz} \textbf{si}\textbf{n} dem ermele \textbf{erwant}.\\ 
 & Dô wart geweinet unde geschrît\\ 
 & \textbf{in} dem palase wît,\\ 
25 & daz von drîzic landen\\ 
 & \textbf{m\textit{ö}htens} \textbf{niht den ougen} enblanden.\\ 
 & er truoc \textit{si} in sînen henden\\ 
 & \textbf{alumbe ze} vier wenden\\ 
 & \textbf{unde} aber wider zuo der tür.\\ 
30 & der knappe spranc hin ûz \textbf{zer tür}.\\ 
\end{tabular}
\scriptsize
\line(1,0){75} \newline
T U V W \newline
\line(1,0){75} \newline
\textbf{1} \textit{Initiale} W   $\cdot$ \textit{Majuskel} T  \textbf{17} \textit{Majuskel} T  \textbf{23} \textit{Majuskel} T  \newline
\line(1,0){75} \newline
\textbf{1} [*]: vnde hette durch sine siecheit V \textbf{2} unde an im] an im vnd W  $\cdot$ warmiu] warm U W \textbf{4} als] Alsus U (V) W  $\cdot$ muose] mvese T \textbf{5} der bellez] [*]: Einen beltz V  $\cdot$ unde der] vnd U vnde [*]: ein V ein W \textbf{6} wære] [w*]: waz V \textbf{11} ein arabesch] von atabie ein W \textbf{12} oben] Obenan V W \textbf{13} enmitten] Mitten U  $\cdot$ dran] drein W \textbf{14} rubîn] robin U W \textbf{15} dâ] Do U V W  $\cdot$ was] [*]: saz V \textbf{16} dâ] Do U W  $\cdot$ jâmer] gros iamer W  $\cdot$ si] \textit{om.} U in W \textbf{17} dar] \textit{om.} U hin W \textbf{18} eine] ein lange W \textbf{19} Von der sy gewunnen sweren muͦt W  $\cdot$ des] Der V \textbf{20} oben] Ob ein U Obenan W  $\cdot$ erhuop] huͦb W \textbf{21} unde lief den schaft] Den schaft lief es W  $\cdot$ nider unz ûf] bit of U vnz [*]: an V vntz an W \textbf{22} [Da*]: Daz es an dem ermel wider want V  $\cdot$ daz sin] Das es in W  $\cdot$ erwant] wider want U (W) \textbf{25} daz] Daz [*]: volk V \textbf{26} möhtens] mohtens T Mocht U (V)  $\cdot$ niht] iz U (V) ir W  $\cdot$ den] \textit{om.} W  $\cdot$ enblanden] nit enblanden U (V) do erblanden W \textbf{27} si in] in T U [*]: sv́ in V  $\cdot$ sînen] baiden W \textbf{28} \textit{nach 231.28:} Ieglichem ritter er sy bot / Sy wanden alle dise not W   $\cdot$ ze] zen V \textbf{29} \textit{Versfolge 231.30-29} W   $\cdot$ unde aber] \textit{om.} W  $\cdot$ der] der selben W \textbf{30} hin ûz zer tür] [*]: hin vz der fúr V do hin fúr W \newline
\end{minipage}
\end{table}
\end{document}
