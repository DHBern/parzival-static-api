\documentclass[8pt,a4paper,notitlepage]{article}
\usepackage{fullpage}
\usepackage{ulem}
\usepackage{xltxtra}
\usepackage{datetime}
\renewcommand{\dateseparator}{.}
\dmyyyydate
\usepackage{fancyhdr}
\usepackage{ifthen}
\pagestyle{fancy}
\fancyhf{}
\renewcommand{\headrulewidth}{0pt}
\fancyfoot[L]{\ifthenelse{\value{page}=1}{\today, \currenttime{} Uhr}{}}
\begin{document}
\begin{table}[ht]
\begin{minipage}[t]{0.5\linewidth}
\small
\begin{center}*D
\end{center}
\begin{tabular}{rl}
\textbf{232} & \textit{\begin{large}G\end{large}}estillet \textbf{was} des volkes nôt,\\ 
 & als in \textbf{der} jâmer ê gebôt,\\ 
 & des si diu glevîn het ermant,\\ 
 & die \textbf{der knappe brâhte} \textbf{in} sîner hant.\\ 
5 & Wil iuch nû niht \textbf{erlangen},\\ 
 & sô wirt hie \textbf{zuo gevangen},\\ 
 & daz ich iuch bringe an die vart,\\ 
 & wie dâ mit zuht gedienet wart.\\ 
 & zende an dem palas\\ 
10 & ein stehelîn tür entslozzen was;\\ 
 & dâ giengen ûz zwei werdiu kint.\\ 
 & nû hœret, wie diu geprüevet sint,\\ 
 & daz si \textbf{wol} gæben minnen solt,\\ 
 & swerz dâ mit dienste het erholt.\\ 
15 & \textbf{daz} wâren juncvrouwen clâr.\\ 
 & zwei schapel über \textbf{blôziu} hâr,\\ 
 & \textbf{bluomen} was ir gebende.\\ 
 & iewederiu ûf \textbf{der} hende\\ 
 & truoc von golde ein kerzestal.\\ 
20 & ir hâr was \textbf{reit, lanc} unt val.\\ 
 & Si truogen \textbf{brinnendigiu} lieht.\\ 
 & \textbf{hie} sule wir vergezzen nieht\\ 
 & umbe der juncvrouwen gewant,\\ 
 & dâ man si kumende inne vant:\\ 
25 & \textbf{De\textit{r}} grævinne von Tenabroc\\ 
 & \textbf{brûn} scharlachen was ir roc.\\ 
 & des selben truog ouch ir gespil.\\ 
 & si \textbf{wâren} gefischieret vil\\ 
 & mit zwein gürteln an der krenke,\\ 
30 & Ob der hüffe ame gelenke.\\ 
\end{tabular}
\scriptsize
\line(1,0){75} \newline
D \newline
\line(1,0){75} \newline
\textbf{1} \textit{Initiale} D  \textbf{5} \textit{Majuskel} D  \textbf{21} \textit{Majuskel} D  \textbf{25} \textit{Majuskel} D  \textbf{30} \textit{Majuskel} D  \newline
\line(1,0){75} \newline
\textbf{1} Gestillet] Destillet D \textbf{25} Der] De D  $\cdot$ Tenabroc] Tenabroch D \newline
\end{minipage}
\hspace{0.5cm}
\begin{minipage}[t]{0.5\linewidth}
\small
\begin{center}*m
\end{center}
\begin{tabular}{rl}
 & gest\textit{i}llet \textbf{was} des volkes nôt,\\ 
 & als in \textbf{des} jâmer ê gebôt,\\ 
 & des si diu glevîn h\textit{e}t ermant,\\ 
 & die \textbf{brâhte der \textit{k}nappe} \textbf{in} sîner hant.\\ 
5 & \begin{large}W\end{large}il iuch nû niht \textbf{belangen},\\ 
 & sô wirt hie \textbf{zu\textit{o} \textit{g}evangen},\\ 
 & daz ich iuch bringe an die vart,\\ 
 & wie dâ mit zuht gedienet wart.\\ 
 & ze ende an dem palas\\ 
10 & ein stehelîn tür entslozzen was;\\ 
 & d\textit{â} giengen ûz zwei werdiu kint.\\ 
 & nû hœret, wie diu gebrüefet sint,\\ 
 & daz si gæben minne solt,\\ 
 & \dag waz\dag  dâ mit dienste hete erholt.\\ 
15 & \textbf{daz} wâren juncvrouwen clâr.\\ 
 & zwei schapel über \textbf{blôz} hâr,\\ 
 & \textbf{blüemîn} was ir gebende.\\ 
 & ietwederiu ûf \textbf{eine\textit{r}} hende\\ 
 & truoc von golde ein \textit{k}erzestal.\\ 
20 & ir hâr was \textbf{rôt, reit} und val.\\ 
 & si truogen \textbf{brinnend\textit{iu}} lieht.\\ 
 & \textbf{hie} sullen wir vergezzen niht\\ 
 & umb der juncvrouwen gewant,\\ 
 & d\textit{â} man si komende inne vant:\\ 
25 & \textbf{diu} grævîn von Ten\textit{e}broc,\\ 
 & \textbf{mit} scharlachen was ir roc.\\ 
 & des selben truoc ouch ir gespil.\\ 
 & si \textbf{gewâren} gefischieret vil\\ 
 & mit zwein gürtelen an der krenke,\\ 
30 & ob der huf ame gelenke.\\ 
\end{tabular}
\scriptsize
\line(1,0){75} \newline
m n o Fr69 \newline
\line(1,0){75} \newline
\textbf{5} \textit{Initiale} m   $\cdot$ \textit{Capitulumzeichen} n  \newline
\line(1,0){75} \newline
\textbf{1} gestillet] Gestiellet m \textbf{2} in] jme n (o)  $\cdot$ des] der o \textbf{3} het] hant m \textbf{4} knappe] kanappe m \textbf{5} nû] ẏm o \textbf{6} zuo gevangen] zuͯ vͯch gefangen m \textbf{8} dâ] do n o  $\cdot$ zuht] zúchten n \textbf{10} stehelîn tür entslozzen] stegelin kar erslossen o \textbf{11} dâ] Do m n o \textbf{12} nû] Muͯ o  $\cdot$ diu] die so n \textbf{13} si] sie wol o \textbf{14} dâ] do n o \textbf{17} blüemîn] Blúmen n \textbf{18} einer] einen m \textbf{19} truoc] Vnd truͦg n  $\cdot$ kerzestal] stercze stal m \textbf{20} rôt] \textit{om.} Fr69  $\cdot$ und] eyn o \textbf{21} brinnendiu] brinenden m \textbf{23} umb] Vnd o \textbf{24} dâ] Do m n \textbf{25} Tenebroc] thenabrog m thenebrog n themabarg o \textbf{27} des] Das o \textbf{28} gewâren gefischieret] geworent gefigieret m worent gefiguriert n werent gefugiere o \textbf{29} krenke] krencken o \textbf{30} gelenke] glencken o \newline
\end{minipage}
\end{table}
\newpage
\begin{table}[ht]
\begin{minipage}[t]{0.5\linewidth}
\small
\begin{center}*G
\end{center}
\begin{tabular}{rl}
 & gestillet \textbf{wart} des volkes nôt,\\ 
 & als in \textbf{der} jâmer ê gebôt,\\ 
 & des si diu glavîn hete ermant,\\ 
 & die \textbf{der knappe truoc} \textbf{in} \textit{sîn}er hant.\\ 
5 & wil iuch nû niht \textbf{erlangen},\\ 
 & sô wirt hie \textbf{angevangen},\\ 
 & daz ich iuch bringe an die vart,\\ 
 & wie dâ mit zuht gedienet wart.\\ 
 & zende an dem palas\\ 
10 & ein stehelîn tür entslozzen was;\\ 
 & dâ giengen ûz zwei werdiu kint.\\ 
 & nû hœrt, wie diu gebrüevet sint,\\ 
 & daz si \textbf{wol} gæben minnen solt,\\ 
 & swerz dâ mit dienste het erholt.\\ 
15 & \textbf{daz} wâren juncvrouwen clâr.\\ 
 & zwei schapel über \textbf{blôzez} hâr,\\ 
 & \textbf{blüemîn} was ir gebende.\\ 
 & ietwederiu ûf \textbf{ir} hende\\ 
 & truoc von golde ein kerzestal.\\ 
20 & ir hâr was \textbf{lanc, reit} unde val.\\ 
 & si truogen \textbf{brinnendiu} lieht.\\ 
 & \textbf{hie} sulen wir vergezzen niht\\ 
 & umbe der juncvrouwen gewant,\\ 
 & dâ man si komende inne vant:\\ 
25 & \textbf{diu} grævîn von Tenebroc,\\ 
 & \textbf{brûn} scharlach was ir roc.\\ 
 & des selben truog ouch ir gespil.\\ 
 & si \textbf{wâren} gefischiert vil\\ 
 & mit zwein gürtelen an der krenke,\\ 
30 & obe der huf an dem gelenke.\\ 
\end{tabular}
\scriptsize
\line(1,0){75} \newline
G I O L M Q R Z Fr21 Fr51 \newline
\line(1,0){75} \newline
\textbf{1} \textit{Initiale} M Z Fr21  \textbf{5} \textit{Initiale} I L  \textbf{9} \textit{Initiale} O R  \textbf{21} \textit{Initiale} I  \newline
\line(1,0){75} \newline
\textbf{1} \textit{Versdoppelung 231.25, 231.27-232.2 und 230.21 (²O) nach 230.21; Lesarten der vorausgehenden Verse mit ¹O bezeichnet} O  \textbf{2} in der] in O L (M) im Fr21  $\cdot$ jâmer ê] yamor ye M \textbf{3} des] wes I Als Q  $\cdot$ glavîn] [glæver]: glævei O \textbf{4} der knappe truoc] Truͤc der chnappe I trvͦch ein chnappe O (L) (M) (R) (Z) (Fr21) der knabe bracht Q  $\cdot$ sîner] der G \textbf{5} erlangen] belangen O (L) \textbf{6} hie] iv hie O (Q) Fr21  $\cdot$ angevangen] zv gevangen Z \textbf{8} wie] Die R  $\cdot$ dâ] do Q \textbf{9} zende] ÷ende O Zen Fr21  $\cdot$ an dem] anden M \textbf{10} tür] tor M  $\cdot$ entslozzen] entschosszen Q enstlozen Fr21 \textbf{11} dâ] Do Q \textbf{12} diu] des Z \textbf{13} wol gæben minnen] uͯch geben wol der mynne L wol geben meinen Q \textbf{14} swerz] Wer ez L (Q) (R)  $\cdot$ dâ] \textit{om.} I do Q  $\cdot$ het] her O  $\cdot$ erholt] verholt O \textbf{15} daz] Ez L  $\cdot$ juncvrouwen] zwuͤ iuncfrowen I \textbf{16} schapel] sappel I  $\cdot$ blôzez] ir blozze Z ir blozez Fr21 \textbf{17} blüemîn] bluͦmen I (O) (L) (M) (Q) (Z) (Fr21)  $\cdot$ was] warn I \textbf{18} ietwederiu] Jewerdriv O \textbf{19} truoc] Trvͦgen O Fr21  $\cdot$ von golde ein] gvldin O vf ir hende ein L ein guldin Q gvldiniv Fr21  $\cdot$ kerzestal] kerstal M \textbf{20} lanc reit] reide I reit lanch O (Fr21) lanc breit M \textbf{21} lieht] lýcht L (M) (Q) \textbf{22} hie] nu I Ouch L  $\cdot$ sulen] ensvln O  $\cdot$ vergezzen] ergessen Q \textbf{24} dâ] Do Q  $\cdot$ komende] chomen O \textbf{25} Tenebroc] Tenebroch G O Fr21 (Fr51) Tenbroc I Tenebrock L tenebrok Q \textbf{26} brûn] Grvͦn Fr51 \textbf{28} si] Jn O Fr21 Die M  $\cdot$ gefischiert] gefuschirt Q ge feẏsieret Fr51 \textbf{29} mit] \textit{om.} I \textbf{30} obe] Vff M  $\cdot$ huf] gvrtel O gvrteln Fr21 \newline
\end{minipage}
\hspace{0.5cm}
\begin{minipage}[t]{0.5\linewidth}
\small
\begin{center}*T
\end{center}
\begin{tabular}{rl}
 & gestillet \textbf{was} des volkes nôt,\\ 
 & als in \textbf{der} jâmer ê gebôt,\\ 
 & des si di\textit{u} gleve hete ermant,\\ 
 & die \textbf{der knappe truoc} \textbf{an} sîner hant.\\ 
5 & \begin{large}W\end{large}il iuch nû niht \textbf{erlangen},\\ 
 & sô wirt hie \textbf{zuo gegangen},\\ 
 & daz ich iuch bringe an die vart,\\ 
 & wie dâ mit zuht gedient wart.\\ 
 & Zende an dem palas\\ 
10 & ein stehelîn tür \textit{e}ntslozzen was;\\ 
 & dâ giengen ûz zwei werdiu kint.\\ 
 & nû hœret, wie die geprüevet sint,\\ 
 & daz si \textbf{wol} gæben minnen solt,\\ 
 & swerz dâ mit dienste het erholt.\\ 
15 & \textbf{ez} wâren juncvrouwen clâr.\\ 
 & Zwei schapel über \textbf{blôzez} hâr,\\ 
 & \textbf{blüemîn} was ir gebende.\\ 
 & ietwederiu ûf \textbf{ir} hende\\ 
 & truoc von golde ein kerzestal.\\ 
20 & ir hâr was \textbf{reit} unde val.\\ 
 & si truogen \textbf{brennend\textit{iu}} lieht.\\ 
 & \textbf{ouch} sul wir vergezzen nieht\\ 
 & umbe der juncvrouwen gewant,\\ 
 & dâ man si komende inne vant:\\ 
25 & \textbf{Diu} grævîn von Tenebroc,\\ 
 & \textbf{brûn} scharlachen was ir roc.\\ 
 & des selben truoc ouch ir gespil.\\ 
 & si \textbf{wâren} gefischieret vil\\ 
 & mit zwein gürteln an der krenke,\\ 
30 & ob der huf an dem gelenke.\\ 
\end{tabular}
\scriptsize
\line(1,0){75} \newline
T U V W \newline
\line(1,0){75} \newline
\textbf{5} \textit{Initiale} T U W  \textbf{9} \textit{Majuskel} T  \textbf{16} \textit{Majuskel} T  \textbf{25} \textit{Majuskel} T  \newline
\line(1,0){75} \newline
\textbf{1} Do ward gestillet ir not W \textbf{3} \textit{Die Verse 232.3-4 fehlen} W   $\cdot$ des] Daz U  $\cdot$ diu] die T \textbf{4} der knappe truoc] druͦc der knappe U (V) \textbf{5} iuch nû] îv nv T vch U (V) es eúch W  $\cdot$ erlangen] [*langen]: belangen V belangen W \textbf{6} gegangen] gevangen U V (W) \textbf{7} iuch] îv T  $\cdot$ bringe] bringen U [brin*]: bringe V \textbf{8} dâ] do V W  $\cdot$ zuht] zúchten W \textbf{9} an] in W \textbf{10} entslozzen] eintslozzen T \textbf{11} dâ] Do U V W \textbf{13} gæben] gebent V gaben W \textbf{14} swerz] Wer iz U (W)  $\cdot$ dâ] do U V W  $\cdot$ het] hat W \textbf{15} ez] Daz V \textbf{17} blüemîn] Bluͦmen U W \textbf{20} reit] reit lanc U (V) lang reit W \textbf{21} brennendiu] brennende T bornende U brinnendige V \textbf{24} dâ] Do V W \textbf{25} Tenebroc] tenebrog V tenebrock W \textbf{26} scharlachen] scharlat V scharlachin W \textbf{28} gefischieret] [gefisceret]: gefiscieret T gefifieret U geveitieret V gefisieret W \textbf{29} gürteln] gúrtelin W \newline
\end{minipage}
\end{table}
\end{document}
