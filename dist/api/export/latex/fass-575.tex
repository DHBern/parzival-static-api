\documentclass[8pt,a4paper,notitlepage]{article}
\usepackage{fullpage}
\usepackage{ulem}
\usepackage{xltxtra}
\usepackage{datetime}
\renewcommand{\dateseparator}{.}
\dmyyyydate
\usepackage{fancyhdr}
\usepackage{ifthen}
\pagestyle{fancy}
\fancyhf{}
\renewcommand{\headrulewidth}{0pt}
\fancyfoot[L]{\ifthenelse{\value{page}=1}{\today, \currenttime{} Uhr}{}}
\begin{document}
\begin{table}[ht]
\begin{minipage}[t]{0.5\linewidth}
\small
\begin{center}*D
\end{center}
\begin{tabular}{rl}
\textbf{575} & \begin{large}S\end{large}i sande zwô juncvrouwen dar\\ 
 & \textbf{und bat si nemen rehte} war,\\ 
 & daz si sanfte slichen,\\ 
 & ê daz si dan entwichen,\\ 
5 & \textbf{daz} si ir bræhten mære,\\ 
 & ob er bî lebene wære\\ 
 & oder ob er wære verscheiden.\\ 
 & daz gebôt si den beiden.\\ 
 & Die süezen meide reine,\\ 
10 & ob ir \textbf{dewederiu} weine?\\ 
 & jâ, si beide sêre\\ 
 & durch rehtes jâmers lêre,\\ 
 & dô si in sus ligen vunden,\\ 
 & daz von sînen wunden\\ 
15 & der schilt mit bluote swebete.\\ 
 & si besâhen, ob er lebete.\\ 
 & Einiu mit ir clâren hant\\ 
 & den helm von sîme houbte bant\\ 
 & und ouch die vintâlen sîn.\\ 
20 & dâ lac ein \textbf{vil} kleinez schiumelîn\\ 
 & vor sîme rôten munde.\\ 
 & \textbf{ze} warten si begunde,\\ 
 & ob er den âtem inder züge\\ 
 & oder ober \textbf{si} \textbf{des} lebens trüge;\\ 
25 & daz lac dannoch in strîte.\\ 
 & ûf sîme kursîte\\ 
 & von zobele wâren zwei gampilûn,\\ 
 & als Ilynot der Bertun\\ 
 & mit grôzem prîse wâpen truoc;\\ 
30 & \textbf{der} brâhte werdecheit genuoc\\ 
\end{tabular}
\scriptsize
\line(1,0){75} \newline
D Fr7 \newline
\line(1,0){75} \newline
\textbf{1} \textit{Initiale} D Fr7  \textbf{9} \textit{Majuskel} D  \textbf{17} \textit{Majuskel} D  \newline
\line(1,0){75} \newline
\textbf{2} si nemen rehte] des rehte nemen Fr7 \textbf{3} slichen] slîchen D \textbf{4} daz si dan] si von dannen Fr7 \textbf{18} \textit{Vers 575.18 fehlt} Fr7  \textbf{19} und] Den helm vnd Fr7  $\cdot$ vintâlen] fantalen Fr7 \textbf{20} vil] \textit{om.} Fr7 \textbf{24} lebens] leben Fr7 \textbf{28} Ilynot] Jlynôt D ilinot Fr7 \textbf{30} brâhte] braht ovch Fr7 \newline
\end{minipage}
\hspace{0.5cm}
\begin{minipage}[t]{0.5\linewidth}
\small
\begin{center}*m
\end{center}
\begin{tabular}{rl}
 & si sante \textbf{dô} zwô juncvrowen dar\\ 
 & \textbf{und bat si reht \textit{n}emen} war,\\ 
 & daz si sanfte slichen,\\ 
 & ê daz si dan entwichen,\\ 
5 & \textbf{und} si ir bræhten mære,\\ 
 & ob er bî leben wære\\ 
 & oder ob er wær verscheiden.\\ 
 & daz gebôt si den beiden.\\ 
 & die süezen megde \textbf{beide} \textit{r}eine,\\ 
10 & ob ir \textbf{deweders} \textbf{iht} weine,\\ 
 & \multicolumn{1}{l}{ - - - }\\ 
 & \multicolumn{1}{l}{ - - - }\\ 
 & dô si in sus ligen vunden,\\ 
 & daz von sînen wunden\\ 
15 & der schilt mit bluote swebte?\\ 
 & si besâhen, ob er lebte.\\ 
 & \textbf{ir} einiu mit ir clâren hant\\ 
 & den hel\textit{m} \textit{v}on sînem houbte bant\\ 
 & und ouch die fantailen sîn.\\ 
20 & d\textit{â} lac ein kleinez schi\textit{u}melîn\\ 
 & vor sînem rôtem munde.\\ 
 & \textbf{nû} warten si begunde,\\ 
 & ob er den âtem iendert züge\\ 
 & oder ob er \textbf{si} \textbf{des} lebens trüge;\\ 
25 & daz lac dannoch i\textit{n} strîte.\\ 
 & ûf sînem kursîte\\ 
 & von zobel\textit{e} wâren zwe\textit{i} g\textit{a}mp\textit{i}l\textit{û}n,\\ 
 & als Il\textit{in}ot der Britun\\ 
 & mit grôzem prîse wâpen truoc;\\ 
30 & \textbf{der} brâhte werdicheit genuoc\\ 
\end{tabular}
\scriptsize
\line(1,0){75} \newline
m n o \newline
\line(1,0){75} \newline
\newline
\line(1,0){75} \newline
\textbf{1} dô] \textit{om.} n o \textbf{2} nemen] nẏemen m [nemem]: nemene n \textbf{4} dan] denne n \textbf{6} bî] bẏ dem n \textbf{9} süezen] suͯsse o  $\cdot$ reine] meine m n [*]: reine o \textbf{10} ir deweders] ir [*]: do weders n irs :::wider: o \textbf{11} \textit{Die Verse 575.11-12 fehlen} m n o  \textbf{18} helm von] helm mit von m \textbf{19} fantailen] fantalen n fantailer o \textbf{20} dâ] Do m n o  $\cdot$ schiumelîn] schimelin m (o) \textbf{21} vor] Do vor n  $\cdot$ rôtem] roten n o \textbf{22} begunde] beguͯnde m \textbf{25} in] ir m n o \textbf{27} zobele] zabelin m  $\cdot$ zwei] zwein m o  $\cdot$ gampilûn] gimppelin m gunpelun n gampeluͯn o \textbf{28} Ilinot] jlmot m ilmot n o  $\cdot$ Britun] brittuͯm m brituͦn n britom o \newline
\end{minipage}
\end{table}
\newpage
\begin{table}[ht]
\begin{minipage}[t]{0.5\linewidth}
\small
\begin{center}*G
\end{center}
\begin{tabular}{rl}
 & \begin{large}S\end{large}i sande zwô juncvrouwen dar,\\ 
 & \textbf{daz si rehte n\textit{e}men} war,\\ 
 & daz si sanfte sl\textit{i}chen,\\ 
 & ê daz si dan entwichen,\\ 
5 & \textbf{daz} si ir bræhten mære,\\ 
 & ob er bî leben wære\\ 
 & ode obe er wære verscheiden.\\ 
 & daz gebôt si den beiden.\\ 
 & die süezen meide reine,\\ 
10 & ob ir \textbf{dewederiu} weine?\\ 
 & jâ, si beide sêre\\ 
 & durch rehtes jâmers lêre,\\ 
 & dô sin sus ligen vunden,\\ 
 & daz von sînen wunden\\ 
15 & der schilt mit bluote swebete.\\ 
 & si besâhen, ob er lebete.\\ 
 & einiu mit ir clâren hant\\ 
 & den helm von sînem houbete bant\\ 
 & unde ouch die vintâlen sîn.\\ 
20 & dâ lac ein kleinez schiumelîn\\ 
 & vor sînem rôtem munde.\\ 
 & \textbf{ze} wartene si begunde,\\ 
 & obe er den âtem iender züge\\ 
 & od ob er \textbf{si} lebens trüge;\\ 
25 & daz lac da\textit{n}noch in strîte.\\ 
 & ûf sînem kursîte\\ 
 & von zobele wâren zwei gabilûn,\\ 
 & als Ilinot der Britun\\ 
 & mit grôzem brîse wâpen truoc;\\ 
30 & \textbf{er} brâhte werdecheit genuoc\\ 
\end{tabular}
\scriptsize
\line(1,0){75} \newline
G I L M Z Fr23 \newline
\line(1,0){75} \newline
\textbf{1} \textit{Initiale} G I L Z Fr23  \textbf{21} \textit{Initiale} I  \newline
\line(1,0){75} \newline
\textbf{1} sande] fand Fr23 \textbf{2} daz] Vnd bat L (M) (Z) Fr23  $\cdot$ nemen] namen G \textbf{3} slichen] slischen G \textbf{4} Vnd sy des berichten L  $\cdot$ daz] \textit{om.} I \textbf{5} daz] E daz L  $\cdot$ ir] \textit{om.} L Z  $\cdot$ bræhten] brehte Z \textbf{6} er bî leben] ir blibin M er bi dem leben Fr23 \textbf{7} \textit{Versfolge 575.8-7} M  \textbf{9} süezen] suͯsze L  $\cdot$ reine] beide rein I \textbf{10} dewederiu] icslichir M entwedere Z ietweder Fr23 \textbf{12} rehtes] rechte L  $\cdot$ lêre] ere I L \textbf{13} dô] Daz L Da M Z  $\cdot$ ligen] ligende L \textbf{14} daz] Da L Z \textbf{15} mit] von I Z \textbf{19} die vintâlen] sintalen M \textbf{20} dâ] Do L \textbf{21} rôtem] roten L (M) \textbf{23} iender] irgen M \textbf{24} Gelýptes bluͯte genvge L \textbf{25} dannoch] da noch G \textbf{27} gabilûn] kampilun M (Z) \textbf{28} Ilinot] ybilon G ilinot I ýlinot L Jlinot M Jbnot Z  $\cdot$ Britun] pritun I brittvn L Brituͯn M \textbf{30} er] Der L M Z \newline
\end{minipage}
\hspace{0.5cm}
\begin{minipage}[t]{0.5\linewidth}
\small
\begin{center}*T
\end{center}
\begin{tabular}{rl}
 & si sante zwô juncvrouwen dar\\ 
 & \textbf{und bat si rehte nemen} war,\\ 
 & daz si sanfte slichen,\\ 
 & ê daz si dan entwichen,\\ 
5 & \textbf{daz} sir bræhten mære,\\ 
 & ob er bî leben wære\\ 
 & oder ob er wære verscheiden.\\ 
 & daz gebôt si den beiden.\\ 
 & die süezen meide rein\textit{e},\\ 
10 & ob ir \textbf{dewederiu} weine?\\ 
 & jâ, si bêde sêre\\ 
 & durch rehtes jâmers lêre,\\ 
 & dô si in sus ligen vunden,\\ 
 & daz von sînen wunden\\ 
15 & der schilt mit bluot swebte.\\ 
 & si besâhen, ober lebte.\\ 
 & einiu mit ir clârer hant\\ 
 & den helm von sînem houbt bant\\ 
 & und \textit{ouch} die vinteilen sîn.\\ 
20 & d\textit{â} lac ein kleinez sch\textit{iu}melîn\\ 
 & vor sînem rôten munde.\\ 
 & \textbf{zuo} warten si begunde,\\ 
 & ob er den âtem indert züge\\ 
 & oder ob er \textbf{iht} lebens trüge;\\ 
25 & daz lac dannoch in strîte.\\ 
 & ûf sînem kursîte\\ 
 & von zobel wâren zwei gampilûn,\\ 
 & als Ylinot der Britun\\ 
 & mit grôzem prîse wâpen tr\textit{uo}c;\\ 
30 & \textbf{der} brâht werdecheit genuoc\\ 
\end{tabular}
\scriptsize
\line(1,0){75} \newline
Q R W V U \newline
\line(1,0){75} \newline
\textbf{1} \textit{Capitulumzeichen} R  \textbf{9} \textit{Initiale} W  \newline
\line(1,0){75} \newline
\textbf{1} \textit{Die Verse 553.1-599.30 fehlen} U  \textbf{3} slichen] [sliffen]: schligen Q \textbf{4} dan] dann Q \textbf{6} bî leben] by dem leben R (W) lebende V \textbf{7} ob er] aber V \textbf{9} reine] reynen Q \textbf{10} dewederiu] dewedre R \textbf{11} bêde] beidu R \textbf{12} rehtes] rechten W rehte V \textbf{13} ligen] ligende V \textbf{15} mit] von R W (V) \textbf{16} besâhen] versvͦchten V \textbf{17} einiu] Eine R [*ine]: Jr eine V  $\cdot$ clârer] claren R (W) V \textbf{18} sînem] dem R \textbf{19} und ouch] Vnd vff Q Vnd dar zuͦ W [*nde]: Vnde oͮch V \textbf{20} dâ lac] do lac Q (R) (V) Vnd W  $\cdot$ schiumelîn] schemelein Q \textbf{21} rôten] rote Q rotem W \textbf{22} warten] wattent R  $\cdot$ begunde] begunden R \textbf{23} indert] nendert R iergent V \textbf{24} iht lebens] sin leben R sy lebens W (V) \textbf{25} daz] Do V \textbf{27} gampilûn] [ganpl*]: ganpelvn V \textbf{28} Ylinot] Jl::ot R ylmot W  $\cdot$ Britun] prittum Q brittvn V \textbf{29} truoc] tranc Q \newline
\end{minipage}
\end{table}
\end{document}
