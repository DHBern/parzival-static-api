\documentclass[8pt,a4paper,notitlepage]{article}
\usepackage{fullpage}
\usepackage{ulem}
\usepackage{xltxtra}
\usepackage{datetime}
\renewcommand{\dateseparator}{.}
\dmyyyydate
\usepackage{fancyhdr}
\usepackage{ifthen}
\pagestyle{fancy}
\fancyhf{}
\renewcommand{\headrulewidth}{0pt}
\fancyfoot[L]{\ifthenelse{\value{page}=1}{\today, \currenttime{} Uhr}{}}
\begin{document}
\begin{table}[ht]
\begin{minipage}[t]{0.5\linewidth}
\small
\begin{center}*D
\end{center}
\begin{tabular}{rl}
\textbf{307} & spien si im vür sîn houbtloch.\\ 
 & Cunneware gab im mêr dennoch:\\ 
 & einen tiweren gürtel fier.\\ 
 & \textbf{mit edelen steinen} manec tier\\ 
5 & \textbf{muose} \textbf{ûzen} ûf \textbf{dem} borten sîn.\\ 
 & diu rinke was ein rubîn.\\ 
 & Wie was der junge âne bart\\ 
 & geschicket, dô er \textbf{gegürtet} wart?\\ 
 & \textbf{diz} mære giht: wol genuoc.\\ 
10 & daz volc im holdez herze truoc.\\ 
 & \textbf{swer in sach,} man \textbf{oder} wîp,\\ 
 & \textbf{die} heten \textbf{wert} sînen lîp.\\ 
 & \textbf{Der künec} messe het gehôrt.\\ 
 & \textbf{man sach Artusen} komen dort\\ 
15 & mit der tavelrunde diet,\\ 
 & der \textbf{neheiner} valscheit nie geriet.\\ 
 & die heten alle ê vernomen,\\ 
 & der rôte rîter wære komen\\ 
 & in Gawans poulûn.\\ 
20 & dar kom Artus der Bertun.\\ 
 & Der zerblûwen Anthanor\\ 
 & spranc dem künege allez vor,\\ 
 & unz er den Waleis ersach.\\ 
 & den vrâgt er: "sît ir\textbf{z}, der mich rach\\ 
25 & unt Cunnewaren de Lalant?\\ 
 & vil prîses giht man iwerer hant.\\ 
 & Keie hât verpfendet,\\ 
 & sîn dröuwen ist \textbf{nû} gelendet.\\ 
 & ich vürhte wênec sînen swanc;\\ 
30 & der zeswe arm ist im \textbf{ze} kranc."\\ 
\end{tabular}
\scriptsize
\line(1,0){75} \newline
D \newline
\line(1,0){75} \newline
\textbf{7} \textit{Majuskel} D  \textbf{13} \textit{Majuskel} D  \textbf{21} \textit{Majuskel} D  \newline
\line(1,0){75} \newline
\newline
\end{minipage}
\hspace{0.5cm}
\begin{minipage}[t]{0.5\linewidth}
\small
\begin{center}*m
\end{center}
\begin{tabular}{rl}
 & spien si im vür sîn houbetloch.\\ 
 & C\textit{un}n\textit{e}ware gap im mê dannoch:\\ 
 & einen tiuren gürtel fier.\\ 
 & \textbf{mit edelen steinen} manic tier\\ 
5 & \textbf{muose} \textbf{ûzen} ûf borten sîn.\\ 
 & diu rinke was ein rubîn.\\ 
 & wie was der junge âne bart\\ 
 & \dag er schicket\dag , dô er \textbf{gegürtet} wart?\\ 
 & \textbf{diz} mære giht: wol genuoc.\\ 
10 & daz volc ime holdez herze truoc.\\ 
 & \textbf{wer in gesach}, man \textbf{oder} wîp,\\ 
 & \textbf{die} heten \textbf{werden} sînen lîp.\\ 
 & \textbf{\begin{large}A\end{large}rtus} messe het gehôrt.\\ 
 & \textbf{den sach man schône} komen dort\\ 
15 & mit \textit{de}r tavelrunde diet,\\ 
 & der \textbf{enkeiner} valscheit nie geriet.\\ 
 & die heten alle ê vernomen,\\ 
 & der rôte ritter wære komen\\ 
 & in Gawanes pavelûn.\\ 
20 & dar kam Artus der Britun.\\ 
 & der zerb\textit{l}ûwene A\textit{ntha}nor\\ 
 & spranc dem künige alle\textit{z} vor,\\ 
 & unz er den Waleis ersach.\\ 
 & den vrâg\textit{e}te er: "sît ir\textbf{z}, der mich \textbf{dô} rach\\ 
25 & und Cu\textit{nn}ew\textit{a}re\textit{n}  de Lalant?\\ 
 & vil prîses giht man iuwerre hant.\\ 
 & Keie hât verpfendet,\\ 
 & sîn drôn ist \textbf{nû} gelendet.\\ 
 & ich vürhte wênic sînen swanc;\\ 
30 & der zesewe arm ist ime kranc."\\ 
\end{tabular}
\scriptsize
\line(1,0){75} \newline
m n o Fr69 \newline
\line(1,0){75} \newline
\textbf{13} \textit{Initiale} m n Fr69  \newline
\line(1,0){75} \newline
\textbf{2} Cunneware] Kingware m Conneware n Kume waren o \textbf{3} tiuren] \textit{om.} n \textbf{5} muose] Musse m Muͯste n (o)  $\cdot$ borten] den porten n (o) \textbf{6} rubîn] robin n \textbf{8} er schicket] Gesticket n o \textbf{9} diz] Dise n  $\cdot$ giht] git o \textbf{10} daz] Disz n o  $\cdot$ herze] herczen o \textbf{11} wer] Swer Fr69 \textbf{12} werden] wert n o \textbf{13} Artus] DO artus n \textbf{15} der] ir m  $\cdot$ tavelrunde] tafelrunden n tafelruͯnder o \textbf{16} enkeiner] in keiner n o  $\cdot$ nie] ie o \textbf{17} vernomen] ver vernomen n \textbf{20} Britun] pritun m brituͦn n britẏm o \textbf{21} zerblûwene] zerbuwene m  $\cdot$ Anthanor] auchonor m anthenor n o \textbf{22} künige] konigez o  $\cdot$ allez] aller m \textbf{24} vrâgete] frogente m  $\cdot$ irz] ir der n ir o  $\cdot$ dô] \textit{om.} n o \textbf{25} Cunnewaren] Cumewere m canewaren n Conne waren o  $\cdot$ de Lalant] delalant m \textbf{26} giht] git o  $\cdot$ iuwerre] ire m uwer n (o) \textbf{27} Keie] Keẏe n o \textbf{28} drôn] truwe n (o)  $\cdot$ nû] nit o \newline
\end{minipage}
\end{table}
\newpage
\begin{table}[ht]
\begin{minipage}[t]{0.5\linewidth}
\small
\begin{center}*G
\end{center}
\begin{tabular}{rl}
 & spien sim vür sîn houbtloch.\\ 
 & \textit{K}u\textit{newar}e gap im mê dannoch:\\ 
 & einen tiuren gürtel fier.\\ 
 & \textbf{von edele\textit{m} \textit{ge}stein\textit{e}} manic tier\\ 
5 & \textbf{muose} \textbf{ûzen} ûf \textbf{dem} borten sîn.\\ 
 & diu rinke was ein rubîn.\\ 
 & \begin{large}W\end{large}ie was der junge âne bart\\ 
 & geschicket, dô er \textbf{gekleidet} wart?\\ 
 & \textbf{daz} mære giht: wol genuoc.\\ 
10 & daz volc im holdez herze truoc.\\ 
 & \textbf{beidiu} man \textbf{unde} wîp,\\ 
 & \textbf{die} heten \textbf{wert} sînen lîp.\\ 
 & \textbf{der künic} messe hete gehôrt.\\ 
 & \textbf{man sach Artusen} komen dort\\ 
15 & mit der tavelrunder diet,\\ 
 & der \textbf{deheine} va\textit{l}scheit nie geriet.\\ 
 & die heten alle ê \textbf{wol} vernomen,\\ 
 & der rôte rîter wære komen\\ 
 & in Gawans pavelûn.\\ 
20 & dar kom Artus der Britun.\\ 
 & der zerblûwen Antanor\\ 
 & spranc dem künige allez vor,\\ 
 & unzer den Waleis ersach.\\ 
 & den vrâgter: "sît ir\textbf{z}, der mich rach\\ 
25 & unde \textit{Ku}ne\textit{war}en de Lalant?\\ 
 & vil brîses giht man iuwerre hant.\\ 
 & Kay hât verpfendet,\\ 
 & sîn dröun ist \textbf{nû} gelendet.\\ 
 & ich vürht wênic sînen swanc;\\ 
30 & der zeswe arm ist im \textbf{ze} kranc."\\ 
\end{tabular}
\scriptsize
\line(1,0){75} \newline
G I O L M Q R Z \newline
\line(1,0){75} \newline
\textbf{7} \textit{Initiale} G I Z  \textbf{13} \textit{Initiale} O L Q R  \textbf{23} \textit{Initiale} I  \newline
\line(1,0){75} \newline
\textbf{1} spien] Spinen Q  $\cdot$ sîn] das M \textbf{2} Kuneware] div froͮwe G kunwar I Kvnware O (M) Cvneware L Z Conware Q Cuͦnware R  $\cdot$ mê] \textit{om.} I Q R  $\cdot$ dannoch] ouch R \textbf{3} fier] fyn M schier R \textbf{4} edelem gesteine] edelen steinen G \textbf{5} muose] muͤst I Musten M  $\cdot$ ûf] an Z  $\cdot$ borten] guͯrtel L \textbf{6} rubîn] rubein Q \textbf{7} junge] juge M \textbf{8} dô] da M Z  $\cdot$ gekleidet] gegvrtet O (L) (M) (Q) (R) (Z) \textbf{9} daz] Ditze O (L) (M) (Q) (Z)  $\cdot$ giht] schit M \textbf{10} herze] hertzen L (R) \textbf{11} beidiu] Wer in sach beide Q Wer in an sach R Swer in gesach Z  $\cdot$ unde] oder Z \textbf{13} der] ÷er O  $\cdot$ messe hete] hatte messe M  $\cdot$ gehôrt] vernomen R \textbf{14} Artusen] Artuͯsen L artus R  $\cdot$ komen dort] doͯrt her komen R \textbf{15} tavelrunder] tavelrvnde L \textbf{16} deheine] deheiner O L (R)  $\cdot$ valscheit] vascheit G \textbf{17} alle ê] ê Alle O (Q) E R  $\cdot$ wol] \textit{om.} I O \textbf{18} der] Das der R \textbf{19} Gawans] Gawanes L gawanis M \textbf{20} Britun] Pritun I Brittvn L brittuͯn Q Bitun R \textbf{21} der] Vnd der R  $\cdot$ Antanor] Antonor R \textbf{22} dem] den R  $\cdot$ allez] als M \textbf{23} unzer] Do er L  $\cdot$ Waleis] walaýs L  $\cdot$ ersach] gischach M \textbf{24} \textit{Vers 307.24 fehlt} R   $\cdot$ vrâgter] fragt er I O Z  $\cdot$ irz] ir L M  $\cdot$ rach] [r]: da rach I \textbf{25} Kunewaren] mine froͮwen G kunwarn I Gvnwaren O Cvnewaren L kunwaren M konwaren Q Cuͦwarten R  $\cdot$ de] von R \textbf{26} giht man] man giht I set man M \textbf{27} Kay] kaẏ G (L) kain I Key O Q R Z Keie M  $\cdot$ hât] het R \textbf{28} dröun] tuͦn R  $\cdot$ gelendet] gelemet R \textbf{29} \textit{Versfolge 307.30-29} O  \textbf{30} zeswe] rechte M (R) zesem Q ceswen Z \newline
\end{minipage}
\hspace{0.5cm}
\begin{minipage}[t]{0.5\linewidth}
\small
\begin{center}*T
\end{center}
\begin{tabular}{rl}
 & spien sim vür sîn houbetloch.\\ 
 & Cunnewar gab im mêr dannoch:\\ 
 & einen tiuren gürtel fier.\\ 
 & \textbf{ûz edelem gesteine} manec tier\\ 
5 & \textbf{muosen} ûf \textbf{dem} borten sîn.\\ 
 & diu rinke was ein rubîn.\\ 
 & Wie was der junge âne bart\\ 
 & geschicket, dô er \textbf{begürtet} wart?\\ 
 & \textbf{diz} mære giht: wol genuoc.\\ 
10 & daz volc im holdez herze truoc.\\ 
 & \textbf{beid\textit{iu}} man \textbf{unde} wîp\\ 
 & heten \textbf{wert} sînen lîp.\\ 
 & \textbf{\begin{large}D\end{large}er künec} messe hete gehôrt.\\ 
 & \textbf{man sach Artusen} komen dort\\ 
15 & mit der tavelrunder diet,\\ 
 & der \textbf{deheine} valscheit nie geriet.\\ 
 & die heten alle ê vernomen,\\ 
 & der rôte rîter wære komen\\ 
 & in Gawans pavelûn.\\ 
20 & dar kom Artus der Britun.\\ 
 & Der zerblûwene Antenor\\ 
 & spranc dem künege allez vor,\\ 
 & unz er den Waleis ersach.\\ 
 & den vrâgeter: "sît ir, der mich rach\\ 
25 & unde Cunnewaren de Lalant?\\ 
 & vil prîses giht man iuwerre hant.\\ 
 & Key hât verpfendet,\\ 
 & sîn dröun ist gelendet.\\ 
 & ich vörhte wênic sînen swanc;\\ 
30 & der zesewe arm ist im \textbf{ze} kranc."\\ 
\end{tabular}
\scriptsize
\line(1,0){75} \newline
T U V W \newline
\line(1,0){75} \newline
\textbf{5} \textit{Initiale} V  \textbf{7} \textit{Majuskel} T  \textbf{13} \textit{Initiale} T U W  \textbf{21} \textit{Majuskel} T  \newline
\line(1,0){75} \newline
\textbf{2} Cunnewar] Kuͦnnewar U [Kv́neware*]: Kv́neware V Kunnewar W  $\cdot$ mêr] \textit{om.} W \textbf{4} ûz] [*]: Mit V \textbf{5} muosen] [mvezen]: mvesen T [Mvͤste*]: Mvͤste V Muͦsse W  $\cdot$ ûf] vzen of U vzzen an V aussenan an W  $\cdot$ borten] borte U \textbf{6} diu] der V (W)  $\cdot$ was] waren U  $\cdot$ rubîn] rubein W \textbf{8} begürtet] [*egv́rt*]: gegv́rtet V \textbf{9} diz] dise U  $\cdot$ giht] [gi*]: giht T git U  $\cdot$ wol] wol vnd W \textbf{10} holdez herze] heldes hertzen W \textbf{11} beidiu] beide T \textbf{15} der] \textit{om.} W \textbf{16} deheine] [*]: deheiner V  $\cdot$ nie] \textit{om.} W \textbf{21} Der] Vnd der W  $\cdot$ Antenor] AnthenoR T (U) (W) [anthe*]: anthenor  V \textbf{22} allez] [all*]: allez V \textit{om.} W \textbf{23} unz] Mit U  $\cdot$ Waleis] walleis V waleisen W  $\cdot$ ersach] sach W \textbf{24} den vrâgeter] \textit{om.} W  $\cdot$ ir] [*]: irs V  $\cdot$ rach] [*]: do rach V \textbf{25} Cunnewaren] Cvͦnnewaren T kuͦnnewaren U kvnnewaren V (W) \textbf{26} iuwerre] eúwer W \textbf{27} Key] Keẏn V  $\cdot$ hât] ward W \textbf{28} dröun] [troͤwe*]: troͤwen V  $\cdot$ gelendet] [*]: gelendet U [*]: nv gelendet V \newline
\end{minipage}
\end{table}
\end{document}
