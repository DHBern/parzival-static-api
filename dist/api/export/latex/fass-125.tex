\documentclass[8pt,a4paper,notitlepage]{article}
\usepackage{fullpage}
\usepackage{ulem}
\usepackage{xltxtra}
\usepackage{datetime}
\renewcommand{\dateseparator}{.}
\dmyyyydate
\usepackage{fancyhdr}
\usepackage{ifthen}
\pagestyle{fancy}
\fancyhf{}
\renewcommand{\headrulewidth}{0pt}
\fancyfoot[L]{\ifthenelse{\value{page}=1}{\today, \currenttime{} Uhr}{}}
\begin{document}
\begin{table}[ht]
\begin{minipage}[t]{0.5\linewidth}
\small
\begin{center}*D
\end{center}
\begin{tabular}{rl}
\textbf{125} & Der vürste in guoten morgen bôt\\ 
 & unt vrâgete, ob si sæhen nôt\\ 
 & eine juncvrouwen lîden.\\ 
 & sine kunden niht vermîden,\\ 
5 & swes er vrâgete, \textbf{daz} wart gesagt:\\ 
 & "zwêne ritter unt ein magt\\ 
 & \textbf{dâ} riten hiute morgen.\\ 
 & diu \textbf{vrouwe} vuor mit sorgen.\\ 
 & mit sporn si vaste ruorten,\\ 
10 & die die juncvrouwen vuorten."\\ 
 & \textbf{Ez} was Meljacanz,\\ 
 & den ergâhte Karnahkarnanz.\\ 
 & mit strîte er im die vrouwen nam.\\ 
 & diu was \textbf{dâ vor} an vreuden lam.\\ 
15 & si hiez Imane\\ 
 & von der \textbf{Beafontane}.\\ 
 & \begin{large}D\end{large}ie bûliute verzageten,\\ 
 & dô die \textbf{helde} vür si jageten.\\ 
 & si sprâchen: "wie ist uns \textbf{sus} geschehen?\\ 
20 & hât unser junchêrre \textbf{ersehen}\\ 
 & \textbf{an} disen rittern \textbf{helme schart},\\ 
 & sô\textbf{ne} hân wir uns niht wol bewart.\\ 
 & wir sulen der küneginne haz\\ 
 & von schulden hœren umb daz,\\ 
25 & wand \textbf{er} \textbf{mit uns dâ her} lief\\ 
 & hiute morgen, dô si dannoch slief."\\ 
 & Der knappe enruochte \textbf{ouch}, wer dô schôz\\ 
 & die hirze kleine \textbf{und} grôz.\\ 
 & er huop sich gein der muoter wider\\ 
30 & unt sagt ir mære; dô viel si nider.\\ 
\end{tabular}
\scriptsize
\line(1,0){75} \newline
D \newline
\line(1,0){75} \newline
\textbf{1} \textit{Majuskel} D  \textbf{11} \textit{Majuskel} D  \textbf{17} \textit{Initiale} D  \textbf{27} \textit{Majuskel} D  \newline
\line(1,0){75} \newline
\textbf{11} Meljacanz] Meliakanz D \textbf{12} Karnahkarnanz] karnachkarnanz D \textbf{15} Imane] Imâne D \textbf{16} Beafontane] Beafontâne D \newline
\end{minipage}
\hspace{0.5cm}
\begin{minipage}[t]{0.5\linewidth}
\small
\begin{center}*m
\end{center}
\begin{tabular}{rl}
 & der vürste in guoten morgen bôt\\ 
 & und vrâgete, ob si sæhen nôt\\ 
 & eine juncvrouwen lîden.\\ 
 & si enkunde\textit{n} niht vermîden,\\ 
5 & wes er vrâgete, \textbf{daz} wart gesaget:\\ 
 & "z\textit{w}êne ritter und eine maget\\ 
 & \textbf{d\textit{â}} riten hiute morgen.\\ 
 & diu \textbf{maget} vuor mit sorgen.\\ 
 & mit sporn si vaste ruorten,\\ 
10 & die die juncvrouwen vuorten."\\ 
 & \textbf{ez} was Meliaganz,\\ 
 & den ergâhete Karnachkarnan\textit{z}.\\ 
 & mit strîte er ime die vrouwen nam.\\ 
 & diu was \textbf{dâ vor} an vröuden lam.\\ 
15 & si hiez Ymane\\ 
 & von der \textbf{Beavantane}.\\ 
 & \begin{large}D\end{large}ie bûliute verzageten,\\ 
 & dô die \textbf{helde} vür si jageten.\\ 
 & si sprâchen: "wie ist uns \textbf{sus} geschehen?\\ 
20 & hât unser junchêrr\textit{e} \textbf{gesehen}\\ 
 & \textbf{ûf} disen rittern \textbf{helmes art},\\ 
 & sô \textbf{en}hân wir uns niht wol bewart.\\ 
 & wir s\textit{ullen} der künigîn haz\\ 
 & von schulden hœren umb daz,\\ 
25 & want \textbf{er} \textbf{mit uns dâ her} lief\\ 
 & hiute morgen, dô si dannoch slief."\\ 
 & der knapp\textit{e} enruochte, wer dô schôz\\ 
 & die hirze kleine \textbf{oder} grôz.\\ 
 & er huop sich gegen der muoter wider\\ 
30 & und sagete ir mære; dô viel si nider.\\ 
\end{tabular}
\scriptsize
\line(1,0){75} \newline
m n o \newline
\line(1,0){75} \newline
\textbf{17} \textit{Initiale} m o   $\cdot$ \textit{Capitulumzeichen} n  \newline
\line(1,0){75} \newline
\textbf{3} juncvrouwen] jungfrouwe n (o) \textbf{4} enkunden] enkunde m kunden n kúnde o \textbf{5} wes] Was m n o \textbf{6} zwêne] Zene m \textbf{7} dâ] Do m n o  $\cdot$ riten] ritter o \textbf{10} juncvrouwen] jungfrouwe n \textbf{11} Meliaganz] melia gancz m meliagantz n meliagancz o \textbf{12} Karnachkarnanz] karnach karnanck m karnoch karnantz n karnach karnancz o \textbf{13} vrouwen] frouwe n (o) \textbf{14} an] in o \textbf{15} \textit{Die Verse 125.15-16 fehlen} n   $\cdot$ Ymane] ẏman o \textbf{16} Beavantane] beafontammen o \textbf{18} helde] helle o \textbf{19} sus geschehen] beschehen n o \textbf{20} hât] Hette n  $\cdot$ junchêrre] jungheren m  $\cdot$ gesehen] gesohen n \textbf{21} rittern] ritter n \textbf{23} sullen] sint m sol o  $\cdot$ künigîn] konig o \textbf{24} von] Vmb n \textbf{27} knappe] knappen m \newline
\end{minipage}
\end{table}
\newpage
\begin{table}[ht]
\begin{minipage}[t]{0.5\linewidth}
\small
\begin{center}*G
\end{center}
\begin{tabular}{rl}
 & \begin{large}D\end{large}er vürste in guoten morgen bôt\\ 
 & und vrâgte \textbf{si}, op si s\textit{æ}hen nôt\\ 
 & eine juncvrouwen lîden.\\ 
 & sine kunden niht vermîden,\\ 
5 & swes er \textbf{si} vrâgete, \textbf{ez} wart gesaget:\\ 
 & "zwêne rîter und ein maget\\ 
 & \textbf{hie} riten h\textit{iu}te morgen.\\ 
 & diu \textbf{vrouwe} vuor mit sorgen.\\ 
 & mit sporen si vaste ruorten,\\ 
10 & die die juncvrouwen vuorten."\\ 
 & \textbf{daz} was Meliahganz,\\ 
 & den ergâhte Karnakarnanz.\\ 
 & mit strîte er im die vrouwen nam.\\ 
 & diu was \textbf{gar} an vröuden lam.\\ 
15 & \textit{si hiez Imanie}\\ 
 & v\textit{on der \textbf{Beafuntanie}}.\\ 
 & die bûliute \textbf{sêre} verzagten,\\ 
 & dô die \textbf{rîter} vür si jagten.\\ 
 & si sprâchen: "wiest uns \textbf{sus} geschehen?\\ 
20 & hât unser junchêrre \textbf{ersehen}\\ 
 & \textbf{ûf} disen rîteren \textbf{helm schart},\\ 
 & sô haben wir uns niht wol bewart.\\ 
 & wir sulen der küniginne haz\\ 
 & von schulden hœren umbe daz,\\ 
25 & wan \textbf{er} \textbf{dâ her mit uns} lief\\ 
 & hiute morgen, dô si dannoch slief."\\ 
 & der knappe enruohte, wer dô schôz\\ 
 & die hirze kleine \textbf{und} grôz.\\ 
 & er huop sich gein der muoter wider\\ 
30 & unde sagte ir mære; dô viel si nider.\\ 
\end{tabular}
\scriptsize
\line(1,0){75} \newline
G I O L M Q R Z Fr36 \newline
\line(1,0){75} \newline
\textbf{1} \textit{Initiale} G O  \textbf{5} \textit{Initiale} I  \textbf{17} \textit{Initiale} L R Z  \textbf{27} \textit{Initiale} I  \newline
\line(1,0){75} \newline
\textbf{1} Der] ÷er O \textbf{2} vrâgte si] vragt I (L) fragt siv O fragencz R  $\cdot$ sæhen] sahen G L sechent R iht sehen Z \textbf{3} eine juncvrouwen] ein magt I \textbf{4} kunden] enchvnden O  $\cdot$ niht] in nicht Q \textbf{5} swes] Wes L M Q R  $\cdot$ er] \textit{om.} L  $\cdot$ si vrâgete] si vraget I (O) (R) vraget L (Q)  $\cdot$ ez] des L  $\cdot$ wart] wer O \textbf{7} hie] Die L  $\cdot$ hiute] hoͮte G hie fuͯr huͯte L  $\cdot$ morgen] enmorgen I L Z \textbf{8} vrouwe] magt R  $\cdot$ vuor] \textit{om.} Z \textbf{9} sporen si] sporne Q \textbf{11} daz] Ez L  $\cdot$ Meliahganz] meliahkanz G MeliaGantz I Melyakanz O Melyakantz L Meliachkanz M meliahkrancz Q Meliahkancz R meliahkantz Z \textbf{12} ergâhte] her iachte M  $\cdot$ Karnakarnanz] karnach Garnanz I karnahkarnantz L karnacarnanz M karnahkarnancz Q (R) karnahkantz Z \textbf{14} was gar] wart sit I was da vor O L M (Q) R Z \textbf{15} \textit{Die Verse 125.15-16 fehlen} G   $\cdot$ Imanie] ymanGe I Jmenîe O vermanie L Jmnienie M Jammange Q Jmange R linmarie Z \textbf{16} Beafuntanie] Beafantanîe O Beahfentanie L Beafuntenie M beafontange Q (R) beasuntarie Z \textbf{17} sêre] \textit{om.} L \textbf{18} dô] Da M Z  $\cdot$ rîter] helde O L M Q (R) Z  $\cdot$ si] sich L \textbf{19} sus] \textit{om.} O Z so L M Q \textbf{20} hât] Hant R \textbf{21} disen rîteren] disem ritter O disem ritterrn Q  $\cdot$ schart] schar I \textbf{22} sô] son I (L) (Q) (R)  $\cdot$ bewart] \textit{om.} I \textbf{23} sulen] [haben]: svln O  $\cdot$ küniginne] kunginnen R \textbf{24} hœren] dulten I doln L \textbf{25} dâ her mit uns] mit vns da her O L M (Q) (R) Z \textbf{26} hiute morgen] hiuͤten morgen I (L) (Z)  $\cdot$ dô si] da sie M do R [s]: da sie Z \textbf{27} enruohte] enruͤcht I (R) en rvht ovch Z  $\cdot$ dô] da mer I da M Z \textit{om.} R \textbf{28} hirze] hierzen I (Q) hercze M  $\cdot$ und] oder O L (M) Q (R) Z \textbf{29} gein] [widdir]: geyn M \textbf{30} sagte] seit I (O) (Q) (R) (Z)  $\cdot$ ir] er M  $\cdot$ dô] da O M Z \newline
\end{minipage}
\hspace{0.5cm}
\begin{minipage}[t]{0.5\linewidth}
\small
\begin{center}*T (U)
\end{center}
\begin{tabular}{rl}
 & der vürste in guoten morgen bôt\\ 
 & und vrâget\textit{e} \textbf{si}, ob si sæhen nôt\\ 
 & eine juncvrouwen lîden.\\ 
 & si enkunden \textbf{ez} niht vermîden,\\ 
5 & wes er \textbf{si} vrâgete, \textbf{daz} wart gesaget:\\ 
 & "zwêne rîter und eine maget,\\ 
 & \textbf{die} riten \textbf{hie vür} hiute morgen.\\ 
 & diu \textbf{vrouwe} vuor mit sorgen.\\ 
 & mit sporn si vaste ruorten,\\ 
10 & die die juncvrouwen vuorten."\\ 
 & \textbf{daz} was Meliakanz,\\ 
 & den ergâhte Garnagarnanz.\\ 
 & mit strîte er im die vrouwen nam.\\ 
 & diu was \textbf{dâ vor} an vreuden lam.\\ 
15 & si hiez Ymanie\\ 
 & \textit{von} der \textbf{Peinfontanie}.\\ 
 & \begin{large}D\end{large}ie b\textit{ûli}u\textit{te} \textbf{sêre} verzageten,\\ 
 & dô die \textbf{helde} vür si jageten.\\ 
 & si sprâchen: "wie ist uns \textbf{sô} geschehen?\\ 
20 & hât unser junchêrre \textbf{ersehen}\\ 
 & \textbf{ûf} disen rîtern \textbf{helmes schart},\\ 
 & sô hân wir uns niht wol bewart.\\ 
 & wir soln der küneginne haz\\ 
 & von schulden hœren umb daz,\\ 
25 & wan \textbf{der} \textbf{mit uns dâ her} lief\\ 
 & hiute morgen, dô si dannoch slief."\\ 
 & der knappe enruochte, wer dô schôz\\ 
 & die hirze kleine \textbf{oder} grôz.\\ 
 & er huop sich gein der muoter wider\\ 
30 & und seit ir mære; dô viel si nider,\\ 
\end{tabular}
\scriptsize
\line(1,0){75} \newline
U V W T \newline
\line(1,0){75} \newline
\textbf{1} \textit{Majuskel} T  \textbf{4} \textit{Majuskel} T  \textbf{6} \textit{Initiale} T  \textbf{17} \textit{Initiale} U V W T  \textbf{27} \textit{Majuskel} T  \newline
\line(1,0){75} \newline
\textbf{2} vrâgete si] vrageten sie U vrage T \textbf{4} si enkunden ez] Sy begunden W Sine kvnden T  $\cdot$ vermîden] verswîgen T \textbf{5} wes] Swez V [swer]: swez T  $\cdot$ daz] es W ez T \textbf{7} die] hie T  $\cdot$ hie vür hiute morgen] hie fv́r mit sorgen V hie fúr heút am morgen W hivte morgen T \textbf{8} Hv́tte frvͤge morgen V  $\cdot$ vuor] rait W \textbf{10} die die] Die dise W \textbf{11} was] was der stoltz W  $\cdot$ Meliakanz] meliakantz V melyaganz W Meliaganz T \textbf{12} Garnagarnanz] gurnagarnanz W \textbf{14} dâ vor an] do vor W \textbf{15} Sy hieß die schoͤne imanie W  $\cdot$ Ymanie] Jmanŷe T \textbf{16} von] Vnd U  $\cdot$ Peinfontanie] [*fonthanie]: Beafonthanie V penfontanie W Beahfontanie T \textbf{17} bûliute] begunden U \textbf{19} si] Die W  $\cdot$ sô] \textit{om.} T \textbf{21} helmes] helm T \textbf{24} hœren] dvlten V (W) \textbf{25} wan der] Wan er V (W) wander T \textbf{26} dannoch] noch T \textbf{27} wer dô] ouch wer da T \textbf{30} seit ir] sagetir T  $\cdot$ mære] das mer W  $\cdot$ si] er T \newline
\end{minipage}
\end{table}
\end{document}
