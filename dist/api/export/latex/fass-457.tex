\documentclass[8pt,a4paper,notitlepage]{article}
\usepackage{fullpage}
\usepackage{ulem}
\usepackage{xltxtra}
\usepackage{datetime}
\renewcommand{\dateseparator}{.}
\dmyyyydate
\usepackage{fancyhdr}
\usepackage{ifthen}
\pagestyle{fancy}
\fancyhf{}
\renewcommand{\headrulewidth}{0pt}
\fancyfoot[L]{\ifthenelse{\value{page}=1}{\today, \currenttime{} Uhr}{}}
\begin{document}
\begin{table}[ht]
\begin{minipage}[t]{0.5\linewidth}
\small
\begin{center}*D
\end{center}
\begin{tabular}{rl}
\textbf{457} & \begin{large}D\end{large}ô \textbf{disiu} rede was getân,\\ 
 & dô sprach aber der guote man:\\ 
 & "ich bin râtes iwer wer.\\ 
 & nû sagt mir, wer \textbf{iu\textit{ch} wîste} her."\\ 
5 & "hêrre, \textbf{ûf dem} walde mir widergienc\\ 
 & ein grâ man, der mich wol enpfienc.\\ 
 & als tet sîn massenîe.\\ 
 & der selbe valsches vrîe\\ 
 & hât mich \textbf{zuo ziu her} gesant.\\ 
10 & ich reit sîne slâ, unz ich iuch vant."\\ 
 & der wirt sprach: "daz \textbf{was} Kahenis.\\ 
 & der ist werdeclîcher vuore wîs.\\ 
 & der vürste ist ein Punturteis.\\ 
 & der rîche künec von \textbf{Kareis}\\ 
15 & sîne swester hât ze wîbe.\\ 
 & nie kiuscher vruht von lîbe\\ 
 & wart geboren denne sîn \textbf{selbes} kint,\\ 
 & diu iu dâ widergangen sint.\\ 
 & der vürste ist \textbf{von} küneges art.\\ 
20 & all\textit{iu} jâr ist \textbf{zuo mir her} sîn vart."\\ 
 & Parzival zem wirte sprach:\\ 
 & "dô ich iuch \textbf{vor} mir \textbf{stênde} sach,\\ 
 & vörht ir iu iht, dô ich zuo \textbf{z}iu reit?\\ 
 & was iu mîn komen dô iht leit?"\\ 
25 & dô sprach er: "\textbf{hêrre}, geloubet mirz,\\ 
 & mich hât der \textbf{ber} und \textbf{ouch} der hirz\\ 
 & erschrecket \textbf{dicker} denne der man.\\ 
 & eine wârheit ich iu sagen kan:\\ 
 & ich\textbf{n} vürhte niht, swaz \textbf{mensche} ist.\\ 
30 & ich hân ouch menschlîchen list.\\ 
\end{tabular}
\scriptsize
\line(1,0){75} \newline
D \newline
\line(1,0){75} \newline
\textbf{1} \textit{Initiale} D  \newline
\line(1,0){75} \newline
\textbf{4} iuch] iv D \textbf{11} Kahenis] [Ga*nîs]: Gabenîs D \textbf{13} Punturteis] Pvntvrtêis D \textbf{14} Kareis] chareis D \textbf{20} alliu] alle D \textbf{21} Parzival] Parcifal D \newline
\end{minipage}
\hspace{0.5cm}
\begin{minipage}[t]{0.5\linewidth}
\small
\begin{center}*m
\end{center}
\begin{tabular}{rl}
 & dô \textbf{diu} rede was getân,\\ 
 & dô sprach aber der guote man:\\ 
 & "ich bin râtes iu\textit{w}er wer.\\ 
 & nû saget mir: wer \textbf{wîset iuch} her?"\\ 
5 & "hêrre, \textbf{ûf disem} walde mir widergienc\\ 
 & ein grâwer man, der mich wol enpfienc.\\ 
 & alsô tet sîn massenîe.\\ 
 & der selbe valsches vrîe\\ 
 & het mich \textbf{her zuo iu} gesant.\\ 
10 & ich reit sîn slâ, unz ich iuch vant."\\ 
 & der wirt sprach: "daz \textbf{was} Kehenis.\\ 
 & der ist wirdeclîcher vuor \textbf{als} wîs.\\ 
 & der vürste ist ein Ponturteise.\\ 
 & der rîche künic von \textbf{Clar\textit{ei}se}\\ 
15 & sîn swester het zuo wîbe.\\ 
 & nie kiuscher vruht von lîbe\\ 
 & wart geborn dan sîniu kint,\\ 
 & diu iu dâ widergangen sint.\\ 
 & der vürste ist \textbf{des} küniges art.\\ 
20 & alliu jâr ist \textbf{her zuo} sîn vart."\\ 
 & Parcifal zem wirte sprach:\\ 
 & "dô ich iuch \textbf{vor} mir \textbf{stên} sach,\\ 
 & vörht ir iu iht, dô ich zuo iu reit?\\ 
 & was iu mîn komen dô iht leit?"\\ 
25 & dô sprach er: "\textbf{nû} gloubet mirz,\\ 
 & mich het der \textbf{ber} und \textbf{ouch} der hirz\\ 
 & erschrecket \textbf{dicker} dan der man.\\ 
 & ein wârheit ich iu sagen kan:\\ 
 & ich vörhte niht, waz \textbf{menschlîch} ist.\\ 
30 & ich hab ouch menschlîch\textit{en} list.\\ 
\end{tabular}
\scriptsize
\line(1,0){75} \newline
m n o \newline
\line(1,0){75} \newline
\textbf{1} \textit{Capitulumzeichen} n  \newline
\line(1,0){75} \newline
\textbf{1} dô diu] Die do o  $\cdot$ getân] getam o \textbf{3} iuwer] uͯber m \textbf{10} ich] das ich o \textbf{12} wirdeclîcher] wurdeklichen o  $\cdot$ als] alle n \textbf{13} Ponturteise] punterteisze o \textbf{14} Clareise] clariese m clareẏse n klareise o \textbf{15} het] hette n \textbf{18} dâ] do n o \textbf{19} des] von n o \textbf{23} vörht] Forchte n  $\cdot$ iu] \textit{om.} o \textbf{24} iu] ich o \textbf{26} ouch] \textit{om.} n o \textbf{30} menschlîchen list] menschlich [lip]: list m \newline
\end{minipage}
\end{table}
\newpage
\begin{table}[ht]
\begin{minipage}[t]{0.5\linewidth}
\small
\begin{center}*G
\end{center}
\begin{tabular}{rl}
 & \begin{large}D\end{large}ô \textbf{diu} rede was getân,\\ 
 & \textit{d}ô sprach aber der guote man:\\ 
 & "ich bin râtes iuwer wer.\\ 
 & nû saget mir, wer \textbf{iuch wîste} her."\\ 
5 & "hêrre, \textbf{ûf dem} walde mir widergienc\\ 
 & ein grâ man, der mich wol enpfienc.\\ 
 & als tet sîn massenîe.\\ 
 & der selbe valsches vrîe\\ 
 & hât mich \textbf{zuo zi\textit{u h}er} gesant.\\ 
10 & ich reit sîn slâ, unze ich iuch vant."\\ 
 & der wirt sprach: "daz \textbf{was} Kahnis.\\ 
 & der ist werdeclîcher vuore \textbf{al}wîs.\\ 
 & der vürste ist ein Ponturteis.\\ 
 & der rîche künic von \textbf{Gareis}\\ 
15 & sîne swester hât ze wîbe.\\ 
 & nie kiuscher vruht von lîbe\\ 
 & \textit{w}art geborn danne sîn \textbf{selbes} kint,\\ 
 & diu iu dâ widergangen sint.\\ 
 & der vürste ist \textbf{von} küneges art.\\ 
20 & all\textit{iu} jâr ist \textbf{zuo mir her} sîn vart."\\ 
 & Parzival ze dem wirte sprach: \\ 
 & "dô ich iuch \textbf{vor} mir \textbf{stênde} sach,\\ 
 & vörh\textit{t} ir iu iht, dô ich zuo \textit{i}u reit?\\ 
 & was iu mîn komen dô iht leit?"\\ 
25 & dô sprach er: "\textbf{hêrre}, geloubet mirz,\\ 
 & mi\textit{ch} hât der \textbf{bêr} unde \textbf{ouch} der hirz\\ 
 & erschrecket \textbf{dicker} danne der man.\\ 
 & ein wârheit ich iu sagen kan:\\ 
 & ich\textbf{ne} vürhte niht, swaz \textbf{mensch} ist.\\ 
30 & ich hân ouch menschlîchen list.\\ 
\end{tabular}
\scriptsize
\line(1,0){75} \newline
G I O L M Z \newline
\line(1,0){75} \newline
\textbf{1} \textit{Initiale} G I O L Z  \textbf{15} \textit{Initiale} I  \newline
\line(1,0){75} \newline
\textbf{1} Dô diu] Diu I ÷a div O Da dise Z \textbf{2} dô] o G da O (M) (Z)  $\cdot$ der] dy M \textbf{4} nû] \textit{om.} I  $\cdot$ mir] \textit{om.} M  $\cdot$ iuch wîste] wiste uͯch L evch wiset Z \textbf{9} zuo ziu her] zoͮ ziv her ziv her G her zuͤ ev I (O) (L) zcu uch her M zv zv her Z \textbf{10} ich iuch] er mich O M \textbf{11} was] ist I  $\cdot$ Kahnis] kiehnis I kæhenis O Kahenis L (Z) kahevis M \textbf{12} werdeclîcher] werltlicher I werder O  $\cdot$ alwîs] wis O \textbf{13} Ponturteis] [portvrteis]: ponrturteis G pentortoys I pvͦntvͦrteis O Pvntuͯrteiz L punturteis M Z \textbf{14} Gareis] Garoys I sareis O Kareiz L kareis M Chareis Z \textbf{17} wart] Nie wart G \textbf{18} dâ] \textit{om.} I \textbf{20} alliu] Alle G O  $\cdot$ zuo mir her] her zuͤ mir I zvͦ mir O \textbf{21} Parzival] Parzifal I L M Parcifal O Z \textbf{22} dô] Da M Z  $\cdot$ vor] bi O L  $\cdot$ stênde] stendin M \textbf{23} vörht] Forhte G  $\cdot$ dô] da M Z  $\cdot$ iu] ziv G \textbf{24} mîn komen dô] do min chomen O \textbf{25} dô] Da O M Z  $\cdot$ er] \textit{om.} I der O L (M) \textbf{26} mich] Mir G (I)  $\cdot$ bêr] ber I O L (M) Z  $\cdot$ ouch] \textit{om.} O M \textbf{28} \textit{Versdoppelung 459.11-12 (²Z) nach 457.28; Lesarten der vorausgehenden Verse mit ¹Z bezeichnet:} Vnder im lac ramschovp vnd varm / Al sine lide die wurden warm Z   $\cdot$ ein] mer I  $\cdot$ sagen] gesagen I \textbf{29} ichne] ich I (O) (L)  $\cdot$ swaz] waz L (M) Z \textbf{30} menschlîchen] mensliche M \newline
\end{minipage}
\hspace{0.5cm}
\begin{minipage}[t]{0.5\linewidth}
\small
\begin{center}*T
\end{center}
\begin{tabular}{rl}
 & dô \textbf{di\textit{u}} rede was getân,\\ 
 & dô sprach aber der guote man:\\ 
 & "ich bin râtes iuwer wer.\\ 
 & nû saget mir, wer \textbf{iuch wîste} her."\\ 
5 & "Hêrre, \textbf{ûffem} walde mir widergie\\ 
 & ein grâwer man, der mich wol enpfie.\\ 
 & als tet sîn massenîe.\\ 
 & der selbe valsches vrîe\\ 
 & hât mich \textbf{her ziu} gesant.\\ 
10 & ich reit sîne slâ, unzich iuch vant."\\ 
 & Der wirt sprach: "daz \textbf{ist} Kahenis.\\ 
 & der ist werdeclîcher vuore \textbf{al}wîs.\\ 
 & der vürste ist ein Puntertoys.\\ 
 & der rîche künec von \textbf{Caroys}\\ 
15 & sîne swester hât ze wîbe.\\ 
 & nie kiuscher vruht von lîbe\\ 
 & wart \textbf{nie} geborn danne sîn \textbf{selbes} kint,\\ 
 & diu iu dâ widergangen sint.\\ 
 & der vürste ist \textbf{von} küneges art.\\ 
20 & alliu jâr ist \textbf{ze mir} sîn vart."\\ 
 & \begin{large}P\end{large}arcifal zem wirte sprach:\\ 
 & "dô ich iuch \textbf{bî} mir \textbf{stânde} sach,\\ 
 & vörhtir iu iht, dô ich zuo iu reit?\\ 
 & was iu mîn komen dô iht leit?"\\ 
25 & Dô sprach er: "\textbf{nû} gloubet mir\textit{z},\\ 
 & mich hât der \textbf{ber} unde der hirz\\ 
 & erschrecket \textbf{ê} danne der man.\\ 
 & eine wârheit ich iu sagen kan:\\ 
 & i\textbf{ne} vörhte niht, waz \textbf{mensche} ist.\\ 
30 & ich hân ouch menschlîchen list.\\ 
\end{tabular}
\scriptsize
\line(1,0){75} \newline
T U V W Q R \newline
\line(1,0){75} \newline
\textbf{1} \textit{Initiale} V Q   $\cdot$ \textit{Capitulumzeichen} R  \textbf{5} \textit{Majuskel} T  \textbf{11} \textit{Majuskel} T  \textbf{21} \textit{Initiale} T W  \textbf{25} \textit{Majuskel} T  \newline
\line(1,0){75} \newline
\textbf{1} \textit{Die Verse 453.1-502.30 fehlen} U   $\cdot$ diu] die T \textbf{2} guote] gut Q (R) \textbf{4} iuch wîste] îv wiste T wiset v́ch V (W) (R) \textbf{5} ûffem] vff den Q \textbf{6} wol] schon W \textit{om.} Q \textbf{7} tet] tet oͮch V \textbf{10} slâ unzich] stig bis Jch R  $\cdot$ iuch] iv T \textbf{11} ist] [*]: waz V  $\cdot$ Kahenis] kehenis V Lahenis Q Kachamis R \textbf{12} alwîs] wis W \textbf{13} Puntertoys] pontertoẏs V ponturteis W punturteis Q punterteis R \textbf{14} Caroys] Carôys T kareẏs V kareis W Q Karyes R \textbf{17} danne] wann W (R)  $\cdot$ selbes] \textit{om.} Q \textbf{18} dâ] do V W die Q \textbf{20} ze mir] har zvͦ mir V zuͦ mir her W (R) \textbf{21} Parcifal] Parzifal V PArtzifal W (Q) Parczifal R \textbf{22} ich iuch bî mir] ich iv bimir T uwer oͮge mich vor v́ch V ich bei mir W \textbf{23} iu] \textit{om.} Q  $\cdot$ iht] nit R \textbf{24} Was eúch minn do icht lait W  $\cdot$ Was euch do ich meyn komer leit Q \textbf{25} Dô sprach er] Er sprach Q R  $\cdot$ nû] herre V W Q R  $\cdot$ mirz] mirs T V W Q R \textbf{27} ê] diker V (W) (Q) (R) \textbf{29} ine] Jch R \textbf{30} hân] kan R  $\cdot$ menschlîchen] menschliche W (Q) \newline
\end{minipage}
\end{table}
\end{document}
