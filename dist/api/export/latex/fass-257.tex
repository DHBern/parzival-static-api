\documentclass[8pt,a4paper,notitlepage]{article}
\usepackage{fullpage}
\usepackage{ulem}
\usepackage{xltxtra}
\usepackage{datetime}
\renewcommand{\dateseparator}{.}
\dmyyyydate
\usepackage{fancyhdr}
\usepackage{ifthen}
\pagestyle{fancy}
\fancyhf{}
\renewcommand{\headrulewidth}{0pt}
\fancyfoot[L]{\ifthenelse{\value{page}=1}{\today, \currenttime{} Uhr}{}}
\begin{document}
\begin{table}[ht]
\begin{minipage}[t]{0.5\linewidth}
\small
\begin{center}*D
\end{center}
\begin{tabular}{rl}
\textbf{257} & \begin{large}D\end{large}â lac ûf ein gereite,\\ 
 & smal ân \textbf{alle} breite,\\ 
 & geschelle unt \textbf{bogen} verrêret,\\ 
 & grôz zadel dran gemêret.\\ 
5 & der vrouwen trûrec, niht ze geil,\\ 
 & ir \textbf{surzengel} was ein seil.\\ 
 & dem was si doch ze wol geborn.\\ 
 & ouch heten die este unt etslîch dorn\\ 
 & ir hemde zervüeret.\\ 
10 & swâ \textbf{daz} mit zerren was gerüeret,\\ 
 & dâ sach er vil \textbf{der stricke},\\ 
 & dâr unde liehte blicke:\\ 
 & ir hût \textbf{noch wîzer} denn ein swan.\\ 
 & si\textbf{ne} \textbf{hete} niht wan knoden an.\\ 
15 & swâ die wâren des velles dach,\\ 
 & in blanker varwe er daz sach.\\ 
 & daz ander leit von sunnen nôt.\\ 
 & swie ez \textbf{ie} kom, ir munt was rôt.\\ 
 & der muose al sölhe varwe tragen,\\ 
20 & man hete \textbf{viwer wol} drûz geslagen.\\ 
 & Swâ man si wolt an rîten,\\ 
 & daz was ze\textbf{r blôzen} sîten.\\ 
 & nante\textbf{s} iemen vilân,\\ 
 & der het ir unreht getân,\\ 
25 & wan si hete wênec an ir.\\ 
 & durch iwer zuht geloubet mir:\\ 
 & si truog ungedienten haz.\\ 
 & wîplîcher güete si nie vergaz.\\ 
 & ich saget \textbf{iu} vil armuot.\\ 
30 & war zuo? diz ist als guot.\\ 
 & \textbf{doch} næme ich sölhen blôzen lîp\\ 
 & vür etslîch wol gekleit wîp.\\ 
\end{tabular}
\scriptsize
\line(1,0){75} \newline
D \newline
\line(1,0){75} \newline
\textbf{1} \textit{Initiale} D  \textbf{21} \textit{Majuskel} D  \newline
\line(1,0){75} \newline
\newline
\end{minipage}
\hspace{0.5cm}
\begin{minipage}[t]{0.5\linewidth}
\small
\begin{center}*m
\end{center}
\begin{tabular}{rl}
 & dâ lac ûf ein gereite,\\ 
 & smal âne breite,\\ 
 & gesch\textit{e}lle und \textbf{bogen} verrêret,\\ 
 & gr\textit{ô}z \textit{z}adel dran gemêret.\\ 
5 & der vrouwen trûric, niht ze geil,\\ 
 & ir \textbf{surzengel} was ein seil.\\ 
 & dem was si doch ze wol geborn.\\ 
 & ouch heten die este und etlîche dorn\\ 
 & ir hemede zervüeret.\\ 
10 & wâ \textbf{ez} mit zerren was gerüeret,\\ 
 & d\textit{â} sach er vil \textbf{der stricke},\\ 
 & dâr under lieht\textit{e} blicke:\\ 
 & ir hût \textbf{noch wîzer} danne ein swane.\\ 
 & si \textbf{en}\textbf{vuorte} niht wanne knoden ane.\\ 
15 & wâ die wâren des velles dach,\\ 
 & in blanker varwe er daz sach.\\ 
 & daz ander leit von sunnen nôt.\\ 
 & wie ez kam, ir munt was rôt.\\ 
 & der muose al soliche varwe tragen,\\ 
20 & man hete \textbf{vil wol \textit{viur}} drûz geslagen.\\ 
 & wâ man si wolte an rîten,\\ 
 & daz was ze\textbf{r blôze\textit{n}} sîten.\\ 
 & nant\textit{e} \textbf{es} ieman vilân,\\ 
 & der het ir unreht getân,\\ 
25 & wanne si hete wênic an ir.\\ 
 & durch iuwere zuht geloubet mir:\\ 
 & si truoc ungedienten haz.\\ 
 & wîplîcher güete si nie vergaz.\\ 
 & ich sagete \textbf{iu} vil \textbf{v\textit{on}} armuot.\\ 
30 & war zuo? diz ist als guot.\\ 
 & \textbf{dô} næm ich solhen blôzen lîp\\ 
 & vür etslîch wol gekleidet wîp.\\ 
\end{tabular}
\scriptsize
\line(1,0){75} \newline
m n o Fr69 \newline
\line(1,0){75} \newline
\newline
\line(1,0){75} \newline
\textbf{1} gereite] gerete o \textbf{3} geschelle] Gescholle m Gestelle o \textbf{4} grôz zadel] Gras cadel m \textbf{6} surzengel] sorzingel m fúr zúgel n (o) \textbf{7} ze] so o \textbf{10} zerren] zorren o \textbf{11} dâ] Do m n o  $\cdot$ der] dicke n o \textbf{12} dâr] Das n  $\cdot$ liehte] liehten m (n) \textbf{14} wanne] dan o \textbf{15} des] die n  $\cdot$ velles] folkes o \textbf{16} blanker] blancke o \textbf{19} muose] musse m muͯste n o  $\cdot$ al] also n \textbf{20} vil wol viur] vil wol m wol fur n >wol< fur o \textbf{22} zer blôzen] zerblosse m zuͦr blaszen o \textbf{23} nante es] Nantes es m Nantes [ob]: es o \textbf{24} ir] ẏe n \textbf{26} iuwere] yere m ir n o \textbf{28} wîplîcher] Wieplich m \textbf{29} sagete] sage n  $\cdot$ von] vs m \textbf{30} als] \textit{om.} n so o \textbf{31} dô] Doch n (o) \textbf{32} gekleidet] bekleidet Fr69 \newline
\end{minipage}
\end{table}
\newpage
\begin{table}[ht]
\begin{minipage}[t]{0.5\linewidth}
\small
\begin{center}*G
\end{center}
\begin{tabular}{rl}
 & dâ lac ûffe ein gereite,\\ 
 & smal âne \textbf{alle} breite,\\ 
 & geschelle und \textbf{bogen} verr\textit{êr}et,\\ 
 & grôz zadel dran ge\textit{m}êret.\\ 
5 & der vrouwen trûric, niht ze geil,\\ 
 & ir \textbf{surzengel} was ein seil.\\ 
 & dem was si doch ze wolgeborn.\\ 
 & ouch heten die este und etslîch dorn\\ 
 & ir hemde zervüeret.\\ 
10 & swâ \textbf{ez} mit zerrene was gerüeret,\\ 
 & dâ sac\textit{h e}r vil \textbf{dicke}\\ 
 & dâr under liehte blicke:\\ 
 & ir hût \textbf{noch wîzer} danne ein swane.\\ 
 & si\textbf{ne} \textbf{vuorte} niht wan knoden ane.\\ 
15 & swâ die wâren des velles dach,\\ 
 & in blanker varwe er daz sach.\\ 
 & daz ander leit von sunnen nôt.\\ 
 & swiez \textbf{ie} kom, ir munt was rôt.\\ 
 & der muose alsolhe varwe tragen,\\ 
20 & man het\textbf{z viur wol} drûz geslagen.\\ 
 & swâ man si wolt an rîten,\\ 
 & daz was ze\textbf{r blôzen} sîten.\\ 
 & nante \textbf{si} iemen vilân,\\ 
 & der het ir unreht getân,\\ 
25 & wan si hete wênic an ir.\\ 
 & \begin{large}D\end{large}urch iuwer zuht geloubet mir:\\ 
 & si truoc ungedienten haz.\\ 
 & wîplîcher güete si ni\textit{e} vergaz.\\ 
 & ich sagte vil \textbf{ir} armuot.\\ 
30 & war zuo? diz ist als guot.\\ 
 & \textbf{doch} næme ich solhen blôzen lîp\\ 
 & vür etslîch wol gekleidet wîp.\\ 
\end{tabular}
\scriptsize
\line(1,0){75} \newline
G I O L M Q R Z Fr21 Fr60 \newline
\line(1,0){75} \newline
\textbf{1} \textit{Initiale} O L Z Fr21 Fr60   $\cdot$ \textit{Capitulumzeichen} R  \textbf{19} \textit{Initiale} I  \textbf{26} \textit{Initiale} G  \textbf{29} \textit{Initiale} Z  \textbf{31} \textit{Initiale} O L M Q Fr21   $\cdot$ \textit{Capitulumzeichen} R  \newline
\line(1,0){75} \newline
\textbf{1} dâ] ÷a O  $\cdot$ lac ûffe] vffe lach L \textbf{2} âne] vnd I \textbf{3} geschelle und bogen] satelbogen vnd shelle I  $\cdot$ verrêret] verret G \textbf{4} zadel] gebresten R  $\cdot$ gemêret] gecheret G \textbf{5} der vrouwen] Die frowe R  $\cdot$ trûric] trurig vnd R trvren Fr21 \textbf{6} surzengel] fvrzingel O (Q) zigel R \textbf{8} die este] este Z  $\cdot$ und] von I  $\cdot$ etslîch] die R \textbf{9} hemde] hemede gar L hende so Fr21 \textbf{10} swâ ez] Swaz O Fr21 Wa ez L (Q) (R) Swaz ez Z \textbf{11} dâ] Do Q  $\cdot$ sach er] sach oͮch er G sach er auch I ersach er O (M) (Q) R (Z) Fr21  $\cdot$ dicke] der striche O (L) (M) (Q) (R) (Z) (Fr21) \textbf{12} liehte] die liehten I lýchte L (M) (Q) \textbf{13} ir] \textit{om.} Fr21  $\cdot$ noch] \textit{om.} I Fr21  $\cdot$ danne ein] an ein Q denen R  $\cdot$ swane] swam I \textbf{14} sine] si I (L) (R) (Z) Seine Q  $\cdot$ vuorte] vurchten M fuͦt R  $\cdot$ knoden] hadern L koten M konden Q knoͯppff R \textbf{15} swâ] Wa L (Q) R  $\cdot$ die] sie Z  $\cdot$ velles] libes R  $\cdot$ dach] tag Q (R) \textbf{16} blanker] blancke Q wisser R \textbf{17} ander] andes R  $\cdot$ sunnen] svnne Fr21 \textbf{18} swiez] swie I Wie ez L (Q) (R)  $\cdot$ ir] der Fr21 \textbf{19} muose] mvͦz Fr21  $\cdot$ alsolhe] solche L (Q) also R  $\cdot$ varwe] frawe Q \textbf{20} man] wan I  $\cdot$ hetz] het Fr21  $\cdot$ wol] \textit{om.} M \textbf{21} swâ] Wo L Q (R)  $\cdot$ si] so L die Z  $\cdot$ wolt an] an wolte I \textbf{22} daz was] da was si I  $\cdot$ zer blôzen] ze blozzer I ze blozen O (L) (R) \textbf{23} nante] Neme R  $\cdot$ si] sich R  $\cdot$ iemen] nyman Q yemen ir R \textbf{24} ir] \textit{om.} R \textbf{25} wan] Wa Fr21  $\cdot$ wênic] Lvtzl O (L) (M) (Q) (Fr21) \textbf{26} geloubet] nv gelovbet O (Q) (Fr21) \textbf{27} ungedienten] vnverdienten Z \textbf{28} wîplîcher] Wibes O L M (Q) R (Fr21)  $\cdot$ nie] niht G \textbf{29} sagte] sag R (Fr21)  $\cdot$ vil] uͯch vil L (Q) (R) (Z) (Fr21) vbil M  $\cdot$ ir] \textit{om.} Z Fr21 \textbf{30} als] alles Q \textbf{31} doch] ÷och O \textbf{32} gekleidet] geuazt I gekleitz Z  $\cdot$ wîp] lib oder wib R \newline
\end{minipage}
\hspace{0.5cm}
\begin{minipage}[t]{0.5\linewidth}
\small
\begin{center}*T
\end{center}
\begin{tabular}{rl}
 & dâ lac ûf ein gereite,\\ 
 & smal âne \textbf{alle} breite,\\ 
 & geschelle unde \textbf{boge} verrêret,\\ 
 & grôz zadel dran gemêret.\\ 
5 & Der vrouwen trûric, niht ze geil,\\ 
 & ir \textbf{durchzingel} was ein seil.\\ 
 & dem was si doch ze wol geborn.\\ 
 & ouch heten die este unde etslîch dorn\\ 
 & ir hemde zervüeret.\\ 
10 & swâ \textbf{ez} mit zerrenne was gerüeret,\\ 
 & dâ sach er vil \textbf{der stricke},\\ 
 & dâ under liehte blicke:\\ 
 & ir hût \textbf{was} \textbf{glîcher} danne ein swan.\\ 
 & si \textbf{vuorte} niht wan knoden an.\\ 
15 & swâ die wâren des velles dach,\\ 
 & in blanker varwe er daz sach.\\ 
 & daz ander leit von sunnen nôt.\\ 
 & swiez \textbf{ie} kom, ir munt was rôt.\\ 
 & der muose also\textit{l}he varwe tragen,\\ 
20 & man hete \textbf{daz viur wol} drûz geslagen.\\ 
 & swâ man si wolte an rîten,\\ 
 & daz was ze \textbf{blôzer} sîten.\\ 
 & nante \textbf{s}ieman vilân,\\ 
 & der h\textit{e}tir unreht getân,\\ 
25 & wan si hete wênic an ir.\\ 
 & durch iuwer zuht geloubet mir:\\ 
 & si truoc u\textit{n}gedienten haz.\\ 
 & wîplîcher güete si nie vergaz.\\ 
 & ich seit \textbf{iu} vil \textbf{von} armuot.\\ 
30 & war zuo? diz ist als guot.\\ 
 & \textbf{doch} næm ich solhen blôzen lîp\\ 
 & vür etslîch wol gekleidet wîp.\\ 
\end{tabular}
\scriptsize
\line(1,0){75} \newline
T U V W \newline
\line(1,0){75} \newline
\textbf{5} \textit{Initiale} W   $\cdot$ \textit{Majuskel} T  \newline
\line(1,0){75} \newline
\textbf{1} dâ lac ûf] Do lag drauff W \textbf{3} boge] bogen V W \textbf{4} zadel] trauren W \textbf{5} Der vrouwen] Die vreuͦwe U [D*]: Der vrowen V  $\cdot$ trûric] pfert waz W \textbf{6} durchzingel] tarengv́rtel V \textbf{7} geborn] erborn V \textbf{8} etslîch] etlicher W \textbf{10} swâ] Wo U W \textbf{11} dâ] Daz U Do W  $\cdot$ der] \textit{om.} W \textbf{12} dâ] [D*]: Dar V Das W \textbf{13} glîcher] wizer U V liechter W \textbf{14} vuorte] in vuͦrte U (V)  $\cdot$ knoden] knoͤpffe W \textbf{15} swâ] Wo U W  $\cdot$ velles] velses U \textbf{16} in blanker] Ir blancken W  $\cdot$ daz] do W \textbf{18} swiez] Wie iz U (W)  $\cdot$ ie] echt W \textbf{19} der muose] der mvese T Die muͦze U  $\cdot$ alsolhe] alsohe T \textbf{20} daz] da V  $\cdot$ wol drûz] do durch W \textbf{21} swâ] Wo U W \textbf{22} ze blôzer] zuͦ der blozer U zerblozen V \textbf{23} nante] Rande W  $\cdot$ vilân] vil an W \textbf{24} hetir] hotir T [hette]: het ir V  $\cdot$ unreht] vnrechte U \textbf{27} ungedienten] vnde gedienten T vnuerdienten W \textbf{29} Jch sagete [*]: v́ch vil ir armuͦt V  $\cdot$ von] von irm U ir W \textbf{32} etslîch] manig W \newline
\end{minipage}
\end{table}
\end{document}
