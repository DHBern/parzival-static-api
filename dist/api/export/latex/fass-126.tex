\documentclass[8pt,a4paper,notitlepage]{article}
\usepackage{fullpage}
\usepackage{ulem}
\usepackage{xltxtra}
\usepackage{datetime}
\renewcommand{\dateseparator}{.}
\dmyyyydate
\usepackage{fancyhdr}
\usepackage{ifthen}
\pagestyle{fancy}
\fancyhf{}
\renewcommand{\headrulewidth}{0pt}
\fancyfoot[L]{\ifthenelse{\value{page}=1}{\today, \currenttime{} Uhr}{}}
\begin{document}
\begin{table}[ht]
\begin{minipage}[t]{0.5\linewidth}
\small
\begin{center}*D
\end{center}
\begin{tabular}{rl}
\textbf{126} & sîner worte si \textbf{sô} sêre erschrac,\\ 
 & daz si unversunnen vor im lac.\\ 
 & Dô diu küneginne\\ 
 & wider kom z\textbf{ir} sinne,\\ 
5 & swie si dâ vor wære verzagt,\\ 
 & \textbf{dô sprach si}: "sun, wer hât gesagt\\ 
 & dir von ritters orden?\\ 
 & wâ bist dûs innen worden?"\\ 
 & "Muoter, ich sach vier man\\ 
10 & noch liehter danne got getân.\\ 
 & die sageten mir von ritterschaft.\\ 
 & Artuses küneclîchiu kraft\\ 
 & sol mich nâch ritters êren\\ 
 & an schildes ambet kêren."\\ 
15 & Sich huop ein niwer jâmer hie.\\ 
 & diu vrouwe enweste rehte, wie\\ 
 & \textbf{daz} si ir \textbf{den list} erdæhte\\ 
 & \textbf{unt} in von dem willen bræhte.\\ 
 & \begin{large}D\end{large}er knappe tump unt wert\\ 
20 & iesch von der muoter dicke \textbf{ein} pfert.\\ 
 & daz begunde si \textbf{in ir} herzen klagen.\\ 
 & si dâhte: "i\textbf{ne} wil im niht versagen,\\ 
 & ez muoz aber \textbf{vil} bœse sîn."\\ 
 & dô \textbf{gedâhte} \textbf{mêr} diu künegîn:\\ 
25 & "der liute vil bî spotte sint.\\ 
 & tôren kleider sol mîn kint\\ 
 & ob sîme liehten lîbe tragen.\\ 
 & wirt er geroufet \textbf{unt} geslagen,\\ 
 & sô kumt er mir her wider wol."\\ 
30 & owê der jæmerlîchen dol!\\ 
\end{tabular}
\scriptsize
\line(1,0){75} \newline
D \newline
\line(1,0){75} \newline
\textbf{3} \textit{Majuskel} D  \textbf{9} \textit{Majuskel} D  \textbf{15} \textit{Majuskel} D  \textbf{19} \textit{Initiale} D  \newline
\line(1,0){75} \newline
\textbf{12} Artuses] Artvs D \newline
\end{minipage}
\hspace{0.5cm}
\begin{minipage}[t]{0.5\linewidth}
\small
\begin{center}*m
\end{center}
\begin{tabular}{rl}
 & sîner worte si \textbf{sô} sêre erschrac,\\ 
 & daz si unversunnen vor im lac.\\ 
 & dô diu küniginne\\ 
 & wider kam ze sinne,\\ 
5 & wie si dâr vor wære verzaget,\\ 
 & \textbf{si sprach}: "sun, wer hât gesaget\\ 
 & dir von ritters orden?\\ 
 & wâ bistû es innen worden?"\\ 
 & "muoter, ich sach vier man\\ 
10 & noch liehter danne got getân.\\ 
 & die sageten mir von ritterschaft.\\ 
 & Artuses küniclîchiu kraft\\ 
 & sol mich nâch ritters ê\textit{r}en\\ 
 & an schiltes ambet kêren."\\ 
15 & sich huop ein niuwe jâmer hie.\\ 
 & diu vrouwe enwuste rehte, wie\\ 
 & \textbf{daz} si ir \textbf{den} \dag lust\dag  erdæhte,\\ 
 & \textbf{daz si} in von dem willen bræhte.\\ 
 & der knappe tump und wert\\ 
20 & iesch von der muoter dicke \textbf{ein} pfert.\\ 
 & daz begund\textit{e} si \textbf{in} herzen klagen.\\ 
 & si dâhte: "ich wil ime niht versagen,\\ 
 & ez muoz aber \textbf{vil} bœse sîn."\\ 
 & dô \textbf{gedâhte} \textbf{\textit{m}êr} diu künigîn:\\ 
25 & "der liute vil bî spotte sint.\\ 
 & tôren kleider sol mîn kint\\ 
 & ob sînem liehten lîbe tragen.\\ 
 & wirt er geroufet \textbf{und} geslagen,\\ 
 & sô kumet er mir her wider wol."\\ 
30 & owê der jâmerlîchen dol!\\ 
\end{tabular}
\scriptsize
\line(1,0){75} \newline
m n o \newline
\line(1,0){75} \newline
\newline
\line(1,0){75} \newline
\textbf{1} sô] gar n \textit{om.} o \textbf{2} lac] do lag n o \textbf{8} es] \textit{om.} o \textbf{13} êren] erden m \textbf{15} niuwe] nuwes n \textbf{16} rehte] \textit{om.} n \textbf{17} erdæhte] erdeckte o \textbf{18} daz] Vnd n o \textbf{20} iesch] Hie o \textbf{21} begunde] begunden m  $\cdot$ herzen] ir hertze n (o) \textbf{22} dâhte] gedochte n (o)  $\cdot$ wil] wil es n (o)  $\cdot$ versagen] [sagen]: versagen o \textbf{24} mêr] [mi*]: mier m \textbf{29} sô] Do n \newline
\end{minipage}
\end{table}
\newpage
\begin{table}[ht]
\begin{minipage}[t]{0.5\linewidth}
\small
\begin{center}*G
\end{center}
\begin{tabular}{rl}
 & sîner worte si \textbf{sô} sêre erschrac,\\ 
 & daz si unversunnen \textit{vor im} lac.\\ 
 & dô diu küniginne\\ 
 & wider kom ze sinne,\\ 
5 & swie si dâ vor wære verzaget,\\ 
 & \textbf{doch sprach si}: "sun, wer hât gesaget\\ 
 & dir von rîters orden?\\ 
 & wâ bistûs innen worden?"\\ 
 & "muoter, ich sach vier man\\ 
10 & noch liehter dane got getân.\\ 
 & die seiten mir von rîterschaft.\\ 
 & Artuses küniclîchiu kraft\\ 
 & sol mich nâch rîters êren\\ 
 & an schiltes ambet kêren."\\ 
15 & sich huop ein niwer jâmer hie.\\ 
 & diu vrouwe enwesse rehte, wie\\ 
 & sir \textbf{der liste} erd\textit{æ}hte\\ 
 & \textbf{unde} in von dem willen br\textit{æ}hte.\\ 
 & der knappe tump und wert\\ 
20 & iesch von der muoter dicke \textbf{ein} pfert.\\ 
 & daz begunde si \textbf{in ir} herzen klagen.\\ 
 & si dâhte: "ich\textbf{ne} wil\textbf{z} im niht versagen,\\ 
 & ez muoz aber \textbf{harte} bœse sîn."\\ 
 & dô \textbf{dâhte} \textbf{mêr} diu künigîn:\\ 
25 & "\begin{large}D\end{large}er liute vil bî spote sint.\\ 
 & tôren kleider sol mîn kint\\ 
 & obe sînem liehten lîbe tragen.\\ 
 & wirt er geroufet \textbf{und} geslagen,\\ 
 & sô kumet er mir her wider wol."\\ 
30 & owê der jæmerlîchen dol!\\ 
\end{tabular}
\scriptsize
\line(1,0){75} \newline
G I O L M Q R Z \newline
\line(1,0){75} \newline
\textbf{3} \textit{Initiale} M  \textbf{9} \textit{Initiale} I  \textbf{19} \textit{Initiale} Q R Z  \textbf{25} \textit{Initiale} G  \newline
\line(1,0){75} \newline
\textbf{1} worte] wortten R  $\cdot$ sô] \textit{om.} M Q R vil Z \textbf{2} unversunnen] vnuersunden Q vnuersint R  $\cdot$ vor im] \textit{om.} G  $\cdot$ lac] Gelac I \textbf{3} dô] Da M Z  $\cdot$ diu] die I \textbf{4} ze] ze ir O (Q) (R) (Z) \textbf{5} swie] Wie L (M) (Q) R  $\cdot$ dâ] do Q  $\cdot$ wære verzaget] verzagte O \textbf{6} doch] do I (L) (Q) Da O M Z  $\cdot$ gesaget] gesagte O \textbf{8} bistûs] bistv des O (L) bistu oz M wistusz Q bistu R \textbf{10} noch] \textit{om.} I  $\cdot$ liehter] lýchter L (M) (Q) liechtten R  $\cdot$ dane] wann R \textbf{12} Artuses] Arcuses Q Artus R  $\cdot$ küniclîchiu] kintliche Q \textbf{13} mich] man Q  $\cdot$ rîters] Ritters orden R \textbf{14} kêren] leren Z \textbf{16} rehte] \textit{om.} Z \textbf{17} sir] daz si ir I (O) (R) (Z) Sý L Das si M Das ir Q  $\cdot$ der liste] den list L den lust R  $\cdot$ erdæhte] erdahte G (L) \textbf{18} unde] Daz sie L  $\cdot$ in] im O  $\cdot$ bræhte] brahte G (L) \textbf{20} iesch] hyesz Q  $\cdot$ muoter] vrowen L  $\cdot$ dicke] \textit{om.} O L \textbf{21} begunde si] begundes Q Z  $\cdot$ herzen] [het]: hertze L hercze M (Q) (Z) \textbf{22} dâhte] Gedaht I (O) (L) (M) (Q) (R)  $\cdot$ ichne] ich I O L Q (R) Z  $\cdot$ wilz im] wil O wel ym M (Z) wil ymsz Q (R)  $\cdot$ versagen] [vertragen]: versagen I \textbf{24} dô] Da M Z  $\cdot$ dâhte] Gedaht I (L) (M)  $\cdot$ mêr] ir I mir M \textbf{25} Der] Die L  $\cdot$ bî] in R \textbf{26} kleider] cheider I \textbf{27} obe] Vff M  $\cdot$ liehten] liehtem I O lichten L (M) claren Q (R) \textbf{30} owê] Awe O \newline
\end{minipage}
\hspace{0.5cm}
\begin{minipage}[t]{0.5\linewidth}
\small
\begin{center}*T (U)
\end{center}
\begin{tabular}{rl}
 & \hspace*{-.7em}\big| daz si unversunnen vor im lac.\\ 
 & \hspace*{-.7em}\big| sîner worte si sêre erschrac.\\ 
 & dô diu küniginne\\ 
 & wider kam zuo sinne,\\ 
5 & wie si dâ vor wære verzaget,\\ 
 & \textbf{doch sprach s\textit{i}}: "sun, wer hât gesaget\\ 
 & dir von rîters orden?\\ 
 & wâ bistû es innen worden?"\\ 
 & "muoter, ich sach vier man\\ 
10 & noch liehter danne got getân.\\ 
 & die sageten mir von rîterschaft.\\ 
 & Artuses küneclîchiu kraft\\ 
 & sol mich nâch ritters êren\\ 
 & an schiltes ambet kêren."\\ 
15 & sich huop ein niuwe jâmer hie.\\ 
 & diu vrouwe enw\textit{uste} rehte, wie\\ 
 & \textbf{daz} si \textbf{den list} erdæhte\\ 
 & \textbf{und} in von dem willen bræhte.\\ 
 & \begin{large}D\end{large}er knappe tump und wert\\ 
20 & iesch von der muoter dicke pfert.\\ 
 & daz begunde si \textbf{von} herzen klagen.\\ 
 & si dâhte: "ich wil \textbf{ez} im niht versagen,\\ 
 & ez muoz aber \textbf{harte} bœse sîn."\\ 
 & dô \textbf{gedâhte} diu künegîn:\\ 
25 & "der liute vil bî spotte sint.\\ 
 & tôren kleider sol mîn kint\\ 
 & ob sîme liehten lîbe tragen.\\ 
 & wirt er geroufet \textbf{oder} geslagen,\\ 
 & sô kumt er mir her wider wol."\\ 
30 & owê der jæmerlîchen dol!\\ 
\end{tabular}
\scriptsize
\line(1,0){75} \newline
U V W T \newline
\line(1,0){75} \newline
\textbf{3} \textit{Majuskel} T  \textbf{15} \textit{Majuskel} T  \textbf{19} \textit{Initiale} U V W T  \newline
\line(1,0){75} \newline
\textbf{2} \textit{Versfolge 126.1-2} T  \textbf{1} sêre] gar V so sêre T \textbf{4} wider] Aber wider W  $\cdot$ zuo] zir T \textbf{5} wie] Swie V T \textbf{6} doch] Do W  $\cdot$ si] sin U \textbf{8} bistû es] bistv V \textbf{9} muoter] vrôuwe T \textbf{15} niuwe] neúwer W niͮwez T \textbf{16} enwuste] in weiz U \textbf{17} daz si den] si einen T \textbf{20} dicke] dike [*]: ein V oft W dîcke ein T \textbf{21} si von] sy im W sim in ir T \textbf{22} ich wil ez im] ich wil ims W ine wils im T \textbf{23} harte bœse] vil boͤse W harte bvese T \textbf{24} dô gedâhte] Da gedahte V Do gedachte ir me W do dahte mer T \textbf{25} Wann alle froͤde was ir sam der wint W \textbf{27} liehten] schoͤnen V \textbf{28} oder] vnde V \textbf{29} mir] \textit{om.} T \textbf{30} jæmerlîchen] iemerliche U \newline
\end{minipage}
\end{table}
\end{document}
