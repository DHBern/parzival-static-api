\documentclass[8pt,a4paper,notitlepage]{article}
\usepackage{fullpage}
\usepackage{ulem}
\usepackage{xltxtra}
\usepackage{datetime}
\renewcommand{\dateseparator}{.}
\dmyyyydate
\usepackage{fancyhdr}
\usepackage{ifthen}
\pagestyle{fancy}
\fancyhf{}
\renewcommand{\headrulewidth}{0pt}
\fancyfoot[L]{\ifthenelse{\value{page}=1}{\today, \currenttime{} Uhr}{}}
\begin{document}
\begin{table}[ht]
\begin{minipage}[t]{0.5\linewidth}
\small
\begin{center}*D
\end{center}
\begin{tabular}{rl}
\textbf{674} & \textit{\begin{large}S\end{large}}ît ir mich \textbf{gesuochet} hât,\\ 
 & \textbf{nû} lêre iuch got ergetzens rât.\\ 
 & in des helfe ir sît geriten,\\ 
 & ob \textbf{der} hât mit mir gestriten,\\ 
5 & dâ wart ich âne wer \textbf{bekant}\\ 
 & unt zer blôzen sîten an gerant.\\ 
 & ob \textbf{der} noch strîtes \textbf{gein mir} gert,\\ 
 & der wirt wol geendet âne swert."\\ 
 & ZArtuse sprach dô Gawan:\\ 
10 & "waz rât ir\textbf{s}, ob wir \textbf{disen} plân\\ 
 & baz mit rîtern überlegen,\\ 
 & sît wirz wol getuon megen?\\ 
 & ich erwirbe wol \textbf{an} der herzogîn,\\ 
 & daz die iwern \textbf{ledec sulen} sîn\\ 
15 & unt daz ir rîterschaft dâ her\\ 
 & kumt mit manegem niwen sper."\\ 
 & "Des volge ich", sprach Artus.\\ 
 & diu herzogîn dô hin ze hûs\\ 
 & sande nâch den werden.\\ 
20 & ich wæne, ûf der erden\\ 
 & \textbf{ie} schœner samnunge wart.\\ 
 & Gein \textbf{herbergen sîner} vart\\ 
 & Gawan urloubes gerte,\\ 
 & des in der künec \textbf{gewerte}.\\ 
25 & die man mit im komen sach,\\ 
 & vuoren dan mit im an \textbf{ir} gemach.\\ 
 & sîn herberge rîche\\ 
 & stuont sô rîterlîche,\\ 
 & daz si was kostebære\\ 
30 & \textbf{unt} \textbf{der} armüete lære.\\ 
\end{tabular}
\scriptsize
\line(1,0){75} \newline
D Fr8 \newline
\line(1,0){75} \newline
\textbf{1} \textit{Initiale} D  \textbf{9} \textit{Initiale} Fr8   $\cdot$ \textit{Majuskel} D  \textbf{17} \textit{Majuskel} D  \textbf{22} \textit{Majuskel} D  \newline
\line(1,0){75} \newline
\textbf{1} Sît] ÷it D \textbf{2} nû] So Fr8 \textbf{9} ZArtuse] ZArtv̂se D Zvͦ Arthuse Fr8 \textbf{11} rîtern] zuchten Fr8 \textbf{13} erwirbe] wirbe Fr8 \textbf{14} Daz die vwer suln ledich sin Fr8 \textbf{16} manegem] manigen Fr8 \textbf{17} Artus] Arthus Fr8 \textbf{18} hin ze] zirme Fr8 \textbf{20} ûf] ie vf Fr8 \textbf{21} ie] \textit{om.} Fr8 \textbf{26} Die vuͦren mit im an genach Fr8 \textbf{30} der] von Fr8 \newline
\end{minipage}
\hspace{0.5cm}
\begin{minipage}[t]{0.5\linewidth}
\small
\begin{center}*m
\end{center}
\begin{tabular}{rl}
 & sît ir mich \textbf{gesuoch\textit{et}} \textit{h}ât,\\ 
 & \textbf{nû} lêre iuch got ergetzens rât,\\ 
 & \textbf{wan} in des helfe ir sît geriten,\\ 
 & ob \textbf{er} het mit mir gestriten,\\ 
5 & d\textit{â} wart ich âne wer \textbf{bekant}\\ 
 & und zer blôzen sîten an gerant.\\ 
 & ob \textbf{er} noch strîtes \textbf{an mich} gert,\\ 
 & der wirt wol ge\textit{end}et âne swert."\\ 
 & zuo Artuse sprach dô Gawan:\\ 
10 & "waz râtet ir, ob \textit{w}ir \textbf{den} plân\\ 
 & baz mit rittern überlege\textit{n},\\ 
 & sît wirz wol getuon mege\textit{n}?\\ 
 & ich erwirbe wol \textbf{an} der herzogîn,\\ 
 & daz die iuwern \textbf{ledic sullent} sîn\\ 
15 & und daz ir ritterschaft dâ her\\ 
 & kumt mit manige\textit{m} niuwen sper."\\ 
 & "des volge ich \textbf{dir}", sprach Artus.\\ 
 & diu herzogîn dô hin zuo \textbf{ir} hûs\\ 
 & sante nâch den werden.\\ 
20 & ich wæne, ûf der erden\\ 
 & \textbf{ie} schœner samenunge wart.\\ 
 & gegen \textbf{sîner herbergen} vart\\ 
 & Gawan urloubes gert,\\ 
 & des in der künic \textbf{wert}.\\ 
25 & die man mit i\textit{m k}omen sach,\\ 
 & vuoren dan mit im an \textbf{sîn} gemach.\\ 
 & sîn herberge rîche\\ 
 & stuont sô ritterlîche,\\ 
 & daz si was kostbære\\ 
30 & \textbf{und} \textbf{grôzer} armuot lære.\\ 
\end{tabular}
\scriptsize
\line(1,0){75} \newline
m n o \newline
\line(1,0){75} \newline
\newline
\line(1,0){75} \newline
\textbf{1} gesuochet hât] gesuchent han hat m \textbf{2} ergetzens] [ergeczen]: ergeczens o \textbf{4} er] der n o \textbf{5} dâ] Do m n o \textbf{7} er] der n o \textbf{8} geendet] gedienet m \textbf{9} dô] \textit{om.} o \textbf{10} wir] ir m n o \textbf{11} überlegen] uͯber legent m \textbf{12} megen] mogent m \textbf{15} ritterschaft] ritteschafft o \textbf{16} manigem] mangen m \textbf{17} Artus] artuͯs o \textbf{20} der] \textit{om.} n \textbf{23} Gawan] Gawanes n  $\cdot$ gert] gerte n \textbf{24} wert] werte n \textbf{25} im komen] ym sach komen m ime E komen n (o) \newline
\end{minipage}
\end{table}
\newpage
\begin{table}[ht]
\begin{minipage}[t]{0.5\linewidth}
\small
\begin{center}*G
\end{center}
\begin{tabular}{rl}
 & \begin{large}S\end{large}ît ir mich \textbf{gesuochet} hât,\\ 
 & \textbf{sô} lêre iuch got ergetzens rât.\\ 
 & in des helfe ir sît geriten,\\ 
 & op \textbf{der} hât mit mir gestriten,\\ 
5 & dô wart ich âne wer \textbf{erkant}\\ 
 & unde ze der blôzen sîten an gerant.\\ 
 & op \textbf{der} noch strîtes \textbf{ane mich} gert,\\ 
 & der wirt wol geendet âne swert."\\ 
 & ze Artus sprach dô Gawan:\\ 
10 & "waz râtet ir\textbf{s}, ob wir \textbf{disen} plân\\ 
 & baz mit rîtern überlegen,\\ 
 & sît wirz wol getuon megen?\\ 
 & ich erwirbe\textbf{z} wol \textbf{dâ ze}r herzogîn,\\ 
 & daz die iwern \textbf{sulen ledic} sîn\\ 
15 & unde daz ir rîterschaft dâ her\\ 
 & kumt mit manigem niwen sper."\\ 
 & "des volge ic\textit{h}", \textit{s}prach Artus.\\ 
 & diu herzogîn dô hin ze \textbf{ir} hûs\\ 
 & sande nâch den werden.\\ 
20 & ich wæne, ûf der erde\textit{n}\\ 
 & \textbf{nie} schœner samenunge wart.\\ 
 & gein \textbf{herbergen sîner} vart\\ 
 & Gawan urloubes gerte,\\ 
 & des in der künic \textbf{gewerte}.\\ 
25 & die man mit im komen sach,\\ 
 & vuoren dan mit im an \textbf{ir} gemach.\\ 
 & sîn herberge rîche\\ 
 & stuont sô rîterlîche,\\ 
 & daz si was kostebære,\\ 
30 & \textbf{der} armüete lære.\\ 
\end{tabular}
\scriptsize
\line(1,0){75} \newline
G I L M Z Fr61 \newline
\line(1,0){75} \newline
\textbf{1} \textit{Initiale} G I L Z  \textbf{17} \textit{Initiale} I  \newline
\line(1,0){75} \newline
\textbf{1} mich] mir Z \textbf{2} sô] Nv L (M) (Z) (Fr61)  $\cdot$ iuch got] ich evch I \textbf{4} der] ir Z \textbf{5} dô] So L Fr61 Dy M Da Z  $\cdot$ ich] ouch M  $\cdot$ erkant] bikant M (Z) \textbf{6} der] \textit{om.} I L Fr61  $\cdot$ blôzen sîten] blozzer seite Fr61 \textbf{7} noch] \textit{om.} M \textbf{9} Artus] Artuse I (M) Artuͯse L Artausen Fr61  $\cdot$ dô] da M \textit{om.} Z \textbf{10} irs] ir I M Z Fr61 \textbf{13} dâ zer] daz der I zuͯr L (M) (Z) (Fr61) \textbf{14} sulen ledic] ledic suln M (Z) mvͤzzen ledich Fr61 \textbf{16} manigem niwen] mangen niwem I manigē M \textbf{17} volge ich sprach] volge ih gerne sprah G  $\cdot$ Artus] Artaus Fr61 \textbf{18} dô] da M Z  $\cdot$ hin ze ir] mir I zcu ir M \textbf{19} den] der L Z \textbf{20} ûf] uff uff M  $\cdot$ erden] erde G \textbf{22} herbergen] herberge I \textbf{23} gerte] gert Fr61 \textbf{24} gewerte] wert Fr61 \textbf{26} dan] \textit{om.} I M Fr61  $\cdot$ ir] \textit{om.} Fr61 \textbf{27} \textit{Die Verse 674.27-28 fehlen} Fr61  \textbf{28} sô] \textit{om.} I \textbf{29} daz] so daz I  $\cdot$ si] [d*]: sy M \textbf{30} der] Vnd der L (M) (Z) vnd Fr61 \newline
\end{minipage}
\hspace{0.5cm}
\begin{minipage}[t]{0.5\linewidth}
\small
\begin{center}*T
\end{center}
\begin{tabular}{rl}
 & sît ir mich \textbf{besuochet} hât,\\ 
 & \textbf{nû} lêr iuch got ergetzens rât.\\ 
 & in des helfe ir sît geriten,\\ 
 & ob \textbf{der} hât mit \textit{mir} gestriten,\\ 
5 & dô wart ich âne were \textbf{erkant}\\ 
 & und zuor blôzen sîten an gerant.\\ 
 & ob \textbf{der} noch strîtes \textbf{an mich} gert,\\ 
 & der wirt wol geendet âne swert."\\ 
 & zuo Artus sprach dô Gawan:\\ 
10 & "waz rât ir, ob wir \textbf{disen} plân\\ 
 & baz mit rittern überlegen,\\ 
 & sît wirz wol getuon megen?\\ 
 & ich erwirb \textbf{ez} wol \textbf{dâ zuo}r herzogîn,\\ 
 & daz die iuweren \textbf{ledic sollen} sîn\\ 
15 & und daz ir ritterschaft dâ her\\ 
 & kumt mit manegem niuwen sper."\\ 
 & "des volge ich", sprach Artus.\\ 
 & diu herzogîn dô hin zuo \textbf{ir} hûs\\ 
 & sant nâch den werden.\\ 
20 & ich wæne, ûf der erden\\ 
 & \textbf{nie} schœner samenunge wart.\\ 
 & gên \textbf{herberge sîne\textit{r}} vart\\ 
 & Gawan urloubes gerte,\\ 
 & des in der künic \textbf{gewerte}.\\ 
25 & die man mit im komen sach,\\ 
 & vuoren dan mit im an \textbf{ir} gemach.\\ 
 & sîn herberge rîche\\ 
 & stuont sô ritterlîche,\\ 
 & daz si was kostebære\\ 
30 & \textbf{und} \textbf{der} armuot lære.\\ 
\end{tabular}
\scriptsize
\line(1,0){75} \newline
Q R W V \newline
\line(1,0){75} \newline
\textbf{9} \textit{Initiale} W  \newline
\line(1,0){75} \newline
\textbf{1} besuochet] gesuͦcht R (W) (V) \textbf{4} hât mit mir] hot mit Q mit mir hat R \textbf{5} erkant] bekant R W V \textbf{6} zuor blôzen] zuͦ blossen W zvͦ [blozen]: blozer V \textbf{8} wirt] wil R  $\cdot$ geendet] gedienet W \textbf{11} rittern] Ritter R \textbf{13} [D*]: Jch erwirbe ez wol do zer herzogin V  $\cdot$ dâ] do R W \textbf{14} ledic sollen] [*]: ledig sollent V \textbf{16} manegem niuwen] mengen núwen R [*]: manigem nvwen V \textbf{17} [D*]: Dez volge ich sprach artus V \textbf{18} dô] \textit{om.} R  $\cdot$ zuo] vff zu R \textbf{19} den] [dem]: den V \textbf{21} wart] were R \textbf{22} [Geg*]: Gegen siner herbergen vart V  $\cdot$ herberge] herbergen W  $\cdot$ sîner vart] seine vart Q one schwere R \textbf{23} Gawan] Gawin R Her gawan W \textbf{25} komen] [*]: e kvmen V \textbf{26} ir] sin R [*]: ir V \textbf{27} sîn] Seiner W \textbf{28} [D*]: Stunt so ritterliche V \textbf{29} [*]: Daz sú waz kostbere V \textbf{30} [*]: Vnd der armuͤte lere V  $\cdot$ der] er R \newline
\end{minipage}
\end{table}
\end{document}
