\documentclass[8pt,a4paper,notitlepage]{article}
\usepackage{fullpage}
\usepackage{ulem}
\usepackage{xltxtra}
\usepackage{datetime}
\renewcommand{\dateseparator}{.}
\dmyyyydate
\usepackage{fancyhdr}
\usepackage{ifthen}
\pagestyle{fancy}
\fancyhf{}
\renewcommand{\headrulewidth}{0pt}
\fancyfoot[L]{\ifthenelse{\value{page}=1}{\today, \currenttime{} Uhr}{}}
\begin{document}
\begin{table}[ht]
\begin{minipage}[t]{0.5\linewidth}
\small
\begin{center}*D
\end{center}
\begin{tabular}{rl}
\textbf{124} & Der knappe sprach durch sînen muot\\ 
 & \textbf{zem vürsten}: "war zuo ist diz guot,\\ 
 & daz \textbf{dich sô wol kan schicken}?\\ 
 & ine mag \textbf{es} niht ab gezwicken."\\ 
5 & der vürste \textbf{im zeigete} \textbf{sân} sîn swert:\\ 
 & "\textbf{nû} sich, swer \textbf{an mich strîtes} gert,\\ 
 & des selben wer ich mich mit slegen.\\ 
 & vür die sîne muoz ich \textbf{an mich} legen\\ 
 & \textbf{unt} vür den \textbf{schuz} unt vü\textit{r} den stich\\ 
10 & muoz ich alsus wâpen mich."\\ 
 & Aber sprach der knappe snel:\\ 
 & "ob die hirze \textbf{trüegen sus} ir vel,\\ 
 & sô\textbf{ne} verwunt ir niht mîn gabylôt,\\ 
 & der \textbf{vellet maneger} \textbf{von} mir tôt."\\ 
15 & \begin{large}D\end{large}ie ritter zurnden, daz er hielt\\ 
 & bî dem \textbf{knappen}, der vil tumpheite wielt.\\ 
 & der vürste sprach: "got hüete dîn.\\ 
 & owî, wan wære \textbf{dîn} schœne mîn!\\ 
 & dir hete got den wunsch gegeben,\\ 
20 & ob dû mit witzen soldest leben.\\ 
 & diu gotes kraft dir virre leit."\\ 
 & die sîne \textbf{unt} ouch er selbe reit\\ 
 & unt g\textit{â}hten \textbf{harte} balde\\ 
 & zeinem velde in dem walde.\\ 
25 & dâ \textbf{vant} der gevüege\\ 
 & vrou Herzeloyden pf\textit{l}üege.\\ 
 & ir volke leider nie geschach,\\ 
 & \textbf{die} er balde eren sach.\\ 
 & si begunden sæn, dar nâch egen,\\ 
30 & ob \textbf{starken} ohsen wegen.\\ 
\end{tabular}
\scriptsize
\line(1,0){75} \newline
D \newline
\line(1,0){75} \newline
\textbf{1} \textit{Majuskel} D  \textbf{11} \textit{Majuskel} D  \textbf{15} \textit{Initiale} D  \newline
\line(1,0){75} \newline
\textbf{9} vür] fv̂f D \textbf{23} gâhten] gæhten D \textbf{26} pflüege] phvͤge D \newline
\end{minipage}
\hspace{0.5cm}
\begin{minipage}[t]{0.5\linewidth}
\small
\begin{center}*m
\end{center}
\begin{tabular}{rl}
 & der knappe sprach durch sînen muot\\ 
 & \textbf{zem vürsten}: "war zuo ist diz guot,\\ 
 & daz \textbf{dich sô wol kan schicken}?\\ 
 & \textit{in}e mac \textbf{es} niht \textbf{her} abe gezwicken."\\ 
5 & der vürste \textbf{ime zougete} \textbf{s\textit{ân}} sîn swert:\\ 
 & "\textbf{nû} sich, wer \textbf{an mich strîtes} gert,\\ 
 & des selben wer ich mich mit slegen.\\ 
 & vür die s\textit{î}ne muoz ich \textbf{\textit{ane} mich} legen\\ 
 & \textbf{und} vür \textit{den} \textbf{schuz} und vür den stich\\ 
10 & muoz ich alsus wâpen mich."\\ 
 & aber sprach der knabe snel:\\ 
 & "ob die hi\textit{r}ze \textbf{trüege\textit{n} sus} ir vel,\\ 
 & sô verwundet ir niht mîn gabilôt,\\ 
 & der \textbf{vellet maniger} \textbf{vor} mi\textit{r} tôt."\\ 
15 & \begin{large}D\end{large}ie ritter zurnden, daz er hielt\\ 
 & bî de\textit{m}, \textit{d}er \textit{vil} tumpheit wielt.\\ 
 & der vürste sprach: "got hüete dîn.\\ 
 & owê, wanne wær \textbf{diu} schœne mîn!\\ 
 & dir hete got den wunsch gegeben,\\ 
20 & o\textit{b} dû mit witzen soltes leben.\\ 
 & diu gotes kraft dir verre leit."\\ 
 & die sîne \textbf{und} ouch er selbe reit\\ 
 & und gâheten \textbf{harte} balde\\ 
 & ze einem velde in dem walde.\\ 
25 & d\textit{â} \textbf{vant} der gevüege\\ 
 & vrouwe Herczeloiden pflüege.\\ 
 & ir volke leider nie geschach,\\ 
 & \textbf{daz} er \textbf{dâ} balde ern sach.\\ 
 & si begunden \dag sich\dag  dar nâch \textit{e}gen,\\ 
30 & \textbf{ir garte} ob \textbf{starken} ohsen wegen.\\ 
\end{tabular}
\scriptsize
\line(1,0){75} \newline
m n o \newline
\line(1,0){75} \newline
\textbf{15} \textit{Initiale} m n  \newline
\line(1,0){75} \newline
\textbf{1} sînen] >sinen< o \textbf{4} ine] Me m Jch n o \textbf{5} zougete] zeigete n  $\cdot$ sân] so m n o \textbf{8} sîne] suͯne m sinne o  $\cdot$ ane] \textit{om.} m \textbf{9} und vür den schuz] Vnd fur schucz m Vnd von den schosz n Vor vnd vor den schos o  $\cdot$ vür den stich] von den stich n o \textbf{12} hirze trüegen] hicz truge m hicze tragen n hirz tragen o \textbf{13} verwundet] [werwindet]: verwindet o \textbf{14} mir] mitte m \textbf{16} bî dem der vil] Bidem das er m \textbf{19} dir] Die n o  $\cdot$ gegeben] geben n o \textbf{20} ob] O m  $\cdot$ soltes] woltest n \textbf{21} verre leit] selber leit ferre o \textbf{22} und ouch er] er ouch n  $\cdot$ selbe reit] reit selbe o \textbf{23} gâheten] gehetten o \textbf{24} velde in] falde [ein]: in o \textbf{25} dâ] Do m n o \textbf{26} Herczeloiden] hertzeloiden n herczeleiden o \textbf{28} dâ] do n o \textbf{29} egen] oͯgen m \textbf{30} ob starken] vnd starcke o \newline
\end{minipage}
\end{table}
\newpage
\begin{table}[ht]
\begin{minipage}[t]{0.5\linewidth}
\small
\begin{center}*G
\end{center}
\begin{tabular}{rl}
 & der knappe sprach durch sînen muot:\\ 
 & "\textbf{jâ, hêrre}, war zuo ist diz guot,\\ 
 & daz \textbf{dich sus wol kan schicken}?\\ 
 & ichne mag \textbf{es} niht abe gezwicken."\\ 
5 & der vürste \textbf{zeigete im} \textbf{sân} sîn swert:\\ 
 & "\textbf{nû} sich, swer \textbf{strîtes an mich} gert,\\ 
 & des selben wer ich mich mit slegen.\\ 
 & vür die sîne muoz ich \textbf{mich an} legen.\\ 
 & vür den \textbf{schuz} und vür den stich\\ 
10 & muoz ich alsus wâpenen mich."\\ 
 & aber sprach der knappe snel:\\ 
 & "obe die hirze \textbf{trüegen sus} ir vel,\\ 
 & sô\textbf{ne} verwunt ir niht mîn gabilôt,\\ 
 & der \textbf{lît vil maniger} \textbf{vor} mir tôt."\\ 
15 & die rîter zurnden, daz er hielt\\ 
 & bî dem \textbf{knappen}, der vil tumpheit wielt.\\ 
 & der vürste sprach: "got hüete dîn.\\ 
 & owê, wan wære \textbf{dîn} schœne mîn!\\ 
 & dir hete got den wunsch gegeben,\\ 
20 & obe dû \textit{m}i\textit{t} witzen soldest leben.\\ 
 & diu gotes kraft dir verre leit."\\ 
 & die sîne \textbf{und} ouch er selbe reit\\ 
 & unde gâhten \textbf{dannen} balde\\ 
 & zeinem velde in dem walde.\\ 
25 & dâ \textbf{sach} der gevüege\\ 
 & vrôn Herzeloide pflüege.\\ 
 & ir volke leider nie geschach,\\ 
 & \textbf{die} er \textbf{vil} balde eren sach.\\ 
 & si begunden sæn \textbf{und} dar nâch egen,\\ 
30 & \textbf{ir gart} obe \textbf{starken} ohsen wegen.\\ 
\end{tabular}
\scriptsize
\line(1,0){75} \newline
G I O L M Q R Z Fr36 \newline
\line(1,0){75} \newline
\textbf{1} \textit{Initiale} O  \textbf{5} \textit{Initiale} I  \textbf{15} \textit{Initiale} I L Q R Z  \newline
\line(1,0){75} \newline
\textbf{1} der] ÷er O \textbf{2} diz] daz I (M) (R) \textbf{3} dich] sich L  $\cdot$ sus] so O L M Q Fr36  $\cdot$ kan] mag R  $\cdot$ schicken] sichen I geschicken L \textbf{4} Jch mag es hie herre alles gewúrken R  $\cdot$ ichne] ich I (O) Fr36  $\cdot$ mag es] mach sin I chans O (M) (Q) (Fr36)  $\cdot$ abe] her ab I (Q)  $\cdot$ gezwicken] zwichen O (Q) (Fr36) \textbf{5} Der furst im san zeigt sin swert Z  $\cdot$ zeigete] zaiget I (O) (L) (Q) (R) (Fr36)  $\cdot$ sân] \textit{om.} I O M R \textbf{6} swer] wer L M Q R Z  $\cdot$ mich] myr M \textbf{8} vür] Vnde fur M  $\cdot$ die] \textit{om.} Q Fr36  $\cdot$ sîne] sinen I (Q) vigent R  $\cdot$ mich an] an mich I O L (M) Z mich Q an R \textbf{9} vür den] Vnde fvͤr den O (L) (M) (Z)  $\cdot$ und vür] vnde M (R) \textbf{10} ich] \textit{om.} O mich M  $\cdot$ wâpenen] gewapenen L \textbf{12} Truͯgen die hirtze sus ir vel L  $\cdot$ hirze] hirsen Q  $\cdot$ trüegen sus] sus tragen I svs trvͦgen O (M) trugen Q trvgen sus Z \textbf{13} sône] So I O L Q R Z  $\cdot$ verwunt] verwindet M  $\cdot$ ir] sy R (Z) \textbf{14} der] ir I  $\cdot$ lît vil maniger] vellet manger O L (M) (Z) vellet manges Q vellett R  $\cdot$ vor] von I (Q) R  $\cdot$ mir] ym M mir mengig R \textbf{15} die] ÷ie I Der M  $\cdot$ zurnden] zcurnite M zúrnent R  $\cdot$ er] der R \textbf{16} dem] den M  $\cdot$ knappen] \textit{om.} I  $\cdot$ der] \textit{om.} R  $\cdot$ vil] \textit{om.} Z  $\cdot$ tumpheit] tvmbe O torheit L tumbliche M  $\cdot$ wielt] hiet O \textbf{17} vürste] \textit{om.} L  $\cdot$ hüete] behuͯt R \textbf{18} owê] Awe O Owý L (M) (Q)  $\cdot$ wan] \textit{om.} R  $\cdot$ dîn] die Q \textbf{19} hete] hat L M R  $\cdot$ gegeben] geben O \textbf{20} mit] bi G  $\cdot$ witzen] wisheit M \textbf{21} dir] die L (M) \textbf{22} die sîne] din sin I Die sinen L (Q)  $\cdot$ und] \textit{om.} R  $\cdot$ ouch er] er L och die er R ovch der Z  $\cdot$ selbe] selbern M selber R  $\cdot$ reit] treit R \textbf{23} gâhten] Gahent I ilten M  $\cdot$ dannen] alle O \textbf{24} zeinem] Zcu deme M  $\cdot$ in dem] inden R \textbf{25} dâ] Do Q  $\cdot$ sach] ersach I vant Z \textbf{26} vrôn] Frov O (L) (Q) (R)  $\cdot$ Herzeloide] herzelaude I herzenlavden O Hertzelauͯden L herloiden M herzelouden Q herczelaudes R herzelovden Z \textbf{27} volke leider] soͯlich leide R  $\cdot$ geschach] gesach Q \textbf{28} die] daz I \textbf{29} sæn und] san L \textbf{30} obe] uff M  $\cdot$ starken] starc I \newline
\end{minipage}
\hspace{0.5cm}
\begin{minipage}[t]{0.5\linewidth}
\small
\begin{center}*T (U)
\end{center}
\begin{tabular}{rl}
 & der knappe sprach durch sînen muot:\\ 
 & "\textbf{jâ, hêrre}, war zuo ist diz guot,\\ 
 & daz \textbf{dû ez sus kanst zesamen \textit{geschicken}}?\\ 
 & ine mag \textbf{ez} niht abe gezwicken."\\ 
5 & \begin{large}D\end{large}er vürste \textbf{zeigetim} sîn swert:\\ 
 & "sich, wer \textbf{strîtes an mich} gert,\\ 
 & des selben wer ich mich mit slegen.\\ 
 & vür die sîne muoz ich \textbf{an mich} legen.\\ 
 & vür den \textbf{slac} und vür den stich\\ 
10 & muoz ich alsus wâpenen mich."\\ 
 & aber sprach der knappe snel:\\ 
 & "ob die hirze \textbf{sus trüegen} ir vel,\\ 
 & sô verwundet ir niht mîn gabilôt,\\ 
 & der \textbf{vellet manege\textit{r}} \textbf{vor} mir tôt."\\ 
15 & die rîter zurnten, daz er hielt\\ 
 & bî dem, der vil tumpheit wielt.\\ 
 & der vürste sprach: "got hüete dîn.\\ 
 & owê, wan wære \textbf{dîn} schœne mîn!\\ 
 & dir hât got den wunsch gegeben,\\ 
20 & ob dû mit witzen soltes leben.\\ 
 & diu gotes kraft dir verre leit."\\ 
 & die sîne, ouch er selbe \textit{r}eit\\ 
 & und gâheten balde\\ 
 & zuo eime velde in dem walde.\\ 
25 & dâ \textbf{sach} der gevüege\\ 
 & vrôn Herzeloyde pflüege.\\ 
 & ir volke leide\textit{r} nie geschach,\\ 
 & \textbf{d\textit{i}e}r balde eren sach.\\ 
 & si begunden sæwen \textbf{und} dâ nâch egen,\\ 
30 & \textbf{ir garten} ob \textbf{den} ohsen wegen.\\ 
\end{tabular}
\scriptsize
\line(1,0){75} \newline
U V W T \newline
\line(1,0){75} \newline
\textbf{1} \textit{Majuskel} T  \textbf{5} \textit{Initiale} U V   $\cdot$ \textit{Majuskel} T  \textbf{11} \textit{Majuskel} T  \textbf{15} \textit{Initiale} W T  \newline
\line(1,0){75} \newline
\textbf{1} der knappe sprach] Sprach der knappe san W \textbf{2} diz] das W \textbf{3} dû ez sus kanst zesamen] du dich suß kanst wol W dich svs wol kan T  $\cdot$ geschicken] \textit{om.} U zwicken W schicken T \textbf{4} Vnd so honestlich schicken W \textbf{5} vürste] riter T  $\cdot$ zeigetim] zoͤiget im V zaiget im W  $\cdot$ sîn] sam sin V (W) sa sin T \textbf{6} sich] nv sich T  $\cdot$ wer] swer V (T) \textbf{9} vür den slac] Eúr den schutz W vnde vur den schvtz T \textbf{12} hirze sus trüegen ir] hirze svs trvͤgen V hirtzen truͤgen ob ir W hirze truegen svs ir T \textbf{13} sô verwundet ir] Si verwundete V \textbf{14} maneger] manegen U (W) [manigen]: maniger  V \textbf{16} dem] dem knappen W (T) \textbf{21} dir verre] dir [verre]: verret V ir kunst an dich W \textbf{22} [D*]: Er vnde die sine dannan reit V · Zuͦhant er mit im dannen rait W  $\cdot$ ouch er] vnde er ôuch T  $\cdot$ reit] treit U [*]: reit T \textbf{23} balde] harte balde V fúrbas balde W dannen balde T \textbf{24} zuo eime] Auff dem W  $\cdot$ in] [i*]: vor V vor W \textbf{25} dâ] Do V W \textbf{26} vrôn] Vro U  $\cdot$ Herzeloyde] Herzeleide U herzelauden V hertzeloyde W herzeloyden T \textbf{27} Ir volck laider nie gesach W  $\cdot$ leider] leide U  $\cdot$ geschach] beschach V \textbf{28} dier] der U Die er do W dier vil T \textbf{29} und] \textit{om.} W \textbf{30} den] starcken W den starken T \newline
\end{minipage}
\end{table}
\end{document}
