\documentclass[8pt,a4paper,notitlepage]{article}
\usepackage{fullpage}
\usepackage{ulem}
\usepackage{xltxtra}
\usepackage{datetime}
\renewcommand{\dateseparator}{.}
\dmyyyydate
\usepackage{fancyhdr}
\usepackage{ifthen}
\pagestyle{fancy}
\fancyhf{}
\renewcommand{\headrulewidth}{0pt}
\fancyfoot[L]{\ifthenelse{\value{page}=1}{\today, \currenttime{} Uhr}{}}
\begin{document}
\begin{table}[ht]
\begin{minipage}[t]{0.5\linewidth}
\small
\begin{center}*D
\end{center}
\begin{tabular}{rl}
\textbf{271} & \begin{large}D\end{large}az ich si \textbf{hulde mîn} verstiez.\\ 
 & dô ich die \textbf{süezen} eine liez,\\ 
 & waz mohte si, swaz ir \textbf{geschach}?\\ 
 & dô si aber von dîner schœne sprach,\\ 
5 & ich wânde, dâ wære \textbf{ein} vriwentschaft bî.\\ 
 & nû lône dir got, s\textit{i} ist valsches vrî.\\ 
 & I\textit{ch} hân \textbf{unvuoge} an ir getân.\\ 
 & \textbf{vür}z fôrest in Brizljan\\ 
 & reit ich \textbf{dô in juven poys}."\\ 
10 & Parzival \textbf{diz} sper von Troys\\ 
 & nam unt vuortez mit im dan.\\ 
 & \textbf{des} vergaz der wilde Taurian,\\ 
 & \textbf{Dodines} bruoder, dâ.\\ 
 & nû sprechet, wie oder wâ\\ 
15 & die helde des nahtes megen sîn.\\ 
 & helme unt \textbf{ir} schilde heten pîn;\\ 
 & die sach man gar \textbf{verhouwen}.\\ 
 & Parzival zer vrouwen\\ 
 & nam urloup unt zir âmîs.\\ 
20 & dô ladete in der vürste wîs\\ 
 & \textbf{mit im} an sîne viwerstat.\\ 
 & \textbf{daz} half \textbf{in} niht, swie vil er\textbf{s} bat.\\ 
 & \textbf{al dâ} schieden die helde sich.\\ 
 & diu âventiure wert \textbf{mære} mich:\\ 
25 & Dô Orilus, der vürste erkant,\\ 
 & kom, dâ er sîne poulûn vant\\ 
 & unt sîner messenîe ein teil,\\ 
 & daz volc \textbf{was} al gelîche geil,\\ 
 & daz suone was worden schîn\\ 
30 & gein der \textbf{sældebernden} herzogîn.\\ 
\end{tabular}
\scriptsize
\line(1,0){75} \newline
D \newline
\line(1,0){75} \newline
\textbf{1} \textit{Initiale} D  \textbf{7} \textit{Majuskel} D  \textbf{25} \textit{Majuskel} D  \newline
\line(1,0){75} \newline
\textbf{6} si] so D \textbf{7} Ich] Jn D \textbf{8} Brizljan] Prizlian D \textbf{12} Taurian] Tavrian D \newline
\end{minipage}
\hspace{0.5cm}
\begin{minipage}[t]{0.5\linewidth}
\small
\begin{center}*m
\end{center}
\begin{tabular}{rl}
 & daz ich si \textbf{hulde mî\textit{n}} verstiez.\\ 
 & dô ich die \textbf{süezen} eine liez,\\ 
 & waz mohte si, waz \textit{ir} \textbf{\textit{b}esch\textit{ach}}?\\ 
 & dô si aber von dîner schœne sprach,\\ 
5 & ich wânde, dâ wære \textbf{ein} v\textit{ri}u\textit{nt}schaft bî.\\ 
 & nû lôn dir got, si ist valsches vrî.\\ 
 & ich hân \textbf{ungevüege} an ir getân.\\ 
 & \textbf{vü\textit{r}}z fôrest in Pricil\textit{a}n\\ 
 & reit ich \textbf{dô in den Jovanpois}."\\ 
10 & Parcifal \textbf{ein} sper von Trois\\ 
 & nam und vuorte ez mit im dan.\\ 
 & \textbf{des} vergaz der wilde Taurian,\\ 
 & \textbf{Dodenas} bruode\textit{r}, \textit{d}â.\\ 
 & nû sprechet, wie oder wâ\\ 
15 & die helde des nahtes mügen sîn.\\ 
 & helme und \textbf{ir} schilt\textit{e} heten pîn;\\ 
 & die sach \textit{ma}n gar \textbf{verhouwen}.\\ 
 & Parcifal zer vrouwen\\ 
 & nam urloup und zuo ir âmîs.\\ 
20 & dô ladete in der vürste wîs\\ 
 & \textbf{mit ime} an sîne viurstat.\\ 
 & \textbf{daz} half \textbf{in} niht, wie vil er\textbf{s} bat.\\ 
 & \textbf{aldâ} schieden die helde sich.\\ 
 & diu âventiure wert \textbf{mære} mich:\\ 
25 & \begin{large}D\end{large}ô Orilus, der vürste erkant,\\ 
 & kam, d\textit{â} er sîne pavelûne vant\\ 
 & und sîner massenîe ein teil,\\ 
 & daz \textit{v}ol\textit{c} \textbf{wart} a\textit{l} gelîche geil,\\ 
 & daz suone \textbf{dô} was worden schîn\\ 
30 & gegen der \textbf{sældebæren} herzogîn.\\ 
\end{tabular}
\scriptsize
\line(1,0){75} \newline
m n o Fr69 \newline
\line(1,0){75} \newline
\textbf{25} \textit{Initiale} m Fr69   $\cdot$ \textit{Capitulumzeichen} n  \newline
\line(1,0){75} \newline
\textbf{1} si] die o  $\cdot$ mîn] mit m \textbf{2} süezen] suͯsse n (o) \textbf{3} mohte si] moͯchte n  $\cdot$ ir beschach] sẏ [*]: geschuͦff m \textbf{5} vriuntschaft] furschaft m \textbf{7} ungevüege] vnfuͦge n \textbf{8} vürz] Fus m  $\cdot$ Pricilan] pricilon m \textbf{9} den] \textit{om.} o  $\cdot$ Jovanpois] iovanpois m [j*]: Jovanpeis n iopansopis o \textbf{12} des] Dasz o  $\cdot$ Taurian] tavrian m tarrian n torian o \textbf{13} Dodenas] Do denas m Do denes n Todenes o  $\cdot$ bruoder dâ] bruder pfalag da m do n o \textbf{14} nû sprechet] Sú sprochent n \textbf{15} mügen] moͯchten n mochten o \textbf{16} schilte] schilttes m \textbf{17} sach man] sachen m \textbf{19} ir] sin n o \textbf{20} ladete] ladet n o Fr69 \textbf{23} schieden] schieden sich o \textbf{24} wert] weret n werent o  $\cdot$ mære] me n o \textbf{25} Orilus] vrelus o \textbf{26} dâ] do m n o  $\cdot$ sîne] \textit{om.} o \textbf{28} volc] wol m  $\cdot$ al gelîche] alle geliche m al glich n (o) \textbf{29} suone dô was worden] sone das do was [sune]: worden o \textbf{30} sældebæren] schoͯnen n o \newline
\end{minipage}
\end{table}
\newpage
\begin{table}[ht]
\begin{minipage}[t]{0.5\linewidth}
\small
\begin{center}*G
\end{center}
\begin{tabular}{rl}
 & daz ich si \textbf{suone mîn} verstiez.\\ 
 & dô ich die \textbf{guoten} eine liez,\\ 
 & waz mohte si, swaz ir \textbf{geschach}?\\ 
 & dô si aber von dîner schœne sprach,\\ 
5 & ich wânde, dâ wære vriuntschaft bî.\\ 
 & nû lône dir got, sist valsches vrî.\\ 
 & ich hân \textbf{ungevüege} an ir getân.\\ 
 & \textbf{durch} daz fôreist in Brizilan\\ 
 & reit ich \textbf{von ir alsô von Poys}."\\ 
10 & Parzival \textbf{daz} sper von Troys\\ 
 & nam unde vuo\textit{r}tez mit im dan.\\ 
 & \textbf{es} vergaz der wilde Thaurian,\\ 
 & \textbf{Toclines} bruoder, dâ.\\ 
 & nû sprecht, wie oder wâ\\ 
15 & die helde des nahtes mugen sîn.\\ 
 & helm unde \textbf{ir} schilte heten pîn;\\ 
 & die sach man gar \textbf{zerhouwen}.\\ 
 & Parzival zer vrouwen\\ 
 & \begin{large}N\end{large}am urloup unde zir âmîs.\\ 
20 & dô ladete in der vürste wîs\\ 
 & \textbf{vil ofte} an sîne viurstat.\\ 
 & \textbf{ez} half \textbf{in} niht, swie vil er\textbf{s} bat.\\ 
 & \textbf{dô} schieden die helde sich.\\ 
 & \textit{d}iu âventiure \textit{wert} mich:\\ 
25 & dô Orillus, der vürste erkant,\\ 
 & kom, dâ er sîn pavelûn vant\\ 
 & unde sîner messenîe ein teil,\\ 
 & daz volc \textbf{was} al gelîche geil,\\ 
 & daz suone was worden schîn\\ 
30 & gein der \textbf{sældebernden} herzogîn.\\ 
\end{tabular}
\scriptsize
\line(1,0){75} \newline
G I O L M Q R Z \newline
\line(1,0){75} \newline
\textbf{19} \textit{Initiale} G I  \textbf{25} \textit{Initiale} O L R Z  \newline
\line(1,0){75} \newline
\textbf{1} suone mîn] der suͤne min I miner svͦn O (L) svne mit M minner hulde Q (R) der hulde min Z \textbf{2} dô] Da M Z  $\cdot$ eine] \textit{om.} O enig ligen R einen Z \textbf{3} si] \textit{om.} Z  $\cdot$ swaz] daz O (M) waz L (Q) (R) Z \textbf{4} dô si aber] Do aber si O (L) (Q) (R) Da si abir M (Z) \textbf{5} vriuntschaft] ein frivntschaft O (M) (Q) (R) (Z) \textbf{6} sist] sis G svst Z  $\cdot$ valsches] wandels O \textbf{8} durch daz] Dvrch O (L) (R) Fur des Q Fvrz Z  $\cdot$ fôreist] frecheit L fur ich M  $\cdot$ in Brizilan] inbrizilian I in Brezilian O L in Brecilian M in brezzilian Q (Z) in Brezilon R \textbf{9} ich] \textit{om.} I  $\cdot$ von] vor I do O L Q R da M Z  $\cdot$ ir alsô von Poys] in manie von poys O (M) Jmmanie von poýs L in dem in nam poys Q in dem Iuuanpois R in dem iovan pois Z \textbf{10} Parzival] [parzifal]: Parzifal I Parcifal O Z Parzifal L M Partzifal Q Parczifal R  $\cdot$ daz] diz O (L) (Q) (R) (Z)  $\cdot$ Troys] Troýs L trois M R Z \textbf{11} nam] nam ez I  $\cdot$ vuortez] foͮtez G fvͦrt itz O  $\cdot$ mit im] von R \textbf{12} es] sin I Ez O M Er Q Des Z  $\cdot$ Thaurian] turian G tharian I Toyrian O Taurian L (Q) R torÿan M \textbf{13} Toclines] Tochelius I Toclicies O Do deines Q Do dines R Todines Z \textbf{14} oder] [der]: oder G \textbf{15} mugen] mohten Z \textbf{16} ir] \textit{om.} I L  $\cdot$ heten] die hatten L \textbf{17} die] Do Q  $\cdot$ gar] \textit{om.} L \textbf{18} \textit{Die Verse 271.18-272.21 fehlen} Q   $\cdot$ Parzival] [parzifal]: Parzifal I Parcifal O Z Parzifal L M Parczifal R \textbf{20} dô ladete] doladet I (O) (R) Da latte L Da ladite M Da ladet Z \textbf{21} vil ofte] Vil dicke M Mit im Z \textbf{22} ez] ezn I (O) (L) (M) (Z)  $\cdot$ in] \textit{om.} L  $\cdot$ swie] wie L (M) R Z  $\cdot$ vil] oft I  $\cdot$ ers] er I er dez L er sy M er sin Z \textbf{23} dô] Da O M Z \textbf{24} sus wert div auenture mich G  $\cdot$ wert] werte I wert mer O (L) (M) (R) (Z) \textbf{25} dô] ÷o O Da M Z  $\cdot$ Orillus] Orilus I (O) M (R) (Z) \textbf{26} pavelûn] paielune I \textbf{28} al gelîche] geliche O L (M) alles R \textbf{29} \textit{Vers 271.29 fehlt} R   $\cdot$ was] was da I \textbf{30} sældebernden] seldebarn I (L) (R) \newline
\end{minipage}
\hspace{0.5cm}
\begin{minipage}[t]{0.5\linewidth}
\small
\begin{center}*T
\end{center}
\begin{tabular}{rl}
 & daz ich si \textbf{mîner hulde} verstiez.\\ 
 & dô ich di\textit{e} \textbf{guoten} eine liez,\\ 
 & waz mohte si, swaz ir \textbf{geschach}?\\ 
 & dô si aber von dîner schœne sprach,\\ 
5 & ich wânde, dâ wære \textbf{ein} vriuntschaft bî.\\ 
 & nû lône dir got, sist valsches vrî.\\ 
 & ich hân \textbf{unvuoge} an ir getân.\\ 
 & \textbf{durch} daz fôreht in Prezilian\\ 
 & reit ich \textbf{vor ir alsô von Poys}."\\ 
10 & Parcifal \textbf{diz} sper von Troys\\ 
 & nam unde vuortez mit im dan.\\ 
 & \textbf{des} vergaz der wilde Taurian,\\ 
 & \textbf{Dodines} bruoder, dâ.\\ 
 & Nû sprechet, \textit{wie} oder wâ\\ 
15 & die helde des nahtes mugen sîn.\\ 
 & helme unde schilte heten pîn;\\ 
 & die sach man gar \textbf{verhouwen}.\\ 
 & Parcifal zer vrouwen\\ 
 & nam urloup unde zir âmîs.\\ 
20 & dô ladete in der vürste wîs\\ 
 & \textbf{vil} an sîne viurstat.\\ 
 & \textbf{ez} \textbf{en}half niht, swie vil er bat.\\ 
 & \textbf{sus} schieden die helde sich.\\ 
 & Diu âventiure wert \textbf{mære} mich:\\ 
25 & dô Orilus, der vürste erkant,\\ 
 & kom, dâ er sîn pavelûn vant\\ 
 & unde sîner massenîe ein teil,\\ 
 & daz volc \textbf{was} al glîche geil,\\ 
 & daz suone was worden schîn\\ 
30 & gegen der \textbf{sældebæren} herzogîn.\\ 
\end{tabular}
\scriptsize
\line(1,0){75} \newline
T U V W \newline
\line(1,0){75} \newline
\textbf{14} \textit{Majuskel} T  \textbf{24} \textit{Majuskel} T  \textbf{25} \textit{Initiale} U W  \newline
\line(1,0){75} \newline
\textbf{1} mîner hulde] der hulde mein W \textbf{2} die] div T  $\cdot$ guoten] guͦte U \textbf{3} swaz] waz U des was W \textbf{5} vriuntschaft] [*]: frúntschaft V \textbf{6} sist] sis duͦ U  $\cdot$ valsches vrî] [*]: valschez vri V \textbf{7} unvuoge an ir] vngevuͦge an dir U \textbf{8} durch daz] Fúrz V  $\cdot$ fôreht] vuͦrte U  $\cdot$ Prezilian] Precilian U (V) \textbf{9} vor ir alsô von Poys] do von pontertoys U [*]: do in den iovanpoẏs V do inna von proys W \textbf{10} Parcifal] Parzifal T V Partzifal W  $\cdot$ diz] [*]: ein V \textbf{12} des] Er W  $\cdot$ Taurian] Tavrian T (V) torian W \textbf{13} Dodines] Der dines U  $\cdot$ dâ] do W \textbf{14} wie] \textit{om.} T \textbf{16} unde] [*]: vnde ir V  $\cdot$ heten] liden U \textbf{17} die] Do U \textbf{18} Parcifal] Parzifal T V Partzfal W  $\cdot$ zer] zuͦ U \textbf{20} ladete] luͦt U \textbf{21} vil an] Vil in U [*]: Mit im an V  $\cdot$ viurstat] vorstat U \textbf{22} ez enhalf] [*]: Daz enhalf V Es halff W  $\cdot$ swie] wie U V W  $\cdot$ er bat] [*]: ers bat V \textbf{23} sus] [*]: al da V \textbf{24} wert] werte U  $\cdot$ mich] sich W \textbf{25} erkant] erkante U \textbf{26} dâ] do U V W  $\cdot$ sîn] [si*]: sine V  $\cdot$ pavelûn] wapen U pauelune V \textbf{28} was] [*]: wart V  $\cdot$ al glîche] engeslichen U alle gleich W \textbf{29} suone] [*]: svͦne T in U svͦne [*]: do V die suͦne W \textbf{30} sældebæren] [seldenber*]: seldenberen V seldenreichen W \newline
\end{minipage}
\end{table}
\end{document}
