\documentclass[8pt,a4paper,notitlepage]{article}
\usepackage{fullpage}
\usepackage{ulem}
\usepackage{xltxtra}
\usepackage{datetime}
\renewcommand{\dateseparator}{.}
\dmyyyydate
\usepackage{fancyhdr}
\usepackage{ifthen}
\pagestyle{fancy}
\fancyhf{}
\renewcommand{\headrulewidth}{0pt}
\fancyfoot[L]{\ifthenelse{\value{page}=1}{\today, \currenttime{} Uhr}{}}
\begin{document}
\begin{table}[ht]
\begin{minipage}[t]{0.5\linewidth}
\small
\begin{center}*D
\end{center}
\begin{tabular}{rl}
\textbf{451} & \begin{large}E\end{large}r neic unt die andern nigen.\\ 
 & dâ wart ir klage niht verswigen.\\ 
 & hin \textbf{rîtet} Herzeloyden vruht.\\ 
 & dem riet sîn manlîchiu zuht\\ 
5 & kiusche unt erbarmunge.\\ 
 & sît Herzeloyde, diu junge,\\ 
 & in het ûf geerbet triwe,\\ 
 & sich huop sînes herzen riwe.\\ 
 & alrêst er dô \textbf{gedâhte},\\ 
10 & wer al die werlt volbrâhte,\\ 
 & an sînen schepfære,\\ 
 & wie gewaltec der wære.\\ 
 & Er sprach: "\textbf{waz}, ob got helfe pfligt,\\ 
 & diu mînem \textbf{trûren} an gesigt?\\ 
15 & wart \textbf{aber er} ie rîter holt,\\ 
 & gediende ie rîter sînen solt\\ 
 & ode mac schilt \textbf{unt} swert\\ 
 & sîner helfe sîn sô wert\\ 
 & unt \textbf{rehtiu manlîchiu} wer,\\ 
20 & daz sîn helfe mich \textbf{vor} sorgen ner,\\ 
 & ist hiute sîn \textbf{helflîcher} tac,\\ 
 & sô helfe er, ob er helfen mac."\\ 
 & \textbf{er kêrte sich wider}, dannen er \textbf{dâ} reit.\\ 
 & si stuonden dannoch, den was leit,\\ 
25 & daz er von in kêrte.\\ 
 & ir triwe si daz lêrte.\\ 
 & die juncvrouwen \textbf{im sâhen} nâch,\\ 
 & gein den ouch \textbf{im} sîn herze jach,\\ 
 & daz er si gerne sæhe,\\ 
30 & wand ir blic in schœne jæhe.\\ 
\end{tabular}
\scriptsize
\line(1,0){75} \newline
D Fr5 \newline
\line(1,0){75} \newline
\textbf{1} \textit{Initiale} D Fr5  \textbf{13} \textit{Majuskel} D  \newline
\line(1,0){75} \newline
\textbf{2} dâ] Do Fr5 \textbf{3} rîtet] reit Fr5  $\cdot$ Herzeloyden] herzelaudin Fr5 \textbf{6} Herzeloyde] herzelaude Fr5 \textbf{9} gedâhte] dahte Fr5 \textbf{16} gediende] Gidienit Fr5 \textbf{17} schilt unt] gischilt oder Fr5 \textbf{19} rehtiu] reht Fr5 \textbf{20} vor] von Fr5 \textbf{23} wider] umbe Fr5  $\cdot$ dâ] \textit{om.} Fr5 \textbf{27} im sâhen] sahin im Fr5 \textbf{28} im] \textit{om.} Fr5 \textbf{30} in schœne jæhe] der was wæhe Fr5 \newline
\end{minipage}
\hspace{0.5cm}
\begin{minipage}[t]{0.5\linewidth}
\small
\begin{center}*m
\end{center}
\begin{tabular}{rl}
 & er neic und die andern nigen.\\ 
 & dô wart ir klage niht verswigen.\\ 
 & \begin{large}H\end{large}in \textbf{rîtet} Herczeloiden vruht.\\ 
 & dem r\textit{ie}t sîn manlîchiu zuht\\ 
5 & kiusche und erbarmunge.\\ 
 & sît Herczeloide, diu junge,\\ 
 & in hete ûf geerbet triuwe,\\ 
 & sich huop sînes herzen riuwe.\\ 
 & allerêrst er dô \textbf{gedâhte},\\ 
10 & wer alle die werlt volbrâhte,\\ 
 & \hspace*{-.7em}\big| \textit{wi}e gewaltic der wære,\\ 
 & \hspace*{-.7em}\big| an sînen schepfære.\\ 
 & er sprach: "\textbf{waz}, ob got helfe pfliget,\\ 
 & diu \dag müede\dag  \textbf{triuwe} an gesiget?\\ 
15 & wart \textbf{aber er} ie ritter holt,\\ 
 & gediend\textit{e} ie ritter sînen solt\\ 
 & oder mac schilt \textbf{oder} swert\\ 
 & sîner helfe sîn sô wert\\ 
 & und \textbf{reht manlîchiu} wer,\\ 
20 & daz sîn helfe mich \textbf{vor} sorgen ner,\\ 
 & ist hiute sîn \textbf{helfe rîcher} tac,\\ 
 & sô helf er, ob er helfen mac."\\ 
 & \textbf{sus kêrte er wider}, dannen er \textbf{dô} reit.\\ 
 & si stuonden dennoch, den was leit,\\ 
25 & daz er von in kêrte.\\ 
 & ir triuwe si daz lêrte.\\ 
 & die juncvrouwen \textbf{im sâhen} nâch,\\ 
 & gegen den ouch \textbf{im} sîn herze jach,\\ 
 & daz er si gerne sæhe,\\ 
30 & wan \textit{i}r b\textit{l}ic in schœne jæhe.\\ 
\end{tabular}
\scriptsize
\line(1,0){75} \newline
m n o \newline
\line(1,0){75} \newline
\textbf{3} \textit{Initiale} m   $\cdot$ \textit{Capitulumzeichen} n  \newline
\line(1,0){75} \newline
\textbf{1} andern nigen] ander neigen o \textbf{2} Do wart ie clagte nit gesweigen o \textbf{3} rîtet] ritten o  $\cdot$ Herczeloiden] hertzoloiden n herczeleide o \textbf{4} riet] reit m riette o  $\cdot$ zuht] zuͯhte o \textbf{6} Herczeloide] hertzoloide n herczeleide o  $\cdot$ diu] de o \textbf{12} \textit{Versfolge 451.11-12} n   $\cdot$ wie] Jme m \textbf{11} schepfære] schoppehere o \textbf{14} müede] miner n o \textbf{16} gediende] Gedienden m \textbf{19} manlîchiu] manlicher n \textbf{21} helfe rîcher] helffeclicher o \textbf{23} wider] do wider n  $\cdot$ er] der o \textbf{24} was] wasz doch o \textbf{26} triuwe] getruͯwe o \textbf{28} ouch im] jme ouch n \textbf{30} ir blic] er bick m \newline
\end{minipage}
\end{table}
\newpage
\begin{table}[ht]
\begin{minipage}[t]{0.5\linewidth}
\small
\begin{center}*G
\end{center}
\begin{tabular}{rl}
 & \begin{large}E\end{large}r neic unde die andern nigen.\\ 
 & dô wart ir klage niht verswigen.\\ 
 & hin \textbf{reit} Herzeloide vruht.\\ 
 & dem riet sîn manlîchiu zuht\\ 
5 & kiusche und erbarmunge.\\ 
 & sît Herzeloide, diu junge,\\ 
 & in het ûf geerbet triuwe,\\ 
 & sich huop sînes herzen riuwe.\\ 
 & alrêrste er dô \textbf{dâhte},\\ 
10 & wer al die werlt volbrâhte,\\ 
 & an sînen schepfære,\\ 
 & wie gewaltic der wære.\\ 
 & er sprach: "\textbf{waz}, ob got helfe pfliget,\\ 
 & diu mînem \textbf{trûren} an gesiget?\\ 
15 & wart \textbf{er aber} ie rîter holt,\\ 
 & gedient ie rîter sînen solt\\ 
 & ode mac schilt \textbf{ode} swert\\ 
 & sîner helfe sîn sô wert\\ 
 & unde \textbf{rehtiu manlîchiu} wer,\\ 
20 & daz sîn helfe mich \textbf{vor} sorgen ner,\\ 
 & ist hiute sîn \textbf{helfeclîcher} tac,\\ 
 & sô helfe er, ob er helfen mac."\\ 
 & \textbf{er kêrte sich wider}, danne er reit.\\ 
 & si stuonden dannoch, den was leit,\\ 
25 & daz er von in kêrte.\\ 
 & ir triuwe si daz lêrte.\\ 
 & die juncvrouwen \textbf{im sâhen} nâch,\\ 
 & gên den ouch \textbf{im} sîn herze jach,\\ 
 & daz er si gerne sæhe,\\ 
30 & wande ir blic in schœne jæhe.\\ 
\end{tabular}
\scriptsize
\line(1,0){75} \newline
G I O L M Z \newline
\line(1,0){75} \newline
\textbf{1} \textit{Initiale} G O L Z  \textbf{3} \textit{Initiale} I  \textbf{13} \textit{Initiale} I  \newline
\line(1,0){75} \newline
\textbf{1} Er] ÷r O  $\cdot$ die andern] in dem L andirn M \textbf{2} dô] Da O L M Z  $\cdot$ klage] clagen L  $\cdot$ verswigen] vermiten O \textbf{3} reit] ritet O M Z  $\cdot$ Herzeloide] herzeloyde G herzenlauden I herzen lavden O Hertzeleuͯde L herczin luden M herzenlovden Z \textbf{4} sîn] eyn M  $\cdot$ manlîchiu] meynliche M \textbf{6} sît] Sin O M  $\cdot$ Herzeloide] herzoloyde G herzenlaude I mvͦter herzelavde O Hertzelouͯde L herzceloude M herzenlovde Z \textbf{7} in het ûf] vf in het I Het vf in O Jn hette uch M \textbf{9} alrêrste] Ouch reit M  $\cdot$ dô] da M  $\cdot$ dâhte] gedahte O (L) (M) Z \textbf{11} sînen] sinem L (M) \textbf{12} der] er Z \textbf{13} waz] \textit{om.} O L M  $\cdot$ got] et got O \textbf{14} mînem] minne G \textbf{15} er aber] aber er I (O) (Z) aber L (M)  $\cdot$ rîter] rittern M \textbf{16} ie] ze O  $\cdot$ rîter] rittern M [ritten]: ritter Z \textbf{17} mac schilt] mac schilte G shilt mac I  $\cdot$ ode] vnde O (L) (M) (Z) \textbf{19} manlîchiu] dienstlichiv O meinliche M \textbf{20} daz] Da Z \textbf{22} helfe er] helf er mir I helfe L \textbf{23} danne] [denn]: dann I  $\cdot$ reit] do reit O da reit L Z \textbf{24} dannoch den] dannoch in L dar noch deme M \textbf{27} im] in I \textbf{28} den] deme M  $\cdot$ ouch im] im auch I ovch O ougen im L \textbf{30} in] \textit{om.} O \newline
\end{minipage}
\hspace{0.5cm}
\begin{minipage}[t]{0.5\linewidth}
\small
\begin{center}*T
\end{center}
\begin{tabular}{rl}
 & Er neic unde die andern nigen.\\ 
 & dâ wart i\textit{r} klage niht verswigen.\\ 
 & hin \textbf{reit} Herzeloyden vruht.\\ 
 & dem riet sîn manlîch\textit{iu} zuht\\ 
5 & kiusche unde erbarmunge.\\ 
 & sît Herzeloyde, diu junge,\\ 
 & in hete ûf geerbet triuwe,\\ 
 & sich huop sînes herzen riuwe.\\ 
 & \begin{large}A\end{large}ller êrst er dô \textbf{gedâhte},\\ 
10 & wer alle die werlt volbrâhte,\\ 
 & an sînen schepfære,\\ 
 & wie gewaltic der wære.\\ 
 & er sprach: "obe got helfe pfliget,\\ 
 & diu mînem \textbf{trûren} an gesiget?\\ 
15 & wart \textbf{aber er} ie rîter holt,\\ 
 & gediende ie rîter sînen solt\\ 
 & oder mac schilt \textbf{unde} swert\\ 
 & sîner helfe sîn sô wert\\ 
 & unde \textbf{rehter manlîchen} wer,\\ 
20 & daz sîn helfe mich \textbf{von} sorgen ner,\\ 
 & ist hiute sîn \textbf{helfeclîcher} tac,\\ 
 & sô helfer, ob er helfen mac."\\ 
 & \textbf{Er kêrte umbe}, dannen er reit.\\ 
 & si stuonden dannoch, den was leit,\\ 
25 & daz er von in kêrte.\\ 
 & ir triuwe si daz lêrte,\\ 
 & die juncvrouwen, \textbf{unde sâhen im} nâch,\\ 
 & gegen den ouch sîn herze jach,\\ 
 & daz er si gerne sæhe,\\ 
30 & wandir blic in schœne jæhe.\\ 
\end{tabular}
\scriptsize
\line(1,0){75} \newline
T U V W Q R \newline
\line(1,0){75} \newline
\textbf{1} \textit{Initiale} Q   $\cdot$ \textit{Capitulumzeichen} R   $\cdot$ \textit{Majuskel} T  \textbf{9} \textit{Initiale} T U  \textbf{23} \textit{Majuskel} T  \newline
\line(1,0){75} \newline
\textbf{2} dâ] Do U V W Q R  $\cdot$ ir] in T \textbf{3} reit] reitet W (R)  $\cdot$ Herzeloyden] herzeleide U herzelauden V hertzeloyden W hertzeloudin Q herczelaude R \textbf{4} manlîchiu] manliche T \textbf{6} Herzeloyde] herzeleide U herzelaͮde V hertzeloyde W herzeloude Q herczelaude R \textbf{7} in] [*]: in V  $\cdot$ hete] \textit{om.} Q \textbf{9} dô] nun W \textbf{10} volbrâhte] vol bra U [wol]: vol brachte Q \textbf{11} sînen] sime U [sin*]: sinen V \textbf{14} mînem] [miner]: minem V meinen W  $\cdot$ trûren] treúwen W \textbf{15} aber er] abir >er< U er aber R  $\cdot$ ie rîter] ritter ye Q \textbf{16} gediende] Gedienet W (Q) (R) \textbf{17} unde] oder V \textbf{18} helfe] hilffen R \textbf{19} rehter manlîchen] rechter manlicher U rehte manliche V (W) (Q) rechtu manlichú R \textbf{20} von] vor U V W R  $\cdot$ sorgen] sorge Q \textbf{21} helfeclîcher] heflicher Q \textbf{23} umbe] sich U V W Q R  $\cdot$ dannen] [*]: wider dannen V wider dannen W (Q) (R)  $\cdot$ reit] do reit V (W) Q R \textbf{24} den] [*]: in V \textbf{27} unde sâhen im] im sahen U V W (Q) (R) \textbf{28} den] \textit{om.} Q  $\cdot$ ouch] im auch W (R) auch im Q \textbf{29} er] es Q \textbf{30} wandir] Wan der U  $\cdot$ in] \textit{om.} Q \newline
\end{minipage}
\end{table}
\end{document}
