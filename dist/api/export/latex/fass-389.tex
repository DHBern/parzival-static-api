\documentclass[8pt,a4paper,notitlepage]{article}
\usepackage{fullpage}
\usepackage{ulem}
\usepackage{xltxtra}
\usepackage{datetime}
\renewcommand{\dateseparator}{.}
\dmyyyydate
\usepackage{fancyhdr}
\usepackage{ifthen}
\pagestyle{fancy}
\fancyhf{}
\renewcommand{\headrulewidth}{0pt}
\fancyfoot[L]{\ifthenelse{\value{page}=1}{\today, \currenttime{} Uhr}{}}
\begin{document}
\begin{table}[ht]
\begin{minipage}[t]{0.5\linewidth}
\small
\begin{center}*D
\end{center}
\begin{tabular}{rl}
\textbf{389} & \textbf{\textit{\begin{large}N\end{large}}iht gesagen}, wâ der was,\\ 
 & wan sîn pflæge ein künec, hiez Anfortas.\\ 
 & dô \textbf{diu} rede von in geschach,\\ 
 & der rôte ritter aber sprach:\\ 
5 & "Ob mîner bete niht \textbf{ergêt},\\ 
 & sô vart, dâ Pelrapeire \textbf{stêt}.\\ 
 & bringet der künegîn iwer sicherheit\\ 
 & unt sagt ir, der durch si dâ streit\\ 
 & mit Kingrune unt mit Clamide,\\ 
10 & dem sî nû nâch dem Grâle wê\\ 
 & unt \textbf{doch wider} nâch ir minne.\\ 
 & nâch bêden ich immer sinne.\\ 
 & \textbf{nû} \textbf{sagt} ir \textbf{sus}, ich sant iuch dar,\\ 
 & ir helde, daz iuch got bewar."\\ 
15 & Mit urloube si riten \textbf{în}.\\ 
 & dô sprach \textbf{ouch} er zen knappen sîn:\\ 
 & "wir sîn gewinnes \textbf{unverzagt}.\\ 
 & nemt, swaz \textbf{hie orse sî} bejagt,\\ 
 & wan einez lât mir an dirre stunt.\\ 
20 & ir seht wol, daz mîne ist sêre wunt."\\ 
 & Dô sprâchen die knappen guot:\\ 
 & "\textbf{hêrre, iwer gnâde}, daz ir uns tuot\\ 
 & iwer helfe \textbf{sô} grœzlîch.\\ 
 & wir sîn nû immer rîch."\\ 
25 & Er \textbf{welt} im einez ûf sîne vart,\\ 
 & mit den kurzen ôren Ingliart,\\ 
 & daz dort von Gawane gienc,\\ 
 & in des er Melyanzen vienc.\\ 
 & dâ \textbf{holt} \textbf{ez} des rôten ritters hant.\\ 
30 & des wart verdürkelt \textbf{etslîch} rant.\\ 
\end{tabular}
\scriptsize
\line(1,0){75} \newline
D \newline
\line(1,0){75} \newline
\textbf{1} \textit{Initiale} D  \textbf{5} \textit{Majuskel} D  \textbf{15} \textit{Majuskel} D  \textbf{21} \textit{Majuskel} D  \textbf{25} \textit{Majuskel} D  \newline
\line(1,0){75} \newline
\textbf{1} Niht] ÷iht D \textbf{9} Kingrune] kyngrvͦne D  $\cdot$ Clamide] Chlamidê D \textbf{26} Ingliart] Jngliart D \newline
\end{minipage}
\hspace{0.5cm}
\begin{minipage}[t]{0.5\linewidth}
\small
\begin{center}*m
\end{center}
\begin{tabular}{rl}
 & \textbf{niht gesagen}, wâ der was,\\ 
 & wanne sîn pflæge ein künic, hiez Anfortas.\\ 
 & dô \textbf{diu} rede von in geschach,\\ 
 & der rôte ritter aber sprach:\\ 
5 & "ob mîner bete \textbf{d\textit{ô}} niht \textbf{ergât},\\ 
 & sô \textit{v}art, d\textit{â} Pelraper\textit{i}e \textbf{stât}.\\ 
 & bringet der künigîn iuwer sicherheit\\ 
 & und saget ir, der durch si dâ streit\\ 
 & mit Kingrune und mit Cla\textit{m}ide,\\ 
10 & dem sî nû nâch dem Grâle wê\\ 
 & und \textbf{doch wider} nâch \textit{ir} min\textit{n}e.\\ 
 & nâch beiden ich iemer sinne.\\ 
 & \textbf{nû} \textbf{sage ich} ir \textbf{sus}, ich sante iuch dar,\\ 
 & ir helde, daz iuch got bewar."\\ 
15 & mit urloube si \textbf{dô} riten \textbf{în}.\\ 
 & dô sprach er zuo den knappen sîn:\\ 
 & "wir sî\textit{n} gewinnes \textbf{unverzaget}.\\ 
 & nemt, waz \textbf{rosse hie sî} bejaget,\\ 
 & wanne einez lât mir an dirre stunt.\\ 
20 & ir s\textit{eh}et wol, daz mîne ist sêre wunt."\\ 
 & dô sprâchen die knappen guot:\\ 
 & "\textbf{hêrre, iuwer gnâde}, daz ir uns tuot\\ 
 & iuwer helf\textit{e} \textbf{alsô} grœzlîch.\\ 
 & wir sî\textit{n} nû iemer \textbf{mêre} rîch."\\ 
25 & er \textbf{welt} im einez ûf sîne vart,\\ 
 & mit den kurzen ôren Ingliart,\\ 
 & daz dort von Gawane gienc,\\ 
 & innen des er Melianzen vienc.\\ 
 & d\textit{â} \textbf{holt} \textbf{ez} des rôten ritters hant.\\ 
30 & des wart verdürkelt \textbf{etlîch} rant.\\ 
\end{tabular}
\scriptsize
\line(1,0){75} \newline
m n o \newline
\line(1,0){75} \newline
\newline
\line(1,0){75} \newline
\textbf{2} hiez] \textit{om.} n \textbf{3} in] jme n (o) \textbf{5} dô] da m \textbf{6} vart] wart m o  $\cdot$ dâ] do m n o  $\cdot$ Pelraperie] pelrapere m pelrapier n o \textbf{8} durch si dâ] durch sú do n do so durch sie o \textbf{9} Kingrune] kv́nigrin n konigrim o  $\cdot$ Clamide] Clanide m \textbf{11} ir minne] die mine m \textbf{13} sage] sagete n sagt o  $\cdot$ ich ir] ir n o  $\cdot$ iuch] ouch úch n \textbf{17} sîn] sint m n o \textbf{19} dirre] der n o \textbf{20} sehet] sullent m  $\cdot$ sêre] \textit{om.} n o \textbf{21} knappen] knappe o \textbf{22} ir] \textit{om.} n \textbf{23} helfe] hellflich m \textbf{24} sîn] sint m n o \textbf{25} welt] welte n o \textbf{26} kurzen ôren] kẏrczen o  $\cdot$ Ingliart] jngliart m n o \textbf{27} Gawane] gawige o \textbf{28} Melianzen] mellianczen m meliantzen n melianczen o \textbf{29} dâ] Do m n Do rat o \newline
\end{minipage}
\end{table}
\newpage
\begin{table}[ht]
\begin{minipage}[t]{0.5\linewidth}
\small
\begin{center}*G
\end{center}
\begin{tabular}{rl}
 & \textbf{gezeigen ninder}, wâ der was,\\ 
 & wan sîn pflæge ein künic, hiez Anfortas.\\ 
 & dô \textbf{disiu} rede von in geschach,\\ 
 & der rôte rîter aber sprach:\\ 
5 & "obe mîner bet niht \textbf{ergê},\\ 
 & sô vart, dâ Pelrapeire \textbf{stê}.\\ 
 & bringet der künigîn iwer sicherheit\\ 
 & \textit{und} saget ir, der durch si dâ streit\\ 
 & mit Kingrune unde mit Clamide,\\ 
10 & dem sî nû nâch dem Grâle wê\\ 
 & unde \textbf{ouch} nâch ir minne.\\ 
 & nâch beiden ich imer sinne.\\ 
 & \textbf{saget} ir \textit{\textbf{von mir}}, ich sande iuch dar,\\ 
 & ir helde, daz iuch got bewar."\\ 
15 & mit urloube si riten \textbf{în}.\\ 
 & dô sprach er zen knappen sîn:\\ 
 & "wir sîn gewinnes \textbf{niht verzaget}.\\ 
 & nemet, swaz \textbf{hie orse sîn} bejaget,\\ 
 & wan einez lât mir an dirre stunt.\\ 
20 & ir seht wol, daz mîn ist sêre wunt."\\ 
 & dô sprâchen die knappen guot:\\ 
 & "\textbf{genâde, hêrre}, daz ir uns tuot\\ 
 & iwer helfe \textbf{sô} grœzlîche.\\ 
 & wir sîn nû imer rîche."\\ 
25 & er \textbf{\textit{er}welt} im einez ûf sîne vart,\\ 
 & mit den kurzen ôren Inguliart,\\ 
 & daz dort von Gawane gienc,\\ 
 & innen des \textbf{dô}r Melianzen vienc.\\ 
 & dâ \textbf{erholte} des rôten rîte\textit{r}s hant,\\ 
30 & des wart verdürkelet \textbf{etslîch} rant.\\ 
\end{tabular}
\scriptsize
\line(1,0){75} \newline
G I O L M Q R Z Fr28 \newline
\line(1,0){75} \newline
\textbf{3} \textit{Initiale} I O L Z   $\cdot$ \textit{Capitulumzeichen} R  \textbf{21} \textit{Initiale} I  \newline
\line(1,0){75} \newline
\textbf{1} \textit{Die Verse 370.13-412.12 fehlen} Q   $\cdot$ Niht gezeigen wa der was Z  $\cdot$ ninder] nyrgen M nyemer R  $\cdot$ der] er I \textbf{2} pflæge] phlac I  $\cdot$ künic] kunc der I  $\cdot$ Anfortas] Amfortas O Amfortaz L anfortes M Anfrotas R \textbf{3} dô] ÷o O Da Z  $\cdot$ in] [i*]: in G im I L (M) Z  $\cdot$ geschach] gesach M \textbf{5} \textit{Versfolge 389.6-5} I   $\cdot$ ergê] erget I ergie M \textbf{6} vart dâ] wart da M wartent wa R  $\cdot$ Pelrapeire] pailrapeir I pelarapeire R  $\cdot$ stê] \textit{om.} M \textbf{7} der] der der L  $\cdot$ iwer] min R \textbf{8} und] \textit{om.} G  $\cdot$ der] de Fr28 \textbf{9} Kingrune] kingrun I kyngrvn O (M) kingruͯne L kúngrune R kingrunen Z kẏngrun Fr28  $\cdot$ unde mit] vnd R  $\cdot$ Clamide] Glamide O chlamide Fr28 \textbf{10} nâch dem] nach den R \textbf{11} ouch] doch L (M) R (Z) Fr28  $\cdot$ nâch] wider nach I (L) (M) R (Z) (Fr28)  $\cdot$ minne] minnen O \textbf{12} ich] \textit{om.} O  $\cdot$ imer] miner M \textbf{13} saget] Nv sagt Z  $\cdot$ von mir] \textit{om.} G svs Z \textbf{14} iuch] uch vnd sy R \textbf{15} în] hin M Fr28 \textbf{16} dô] Da O M  $\cdot$ er] auch er I (O) (Z) er ouch L (R)  $\cdot$ zen] zcu deme M \textbf{17} sîn] en sint M sind R  $\cdot$ gewinnes] siges Fr28 \textbf{18} nemt swaz ors ich hie han beiagt I  $\cdot$ swaz] waz L (R) Fr28 swar M  $\cdot$ hie orse] rosse hie M (Fr28) hie Rosen R  $\cdot$ sîn] si O (L) (R) Z Fr28 synt M \textbf{20} sêre] \textit{om.} I O L  $\cdot$ wunt] gwunt Fr28 \textbf{21} dô] Da M R \textbf{22} \textit{Versfolge 389.24-23} Fr28   $\cdot$ genâde] Jwer genade O (L) (M) (R) (Z) (Fr28)  $\cdot$ hêrre] herren R \textbf{23} sô] \textit{om.} M \textbf{24} sîn] sind R  $\cdot$ nû] \textit{om.} I \textbf{25} erwelt] welt G wolde M \textbf{26} kurzen] Rotten R  $\cdot$ Inguliart] luzguliart I Jngvliart O L Jngliart R \textbf{27} von] vor O  $\cdot$ Gawane] Gawan I O Z \textbf{28} Melianzen] Melyanzen O Melianczen R meliantzen Z melian::: Fr28 \textbf{29} dâ] Do R (Fr28)  $\cdot$ erholte] erholz I er holt O holt es R erholt ez Z  $\cdot$ rîters] rites G \textbf{30} des] Da O Dest L Der Z  $\cdot$ wart] war L  $\cdot$ verdürkelet] durchel I (M) verdrvcht O (L) verdurket R  $\cdot$ etslîch] [etlichich]: shiltes I etisliches schildes M (Fr28) \newline
\end{minipage}
\hspace{0.5cm}
\begin{minipage}[t]{0.5\linewidth}
\small
\begin{center}*T
\end{center}
\begin{tabular}{rl}
 & \textbf{gezeigen nie\textit{n}der}, wâ der was,\\ 
 & wan sîn pflæge ein künec, hieze Anfortas.\\ 
 & \begin{large}D\end{large}ô \textbf{dis\textit{iu}} rede von in geschach,\\ 
 & der rôte rîter aber \textbf{dô} sprach:\\ 
5 & "ob mîner bete niht \textbf{ergêt},\\ 
 & sô vart, dâ Peilrapere \textbf{stêt}.\\ 
 & bringet der küneginne iuwer sicherheit\\ 
 & unde saget ir, der durch si dâ streit\\ 
 & mit Kyngrune unde mit Clamide,\\ 
10 & dem sî nû nâch dem Grâle wê\\ 
 & unde \textbf{doch wider} nâch ir minne.\\ 
 & nâch beiden ich iemer sinne.\\ 
 & \textbf{saget} ir \textbf{von mir}, ich santiuch dar,\\ 
 & ir helde, daz iuch got bewar."\\ 
15 & mit urloube si riten \textbf{hin}.\\ 
 & Dô sprach \textbf{ouch} er zen knappen sîn:\\ 
 & "wir sîn gewinnes \textbf{niht verzaget}.\\ 
 & nemt, swaz \textbf{orse sî hie} bejaget,\\ 
 & wan einez lât mir an dirre stunt.\\ 
20 & ir seht wol, daz mîne ist sêre wunt."\\ 
 & Dô sprâchen die knappen guot:\\ 
 & "\textbf{iuwer gnâde, hêrre}, daz ir uns tuot\\ 
 & iuwer helfe \textbf{sô} grœzlîche.\\ 
 & wir sîn nû iemer rîche."\\ 
25 & er \textbf{welt}im einez ûf sîne vart,\\ 
 & mit den kurzen ôren Ingliart,\\ 
 & daz dort von Gawane gie,\\ 
 & innen des \textbf{dô} er Melyanzen vie.\\ 
 & dâ \textbf{erwarp} \textbf{ez} des rôten rîters hant.\\ 
30 & des \textit{wart} verdürkelt \textbf{manec} rant.\\ 
\end{tabular}
\scriptsize
\line(1,0){75} \newline
T V W \newline
\line(1,0){75} \newline
\textbf{3} \textit{Initiale} T V W  \textbf{16} \textit{Majuskel} T  \textbf{21} \textit{Majuskel} T  \newline
\line(1,0){75} \newline
\textbf{1} niender] nieder T \textbf{2} hieze] [*]: der hies V \textbf{3} disiu] dise T  $\cdot$ in] im V W \textbf{4} dô] \textit{om.} V W \textbf{5} niht] [*]: do niht V \textbf{6} dâ] do V W  $\cdot$ Peilrapere] [*lrapere]: pelrapere V pelrapeir W \textbf{8} durch si dâ streit] sy do erstrait W \textbf{9} Kyngrune] kẏngrune V kingrun W  $\cdot$ Clamide] Clamîde T klamide W \textbf{10} Grâle] gral W \textbf{13} santiuch] santiv T \textbf{14} iuch] iv T \textbf{15} riten hin] [*]: do ritten in V \textbf{16} ouch er] [*]: er V \textbf{18} swaz orse sî hie] was hie ros sein W \textbf{19} lât] daz lant V  $\cdot$ dirre] der W \textbf{24} sîn] sint V \textbf{25} weltim] welt im V W \textbf{26} Ingliart] [*]: gringliart V ingliart W \textbf{28} Melyanzen] [melẏ*nzen]: melẏanzen V melianzen W \textbf{29} dâ] Do W  $\cdot$ erwarp ez] [*]: holte ez V holt er W \textbf{30} wart] \textit{om.} T  $\cdot$ manec rant] etschlich hant W \newline
\end{minipage}
\end{table}
\end{document}
