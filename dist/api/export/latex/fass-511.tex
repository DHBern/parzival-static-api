\documentclass[8pt,a4paper,notitlepage]{article}
\usepackage{fullpage}
\usepackage{ulem}
\usepackage{xltxtra}
\usepackage{datetime}
\renewcommand{\dateseparator}{.}
\dmyyyydate
\usepackage{fancyhdr}
\usepackage{ifthen}
\pagestyle{fancy}
\fancyhf{}
\renewcommand{\headrulewidth}{0pt}
\fancyfoot[L]{\ifthenelse{\value{page}=1}{\today, \currenttime{} Uhr}{}}
\begin{document}
\begin{table}[ht]
\begin{minipage}[t]{0.5\linewidth}
\small
\begin{center}*D
\end{center}
\begin{tabular}{rl}
\textbf{511} & \begin{large}I\end{large}ch wiste gerne, ob ir \textbf{der} sît,\\ 
 & der durch mich getorste \textbf{lîden} strît.\\ 
 & daz verbert, \textbf{bedurft} ir êre.\\ 
 & solt ich iu râten mêre,\\ 
5 & spræchet ir denne der volge 'jâ',\\ 
 & sô suocht ir minne anderswâ.\\ 
 & ob ir mîner minne gert,\\ 
 & minne unt vreude \textbf{ir sît} entwert.\\ 
 & ob ir mich hinnen vüeret,\\ 
10 & grôz sorge iuch dâ nâch rüeret."\\ 
 & Dô sprach mîn hêrre Gawan:\\ 
 & "Wer \textbf{mac minne ungedient} hân\\ 
 & - muoz ich iu daz künden? -,\\ 
 & der treit si hin mit sünden.\\ 
15 & swem ist ze werder minne gâch,\\ 
 & \textbf{dâ} hœret dienst vor und nâch."\\ 
 & Si sprach: "welt ir mir dienst geben,\\ 
 & sô muozet ir \textbf{werlîche} leben\\ 
 & und megt doch laster wol bejagen.\\ 
20 & mîn dienst bedarf decheines  zagen.\\ 
 & vart jenen pfat - ez ist niht ein wec -\\ 
 & dort über \textbf{jenen} hôhen stec\\ 
 & in \textbf{jenen} boumgarten.\\ 
 & \textbf{mînes} pferdes sult ir dâ warten.\\ 
25 & dâ \textbf{hœrt ir und seht ir} manege diet,\\ 
 & die tanzent und singent liet,\\ 
 & Tambûren, floitieren.\\ 
 & swie si iuch condwieren,\\ 
 & gêt durch si, dâ mîn pfert \textbf{dort} stêt,\\ 
30 & unt lœst ez ûf: nâch iu ez gêt."\\ 
\end{tabular}
\scriptsize
\line(1,0){75} \newline
D \newline
\line(1,0){75} \newline
\textbf{1} \textit{Initiale} D  \textbf{11} \textit{Majuskel} D  \textbf{12} \textit{Majuskel} D  \textbf{17} \textit{Majuskel} D  \textbf{27} \textit{Majuskel} D  \newline
\line(1,0){75} \newline
\newline
\end{minipage}
\hspace{0.5cm}
\begin{minipage}[t]{0.5\linewidth}
\small
\begin{center}*m
\end{center}
\begin{tabular}{rl}
 & ich wuste gerne, ob ir \textbf{daz} sît,\\ 
 & der durch mich getorste strît.\\ 
 & daz verbert, \textbf{bedurft} ir êre.\\ 
 & solt ich iu râten mêre,\\ 
5 & spræchet ir dan der volge 'jâ',\\ 
 & sô suocht ir minne anderswâ.\\ 
 & ob ir mîner minne gert,\\ 
 & minne und vröude \textbf{ir sît} entwert.\\ 
 & ob ir mich hinnen vüeret,\\ 
10 & grôz sorge iuch dar nâch rüeret."\\ 
 & dô sprach mîn hêr Gawan:\\ 
 & "wer \textbf{minne ungedienet wil} hân\\ 
 & - muoz ich iu daz künden? -,\\ 
 & der treit si hin mit sünden.\\ 
15 & wem ist zuo werder minne \textit{g}âch,\\ 
 & \textbf{dem} hœret dienst vor und nâch."\\ 
 & si sprach: "welt ir mir dienst geben,\\ 
 & sô müezet ir \textbf{wirdeclîchen} leben\\ 
 & und magt doch laster wol bejagen.\\ 
20 & mîn dienst bedarf dekeines zagen.\\ 
 & vart jenen pfat - ez ist niht ein wec -\\ 
 & dort über \textbf{einen} hôhen stec\\ 
 & in \textbf{einen} boumgarten.\\ 
 & \textbf{eines} pferdes solt ir d\textit{â} warten.\\ 
25 & d\textit{â} \textbf{hœrt und seht ir} manige diet,\\ 
 & die tanzen\textit{t} und singent liet,\\ 
 & tambûren, floitieren.\\ 
 & wie si iuch condwieren,\\ 
 & gât durch si, d\textit{â} mîn pfert stât,\\ 
30 & und lœset ez ûf: nâch iu ez gât."\\ 
\end{tabular}
\scriptsize
\line(1,0){75} \newline
m n o \newline
\line(1,0){75} \newline
\newline
\line(1,0){75} \newline
\textbf{2} strît] liden strit n o \textbf{5} jâ] :a o \textbf{6} sô] S: o \textbf{11} hêr] herre her n \textbf{14} der] Das n \textbf{15} gâch] joch m n \textbf{16} dem] Der o \textbf{20} bedarf] bedarfft o  $\cdot$ dekeines] do keins n \textbf{21} \textit{Die Versanfänge 511.21-30 fehlen (abgerissen)} o   $\cdot$ jenen] innen o  $\cdot$ niht ein wec] enweg o \textbf{23} einen] einem n (o) \textbf{24} solt] solte o  $\cdot$ dâ] do m n o \textbf{25} dâ] Do m So n  $\cdot$ ir] \textit{om.} n \textbf{26} tanzent] tanczen m (n) (o)  $\cdot$ singent liet] singen liecht o \textbf{27} tambûren] Dumburen n  $\cdot$ floitieren] floritien o \textbf{28} condwieren] Condiweieren o \textbf{29} dâ] do m n \textbf{30} nâch] noch noch o \newline
\end{minipage}
\end{table}
\newpage
\begin{table}[ht]
\begin{minipage}[t]{0.5\linewidth}
\small
\begin{center}*G
\end{center}
\begin{tabular}{rl}
 & \begin{large}I\end{large}ch wesse gerne, obe ir \textbf{daz} sît,\\ 
 & der durch mich getorste \textbf{lîden} strît.\\ 
 & daz verbert, \textbf{bedurft} ir êre.\\ 
 & solde ich iu râten mêre,\\ 
5 & spr\textit{æ}chet ir danne \textit{der volge} 'jâ',\\ 
 & sô suoch\textit{t} ir minne anderswâ.\\ 
 & obe ir mîner minne gert,\\ 
 & minne unde vröude \textbf{sît ir} entwert.\\ 
 & obe ir mich hinnen vüeret,\\ 
10 & grô\textit{z} sorge iuch dâ nâch rüeret."\\ 
 & d\textit{ô} sprach mîn hêrre Gawan:\\ 
 & "swer \textbf{mac minne ungedienet} hân\\ 
 & - muoz ich iu daz künden? -,\\ 
 & der treit si hin mit sünden.\\ 
15 & swem ist ze werder minne gâch,\\ 
 & \textbf{dâ} hœret dienst vor unde nâch."\\ 
 & si sprach: "welt ir mir dienst geben,\\ 
 & sô müezet ir \textbf{werdeclîchen} leben\\ 
 & unt muget doch laster wol bejagen.\\ 
20 & mîn dienst bedarf neheines zagen.\\ 
 & vart jenen pfat - ez ist niht ein wec -\\ 
 & dort über \textbf{jenen} hôhen stec\\ 
 & in \textbf{einen} boumgarten.\\ 
 & \textbf{mînes} pferdes sult ir dâ warten.\\ 
25 & dâ \textbf{hœrt unde sehet} manige diet,\\ 
 & die tanzent unde singen\textit{t} \textit{l}iet,\\ 
 & tambûren \textbf{unde} floitieren.\\ 
 & swie si iuch condewieren,\\ 
 & gêt durch si, dâ mîn pfert \textbf{dort} stêt,\\ 
30 & \textit{und lœst ez ûf: nâch iu ez gêt}."\\ 
\end{tabular}
\scriptsize
\line(1,0){75} \newline
G I L M Z Fr22 \newline
\line(1,0){75} \newline
\textbf{1} \textit{Initiale} G I L Z Fr22  \textbf{21} \textit{Initiale} I  \newline
\line(1,0){75} \newline
\textbf{1} obe ir daz] abir der M \textbf{3} verbert] werbent L vorbort M \textbf{4} solde] sol I \textbf{5} spræchet] sprachet G  $\cdot$ der volge] \textit{om.} G \textbf{6} suocht] suehte G (Fr22) \textbf{8} minne unde vröude] Minnin vnde vrewedin Fr22  $\cdot$ sît ir] ir sit M Z Fr22 \textbf{10} grôz] groze G \textbf{11} dô] da G \textbf{12} swer] Wer L M Z  $\cdot$ mac minne] mýnne mag L libe wil M \textbf{14} hin] \textit{om.} L Fr22 \textbf{15} swem] Wem L (M)  $\cdot$ werder] werdin M (Z) \textbf{16} dienst] dienst zvͦ I zcu dienst M \textbf{18} werdeclîchen] werliche L (M) (Z) (Fr22) \textbf{19} doch laster wol] laster doch I \textbf{20} neheines] ich eines M [keiner]: keines Z \textbf{21} jenen] ienez I einen M \textbf{22} jenen] einen L (M) \textbf{23} in einen baugarten I  $\cdot$ Jn einē boymgartin M \textbf{24} Dar sult ir myns pherdis warten M \textbf{25} hœrt] hort ir I (L) (Z)  $\cdot$ sehet] set ir M \textbf{26} singent liet] singent mænige liet G \textbf{27} unde] \textit{om.} L Z \textbf{28} swie] Wie L M \textbf{29} mîn] myne M  $\cdot$ dort] \textit{om.} I M \textbf{30} \textit{Vers 511.30 fehlt} G   $\cdot$ lœst ez] loset M \newline
\end{minipage}
\hspace{0.5cm}
\begin{minipage}[t]{0.5\linewidth}
\small
\begin{center}*T
\end{center}
\begin{tabular}{rl}
 & ich wiste gerne, ob ir \textbf{der} sît,\\ 
 & der durch mich getorste \textbf{lîden} strît.\\ 
 & daz verbert, \textbf{durft} ir êre.\\ 
 & solt ich iu râten mêre,\\ 
5 & spræchet ir danne der volge 'jâ',\\ 
 & sô suochet ir minne anderswâ.\\ 
 & ob ir mîner minne gert,\\ 
 & minne unde vröude \textbf{ir sît} entwert.\\ 
 & ob ir mich hinnen vüeret,\\ 
10 & grôz sorge iuch dar nâch rüeret."\\ 
 & Dô sprach mîn hêr Gawan:\\ 
 & "swer \textbf{minne mac ungedienet} hân\\ 
 & - muoz ich iu daz künden? -,\\ 
 & der treit si hin mit sünden.\\ 
15 & swem ist ze werder minne gâch,\\ 
 & \textbf{dâ} hœret dienst vor unde nâch."\\ 
 & \begin{large}S\end{large}i sprach: "welt ir mir dienst geben,\\ 
 & sô müezet ir \textbf{werlîche} leben\\ 
 & unde muget doch laster wol bejagen.\\ 
20 & mîn dienst bedarf deheines zagen.\\ 
 & var\textit{t} jenen pfat - ez ist niht ein wec -\\ 
 & dort über \textbf{jenen} hôhen stec\\ 
 & in \textbf{einem} boumgarten.\\ 
 & \textbf{mînes} pferdes sult ir dâ warten.\\ 
25 & dâ \textbf{seht ir unde hœret} maneg\textit{e} diet,\\ 
 & die tanzent unde singent liet,\\ 
 & tambûre\textit{n}, floitieren.\\ 
 & swie si iuch condewieren,\\ 
 & gêt durch si, dâ mîn pfert \textbf{dort} stêt,\\ 
30 & unde lœset ez ûf: nâch iu ez gêt."\\ 
\end{tabular}
\scriptsize
\line(1,0){75} \newline
T U V W O Q R Fr40 \newline
\line(1,0){75} \newline
\textbf{1} \textit{Initiale} O Q Fr40  \textbf{11} \textit{Majuskel} T  \textbf{17} \textit{Initiale} T U  \newline
\line(1,0){75} \newline
\textbf{1} ich] ÷ch O \textbf{2} mich getorste] minne torst W mich getroste Q \textbf{3} verbert] verbergent W verberck Q  $\cdot$ durft] bedurfte U bedort V (O) (R) (Fr40) \textbf{6} sô] Do Q  $\cdot$ suochet] [svͦch*]: svͦchtent V svlt O \textbf{8} ir sît] sint ir V (R) (Fr40)  $\cdot$ entwert] gewert R \textbf{9} vüeret] [vner*]: vueret U \textbf{10} iuch] iv T  $\cdot$ dar nâch] [dannoch]: darnoch V dannoch Fr40 \textbf{11} sprach] sprach do Q  $\cdot$ Gawan] her Gawain R \textbf{12} swer] Wer U W Q R (Fr40)  $\cdot$ minne] mein Q  $\cdot$ mac] [*]: wil V wil O  $\cdot$ ungedienet] vngediet R \textbf{15} swem] Wem U W Q R  $\cdot$ ze] nach W O \textit{om.} Fr40  $\cdot$ werder] werden V \textbf{16} dâ] Dar zu R  $\cdot$ hœret] gehoͤret V (Q) (R) (Fr40) \textbf{17} mir] vns Fr40 \textbf{18} werlîche] [*]: werdecliche V wertlich Q \textbf{19} muget] must Q  $\cdot$ doch] \textit{om.} Q \textbf{20} bedarf] gehoͯrtt R  $\cdot$ deheines] keinem R \textbf{21} vart] var T  $\cdot$ jenen] einen U  $\cdot$ ez] [ei]: ez en O \textbf{22} dort] \textit{om.} O  $\cdot$ jenen] einen U \textbf{23} Jn einen (einē V ) bauͦmgarten U (V) (W) (Q) (Fr40)  $\cdot$ Jnden bavngarten O \textbf{25} manege] manegiv T \textbf{27} tambûren] tambvre T Tambvren vnd V (W)  $\cdot$ floitieren] florieren U \textbf{28} swie] Wie U W (Q) R  $\cdot$ iuch] iv T auch Q \textbf{29} mîn] [nuͯn]: mein Q  $\cdot$ dort] do W \textit{om.} O Q R \textbf{30} iu] \textit{om.} U \newline
\end{minipage}
\end{table}
\end{document}
