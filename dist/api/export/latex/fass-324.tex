\documentclass[8pt,a4paper,notitlepage]{article}
\usepackage{fullpage}
\usepackage{ulem}
\usepackage{xltxtra}
\usepackage{datetime}
\renewcommand{\dateseparator}{.}
\dmyyyydate
\usepackage{fancyhdr}
\usepackage{ifthen}
\pagestyle{fancy}
\fancyhf{}
\renewcommand{\headrulewidth}{0pt}
\fancyfoot[L]{\ifthenelse{\value{page}=1}{\today, \currenttime{} Uhr}{}}
\begin{document}
\begin{table}[ht]
\begin{minipage}[t]{0.5\linewidth}
\small
\begin{center}*D
\end{center}
\begin{tabular}{rl}
\textbf{324} & \textit{\begin{large}B\end{large}}eacors al vaste bat.\\ 
 & der gast stuont an sîner stat.\\ 
 & er sprach: "mir biutet kampf ein man,\\ 
 & des ich neheine künde hân.\\ 
5 & i\textbf{ne} hân ouch niht ze sprechen dar.\\ 
 & \textbf{starc}, küene, wol gevar,\\ 
 & getriwe unde rîche,\\ 
 & hât er diu volleclîche,\\ 
 & er mac borgen deste baz.\\ 
10 & i\textbf{ne} \textbf{trage} \textbf{gein im} decheinen haz.\\ 
 & Er was mîn hêrre unt mîn mâc,\\ 
 & durch den ich hebe disen bâc.\\ 
 & unser veter \textbf{gebruoder} hiezen,\\ 
 & die \textbf{nihtes} ein ander liezen.\\ 
15 & \textbf{Nehein} man gekrœnet wart\\ 
 & \textbf{nie}, \textbf{ich} enhet im vollen art\\ 
 & \textbf{mit kampfe} rede ze bieten,\\ 
 & \textbf{mich} râche gein im nieten.\\ 
 & Ich bin ein vürste ûz Ascalun,\\ 
20 & der lantgrâve von Schampfanzun,\\ 
 & unt heize Kingrimursel.\\ 
 & ist hêr Gawan lobes \textbf{snel},\\ 
 & \textbf{der} mac sich \textbf{anders} niht entsagen,\\ 
 & er\textbf{n} \textbf{müeze} \textbf{kampf dâ} gein mir tragen.\\ 
25 & \textbf{Ouch gib ich} im vride über al daz lant,\\ 
 & niwan von mîn eines hant.\\ 
 & mit triwen ich vride geheize\\ 
 & ûzerhalp des kampfes kreize.\\ 
 & got hüete al der ich lâze hie,\\ 
30 & wan eines: er weiz wol se\textit{l}be wie."\\ 
\end{tabular}
\scriptsize
\line(1,0){75} \newline
D \newline
\line(1,0){75} \newline
\textbf{1} \textit{Initiale} D  \textbf{11} \textit{Majuskel} D  \textbf{15} \textit{Majuskel} D  \textbf{19} \textit{Majuskel} D  \textbf{25} \textit{Majuskel} D  \newline
\line(1,0){75} \newline
\textbf{1} Beacors] Deachors D \textbf{19} Ascalun] Ascalv̂n D \textbf{20} Schampfanzun] Scampfanzv̂n D \textbf{30} selbe] sebe D \newline
\end{minipage}
\hspace{0.5cm}
\begin{minipage}[t]{0.5\linewidth}
\small
\begin{center}*m
\end{center}
\begin{tabular}{rl}
 & \begin{large}B\end{large}ea\textit{c}urs alvaste bat.\\ 
 & der gast stuont an sîner stat.\\ 
 & er sprach: "mir biutet kampf ein man,\\ 
 & des ich enkeine künde hân.\\ 
5 & ich hân ouch niht ze sprechen dar.\\ 
 & \textbf{star\textit{c}}, küene, wol gevar,\\ 
 & getriuwe und rîche,\\ 
 & hât er diu volle\textit{c}lîche,\\ 
 & er mac borgen deste baz.\\ 
10 & ich \textbf{trage} \textbf{ûf in} dekeine\textit{n} haz.\\ 
 & er was mîn hêrre und mîn mâc,\\ 
 & durch den ich hebe disen bâc.\\ 
 & unser vetere \textbf{gebruoder} hiezen,\\ 
 & die \textbf{nihtes} ein ander liezen.\\ 
15 & \textbf{nie k\textit{ein}} \textit{m}an gek\textit{r}œnet \textit{w}a\textit{rt},\\ 
 & \textbf{ich} enhete ime \textbf{wol} vollen \textit{art}\\ 
 & \textbf{mit kampfe} rede ze bieten,\\ 
 & \textbf{mich} râche gegen ime nieten.\\ 
 & ich bin ein vürste ûz Ascalun,\\ 
20 & der lantgrâve von S\textit{ch}a\textit{n}fanzun,\\ 
 & und heize Kingrimursel.\\ 
 & ist hêr Gawa\textit{n} lobes \textbf{snel},\\ 
 & \textbf{der} mac sich \textbf{anders} niht entsagen,\\ 
 & er \textbf{müeze} \textbf{kampf d\textit{â}} gegen mir tragen.\\ 
25 & \textbf{ouch gip ich} ime vride über al daz lant,\\ 
 & niuwan von mîn eines hant.\\ 
 & mit triuwe ich vride geheize\\ 
 & ûzerhalp des kampfes kreize.\\ 
 & got hüete al der ich lâze hie,\\ 
30 & wanne eines: er weiz wol selbe wie."\\ 
\end{tabular}
\scriptsize
\line(1,0){75} \newline
m n o \newline
\line(1,0){75} \newline
\textbf{1} \textit{Initiale} m n  \newline
\line(1,0){75} \newline
\textbf{1} Beacurs] [Bet]: Beathurs m BEahcúrs n Beahcurs o \textbf{2} stuont] sant o \textbf{4} enkeine] nẏe kein n nuͯ kan o \textbf{6} starc] Starcke m  $\cdot$ wol] vnd wol n o \textbf{8} hât] Hette n  $\cdot$ volleclîche] vollenkenliche m \textbf{10} dekeinen] dekeine m do keinen n \textbf{12} bâc] tag o \textbf{14} ein ander] eẏander o \textbf{15} Nie kam in angekoͯnet bas m \textbf{16} art] was m \textbf{19} Ascalun] ascalún m o \textbf{20} Schanfanzun] Samfanzún m scanfazun n von scanfazẏm o \textbf{21} heize] hiesz n o  $\cdot$ Kingrimursel] kingramúrsel n konigramuͯrsel o \textbf{22} Gawan] gawas m gewan o \textbf{24} müeze] muͦsz n (o)  $\cdot$ dâ] do m n o \textbf{25} vride] freuͯide o \textbf{26} Niht wan von mynns [ey]: hant o  $\cdot$ mîn] mins m n \textbf{27} triuwe ich vride] trauwe ich freuide o \textbf{29} hüete] behuͯte n  $\cdot$ al] alle n  $\cdot$ der ich] [ich]: die ich n die ich o \textbf{30} selbe] selbes n \newline
\end{minipage}
\end{table}
\newpage
\begin{table}[ht]
\begin{minipage}[t]{0.5\linewidth}
\small
\begin{center}*G
\end{center}
\begin{tabular}{rl}
 & Beakurs al vaste bat.\\ 
 & der gast stuont an sîner stat.\\ 
 & \begin{large}E\end{large}r sprach: "mir biutet kampf ein man,\\ 
 & des ich deheine künde hân.\\ 
5 & ich\textbf{ne} hân ouch niht ze sprechenne dar.\\ 
 & \textbf{stæte}, küene, wolgevar,\\ 
 & getriuwe unde rîche,\\ 
 & hât er diu volliclîche,\\ 
 & er mac borgen deste baz.\\ 
10 & ich\textbf{ne} \textbf{hân} \textbf{gein im} deheinen haz.\\ 
 & er was mîn hêrre unde \textbf{ouch} mîn mâc,\\ 
 & durch den ich hebe disen bâc.\\ 
 & unser vatere \textbf{bruoder} hiezen,\\ 
 & die \textbf{nihtes} ein ander liezen.\\ 
15 & \textbf{sô hôher} man gekrœnt wart\\ 
 & \textbf{nie}, \textbf{ich} enhetim vollen art\\ 
 & \textbf{in kampfe} rede ze bieten,\\ 
 & \textbf{mich} râche gein im nieten.\\ 
 & ich bin ein vürste ûz Aschalun,\\ 
20 & der lantgrâve von Tschanfenzun,\\ 
 & unt heize Kingrimursel.\\ 
 & ist hêr Gawan lobes \textbf{snel},\\ 
 & \textbf{er} mac sich \textbf{anders} niht entsagen,\\ 
 & er \textbf{en}\textbf{welle} \textbf{dâ kampf} gein mir tragen.\\ 
25 & \textbf{ich gibe} im vride über al daz lant,\\ 
 & niwan von mîn eines hant.\\ 
 & mit triuwen ich vride geheize\\ 
 & ûzerhalp des kampfes kreize.\\ 
 & got hüete al der ich lâze hie,\\ 
30 & wan eines: er weiz wol selbe wie."\\ 
\end{tabular}
\scriptsize
\line(1,0){75} \newline
G I O L M Q R Z Fr22 Fr39 Fr40 \newline
\line(1,0){75} \newline
\textbf{1} \textit{Initiale} I M  \textbf{3} \textit{Initiale} G   $\cdot$ \textit{Capitulumzeichen} R  \textbf{11} \textit{Initiale} O L Fr22 Fr39  \textbf{19} \textit{Initiale} I Z  \textbf{27} \textit{Initiale} Fr40  \newline
\line(1,0){75} \newline
\textbf{1} Beakurs] beacurs G (I) (O) (M) Beakuͯrs L Beachurs Z [Beakus]: Beakurs Fr39  $\cdot$ al] als R \textbf{3} kampf] chanphe I kamphes M \textbf{5} ichne] ich I (O) (L) (Q) (R) (Fr39) Vnd Z  $\cdot$ ouch] doch I (M)  $\cdot$ niht] \textit{om.} Z  $\cdot$ ze sprechenne] gesprochen R \textbf{6} stæte] Starch O (M) (Q) (R) (Z) (Fr22) (Fr40) Starke L (Fr39)  $\cdot$ wolgevar] wol getan L Fr39 \textbf{7} \textit{Versfolge 324.8-7} L Fr22 Fr39   $\cdot$ Dar zuͯ so ist er riche L (Fr39) Ist er mvͦtis riche Fr22  $\cdot$ getriuwe] Triwe I \textbf{8} diu] \textit{om.} I die O Fr39 (Fr40) \textbf{9} mac] mag wol Q Fr40 \textbf{10} ichne hân] ich han I (O) (Fr22) Jch entrag Q (Z) (Fr40) Jch trag R  $\cdot$ im] dir L en M \textbf{11} er] ÷r O  $\cdot$ ouch mîn mâc] min mac I (Q) (R) (Fr40) ich sin man L (Fr22) (Fr39) mac M \textbf{12} Duͯrch den ich den (disin Fr22 ) kampf wil han L (Fr22) (Fr39)  $\cdot$ durch den] Den durch Q  $\cdot$ hebe] habe M (R)  $\cdot$ disen] den Q  $\cdot$ bâc] wag R \textbf{13} vatere] beyder Q \textbf{14} nihtes] nih I  $\cdot$ ander] andren R \textbf{15} sô hôher] Kein Z \textbf{16} nie] Hie Z  $\cdot$ ich] \textit{om.} L  $\cdot$ enhetim] het im I (R) het in Z inhête in Fr22  $\cdot$ vollen] vollir Fr22 \textbf{17} in kampfe] in kamphes I (M) Jm champfes O (L) (Fr39) Kampfes Q R (Fr40) Mit kampfe Z  $\cdot$ ze] zen O (Fr22) on M \textbf{18} mich] nun I Mit L M Q (Fr39) (Fr40) \textbf{19} Aschalun] ascalun G (L) M R (Z) (Fr39) ashaluͤn I ascaluͯn Q Ascalv̂n Fr22 \textbf{20} Tschanfenzun] tschanvenzun G schanfanzuͤn I thschanvezvn O tshanfentvn L scanfenzcun M schampfenzuͯn Q schanfenzun R tschanzfenzvn Z Tschanfanzv̂n Fr22 tshanfencvn Fr39 \textbf{21} heize] hiesze L herze M hies R  $\cdot$ Kingrimursel] kingrunmursel I kýngrýmvrsel L [kyngrymusel]: kyngrymursel M kringruͯn ursel Q kẏngrimursel R kyngrimur sel Fr40 \textbf{22} hêr] \textit{om.} Z  $\cdot$ snel] hel I \textbf{23} er] ern I (Q) (Z) (Fr39) (Fr40) \textbf{24} enwelle] musz Q (R) (Z) muze Fr40  $\cdot$ dâ kampf] den campfe O (L) (R) (Fr39) das kamph M den kampf do Q kampf Z den kampf da Fr40 \textbf{25} ich gibe] Ovch gib ich O (L) (M) (Q) (R) (Z) (Fr39) (Fr40)  $\cdot$ im] \textit{om.} O  $\cdot$ al daz] al dis L allis M alles daz R das Fr39 \textbf{26} niwan] Nv wan L Nymat Q  $\cdot$ von mîn] \textit{om.} O von myns M (R) vor min Q  $\cdot$ eines] eines >von< O eigen M eingen R \textbf{27} triuwen] truwe M \textbf{28} kampfes] ringes R \textbf{29} al der] aller der die I aller der O aber die L Fr39 aldy M al [der]: die R alle der die Z \textbf{30} wan] an I  $\cdot$ weiz] \textit{om.} Z  $\cdot$ selbe] selben M \newline
\end{minipage}
\hspace{0.5cm}
\begin{minipage}[t]{0.5\linewidth}
\small
\begin{center}*T
\end{center}
\begin{tabular}{rl}
 & Beakurs al vaste bat.\\ 
 & Der gast stuont an sîner stat.\\ 
 & er sprach: "mir biutet kampf ein man,\\ 
 & des ich deheine künde hân.\\ 
5 & ich hân ouch niht ze sprechene dar.\\ 
 & \textbf{sterke}, küene, wol gevar,\\ 
 & \hspace*{-.7em}\big| hât er diu volleclîche,\\ 
 & \hspace*{-.7em}\big| \textbf{unde ist} getriuwe unde rîche,\\ 
 & er mac borgen deste baz.\\ 
10 & ich \textbf{hân} \textbf{gegen im} deheinen haz.\\ 
 & er was mîn hêrre unde \textbf{ouch} mîn mâc,\\ 
 & durch den ich hebe disen bâc.\\ 
 & unser vetere \textbf{bruodere} hiezen,\\ 
 & die \textbf{niht} ein ander liezen.\\ 
15 & \textbf{sô hôher} man gekrœnet wart\\ 
 & \textbf{nie}, \textbf{er}n hete im vollen art\\ 
 & \textbf{im kampfes} rede ze bieten,\\ 
 & \textbf{mit} râche gegen im \textbf{ze} nieten.\\ 
 & ich bin ein vürste ûz Ascalun,\\ 
20 & der lantgrâve von Tschampfenzun,\\ 
 & unde heize Kyngrimursel.\\ 
 & ist hêr Gawan lobes \textbf{hel},\\ 
 & \textbf{er}\textbf{n} mac sich niht entsagen,\\ 
 & er\textbf{n} \textbf{welle} \textbf{den} \textbf{kampf dâ} gegen mir tragen.\\ 
25 & \textbf{ich gib}im vride über al daz lant,\\ 
 & niht wan von mîn eines hant.\\ 
 & mit triuwen ich \textbf{im} vride geheize\\ 
 & ûzer\textit{h}alp des kampfes kreize.\\ 
 & got hüete al der ich lâze hie,\\ 
30 & wen eines: er weiz wol selbe wie."\\ 
\end{tabular}
\scriptsize
\line(1,0){75} \newline
T U V W \newline
\line(1,0){75} \newline
\textbf{1} \textit{Majuskel} T  \textbf{2} \textit{Majuskel} T  \textbf{19} \textit{Initiale} V  \newline
\line(1,0){75} \newline
\textbf{1} Beakurs] Beakuͦrs U [Beakvr*]: Beakvrs V Beachurs W  $\cdot$ al] al zuͦ U \textbf{3} er] Der W \textbf{6} sterke] [*]: Starc V Starck W \textbf{8} \textit{Versfolge 324.7-8} W  \textbf{7} unde ist] \textit{om.} W \textbf{10} hân] enhan V \textbf{11} er] Fr W  $\cdot$ ouch] \textit{om.} W \textbf{13} bruodere] [*]: gebruͦder V  $\cdot$ hiezen] [wâren]: hiezen T \textbf{14} niht] nit des U [*]: nihtes V \textbf{15} gekrœnet] nie gecroͤnet V nie kroͤnet W \textbf{16} Nie er in hete in voller art U  $\cdot$ [*]: Jch enhette in wol uollen art V · In hette voͤlliglicher art W \textbf{17} im] Jn U V (W) \textbf{18} mit] Mein W  $\cdot$ ze] \textit{om.} V \textbf{19} Ascalun] Ascalv̂n T Aschaluͦn U astalun W \textbf{20} Tschampfenzun] Tschampfenzv̂n T Tschamfenzuͦn U Scanphenzvn V schanfenzun W \textbf{21} unde] Ich W  $\cdot$ Kyngrimursel] kuͦngrimorsel U kẏngrimursel V kingrimursel W \textbf{22} hel] [*]: snel V schnel W \textbf{23} ern mac] Er [*mag]: enmag V Er mag W  $\cdot$ niht] anders niht V (W) \textbf{24} welle] [*]: mvͤse V  $\cdot$ dâ] do U W \textbf{25} gibim vride] gen im vreden U  $\cdot$ al daz] alle daz U alles daz V alles W \textbf{26} niht wan] [N*]: Nvwan V Nicht W \textbf{28} ûzerhalp] v̂zer talp T Vier halp U \textbf{29} al] aller U V W \textbf{30} wen] Was U  $\cdot$ selbe] selbes W \newline
\end{minipage}
\end{table}
\end{document}
