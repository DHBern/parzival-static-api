\documentclass[8pt,a4paper,notitlepage]{article}
\usepackage{fullpage}
\usepackage{ulem}
\usepackage{xltxtra}
\usepackage{datetime}
\renewcommand{\dateseparator}{.}
\dmyyyydate
\usepackage{fancyhdr}
\usepackage{ifthen}
\pagestyle{fancy}
\fancyhf{}
\renewcommand{\headrulewidth}{0pt}
\fancyfoot[L]{\ifthenelse{\value{page}=1}{\today, \currenttime{} Uhr}{}}
\begin{document}
\begin{table}[ht]
\begin{minipage}[t]{0.5\linewidth}
\small
\begin{center}*D
\end{center}
\begin{tabular}{rl}
\textbf{155} & Der helt was zornes dræte.\\ 
 & er sluog in, daz im wæte\\ 
 & \textit{v}ome schafte ûzer swarten bluot.\\ 
 & Parzival, der \textbf{knappe} guot,\\ 
5 & stuont al zornic ûf dem plân.\\ 
 & \textbf{sîn gabylôt begreif} er sân.\\ 
 & dâ der helm unt diu barbier\\ 
 & \textbf{sich} \textbf{locheten} \textbf{ob dem} härsnier,\\ 
 & \multicolumn{1}{l}{ - - - }\\ 
 & \multicolumn{1}{l}{ - - - }\\ 
 & durch\textbf{z ouge} \textbf{in sneit} daz gabylôt\\ 
10 & unt durch den nac, \textbf{sô} daz er tôt\\ 
 & viel, \textbf{der valscheite} widersaz.\\ 
 & wîbe siufzen, herzen jâmers kraz\\ 
 & gab Ithers tôt von Gaheviez,\\ 
 & \textbf{der} wîben \textbf{nazziu ougen} liez.\\ 
15 & swelhiu sîner minne enpfant,\\ 
 & durch die vreude ir was gerant\\ 
 & \textbf{unt} ir schimpf \textbf{entschumpfieret},\\ 
 & gein der \textbf{riuhe} gecondwieret.\\ 
 & Parzival, der tumbe,\\ 
20 & kêrten \textbf{dicke} al umbe.\\ 
 & er kunde im ab \textbf{geziehen} niht\\ 
 & - daz was ein wunderlîch geschiht -\\ 
 & \textbf{helmes snüere} \textbf{noch} \textbf{sîniu} schinnelier.\\ 
 & mit sînen blanken handen fier\\ 
25 & kund ers niht \textbf{ûf} gestricken\\ 
 & noch sus her ab gezwicken.\\ 
 & \textbf{vil} dicke\textbf{rs} doch versuochte,\\ 
 & wîsheit der unberuochte.\\ 
 & \textit{\begin{large}D\end{large}}az ors unt \textbf{daz} pferdelîn\\ 
30 & erhuoben einen \textbf{sô hôhen} grîn,\\ 
\end{tabular}
\scriptsize
\line(1,0){75} \newline
D \newline
\line(1,0){75} \newline
\textbf{1} \textit{Majuskel} D  \textbf{29} \textit{Initiale} D  \newline
\line(1,0){75} \newline
\textbf{3} vome] wome D \textbf{13} Ithers] Jthers D  $\cdot$ Gaheviez] Gahevîez D \textbf{29} Daz] ÷az \textit{nachträglich korrigiert zu:} Daz D \newline
\end{minipage}
\hspace{0.5cm}
\begin{minipage}[t]{0.5\linewidth}
\small
\begin{center}*m
\end{center}
\begin{tabular}{rl}
 & der helt was zornes dræte.\\ 
 & er sluoc \textit{in}, daz ime wæte\\ 
 & von dem schafte ûzer swarte\textit{n} bluot.\\ 
 & Parcifal, der \textbf{knappe} guot,\\ 
5 & stuont al zornic ûf dem plân.\\ 
 & \textbf{sîn gabilôt begreif} er sân.\\ 
 & d\textit{â} der helm und diu barbier\\ 
 & \textbf{sich} \textbf{locheten} \textbf{ob dem} hersenier,\\ 
 & \multicolumn{1}{l}{ - - - }\\ 
 & \multicolumn{1}{l}{ - - - }\\ 
 & durch \textbf{daz ouge} \textbf{in sneit} daz gabilôt\\ 
10 & und durch den nac, daz er \textbf{lac}. tôt\\ 
 & viel \textbf{der valscheit} widersaz.\\ 
 & wîbe siufzen, herzen jâmers kra\textit{z}\\ 
 & gap I\textit{t}hers tôt von Gah\textit{e}v\textit{ie}z,\\ 
 & \textbf{der} wîben \textbf{ouge nazzen} liez.\\ 
15 & welhiu sîner minne enpfan\textit{t},\\ 
 & durch die vröude ir was gerant,\\ 
 & ir schimpf \textbf{wart} \textbf{geschumpfieret},\\ 
 & gegen der \textbf{riuhe} gecundewieret.\\ 
 & Parcifal, der tumbe,\\ 
20 & \textit{k}êrte in \textbf{dicke} alumbe.\\ 
 & er kunde ime abe \textbf{ziehen} niht\\ 
 & - daz was ein wunderlîch geschiht -\\ 
 & \textbf{helmes snüere} \textbf{noch} \textbf{sîniu} schinnelier.\\ 
 & mit sînen blanken handen fier\\ 
25 & kund ers niht \textbf{ûf} gestricken\\ 
 & noch sus her abe gezwicken.\\ 
 & \textbf{vil} dicke \textbf{es} doch versuochte\\ 
 & wîsheit der unberuochte.\\ 
 & \dag danne ers\dag  und \textbf{d\textit{a}z} pferdelîn\\ 
30 & erhuoben einen \textbf{solhen} grîn,\\ 
\end{tabular}
\scriptsize
\line(1,0){75} \newline
m n o \newline
\line(1,0){75} \newline
\newline
\line(1,0){75} \newline
\textbf{2} in] \textit{om.} m \textbf{3} ûzer] vff der o  $\cdot$ swarten] swarttem m \textbf{4} Parcifal] Parcipfal n \textbf{6} begreif] ergreiff n o \textbf{7} dâ] Do m n o  $\cdot$ diu] der n \textbf{8} sich locheten] Sicheten o  $\cdot$ ob] vff n o \textbf{9} sneit] \textit{om.} n  $\cdot$ daz] sin n da o \textbf{10} lac] lag vor ir n \textbf{12} herzen] hertze n  $\cdot$ kraz] craft m \textbf{13} Ithers] ichers m ichters n  $\cdot$ Gaheviez] gahiveis m o gahwies n \textbf{14} ouge nazzen] nasz ougen n (o) \textbf{15} enpfant] enpfang m \textbf{17} geschumpfieret] [enps]: entschumppfieret n entschiempfieret o \textbf{18} riuhe] ruwe o  $\cdot$ gecundewieret] gedondieret n o \textbf{19} Parcifal] Parcival m Parcipfal n \textbf{20} kêrte] Derte m  $\cdot$ dicke] deck o \textbf{21} ziehen] geziehen n o \textbf{25} gestricken] gestricket o \textbf{27} es] ers n o \textbf{29} und] \textit{om.} o  $\cdot$ daz] des m \newline
\end{minipage}
\end{table}
\newpage
\begin{table}[ht]
\begin{minipage}[t]{0.5\linewidth}
\small
\begin{center}*G
\end{center}
\begin{tabular}{rl}
 & der helt was zornes dræte.\\ 
 & er sluog in, daz im wæte\\ 
 & vome schafte ûz der swarten bluot.\\ 
 & Parzival, der \textbf{helt} guot,\\ 
5 & stuont al zornic ûf dem plân.\\ 
 & \textbf{zem gabilôte greif} er sân.\\ 
 & dâ der helm und diu barbier\\ 
 & \textbf{sich} \textbf{lûchent} \textbf{umbe den} härsnier,\\ 
 & \multicolumn{1}{l}{ - - - }\\ 
 & \multicolumn{1}{l}{ - - - }\\ 
 & durch \textbf{daz ouge} \textbf{in sneit} daz gabilôt\\ 
10 & unt durch den nac, \textbf{sô} daz er tôt\\ 
 & viel, \textbf{der v\textit{e}lsche} widersaz.\\ 
 & wîbe siuften, herzen jâmers kraz\\ 
 & gap Ithers tôt vo\textit{n} Kaheviez,\\ 
 & \textbf{der} wîben \textbf{nazziu ougen} liez.\\ 
15 & swelhiu sîner minne enpfant,\\ 
 & durch die vröude ir was gerant\\ 
 & \textbf{unde} ir schimpf \textbf{enschumpfiert},\\ 
 & gein der \textbf{riwe} gecondewiert.\\ 
 & Parzival, der tumbe,\\ 
20 & kêrt in \textbf{dicke} al umbe.\\ 
 & er kunde im abe \textbf{geziehen} niht\\ 
 & - daz was ein wunderlîch geschiht -\\ 
 & \textbf{helmsnüere} \textit{\textbf{noch}} \textit{\textbf{diu}} tschillier.\\ 
 & mit sînen blanken handen fier\\ 
25 & kunde ers niht \textbf{abe} gestricken\\ 
 & noch sus her abe gezwicken.\\ 
 & \textbf{vil} dicke\textbf{\textit{r}z} doch versuohte,\\ 
 & wîsheit der unberuohte.\\ 
 & daz ors unde \textbf{sîn} pferdelîn\\ 
30 & erhuoben einen \textbf{sô hôhen} grîn,\\ 
\end{tabular}
\scriptsize
\line(1,0){75} \newline
G I O L M Q R Z Fr36 \newline
\line(1,0){75} \newline
\textbf{9} \textit{Initiale} M   $\cdot$ \textit{Capitulumzeichen} L  \textbf{15} \textit{Initiale} I  \textbf{27} \textit{Initiale} I O Z Fr36  \textbf{29} \textit{Initiale} L  \newline
\line(1,0){75} \newline
\textbf{2} im wæte] im vf wate L es im we tette R \textbf{3} schafte] schopfe O  $\cdot$ swarten] swarte I O L Z  $\cdot$ bluot] daz blvͦt O \textbf{4} Parzival] Parzifal I Parcifal O L Z Parzeval M Partzifal Q Parczifal R  $\cdot$ helt] chnappe I (M) (Z) \textbf{5} stuont al] Wart also Q  $\cdot$ dem] den O M dē Q \textbf{6} zem] Sin L  $\cdot$ greif] begreif L \textbf{7} dâ] Do Q  $\cdot$ diu] der I M Z daz L  $\cdot$ barbier] visier R \textbf{8} lûchent] lvhten O locherton L lochten M (Q) (Z) lochrete R  $\cdot$ umbe den] ob dem O L R Z uff dem M (Q)  $\cdot$ härsnier] barbir R \textbf{9} ouge] \textit{om.} Z  $\cdot$ in] im L \textit{om.} R  $\cdot$ daz gabilôt] sin Gabilot I der gabilot Z \textbf{10} sô] also I \textit{om.} L  $\cdot$ tôt] lac tot I \textbf{11} viel] Wie M  $\cdot$ der] daz L  $\cdot$ velsche] valsche G O M (Q) valsches I (L) (R) falscheit Z  $\cdot$ widersaz] wider sazt I wider scaz M \textbf{12} wîbe] Bibes O (L) (M) Weyben Q  $\cdot$ siuften] schufften M susszen Q  $\cdot$ herzen] hercze R  $\cdot$ kraz] craft I (R) tratz Q \textbf{13} Ithers] iters I Jthers O jhters L (R) Jters M ichers Q Z  $\cdot$ von] vo G  $\cdot$ Kaheviez] gahaviez G Gahafiez I kahaviez O kachaviez M [gahevitz]: gahevietz Q kacheveis R \textbf{14} der] den L  $\cdot$ wîben nazziu ougen] wibes avgen naziv O weybes auge nasse Q (R) \textbf{15} swelhiu] Welche L (M) Q (R) \textbf{16} die vröude ir] der frævde er O der frewde Q in Z \textbf{17} enschumpfiert] was entschvmpfieret O \textbf{18} gecondewiert] chonduwieret I \textbf{19} Parzival] [parzifal]: Parzifal I Parcifal O L Z Partziual M Partzifal Q Parczifal R \textbf{20} al] \textit{om.} I O M Q \textbf{21} er] ern I (L) (M) (Q) (Z)  $\cdot$ im abe] ab im I in abe L (R) (Z)  $\cdot$ geziehen] zihen Q \textbf{23} helmsnüere] Her swur Q Helms schnuͦr R  $\cdot$ noch diu] \textit{om.} G noch siniv O (L) (M) (Q) (R) (Z)  $\cdot$ tschillier] sinillir Q schaler R \textbf{24} mit] [Min]: Mit G  $\cdot$ blanken handen] handen blanchen vnd I [blaͤnchen]: blanchen [ar*]: armen O wisen henden R  $\cdot$ fier] fer R \textbf{25} ers] er Q  $\cdot$ abe] vf O L (M) (Q) (R) Z (Fr36)  $\cdot$ gestricken] gestriktten R \textbf{26} noch sus] Sust och nit R  $\cdot$ gezwicken] gewiken R \textbf{27} vil] ÷il O Wie R  $\cdot$ dickerz] ditchez G  $\cdot$ doch] \textit{om.} M \textbf{28} wîsheit] Mit wiͤsheit O Witze L  $\cdot$ der] den M  $\cdot$ unberuohte] vngebruchte Q \textbf{29} sîn] \textit{om.} I daz L Z \textbf{30} einen sô] einem I einē so L M  $\cdot$ hôhen] grossen R \newline
\end{minipage}
\hspace{0.5cm}
\begin{minipage}[t]{0.5\linewidth}
\small
\begin{center}*T (U)
\end{center}
\begin{tabular}{rl}
 & der helt was zornes drâte.\\ 
 & er sluoc in, daz im wâte\\ 
 & vonme schafte ûz der swarten bluot.\\ 
 & Parcifal, der \textbf{knappe} guot,\\ 
5 & stuont al zornic ûf dem plân.\\ 
 & \textbf{zuo dem gabilôte greif} er sân.\\ 
 & d\textit{â} der helm und diu ba\textit{rb}ier\\ 
 & \textbf{lochent} \textbf{ob dem} hersenier,\\ 
 & dô begunde er sîn pflegen\\ 
 & mit schüzzen, biz daz den degen\\ 
 & durch \textbf{diu ougen} \textbf{versneit} daz gabelôt\\ 
10 & und durch den nac, \textbf{sô} daz er tôt\\ 
 & viel, \textbf{der valsches} widersaz.\\ 
 & wîbe siufzen, herzen jâmers kraz\\ 
 & gap Ithers tôt von Kaheviez,\\ 
 & \textbf{den} wîben \textbf{nazziu ouge\textit{n}} liez.\\ 
15 & welchiu sîner minne e\textit{n}pfant,\\ 
 & durch die vröude ir was gerant\\ 
 & \textbf{und} ir schimpf \textbf{enschumpfieret},\\ 
 & gein der \textbf{riuwe} gecundewieret.\\ 
 & \begin{large}P\end{large}arcifal, der tumbe,\\ 
20 & kêrtin \textbf{ofte} alumbe.\\ 
 & er \textbf{en}kunde \textbf{ez} im abe \textbf{geziehen} niht\\ 
 & - daz was ein wunderlîch geschiht -\\ 
 & \textbf{helmsnüere}, schillier.\\ 
 & mit sînen blanken handen fier\\ 
25 & kunders niht \textbf{ûf} gestricken\\ 
 & noch sus her abe gezwicken.\\ 
 & dicke \textbf{er ez} doch versuochte,\\ 
 & wîsheit d\textit{e}r unberuochte.\\ 
 & daz ors und \textbf{sîn} pferdelîn\\ 
30 & erhuoben einen \textbf{sô hôhen} grîn,\\ 
\end{tabular}
\scriptsize
\line(1,0){75} \newline
U V W T \newline
\line(1,0){75} \newline
\textbf{1} \textit{Majuskel} T  \textbf{7} \textit{Majuskel} T  \textbf{15} \textit{Majuskel} T  \textbf{19} \textit{Initiale} U V W T  \textbf{29} \textit{Majuskel} T  \newline
\line(1,0){75} \newline
\textbf{3} der swarten] [*]: siner swarte V \textbf{4} Parcifal] parzifal V (T) Partzifal W \textbf{5} dem] den V W \textbf{6} dem] seinem W \textbf{7} dâ] Do U W  $\cdot$ helm] heln heln V  $\cdot$ diu] das W die T  $\cdot$ barbier] banier U \textbf{8} lochent] sich [loͤs*]: loͤsent V Sich loͤcherten W losten T \textbf{8} \textit{Die Verse 155.8¹-8² fehlen} T   $\cdot$ pflegen] sere pflegen W \textbf{8} schüzzen] schiessen W  $\cdot$ biz] vnz V  $\cdot$ den] der W \textbf{9} diu] \textit{om.} V (W)  $\cdot$ versneit] in sneit V [s*]: snêit im T  $\cdot$ daz] sein W \textbf{10} und] \textit{om.} V  $\cdot$ sô] \textit{om.} T \textbf{11} viel nider von des kindes craft T  $\cdot$ valsches widersaz] valsche widerfart W \textbf{12} Vroͮwen svften sigehaft T  $\cdot$ wîbe] wiben V Weibes W  $\cdot$ kraz] zart W \textbf{13} Ithers] Jthers U T yters V ythers W  $\cdot$ Kaheviez] Caheviez U kahevies V gahafies W \textbf{14} den] Der V W (T)  $\cdot$ ougen] auge U \textbf{15} welchiu] Swelche V [*]: Swelhiv T  $\cdot$ enpfant] erfant U \textbf{16} die vröude ir] der vroude T \textbf{17} und ir schimpf] Vil ir schinphes V \textbf{19} Parcifal] PArzifal U V T PArtzifal W \textbf{20} ofte] [*]: dicke V dicke T \textbf{21} er enkunde ez] Ern kondes V (W) \textbf{23} schillier] [*]: noch schillier V noch schellier T \textbf{25} kunders] Kunde er W kvnderz T  $\cdot$ ûf] ab W \textbf{26} sus] \textit{om.} T  $\cdot$ gezwicken] gewikken V \textbf{27} dicke] vil [*]: dicke T  $\cdot$ er ez] ers V T \textbf{28} der unberuochte] dar vmb ruͦchte U (V) \textbf{30} erhuoben] hvͦben T  $\cdot$ sô hôhen] solhen T \newline
\end{minipage}
\end{table}
\end{document}
