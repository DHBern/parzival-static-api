\documentclass[8pt,a4paper,notitlepage]{article}
\usepackage{fullpage}
\usepackage{ulem}
\usepackage{xltxtra}
\usepackage{datetime}
\renewcommand{\dateseparator}{.}
\dmyyyydate
\usepackage{fancyhdr}
\usepackage{ifthen}
\pagestyle{fancy}
\fancyhf{}
\renewcommand{\headrulewidth}{0pt}
\fancyfoot[L]{\ifthenelse{\value{page}=1}{\today, \currenttime{} Uhr}{}}
\begin{document}
\begin{table}[ht]
\begin{minipage}[t]{0.5\linewidth}
\small
\begin{center}*D
\end{center}
\begin{tabular}{rl}
\textbf{709} & \begin{large}L\end{large}ieht wâren ir covertiure.\\ 
 & Parzival, der gehiure,\\ 
 & wart in bêden hern geprîset sô,\\ 
 & sîne vriwent \textbf{des} mohten wesen vrô.\\ 
5 & Si jâhen in Gramoflanzes her,\\ 
 & daz ze keiner zît sô wol ze wer\\ 
 & nie k\textit{œ}m rîter dechein,\\ 
 & den \textbf{diu sunne ie} überschein.\\ 
 & \textbf{swaz} ze bêden \textbf{sîten} \textbf{dâ} wære getân,\\ 
10 & den prîs müeser al eine hân.\\ 
 & dennoch si \textbf{sîn} erkanten niht,\\ 
 & dem ieslîch munt \textbf{dâ} prîses giht.\\ 
 & Gramoflanze si rieten,\\ 
 & er m\textit{ö}hte wol enbieten\\ 
15 & Artuse, daz er næme war,\\ 
 & daz dechein \textbf{ander} man ûz sîner schar\\ 
 & gein im kœme durch vehten,\\ 
 & daz er im sande den rehten:\\ 
 & Gawan, \textbf{des} \textbf{künec} Lotes sun,\\ 
20 & \textbf{mit} \textbf{dem} wolt er den kampf tuon.\\ 
 & Die boten wurden \textbf{dan} gesant,\\ 
 & zwei \textbf{wîsiu} kint \textbf{höfsch} erkant.\\ 
 & der künec sprach: "\textbf{nû sult ir} \textbf{spehen},\\ 
 & \textbf{wem} ir dâ prîses wellet jehen\\ 
25 & under al \textbf{den} clâren vrouwen.\\ 
 & ir sult ouch sunder schouwen,\\ 
 & bî welher Bene sitze.\\ 
 & nemt daz \textbf{in} iwer witze,\\ 
 & in \textbf{welhen gebærden} \textbf{diu} sî,\\ 
30 & \textbf{won} ir vreude oder trûren bî,\\ 
\end{tabular}
\scriptsize
\line(1,0){75} \newline
D Fr66 \newline
\line(1,0){75} \newline
\textbf{1} \textit{Initiale} D  \textbf{5} \textit{Majuskel} D  \textbf{21} \textit{Majuskel} D  \newline
\line(1,0){75} \newline
\textbf{2} Parzival] Parcifal D \textbf{5} Gramoflanzes] Gramoflanzs D \textbf{7} kœm] chom D \textbf{14} möhte] mohte D \textbf{19} Lotes] Lots D \newline
\end{minipage}
\hspace{0.5cm}
\begin{minipage}[t]{0.5\linewidth}
\small
\begin{center}*m
\end{center}
\begin{tabular}{rl}
 & lieht wâren ir c\textit{o}vertiure.\\ 
 & Parcifal, der gehiure,\\ 
 & wart in beiden  gebrîset sô,\\ 
 & sîne vriunt \textbf{des}  wesen vrô.\\ 
5 & si jâhen in Gramolanzes her,\\ 
 & daz zuo keiner zît sô wol zuo wer\\ 
 & nie \dag kein\dag  ritter dekein,\\ 
 & den \textbf{ie diu sunne} überschein.\\ 
 & \textbf{waz} zuo beiden \textbf{sîten} wære getân,\\ 
10 & den prîs müest er aleine hân.\\ 
 & dannoch si \textbf{in} erkanten niht,\\ 
 & dem ieglîch munt \textbf{d\textit{â}} prîses giht.\\ 
 & Gramolanz si rieten,\\ 
 & er m\textit{ö}hte wol enbieten\\ 
15 & Artuse, daz er næme war,\\ 
 & daz kein \textbf{ander} man ûz sîner schar\\ 
 & gegen im kæme durch vehten,\\ 
 & daz er im sante den reh\textit{t}en:\\ 
 & Gawane\textit{n}, \textbf{künic} Lotes sun,\\ 
20 & \textbf{mit} \textbf{dem} wolt er den kampf tuon.\\ 
 & die boten wurden \textbf{dan} gesant,\\ 
 & zwei \textbf{höfschiu} kint \textbf{wîse} erkant.\\ 
 & der künic sprach: "\textbf{ir sullet} \textbf{spehen},\\ 
 & \textbf{wem} ir d\textit{â} prîses wellet jehen\\ 
25 & under allen clâren vrouwen.\\ 
 & ir solt ouch sunder schouwen,\\ 
 & bî welicher Bene sitze.\\ 
 & nemt daz \textbf{mit} iuwer witze,\\ 
 & in \textbf{welicher gebærde} \textbf{si} sî,\\ 
30 & \textbf{won} ir vröude oder trûren bî,\\ 
\end{tabular}
\scriptsize
\line(1,0){75} \newline
m n o Fr69 \newline
\line(1,0){75} \newline
\newline
\line(1,0){75} \newline
\textbf{1} covertiure] conuertuͯr m (n) (o) \textbf{5} si] Do o  $\cdot$ Gramolanzes] gramolantzes m n o \textbf{7} dekein] do kein n \textbf{11} in erkanten] erkante n in erkante o \textbf{12} dâ] do m n o \textbf{13} Gramolanz] Gramolantz m n Gramolancz o  $\cdot$ rieten] ritten o \textbf{14} möhte] mohtte m (o) \textbf{15} Artuse] Artúse o \textbf{18} rehten] rechen m \textbf{19} Gawanen] Gawanes m  $\cdot$ Lotes] lotz m n locz o \textbf{21} wurden] worent o \textbf{22} kint] \textit{om.} o \textbf{24} dâ] do m n o \textbf{25} allen] allen den n (o) \textbf{27} Bene] ::: Fr69 \textbf{28} nemt] Vnde nement n  $\cdot$ mit] in n o Fr69 \textbf{29} jn welhen geberen dv́ si Fr69 \textbf{30} vröude] frouͯiden o \newline
\end{minipage}
\end{table}
\newpage
\begin{table}[ht]
\begin{minipage}[t]{0.5\linewidth}
\small
\begin{center}*G
\end{center}
\begin{tabular}{rl}
 & lieht wâren ir covertiure.\\ 
 & \begin{large}P\end{large}arcival, der gehiure,\\ 
 & wart in bêden hern gebrîset sô,\\ 
 & sîne vriunt \textbf{es} mohten wesen vrô.\\ 
5 & si jâhen in Gramoflanzes her,\\ 
 & daz zuo deheiner zît sô wol ze wer\\ 
 & nie kœme rîter dehein,\\ 
 & den \textbf{diu sunne ie} überschein.\\ 
 & \textbf{swie ez} ze bêden \textbf{tagen} \textbf{dâ} wære getân,\\ 
10 & den brîs müeser aleine hân.\\ 
 & dannoch si \textbf{sîn} erkanden niht,\\ 
 & dem ieslîch munt \textbf{des} brîses giht.\\ 
 & Gramoflanz si rieten,\\ 
 & er möht wol enbieten\\ 
15 & Artus, daz er næme war,\\ 
 & daz dehein man ûz sîner schar\\ 
 & gein im kœme durch vehten,\\ 
 & daz er im sande den rehten:\\ 
 & Gawan, \textbf{des} \textbf{künic} Lotes sun,\\ 
20 & \textbf{gein} \textbf{dem} wolde er den kampf tuon.\\ 
 & die boten wurden \textbf{dan} gesant,\\ 
 & zwei \textbf{wîsiu} kint \textbf{höfsch} erkant.\\ 
 & der künic sprach: "\textbf{nû sult ir} \textbf{ouch sehen},\\ 
 & \textbf{welher} ir dâ brîses welt jehen\\ 
25 & under al \textbf{den} clâren vrouwen.\\ 
 & ir sült ouch sunder schouwen,\\ 
 & bî welher Bene sitze.\\ 
 & nemt daz \textbf{in} iwer witze,\\ 
 & in \textbf{welher gebære} \textbf{si} sî,\\ 
30 & \textbf{wone} ir vröude oder trûren bî,\\ 
\end{tabular}
\scriptsize
\line(1,0){75} \newline
G I L M Z Fr18 \newline
\line(1,0){75} \newline
\textbf{1} \textit{Initiale} L Z Fr18  \textbf{2} \textit{Initiale} G  \textbf{19} \textit{Initiale} I  \newline
\line(1,0){75} \newline
\textbf{2} Parcival] Parcifal G Z Fr18 Parzifal I L M \textbf{3} hern] her M \textbf{4} es] sin I des L Z Fr18 \textbf{5} jâhen] sprachin M  $\cdot$ in] \textit{om.} I  $\cdot$ Gramoflanzes] gramorflanzes M gramoflantzes Z \textbf{7} kœme] kom L \textbf{8} überschein] beschein L vber scheine Z \textbf{9} swie] Wie L (M)  $\cdot$ wære] wart L \textbf{10} müeser] muͦs er I (Z) (Fr18) er muͯste L \textbf{12} des] da Z  $\cdot$ brîses] besten L \textbf{13} Gramoflanz] Gramoflanze L Gramorflanze M Gramoflantz Z Gramoflantze Fr18 \textbf{14} möht] mochte L M (Z) (Fr18)  $\cdot$ enbieten] einbieten L irbieten M \textbf{15} Artus] Artuse I L (Fr18) \textbf{17} gein im kœme] Gein im kom L Queme gein im Z  $\cdot$ durch] nach M \textbf{19} künic] kunges I (L)  $\cdot$ Lotes] lots G lotis M \textbf{20} gein dem] Mit im L (M) Z Fr18  $\cdot$ den] dē M dem Fr18 \textbf{21} dan] dar I \textbf{23} ouch] \textit{om.} I L M Z Fr18  $\cdot$ sehen] spehen L (M) Z (Fr18) \textbf{24} welher ir] welhem ir I Welher Fr18 \textbf{25} al] alle M \textbf{26} ouch] doch L \textbf{27} sitze] siczen M \textbf{29} welher gebære] welchen gebarden L (M) (Z) (Fr18)  $\cdot$ si] diu I (L) (M) (Z) disiv Fr18 \textbf{30} vröude] vrouden M \newline
\end{minipage}
\hspace{0.5cm}
\begin{minipage}[t]{0.5\linewidth}
\small
\begin{center}*T
\end{center}
\begin{tabular}{rl}
 & lieht wâren ir covertiure.\\ 
 & Parcifal, der gehiure,\\ 
 & wart in beiden hern geprîset sô,\\ 
 & sîne vriunt \textbf{des} mohten wesen vrô.\\ 
5 & si jâhen in Gramoflanzes her,\\ 
 & daz \textit{zuo} dekeiner zît sô wol zuo wer\\ 
 & nie k\textit{æ}m rîter dehein,\\ 
 & den \textbf{diu sunne ie} überschein.\\ 
 & \textbf{waz} zuo beiden \textbf{sîten} \textbf{d\textit{â}} wære getân,\\ 
10 & den prîs mües er aleine hân.\\ 
 & dannoch si \textbf{sîn} erkanten niht,\\ 
 & dem ieclîch munt \textbf{des} prîses giht.\\ 
 & \begin{large}G\end{large}ramoflanze si rieten,\\ 
 & er m\textit{ö}hte wol enbieten\\ 
15 & Artuse, daz er næme war,\\ 
 & daz dekein man ûz sîner schar\\ 
 & gein im k\textit{æ}m durch vehten,\\ 
 & daz er im sante den rehten:\\ 
 & Gawanen, \textbf{des} \textbf{küneges} Lotes suon,\\ 
20 & \textbf{mit} \textbf{im} wolt er den kampf tuon.\\ 
 & die boten wurden gesant,\\ 
 & zwei \textbf{wîsiu} kint \textbf{hövesch} erkant.\\ 
 & der künec sprach: "\textbf{nû solt ir} \textbf{spehen},\\ 
 & \textbf{welher} ir d\textit{â} prîses w\textit{e}l\textit{t} jehen\\ 
25 & under al \textbf{den} clâren vrouwen.\\ 
 & ir solt ouch sunder schouwen,\\ 
 & bî welher Bene sitze.\\ 
 & nemet daz \textbf{in} iuwer witze,\\ 
30 & \hspace*{-.7em}\big| \textbf{wont} ir vreude oder trûren bî\\ 
 & \hspace*{-.7em}\big| \textbf{oder} in \textbf{welhen gebærden} \textbf{si} sî,\\ 
\end{tabular}
\scriptsize
\line(1,0){75} \newline
U V W Q R \newline
\line(1,0){75} \newline
\textbf{5} \textit{Initiale} W  \textbf{13} \textit{Initiale} U V  \newline
\line(1,0){75} \newline
\textbf{1} ir covertiure] die kouentᵫre R \textbf{2} Parcifal] Parzifal U Parzefal V Her partzifal W Partzifal Q Parczifal R \textbf{4} vriunt] fúnd R  $\cdot$ des mohten] des moͤhten V mochtens Q des R  $\cdot$ wesen] waren R \textbf{5} Gramoflanzes] gramaflanzes V gramoflantzes W Gramaflanczes R \textbf{6} zuo] \textit{om.} U W  $\cdot$ dekeiner] keine W  $\cdot$ wol] vil R \textbf{7} kæm] quam U (Q) keine R  $\cdot$ dehein] keine W \textbf{8} ie] nye Q \textbf{9} waz] Swaz V  $\cdot$ beiden] beten Q  $\cdot$ dâ] do U V W Q \textbf{10} mües] muͦß W  $\cdot$ er] \textit{om.} R \textbf{12} des prîses] do preses W \textbf{13} Gramoflanze] Gramaflanze V Gramoflantzen W Grameflancze R \textbf{14} möhte] mochte U Q  $\cdot$ enbieten] enbeitten R \textbf{15} Artuse] Artus Q R  $\cdot$ er] der Q \textbf{16} man ûz] [*]: ander man vz V man vser R \textbf{17} kæm] quam U \textbf{18} den] der Q \textbf{19} Gawan kúnig lottes sun W  $\cdot$ Gawanen] Gawan Q Gawin R \textbf{20} im] dem W \textbf{21} wurden] wurdent dan V (Q) (R) \textbf{22} wîsiu] wise R  $\cdot$ hövesch] hofflich R \textbf{24} dâ] do U V W zu R  $\cdot$ prîses] prise R  $\cdot$ welt] wil U \textbf{25} al den] al der Q (R) \textbf{26} ouch] eúch W \textbf{30} \textit{Versfolge 709.29-30} W Q R   $\cdot$ wont] Wone W (Q) (R) \textbf{29} oder] \textit{om.} W Q R  $\cdot$ welhen] welcher W  $\cdot$ gebærden] geberde W begerden Q  $\cdot$ si] [*]: die V die W Q (R) \newline
\end{minipage}
\end{table}
\end{document}
