\documentclass[8pt,a4paper,notitlepage]{article}
\usepackage{fullpage}
\usepackage{ulem}
\usepackage{xltxtra}
\usepackage{datetime}
\renewcommand{\dateseparator}{.}
\dmyyyydate
\usepackage{fancyhdr}
\usepackage{ifthen}
\pagestyle{fancy}
\fancyhf{}
\renewcommand{\headrulewidth}{0pt}
\fancyfoot[L]{\ifthenelse{\value{page}=1}{\today, \currenttime{} Uhr}{}}
\begin{document}
\begin{table}[ht]
\begin{minipage}[t]{0.5\linewidth}
\small
\begin{center}*D
\end{center}
\begin{tabular}{rl}
\textbf{409} & ez was grôz unt swære.\\ 
 & man sagt von ir diu mære:\\ 
 & \textit{\begin{large}S\end{large}}wen dâ \textbf{erreichte} ir wurfes swanc,\\ 
 & der strûchte âne sînen danc.\\ 
5 & diu küneginne rîche\\ 
 & streit dâ ritterlîche.\\ 
 & bî Gawane si werlîche schein,\\ 
 & daz diu \textbf{koufwîp} ze Tolenstein\\ 
 & an der vasnaht nie baz gestriten,\\ 
10 & wan si tuontz \textbf{von} gampelsiten\\ 
 & unt müent ân nôt ir lîp.\\ 
 & swâ \textbf{harnaschrâmec} wirt ein wîp,\\ 
 & diu hât ir rehtes vergezzen,\\ 
 & \textbf{sol man} ir kiusche mezzen,\\ 
15 & sine tuo ez denne durch \textbf{ir} triwe.\\ 
 & Antikonien riwe\\ 
 & wart ze Schanpfanzun erzeiget\\ 
 & unt ir hôher muot geneiget.\\ 
 & in strîte si sêre weinde.\\ 
20 & wol si daz bescheinde,\\ 
 & daz \textbf{vriwentlîch} liebe ist stæte.\\ 
 & waz Gawan dô tæte?\\ 
 & swenne im diu muoze geschach,\\ 
 & daz er die magt rehte ersach,\\ 
25 & ir munt, ir ougen unt ir nasen,\\ 
 & baz geschicket an spizze \textbf{hasen},\\ 
 & ich wæne, den gesâhet ir nie,\\ 
 & denne si was dort unt hie\\ 
 & zwischen der hüffe unt \textbf{ir} brust.\\ 
30 & \textbf{minne gerende} gelust\\ 
\end{tabular}
\scriptsize
\line(1,0){75} \newline
D \newline
\line(1,0){75} \newline
\textbf{3} \textit{Initiale} D  \newline
\line(1,0){75} \newline
\textbf{3} Swen] ÷wen D \textbf{16} Antikonien] Antikonîen D \textbf{17} Schanpfanzun] Scanpfanzvn D \newline
\end{minipage}
\hspace{0.5cm}
\begin{minipage}[t]{0.5\linewidth}
\small
\begin{center}*m
\end{center}
\begin{tabular}{rl}
 & ez was grôz und swære.\\ 
 & man sagt von ir diu mære:\\ 
 & w\textit{e}n dâ \textbf{reichete} ir wurfes swanc,\\ 
 & der strûchete âne sînen danc.\\ 
5 & diu kün\textit{i}g\textit{în} rîche\\ 
 & streit dâ ritterlîche.\\ 
 & bî Gawane si \textbf{sô} werlîche schein,\\ 
 & daz diu \textbf{kampfwîp} ze Tole\textit{n}stein\\ 
 & an der vastnaht nie baz gestriten,\\ 
10 & wand si tuont ez \textbf{von} gampelsiten\\ 
 & und müent âne nôt ir lîp.\\ 
 & wâ \textbf{harnaschrâmic} wirt ein wîp,\\ 
 & diu hât ir rehtes vergezzen.\\ 
 & \textbf{man sol} ir kiusche mezzen,\\ 
15 & si entuo ez \textbf{niht} danne durch \textbf{ir} triuwe.\\ 
 & Anticonien riuwe\\ 
 & wart ze Schanfanzun erzeiget\\ 
 & und ir hôher muot geneiget.\\ 
 & in strîte si sêre weinde.\\ 
20 & wol si daz bescheinde,\\ 
 & daz \textbf{vriuntlîch} liebe ist stæte.\\ 
 & waz Gawan dô tæte?\\ 
 & wenne ime diu muoze geschach,\\ 
 & daz er die maget rehte ersach,\\ 
25 & ir munt, ir ougen und ir nasen,\\ 
 & baz geschicket an spizze \textbf{hasen},\\ 
 & ich wæne, den gesâhet ir nie,\\ 
 & danne si was dort und hie\\ 
 & zwischen der huf und \textbf{ir} brust.\\ 
30 & \textbf{mînen gerenden} gelust\\ 
\end{tabular}
\scriptsize
\line(1,0){75} \newline
m n o \newline
\line(1,0){75} \newline
\newline
\line(1,0){75} \newline
\textbf{1} grôz] grosse n  $\cdot$ und] [od]: vnd n \textbf{2} sagt] sagete n \textbf{3} wen] Wan m o Wenne n  $\cdot$ dâ] do n o \textbf{4} strûchete] struchet n (o) \textbf{5} künigîn] kunnge m \textbf{6} dâ] do n o \textbf{7} Gawane] gawan n o \textbf{8} kampfwîp] kouff wip n (o)  $\cdot$ Tolenstein] toleristein m tolen stein n dol entscheyn o \textbf{11} müent] maget o \textbf{13} hât] hette n o  $\cdot$ ir rehtes] irs rechten n o \textbf{15} entuo] du n (o)  $\cdot$ niht] \textit{om.} n o \textbf{16} Anticonien] Antikonien m Antitonie n o \textbf{17} Schanfanzun] scanfanzún m scanfanzun n scamfazẏm o  $\cdot$ erzeiget] erzoget o \textbf{19} si] so n o  $\cdot$ weinde] [geneiget]: weinde n \textbf{22} Gawan] gawann o \textbf{26} geschicket] beschicket o \textbf{27} gesâhet] gesohen m n gesehet o \textbf{30} gelust] er geluͯst o \newline
\end{minipage}
\end{table}
\newpage
\begin{table}[ht]
\begin{minipage}[t]{0.5\linewidth}
\small
\begin{center}*G
\end{center}
\begin{tabular}{rl}
 & ez was grôz unde swære.\\ 
 & man saget von ir diu mære:\\ 
 & swen dâ \textbf{erreichte} ir wurfes swanc,\\ 
 & der strûhte âne sînen danc.\\ 
5 & \begin{large}D\end{large}iu küniginne rîche\\ 
 & streit dâ rîterlîche.\\ 
 & bî Gawane si werlîchen schein,\\ 
 & daz diu \textbf{koufwîp} ze Tollenstein\\ 
 & an der vasnaht nie baz gestriten,\\ 
10 & wan si tuont ez \textbf{von} gampelsiten\\ 
 & unde müent âne nôt ir lîp.\\ 
 & swâ \textbf{harnaschrâmec} wirt ein wîp,\\ 
 & diu hât ir rehtes vergezzen,\\ 
 & \textbf{sol man} ir kiusche mezzen,\\ 
15 & sine tuo ez dane durch triwe.\\ 
 & Antikonien riwe\\ 
 & wart ze Tschanfenzun erzeiget\\ 
 & unt ir hôher muot geneiget.\\ 
 & in strîte si sêre weinde.\\ 
20 & \textit{w}ol si daz bescheinde,\\ 
 & daz \textbf{vriuntlîch} liebe ist stæte.\\ 
 & waz Gawan dô tæte?\\ 
 & swenne im diu muoze geschach,\\ 
 & daz er die maget reht ersach,\\ 
25 & ir munt, ir ougen unde ir nasen,\\ 
 & baz geschict an spizze \textbf{hasen},\\ 
 & ich wæne, den gesâhet ir nie,\\ 
 & danne si was dort unde hie\\ 
 & zwischen der huf unde \textbf{ir} brust.\\ 
30 & \textbf{minne gernde} gelust\\ 
\end{tabular}
\scriptsize
\line(1,0){75} \newline
G I O L M Q R Z Fr22 \newline
\line(1,0){75} \newline
\textbf{3} \textit{Initiale} I O L Z Fr22   $\cdot$ \textit{Capitulumzeichen} R  \textbf{5} \textit{Initiale} G  \textbf{17} \textit{Initiale} I  \newline
\line(1,0){75} \newline
\textbf{1} \textit{Die Verse 370.13-412.12 fehlen} Q  \textbf{2} diu] daz I \textbf{3} swen] Swem I ÷wen O Wen L Wan M Wenn R  $\cdot$ erreichte] erraichet O reichtten R reicht Z  $\cdot$ ir] ein R \textbf{4} Der sturczte sich one gedank R  $\cdot$ strûhte] strvht O (Z)  $\cdot$ âne] vnder Z \textbf{6} \textit{nach 409.6:} Vnd gar vnuerwegenliche R   $\cdot$ dâ] do R \textbf{7} Gawane] Gawan I O L (M) R (Z)  $\cdot$ werlîchen] zewer I  $\cdot$ schein] stuͦnd R \textbf{8} daz] Als R  $\cdot$ ze Tollenstein] zetollen steine G zetolenstein I ze tolnstein O zuͯ Toln stein L zcu tolnsteyn M koͯln stond R zv Tollenstein Z zv Thohnstein Fr22 \textbf{9} nie] ward nie R \textbf{10} gampelsiten] gamel sitten R \textbf{11} müent] mvͦn::: Fr22 \textbf{12} swâ] Wa L M Wa von R \textbf{14} kiusche] chuͤste I \textbf{15} sine] Sy R  $\cdot$ tuo ez] tvnz L  $\cdot$ dane] \textit{om.} O  $\cdot$ durch] dvrch ir O (L) (M) (Z) von R \textbf{16} Antikonien] Anticonien I Antigonîen O Anthichonien M Antykonien R \textbf{17} ze Tschanfenzun] zetschanfenzun G zeshanphanzuͤn I ze schamfazvͦn O zcu schamphenzcun M ze schanfernzen R zv Tschanfanzvn Z \textbf{20} wol] vil wol G  $\cdot$ daz] daz sie Z \textbf{21} vriuntlîch] frivntliche O \textbf{22} dô] da I L M Z \textbf{23} swenne] Wenne L (M) (R) \textbf{24} er] Gawain I \textbf{25} unde] \textit{om.} I R  $\cdot$ nasen] nase O \textbf{26} an] ainem L one R  $\cdot$ hasen] hase O \textbf{27} den] der R  $\cdot$ gesâhet] geseht O gesaget L  $\cdot$ ir] er R \textbf{29} der] \textit{om.} R ir Z  $\cdot$ ir] der I (O) (L) (R) \textbf{30} minne gernde] minne gernder I Minnen gernde Z \newline
\end{minipage}
\hspace{0.5cm}
\begin{minipage}[t]{0.5\linewidth}
\small
\begin{center}*T
\end{center}
\begin{tabular}{rl}
 & ez was grôz unde swære.\\ 
 & man saget von ir diu mære:\\ 
 & swen dâ \textbf{erreichete} ir wurfes swanc,\\ 
 & der strûhte âne sînen danc.\\ 
5 & \begin{large}D\end{large}iu küneginne rîche\\ 
 & streit dâ rîterlîche.\\ 
 & bî Gawane si werlîche schein,\\ 
 & daz diu \textbf{kampfwîp} ze Tolenstein\\ 
 & an der vastnaht nie baz gestriten,\\ 
10 & wan si tuont ez \textbf{mit} gampelsiten\\ 
 & unde müent âne nôt ir lîp.\\ 
 & swâ \textbf{râmec harnasch} wirt ein wîp,\\ 
 & di\textit{u} hât ir rehtes vergezzen,\\ 
 & \textbf{sol man} ir kiusche mezzen,\\ 
15 & si entuoz danne durch \textbf{ir} triuwe.\\ 
 & Antickonien riuwe\\ 
 & wart ze Tschampfenzun erzeiget\\ 
 & unde ir hôher muot geneiget.\\ 
 & in strîte si sêre weinde.\\ 
20 & wol si daz bescheinde,\\ 
 & daz \textbf{vriunt} lieb\textit{e} ist stæte.\\ 
 & Waz Gawan dô tæte?\\ 
 & swenne im diu muoze geschach,\\ 
 & daz er die maget rehte ersach,\\ 
25 & ir munt, ir ougen unde ir nase,\\ 
 & baz geschicket an spi\textit{z}ze \textbf{hase},\\ 
 & ich wæne, den gesâhet ir nie,\\ 
 & danne si was dort unde hie\\ 
 & zwischen der huf unde \textbf{der} brust.\\ 
30 & \textbf{minne gernde} gelust\\ 
\end{tabular}
\scriptsize
\line(1,0){75} \newline
T U V W \newline
\line(1,0){75} \newline
\textbf{3} \textit{Initiale} V  \textbf{5} \textit{Initiale} T U W  \textbf{22} \textit{Majuskel} T  \newline
\line(1,0){75} \newline
\textbf{1} unde] oder W \textbf{3} \textit{Die Verse 409.3-4 fehlen} U   $\cdot$ swen dâ] Swen do V Wenn do W \textbf{4} strûhte] strackte W \textbf{6} dâ] do U V W \textbf{7} Gawane] gawâne T gawan W  $\cdot$ si] sv́ so V \textbf{8} kampfwîp] koͮfwip V (W)  $\cdot$ Tolenstein] Tolnstein U \textbf{10} mit] von U V W \textbf{12} swâ râmec harnasch] Wo (Swa V ) harnasch ramec U (V) (W) \textbf{13} diu] die T  $\cdot$ ir rehtes] irs rechten U \textbf{14} sol man] Man sol V \textbf{15} entuoz] in duͦn iz U (W) \textbf{16} Antickonien] Antikonien U V W  $\cdot$ riuwe] treúwe W \textbf{17} Tschampfenzun] Tscampfenzv̂n T Tschamfenzuͦn U schanpfanzvn V \textbf{21} vriunt liebe] vrivnt liebiv T frvntliche liebe V (W) \textbf{23} swenne] Wan U (W) \textbf{25} nase] nasen U V W \textbf{26} spizze] spitze T U  $\cdot$ hase] hasen U V W \textbf{27} den gesâhet] do gesehet W \textbf{29} der huf] dem huͦff W \textbf{30} gernde] gerender W \newline
\end{minipage}
\end{table}
\end{document}
