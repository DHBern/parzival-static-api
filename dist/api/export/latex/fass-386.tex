\documentclass[8pt,a4paper,notitlepage]{article}
\usepackage{fullpage}
\usepackage{ulem}
\usepackage{xltxtra}
\usepackage{datetime}
\renewcommand{\dateseparator}{.}
\dmyyyydate
\usepackage{fancyhdr}
\usepackage{ifthen}
\pagestyle{fancy}
\fancyhf{}
\renewcommand{\headrulewidth}{0pt}
\fancyfoot[L]{\ifthenelse{\value{page}=1}{\today, \currenttime{} Uhr}{}}
\begin{document}
\begin{table}[ht]
\begin{minipage}[t]{0.5\linewidth}
\small
\begin{center}*D
\end{center}
\begin{tabular}{rl}
\textbf{386} & \textbf{\textit{\begin{large}L\end{large}}yppaut, der vürste}, des landes wirt,\\ 
 & sîn \textbf{manlîch} ellen niht verbirt;\\ 
 & \textbf{gein} dem streit der künec von Gors.\\ 
 & dâ muosen beidiu liute unt ors\\ 
5 & von \textbf{geschütze} lîden pîne,\\ 
 & dâ die Kahetine\\ 
 & unt die sarjande von Semblidac,\\ 
 & ieslîcher sîner künste pflac.\\ 
 & turkople kunden wenken;\\ 
10 & die burgære muosen denken,\\ 
 & waz vîende von ir letzen schiet.\\ 
 & si heten sarjande \textbf{ad} piet.\\ 
 & ir zingel wâren sô behuot,\\ 
 & als dâ man noch daz beste tuot.\\ 
15 & swelch wert man dâ den lîp verlôs,\\ 
 & Obien zorn \textbf{unsanfte er} kôs,\\ 
 & wande ir tumbiu lôsheit\\ 
 & vil liute brâht in arbeit.\\ 
 & \textbf{wes} engalt \textbf{der vürste} Lyppaut?\\ 
20 & \textbf{sîn hêrre}, der \textbf{alte} künec Schaut,\\ 
 & hetes in erlâzen gar.\\ 
 & dô \textbf{begunde} müeden \textbf{ouch} \textbf{di\textit{u}} schar.\\ 
 & dennoch streit vaste Melyacanz.\\ 
 & ob sîn schilt wære ganz?\\ 
25 & \textbf{des} \textbf{en}was niht hende breit beliben.\\ 
 & dô het in \textbf{verre} \textbf{hin dan} \textbf{getriben}\\ 
 & der herzoge Kardefablet.\\ 
 & der turnei al stille stêt\\ 
 & ûf einem blüemînem plân.\\ 
30 & dô kom \textbf{ouch} mîn hêr Gawan.\\ 
\end{tabular}
\scriptsize
\line(1,0){75} \newline
D \newline
\line(1,0){75} \newline
\textbf{1} \textit{Initiale} D  \newline
\line(1,0){75} \newline
\textbf{1} Lyppaut] ÷yppaot D \textbf{7} Semblidac] Semblydach D \textbf{19} Lyppaut] Lyppâot D \textbf{20} Schaut] Scôt D \textbf{22} diu] di D \newline
\end{minipage}
\hspace{0.5cm}
\begin{minipage}[t]{0.5\linewidth}
\small
\begin{center}*m
\end{center}
\begin{tabular}{rl}
 & \textbf{\begin{large}D\end{large}er vürste Lippo\textit{u}t}, des landes wirt,\\ 
 & \textit{s}în \textbf{manlîch} ellen niht verbirt;\\ 
 & \textbf{gegen} dem str\textit{e}it der künic von Gros.\\ 
 & dô muosen beidiu liut und ros\\ 
5 & von \textbf{geschützen} lîden pîne,\\ 
 & dâ die Kahetine\\ 
 & und die sarjande von Semblidac,\\ 
 & ieglîcher sîner kü\textit{n}s\textit{t}e pflac.\\ 
 & turkopele kunden wenken;\\ 
10 & die burgære muosen denken,\\ 
 & waz vîende von ir letzen schiet.\\ 
 & si h\textit{e}ten sarjande \textbf{ad} piet.\\ 
 & ir zingele wâren sô behuot,\\ 
 & als dâ man noch daz beste \textit{tuot}.\\ 
15 & welich wert man d\textit{â} den lîp verlôs,\\ 
 & Obi\textit{e}n zorn \textbf{er unsanfte} kôs,\\ 
 & wand ir tumbiu lôsheit\\ 
 & vil liute brâhte in arbeit.\\ 
 & \textbf{wes} e\textit{n}galte \textbf{der vürste} Lipp\textit{ou}t?\\ 
20 & \textbf{sîn hêrre}, der \textbf{alte} künic Schout,\\ 
 & het\textit{e} es in erlâzen gar.\\ 
 & dô \textbf{begunden} müeden \textbf{ouch} \textbf{die} schar.\\ 
 & dannoch streit vaste Melia\textit{g}anz.\\ 
 & ob sîn schilt wære ganz?\\ 
25 & \textbf{des} \textbf{en}w\textit{a}s niht hende breit bli\textit{b}en.\\ 
 & dô hete in \textbf{verre} \textbf{vertriben}\\ 
 & der herzoge Kardefablet.\\ 
 & der turnei al stille stêt\\ 
 & ûf einem blüemînen plân.\\ 
30 & dô kam mîn hêr Gawan.\\ 
\end{tabular}
\scriptsize
\line(1,0){75} \newline
m n o \newline
\line(1,0){75} \newline
\textbf{1} \textit{Illustration mit Überschrift:} Also gawan mit meliantzen streit vor dem her vnd in in not bracht n   $\cdot$ \textit{Initiale} m n  \newline
\line(1,0){75} \newline
\textbf{1} \textit{Die Verse 386.1-23 fehlen} o   $\cdot$ Lippout] lippoat m lippaot n \textbf{2} sîn] Din m  $\cdot$ ellen] ellende n \textbf{3} dem] \textit{om.} n  $\cdot$ streit] strit m  $\cdot$ Gros] gors n \textbf{4} muosen] muͯssen m muͯsten n \textbf{5} geschützen] geschútze n \textbf{6} dâ] Do n  $\cdot$ Kahetine] kahettine m kahethin n \textbf{7} Semblidac] Semblidag m n \textbf{8} künste] kusche m kúnffte n \textbf{9} turkopele] Turcoppole m \textbf{10} muosen] mussen m muͯsten n  $\cdot$ denken] gedencken n \textbf{12} heten] htten m \textbf{14} dâ man noch] do man n  $\cdot$ tuot] \textit{om.} m \textbf{15} dâ] do m n \textbf{16} Obien] Obian m Obẏen n \textbf{18} liute] lichte n \textbf{19} engalte] egaltte m  $\cdot$ Lippout] lippoat m lippaot n \textbf{20} Schout] scovt m scaot n \textbf{21} hete] Hettes m \textbf{23} Meliaganz] meliacancz m meliacantz n \textbf{25} des] Das o  $\cdot$ enwas] enweis m  $\cdot$ bliben] bliden m \textbf{26} vertriben] hin dan getriben n o \textbf{27} Kardefablet] karde fablet m kardaflabet o \textbf{28} turnei] storneẏ o  $\cdot$ al] alle n \textbf{29} blüemînen] bluͯmelin o \textbf{30} mîn] [mir]: min m \newline
\end{minipage}
\end{table}
\newpage
\begin{table}[ht]
\begin{minipage}[t]{0.5\linewidth}
\small
\begin{center}*G
\end{center}
\begin{tabular}{rl}
 & \textbf{Libaut}, des landes wirt,\\ 
 & sîn \textbf{manheit} ellen niht verbirt;\\ 
 & \textbf{mit} dem streit der künic von Gors.\\ 
 & dâ muosen beidiu liute unde ors\\ 
5 & von \textbf{geschôze} lîden pîne,\\ 
 & dâ die Kahadine\\ 
 & unt die sarjande von Semlidac,\\ 
 & ieslîcher sîner künste pflac.\\ 
 & turkopel kunden wenken;\\ 
10 & die burgære muosen denken,\\ 
 & waz vînde von ir letze schiet.\\ 
 & si heten sarjande \textbf{an} piet.\\ 
 & \multicolumn{1}{l}{ - - - }\\ 
 & \multicolumn{1}{l}{ - - - }\\ 
15 & swelch wert man dâ den lîp verlôs,\\ 
 & Obien zorn \textbf{unsanfte er} kôs,\\ 
 & wan ir tumbiu lôsheit\\ 
 & vil liute brâht in arbeit.\\ 
 & \textbf{es} engalt \textbf{ir vater} Libaut.\\ 
20 & \textbf{sîn hêrre}, der künic Tschaut,\\ 
 & het es in erlâzen gar.\\ 
 & dô \textbf{begunden} müeden \textbf{al} \textbf{die} schar.\\ 
 & dannoch streit vaste Meliahganz,\\ 
 & op sîn schilt wære ganz?\\ 
25 & \textbf{sîn} was niht hende breit beliben.\\ 
 & dô hete in \textbf{verre} \textbf{hin dan} \textbf{getriben}\\ 
 & der herzoge Kardefablet.\\ 
 & der turnei al stille stêt\\ 
 & ûf einem blüemînen plân.\\ 
30 & dô kom mîn hêr Gawan.\\ 
\end{tabular}
\scriptsize
\line(1,0){75} \newline
G I O L M Q R Z \newline
\line(1,0){75} \newline
\textbf{1} \textit{Initiale} I O L M R Z  \newline
\line(1,0){75} \newline
\textbf{1} \textit{Die Verse 370.13-412.12 fehlen} Q   $\cdot$ Libaut] ÷ybavt O Lýbavt L Libayt M Rybant R Lybavt Z  $\cdot$ des] der fvrste des O (L) (M) (R) der kvnic des Z \textbf{2} sîn] Des O (L) M R Z  $\cdot$ manheit] mannes M  $\cdot$ ellen] ellens I \textbf{3} dem] \textit{om.} O L  $\cdot$ Gors] chors I goͤrs O (Z) Gorz L gros M (R) \textbf{4} dâ] mit dem I Do R \textbf{5} geschôze] geschoszen L stritten R \textbf{6} dâ] Do R  $\cdot$ die] div O  $\cdot$ Kahadine] kaledine M Kachedine R \textbf{7} Semlidac] semlidach G O L samblidac I Semblidak R \textbf{9} kunden] kunde Z \textbf{10} denken] gedenchen L [wenkin]: deken M \textbf{11} von] vor R  $\cdot$ schiet] [liez]: schiet M \textbf{12} an piet] anphiet G anbiet I apiet O (L) (M) (R) (Z) \textbf{13} \textit{Die Verse 386.13-14 fehlen} G I   $\cdot$ Jr cingel warn so behvͦt O (L) (M) (R) (Z) \textbf{14} Als da man (Als man da L Alda man M ) noch daz beste tvͦt O (M) (R) (Z) \textbf{15} swelch] Welch L (M) (R)  $\cdot$ wert] wart O  $\cdot$ dâ] \textit{om.} I O \textbf{16} Obien] ob in I Obyen O R Z  $\cdot$ unsanfte] er vnsanft R \textbf{17} tumbiu] tumbe R  $\cdot$ lôsheit] Lolheit L boszheit R \textbf{18} brâht] brach M \textbf{19} es engalt] ezn clagt I Des engalt O Wes engalt Z  $\cdot$ Libaut] Lybavt O Z Lýbavt L libayt M Lybant R \textbf{20} künic] alte [k*]: kvͯnig L alde konnig M (R) (Z)  $\cdot$ Tschaut] Scohut I tschovt O Tshavt L scoyt M schant R Tschavt Z \textbf{22} dô] Da M Z  $\cdot$ begunden müeden] begund muͯde R  $\cdot$ al die] alle I ovch die O (L) (M) (R) (Z) \textbf{23} Meliahganz] Miliaganz I Melyakanz O Meliahkanz L (Z) Meliachkansz M Meliahkancz R \textbf{24} wære] noch iht were I \textbf{25} sîn] Esn O (L) (M) (Z) Es R  $\cdot$ hende] \textit{om.} O \textbf{26} dô] Da M Z  $\cdot$ dan] \textit{om.} M \textbf{27} Kardefablet] kat defablet G kat de fablet I kadefablet L \textbf{29} einem] einen O R einē L (M)  $\cdot$ blüemînen] pluͤmen I (O) pluͯmigen R \textbf{30} dô] Da M Z  $\cdot$ kom] qvam ouch Z  $\cdot$ Gawan] [ywan]: Gawan I \newline
\end{minipage}
\hspace{0.5cm}
\begin{minipage}[t]{0.5\linewidth}
\small
\begin{center}*T
\end{center}
\begin{tabular}{rl}
 & \textbf{\begin{large}L\end{large}ybaut, der vürste}, des landes wirt,\\ 
 & sîn \textbf{manlîch} ellen niht verbirt;\\ 
 & \textbf{mit} dem streit der künec von Gors.\\ 
 & dô muosen beidiu liute unde ors\\ 
5 & von \textbf{schozze} lîden pîne,\\ 
 & dâ die Kahedine\\ 
 & unde die sarjande von Semblidac,\\ 
 & iegeslîcher sîner künste pflac.\\ 
 & turkopel kunden wenken;\\ 
10 & die burgære muosen denken,\\ 
 & waz vîende von ir letze schiet.\\ 
 & si heten sarjande \textbf{ah} piet.\\ 
 & \multicolumn{1}{l}{ - - - }\\ 
 & \multicolumn{1}{l}{ - - - }\\ 
15 & swelch wert man dâ den lîp verlôs,\\ 
 & Obyen zorn \textbf{unsanfte er} kôs,\\ 
 & wand ir tumbiu lôsheit\\ 
 & vil liute brâhte in arbeit.\\ 
 & \textbf{des} engalt \textbf{ouch} \textbf{\textit{ir} vater} Lybaut.\\ 
20 & der \textbf{alte} künec Tschaut\\ 
 & hetes in erlâzen gar.\\ 
 & dô \textbf{begunde} müeden \textbf{ouch} \textbf{diu} schar.\\ 
 & Dannoch streit vaste Meliahganz.\\ 
 & ob sîn schilt wære ganz?\\ 
25 & \textbf{des} was niht h\textit{e}nde breit bliben.\\ 
 & dô hetin \textbf{hin dan} \textbf{getriben}\\ 
 & der herzoge Gardefablet.\\ 
 & der turnei alstille stêt\\ 
 & ûf einem blüemînen plân.\\ 
30 & Dô kom \textbf{ouch} mîn hêr Gawan.\\ 
\end{tabular}
\scriptsize
\line(1,0){75} \newline
T V W \newline
\line(1,0){75} \newline
\textbf{1} \textit{Initiale} T W  \textbf{23} \textit{Majuskel} T  \textbf{30} \textit{Majuskel} T  \newline
\line(1,0){75} \newline
\textbf{1} Lybaut] Lẏbaut V LYbout W \textbf{2} sîn manlîch] Des manhait W \textbf{3} künec] fúrste W  $\cdot$ Gors] gorß W \textbf{4} dô] Da V \textbf{5} schozze] schossen V geschosse W \textbf{6} dâ] Do W \textbf{7} Semblidac] [semblida*]: semblidag V semblidag W \textbf{12} ah piet] [*]: apiet V alpiet W \textbf{13} \textit{Die Verse 386.13-14 fehlen} T   $\cdot$ Jr zingel waren so behvͦt V (W) \textbf{14} Also do men noch daz beste tvͦt V (W) \textbf{15} swelch] Welich W  $\cdot$ dâ] \textit{om.} V do W \textbf{19} des] [*]: Wez V Es W  $\cdot$ ouch] \textit{om.} V W  $\cdot$ ir vater] sin vater T [*]: der fúrste V  $\cdot$ Lybaut] Lybâvt T lẏppaovt V lybout W \textbf{20} der] [*]: Sin herre der V Sein herre der W  $\cdot$ alte] \textit{om.} W  $\cdot$ Tschaut] Tscevt T [*]: schovt V tschout W \textbf{22} müeden ouch] oͮch mvͤden V er muͤden auch W \textbf{23} Meliahganz] [Melia*]: Meliahgânz T [*]: meliakanz V melyagantz W \textbf{25} des was] [D*waz]: Dez enwaz V Es enwas W  $\cdot$ hende] hinde T \textbf{26} hin dan] [*]: verre hindan V verre hindan W \textbf{27} Gardefablet] kardefablet W \textbf{29} blüemînen] geblvͤmeten V \textbf{30} ouch] \textit{om.} W \newline
\end{minipage}
\end{table}
\end{document}
