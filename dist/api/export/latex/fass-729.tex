\documentclass[8pt,a4paper,notitlepage]{article}
\usepackage{fullpage}
\usepackage{ulem}
\usepackage{xltxtra}
\usepackage{datetime}
\renewcommand{\dateseparator}{.}
\dmyyyydate
\usepackage{fancyhdr}
\usepackage{ifthen}
\pagestyle{fancy}
\fancyhf{}
\renewcommand{\headrulewidth}{0pt}
\fancyfoot[L]{\ifthenelse{\value{page}=1}{\today, \currenttime{} Uhr}{}}
\begin{document}
\begin{table}[ht]
\begin{minipage}[t]{0.5\linewidth}
\small
\begin{center}*D
\end{center}
\begin{tabular}{rl}
\textbf{729} & \begin{large}D\end{large}iu winde von dem huote.\\ 
 & Arnive, diu guote,\\ 
 & \textbf{Sangive} unt Cundrie,\\ 
 & die \textbf{hete} Artus gebeten ê\\ 
5 & an dirre suone teidinc.\\ 
 & swer \textbf{prüevet} daz vür kleiniu dinc,\\ 
 & der grœze, swaz er welle.\\ 
 & Jofreit, Gawans geselle,\\ 
 & vuorte die herzoginne lieht erkant\\ 
10 & underz poulûn an sîner hant.\\ 
 & diu pflac durch zuht der sinne,\\ 
 & die drî küneginne\\ 
 & lie si vor ir \textbf{gên} dar în.\\ 
 & die kuste Brandelidelin.\\ 
15 & Orgeluse in ouch \textbf{mit kusse} enpfienc.\\ 
 & Gramoflanz durch suone gienc\\ 
 & und ûf genâde gein \textbf{ir} dar.\\ 
 & ir \textbf{süezer} munt rôt gevar\\ 
 & den künec durch suone kuste,\\ 
20 & \textbf{dâr umbe} si \textbf{weinens} \textbf{luste}.\\ 
 & si \textbf{dâhte} an Cidegastes tôt.\\ 
 & dô twanc si wîplîchiu nôt\\ 
 & \textbf{nâch im} dennoch ir riwe.\\ 
 & welt ir des \textbf{iht} vür triwe?\\ 
25 & Gawan unt Gramoflanz\\ 
 & mit kusse ir suone \textbf{ouch} machten ganz.\\ 
 & Artus gab Itonje\\ 
 & Gramoflanz ze rehter ê.\\ 
 & dâ het er vil gedienet nâch.\\ 
30 & Bene was vrô, dô daz geschach.\\ 
\end{tabular}
\scriptsize
\line(1,0){75} \newline
D \newline
\line(1,0){75} \newline
\textbf{1} \textit{Initiale} D  \newline
\line(1,0){75} \newline
\textbf{2} Arnive] Arnîve D \textbf{3} Sangive] Sangîve D  $\cdot$ Cundrie] Cvndrîe D \textbf{27} Itonje] Jtoniê D \newline
\end{minipage}
\hspace{0.5cm}
\begin{minipage}[t]{0.5\linewidth}
\small
\begin{center}*m
\end{center}
\begin{tabular}{rl}
 & diu winde von dem huote.\\ 
 & A\textit{r}n\textit{iv}e, diu guote,\\ 
 & \textbf{Sangive} und Condrie,\\ 
 & die \textbf{het} Artus gebeten ê\\ 
5 & an diser suone tegedinc.\\ 
 & wer \textbf{brüevet} d\textit{az} vür kleiniu dinc,\\ 
 & der grœze, waz er welle.\\ 
 & Jofrit, Gawanes geselle,\\ 
 & vuorte die herzogîn lieht erkant\\ 
10 & under daz pavelûn an sîner hant.\\ 
 & diu pflac durch zuht der sinne,\\ 
 & die drîe küniginne\\ 
 & liez si vor ir \textbf{ougen} dar în.\\ 
 & die kuste Brandelidelin.\\ 
15 & Urgeluse in ouch \textbf{mit kusse} enpfienc.\\ 
 & Gramolantz durch suone gienc\\ 
 & und ûf gnâde gegen \textbf{in} dar.\\ 
 & ir \textbf{süezer} munt rôt gevar\\ 
 & den künic durch s\textit{uo}n\textit{e} kuste,\\ 
20 & \textbf{dâr umb} si \textbf{weinens} \textbf{luste}.\\ 
 & si \textbf{dâht} an Zidegastes tôt.\\ 
 & dô t\textit{w}anc si wîplîch nôt\\ 
 & \textbf{nâch im} dannoch ir riuwe.\\ 
 & wolt ir, des \textbf{jeht ir} vür triuwe.\\ 
25 & Gawan und Gramolantz\\ 
 & mit kusse ir suone \textbf{ouch} mach\textit{t}en \textit{g}anz.\\ 
 & Artus gap Itonie\\ 
 & Gramolantz zuo rehter ê.\\ 
 & d\textit{â} het er vil gedienet nâch.\\ 
30 & Bene was vrô, dô daz geschach.\\ 
\end{tabular}
\scriptsize
\line(1,0){75} \newline
m n o \newline
\line(1,0){75} \newline
\newline
\line(1,0){75} \newline
\textbf{2} Arnive] Arune m Arniwe n \textbf{3} Sangive] Sangwe n [Sanwe]: Sangwe o  $\cdot$ Condrie] cundrie o \textbf{6} daz] do m \textbf{13} în] im o \textbf{14} Brandelidelin] brandeledilen o \textbf{15} kusse] zorn o \textbf{16} Gramolantz] Gramolancz o \textbf{19} suone] sin m o \textbf{21} dâht] gedacht n  $\cdot$ Zidegastes] zitigastes n \textbf{22} twanc] tang m \textbf{24} jeht ir] jehen n echt ir o \textbf{25} Gramolantz] gramolancz o \textbf{26} ouch] auch auch o  $\cdot$ machten ganz] machen glantz m \textbf{27} Itonie] Ithonie n jtonie o \textbf{28} Gramolantz] Gramolancz o \textbf{29} dâ] Do m n o \newline
\end{minipage}
\end{table}
\newpage
\begin{table}[ht]
\begin{minipage}[t]{0.5\linewidth}
\small
\begin{center}*G
\end{center}
\begin{tabular}{rl}
 & \begin{large}D\end{large}iu winde von dem huote.\\ 
 & Arnive, diu guote,\\ 
 & \textbf{Sagive} unde Gundrie,\\ 
 & die \textbf{heten} Artus gebeten ê\\ 
5 & an dirre suone teidinc.\\ 
 & swer \textbf{prüeve} daz vür kleiniu dinc,\\ 
 & der grœze, swaz er welle.\\ 
 & Jofreit, Gawans geselle,\\ 
 & vuorte die herzoginne lieht erkant\\ 
10 & underz pavelûn an sîner hant.\\ 
 & diu pflac durch zuht der sinne,\\ 
 & die drî küneginne\\ 
 & lie si vor ir \textbf{gên} dar în.\\ 
 & die kuste Brandelidelin.\\ 
15 & Orgeluse in ouch \textbf{mit kusse} enpfienc.\\ 
 & Gramoflanz durch suone gienc\\ 
 & unde ûf gnâde gein \textbf{ir} dar.\\ 
 & ir \textbf{dicker} munt rôt gevar\\ 
 & den künic durch suone kuste,\\ 
20 & \textbf{des} si \textbf{doch} \textbf{wênec} \textbf{luste}.\\ 
 & si \textbf{dâhte} an Zidegastes tôt.\\ 
 & dâ twanc si wîplîchiu nôt\\ 
 & dannoch \textbf{in} ir riwe.\\ 
 & welt ir des \textbf{iht jehen} vür triwe?\\ 
25 & Gawan unde Gramoflanz\\ 
 & mit kusse ir suone macheten ganz.\\ 
 & Artus gab Itonie\\ 
 & Gramoflanz ze rehter ê.\\ 
 & dâ het er vil gedient nâch.\\ 
30 & Bene was vrô, dô daz geschach.\\ 
\end{tabular}
\scriptsize
\line(1,0){75} \newline
G I L M Z Fr20 Fr24 \newline
\line(1,0){75} \newline
\textbf{1} \textit{Initiale} G L Z Fr20  \textbf{5} \textit{Initiale} I  \textbf{25} \textit{Initiale} I  \textbf{27} \textit{Initiale} Fr24  \newline
\line(1,0){75} \newline
\textbf{1} Diu] ÷iv Fr20 \textbf{2} Arnive] arniue I  $\cdot$ guote] vil guͤte I \textbf{3} Sagive] seive G (M) Fr20 Saife I Seyve Z  $\cdot$ Gundrie] kvndrie G L (M) Z Fr24 Gundr::: I chundrie Fr20 \textbf{4} die] Sie M  $\cdot$ Artus] Artvsen L (M) \textbf{6} swer] Wer L M  $\cdot$ prüeve daz] bruͤuet daz I (M) pruͯfte daz L daz Fr20 :az prvͦue Fr24 \textbf{7} grœze] grvze Fr20  $\cdot$ swaz] waz L (M) \textbf{8} Jofreit] Jofreýt L Lofreit M Jofrit Fr20 \textbf{9} lieht] \textit{om.} L licht M \textbf{13} si] \textit{om.} L  $\cdot$ ir] yn M \textbf{14} Brandelidelin] brandalidelin I Brantlidelin L Brandlidelin M Fr24 \textbf{15} Orgeluse] Orgillvsie G Orguluse I Orgelýse L Orgelose M Orguluͦse Fr24 \textbf{16} Gramoflanz] Gramoflantz Z \textbf{17} ir] yn M \textbf{19} kuste] da kvste Z \textbf{20} luste] geluͯste L \textbf{21} Zidegastes] Citegastes G L zitegasstes I zcitegastes M Cidegastes Z Cydegastes Fr24 \textbf{22} dâ] do I \textbf{23} \textit{Versfolge 729.24-23} Z  \textbf{24} iht jehen] iehen I Z ieht L icht M geht Fr24 \textbf{25} Gramoflanz] Gramoflantz Z \textbf{26} kusse] chussen I (M)  $\cdot$ macheten] machint M \textbf{27} Itonie] Itonîe G Jconie Z \textbf{28} Gramoflanz] Gramoflanzen I Gramoflantz Z  $\cdot$ ze] mit I \textbf{29} er] \textit{om.} M \textbf{30} was vrô dô] vroide M \newline
\end{minipage}
\hspace{0.5cm}
\begin{minipage}[t]{0.5\linewidth}
\small
\begin{center}*T
\end{center}
\begin{tabular}{rl}
 & diu winde von dem huote.\\ 
 & Arnyve, diu guote,\\ 
 & \textbf{Seyve} und Kundrie,\\ 
 & die \textbf{hete} Artus gebeten ê\\ 
5 & an dirre suon\textit{e tei}dinc.\\ 
 & wer \textbf{prüevet} daz vür kleiniu dinc,\\ 
 & der grœze, waz er welle.\\ 
 & Jofreit, Gawanes geselle,\\ 
 & vuorte die herzoginne lieht erkant\\ 
10 & under daz pavelûn an sîner hant.\\ 
 & diu pflac durch zuht der sinne,\\ 
 & die drîe küneginne\\ 
 & liez si vor ir \textbf{gân} dar în.\\ 
 & die kuste Brandelidelin.\\ 
15 & Orgeluse in ouch \textbf{dar nâch} entvienc.\\ 
 & Gramoflanz durch suone gienc\\ 
 & und ûf gnâde gein \textbf{ir} dar.\\ 
 & ir \textbf{süezer} munt rôt gevar\\ 
 & den künec durch suone kuste,\\ 
20 & \textbf{des} si \textbf{doch} \textbf{wênic} \textbf{geluste}.\\ 
 & si \textbf{gedâhte} an Cydegastes tôt.\\ 
 & dô twanc si wîplîchiu nôt\\ 
 & dannoch \textbf{in} i\textit{r} riuwe.\\ 
 & wolt ir des \textbf{jehen} vür triuwe?\\ 
25 & Gawan und Gramoflanz\\ 
 & mit kusse ir suone machten ganz.\\ 
 & \begin{large}A\end{large}rtus gap Itonie\\ 
 & Gramoflanz zuo rehter ê.\\ 
 & d\textit{â} heter vil gedienet nâch.\\ 
30 & Bene was vr\textit{ô}, dô daz geschach.\\ 
\end{tabular}
\scriptsize
\line(1,0){75} \newline
U V W Q R \newline
\line(1,0){75} \newline
\textbf{25} \textit{Initiale} W  \textbf{27} \textit{Initiale} U V  \newline
\line(1,0){75} \newline
\textbf{2} Arnyve] arniue V (Q) Arnyue W Arnẏue R \textbf{3} Seyve] Seyue V W R Seyre Q  $\cdot$ Kundrie] kuͦndrie U \textbf{4} hete] hetten V  $\cdot$ Artus] [artus*]: artusen V \textbf{5} \textit{Versfolge 729.6-5} U W   $\cdot$ dirre] dise W  $\cdot$ teidinc] dise dinc U \textbf{6} wer] Swer V  $\cdot$ prüevet] prieue V (W) (Q) (R) \textbf{7} waz] swaz V wo W \textbf{8} Jofreit] Joffrit V Iofrid W Jofrid R  $\cdot$ Gawanes] herr gawans W Gawins R \textbf{9} lieht] licht Q \textbf{10} under] Vnd Q  $\cdot$ pavelûn] [*]: gezelt V \textbf{14} kuste] kusten R  $\cdot$ Brandelidelin] brandelidelein W brandlidelein Q \textbf{15} Orgeluse] Orguluse R  $\cdot$ ouch] \textit{om.} R \textbf{16} Gramoflanz] Gramaflanz V Gramoflantz W Q Gramoflancz R \textbf{20} [d*]: darvmbe sv́ weinenz luste V  $\cdot$ doch] so W  $\cdot$ wênic] lúczel R  $\cdot$ geluste] luste W Q R \textbf{21} gedâhte] gedach R  $\cdot$ Cydegastes] gẏdegastes V cytegastes W cydegastes Q Cidegastes R \textbf{22} wîplîchiu] liplich R \textbf{23} [*]: Nach im dennoch ir ruwe V  $\cdot$ ir] irm U \textbf{24} des] das W  $\cdot$ jehen] iecht W Q ich R  $\cdot$ vür triuwe] [*]: ir fúr truwe V \textbf{25} Gawan] Gawin R  $\cdot$ Gramoflanz] gramaflancz V gramoflantz W Q Gramoflancz R \textbf{26} ir] irn R \textbf{27} Itonie] Jtonie U R ẏconie V ytonie W Q \textbf{28} Gramoflanz] Gramaflancze V Gramoflantz W Gramoflantze Q Gramoflancz R  $\cdot$ zuo rehter] zvͦ der rehten V zu rechtte R \textbf{29} dâ] Do U V W Q R  $\cdot$ heter] hat er R \textbf{30} vrô] vrov U \textit{om.} Q \newline
\end{minipage}
\end{table}
\end{document}
