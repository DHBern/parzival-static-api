\documentclass[8pt,a4paper,notitlepage]{article}
\usepackage{fullpage}
\usepackage{ulem}
\usepackage{xltxtra}
\usepackage{datetime}
\renewcommand{\dateseparator}{.}
\dmyyyydate
\usepackage{fancyhdr}
\usepackage{ifthen}
\pagestyle{fancy}
\fancyhf{}
\renewcommand{\headrulewidth}{0pt}
\fancyfoot[L]{\ifthenelse{\value{page}=1}{\today, \currenttime{} Uhr}{}}
\begin{document}
\begin{table}[ht]
\begin{minipage}[t]{0.5\linewidth}
\small
\begin{center}*D
\end{center}
\begin{tabular}{rl}
\textbf{473} & \begin{large}D\end{large}er site ist niht dem Grâle reht.\\ 
 & dâ muoz der rîter unt der kneht\\ 
 & \textbf{bewart sîn} vor lôsheit.\\ 
 & diemuot \textbf{die} hôchvart überstreit.\\ 
5 & dâ \textit{w}on\textit{t} ein werdiu bruoderschaft;\\ 
 & die hânt mi\textit{t} \textbf{werlîcher} kraft\\ 
 & erwert mit ir handen\\ 
 & der \textbf{diet} von al \textbf{den} landen,\\ 
 & daz der Grâl ist \textbf{unerkennet},\\ 
10 & wan die dar sint benennet\\ 
 & ze Munsalvæsche ans Grâles schar.\\ 
 & wan einer kom \textbf{unbenennet} dar;\\ 
 & der selbe was ein tumber man\\ 
 & unt vuorte ouch sünde mit im dan,\\ 
15 & daz er niht zem wirte sprach\\ 
 & umben kumber, den er an im sach.\\ 
 & ich \textbf{en}\textbf{sol} niemen schelten,\\ 
 & doch muoz er \textbf{sünde} engelten,\\ 
 & daz er niht vrâgte des wirtes schaden.\\ 
20 & \textbf{er} was mit kumber sô \textbf{geladen},\\ 
 & ez \textbf{en}wart nie \textbf{erkant} \textbf{sô} hôher pîn.\\ 
 & dâ \textbf{von} kom \textit{r}oys Læhelin\\ 
 & ze Brumbane an den sê geriten.\\ 
 & durch tjoste het sîn dâ gebiten\\ 
25 & Lybbeals, der werde helt,\\ 
 & des tôt mit tjoste \textbf{was} erwelt.\\ 
 & er was \textbf{erborn} von Prienlascors.\\ 
 & Læhelin des heldes ors\\ 
 & dannen \textit{z}ôch mit sîner hant.\\ 
30 & dâ wart der rêroup \textbf{bekant}.\\ 
\end{tabular}
\scriptsize
\line(1,0){75} \newline
D \newline
\line(1,0){75} \newline
\textbf{1} \textit{Initiale} D  \newline
\line(1,0){75} \newline
\textbf{5} wont] von D \textbf{6} mit] mir D \textbf{11} Munsalvæsche] Mvnsælvæsche D \textbf{22} roys] Boys D \textbf{23} Brumbane] Brvmbanie D \textbf{29} zôch] doch D \newline
\end{minipage}
\hspace{0.5cm}
\begin{minipage}[t]{0.5\linewidth}
\small
\begin{center}*m
\end{center}
\begin{tabular}{rl}
 & der site ist niht dem Grâl reht.\\ 
 & d\textit{â} muoz der \textbf{reht} ritter und der kneht\\ 
 & \textbf{bewart sîn} vor l\textit{ô}sheit.\\ 
 & diemuot \textbf{ir} hôchvart überstreit.\\ 
5 & d\textit{â} wonte ein werdiu bruoderschaft;\\ 
 & die hânt mit \textbf{werlîcher} kraft\\ 
 & erwert mit ir handen\\ 
 & der \textbf{tæte} von allen landen,\\ 
 & daz der Grâl ist \textbf{unerkennet},\\ 
10 & wan die dar sint benennet\\ 
 & ze Mun\textit{t}salvasche an des Grâles schar.\\ 
 & wan einer kam \textbf{u\textit{n}benennet} dar;\\ 
 & der selbe was ein tumber man\\ 
 & und vuorte ouch sünde mit im dan,\\ 
15 & daz er niht zuom wirte sprach\\ 
 & umb den kumber, den er an im sach.\\ 
 & ich \textbf{so\textit{l}} niemen schelten,\\ 
 & doch muoz er \textbf{sünde} engelten,\\ 
 & daz er niht vrâgte des wirtes schaden.\\ 
20 & \textbf{der} was mit kumber sô \textbf{beladen},\\ 
 & ez wart nie \textbf{erkant} \textbf{sô} hôhiu pîn.\\ 
 & dâ \textbf{vor} kam rois Lehelin\\ 
 & zuo Bru\textit{n}bane an den sê geriten.\\ 
 & durch juste het sîn d\textit{â} gebiten\\ 
25 & Lippeals, der werde helt,\\ 
 & des tôt mit juste \textbf{was} erwelt.\\ 
 & er was \textbf{erborn} von Prie\textit{n}lasc\textit{or}s.\\ 
 & Lehelin des heldes ors\\ 
 & dannen zôch mit sîner hant.\\ 
30 & d\textit{â} wart der rêro\textit{up} \textbf{erkant}.\\ 
\end{tabular}
\scriptsize
\line(1,0){75} \newline
m n o \newline
\line(1,0){75} \newline
\newline
\line(1,0){75} \newline
\textbf{2} dâ] Do m n o  $\cdot$ reht] \textit{om.} n o \textbf{3} sîn] sint o  $\cdot$ lôsheit] lausheit m boszheit o \textbf{4} ir] ẏe n e o  $\cdot$ überstreit] vber treit o \textbf{5} dâ] Do m n o \textbf{8} allen landen] ellen [h]: landen o \textbf{9} Grâl] grole n \textbf{10} die] sie o \textbf{11} Muntsalvasche] mundsaluasce m montsaluasce n muntsaluasce o \textbf{12} unbenennet] vnd benennet m vnbekennet n  $\cdot$ dar] [der]: dar o \textbf{16} umb] Jm o \textbf{17} sol] so m \textbf{20} beladen] geladen o \textbf{23} Brunbane] brumbane m n  $\cdot$ brunbane o \textbf{24} dâ] do m n o \textbf{25} Lippeals] Lippe als o \textbf{27} Prienlascors] priealascros m \textbf{30} dâ] Do m n o  $\cdot$ rêroup] rerons m \newline
\end{minipage}
\end{table}
\newpage
\begin{table}[ht]
\begin{minipage}[t]{0.5\linewidth}
\small
\begin{center}*G
\end{center}
\begin{tabular}{rl}
 & \begin{large}D\end{large}er site ist niht dem Grâl reht.\\ 
 & dâ muoz der rîter unde der kneht\\ 
 & \textbf{bewart sîn} vor lôsheit.\\ 
 & diemuot \textbf{die} hôchvart überstreit.\\ 
5 & dâ wont ein werdiu bruodersc\textit{h}aft;\\ 
 & die hânt mit \textbf{werlîcher} kraft\\ 
 & erwert mit ir handen\\ 
 & der \textbf{diet} von al \textbf{den} landen,\\ 
 & daz der Grâl ist \textbf{unbekennet},\\ 
10 & wan die dar sint benennet\\ 
 & ze Muntsalvatsche ans Grâles schar.\\ 
 & wan einer kom \textbf{unbenennet} dar;\\ 
 & der selbe was ein tumber man\\ 
 & unde vuorte ouch sünde mit im dan,\\ 
15 & daz er niht ze dem wirte sprach\\ 
 & umbe\textit{n k}umber, den er an im sach.\\ 
 & ich \textbf{en}\textbf{sol} niemen schelten,\\ 
 & doch muoz er \textbf{sünd\textit{e}} engelten,\\ 
 & daz er niht vrâget des wirtes schaden.\\ 
20 & \textbf{er} was mit kumber sô \textbf{geladen},\\ 
 & ez \textbf{en}wart nie \textbf{erkant} \textbf{sô} hô\textit{h}er pîn.\\ 
 & dâ \textbf{vor} kom roys Lehelin\\ 
 & ze Brunbanie an den sê geriten.\\ 
 & durch tjost het sîn dâ gebiten\\ 
25 & Liebeals, der werde helt,\\ 
 & des tôt mit tjoste \textbf{wa\textit{rt}} erwelt.\\ 
 & er was \textbf{erborn} von Prienlacors.\\ 
 & Lehelin des heldes ors\\ 
 & dannen zôch mit sîner hant.\\ 
30 & dâ wart der rêroup \textbf{bekant}.\\ 
\end{tabular}
\scriptsize
\line(1,0){75} \newline
G I O L M Z Fr18 Fr49 \newline
\line(1,0){75} \newline
\textbf{1} \textit{Initiale} G I O L M Z Fr18  \textbf{13} \textit{Initiale} I  \newline
\line(1,0){75} \newline
\textbf{1} Der] ÷er O \textbf{2} dâ] Daz Z  $\cdot$ unde] vnd ouch Z \textbf{3} bewart sîn] Bewarn sich O L (Fr18)  $\cdot$ vor] von I \textbf{4} die] \textit{om.} O ýe L \textbf{5} bruoderschaft] brudersc:aft G botschafft M \textbf{6} hânt] hat M \textbf{8} den] \textit{om.} Z \textbf{9} unbekennet] vnerchennet O (L) (M) (Z) (Fr18) \textbf{10} dar] gar O  $\cdot$ benennet] genennet O \textbf{11} ze Muntsalvatsche] zemvntsalvatsche G (Fr18) zemuntshaluasce I Zcu Munsalvatsche M Zv montsalvatsche Z \textbf{12} unbenennet] vngenant O L Fr18 \textbf{13} ein] \textit{om.} Z \textbf{14} sünde] sunden M \textbf{16} umben kumber] vmben den chumbir G \textbf{17} ich ensol] ich sol I (O) Jch ensolte L (M) (Fr18) \textbf{18} sünde] svnden G  $\cdot$ engelten] erkelden M \textbf{19} vrâget] vragte I (L) (M) (Z) (Fr18)  $\cdot$ des] der Fr18 \textbf{20} geladen] beladen Z \textbf{21} enwart] wart O Fr18 (Fr49)  $\cdot$ nie erkant] erkant nie L  $\cdot$ hôher] hohier G \textbf{22} vor] von L  $\cdot$ Lehelin] lohelin G Læhelin O Lehelein Fr49 \textbf{23} ze Brunbanie] zebrvnbanie G ze prupanie I (Fr49) Zebrvmbanîe O Zuͯ Brvmbanie L (Z) Zcu Brumbane M ZeBrvmbanie Fr18 \textbf{25} Liebeals] Liebe als G libeals I (Fr49) Libbeals O L M Z Lybbeals Fr18 \textbf{26} wart] was G \textbf{27} erborn] geborn I O L (M) Fr18 gebor Fr49  $\cdot$ Prienlacors] prinlacors I Fr49 Brienlayoͤrs O prienlaiorsz L prienlascors M prienlascoͤrs Z prinlaẏors Fr18 \textbf{28} Lehelin] Læhelin O \textbf{29} dannen zôch] Dannoch nam L \textbf{30} bekant] er chant O (L) (Fr18) \newline
\end{minipage}
\hspace{0.5cm}
\begin{minipage}[t]{0.5\linewidth}
\small
\begin{center}*T
\end{center}
\begin{tabular}{rl}
 & der site ist niht dem Grâle reht.\\ 
 & dâ muoz der rîter unde der kneht\\ 
 & \textbf{bewarn sich} vor lôsheit.\\ 
 & diemuot hôchvart \textbf{ie} überstreit.\\ 
5 & dâ wont ein werdiu bruoderschaft;\\ 
 & die hânt mit \textbf{werdeclîcher} kraft\\ 
 & erwert mit ir handen\\ 
 & der \textbf{diet} von allen landen,\\ 
 & daz der Grâl ist \textbf{unbekant},\\ 
10 & wan \textbf{den}, die dar sint benant\\ 
 & ze Munsalvasche ans Grâles schar.\\ 
 & wan einer kom \textbf{ungewarnet} dar;\\ 
 & der selbe was ein tumber man\\ 
 & unde vuorte ouch sünde mit im dan,\\ 
15 & daz er niht zem wirte sprach\\ 
 & umbe den kumber, den er an im sach.\\ 
 & ich \textbf{solte} nieman schelten,\\ 
 & doch muoz er \textbf{sünden} engelten,\\ 
 & daz er niht vrâgete \textbf{umbe} des wirtes schaden.\\ 
20 & \textbf{er} was mit kumber sô \textbf{beladen},\\ 
 & ez \textbf{en}wart nie \textbf{bekant} hôher pîn.\\ 
 & Dâ \textbf{vor} kom roys Lehelin\\ 
 & ze Brumbanie an den sê geriten.\\ 
 & durch tjost hete sîn dâ gebiten\\ 
25 & Lebbeals, der werde helt,\\ 
 & des tôt mit tjost \textbf{wart} erwelt.\\ 
 & er was \textbf{geborn} von Prienlaiors.\\ 
 & Lehelin des heldes ors\\ 
 & dannen zôch mit sîner hant.\\ 
30 & dâ wart der rêroup \textbf{bekant}.\\ 
\end{tabular}
\scriptsize
\line(1,0){75} \newline
T U V W Q R Fr42 \newline
\line(1,0){75} \newline
\textbf{1} \textit{Initiale} Q  \textbf{22} \textit{Majuskel} T  \newline
\line(1,0){75} \newline
\textbf{1} \textit{Die Verse 453.1-502.30 fehlen} U  \textbf{2} dâ] Do V W Q \textbf{3} vor] [ver]: vor Fr42  $\cdot$ lôsheit] lashait W boszheit R \textbf{4} hôchvart ie] hochvart [*]: ie V ye hoffart R \textbf{5} dâ] Do W Q  $\cdot$ wont] wonde Q von Fr42  $\cdot$ werdiu] werde R \textbf{6} hânt] hat R  $\cdot$ werdeclîcher] werlicher V W (R) [werticher]: werlicher  Q werlichivcher Fr42 \textbf{7} erwert] Entwert R \textbf{9} unbekant] vnerkant V Q R \textbf{10} Wande [*]: den die dar sint benant V  $\cdot$ den] \textit{om.} W Fr42  $\cdot$ die] \textit{om.} R [dier]: die Fr42  $\cdot$ dar] do Q \textbf{11} Munsalvasche] Mvnsalvasce T U [mvntsal*]: mvntsalvasche V monsaluatz W muntsalvasche Q Munsauashe R mvnsalvahse Fr42  $\cdot$ ans] [*]: an dez V an Q \textbf{12} einer] ein ritter W (Fr42)  $\cdot$ ungewarnet] vngenennet W vnbenant Q (R) [vngennet]: vngenennet  Fr42 \textbf{14} vuorte] furt Q (R)  $\cdot$ sünde] sunder Q \textbf{17} \textit{Vers 473.17 wohl von anderer Hand} Fr42   $\cdot$ ich] in Fr42  $\cdot$ solte] [ensol*]: ensolte V ensolte W \textbf{18} sünden] sv́nde V (W) (R) (Fr42) sunder Q \textbf{19} vrâgete] fragt W  $\cdot$ umbe] \textit{om.} W Q R Fr42 \textbf{20} er] [*r]: Der V  $\cdot$ kumber] kume R \textbf{21} ez enwart] Es ward R  $\cdot$ bekant] erkant R  $\cdot$ hôher] so hohe W so hoher Q (R) (Fr42) \textbf{22} vor] von W Q Fr42  $\cdot$ roys] kv́nig V  $\cdot$ Lehelin] lehalein W lechelin R \textbf{23} Brumbanie] Brvmbange T U Fr42 [Brvnban*]: Brvnbanne V brubange W brunbange Q Brumgange R \textbf{24} sîn] in Fr42  $\cdot$ dâ] do V W Q  $\cdot$ gebiten] erbitten W (Fr42) \textbf{25} Lebbeals] Libealz V Libbeals W Libe als Q Libeals R lybbeals Fr42  $\cdot$ werde] werld Q \textbf{26} wart] [wa*]: waz V  $\cdot$ erwelt] geuelt W erfelt Q \textbf{27} geborn] erborn W Q  $\cdot$ Prienlaiors] [p*]: prialascors V pizenlaiors W prienlaioes Q \textbf{28} Lehelin] Lehalein W Lechlin R  $\cdot$ heldes] holdes W \textbf{29} dannen] Dannoch W (Fr42) \textbf{30} dâ] Do V W Q R  $\cdot$ bekant] erkant V W Q R Fr42 \newline
\end{minipage}
\end{table}
\end{document}
