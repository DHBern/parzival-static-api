\documentclass[8pt,a4paper,notitlepage]{article}
\usepackage{fullpage}
\usepackage{ulem}
\usepackage{xltxtra}
\usepackage{datetime}
\renewcommand{\dateseparator}{.}
\dmyyyydate
\usepackage{fancyhdr}
\usepackage{ifthen}
\pagestyle{fancy}
\fancyhf{}
\renewcommand{\headrulewidth}{0pt}
\fancyfoot[L]{\ifthenelse{\value{page}=1}{\today, \currenttime{} Uhr}{}}
\begin{document}
\begin{table}[ht]
\begin{minipage}[t]{0.5\linewidth}
\small
\begin{center}*D
\end{center}
\begin{tabular}{rl}
\textbf{97} & als \textbf{dô} ich mînem wîbe entran,\\ 
 & die ich \textbf{ouch} mit rîterschaft gewan.\\ 
 & dô si mich ûf von strîte bant,\\ 
 & ich liez \textbf{liute} unde lant."\\ 
5 & \textbf{Si sprach}: "\textbf{hêrre}, \textbf{nû} \textbf{nemet} iu \textbf{selbe} \textbf{ein} zil.\\ 
 & ich lâz iu iwers willen vil."\\ 
 & "ich wil \textbf{vrumen noch} vil der sper enzwei.\\ 
 & aller mânôt glîch einen turnei,\\ 
 & des sult ir, vrouwe, \textbf{ruochen},\\ 
10 & daz ich den muoze suochen."\\ 
 & diz lobt si, wart mir gesagt.\\ 
 & er enpfienc diu lant unt \textbf{ouch} die magt.\\ 
 & \textit{\begin{large}D\end{large}}isiu driu junchêrrelîn\\ 
 & Ampflisen, der künegîn,\\ 
15 & hie stuonden unt ir kappelân,\\ 
 & dâ volge unt urteil wart getân.\\ 
 & \textbf{al} \textbf{dâ} \textbf{erz hôrte unt sach},\\ 
 & heinlîche er Gahmureten sprach:\\ 
 & "man tet mîner vrouwen kunt,\\ 
20 & daz ir vor Patelamunt\\ 
 & den hœhsten prîs behieltet\\ 
 & unt \textbf{dâ} zweier \textbf{krône} wieltet.\\ 
 & si hât ouch lant und muot\\ 
 & und gît \textbf{iu} lîp und guot."\\ 
25 & "Dô si mir gap \textbf{die} rîterschaft,\\ 
 & dô \textbf{muos} ich nâch \textbf{der} ordens kraft,\\ 
 & als mir des schildes ambet sagt,\\ 
 & dar \textbf{bî} belîben unverzagt.\\ 
 & wan daz ich schilt von ir gewan,\\ 
30 & ez wære noch anders ungetân.\\ 
\end{tabular}
\scriptsize
\line(1,0){75} \newline
D \newline
\line(1,0){75} \newline
\textbf{5} \textit{Majuskel} D  \textbf{13} \textit{Initiale} D  \textbf{25} \textit{Majuskel} D  \newline
\line(1,0){75} \newline
\textbf{13} Disiu] ÷isiv D \textbf{18} Gahmureten] Gahmvreten D \textbf{20} Patelamunt] Pantelamvnt D \newline
\end{minipage}
\hspace{0.5cm}
\begin{minipage}[t]{0.5\linewidth}
\small
\begin{center}*m
\end{center}
\begin{tabular}{rl}
 & alsô \textbf{dô} ich mînem wîp entran,\\ 
 & die ich \textbf{ouch} mit ritterschaft gewan.\\ 
 & dô si mich ûf von strîte bant,\\ 
 & ich liez \textbf{si}, \textbf{ir} \textbf{liute} und lant."\\ 
5 & \textbf{si sprach}: "\textbf{hêrre}, \textbf{nemt} iu \textbf{ein} zil.\\ 
 & ich lâze i\textit{u} i\textit{uwe}res willen vil."\\ 
 & "ich wil \textbf{vrumen noch} vil der sper in zwei.\\ 
 & aller mânôde gelîch einen turnei,\\ 
 & des sullet ir, vrouwe, \textbf{ruochen},\\ 
10 & daz ich den \textit{m}uoze suochen."\\ 
 & d\textit{i}z lobete si, wart mir gesaget.\\ 
 & er enpfienc diu lant und die maget\\ 
 & \begin{large}D\end{large}isiu driu junchêrrelîn\\ 
 & Ampf\textit{l}isen, der künigîn,\\ 
15 & hie stuonden und ir kappelân,\\ 
 & d\textit{â} volge und urteil wart getân.\\ 
 & \textbf{dô} \textbf{daz der kappelân gesach},\\ 
 & heimlîche er Gahmureten sprach:\\ 
 & "man tet mîner vrouwen kunt,\\ 
20 & \multicolumn{1}{l}{ - - - }\\ 
 & den hœhesten prîs behieltet\\ 
 & und \textbf{d\textit{â}} zweier \textbf{krône} w\textit{ie}ltet.\\ 
 & si hât ouch lant und muot\\ 
 & und gibt \textbf{iu} lîp und guot."\\ 
25 & \textbf{er sprach}: "dô si mir gap ritterschaft,\\ 
 & dô \textbf{muos} ich nâch \textbf{der} ordens kraft,\\ 
 & als mir des schiltes ambet saget,\\ 
 & d\textit{â} belîben unverzaget.\\ 
 & wanne daz ich schilt von ir gewan,\\ 
30 & ez wære noch anders ungetân.\\ 
\end{tabular}
\scriptsize
\line(1,0){75} \newline
m n o \newline
\line(1,0){75} \newline
\textbf{13} \textit{Initiale} m o   $\cdot$ \textit{Capitulumzeichen} n  \newline
\line(1,0){75} \newline
\textbf{5} ein] selb n (o) \textbf{6} iu iuweres] in [v]: yeres m in irs n o \textbf{7} noch] \textit{om.} n o \textbf{8} mânôde] man vnd n man oder o \textbf{10} muoze] nusse m \textbf{11} diz] Des m  $\cdot$ lobete si] lob n loͯbet sie o \textbf{14} Ampflisen] Ampfissen m Anpflisen n o \textbf{16} dâ] Do m n Die o \textbf{17} daz] \textit{om.} o \textbf{18} Gahmureten] gahmuͯretten m gamúreten n o \textbf{20} \textit{Vers 97.20 fehlt} m o   $\cdot$ Wol zú der selben stunt n \textbf{21} hœhesten] groͯsten n (o) \textbf{22} dâ] do m n o  $\cdot$ wieltet] weiltent m weltent o \textbf{24} gibt] [git]: gipt m  $\cdot$ iu] auch o \textbf{26} muos] muͯste n (o)  $\cdot$ ordens] orden n o \textbf{28} dâ] Do m n o  $\cdot$ belîben] by bliben n (o) \newline
\end{minipage}
\end{table}
\newpage
\begin{table}[ht]
\begin{minipage}[t]{0.5\linewidth}
\small
\begin{center}*G
\end{center}
\begin{tabular}{rl}
 & als ich mînem wîbe entran,\\ 
 & die ich \textbf{ouch} mit rîterschaft gewan.\\ 
 & dô si mich ûf von strîte bant,\\ 
 & ich liez \textbf{ir} \textbf{lîp} und lant."\\ 
5 & "\textbf{hêrre}, \textbf{nemet} iu \textbf{selbe} zil.\\ 
 & ich lâze iu iwers willen vil."\\ 
 & "ich wil \textbf{vrumen noch} vil der sper enzwei.\\ 
 & aller mân gelîch einen turnei,\\ 
 & des sult ir, vrouwe, \textbf{ruochen},\\ 
10 & daz ich d\textit{en} muoze suochen."\\ 
 & diz lobte si, wart mir gesaget.\\ 
 & er enpfie d\textit{iu} lant und \textbf{ouch} die maget.\\ 
 & disiu driu junchêrrelîn\\ 
 & Anphlisen, der künigîn,\\ 
15 & hie stuonden und ir kappelân,\\ 
 & dâ volge und urteil wart getân.\\ 
 & \textbf{der pfaffe ez hôrte und sach}.\\ 
 & heinlîche er Gahmureten sprach:\\ 
 & "man tet mîner vrouwen kunt,\\ 
20 & daz ir vor Patelamunt\\ 
 & den hœhesten prîs behieltet\\ 
 & unt zweier \textbf{lande} wieltet.\\ 
 & si hât ouch lant und muot\\ 
 & unde gît \textbf{iu} lîp und guot."\\ 
25 & "dô si mir gap \textbf{die} rîterschaft,\\ 
 & dô \textbf{muose} ich nâch \textbf{der} ordenes kraft,\\ 
 & als mir des schiltes ambet saget,\\ 
 & dar \textbf{bî} belîben unverzaget.\\ 
 & wan daz ich schilt von ir gewan,\\ 
30 & ez wære noch anders ungetân.\\ 
\end{tabular}
\scriptsize
\line(1,0){75} \newline
G I O L M Q R Z Fr36 \newline
\line(1,0){75} \newline
\textbf{1} \textit{Initiale} O  \textbf{13} \textit{Initiale} L R Z Fr36  \textbf{15} \textit{Initiale} I  \textbf{23} \textit{Initiale} M  \newline
\line(1,0){75} \newline
\textbf{1} als] ÷ls O  $\cdot$ ich] do ich O L M Q da Jch R (Z)  $\cdot$ mînem] von myme M \textbf{3} dô] Da M Z  $\cdot$ si] ich Q  $\cdot$ mich] mich hute M  $\cdot$ ûf] ovch Z  $\cdot$ von] \textit{om.} O \textbf{4} ich] Vnd M  $\cdot$ lîp] lvͤte O (Z)  $\cdot$ lant] or lant M \textbf{5} \textit{Versfolge 97.7-10, dann 97.5-6} I   $\cdot$ hêrre] Sie sprach herre O L (M) Q (R) Z  $\cdot$ selbe] selben M selber Q  $\cdot$ zil] ein zil I Z \textbf{6} willen] willens R \textbf{7} vrumen noch vil] noch vil vruͤmen I (Q) noch frommen vil L froymen vil M  $\cdot$ enzwei] enzwer Q \textbf{8} aller] \textit{om.} I  $\cdot$ mân gelîch] mænlich O (Z) mendelich M menigliche Q menglichs R \textbf{9} ir] \textit{om.} M \textbf{10} den] die G  $\cdot$ muoze] musz Q \textbf{11} diz] Daz O (R) Fr36  $\cdot$ lobte] lop I Q Fr36 lobt O R Z  $\cdot$ si] si im I \textit{om.} Q sir Fr36  $\cdot$ wart] ist I  $\cdot$ gesaget] gesant M \textbf{12} enpfie] phinck Q  $\cdot$ diu lant] daz lant G die lant I die [magt]: lant R  $\cdot$ ouch] \textit{om.} I O L M Q R Fr36  $\cdot$ die] div Fr36 \textbf{13} disiu] ÷isiv Fr36  $\cdot$ junchêrrelîn] [ch]: ivnch herrelin G Junge herlin M \textbf{14} Anphlisen] anphisen I Amphilisen O Anfolẏsen L Ansibisen M Anflissen Q Amflysen R Amflisen Z anphelise Fr36 \textbf{15} hie] Die O M  $\cdot$ stuonden] stuenden I (L) \textbf{16} dâ] Do Q  $\cdot$ urteil wart] vrtailde was I  $\cdot$ getân] [gethor]: gethon Q \textbf{17} ez] \textit{om.} R  $\cdot$ hôrte] hort I O Q \textbf{18} er] er ze O (Q) (Z)  $\cdot$ Gahmureten] Gamvreten O (Q) (Z) Gahmuͯreten L gamurete M Gahmurten R  $\cdot$ sprach] gesprach I \textbf{20} vor] von I Q R  $\cdot$ Patelamunt] patalamunt I petalamvnt L patelamúnt Q \textbf{21} hœhesten] besten O  $\cdot$ behieltet] gehilte Q behalttet R \textbf{22} unt] Vnde da O (Z) Vnd do Q R  $\cdot$ lande] chrone O (L) (Q) (R) (Z) cronen M  $\cdot$ wieltet] wildet M wilte Q walttet R \textbf{23} hât] \textit{om.} Q  $\cdot$ muot] gvt Z \textbf{24} guot] mut Z \textbf{25} dô] Da M Z \textbf{26} dô] Da M Z  $\cdot$ muose] muͤs I  $\cdot$ der] \textit{om.} I  $\cdot$ ordenes] orden O ordnuns R \textbf{27} des] daz O L (R) Z do Q  $\cdot$ schiltes] schilt L  $\cdot$ saget] sagte Q \textbf{28} unverzaget] vnuerzagte Q \textbf{29} von ir gewan] vor genam I \newline
\end{minipage}
\hspace{0.5cm}
\begin{minipage}[t]{0.5\linewidth}
\small
\begin{center}*T (U)
\end{center}
\begin{tabular}{rl}
 & als \textbf{dô} ich mîme wîbe entran,\\ 
 & die ich mit ritterschaft gewan.\\ 
 & dô si mich ûf von strîte bant,\\ 
 & ich liez \textbf{ir} \textbf{lîp} und lant."\\ 
5 & \textbf{si sprach}: "\textbf{nû} \textbf{gêt} iu \textbf{selber} \textbf{ein} zil.\\ 
 & ich lâze iu iuwers willen vil."\\ 
 & "ich wil \textbf{noch vrumen} vil der sper enzwei.\\ 
 & aller mâne glîch einen turnei,\\ 
 & des solt ir, vrouwe, \textbf{geruochen},\\ 
10 & daz ich den muoze suochen."\\ 
 & diz lobete si, wart mir gesaget.\\ 
 & er entvienc diu lant und \textbf{ouch} die maget.\\ 
 & \begin{large}D\end{large}isiu driu junchêrrelîn\\ 
 & Anflisen, der künegîn,\\ 
15 & hie stuonden und ir kappelân,\\ 
 & dâ volge und urteil wart getân.\\ 
 & \textbf{der pfaffe ez hôrte und sach}.\\ 
 & heimelîch er Gahmureten sprach:\\ 
 & "man tet mîner vrouwen kunt,\\ 
20 & daz ir vor Patelamunt\\ 
 & den hœhesten prîs behielte\textit{t}\\ 
 & und \textbf{dâ} zweier \textbf{krône} w\textit{i}elte\textit{t}.\\ 
 & si het ouch lant und muot\\ 
 & und gît \textbf{ouch} lîp und guot."\\ 
25 & "dô si mir gap \textbf{die} ritterschaft,\\ 
 & dô \textbf{muoz} ich nâch \textbf{des} ordens kraft,\\ 
 & als mir des schiltes ambet saget,\\ 
 & dar \textbf{bî} blîben unverzaget.\\ 
 & wan daz ich schi\textit{l}t von ir gewan,\\ 
30 & ez wære noch anders ungetân.\\ 
\end{tabular}
\scriptsize
\line(1,0){75} \newline
U V W T \newline
\line(1,0){75} \newline
\textbf{5} \textit{Majuskel} T  \textbf{7} \textit{Majuskel} T  \textbf{11} \textit{Majuskel} T  \textbf{13} \textit{Initiale} U V W   $\cdot$ \textit{Majuskel} T  \textbf{17} \textit{Majuskel} T  \textbf{25} \textit{Majuskel} T  \newline
\line(1,0){75} \newline
\textbf{2} ich] ich auch W \textbf{3} ûf] \textit{om.} W \textbf{5} nû gêt iu selber] nv nement selbe V nempt euch selber W herre nemt selbe T \textbf{7} wil noch vrumen] vrvme noch T  $\cdot$ der] \textit{om.} W \textbf{8} aller] Alle W  $\cdot$ mâne glîch einen] manneglich U [m*nlich]: monlich ein V menede ein W \textbf{9} \textit{Versfolge 97.10-9} T   $\cdot$ ir] \textit{om.} W \textbf{12} ouch] \textit{om.} W T  $\cdot$ die] div T \textbf{14} Anflisen] Anflizen U Anfolisen W \textbf{16} dâ] Do W  $\cdot$ wart] hat W \textbf{17} hôrte] hort W  $\cdot$ sach] sprach W \textbf{18} heinlich zuͦ Gamuret er sprach V · Zuͦ gamuret vnd iach W  $\cdot$ Gahmureten] [*]: gahmuren T \textbf{19} mîner] seiner W \textbf{20} ir] er W  $\cdot$ Patelamunt] Patelamuͦnt U \textbf{21} hœhesten] besten W T  $\cdot$ behieltet] behielten U V behielte W \textbf{22} dâ] do V W  $\cdot$ krône] cronen V  $\cdot$ wieltet] welten U wielten V wielte W \textbf{23} ouch] \textit{om.} T  $\cdot$ muot] guͦt W (T) \textbf{24} und] Sy W  $\cdot$ ouch] v́ch V (W) (T)  $\cdot$ guot] muͦt W (T) \textbf{25} dô si] Dyse W  $\cdot$ mir gap die] mich werte T \textbf{26} dô] So W  $\cdot$ muoz ich] muͤst ich V mvesich T  $\cdot$ des] der W (T) \textbf{29} schilt] schit U \textbf{30} noch] \textit{om.} V \newline
\end{minipage}
\end{table}
\end{document}
