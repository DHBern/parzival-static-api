\documentclass[8pt,a4paper,notitlepage]{article}
\usepackage{fullpage}
\usepackage{ulem}
\usepackage{xltxtra}
\usepackage{datetime}
\renewcommand{\dateseparator}{.}
\dmyyyydate
\usepackage{fancyhdr}
\usepackage{ifthen}
\pagestyle{fancy}
\fancyhf{}
\renewcommand{\headrulewidth}{0pt}
\fancyfoot[L]{\ifthenelse{\value{page}=1}{\today, \currenttime{} Uhr}{}}
\begin{document}
\begin{table}[ht]
\begin{minipage}[t]{0.5\linewidth}
\small
\begin{center}*D
\end{center}
\begin{tabular}{rl}
\textbf{719} & \begin{large}N\end{large}û helfet mir, ir zwêne,\\ 
 & und ouch \textbf{dû}, vriwendîn Bene,\\ 
 & daz der künec \textbf{her} zuo mir rîte\\ 
 & unt den kampf doch morgen strîte.\\ 
5 & \textbf{Mînen} neven Gawan\\ 
 & bringe ich gein im ûf den plân.\\ 
 & \textbf{rîtet} der künec hiute in mîn her,\\ 
 & er ist morgen \textbf{al} deste baz ze wer.\\ 
 & hie gît diu minne im einen schilt,\\ 
10 & \textbf{des} sînen kampfgenôz bevilt.\\ 
 & ich meine gein minne hôhen muot,\\ 
 & der bî \textbf{den vîenden} schaden tuot.\\ 
 & Er sol höfsche liute bringen.\\ 
 & ich wil \textbf{hie} teidingen\\ 
15 & zwischen im und der herzogîn.\\ 
 & nû wer\textit{b}etz, \textbf{trûtgeselle} mîn,\\ 
 & mit \textbf{vuoge}; des habt ir êre.\\ 
 & \textbf{ich sol iu klagen mêre}:\\ 
 & waz \textbf{hân ich unsælic man}\\ 
20 & \textbf{dem künege} Gramoflanz getân,\\ 
 & \textbf{sît} er gein mîme künne pfligt,\\ 
 & daz in lîhte unhôhe wigt,\\ 
 & minne und unminne grôz?\\ 
 & ein ieslîch künec, mîn genôz,\\ 
25 & \textbf{mîn gerne m\textit{ö}hte} schônen.\\ 
 & wil er nû mit hazze lônen\\ 
 & ir bruoder, diu in minnet,\\ 
 & \textbf{ob} er sich\textbf{s} versinnet,\\ 
 & sîn herze tuot von minnen wanc,\\ 
30 & \textbf{swenne}z in lêret den gedanc."\\ 
\end{tabular}
\scriptsize
\line(1,0){75} \newline
D \newline
\line(1,0){75} \newline
\textbf{1} \textit{Initiale} D  \textbf{5} \textit{Majuskel} D  \textbf{13} \textit{Majuskel} D  \newline
\line(1,0){75} \newline
\textbf{2} Bene] Bêne D \textbf{16} werbetz] werbtetz D \textbf{25} möhte] mohte D \newline
\end{minipage}
\hspace{0.5cm}
\begin{minipage}[t]{0.5\linewidth}
\small
\begin{center}*m
\end{center}
\begin{tabular}{rl}
 & nû helfet mir, ir zwêne,\\ 
 & und ouch \textbf{mîn} vriundîn Bene,\\ 
 & daz der künic \textbf{her} zuo mir rîte\\ 
 & \dag ûf\dag  den kampf doch morgen strîte.\\ 
5 & \textbf{mînen} neven Gawan\\ 
 & bringe ich gegen im ûf den plân.\\ 
 & \textbf{rît} der künic hiute in mî\textit{n} her,\\ 
 & er ist morgen deste baz zuo wer.\\ 
 & hie gît diu minne im einen schilt,\\ 
10 & \textbf{daz} sînen kampfgenôz bevilt.\\ 
 & ich mein gegen minne hôhen muot,\\ 
 & der bî \textbf{den vîe\textit{n}den} schaden tuot.\\ 
 & er sol höfsch liute bringen.\\ 
 & ich wil \textbf{hie} tegedingen\\ 
15 & zwischen im und der herzogîn.\\ 
 & nû wer\textit{b}etz, \textbf{trûtgesellen} mîn,\\ 
 & mit \textbf{vuoge}; des habt ir êre.\\ 
 & \textbf{mich wundert harte sêre},\\ 
 & waz \textbf{ich übels m\textit{ö}hte hân}\\ 
20 & \textbf{wider} Gramolantzen getân,\\ 
 & \textbf{sît} er gegen mîm\textit{e} künne pfliget,\\ 
 & daz in lîht unhôhe wiget,\\ 
 & minne und unminne grôz?\\ 
 & ein ieglîch künic, mîn genôz,\\ 
25 & \textbf{mîn gerne m\textit{ö}hte} schônen.\\ 
 & wil er nû mit hazze lônen\\ 
 & ir bruoder, diu in minnet,\\ 
 & \textbf{ob} er sich versinnet,\\ 
 & sîn herz tuot von minnen wanc,\\ 
30 & \textbf{wenn} ez \textit{i}n lêret den gedanc."\\ 
\end{tabular}
\scriptsize
\line(1,0){75} \newline
m n o Fr69 \newline
\line(1,0){75} \newline
\newline
\line(1,0){75} \newline
\textbf{2} vriundîn] fruͯnde o \textbf{4} doch] \textit{om.} n  $\cdot$ strîte] stete o \textbf{7} mîn] minem m \textbf{10} kampfgenôz] kampff benos o \textbf{11} minne] myͯnem o \textbf{12} vîenden] fierden m (o) \textbf{14} ich] Jich Fr69 \textbf{15} zwischen] Zwúschentz n \textbf{16} werbetz] werbens m n werbent Fr69  $\cdot$ trûtgesellen] trut geselle Fr69 \textbf{17} des] das o \textbf{19} ich übels] ich v́bels ich v́bels n  $\cdot$ möhte] mohtte m (o) \textbf{20} Gramolantzen] gramolanczes o \textbf{21} mîme] mimen m mẏnnen o  $\cdot$ künne] kuͯnen o \textbf{22} lîht] liecht o \textbf{24} ein] Sin o \textbf{25} möhte] mohtte m (o)  $\cdot$ schônen] schowen o \textbf{26} nû] \textit{om.} n \textbf{29} sîn] Ein n \textbf{30} in lêret] enleret m n erleret o \newline
\end{minipage}
\end{table}
\newpage
\begin{table}[ht]
\begin{minipage}[t]{0.5\linewidth}
\small
\begin{center}*G
\end{center}
\begin{tabular}{rl}
 & \begin{large}N\end{large}û helfet mir, ir zwêne,\\ 
 & unde ouch \textbf{dû}, vriundîn Bene,\\ 
 & daz der künec \textbf{doch} zuo mir rîte\\ 
 & unde den kampf doch morgen strîte.\\ 
5 & \textbf{mînen} neven Gawan\\ 
 & bringe ich gein im ûf den plân.\\ 
 & \textbf{rîte\textit{t}} \textbf{aber} der künec hiut in mîn her,\\ 
 & er ist morgen deste baz ze wer.\\ 
 & hie gît diu minne im einen schilt,\\ 
10 & \textbf{des} sînen kampfgenôz bevilt.\\ 
 & ich meine gein minne hôhen muot,\\ 
 & der bî \textbf{den vîenden} schaden tuot.\\ 
 & er sol höfsche lûte bringen.\\ 
 & ich wil \textbf{hie} teidingen\\ 
15 & zwischen im unde der herzogîn.\\ 
 & nû werbet ez, \textbf{trûtgeselle} mîn,\\ 
 & mit \textbf{vuoge}; des habet ir êre.\\ 
 & \textbf{ich solde iu klagen mêre}:\\ 
 & waz \textbf{hân ich unsælic man}\\ 
20 & \textbf{dem künige} Gramoflanz getân,\\ 
 & \textbf{daz} er gein mînem künne pfliget,\\ 
 & daz in \textbf{doch} lîhte unhôhe wiget,\\ 
 & minne unde unminne grôz?\\ 
 & ein ieslîch künec, mîn genôz,\\ 
25 & \textbf{möhte mîn gerne} schônen.\\ 
 & wil er nû mit hazze lônen\\ 
 & ir bruoder, diu in minnet,\\ 
 & \textbf{swenne} er sich versinnet,\\ 
 & sîn herze tuot von minne wanc,\\ 
30 & \textbf{ob} ez in lêrt den gedanc."\\ 
\end{tabular}
\scriptsize
\line(1,0){75} \newline
G I L M Z Fr20 Fr24 \newline
\line(1,0){75} \newline
\textbf{1} \textit{Initiale} G I Z Fr20  \textbf{21} \textit{Initiale} I  \newline
\line(1,0){75} \newline
\textbf{1} mir] \textit{om.} I \textbf{2} dû vriundîn] diu friundin I (Z) (Fr20) (Fr24) die vrowe L myn frundynne M \textbf{3} doch] noch I \textit{om.} L her M Z Fr24 \textbf{4} doch] dorch M \textbf{7} rîtet] riter G Rite M (Z) (Fr24) \textbf{8} deste] des Fr24 \textbf{9} diu minne im] yme dy mynne M \textbf{11} gein] durch Z \textbf{16} trûtgeselle] trutgesellen I (Z) (Fr24) \textbf{17} vuoge] fugen Z  $\cdot$ habet ir] hat er I \textbf{20} Gramoflanz] Gramoflanze I gramorflanze M gramoflantz Z \textbf{22} lîhte] lieht L (M) o\textit{m. } I  $\cdot$ unhôhe] vnsanfte I \textbf{23} Haszes vz der masze groz L \textbf{24} ieslîch] ieglich I (M) ieschlich Fr20 \textbf{25} möhte] Mochte L M (Z) (Fr20) (Fr24) \textbf{26} nû] \textit{om.} Fr20 \textbf{28} swenne] Wenne L (M) \textbf{29} minne] minnen I (M) (Z) \newline
\end{minipage}
\hspace{0.5cm}
\begin{minipage}[t]{0.5\linewidth}
\small
\begin{center}*T
\end{center}
\begin{tabular}{rl}
 & nû helfet mir, ir zwêne,\\ 
 & und ouch \textbf{diu} vriundinne Bene,\\ 
 & daz der künec \textbf{her} zuo mir rîte\\ 
 & und den kampf doch morne strîte\\ 
5 & \textbf{mit mîme} neven Gawan;\\ 
 & \textbf{den} bringe ich gein im ûf den plân.\\ 
 & \textbf{rîte} \textbf{aber} der künec hiute in mîn her,\\ 
 & er ist morne deste baz zuo wer.\\ 
 & hie gît diu minne im einen schilt,\\ 
10 & \textbf{des} sînen kampfgenôz bevilt.\\ 
 & ich meine gein \textit{minne} hôhen muot,\\ 
 & der bî \textbf{dem vînde} schaden tuot.\\ 
 & er sol hövesche liute bringen.\\ 
 & ich wil te\textit{id}ingen\\ 
15 & zwischen im und der herzogîn.\\ 
 & nû werbet ez, \textbf{trûtgeselle} mîn,\\ 
 & mit \textbf{gevuoge}; des hât ir êre.\\ 
 & \textbf{ich solt iu klagen mêre}:\\ 
 & waz \textbf{hân ich unsælic man}\\ 
20 & \textbf{dem künege} Gramoflanz getân,\\ 
 & \textbf{daz} er gein mîme künne pfliget,\\ 
 & daz in \textbf{doch} lîht unhôhe wiget,\\ 
 & minne und unminne grôz?\\ 
 & ein ieclîch künec, mîn genôz,\\ 
25 & \textbf{möhte mîn gerne} schônen.\\ 
 & wil er nû mit hazze lônen\\ 
 & ir bruodere, diu in minnet,\\ 
 & \textbf{wanne} er sich versinnet,\\ 
 & sîn herze tuot von minnen wanc,\\ 
30 & \textbf{o\textit{b}} e\textit{z i}n lêret den gedanc."\\ 
\end{tabular}
\scriptsize
\line(1,0){75} \newline
U V W Q R \newline
\line(1,0){75} \newline
\textbf{19} \textit{Initiale} W  \newline
\line(1,0){75} \newline
\textbf{1} ir] die R \textbf{2} diu] [d*]: min V  $\cdot$ vriundinne] die frúndinne W froͯdenbere R \textbf{4} morne] morgens V \textbf{5} mit mîme] [M*]: Minen V Meinen W (Q) (R)  $\cdot$ Gawan] [*]: Gawan V \textbf{6} den bringe] Bringe W (Q) (R) \textbf{7} rîte] Ritet V \textbf{8} morne deste] dester morgen Q \textbf{9} diu minne im] die [minn*]: minne im V Im die minne R \textbf{10} kampfgenôz] kampff genoß nit W kampfen nosen R \textbf{11} gein minne] gein U gegen meinen W [geminnen]: gemeinnen Q \textbf{12} dem vînde] den vienden V (W) (Q) (R) \textbf{13} hövesche] hofflich R \textbf{14} teidingen] degingen U hie tegedingen V (Q) (R) \textbf{16} trûtgeselle] trauten gesellen W trawt gesellen Q trut gesellen Zem besten R \textbf{17} gevuoge] fuͦge V W (Q) (R) \textbf{18} [*]: Mich wundert harte sere V \textbf{19} Waz [h* ich *]: ich v́belz moͤge han V \textbf{20} [D*]: wider gramolanzen getan V  $\cdot$ Gramoflanz] gramoflanten W gramoflantz Q Gramoflancz R \textbf{21} daz er] [*r]: Sit er V  $\cdot$ mîme] meinē W Q minen R  $\cdot$ künne] kunge R \textbf{22} lîht] liecht U  $\cdot$ unhôhe] in hoͤhe W \textbf{23} minne und] Meine Q \textbf{24} ieclîch] itzlich Q \textbf{27} ir] Jren Q \textbf{28} wanne] Swenne V  $\cdot$ sich] sich des V \textbf{29} minnen] minne W R \textbf{30} ob ez] Oder sie U [*]: Swenz V  $\cdot$ den] der W \newline
\end{minipage}
\end{table}
\end{document}
