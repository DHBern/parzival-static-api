\documentclass[8pt,a4paper,notitlepage]{article}
\usepackage{fullpage}
\usepackage{ulem}
\usepackage{xltxtra}
\usepackage{datetime}
\renewcommand{\dateseparator}{.}
\dmyyyydate
\usepackage{fancyhdr}
\usepackage{ifthen}
\pagestyle{fancy}
\fancyhf{}
\renewcommand{\headrulewidth}{0pt}
\fancyfoot[L]{\ifthenelse{\value{page}=1}{\today, \currenttime{} Uhr}{}}
\begin{document}
\begin{table}[ht]
\begin{minipage}[t]{0.5\linewidth}
\small
\begin{center}*D
\end{center}
\begin{tabular}{rl}
\textbf{495} & \begin{large}S\end{large}us gît man vome Grâle dan\\ 
 & offenlîchen meide, verholne die man,\\ 
 & durch vruht ze dienste wider dar,\\ 
 & ob ir \textbf{kint} des Grâles schar\\ 
5 & mit dienste suln mêren;\\ 
 & daz kan si got wol lêren.\\ 
 & swer \textbf{sich dienstes geinme Grâle hât} bewegen,\\ 
 & \textbf{gein} \textbf{wîben} \textbf{minne er} muoz verpflegen.\\ 
 & \textbf{wan} der künec sol haben eine\\ 
10 & \textbf{ze rehte}, eine konen reine,\\ 
 & unt ander, die got gesant\\ 
 & ze hêrren in hêrrenlôsiu lant.\\ 
 & über daz gebot ich mich bewac,\\ 
 & daz ich \textbf{nâch} minnen \textbf{dienstes} pflac.\\ 
15 & mir \textbf{geriet} mîn \textbf{vlæteclîchiu} jugent\\ 
 & unt eines werden wîbes tugent,\\ 
 & daz ich in ir dienst reit,\\ 
 & \textbf{dâ} \textbf{ich} dicke \textbf{herteclîchen} streit.\\ 
 & die wilden âventiure\\ 
20 & \textbf{mich dûhten} \textbf{sô} gehiure,\\ 
 & daz ich selten turnierte.\\ 
 & ir minne condwierte\\ 
 & \textbf{mir} vreude inz herze mîn;\\ 
 & durch si tet ich vil \textbf{strîtens} schîn.\\ 
25 & des twanc mich ir minnen kraft\\ 
 & gein der wilden verren rîterschaft.\\ 
 & ir minne ich alsus koufte:\\ 
 & der heiden unt der getoufte\\ 
 & wâren \textbf{mir strîtes} al gelîch;\\ 
30 & si \textbf{dûhte} mich lônes rîch.\\ 
\end{tabular}
\scriptsize
\line(1,0){75} \newline
D Fr11 Fr31 \newline
\line(1,0){75} \newline
\textbf{1} \textit{Initiale} D Fr11 Fr31  \newline
\line(1,0){75} \newline
\textbf{2} meide] divͯ mayde Fr11 die mægede Fr31 \textbf{5} mêren] mer::: Fr11 mere Fr31 \textbf{6} kan] sol Fr31 \textbf{8} wîben minne er] wi::: er ::: Fr11 wibe er minne Fr31 \textbf{9} wan] \textit{om.} Fr31 \textbf{10} ze rehte] \textit{om.} Fr31 \textbf{11} gesant] hat gesant Fr11 Fr31 \textbf{13} gebot] bot Fr11 \textbf{14} minnen dienstes] mynnͯ diͯnstes Fr11 minne Fr31 \textbf{15} geriet] riet Fr11  $\cdot$ vlæteclîchiu] flaetig Fr11 (Fr31) \textbf{16} werden] \textit{om.} Fr11 \textbf{18} herteclîchen] hertznlichn Fr11 \textbf{19} die] Divͯ Fr11 \textbf{23} mir] Jr Fr31  $\cdot$ herze] hertzn Fr11 \textbf{24} strîtens] strites Fr11 Fr31 \textbf{25} minnen] minne Fr31 \textbf{26} der] den Fr11 \textbf{27} alsus] also Fr11 \textbf{30} si dûhte] divͯ dauchtn Fr11 Sie dvhten Fr31 \newline
\end{minipage}
\hspace{0.5cm}
\begin{minipage}[t]{0.5\linewidth}
\small
\begin{center}*m
\end{center}
\begin{tabular}{rl}
 & sus gît man von dem Grâle dan\\ 
 & offenlîch megde, verholn die man,\\ 
 & durch vruht zuo dienst wider dar,\\ 
 & ob ir \textbf{kint} des Grâles schar\\ 
5 & mit dienste sollen mêren;\\ 
 & daz \textit{kan} si got wol lêren.\\ 
 & wer \textbf{sich dienstes dem Grâl het} bewegen,\\ 
 & \textbf{wîbe} \textbf{minne er} m\textit{uo}z verpflegen.\\ 
 & \textbf{wan} der künic sol haben eine\\ 
10 & \textbf{zuo reht}, ein konen reine,\\ 
 & und ander, die got \textbf{hât} gesant\\ 
 & zuo hêrren in hêrrenlôs\textit{iu} lant.\\ 
 & über daz gebot ich mich bewac,\\ 
 & daz ich \textbf{nâch} minnen \textbf{dienste} pflac.\\ 
15 & \begin{large}M\end{large}ir \textbf{geriet} mîn \dag gevlehteclîche\dag  jugent\\ 
 & und eines werde\textit{n} \textit{w}îbes tugent,\\ 
 & daz ich in ir dienste reit\\ 
 & \textbf{und} dicke \textbf{herzelîchen} streit.\\ 
 & die wilden âventiure\\ 
20 & \textbf{mich dûhte\textit{n}} \textit{g}ehiure,\\ 
 & daz ich selten turnierte.\\ 
 & ir minne cond\textit{ew}ierte\\ 
 & \textbf{mit} vröuden in daz herze mîn;\\ 
 & durch si tet ich vil \textbf{strîtes} schîn.\\ 
25 & des twanc mich ir minne kraft\\ 
 & gegen der wilden verren ritterschaft.\\ 
 & ir minne ich alsus kouft:\\ 
 & der heiden und der getouft\\ 
 & wâr\textit{e}n \textbf{mit strîten} alglîch;\\ 
30 & si \textbf{gedâht} mich lônes rîch.\\ 
\end{tabular}
\scriptsize
\line(1,0){75} \newline
m n o \newline
\line(1,0){75} \newline
\textbf{15} \textit{Initiale} m  \newline
\line(1,0){75} \newline
\textbf{1} sus] Sos o \textbf{3} vruht] \textit{om.} o \textbf{6} kan] \textit{om.} m \textbf{8} muoz] mirs m \textbf{10} zuo] Zuͯm n  $\cdot$ konen] tonnen o \textbf{12} hêrrenlôsiu lant] herrelosen lant m n herrelosen [h]: lant o \textbf{16} werden wîbes] werden mannes wibez m \textbf{20} Mich tuhtten der gehuͯre m  $\cdot$ dûhten] dúchten o \textbf{22} condewierte] conduriertte m do conduwierte n condiwirte o \textbf{26} wilden] willen o \textbf{27} alsus] also o \textbf{29} wâren] Woran m  $\cdot$ strîten] strite n o  $\cdot$ alglîch] alle glich n \newline
\end{minipage}
\end{table}
\newpage
\begin{table}[ht]
\begin{minipage}[t]{0.5\linewidth}
\small
\begin{center}*G
\end{center}
\begin{tabular}{rl}
 & \begin{large}S\end{large}us gît man vonem Grâle dan\\ 
 & offenlîche \textbf{die} meide, verholn die man,\\ 
 & durch vruht ze dienste wider dar,\\ 
 & obe ir \textbf{kint} des Grâles schar\\ 
5 & mit dienste suln mêren;\\ 
 & daz kan si got wol lêren.\\ 
 & swer \textbf{sich dienstes dem Grâle hât} bewegen,\\ 
 & \textbf{gein} \textbf{wîbe} \textbf{er minne} muoz verpflegen.\\ 
 & der künic sol haben eine,\\ 
10 & ein konen reine,\\ 
 & unde ander, die got \textbf{hât} gesant\\ 
 & ze hêrren in hêrrenlôsiu lant.\\ 
 & über daz gebot ich mich bewac,\\ 
 & daz ich \textbf{nâch} minne \textbf{dienstes} pflac.\\ 
15 & mir \textbf{geriet} mîn \textbf{vlætigiu} jugent\\ 
 & unde eines werde\textit{n} wîbes tugent,\\ 
 & daz ich in ir dienste reit,\\ 
 & \textbf{dâ} \textbf{ich} dicke \textbf{herticlîchen} streit.\\ 
 & die wilden âventiure\\ 
20 & \textbf{mich dûhten} \textbf{sô} gehiure,\\ 
 & daz ich selten turnierte.\\ 
 & ir minne condewierte\\ 
 & \textbf{mir} vröude in daz herze mîn;\\ 
 & durch si tet ich vil \textbf{strîtes} schîn.\\ 
25 & des twanc \textit{m}i\textit{ch} ir minnen kraft\\ 
 & gein der wilden verren rîterschaft.\\ 
 & ir minne ich alsus koufte:\\ 
 & der heiden unde der getoufte\\ 
 & wâren \textbf{mir strîtes} al gelîche;\\ 
30 & si \textbf{dûhte} mich lônes rîche.\\ 
\end{tabular}
\scriptsize
\line(1,0){75} \newline
G I L M Z Fr49 Fr61 \newline
\line(1,0){75} \newline
\textbf{1} \textit{Initiale} G I L Z  \textbf{19} \textit{Initiale} I  \newline
\line(1,0){75} \newline
\textbf{1} man] [wan]: man G \textbf{2} die] \textit{om.} L Z \textbf{3} durch] Zcu M \textbf{6} daz] dez Fr49  $\cdot$ lêren] geleren L \textbf{7} Wer deme grale sich dienstes hat bewegen M  $\cdot$ swer] Wer L Z (Fr49)  $\cdot$ dienstes] dienst I (Fr49) dienstes gein Z \textbf{8} wîbe] wiben I L M  $\cdot$ er minne] mýnne er L (M) (Z) \textbf{9} der] Wan der L (M) Z \textbf{10} Zuͯ rechte eine cronen reyne M  $\cdot$ ein] Zuͯ rechte eine L (Z) \textbf{12} hêrrenlôsiu] herre losie L herczelose M \textbf{14} dienstes] dinste M \textbf{15} geriet] riet Fr61  $\cdot$ vlætigiu] [flæget]: flætigev G fledecliche M (Z) werleichev Fr61 \textbf{16} unde] an Fr61  $\cdot$ werden] werdes G \textbf{18} dâ] Das M (Fr61)  $\cdot$ herticlîchen] hercziclichir M \textbf{20} dûhten] duͯchte L \textbf{23} mir vröude] Mit froude L Mit frevden Z \textbf{24} vil] \textit{om.} Fr61 \textbf{25} mich ir] in ir G mich in L  $\cdot$ minnen kraft] ritershaft I mynne craft L (M) (Fr61) \textbf{26} der] den M  $\cdot$ wilden] wonden L  $\cdot$ verren] \textit{om.} Fr61 \textbf{27} minne] libe M \textbf{29} mir strîtes] mit strite M  $\cdot$ al] \textit{om.} Fr61 \textbf{30} dûhte] duchten M \newline
\end{minipage}
\hspace{0.5cm}
\begin{minipage}[t]{0.5\linewidth}
\small
\begin{center}*T
\end{center}
\begin{tabular}{rl}
 & \begin{large}S\end{large}us gît man vonme Grâle dan\\ 
 & offenlîche \textbf{die} megde, verholne die man,\\ 
 & durch vruht ze dienste wider dar,\\ 
 & ob ir \textbf{vruht} des Grâles schar\\ 
5 & mit dienste suln mêren;\\ 
 & daz kan si got wol lêren.\\ 
 & swer \textbf{aber} \textbf{dem Grâle hât dienstes sich} bewegen,\\ 
 & \textbf{gegen} \textbf{wîben} \textbf{minne er} muoz verpflegen.\\ 
 & \textbf{wan} der künec sol haben eine\\ 
10 & \textbf{ze rehte}, eine konen reine,\\ 
 & unde an\textit{der}, die got \textbf{hât} gesant\\ 
 & ze hêrren in hêrrenlôs\textit{iu} lant.\\ 
 & über daz gebot ich mich bewac,\\ 
 & daz ich \textbf{durch} minne \textbf{dienstes} pflac.\\ 
15 & mir \textbf{riet} mîn \textbf{vlætig\textit{iu}} jugent\\ 
 & unde eines werden wîbes tugent,\\ 
 & daz ich in ir dienst reit,\\ 
 & \textbf{dâ} \textbf{ich} dicke \textbf{herteclîche} streit.\\ 
 & die wilden âventiure\\ 
20 & \textbf{sîn mohten} \textbf{sô} gehiure,\\ 
 & daz ich selten turnierte.\\ 
 & ir minne condewierte\\ 
 & \textbf{mit} vröude in daz herze mîn;\\ 
 & durch si tet ich vil \textbf{strîtes} schîn.\\ 
25 & des twanc mich ir minnen kraft\\ 
 & gegen der wilden verren rîterschaft.\\ 
 & ir minne ich alsus koufte:\\ 
 & der heide\textit{n} unde der getoufte\\ 
 & wâren \textbf{mir strîtes} alglîch;\\ 
30 & si \textbf{dûhte} mich \textbf{sô} lônes rîch.\\ 
\end{tabular}
\scriptsize
\line(1,0){75} \newline
T U V W O Q R Fr40 \newline
\line(1,0){75} \newline
\textbf{1} \textit{Initiale} T V O Fr40  \newline
\line(1,0){75} \newline
\textbf{1} \textit{Die Verse 453.1-502.30 fehlen} U   $\cdot$ Sus] ÷vs O [*ls]: Sus Q \textbf{2} megde] meg W \textbf{4} vruht] kint W (O) Q (R) Fr40 \textbf{7} swer] Wer W Q R  $\cdot$ aber] \textit{om.} W  $\cdot$ dem] den W R  $\cdot$ hât dienstes sich] sich diensts hat W (Q) (R) sich dienst hat O dinstes sich hat Fr40  $\cdot$ bewegen] [begen]: bewegen T \textbf{8} wîben] weybe Q  $\cdot$ minne er] er minne V W O Q R  $\cdot$ verpflegen] pflegen R \textbf{9} wan] Wa R \textbf{10} rehte] rechtten R  $\cdot$ eine konen] eine schoͤne V ein erkorn R \textbf{11} ander die] andie T \textbf{12} hêrrenlôsiu] herrelôse T hoͯrlosse R \textbf{13} bewac] bewanck Q \textbf{15} vlætigiu] vletige T \textbf{16} werden wîbes] wibes werdiv O \textbf{17} in ir] mir Q \textbf{18} dâ] [Daz]: Da V Do W Q  $\cdot$ herteclîche] hertzelichen W \textbf{19} wilden] weile die Q selben R \textbf{20} sîn mohten] Mich duhte V (R) Mich dauchten W (O) (Q)  $\cdot$ gehiure] vngehúrre R \textbf{21} turnierte] turrite Q \textbf{22} condewierte] kunduierte W \textbf{23} mit] Mir V W O Q  $\cdot$ vröude] froͯwden R \textbf{25} mich] \textit{om.} W  $\cdot$ minnen] minne O \textbf{26} wilden] widen R \textbf{27} alsus] also W Q R \textbf{28} heiden] [heideiden]: heidenen T heidene V haide W  $\cdot$ getoufte] getauften Q (R) \textbf{29} alglîch] geliche R \textbf{30} dûhte] duchttent R  $\cdot$ sô] \textit{om.} V W O Q R \newline
\end{minipage}
\end{table}
\end{document}
