\documentclass[8pt,a4paper,notitlepage]{article}
\usepackage{fullpage}
\usepackage{ulem}
\usepackage{xltxtra}
\usepackage{datetime}
\renewcommand{\dateseparator}{.}
\dmyyyydate
\usepackage{fancyhdr}
\usepackage{ifthen}
\pagestyle{fancy}
\fancyhf{}
\renewcommand{\headrulewidth}{0pt}
\fancyfoot[L]{\ifthenelse{\value{page}=1}{\today, \currenttime{} Uhr}{}}
\begin{document}
\begin{table}[ht]
\begin{minipage}[t]{0.5\linewidth}
\small
\begin{center}*D
\end{center}
\begin{tabular}{rl}
\textbf{754} & \textbf{lît} hie bî \textbf{mit werder} diet,\\ 
 & von \textbf{den} ich mich hiute schiet,\\ 
 & mit grôzer \textbf{minneclîcher} schar.\\ 
 & wir sehen dâ vrouwen wol gevar."\\ 
5 & Dô der heiden hôrte nennen wîp\\ 
 & - diu wâren \textbf{êt} sîn selbes lîp -\\ 
 & er sprach: "\textbf{dâ} vüere mich \textbf{hin} mit dir.\\ 
 & dar zuo soltû sagen mir\\ 
 & mære, der ich vrâge.\\ 
10 & sehe wir unser mâge,\\ 
 & sô wir zArtuse komen?\\ 
 & von des vuore ich hân vernomen,\\ 
 & daz er \textbf{sî prîses} rîche\\ 
 & unt \textbf{er} var \textbf{ouch} werdeclîche."\\ 
15 & \textbf{Dô sprach aber} Parzival:\\ 
 & "wir sehen dâ vrouwen lieht gemâl.\\ 
 & sich failiert niht unser vart;\\ 
 & wir \textit{v}inden unsern rehten art,\\ 
 & liute, von den wir sîn \textbf{erborn},\\ 
20 & \textbf{etslîches} houbet ze\textbf{r} krône erkorn."\\ 
 & Ir deweder dô niht langer saz.\\ 
 & Parzival des niht vergaz,\\ 
 & er\textbf{n} holte sînes bruoder swert.\\ 
 & daz stiez er dem degen wert\\ 
25 & wider in die scheiden.\\ 
 & dâ wart von in beiden\\ 
 & \begin{large}Z\end{large}ornlîcher haz vermiten\\ 
 & unt geselleclîche \textbf{dan} geriten.\\ 
 & ê si zArtuse \textbf{wâren} komen,\\ 
30 & dâ \textbf{was} ouch mære von in vernomen.\\ 
\end{tabular}
\scriptsize
\line(1,0){75} \newline
D Fr12 \newline
\line(1,0){75} \newline
\textbf{5} \textit{Majuskel} D  \textbf{15} \textit{Majuskel} D  \textbf{21} \textit{Majuskel} D  \textbf{27} \textit{Initiale} D  \newline
\line(1,0){75} \newline
\textbf{14} er] \textit{om.} Fr12 \textbf{15} Parzival] Parcifal D Fr12 \textbf{17} failiert] enwent Fr12 \textbf{18} vinden] winden D \textbf{22} Parzival] Parcifal D Fr12 \textbf{25} die] sine Fr12 \newline
\end{minipage}
\hspace{0.5cm}
\begin{minipage}[t]{0.5\linewidth}
\small
\begin{center}*m
\end{center}
\begin{tabular}{rl}
 & \textbf{het} hie bî \textbf{mir werden} diet,\\ 
 & von \textbf{den} ich mich hiute schiet,\\ 
 & mit grôzer \textbf{wünneclîcher} schar.\\ 
 & wir sehen d\textit{â} vrowen wol gevar."\\ 
5 & dô der heiden hôrt nennen wîp\\ 
 & - die wâren \textbf{eht} sî\textit{n} selbes lîp -\\ 
 & er sprach: "\textbf{dar} vüere mich mit dir.\\ 
 & dar zuo soltû sagen mir\\ 
 & mære, der ich \textbf{dich} vrâge.\\ 
10 & sehen wir unser mâge,\\ 
 & sô wir zuo Artuse komen?\\ 
 & von des vuor ich hân vernomen,\\ 
 & daz er \textbf{sî prîses} rîch\\ 
 & und var \textbf{ouch} wirdeclîch."\\ 
15 & \textbf{dô sprach aber} Parcifal:\\ 
 & "wir sehen d\textit{â} vrowen lieht gemâl.\\ 
 & sich fâlieret niht unser vart;\\ 
 & wir vinden unse\textit{r}n rehten art,\\ 
 & liute, von de\textit{n} wir sîn \textbf{erborn},\\ 
20 & \textbf{etlîch} houbt ze\textbf{r} \textit{k}rôn erkorn."\\ 
 & ir deweder dô niht langer saz.\\ 
 & Parcifal des niht vergaz,\\ 
 & er holt sînes bruoder swert.\\ 
 & daz stiez er dem degen wert\\ 
25 & wider in die scheiden.\\ 
 & dô wart von in beiden\\ 
 & zorneclîcher haz vermiten\\ 
 & und geselleclîch \textbf{dan} geriten.\\ 
 & ê si zuo Artuse \textbf{wæren} komen,\\ 
30 & dô \textbf{was} ouch mære von in vernomen.\\ 
\end{tabular}
\scriptsize
\line(1,0){75} \newline
m n o V V' \newline
\line(1,0){75} \newline
\textbf{5} \textit{Überschrift:} Hie kvmmet Parzefal mit sime bruͦdere Ferevis Anschevin zvͦ kv́nig artus hof Do sv́ mittenander gestritten hettent V  Hie komet parzifal mit sime bruder zv kvnig artuse vnd werdin gar wol enpfangen V'   $\cdot$ \textit{Initiale} V V'  \newline
\line(1,0){75} \newline
\textbf{1} het] Lit V (V')  $\cdot$ mir] mit V V'  $\cdot$ werden] werder n o V V' \textbf{4} dâ] do m n o V V'  $\cdot$ vrowen] freuden V' \textbf{6} sîn] sich m \textbf{7} mich] mich hin V V' \textbf{9} Herre dez ich dich dich froge V' \textbf{11} Artuse] kvnig artuse V' \textbf{12} vuor ich] ich fuͦr o  $\cdot$ hân] hant o \textbf{15} aber] ouch n abir do V'  $\cdot$ Parcifal] parzefal V parzifal V' \textbf{16} dâ] do m n o V \textit{om.} V' \textbf{18} unsern] vnsen m \textbf{19} den] der m n o  $\cdot$ sîn] sint m n o V V' \textbf{20} zer] zvͦ V (V')  $\cdot$ krôn] ron m \textbf{21} deweder] do weder n itweder V'  $\cdot$ langer] langen o \textbf{22} Parcifal] Parzefal V Parzifal V'  $\cdot$ des] do dez V' \textbf{23} holt] enholte V V' \textbf{24} stiez] stiesse n :ies o \textbf{27} vermiten] [mit]: vermiten V' \textbf{29} ê] Vnde e V (V')  $\cdot$ wæren komen] woren kommen o [*]: werent komen V [quoment]: quomen V' \newline
\end{minipage}
\end{table}
\newpage
\begin{table}[ht]
\begin{minipage}[t]{0.5\linewidth}
\small
\begin{center}*G
\end{center}
\begin{tabular}{rl}
 & \textbf{lît} hie bî \textbf{mit werder} diet,\\ 
 & von \textbf{dem} ich mich hiute schiet,\\ 
 & \begin{large}M\end{large}it grôzer \textbf{minneclîcher} schar.\\ 
 & wir sehen dâ vrouwen wol gevar."\\ 
5 & dô der heiden hôrte nennen wîp\\ 
 & - die wâren \textbf{êt} sîn selbes lîp -\\ 
 & er sprach: "\textbf{dar} vüere mich mit dir.\\ 
 & dar zuo soltû sagen mir\\ 
 & mære, der ich \textbf{dich} vrâge.\\ 
10 & sehe wir unser mâge,\\ 
 & sô wir ze Artuse komen?\\ 
 & von des vuore ich hân vernomen,\\ 
 & daz er \textbf{sî brîses} rîche\\ 
 & unde \textbf{er} var werdeclîche."\\ 
15 & \textbf{aber sprach dô} Parzival:\\ 
 & "wir sehen dâ vrouwen lieht gemâl.\\ 
 & sich fâliert niht unser vart;\\ 
 & wir vinden \textbf{dâ} unsern rehten art,\\ 
 & liute, von den wir sîn \textbf{geborn},\\ 
20 & \textbf{etslîch} houbet ze krône erkorn."\\ 
 & ir deweder dô niht langer saz.\\ 
 & Parzival des niht vergaz,\\ 
 & er\textbf{ne} holte sînes bruoder swert.\\ 
 & daz stiez er \textit{dem} deg\textit{en wert}\\ 
25 & wider in die scheiden.\\ 
 & dâ wart von in beiden\\ 
 & zornlîcher haz vermiten\\ 
 & unde geselleclîche \textbf{dan} geriten.\\ 
 & ê si ze Artuse \textbf{wâren} komen,\\ 
30 & dô \textbf{was} ouch mære von in vernomen.\\ 
\end{tabular}
\scriptsize
\line(1,0){75} \newline
G I L M Z Fr43 \newline
\line(1,0){75} \newline
\textbf{1} \textit{Initiale} L Z  \textbf{3} \textit{Initiale} G I  \textbf{29} \textit{Initiale} I  \newline
\line(1,0){75} \newline
\textbf{1} bî] \textit{om.} L \textbf{2} dem] dē M den Z \textbf{5} dô] Da M Z  $\cdot$ hôrte] horte da I \textbf{6} êt] \textit{om.} Z  $\cdot$ sîn] sins M Z \textbf{7} er sprach] \textit{om.} L  $\cdot$ mich] mich hin M Z \textbf{9} mære der] Mere des L (M) Z  $\cdot$ dich] \textit{om.} L \textbf{10} wir] wie M \textbf{11} Artuse] Artuͯse L artus Z \textbf{12} vuore] sýt L  $\cdot$ ich] ich e Z \textbf{13} sî brîses] prises sy M (Z) \textbf{14} er] \textit{om.} L  $\cdot$ var] var ouch L M Z \textbf{15} sprach dô] do sprach I sprach da M (Z)  $\cdot$ Parzival] parcifal G Z parzifal I L M \textbf{16} lieht] licht L M \textbf{18} dâ] \textit{om.} L M  $\cdot$ unsern] vnserr I vnse M (Z) \textbf{19} sîn geborn] si erborn I \textbf{20} etslîch] Ettisliche M  $\cdot$ houbet] horet L  $\cdot$ erkorn] geborn L \textbf{21} deweder] ietdweder I widir M  $\cdot$ dô] da M Z  $\cdot$ langer] lange I Z \textbf{22} Parzival] parcifal G (Z) Parzifal I L M \textbf{24} stiez] \textit{om.} M  $\cdot$ dem degen wert] deg -*- G \textbf{25} wider] Stiesz widir M  $\cdot$ die] sin L  $\cdot$ scheiden] shaide I \textbf{26} dâ] do I \textbf{29} Artuse] artus I Z  $\cdot$ wâren] wern I (Z) \textbf{30} dô] Da M Z  $\cdot$ in] ir Z \newline
\end{minipage}
\hspace{0.5cm}
\begin{minipage}[t]{0.5\linewidth}
\small
\begin{center}*T
\end{center}
\begin{tabular}{rl}
 & \textbf{liget} hie bî \textbf{mit werder} diet,\\ 
 & von \textbf{den} ich mich hiute schiet,\\ 
 & mit grôzer \textbf{wünneclîcher} schar.\\ 
 & wir sehen dâ vrouwen wol gevar."\\ 
5 & \begin{large}D\end{large}ô der heiden hôrte nennen wîp\\ 
 & - die wâren sîn selbes lîp -\\ 
 & er sprach: "\textbf{dar} vüere mich mit dir.\\ 
 & dar zuo soltû sagen mir\\ 
 & mære, der ich \textbf{dich} vrâge.\\ 
10 & sehen wir unser mâge,\\ 
 & sô wir zuo Artuse komen?\\ 
 & von des vuore ich hân vernomen,\\ 
 & daz er \textbf{prîses sî} rîche\\ 
 & und \textbf{er} var \textbf{ouch} wirdeclîche."\\ 
15 & \textbf{aber sprach dô} Parcifal:\\ 
 & "wir sehen d\textit{â} \textit{v}rouwen lieht gemâl.\\ 
 & sich fallieret niht unser vart;\\ 
 & wir vinden \textbf{dâ} unsern rehten art,\\ 
 & liute, von den wir sîn \textbf{geborn},\\ 
20 & \textbf{etslîch} houbet zuo \textbf{der} krône erkorn."\\ 
 & ir ietweder \textbf{sîte} dô niht langer saz.\\ 
 & Parcifal des niht vergaz,\\ 
 & er \textbf{en}holte sînes bruoder swert.\\ 
 & daz stiez er dem degen wert\\ 
25 & wider in die scheiden.\\ 
 & dô wart von in beiden\\ 
 & zornlîcher haz vermiten\\ 
 & und geselleclîche \textbf{dô} geriten.\\ 
 & ê si zuo Artuse \textbf{wæren} komen,\\ 
30 & dô \textbf{wâren} ouch mære von in vernomen.\\ 
\end{tabular}
\scriptsize
\line(1,0){75} \newline
U W Q R \newline
\line(1,0){75} \newline
\textbf{5} \textit{Initiale} U  \newline
\line(1,0){75} \newline
\textbf{2} den] dem W R \textbf{4} dâ] do W Q \textbf{6} wâren] waren echt W (R) warren auch Q \textbf{9} der] das W des Q R \textbf{10} unser] v́nsz R \textbf{11} Artuse] Artus R \textbf{12} ich hân] ich ban W han ich R \textbf{13} prîses sî] sig prises R \textbf{15} Parcifal] herr partzifal W partzifal Q parczifal R \textbf{16} dâ] do U W Q  $\cdot$ vrouwen] ruͦwen U  $\cdot$ lieht] licht Q \textbf{18} dâ] \textit{om.} W Q R  $\cdot$ unsern rehten] vnszrú rechtte R \textbf{20} \textit{statt 754.20 (korrigierende Versdoppelung):} Ettlich die da tragent Cron / Ettlich die ze der crone [*]: erkorn R   $\cdot$ der] \textit{om.} W Q  $\cdot$ erkorn] enkoren W \textbf{21} ir ietweder] Ir deweder W Jtweder Q  $\cdot$ sîte] \textit{om.} W Q R  $\cdot$ dô] da R \textbf{22} Parcifal] Her partzifal W Partzifal Q Parczifal R \textbf{23} er enholte] Er holte W (R) Ern holt Q \textbf{25} die scheiden] sin scheide R \textbf{28} dô] dan W R \textbf{29} Artuse] artusen Q Artus R  $\cdot$ wæren] waren W \textbf{30} wâren] was Q R \newline
\end{minipage}
\end{table}
\end{document}
