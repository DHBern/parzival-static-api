\documentclass[8pt,a4paper,notitlepage]{article}
\usepackage{fullpage}
\usepackage{ulem}
\usepackage{xltxtra}
\usepackage{datetime}
\renewcommand{\dateseparator}{.}
\dmyyyydate
\usepackage{fancyhdr}
\usepackage{ifthen}
\pagestyle{fancy}
\fancyhf{}
\renewcommand{\headrulewidth}{0pt}
\fancyfoot[L]{\ifthenelse{\value{page}=1}{\today, \currenttime{} Uhr}{}}
\begin{document}
\begin{table}[ht]
\begin{minipage}[t]{0.5\linewidth}
\small
\begin{center}*D
\end{center}
\begin{tabular}{rl}
\textbf{312} & unz ûf daz \textbf{siufzebære} zil.\\ 
 & hie kom, von der ich sprechen wil,\\ 
 & \textbf{ein magt, gein triwen} wol gelobt,\\ 
 & wan daz ir zuht \textbf{was} vertobt.\\ 
5 & ir mære \textbf{tet} vil liuten leit.\\ 
 & \textbf{nû} hœret, wie diu juncvrouwe reit,\\ 
 & Einen mûl, hôch als ein kastelân,\\ 
 & val unt dennoch sus getân:\\ 
 & nas snitec unt verbrant,\\ 
10 & als ungerschiu marc erkant.\\ 
 & ir \textit{z}oum unt ir gereite\\ 
 & was geworht mit arbeite,\\ 
 & tiwer unde rîche.\\ 
 & ir mûl gienc volleclîche.\\ 
15 & si was niht \textbf{vrouwelîche} \textbf{var}.\\ 
 & \textbf{wê}, waz solt ir komen dar?\\ 
 & si kom iedoch. \textbf{daz} muose \textbf{êt} sîn.\\ 
 & \textbf{Artuses her} si brâhte pîn.\\ 
 & Der \textbf{meide} ir \textbf{kunst} des verjach,\\ 
20 & alle sprâche si wol sprach,\\ 
 & latîn, heidensch, franzois.\\ 
 & si was der witze kurtois,\\ 
 & dîaletice unt jêometrî,\\ 
 & ir wâren ouch die liste bî\\ 
25 & von astronomîe.\\ 
 & \textbf{Si hiez} Cundrie,\\ 
 & surziere was ir zuonam,\\ 
 & in dem munde niht diu lam,\\ 
 & wan \textbf{der} gereit ir genuoc.\\ 
30 & vil hôher vreude si nider sluoc.\\ 
\end{tabular}
\scriptsize
\line(1,0){75} \newline
D \newline
\line(1,0){75} \newline
\textbf{7} \textit{Majuskel} D  \textbf{19} \textit{Majuskel} D  \textbf{26} \textit{Majuskel} D  \newline
\line(1,0){75} \newline
\textbf{10} ungerschiu] vngersciv D \textbf{11} zoum] toͮm D \textbf{18} Artuses] Artvs D \textbf{21} franzois] franzoys D \textbf{26} Cundrie] Cvndrîe D \newline
\end{minipage}
\hspace{0.5cm}
\begin{minipage}[t]{0.5\linewidth}
\small
\begin{center}*m
\end{center}
\begin{tabular}{rl}
 & \multicolumn{1}{l}{ - - - }\\ 
 & \multicolumn{1}{l}{ - - - }\\ 
 & \textbf{gegen zuht vil dicke} wol gelobet,\\ 
 & wan daz ir zuht \textbf{hie wirt} vertobet.\\ 
5 & ir mære \textbf{tuot} vil liuten leit.\\ 
 & \textbf{nû} hœret, wie diu juncvrouwe reit,\\ 
 & einen mûl, hôch als ein kastelân,\\ 
 & val und dannoch sus getân:\\ 
 & nassnitic und verbrant,\\ 
10 & als unge\textit{r}schiu ma\textit{rc} erkant.\\ 
 & ir z\textit{ou}m und ir gereite\\ 
 & was geworht mit arbeite,\\ 
 & tiure und \textbf{dar zuo} rîc\textit{h}e.\\ 
 & ir mûl gienc volleclîche.\\ 
15 & si was niht \textbf{vrouwelîche} \textbf{var}.\\ 
 & \textbf{wê}, waz solt ir komen dar?\\ 
 & si kam iedoch. \textbf{ez} muose sîn.\\ 
 & \textbf{Artuses her} si brâhte pîn.\\ 
 & der \textbf{megde} ir \textbf{kunst} des verjach,\\ 
20 & alle sprâche si wol sprach,\\ 
 & l\textit{atîn}, heidensch, franzois.\\ 
 & si was der witze kurtois,\\ 
 & dîal\textit{et}i\textit{c}e und \textit{gê}ometrî,\\ 
 & ir wâren ouch die liste bî\\ 
25 & von astronomîe,\\ 
 & \textbf{diu} Condrie,\\ 
 & \textit{s}urziere was ir zuoname,\\ 
 & in dem munde niht diu lame,\\ 
 & wand \textbf{er} geret ir genuoc.\\ 
30 & vil hôher vröude si nider sluoc.\\ 
\end{tabular}
\scriptsize
\line(1,0){75} \newline
m n o \newline
\line(1,0){75} \newline
\newline
\line(1,0){75} \newline
\textbf{1} \textit{Die Verse 312.1-2 fehlen} m n o  \textbf{3} vil] sie [wol]: vil o  $\cdot$ dicke] [dich]: dicke n \textbf{4} vertobet] vertauͯbet o \textbf{6} juncvrouwe] jungfroͧwen n  $\cdot$ reit] [seit]: reit o \textbf{9} nassnitic] Nassnigetig o  $\cdot$ verbrant] verbant n \textbf{10} ungerschiu] [e*]: vngesche m vngersz n vngerst o  $\cdot$ marc] magt m \textbf{11} zoum] zuom m  $\cdot$ ir gereite] das gereite n \textbf{12} geworht] geforcht o \textbf{13} rîche] [richt]: richte m \textbf{14} ir mûl] Jre mvle n \textbf{15} var] gevar n (o) \textbf{17} iedoch] e doch o  $\cdot$ muose] musse m muͯst n o \textbf{18} Artuses] Artus m n o \textbf{21} latîn] Luter m  $\cdot$ franzois] franczos m vnd frantzois n vnd franczosis o \textbf{22} kurtois] kurtoris o \textbf{23} dîaletice] Dialcrite m (n) Du altrite o  $\cdot$ gêometrî] Stometri m scoͯmetri n stomestri o \textbf{24} die] der n  $\cdot$ liste] listi o \textbf{25} astronomîe] astromonie m \textbf{26} diu] Sú hiesz n Sus hies o  $\cdot$ Condrie] Cundrie n connodrie o \textbf{27} surziere] Furcziere m Surtzier n Surczier o \textbf{28} in] An o  $\cdot$ diu] de o \newline
\end{minipage}
\end{table}
\newpage
\begin{table}[ht]
\begin{minipage}[t]{0.5\linewidth}
\small
\begin{center}*G
\end{center}
\begin{tabular}{rl}
 & unz ûf daz \textbf{siuftebære} zil.\\ 
 & hie kom, von der ich sprechen wil,\\ 
 & \textbf{ein ma\textit{get}, gein triwen} wol gelobt,\\ 
 & wan daz ir zuht \textbf{was} vertobt.\\ 
5 & ir mære \textbf{tet} vil liuten leit.\\ 
 & hœret, wie diu juncvrouwe reit,\\ 
 & einen mûl, hôch als ein kastelân,\\ 
 & val unde dannoch sus getân:\\ 
 & nase snitic unde verbrant,\\ 
10 & al\textit{s} \textit{u}ngers\textit{chiu} marc erkant.\\ 
 & ir zoum unde ir \textbf{pferdes} gereite\\ 
 & was geworht mit arbeite,\\ 
 & tiur unde rîche.\\ 
 & ir mûl gie volleclîche.\\ 
15 & si\textbf{ne} was niht \textbf{vroulîch} \textbf{gevar}.\\ 
 & \textbf{owê}, waz solt ir komen dar?\\ 
 & si kom iedoch. \textbf{daz} muos \textbf{êt} sîn.\\ 
 & \textbf{Artuses her} si brâhte pîn.\\ 
 & der \textbf{vrouwen} ir \textbf{zuht} des ver\textit{j}ach,\\ 
20 & alle sprâche si wol sprach,\\ 
 & \begin{large}L\end{large}atîne, heidensch, franzois.\\ 
 & si was der witze kurtois,\\ 
 & dî\textit{a}let\textit{i}ke un\textit{de} gêometrîe.\\ 
 & \multicolumn{1}{l}{ - - - }\\ 
25 & \multicolumn{1}{l}{ - - - }\\ 
 & \textbf{si hiez} Gundrie,\\ 
 & surzier was ir zuoname,\\ 
 & in dem munde niht diu lame,\\ 
 & wan \textbf{er} geredet ir genouc.\\ 
30 & vil hôher vröude si nider sluoc.\\ 
\end{tabular}
\scriptsize
\line(1,0){75} \newline
G I O L M Q R Z Fr64 \newline
\line(1,0){75} \newline
\textbf{5} \textit{Initiale} M  \textbf{15} \textit{Initiale} O R Z  \textbf{17} \textit{Initiale} I  \textbf{21} \textit{Initiale} G  \newline
\line(1,0){75} \newline
\textbf{1} siuftebære] súnsz zerede R sevftzeberez Z \textbf{3} Ein trúwen wol gelopt R  $\cdot$ maget] man G  $\cdot$ gein triwen] getrewen Q von \sout{der ich} trewen Z  $\cdot$ gelobt] gelop I \textbf{4} ir] [sin]: ir G er R \textbf{5} ir] Die L \textbf{6} reit] seit Q \textbf{7} \textit{Die Verse 312.7-313.4 fehlen} L   $\cdot$ einen] ein I (O) (M) (R) (Z)  $\cdot$ als] sam Z  $\cdot$ ein] einen Q \textbf{9} nase] Hasz Q  $\cdot$ snitic] gesniten I sliffic M \textbf{10} als ungerschiu] als ein vngers G Alse vngirscen M Als vngerische Q Z Als vngresche R \textbf{11} ir pferdes] ir pherit I ir O M Q Z \textit{om.} R \textbf{12} geworht] geworhte O \textbf{15} sine] ÷i O Sie M Q (R) Z  $\cdot$ vroulîch] froͯlich R \textbf{16} owê] Awe O We Z  $\cdot$ solt] wolt I sal M \textbf{17} si] Die Q  $\cdot$ daz] ez I  $\cdot$ muos êt] muͤz I muste M Z muͦst R \textbf{18} Artuses] Artus Q R Z  $\cdot$ pîn] in pin I \textbf{19} ichn waiz waz si an in rach I  $\cdot$ der vrouwen] Der mæide O (M) (Q) (Z) Dar meige R  $\cdot$ zuht] kunst Q R Z  $\cdot$ verjach] verach G \textbf{21} heidensch] heidens M \textbf{22} kurtois] en teils M gyrtois Q kurtosys R \textbf{23} dîaletike] dioletche G Dyalanke Q Dyaletica R  $\cdot$ unde] vn G vnd die I \textbf{24} \textit{Die Verse 312.24-25 fehlen} G I   $\cdot$ Jr warn di liste bi O (Z)  $\cdot$ Jr warin ouch dise liste bÿ M  $\cdot$ Es waren auch die liste bey Q  $\cdot$ Jr waurent och die liste bẏ R \textbf{25} Von antromie O  $\cdot$ Von astronomie M Q (R) Z \textbf{26} si] si selb I  $\cdot$ Gundrie] kungri I kvndrie O (Q) (Z) kondrie M kondrye R \textbf{27} surzier] surscier I Svrzir O Z Lach zurzir M Suzie Q Surczir R \textbf{28} niht] was sie nicht M  $\cdot$ diu lame] zelam O lam M \textbf{29} er] der Q R  $\cdot$ geredet] geredete I (M) redet Z \textbf{30} hôher] hohe O (Q)  $\cdot$ vröude] frevden Z  $\cdot$ si] \textit{om.} I \newline
\end{minipage}
\hspace{0.5cm}
\begin{minipage}[t]{0.5\linewidth}
\small
\begin{center}*T
\end{center}
\begin{tabular}{rl}
 & unz ûf daz \textbf{riuwebære} zil.\\ 
 & Hie kom, von der ich sprechen wil,\\ 
 & \textbf{ein maget, gegen triwen} wol gelobet,\\ 
 & wan daz ir zuht \textbf{was} vertobet.\\ 
5 & ir mære \textbf{tet} vil liuten leit.\\ 
 & hœret, wie diu juncvrouwe reit,\\ 
 & Einen mûl, hôch als ein kastelân,\\ 
 & val unde dannoch sus getân:\\ 
 & nase snitic unde verbrant,\\ 
10 & als ungerschiu marc erkant.\\ 
 & ir zoum unde ir gereite\\ 
 & was geworht mit arbeite,\\ 
 & tiure unde rîche.\\ 
 & ir mûl gienc volleclîche.\\ 
15 & si was niht \textbf{vrœlîche} \textbf{var}.\\ 
 & \textbf{ouwê}, waz solte ir komen dar?\\ 
 & si kom iedoch. \textbf{ez} muose sîn.\\ 
 & \textbf{Artuse} si brâhte pîn.\\ 
 & der \textbf{maget} ir \textbf{kunst} des verjach,\\ 
20 & alle sprâche si wol sprach,\\ 
 & latîn, heidensch \textbf{unde} franzois.\\ 
 & si was der witze kurtois,\\ 
 & Dyaletike unde gêometrî,\\ 
 & ir wâren ouch die liste bî\\ 
25 & von astronomîe.\\ 
 & \textbf{si hiez} Kundrie,\\ 
 & Surziere was ir zuoname,\\ 
 & in dem munde niht diu lame,\\ 
 & wan \textbf{der} geredet ir genuoc.\\ 
30 & vil hôher vröude si nider sluoc.\\ 
\end{tabular}
\scriptsize
\line(1,0){75} \newline
T U V W \newline
\line(1,0){75} \newline
\textbf{1} \textit{Überschrift:} Hie kam kundrie die iunckfrawe von montsaluatz fúr die tauelrunde vnd schalt her partzfalen mit boͤsen worten W   $\cdot$ \textit{Platz für Illustration ausgespart} W   $\cdot$ \textit{Initiale} W  \textbf{2} \textit{Majuskel} T  \textbf{7} \textit{Majuskel} T  \textbf{15} \textit{Initiale} V  \textbf{23} \textit{Majuskel} T  \textbf{27} \textit{Majuskel} T  \newline
\line(1,0){75} \newline
\textbf{1} \textit{Die Verse 312.1-2 fehlen} V   $\cdot$ unz] Mit U BIs W  $\cdot$ riuwebære] sufcebere U leichte W \textbf{2} kom] waz W \textbf{3} Gegen zvht vil dicke wol gelobet V  $\cdot$ gegen] in W \textbf{4} was] wurt hie V \textbf{5} tet] tuͦt V \textbf{6} hœret] Nun hoͤrent W \textbf{7} Einen] Ein W \textbf{9} snitic] geschúrpffet W \textbf{10} ungerschiu] vngersce T [vn*]: vngersche U vngersche V vngerische W \textbf{12} geworht] gemacht U \textbf{13} unde] vnde darzvͦ V \textbf{15} si] [S*]: Sv́ V  $\cdot$ was] in was U  $\cdot$ vrœlîche] frowelich V  $\cdot$ var] geuar V W \textbf{17} muose] mvese T muͦz U mvͤste eht V \textbf{18} Artuse] Artuses her V Artuse her W \textbf{19} Der megede kvnst [de*]: dez veriach V  $\cdot$ der] Die U Ir W \textbf{21} unde] \textit{om.} U V W \textbf{23} dyaletike] Dyaletica V Dyaletici W \textbf{24} liste] leste U \textbf{26} Kundrie] Kvndrîe T kuͦndrie U \textbf{27} Surziere] Suͦrziere U Surrzier W \textbf{29} Wann do mitte gerette sy genuͦg W  $\cdot$ der] er V  $\cdot$ geredet] gerete U \textbf{30} hôher] hohe W  $\cdot$ vröude] vreiden U \newline
\end{minipage}
\end{table}
\end{document}
