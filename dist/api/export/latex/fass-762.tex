\documentclass[8pt,a4paper,notitlepage]{article}
\usepackage{fullpage}
\usepackage{ulem}
\usepackage{xltxtra}
\usepackage{datetime}
\renewcommand{\dateseparator}{.}
\dmyyyydate
\usepackage{fancyhdr}
\usepackage{ifthen}
\pagestyle{fancy}
\fancyhf{}
\renewcommand{\headrulewidth}{0pt}
\fancyfoot[L]{\ifthenelse{\value{page}=1}{\today, \currenttime{} Uhr}{}}
\begin{document}
\begin{table}[ht]
\begin{minipage}[t]{0.5\linewidth}
\small
\begin{center}*D
\end{center}
\begin{tabular}{rl}
\textbf{762} & \begin{large}J\end{large}ofreit was wider komen.\\ 
 & von dem \textbf{hete} Artus vernomen,\\ 
 & wie er werben solde,\\ 
 & \textbf{ob} er enpfâhen wolde\\ 
5 & sînen neven, den heiden.\\ 
 & daz sitzen wart \textbf{bescheiden}\\ 
 & an Gawans ringe\\ 
 & mit \textbf{höfschlîchem dinge}.\\ 
 & Diu massenîe der herzogîn\\ 
10 & unt die \textbf{gesellen} under in\\ 
 & ze Gawans zeswen \textbf{saz}.\\ 
 & anderthalben \textbf{si} mit vreuden \textbf{az},\\ 
 & ritter, \textbf{Clinschors} diet.\\ 
 & \textbf{der} vrouwen sitzen man beschiet.\\ 
15 & über gein Gawane an \textbf{den} ort\\ 
 & sâzen \textbf{Clinschors} vrouwen dort;\\ 
 & der was manegiu lieht gemâl.\\ 
 & Feirefiz unt Parzival\\ 
 & sâzen mitten zwischen \textbf{den} vrouwen;\\ 
20 & man moht \textbf{dâ} clârheit schouwen.\\ 
 & Der \textbf{Turkoyte} Florant\\ 
 & unt Sangive, diu wert erkant,\\ 
 & unt der herzoge von Gowerzin\\ 
 & unt Cundrie, daz wîp sîn,\\ 
25 & über gein ein ander sâzen.\\ 
 & ich wæne des, \textbf{niht} vergâzen\\ 
 & Gawan und Jofreit\\ 
 & ir alten gesellecheit.\\ 
 & si âzen mit ein ander.\\ 
30 & diu herzogîn mit \textbf{blicken} glander\\ 
\end{tabular}
\scriptsize
\line(1,0){75} \newline
D Fr12 \newline
\line(1,0){75} \newline
\textbf{1} \textit{Initiale} D  \textbf{9} \textit{Majuskel} D  \textbf{21} \textit{Majuskel} D  \newline
\line(1,0){75} \newline
\textbf{1} Jofreit] Iofreit D \textbf{13} Clinschors] Clinscors D \textbf{15} Gawane] Gawan Fr12 \textbf{16} Clinschors] Clinscors D Clin:ors Fr12 \textbf{17} lieht] vil ::: Fr12 \textbf{18} Parzival] Parcifal D Par:::al Fr12 \textbf{21} Turkoyte] :::ote Fr12  $\cdot$ Florant] flor:nt Fr12 \textbf{26} des] :as dez Fr12 \newline
\end{minipage}
\hspace{0.5cm}
\begin{minipage}[t]{0.5\linewidth}
\small
\begin{center}*m
\end{center}
\begin{tabular}{rl}
 & Jofr\textit{e}it was wider komen.\\ 
 & von dem \textbf{hât} Artus vernomen,\\ 
 & wie er werben solte,\\ 
 & \textbf{ob} er enpfâhen wolte\\ 
5 & sînen neven, den heiden.\\ 
 & daz sitzen wart \textbf{bescheiden}\\ 
 & an Gawans ringe\\ 
 & mit \textbf{hovelîchem dinge}.\\ 
 & diu massenîe der herzogîn\\ 
10 & und die \textbf{geselleschaft} under in\\ 
 & zuo Gawans zesewen \textbf{saz}.\\ 
 & anderhalp mit vröuden \textbf{az}\\ 
 & ritte\textit{r}, \textbf{\textit{C}linsors} diet.\\ 
 & \textbf{der} vrowen sitzen man beschiet.\\ 
15 & über gegen Gawan an \textbf{den} ort\\ 
 & sâzen \textbf{Clinsors} vrowen dort;\\ 
 & der was manigiu lieht gemâl.\\ 
 & Ferefiz und Parcifal\\ 
 & sâzen mitten zwischen \textbf{die} vrouwen;\\ 
20 & man mohte \textbf{d\textit{â}} clârheit schouwen.\\ 
 & der \textbf{Tur\textit{k}oite} Florant\\ 
 & und Sang\textit{i}ve, diu wert erkant,\\ 
 & und der herzoge von Gowertzin\\ 
 & und Condrie, daz wîp sîn,\\ 
25 & über gegen ein ander sâzen.\\ 
 & ich wæne des, \textbf{\textit{i}ht} vergâzen\\ 
 & Gawan und Jofr\textit{e}it\\ 
 & ir alten gesellicheit.\\ 
 & si âzen mit ein ander.\\ 
30 & diu herzogîn mit \textbf{blic} glander\\ 
\end{tabular}
\scriptsize
\line(1,0){75} \newline
m n o V V' W \newline
\line(1,0){75} \newline
\textbf{1} \textit{Initiale} W  \newline
\line(1,0){75} \newline
\textbf{1} \textit{statt 762.1-28:} Do artuse die mere worden kvnt / Do reit Ioffrid wider zv gawin an der stunt V'   $\cdot$ Jofreit] Jofrit m o V Joffrit n iOfrid W  $\cdot$ was wider] ist dar n \textbf{2} hât] het n (o) (V) W \textbf{5} sînen] [Sinem]: Sinen V \textbf{6} sitzen] sit V \textbf{7} Gawans] Gawens V  $\cdot$ ringe] [ringen]: ringe V \textbf{8} hovelîchem] kostlichem n hoͤueschlichem V  $\cdot$ dinge] dingen o [dingen]: dinge V \textbf{9} Der hirczogin der massenie o \textbf{11} saz] [sa:en]: sasen V \textbf{12} az] [a*en]: asen V \textbf{13} ritter] ritter vnd m Die ritter W  $\cdot$ Clinsors] klynßhors W \textbf{14} der] Die o  $\cdot$ beschiet] beschicht o baschiet W \textbf{15} Gawan] Gawane V  $\cdot$ den ort] dem art n daz ort V ein ort W \textbf{16} Clinsors] klynßhors W  $\cdot$ dort] dot n \textbf{18} Ferefiz] ferefis m Ferrefis n o Ferevis V Ferafis W  $\cdot$ Parcifal] parzefal V herr partzifal W \textbf{19} mitten] mitte o  $\cdot$ die] den V W \textbf{20} mohte] moͤhte V (W)  $\cdot$ dâ] do m n V die o \textit{om.} W  $\cdot$ clârheit] frawen W \textbf{21} Turkoite] turtoite m n turteẏ o turkate V turkoyte W \textbf{22} \textit{Die Verse 762.22-764.27 fehlen} o   $\cdot$ Sangive] sangwe m n sagiue V seyue W \textbf{23} Gowertzin] Gowertzschin V gouerszin W \textbf{24} und] Die n  $\cdot$ Condrie] kvndrie V (W) \textbf{26} iht] eht m nicht W \textbf{27} Jofreit] jofrit m Joffreit V iofreit W \textbf{29} \textit{Die Verse 762.29-764.22 fehlen} V'   $\cdot$ si âzen] Sú sossent n (W) Soszent V \textbf{30} blic] blicken V \newline
\end{minipage}
\end{table}
\newpage
\begin{table}[ht]
\begin{minipage}[t]{0.5\linewidth}
\small
\begin{center}*G
\end{center}
\begin{tabular}{rl}
 & \begin{large}J\end{large}ofreit was wider komen.\\ 
 & von dem \textbf{het} Artus vernomen,\\ 
 & wie er werben solde,\\ 
 & \textbf{ob} er enpfâhen wolde\\ 
5 & sînen neven, den heiden.\\ 
 & daz sitzen wart \textbf{beiden}\\ 
 & an Gawans ringe\\ 
 & mit \textbf{höfschlîchem dinge}.\\ 
 & diu messenîe der herzogîn\\ 
10 & unde die \textbf{gesellen} under in\\ 
 & ze Gawans zeswen \textbf{sâzen}.\\ 
 & anderhalp mit vröuden \textbf{âzen}\\ 
 & rîter \textbf{unde kleine} diet.\\ 
 & \textbf{den} vrouwen sitzen man beschiet.\\ 
15 & über gein Gawan an \textbf{den} ort\\ 
 & sâzen \textbf{kleine} vrouwen dort;\\ 
 & der was manigiu lieht gemâl.\\ 
 & Feirafiz unde Parzival\\ 
 & sâzen mitten zwischen \textbf{den} vrouwen;\\ 
20 & man mohte clârheit schouwen.\\ 
 & der \textbf{Turkoite} Florant\\ 
 & unde Sagive, diu wert erkant,\\ 
 & unde der herzoge von Gowerzin\\ 
 & unde Gundrie, daz wîp sîn,\\ 
25 & über gein ein ander sâzen.\\ 
 & ich wæne des, \textbf{iht} vergâzen\\ 
 & Gawan unde Jofreit\\ 
 & ir alten gesellecheit.\\ 
 & si âzen mit ein ander.\\ 
30 & diu herzogîn mit \textbf{blicken} glander\\ 
\end{tabular}
\scriptsize
\line(1,0){75} \newline
G I L M Z Fr45 \newline
\line(1,0){75} \newline
\textbf{1} \textit{Initiale} G L M Z  \textbf{13} \textit{Initiale} I  \newline
\line(1,0){75} \newline
\textbf{1} Jofreit] Iofreit G Jofreyd Fr45 \textbf{3} \textit{Versfolge 762.4-3} M Fr45  \textbf{4} enpfâhen] erpfahen Z \textbf{5} neven] \textit{om.} L  $\cdot$ heiden] chuͦnen heiden I \textbf{6} wart] daz wart in I  $\cdot$ beiden] bescheýden L (Z) (Fr45) \textbf{7} an Gawans] an des chuͦnen Gawans I An Gawanz L An gawanes M Jn artuses Fr45 \textbf{8} höfschlîchem] vil hofslichem I hoffelicheme M \textbf{9} der] diu I \textbf{11} Gawans] Gawanz L gawansz M  $\cdot$ zeswen] Geselle I \textbf{12} mit] sus mit M Fr45  $\cdot$ vröuden] frowen I (M) \textbf{13} rîter unde kleine] Ritern vnd der chlainen I Ritter vnd clare L Die ritter clinsors diet M (Z) (Fr45) \textbf{14} den] der Fr45 \textbf{15} über gein] vber G Gein Fr45  $\cdot$ Gawan] Gawane L  $\cdot$ an] vͦber an Fr45  $\cdot$ den] daz L \textbf{16} kleine] clare L Z clinsors M (Fr45) \textbf{17} manigiu] vil mangev I  $\cdot$ lieht] licht M \textbf{18} Feirafiz] Ferefiz L Feirefisz M Feirefiz Z Feẏrafẏz Fr45  $\cdot$ unde] vnd der lieht I  $\cdot$ Parzival] parcifal G Z Parzifal I (L) (M) persciual Fr45 \textbf{19} mitten] en mitten I (L) o\textit{m. } Fr45  $\cdot$ den] dy M (Z) \textbf{20} clârheit] da chlarhait I (M) (Z) (Fr45) \textbf{21} Turkoite] Turchoyde I Tuͯrkoýte L Tvrkoit Z tuͦrkoẏte Fr45  $\cdot$ Florant] floriant I \textbf{22} Sagive] saife I sayve M Seyve Z Sagiue Fr45  $\cdot$ wert] werde M \textbf{23} unde] \textit{om.} L  $\cdot$ Gowerzin] Gouerzin I \textbf{24} Gundrie] kvndrie G L (Z) (Fr45) \textbf{26} ich wæne] Jht Z  $\cdot$ des] si des I \textit{om.} L  $\cdot$ iht] ich Fr45 \textbf{27} Jofreit] Iofreit G \textbf{28} ir alten] duͦrch alde Fr45 \textbf{29} âzen] azen da I \textbf{30} mit blicken] \textit{om.} L mit blich M  $\cdot$ glander] [chander]: chlander I \newline
\end{minipage}
\hspace{0.5cm}
\begin{minipage}[t]{0.5\linewidth}
\small
\begin{center}*T
\end{center}
\begin{tabular}{rl}
 & Jofreit was wider komen.\\ 
 & von dem \textbf{het} Artus vernomen,\\ 
 & wie er werben solte,\\ 
 & \textbf{oder} er enpfâhen wolte\\ 
5 & sînen neven, den heiden.\\ 
 & daz sitzen wart \textbf{bescheiden}\\ 
 & an Gawanes ringe\\ 
 & mit \textbf{höveschlîchen dingen}.\\ 
 & diu massenîe der herzogîn\\ 
10 & und die \textbf{gesellen} under in\\ 
 & zuo Gawanes z\textit{es}wen \textbf{sâzen}.\\ 
 & anderhalp \textbf{sus} mit vreuden \textbf{âzen}\\ 
 & \textbf{die} rîter, \textbf{Clynsors} diet.\\ 
 & \textbf{den} vrouwen sitzen man beschiet.\\ 
15 & über gein Gawan an \textbf{dem} ort\\ 
 & sâzen \textbf{Clynsors} vrouwen dort;\\ 
 & der was manegiu lieht gemâl.\\ 
 & Ferefis und Parcifal\\ 
 & sâzen mitten zwischen \textbf{die} vrouwen;\\ 
20 & man mohte \textbf{d\textit{â}} clârheit schouwen.\\ 
 & der \textbf{kurtoise} Florant\\ 
 & und Seyve, diu werde erkant,\\ 
 & und der herzoge von Gauwerzin\\ 
 & und Kundrie, daz wîp sîn,\\ 
25 & über gein \textit{ein} ander sâzen.\\ 
 & ich wæne des, \textbf{iht} vergâzen\\ 
 & Gawan und Jofreit\\ 
 & ir alten gesellecheit.\\ 
 & si âzen mit ein ander.\\ 
30 & diu herzogîn mit \textbf{blicken} glander\\ 
\end{tabular}
\scriptsize
\line(1,0){75} \newline
U Q R \newline
\line(1,0){75} \newline
\textbf{1} \textit{Initiale} Q R  \textbf{29} \textit{Initiale} R  \newline
\line(1,0){75} \newline
\textbf{1} Jofreit] Iofrit R \textbf{2} het] hat R \textbf{4} oder] ob Q R \textbf{7} Gawanes] Gawins R \textbf{11} Gawanes] gawans Q Gawins R  $\cdot$ zeswen] zuͦ wen U \textbf{12} anderhalp] Aderhalb R  $\cdot$ sus] \textit{om.} Q \textbf{13} Clynsors] clinszhors Q Clinshor R \textbf{15} Gawan] gawanen Q Gawin R  $\cdot$ dem] den Q (R) \textbf{16} sâzen] Satzten Q  $\cdot$ Clynsors] clinszhors Q Clinshor R \textbf{17} lieht] licht Q \textbf{18} Ferefis] feirefisz Q Feirefis R  $\cdot$ Parcifal] partzifal Q parczifal R \textbf{19} die] den R \textbf{20} dâ] do U Q \textbf{21} kurtoise] [turkoit*]: turkoiten Q Turkoite R \textbf{22} Seyve] seyne Q Seyue R \textbf{23} Gauwerzin] Gauͦwerzin U kawerzin Q Gowerzin R \textbf{24} Kundrie] kuͦndrie U kúndrie Q  $\cdot$ daz wîp] dem wibe R \textbf{25} ein] \textit{om.} U  $\cdot$ ander] andren R \textbf{26} des iht] daz sẏ nit R \textbf{27} Jofreit] iofreit Q \textbf{28} ir alten] Jrre alte Q  $\cdot$ gesellecheit] gesellenheit R \textbf{30} blicken] kliken R \newline
\end{minipage}
\end{table}
\end{document}
