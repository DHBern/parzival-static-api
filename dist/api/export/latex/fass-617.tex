\documentclass[8pt,a4paper,notitlepage]{article}
\usepackage{fullpage}
\usepackage{ulem}
\usepackage{xltxtra}
\usepackage{datetime}
\renewcommand{\dateseparator}{.}
\dmyyyydate
\usepackage{fancyhdr}
\usepackage{ifthen}
\pagestyle{fancy}
\fancyhf{}
\renewcommand{\headrulewidth}{0pt}
\fancyfoot[L]{\ifthenelse{\value{page}=1}{\today, \currenttime{} Uhr}{}}
\begin{document}
\begin{table}[ht]
\begin{minipage}[t]{0.5\linewidth}
\small
\begin{center}*D
\end{center}
\begin{tabular}{rl}
\textbf{617} & Sît daz lît sô helfelôs,\\ 
 & den ich nâch Cidegaste \textbf{erkôs}\\ 
 & zergetzen unt durch rechen.\\ 
 & hêrre, nû hœret sprechen,\\ 
5 & wâ mit erwarp Clinschor\\ 
 & den rîchen krâm vor iwerm tor.\\ 
 & Dô der clâre Anfortas\\ 
 & minne und vreude erwendet was,\\ 
 & der mir die gâbe sande,\\ 
10 & dô \textit{v}orht ich die schande.\\ 
 & Clinschore ist stæteclîchen bî\\ 
 & der list von nigromanzî,\\ 
 & daz er mit zouber twingen kan\\ 
 & beidiu wîb und man.\\ 
15 & swaz \textbf{er} \textbf{werder diet} \textbf{gesiht},\\ 
 & die\textbf{ne} læt er âne kumber niht.\\ 
 & durch vride ich Clinschore dar\\ 
 & \textbf{gap mînen krâm} nâch rîcheite \textbf{var}.\\ 
 & swenne diu âventiure würde erliten,\\ 
20 & swer den prîs het erstriten,\\ 
 & an den solt ich \textbf{minne} suochen.\\ 
 & wolt er mîn niht geruochen,\\ 
 & der \textbf{krâm} wære an der stunde mîn.\\ 
 & der sol sus unser zweier sîn.\\ 
25 & des swuoren, die dâ wâren.\\ 
 & dâ mite ich wolde vâren\\ 
 & Gramoflanzes durch den list,\\ 
 & der leider noch \textbf{ungeendet} ist.\\ 
 & het er die âventiure geholt,\\ 
30 & sô müeser sterben hân gedolt.\\ 
\end{tabular}
\scriptsize
\line(1,0){75} \newline
D Z Fr68 \newline
\line(1,0){75} \newline
\textbf{1} \textit{Initiale} Fr68   $\cdot$ \textit{Majuskel} D  \textbf{7} \textit{Majuskel} D  \newline
\line(1,0){75} \newline
\textbf{1} lît] er lit Z erlit Fr68 \textbf{2} Cidegaste] Citegast Z citegaste Fr68 \textbf{3} durch] zv Z \textbf{4} sprechen] sprechet Z \textbf{5} Clinschor] Clinscor D Clingezor Z clinsdior Fr68 \textbf{6} krâm] \textit{om.} Z \textbf{7} Dô] Da Z  $\cdot$ Anfortas] anphortas Fr68 \textbf{10} dô vorht] do worht D Da forht Z \textbf{11} Clinschore] Clinscore D Clingezor Z \textbf{13} zouber] :::e Fr68 \textbf{17} Clinschore] Clinscore D Clingezor Z clinsdiore Fr68 \textbf{18} nâch rîcheite var] wol gevar Fr68 \textbf{19} swenne] swennen Fr68 \textbf{20} het] da hete Fr68 \textbf{22} wolt er mîn] Wold er minne Z vnde wolders Fr68 \textbf{23} so were der kram aber min Fr68 \textbf{24} zweier] beider Fr68 \textbf{25} des] daz Fr68 \textbf{27} Gramoflanzes] Gramoflantz Z \textbf{30} müeser] must er Z (Fr68) \newline
\end{minipage}
\hspace{0.5cm}
\begin{minipage}[t]{0.5\linewidth}
\small
\begin{center}*m
\end{center}
\begin{tabular}{rl}
 & sît daz \textbf{er} lît sô helfelôs,\\ 
 & den ich nâch Zidegast \textbf{erkôs}\\ 
 & zergetzen und durch rechen.\\ 
 & hêrre, nû hœret sprechen,\\ 
5 & wâ mit erwarp Clinsor\\ 
 & den rîchen krâm vor iuwerm tor.\\ 
 & dô der clâre Anfortas\\ 
 & minne und vröude er\textit{we}n\textit{d}et was,\\ 
 & der mir \textbf{den} die gâbe sande,\\ 
10 & dô vorht ich die sch\textit{an}de.\\ 
 & Clinsor ist stæteclîchen bî\\ 
 & der list von nigromanzî,\\ 
 & daz er mit zouber twingen kan\\ 
 & beidiu wîp und man.\\ 
15 & waz \textbf{er} \textbf{we\textit{r}der diet} \textbf{gesiht},\\ 
 & die lât er âne kumber niht.\\ 
 & durch vride ich Clinsor dar\\ 
 & \textbf{gap mî\textit{n}e\textit{n k}râm} nâch rîchei\textit{t} \textbf{\textit{v}ar}.\\ 
 & wen diu âventiur würde erliten,\\ 
20 & wer den prîs het erstriten,\\ 
 & an den solt ich \textbf{minne} suochen.\\ 
 & wolt er mîn niht geruochen,\\ 
 & der \textbf{krâm} wær an der stunde mîn.\\ 
 & der sol sus unser zweier sîn.\\ 
25 & des swuoren, die d\textit{â} wâren.\\ 
 & dâ mit ich wolte vâren\\ 
 & Gram\textit{o}la\textit{n}zes durch den list,\\ 
 & der \textit{l}eider noch \textbf{\textit{un}verendet} ist.\\ 
 & het er die âventiur ge\textit{h}olt,\\ 
30 & sô müest er sterben hân gedolt.\\ 
\end{tabular}
\scriptsize
\line(1,0){75} \newline
m n o \newline
\line(1,0){75} \newline
\newline
\line(1,0){75} \newline
\textbf{2} Zidegast] zitegast n \textbf{8} erwendet] ermynnet m er wenden o \textbf{9} den] \textit{om.} n o \textbf{10} schande] schmde m \textbf{12} nigromanzî] nigromazẏ m (n) ::gromaci o \textbf{13} mit] \textit{om.} n \textbf{15} werder] weder m \textbf{18} Gab mynnen bi kram no richeit dar vnd far m \textbf{20} wer] Wor o  $\cdot$ het] hat n  $\cdot$ erstriten] gestritten o \textbf{25} dâ] do m n o  $\cdot$ wâren] wen n \textbf{26} ich] \textit{om.} n \textbf{27} Gramolanzes] Gramunlaczes m Gramonlantzes n Gramonlanczes o  $\cdot$ list] lip list n o \textbf{28} Der beider noch verendet ist m \textbf{29} geholt] gedolt m \textbf{30} sô] Sú n Sús o \newline
\end{minipage}
\end{table}
\newpage
\begin{table}[ht]
\begin{minipage}[t]{0.5\linewidth}
\small
\begin{center}*G
\end{center}
\begin{tabular}{rl}
 & sît daz \textbf{er} lît sô helfelôs,\\ 
 & den ich nâch Zidegaste \textbf{erkôs}\\ 
 & zergetzen unde durch rechen.\\ 
 & hêrre, nû hœret sprechen,\\ 
5 & wâ mit erwarp Clinsor\\ 
 & den rîchen krâm vor iuwerm tor.\\ 
 & dô der clâre Anfortas\\ 
 & minne unde vröude erwendet was,\\ 
 & der mir die gâbe sande,\\ 
10 & dô vorht ich die schande.\\ 
 & Clinsor ist stæteclîch bî\\ 
 & der list von nigromanz\textit{î},\\ 
 & daz er mit zouber dwingen kan\\ 
 & beidiu wîp unde man.\\ 
15 & swaz \textbf{er} \textbf{werdecheit} \textbf{gesiht},\\ 
 & die\textbf{ne} lât er âne kumber niht.\\ 
 & durch vride ich Clinsor dar \textbf{gap mînen kranz}.\\ 
 & nâch rîcheit \textbf{würde ganz},\\ 
 & swenne diu âventiure würde er\textit{l}iten,\\ 
20 & swer den prîs het erstriten,\\ 
 & an den solt ich \textbf{helfe} suochen.\\ 
 & wolde er mîn niht geruochen,\\ 
 & der \textbf{kranz} wære an der stunde mîn.\\ 
 & der sol sus unser zweier sîn.\\ 
25 & des swuoren, die dâ wâren.\\ 
 & dâ mit ich wolde vâren\\ 
 & Gramoflanzes durch den list,\\ 
 & der leider noch \textbf{ungeendet} ist.\\ 
 & het er die âventiure geholt,\\ 
30 & sô müese er sterben hân gedolt.\\ 
\end{tabular}
\scriptsize
\line(1,0){75} \newline
G I L M Z \newline
\line(1,0){75} \newline
\textbf{1} \textit{Initiale} L  \textbf{9} \textit{Initiale} I  \newline
\line(1,0){75} \newline
\textbf{2} den] Dan M  $\cdot$ Zidegaste] zitegaste I Citegaste L zcitegaste M Citegast Z  $\cdot$ erkôs] kos I L \textbf{3} durch] zv Z \textbf{4} sprechen] sprechet Z \textbf{5} Clinsor] clinshor G Clinisor L Clingezor Z \textbf{6} den] die I  $\cdot$ krâm] \textit{om.} Z \textbf{7} dô] Da M Z  $\cdot$ Anfortas] Amfortas L \textbf{10} dô] Da L M \textbf{11} Clinsor] Chlinshor G Clinisor L Clingezor Z \textbf{12} list] ist L  $\cdot$ nigromanzî] nigromanz G \textbf{15} swaz] Waz L (M)  $\cdot$ werdecheit] werder diet Z  $\cdot$ gesiht] [chan]: gesihet G sýht L \textbf{16} diene] die I  $\cdot$ âne] an yme M \textbf{17} \textit{Vers 617.17 auf zwei Zeilen verteilt} G   $\cdot$ Clinsor] chlinshor G Clinisor L Clingezor Z  $\cdot$ dar] \textit{om.} I  $\cdot$ gap mînen kranz] \textit{om.} L M Z \textbf{18} Gap mýnen kram nach richeit var L (M) (Z) \textbf{19} swenne] Wenne L (M)  $\cdot$ erliten] erbiten G \textbf{20} swer] Wer L M \textbf{21} helfe] minne Z \textbf{22} wolde er] Wolt ir M  $\cdot$ mîn] minne Z \textbf{23} kranz] kram L (M) Z  $\cdot$ stunde] stunden I \textbf{27} Gramoflanzes] Gramoflanz L Gramorflanz M Gramoflantz Z \textbf{29} het] Hat M \textbf{30} müese] muͦse G \newline
\end{minipage}
\hspace{0.5cm}
\begin{minipage}[t]{0.5\linewidth}
\small
\begin{center}*T
\end{center}
\begin{tabular}{rl}
 & \begin{large}S\end{large}ît daz \textbf{er} liget sô helfelôs,\\ 
 & den ich nâch Cydegaste \textbf{kôs}\\ 
 & zuo ergetzene und durch rechen.\\ 
 & hêrre, nû hœret sprechen,\\ 
5 & wâ mit erwarp Clynsor\\ 
 & den rîchen krâm vor iuwerm tor.\\ 
 & dô der clâre Anfortas\\ 
 & minne und vreude erwendet was,\\ 
 & der mir die gâbe sande,\\ 
10 & dô vorht ich die schande.\\ 
 & Clynsor ist stæteclîchen bî\\ 
 & der list von nigromanzî,\\ 
 & daz er mit zouber twingen kan\\ 
 & beidiu wîp und man.\\ 
15 & waz \textbf{im} \textbf{wirdecheit} \textbf{geschiht},\\ 
 & die \textbf{en}lât er âne kumber niht.\\ 
 & durch vride ich Clynsor dar\\ 
 & \textbf{gap mînen krâm} nâch rîcheit \textbf{var}.\\ 
 & wan diu âventiure würde erliten,\\ 
20 & wer den prîs hete erstriten,\\ 
 & an de\textit{n} solt ich \textbf{minne} suochen.\\ 
 & wolt er mîn niht geruochen,\\ 
 & der \textbf{krâm} wære an der stunt mîn.\\ 
 & der sol sus unser zweier sîn.\\ 
25 & des swuoren, die d\textit{â} wâren.\\ 
 & dâ mit ich wolte vâren\\ 
 & Gramoflanzes durch den list,\\ 
 & der leider noch \textbf{ungeendet} ist.\\ 
 & h\textit{e}t er die âventiure \textit{geholt},\\ 
30 & \textit{sô müest er sterben hân} gedolt.\\ 
\end{tabular}
\scriptsize
\line(1,0){75} \newline
U V W Q R Fr39 \newline
\line(1,0){75} \newline
\textbf{1} \textit{Initiale} U V W Fr39  \newline
\line(1,0){75} \newline
\textbf{2} Cydegaste] gẏdegaste V zytegaste W Cidegast Q [Cidast]: Cidegast R Cidegast Fr39  $\cdot$ kôs] [*kos]: erkos V erkosz Q (R) \textbf{3} und] vntz Q  $\cdot$ durch] mich R \textbf{5} Clynsor] clinsor V klyshor W clinshor Q Clingshor R \textbf{8} vreude] froͤden W  $\cdot$ erwendet] entwendet W Q \textbf{10} dô] Da V  $\cdot$ die] der W \textbf{11} Clynsor] Clinsor V Klynshor W Clinshor Q R [Clinh]: Clinshor Fr39 \textbf{14} beidiu] beide Fr39 \textbf{15} waz] Swaz V (Fr39)  $\cdot$ im] er V W Q R Fr39  $\cdot$ wirdecheit] werder diet V  $\cdot$ geschiht] gesiht V (W) Fr39 [geschicht]: gesicht Q ersicht R \textbf{17} Clynsor] clinsor V klynshor W clinshor Q (R) (Fr39) \textbf{18} mînen] minem Fr39 \textbf{19} wan] Swenne V Fr39  $\cdot$ âventiure] wirdikeit R  $\cdot$ würde] wer W R \textbf{20} wer] Swer V (Fr39)  $\cdot$ hete] hott Q \textbf{21} den] dem U  $\cdot$ ich] ir W \textbf{22} er] ir R \textbf{23} krâm] \textit{om.} Fr39  $\cdot$ stunt] [stvn*]: stvnde V stunden Q \textbf{24} der] Sust R  $\cdot$ sus] als Q er R  $\cdot$ unser] vnsz R \textbf{25} dâ] do U W Q Fr39 \textbf{27} Gramoflanzes] Gramaflanzen V Gramoflantze W Zu gramoflanze Q Gramofalencze R \textbf{29} \textit{Verse 617.29-30 kontrahiert zu:} Hat der die Abenture gedolt U  \textbf{30} müest] muͦst W (Q) muͯse R \newline
\end{minipage}
\end{table}
\end{document}
