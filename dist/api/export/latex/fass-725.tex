\documentclass[8pt,a4paper,notitlepage]{article}
\usepackage{fullpage}
\usepackage{ulem}
\usepackage{xltxtra}
\usepackage{datetime}
\renewcommand{\dateseparator}{.}
\dmyyyydate
\usepackage{fancyhdr}
\usepackage{ifthen}
\pagestyle{fancy}
\fancyhf{}
\renewcommand{\headrulewidth}{0pt}
\fancyfoot[L]{\ifthenelse{\value{page}=1}{\today, \currenttime{} Uhr}{}}
\begin{document}
\begin{table}[ht]
\begin{minipage}[t]{0.5\linewidth}
\small
\begin{center}*D
\end{center}
\begin{tabular}{rl}
\textbf{725} & \begin{large}D\end{large}er künec Brandelidelin\\ 
 & \textbf{saz} zuo Ginovern, der künegîn.\\ 
 & ouch saz der künec Gramoflanz\\ 
 & zuo der, diu \textbf{ir} liehten glanz\\ 
5 & mit weinen hete begozzen.\\ 
 & \textbf{daz} hete si sîn genozzen.\\ 
 & Er \textbf{en}\textbf{welle} \textbf{unschulde} rechen,\\ 
 & \textbf{sus} muoser hin zir sprechen,\\ 
 & sîn dienst nâch minnen bieten.\\ 
10 & si kunde ouch sich des nieten,\\ 
 & daz si im dankte umbe sîn komen.\\ 
 & ir rede von niemen wart vernomen.\\ 
 & si sâhen ein ander gerne.\\ 
 & swenne ich nû \textbf{rede} gelerne,\\ 
15 & sô prüeve ich, waz si \textbf{spræchen} dâ,\\ 
 & eintweder nein oder jâ.\\ 
 & Artus ze Brandelidelin\\ 
 & sprach: "ir habt dem wîbe mîn\\ 
 & iwer mære \textbf{nû} genuoc gesagt."\\ 
20 & er vuorte den helt unverzagt\\ 
 & in ein minre gezelt\\ 
 & kurzen wec über\textbf{z} velt.\\ 
 & Gramoflanz saz stille\\ 
 & - daz was \textbf{Artuses} wille -\\ 
25 & und ander die gesellen sîn.\\ 
 & dâ gâben vrouwen \textbf{clâren} schîn,\\ 
 & daz die rîter wênec \textbf{dâ} verdrôz.\\ 
 & ir kurzwîle was sô grôz,\\ 
 & si m\textit{ö}hte ein man \textbf{noch} gerne dolen,\\ 
30 & der nâch \textbf{sorgen vreude} wolt erholen.\\ 
\end{tabular}
\scriptsize
\line(1,0){75} \newline
D \newline
\line(1,0){75} \newline
\textbf{1} \textit{Initiale} D  \textbf{7} \textit{Majuskel} D  \newline
\line(1,0){75} \newline
\textbf{24} Artuses] Artvss D \textbf{29} möhte] mohte D \newline
\end{minipage}
\hspace{0.5cm}
\begin{minipage}[t]{0.5\linewidth}
\small
\begin{center}*m
\end{center}
\begin{tabular}{rl}
 & der künic Brandelidelin\\ 
 & \textbf{saz} zuo Genovern, der künigîn.\\ 
 & ouch saz der künic Gramolantz\\ 
 & zuo der, diu liehten glanz\\ 
5 & mit weinen het \textit{b}e\textit{g}ozzen.\\ 
 & \textbf{des} het si sîn genozzen.\\ 
 & er \textbf{wolte} \textbf{unschuldige} rechen,\\ 
 & \textbf{sô} muos er hin zuo ir sprechen,\\ 
 & sîn dienst nâch minne bieten.\\ 
10 & si kunde ouch sich des nieten,\\ 
 & daz si im dankte umb sîn komen.\\ 
 & ir rede von niemen wart vernomen.\\ 
 & si sâhen ein ander gerne.\\ 
 & wan ich nû \textbf{rede} gelerne,\\ 
15 & sô prüeve ich, waz si \textbf{sprâchen} dâ,\\ 
 & eintweder \textit{n}ein oder jâ.\\ 
 & \begin{large}A\end{large}rtus zuo Brandelidelin\\ 
 & sprach: "ir habt dem wîbe mîn\\ 
 & iuwer mær \textbf{nû} genuoc gesaget."\\ 
20 & er vuorte den helt unverzaget\\ 
 & in ein minner gezelt\\ 
 & kurzen wec über \textbf{daz} velt.\\ 
 & Gramolantz saz stille\\ 
 & - daz was \textbf{ouch} \textbf{wol} \textbf{sîn} wille -\\ 
25 & und \textbf{al} ander die gesellen sîn.\\ 
 & d\textit{â} gâben vrowen \textbf{sô} \textbf{clâren} schîn,\\ 
 & daz die ritter wênic \textbf{d\textit{â}} verdrôz.\\ 
 & ir kurzewîle was sô grôz,\\ 
 & si m\textit{ö}hte ein man \textbf{noch} gerne doln,\\ 
30 & der nâch \textbf{sorge}  wolte erholn.\\ 
\end{tabular}
\scriptsize
\line(1,0){75} \newline
m n o \newline
\line(1,0){75} \newline
\textbf{17} \textit{Initiale} m   $\cdot$ \textit{Capitulumzeichen} n  \newline
\line(1,0){75} \newline
\textbf{1} Brandelidelin] brandeledelin o \textbf{2} saz] Das n  $\cdot$ Genovern] ginofer o \textbf{3} künic] konigin o  $\cdot$ Gramolantz] gramolancz o \textbf{4} diu] die iren n ir o \textbf{5} begozzen] genossen m \textbf{6} des] Das o \textbf{7} unschuldige] vnschulde n o \textbf{10} si] So o  $\cdot$ nieten] genieten n \textbf{14} wan] Wenne n \textbf{15} dâ] do n \textbf{16} nein] mein m \textbf{18} wîbe] libe o \textbf{23} Gramolantz] Gramolancz o \textbf{26} dâ] Do m n o \textbf{27} dâ] do m n o \textbf{29} möhte] mohtte m (o) \textbf{30} nâch] noch m n o \newline
\end{minipage}
\end{table}
\newpage
\begin{table}[ht]
\begin{minipage}[t]{0.5\linewidth}
\small
\begin{center}*G
\end{center}
\begin{tabular}{rl}
 & \textbf{dâ saz} der künec Brandelidelin\\ 
 & zuo Schinovern, der künigîn.\\ 
 & ouch saz der künec Gramoflanz\\ 
 & zuo der, diu \textbf{ir} liehten glanz\\ 
5 & mit weinen hete begozzen.\\ 
 & \textbf{daz} hete si sîn genozzen.\\ 
 & er\textbf{ne} \textbf{welle} \textbf{unschulde} rechen,\\ 
 & \textbf{sus} muoser hin zir sprechen,\\ 
 & sîn dienst nâch minne bieten.\\ 
10 & si kunde ouch sich des nieten,\\ 
 & daz si im dankte umbe sîn komen.\\ 
 & ir rede von niemen wart vernomen.\\ 
 & si sâhen ein ander gerne.\\ 
 & swenne ich nû \textbf{reden} gelerne,\\ 
15 & sô prüeve ich, swaz si \textbf{sprâchen} dâ,\\ 
 & einweder nein oder jâ.\\ 
 & \begin{large}A\end{large}rtus ze Brandelidelin\\ 
 & sprach: "ir habet dem wîbe mîn\\ 
 & iwer mære genuoc gesaget."\\ 
20 & er vuorte den helt unverzaget\\ 
 & in ein minner gezelt\\ 
 & kurzen wec über velt.\\ 
 & Gramoflanz saz stille\\ 
 & - daz was \textbf{Artuses} wille -\\ 
25 & unde ander die gesellen sîn.\\ 
 & dâ gâben vrouwen \textbf{liehten} schîn,\\ 
 & daz die rîter wênec \textbf{bî im} verdrôz.\\ 
 & ir kurzewîle was sô grôz,\\ 
 & si m\textit{ö}hte ein man gerne dolen,\\ 
30 & der nâch \textbf{vröude, sorge} wolde erholen.\\ 
\end{tabular}
\scriptsize
\line(1,0){75} \newline
G I L M Z Fr20 Fr24 \newline
\line(1,0){75} \newline
\textbf{1} \textit{Initiale} Z  \textbf{9} \textit{Initiale} I  \textbf{17} \textit{Initiale} G L Z Fr20  \textbf{27} \textit{Initiale} I  \newline
\line(1,0){75} \newline
\textbf{1} dâ] do I (L)  $\cdot$ Brandelidelin] brandalidelin I Branlidelin L bran::: Fr20 \textbf{2} Schinovern] Ginovern G (M) Ginofern I Gynovern L [der]: gynovern Z chvnnewarin \textit{nachträglich korrigiert zu:} schenoveren Fr20  $\cdot$ der] die L M \textbf{3} Gramoflanz] Gramoflanc I gramoflantz Z gra::: Fr20 \textbf{4} der diu ir] ir die den L  $\cdot$ liehten] lichten L M  $\cdot$ glanz] kranz L \textbf{6} daz] des I \textbf{7} erne welle] Her wolde M  $\cdot$ unschulde] unschude Fr20 \textbf{9} \textit{Die Verse 725.9-14 fehlen} Fr20   $\cdot$ minne] mynnen L \textbf{11} dankte] [danken]: danket Z \textbf{13} ein ander] einander an I \textbf{14} swenne] Wenne L (M)  $\cdot$ reden] rede L M \textbf{15} so froͮwe ich [w*]: weiz sie sp::: Fr20  $\cdot$ sô] si I  $\cdot$ ich] \textit{om.} Z  $\cdot$ sprâchen] spehen Z \textbf{17} Artus] ÷rtus Fr20  $\cdot$ brandelidelin] brandalidelin I Branlidelin L brandelide::: Fr20 \textbf{18} sprach] \textit{om.} Fr20 \textbf{19} genuoc] nv gnvch L (M) (Z) \textbf{21} minner] ander I \textbf{22} kurzen] einen churzen I \textbf{23} Gramoflanz] Gramoflantz Z \textbf{24} Artuses] Artus G (Z) Artuͯses L artusis Fr20 \textbf{25} die] der L \textbf{26} dâ gâben] den Gaben die I  $\cdot$ liehten] lichten L (M) \textbf{27} daz] \textit{om.} L  $\cdot$ im] in I L (M) Z Fr24  $\cdot$ verdrôz] erdroz L bedroz Fr24 \textbf{28} ir] wande ir I \textbf{29} möhte] mohte G (I) (L) (Z) (Fr20) (Fr24) mochten M \textbf{30} nâch] sich nach L  $\cdot$ vröude sorge] freuden sorge I sorge frovde L (M) frevden sorgen Z frevde sorgen Fr24  $\cdot$ erholen] holn I \newline
\end{minipage}
\hspace{0.5cm}
\begin{minipage}[t]{0.5\linewidth}
\small
\begin{center}*T
\end{center}
\begin{tabular}{rl}
 & \textbf{\begin{large}D\end{large}ô saz} der künec Brandelidelin\\ 
 & zuo Gynovern, der künegîn.\\ 
 & ouch saz der künec Gramoflanz\\ 
 & zuo der, diu \textbf{ir} liehten \textit{g}lanz\\ 
5 & mit weinen hete begozzen.\\ 
 & \textbf{des} hete si sîn genozzen.\\ 
 & er \textbf{en}\textbf{welle} \textbf{unschulde} rechen,\\ 
 & \textbf{sus} muos er hin zuo ir sprechen,\\ 
 & sîn dienst nâch minne bieten.\\ 
10 & si kunde ouch sich des nieten,\\ 
 & daz si im dankete umb sîn komen.\\ 
 & ir rede von nieman wart vernomen.\\ 
 & si sâhen ein ander gerne.\\ 
 & wan ich nû \textbf{rede} gelerne,\\ 
15 & sô prüeve ich, waz si \textbf{sprâchen} dâ,\\ 
 & einweder nein oder jâ.\\ 
 & Artus zuo Brandelidelin\\ 
 & sprach: "ir hât dem wîbe mîn\\ 
 & iuwer mære genuoc gesaget."\\ 
20 & er vuorte den helt unverzaget\\ 
 & in ei\textit{n} m\textit{in}re gezelt\\ 
 & kurzen wec über \textbf{daz} velt.\\ 
 & Gramoflanz saz stille\\ 
 & - daz was \textbf{Artuses} wille -\\ 
25 & und ander die gesellen sîn.\\ 
 & d\textit{â} gâben vrouwen \textbf{liehten} schîn,\\ 
 & daz die rîter wênic \textbf{bî im} verdrôz.\\ 
 & ir kurzewîle was sô grôz,\\ 
 & si möhte ein man gerne doln,\\ 
30 & d\textit{e}r nâch \textbf{sorge vreude} wolt erholn.\\ 
\end{tabular}
\scriptsize
\line(1,0){75} \newline
U V W Q R \newline
\line(1,0){75} \newline
\textbf{1} \textit{Initiale} U V  \textbf{3} \textit{Initiale} W  \textbf{17} \textit{Initiale} Q  \newline
\line(1,0){75} \newline
\textbf{1} Brandelidelin] brandlidelin Q \textbf{2} Gynovern] tschinouern W ginouer Q \textbf{3} ouch] dOch W  $\cdot$ künec] kúni W  $\cdot$ Gramoflanz] Gramaflanz V gramoflantz W Q Gramoflancz R \textbf{4} liehten] lichten Q  $\cdot$ glanz] Gramoflanz U \textbf{6} hete si sîn] sie sein hett Q \textbf{7} unschulde] vnschuldig R \textbf{8} sus muos] [S* muͤ*]: So muͤs V Als musz Q  $\cdot$ hin] \textit{om.} R \textbf{11} si im] im W vm Q  $\cdot$ dankete] dancket W Q (R) \textbf{14} wan] Swen V  $\cdot$ rede] reden R \textbf{15} si sprâchen dâ] sv́ sprachent do V (W) (Q) ich sprechen kan R \textbf{17} Brandelidelin] brandlidelein Q Brandeleslin R \textbf{19} genuoc] nuͦ genuͦg W (Q) (R) \textbf{21} ein minre] eine muͦre U \textbf{22} daz] \textit{om.} R \textbf{23} Gramoflanz] Gramaflanz V Gramoflantz Q Gramofrancz R  $\cdot$ stille] alle stille R \textbf{24} Daz [*]: waz artus wille V  $\cdot$ Artuses] artus W (R) artusesz Q \textbf{25} ander die] [*]: ander die V \textbf{26} dâ] Do U V W Q R  $\cdot$ vrouwen liehten] frawe lichten Q \textbf{27} im] den frowen V in W Q R \textbf{29} möhte] mochte W mochtten R \textbf{30} der] Dar U  $\cdot$ sorge vreude] sorgen vroͮede V froͯde sorgen R  $\cdot$ erholn] holn W \newline
\end{minipage}
\end{table}
\end{document}
