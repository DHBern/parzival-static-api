\documentclass[8pt,a4paper,notitlepage]{article}
\usepackage{fullpage}
\usepackage{ulem}
\usepackage{xltxtra}
\usepackage{datetime}
\renewcommand{\dateseparator}{.}
\dmyyyydate
\usepackage{fancyhdr}
\usepackage{ifthen}
\pagestyle{fancy}
\fancyhf{}
\renewcommand{\headrulewidth}{0pt}
\fancyfoot[L]{\ifthenelse{\value{page}=1}{\today, \currenttime{} Uhr}{}}
\begin{document}
\begin{table}[ht]
\begin{minipage}[t]{0.5\linewidth}
\small
\begin{center}*D
\end{center}
\begin{tabular}{rl}
\textbf{95} & mir gap diu gehiure\\ 
 & von \textbf{dem} lande die besten stiure.\\ 
 & ich was dô ermer denne nuo.\\ 
 & dâ greif ich willeclîchen zuo.\\ 
5 & zelt mich \textbf{noch} vür \textbf{die} armen.\\ 
 & ich solt iuch, \textbf{vrouwe}, erbarmen.\\ 
 & mir ist mîn \textbf{werder} bruoder tôt.\\ 
 & durch iwer zuht lât mich âne nôt.\\ 
 & kêrt minne, dâ \textbf{diu} vreude sî,\\ 
10 & \textbf{wan} mir wont \textbf{niwan} \textbf{jâmer} bî."\\ 
 & "\textit{\begin{large}L\end{large}}ât mich den lîp niht langer zern.\\ 
 & sagt an, \textbf{wâ mite} welt ir iuch \textbf{wern}?"\\ 
 & "ich sage nâch iwerre vrâge ger.\\ 
 & ez \textbf{wart} ein turnei dâ her\\ 
15 & gesprochen, des enwart hie niht.\\ 
 & manec geziuc mir des giht."\\ 
 & "den hât ein vesperîe erlemt.\\ 
 & \textbf{die vrechen sîn sô hie} gezemt,\\ 
 & daz der turnei dâr von verdarp."\\ 
20 & "iwerre stete wer ich warp\\ 
 & mit den, die\textbf{z} \textbf{guot} \textbf{hie} hânt getân.\\ 
 & ir sult mich nôtrede erlân.\\ 
 & ez tet hie manec ritter baz.\\ 
 & iwer reht ist gein mir laz.\\ 
25 & niwan iwer gemeiner gruoz,\\ 
 & ob ich den von iu \textbf{haben} muoz."\\ 
 & \textbf{Als mir} diu âventiure sagt,\\ 
 & dô nam der ritter unt diu magt\\ 
 & einen rihtære über \textbf{der vrouwen} klage.\\ 
30 & dô nâhet ez dem \textbf{mitten tage}.\\ 
\end{tabular}
\scriptsize
\line(1,0){75} \newline
D \newline
\line(1,0){75} \newline
\textbf{11} \textit{Initiale} D  \textbf{27} \textit{Majuskel} D  \newline
\line(1,0){75} \newline
\textbf{11} Lât] ÷at D \newline
\end{minipage}
\hspace{0.5cm}
\begin{minipage}[t]{0.5\linewidth}
\small
\begin{center}*m
\end{center}
\begin{tabular}{rl}
 & mir gap d\textit{iu} gehiure\\ 
 & von \textbf{\textit{irm}e} lande die beste stiure.\\ 
 & ich was dô armer denne nû.\\ 
 & dô greif ich williclîchen zuo.\\ 
5 & \textbf{man} zelt mich \textbf{noch} vür armen.\\ 
 & ich solte iuch, \textbf{vrouwe}, erbarmen.\\ 
 & mir ist mîn bruoder tôt.\\ 
 & durch iuwer zuht lât mich âne nôt.\\ 
 & kêret minne, d\textit{â} vröude sî,\\ 
10 & \textbf{wand} mir wonet \textbf{\textit{niht} wan} \textbf{jâmer} bî."\\ 
 & "lât mich den lîp niht langer zern.\\ 
 & saget an, \textbf{wâ mite} wellet ir iuch \textbf{wern}?"\\ 
 & "ich sage nâch i\textit{uwe}rre vrâge ger.\\ 
 & ez \textbf{was} ein turnei dâ her\\ 
15 & gesprochen, des enwart hie niht.\\ 
 & manic \dag getwerc\dag  mir des giht."\\ 
 & "den hât ein vesperîe erlemet.\\ 
 & \textbf{die vrechen sint hie} gezemet,\\ 
 & daz  turnei dâ von verdarp."\\ 
20 & "i\textit{uwe}rre stete wer ich warp\\ 
 & mit den, die \textbf{daz} \textbf{guot} hânt getân.\\ 
 & ir sullet mich nôtrede erlân.\\ 
 & ez tet hie manic ritter baz.\\ 
 & iuwer reht ist gegen mir laz.\\ 
25 & niuwen iuwe\textit{r} gemeiner gruoz,\\ 
 & ob ich den von iu \textbf{enpfâhen} muoz."\\ 
 & \textbf{\begin{large}A\end{large}ls mir} diu âventiure saget,\\ 
 & dô nam der ritter und diu maget\\ 
 & einen rihtære über \textbf{der vrowen} klage.\\ 
30 & dô nâhete ez dem \textbf{mittage}.\\ 
\end{tabular}
\scriptsize
\line(1,0){75} \newline
m n o \newline
\line(1,0){75} \newline
\textbf{27} \textit{Initiale} m n  \newline
\line(1,0){75} \newline
\textbf{1} \textit{Die Verse 95.1-2 fehlen} n   $\cdot$ diu] der m \textbf{2} irme] uͯwe m \textbf{3} armer] armen o  $\cdot$ nû] wo n o \textbf{5} zelt] zalt n  $\cdot$ armen] armer o \textbf{7} bruoder] werder bruͦder n [werden]: werder bruͦder o \textbf{8} nôt] [mot]: not o \textbf{9} kêret] Leret n o  $\cdot$ dâ] do m n o \textbf{10} niht] im m  $\cdot$ wan] dan o \textbf{12} saget] Sage n \textbf{13} iuwerre] ire m ir n o  $\cdot$ vrâge] froͯger o \textbf{14} was] wer o \textbf{17} hât] hette o \textbf{18} gezemet] so gezemet n o \textbf{19} von] \textit{om.} n \textbf{20} iuwerre] Jre m Jr n o \textbf{21} hânt] hie hant n han o \textbf{25} iuwer] uwern m \textbf{29} rihtære] [rittere]: rihtere m \textbf{30} nâhete] nohet n (o)  $\cdot$ mittage] mittem tage n mitten tage o \newline
\end{minipage}
\end{table}
\newpage
\begin{table}[ht]
\begin{minipage}[t]{0.5\linewidth}
\small
\begin{center}*G
\end{center}
\begin{tabular}{rl}
 & mir gap diu gehiure\\ 
 & von \textbf{dem} lande die besten stiure.\\ 
 & \hspace*{-.7em}\big| dô greif ich williclîchen zuo.\\ 
 & \hspace*{-.7em}\big| ich was dô ermer dane nû.\\ 
5 & \begin{large}Z\end{large}elt mich vür \textbf{die} armen.\\ 
 & ich solt iuch, \textbf{vrouwe}, erbarmen.\\ 
 & mir ist mîn \textbf{werder} bruoder tôt.\\ 
 & durch iwer zuht lât mich ân nôt.\\ 
 & kêrt minne, dâ \textbf{diu} vröude sî.\\ 
10 & mir \textit{won}t \textbf{niht wan} \textbf{kumber} bî."\\ 
 & "lât mich den lîp niht langer zern.\\ 
 & saget an, \textbf{wie} welt ir\textbf{s} iuch \textbf{erwern}?"\\ 
 & "ich sage nâch iwer vrâge ger.\\ 
 & ez \textbf{wart} ein turnei dâ her\\ 
15 & gesprochen, desne wart hie niht.\\ 
 & manic geziuc mir des giht."\\ 
 & "den hât ein vesperîe erlemet.\\ 
 & \textbf{hie sint die vrechen sô} gezemet,\\ 
 & daz der turnei dâr von verdarp."\\ 
20 & "iwere stete wer ich warp\\ 
 & mit den, die\textbf{z} \textbf{guot} \textbf{hie} hânt getân.\\ 
 & ir sult mich nôtrede erlân.\\ 
 & ez tet hie manic rîter baz.\\ 
 & iwer reht ist gein mi\textit{r} \textit{l}az.\\ 
25 & niwan iwer gemeiner gruoz,\\ 
 & obe ich den von iu \textbf{\textit{entv}â\textit{h}en} muoz."\\ 
 & \textbf{als uns} diu âventiure saget,\\ 
 & dô nam der rîter und diu maget\\ 
 & einen rihtære über \textbf{ir beider} klage.\\ 
30 & dô nâhet ez dem \textbf{mitten tage}.\\ 
\end{tabular}
\scriptsize
\line(1,0){75} \newline
G I O L M Q R Z Fr36 Fr56 \newline
\line(1,0){75} \newline
\textbf{1} \textit{Initiale} O  \textbf{5} \textit{Initiale} G  \textbf{11} \textit{Initiale} I L R Z  \textbf{27} \textit{Capitulumzeichen} L  \newline
\line(1,0){75} \newline
\textbf{1} mir] ÷ir O \textbf{2} dem] \textit{om.} I Q \textbf{4} dô] Da Z \textbf{3} dô] da M Z  $\cdot$ nû] tuͦ R \textbf{5} Zelt] Man zelt Z  $\cdot$ mich] mich nu I mich noch L Q R mich doch Z \textbf{6} solt] sol I Q \textbf{7} werder] werde I \textbf{8} lât] langit M lach Fr56  $\cdot$ ân] en O \textbf{9} \textit{Die Verse 95.9-10 fehlen} Z   $\cdot$ dâ] do O Q  $\cdot$ diu] \textit{om.} O Q R \textbf{10} wont] ist G  $\cdot$ wan] dan L M von Q  $\cdot$ kumber] iamer O L M (Q) (R) Fr56 \textbf{12} wie welt irs iuch] wie woltet ir evch I wa mit ir ivch welt O (L) (Q) (Z) wy wolde uch mir M wa mit [wo*]: wolt ir uch R  $\cdot$ erwern] wern O L M (Q) R Z \textbf{14} wart] was R \textbf{15} desne wart] das enwart R  $\cdot$ hie] \textit{om.} I \textbf{16} des] daz R \textbf{17} \textit{Versdoppelung 95.17-22 nach 95.22} O   $\cdot$ ein] diu I  $\cdot$ erlemet] so gelemt I \textbf{18} hie sint] daz I  $\cdot$ vrechen] vromden L (M)  $\cdot$ sô] sint I also M \textbf{19} der] \textit{om.} M \textbf{20} iwere] Wer R  $\cdot$ stete] tat M  $\cdot$ ich] úch R \textbf{21} diez] ditz Z  $\cdot$ hie] \textit{om.} I O L M Q R \textbf{22} nôtrede] ane rede M leydes Q  $\cdot$ erlân] lan M (R) \textbf{23} ez] Er Z \textbf{24} mir laz] mir vil laz G \textbf{26} iu] ir R  $\cdot$ entvâhen] haben G \textbf{27} als uns] Alsus M \textbf{28} dô] Da M  $\cdot$ nam] man R  $\cdot$ der] dy M  $\cdot$ und diu maget] vnverzagt I (M) \textbf{29} rihtære] ritter Q  $\cdot$ über] vnb I (R)  $\cdot$ ir beider] >der< frowen O (L) (M) (R) Z der frawe Q \textbf{30} dô] Da O M  $\cdot$ nâhet] nahent I nahete M  $\cdot$ dem] den Z  $\cdot$ mitten tage] mittem tag I (O) (L) (Z) (Fr36) mittage Q \newline
\end{minipage}
\hspace{0.5cm}
\begin{minipage}[t]{0.5\linewidth}
\small
\begin{center}*T (U)
\end{center}
\begin{tabular}{rl}
 & mir gap diu gehiure\\ 
 & von\textbf{me} lande die beste stiure.\\ 
 & \hspace*{-.7em}\big| dô greif ich willeclîchen zuo.\\ 
 & \hspace*{-.7em}\big| ich was dô ermere danne nuo.\\ 
5 & \textbf{nû} zelt mich \textbf{noch} vür \textbf{die} armen.\\ 
 & ich solt iuch \textbf{noch} erbarmen.\\ 
 & mir ist mîn \textbf{werder} bruoder tôt.\\ 
 & durch iuwer zuht lât mich âne nôt.\\ 
 & kêrt minne, d\textit{â} \textbf{diu} vreude sî.\\ 
10 & mir wonet \textbf{niht dan} \textbf{jâmer} bî."\\ 
 & "\begin{large}L\end{large}ât mich den lîp niht langer zern.\\ 
 & saget an, \textbf{wâ mit} welt ir i\textit{uch} \textbf{wern}?"\\ 
 & "\textbf{vrouwe}, ich sage \textbf{iu} nâch iuwerre vrâge ger.\\ 
 & ez \textbf{wart} ein turnei dâ her\\ 
15 & gesprochen, d\textit{e}s enwart hie niht.\\ 
 & manec geziuc mir des giht."\\ 
 & "den hât ein vesperîe erlemet.\\ 
 & \textbf{hie sint die vrechen} gezemet,\\ 
 & daz der turnei dâr von verdarp."\\ 
20 & "iuwer stete wer ich warp\\ 
 & mit den, die \textbf{ez} \textbf{wol} hânt getân.\\ 
 & ir solt mich nôtrede erlân.\\ 
 & ez tet hie manec ritter baz.\\ 
 & iuwer reht ist gein mir \textbf{zuo} laz.\\ 
25 & niht wan iuwer gemeiner gruoz,\\ 
 & ob ich den von iu \textbf{entvâhen} muoz."\\ 
 & \textbf{alsus} diu âventiure saget,\\ 
 & dô \textit{n}a\textit{m} der ritter und diu maget\\ 
 & einen ri\textit{h}ter über \textbf{der vrouwen} klage.\\ 
30 & dô nâhtez dem \textbf{mitten tage}.\\ 
\end{tabular}
\scriptsize
\line(1,0){75} \newline
U V W T \newline
\line(1,0){75} \newline
\textbf{6} \textit{Majuskel} T  \textbf{11} \textit{Initiale} U V W   $\cdot$ \textit{Majuskel} T  \textbf{27} \textit{Initiale} T  \newline
\line(1,0){75} \newline
\textbf{1} diu] der W \textbf{2} beste] besten W T \textbf{4} dô] da T \textbf{3} \textit{Versfolge 95.4-6-3-5} W   $\cdot$ nuo] ich nun sei W \textbf{5} Doch bin ich noch den armen bei W  $\cdot$ noch] doch V \textit{om.} T \textbf{6} Ich meine die mir das beste thuͦ W  $\cdot$ Lât mich iv̂ vrouwe erbarmeN T  $\cdot$ noch] [*]: vrowe V \textbf{7} Darzuͦ ist mein bruͦder laider tot W \textbf{9} minne] uwer minne V (T)  $\cdot$ dâ] do U V W  $\cdot$ diu] \textit{om.} W T \textbf{10} dan] wan V (W) T \textbf{12} an] \textit{om.} W  $\cdot$ welt ir iuch wern] wolt ir inwern U ir eúch woͤlt wern W \textbf{13} vrouwe] \textit{om.} T  $\cdot$ iu] \textit{om.} T  $\cdot$ vrâge] \textit{om.} W \textbf{15} des enwart] disin wart U des ward W \textbf{16} des] das W \textbf{18} gezemet] so [gezem*]: gezemet V so gezemet W (T) \textbf{20} iûwer state iû lop er warp T  $\cdot$ ich warp] eúch erwarb W \textbf{21} den die] des W  $\cdot$ wol] guͦt W (T) \textbf{24} ist] ist hie W  $\cdot$ zuo] \textit{om.} W T \textbf{25} gemeiner] gemainen W \textbf{27} alsus] Als vns V (W) T \textbf{28} nam] man U \textbf{29} rihter] ritter U  $\cdot$ über der vrouwen] vmb die W \textbf{30} nâhtez] nohet es V (T)  $\cdot$ mitten] mittem T \newline
\end{minipage}
\end{table}
\end{document}
