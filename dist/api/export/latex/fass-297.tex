\documentclass[8pt,a4paper,notitlepage]{article}
\usepackage{fullpage}
\usepackage{ulem}
\usepackage{xltxtra}
\usepackage{datetime}
\renewcommand{\dateseparator}{.}
\dmyyyydate
\usepackage{fancyhdr}
\usepackage{ifthen}
\pagestyle{fancy}
\fancyhf{}
\renewcommand{\headrulewidth}{0pt}
\fancyfoot[L]{\ifthenelse{\value{page}=1}{\today, \currenttime{} Uhr}{}}
\begin{document}
\begin{table}[ht]
\begin{minipage}[t]{0.5\linewidth}
\small
\begin{center}*D
\end{center}
\begin{tabular}{rl}
\textbf{297} & an swem di\textit{u} kurtôsîe\\ 
 & unt diu werde kumpânîe\\ 
 & lac, den kunder êren,\\ 
 & sîn dienst gein im kêren.\\ 
5 & Ich gihe von im der mære,\\ 
 & er was ein merkære.\\ 
 & \textbf{er} tet vil rûhes willen schîn\\ 
 & ze scherme dem hêrren sîn.\\ 
 & Partierre unt valsche diet,\\ 
10 & von den werden er die schiet.\\ 
 & er was ir vuore ein strenger hagel,\\ 
 & noch scharpfer dan \textbf{der} bîn ir zagel.\\ 
 & seht, die verkêrten Keien prîs.\\ 
 & \textbf{der} was manlîcher triwen wîs.\\ 
15 & vil \textbf{hazzes} er von in gewan.\\ 
 & Von Duringen \textbf{vürste} Herman,\\ 
 & etslîch dîn ingesinde ich maz,\\ 
 & daz 'ûzgesinde' hieze baz.\\ 
 & dir wære ouch eines Keien nôt,\\ 
20 & sît wâriu milte dir gebôt\\ 
 & sô manecvalten anehanc,\\ 
 & etswâ smæhlîch gedranc\\ 
 & unt etswâ werdez dringen.\\ 
 & des muoz hêr Walther singen:\\ 
25 & "guoten tac, bœse unde guot."\\ 
 & swâ man solhen sanc nû tuot,\\ 
 & des sint die valschen geêret.\\ 
 & Keie hets in niht gelêret\\ 
 & noch hêr Heinrich von Rispach.\\ 
30 & hœret wunders mêr, waz dort geschach\\ 
\end{tabular}
\scriptsize
\line(1,0){75} \newline
D \newline
\line(1,0){75} \newline
\textbf{5} \textit{Majuskel} D  \textbf{9} \textit{Majuskel} D  \textbf{16} \textit{Majuskel} D  \newline
\line(1,0){75} \newline
\textbf{1} diu] die D \textbf{29} Rispach] Rîspach D \newline
\end{minipage}
\hspace{0.5cm}
\begin{minipage}[t]{0.5\linewidth}
\small
\begin{center}*m
\end{center}
\begin{tabular}{rl}
 & an wem diu kurtois\textit{î}e\\ 
 & und diu werd\textit{e} k\textit{o}mpânîe\\ 
 & lac, den \textit{k}under êren,\\ 
 & sîn \textit{dienst} gegen ime kêren.\\ 
5 & ich gihe von ime der mære,\\ 
 & er was ein merkære\\ 
 & \textbf{und} tet vil rûh\textit{e}s willen schîn\\ 
 & ze scherme dem hêrren sîn.\\ 
 & par\textit{tie}rre und valsche diet,\\ 
10 & von den werden er die schiet.\\ 
 & er was ir vuor ein strenger hagel,\\ 
 & noch scharpfer danne \textbf{der} bîn ir zagel.\\ 
 & seht, die verkêrten Keien prîs.\\ 
 & \textbf{der} was manlîcher triuwen wîs.\\ 
15 & vil \textbf{hazzens} er von in gewan.\\ 
 & von Duringen \textbf{vürste} Herman,\\ 
 & etslîch dîn \textit{i}ngesinde ich maz,\\ 
 & daz 'ûzgesinde' hieze baz.\\ 
 & dir wære ouch eines Keien nôt,\\ 
20 & sît wâriu milte \textit{d}ir gebôt\\ 
 & sô manicvalten anehanc,\\ 
 & etswâ s\textit{m}æhelîch gedranc\\ 
 & und etwâ werd\textit{e}z dringen.\\ 
 & des muoz hêr Walther singen:\\ 
25 & "guoten tac, bœse und guot."\\ 
 & wâ man solichen sanc nû tuot,\\ 
 & des sint die valschen geêret.\\ 
 & Keie hets in niht gelêret\\ 
 & noch hêr Heinrich von Ris\textit{p}ach.\\ 
30 & hœrt wunders mêr, waz dort geschach\\ 
\end{tabular}
\scriptsize
\line(1,0){75} \newline
m n o Fr69 \newline
\line(1,0){75} \newline
\newline
\line(1,0){75} \newline
\textbf{1} kurtoisîe] kurtoise m \textbf{2} werde] werden m  $\cdot$ kompânîe] campanie m camponie n o \textbf{3} kunder] tunder m \textbf{4} dienst] \textit{om.} m \textbf{6} merkære] me merckherre o \textbf{7} rûhes] ruchens m \textbf{9} partierre] Parcifare m \textbf{11} ir] \textit{om.} n \textbf{12} bîn] by n \textbf{13} die] sie o  $\cdot$ Keien] keyen n keẏen o \textbf{14} manlîcher] maniger o \textbf{15} hazzens] hasses n (o)  $\cdot$ er] es o  $\cdot$ in] ẏme o \textbf{16} Duringen] dúringen n doringen o  $\cdot$ vürste] fursten o  $\cdot$ Herman] horrman o \textbf{17} ingesinde] eingesinde m  $\cdot$ maz] was n (o) \textbf{19} eines] keines o  $\cdot$ Keien] keẏen n o \textbf{20} dir] ir m \textbf{21} manicvalten anehanc] manig faltig one [fang]: hang o \textbf{22} etswâ] Etwenne n (o)  $\cdot$ smæhelîch] sinehelich m sin helich n o  $\cdot$ gedranc] gedang n getrang o \textbf{23} werdez] werders m \textbf{26} sanc] dang n \textbf{28} Keie] Keẏ n  $\cdot$ hets in] het sin m n \textbf{29} \textit{Verse 297.24-29 kontrahiert zu:} Dez muͯs her heinrich von rispach o   $\cdot$ Rispach] rissprach m riszbach n \textbf{30} wunders] wunder n (o)  $\cdot$ geschach] beschach n o \newline
\end{minipage}
\end{table}
\newpage
\begin{table}[ht]
\begin{minipage}[t]{0.5\linewidth}
\small
\begin{center}*G
\end{center}
\begin{tabular}{rl}
 & an swem diu kurtôsîe\\ 
 & unde diu werde kumpânîe\\ 
 & lac, den kunder êren,\\ 
 & sîn dienst gein im kêren.\\ 
5 & ich gihe von im der mære,\\ 
 & er was ein merkære.\\ 
 & \textbf{er} tet vil rûhes willen schîn\\ 
 & ze scherme dem hêrren sîn.\\ 
 & partierære unde valsche diet,\\ 
10 & von den werden er die schiet.\\ 
 & er was ir vuore ein strenge hagel,\\ 
 & noch scherpfer danne \textbf{ein} bîn, ir zagel.\\ 
 & seht, die verkêrten \textit{Kayn} prîs.\\ 
 & \textbf{er} was manlîcher triwen wîs.\\ 
15 & vil \textbf{hazzes} er von in gewan.\\ 
 & von Durngen \textbf{vürste} Herman,\\ 
 & etlîch dîn ingesinde ich maz,\\ 
 & daz 'ûzgesinde' hieze baz.\\ 
 & dir wære ouch eines Kayn nôt,\\ 
20 & sît wâriu milte dir gebôt\\ 
 & sô manicvalten anehanc,\\ 
 & etswâ smæhlîch gedranc\\ 
 & unt etswâ werdez dringen.\\ 
 & des muoz hêr Walther singen:\\ 
25 & "guoten tac, bœse unde guot."\\ 
 & swâ man solchen sanc nû tuot,\\ 
 & des sint die valschen geêret.\\ 
 & Kay hetes in niht gelêret\\ 
 & noch hêr Heinrich von Rispach.\\ 
30 & hœrt wunders mê, waz dort geschach\\ 
\end{tabular}
\scriptsize
\line(1,0){75} \newline
G I O L M Q R Z \newline
\line(1,0){75} \newline
\textbf{1} \textit{Initiale} I  \textbf{5} \textit{Initiale} L Z  \textbf{9} \textit{Initiale} O Q  \textbf{21} \textit{Initiale} I  \textbf{25} \textit{Initiale} M  \newline
\line(1,0){75} \newline
\textbf{1} swem] wem L (M) Q wenn R \textbf{2} werde] werder L  $\cdot$ kumpânîe] kapanie I \textbf{3} êren] erren R \textbf{4} \textit{Vers 297.4 fehlt} Q   $\cdot$ sîn] Sinen L [Siner]: Sinen Z \textbf{5} gihe] se M vergich R [gibe]: gihe Z  $\cdot$ von] \textit{om.} R  $\cdot$ der] \textit{om.} I \textbf{7} rûhes] ruhen I tuhes M riches R \textbf{8} dem] den L \textbf{9} partierære] ÷Artîrer O Paratierre L Garrirre Q \textbf{11} hagel] bagil M \textbf{12} Noch sie arpher wan der irczagil M  $\cdot$ ein bîn ir] ein pigen I der pîn der O derbý ir L (R) der bin ir Q Z \textbf{13} die] den L dei M  $\cdot$ Kayn] sinen G chain I keyn O M kayen L key Q keyen R Z \textbf{14} er] Es Q \textbf{15} in] im Q R \textbf{16} Durngen] durgen G durin::: I dvringen O Z dvrungen L doringen M durúngen Q dúrringen R  $\cdot$ vürste] fursten Q graffe R marcgrefe Z \textbf{17} ingesinde] gesinde I  $\cdot$ ich] ez I \textbf{18} ûzgesinde] vusz gesinde Q \textbf{19} dir] Disz Q  $\cdot$ eines] an R  $\cdot$ Kayn] kaẏn G kains I key O Q kaýen L keyn M (Z) keyen R \textbf{20} dir] disz Q \textbf{21} manicvalten] mancualtich I mannic valde M menigen R \textbf{22} etswâ] swa I Erschwa R  $\cdot$ gedranc] gedanch L dranc M \textbf{23} werdez] werdens Q  $\cdot$ dringen] gedrengen M \textbf{24} hêr] et O  $\cdot$ Walther] walter L M [warter]: walter Q \textbf{26} swâ] Wo L (M) Q (R) Z  $\cdot$ solchen sanc nû] nu solhen sanc I sulchen gesanck nun Q soͯlich sang nun R \textbf{28} Kay] kaẏ G kain I Key O Q R Z Kaý L Keie M  $\cdot$ hetes in] het sin I hatte en des M het ins Q (Z) hett sy es R \textbf{29} hêr] der M Z  $\cdot$ Heinrich] hainrich I  $\cdot$ Rispach] rispac I \textbf{30} wunders mê] wunder I wnders mær O  $\cdot$ waz] daz L (R)  $\cdot$ dort] \textit{om.} Z  $\cdot$ geschach] me geshach I \newline
\end{minipage}
\hspace{0.5cm}
\begin{minipage}[t]{0.5\linewidth}
\small
\begin{center}*T
\end{center}
\begin{tabular}{rl}
 & an swem diu kurtôsîe\\ 
 & unde diu werde kompânîe\\ 
 & lac, den kunder êren,\\ 
 & sîn dienst gegen im kêren.\\ 
5 & ich gihe von im der mære,\\ 
 & er was ein merkære.\\ 
 & \textbf{er} tet vil rûhes willen schîn\\ 
 & ze schirme dem hêrren sîn.\\ 
 & partierre unde valsche diet,\\ 
10 & von den werden er die schiet.\\ 
 & er was ir vuore ein strenger hagel,\\ 
 & noch scherpfer danne \textbf{der} bîn ir zagel.\\ 
 & seht, die verkêrten Keys prîs.\\ 
 & \textbf{er} was manlîcher triuwen wîs.\\ 
15 & vil \textbf{hazzes} er von in gewan.\\ 
 & von Duringen \textbf{lantgrâve} Herman,\\ 
 & etslîch dîn ingesinde ich maz,\\ 
 & daz 'ûzgesinde' hieze baz.\\ 
 & dir wære ouch eines Key nôt,\\ 
20 & sît wâr\textit{iu} milte dir gebôt\\ 
 & sô manecvalten anehanc,\\ 
 & etswâ smæhlîchen gedranc\\ 
 & unde etswâ werdez dringen.\\ 
 & Des muoz hêr Walter singen:\\ 
25 & "guoten tac, bœse unde guot."\\ 
 & swâ man \textbf{noch} sölhen sanc nû tuot,\\ 
 & des sint die valschen geêret.\\ 
 & Key hets in niht gelêret\\ 
 & noch hêr Heinrich von Rispach.\\ 
30 & Hœret wunder\textit{s} mêr, waz dort geschach\\ 
\end{tabular}
\scriptsize
\line(1,0){75} \newline
T U V W \newline
\line(1,0){75} \newline
\textbf{14} \textit{Initiale} V  \textbf{24} \textit{Majuskel} T  \textbf{30} \textit{Majuskel} T  \newline
\line(1,0){75} \newline
\textbf{1} swem] wem U wem aber lag W  $\cdot$ kurtôsîe] kurtoyse W \textbf{3} lac] \textit{om.} W  $\cdot$ êren] geeren W \textbf{4} sîn] Seinen W \textbf{5} gihe] ginc U \textbf{6} er] Fr W  $\cdot$ merkære] rechter merkere W \textbf{7} er] Vnde V \textbf{8} hêrren] hertzen W \textbf{11} vuore] feúre W \textbf{12} noch] Vil V  $\cdot$ der bîn ir] des pigen W \textbf{13} verkêrten] verkeren W  $\cdot$ Keys] keyns U keins V W \textbf{14} was manlîcher] manlich W  $\cdot$ triuwen] trv́we V (W) \textbf{15} hazzes] hassen V  $\cdot$ in] im U \textbf{16} Fúrste von dúringen herman W  $\cdot$ Duringen] tv́ringen V \textbf{17} etslîch] Etschlicb W  $\cdot$ dîn ingesinde] [d*]: din in gesinde V dein gesinde W \textbf{18} hieze] hies es W \textbf{19} dir] [Diz]: Dir V  $\cdot$ Key] [k*]: keie V \textbf{20} wâriu] ware T \textbf{21} manecvalten] manig [valt*]: valtigen V \textbf{22} smæhlîchen] schwachlich W \textbf{23} werdez] werder W \textbf{24} Des] Das W  $\cdot$ Walter] walther U V \textbf{25} tac bœse] dac bosen U [ta*]: tag boͤse V \textbf{26} swâ] Wo U W  $\cdot$ noch] \textit{om.} W  $\cdot$ nû] \textit{om.} W \textbf{28} Key] [K*]: Key V  $\cdot$ hets in] het sin T het iz in U hetz [*]: in V \textbf{29} Rispach] Reispach V \textbf{30} wunders mêr] wunderz mer T wunder W  $\cdot$ waz] das W \newline
\end{minipage}
\end{table}
\end{document}
