\documentclass[8pt,a4paper,notitlepage]{article}
\usepackage{fullpage}
\usepackage{ulem}
\usepackage{xltxtra}
\usepackage{datetime}
\renewcommand{\dateseparator}{.}
\dmyyyydate
\usepackage{fancyhdr}
\usepackage{ifthen}
\pagestyle{fancy}
\fancyhf{}
\renewcommand{\headrulewidth}{0pt}
\fancyfoot[L]{\ifthenelse{\value{page}=1}{\today, \currenttime{} Uhr}{}}
\begin{document}
\begin{table}[ht]
\begin{minipage}[t]{0.5\linewidth}
\small
\begin{center}*D
\end{center}
\begin{tabular}{rl}
\textbf{309} & \textbf{\textit{\begin{large}O\end{large}}uch moht ers} sîn von schulden vrô.\\ 
 & Parzival si werte dô.\\ 
 & Nû \textbf{râtet}, hœret unt jeht,\\ 
 & ob tavelrunde meg ir reht\\ 
5 & des tages behalden, wande ir pflac\\ 
 & Artus, bî dem ein site lac:\\ 
 & nehein rîter vor im az\\ 
 & des tages, \textbf{swenne} âventiure vergaz,\\ 
 & daz si sînen hof vermeit.\\ 
10 & im ist âventiure nû bereit.\\ 
 & daz lop muose tavelrunde hân.\\ 
 & swie si \textbf{wære dâ ze Nantes} \textbf{lân},\\ 
 & Man sprach ir reht ûf bluomen velt.\\ 
 & dâne irte stûde noch gezelt.\\ 
15 & der künec Artus daz gebôt\\ 
 & ze êren dem rîter rôt.\\ 
 & sus nam sîn werdecheit dâ lôn.\\ 
 & \textbf{ein} pfelle von Acraton,\\ 
 & ûz heidenschefte verre brâht,\\ 
20 & \textbf{wart} zeime zil \textbf{al dâ} gedâht,\\ 
 & niht \textbf{breit}, sinwel gesniten,\\ 
 & al nâch tavelrunde siten,\\ 
 & wande in ir zuht des verjach:\\ 
 & nâch gegenstuole \textbf{dâ} niemen sprach.\\ 
25 & \textbf{die gesitze} \textbf{wâren} \textbf{al} gelîche hêr.\\ 
 & der künec Artus gebôt in mêr,\\ 
 & daz man werde \textbf{rîter} unt \textbf{werde} vrouwen\\ 
 & an dem ringe \textbf{müese} schouwen.\\ 
 & die man dâ gein prîse maz,\\ 
30 & magt, \textbf{wîb unt man ze hove dô} az.\\ 
\end{tabular}
\scriptsize
\line(1,0){75} \newline
D \newline
\line(1,0){75} \newline
\textbf{1} \textit{Initiale} D  \textbf{3} \textit{Majuskel} D  \textbf{13} \textit{Majuskel} D  \newline
\line(1,0){75} \newline
\textbf{1} Ouch] [÷*h]: ÷vch \textit{nachträglich korrigiert zu:} Ovch D \newline
\end{minipage}
\hspace{0.5cm}
\begin{minipage}[t]{0.5\linewidth}
\small
\begin{center}*m
\end{center}
\begin{tabular}{rl}
 & \textbf{ouch moht ers} sîn von schulden vrô.\\ 
 & \textit{Parcifal si werte dô}.\\ 
 & \begin{large}N\end{large}û \textbf{râtet}, hœret und jeht,\\ 
 & ob tavelrunder muge ir reht\\ 
5 & des tages behalten, wand ir pflac\\ 
 & Artus, bî dem ein site lac:\\ 
 & \textbf{ei}, kein ritter vor im az\\ 
 & des tages, \textbf{wenne} âventiure vergaz,\\ 
 & daz si sînen hof vermeit.\\ 
10 & im ist âventiure nû bereit.\\ 
 & daz lop muos tavelrunde hân.\\ 
 & wie si \textbf{ze Nantes wære} \textbf{verlân},\\ 
 & man sprach ir reht ûf bluomen velt.\\ 
 & dâ en\textit{ir}te stûde noch gezelt.\\ 
15 & der künic Artus daz gebôt\\ 
 & ze êren dem ritter rôt.\\ 
 & sus nam sîn werdicheit dâ l\textit{ô}n.\\ 
 & \textbf{ein} pfelle von Acraton,\\ 
 & ûz heidenschaft verre brâht,\\ 
20 & \textbf{wart} zeinem zil \textbf{aldâ} gedâht,\\ 
 & niht \textbf{breit}, sinwel gesniten,\\ 
 & al nâch \textbf{der} t\textit{a}vel\textit{run}de siten,\\ 
 & wand in ir zuht des verjach:\\ 
 & nâch gegenstuole \textbf{d\textit{â}} niemen sprach.\\ 
25 & \textbf{diu gesæ\textit{z}e} \textbf{wâren} \textbf{alle} glîche \textbf{und} hêr.\\ 
 & der künic Artus gebôt i\textit{n} mêr,\\ 
 & daz man werde \textbf{ritterschaft} und vrouwen\\ 
 & an dem ringe \textbf{müese} schouwen.\\ 
 & die man d\textit{â} gegen prîse maz,\\ 
30 & magt, \textbf{wî\textit{p} und man dô ze hof} az.\\ 
\end{tabular}
\scriptsize
\line(1,0){75} \newline
m n o Fr69 \newline
\line(1,0){75} \newline
\textbf{3} \textit{Initiale} m   $\cdot$ \textit{Capitulumzeichen} n  \newline
\line(1,0){75} \newline
\textbf{2} \textit{Vers 309.2 fehlt} m  \textbf{3} Nû râtet hœret] Nuͯ horent vnd rotent n Vnd ratent harent o \textbf{4} tavelrunder] tafelrunde n taferuͯnde o \textbf{5} ir] er n \textbf{6} Artus] Artuͯs o \textbf{7} ei] Vnd n o  $\cdot$ az] was n \textbf{9} vermeit] gemeit o \textbf{11} muos] muͯsz n myn o  $\cdot$ tavelrunde] tafelruͯnder o \textbf{12} Nantes] nantez o \textbf{14} dâ] Do n o  $\cdot$ enirte] enwette m  $\cdot$ stûde] weder stude n \textbf{17} dâ lôn] da lan m do lan n (o) \textbf{18} Acraton] acratan n o \textbf{22} tavelrunde] taufelde m \textbf{24} dâ] do m n o \textbf{25} diu] Der o  $\cdot$ gesæze] gesecze m  $\cdot$ und] \textit{om.} n o \textbf{26} Artus] artús o  $\cdot$ in] ym m \textbf{27} ritterschaft] ritter n o \textbf{28} müese] muͯsse m muͯstent o \textbf{29} dâ] do m n o \textbf{30} wîp und] wibe vnd m vnd wip vnd n vnd wip o \newline
\end{minipage}
\end{table}
\newpage
\begin{table}[ht]
\begin{minipage}[t]{0.5\linewidth}
\small
\begin{center}*G
\end{center}
\begin{tabular}{rl}
 & \textbf{er mohts} sîn von schulden vrô.\\ 
 & Parzival si werte dô.\\ 
 & nû \textbf{sprechet}, hœret unde jeht,\\ 
 & obe tavelrunder muge ir reht\\ 
5 & des tages behalten, \textit{wand} ir pflac\\ 
 & Artus, bî dem ein site lac:\\ 
 & dehein rîter vor im az\\ 
 & des tages, \textbf{sô} âventiure vergaz,\\ 
 & daz si sînen hof vermeit.\\ 
10 & im ist âventiure nû bereit.\\ 
 & daz lop muose tavelrunder hân.\\ 
 & swie si \textbf{wære ze Nantis} \textbf{lân},\\ 
 & man sprach ir reht ûf bluomen velt.\\ 
 & dâne irte stûde noch gezelt.\\ 
15 & der künic Artus daz gebôt\\ 
 & zêren dem rîter rôt.\\ 
 & sus nam sîn werdicheit dâ lôn.\\ 
 & \textbf{ein} pfelle von Acraton,\\ 
 & ûz heidenschefte verre brâht,\\ 
20 & \textbf{des was dâ} zeinem zil gedâht,\\ 
 & niht \textbf{breit}, sinewel gesniten,\\ 
 & al nâch \textbf{der} tavelrunder siten,\\ 
 & wan in ir zuht des verjach:\\ 
 & nâch g\textit{e}gens\textit{tuole} \textbf{dâ} niemen sprach.\\ 
25 & \textbf{ir gesitz} \textbf{was} \textbf{al}gelîche hêr.\\ 
 & der künic Artus gebôt in mêr,\\ 
 & daz man \textit{werde} \textbf{rîter} unde vrouwen\\ 
 & an dem ringe \textbf{solt} schouwen.\\ 
 & die man dâ gein brîse maz,\\ 
30 & maget, \textbf{wîp unde man ze hove dô} az.\\ 
\end{tabular}
\scriptsize
\line(1,0){75} \newline
G I O L M Q R Z \newline
\line(1,0){75} \newline
\textbf{3} \textit{Initiale} L  \textbf{11} \textit{Initiale} I O Q R Z  \textbf{15} \textit{Initiale} M  \textbf{29} \textit{Initiale} I  \newline
\line(1,0){75} \newline
\textbf{1} er mohts] er moht I Auch mocht ers Q (R)  $\cdot$ schulden] hertzen Z \textbf{2} Parzival] [parzifal]: Parzifal I Barcifal O Parzifal L M Partzifal Q Parczifal R Parcifal Z  $\cdot$ si] sich Q  $\cdot$ werte] warte L gewerte Z \textbf{4} tavelrunder] Tavelrvnde L \textbf{5} behalten] bihalde M  $\cdot$ wand ir] ir G wan her M  $\cdot$ pflac] pflagt Q \textbf{6} site] si O  $\cdot$ lac] lag vnd des pflag R \textbf{7} az] saz L \textbf{8} sô] so er I swenne O (M) Z wenne L wan er Q (R) \textbf{9} daz] Des L  $\cdot$ si] \textit{om.} I sy sy R \textbf{10} im] nu I  $\cdot$ ist] \textit{om.} Z  $\cdot$ âventiure nû] auentvre I nv aventiwer O (L) (Q) (R) \textbf{11} daz] ÷az O  $\cdot$ muose] muͤsen I  $\cdot$ tavelrunder] Tavelrvnde L (R) \textbf{12} swie] Wie L Q R  $\cdot$ wære] wern I wart O  $\cdot$ Nantis] nantes I (L) (Q) (Z) Natis R  $\cdot$ lân] gelan L M Q \textbf{13} \textit{Versdoppelung 309.13-14 nach 78.4} L   $\cdot$ sprach] sprech I  $\cdot$ velt] gelt M \textbf{14} dâne] Da M Do en Q  $\cdot$ irte] irt nieman noch I irreten M rite Q  $\cdot$ stûde] studen I (M)  $\cdot$ noch] noch daz O \textbf{15} daz] do R \textbf{16} zêren dem] Zu eren den Q Essen den R \textbf{17} sîn] si O  $\cdot$ dâ] do O L Q ze R \textbf{18} ein] einen I (R)  $\cdot$ Acraton] acredon G agraton I achgregon O Ahgredon L achgredon M ackriton R akraton Z \textbf{19} verre] wart M \textbf{20} des was] Des ward R Wart Z  $\cdot$ dâ] \textit{om.} I Z do Q R  $\cdot$ gedâht] alda gedaht Z \textbf{21} breit] bereit R \textbf{22} al] Als R  $\cdot$ der] \textit{om.} Z  $\cdot$ tavelrunder] tauelrunde I \textbf{23} in] \textit{om.} O  $\cdot$ des] dasz Q \textbf{24} gegenstuole] gagensidel G  $\cdot$ dâ] do R  $\cdot$ niemen] nẏem R  $\cdot$ sprach] sach O \textbf{25} ir gesitz] Jr gesittes R Die gesitze Z  $\cdot$ was] waren Z  $\cdot$ algelîche] geliche L (M)  $\cdot$ hêr] herer I \textbf{26} Artus] Artuͯs L  $\cdot$ in] im Q \textbf{27} man] \textit{om.} M  $\cdot$ werde] \textit{om.} G  $\cdot$ vrouwen] werde frowen O (L) (M) (Q) (R) \textbf{28} Man solde andeme ringe schouwen M \textbf{29} dâ] do I Q R \textbf{30} wîp unde man] wip man O (Q) man wip M  $\cdot$ ze hove dô] daze hove O (L) zcu hofe da M (Z) \newline
\end{minipage}
\hspace{0.5cm}
\begin{minipage}[t]{0.5\linewidth}
\small
\begin{center}*T
\end{center}
\begin{tabular}{rl}
 & \textbf{des mohten si} sîn von schulden vrô.\\ 
 & Parcifal si\textbf{z} werte dô.\\ 
 & \begin{large}N\end{large}û \textbf{sprechet}, hœret unde jeht,\\ 
 & ob \textbf{die} tavelrunder muge ir reht\\ 
5 & des tages behalten, wan ir pflac\\ 
 & Artus, bî dem ein site lac,\\ 
 & \textbf{daz} dehein rîter vor im az\\ 
 & des tages, \textbf{swenne}\textbf{r} âventiur vergaz\\ 
 & \textbf{unde} daz si sînen hof vermeit.\\ 
10 & Im ist âventiure nû bereit.\\ 
 & daz lop muose tavelrunder hân.\\ 
 & swie \textbf{daz} si \textbf{wære ze Nantes} \textbf{lân},\\ 
 & man sprach ir reht ûf bluomen velt.\\ 
 & dâ enirrete stûde noch gezelt.\\ 
15 & der künec Artus daz gebôt\\ 
 & zêren dem rîter rôt.\\ 
 & Sus nam sîn werdecheit dâ lôn,\\ 
 & \textbf{einen} pfelle von Acraton,\\ 
 & ûz heidenschaft verre brâht.\\ 
20 & \textbf{des was dâ} zeinem zil gedâht,\\ 
 & niht \textbf{lanc}, sinewel gesniten,\\ 
 & al nâch \textbf{der} tavelrunder siten,\\ 
 & wand in ir zuht des verjach:\\ 
 & nâch gegenstuole niemen sprach.\\ 
25 & \textbf{ir gesitz} \textbf{was} glîche hêr.\\ 
 & der künec Artus gebôt in mêr,\\ 
 & daz man werde \textbf{rîter} unde vrouwen\\ 
 & an dem ringe \textbf{lieze}\textbf{n} schouwen.\\ 
 & die man dâ gegen prîse maz,\\ 
30 & maget \textbf{unde wîp ze hove} az.\\ 
\end{tabular}
\scriptsize
\line(1,0){75} \newline
T U V W \newline
\line(1,0){75} \newline
\textbf{3} \textit{Initiale} T U  \textbf{10} \textit{Majuskel} T  \textbf{17} \textit{Initiale} W   $\cdot$ \textit{Majuskel} T  \newline
\line(1,0){75} \newline
\textbf{1} des] Sy W  $\cdot$ mohten si] [mohter]: mohten si T moͤhte er V moͤchten W \textbf{2} Parcifal] parzifal T (V) Partzifal W  $\cdot$ siz] sú V sich W \textbf{3} jeht] secht W \textbf{4} die tavelrunder] tauelrunde V \textbf{5} wan] ob W \textbf{8} swenner] wan er U (W) \textbf{9} unde] \textit{om.} W \textbf{11} daz] Dein W  $\cdot$ muose] mvese T muͤste V  $\cdot$ tavelrunder] [tauelrund]: tauelrunder V die tauelrunde W \textbf{12} swie] Wie U W  $\cdot$ daz] \textit{om.} U V W  $\cdot$ ze Nantes lân] zenantes verlan V zuͦ nantis gelan W \textbf{13} \textit{Versdoppelung 309.13-14 nach 78.4} V   $\cdot$ ir] \textit{om.} W \textbf{14} dâ] Do U V W  $\cdot$ enirrete] enirret sy W \textbf{15} daz] do W \textbf{17} dâ] do V W \textbf{18} einen] Ein V Finen W  $\cdot$ Acraton] Acratôn T acharon W \textbf{20} des was] [D*]: Dez wart V  $\cdot$ dâ] do V W \textbf{21} lanc] breit V (W) \textbf{22} al] Alle U  $\cdot$ tavelrunder] [tauelrund*]: tauelrunder V \textbf{23} ir] \textit{om.} W \textbf{25} gesitz] sitzen W  $\cdot$ was] was all W \textbf{26} gebôt] >daz< gebot U \textbf{28} liezen] solte U V W \textbf{29} dâ] \textit{om.} U V do W \textbf{30} unde wîp] wip vnd man U (V) (W)  $\cdot$ az] do as V \newline
\end{minipage}
\end{table}
\end{document}
