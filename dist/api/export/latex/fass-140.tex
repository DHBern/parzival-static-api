\documentclass[8pt,a4paper,notitlepage]{article}
\usepackage{fullpage}
\usepackage{ulem}
\usepackage{xltxtra}
\usepackage{datetime}
\renewcommand{\dateseparator}{.}
\dmyyyydate
\usepackage{fancyhdr}
\usepackage{ifthen}
\pagestyle{fancy}
\fancyhf{}
\renewcommand{\headrulewidth}{0pt}
\fancyfoot[L]{\ifthenelse{\value{page}=1}{\today, \currenttime{} Uhr}{}}
\begin{document}
\begin{table}[ht]
\begin{minipage}[t]{0.5\linewidth}
\small
\begin{center}*D
\end{center}
\begin{tabular}{rl}
\textbf{140} & \multicolumn{1}{l}{ - - - }\\ 
 & \multicolumn{1}{l}{ - - - }\\ 
 & Ê si den knappen rîten lieze,\\ 
 & si vrâgete in, wie er hieze,\\ 
5 & \textbf{unt} jach, er trüege den gotes vlîz.\\ 
 & "bon fiz, iera fiz, beafiz,\\ 
 & \textbf{alsus} hât mich genennet,\\ 
 & \textbf{der} mich dâ heime erkennet."\\ 
 & \begin{large}D\end{large}ô \textbf{diu} rede was getân,\\ 
10 & si erkant in bî dem namen sân.\\ 
 & nû hœrt in \textbf{rehter} nennen,\\ 
 & daz i\textit{r} \textbf{wol} muget erkennen,\\ 
 & wer dirre âventiure hêrre sî.\\ 
 & der hielt der juncvrouwen bî.\\ 
15 & ir rôter munt sprach sunder twâl:\\ 
 & "\textbf{deiswâr}, dû heizest Parzival.\\ 
 & der nam i\textit{st} rehte ›mitten durch‹.\\ 
 & grôz liebe i\textit{e}r \textbf{solhe} herzen vurch\\ 
 & mit dîner muoter triwe.\\ 
20 & dîn vater liez ir riwe.\\ 
 & ich engihe dirs niht ze ruome,\\ 
 & dîn muoter ist mîn muome,\\ 
 & \textbf{unt} sag dir sunder valschen list\\ 
 & die rehten wârheit, wer dû bist.\\ 
25 & Dîn vater was ein Anschevin.\\ 
 & ein Waleis von der muoter \textbf{dîn}\\ 
 & \textbf{bistû} geborn von Kanvoleiz.\\ 
 & die rehten wârheit ich des weiz.\\ 
 & dû bist ouch künec ze Norgals.\\ 
30 & in der houptstat ze Kingrivals\\ 
\end{tabular}
\scriptsize
\line(1,0){75} \newline
D \newline
\line(1,0){75} \newline
\textbf{3} \textit{Majuskel} D  \textbf{9} \textit{Initiale} D  \textbf{25} \textit{Majuskel} D  \newline
\line(1,0){75} \newline
\textbf{1} \textit{Die Verse 140.1-2 fehlen} D  \textbf{9} Dô] ÷o \textit{nachträglich ergänzt zu:} Do D \textbf{12} ir] in D \textbf{17} ist] ir D \textbf{18} ier] ir D \textbf{25} Anschevin] Anscevin D \newline
\end{minipage}
\hspace{0.5cm}
\begin{minipage}[t]{0.5\linewidth}
\small
\begin{center}*m
\end{center}
\begin{tabular}{rl}
 & \multicolumn{1}{l}{ - - - }\\ 
 & \multicolumn{1}{l}{ - - - }\\ 
 & ê si den knappen rîten lieze,\\ 
 & si vrâgete in, wie er hieze,\\ 
5 & \textbf{und} jach, er trüege den gotes vlîz.\\ 
 & "bonfi\textit{z}, ierafi\textit{z}, beafi\textit{z},\\ 
 & \textbf{als} hât mich genennet,\\ 
 & \textbf{der} mich dâ heime erkennet."\\ 
 & dô \textbf{diu} rede was getân,\\ 
10 & si erkanten bî dem namen sân.\\ 
 & nû hœret in \textbf{rehter} nennen.\\ 
 & \dag si mügen\dag  \textbf{wol} erkennen,\\ 
 & wer dirre âventiure hêrre s\textit{î}.\\ 
 & der h\textit{i}elt der juncvrouwen bî.\\ 
15 & ir rôter munt sprach sunder twâl:\\ 
 & daz  wâr, dû heizest Parcifal.\\ 
 & der name ist reht ›enmitten durch‹.\\ 
 & grôziu liebe \dag er\dag  \textbf{soliche} herzvurch\\ 
 & mit dîner muoter triuwe.\\ 
20 & dîn vater liez ir riuwe.\\ 
 & \textit{in}e gihe dirs niht ze ruome,\\ 
 & dîn muoter ist mîn muome,\\ 
 & \textbf{und} sage dir sunder valschen list\\ 
 & die rehten wârheit, wer dû bist.\\ 
25 & dîn vater was ein A\textit{n}schevin.\\ 
 & ein Waleis von der muoter \textbf{sîn}\\ 
 & \textbf{bist dû} geborn von Kanvoleiz.\\ 
 & die rehten wârheit ich des weiz.\\ 
 & dû bist ouch künic ze Norgals.\\ 
30 & in der houbetstat ze K\textit{i}ngrivals\\ 
\end{tabular}
\scriptsize
\line(1,0){75} \newline
m n o \newline
\line(1,0){75} \newline
\newline
\line(1,0){75} \newline
\textbf{1} \textit{Die Verse 140.1-2 fehlen} m n o  \textbf{3} lieze] lesse n \textbf{5} jach] jach do n \textbf{6} bonfiz] Bonvir m Bonfúr n Bonfir o  $\cdot$ ierafiz] ierafir m o jerafuͯr n  $\cdot$ beafiz] beafir m o beaflẏsz n \textbf{7} hât] hett o \textbf{11} nennen] nennent o \textbf{13} sî] sin m \textbf{14} hielt] helt m \textbf{16} Parcifal] parcival m \textbf{17} enmitten] in mitte o \textbf{18} er soliche] in sollichem n (o)  $\cdot$ herzvurch] hertzen furch n (o) \textbf{20} liez] liesse n \textbf{21} ine gihe] Me gihe m Jch engie o \textbf{23} \textit{Die Verse 140.23-24 fehlen} n o  \textbf{24} rehten] rechte m \textbf{25} \textit{Versfolge 140.26-25} n o   $\cdot$ Anschevin] auscevin m anschefin n \textbf{26} Waleis] walleis m  $\cdot$ sîn] din n o \textbf{27} bist dû] Du bist n o  $\cdot$ Kanvoleiz] kanvoleis m kanfoleis n o \textbf{28} rehten] rechte n (o)  $\cdot$ ich] \textit{om.} n o  $\cdot$ des] das o \textbf{30} \textit{Vers 140.30 fehlt} n   $\cdot$ Kingrivals] kungrivals m kingenals o \newline
\end{minipage}
\end{table}
\newpage
\begin{table}[ht]
\begin{minipage}[t]{0.5\linewidth}
\small
\begin{center}*G
\end{center}
\begin{tabular}{rl}
 & dû bist geborn von triwen,\\ 
 & daz er dich sus kan riwen."\\ 
 & ê si den knappen rîten lieze,\\ 
 & si vrâgte in, wier hieze.\\ 
5 & \textbf{si} jach, er trüege den gotes vlîz.\\ 
 & "bon fiz, tschier fiz, beanfiz,\\ 
 & \textbf{\textit{al}sus} hât mich genennet,\\ 
 & \textbf{swer} mich dâ heime erkennet."\\ 
 & dô \textbf{diu} rede was getân,\\ 
10 & si erkande in bî dem namen sân.\\ 
 & \multicolumn{1}{l}{ - - - }\\ 
 & \multicolumn{1}{l}{ - - - }\\ 
 & \multicolumn{1}{l}{ - - - }\\ 
 & \multicolumn{1}{l}{ - - - }\\ 
15 & ir rôter munt sprach sunder twâl:\\ 
 & "\textbf{dêswâr}, dû heizest Parzival.\\ 
 & der name ist rehte ›enmitten durch‹.\\ 
 & grôz liebe ier \textbf{solch} herzen vurch\\ 
 & mit dîner muoter triwe.\\ 
20 & dîn vater liez ir riwe.\\ 
 & ichne gihe dirs niht ze ruome,\\ 
 & dîn muoter ist mîn muome.\\ 
 & \textbf{ich} sage dir sunder valschen list\\ 
 & die rehten wârheit, wer dû bist.\\ 
25 & dîn vater was ein Antschevin.\\ 
 & ein Waleise von der muoter \textbf{dîn}\\ 
 & \textbf{\textit{\begin{large}B\end{large}ist dû}} geborn von Kanvoleiz.\\ 
 & die rehten wârheit ich des weiz.\\ 
 & dû bist ouch künic ze Nurgals.\\ 
30 & in der houbetstat ze Kinrivals\\ 
\end{tabular}
\scriptsize
\line(1,0){75} \newline
G I O L M Q R Z \newline
\line(1,0){75} \newline
\textbf{5} \textit{Initiale} M  \textbf{9} \textit{Initiale} I L Q R Z  \textbf{27} \textit{Initiale} G  \newline
\line(1,0){75} \newline
\textbf{1} geborn] gebarn O \textbf{2} sus kan riwen] kan ausz reyben Q \textbf{3} ê si] Eli M  $\cdot$ lieze] liez M (Q) (R) Z \textbf{4} vrâgte in] fragt in I fragte Q fragent in R frageten e Z  $\cdot$ hieze] hiez I M (Q) (R) Z \textbf{5} si jach] Sit L Sie sprach M  $\cdot$ trüege] trvͦg O (Q) truͯgen R \textbf{6} tschier fiz] scheraphiz I Thierfiz L schere fiz O  $\cdot$ beanfiz] bea fiz O (L) (M) (Q) (R) (Z) \textbf{7} alsus] sus G \textbf{8} swer] der I (L) Wer M Q R \textbf{9} dô] Da M Z  $\cdot$ diu] disu R  $\cdot$ was] was was Z \textbf{10} erkande] bechande I kant Q  $\cdot$ in] \textit{om.} L  $\cdot$ dem] \textit{om.} R  $\cdot$ sân] sein Q (R) \textbf{11} \textit{Die Verse 140.11-14 fehlen} G I O M Z   $\cdot$ Nv horent in rechter (rehte Q ) nennen L (Q) (R) \textbf{12} Daz ir in muͯget (wol mocht Q wol mugent R ) erkennen L (Q) (R) \textbf{13} Wer dirre (diese Q ) aventuͯre herre sẏ L (Q) (R) \textbf{14} Der hielt der jvngfrauwen bý L (Q) (R) \textbf{16} dêswâr] Disz war M Entzwar Q Zwar Z  $\cdot$ dû heizest] so bistuz I  $\cdot$ Parzival] Parzifal I (M) Parcifal O (Q) (Z) Partzifal L parczifal R \textbf{17} der] Din O L (M) (Q)  $\cdot$ enmitten] Emitten L eyn mitten M  $\cdot$ durch] druch M \textbf{18} ier] er R Z  $\cdot$ vurch] wurch I fruch M R \textbf{19} \textit{Versfolge 140.21-22-19-20} I   $\cdot$ din muͤter vil getriwe I \textbf{21} \textit{Vers 140.21 fehlt} R   $\cdot$ ichne gihe dirs] ich gihe sin I Jch gihe dir O Jch en thu dirsz M \textbf{23} sage dir] \textit{om.} Z  $\cdot$ valschen] valsche M \textbf{24} wer] wie M \textbf{25} ein] von M  $\cdot$ Antschevin] anschevin G antsheuin I anshevin O (L) Z anscevin M anscherin Q anschwovin R \textbf{26} Waleise] waleis I O L Q R Z waleisz M \textbf{27} Bist dû] Dv bist G  $\cdot$ Kanvoleiz] Ganfoleis I kanvolays L kamvoleis M kanűoleis Q kanuoleis R Ranwoleis Z \textbf{28} \textit{Vers 140.28 fehlt} R  \textbf{29} ouch] \textit{om.} I  $\cdot$ ze] von Q  $\cdot$ Nurgals] NorGals I Nvͦrgals O Novrgals L núrgals Q \textbf{30} houbetstat] werden stat O  $\cdot$ Kinrivals] kingrifals I kingrivals O Z [kingivals]: kingrivals L kingriuals M Q kungriuals R \newline
\end{minipage}
\hspace{0.5cm}
\begin{minipage}[t]{0.5\linewidth}
\small
\begin{center}*T (U)
\end{center}
\begin{tabular}{rl}
 & dû bist geborn von triuwen,\\ 
 & daz er dich sus kan riuwen."\\ 
 & ê si den knaben rîten lieze,\\ 
 & si vrâgete in, wie er hieze.\\ 
5 & \textbf{si} jach, er trüege den gotes vlîz.\\ 
 & "bonfiz, scherafiz, befiz,\\ 
 & \textbf{alsus} hâte mich genennet,\\ 
 & \textbf{wer} mich dâ heime erkennet."\\ 
 & dô \textbf{disiu} rede was getân,\\ 
10 & si erkant in bî dem namen sân.\\ 
 & nû hœret in \textbf{rehte} nennen,\\ 
 & daz ir \textbf{in} moget erkennen,\\ 
 & wer dirre âventiure hêrre sî.\\ 
 & der hielt der juncvrouwen bî.\\ 
15 & ir rôter munt sprach sunder twâl:\\ 
 & "\textbf{dêswâr}, dû heizest Parcifal.\\ 
 & \textit{der name ist reht ›enmitten durch‹.}\\ 
 & \textit{grôziu liebe i}e\textit{r \textbf{sælic} herzen vurch}\\ 
 & mit dîner muoter triuwe.\\ 
20 & dîn vater liez ir riuwe.\\ 
 & ich engihe dir es niht zuo ruome,\\ 
 & dîn muoter ist mîniu muome.\\ 
 & \textbf{ich} sage dir \textbf{ez} sunder valschen list,\\ 
 & die rehten wârheit, wer dû bist.\\ 
25 & dîn vater was ein Anschevin.\\ 
 & ein Waleis von der muoter \textbf{dîn},\\ 
 & \textbf{dû bist} geborn von Kanvoleiz.\\ 
 & die rehten wârheit ich des weiz.\\ 
 & dû bist ouch künec ze Norgals.\\ 
30 & in der houbetstat zuo Kingrivals\\ 
\end{tabular}
\scriptsize
\line(1,0){75} \newline
U V W T \newline
\line(1,0){75} \newline
\textbf{3} \textit{Majuskel} T  \textbf{6} \textit{Majuskel} T  \textbf{9} \textit{Majuskel} T  \textbf{11} \textit{Majuskel} T  \textbf{15} \textit{Initiale} W   $\cdot$ \textit{Majuskel} T  \newline
\line(1,0){75} \newline
\textbf{3} Êsin riten lîeze T \textbf{4} vrâgete] fragt W \textbf{5} si jach er trüege] Vnd treist W si iach er iach T \textbf{6} Er sprach gvͦt svn lieber svn beavis V  $\cdot$ scherafiz] befis W Beafiz T  $\cdot$ befiz] beafis W \textbf{7} hâte] hat V W T \textbf{8} wer] Der V swer T  $\cdot$ mich dâ heime] da heime mich T \textbf{9} Dô disiu] Dise W Do div T \textbf{11} \textit{Die Verse 140.11-14 sind am rechten Spaltenrand nachgetragen} T  \textbf{12} in] \textit{om.} T \textbf{16} dêswâr] Ist es war W benamen T  $\cdot$ Parcifal] Parzifal U (V) T partziual W \textbf{17} \textit{Die Verse 140.17-18 fehlen} U  \textbf{18} Groze liͤbe [*]: ir selig herzen furch V  $\cdot$ Vor grosser liebe das hertze furch W \textbf{19} \textit{Versdoppelung} T   $\cdot$ mit] Von W \textbf{21} ich engihe] Jch gihe V Ich sage W  $\cdot$ zuo] von W \textbf{23} ez] \textit{om.} W T \textbf{24} rehten] rechte U \textbf{25} Anschevin] Anschovin U Anscheuin V antscheuin W Anscevin T \textbf{26} Waleis] walleis V \textbf{27} dû bist] Bistu W (T)  $\cdot$ Kanvoleiz] kanvoleis V kanuoleis W \textbf{28} rehten] rechte U \textbf{30} Kingrivals] kyngrivals V kingriuals W \newline
\end{minipage}
\end{table}
\end{document}
