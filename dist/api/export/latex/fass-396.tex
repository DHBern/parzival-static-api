\documentclass[8pt,a4paper,notitlepage]{article}
\usepackage{fullpage}
\usepackage{ulem}
\usepackage{xltxtra}
\usepackage{datetime}
\renewcommand{\dateseparator}{.}
\dmyyyydate
\usepackage{fancyhdr}
\usepackage{ifthen}
\pagestyle{fancy}
\fancyhf{}
\renewcommand{\headrulewidth}{0pt}
\fancyfoot[L]{\ifthenelse{\value{page}=1}{\today, \currenttime{} Uhr}{}}
\begin{document}
\begin{table}[ht]
\begin{minipage}[t]{0.5\linewidth}
\small
\begin{center}*D
\end{center}
\begin{tabular}{rl}
\textbf{396} & \begin{large}M\end{large}elyanz durch daz dar nâher gienc.\\ 
 & diu magt Gawanen \textbf{zuo zir gevienc}.\\ 
 & Obilote \textbf{doch} sicherheit geschach,\\ 
 & \textbf{dâ ez} manec \textbf{werder ritter} sach.\\ 
5 & "Hêr künec, nû habt ir missetân,\\ 
 & sol mîn ritter sîn ein koufman,\\ 
 & des mich mîn swester vil an streit,\\ 
 & daz ir im gâbet sicherheit",\\ 
 & \textbf{sus} sprach diu \textbf{magt} Obilot.\\ 
10 & Melyanze si dâ nâch gebôt,\\ 
 & daz er sicherheit verjæhe,\\ 
 & diu in ir hant geschæhe,\\ 
 & ir swester Obien.\\ 
 & "zeiner âmîen\\ 
15 & sult ir si \textbf{hân} durch ritters prîs.\\ 
 & zeinem hêrren \textbf{unt} zeinem âmîs\\ 
 & sol si iuch immer gerne hân.\\ 
 & i\textbf{ne} wils iuch \textbf{enwederhalp} \textbf{erlân}."\\ 
 & Got ûz \textbf{ir jungen} munde sprach.\\ 
20 & ir bete bêdenthalp geschach.\\ 
 & dâ meisterte vrou minne\\ 
 & mit \textbf{ir} \textbf{krefteclîchem} sinne\\ 
 & unt \textbf{herzenlîchiu} triwe\\ 
 & der zweier \textbf{liebe} al niwe.\\ 
25 & Obien hant vürn mantel sleif,\\ 
 & \textbf{dâ si} Melyanzes \textbf{arm} begreif.\\ 
 & \textbf{al weinende kust} ir \textbf{rôter} munt,\\ 
 & dâ \textbf{der} \textbf{was} \textbf{von} \textbf{der} tjoste wunt.\\ 
 & manec zaher \textbf{im} den arm begôz,\\ 
30 & der von ir liehten ougen vlôz.\\ 
\end{tabular}
\scriptsize
\line(1,0){75} \newline
D \newline
\line(1,0){75} \newline
\textbf{1} \textit{Initiale} D  \textbf{5} \textit{Majuskel} D  \textbf{19} \textit{Majuskel} D  \newline
\line(1,0){75} \newline
\textbf{2} Gawanen] Gawann D \newline
\end{minipage}
\hspace{0.5cm}
\begin{minipage}[t]{0.5\linewidth}
\small
\begin{center}*m
\end{center}
\begin{tabular}{rl}
 & Mel\textit{i}anz durch daz dar nâher gienc.\\ 
 & diu maget Gawanen \textbf{zuo ir gevienc}.\\ 
 & Obilote \textbf{doch} sicherheit geschach,\\ 
 & \textbf{daz ez} manic \textbf{ritter wert} sach.\\ 
5 & "hêr künic, nû habet ir missetân,\\ 
 & sol mîn ritter sîn ein koufman,\\ 
 & des mich mîn swester vil an streit,\\ 
 & daz ir ime gâbet sicherheit",\\ 
 & \textbf{sus} sprach diu \textbf{maget} Obilot.\\ 
10 & Mel\textit{i}anz si dâ nâch gebôt,\\ 
 & daz er \textbf{der} sicherheit verjæhe,\\ 
 & diu i\textit{n} \textit{i}r hant geschæhe,\\ 
 & ir sw\textit{e}ster Obien.\\ 
 & "zuo einer âmîen\\ 
15 & sullet ir \textit{si} \textbf{hân} durch ritters prîs.\\ 
 & ze einem hêrren \textbf{und} ze einem âmîs\\ 
 & sol \textit{si} iuch iemer gerne hân.\\ 
 & ich wil es iuch \textbf{enwederhalp} \textbf{lân}."\\ 
 & got ûz\textbf{er süezem} munde sprach.\\ 
20 & ir b\textit{e}te beidenthalp geschach.\\ 
 & dâ meister\textit{t} vrouwe minne\\ 
 & mit \textbf{ir} \textbf{krefteclîchem} sinne\\ 
 & und \textbf{herzlîchiu} triuwe\\ 
 & der zweier \textbf{minne} \textit{alniuwe}.\\ 
25 & Obien hant vür den mantel sleif,\\ 
 & \textbf{dâ si} Melianzes \textbf{hant} begreif.\\ 
 & \textbf{alweinende in kuste} ir \textbf{rôter} munt,\\ 
 & dâ \textbf{der} \textbf{wart} \textbf{von} juste wunt.\\ 
 & manic zaher \textbf{ime} den arm begôz,\\ 
30 & der von ir liehten ougen vlôz.\\ 
\end{tabular}
\scriptsize
\line(1,0){75} \newline
m n o \newline
\line(1,0){75} \newline
\newline
\line(1,0){75} \newline
\textbf{1} Melianz] Meleancz m Meliantz n Meliancz o \textbf{2} Gawanen] gewane o  $\cdot$ ir] mir o \textbf{3} Obilote] Obilotte m Obilot n o \textbf{4} ritter wert] werder ritter n (o) \textbf{6} sîn] si o \textbf{8} gâbet] gebent n o \textbf{9} Obilot] abilot n o \textbf{10} Melianz] Meleancz m Meliantz n Meliancz o \textbf{11} Des er der sicherheit weriehen o \textbf{12} in ir] jn in ir m \textbf{13} swester] swster m lieben swester n  $\cdot$ Obien] obẏen n obigen o \textbf{14} zuo] Von o  $\cdot$ âmîen] lieben amẏen n \textbf{15} \textit{Versdoppelung 396.15-16 (²o) nach 396.15; Lesarten der vorausgehenden Verse mit ¹o bezeichnet} o   $\cdot$ ir] \textit{om.} \textsuperscript{1}\hspace{-1.3mm} o  $\cdot$ si] \textit{om.} m \textbf{16} und] \textit{om.} n o \textbf{17} si] ich m \textbf{18} enwederhalp] ẏe wider halp n (o) \textbf{19} ûzer süezem] vs irem suͯssen n (o) \textbf{20} bete] beitte m  $\cdot$ beidenthalp] bederhalp o \textbf{21} dâ] Do n o  $\cdot$ meistert] meister m \textbf{23} herzlîchiu] hertzelicher n herczoleider o \textbf{24} minne] liebe n o  $\cdot$ alniuwe] \textit{om.} m alle nuwe o \textbf{25} Obien] Obẏen n \textbf{26} dâ] Do n o  $\cdot$ Melianzes] [melia*]: melianczes m meliantzes n melianczes o \textbf{27} alweinende] Al weinenden o \textbf{28} dâ] Do n o  $\cdot$ wart] was n o \textbf{29} den] dem o \newline
\end{minipage}
\end{table}
\newpage
\begin{table}[ht]
\begin{minipage}[t]{0.5\linewidth}
\small
\begin{center}*G
\end{center}
\begin{tabular}{rl}
 & Melianz durch daz dar nâher gienc.\\ 
 & diu maget Gawanen \textbf{vaste umbevienc}.\\ 
 & Obilote \textbf{dâ} sicherheit geschach,\\ 
 & \textbf{da\textit{z} ez} manic \textbf{wert rîter} sach.\\ 
5 & "hêr künic, nû habet ir missetân,\\ 
 & sol mîn rîter sîn ein koufman,\\ 
 & des \textit{mich} mîn swester vil \textit{an} streit,\\ 
 & daz ir im gâbet sicherheit",\\ 
 & \textbf{sus} sprach diu \textbf{junge} Obilot.\\ 
10 & Melianze si dar nâch gebôt,\\ 
 & daz er sicherheit verjæhe,\\ 
 & diu in ir hant geschæhe,\\ 
 & ir swester Obien.\\ 
 & "zeiner âmîen\\ 
15 & sult ir si \textbf{nemen} durch rîters prîs.\\ 
 & zeinem hêrren \textbf{unde} zeinem âmîs\\ 
 & sol si iuch imer gerne hân.\\ 
 & ich wil es iuch \textbf{dewederhalp} \textbf{erlân}."\\ 
 & got ûz \textbf{der ju\textit{n}gen} munde sprach.\\ 
20 & ir bet \textit{bêdent}halp geschach.\\ 
 & dô meisterte vrô minne\\ 
 & mit \textbf{vriuntlîchem} sinne\\ 
 & unde \textbf{herzenlîchiu} triwe\\ 
 & der zweier \textbf{liebe} al niwe.\\ 
25 & \begin{large}O\end{large}bien hant vü\textit{r} den mandel sleif,\\ 
 & Melianzes \textbf{arm si} begreif\\ 
 & \textbf{unde dructe in an} ir \textbf{rôten} munt,\\ 
 & \textbf{al} dâ \textbf{er} \textbf{was} \textbf{ze}\textbf{r} tjoste wunt.\\ 
 & manic zaher \textbf{ir} den arm begôz,\\ 
30 & der von ir liehten ougen vlôz.\\ 
\end{tabular}
\scriptsize
\line(1,0){75} \newline
G I O L M Q R Z Fr28 \newline
\line(1,0){75} \newline
\textbf{1} \textit{Initiale} I O L M Z   $\cdot$ \textit{Capitulumzeichen} R  \textbf{19} \textit{Initiale} I  \textbf{25} \textit{Initiale} G  \newline
\line(1,0){75} \newline
\textbf{1} \textit{Die Verse 370.13-412.12 fehlen} Q   $\cdot$ Melianz] ÷elyanz O MEliantz L (Z) Meliancz R  $\cdot$ dar] er R her Z  $\cdot$ nâher] nahe O \textbf{2} Gawanen] Gawan I (M) her Gawan R  $\cdot$ vaste] \textit{om.} R \textbf{3} Obilote] [obilot]: obilote G Obylot O Z Oblet R Obẏlot Fr28  $\cdot$ dâ] do R \textbf{4} daz] da G  $\cdot$ wert] \textit{om.} Z  $\cdot$ sach] gesach I \textbf{5} hêr künic] Der konnick sprach M Her chuͦnig sprach sie Fr28  $\cdot$ ir] dir I \textbf{6} sol] So R  $\cdot$ sîn] sy M \textbf{7} mich] \textit{om.} G  $\cdot$ an streit] gestreit G \textbf{8} im] \textit{om.} O  $\cdot$ gâbet] iahet I gebt R gaben Fr28 \textbf{9} Obilot] Obylot O (Z) oblett R obẏlot Fr28 \textbf{10} Melianze] Melyanz O Melianz M Meliancze R Meliantze Z  $\cdot$ si dar nâch] ir das M sie daz Fr28 \textbf{11} er] er ir I ir M  $\cdot$ verjæhe] fuͦr lehe Fr28 \textbf{13} Obien] obion I obyen O R Z obẏen Fr28 \textbf{14} âmîen] Amyon I \textbf{15} ir si] irs L ir sin M  $\cdot$ nemen] neme M han Z \textbf{17} imer] myner M  $\cdot$ gerne] ýemmer L \textbf{18} wil es iuch] wil evchs I enwils uͯch L (Z) (Fr28) wil úch R  $\cdot$ dewederhalp] wederthalb O (M) (Z) niewederhalb Fr28  $\cdot$ erlân] [alre]: erlan G [lan]: erlan M \textbf{19} der] ir O L M (R) Z Fr28  $\cdot$ jungen] ivgen G \textbf{20} ir] ir ir I  $\cdot$ bêdenthalp] iewederhalp G  $\cdot$ geschach] scah Fr28 \textbf{21} dô] Da O Z  $\cdot$ meisterte] maistert I (O) (L) (Z) meister M \textbf{22} vriuntlîchem] Mit ir krefticlichem Z \textbf{23} herzenlîchiu] herzenlicher I hertliche R hertzicliche Z \textbf{24} al] \textit{om.} O \textbf{25} Obien] [Oben]: Obien G Obyen O Z obẏen Fr28  $\cdot$ vür] vuf G \textbf{26} Melianzes] Melyanzes O Melianczes R Meliantzes Z  $\cdot$ arm] hant I \textbf{27} dructe] drucke Z  $\cdot$ ir] irm R \textbf{28} zer tjoste] von der tioste I zetyost O von strit R  $\cdot$ wunt] gwunt Fr28 \textbf{29} ir] im O L (M) (R) Z Fr28 \newline
\end{minipage}
\hspace{0.5cm}
\begin{minipage}[t]{0.5\linewidth}
\small
\begin{center}*T
\end{center}
\begin{tabular}{rl}
 & Melyanz durch daz dar nâher gienc.\\ 
 & diu maget Gawanen \textbf{vaste umbevienc}.\\ 
 & Obylote \textbf{dâ} sicherheit geschach,\\ 
 & \textbf{dâz} manec \textbf{werder rîter} sach.\\ 
5 & "Hêr künec, nû habt ir missetân,\\ 
 & sol mîn rîter sîn ein koufman,\\ 
 & des mich mîn swester vil an streit,\\ 
 & daz irme gâbet sicherheit",\\ 
 & sprach diu \textbf{junge} Obylot.\\ 
10 & Melyanze si dâ nâch gebôt,\\ 
 & daz er sicherheit verjæhe,\\ 
 & di\textit{u} in ir hant geschæhe,\\ 
 & ir swester Obien.\\ 
 & "zeiner âmîen\\ 
15 & sult ir si \textbf{hân} durch rîters prîs.\\ 
 & zeinem hêrren, zeinem âmîs\\ 
 & sol siuch iemer gerne hân.\\ 
 & i\textbf{ne} wils iuch \textbf{wederhalp} \textbf{erlân}."\\ 
 & Got ûz \textbf{ir jungem} munde sprach.\\ 
20 & ir bete beidenthalp geschach.\\ 
 & Dô meisterte v\textit{rou} minne\\ 
 & mit \textbf{vriuntlîchem} sinne\\ 
 & unde \textbf{herzeliebiu} triuwe\\ 
 & der zweier \textbf{liep} al niuwe.\\ 
25 & Obyen hant vür den mantel sleif,\\ 
 & Melyanzes \textbf{arm si} begreif\\ 
 & \textbf{unde druhtin an} ir \textbf{rôten} munt,\\ 
 & \textbf{al}dâ \textbf{er} \textbf{was} \textbf{ze}\textbf{r} tjost wunt.\\ 
 & manec zaher \textbf{im} den arm begôz,\\ 
30 & der von ir liehten ougen vlôz.\\ 
\end{tabular}
\scriptsize
\line(1,0){75} \newline
T V W \newline
\line(1,0){75} \newline
\textbf{1} \textit{Initiale} W  \textbf{19} \textit{Majuskel} T  \textbf{21} \textit{Majuskel} T  \newline
\line(1,0){75} \newline
\textbf{1} Melyanz] Melianz V  $\cdot$ daz dar] in W \textbf{2} Gawanen] Gawan V  $\cdot$ vaste] \textit{om.} W \textbf{3} Obylote] Obelote T Obiloten V Obylot W  $\cdot$ dâ] do V W \textbf{4} dâz] Das ez V (W) \textbf{9} sprach] Sus sprach V W  $\cdot$ Obylot] obilot V \textbf{10} Melyanze] Melianze V Melianz W \textbf{11} sicherheit] der sicherheit V \textbf{12} diu] die T Vnde die V \textbf{13} ir] [Jrre]: Mine V  $\cdot$ Obien] Obyen T (W) \textbf{15} hân] nemen W  $\cdot$ rîters] ritter W \textbf{16} hêrren] [herr*]: herren vnde V herren vnd W \textbf{17} sol siuch] solsiv T Sol ich in eúch W \textbf{18} wils iuch wederhalp] wilsiv weder halp T wils v́ch dewederhalb V wil es wederthalb eúch nit W \textbf{21} meisterte] maistert W  $\cdot$ vrou] vor T \textbf{23} herzeliebiu] herzenliche V (W) \textbf{25} Obyen] Obẏen V \textbf{26} Melyanzes] Melianzes V W \textbf{27} druhtin] druckt in W \textbf{29} zaher] trehen V \newline
\end{minipage}
\end{table}
\end{document}
