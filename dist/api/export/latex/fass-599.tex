\documentclass[8pt,a4paper,notitlepage]{article}
\usepackage{fullpage}
\usepackage{ulem}
\usepackage{xltxtra}
\usepackage{datetime}
\renewcommand{\dateseparator}{.}
\dmyyyydate
\usepackage{fancyhdr}
\usepackage{ifthen}
\pagestyle{fancy}
\fancyhf{}
\renewcommand{\headrulewidth}{0pt}
\fancyfoot[L]{\ifthenelse{\value{page}=1}{\today, \currenttime{} Uhr}{}}
\begin{document}
\begin{table}[ht]
\begin{minipage}[t]{0.5\linewidth}
\small
\begin{center}*D
\end{center}
\begin{tabular}{rl}
\textbf{599} & \begin{large}Û\end{large}f strîtes gedense;\\ 
 & daz \textbf{tæte} iu wê zer gense.\\ 
 & iu mac durch \textbf{rüemen} wesen liep\\ 
 & der schilt dürkel als ein sip,\\ 
5 & den iu sô manec pfîl \textbf{zerbrach}.\\ 
 & \textbf{an} disen zîten ungemach\\ 
 & \textbf{muget} ir gerne vliehen.\\ 
 & lât iu den vinger ziehen,\\ 
 & Rîtet wider ûf zen vrouwen.\\ 
10 & wie get\textit{ö}rst ir \textbf{geschouwen}\\ 
 & strît, den ich werben solde,\\ 
 & ob iwer herze wolde\\ 
 & mir dienen nâch minne."\\ 
 & Er sprach zer \textbf{herzoginne}:\\ 
15 & "vrouwe, \textbf{hân ich} wunden,\\ 
 & die hânt \textbf{hie} helfe vunden.\\ 
 & ob \textbf{iwer} helfe kan gezemen,\\ 
 & daz ir mîn dienst \textbf{ruochet} nemen,\\ 
 & sô wart nie nôt sô hert erkant,\\ 
20 & i\textbf{ne} sî ze dienste \textbf{iu} \textbf{dar} benant."\\ 
 & Si sprach: "ich lâze iuch rîten,\\ 
 & mêr nâch prîse strîten\\ 
 & mit mir geselleclîche."\\ 
 & des wart an vreuden rîche\\ 
25 & der stolze, werde Gawan.\\ 
 & den Turkoten sant er \textbf{dan}\\ 
 & \textbf{mit} sînem wirte Plippalinot\\ 
 & ûf die burc. er enbôt,\\ 
 & daz \textbf{sîn} mit \textbf{wirde} næmen war\\ 
30 & \textbf{al die} vrouwen wol gevar.\\ 
\end{tabular}
\scriptsize
\line(1,0){75} \newline
D Z \newline
\line(1,0){75} \newline
\textbf{1} \textit{Initiale} D  \textbf{5} \textit{Initiale} Z  \textbf{9} \textit{Majuskel} D  \textbf{14} \textit{Majuskel} D  \textbf{21} \textit{Majuskel} D  \newline
\line(1,0){75} \newline
\textbf{2} tæte] tut Z \textbf{3} rüemen] rvm Z  $\cdot$ liep] lip D [lop]: liep Z \textbf{4} sip] schiep Z \textbf{10} getörst] getorst D Z \textbf{14} herzoginne] kuniginne Z \textbf{16} hie] \textit{om.} Z \textbf{18} mîn dienst ruochet] minen dienst geruchet Z \textbf{22} mêr] Vnd mer Z \textbf{23} mir] mit Z \textbf{26} Turkoten] Tvrkoiten Z \textbf{27} mit] Bi Z  $\cdot$ Plippalinot] plipalinot Z \textbf{29} \textit{Versfolge 599.30-29} Z   $\cdot$ sîn] sie sin Z \textbf{30} die] den Z \newline
\end{minipage}
\hspace{0.5cm}
\begin{minipage}[t]{0.5\linewidth}
\small
\begin{center}*m
\end{center}
\begin{tabular}{rl}
 & ûf strîtes gedense;\\ 
 & daz \textbf{tæte} iu wê zer gense.\\ 
 & iu mac durch \textbf{rüemen} wesen liep\\ 
 & der schilt dü\textit{r}kelt als ein sip,\\ 
5 & den iu sô manic pfîl \textbf{zerbrach}.\\ 
 & \textbf{zuo} disen zîten ungemach\\ 
 & \textbf{m\textit{ö}ht} ir gerne vliehen.\\ 
 & lât iu den vinger ziehen,\\ 
 & rîtet wider ûf zen vrouwen.\\ 
10 & wie g\textit{e}t\textit{ö}rstet ir \textbf{beschouwen}\\ 
 & strît, den ich werben solte,\\ 
 & ob iuwer herz wolte\\ 
 & mir dienen nâch minne."\\ 
 & er sprach zer \textbf{herzoginne}:\\ 
15 & "vrouwe, \textbf{hab ich} wunden,\\ 
 & die hânt \textbf{hie} helfe vunden.\\ 
 & ob \textbf{iuwer} helfe kan gezemen,\\ 
 & daz ir mîn dienst \textbf{ruochet} nemen,\\ 
 & sô wart \textit{n}ie nôt sô herte erkant,\\ 
20 & ich sî zuo dienst \textbf{iu} benant."\\ 
 & si sprach: "ich lâz iuch rîten,\\ 
 & mê nâch prîse strîten\\ 
 & mit mir geselleclîch."\\ 
 & des wart an vröuden rîch\\ 
25 & der stolz, werde Gawan.\\ 
 & den \textit{T}ur\textit{k}oiten s\textit{an}ter \textbf{dan}\\ 
 & \textbf{mit} sînem wirt Plipp\textit{a}l\textit{in}ot\\ 
 & ûf die burc. er enbôt,\\ 
 & daz \textbf{sîn} mit \textbf{wirde} næmen war\\ 
30 & \textbf{alle die} vrouwen wol gevar.\\ 
\end{tabular}
\scriptsize
\line(1,0){75} \newline
m n o \newline
\line(1,0){75} \newline
\newline
\line(1,0){75} \newline
\textbf{3} wesen] wen o \textbf{4} dürkelt] durchkelt m \textbf{5} sô] do n \textbf{7} möht] Moht m (o) \textbf{9} rîtet] Riten o \textbf{10} getörstet] gestorstet m getursten n \textbf{14} herzoginne] herczoginnen o \textbf{18} ruochet] rúchen n \textbf{19} nie] mẏe m \textbf{24} des] Das o  $\cdot$ vröuden] freiuden o \textbf{26} Turkoiten] kurtoitten m (n) kortoiten o  $\cdot$ santer] snitter m \textbf{27} Plippalinot] plippolmot m n plippolinat o \textbf{29} mit] nit o \newline
\end{minipage}
\end{table}
\newpage
\begin{table}[ht]
\begin{minipage}[t]{0.5\linewidth}
\small
\begin{center}*G
\end{center}
\begin{tabular}{rl}
 & ûf strîtes gedense;\\ 
 & daz \textbf{tuot} iu wê zer gense.\\ 
 & iu mac durch \textbf{ruom wol} wesen liep\\ 
 & der schilt dürkel als ein sip,\\ 
5 & den iu sô manic pfîl \textbf{brach}.\\ 
 & \textbf{an} disen zîten ungemach,\\ 
 & \textbf{daz} \textbf{muget} ir gerne vliehen.\\ 
 & lât iu den vinger ziehen,\\ 
 & rîtet wider ûf ze den vrouwen.\\ 
10 & \textbf{saget}, wie get\textit{ö}rst ir \textbf{schouwen}\\ 
 & strît, den ich werben solde,\\ 
 & ob iuwer herze wolde\\ 
 & mir dienen nâch minne."\\ 
 & er sprach ze der \textbf{küneginne}:\\ 
15 & "vrouwe, \textbf{ich hân} \textit{w}unden,\\ 
 & die hânt helfe \textit{v}unden.\\ 
 & ob \textbf{iuch} \textit{helfe} kan gezemen,\\ 
 & daz ir mîn dienst \textbf{geruochet} nemen,\\ 
 & sô\textbf{ne} wart nie nôt sô hert erkant,\\ 
20 & i\textbf{ne} sî ze dieneste \textbf{dar} benant."\\ 
 & si sprach: "ich lâz iuch rîten\\ 
 & \textbf{unde} mê nâch prîse strîten\\ 
 & mit mir geselleclîche."\\ 
 & des wart an vröuden rîche\\ 
25 & der stolze, werde Gawan.\\ 
 & \textit{den} \textit{Turk}oi\textit{t}en sande er \textbf{sân}\\ 
 & \textbf{bî} sîneme wirte Pliplalinot\\ 
 & ûf die burc. er enbôt\\ 
30 & \hspace*{-.7em}\big| \textbf{\textit{d}en} vrouwen wolgevar,\\ 
 & \hspace*{-.7em}\big| daz si\textbf{s} mit \textbf{wirden} næmen war.\\ 
\end{tabular}
\scriptsize
\line(1,0){75} \newline
G I L M Z \newline
\line(1,0){75} \newline
\textbf{5} \textit{Initiale} I L M Z  \textbf{21} \textit{Initiale} I  \newline
\line(1,0){75} \newline
\textbf{3} wol] \textit{om.} L M Z  $\cdot$ liep] [lop]: liep Z \textbf{4} schilt] shilt ist I  $\cdot$ sip] schiep Z \textbf{5} brach] durc [bl]: brach I zerbrach L (M) (Z) \textbf{7} daz] \textit{om.} L Z \textbf{9} rîtet] \textit{om.} I  $\cdot$ ûf] \textit{om.} I \textbf{10} saget] \textit{om.} Z  $\cdot$ getörst ir] getorst ir G L Z [gerst]: getorst I gesi tostir M  $\cdot$ schouwen] ie beshawen I geschouwen M Z \textbf{13} minne] mýnnen L (M) \textbf{14} küneginne] hertzoginen L (M) \textbf{15} ich hân] han ich L M Z  $\cdot$ wunden] funden G \textbf{16} vunden] wunden G enphunden I \textbf{17} helfe] \textit{om.} G ewer helfe Z \textbf{18} geruochet] ruͤchet I (L) geruget M \textbf{19} sône] So Z \textbf{20} dar] ev dar Z \textbf{22} prîse] [strite]: prise I \textbf{23} mir] mit Z \textbf{24} wart an] anden M \textbf{26} den Turkoiten] lyshoisen G den Turchoyden I Lýtschosen L Lysoien M  $\cdot$ sân] dan L Z soen M \textbf{27} Pliplalinot] plipalinot I L M Z \textbf{30} den] Nach den G Al den L (M) Z \textbf{29} sis] si sin I (Z)  $\cdot$ mit wirden] \textit{om.} I mit wirde L Z  $\cdot$ næmen] nemen wol I namen L \newline
\end{minipage}
\hspace{0.5cm}
\begin{minipage}[t]{0.5\linewidth}
\small
\begin{center}*T
\end{center}
\begin{tabular}{rl}
 & ûf strîtes gedense;\\ 
 & daz \textbf{tuot} iu wê zuor gense.\\ 
 & iu mac durch \textbf{ruom} wesen liep\\ 
 & der schilt dürkel als ein sip,\\ 
5 & den iu sô manic pfî\textit{l} \textbf{zerbrach}.\\ 
 & \textbf{an} disen zîten ungemach,\\ 
 & \textbf{daz} \textbf{mugt} ir gerne vliehen.\\ 
 & lât iu den vinger ziehen,\\ 
 & rît wider ûf zuon vrouwen.\\ 
10 & \textbf{sagt}, wie get\textit{ö}rst ir \textbf{schouwen}\\ 
 & strît, den ich werben solde,\\ 
 & ob iuwer herze wolde\\ 
 & mir dienen nâch minne."\\ 
 & er sprach zuor \textbf{herzoginne}:\\ 
15 & "vrou, \textbf{hân ich} wunden,\\ 
 & die hânt helfe vunden.\\ 
 & ob \textbf{iuch} helfe kan gezemen,\\ 
 & daz ir mîn dienst \textbf{geruochet} nemen,\\ 
 & sô wart nie nôt sô hert erkant,\\ 
20 & ich\textbf{n} sî zuo dienste \textbf{dar} bena\textit{nt}."\\ 
 & si sprach: "ich lâz iuch rîten\\ 
 & \textbf{und} mêr nâch prîse strîten\\ 
 & mit mir geselleclîche."\\ 
 & des wart an vreuden rîche\\ 
25 & der stolze, werde Gawan.\\ 
 & den Turkoyten sant er \textbf{dan}\\ 
 & \textbf{bî} sînem wirte Plipalinot\\ 
 & ûf die burc. er enbôt\\ 
30 & \hspace*{-.7em}\big| \textbf{al den} vrouwen wol gevar,\\ 
 & \hspace*{-.7em}\big| daz si\textbf{s} mit \textbf{wirden} næmen war.\\ 
\end{tabular}
\scriptsize
\line(1,0){75} \newline
Q R W V U \newline
\line(1,0){75} \newline
\textbf{5} \textit{Capitulumzeichen} R  \textbf{21} \textit{Initiale} V  \newline
\line(1,0){75} \newline
\textbf{1} \textit{Die Verse 553.1-599.30 fehlen} U  \textbf{4} dürkel] dunckel R dúrckelt W [dunkelt]: durkel V \textbf{5} pfîl] pfeyle Q \textbf{10} [*]: Wie getorstent ir geschowe V  $\cdot$ getörst] getorst Q R (V) \textbf{13} minne] mine W \textbf{15} vrou] Erawe W \textbf{16} vunden] [*]: hie fvnden V \textbf{17} iuch] [*]: uwer V  $\cdot$ kan] mag R [*]: kan V \textbf{18} mîn dienst] nun hilffe W minen dienst V  $\cdot$ geruochet] geruͦchen R \textbf{19} wart] enwart R (W) V \textbf{20} ichn] Jch R (W)  $\cdot$ dar benant] dar benam Q [*]: v́ch dar benant V \textbf{24} rîche] reichen Q \textbf{25} Gawan] gewan R \textbf{26} Turkoyten] turkoiten Q tvrkorten V  $\cdot$ sant er] er sante V \textbf{30} al den] Allen R W \textbf{29} sis mit] sies nit R sir mit W sv́ sin mit V  $\cdot$ wirden] wirde W (V) \newline
\end{minipage}
\end{table}
\end{document}
