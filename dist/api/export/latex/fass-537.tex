\documentclass[8pt,a4paper,notitlepage]{article}
\usepackage{fullpage}
\usepackage{ulem}
\usepackage{xltxtra}
\usepackage{datetime}
\renewcommand{\dateseparator}{.}
\dmyyyydate
\usepackage{fancyhdr}
\usepackage{ifthen}
\pagestyle{fancy}
\fancyhf{}
\renewcommand{\headrulewidth}{0pt}
\fancyfoot[L]{\ifthenelse{\value{page}=1}{\today, \currenttime{} Uhr}{}}
\begin{document}
\begin{table}[ht]
\begin{minipage}[t]{0.5\linewidth}
\small
\begin{center}*D
\end{center}
\begin{tabular}{rl}
\textbf{537} & \begin{large}N\end{large}û diz was unwendec.\\ 
 & der komende was genendec.\\ 
 & alsô was ouch, der dâ beite.\\ 
 & zer tjost er sich bereite.\\ 
5 & dô sazter die glevîn\\ 
 & \textbf{vorn} ûf des satels vilzelîn,\\ 
 & \textbf{des} Gawan vor het \textbf{erdâht}.\\ 
 & sus \textbf{wart} ir bêder tjoste brâht.\\ 
 & diu tjost \textbf{ieweder} sper zerbrach,\\ 
10 & \textbf{daz} man die helde ligen sach.\\ 
 & Dô strûchte der \textbf{baz} geriten man,\\ 
 & daz er unt \textbf{mîn hêr} Gawan\\ 
 & ûf den bluomen lâgen.\\ 
 & wes si dô bêde pflâgen?\\ 
15 & \textbf{ûf springens} mit den swerten.\\ 
 & si bêde \textbf{strîtes} gerten.\\ 
 & die schilde wâren unvermiten.\\ 
 & \textbf{die} wurden alsô hin gesniten,\\ 
 & \textbf{ir beleip in} \textbf{lützel} vor der hant,\\ 
20 & wan der schilt ist immer strîtes pfant.\\ 
 & man sach dâ \textbf{blicke} unt helmes viuwer.\\ 
 & ir megts im jehen vür âventiuwer,\\ 
 & swen got den sic \textbf{dan} læzet tragen,\\ 
 & der muoz vil prîses ê bejagen.\\ 
25 & Sus \textbf{tûwerten} si mit strîte\\ 
 & ûf des angers wîte.\\ 
 & es wæren müede zwêne smide,\\ 
 & ob si halt \textbf{heten} \textbf{starker} lide,\\ 
 & von alsô manegen \textbf{grôzem} slage.\\ 
30 & sus rungen si nâch prîses bejage.\\ 
\end{tabular}
\scriptsize
\line(1,0){75} \newline
D Fr31 \newline
\line(1,0){75} \newline
\textbf{1} \textit{Initiale} D  \textbf{11} \textit{Majuskel} D  \textbf{25} \textit{Majuskel} D  \newline
\line(1,0){75} \newline
\textbf{29} manegen grôzem] mangem Fr31 \newline
\end{minipage}
\hspace{0.5cm}
\begin{minipage}[t]{0.5\linewidth}
\small
\begin{center}*m
\end{center}
\begin{tabular}{rl}
 & nû diz was unwendic.\\ 
 & der komende was genendic.\\ 
 & alsô was ouch, der d\textit{â} beite.\\ 
 & zuor juste er sich bereite.\\ 
5 & dô sat er die glevîn\\ 
 & \textbf{v\textit{or}nen} ûf des satel\textit{s} vilzelîn.\\ 
 & \textbf{daz} Gawan vor het \textbf{gedâht}.\\ 
 & sus \textbf{wart} ir b\textit{ei}der juste brâht.\\ 
 & diu just \textbf{ietweder} sper zerbrach,\\ 
10 & \textbf{daz} man die helde ligen sach.\\ 
 & dô strûhte der \textbf{\textit{b}az} geriten man,\\ 
 & daz er und \textbf{ouch} Gawan\\ 
 & \textbf{beide} ûf den bluomen lâgen.\\ 
 & wes si dô beid\textit{e} \textit{p}flâgen?\\ 
15 & \textbf{ûf spr\textit{i}nge\textit{n}s} mit den swerten.\\ 
 & si beide \textbf{strîtes} gerten.\\ 
 & die schilt wâren unvermiten\\ 
 & \textbf{und} wurden alsô hin gesniten,\\ 
 & \textbf{in bleip} \textbf{lützel} vor der hant;\\ 
20 & wan der schilt ist iemer strîtes pfant.\\ 
 & man sach d\textit{â} \textbf{buoc} und helmes viur.\\ 
 & ir megt es im jehen vür âventiur,\\ 
 & wen got den sige \textbf{dannen} l\textit{â}t tragen,\\ 
 & der muoz vil prîses ê bejagen.\\ 
25 & sus \textbf{tûrten} si mit strîte\\ 
 & ûf d\textit{e}s angers wîte.\\ 
 & es w\textit{æ}ren müede zwên s\textit{m}ide,\\ 
 & ob si halt \textbf{heten} \textbf{starker} lide,\\ 
 & von alsô manige\textit{m} \textbf{grôzem} slage.\\ 
30 & sus rungen si nâch prîses bejage.\\ 
\end{tabular}
\scriptsize
\line(1,0){75} \newline
m n o \newline
\line(1,0){75} \newline
\newline
\line(1,0){75} \newline
\textbf{1} diz] des o \textbf{3} dâ] do m n o  $\cdot$ beite] beit o \textbf{6} vornen] Fronen m  $\cdot$ des satels] des sattel m das sattel o \textbf{7} daz] Do o \textbf{8} beider] bruͯder m beide o \textbf{11} baz] has m \textbf{13} lâgen] gelogen n (o) \textbf{14} dô] da o  $\cdot$ beide pflâgen] beẏde clagen vnd pflagen m \textbf{15} springens] sprunges m \textbf{21} dâ] do m n o  $\cdot$ buoc] blick n o \textbf{23} lât] lut m \textbf{24} muoz] muͯs m \textbf{26} des] das m o \textbf{27} wæren] woren m werdent n  $\cdot$ smide] snẏde m \textbf{28} lide] gelide n \textbf{29} manigem] manig ein m manigen o  $\cdot$ grôzem] groszen o \newline
\end{minipage}
\end{table}
\newpage
\begin{table}[ht]
\begin{minipage}[t]{0.5\linewidth}
\small
\begin{center}*G
\end{center}
\begin{tabular}{rl}
 & \begin{large}N\end{large}û diz was unwendic.\\ 
 & der komende was ge\textit{ne}ndic.\\ 
 & als was ouch, der dâ beite.\\ 
 & zer tjost er sich bereite.\\ 
5 & dô sazete er die glevîn\\ 
 & \textbf{vor} ûf des satels vilzelîn,\\ 
 & \textbf{des} Gawan vor hete \textbf{erdâht}.\\ 
 & sus \textbf{was} ir bêder tjoste brâht.\\ 
 & diu tjoste \textbf{ietweder} sper zerbrach,\\ 
10 & \textbf{daz} man die helde ligen sach.\\ 
 & dô strû\textit{h}te der \textbf{baz} geriten man,\\ 
 & daz er unde \textbf{mîn hêrre} Gawan\\ 
 & ûf den bluomen lâgen.\\ 
 & wes si dô bêde pflâgen?\\ 
15 & \textbf{ûfsprunges} mit den swerten.\\ 
 & si bêde \textbf{strîtes} gerten.\\ 
 & die schilde wâren unvermiten.\\ 
 & \textbf{die} wurden alsô hin gesniten,\\ 
 & \textbf{i\textit{r} beleip i\textit{n}} \textbf{lützel} vor der hant,\\ 
20 & wan der schilt ist immer strîtes pfant.\\ 
 & man sach dâ \textbf{blicke} unde helmes viur.\\ 
 & ir mugts im jehen vür âventiur,\\ 
 & swen got den sic \textbf{da\textit{nnân}} læzet tragen,\\ 
 & der muoz vil brîses ê bejagen.\\ 
25 & sus \textbf{tûrten} si mit strîte\\ 
 & ûf des angers wîte.\\ 
 & es \textit{wæren müede} zwêne smide,\\ 
 & op si halt \textbf{heten} \textbf{starker} lide,\\ 
 & von alsô \textit{manigem} \textbf{grôzem} slage.\\ 
30 & sus rungen si nâch prîses bejage.\\ 
\end{tabular}
\scriptsize
\line(1,0){75} \newline
G I L M Z Fr19 \newline
\line(1,0){75} \newline
\textbf{1} \textit{Initiale} G I L Z Fr19  \textbf{21} \textit{Initiale} I  \newline
\line(1,0){75} \newline
\textbf{2} komende] chune I  $\cdot$ was] \textit{om.} L  $\cdot$ genendic] gendech G \textbf{3} was] \textit{om.} I  $\cdot$ ouch] \textit{om.} M  $\cdot$ dâ] \textit{om.} M \textbf{5} dô] Da L M  $\cdot$ glevîn] klavelin I \textbf{6} vor ûf] Vorne vf L (Z) (Fr19) Vorne M  $\cdot$ des] daz L \textbf{7} erdâht] gedaht I L (M) \textbf{8} was] wart I L Z \textbf{11} dô] Da Z  $\cdot$ strûhte] strufte G struͯch L \textbf{12} unde] vnder I  $\cdot$ hêrre Gawan] hern Gawan I ergawan M \textbf{14} dô] da M Z \textbf{15} ûfsprunges] vf sprungens I (L) (Z) Vff Springens M ::: gens Fr19  $\cdot$ den swerten] deme swerte M \textbf{17} unvermiten] vnfer miten G \textbf{18} die] si I  $\cdot$ wurden] warn Z  $\cdot$ hin gesniten] hin versniten I versnîten Fr19 \textbf{19} ir beleip in] in beleip ir G \textbf{21} man] Mn I  $\cdot$ dâ] swert L  $\cdot$ blicke unde helmes] blichen von helmen I \textbf{22} mugts] mugt sin I mvgsz L  $\cdot$ im] in I \textbf{23} swen] Wen L M  $\cdot$ dannân] da G \textbf{24} brîses] brise I strites L \textbf{25} tûrten] tuwierten I \textbf{27} wæren müede zwêne] muede warin zwene G weren zcwene Mude M \textbf{29} alsô] als I  $\cdot$ manigem grôzem] grozem G mangen groszen L mannigeme M grozzem manigem Z manigem grozen Fr19 \newline
\end{minipage}
\hspace{0.5cm}
\begin{minipage}[t]{0.5\linewidth}
\small
\begin{center}*T
\end{center}
\begin{tabular}{rl}
 & Nû diz was unwendic.\\ 
 & der komende was genendic.\\ 
 & alse was ouch, der dâ beite.\\ 
 & zer tjoste er sich bereite.\\ 
5 & dô sazter die glevîn\\ 
 & \textbf{vor} ûf des satels vilzelîn,\\ 
 & \textbf{des} Gawan vor hâte \textbf{gedâht}.\\ 
 & sus \textbf{wart} ir beider tjost brâht.\\ 
 & diu tjost \textbf{ietweders} sper zerbrach,\\ 
10 & \textbf{dô} man die helde ligen sach.\\ 
 & dô strûhte der \textbf{wol} geritene man,\\ 
 & daz er unde \textbf{mîn hêr} Gawan\\ 
 & ûf den bluomen lâgen.\\ 
 & wes si dô beide pflâgen?\\ 
15 & \textbf{ûf springens} mit den swerten.\\ 
 & \textbf{prîses} si beide gerten.\\ 
 & die schilte wâren unvermiten.\\ 
 & \textbf{die} wurden alsô hin gesniten,\\ 
 & \textbf{in bleip ir} \textbf{wênic} vor der hant,\\ 
20 & wan der schilt ist iemer strîtes pfant.\\ 
 & man sach dâ \textbf{blicke} unde helmes viur.\\ 
 & ir muget\textit{s} im jehen vür âventiur,\\ 
 & swen got den sic \textbf{dâ} lât tragen,\\ 
 & der muoz vil prîses ê bejagen.\\ 
25 & sus \textbf{dructen} si mit strîte\\ 
 & ûf des angers wîte.\\ 
 & es wæren müede zwêne smide,\\ 
 & ob si halt \textbf{trüegen} \textbf{starkiu} lide,\\ 
 & von alsô manegem \textbf{starken} slage.\\ 
30 & sus rungen si nâch prîses bejage.\\ 
\end{tabular}
\scriptsize
\line(1,0){75} \newline
T U V W O Q R \newline
\line(1,0){75} \newline
\textbf{1} \textit{Initiale} W O R   $\cdot$ \textit{Majuskel} T  \newline
\line(1,0){75} \newline
\textbf{1} Nû] ÷o O Do Q \textbf{2} genendic] gendig R \textbf{3} ouch] \textit{om.} V  $\cdot$ der] der der W  $\cdot$ dâ] do U V W \textbf{4} bereite] gar eben beraite W breite R \textbf{5} dô] Da O \textbf{6} vor] Vorn W (Q) Vornan R  $\cdot$ des satels] daz satel O den satel R  $\cdot$ vilzelîn] sin R \textbf{7} des] Daz V  $\cdot$ Gawan] gawin R  $\cdot$ vor] do vor U  $\cdot$ gedâht] erdacht U (O) R \textbf{8} sus] Als Q  $\cdot$ brâht] volbracht R \textbf{9} ietweders] beider site des U ietweder W O (Q) (R)  $\cdot$ sper] spers U  $\cdot$ zerbrach] brach W \textbf{10} dô] [*]: Daz V Das W (O) Q R  $\cdot$ ligen] beid ligen W \textbf{11} strûhte] [*]: struhte V strauchet W (Q)  $\cdot$ wol] [*]: wol V \textbf{12} mîn hêr] [*]: min her V \textbf{14} wes] Waz V \textbf{15} springens] sprvngen sv́ V (R) \textbf{16} si] sy do R \textbf{18} wurden] frewde Q  $\cdot$ gesniten] verschnitten W \textbf{19} in bleip ir] Jr bleip U Jr bleip in V (Q) Auß belaib in W  $\cdot$ vor] in O \textbf{20} iemer] \textit{om.} U \textbf{21} dâ] do U V W Q R  $\cdot$ blicke] blichen O blikre R  $\cdot$ unde] von R  $\cdot$ helmes] helles W \textbf{22} mugets im] mvgetz im T (U) (Q) (R) moͤgentz [*]: im  V mvgt ims O \textbf{23} swen] Wen U W R Wem Q  $\cdot$ den] nv den V  $\cdot$ dâ] \textit{om.} U dan V W O Q \textbf{24} prîses] [*]: prises V  $\cdot$ ê] \textit{om.} Q  $\cdot$ bejagen] eriagen W \textbf{25} sus] Als Q  $\cdot$ dructen] turten U W (O) (R) vahtent V riten Q \textbf{27} wæren] were Q \textbf{28} halt] \textit{om.} V  $\cdot$ trüegen] heten U (V) (W) O (Q) (R)  $\cdot$ starkiu] stercher O  $\cdot$ lide] glide W \textbf{29} alsô manegem] manegem also U (V) (W) (O) (Q) (R)  $\cdot$ starken] starke U starchem O (Q)  $\cdot$ slage] clage V \textbf{30} sus] Als Q \newline
\end{minipage}
\end{table}
\end{document}
