\documentclass[8pt,a4paper,notitlepage]{article}
\usepackage{fullpage}
\usepackage{ulem}
\usepackage{xltxtra}
\usepackage{datetime}
\renewcommand{\dateseparator}{.}
\dmyyyydate
\usepackage{fancyhdr}
\usepackage{ifthen}
\pagestyle{fancy}
\fancyhf{}
\renewcommand{\headrulewidth}{0pt}
\fancyfoot[L]{\ifthenelse{\value{page}=1}{\today, \currenttime{} Uhr}{}}
\begin{document}
\begin{table}[ht]
\begin{minipage}[t]{0.5\linewidth}
\small
\begin{center}*D
\end{center}
\begin{tabular}{rl}
\textbf{436} & \begin{large}D\end{large}urch minne, diu an im erstarp,\\ 
 & daz si der vürste niht erwarp,\\ 
 & si minnete sînen tôten lîp.\\ 
 & ob si worden wære sîn wîp,\\ 
5 & \textbf{dâ hete} sich vrou Lunete\\ 
 & \textbf{gesûmet} an \textbf{sô} \textbf{gæher bete},\\ 
 & als si riet ir \textbf{selber} vrouwen.\\ 
 & man mac \textbf{doch} dicke schouwen\\ 
 & vroun Luneten rîten zuo\\ 
10 & etslîchem râte gar ze vruo.\\ 
 & Swelch wîp nû durch geselleschaft\\ 
 & und durch \textbf{ir} zühte kraft\\ 
 & pfliht \textbf{verbirt} an vremder minne,\\ 
 & als ich mich\textbf{s} versinne,\\ 
15 & læt si\textbf{z} bî ir mannes leben,\\ 
 & dem wart an ir der wunsch gegeben.\\ 
 & dechein beiten stêt ir alsô wol.\\ 
 & daz \textbf{erziuge} ich, \textbf{ob} ich sol.\\ 
 & dâr nâch tuo, als \textbf{siz} lêre.\\ 
20 & beheltet si dennoch êre,\\ 
 & si\textbf{ne} treit deheinen \textbf{sô liehten} kranz,\\ 
 & gêt si durch vreude an den tanz.\\ 
 & Wes \textbf{mizze} ich vreude \textbf{gein} der nôt,\\ 
 & als Sigunen ir triwe \textbf{gebôt}?\\ 
25 & daz m\textit{ö}ht ich gerne lâzen.\\ 
 & über ronen, âne strâzen\\ 
 & Parzival vürz venster reit\\ 
 & al ze nâhe, daz was im leit.\\ 
 & dô wolt er vrâgen umben walt,\\ 
30 & oder war sîn reise wære \textbf{gezalt}.\\ 
\end{tabular}
\scriptsize
\line(1,0){75} \newline
D Fr31 \newline
\line(1,0){75} \newline
\textbf{1} \textit{Initiale} Fr31  \newline
\line(1,0){75} \newline
\textbf{1} diu] div ir Fr31 \textbf{3} sînen] [*inen]: sinen D \textbf{8} doch] noch Fr31 \textbf{10} etslîchem râte] Ette::: raten Fr31 \textbf{12} und] Verbi:: vnd Fr31  $\cdot$ zühte] minne Fr31 \textbf{14} michs versinne] mich versenne Fr31 \textbf{15} bî ir] dvrch Fr31 \textbf{16} der] \textit{om.} Fr31 \textbf{19} als] si als ich Fr31 \textbf{20} beheltet] Beh:::tent Fr31  $\cdot$ êre] ir ere Fr31 \textbf{21} sine treit] Si tr:ait Fr31 \textbf{24} Sigunen] Sigvͦnen D \textbf{25} möht] moht D (Fr31) \textbf{27} Parzival] Parcifal D Parzifal Fr31  $\cdot$ vürz] fvͦr diz Fr31  $\cdot$ venster] venst Fr31 \newline
\end{minipage}
\hspace{0.5cm}
\begin{minipage}[t]{0.5\linewidth}
\small
\begin{center}*m
\end{center}
\begin{tabular}{rl}
 & durch minne, diu an im erstarp,\\ 
 & daz si der vürste niht erwarp,\\ 
 & si minnete sînen tôten lîp,\\ 
 & \textbf{als} ob si worden wære sîn wîp.\\ 
5 & \textbf{dô hette} sich vrouwe Lunette\\ 
 & \textbf{gesûmet} an \textbf{sô} \textbf{gæher bete},\\ 
 & als si riet ir \textbf{selber} vrouwen.\\ 
 & man mac \textbf{noch} dicke schouwen\\ 
 & vrouwe L\textit{un}etten rîten zuo\\ 
10 & etslîchem râte gar ze vruo.\\ 
 & wellich wîp nû durch geselleschaft\\ 
 & \textbf{verbirt} und durch \textbf{ir} zühte kraft\\ 
 & pfliht an vrömder minne,\\ 
 & als ich mich\textbf{s} versinne,\\ 
15 & lât si\textbf{z} bî ir mannes leben,\\ 
 & dem wart an ir der wunsch gegeben.\\ 
 & dekein beiten stât ir als \textbf{uns} wol.\\ 
 & daz \textbf{erziuge} ich, \textbf{ob} ich sol.\\ 
 & dâr nâch tuo, als \textbf{siz} lêre.\\ 
20 & behaltet si dannoch \textbf{ir} êre,\\ 
 & s\textit{iu} treit dekeinen \textbf{sollichen} kranz,\\ 
 & gât si durch \textit{vröude} an den tanz.\\ 
 & wes \textbf{wuns\textit{ch}} ich vröude \textbf{gegen} der nôt,\\ 
 & als Sigunen ir triuwe \textbf{bôt}?\\ 
25 & daz m\textit{ö}ht ich gerne lâzen.\\ 
 & über r\textit{o}nen, âne strâzen\\ 
 & Parcifal vür daz venster reit\\ 
 & alze nâhe, daz was im leit.\\ 
 & dô wolte er vrâgen umben walt,\\ 
30 & oder war sîn reise wær \textbf{gezalt}.\\ 
\end{tabular}
\scriptsize
\line(1,0){75} \newline
m n o \newline
\line(1,0){75} \newline
\newline
\line(1,0){75} \newline
\textbf{1} an] \textit{om.} n \textbf{5} Lunette] lúnete n luͯnete o \textbf{6} sô] [sin]: so o  $\cdot$ gæher] hoher n \textbf{7} selber] selbes n o \textbf{8} noch] uch o \textbf{9} Lunetten] limetten m lúneten n luͯneten o \textbf{12} zühte] zucht n \textbf{15} siz] sich o  $\cdot$ mannes] mannens m \textbf{16} gegeben] geben n o \textbf{17} dekein] Do kein n  $\cdot$ stât ir als uns] stat ir also n ir die stat o \textbf{19} siz] sich o \textbf{21} siu] So m  $\cdot$ treit] streit o  $\cdot$ dekeinen] so keinen o  $\cdot$ sollichen] so liechten n o \textbf{22} vröude] \textit{om.} m \textbf{23} wunsch] vns m \textbf{24} Sigunen] sigune o  $\cdot$ bôt] gebot n o \textbf{25} möht] moht m (o)  $\cdot$ ich] es o \textbf{26} ronen] \sout{ni} runen m \textbf{30} gezalt] gestalt o \newline
\end{minipage}
\end{table}
\newpage
\begin{table}[ht]
\begin{minipage}[t]{0.5\linewidth}
\small
\begin{center}*G
\end{center}
\begin{tabular}{rl}
 & \begin{large}D\end{large}urch minne, diu an im erstarp,\\ 
 & daz si der vürste niht erwarp,\\ 
 & si minnete s\textit{î}nen tôten lîp.\\ 
 & ob si worden wære sîn wîp,\\ 
5 & \textbf{dâ hete} sich vrô Lunet\\ 
 & \textbf{gesûmet} an \textbf{ir} \textbf{gâhen bet},\\ 
 & als si riet ir \textbf{selben} vrouwen.\\ 
 & man mac \textbf{noch} dicke schouwen\\ 
 & vrouwe Luneten rîten zuo\\ 
10 & etslîchem r\textit{â}t gar ze vruo.\\ 
 & swelch wîb nû durch geselleschaft\\ 
 & \textbf{verbirt} und durch \textbf{ir} zühte kraft\\ 
 & pflihte an vrömeder minne,\\ 
 & als ich mi\textit{ch} \textit{ve}rsinne,\\ 
15 & lât si\textbf{s} bî ir mannes leben,\\ 
 & dem wart an ir der wunsch gegeben.\\ 
 & dehein beiten stêt ir als wol.\\ 
 & daz \textbf{erziuge} \textit{ich}, \textbf{als} ich sol.\\ 
 & dâr nâch tuo, als \textbf{ez} lêre.\\ 
20 & behalt si dannoch êre,\\ 
 & si\textbf{ne} treit deheinen \textbf{sô liehten} kranz,\\ 
 & gêt si durch vröude an den tanz.\\ 
 & wes \textbf{mizze} ich vröude \textbf{zuo} der nôt,\\ 
 & als Sigunen ir triuwe \textbf{gebôt}?\\ 
25 & daz m\textit{ö}hte ich gerne lâzen.\\ 
 & über ronen, ân strâzen\\ 
 & Parcival vür daz venster reit\\ 
 & al ze nâhen, daz was im leit.\\ 
 & dô wolde er vrâgen umbe den walt,\\ 
30 & ode war sîn reise wære \textbf{gezalt}.\\ 
\end{tabular}
\scriptsize
\line(1,0){75} \newline
G I L M Z Fr25 \newline
\line(1,0){75} \newline
\textbf{1} \textit{Initiale} I L Z  \textbf{17} \textit{Initiale} I M  \newline
\line(1,0){75} \newline
\textbf{1} minne] libe M \textit{om.} Z \textbf{3} minnete] mẏnnet L (Z) meynte M  $\cdot$ sînen] senen G \textbf{5} Lunet] luͯnete M \textbf{6} gesûmet] Gesuͯnet M  $\cdot$ ir gâhen] so gaher L M (Z) \textbf{7} selben] selber I L (M) Z \textbf{9} vrouwe] froͮn I (Z)  $\cdot$ Luneten] luͯneten M \textbf{10} etslîchem] ethslichen I  $\cdot$ rât] rait G  $\cdot$ gar] al I \textbf{11} swelch] Welch L (M) \textbf{12} verbirt] Wirbet M  $\cdot$ ir] \textit{om.} M \textbf{13} pflihte] phliget I (L)  $\cdot$ vrömeder] vorderr I \textbf{14} mich versinne] mi:: ursinne G michs ver sinne Z \textbf{15} ir] irsz M \textbf{16} wart] wirt L  $\cdot$ an ir der wunsch] der wunsh an ir I \textbf{17} dehein] Kein Z  $\cdot$ beiten] leiten L  $\cdot$ ir als] so I \textbf{18} ich als ich] als ih G ich ab ich M (Z) \textbf{19} ez] sis L (M) (Z) \textbf{20} behalt] Behildet M  $\cdot$ êre] \sout{als ez lere} [*]: ere G ir ere I \textbf{21} deheinen] dehain I (L) (Fr25) icheinen M keinen Z  $\cdot$ sô] also Z  $\cdot$ liehten] lichten L liben M \textbf{22} vröude] \textit{om.} Fr25 \textbf{23} zuo der] gein der L M (Fr25) gein Z \textbf{24} Sigunen] sigenun G Sygvnen L (M)  $\cdot$ ir] \textit{om.} I \textbf{25} möhte] mohte G L (M) (Z) \textbf{26} ronen ân] ron vnd ane I ronen vnd vber L wegk vnd ubir M \textbf{27} Parcival] Parziual G [parzifal]: Parzifal I Parzifal L M Parcifal Z :arcifal Fr25  $\cdot$ daz] [dro]: daz L \textbf{28} al] :l da Fr25 \textbf{29} dô] Da Z  $\cdot$ umbe] vnd L \newline
\end{minipage}
\hspace{0.5cm}
\begin{minipage}[t]{0.5\linewidth}
\small
\begin{center}*T
\end{center}
\begin{tabular}{rl}
 & \begin{large}D\end{large}urch minne, diu an im erstarp,\\ 
 & daz si der vürste niht erwarp,\\ 
 & si minnete sînen tôten lîp.\\ 
 & ob si worden wære sîn wîp,\\ 
5 & \textbf{daz} sich vrou Lunete\\ 
 & \textbf{versûmete} an \textbf{sô} \textbf{hôher bete},\\ 
 & als si riet ir vrouwen.\\ 
 & man mac \textbf{noch} dicke schouwen\\ 
 & vroun Luneten rîten zuo\\ 
10 & etslîchem râte gar ze vruo.\\ 
 & Swelch wîp nû durch geselleschaft\\ 
 & \textbf{verbirt} und durch zühte kraft\\ 
 & pflihte an vremde minne,\\ 
 & als ich mich versinne.\\ 
15 & lât si\textbf{z} bî ir mannes leben,\\ 
 & dem wart an ir der wunsch gegeben.\\ 
 & dehein beiten stât ir alse wol.\\ 
 & daz \textbf{beziug}ich, \textbf{als} ich sol.\\ 
 & dâ nâch tuo, alse \textbf{siz} lêre.\\ 
20 & behaltet si dannoch \textbf{ir} êre,\\ 
 & si\textbf{ne} treit deheinen \textbf{sô liehten} kranz,\\ 
 & gêt si durch vröuden an den tanz.\\ 
 & wes \textbf{mizz}ich vröude \textbf{gegen} der nôt,\\ 
 & alse Sygune ir triuwe \textbf{gebôt}?\\ 
25 & daz m\textit{ö}ht ich gerne lâzen.\\ 
 & über ronen, âne strâzen\\ 
 & Parcifal vür daz venster reit\\ 
 & alze nâhen, daz was im leit.\\ 
 & dô wolt er vrâgen umbe den walt,\\ 
30 & oder war sîn reise wære \textbf{bezalt}.\\ 
\end{tabular}
\scriptsize
\line(1,0){75} \newline
T U V W O Q R \newline
\line(1,0){75} \newline
\textbf{1} \textit{Initiale} U V O   $\cdot$ \textit{Capitulumzeichen} R  \textbf{11} \textit{Majuskel} T  \newline
\line(1,0){75} \newline
\textbf{1} Durch] ÷vrch O  $\cdot$ im erstarp] im starp U ir erstarb W Jmer \sout{stab} starb R \textbf{2} daz] Do W \textbf{3} minnete] minnet Q R \textbf{5} daz sich] Do hete sich U (V) (O) (Q) Do hette sy W (R)  $\cdot$ Lunete] Luͦnete U lunet V W Q Lonet O lunett R \textbf{6} versûmete] Versuͦmet U Gesumet V (O) (Q) R Gesuͤnet W  $\cdot$ hôher] goher V gaher W (O) Q (R) \textbf{7} ir] ir selber O \textbf{9} vroun] Vrov U (Q) (R)  $\cdot$ Luneten] Luͦneten U lunaeten Q lunetten R  $\cdot$ rîten] rieten O \textbf{10} etslîchem râte] Etlichen thete W  $\cdot$ vruo] vro U \textbf{11} Swelch] Welch U W Welchs Q Welhe R  $\cdot$ wîp nû] frow R \textbf{12} zühte] zuchtes U ir zúhte V (Q) (W) (R) \textbf{13} pflihte] Pfligt O  $\cdot$ vremde] froͤmder V (O) (Q) (R) \textbf{15} lât] Leit U  $\cdot$ bî] durch R  $\cdot$ ir] irs U W Q \textbf{16} der wunsch] der wunchst Q des wunschs R  $\cdot$ gegeben] geben W R \textbf{17} dehein] Dekein U Kein W Q  $\cdot$ stât ir alse] stot also V (R) stet alrechte W \textbf{18} beziugich] erzeigen ich U (V) (W) (O) (Q) (R)  $\cdot$ als] ob U O Q \textbf{19} dâ nâch] Dannoch V  $\cdot$ tuo] \textit{om.} W \textbf{20} behaltet] Behalte W  $\cdot$ ir] \textit{om.} V W O Q R \textbf{21} sine] Sú V (O) (R)  $\cdot$ deheinen] dekeinen U keinen W dehein O keyn Q  $\cdot$ sô] \textit{om.} U W \textbf{22} vröuden] vreide U (V) (W) (Q) (R) o\textit{m. } O \textbf{23} wes] Des R  $\cdot$ vröude] frewe Q  $\cdot$ gegen der] ander R \textbf{24} An sigúnen sie trewe gebot Q  $\cdot$ Sygune] syguͦne U sigunen V (W) (O) Sygunen R  $\cdot$ triuwe] trúwen W  $\cdot$ gebôt] bott R \textbf{25} möht] moht T (U) O (Q)  $\cdot$ ich] doch R \textbf{26} âne] vnd one R \textbf{27} Parcifal] Parzifal T Parzefal V Partzifal W Q Barcifal O Parczifal R \textbf{28} alze nâhen] Also nahen W Al da zenahen O \textbf{30} wære] do wer W  $\cdot$ bezalt] gezalt V W O (R) gestalt Q \newline
\end{minipage}
\end{table}
\end{document}
