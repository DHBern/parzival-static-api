\documentclass[8pt,a4paper,notitlepage]{article}
\usepackage{fullpage}
\usepackage{ulem}
\usepackage{xltxtra}
\usepackage{datetime}
\renewcommand{\dateseparator}{.}
\dmyyyydate
\usepackage{fancyhdr}
\usepackage{ifthen}
\pagestyle{fancy}
\fancyhf{}
\renewcommand{\headrulewidth}{0pt}
\fancyfoot[L]{\ifthenelse{\value{page}=1}{\today, \currenttime{} Uhr}{}}
\begin{document}
\begin{table}[ht]
\begin{minipage}[t]{0.5\linewidth}
\small
\begin{center}*D
\end{center}
\begin{tabular}{rl}
\textbf{114} & beidiu siufzen unt lachen\\ 
 & kunde ir munt \textbf{vil} wol \textbf{gemachen}.\\ 
 & si vreute sich ir sunes geburt.\\ 
 & ir schimpf ertranc in riwen vurt.\\ 
5 & \begin{large}S\end{large}wer nû wîben sprichet baz,\\ 
 & \textbf{deiswâr}, daz lâz ich âne haz.\\ 
 & ich \textbf{vriesche} gerne ir \textbf{vreude} breit,\\ 
 & wan einer bin ich unbereit\\ 
 & dienstlîcher triwe.\\ 
10 & mîn zorn ist immer niwe\\ 
 & gein ir, sît ich \textbf{si an wanke} sach.\\ 
 & ich bin \textit{W}olfram von Eschenbach\\ 
 & unt kan ein teil mit sange\\ 
 & \textbf{unt} bin ein habendiu zange\\ 
15 & \textbf{mînen} zorn gein einem wîbe.\\ 
 & diu hât mîme lîbe\\ 
 & erboten solhe missetât,\\ 
 & i\textbf{ne} hân si hazzens keinen rât.\\ 
 & dâr umbe \textbf{hânt mîn die} andern haz.\\ 
20 & owê, wâr umbe \textbf{tuont} si daz?\\ 
 & al ein \textbf{sî} mir \textbf{ir} hazzen leit,\\ 
 & ez \textbf{ist iedoch} ir wîpheit,\\ 
 & Sît ich mich versprochen hân\\ 
 & unt an mir selben missetân.\\ 
25 & daz lîhte nimmer mêr geschiht.\\ 
 & \textbf{iedoch} \textbf{en}suln si sich vergâhen niht\\ 
 & mit hurte an mîn hamît.\\ 
 & si vindent werlîchen strît.\\ 
 & I\textbf{ne} hân des niht vergezzen,\\ 
30 & ine künne wol gemezzen\\ 
\end{tabular}
\scriptsize
\line(1,0){75} \newline
D Fr33 \newline
\line(1,0){75} \newline
\textbf{5} \textit{Initiale} D Fr33  \textbf{23} \textit{Majuskel} D  \textbf{29} \textit{Majuskel} D  \newline
\line(1,0){75} \newline
\textbf{1} beidiu] \textit{om.} Fr33 \textbf{2} vil] \textit{om.} Fr33 \textbf{4} ir schif ir tranc ir ruwen ::: Fr33 \textbf{5} Swer] Swor Fr33 \textbf{7} vreude] ere Fr33 \textbf{8} einer] eine Fr33 \textbf{9} dienstlîcher] Distlicher Fr33 \textbf{12} Wolfram] Volfram D  $\cdot$ Eschenbach] Esch:::ach Fr33 \textbf{15} zorn] haz Fr33 \textbf{19} hânt mîn] h:: ich Fr33 \textbf{21} al ein] Ale::e Fr33  $\cdot$ ir hazzen] min haz so Fr33 \textbf{22} S: ist iz doch ir wipheit Fr33 \textbf{24} selben] \textit{om.} Fr33 \textbf{25} Daz nimmer lihte geschit Fr33 \textbf{26} iedoch] doch Fr33 \textbf{27} mîn] mime Fr33 \newline
\end{minipage}
\hspace{0.5cm}
\begin{minipage}[t]{0.5\linewidth}
\small
\begin{center}*m
\end{center}
\begin{tabular}{rl}
 & beidiu siufzen und lachen\\ 
 & kunde ir munt wol \textbf{gemachen}.\\ 
 & si vröwete sich ir sunes geburt.\\ 
 & ir schimpf ertranc in riuwen vurt.\\ 
5 & \begin{large}W\end{large}er nû wîben sprichet baz,\\ 
 & \textbf{zwâr}, daz lâz ich âne haz.\\ 
 & ich \textbf{vreische} gerne ir \textbf{vreude} breit,\\ 
 & wanne einer bin ich unbereit\\ 
 & dienstlîcher triuwe.\\ 
10 & mîn zorn ist iemer niuwe\\ 
 & gegen ir, sît ich \textbf{si an wanke} sach.\\ 
 & ich bin Wolf\textit{ra}m von Eschenbach\\ 
 & und kan ein teil mit sange.\\ 
 & \textbf{ich} bin ein habend\textit{iu} zange\\ 
15 & \textbf{mînen} zorn gegen einem wîbe.\\ 
 & diu hât mînem lîbe\\ 
 & erboten soliche missetât,\\ 
 & i\textbf{ne} hân si hazzens keinen rât.\\ 
 & dâr umbe \textbf{hân ich den} andern haz.\\ 
20 & owê, wâr umbe \textbf{tuot} si daz?\\ 
 & alein \textbf{sî} mir hazzen l\textit{e}it,\\ 
 & ez \textbf{ist iedoch} ir wîpheit,\\ 
 & sît ich mich versprochen hân\\ 
 & und an mir selben missetân.\\ 
25 & daz lîhte niemer mê geschiht.\\ 
 & \textbf{daz} \textbf{en}sullen si sich vergâhen niht\\ 
 & mit h\textit{u}rte an mîn hamît.\\ 
 & si vindent werlîchen strît.\\ 
 & \textit{i}\textbf{\textit{n}e} hân des niht vergezzen,\\ 
30 & \textit{in}e künne wol gemezzen\\ 
\end{tabular}
\scriptsize
\line(1,0){75} \newline
m n o \newline
\line(1,0){75} \newline
\textbf{5} \textit{Initiale} m o   $\cdot$ \textit{Capitulumzeichen} n  \newline
\line(1,0){75} \newline
\textbf{3} \textit{Versfolge 114.4-3} n o   $\cdot$ vröwete] frouwet n (o)  $\cdot$ ir] irs m (n) o \textbf{7} vreische] friste n \textbf{10} niuwe] nuͯ o \textbf{12} Wolfram] wolvarm m wolffram n o \textbf{14} habendiu] habendes m \textbf{18} ine] Me m n o  $\cdot$ hân] [kan]: han ich o \textbf{19} den] der n o \textbf{20} tuot] duͦnt n (o) \textbf{21} leit] lit m \textbf{24} selben] selber n o \textbf{26} daz ensullen] Doch soͯllent n (o) \textbf{27} hurte] herte m n o \textbf{29} ine] Me m \textbf{30} ine] Me m Jch n o \newline
\end{minipage}
\end{table}
\newpage
\begin{table}[ht]
\begin{minipage}[t]{0.5\linewidth}
\small
\begin{center}*G
\end{center}
\begin{tabular}{rl}
 & bêdiu sûften und lachen\\ 
 & kunde ir munt \textbf{vil} wol \textbf{gemachen}.\\ 
 & si vröute sich ir sunes geburt.\\ 
 & ir schimpf ertranc in riwen vurt.\\ 
5 & \begin{large}S\end{large}wer nû wîben spricht baz,\\ 
 & \textit{\textbf{dêswâr}}, daz lâze ic\textit{h â}ne haz.\\ 
 & ich \textbf{vriesche} gerne ir \textbf{\textit{v}r\textit{öud}e} breit,\\ 
 & wan einer bin ich unbereit\\ 
 & dienstlîcher triwe.\\ 
10 & mîn zorn ist imer niwe\\ 
 & gein ir, sît ich \textbf{si an wanke} sach.\\ 
 & ich bin Wolfram von Eschenbach\\ 
 & unt kan ein teil mit sange.\\ 
 & \textit{\textbf{ich}} bin ein habendiu zange\\ 
15 & \textbf{mit} zorne gein einem wîbe.\\ 
 & diu hât mînem lîbe\\ 
 & erboten solhe missetât,\\ 
 & ich hân si hazzenes deheinen rât.\\ 
 & dâr umbe \textbf{hân ich der} andern haz.\\ 
20 & owê, wâr umbe \textbf{tuont} si daz?\\ 
 & al ein \textbf{ist} mir \textbf{ir} hazzen leit,\\ 
 & ez \textbf{ist iedoch} ir wîpheit,\\ 
 & sît ich mich versprochen hân\\ 
 & unde an mir se\textit{l}be\textit{n} missetân.\\ 
25 & daz lîhte nimer mêr geschiht.\\ 
 & \textbf{doch} sulen si sich vergâhen niht\\ 
 & mit hurte an mîn hamît.\\ 
 & si vindent werlîchen strît.\\ 
 & ich hân des niht vergezzen,\\ 
30 & ichne kunne wol gemezzen\\ 
\end{tabular}
\scriptsize
\line(1,0){75} \newline
G I O L M Q R Z \newline
\line(1,0){75} \newline
\textbf{1} \textit{Initiale} O M  \textbf{5} \textit{Initiale} G I L R Z  \textbf{23} \textit{Initiale} I  \newline
\line(1,0){75} \newline
\textbf{1} bêdiu] ÷eidiv O  $\cdot$ sûften] susszzen Q súnszen R \textbf{2} vil] \textit{om.} L M Z  $\cdot$ gemachen] machen I L (Q) Z [lachin]: machin M \textbf{3} vröute] freut I (O) (M) (Q) (Z)  $\cdot$ ir] irs L (M) Q Z  $\cdot$ sunes] kindes I (Q) suͯssen R \textbf{4} schimpf] frevde O (L) (Q) (R)  $\cdot$ riwen] iamers O L (Q) trúwen R \textbf{5} Swer] Wer L M Q R  $\cdot$ wîben spricht] wibe sprichet I sprichet weyben Q [wibens]: wiben sprichet Z \textbf{6} daz laze ich weiz got âne haz G  $\cdot$ dêswâr] Vor war M Entzwar Q Zwar Z \textbf{7} ich] Jr Q  $\cdot$ vriesche] freische L M (Q)  $\cdot$ gerne] gen R  $\cdot$ vröude] ere G ir froͯde R \textbf{9} dienstlîcher] Dienstliche R \textbf{10} mîn] Sin O [Meiner]: Mein Q  $\cdot$ zorn] zornen Q  $\cdot$ imer] ir I iamer O en mer M  $\cdot$ niwe] [n*wr]: newr Q \textbf{11} ich si] sie L ichtz Q  $\cdot$ an wanke] in wanche I an wanche ê O an wanken R \textbf{12} Wolfram] Wolvram I wolffran L wolferam Q Wolffram R  $\cdot$ Eschenbach] esschenpach O Eschelbach L eschinbach M esschenbach Z \textbf{13} unt] Jch O \textbf{14} ich] vnt G \textbf{15} mit zorne] Minen zorn L R Z Mein zornen Q  $\cdot$ gein] gei O \textbf{16} mînem] myne M minen R \textbf{18} hân] enhan L (M) R (Z)  $\cdot$ si] \textit{om.} I M  $\cdot$ hazzenes] hazzes I Q heissent R  $\cdot$ deheinen] sie nikeyn M \textbf{19} der andern] der ander I (R) den anderen Q  $\cdot$ haz] baz M \textbf{20} owê] Awe O  $\cdot$ tuont] tuͦt R \textbf{21} ir] min Z \textbf{24} selben] [*bem]: sebem G selbern M selbe R  $\cdot$ missetân] missetat Q \textbf{26} sulen] ensullen Q \textbf{27} hurte] hurten I  $\cdot$ mîn] im Q \textbf{28} vindent] en vinden Z  $\cdot$ werlîchen] vreislichen I \textbf{29} ich hân] Jchn han O (M) (R) (Z)  $\cdot$ niht] werlichen nicht M \textbf{30} ichne] ich I (L)  $\cdot$ wol gemezzen] vil wol mezzen Z \newline
\end{minipage}
\hspace{0.5cm}
\begin{minipage}[t]{0.5\linewidth}
\small
\begin{center}*T (U)
\end{center}
\begin{tabular}{rl}
 & beidiu siufzen und lachen\\ 
 & kunde ir munt wol \textbf{machen}.\\ 
 & si vreute sich ir sunes geburt.\\ 
 & ir schimpf ertranc in riuwen vurt.\\ 
5 & \begin{large}W\end{large}er nû wîben sprichet baz,\\ 
 & \textbf{dêswâr}, daz lâz ich âne haz.\\ 
 & ich \textbf{vriesche} gerne ir \textbf{êre} breit,\\ 
 & wan einer bin ich unbereit\\ 
 & dienstlîcher triuwe.\\ 
10 & mîn zorn ist imer niuwe\\ 
 & gein ir, sît ich \textbf{einen wanc} sach.\\ 
 & ich bin Wolfram von Eschebach\\ 
 & und kan ein teil mit sange.\\ 
 & \textbf{ich} bin ein habendiu zange\\ 
15 & \textbf{mit} zorne gein eime wîbe.\\ 
 & diu \textbf{selbe} hât mîme lîbe\\ 
 & erboten soliche missetât,\\ 
 & ich hân si hazzens dekeinen rât.\\ 
 & dâr umb \textbf{lîde ich der} andern haz.\\ 
20 & owê, wâr umb \textbf{tuont} si daz?\\ 
 & aleine \textbf{ist} mir \textbf{ir} hazzen leit,\\ 
 & ez \textbf{krenket doch} ir wîpheit,\\ 
 & sît ich mich versprochen hân\\ 
 & und an mir selbe missetân.\\ 
25 & daz lîhte niemer mê geschiht.\\ 
 & \textbf{doch} \textbf{en}soln si sich vergâhen niht\\ 
 & mit \textbf{ir} hurte an \textit{mî}n hamît.\\ 
 & s\textit{i} vindent werlîchen strît.\\ 
 & ich \textbf{en}hân des niht vergezzen,\\ 
30 & ich enkunne wol gemezzen\\ 
\end{tabular}
\scriptsize
\line(1,0){75} \newline
U V W T \newline
\line(1,0){75} \newline
\textbf{1} \textit{Majuskel} T  \textbf{3} \textit{Majuskel} T  \textbf{5} \textit{Initiale} U V W T  \textbf{9} \textit{Majuskel} T  \newline
\line(1,0){75} \newline
\textbf{2} munt] mvͦt T \textbf{3} vreute] frauwet W  $\cdot$ ir] irs U V (W) \textbf{4} schimpf] svfc T \textbf{5} Wer] Swer V T \textbf{7} vriesche] verneme V freische W (T)  $\cdot$ êre] froͤde W (T) \textbf{10} ist] ist anders W \textbf{11} einen wanc] [*]: sv́ an [wan*]: wanke V si an wancke T \textbf{12} Wolfram] wolfruͦm U  $\cdot$ Eschebach] eschenbach W Escebach T \textbf{15} mit zorne] min zorn ist T \textbf{16} selbe] \textit{om.} T \textbf{17} soliche] semliche W solhiv T \textbf{18} hân] enhan W  $\cdot$ hazzens] hassendes V \textbf{19} lîde] han T \textbf{20} tuont] tuͦt V \textbf{21} aleine ist mir ir] Das allein ist in mein W \textbf{22} ez krenket doch] Das ist doch von W ez ist iedoch T \textbf{24} selbe] selben V W selbem T \textbf{25} niemer mê] nieme W \textbf{26} ensoln] súllen W (T) \textbf{27} ir] \textit{om.} T  $\cdot$ mîn] ran U \textbf{28} si] Sin U  $\cdot$ vindent werlîchen] vindet werlich W \textbf{29} enhân] han V W T \textbf{30} enkunne] kúnne V \newline
\end{minipage}
\end{table}
\end{document}
