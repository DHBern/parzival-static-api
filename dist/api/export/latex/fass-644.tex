\documentclass[8pt,a4paper,notitlepage]{article}
\usepackage{fullpage}
\usepackage{ulem}
\usepackage{xltxtra}
\usepackage{datetime}
\renewcommand{\dateseparator}{.}
\dmyyyydate
\usepackage{fancyhdr}
\usepackage{ifthen}
\pagestyle{fancy}
\fancyhf{}
\renewcommand{\headrulewidth}{0pt}
\fancyfoot[L]{\ifthenelse{\value{page}=1}{\today, \currenttime{} Uhr}{}}
\begin{document}
\begin{table}[ht]
\begin{minipage}[t]{0.5\linewidth}
\small
\begin{center}*D
\end{center}
\begin{tabular}{rl}
\textbf{644} & \begin{large}D\end{large}iu würze was bî dem blank\textit{en} brûn.\\ 
 & muoterhalp der Bertun,\\ 
 & Gawan, fillu roy Lot,\\ 
 & süezer senfte vür sûre nôt\\ 
5 & er mit werder \textbf{helfe} pflac\\ 
 & helfeclîche unz an den tac.\\ 
 & Sîn helfe was \textbf{doch sô} gedigen,\\ 
 & \textbf{deiz al daz} volc \textbf{was} verswigen.\\ 
 & sît nam er mit vreuden war\\ 
10 & \textbf{al} der \textbf{rîter} unt der \textbf{vrouwen} \textbf{gar},\\ 
10 & \multicolumn{1}{l}{ - - - }\\ 
 & \textbf{sô} daz ir trûren \textbf{vil nâch} verdarp.\\ 
 & Nû \textbf{hœret ouch, wie der knappe warp,}\\ 
 & \multicolumn{1}{l}{ - - - }\\ 
 & \textbf{den} Gawan hete gesant\\ 
 & \textbf{hin} ze Lœver in daz lant,\\ 
15 & ze \textbf{Bems} bî der \textbf{Korca}.\\ 
 & \textbf{der künec Artus was al} dâ\\ 
 & \textbf{unt des wîp}, diu künegîn,\\ 
 & unt maneger \textbf{vrouwen liehter} schîn\\ 
 & \textbf{unt} der messenîe ein vluot.\\ 
20 & nû hœret \textbf{ouch}, wie der knappe tuot:\\ 
 & Diz was eines morgens vruo,\\ 
 & sîner botschefte greif er zuo.\\ 
 & \textbf{diu} künegîn \textbf{zer kappeln} was,\\ 
 & \textbf{an ir venje} si \textbf{den} salter las.\\ 
25 & der knappe vür si kniete,\\ 
 & \textbf{er} bôt ir \textbf{vreuden} miete.\\ 
 & einen brief \textbf{si nam} \textbf{ûz sîner} hant,\\ 
 & dâr an si geschriben vant\\ 
 & \textbf{schrift}, die si bekante,\\ 
30 & ê \textbf{sînen hêrren} nante\\ 
\end{tabular}
\scriptsize
\line(1,0){75} \newline
D Fr1 \newline
\line(1,0){75} \newline
\textbf{1} \textit{Initiale} D  \textbf{7} \textit{Initiale} Fr1   $\cdot$ \textit{Majuskel} D  \textbf{12} \textit{Majuskel} D  \textbf{21} \textit{Majuskel} D  \newline
\line(1,0){75} \newline
\textbf{1} blanken] blanch D \textbf{2} Bertun] bertvͦn D \textbf{3} Gawan svn des kvnec Lôt Fr1 \textbf{12} hœret] vernemt Fr1 \textbf{14} Lœver] Loͤver D Fr1 \textbf{15} Bems] Beems Fr1  $\cdot$ Korca] Chorcha D Choͤrcha Fr1 \textbf{20} ouch] \textit{om.} Fr1 \newline
\end{minipage}
\hspace{0.5cm}
\begin{minipage}[t]{0.5\linewidth}
\small
\begin{center}*m
\end{center}
\begin{tabular}{rl}
 & diu würze was bî dem blanken brûn.\\ 
 & muoterhalp der Britu\textit{n},\\ 
 & Gawan, fili roi Lot,\\ 
 & süezer senfte vür sûre nôt\\ 
5 & er mit werder \textbf{helfe} pflac\\ 
 & helflîch unz an den tac.\\ 
 & sîn helfe was \textbf{doch sô} g\textit{e}digen,\\ 
 & \textbf{daz allem} volc \textbf{was} verswigen.\\ 
 & sît nam er mit vröuden war\\ 
10 & der \textbf{ritter} und der \textbf{vrowe\textit{n}} \textbf{\textit{g}ar}.\\ 
10 & s\textit{ô} süezeclîch er mit i\textit{n} warp,\\ 
 & daz \textbf{gar} ir trûren \textbf{dô} verdarp.\\ 
 & \begin{large}N\end{large}û \textbf{lâzen wir diz mære hie}\\ 
 & und kêren wider, d\textit{â} ich ez lie,\\ 
 & \textbf{ich mein, dô} Gawan hete gesant\\ 
 & \textbf{den boten} zuo Lov\textit{e}r in daz lant.\\ 
 & \hspace*{-.7em}\big| \textbf{vernemet, waz er würbe} dâ.\\ 
15 & \hspace*{-.7em}\big| zuo \textbf{Bems} bî der \textbf{K\textit{o}rca}\\ 
 & \textbf{was Artus und} diu künigîn\\ 
 & und maniger \textbf{liehter \textit{vrouwen}} schîn,\\ 
 & der \textbf{werde\textit{n}} massenîe ei\textit{n v}luot.\\ 
20 & nû hœret, wie der knappe tuot:\\ 
 & diz was eines morge\textit{n}s vruo,\\ 
 & sîner botschaft greif er zuo.\\ 
 & \textbf{dô er zuor} künigîn \textbf{komen} was,\\ 
 & \textbf{in der kappel} si \textbf{ir} salter las.\\ 
25 & der knappe vür si kniete\\ 
 & \textbf{und} bôt ir \textbf{vreude} miete.\\ 
 & einen brief \textbf{gap er \textit{ir}} \textbf{in die} hant,\\ 
 & dâr an si geschriben vant\\ 
 & \textbf{geschrift}, die si bekande,\\ 
30 & ê \textbf{sînen hêrren} nande\\ 
\end{tabular}
\scriptsize
\line(1,0){75} \newline
m n o Fr69 \newline
\line(1,0){75} \newline
\textbf{12} \textit{Capitulumzeichen} n  \newline
\line(1,0){75} \newline
\textbf{1} brûn] b:nn o \textbf{2} Britun] brittum m brituͦn n britum o \textbf{3} roi Lot] roẏlot n roilat o \textbf{4} süezer] Suͯsse m n o \textbf{5} werder] senffter n \textbf{7} doch] doc Fr69  $\cdot$ gedigen] genedigen m \textbf{10} und] vnd ouch n  $\cdot$ vrowen gar] frowen dar vnd gar m \textbf{10} sô] Su m  $\cdot$ in] ẏm m \textbf{11} trûren] truwen n \textbf{12} diz] das n \textbf{12} dâ] do m n o \textbf{13} gesant] genant o \textbf{14} Lover] louor m loner n louer o \textbf{16} dâ] do n \textbf{15} Bems] beems m n o  $\cdot$ Korca] karca m o karco n \textbf{17} Artus] artuͯs o \textbf{18} maniger] manig n o  $\cdot$ vrouwen] \textit{om.} m \textbf{19} werden] werde m  $\cdot$ ein vluot] ein fluht fluͯt m \textbf{21} morgens] morgengens m \textbf{24} kappel] Cappellen n  $\cdot$ salter] satter o \textbf{25} \textit{Versdoppelung 644.25-30 (²o) nach 644.30; Lesarten der vorausgehenden Verse mit ¹o bezeichnet} o   $\cdot$ kniete] knuwe o \textbf{26} miete] ner:e \textsuperscript{1}\hspace{-1.3mm} o mete \textsuperscript{2}\hspace{-1.3mm} o \textbf{27} ir] \textit{om.} \textit{(krit. Text emendiert nach V#'* ͫ)} m n o \textbf{29} geschrift] Die geschrifft n  $\cdot$ bekande] erkante n \textbf{30} ê] E sie \textsuperscript{2}\hspace{-1.3mm} o \newline
\end{minipage}
\end{table}
\newpage
\begin{table}[ht]
\begin{minipage}[t]{0.5\linewidth}
\small
\begin{center}*G
\end{center}
\begin{tabular}{rl}
 & \begin{large}D\end{large}iu würze was bî dem blanken brûn.\\ 
 & muoterhalp der Britun,\\ 
 & Gawan, fil\textit{i} roys Lot,\\ 
 & süezer senfte vür sûre nôt\\ 
5 & er mit werder \textbf{vröude} pflac\\ 
 & helflîch unze an den tac.\\ 
 & sîn helfe wa\textit{s} \textbf{alsô} gedigen,\\ 
 & \textbf{daz al daz} volc \textbf{wart} \textbf{gar} verswigen.\\ 
 & sît nam er mit vröuden war\\ 
10 & \textbf{al} der \textbf{vrouwen} unde der \textbf{rîter} \textbf{schar},\\ 
10 & \multicolumn{1}{l}{ - - - }\\ 
 & \textbf{sô} daz ir trûren \textbf{vil nâch} verdarp.\\ 
 & nû \textbf{hœrt ouch, wie der knappe warp,}\\ 
 & \multicolumn{1}{l}{ - - - }\\ 
 & \textbf{den} Gawan hete gesant\\ 
 & \textbf{hin} ze Lover inz lant,\\ 
15 & ze \textbf{Sabins} bî der \textbf{Chronica}.\\ 
 & \textbf{der künic Artus was al} dâ\\ 
 & \textbf{unde des wîp}, diu künegîn,\\ 
 & unde maniger \textbf{liehten vrouwen} schîn\\ 
 & \textbf{unde ouch} der massenîe ein vluot.\\ 
20 & nû hœret \textbf{ouch}, wie der knappe tuot:\\ 
 & d\textit{it}z was eines morgens vruo,\\ 
 & sîner botschefte greif er zuo.\\ 
 & \textbf{diu} künegîn \textbf{ze der kappeln} was,\\ 
 & \textbf{an ir venje} si \textbf{den} salter las.\\ 
25 & der knappe vür si kniete,\\ 
 & \textbf{er} bôt ir \textbf{vrömede} miete.\\ 
 & eine\textit{n} brief \textbf{si nam} \textbf{ûz sîner} hant,\\ 
 & dâr an si \textit{ge}schriben vant\\ 
 & \textbf{schrift}, die si bekande,\\ 
30 & ê \textbf{si} \textbf{sîn herze} nande,\\ 
\end{tabular}
\scriptsize
\line(1,0){75} \newline
G I L M Z Fr18 \newline
\line(1,0){75} \newline
\textbf{1} \textit{Initiale} G I L Z Fr18  \textbf{21} \textit{Initiale} I  \newline
\line(1,0){75} \newline
\textbf{1} dem] den L  $\cdot$ blanken] [blachen]: blanchen G \textbf{2} Britun] pritun I Brittvͯn L brytun M brẏtvn Fr18 \textbf{3} fili roys] filioroys G fiz Lv Roýs L fillurois Z fẏllẏirẏs Fr18 \textbf{5} vröude] freuden I \textbf{6} unze] vnde M \textbf{7} was] wart G  $\cdot$ gedigen] [geb]: gedigen Z \textbf{8} wart] waz L  $\cdot$ gar] \textit{om.} I Z \textbf{10} al] \textit{om.} L  $\cdot$ vrouwen] ritter L (M) Z Fr18  $\cdot$ rîter schar] vrowen gar L (M) (Z) (Fr18) \textbf{11} vil nâch] gar L nach M \textbf{12} wie] waz I \textbf{14} hin] \textit{om.} I  $\cdot$ Lover] louer G Louers I Leover L loͤfer Z \textbf{15} ze Sabins] zebeins G ze Sebins I Zv benis Z Ze Rabins Fr18  $\cdot$ Chronica] chorcha G L kronica I korcha M Fr18 corhta Z \textbf{16} Artus was] [arturwas]: artuswas G was I Artuͯs waz L \textbf{17} des] sin L \textbf{18} liehten] lýchten L (M) \textbf{19} ouch der] der werden L (M) Z Fr18 \textbf{20} nû] [siner]: Nu G  $\cdot$ ouch] >oh< G \textit{om.} L  $\cdot$ wie] waz I \textbf{21} ditz] daz G \textbf{23} ze] in I  $\cdot$ kappeln] kappel Fr18 \textbf{24} an ir] Andy M  $\cdot$ si den salter] ir gebet si I \textbf{26} vrömede] frovde L (M) (Z) Fr18 \textbf{27} einen] Einem G  $\cdot$ si nam] nam si I den nam sý L  $\cdot$ hant] hat M \textbf{28} geschriben] scriben G bischriben M \textbf{29} schrift] schrifte G  $\cdot$ bekande] erkande L \textbf{30} si] er Z  $\cdot$ sîn herze] der knappe L sinen hern M (Z) (Fr18) \newline
\end{minipage}
\hspace{0.5cm}
\begin{minipage}[t]{0.5\linewidth}
\small
\begin{center}*T
\end{center}
\begin{tabular}{rl}
 & diu würze was bî dem blanken brûn.\\ 
 & muoterhalp der Britun,\\ 
 & Gawan, fil rois Lot,\\ 
 & süezer senfte vür sûre nôt\\ 
5 & er mit werder \textbf{helfe} pflac\\ 
 & helflîch unz an den tac.\\ 
 & sîn helfe was \textbf{doch sô} gedigen,\\ 
 & \textbf{daz allez daz} volc \textbf{was} \textbf{gar} verswigen.\\ 
 & sît nam er mit vreuden war\\ 
10 & \textbf{al} der \textbf{ritter} und de\textit{r} \textbf{\textit{v}rouwen} \textit{\textbf{gar}},\\ 
10 & \multicolumn{1}{l}{ - - - }\\ 
 & \textbf{sô} daz ir trûren \textbf{vil nâch} verdarp.\\ 
 & nû \textbf{hœret ouch, wie der knabe \textit{w}a\textit{rp}},\\ 
 & \multicolumn{1}{l}{ - - - }\\ 
 & \textbf{den} Gawan het gesant\\ 
 & \textbf{hin} zuo Lover in daz lant,\\ 
15 & zuo \textbf{Benis} bî der \textbf{Koicha}.\\ 
 & \textbf{der künec Artus was al}dâ\\ 
 & \textbf{und sîn wîp}, diu künigîn,\\ 
 & und maneger \textbf{li\textit{e}hten vrouwen} schîn\\ 
 & \textbf{und} der \textbf{werden} massenîen ein vluot.\\ 
20 & nû hœret \textbf{ouch}, wie der knabe tuot:\\ 
 & diz was eines morgens vruo,\\ 
 & sîner botschefte greif er zuo.\\ 
 & \textbf{diu} künigîn \textbf{zuor kappeln} was,\\ 
 & \textbf{an ir venje} si \textbf{den} salter las.\\ 
25 & der knabe vür si kniete,\\ 
 & \textbf{er} bôt ir \textbf{vreude} miete.\\ 
 & einen brief \textbf{si entpfienc} \textbf{ûz sîner} hant,\\ 
 & dâr an si geschriben vant\\ 
 & \textbf{schrift}, die si bekante,\\ 
30 & ê \textbf{er} \textbf{sînen hêrren} nante,\\ 
\end{tabular}
\scriptsize
\line(1,0){75} \newline
Q R W V \newline
\line(1,0){75} \newline
\textbf{1} \textit{Initiale} Q  \textbf{3} \textit{Initiale} W  \textbf{21} \textit{Initiale} R V  \newline
\line(1,0){75} \newline
\textbf{1} dem] den W \textbf{2} der] ein rechter W  $\cdot$ Britun] brittum Q [britt*]: brittvn V \textbf{3} Gawan dez kv́niges [*]: svn lot  V  $\cdot$ Gawan] Gewan R  $\cdot$ fil rois] fillirois R (W) \textbf{4} senfte] senffter W  $\cdot$ nôt] bot R \textbf{5} helfe] froͯden R \textbf{7} doch] echt W \textit{om.} V \textbf{8} allez daz] alles W \textbf{10} der ritter] ritter W  $\cdot$ der vrouwen gar] der gar frawen Q der frowen schar R \textbf{11} ir] er R [*]: ir V  $\cdot$ vil nâch] [*]: vil noch V \textbf{12} [*]: Nv hoͤrent oͮch wie der knappe warp V  $\cdot$ ouch] \textit{om.} W  $\cdot$ warp] sprach Q \textbf{13} [*]: Den gawan hette gesant V  $\cdot$ Gawan] Gawin R  $\cdot$ het] hat R \textbf{14} hin zuo] Hin gen W [*]: Hin ze V  $\cdot$ Lover] louer Q W lofier R [*]: louer V \textbf{15} Benis] [*]: benis V  $\cdot$ Koicha] kocha Q korka R korcha W [*]: korka V \textbf{16} [*t]: Der kv́nig artus waz alda V \textbf{17} [*t]: Vnd dez wip die kv́nigin V  $\cdot$ sîn] des R W \textbf{18} liehten] lichten Q liechtter R (W) [*]: liehten  V  $\cdot$ vrouwen] [*]: vrowen V \textbf{19} [*]: Vnd der werden massenie ein fluͦt V  $\cdot$ massenîen] messnie R (W) \textbf{21} eines morgens] enmorgen R \textbf{23} [D*]: Do er zer kv́niginne komen waz V  $\cdot$ kappeln] kappel R \textbf{26} vreude] froͤden W  $\cdot$ miete] mitte Q \textbf{27} Einen brief [*]: gap er ir in die hant V  $\cdot$ si entpfienc] nan sy R sy nam W  $\cdot$ ûz] vsser R \textbf{28} si] sy nun W \textbf{29} schrift] Geschrifftt R  $\cdot$ bekante] wol kande W erkande V \textbf{30} nante] [vande]: nande Q \newline
\end{minipage}
\end{table}
\end{document}
