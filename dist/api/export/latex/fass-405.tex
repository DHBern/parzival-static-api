\documentclass[8pt,a4paper,notitlepage]{article}
\usepackage{fullpage}
\usepackage{ulem}
\usepackage{xltxtra}
\usepackage{datetime}
\renewcommand{\dateseparator}{.}
\dmyyyydate
\usepackage{fancyhdr}
\usepackage{ifthen}
\pagestyle{fancy}
\fancyhf{}
\renewcommand{\headrulewidth}{0pt}
\fancyfoot[L]{\ifthenelse{\value{page}=1}{\today, \currenttime{} Uhr}{}}
\begin{document}
\begin{table}[ht]
\begin{minipage}[t]{0.5\linewidth}
\small
\begin{center}*D
\end{center}
\begin{tabular}{rl}
\textbf{405} & \begin{large}D\end{large}ô Gawan die magt ersach,\\ 
 & der bote gienc nâher und sprach,\\ 
 & \textbf{al daz} der künec werben hiez.\\ 
 & diu küneginne dô niht enliez,\\ 
5 & si\textbf{ne spræche}: "hêrre, gêt nâher mir.\\ 
 & mîner zühte meister, daz sît ir.\\ 
 & nû gebiet unde lêret.\\ 
 & wirt iu kurzwîle gemêret,\\ 
 & daz muoz an iwerem gebote sîn.\\ 
10 & sît daz iuch der bruoder mîn\\ 
 & mir bevolhen hât sô wol,\\ 
 & ich küsse \textbf{iuch}, ob ich küssen sol.\\ 
 & nû gebiet nâch \textbf{iwerer} mâzen\\ 
 & mîn tuon \textbf{oder} mîn lâzen."\\ 
15 & mit grôzer zuht si vor im stuont.\\ 
 & \textbf{Gawan} sprach: "vrouwe, iwer munt\\ 
 & ist sô küssenlîch getân,\\ 
 & ich sol iweren \textbf{kus mit gruoze} hân."\\ 
 & Ir munt was heiz, dicke unt rôt,\\ 
20 & dâr an Gawan den sînen bôt.\\ 
 & dâ ergienc ein kus ungastlîch.\\ 
 & zuo der meide zühte rîch\\ 
 & saz der wol geborne gast.\\ 
 & süezer rede in niht gebrast\\ 
25 & bêdenthalp mit triwen.\\ 
 & si \textbf{kunde} wol geniwen,\\ 
 & \textbf{er sîne bete, si} ir versagen.\\ 
 & daz begunder herzenlîche klagen.\\ 
 & \textbf{ouch} bat er si genâden vil.\\ 
30 & diu magt sprach, als ich \textbf{iu} sagen wil:\\ 
\end{tabular}
\scriptsize
\line(1,0){75} \newline
D Fr5 \newline
\line(1,0){75} \newline
\textbf{1} \textit{Initiale} D  \textbf{19} \textit{Majuskel} D  \newline
\line(1,0){75} \newline
\newline
\end{minipage}
\hspace{0.5cm}
\begin{minipage}[t]{0.5\linewidth}
\small
\begin{center}*m
\end{center}
\begin{tabular}{rl}
 & \begin{large}D\end{large}ô Gawan die maget ersach,\\ 
 & der bote gienc nâher und sprach,\\ 
 & \textbf{al daz} der künic werben hiez.\\ 
 & diu künigîn dô niht enliez,\\ 
5 & si \textbf{e\textit{n}spræche}: "hêr, gât nâher mir.\\ 
 & mîner zuht meister, daz sît ir.\\ 
 & nû gebiet und lêret.\\ 
 & wirt iu kurzewîle gemêret,\\ 
 & daz muoz an iuwer\textit{m} gebote sîn.\\ 
10 & sît daz iuch der bruoder mîn\\ 
 & mir bevolhen hât sô wol,\\ 
 & ich küsse, ob ich küssen sol.\\ 
 & nû gebietet nâch \textbf{iuwerre} mâzen\\ 
 & mîn tuon \textbf{oder} mîn lâzen."\\ 
15 & mit grôzer zuht si vor ime stuont.\\ 
 & \textbf{Gawan} sprach: "vrouwe, iuwer munt\\ 
 & ist sô küssenlîch getân,\\ 
 & ich sol iuwer\textit{n} \textbf{ku\textit{s} mit gruoze} hân."\\ 
 & ir munt was heiz, dicke und rôt,\\ 
20 & dâr an Gawan den sînen bôt.\\ 
 & dâ ergienc ein kus ungastlîch.\\ 
 & zuo der megde zühte rîch\\ 
 & saz der wol geborne gast.\\ 
 & süezer rede in niht gebrast\\ 
25 & beidenthalben mit triuwen.\\ 
 & \textit{si \textbf{kunde} wol geniuwen}\\ 
 & \textbf{gegen sîner bete} ir versagen.\\ 
 & daz begunde \textit{er} herzelîchen klagen.\\ 
 & \textbf{doch} bat er si genâden vil.\\ 
30 & diu maget sprach, als ich \textbf{iu} sagen wil:\\ 
\end{tabular}
\scriptsize
\line(1,0){75} \newline
m n o \newline
\line(1,0){75} \newline
\textbf{1} \textit{Illustration mit Überschrift:} Also her gawan die schoͯne maget ersach do von er grosse freide enpfing n (o)   $\cdot$ \textit{Initiale} m n o  \newline
\line(1,0){75} \newline
\textbf{3} al] Also n  $\cdot$ werben] in werben n o \textbf{5} enspræche] entspreche m sprach n o \textbf{7} gebiet] gebieten n \textbf{9} iuwerm] uwern m \textbf{11} bevolhen] befollent o \textbf{12} küsse] kush o \textbf{13} gebietet] gebieten n gebiete o  $\cdot$ iuwerre] yre m \textbf{15} stuont] \textit{om.} o \textbf{18} ich] Es n o  $\cdot$ iuwern kus] iuwerm kuͯsse m ein kusz n (o)  $\cdot$ gruoze] grunt n gr:sz o  $\cdot$ hân] ergan n o \textbf{19} heiz] heisse n o \textbf{21} dâ] Do n o \textbf{22} zühte] zúchten n (o) \textbf{26} \textit{Vers 405.26 fehlt} m   $\cdot$ kunde] konden o \textbf{27} ir] er n \textbf{28} er] \textit{om.} m ir o  $\cdot$ klagen] [sagen]: clagen o \newline
\end{minipage}
\end{table}
\newpage
\begin{table}[ht]
\begin{minipage}[t]{0.5\linewidth}
\small
\begin{center}*G
\end{center}
\begin{tabular}{rl}
 & dô Gawan die maget ersach,\\ 
 & der bote gie nâher unde sprach,\\ 
 & \textbf{al daz} der künic werben hiez,\\ 
 & diu künigîn dô niht enliez,\\ 
5 & si \textbf{sprach}: "hêrre, gêt nâher mir.\\ 
 & mîner zühte meister, daz sît ir.\\ 
 & nû gebiet unde lêret.\\ 
 & wirt iu kurzewîle gemêret,\\ 
 & daz muoz an iwerem gebote sîn.\\ 
10 & sît daz iuch der bruoder mîn\\ 
 & mir bevolhen hât sô wol,\\ 
 & ich küsse \textbf{iuch}, obe ich küssen sol.\\ 
 & nû gebiet nâch \textbf{iweren} mâzen\\ 
 & mîn tuon \textbf{unde} mîn lâzen."\\ 
15 & mit grôzer zuht si vor im stuont.\\ 
 & \textbf{er} sprach: "vrouwe, iwer munt\\ 
 & ist sô küslîch getân,\\ 
 & ich \textit{so}l iweren \textbf{gruoz mit kusse} hân."\\ 
 & ir munt was heiz, dicke unde rôt,\\ 
20 & dâr an Gawan den sînen bôt.\\ 
 & dâ ergienc ein kus ungastlîch.\\ 
 & zuo der meide zühte rîch\\ 
 & saz der wolgeborne gast.\\ 
 & süezer rede in niht gebrast\\ 
25 & beidenthalp mit triwen.\\ 
 & si \textbf{kunde} wol geniwen,\\ 
 & \textbf{er sîne bete, si} ir versagen.\\ 
 & daz begunder herzenlîchen klagen.\\ 
 & \textbf{ouch} bat er si genâden vil.\\ 
30 & diu maget sprach, als ich sagen wil:\\ 
\end{tabular}
\scriptsize
\line(1,0){75} \newline
G I O L M Q R Z Fr22 \newline
\line(1,0){75} \newline
\textbf{1} \textit{Initiale} I O L Z   $\cdot$ \textit{Capitulumzeichen} R  \textbf{15} \textit{Initiale} I  \newline
\line(1,0){75} \newline
\textbf{1} \textit{Die Verse 370.13-412.12 fehlen} Q   $\cdot$ dô] ÷o O Da M Z \textbf{2} nâher] nach ir M \textbf{3} daz] \textit{om.} L  $\cdot$ werben] werden O \textbf{4} dô] da M Z \textbf{5} si sprach] Sine spræch O (L) (M) (Z)  $\cdot$ nâher] nahe baz O \textbf{11} bevolhen] beuohen R \textbf{12} küsse] chuste I  $\cdot$ ich] ich uch M \textbf{13} iweren] iwer O (L) (R) (Z) \textbf{14} unde] oder O L (M) Z \textbf{16} er] Gawan O L (M) Z  $\cdot$ vrouwe] \textit{om.} M \textbf{18} sol] wil G  $\cdot$ gruoz mit kusse] chus mit gruze I (L) (Z) chvs mir grvͦze O grusz mit kussen M  $\cdot$ hân] kan L \textbf{19} heiz dicke] diche haiz I heisze dicke L \textbf{20} den sînen] sinen mund R \textbf{21} kus] chauf I \textbf{22} zühte rîch] zuhten rich I (R) zuͯchtecliche L \textbf{24} in niht] im nie I im nit R \textbf{25} beidenthalp] Bidenthalb R \textbf{26} kunde] chunden I (O) (L) (M) (Z) (Fr22) \textbf{30} ich] ich nu I si O ich uͯch L \newline
\end{minipage}
\hspace{0.5cm}
\begin{minipage}[t]{0.5\linewidth}
\small
\begin{center}*T
\end{center}
\begin{tabular}{rl}
 & Dô Gawan die maget ersach,\\ 
 & der bote gie nâher unde sprach,\\ 
 & \textbf{als in} der künec werben hiez.\\ 
 & Diu künegîn dô niht enliez,\\ 
5 & si\textbf{ne spræche}: "hêrre, gêt nâher mir.\\ 
 & mîner zühte meister, daz sît ir.\\ 
 & nû gebiet unde lêret.\\ 
 & wirt iu kurzewîle gemêret,\\ 
 & daz muoz an iuwerm gebote sîn.\\ 
10 & sît daz iuch der bruoder mîn\\ 
 & mir bevolhen hât sô wol,\\ 
 & ich küss\textbf{iuch}, ob ich \textbf{iuch} küssen sol.\\ 
 & nû gebiet nâch \textbf{iuwern} mâzen\\ 
 & mîn tuon \textbf{oder} mîn lâzen."\\ 
15 & mit grôzer zuht si vor im stuont.\\ 
 & \textbf{Gawan} sprach: "vrouwe, iuwer munt\\ 
 & ist sô küss\textit{e}nlîch getân,\\ 
 & ich sol iuwern \textbf{kus mit gruoze} hân."\\ 
 & Ir munt was heiz, dicke unde rôt,\\ 
20 & dâr an Gawan den sînen bôt.\\ 
 & dâ ergie ein kus ungastlîche.\\ 
 & zuo der megde zühte rîche\\ 
 & saz der wol geborne gast.\\ 
 & süezer rede in niht gebrast\\ 
25 & beidenthalp mit triuwen.\\ 
 & si \textbf{kunden} wol geniuwen,\\ 
 & \textbf{er sîne bete, si} ir versagen.\\ 
 & daz begunder herzeclîche klagen.\\ 
 & \textbf{ouch} bat er si gnâden vil.\\ 
30 & diu maget sprach, als ich \textbf{iu} sagen wil:\\ 
\end{tabular}
\scriptsize
\line(1,0){75} \newline
T U V W \newline
\line(1,0){75} \newline
\textbf{1} \textit{Majuskel} T   $\cdot$ \textit{Initiale} W  \textbf{4} \textit{Majuskel} T  \textbf{17} \textit{Initiale} V  \textbf{19} \textit{Majuskel} T  \newline
\line(1,0){75} \newline
\textbf{3} als] Aldaz V (W)  $\cdot$ in] \textit{om.} W \textbf{4} dô] auch do W \textbf{5} sine] Sie U Seine W \textbf{6} zühte] zucht W \textbf{7} gebiet] gebet U \textbf{10} iuch] îv T \textbf{12} küssiuch] kvssiv T  $\cdot$ iuch küssen] iv kvssen T kússen W \textbf{13} iuwern] uwerre V \textbf{15} im] in W \textbf{16} vrouwe] \textit{om.} W \textbf{17} küssenlîch] kvsscenlich T kv́slich V \textbf{18} ich] ch V \textbf{21} dâ] Do V W \textbf{22} zühte] zv́hten V \textbf{27} sîne] sie U  $\cdot$ si] ee W \textbf{28} daz] Do W \textbf{29} ouch] Doch V \textbf{30} iu] \textit{om.} W \newline
\end{minipage}
\end{table}
\end{document}
