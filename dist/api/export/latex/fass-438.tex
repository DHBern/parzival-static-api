\documentclass[8pt,a4paper,notitlepage]{article}
\usepackage{fullpage}
\usepackage{ulem}
\usepackage{xltxtra}
\usepackage{datetime}
\renewcommand{\dateseparator}{.}
\dmyyyydate
\usepackage{fancyhdr}
\usepackage{ifthen}
\pagestyle{fancy}
\fancyhf{}
\renewcommand{\headrulewidth}{0pt}
\fancyfoot[L]{\ifthenelse{\value{page}=1}{\today, \currenttime{} Uhr}{}}
\begin{document}
\begin{table}[ht]
\begin{minipage}[t]{0.5\linewidth}
\small
\begin{center}*D
\end{center}
\begin{tabular}{rl}
\textbf{438} & \textbf{\begin{large}S\end{large}i truog einen salter} \textbf{in} der hant.\\ 
 & Parzival, der wîgant,\\ 
 & ein kleinez vingerlîn dâ kôs,\\ 
 & daz si durch arbeit nie \textbf{verlôs},\\ 
5 & \textbf{sine} behieltez durch rehter minne \textbf{rât}.\\ 
 & daz steinlîn was ein grânât.\\ 
 & des blic gab \textbf{ûz} \textbf{der vinster} schîn\\ 
 & als ein \textbf{ander} gensterlîn.\\ 
 & senlîch was ir gebende.\\ 
10 & "dâ ûzen bî der wende",\\ 
 & sprach si, "hêrre, dâ stêt ein banc.\\ 
 & ruochet sitzen, \textbf{lêrz} iuch iwer \textbf{gedanc}\\ 
 & unt ander unmuoze.\\ 
 & daz ich her ziwerem gruoze\\ 
15 & bin komen, daz \textbf{vergelt iu} got.\\ 
 & der giltet getriulîchen urbot."\\ 
 & Der helt ir râtes niht vergaz.\\ 
 & vür daz venster er dô saz.\\ 
 & er bat ouch dinne sitzen sie.\\ 
20 & si sprach: "nû hân ich selten hie\\ 
 & gesezzen bî decheinem man."\\ 
 & der helt si vrâgen began\\ 
 & umbe ir \textbf{site} und umb ir pflege:\\ 
 & "daz ir sô verre von dem wege\\ 
25 & sitzet in dirre wilde!\\ 
 & ich hânz vür unbilde,\\ 
 & vrouwe, wes ir iuch begêt,\\ 
 & sît hie \textbf{niht bûwes umb iuch} stêt."\\ 
 & \textbf{\begin{large}S\end{large}i sprach}: "\textbf{dâ} kumt mir vonme Grâl\\ 
30 & \textbf{mîn} spîse dâ her \textbf{al} sunder twâl.\\ 
\end{tabular}
\scriptsize
\line(1,0){75} \newline
D Fr31 \newline
\line(1,0){75} \newline
\textbf{1} \textit{Initiale} D  \textbf{17} \textit{Majuskel} D  \textbf{29} \textit{Initiale} D  \newline
\line(1,0){75} \newline
\textbf{1} einen] ein Fr31  $\cdot$ in der] ander Fr31 \textbf{2} Parzival] Parcifal D Parzifal Fr31 \textbf{5} sine behieltez] Si bihielt ez Fr31 \textbf{6} steinlîn] steinli Fr31 \textbf{8} als] Rehte als Fr31  $\cdot$ ander gensterlîn] zande ginastelin Fr31 \textbf{16} getriulîchen] getriweliches Fr31 \textbf{28} umb iuch stêt] bi::: Fr31 \textbf{29} mir vonme Grâl] mi::: Fr31 \textbf{30} al sunder twâl] vonme::: Fr31 \newline
\end{minipage}
\hspace{0.5cm}
\begin{minipage}[t]{0.5\linewidth}
\small
\begin{center}*m
\end{center}
\begin{tabular}{rl}
 & \textbf{einen salter truoc si} \textbf{in} der hant.\\ 
 & Parcifal, der wîgant,\\ 
 & ein kleinez vingerlîn d\textit{â} kôs,\\ 
 & daz si durch arbeit nie \textbf{verlôs},\\ 
5 & \textbf{si} behielte ez durch rehter minne \textbf{art}.\\ 
 & daz steinlîn was ein grânât.\\ 
 & des blic gap \textbf{ûz} \textbf{der v\textit{i}nster} schîn\\ 
 & \textbf{reht} als ein \textbf{viures} gensterlîn.\\ 
 & sen\textit{e}lîch was ir gebende.\\ 
10 & "dâ ûzen bî der wende",\\ 
 & sprach si, "hêrre, dâ stât ein banc.\\ 
 & ruochet sitzen, \textbf{lêre ez} iuch iuwer \textbf{gedanc}\\ 
 & und ander unmuoze.\\ 
 & daz ich her zuo i\textit{we}rem gruoze\\ 
15 & bin komen, daz \textbf{vergelte} got.\\ 
 & der giltet getriuweclîchen urbot."\\ 
 & der helt ir râtes niht vergaz.\\ 
 & vür daz venster er dô saz.\\ 
 & er bat ouch dinne sitzen sie.\\ 
20 & si sprach: "nû hân ich selten hie\\ 
 & gesezzen bî dekeinem man."\\ 
 & der helt si vrâgen  began\\ 
 & umb ir \textbf{strîte} und umb ir pflege:\\ 
 & "daz ir sô verre von dem wege\\ 
25 & sitzet in dirre wilde!\\ 
 & ich hân ez vür \textbf{ein} unbilde,\\ 
 & vrouwe, wes ir iuch begêt,\\ 
 & sît hie \textbf{niht bûwes umb iuch} stêt."\\ 
 & \textbf{si sprach}: "\textbf{dô} kumet mir von dem Grâl\\ 
30 & \textbf{mîn} spîse dâ her sunder twâl.\\ 
\end{tabular}
\scriptsize
\line(1,0){75} \newline
m n o \newline
\line(1,0){75} \newline
\newline
\line(1,0){75} \newline
\textbf{1} einen] Eẏn o  $\cdot$ salter] plaster o \textbf{3} dâ] do m n o \textbf{5} behielte] behielt n o  $\cdot$ rehter minne art] minen rat n mẏnne rat o \textbf{7} vinster schîn] vensterschin m \textbf{9} senelîch] Senalich m \textbf{10} dâ] Do n o \textbf{11} hêrre] herren o  $\cdot$ dâ] do n \textbf{12} lêre] lert n o  $\cdot$ iuch] \textit{om.} n o  $\cdot$ iuwer] \sout{sẏ} vwer o \textbf{13} und] Vnd ouch n  $\cdot$ unmuoze] [vnsse]: vnmusse o \textbf{14} daz] Die n o  $\cdot$ iwerem] irem m \textbf{15} got] úch got n o \textbf{16} giltet] gebútet n  $\cdot$ getriuweclîchen] getrúwelichen n getrubelich o \textbf{17} ir] ires n (o) \textbf{18} \textit{Versdoppelung} o   $\cdot$ dô saz] gesasz n (o) \textbf{19} er] [F]: Er n \textbf{20} nû hân] nuͯ han nuͯ han o \textbf{21} dekeinem] do keinem n dekeinen o \textbf{22} began] do began n o \textbf{23} strîte] sitte n (o)  $\cdot$ umb] im o \textbf{26} hân] habe n (o) \textbf{30} spîse] \textit{om.} o  $\cdot$ dâ] do n o \newline
\end{minipage}
\end{table}
\newpage
\begin{table}[ht]
\begin{minipage}[t]{0.5\linewidth}
\small
\begin{center}*G
\end{center}
\begin{tabular}{rl}
 & \textbf{\begin{large}S\end{large}i truoc einen salter} \textbf{in} der hant.\\ 
 & Parcival, der wîgant,\\ 
 & ein kleinez vingerlîn dâ kôs,\\ 
 & daz si durch arbeit nie \textbf{verkôs},\\ 
5 & \textbf{sine} behieltez durch rehter minnen \textbf{art}.\\ 
 & daz steinlîn was ein grânât.\\ 
 & des blic gap \textbf{ûz} \textbf{der vinster} schîn\\ 
 & \textbf{rehte} als ein \textbf{ander} gensterlîn.\\ 
 & senelîch was ir gebende.\\ 
10 & "dâ ûzen bî der wende",\\ 
 & sprach si, "hêrre, dâ stêt ein banc.\\ 
 & ruochet sitzen, \textbf{lêrtez} iuch iuwer \textbf{danc}\\ 
 & unde ander \textbf{iuwer} unmuoze.\\ 
 & daz ich her ze iuwer\textit{m} gruoz\\ 
15 & bin komen, daz \textbf{vergelt iu} got.\\ 
 & der giltet getriuwelîchen \textit{urbo}t."\\ 
 & der helt ir rât\textit{e}s \textit{ni}h\textit{t ver}gaz.\\ 
 & vür daz venster er dô saz.\\ 
 & er bat ouch dinne sitze\textit{n} sie.\\ 
20 & si sprach: "nû hân ich selten hie\\ 
 & gesezzen bî deheinem man."\\ 
 & der helt si vrâgen began\\ 
 & umbe ir \textbf{site} unde umbe ir pflege:\\ 
 & "daz ir sô verre von dem wege\\ 
25 & sitzet in dirre wilde!\\ 
 & ich hânz vür unbilde,\\ 
 & vrouwe, wes ir iuwich begêt,\\ 
 & sît hie \textbf{bouwes umbe iuch niht} stêt."\\ 
 & \textbf{\begin{large}D\end{large}ô sprach si}: "mir kumet vome Grâl\\ 
30 & \textbf{ein} spîse dâ her \textbf{al} sunder twâl.\\ 
\end{tabular}
\scriptsize
\line(1,0){75} \newline
G I L M Z Fr25 \newline
\line(1,0){75} \newline
\textbf{1} \textit{Initiale} G I L Z Fr25  \textbf{17} \textit{Initiale} I  \textbf{29} \textit{Initiale} G  \newline
\line(1,0){75} \newline
\textbf{1} einen] ein I (M)  $\cdot$ salter in] salte:::n Fr25 \textbf{2} Parcival] Parciuâl G Parzifal I L M Parcifal Z (Fr25) \textbf{3} ein kleinez] Einz chleinez G ein chlein I \textbf{4} arbeit] \textit{om.} Z  $\cdot$ verkôs] verlos I L (M) Fr25 \textbf{5} sine behieltez] sin hetez I Sie bihildesz M ::: behielt ez Fr25  $\cdot$ minnen] minn I minne (L) (M) Fr25  $\cdot$ art] \textit{om.} Fr25 tat L rat Z \textbf{6} daz steinlîn] der stain I  $\cdot$ ein] \textit{om.} I  $\cdot$ grânât] gra:at G granart I \textbf{7} ûz] zuͯ L (M) (Z) Fr25  $\cdot$ der vinster] dem venster I dem vinster L do venster Fr25 \textbf{8} ander] \textit{om.} I Fr25 \textbf{10} dâ ûzen] da vz I \textbf{11} dâ] \textit{om.} L \textbf{12} lêrtez] liez I lerisz M (Z)  $\cdot$ danc] gedanc I (L) M Z (Fr25) \textbf{13} iuwer] \textit{om.} I L Z Fr25 \textbf{14} daz] Da Z  $\cdot$ ze iuwerm] ze iuwern G (M) zuͯ uwermer L \textbf{15} vergelt] [vegelt]: vergelt L \textbf{16} giltet] gilten M  $\cdot$ getriuwelîchen] trewelichen Z (Fr25) truͯwe L getruwen M  $\cdot$ urbot] :::t G \textbf{17} râtes niht vergaz] rat:s :::h:::gaz G \textbf{18} dô] da M Z \textbf{19} sitzen] sitze G \textbf{21} deheinem] icheineme M keinem Z \textbf{22} der] [Des]: Der G \textbf{23} ir site] sitzen I  $\cdot$ umbe] \textit{om.} M \textbf{24} daz] [Da*]: Daz G \textbf{25} wilde] [werlde]: wilde G \textbf{27} begêt] hie beget I \textbf{28} hie bouwes] nih buͦwes hie I hie niht buwes L (M) Z (Fr25)  $\cdot$ umbe] bi I  $\cdot$ niht stêt] stet I L M Z Fr25 \textbf{29} Dô sprach si] Sẏ sprach da L (M) (Z) (Fr25)  $\cdot$ mir kumet] kvmt mir L (M) (Z) Fr25  $\cdot$ vome Grâl] \textit{om.} I \textbf{30} ein] min I (L) (M) (Z) :::n Fr25  $\cdot$ dâ] \textit{om.} I L  $\cdot$ her] \textit{om.} L  $\cdot$ al] \textit{om.} L M Fr25 \newline
\end{minipage}
\hspace{0.5cm}
\begin{minipage}[t]{0.5\linewidth}
\small
\begin{center}*T
\end{center}
\begin{tabular}{rl}
 & \textbf{si truoc einen salter} \textbf{an} der hant.\\ 
 & Parcifal, der wîgant,\\ 
 & ein kleinez vingerlîn dâ kôs,\\ 
 & daz si durch arbeit nie \textbf{verlôs},\\ 
5 & \textbf{si} behieltez durch rehter minnen \textbf{art}.\\ 
 & daz steinlîn was ein grânât.\\ 
 & des blic gab \textbf{zuo} \textbf{de\textit{m} venster} schîn\\ 
 & \textbf{reht} als ein \textbf{ander} ganeisterlîn.\\ 
 & senelîch was ir gebende.\\ 
10 & "dâ ûzen bî der wende",\\ 
 & sprach si, "hêrre, dâ stât ein banc.\\ 
 & ruochet sitzen, \textbf{lêret\textit{z}} \textit{i}u iuwer \textbf{gedanc}\\ 
 & und ander unmuoze.\\ 
 & daz ich her ziuwerm gruoze\\ 
15 & bin komen, daz \textbf{vergeltiu} got.\\ 
 & der giltet getriuwelîchen urbot."\\ 
 & \begin{large}D\end{large}er helt ir râtes niht vergaz.\\ 
 & vür daz venster er dô saz.\\ 
 & er bat ouch dâ inne sitzen sie.\\ 
20 & si sprach: "nû hân ich selten hie\\ 
 & gesezzen bî deheinem man."\\ 
 & der helt si vrâgen began\\ 
 & umbir \textbf{site} und umbir pflege:\\ 
 & "daz ir sô verre von dem wege\\ 
25 & sitzet in dirre wilde!\\ 
 & ich hânz vür unbilde,\\ 
 & vrouwe, wes ir iu begêt,\\ 
 & sît hie \textbf{niht bûwes umbiu} stêt."\\ 
 & \textbf{Si sprach}: "\textbf{dâ} kumt mir vonme Grâl\\ 
30 & \textbf{mîne} spîse dâ her sunder twâl.\\ 
\end{tabular}
\scriptsize
\line(1,0){75} \newline
T U V W O Q R \newline
\line(1,0){75} \newline
\textbf{1} \textit{Initiale} W O Q  \textbf{17} \textit{Initiale} T U  \textbf{29} \textit{Majuskel} T  \newline
\line(1,0){75} \newline
\textbf{1} Si] ÷i O  $\cdot$ einen] ein W  $\cdot$ an] in U V W O (Q) (R) \textbf{2} Parcifal] Parzifal T V Partzifal W Q Barcifal O Parczifal R \textbf{3} dâ] do U V W O R \textbf{4} si durch] [*urch]: sv́ durch V durch R \textbf{5} si behieltez] [S*hielt]: Sv́ behielt ez V  $\cdot$ minnen] minne W O (R)  $\cdot$ art] [*]: rat V rat W \textit{om.} O \textbf{6} grânât] granart Q \textbf{7} zuo] [*]: vz V ausz Q  $\cdot$ dem venster] der venster T W [dem v*nster]: der vinster V der finster Q \textbf{8} ein ander] [*]: ein fv́res V ein O  $\cdot$ ganeisterlîn] fensterlein W angesterlin O sternlin R \textbf{9} senelîch] Schemlich R  $\cdot$ gebende] gewende O \textbf{10} dâ] Do W Das Q \textbf{11} sprach si] Sy sprach R  $\cdot$ dâ] do U V W Q \textbf{12} ruochet] Gervͦchet O (R)  $\cdot$ lêretz iu iuwer gedanc] leret siv îuwer gedanc T [ler*]: leretzv́ch uwer gedang V lert úchs eúwer gedanck W lers euch ewr gedanck Q lert es uch úwer gedank R \textbf{13} und ander] [Vndander]: Vnd ander Q \textbf{14} ich] \textit{om.} O  $\cdot$ her] herre Q R  $\cdot$ gruoze] vnmuͯsse R \textbf{15} bin] Sind R  $\cdot$ vergeltiu] vergt uch R \textbf{16} getriuwelîchen urbot] getrv́welich [*bot]: vrbot V triwelichen vrbot O \textbf{17} Der] Des W  $\cdot$ ir râtes] irs ratens W irs Rattes R \textbf{19} bat] batt sy R  $\cdot$ dâ inne] do inne U Q do innen W da innen R \textbf{21} deheinem] dekeime U keinem W  $\cdot$ man] [*an]: man Q \textbf{23} umbir site] Vmb irn sitten R \textbf{24} ir] sy R \textbf{25} sitzet] Siczent R \textbf{26} unbilde] ein vngebilde Q ein vnbilde R \textbf{27} vrouwe] Erawe W [Frower]: Frowe O  $\cdot$ begêt] hie begat R \textbf{28} hie niht bûwes umbiu] vmb úch nit buwes R \textbf{29} dâ] do U V W O Q \textbf{30} mîne] Min V R Mit O  $\cdot$ dâ her] do her V W Q \textit{om.} R \newline
\end{minipage}
\end{table}
\end{document}
