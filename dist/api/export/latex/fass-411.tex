\documentclass[8pt,a4paper,notitlepage]{article}
\usepackage{fullpage}
\usepackage{ulem}
\usepackage{xltxtra}
\usepackage{datetime}
\renewcommand{\dateseparator}{.}
\dmyyyydate
\usepackage{fancyhdr}
\usepackage{ifthen}
\pagestyle{fancy}
\fancyhf{}
\renewcommand{\headrulewidth}{0pt}
\fancyfoot[L]{\ifthenelse{\value{page}=1}{\today, \currenttime{} Uhr}{}}
\begin{document}
\begin{table}[ht]
\begin{minipage}[t]{0.5\linewidth}
\small
\begin{center}*D
\end{center}
\begin{tabular}{rl}
\textbf{411} & \textbf{Gawan dô muose} entwîchen,\\ 
 & doch unlasterlîchen.\\ 
 & \begin{large}U\end{large}nders turnes tür er wart getân.\\ 
 & nû seht, dô kom der selbe man,\\ 
5 & der in \textbf{kampflîche an ê sprach}.\\ 
 & vor Artuse daz geschach.\\ 
 & der lantgrâve Kyngrimursel\\ 
 & gram durch swarten unt durch vel.\\ 
 & durch Gawans nôt sîne hende er want.\\ 
10 & wan \textbf{des was} sîn triwe pfant,\\ 
 & daz er dâ solte haben vride,\\ 
 & ez enwære, \textbf{daz} eines mannes lide\\ 
 & in \textbf{in} kampfe twungen.\\ 
 & die alten unt die jungen\\ 
15 & treib er vonme turne wider.\\ 
 & den hiez der künec brechen nider.\\ 
 & Kyngrimursel dô sprach\\ 
 & hin ûf, dâ er Gawanen sach:\\ 
 & "helt, gib mir vride \textbf{zuo dir} \textbf{dar} în.\\ 
20 & ich wil geselleclîchen pîn\\ 
 & mit dir hân in dirre nôt.\\ 
 & mich muoz der künec slahen tôt\\ 
 & oder ich behalte dir dîn leben."\\ 
 & Gawan den vride begunde geben.\\ 
25 & Der lantgrâve spranc zuo \textbf{z}im dar.\\ 
 & des zwîvelte diu ûzer schar.\\ 
 & er was ouch burcgrâve al dâ.\\ 
 & si wæren junc oder grâ,\\ 
 & \textbf{die} blûgeten an ir strîte.\\ 
30 & Gawan spranc an die wîte,\\ 
\end{tabular}
\scriptsize
\line(1,0){75} \newline
D \newline
\line(1,0){75} \newline
\textbf{3} \textit{Initiale} D  \textbf{25} \textit{Majuskel} D  \newline
\line(1,0){75} \newline
\newline
\end{minipage}
\hspace{0.5cm}
\begin{minipage}[t]{0.5\linewidth}
\small
\begin{center}*m
\end{center}
\begin{tabular}{rl}
 & \textbf{dô muose Gawan} entwîchen,\\ 
 & doch unlasterlîchen.\\ 
 & under \textit{des} turnes tür er wart getân.\\ 
 & nû seht, dô kom der selbe man,\\ 
5 & der \textit{in} \textbf{ê kampflîche an sprach}.\\ 
 & vor Artuse daz geschach.\\ 
 & der lantgrâve Kingrimursel\\ 
 & gram durch swarten und durch vel.\\ 
 & durch Gawan\textit{e}s nôt sîne hende er \textit{w}ant.\\ 
10 & wanne \textbf{des was} sîn triuwe pfant,\\ 
 & daz er dâ solte haben vride,\\ 
 & ez enwære, \textbf{daz} eines mannes lide\\ 
 & in \textbf{in} kampfe twungen.\\ 
 & die alten und die jungen\\ 
15 & treip er von dem turne wider.\\ 
 & den hie der künic brechen nider.\\ 
 & Kingri\textit{m}ursel dô sprach\\ 
 & hin ûf, dâ er Gawanen sach:\\ 
 & "helt, gip mir vride \textbf{zuo dir} \textbf{dar} în.\\ 
20 & ich wil geselleclîchen pîn\\ 
 & mit dir hân in dirre nôt.\\ 
 & mich muoz der künic slahen tôt\\ 
 & oder ich behalte dir dîn leben."\\ 
 & Gawan den vride begunde geben.\\ 
25 & der lantgr\textit{â}v\textit{e} spranc zuo ime dar.\\ 
 & des zwîfelte diu ûzere schar.\\ 
 & er was ouch b\textit{ur}cgrâve aldâ.\\ 
 & si wæren junc oder grâ,\\ 
 & \textbf{si} blûgeten an ir strîte.\\ 
30 & Gawan spranc an die wîte,\\ 
\end{tabular}
\scriptsize
\line(1,0){75} \newline
m n o \newline
\line(1,0){75} \newline
\newline
\line(1,0){75} \newline
\textbf{1} muose] musse m muͯste n (o) \textbf{3} under des turnes] Vnd erturnes m  $\cdot$ tür] tor n \textbf{5} in ê] nie m ine n o  $\cdot$ kampflîche] campliche o \textbf{6} Artuse] artuͯse o \textbf{7} Kingrimursel] kingrumúrsel n kuͯnde gruͯmel o \textbf{9} Gawanes] gawanens m gawans n gawes o  $\cdot$ want] vant m \textbf{11} dâ] do n o \textbf{12} enwære] were n o  $\cdot$ lide] [sit]: lide o \textbf{13} twungen] twingen n o \textbf{14} alten] alte n o \textbf{17} Kingrimursel] Kingrinvrsel m Kingrumúrsel n Kingrimisel o \textbf{18} dâ] do n o \textbf{19} helt] Halt n Hilt o \textbf{20} geselleclîchen] gesellecliche n [gesec]: geselecliche o \textbf{21} dir] der n \textbf{24} vride] friden n o  $\cdot$ begunde] [muste]: begunde o \textbf{25} lantgrâve] lantgreffin m langgroffe n  $\cdot$ ime] in o \textbf{26} zwîfelte] zer spielte n (o) \textbf{27} er] Es n o  $\cdot$ burcgrâve] brugraffe m \textbf{30} Gawan] Gawann o \newline
\end{minipage}
\end{table}
\newpage
\begin{table}[ht]
\begin{minipage}[t]{0.5\linewidth}
\small
\begin{center}*G
\end{center}
\begin{tabular}{rl}
 & \textbf{Gawan dô muose} entwîchen,\\ 
 & \textit{d}och unlasterlîchen.\\ 
 & unders turnes tür er wart getân.\\ 
 & nû seht, dô kom der selbe man,\\ 
5 & der in \textbf{kampflîche an sprach}.\\ 
 & vor Artuse daz geschach.\\ 
 & \begin{large}D\end{large}er lantgrâve Kingrimursel,\\ 
 & gram durch swarten unde durch vel.\\ 
 & durch Gawanes nô\textit{t} \textit{s}îne hende \textit{er} want,\\ 
10 & wan \textbf{des was} sîn triwe pfant,\\ 
 & daz er dâ solte haben vride,\\ 
 & ez enwære, \textbf{daz} eines mannes lide\\ 
 & in \textbf{in} kampfe twungen.\\ 
 & die alten unde die jungen\\ 
15 & treip er vome turne wider.\\ 
 & den hiez der künic brechen nider.\\ 
 & Kingrimursel dô sprach\\ 
 & hin ûf, dâ er Gawanen sach:\\ 
 & "helt, gip mir vride \textbf{zuo dir} \textbf{hin} în.\\ 
20 & ich wil geselliclîchen pîn\\ 
 & mit dir hân in dirre nôt.\\ 
 & mich muoz der künic slahen tôt\\ 
 & oder ich behalte dir dîn leben."\\ 
 & Gawan den vride begunde geben.\\ 
25 & der lantgrâve spranc zuo im dar.\\ 
 & des zwîfelte diu ûzer schar.\\ 
 & er was ouch burcgrâve al dâ.\\ 
 & si w\textit{æ}ren junc oder grâ,\\ 
 & \textbf{die} blûgten an ir strîte.\\ 
30 & Gawan spranc an die wîte,\\ 
\end{tabular}
\scriptsize
\line(1,0){75} \newline
G I O L M Q R Z \newline
\line(1,0){75} \newline
\textbf{3} \textit{Initiale} I O L Z   $\cdot$ \textit{Capitulumzeichen} R  \textbf{7} \textit{Initiale} G  \textbf{17} \textit{Initiale} I  \newline
\line(1,0){75} \newline
\textbf{1} \textit{Die Verse 370.13-412.12 fehlen} Q   $\cdot$ Gawan] Gawin R  $\cdot$ dô] \textit{om.} I da L Z der M \textbf{2} doch] iedoch G \textbf{3} unders] ÷nder des O Von des R \textbf{4} dô] da M \textbf{5} kampflîche an sprach] champflich an e sprach O E kempfliche ane sprach L kemphliche e an sprach M (Z) kamplich ane sprach R \textbf{6} Artuse] Atus R \textbf{7} Kingrimursel] Kyngrimvrsel O kyngrymursel M kᵫngrumursel R \textbf{8} swarten] swart I  $\cdot$ durch] \textit{om.} M \textbf{9} Gawanes] Gawans I O (M) Z Gawanz L Gawins R  $\cdot$ nôt sîne hende er] not er sine hende G not die hende er L \textbf{10} wan] Was Z \textbf{11} dâ solte] solde da I \textbf{12} enwære daz] ward ob R \textbf{13} in in] Jn eim L  $\cdot$ twungen] twuͯgen L \textbf{15} treip] Die treip L \textbf{17} Kingrimursel] Kyngrimvrsel O Kingrýmvrsel L Kyngrymursel M Kungrumursel R  $\cdot$ dô] da M \textbf{18} hin] Gin L  $\cdot$ dâ] do R  $\cdot$ Gawanen] Gawan I O (M) (Z) Gawinen R \textbf{19} hin] dar O L M R Z \textbf{20} geselliclîchen] gesellenkliche R \textbf{22} tôt] ze tot O \textbf{24} Gawan] Gawa M Gawin R  $\cdot$ den vride begunde] begunde den fride I den frid gunde R \textbf{25} im] zim O \textbf{26} zwîfelte] zwifelt O (R) Z \textbf{27} er] Es R  $\cdot$ burcgrâve] buͯrgare L  $\cdot$ al] \textit{om.} I \textbf{28} si wæren] si waren G (L) (M) Er were R  $\cdot$ oder] vnde M \textbf{29} die] Si O  $\cdot$ blûgten] blvͦgenten O blovgen L bluweten R \textbf{30} Gawan] Gawin R  $\cdot$ spranc] chom I \newline
\end{minipage}
\hspace{0.5cm}
\begin{minipage}[t]{0.5\linewidth}
\small
\begin{center}*T
\end{center}
\begin{tabular}{rl}
 & \textbf{Gawan dô muose} entwîchen,\\ 
 & doch unlasterlîchen.\\ 
 & unders turnes tür er wart getân.\\ 
 & Nû seht, dô kom der selbe man,\\ 
5 & der in \textbf{kampflîche an ê sprach}.\\ 
 & vor Artuse daz geschach.\\ 
 & der lantgrâve Kyngrimursel\\ 
 & gram durch swarten unde durch vel,\\ 
10 & \hspace*{-.7em}\big| wan \textbf{dâ wære} sîn triuwe pfant\\ 
 & \hspace*{-.7em}\big| durch Gawans nôt sîn hende er want,\\ 
 & daz er dâ solte haben vride,\\ 
 & ez enwære, \textbf{dâ} eines mannes lide\\ 
 & in \textbf{an} kampfe twungen.\\ 
 & die alten unde die jungen,\\ 
15 & \textbf{die} treib er von dem turne wider.\\ 
 & den hiez der künec brechen nider.\\ 
 & \begin{large}K\end{large}yngrimursel dô sprach\\ 
 & hin ûf, dâ er Gawanen sach:\\ 
 & "helt, gip mir vride \textbf{hin ûf} \textbf{dar} în.\\ 
20 & ich wil geselleclîchen pîn\\ 
 & mit dir hân in dirre nôt.\\ 
 & mich muoz der künec slahen tôt\\ 
 & oder ich behalte dir dîn leben."\\ 
 & Gawan den vride begunde geben.\\ 
25 & Der lan\textit{t}grâve \textit{sprang} zim\textit{e d}ar.\\ 
 & des zwîvelte di\textit{u} ûzer schar.\\ 
 & er was ouch burcgrâve aldâ.\\ 
 & si wæren junc oder grâ,\\ 
 & \textbf{diu} blûgeten an ir strîte.\\ 
30 & Gawan spranc an die wîte,\\ 
\end{tabular}
\scriptsize
\line(1,0){75} \newline
T U V W \newline
\line(1,0){75} \newline
\textbf{4} \textit{Majuskel} T  \textbf{17} \textit{Initiale} T U W  \textbf{25} \textit{Majuskel} T  \newline
\line(1,0){75} \newline
\textbf{1} Gawan] Gawau W  $\cdot$ muose] mvese T \textbf{3} tür] torn U \textbf{5} kampflîche an ê sprach] e kampflich ane sprach V \textbf{7} Kyngrimursel] kyngrimorsel U kẏngrimursel V kingrimursel W \textbf{10} \textit{Versfolge 411.9-10} W   $\cdot$ Wann des was sein trúw sein pfand W  $\cdot$ dâ wære] do were U dez waz V \textbf{11} dâ] do U V W  $\cdot$ solte haben] haben solte V \textbf{12} dâ] daz U (V) (W) \textbf{13} an] in V \textbf{15} die] \textit{om.} W \textbf{17} Kyngrimursel] Kyngrimuͦrsel U Kẏngrimursel V \textbf{18} dâ] do U V W  $\cdot$ Gawanen] gawan W Gawane V \textbf{19} hin ûf] hin of zuͦ dir U zvͦ dir V zuͦ dir hin W \textbf{23} behalte] wil behalten W \textbf{25} Der Langrave (lantgrave U ) zim do sprach dar T (U) \textbf{26} diu] die T \textbf{28} oder] alde U \textbf{29} blûgeten] bluͦgete U \newline
\end{minipage}
\end{table}
\end{document}
