\documentclass[8pt,a4paper,notitlepage]{article}
\usepackage{fullpage}
\usepackage{ulem}
\usepackage{xltxtra}
\usepackage{datetime}
\renewcommand{\dateseparator}{.}
\dmyyyydate
\usepackage{fancyhdr}
\usepackage{ifthen}
\pagestyle{fancy}
\fancyhf{}
\renewcommand{\headrulewidth}{0pt}
\fancyfoot[L]{\ifthenelse{\value{page}=1}{\today, \currenttime{} Uhr}{}}
\begin{document}
\begin{table}[ht]
\begin{minipage}[t]{0.5\linewidth}
\small
\begin{center}*D
\end{center}
\begin{tabular}{rl}
\textbf{805} & \textit{\begin{large}D\end{large}}iu \textbf{magetuomlîch minne im} gap,\\ 
 & \textbf{dô} si lebte, unt \textbf{sluogen} zuo daz grap.\\ 
 & Condwiramurs begunde klagen\\ 
 & ir vetern tohter, hôrt ich sagen,\\ 
5 & \textbf{unt} \textbf{wart} \textbf{vil} vreuden âne,\\ 
 & wande si Schoysiane,\\ 
 & der tôten meide muoter, zôch\\ 
 & kint wesende - dâr \textbf{umbe} si vreude vlôch -,\\ 
 & diu Parzivals muome was,\\ 
10 & ob der Provenzâl die wârheit las.\\ 
 & Der herzoge Kyot\\ 
 & wesse wênec umbe sîner tohter tôt,\\ 
 & des \textbf{künec} Kardeyzes magezoge.\\ 
 & \textbf{ez} ist niht \textbf{krump} \textbf{alsô} \textbf{der} boge,\\ 
15 & \textbf{diz mære} ist wâr unt sleht.\\ 
 & \textbf{si tâten dô der reise} \textbf{ir} reht,\\ 
 & \textbf{bî} naht \textbf{gein Munsalvæsche si} riten.\\ 
 & \textbf{dâ} het ir \textbf{Feirefiz} \textbf{gebiten}\\ 
 & mit \textbf{kurzwîle} \textbf{die} stunde.\\ 
20 & vil kerzen man \textbf{dô} enzunde,\\ 
 & \textbf{reht} als ob brünne gar der walt.\\ 
 & Ein templeis von Patrigalt\\ 
 & gewâpent bî der küneginne reit.\\ 
 & der hof was wît und breit.\\ 
25 & dâr \textbf{ûffe stuont} manec \textbf{sunderschar}.\\ 
 & \textbf{si} enpfiengen die küneginne gar\\ 
 & \textbf{unt} den wirt unt den sun sîn.\\ 
 & dô truoc man Loherangrin\\ 
 & gein sînem veter Feirefiz.\\ 
30 & dô der was swarz unt wîz,\\ 
\end{tabular}
\scriptsize
\line(1,0){75} \newline
D \newline
\line(1,0){75} \newline
\textbf{1} \textit{Initiale} D  \textbf{11} \textit{Majuskel} D  \textbf{22} \textit{Majuskel} D  \newline
\line(1,0){75} \newline
\textbf{1} Diu] ÷iv D \textbf{3} Condwiramurs] Conwir amvrs D \textbf{6} Schoysiane] Scoysiane D \textbf{9} Parzivals] Parcifals D \textbf{17} Munsalvæsche] Mvnsalvæsce D \newline
\end{minipage}
\hspace{0.5cm}
\begin{minipage}[t]{0.5\linewidth}
\small
\begin{center}*m
\end{center}
\begin{tabular}{rl}
 & diu \textbf{m\textit{a}g\textit{et}lîche minne im} gap,\\ 
 & \textbf{dô} si lebte, und \textbf{sluogen} zuo daz grap.\\ 
 & Condwier amu\textit{r}s begunde klagen\\ 
 & ir veter tohter, hôrt ich sa\textit{g}en,\\ 
5 & \textbf{und} \textbf{war\textit{t}} \textbf{vil} vröuden âne,\\ 
 & wand si Schoisi\textit{a}ne,\\ 
 & der tôten megde muoter, zôch\\ 
 & kint wesende - dâr \textbf{umbe} si vröude vlôch -,\\ 
 & diu Parcifals m\textit{uom}e was,\\ 
10 & ob der Prov\textit{e}nzâl die wârheit las.\\ 
 & der herzoge Kyot\\ 
 & weste wênic umb sîner tohter tôt,\\ 
 & des \textbf{künic} \textit{C}ard\textit{e}i\textit{z}es magezoge.\\ 
 & \textbf{ez} ist niht \textbf{krumb} \textbf{als} \textbf{ein} boge,\\ 
15 & \textbf{diz mær}, \textbf{ez} ist wâr und sleht.\\ 
 & \textbf{si tâten dô der reise} reht,\\ 
 & \textbf{bî} naht \textbf{gegen Mun\textit{t}salvasche si} riten.\\ 
 & \textbf{d\textit{â}} het ir \textbf{Ferefiz} \textbf{gebiten}\\ 
 & mit \textbf{kurzewîle} \textbf{dise} stunde.\\ 
20 & vil kerzen man \textbf{dô} enzunde,\\ 
 & \textbf{reht} als ob brünne gar der walt.\\ 
 & ein templeis von \textit{P}a\textit{t}r\textit{i}galt\\ 
 & gewâpent bî der künigîn reit.\\ 
 & der hof was wît und breit.\\ 
25 & dâr \textbf{ûf stuont} manic \textbf{sunderschar}.\\ 
 & \textbf{si} enpfiengen die künigîn gar,\\ 
 & den wirt und den sun sîn.\\ 
 & dô truoc man Lohelangrin\\ 
 & gegen sînem vetern Ferefiz.\\ 
30 & dô der was swarz und wîz,\\ 
\end{tabular}
\scriptsize
\line(1,0){75} \newline
m n o V V' W \newline
\line(1,0){75} \newline
\textbf{3} \textit{Überschrift:} Hie kvmmet kv́nig Parzefal mit Sineme wibe kvndewiramurs vnde mit sineme svnne Lohelangrin Zvͦme grole V  Hie kvmet parzifal mit siner frouwen zv dem gral V'   $\cdot$ \textit{Initiale} V V'  \newline
\line(1,0){75} \newline
\textbf{1} diu] Sie o  $\cdot$ magetlîche] megliche m magetuͦmliche V (V')  $\cdot$ im] >sy< ime V' \textbf{2} lebte] lebten o  $\cdot$ und sluogen] man sluc V' \textbf{3} Condwier amurs] Condwier amurus m Condewier amirs n Cunwir amirs o Kvndewiramurs V (W) KVndewiramors V' \textbf{4} veter] vettern V  $\cdot$ hôrt] [horl]: hort m hor V'  $\cdot$ sagen] sagagen m \textbf{5} wart] war m  $\cdot$ vröuden] freuͯide o \textbf{6} Schoisiane] scoisiasne m scosiane n o sociane V V' tschosiane W \textbf{8} wesende] wesen V'  $\cdot$ vröude] frouͯiden o (V') \textbf{9} Parcifals] Parzefals V parzifals V' herr partzifals W  $\cdot$ muome] mẏnne m (W) niemer o \textbf{10} Provenzâl] provinzal m provinzol n profinckal o prouenzal V W [*]: prouenzal V'  $\cdot$ las] >ich< laz V' \textbf{11} Kyot] kẏot m n o \textbf{12} weste] Wieste V' \textbf{13} \textit{Die Verse 805.13-16 fehlen} V'   $\cdot$ des künic] Des kv́niges n (V) (W) Der konig o  $\cdot$ Cardeizes] mardices m marditzes n madiczes o kardeiszez V kardeis W \textbf{15} diz] Dise n W  $\cdot$ ez] \textit{om.} n  $\cdot$ wâr] wore n \textbf{16} reht] ir reht V \textbf{17} \textit{statt 805.17-20:} zv houe sy do riten / Do hatte ir kvnic artus gebiten / Mit kurzewile stunde gar (vgl. 805.26: gar) / Kvnic artus quam gein in dar (Fortsetzung in 805.26) V'   $\cdot$ Bei nacht sy gen montsaluatz ritten W  $\cdot$ Muntsalvasche] Mundsaluasce m muntsaluasce n (o) mvnsalfasche V  $\cdot$ si] \textit{om.} o \textbf{18} dâ] Do m n o V W  $\cdot$ Ferefiz] ferefis m o V ferrefis n artus vnde ferefis V ferafis W \textbf{19} dise] die n V \textbf{20} man dô] do man o \textbf{21} als ob] ob o als W  $\cdot$ brünne] brante W \textbf{22} Patrigalt] gargalt m n o W \textbf{26} Vnd enphinc die kvnigin frolich \textit{(Fortsetzung von 805.20)} V' \textbf{27} Vnd die herren alle gelich V'  $\cdot$ den wirt] Vnde den wirt V  $\cdot$ sîn] mynn o \textbf{28} \textit{Die Verse 805.28-806.30 fehlen} V'   $\cdot$ Lohelangrin] lohelangrein W \textbf{29} vetern] vetter o (V) W  $\cdot$ Ferefiz] ferefis m o V ferrevis n ferafiß W \textbf{30} dô der was] Do wasz er o Der do was W \newline
\end{minipage}
\end{table}
\newpage
\begin{table}[ht]
\begin{minipage}[t]{0.5\linewidth}
\small
\begin{center}*G
\end{center}
\begin{tabular}{rl}
 & \begin{large}D\end{large}iu \textbf{im magetuomlîche minne} gap,\\ 
 & \textbf{die wîl} si lebet, \textit{unde} \textbf{sluoc} zuo daz grap.\\ 
 & Kondwiramurs begunde klagen\\ 
 & ir veteren tohter, hôrt ich sagen,\\ 
5 & \textbf{daz si} \textbf{was} vröuden âne,\\ 
 & wande si Schoysiane,\\ 
 & der tôten meide muoter, zôch\\ 
 & kint wesende - dâr \textbf{umbe} si vröude vlôch -,\\ 
 & diu Parcivals muome was,\\ 
10 & op der Provenzâl die wârheit las.\\ 
 & der herzoge Kiot\\ 
 & wesse wênic umbe sîner tohter tôt,\\ 
 & des \textbf{küniges} Kardeizes magezoge.\\ 
 & \textbf{ditze mære} ist niht \textbf{sô} \textbf{der} boge,\\ 
15 & \textbf{ez} ist wâr unde sleht.\\ 
 & \textbf{der reise tâten si dô} reht,\\ 
 & \textbf{die} naht \textbf{si gein Muntsalfatsche} riten.\\ 
 & mit \textbf{vröuden} het \textit{i}r \textbf{dâ} \textbf{erbiten}\\ 
 & \textbf{Feirafiz} \textbf{die} stunde.\\ 
20 & vil kerzen man enzunde,\\ 
 & als obe brünne gar der walt.\\ 
 & ein templeis \textit{von} Patrigalt\\ 
 & gewâpent bî der küniginne reit.\\ 
 & der hof was wît unde breit.\\ 
25 & dâ \textbf{stuont ûf} manic \textbf{sunderschar}.\\ 
 & \textbf{die} enpfiengen die künigîn gar,\\ 
 & den wirt unde den sun sîn.\\ 
 & dô truoc man Loherangrin\\ 
 & gein sînem veteren Feirafiz.\\ 
30 & dô der was swarz unde wîz,\\ 
\end{tabular}
\scriptsize
\line(1,0){75} \newline
G I L Z Fr48 \newline
\line(1,0){75} \newline
\textbf{1} \textit{Initiale} G Z Fr48  \textbf{3} \textit{Initiale} I  \textbf{15} \textit{Initiale} L  \textbf{23} \textit{Initiale} I  \newline
\line(1,0){75} \newline
\textbf{1} \textit{Die Verse 805.1-14 fehlen} L  \textbf{2} lebet] lebete I (Fr48)  $\cdot$ unde] man G  $\cdot$ sluoc] slvgen Z (Fr48) \textbf{3} Kondwiramurs] kvndewiramvrs G Conduwiramurs I Kvndwiramurs Z (Fr48) \textbf{4} hôrt] hoͤr I \textbf{5} was] wart Z Fr48 \textbf{6} Schoysiane] schoisiane I (Fr48) tschosiane Z \textbf{9} Parcivals] parzivals G Parzifals I parcifals Z \textbf{10} Provenzâl] prouenzal I \textbf{11} Kiot] kyot Z \textbf{13} küniges] chunc I (Z)  $\cdot$ Kardeizes] kardaiz I kardeiz Z \textbf{15} ez ist wâr] Dis mere ist ware L \textbf{16} Sie taten da der reise ir reht Z \textbf{17} si] \textit{om.} I Z  $\cdot$ Muntsalfatsche] mvntschaluatsch G Muntsaluasce I Mvntsalvatsche L montsalvatsche Z  $\cdot$ riten] si riten I \textbf{18} mit vröuden] Mit freude I (L) Sie mit frevden Z  $\cdot$ ir] er G  $\cdot$ erbiten] gebiten L Z \textbf{19} Feirafiz] firaviz G Ferefiz L Feirefiz Z \textbf{20} enzunde] in zunde I \textbf{22} von] der G \textbf{25} dâ stuont ûf] Drvfe stvnt L \textbf{27} den wirt] Vnd den wirt Z \textbf{28} dô] Da Z  $\cdot$ Loherangrin] Leheringrin I joherangrin L Lohagrin Z \textbf{29} Feirafiz] firaviz G ferefiz L feirefiz Z \textbf{30} dô der] Sit er L Da der Z \newline
\end{minipage}
\hspace{0.5cm}
\begin{minipage}[t]{0.5\linewidth}
\small
\begin{center}*T
\end{center}
\begin{tabular}{rl}
 & diu \textbf{im magetuomlîche minne} gap,\\ 
 & \textbf{die wîle} si lebete, und \textbf{sluoc} zuo daz grap.\\ 
 & Kundewiramurs begunde klagen\\ 
 & ir vetern tohter, hôrt ich sagen,\\ 
5 & \textbf{daz si} \textbf{wart} vreuden âne,\\ 
 & wanne si Schosiane,\\ 
 & der tôten magede muoter, zôch\\ 
 & kint wesende - dâ \textbf{von} si vreude vlôch -,\\ 
 & diu Parcifals muome was,\\ 
10 & ob der Provenzâl die wârheit las.\\ 
 & der herzoge Kyot\\ 
 & wiste wênic umb sîner tohter tôt,\\ 
 & des \textbf{küneges} Kardeizes magezoge.\\ 
 & \textbf{di\textit{z} mære} ist niht \textbf{sô} \textbf{der} boge,\\ 
15 & \textbf{ez} ist wâr und sleht.\\ 
 & \textbf{der reisen tâten si} \textbf{ir} reht,\\ 
 & \textbf{die} naht \textbf{si gein Munsalvasche} riten.\\ 
 & mit \textbf{vreuden} het ir \textbf{dâ} \textbf{gebiten}\\ 
 & \textbf{Ferefis} \textbf{die} stunde.\\ 
20 & vil kerzen man \textbf{dô} enzunde,\\ 
 & als ob br\textit{ü}n\textit{n}e gar der walt.\\ 
 & ein templeis von Patrigalt\\ 
 & gewâpent bî der küneginne reit.\\ 
 & der hof was wît und breit.\\ 
25 & dâ \textbf{stuont ûffe} ma\textit{n}egiu \textbf{wîte schar}.\\ 
 & \textbf{die} enpfiengen die küneginne gar,\\ 
 & den wirt und den sun sîn.\\ 
 & dô truoc man Lohrangrin\\ 
 & gein sîme vetern Ferefis.\\ 
30 & dô der was swarz und wîz,\\ 
\end{tabular}
\scriptsize
\line(1,0){75} \newline
U Q R \newline
\line(1,0){75} \newline
\newline
\line(1,0){75} \newline
\textbf{1} magetuomlîche] megtlichú R \textbf{2} lebete] lept R  $\cdot$ grap] [gab]: grabp Q \textbf{3} Kundewiramurs] Kuͦndewiramuͦrs U kundwiramurs Q Kúndwuramuͦrs R \textbf{4} vetern] vetter R  $\cdot$ ich] in R \textbf{6} si] sie wart Q  $\cdot$ Schosiane] tschosiane Q Shoisiane R \textbf{8} wesende] wissende R \textbf{9} Parcifals] Parzifals U partzifals Q parczifals R  $\cdot$ muome] muͦtter R \textbf{10} Provenzâl] prouenzale R  $\cdot$ die] du R \textbf{11} Kyot] kyott Q \textbf{12} tôt] [tod]: not R \textbf{13} küneges] [kindes]: kúnges R  $\cdot$ magezoge] Magt ogt R \textbf{14} diz] Dise U \textbf{16} reisen] reysze Q (R) \textbf{17} Munsalvasche] muͦntsalvatsche U muntsaluasche Q Munsaluashe R \textbf{18} dâ] do Q \textbf{19} Ferefis] feyrefisz Q Feirefis R \textbf{20} dô] \textit{om.} Q R  $\cdot$ enzunde] anzunde R \textbf{21} brünne] brente U  $\cdot$ der] ver Q \textbf{22} Patrigalt] Patergalt U [patrigatt]: patrigalt R \textbf{23} küneginne] kunginen R \textbf{25} manegiu] mage U maniger Q  $\cdot$ wîte] sunder Q R \textbf{28} Lohrangrin] Lorangrin U [loheangi*]: loheangrin R \textbf{29} Ferefis] feyrefisz Q feirefis R \textbf{30} was] was da R \newline
\end{minipage}
\end{table}
\end{document}
