\documentclass[8pt,a4paper,notitlepage]{article}
\usepackage{fullpage}
\usepackage{ulem}
\usepackage{xltxtra}
\usepackage{datetime}
\renewcommand{\dateseparator}{.}
\dmyyyydate
\usepackage{fancyhdr}
\usepackage{ifthen}
\pagestyle{fancy}
\fancyhf{}
\renewcommand{\headrulewidth}{0pt}
\fancyfoot[L]{\ifthenelse{\value{page}=1}{\today, \currenttime{} Uhr}{}}
\begin{document}
\begin{table}[ht]
\begin{minipage}[t]{0.5\linewidth}
\small
\begin{center}*D
\end{center}
\begin{tabular}{rl}
\textbf{310} & \textbf{\begin{large}D\end{large}ô} kom vrou Gynover dar\\ 
 & mit maneger vrouwen lieht gevar,\\ 
 & \textbf{mit ir} manec edel vürstîn.\\ 
 & die truogen minneclîchen schîn.\\ 
5 & \textbf{ouch was der rinc} \textbf{genomen} sô wît,\\ 
 & daz âne gedrenge unt âne strît\\ 
 & \textbf{manec} vrouwe bî ir âmîs saz.\\ 
 & Artus, der \textbf{valsches} laz,\\ 
 & \textbf{brâht} den Waleis \textbf{an} der hant.\\ 
10 & Vrou Cunneware de Lalant\\ 
 & gieng im \textbf{anderthalben} bî.\\ 
 & diu was dô \textbf{trûrens worden} vrî.\\ 
 & Artus an den Waleis sach.\\ 
 & nû sult ir hœren, wie er sprach:\\ 
15 & "ich wil iweren clâren lîp\\ 
 & \textbf{lâzen küssen} mîn altez wîp.\\ 
 & des \textbf{en}dorft ir doch \textbf{hie} \textbf{niemen} biten,\\ 
 & sît ir von Pelrapeire geriten,\\ 
 & \textbf{wan} dâ ist des kusses \textbf{hœhster} zil.\\ 
20 & eines dinges ich iuch bitten wil:\\ 
 & kom ich immer in iwer hûs,\\ 
 & gelt disen kus", sprach Artus.\\ 
 & "ich tuon, swes ir mich bittet, dâ",\\ 
 & sprach der Waleis, "unt \textbf{ouch} anderswâ."\\ 
25 & Ein \textbf{lützel} \textbf{gein im si dô} gienc,\\ 
 & diu künegîn in mit kusse enpfienc.\\ 
 & "nû verkiuse ich hie mit triwen",\\ 
 & \textbf{sprach si}, "daz ir mit riwen\\ 
 & \textbf{mich} liezet. die het ir mir gegeben,\\ 
30 & dô ir \textbf{dem künege} Ither nâmt sîn leben."\\ 
\end{tabular}
\scriptsize
\line(1,0){75} \newline
D \newline
\line(1,0){75} \newline
\textbf{1} \textit{Initiale} D  \textbf{10} \textit{Majuskel} D  \textbf{25} \textit{Majuskel} D  \newline
\line(1,0){75} \newline
\textbf{30} Ither] Jther D \newline
\end{minipage}
\hspace{0.5cm}
\begin{minipage}[t]{0.5\linewidth}
\small
\begin{center}*m
\end{center}
\begin{tabular}{rl}
 & \textbf{dô} kam vrouwe Ginovere dar\\ 
 & mit maniger vrouwen lieht gevar,\\ 
 & \textbf{mit ir} manic edel vürstîn.\\ 
 & die truogen minneclîchen schîn.\\ 
5 & \textbf{ouch was der rinc} sô wît,\\ 
 & daz âne gedrenge und âne strît\\ 
 & \textbf{ieglîch} vrouwe bî ir \textbf{kint} âmîs saz.\\ 
 & Artus, der \textbf{valsches} laz,\\ 
 & \textbf{brâhte} den Waleis \textbf{an} der hant.\\ 
10 & vrouwe Cu\textit{nne}w\textit{a}re de Lalant\\ 
 & gienc ime \textbf{anderhalben} bî.\\ 
 & diu was dô \textbf{trûrens worden} vrî.\\ 
 & \begin{large}A\end{large}rtus an den Waleis sach.\\ 
 & nû sollet ir hœren, wie er sprach:\\ 
15 & "ich wil iuwern clâren lîp\\ 
 & \textbf{küssen lâ\textit{n}} mî\textit{n} altez wîp.\\ 
 & des \textbf{en}dorf\textit{t} \textit{ir} doch \textbf{hie} \textbf{niemen} biten,\\ 
 & sît ir von Pelraperie geriten,\\ 
 & \textbf{wan} d\textit{â} ist des kusses \textbf{hœheste\textit{r}} zil.\\ 
20 & eines dinges ich iuch bitten wil:\\ 
 & kume ich iemer in iuwer hûs,\\ 
 & \textbf{sô} gelt disen kus", sprach Ar\textit{t}us.\\ 
 & "ich tuon, wes ir mich bittet, d\textit{â}",\\ 
 & sprach der Waleis, "und \textbf{ouch} anderswâ."\\ 
25 & ein \textbf{lütze\textit{l}} \textbf{si gegen ime dô} gienc,\\ 
 & diu künigîn in mit kusse enpfienc.\\ 
 & "nû verkiu\textit{s} ich hie mit triuwen",\\ 
 & \textit{\textbf{sprach si}}\textit{, "daz ir mit riuwen}\\ 
 & \textbf{mich} liezet. die het \textit{i}r mir gegeben,\\ 
30 & dô ir \textbf{dem künic} I\textit{t}hern nâmet sîn leben."\\ 
\end{tabular}
\scriptsize
\line(1,0){75} \newline
m n o \newline
\line(1,0){75} \newline
\textbf{13} \textit{Initiale} m n o  \newline
\line(1,0){75} \newline
\textbf{1} Ginovere] ginofer n o  $\cdot$ dar] [das]: dar m \textbf{5} rinc sô] nuwe ring genomen n (o) \textbf{7} kint] \textit{om.} n o \textbf{9} der] den n \textbf{10} Cunneware] Cumuwere m conneware n Conne waren o \textbf{13} den] der n dem o  $\cdot$ Waleis] waleise o \textbf{14} er] \textit{om.} o \textbf{16} lân mîn] lant mich \textit{nachträglich korrigiert zu:} lant disz m \textbf{17} endorft ir] endorfte m endarff o  $\cdot$ niemen] nẏmans n \textbf{18} Pelraperie] [per]: pelrapeir n pelrapier o \textbf{19} dâ] do m n  $\cdot$ kusses] kostez o  $\cdot$ hœhester] hoͯhesten m \textbf{22} Artus] arimus m arinus n arenuͯs o \textbf{23} ir] [ich]: ir m  $\cdot$ bittet] bitten o  $\cdot$ dâ] do m n o \textbf{24} anderswâ] anders swo o \textbf{25} lützel] luczen m  $\cdot$ gegen ime dô] do gegen ẏm o \textbf{26} enpfienc] do enpfing n \textbf{27} verkius] verkusch m  $\cdot$ mit] myn o \textbf{28} \textit{Vers 310.28 fehlt} m  \textbf{29} ir] er m  $\cdot$ gegeben] geben n gegen o \textbf{30} Ithern] ichern m ichter n  $\cdot$ nâmet] nomen m (o) \newline
\end{minipage}
\end{table}
\newpage
\begin{table}[ht]
\begin{minipage}[t]{0.5\linewidth}
\small
\begin{center}*G
\end{center}
\begin{tabular}{rl}
 & \textbf{ouch} kom vrou Schinovere dar\\ 
 & mit maniger vrouwen lieht gevar,\\ 
 & \textbf{mit ir} manic edele vürstîn.\\ 
 & die truogen minniclîchen schîn.\\ 
5 & \textbf{der rinc was} \textbf{wol genomen} sô wît,\\ 
 & daz âne gedrenge unde âne strît\\ 
 & \textbf{manic} vrouwe bî ir âmîs saz.\\ 
 & Artus, der \textbf{valsches} laz,\\ 
 & \textbf{dô} \textbf{brâhte} den Waleis \textbf{an} der hant.\\ 
10 & vrou Kuneware de Lalant\\ 
 & \textit{g}ie\textit{n}c \textit{im} \textit{\textbf{anderhalben}} \textit{bî}.\\ 
 & \textit{diu was dô \textbf{worden trûrens} vrî}.\\ 
 & Artus an den Waleis sach.\\ 
 & nû sult ir hœren, wie er sprach:\\ 
15 & "ich wil iuweren clâren lîp\\ 
 & \textbf{lâzen küssen} mîn altez wîp.\\ 
 & des durfet ir doch \textbf{niemen} biten,\\ 
 & sît ir von Pelrapeire geriten;\\ 
 & dâ ist des kusses \textbf{hœhestez} zil.\\ 
20 & eines dinges ich iuch biten wil:\\ 
 & kom ich imer in iuwer hûs,\\ 
 & gelt disen kus", sprach Artus.\\ 
 & "ich tuon, swes ir mich bitet, dâ",\\ 
 & sprach der Waleis, "unde anderswâ."\\ 
25 & ein \textbf{wênic} \textbf{sim dar nâher} gienc,\\ 
 & diu künigîn in mit kusse enpfienc.\\ 
 & "nû verkiuse ich hie mit triuwen",\\ 
 & \textbf{sprach si}, "daz ir \textbf{\textit{mich}} mit riuwen\\ 
 & liezet. die het ir mir \textit{ge}geben,\\ 
30 & dô ir Ither nâmet sîn leben."\\ 
\end{tabular}
\scriptsize
\line(1,0){75} \newline
G I O L M Q R Z \newline
\line(1,0){75} \newline
\textbf{1} \textit{Initiale} L  \textbf{13} \textit{Initiale} R Z  \textbf{19} \textit{Initiale} I  \textbf{27} \textit{Initiale} L  \newline
\line(1,0){75} \newline
\textbf{1} ouch] Da Z  $\cdot$ Schinovere] shinofer I Gvnover O Cvneware L ginover M Gynouer Q Gynower R gynofere Z \textbf{2} lieht] liht O (L) (M) (Q) \textbf{3} ir manic edele] maniger edelen O \textbf{4} minniclîchen] lieht geuarwen I mundiglichen Q \textbf{5} wol genomen] vol genomen R genomen wol Z \textbf{6} daz] Dar M \textbf{7} \textit{Versfolge 310.8-7} G I  \textbf{8} der] des M  $\cdot$ valsches] valsche I \textbf{9} dô] \textit{om.} O L M Q R Z  $\cdot$ Waleis] waleisz L \textbf{10} Kuneware] kunware I (O) M Cvneware L konware Q Cuͯnware R Cvnewaren Z  $\cdot$ de] der O von R  $\cdot$ Lalant] lalan Q \textbf{11} \textit{Die Verse 310.11-12 fehlen} G   $\cdot$ gienc] Gein I  $\cdot$ im anderhalben bî] anderthalb [bi im]: im bi O \textbf{12} dô] da M R Z \textit{om.} Q \textbf{13} Artus] Artuͯs L  $\cdot$ den] dem R  $\cdot$ Waleis] waleýs L walies R \textbf{14} sult] mugent R \textbf{16} lâzen küssen] Kuͯszen lan L  $\cdot$ altez] \textit{om.} I \textbf{17} des] desn I (O) (L) (Q) (Z)  $\cdot$ doch niemen] doch hie nieman I (O) (R) nieman L nymanden M hie nymantz Q \textbf{18} Pelrapeire] [pa*]: pailrapier I pelrapaire O pelrapere M pelrapire R \textbf{19} dâ ist] Wan da ist O L R Z Wen isz da M Wann do ist Q  $\cdot$ kusses] kuͤssens Z  $\cdot$ hœhestez] hohsteste I hoste O L (R) houbitis M hohen Q \textbf{20} iuch] \textit{om.} O L M Q Z \textbf{21} \textit{Vers 310.21 fehlt} R   $\cdot$ imer] vmber Q \textbf{22} gelt] so gelt I Gelten Q  $\cdot$ disen] den I \textbf{23} swes ir] wez ir L (M) (Q) (Z) wenir R  $\cdot$ mich] micht Q \textbf{24} unde] vnd ouch Z \textbf{25} sim dar nâher] naher si im do I si naher im do O sie im do naher L Q sy do nacher R sie naher im da Z \textbf{26} in] yn do M die do Q  $\cdot$ kusse] kussen I (M) Z kusse in Q \textbf{27} hie] Nu M  $\cdot$ mit] min R  $\cdot$ triuwen] triwe I truren R \textbf{28} sprach si] \textit{om.} I  $\cdot$ ir mich mit] [ih]: ir mit G ir [mir]: mit O ir Mugit mich M [micht mt]: micht mit Q ir mich liezzet mit Z \textbf{29} liezet] Lezzen M \textit{om.} Z  $\cdot$ het] hattit M  $\cdot$ gegeben] geben G \textbf{30} dô] Da M Z  $\cdot$ ir] \textit{om.} I ir dem chvnige O (L) (M) (Z)  $\cdot$ Ither] [J*]: Jthern I Jthern O (Q) Jhter L R jther M  $\cdot$ sîn] daz I L (R) \newline
\end{minipage}
\hspace{0.5cm}
\begin{minipage}[t]{0.5\linewidth}
\small
\begin{center}*T
\end{center}
\begin{tabular}{rl}
 & \textbf{ouch} kom vrou Gynover dar\\ 
 & mit maneger vrouwen lieht gevar\\ 
 & \textbf{unde} manec edele vürstîn.\\ 
 & die truogen minneclîchen schîn.\\ 
5 & \textbf{\begin{large}D\end{large}er rinc was} \textbf{genomen wol} sô wît,\\ 
 & daz âne gedrenge unde âne strît\\ 
 & \dag \hspace*{-.7em}\big| Artus der \textbf{valscheite} laz\dag \\ 
 & \hspace*{-.7em}\big| \textbf{manec} vrouwe bî ir âmîs saz.\\ 
10 & \hspace*{-.7em}\big| vrou Cunnewar de Lalant\\ 
 & \hspace*{-.7em}\big| \textbf{vuorte} den Waleis \textbf{bî} der hant.\\ 
 & \textbf{diu} gienc im \textbf{ein sîte} bî.\\ 
 & diu was dô \textbf{trûrens} vrî.\\ 
 & Artus an den Waleis sach.\\ 
 & nû sult ir hœren, wier sprach:\\ 
15 & "ich wil iuwern clâren lîp\\ 
 & \textbf{lâzen küssen} mîn altez wîp.\\ 
 & des \textbf{en}durfet ir doch \textbf{niht} biten,\\ 
 & sît ir von Peilrapere geriten,\\ 
 & \textbf{wan} dâ ist des kusses \textbf{hœheste} zil.\\ 
20 & eines dinges ich iuch biten wil:\\ 
 & kom ich iemer in iuwer hûs,\\ 
 & geltet disen kus", sprach Artus.\\ 
 & "Ich tuon, swes ir mich bitet, dâ",\\ 
 & sprach der Waleis, "unde anderswâ."\\ 
25 & Ein \textbf{wênec} \textbf{nâher sim dô} gie,\\ 
 & diu künegîn in mit kusse enpfie.\\ 
 & \textbf{Si sprach}: "nû verkiusich hie mit triuwen,\\ 
 & daz ir \textbf{mich} mit riuwen\\ 
 & liezet. die het ir mir gegeben,\\ 
30 & dô ir Ithere nâmet sîn leben."\\ 
\end{tabular}
\scriptsize
\line(1,0){75} \newline
T U V W \newline
\line(1,0){75} \newline
\textbf{5} \textit{Initiale} T U  \textbf{23} \textit{Majuskel} T  \textbf{25} \textit{Majuskel} T  \textbf{27} \textit{Majuskel} T  \newline
\line(1,0){75} \newline
\textbf{1} ouch] Do V Vnd W  $\cdot$ Gynover] Genover T U gynovere V tschinouer W \textbf{5} genomen] \textit{om.} V \textbf{8} \textit{Versfolge 310.7-8-9-10} V   $\cdot$ valscheite] valsches V \textbf{7} manec] Jekliche V \textbf{10} Cunnewar] kvnnewar T (W) kuͦnneware U kvnneware V \textbf{9} vuorte] Brachte W  $\cdot$ bî] an V W \textbf{11} ein sîte] anderhalben V (W) \textbf{12} vrî] worden vri V \textbf{18} Peilrapere] pelraper T pelrepere V pelrapeir W \textbf{19} dâ] do W  $\cdot$ kusses] kússens W  $\cdot$ hœheste] hoͤhestes V (W) \textbf{20} iuch] îv T \textbf{22} geltet] [gelter]: geltet T Gelten W \textbf{23} swes] wes U W  $\cdot$ bitet] biten U  $\cdot$ dâ] do U V W \textbf{24} unde] vnde oͮch V \textbf{25} wênec nâher sim] lútzel gegen im sú V \textbf{26} mit] nit U \textbf{27} Si sprach] \textit{om.} V  $\cdot$ mit] \textit{om.} W \textbf{30} Ithere] Jthere T dem kuͦnege Jthere U dem kv́nige ẏtern V kúnig ythern W  $\cdot$ sîn] daz W \newline
\end{minipage}
\end{table}
\end{document}
