\documentclass[8pt,a4paper,notitlepage]{article}
\usepackage{fullpage}
\usepackage{ulem}
\usepackage{xltxtra}
\usepackage{datetime}
\renewcommand{\dateseparator}{.}
\dmyyyydate
\usepackage{fancyhdr}
\usepackage{ifthen}
\pagestyle{fancy}
\fancyhf{}
\renewcommand{\headrulewidth}{0pt}
\fancyfoot[L]{\ifthenelse{\value{page}=1}{\today, \currenttime{} Uhr}{}}
\begin{document}
\begin{table}[ht]
\begin{minipage}[t]{0.5\linewidth}
\small
\begin{center}*D
\end{center}
\begin{tabular}{rl}
\textbf{323} & \textit{\begin{large}B\end{large}}eacors, der stolze man,\\ 
 & des bruoder was hêr Gawan,\\ 
 & der \textbf{spranc} ûf, sprach zehant:\\ 
 & "\textbf{hêrre}, ich sol dâ wesen pfant,\\ 
5 & swar Gawane ist der kampf geleget.\\ 
 & sîn velschen mich unsanfte reget.\\ 
 & welt irs niht erlâzen \textbf{in},\\ 
 & habt iuch an mich: sîn pfant ich bin,\\ 
 & ich sol \textbf{vür in ze kampfe} stên.\\ 
10 & ez mac mit rede niht ergên,\\ 
 & daz hôher prîs geneiget sî,\\ 
 & der Gawan \textbf{ist ledeclîche} bî."\\ 
 & Er kêrte, al dâ sîn bruoder saz.\\ 
 & \textbf{vuozvallens} er \textbf{dâ} niht vergaz.\\ 
15 & \textbf{den bat er sus}, nû hœret \textbf{wie}:\\ 
 & "gedenke, bruoder, daz dû \textbf{ie}\\ 
 & mir hülfe \textbf{grôzer} werdecheit.\\ 
 & lâ mich vür dîn arbeit\\ 
 & ein kampflîche\textit{z} gîsel wesen.\\ 
20 & ob ich \textbf{in} kampfe sol genesen,\\ 
 & des hâstû immer êre."\\ 
 & er bat in vürbaz mêre\\ 
 & durch brüederlîchen rîters prîs.\\ 
 & Gawan sprach: "ich bin sô wîs,\\ 
25 & daz ich dich, bruoder, niht gewer\\ 
 & dîner brüederlîchen ger.\\ 
 & ine weiz, warumbe ich \textbf{strîten} sol,\\ 
 & \textbf{ouch} \textbf{en}tuot mir \textbf{strîten} niht sô wol;\\ 
 & \textbf{ungern wolt ich} dir versagen,\\ 
30 & wan daz ich\textbf{z laster m\textit{üe}se} tragen."\\ 
\end{tabular}
\scriptsize
\line(1,0){75} \newline
D \newline
\line(1,0){75} \newline
\textbf{1} \textit{Initiale} D  \textbf{13} \textit{Majuskel} D  \newline
\line(1,0){75} \newline
\textbf{1} Beacors] Deacors D \textbf{19} kampflîchez] champfliches D \textbf{30} müese] mvͦse D \newline
\end{minipage}
\hspace{0.5cm}
\begin{minipage}[t]{0.5\linewidth}
\small
\begin{center}*m
\end{center}
\begin{tabular}{rl}
 & \begin{large}B\end{large}ea\textit{c}urs, der stolze man,\\ 
 & des bruoder was hêr Gawan,\\ 
 & der \textbf{spranc} ûf \textbf{und} sprach zehant:\\ 
 & "\textbf{h\textit{êrre}}, \textit{ich} sol d\textit{â} wesen pfant,\\ 
5 & wâ Gawane ist der kampf geleget.\\ 
 & sîn velsche\textit{n} \textit{m}ich unsanfte reget.\\ 
 & wellet irs niht erlâzen \textbf{sîn},\\ 
 & habet iuch an mich: sîn pfant ich bin,\\ 
 & ich sol \textbf{vür in ze pfande} stân.\\ 
10 & ez \textbf{en}mac mit rede niht ergân,\\ 
 & daz hôher prîs geneiget sî,\\ 
 & der Gawane \textbf{ist ledeclîchen} bî."\\ 
 & er kêrte, aldâ sîn bruoder saz.\\ 
 & \textbf{vuozvallens} er \textbf{d\textit{â}} niht vergaz.\\ 
15 & \textbf{den bat er sus}, nû hœret \textbf{hie}:\\ 
 & "gedenke, bruoder, daz dû \textbf{nie}\\ 
 & mir hülfe \textbf{grôzer} wirdicheit\\ 
 & \textbf{und} lâz mich vür dîne arbeit\\ 
 & ein kampflîchez gîsel wesen.\\ 
20 & ob ich \textbf{in} kampfe sol genesen,\\ 
 & d\textit{e}s hâstû iemer êre."\\ 
 & er bat in vürbaz mêre\\ 
 & durch brüederlîche\textit{n} ritters prîs.\\ 
 & Gawan sprach: "ich bin s\textit{ô} wîs,\\ 
25 & daz ich dich, bruoder, niht gewer\\ 
 & dîner brüederlîche\textit{n} ger.\\ 
 & \textit{in}e weiz, war umbe ich \textbf{vehten} sol,\\ 
 & \textbf{ouch} \textbf{en}tuot mir \textbf{vehten} niht sô wol;\\ 
 & \textbf{ich wolte ungerne} dir\textbf{z} versagen,\\ 
30 & wand daz ich \textbf{mües es laster} tragen."\\ 
\end{tabular}
\scriptsize
\line(1,0){75} \newline
m n o \newline
\line(1,0){75} \newline
\textbf{1} \textit{Initiale} m   $\cdot$ \textit{Capitulumzeichen} n  \newline
\line(1,0){75} \newline
\textbf{1} Beacurs] Beathturs m Beacúrs n \textbf{2} Gawan] gewann o \textbf{4} hêrre ich] Hant gawane die m  $\cdot$ dâ] do m n o \textbf{5} Gawane] gawan n gewan o  $\cdot$ geleget] gelgt m \textbf{6} velschen mich] velschen sich mich m \textbf{7} sîn] in n o \textbf{9} stân] sten m \textbf{10} enmac] mag n o \textbf{12} Gawane] gawan n gewan o \textbf{13} aldâ] aldo do n \textbf{14} dâ] do m n \textit{om.} o \textbf{15} er] der n  $\cdot$ hie] wie n o \textbf{17} hülfe] helffe o \textbf{18} dîne] din n o \textbf{19} gîsel] geselle n \textbf{21} des] Das m (o) \textbf{23} brüederlîchen] bruderlicher m \textbf{24} Gawan] Gewan o  $\cdot$ sô wîs] swis m \textbf{26} brüederlîchen] bruderlicher m \textbf{27} ine] Me m \textbf{28} entuot] duͦt n (o) \textbf{29} versagen] gesagen o \textbf{30} mües es] muͯsz n mus musse o \newline
\end{minipage}
\end{table}
\newpage
\begin{table}[ht]
\begin{minipage}[t]{0.5\linewidth}
\small
\begin{center}*G
\end{center}
\begin{tabular}{rl}
 & Beakurs, der stolze man,\\ 
 & des bruoder was hêr Gawan,\\ 
 & der \textbf{stuont} ûf \textbf{unde} sprach zehant:\\ 
 & "\textbf{hêrre}, ich sol dâ wesen pfant,\\ 
5 & swar Gawane ist der kampf geleget.\\ 
 & sîn velschen mich unsanfte r\textit{e}get;\\ 
 & \multicolumn{1}{l}{ - - - }\\ 
 & \multicolumn{1}{l}{ - - - }\\ 
 & ich sol \textit{\textbf{ze kampfe vür in}} stên.\\ 
10 & ez\textit{\textbf{n}} mac \textit{mit red}e niht ergên,\\ 
 & daz hôher brîs geneiget sî,\\ 
 & der Gawane \textbf{lediclîche ist} bî."\\ 
 & er kêrte, \textit{al} dâ sîn bruoder saz.\\ 
 & \textbf{vuozvallens} er \textit{\textbf{dâ}} niht vergaz.\\ 
15 & \textbf{er bat in sus}, nû hœret \textbf{wie}:\\ 
 & "gedenke, bruoder, daz dû \textbf{ie}\\ 
 & mir hülfe \textbf{rehter} werdicheit.\\ 
 & lâ mich vür dîn arbeit\\ 
 & ein kampflîchez gîsel wesen.\\ 
20 & obe ich \textbf{an} kampfe su\textit{l} genesen,\\ 
 & des hâstû immer êre."\\ 
 & er bat in vürbaz mêre\\ 
 & durch brüederlîchen rîters prîs.\\ 
 & \textbf{hêr} Gawan sprach: "ich bin sô wîs,\\ 
25 & daz ich dich, bruoder, niht gewer\\ 
 & dîner brüederlîchen ger.\\ 
 & ichne weiz, war umbe ich \textbf{strîten} sol,\\ 
 & \textbf{doch} tuot mir \textbf{strîten} niht sô wol;\\ 
 & \textbf{ungerne wolt ich} dir\textbf{z} versagen,\\ 
30 & wan daz ich \textbf{müese daz laster} tragen."\\ 
\end{tabular}
\scriptsize
\line(1,0){75} \newline
G I O L M Q R Z Fr22 Fr39 Fr40 \newline
\line(1,0){75} \newline
\textbf{1} \textit{Initiale} I  \textbf{11} \textit{Initiale} L Fr39  \textbf{23} \textit{Initiale} I  \textbf{27} \textit{Initiale} Q Fr40  \newline
\line(1,0){75} \newline
\textbf{1} Beakurs] beacurs G (I) (O) (M) Beakuͯrs L Beachvrs Z \textbf{2} hêr Gawan] ergawan M \textbf{4} dâ wesen] do wesen fv̂r in O weisen do Q do wesen Fr39 \textbf{5} swar] War L R Wo Q  $\cdot$ Gawane] Gawan I O L Z Fr39 (Fr40) [man]: gawan  M gewan Q her Gawan R  $\cdot$ ist der kampf] der kamp ist R  $\cdot$ geleget] geleit G \textbf{6} velschen] valsch I velsche M  $\cdot$ mich] mit I sich Q  $\cdot$ reget] reiget G wegt O \textbf{7} \textit{Die Verse 323.7-8 fehlen} G I O L M Z Fr39   $\cdot$ Wolt irs nicht er lossen in Q (Fr40) Welt irs in erlaszen nit in R \textbf{8} Habt euch an mich sin (des R ) pfant ich bin Q (R) (Fr40) \textbf{9} ze kampfe vür in] vur in ze champhe G \textbf{10} ezn] ez G  $\cdot$ mit rede] so lihte G rede O mit reden Q mit der red R  $\cdot$ ergên] erget I \textbf{11} daz] ÷as Fr39  $\cdot$ geneiget] gevnerit M  $\cdot$ sî] sein Q \textbf{12} der] Dar L Des Q  $\cdot$ Gawane] Gawan I O L R Z Fr39  $\cdot$ lediclîche ist] ist ledechlichen I (Q) (R) (Z) ewecliche ist L (Fr39) \textbf{13} kêrte] chert I (O) (Z)  $\cdot$ al] \textit{om.} G  $\cdot$ saz] was R \textbf{14} dâ] \textit{om.} G doch L M Fr39 \textbf{15} in] \textit{om.} L Q Z Fr39 \textbf{17} mir hülfe] Mit hilfe Q (R) \textbf{18} dîn] dine M \textbf{19} ein kampflîchez] Einen kemphlichen L (Fr39) Ein kempflicher R \textbf{20} sul] sule G  $\cdot$ an] in Z am Fr39 \textbf{21} des] Das R  $\cdot$ immer] mer I \textbf{22} vürbaz] wuͦrbaz Fr22 \textbf{23} rîters] rittern M  $\cdot$ prîs] wis R \textbf{24} hêr] \textit{om.} Z  $\cdot$ Gawan] Gawein I  $\cdot$ ich bin] bin ich Q ich bin wol R \textbf{25} dich bruoder] brudir M bruder dir Q \textbf{26} brüederlîchen] bruderliche Q \textbf{27} ichne] inen I Jch Q \textbf{28} doch] Ouch Z  $\cdot$ tuot] entuͯt L (M) (Z) (Fr39)  $\cdot$ strîten] stritens I \textbf{29} dirz] dir O L M Q R Z Fr22 Fr39 \textbf{30} wan daz] Wan O  $\cdot$ müese] muͤz I mvͦse O (R) (Fr22) musz M Q (Z)  $\cdot$ daz laster] laster L Fr39  $\cdot$ tragen] haben I \newline
\end{minipage}
\hspace{0.5cm}
\begin{minipage}[t]{0.5\linewidth}
\small
\begin{center}*T
\end{center}
\begin{tabular}{rl}
 & \begin{large}B\end{large}eakurs, der stolze man,\\ 
 & des bruoder was hêr Gawan,\\ 
 & der \textbf{stuont} ûf \textbf{unde} sprach zehant:\\ 
 & "ich sol dâ wesen pfant,\\ 
5 & swar Gawane ist der kampf geleget.\\ 
 & sîn velschen mich unsanfte reget;\\ 
 & \multicolumn{1}{l}{ - - - }\\ 
 & \multicolumn{1}{l}{ - - - }\\ 
 & ich sol \textbf{ze kampfe vür in} stân.\\ 
10 & ez mac mit rede niht ergân,\\ 
 & daz hôher prîs geneiget sî,\\ 
 & der Gawane \textbf{ledeclîche ist} bî."\\ 
 & er kêrte, al dâ sîn bruoder saz.\\ 
 & \textbf{vür vallens} er niht vergaz\\ 
15 & \textbf{unde sprach alsus}, nû hœret \textbf{wie}:\\ 
 & "gedenke, bruoder, daz dû \textbf{ie}\\ 
 & mir hülfe \textbf{rehter} werdecheit.\\ 
 & lâ mich vür dîne arbeit\\ 
 & ein kampflîchez gîsel wesen.\\ 
20 & ob ich \textbf{an} kampfe sul genesen,\\ 
 & des hâstû iemer êre."\\ 
 & er bat in vürbaz mêre\\ 
 & durch brüederlîchen rîters prîs.\\ 
 & \textbf{Hêr} Gawan sprach: "ich bin sô wîs,\\ 
25 & daz ich dich, bruoder, niht gewer\\ 
 & dîner brüederlîchen ger.\\ 
 & ine weiz, warumbich \textbf{strîten} sol,\\ 
 & \textbf{doch} tuot mir \textbf{strîten} niht sô wol;\\ 
 & \textbf{ungerne woltich} dir\textbf{z} versagen,\\ 
30 & wan daz ich \textbf{muoz daz laster} tragen."\\ 
\end{tabular}
\scriptsize
\line(1,0){75} \newline
T U V W \newline
\line(1,0){75} \newline
\textbf{1} \textit{Initiale} T U V  \textbf{24} \textit{Majuskel} T  \newline
\line(1,0){75} \newline
\textbf{1} Beakurs] Beakuͦrs U Beachurs W \textbf{4} ich] [*]: herre ich V  $\cdot$ dâ] do U V W \textbf{5} swar] War U W  $\cdot$ Gawane] gawan W  $\cdot$ ist der] den W  $\cdot$ kampf geleget] [*]: canpf geleget V kampff leget W \textbf{6} velschen] valsch W \textbf{7} \textit{Die Verse 323.7-8 fehlen} T U W   $\cdot$ Went irs niht erlazen in V \textbf{8} Habent v́ch an mich sin phant ich bin V \textbf{10} ez] Er W \textbf{12} Gawane] [Gaw*]: Gawane V gawan W  $\cdot$ ledeclîche ist] ist ledigliche W \textbf{13} al dâ] als W \textbf{14} vür] Vuͦz U (V) \textbf{15} unde sprach alsus] Den er suß bat W \textbf{16} daz dû ie] die U \textbf{17} hülfe] hiffe W \textbf{19} ein kampflîchez] Einen kampfflichen W  $\cdot$ wesen] geben W \textbf{20} an] den W  $\cdot$ sul genesen] [*]: sol genesen V múg beheben W \textbf{23} brüederlîchen] bruͦderliche U \textbf{25} \textit{Versfolge 323.26-25} V  \textbf{26} brüederlîchen] bruderlicher U \textbf{27} ine weiz] [J*]: Jch weis V \textbf{29} ungerne woltich] [J*]: Jch enwolte vngerne V  $\cdot$ dirz] dir W \textbf{30} muoz daz] dez mvͤste V \newline
\end{minipage}
\end{table}
\end{document}
