\documentclass[8pt,a4paper,notitlepage]{article}
\usepackage{fullpage}
\usepackage{ulem}
\usepackage{xltxtra}
\usepackage{datetime}
\renewcommand{\dateseparator}{.}
\dmyyyydate
\usepackage{fancyhdr}
\usepackage{ifthen}
\pagestyle{fancy}
\fancyhf{}
\renewcommand{\headrulewidth}{0pt}
\fancyfoot[L]{\ifthenelse{\value{page}=1}{\today, \currenttime{} Uhr}{}}
\begin{document}
\begin{table}[ht]
\begin{minipage}[t]{0.5\linewidth}
\small
\begin{center}*D
\end{center}
\begin{tabular}{rl}
\textbf{648} & \textit{\begin{large}R\end{large}}eht umbe den mitten morgen\\ 
 & offenlîche unt unverborgen\\ 
 & \textbf{ûf den hof der knappe} reit.\\ 
 & die höfschen prüeveten sîniu kleit\\ 
5 & wol nâch knappeclîchen siten.\\ 
 & ze bêden sîten was versniten\\ 
 & daz ors mit \textbf{sporn} sêre.\\ 
 & nâch der küneginne lêre\\ 
 & \textbf{er balde} von dem orse spranc.\\ 
10 & umb in \textbf{huop sich} grôz gedranc.\\ 
 & \textbf{kappe, swert} unt sporn\\ 
 & untz ors, \textbf{würden} diu verlorn,\\ 
 & dâ kêrt er sich wênec an.\\ 
 & der knappe huop sich balde dan,\\ 
15 & dâ die werden rîter stuonden,\\ 
 & \textbf{die} vrâgen in begunden\\ 
 & von âventiure mære.\\ 
 & \textbf{Si} \textbf{jehent}, daz reht dâ wære,\\ 
 & ze hove \textbf{az} \textbf{weder} wîp \textbf{noch} man,\\ 
20 & \textbf{ê} der hof sîn reht gewan,\\ 
 & Âventiure sô werdeclîch,\\ 
 & diu âventiure wære gelîch.\\ 
 & Der knappe sprach: "i\textbf{ne} sag iu niht.\\ 
 & mîn unmuoze mir des giht.\\ 
25 & daz sult ir mir durch zuht vertragen\\ 
 & unt \textbf{ruochet} mir vome künege sagen;\\ 
 & den \textbf{het} ich gern \textbf{gesprochen} ê.\\ 
 & mir tuot mîn unmuoze wê.\\ 
 & ir vreischet \textbf{wol}, waz ich \textbf{mære} sage.\\ 
30 & got lêre iuch helfe unt kumbers klage."\\ 
\end{tabular}
\scriptsize
\line(1,0){75} \newline
D \newline
\line(1,0){75} \newline
\textbf{1} \textit{Initiale} D  \textbf{18} \textit{Majuskel} D  \textbf{21} \textit{Majuskel} D  \textbf{23} \textit{Majuskel} D  \newline
\line(1,0){75} \newline
\textbf{1} Reht] ÷eht D \newline
\end{minipage}
\hspace{0.5cm}
\begin{minipage}[t]{0.5\linewidth}
\small
\begin{center}*m
\end{center}
\begin{tabular}{rl}
 & reht umb den mitten morgen\\ 
 & offenlîch und unverborgen\\ 
 & \textbf{ûf den hof er balde} reit.\\ 
 & die höveschen prüeveten sîniu kleit\\ 
5 & wol nâch knappelîchen siten.\\ 
 & zuo beiden sîten was ver\textit{sn}iten\\ 
 & daz ros mit \textbf{sporn} sêre.\\ 
 & nâch der künigîn lêre\\ 
 & \textbf{er balde} von dem rosse spranc.\\ 
10 & umb in \textbf{huop sich} grôz gedranc.\\ 
 & \textbf{kappe, swert} und sporn\\ 
 & und daz ros, \textbf{würden} diu verlorn,\\ 
 & dâ kêrte er sich \textbf{vil} wênic an.\\ 
 & der knappe huop sich balde dan,\\ 
15 & d\textit{â} die werden ritter stuonden\\ 
 & \textbf{und} vrâgen in begunden\\ 
 & von âventiure mære.\\ 
 & \textbf{si} \textbf{jâhen}, daz reht d\textit{â} wære,\\ 
 & \textbf{daz} zuo hove \dag ezze\dag  wîp \textbf{und} man,\\ 
20 & \textbf{ê daz} der hof sîn reht gewan,\\ 
 & âventiure sô werdeclîch,\\ 
 & diu âventiure wære gelîch.\\ 
 & der knappe sprach: "ich sage iu niht.\\ 
 & mîn unmuoze mir des giht.\\ 
25 & daz sullet ir mir durch zuht vertragen\\ 
 & und \textbf{ruochet} mir von dem künige sagen;\\ 
 & den \textbf{het} ich gerne \textbf{gesproch\textit{en}} ê.\\ 
 & mir tuot mîn unmuoze wê.\\ 
 & ir vreischet \textbf{wol}, waz ich \textbf{mære} sage.\\ 
30 & got lêre iuch helf und kumbers klage."\\ 
\end{tabular}
\scriptsize
\line(1,0){75} \newline
m n o Fr69 \newline
\line(1,0){75} \newline
\newline
\line(1,0){75} \newline
\textbf{4} höveschen prüeveten] houesten prufen o  $\cdot$ sîniu] sin m n o \textbf{5} knappelîchen] knappelichem n \textbf{6} versniten] vermitten m \textbf{10} grôz] grosse o \textbf{12} diu] do n \textbf{15} dâ] Do m n o \textbf{18} si jâhen] Sit o  $\cdot$ dâ] do m n o \textbf{20} gewan] gawan o \textbf{21} \textit{Die Verse 648.21-22 fehlen} o  \textbf{23} ich] in Fr69  $\cdot$ niht] niczt o \textbf{27} gesprochen] gesproch m \newline
\end{minipage}
\end{table}
\newpage
\begin{table}[ht]
\begin{minipage}[t]{0.5\linewidth}
\small
\begin{center}*G
\end{center}
\begin{tabular}{rl}
 & \begin{large}R\end{large}ehte umbe den mitten morgen\\ 
 & offenlîche unde unverborgen\\ 
 & \textbf{der knappe ûf den hof} reit.\\ 
 & die höfschen prüev\textit{e}te\textit{n} sîniu kleit\\ 
5 & wol nâch knappelîchen siten.\\ 
 & ze bêden sîten was versniten\\ 
 & daz ors mit \textbf{sporen} sêre.\\ 
 & nâch der künegîn lêre\\ 
 & \textbf{balde er} von dem orse spranc.\\ 
10 & umbe in \textbf{wart dâ} grôz gedranc.\\ 
 & \textbf{sîn} \textbf{swert, kappe} und\textit{e s}porn\\ 
 & unde daz ors, \textbf{werdent} diu verlorn,\\ 
 & dâ kêrte er sich wênic an.\\ 
 & der knappe huop sich balde dan,\\ 
15 & dâ die werde\textit{n} rîter stuonden,\\ 
 & \textbf{die} vrâgen in begunden\\ 
 & von âventiure mære.\\ 
 & \textbf{die} \textbf{jâhen}, daz reht dâ wære,\\ 
 & ze hov\textit{e} \textbf{\textit{w}eder} wîp \textbf{noch} man\\ 
20 & \textbf{e\textit{nbize}}, \textbf{\textit{un}z} d\textit{e}r hof sîn reht gewan,\\ 
 & âventiure sô werde\textit{c}lîch,\\ 
 & diu âventiure wære gelîch.\\ 
 & der knappe sprach: "ich \textbf{en}sage iu niht.\\ 
 & mîn unmuoze mir des giht.\\ 
25 & daz sult ir mir durch zuht vertragen\\ 
 & unde \textbf{ruochet} mir von dem künige sagen;\\ 
 & den \textbf{wolde} \textit{ich} gerne \textbf{sprechen} ê.\\ 
 & mir tuot mîn unmuoze wê.\\ 
 & ir vreischet \textbf{schiere}, waz ich sage.\\ 
30 & got lêre iuch helfe unde kumbers klage."\\ 
\end{tabular}
\scriptsize
\line(1,0){75} \newline
G I L M Z \newline
\line(1,0){75} \newline
\textbf{1} \textit{Initiale} G I L Z  \textbf{17} \textit{Initiale} I  \newline
\line(1,0){75} \newline
\textbf{1} den] einen I \textit{om.} L \textbf{2} unde] \textit{om.} L  $\cdot$ unverborgen] nih verborgen I \textbf{4} prüeveten] proͮvente G  $\cdot$ kleit] leit M \textbf{9} Von dem rvnzite er spranch L \textbf{10} dâ] \textit{om.} I \textbf{11} kappe] kappen L  $\cdot$ unde sporn] vnde de sporen G \textbf{12} unde] \textit{om.} I  $\cdot$ daz ors] rvnzit L \textbf{13} kêrte] kert I L Z \textbf{15} werden] werde G M \textbf{16} die vrâgen in] fragen in die I \textbf{18} die] vnde I Sý L (M) (Z)  $\cdot$ jâhen] iehent L Z sprachin M \textbf{19} ze hove] zehove da G Zuͯ hove az L (Z) Da zcu hofe esze M \textbf{20} enbize unz] Ę daz G (L) (M) (Z)  $\cdot$ der] dir G  $\cdot$ gewan] genam I \textbf{21} âventiure] von Auenture I  $\cdot$ werdeclîch] werdelich G \textbf{23} knappe] \textit{om.} L  $\cdot$ ensage] sag I (L) (M) (Z)  $\cdot$ iu] \textit{om.} M \textbf{24} mir] \textit{om.} I \textbf{25} zuht] \textit{om.} L \textbf{26} ruochet] geruchit M (Z) \textbf{27} ich] \textit{om.} G  $\cdot$ sprechen] gesprechen I \newline
\end{minipage}
\hspace{0.5cm}
\begin{minipage}[t]{0.5\linewidth}
\small
\begin{center}*T
\end{center}
\begin{tabular}{rl}
 & rehte umb den mitten morgen\\ 
 & offenlîch und unverborgen\\ 
 & \textbf{der knabe ûf den hof} reit.\\ 
 & die hövischen prüeveten sîniu kleit\\ 
5 & wol nâch knabelîchen siten.\\ 
 & zuo bêden sîten was versniten\\ 
 & daz ros mit \textbf{sporne} sêre.\\ 
 & nâch der künigîn lêre\\ 
 & \textbf{bald er} von dem rosse spranc.\\ 
10 & umb in \textbf{wart d\textit{â}} grôz gedranc.\\ 
 & \textbf{sîn} \textbf{swert, kappe} und sporn\\ 
 & und daz ros, \textbf{werdent} diu verlorn,\\ 
 & dâ kêrt er sich wênic an.\\ 
 & der knabe huop sich balde dan,\\ 
15 & d\textit{â} die werden ritter stuonden,\\ 
 & \textbf{die} vrâgen in begunden\\ 
 & von âventiure mære.\\ 
 & \textbf{si} \textbf{jehent}, daz reht d\textit{â} wære,\\ 
 & \textbf{dâ} zuo hove \textbf{en}\textbf{ezze} wîp \textbf{noch} man,\\ 
20 & \textbf{ê} der hof sîn reht gewan,\\ 
 & âventiure sô werdeclîch,\\ 
 & diu âventiure wære glîch.\\ 
 & der knabe sprach: "ich sag iu niht.\\ 
 & mîn unmuoze mir des giht.\\ 
25 & daz solt ir mir durc\textit{h} zuht vertragen\\ 
 & und \textbf{geruochet} mir von dem künige sagen;\\ 
 & den \textbf{wolt} ich gerne \textbf{besprechen} ê.\\ 
 & mir tuot mîn unmuoze wê.\\ 
 & ir vreischet \textbf{sc\textit{h}ie\textit{r}}, waz ich sage.\\ 
30 & got lêr iuch helfe und kumbers klage."\\ 
\end{tabular}
\scriptsize
\line(1,0){75} \newline
Q R W V \newline
\line(1,0){75} \newline
\textbf{1} \textit{Capitulumzeichen} R  \newline
\line(1,0){75} \newline
\textbf{2} und] vnd gar W \textbf{4} sîniu] seyn Q \textbf{5} knabelîchen] kempflichem R knappelichem W \textbf{8} künigîn] kunginen R \textbf{10} dâ] do Q R V \textit{om.} W \textbf{11} kappe] der knappen R \textbf{12} werdent] wurdent R [*]: wurden V \textbf{13} kêrt] kertte R (W) (V) \textbf{14} huop sich] kerte W \textbf{15} dâ] Do Q W V \textbf{16} die] Die zuͦ W  $\cdot$ begunden] begvnde V \textbf{18} jehent] Jachent R (W) (V)  $\cdot$ dâ] do Q W \textit{om.} V \textbf{19} dâ] Do R V Das W  $\cdot$ enezze] enige R ez eneze V  $\cdot$ noch] oder R \textbf{20} ê] [*]: E daz V \textbf{23} ich sag iu niht] [*]: ich sage v́ch niht V \textbf{24} mir] \textit{om.} R \textbf{25} durch] durcht Q \textbf{26} geruochet] gezuͦcht W rvͦchent V \textbf{27} den] dem R  $\cdot$ besprechen] gespreche R besprochen W gesprechen V \textbf{29} vreischet] vernempt R pruͤfet W [v*ent]: vreiscent V  $\cdot$ schier] schrie Q  $\cdot$ sage] [*]: mere sage V \textbf{30} und kumbers] vnd och des kumers R gen kumber W \newline
\end{minipage}
\end{table}
\end{document}
