\documentclass[8pt,a4paper,notitlepage]{article}
\usepackage{fullpage}
\usepackage{ulem}
\usepackage{xltxtra}
\usepackage{datetime}
\renewcommand{\dateseparator}{.}
\dmyyyydate
\usepackage{fancyhdr}
\usepackage{ifthen}
\pagestyle{fancy}
\fancyhf{}
\renewcommand{\headrulewidth}{0pt}
\fancyfoot[L]{\ifthenelse{\value{page}=1}{\today, \currenttime{} Uhr}{}}
\begin{document}
\begin{table}[ht]
\begin{minipage}[t]{0.5\linewidth}
\small
\begin{center}*D
\end{center}
\begin{tabular}{rl}
\textbf{638} & \begin{large}N\end{large}û begunde ouch \textbf{strûchen} der tac,\\ 
 & daz sîn schîn vil nâch \textbf{gelac}\\ 
 & \textbf{unt} daz man durch die wolken sach,\\ 
 & \textbf{des} man der naht ze boten jach,\\ 
5 & manegen stern, der balde gienc,\\ 
 & wander der naht herberge vienc.\\ 
 & nâch der \textbf{nahte} baniere\\ 
 & kom si selbe schiere.\\ 
 & Manec tiuriu krône\\ 
10 & was gehangen schône\\ 
 & al umbe ûf \textbf{den} palas,\\ 
 & diu schiere wol \textbf{bekerzet} was.\\ 
 & \textbf{ûf al} die tische sunder\\ 
 & truoc man kerzen dar ein wunder.\\ 
15 & Dar zuo diu âventiure giht,\\ 
 & diu herzoginne wære sô lieht,\\ 
 & wære der kerzen keiniu brâht,\\ 
 & dâ wære doch \textbf{ninder} bî ir naht.\\ 
 & \multicolumn{1}{l}{ - - - }\\ 
20 & \multicolumn{1}{l}{ - - - }\\ 
 & Man welle im \textbf{unrehtes} jehen,\\ 
 & sô \textbf{habt ir} selten ê gesehen\\ 
 & decheinen wirt sô vreuden rîch.\\ 
 & ez was den vreuden \textbf{dâ} gelîch.\\ 
25 & alsus mit vreudehafter ger\\ 
 & die rîter dar, die vrouwen her\\ 
 & dicke an ein ander blicten.\\ 
 & die von \textbf{der} vremde erschricten,\\ 
 & werdents iemer \textbf{heinlîcher} baz,\\ 
30 & daz sol ich lâzen âne haz.\\ 
\end{tabular}
\scriptsize
\line(1,0){75} \newline
D Z Fr1 \newline
\line(1,0){75} \newline
\textbf{1} \textit{Initiale} D Z Fr1  \textbf{9} \textit{Majuskel} D  \textbf{15} \textit{Majuskel} D  \textbf{21} \textit{Majuskel} D  \newline
\line(1,0){75} \newline
\textbf{2} daz] so daz Fr1 \textbf{6} wander] wander er Fr1  $\cdot$ herberge] herbege Fr1 \textbf{11} den] dem Z \textbf{15} diu] di Fr1 \textbf{18} wære] en were Z \textbf{19} \textit{Die Verse 638.19-20 fehlen} D Fr1   $\cdot$ Jr blic wol selbe kunde tagen Z \textbf{20} Sus hort ich von der svzzen sagen Z \textbf{21} im] im dann Fr1  $\cdot$ unrehtes] vnrehte Z \newline
\end{minipage}
\hspace{0.5cm}
\begin{minipage}[t]{0.5\linewidth}
\small
\begin{center}*m
\end{center}
\begin{tabular}{rl}
 & \begin{large}N\end{large}û begunde ouch \textbf{strîchen} der tac,\\ 
 & daz sîn schîn vil nâhe \textbf{lac}\\ 
 & \textbf{und} daz man durch die wolken sach,\\ 
 & \textbf{daz} man der naht zuo boten jach,\\ 
5 & manigen sternen, der balde gienc,\\ 
 & wan er der naht  herberge vienc.\\ 
 & nâch der banier\\ 
 & kam si selbe schier.\\ 
 & manic tiuriu krône\\ 
10 & was gehangen schône\\ 
 & alumb ûf \textbf{dem} palas,\\ 
 & diu schie\textit{re} wol \textbf{bekerzet} was.\\ 
 & \textbf{ûf alle} die tische sunder\\ 
 & truoc man kerzen dar \textit{e}in wunder.\\ 
15 & dar zuo diu âventiure giht,\\ 
 & diu herzogîn wær sô lieht,\\ 
 & wær der kerzen keiniu brâht,\\ 
 & d\textit{â} wær doch \textbf{nindert} bî ir naht.\\ 
 & ir blic wol selbe kunde tagen,\\ 
20 & sus \textbf{hœre} ich von der süezen sagen.\\ 
 & man welle im \textbf{unrehtes} j\textit{e}hen,\\ 
 & sô \textbf{haben\textit{t} si} selten ê gesehen\\ 
 & dekeinen wirt sô vröuden rîch.\\ 
 & ez was den vröuden \textbf{sô} gelîch.\\ 
25 & alsus mit vröudehafter ger\\ 
 & die ritter \textit{da}r, die vrowen \textit{he}r\\ 
 & dicke an ein ander blic\textit{t}en.\\ 
 & die von \textbf{der} vremde erschric\textit{t}en,\\ 
 & werden\textit{t} si iemer \textbf{heimlîcher} baz,\\ 
30 & daz sol ich lâzen âne haz.\\ 
\end{tabular}
\scriptsize
\line(1,0){75} \newline
m n o \newline
\line(1,0){75} \newline
\textbf{1} \textit{Initiale} m n  \newline
\line(1,0){75} \newline
\textbf{1} ouch] sich o  $\cdot$ strîchen] struchen n \textbf{3} daz] dis o \textbf{5} \textit{Versdoppelung 638.5-8 nach 638.8} o  \textbf{9} Manigen túr krone o \textbf{10} gehangen] ge fangen n \textbf{11} dem] den n o \textbf{12} schiere] schie m  $\cdot$ bekerzet] gekerczet o \textbf{13} die] \textit{om.} n \textbf{14} ein] in m \textbf{17} keiniu] kerczen o \textbf{18} dâ] Do m n o \textbf{19} blic] blicke n \textbf{21} unrehtes] vnrecht o  $\cdot$ jehen] yhen m \textbf{22} habent] haben m n o \textbf{23} dekeinen] Do keinen n \textbf{24} sô] do n o \textbf{26} dar] her m  $\cdot$ her] dar m \textbf{27} blicten] blicken m \textbf{28} erschricten] erschricken m erstrickten o \textbf{29} werdent] werden m \newline
\end{minipage}
\end{table}
\newpage
\begin{table}[ht]
\begin{minipage}[t]{0.5\linewidth}
\small
\begin{center}*G
\end{center}
\begin{tabular}{rl}
 & \textit{\begin{large}N\end{large}û} begunde ouch \textbf{strûchen} der tac,\\ 
 & daz sîn schîn vil nâch \textbf{gelac}\\ 
 & \textbf{unde} daz man durch diu wolken sac\textit{h},\\ 
 & \textbf{des} man der naht ze boten jac\textit{h},\\ 
5 & manigen sternen, der \textbf{vil} balde gienc,\\ 
 & wand er der naht herberge vienc.\\ 
 & nâch der \textbf{naht} baniere\\ 
 & kom si selbe schiere.\\ 
 & manic tiuwer krône\\ 
10 & was gehangen schône\\ 
 & al umbe ûf \textbf{dem} palas,\\ 
 & diu schier wol \textbf{gekerzet} was.\\ 
 & \textbf{al ûf} die tische sunder\\ 
 & truoc man kerzen dar ein wunder.\\ 
15 & dar zuo di\textit{u} âventiure giht,\\ 
 & diu herzogîn wære sô lieht,\\ 
 & wære de\textit{r} kerzen deheiniu brâht,\\ 
 & dâ\textbf{ne} wære doch \textbf{ninder} bî ir naht.\\ 
 & ir blic wol selbe kunde tagen,\\ 
20 & sus \textbf{hôrt} ich von der süezen sagen.\\ 
 & man welle im \textbf{unreht} jehen,\\ 
 & sô \textbf{habet ir} selten ê gesehen\\ 
 & deheinen wirt sô vröuden rîch.\\ 
 & ez was den vröuden \textbf{dâ} gelîch.\\ 
25 & alsus mit vröudehafter ger\\ 
 & die rîter dâ, die vrouwen her\\ 
 & dicke an ein ander blicten.\\ 
 & die von vremede erschricten,\\ 
 & werdent si immer \textbf{heinlîch} baz,\\ 
30 & daz sol ich lâzen âne haz.\\ 
\end{tabular}
\scriptsize
\line(1,0){75} \newline
G I L M Z Fr18 \newline
\line(1,0){75} \newline
\textbf{1} \textit{Initiale} G Z Fr18  \textbf{13} \textit{Initiale} I  \newline
\line(1,0){75} \newline
\textbf{1} Nû] Dô G  $\cdot$ strûchen] sigen M (Fr18) \textbf{2} nâch] nahen I Fr18 \textbf{3} sach] sac G \textbf{4} jach] iac G \textbf{5} sternen] stern I M Z Fr18  $\cdot$ vil] \textit{om.} L M Z Fr18 \textbf{6} naht] nach Fr18 \textbf{8} selbe] selbin M \textbf{11} dem] den I (L) (M) ::: Fr18 \textbf{12} gekerzet] gehertzet L bekertzet Z \textbf{13} al ûf] Uf I Vf alle L (M) (Z) (Fr18)  $\cdot$ die] \textit{om.} L M  $\cdot$ sunder] besvnder L \textbf{14} kerzen dar] kertzen L dar kerczen M der kerrzen Fr18  $\cdot$ ein] \textit{om.} L Fr18 \textbf{15} diu] die G \textbf{16} lieht] licht L M \textbf{17} der] den G \textbf{18} dâne] ezn I  $\cdot$ doch] tac M  $\cdot$ bî ir] \textit{om.} L \textbf{19} selbe] selbin M ::: Fr18 \textbf{20} sus] Daz L \textbf{21} welle] enwelle L \textbf{24} dâ] wol I do Fr18 \textbf{25} vröudehafter] vroiden hatter M \textbf{26} dâ] hin I \textbf{27} ein] \textit{om.} I  $\cdot$ blicten] blichen I (M) \textbf{28} vremede] der frevde L der vremden M der fremde Z (Fr18)  $\cdot$ erschricten] ershrichen I \textbf{29} immer] niemer L myner M  $\cdot$ heinlîch] heimlicher Z \newline
\end{minipage}
\hspace{0.5cm}
\begin{minipage}[t]{0.5\linewidth}
\small
\begin{center}*T
\end{center}
\begin{tabular}{rl}
 & \begin{large}N\end{large}û begunde ouch \textbf{sîgen} der tac,\\ 
 & daz sîn schîn vil nâch \textbf{gelac},\\ 
 & daz man durch die wolken sach,\\ 
 & \textbf{daz} man der naht zuo boten jach,\\ 
5 & manegen sternen, der balde gienc,\\ 
 & wan \textit{er} der \textit{naht} herberge vienc.\\ 
 & nâch der \textbf{naht} baniere\\ 
 & kam si selbe schiere.\\ 
 & manegiu tiuriu krône\\ 
10 & was gehangen schône\\ 
 & al umb ûf \textbf{dem} palas,\\ 
 & diu schiere wol \textbf{gekerzet} was.\\ 
 & \textbf{ûf alle} die tische sunder\\ 
 & truoc man kerzen dar ein wunder.\\ 
15 & dar zuo diu âventiure giht,\\ 
 & diu herzoginne wære sô lieht,\\ 
 & wære der kerzen dekeiniu brâht,\\ 
 & dâ\textbf{n} wære doch \textbf{niemer} bî ir naht.\\ 
 & ir blic wol selbe kunde tagen,\\ 
20 & sus \textbf{hôrt} ich von der süezen sagen.\\ 
 & man \textbf{en}wolle im \textbf{unreht} jehen,\\ 
 & sô \textbf{hât ir} selten ê gesehen\\ 
 & dekeinen wirt sô vreuden rîch.\\ 
 & ez was den vreuden \textbf{d\textit{â}} glîch.\\ 
25 & alsus mit vreudehafter ger\\ 
 & die rîter dar, die vrouwen her\\ 
 & dicke an ein ander blic\textit{t}en.\\ 
 & die von \textbf{der} vremde erschricten,\\ 
 & werdent si \textit{ie}m\textit{e}r \textbf{heimelîch} baz,\\ 
30 & daz sol ich lâzen âne haz.\\ 
\end{tabular}
\scriptsize
\line(1,0){75} \newline
U V W Q R Fr40 \newline
\line(1,0){75} \newline
\textbf{1} \textit{Initiale} U V Fr40   $\cdot$ \textit{Capitulumzeichen} R  \newline
\line(1,0){75} \newline
\textbf{1} sîgen] strauchen W seigen Fr40 \textbf{2} daz sîn] Des W  $\cdot$ gelac] lag W \textbf{3} \textit{Versfolge 638.4-3} W   $\cdot$ daz] Vnde daz V (W) (Q) (R) (Fr40)  $\cdot$ wolken] [wuken]: wllken R \textbf{4} \textit{Vers 638.4 fehlt} R   $\cdot$ daz] Des Q (Fr40)  $\cdot$ der] di Fr40  $\cdot$ boten] betten W \textbf{5} sternen] sterne W Q Fr40 \textbf{6} wan] [*]: wande V [Wo*]: Wo R  $\cdot$ er der naht] der U [*]: er der naht V \textbf{7} nâch] Nacht Q (Fr40)  $\cdot$ naht] \textit{om.} W \textbf{8} selbe] selbes W \textbf{9} \textit{Die Verse 638.9-10 fehlen} Q Fr40   $\cdot$ tiuriu] túrre R \textbf{11} dem] den W Q Fr40 \textbf{12} gekerzet] bekertzet W \textbf{13} sunder] besunder W \textbf{14} man] \textit{om.} R \textbf{15} dar] Das W Do Q \textbf{16} diu] Der Q  $\cdot$ lieht] licht Q (Fr40) \textbf{17} dekeiniu] [dehein*]: deheine V deheinen R eine Fr40  $\cdot$ brâht] bricht R \textbf{18} dân wære] Do enwere V Do wer W Da were R  $\cdot$ niemer] niender V (Q) (R) (Fr40) nirgent W  $\cdot$ ir] der R  $\cdot$ naht] nicht Q n::: Fr40 \textbf{19} selbe] selber V  $\cdot$ tagen] [tragen]: tagen Q iagen Fr40 \textbf{20} hôrt] hoͤr W  $\cdot$ süezen] werden W \textbf{21} enwolle] welle V W entwolt Q  $\cdot$ unreht] denne [*]: vnrehtez V dann vnrechtes W \textbf{22} hât ir] haben sy W habten ir Q  $\cdot$ ê] ie W  $\cdot$ gesehen] geschehen Fr40 \textbf{24} dâ] do U V W Q R \textbf{25} vreudehafter] froͤden W frúntschaffter R \textbf{27} ander] andren R  $\cdot$ blicten] blicken U W \textbf{28} der] dem Q  $\cdot$ vremde] froͯden R  $\cdot$ erschricten] [erschrike*]: erschriketen V erschricken W Q \textbf{29} iemer] mir U [*]: iemer V  $\cdot$ heimelîch] [heinlich*]: heinlicher V haymlicher W \newline
\end{minipage}
\end{table}
\end{document}
