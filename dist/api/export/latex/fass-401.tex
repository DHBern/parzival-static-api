\documentclass[8pt,a4paper,notitlepage]{article}
\usepackage{fullpage}
\usepackage{ulem}
\usepackage{xltxtra}
\usepackage{datetime}
\renewcommand{\dateseparator}{.}
\dmyyyydate
\usepackage{fancyhdr}
\usepackage{ifthen}
\pagestyle{fancy}
\fancyhf{}
\renewcommand{\headrulewidth}{0pt}
\fancyfoot[L]{\ifthenelse{\value{page}=1}{\today, \currenttime{} Uhr}{}}
\begin{document}
\begin{table}[ht]
\begin{minipage}[t]{0.5\linewidth}
\small
\begin{center}*D
\end{center}
\begin{tabular}{rl}
\textbf{401} & \begin{large}E\end{large}in ander ors man im dô lêch.\\ 
 & des sînen er sich gar verzêch.\\ 
 & man hieng ouch ander kleider an in.\\ 
 & \textbf{jenez} was der valkenære gewin.\\ 
5 & hie kom Gawan zuo geriten.\\ 
 & âvoy, \textbf{nû} wart dâ niht vermiten,\\ 
 & er\textbf{n} würde baz enpfangen\\ 
 & den ze Karidœl \textbf{wære} ergangen\\ 
 & Erekes enpfâhen,\\ 
10 & dô er begunde nâhen\\ 
 & Artuse nâch sîme strîte\\ 
 & unt \textbf{dô} vrou Enite\\ 
 & sîner vreude was ein cundewier,\\ 
 & sît im Maliclisier,\\ 
15 & daz \textbf{twerc}, \textbf{sîn} \textbf{vel} unsanfte brach\\ 
 & mit der geisel, \textbf{dâ ez} Gynover sach,\\ 
 & \textbf{unt} dô ze Tulmeyn \textbf{ein} strît\\ 
 & ergienc in dem kreize wît\\ 
 & umben sperwære.\\ 
20 & Ider fil Noyt, der mære,\\ 
 & im \textbf{sîne} sicherheit dâ bôt.\\ 
 & \textbf{er muos\textit{e} si im bieten} vür den tôt.\\ 
 & Die rede lât sîn \textbf{unt} hœrt \textbf{si} ouch hie.\\ 
 & ich wæne, sô \textbf{vriescht} ir nie\\ 
25 & \textbf{werden} anpfanc noch gruoz.\\ 
 & ouwê, des wirt unsanfte buoz\\ 
 & des werden Lotes kinde!\\ 
 & râtet irz, ich erwinde\\ 
 & unt sag iu vürbaz niht mêre.\\ 
30 & durch \textbf{trûren} tuon ich widerkêre.\\ 
\end{tabular}
\scriptsize
\line(1,0){75} \newline
D \newline
\line(1,0){75} \newline
\textbf{1} \textit{Initiale} D  \textbf{23} \textit{Majuskel} D  \newline
\line(1,0){75} \newline
\textbf{8} Karidœl] karidoͤl D \textbf{9} Erekes] Ereckes D \textbf{11} Artuse] Artvͦse D \textbf{12} Enite] Enîte D \textbf{14} Maliclisier] Malicliscîer D \textbf{20} Ider] Jder D \textbf{22} muose] mvͦsese D \textbf{27} Lotes] Lots D \newline
\end{minipage}
\hspace{0.5cm}
\begin{minipage}[t]{0.5\linewidth}
\small
\begin{center}*m
\end{center}
\begin{tabular}{rl}
 & ein ande\textit{r} ros man ime dô lêch.\\ 
 & des sînen er sich gar verzêch.\\ 
 & man hienc ouch ander kleit an in.\\ 
 & \textbf{jenez} was der valkenære gewin.\\ 
5 & \textit{h}ie kam Gawan zuo geriten.\\ 
 & â\textit{v}oy, \textbf{im} wart d\textit{â} niht vermiten,\\ 
 & er würde baz enpfangen\\ 
 & danne ze Karidol \textbf{was} ergangen\\ 
 & Ereckes enpfâhen,\\ 
10 & dô er begunde nâhen\\ 
 & Artuse nâch sînem strîte\\ 
 & und \textbf{dô} vrouwe Enite\\ 
 & sîner vrœde was ein condewier,\\ 
 & sît i\textit{m} Malachlisier,\\ 
15 & daz \textbf{twerc}, \textbf{sîn} \textbf{vel} unsanfte brach\\ 
 & mit der geisel, \textbf{d\textit{â} ez} Ginover sach,\\ 
 & dô ze Tulmein \textbf{der} strît\\ 
 & ergienc in dem kreize wît\\ 
 & umb den sperwære.\\ 
20 & Iders fil N\textit{o}t, der mære,\\ 
 & im sicherheit d\textit{â} bôt.\\ 
 & \textbf{die muose er bieten} v\textit{ür} den tôt.\\ 
 & die red lât sîn \textbf{und} hœret\textbf{z} ouch hie.\\ 
 & ich wæne, sô \textbf{\textit{ge}vr\textit{ies}ch\textit{e}t} ir nie\\ 
25 & \textbf{werdern} anpfan\textit{c} noch gruoz.\\ 
 & ouwê, des wirt \textit{un}sanfte buoz\\ 
 & des werden Lotes kinde!\\ 
 & râtet irz, ich erwinde\\ 
 & und sage iu vürbaz niht mêre.\\ 
30 & durch \textbf{trûren} tuon ich widerkêre.\\ 
\end{tabular}
\scriptsize
\line(1,0){75} \newline
m n o \newline
\line(1,0){75} \newline
\newline
\line(1,0){75} \newline
\textbf{1} ander] ande m \textbf{4} jenez] [Jener]: Jenes m \textbf{5} hie] Die m \textbf{6} âvoy im wart dâ] Anoi ym wart do m Ovi do wart n o \textbf{8} Karidol] karadel o  $\cdot$ was] were n (o)  $\cdot$ ergangen] [erhangen]: erganhen o \textbf{9} Ereckes] Erekes o  $\cdot$ enpfâhen] entpflogen o \textbf{11} Artuse] Artuͯse o \textbf{13} vrœde] freuͯiden o \textbf{14} im] ẏn m  $\cdot$ Malachlisier] malaclisier n malachaser o \textbf{16} dâ] do m n o  $\cdot$ Ginover] ginoffer m ginofer n o \textbf{17} Tulmein] tulinem n tulmen o \textbf{20} Iders] Jders m Jeders n o  $\cdot$ fil Not] vilnoit m filnort n sie not o \textbf{21} im] Jch im o  $\cdot$ dâ] do m n o  $\cdot$ bôt] bat o \textbf{22} muose] muͯsse m muͯste n  $\cdot$ bieten] bitten o  $\cdot$ vür den] froͯden m \textbf{23} hœretz ouch] hoͯren n horent o \textbf{24} wæne] wenet o  $\cdot$ gevrieschet] freicheit m gefreischet o \textbf{25} anpfanc] annpfangen m \textbf{26} des] das o  $\cdot$ unsanfte] yme sanfftte m \textbf{28} râtet] [Rotes]: Rotet o  $\cdot$ erwinde] enwinde o \textbf{29} iu] uch sage o \newline
\end{minipage}
\end{table}
\newpage
\begin{table}[ht]
\begin{minipage}[t]{0.5\linewidth}
\small
\begin{center}*G
\end{center}
\begin{tabular}{rl}
 & ein ander ors man im dô lêch.\\ 
 & des sînen er sich gar verzêch.\\ 
 & man hienc ouch ander kleit an in.\\ 
 & \textbf{daz} was der valkenære gewin.\\ 
5 & hie kom Gawan zuo geriten.\\ 
 & âvoy, \textbf{nû} wart dâ niht vermiten,\\ 
 & er\textbf{ne} würde baz enpfangen\\ 
 & dane ze Karidol \textbf{wære} ergangen\\ 
 & Erekes enpfâhen,\\ 
10 & dô er begunde nâhen\\ 
 & \begin{large}A\end{large}rtuse nâch sîme strîte\\ 
 & unde \textbf{dô} vrou Enite\\ 
 & sîner vröuden was ein condewier,\\ 
 & sît i\textit{m} Maliclisier,\\ 
15 & daz \textbf{getwerc}, \textbf{sîn} \textbf{vel} unsanfte brach\\ 
 & mit der geisel, \textbf{dâz} Schinovere sach,\\ 
 & \textbf{unt} dô ze Tolmein \textbf{ein} strît\\ 
 & ergienc in dem kreize wît\\ 
 & umbe den sparwære.\\ 
20 & Ider fil Not, der mære,\\ 
 & im \textbf{sîne} sicherheit dâ bôt.\\ 
 & \textbf{er muose im bieten} vür den tôt.\\ 
 & die rede lât sîn. \textbf{nû} hœrt\textbf{z} ouch hie.\\ 
 & ich wæne, s\textit{ô} \textbf{gevrieschet} ir nie\\ 
25 & \textbf{werdern} antvanc noch gruoz.\\ 
 & owê, des wirt unsanfte buoz\\ 
 & des werden Lotes kinde!\\ 
 & rât irz, ich erwinde\\ 
 & unt sage iu vürbaz niht mêre.\\ 
30 & durch \textbf{trûren} tuon ich widerkêre.\\ 
\end{tabular}
\scriptsize
\line(1,0){75} \newline
G I O L M Q R Z \newline
\line(1,0){75} \newline
\textbf{1} \textit{Initiale} I O L  \textbf{11} \textit{Initiale} G  \textbf{17} \textit{Initiale} I  \newline
\line(1,0){75} \newline
\textbf{1} \textit{Die Verse 370.13-412.12 fehlen} Q   $\cdot$ ein] ÷in O  $\cdot$ dô] \textit{om.} O doch L da M Z \textbf{2} gar] da Z \textbf{3} in] im L \textbf{4} daz] Jenez O (L) (M) (R) Z \textbf{5} Gawan] gewan R  $\cdot$ zuo] \textit{om.} O \textbf{6} âvoy] Awe O \textbf{7} erne] Er O R  $\cdot$ baz] da baz I \textbf{8} Karidol] koridol L  $\cdot$ ergangen] er gegangen L \textbf{9} Erekes] ereches I (O) (L) Ereckes R Z \textbf{10} dô] Da M Z \textbf{11} Artuse] Artus L M \textbf{12} dô] da M Z  $\cdot$ Enite] Enîte O enyte R \textbf{13} vröuden] frevde O (L) Z \textbf{14} im] in G irz I  $\cdot$ Maliclisier] maledizier I Maliachlisir O Malicliser L Maliclisir R malitlisir Z \textbf{15} sîn vel] \textit{om.} I sin vil M \textbf{16} \textit{Vers 401.16 fehlt} Z   $\cdot$ geisel] Gaislen I (R)  $\cdot$ dâz] daz ez I O (L) (M)  $\cdot$ Schinovere] kinovere G Ginouer I Gynover O Genover L ginover M Gynower R  $\cdot$ sach] ersach O \textbf{17} dô] da M Z  $\cdot$ Tolmein] tulmei I tvlinen O Tvlmein L (M) Z Toͯlmein R \textbf{18} dem] einem I  $\cdot$ kreize] [strite]: kreizze Z \textbf{19} den] ein Z \textbf{20} Ider] Jder O L Z Jdel M Jrde R  $\cdot$ fil Not] filnot G I R fil noyt O vil novt L fil Noit M Z \textbf{21} \textit{Die Verse 401.21-24 fehlen} L   $\cdot$ im] vmb I  $\cdot$ dâ] \textit{om.} I do O R \textbf{22} muose] must si I (O) muͦst es R mustez Z  $\cdot$ den] der Z \textbf{23} nû] vnd I (O) (M) Z  $\cdot$ hœrtz ouch] horet auch I horit M hoͯrren R  $\cdot$ hie] wie Z \textbf{24} sô] sone G das R  $\cdot$ gevrieschet] frieschet O (M) (Z) freisheit R \textbf{25} werdern] Wedern L Wedir ern M Werdem R \textbf{26} des] das R  $\cdot$ wirt] wart O \textbf{27} Lotes] lotis M \textbf{29} sage] ensag Z \newline
\end{minipage}
\hspace{0.5cm}
\begin{minipage}[t]{0.5\linewidth}
\small
\begin{center}*T
\end{center}
\begin{tabular}{rl}
 & Ein ander ors man im dô lêch.\\ 
 & des sînen er sich gar verzêch.\\ 
 & man hienc ouch ander kleit an in.\\ 
 & \textbf{jenez} was der valkenær gewin.\\ 
5 & Hie kom \textbf{hêr} Gawan zuo geriten.\\ 
 & âvoy, \textbf{nû} wart dâ niht vermiten,\\ 
 & er\textbf{n} würde baz enpfangen\\ 
 & denne ze Karidol \textbf{wær\textit{e}} \textit{erg}angen\\ 
 & Ereckes enpfâhen,\\ 
10 & dô er begunde nâhen\\ 
 & Artuse nâch sînem strîte\\ 
 & unde vrou Enite\\ 
 & sîner vröuden was ein cundewier,\\ 
 & sît im Malichschier,\\ 
15 & daz \textbf{getwerc}, \textbf{sînen} \textbf{rücke} unsanf\textit{t}e brach\\ 
 & mit der geisel, \textbf{daz e\textit{z}} \textbf{vrou} Gynover sach,\\ 
 & \textbf{unde} dô ze Tulmein \textbf{ein} strît\\ 
 & ergienc in dem kreize wît\\ 
 & umbe den sperwære.\\ 
20 & Yder filly Not, der mære,\\ 
 & im \textbf{sîne} sicherheit dâ bôt.\\ 
 & \textbf{er muose sîn beiten} vür den tôt.\\ 
 & Die rede lât sîn \textbf{unde} hœret \textbf{si} ouch hie.\\ 
 & ich wæne, sô \textbf{vrieschet} ir nie\\ 
25 & \textbf{werdern} anpfanc noch gruoz.\\ 
 & ouwê, des wirt unsanfte buoz\\ 
 & des werden Lotes kinde!\\ 
 & râtet irz, ich erwinde\\ 
 & unde sagiu vürbaz niht mêr.\\ 
30 & durch \textbf{triuwe} tuon ich widerkêr.\\ 
\end{tabular}
\scriptsize
\line(1,0){75} \newline
T U V W \newline
\line(1,0){75} \newline
\textbf{1} \textit{Initiale} W   $\cdot$ \textit{Majuskel} T  \textbf{5} \textit{Initiale} V   $\cdot$ \textit{Majuskel} T  \textbf{23} \textit{Majuskel} T  \textbf{29} \textit{Initiale} V  \newline
\line(1,0){75} \newline
\textbf{2} gar] do gar W  $\cdot$ verzêch] verzich W \textbf{3} kleit] cleider U V (W) \textbf{4} jenez] Jch U \textbf{5} Hie kam gawan her geritten W \textbf{6} dâ] do U V W \textbf{7} ern] Er W \textbf{8} wære ergangen] werer gevangen T \textbf{12} unde] Vnd do W  $\cdot$ Enite] enẏte V \textbf{13} vröuden] vreide U (V) \textbf{14} Malichschier] Maliclichir U maliclisier V masiklisier W \textbf{15} sînen rücke] \textit{om.} U sin vel V (W)  $\cdot$ unsanfte] [vnsanfe]: vnsanffe T im W \textbf{16} geisel] geischelen V (W)  $\cdot$ ez] er T  $\cdot$ vrou] \textit{om.} W  $\cdot$ Gynover] gŷnover T gynovere V tschinouer W \textbf{17} Tulmein] Tvlmêin T tvlmei V tulmin U  $\cdot$ ein] der V \textbf{20} Yder] Jder T Jders V  $\cdot$ filly Not] Fillynot T fili not U V vil not W \textbf{21} dâ] do U V W \textbf{22} er muose] er mvese T Die muͤste V Er muͦsses W  $\cdot$ sîn beiten] sie im bieten U er bieten V im bieten W \textbf{23} si] es W \textbf{24} vrieschet] in gevrieschet U enfrieschet V gefreischet W \textbf{25} anpfanc] vmb vanc U entpfang W \textbf{29} vürbaz niht] nit vuͦrbaz U fúr nicht W \textbf{30} triuwe] truͦre U truren V (W) \newline
\end{minipage}
\end{table}
\end{document}
