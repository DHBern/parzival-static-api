\documentclass[8pt,a4paper,notitlepage]{article}
\usepackage{fullpage}
\usepackage{ulem}
\usepackage{xltxtra}
\usepackage{datetime}
\renewcommand{\dateseparator}{.}
\dmyyyydate
\usepackage{fancyhdr}
\usepackage{ifthen}
\pagestyle{fancy}
\fancyhf{}
\renewcommand{\headrulewidth}{0pt}
\fancyfoot[L]{\ifthenelse{\value{page}=1}{\today, \currenttime{} Uhr}{}}
\begin{document}
\begin{table}[ht]
\begin{minipage}[t]{0.5\linewidth}
\small
\begin{center}*D
\end{center}
\begin{tabular}{rl}
\textbf{243} & \begin{large}D\end{large}em bette \textbf{armuot was} tiwer.\\ 
 & als er \textbf{glohte} \textbf{in} eime viwer,\\ 
 & lac drûffe ein pfellel \textbf{lieht} gemâl.\\ 
 & die ritter \textbf{bat} dô Parzival\\ 
5 & wider varen an ir gemach,\\ 
 & dô er \textbf{dâ niht mêr} bette sach.\\ 
 & mit urloube si \textbf{vuoren} dan.\\ 
 & hie hebt sich ander dienst an.\\ 
 & Vil kerzen unt diu varwe sîn,\\ 
10 & \textbf{die} gâben ze gegenstrîte schîn.\\ 
 & waz mohte \textbf{liehter sîn} der tac?\\ 
 & vor sînem bette ein \textbf{anderz} lac,\\ 
 & dâr \textbf{ûfe} ein kulter, dâ er \textbf{dâ} saz.\\ 
 & junchêrren snel unt niht \textbf{ze} laz,\\ 
15 & \textbf{maneger} im dar nâher spranc.\\ 
 & si entschuochten bein, di\textit{u} wâren blanc.\\ 
 & ouch zôch im mêr gewandes abe\\ 
 & manec wol geborner knabe.\\ 
 & vlætec wâren \textbf{diu selben} kindelîn.\\ 
20 & \textbf{dar nâch gienc dô zer tür dar} în\\ 
 & \textbf{vil} clâre juncvrouwen,\\ 
 & die \textbf{solten dennoch} schouwen,\\ 
 & wie man des heldes pflæge\\ 
 & unt ob er sanfte læge.\\ 
25 & Als mir diu âventiure gewuoc,\\ 
 & vor ieslîcher ein knappe truoc\\ 
 & eine kerzen, diu wol bran.\\ 
 & Parzival, der \textbf{snelle} man,\\ 
 & spranc under\textit{z} declachen.\\ 
30 & si \textbf{sagten}: "ir sult wachen\\ 
\end{tabular}
\scriptsize
\line(1,0){75} \newline
D \newline
\line(1,0){75} \newline
\textbf{1} \textit{Initiale} D  \textbf{9} \textit{Majuskel} D  \textbf{25} \textit{Majuskel} D  \newline
\line(1,0){75} \newline
\textbf{16} diu] di D \textbf{29} underz] vnders D \newline
\end{minipage}
\hspace{0.5cm}
\begin{minipage}[t]{0.5\linewidth}
\small
\begin{center}*m
\end{center}
\begin{tabular}{rl}
 & dem bette \textbf{was armuot} tiure.\\ 
 & alsô er \textbf{glüete} \textbf{in} einem viure,\\ 
 & lac drûf ein pfelle \textbf{lieht} gemâl.\\ 
 & die ritter \textbf{bat} dô Parcifal\\ 
5 & wider varen an ir gemach,\\ 
 & dô er \textbf{d\textit{â} niht mêre} bett\textit{e} \textit{s}ach.\\ 
 & mit urloube si \textbf{vuoren} dan.\\ 
 & hie hebet sich ander dienest an.\\ 
 & vil kerzen und diu varwe sîn,\\ 
10 & \textbf{die} gâben ze gegenstrîte schîn.\\ 
 & waz mohte \textbf{liehter sîn} der tac?\\ 
 & vor sînem bette ein \textbf{anderz} lac,\\ 
 & dâr ein kulter, d\textit{â} er \textbf{dâ ûf} saz.\\ 
 & junchêrren snelle und niht \textbf{ze} laz,\\ 
15 & \textbf{maniger} ime dar nâher spranc.\\ 
 & si entschuoheten bein, diu wâren blanc.\\ 
 & ouch zôch ime mêr gewandes abe\\ 
 & manic wolgeborner knabe.\\ 
 & vlætic wâren \textbf{diu selben} kindelîn.\\ 
20 & \textbf{dar nâch zer tür dô giengen} în\\ 
 & \textbf{vil} klâr\textit{e} \textit{j}uncvrouwen,\\ 
 & die \textbf{solten dennoch} schouwen,\\ 
 & wie man des heldes pflæge\\ 
 & und ob er sanfte læge.\\ 
25 & als mir diu âventiure gewuoc,\\ 
 & vor ieglîcher ein knappe truoc\\ 
 & eine kerzen, diu wol bran.\\ 
 & Parcifal, der \textbf{snelle} man,\\ 
 & spranc under daz deckelachen.\\ 
30 & si \textbf{sprâchen}: "ir sullet wachen\\ 
\end{tabular}
\scriptsize
\line(1,0){75} \newline
m n o Fr69 \newline
\line(1,0){75} \newline
\newline
\line(1,0){75} \newline
\textbf{2} er glüete in einem] ergluͯget niemem n er gluget in eẏnem o \textbf{3} lieht] [sus]: liecht o \textbf{4} bat] bat bat o \textbf{6} dâ] do m Fr69 \textit{om.} n o  $\cdot$ bette sach] bette vant vnd sach m \textbf{10} ze] do o \textbf{11} mohte] moͯchte n \textbf{12} sînem] sẏnen o  $\cdot$ anderz] ander o \textbf{13} dâr] Dar vff n (o) (Fr69)  $\cdot$ dâ er dâ ûf] do er do vff m do er n o da er da Fr69 \textbf{15} ime] zim Fr69 \textbf{16} diu wâren] \textit{om.} n o \textbf{19} diu] die die o \textbf{21} Vil klâre jung jungfrouwen m  $\cdot$ Vil clare juͯncfrowelen o \textbf{25} gewuoc] wuͦg n \textbf{27} kerzen] kertze n (o) \newline
\end{minipage}
\end{table}
\newpage
\begin{table}[ht]
\begin{minipage}[t]{0.5\linewidth}
\small
\begin{center}*G
\end{center}
\begin{tabular}{rl}
 & dem bette \textbf{was armuote} tiur.\\ 
 & alser \textbf{gleste} \textbf{ûz} einem viur,\\ 
 & lac drûffe ein pfelle \textbf{wol} gemâl.\\ 
 & die rîter \textbf{bat} dô Parzival\\ 
5 & wider varen an ir gemach,\\ 
 & dôr \textbf{dâ nimer} bette sach.\\ 
 & mit urloube si \textbf{schieden} dan.\\ 
 & hie hebt sich ander dienst an.\\ 
 & vil kerzen unt diu varwe sîn,\\ 
10 & \textbf{die} gâben ze gegenstrîte schîn.\\ 
 & waz mahte \textbf{liehter sîn} der tac?\\ 
 & vor sînem bette ein \textbf{anderz} lac,\\ 
 & dâr \textbf{ûfe} ein kulter, dâ er saz.\\ 
 & junchêrren snel unde niht laz,\\ 
15 & \textbf{ein teil ir} im dar nâher spranc.\\ 
 & si entschuohten bein, diu wâren blanc.\\ 
 & ouch zôch im mê gewandes abe\\ 
 & manic wol geborner knabe.\\ 
 & vlætic wâren \textbf{diu} kindelîn.\\ 
20 & \textbf{nû seht}, \textbf{dort kom zer tür her} în\\ 
 & \textbf{\begin{large}V\end{large}ier} clâre juncvrouwen,\\ 
 & die \textbf{dannoch wolten} schouwen,\\ 
 & wie man des heldes pflæge\\ 
 & unde ober sanfte læge.\\ 
25 & als mir diu âventiure gewuoc,\\ 
 & vor ieslî\textit{ch}er ein knappe truoc\\ 
 & eine kerzen, diu wol bran.\\ 
 & Parzival, der \textbf{snelle} man,\\ 
 & spranc underz declachen.\\ 
30 & si \textbf{sprâchen}: "ir sult wachen\\ 
\end{tabular}
\scriptsize
\line(1,0){75} \newline
G I O L M Q R Z Fr54 \newline
\line(1,0){75} \newline
\textbf{1} \textit{Initiale} L Q Z  \textbf{9} \textit{Initiale} I R  \textbf{21} \textit{Initiale} G  \newline
\line(1,0){75} \newline
\textbf{1} bette] ritter Q  $\cdot$ armuote] ir ::: Fr54 \textbf{2} alser gleste] als [geste]: geleste I Als erglest Q Ez ergleste Z  $\cdot$ ûz] vns Q vsser R  $\cdot$ einem] dem I \textbf{3} wol] lieht O (R) Z lýcht L (M) (Q) (Fr54) \textbf{4} die] [Der]: Die O Der Fr54  $\cdot$ dô] da M  $\cdot$ Parzival] parzifal I M Parcifal O (L) (Z) patzifal Q parczifal R partzival Fr54 \textbf{5} varen] gen L varn varn Z \textbf{6} dôr dâ] Do er do L Q Da er da M Z  $\cdot$ nimer] nih mer I (O) (L) (M) (Q) (R) (Z) \textbf{7} si] si sich I do sy R \textbf{8} ander] ein ander O \textbf{9} varwe] frawe Q (R) \textbf{10} die] \textit{om.} Fr54  $\cdot$ gegenstrîte] gahen strite I \textbf{11} waz] Wez L Wie Fr54  $\cdot$ liehter] lýchter L (M) (Q)  $\cdot$ sîn] sin danne O (Q) (Z) syn wan M \textbf{12} anderz] [andez]: andrez I andir M (Q) \textbf{13} kulter] teppich Fr54  $\cdot$ dâ] do Q daruff R \textbf{14} junchêrren] iuncfrowen I  $\cdot$ laz] zelaz O (L) (M) (Q) (R) [*iht]: zv laz  Z >ze< laz Fr54 \textbf{15} ein teil ir] Einer L Manger Q (R) Ein tril ir Fr54  $\cdot$ im] \textit{om.} I  $\cdot$ dar] do R \textbf{16} si] Die O M (Q) R Sine Fr54  $\cdot$ entschuohten] geschuͦhten Fr54  $\cdot$ blanc] lanck Q \textbf{17} im] her ym M  $\cdot$ gewandes] gewande Z \textbf{18} knabe] chab I \textbf{19} vlætic] Ê ledich O Vnledig Q  $\cdot$ diu] disiv O (M) Fr54 die selben Z \textbf{20} nû] \textit{om.} R  $\cdot$ dort] do I  $\cdot$ kom] komen L (M) (R) (Z) kompt Q \textbf{23} heldes] helde M helden R \textbf{24} ober] ob er er L \textbf{25} mir] vns R  $\cdot$ gewuoc] giht Q geruͦcht R \textbf{26} ieslîcher] ieslier G \textbf{27} kerzen] kerz I (L) \textbf{28} Parzival] parzifal I (L) (M) Parcifal O Z Partzifal Q Parczifal R \textbf{30} sult] muͤzt I \newline
\end{minipage}
\hspace{0.5cm}
\begin{minipage}[t]{0.5\linewidth}
\small
\begin{center}*T
\end{center}
\begin{tabular}{rl}
 & \begin{large}D\end{large}em bette \textbf{was armuot} tiure.\\ 
 & Alser \textbf{gleste} \textbf{ûz} einem viure,\\ 
 & lac drûfe ein pfelle \textbf{lieht} gemâl.\\ 
 & die rîter \textbf{hiez} dô Parcifal\\ 
5 & wider varn anir gemach,\\ 
 & dô er \textbf{niht mêre dâ} bette sach.\\ 
 & Mit urloube si \textbf{schieden} dan.\\ 
 & hie hebet sich ander dienst an.\\ 
 & Vil kerzen unde diu varwe sîn\\ 
10 & gâben ze gegenstrîte schîn.\\ 
 & waz mohte \textbf{sîn liehter} \textbf{danne} der tac?\\ 
 & vor sînem bette ein \textbf{tepich} lac,\\ 
 & dâr \textbf{ûf} ein kulter, dâ er saz.\\ 
 & Junchêrren snel unde niht \textbf{ze} laz,\\ 
15 & \textbf{genuoge ir} im dar nâher spranc.\\ 
 & si entschuohten \textbf{im diu} bein, diu wâren blanc.\\ 
 & ouch zôch im mê gewandes abe\\ 
 & manec wol geborner knabe.\\ 
 & vlætic wâren \textbf{dis\textit{iu}} kindelîn.\\ 
20 & \textbf{dar nâch gie zer tür dar} în\\ 
 & \textbf{vier} clâre juncvrouwen,\\ 
 & die \textbf{solten dannoch} schouwen,\\ 
 & wie man des heldes pflæge\\ 
 & unde ober sanfte læge.\\ 
25 & Als mir diu âventiure gewuoc,\\ 
 & vor ieglîcher ein knappe truoc\\ 
 & eine kerze, diu wol bran.\\ 
 & Parcifal, der \textbf{werde} man,\\ 
 & spranc underz deckelachen.\\ 
30 & Si \textbf{sprâchen}: "ir sult wachen\\ 
\end{tabular}
\scriptsize
\line(1,0){75} \newline
T U V W \newline
\line(1,0){75} \newline
\textbf{1} \textit{Initiale} T U V W  \textbf{2} \textit{Majuskel} T  \textbf{7} \textit{Majuskel} T  \textbf{9} \textit{Majuskel} T  \textbf{14} \textit{Majuskel} T  \textbf{25} \textit{Majuskel} T  \textbf{28} \textit{Majuskel} T  \textbf{30} \textit{Majuskel} T  \newline
\line(1,0){75} \newline
\textbf{3} lieht] wol W \textbf{4} hiez] [*]: bat V  $\cdot$ Parcifal] parzifal T V partzifal W \textbf{6} niht mêre dâ bette] nit mere bette U [*]: da niht mer bette V do nit hertte W \textbf{7} si] \textit{om.} W \textbf{8} dienst an] dienstman W \textbf{10} gâben] [*]: Die gaben V \textbf{11} mohte] moͤhte V were W  $\cdot$ sîn liehter] liechter sin U (V) liechter W  $\cdot$ danne] \textit{om.} V \textbf{12} Do auff ein kulter guͦter lag W  $\cdot$ tepich] [*]: anderz V \textbf{13} Do auff er allaine sas W  $\cdot$ ein] eine U  $\cdot$ dâ] do U  $\cdot$ er saz] [*]: er da saz V \textbf{15} genuoge ir] [*]: Maniger V  $\cdot$ dar] \textit{om.} U do W  $\cdot$ nâher] nahen W \textbf{16} im diu bein] [*ng]: bein V  $\cdot$ diu wâren] \textit{om.} W \textbf{18} geborner] geborne W \textbf{19} disiu] dise T [*]: die selben V die W \textbf{20} Darnach [*]: zer túr do giengen in V  $\cdot$ dar în] hin ein W \textbf{22} Den nach solten sy schawen W \textbf{27} kerze] kertzen W \textbf{28} Parcifal] Parzival T [*]: Parzifal V Partzifal W  $\cdot$ werde] [*]: snelle V kúne W \newline
\end{minipage}
\end{table}
\end{document}
