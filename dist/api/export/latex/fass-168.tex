\documentclass[8pt,a4paper,notitlepage]{article}
\usepackage{fullpage}
\usepackage{ulem}
\usepackage{xltxtra}
\usepackage{datetime}
\renewcommand{\dateseparator}{.}
\dmyyyydate
\usepackage{fancyhdr}
\usepackage{ifthen}
\pagestyle{fancy}
\fancyhf{}
\renewcommand{\headrulewidth}{0pt}
\fancyfoot[L]{\ifthenelse{\value{page}=1}{\today, \currenttime{} Uhr}{}}
\begin{document}
\begin{table}[ht]
\begin{minipage}[t]{0.5\linewidth}
\small
\begin{center}*D
\end{center}
\begin{tabular}{rl}
\textbf{168} & Der gast an daz bette schreit.\\ 
 & al wîz gewant \textbf{im was} bereit.\\ 
 & von golde unde sîdîn\\ 
 & einen \textbf{bruochgürtel} zôch man drîn.\\ 
5 & scharlachens hosen rôt man streich\\ 
 & an in, dem \textbf{ellen nie} gesweich.\\ 
 & \textit{\begin{large}Â\end{large}}voy, wie stuonden sîniu bein!\\ 
 & \textbf{reht} geschickede ab \textbf{in} schein.\\ 
 & brûn scharlachen wol gesniten,\\ 
10 & dem was furrieren niht vermiten,\\ 
 & beidiu \textbf{innen} \textbf{härmîn} blanc,\\ 
 & roc unt mantel wâren lanc.\\ 
 & \textbf{breit}, swarz unt grâ\\ 
 & zobel dâr vor \textbf{man kôs} \textbf{al} dâ.\\ 
15 & \textbf{daz} leite an der gehiure.\\ 
 & under einen g\textit{ür}te\textit{l} tiure\\ 
 & wart \textit{er} gefischieret\\ 
 & unt wol gezimieret\\ 
 & mit einem tiurem vürspan.\\ 
20 & sîn munt dâ \textbf{bî} \textbf{vor} rœte bran.\\ 
 & Dô kom der wirt mit triwen kraft,\\ 
 & nâch dem gie stolziu rîterschaft.\\ 
 & \textbf{der enpfienc} den gast. dô daz geschach,\\ 
 & der ritter ieslîcher sprach,\\ 
25 & si\textbf{ne} gesæhen \textbf{nie} sô schœnen lîp.\\ 
 & mit triwen lobten si daz wîp,\\ 
 & diu \textbf{gap der werlde} \textbf{al}sölhe vruht.\\ 
 & durch wârheit unt umb \textbf{ir} zuht\\ 
 & si \textbf{jâhen}: "er wirt wol gewert,\\ 
30 & swâ sîn dienst gnâden gert.\\ 
\end{tabular}
\scriptsize
\line(1,0){75} \newline
D \newline
\line(1,0){75} \newline
\textbf{1} \textit{Majuskel} D  \textbf{7} \textit{Initiale} D  \textbf{21} \textit{Majuskel} D  \newline
\line(1,0){75} \newline
\textbf{7} Âvoy] ÷voy D \textbf{16} gürtel] gvlter D \textbf{17} er] \textit{om.} D \newline
\end{minipage}
\hspace{0.5cm}
\begin{minipage}[t]{0.5\linewidth}
\small
\begin{center}*m
\end{center}
\begin{tabular}{rl}
 & der gast an daz bette schreit.\\ 
 & al wîz gewant \textbf{ime was} bereit.\\ 
 & von golt und \textbf{ouch} sîdîn\\ 
 & einen \textbf{gürtel} zôch man drîn.\\ 
5 & scharlachens hosen rôt man streich\\ 
 & an in, dem \textbf{ellen nie} geswei\textit{ch}.\\ 
 & â\textit{v}oy, wie stuonden sîniu bein!\\ 
 & \textbf{reht} geschickede ab \textbf{in} schein.\\ 
 & brûn scharlachen wol gesniten,\\ 
10 & dem was furrieren niht vermiten,\\ 
 & beidiu \textbf{innen} \textbf{her\textit{mîn}} blanc,\\ 
 & roc und mantel wâren lanc.\\ 
 & \textbf{breit}, swarz und grâ\\ 
 & zobel dâr vor \textbf{kôs man} d\textit{â}.\\ 
15 & \textbf{daz} legete an der gehiure.\\ 
 & under einen gürtel tiure\\ 
 & wart er gefischieret\\ 
 & und wol gezimieret\\ 
 & mit einem tiuren vürspa\textit{n}.\\ 
20 & sîn munt dâ \textbf{bî} \textbf{von} rœte bran.\\ 
 & \begin{large}D\end{large}ô kam der wirt mit triuwen kraft,\\ 
 & nâch dem gienc stolziu ritterschaft.\\ 
 & \textbf{der entpfienc} den gast. dô daz geschach,\\ 
 & der ritter ieglîcher sprach,\\ 
25 & si gesæhen \textbf{niht} sô schœnen lîp.\\ 
 & mit triuwen lobeten si daz wîp,\\ 
 & diu \textbf{gap der wer\textit{l}de} \textbf{al}soliche vruht.\\ 
 & durch wârheit und umb \textbf{ir} zuht\\ 
 & si \textbf{jâhen}: "er wirt wol gewert,\\ 
30 & wâ sîn dienest genâden gert.\\ 
\end{tabular}
\scriptsize
\line(1,0){75} \newline
m n o Fr69 \newline
\line(1,0){75} \newline
\textbf{21} \textit{Initiale} m Fr69  \newline
\line(1,0){75} \newline
\textbf{1} bette] beste bette n (o) \textbf{3} ouch] von n o \textbf{4} drîn] dar an o \textbf{5} scharlachens] Scharlachen o  $\cdot$ rôt] rat o  $\cdot$ streich] do streich n \textbf{6} in] \textit{om.} n o  $\cdot$ gesweich] gesweis m \textbf{7} âvoy] Anoi m Ey n (o)  $\cdot$ sîniu] sinen n \textbf{10} furrieren] vor eren o \textbf{11} beidiu] Beiden o  $\cdot$ hermîn] her vm m \textbf{14} dâr vor kôs] den verkosz n (o)  $\cdot$ dâ] do m n \textbf{15} legete] leit n o \textbf{16} einen gürtel] einem kulter Fr69 \textbf{17} gefischieret] figurieret n o \textbf{19} vürspan] vuͯr spang m \textbf{22} stolziu] ein stoltz n \textbf{23} der] Den o  $\cdot$ geschach] beschach o \textbf{25} gesæhen] gesohen o  $\cdot$ niht] nye n (o) \textbf{27} werlde] werde m (n) (o)  $\cdot$ vruht] [fcht]: fvcht Fr69 \textbf{30} genâden gert] genode gerst n \newline
\end{minipage}
\end{table}
\newpage
\begin{table}[ht]
\begin{minipage}[t]{0.5\linewidth}
\small
\begin{center}*G
\end{center}
\begin{tabular}{rl}
 & der gast anz bette schreit.\\ 
 & alwîz gewant \textbf{was im} bereit.\\ 
 & von golde unde sîdîn\\ 
 & einen \textbf{bruochgürtel} zôch man drîn.\\ 
5 & scharlachens hosen \textit{rôte} man streich\\ 
 & an in, dem \textbf{ellen nie} gesweich.\\ 
 & âvoy, wie stuonden sîniu bein!\\ 
 & \textbf{rehte} geschicket abe \textbf{im} schein.\\ 
 & brûn scharlach wol gesniten,\\ 
10 & dem was furrieren niht vermiten,\\ 
 & beidiu \textbf{innen} \textbf{härmîn} blanc,\\ 
 & roc unde mandel wâren lanc.\\ 
 & \textbf{brûn}, swarz unde grâ\\ 
 & zobel dâr vor \textbf{man kôs} \textbf{al} dâ.\\ 
15 & \textbf{daz} leit an der gehiure.\\ 
 & under einen gürtel tiure\\ 
 & wart er gefischieret\\ 
 & unde wol gezimieret\\ 
 & mit einem tiuren vürspan.\\ 
20 & sîn munt dâ \textbf{bî} \textbf{von} rœte bran.\\ 
 & dô kom der wirt mit triwen kraft,\\ 
 & nâch dem gie stolziu rîterschaft.\\ 
 & \textbf{er gruozte} den gast. dô daz geschach,\\ 
 & der rîter ieslîcher sprach,\\ 
25 & si\textbf{ne} gesæhen \textbf{nie} sô schœnen lîp.\\ 
 & mit triwen lobten si daz wîp,\\ 
 & \begin{large}D\end{large}iu \textbf{gap der werlde} solhe vruht.\\ 
 & durch wârheit unde umbe \textbf{ir} zuht\\ 
 & si \textbf{sprâchen}: "er wirt wol gewert,\\ 
30 & swâ sîn dienst genâden gert.\\ 
\end{tabular}
\scriptsize
\line(1,0){75} \newline
G I O L M Q R Z Fr17 Fr21 \newline
\line(1,0){75} \newline
\textbf{1} \textit{Initiale} Q  \textbf{5} \textit{Initiale} O  \textbf{15} \textit{Initiale} L  \textbf{21} \textit{Initiale} I  \textbf{27} \textit{Initiale} G  \newline
\line(1,0){75} \newline
\textbf{2} alwîz] Als weysz Q \textbf{3} golde] goldin R  $\cdot$ unde sîdîn] vnd von sidin I (O) L (M) (Z) von seyden Q \textbf{4} einen] Eine O  $\cdot$ man] man im O \textbf{5} scharlachens] ÷Arlaches O Scharlats L Rot scharlages Q  $\cdot$ rôte] swarz G \textit{om.} O Q  $\cdot$ streich] im da an streich O sreich M (Q) \textbf{6} dem] den Q  $\cdot$ ellen nie] nie ellen I ellende nie O alle nie R  $\cdot$ gesweich] entweich I \textbf{7} âvoy] Awi O  $\cdot$ sîniu] im [s*]: sinev I \textbf{8} geschicket] geschit O  $\cdot$ abe] an M ob R \textbf{9} scharlach] schachin Q \textbf{10} furrieren] fuͯttrin R \textbf{11} blanc] wis R \textbf{12} unde] \textit{om.} Z  $\cdot$ mandel] mandes I  $\cdot$ wâren lanc] warn im lanc I mit vlis R \textbf{13} brûn] Breit O L M Q R Z (Fr17) Fr21 \textbf{14} zobel] Einen zobel Q (R)  $\cdot$ dâr vor man kôs] da vor chos man sa I man dar vf chos O kos man dar vor L den man vor los M der man vor kos Q man kos dervor R man dervor kos Fr21 \textbf{15} leit] leit er L gegleit R \textbf{16} Einen gᵫrtel der was tᵫre R  $\cdot$ einen] einem L (M) einē Q  $\cdot$ gürtel] gvlter O \textbf{17} gefischieret] gefeitiert O gefrischieret R \textbf{19} einem] einen I einer L R einē Q  $\cdot$ tiuren] thure M tewrem Q  $\cdot$ vürspan] fuͯrspang L (R) \textbf{20} sîn] Sint M  $\cdot$ von] mit M vor Z  $\cdot$ bran] brang R \textbf{21} dô] Da M (Fr17)  $\cdot$ triwen] gantzer Q \textit{om.} Z \textbf{22} dem] den M  $\cdot$ stolziu] groz::: I stolcze R groziv Fr17 \textbf{23} er] Der L  $\cdot$ gruozte] gruͤzt I (O) (Q) (Z) (Fr21)  $\cdot$ daz geschach] ern gesach O (Q) (Fr21) er in sach R \textbf{24} rîter] \textit{om.} R \textbf{25} sine] Si O (L) (M) (R) (Z) Die Q  $\cdot$ gesæhen] gesahen L (M) \textbf{27} Diu] diu da I (Z)  $\cdot$ solhe] solhiv O alsulche Q \textbf{28} umbe] durch I (L) \textbf{29} er wirt] wir R \textbf{30} swâ] Wo L (M) (Q) (R)  $\cdot$ sîn] sie Z  $\cdot$ genâden] [gn*]: gnade L \newline
\end{minipage}
\hspace{0.5cm}
\begin{minipage}[t]{0.5\linewidth}
\small
\begin{center}*T
\end{center}
\begin{tabular}{rl}
 & Der gast an daz bette schreit.\\ 
 & alwîz gewant \textbf{was im} bereit.\\ 
 & \hspace*{-.7em}\big| einen \textbf{bruochgürtel} zôch man drîn,\\ 
 & \hspace*{-.7em}\big| \textbf{der was} von golde unde sî\textit{d}în.\\ 
5 & Scharlachens hosen rô\textit{t} man streich\\ 
 & an in, dem \textbf{nie e\textit{ll}e\textit{n}} gesweich.\\ 
 & Âvoy, wie stuonden sîniu bein!\\ 
 & \textbf{wol} geschicket ab \textbf{im} schein.\\ 
 & brûn scharlachen wol gesniten,\\ 
10 & dem was furrieren niht vermiten,\\ 
 & beidiu \textbf{von} \textbf{harmen} blanc,\\ 
 & roc unde mantel wâren lanc.\\ 
 & \textbf{breit}, swarz unde grâ\\ 
 & \textbf{einen} zobel dâr vor \textbf{kôs man} dâ.\\ 
15 & \textbf{dâ} leit an der gehiure.\\ 
 & under einen gürtel tiure\\ 
 & wart er gefischieret\\ 
 & unde wol gezimieret\\ 
 & mit einem tiuren vürspan.\\ 
20 & sîn munt dâ \textbf{vor} \textbf{von} rœte bran.\\ 
 & \begin{large}D\end{large}ô kom der wirt mit triuwen kraft,\\ 
 & nâch dem gienc stolz\textit{iu} rîterschaft.\\ 
 & \textbf{er gruozte} den gast. dô daz geschach,\\ 
 & der rîter iegelîcher sprach,\\ 
25 & si\textbf{ne} gesæhen \textbf{nie} sô schœnen lîp.\\ 
 & mit triuwen lobeten si daz wîp,\\ 
 & diu \textbf{der werlte gap} \textbf{al}solhe vruht.\\ 
 & durch wârheit unde umbe zuht\\ 
 & si \textbf{sprâchen}: "er wirt wol gewert,\\ 
30 & swâ sîn dienst gnâden gert.\\ 
\end{tabular}
\scriptsize
\line(1,0){75} \newline
T U V W \newline
\line(1,0){75} \newline
\textbf{1} \textit{Majuskel} T  \textbf{5} \textit{Majuskel} T  \textbf{7} \textit{Majuskel} T  \textbf{21} \textit{Initiale} T U  \newline
\line(1,0){75} \newline
\textbf{1} an] in W \textbf{2} alwîz] Alle wiz U \textbf{4} \textit{Versfolge 168.3-4} W   $\cdot$ Einen vndergúrtel den er mocht leiden W \textbf{3} der was] \textit{om.} W  $\cdot$ sîdîn] sitlin T von seiden W \textbf{5} rôt] roc T \textit{om.} W \textbf{6} ellen] engel T [*llen]: ellen V ellends W \textbf{8} wol] Rechte W  $\cdot$ ab im] von [im]: in U \textbf{11} \textit{Die Verse 168.11-12 fehlen} W   $\cdot$ von harmen] [*en]: innen hermin V \textbf{13} breit] Blaich W \textbf{14} dâr vor kôs man dâ] den verkos man do U [*]: man kos der vor alda V koß man der vor da W \textbf{15} dâ] Daz V (W)  $\cdot$ leit] laite W \textbf{16} under] Vnd U \textbf{17} wart er] Er ward wol W  $\cdot$ gefischieret] [ge*]: gefigieret V \textbf{19} einem] [einen]: einem V \textbf{20} vor] [*]: bi V \textbf{22} stolziu] stolze T \textbf{23} geschach] beschach V W \textbf{25} sine] Sy W  $\cdot$ sô schœnen] schoͤneren V \textbf{27} Die do gab der welte soͤlche frucht W \textbf{28} zuht] [*]: ir zuht V \textbf{29} sprâchen] [*]: Jahen V \textbf{30} swâ] Wa U (W) \newline
\end{minipage}
\end{table}
\end{document}
