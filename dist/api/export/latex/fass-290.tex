\documentclass[8pt,a4paper,notitlepage]{article}
\usepackage{fullpage}
\usepackage{ulem}
\usepackage{xltxtra}
\usepackage{datetime}
\renewcommand{\dateseparator}{.}
\dmyyyydate
\usepackage{fancyhdr}
\usepackage{ifthen}
\pagestyle{fancy}
\fancyhf{}
\renewcommand{\headrulewidth}{0pt}
\fancyfoot[L]{\ifthenelse{\value{page}=1}{\today, \currenttime{} Uhr}{}}
\begin{document}
\begin{table}[ht]
\begin{minipage}[t]{0.5\linewidth}
\small
\begin{center}*D
\end{center}
\begin{tabular}{rl}
\textbf{290} & der \textbf{noch dort ûze} tjoste gert.\\ 
 & sîn lîp ist \textbf{ouch} wol prîses wert."\\ 
 & \begin{large}K\end{large}eie, der küene man,\\ 
 & brâht \textbf{diz} mære vür den künec sân,\\ 
5 & Segramors wære gestochen abe\\ 
 & unt dort ûze hielt ein strenger knabe,\\ 
 & der gerte tjoste \textbf{reht} als ê.\\ 
 & er sprach: "\textbf{hêrre}, mir tuot \textbf{immer} wê,\\ 
 & sol er\textbf{s} genozzen scheiden hin.\\ 
10 & ob ich iu sô \textbf{wirdec} bin,\\ 
 & lât mich versuochen, wes er ger,\\ 
 & sît er mit ûf gerihtem sper\\ 
 & dort \textbf{habt} vor iwerem wîbe.\\ 
 & nimmer ich belîbe\\ 
15 & in iwerem dienste mêre;\\ 
 & tavelrunde hât unêre,\\ 
 & ob man\textbf{z} \textbf{im niht bezîte} wert.\\ 
 & ûf unsern prîs sîn ellen zert.\\ 
 & \textbf{Nû} gebt mir strîtes urloup.\\ 
20 & wære wir \textbf{alle} blint oder toup,\\ 
 & \textbf{ir} soltz im weren, \textbf{des} wære zît."\\ 
 & Artus erloubete \textbf{Keien} strît.\\ 
 & Gewâpent wart der scheneschalt.\\ 
 & dô wolder swenden den walt\\ 
25 & mit tjoste ûf disen \textbf{kumenden} gast.\\ 
 & der truoc der minne \textbf{grôzen} last.\\ 
 & daz vuogte im snê unde bluot.\\ 
 & \textbf{ez} ist sünde, swer im \textbf{mêr nû} tuot.\\ 
 & ouch hât\textbf{s} diu minne kranken prîs:\\ 
30 & diu stiez ûf in ir krefte rîs.\\ 
\end{tabular}
\scriptsize
\line(1,0){75} \newline
D \newline
\line(1,0){75} \newline
\textbf{3} \textit{Initiale} D  \textbf{19} \textit{Majuskel} D  \textbf{23} \textit{Majuskel} D  \newline
\line(1,0){75} \newline
\textbf{3} Keie] Keye D \newline
\end{minipage}
\hspace{0.5cm}
\begin{minipage}[t]{0.5\linewidth}
\small
\begin{center}*m
\end{center}
\begin{tabular}{rl}
 & der \textbf{noch dort ûz} juste gert.\\ 
 & sîn lîp ist \textbf{ouch} wol prîses wert."\\ 
 & \begin{large}K\end{large}eie, der küene man,\\ 
 & brâht \textbf{\textit{da}z} mære vür den künic sân,\\ 
5 & Segramors wære gestochen abe\\ 
 & und dort ûze hielt ein strenger knabe,\\ 
 & der gerte juste \textbf{reh\textit{t}} als \textbf{ouch} ê.\\ 
 & er sprach: "\textbf{hêrre}, mir tuot wê,\\ 
 & so\textit{l} er\textbf{s} genozzen scheiden hin.\\ 
10 & ob ich iu sô \textbf{wirdic} bin,\\ 
 & lât mich versuochen, wes er ger,\\ 
 & sît er mit ûf gerihtem sper\\ 
 & dort \textbf{habt} vor iuwerm wîbe.\\ 
 & niemer ich blîbe\\ 
15 & in iuwerem dienste mêre;\\ 
 & tavelrunde hât unêre,\\ 
 & ob man\textbf{s} \textbf{im niht bî zîte} wert.\\ 
 & ûf unseren prîs sîn ellen zert.\\ 
 & \textbf{nû} gebet mir strîtes urloup.\\ 
20 & wæren wir \textbf{alle} blint oder toup,\\ 
 & \textbf{ir} sol\textit{t}z ime weren, \textbf{daz} wære zît."\\ 
 & Artus erloubete \textbf{Keien} strît.\\ 
 & gewâpent wart der schinischalt.\\ 
 & dô wolt er swenden den w\textit{a}lt\\ 
25 & mit juste ûf disen \textbf{komenden} gast.\\ 
 & der truoc der minne \textbf{grôzen} last.\\ 
 & daz vuogt ime snê und bluot.\\ 
 & \textbf{es} ist sünde, wer im \textbf{mê nû} tuot.\\ 
 & ouch hât diu minne kranken prîs:\\ 
30 & diu stie\textit{z} ûf in ir krefte rîs.\\ 
\end{tabular}
\scriptsize
\line(1,0){75} \newline
m n o Fr69 \newline
\line(1,0){75} \newline
\textbf{3} \textit{Initiale} m   $\cdot$ \textit{Capitulumzeichen} n  \newline
\line(1,0){75} \newline
\textbf{3} Keie] Keẏe n o \textbf{4} daz] es m  $\cdot$ vür den künic sân] vor der konigin dann o \textbf{5} wære] was n \textbf{7} gerte] gert n o  $\cdot$ reht] rechtes m  $\cdot$ als ouch ê] also E n alse o \textbf{9} sol] So m \textbf{11} er] [es]: er n \textbf{12} gerihtem] gerechtem o \textbf{13} habt] halt o \textbf{14} blîbe] do blibe n \textbf{15} iuwerem] irem m n (o) \textbf{16} tavelrunde] Tafelrunder o \textbf{17} mans] man n (o) \textbf{18} ellen] zellen n \textbf{19} gebet] gelt n  $\cdot$ mir] \textit{om.} n o \textbf{20} wir] wir abe o \textbf{21} soltz] solttencz m súllent n  $\cdot$ weren] were n  $\cdot$ daz] es n ez o  $\cdot$ wære] ist n \textbf{22} erloubete] erloup n (o)  $\cdot$ Keien] keẏen den n o \textbf{23} schinischalt] schiniscalt m sciniscalt n scinitscalt o \textbf{24} walt] wanlt m \textbf{26} minne] mẏnnen n o \textbf{27} vuogt] fugt m fúget n fuͯg o \textbf{28} mê nû] mere n nuͯ mer o \textbf{29} hât] hette n \textbf{30} stiez] stiesse m tiesse o  $\cdot$ ir krefte] crefften n (o)  $\cdot$ rîs] rich o \newline
\end{minipage}
\end{table}
\newpage
\begin{table}[ht]
\begin{minipage}[t]{0.5\linewidth}
\small
\begin{center}*G
\end{center}
\begin{tabular}{rl}
 & der \textbf{dort noch \textit{û}ze} tjoste gert.\\ 
 & sîn lîp ist \textbf{doch} wol brîses wert."\\ 
 & Kay, der küene man,\\ 
 & brâhte \textbf{\textit{da}z} \textit{mære} vür den künic sân,\\ 
5 & Segremors wære gestochen abe\\ 
 & \textit{und} dort ûze hielt ein strenger knabe,\\ 
 & der gert tjoste \textbf{reht} als ê.\\ 
 & er sprach: "\textbf{hêrre}, mir tuot wê,\\ 
 & sol er\textbf{s} genozzen scheiden hin.\\ 
10 & obe ich iu sô \textbf{wirdic} bin,\\ 
 & lât mich versuochen, wes er ger,\\ 
 & sît er mit ûf gerihtem sper\\ 
 & \begin{large}D\end{large}ort \textbf{halt} vor iuwerem wîbe.\\ 
 & nimer ich belîbe\\ 
15 & in iuwerem dienste mêre;\\ 
 & tavelrunder hât\textbf{s} unêre,\\ 
 & obe man\textbf{z} \textbf{im niht enzît} wert.\\ 
 & ûf unseren brîs sîn ellen zert.\\ 
 & gebt mir strîtes urloup.\\ 
20 & w\textit{æ}ren wir \textit{\textbf{alle}} blint oder toup,\\ 
 & \textbf{man} soltz im weren, \textbf{des} wære zît."\\ 
 & Artus erloubte \textbf{im} \textbf{den} strît.\\ 
 & gewâpent wart der seneschalt.\\ 
 & dô wolt er swenden den walt\\ 
25 & mit tjost ûf disen \textbf{küenen} gast.\\ 
 & der truoc der minne \textbf{swæren} last.\\ 
 & daz vuogte im snê unde bluot.\\ 
 & \textbf{es} ist sünde, swer im \textbf{nû mêre} tuot.\\ 
 & \multicolumn{1}{l}{ - - - }\\ 
30 & \multicolumn{1}{l}{ - - - }\\ 
\end{tabular}
\scriptsize
\line(1,0){75} \newline
G I O L M Q R Z Fr40 \newline
\line(1,0){75} \newline
\textbf{1} \textit{Initiale} I  \textbf{3} \textit{Initiale} L Z  \textbf{5} \textit{Initiale} O Q  \textbf{13} \textit{Initiale} G  \textbf{23} \textit{Initiale} I  \newline
\line(1,0){75} \newline
\textbf{1} \textit{Die Verse 288.15-293.2 fehlen} R   $\cdot$ \textit{Die Verse 290.1-2 fehlen} Q   $\cdot$ der dort noch devze tioste gert G  $\cdot$ Der noch dort (vzze Z ) tioste gert I (Z)  $\cdot$ Der tyost noch dort vsze gert L \textbf{2} doch] vil I \textbf{3} Kay] kaẏ G kain I Key O (Z) Keye M \textbf{4} brâhte daz mære] brahtez G Brachte die mere L \textbf{5} Segremors] ÷egremors O Saýgremors L Sigremors M \textbf{6} und] \textit{om.} G  $\cdot$ ûze] \textit{om.} I  $\cdot$ strenger] strenge M \textbf{7} gert] gerte L Q  $\cdot$ reht] \textit{om.} O L M Q  $\cdot$ als] alsam Q \textbf{8} er sprach hêrre] Herre sprach er O L (M) Q  $\cdot$ mir tuot] tut myr M mir tut immer Z \textbf{9} ers] er I L Q  $\cdot$ genozzen scheiden] ganzer Ganzer schæiden O scheiden genoszen L  $\cdot$ hin] sin L \textbf{11} lât] so lat I \textbf{13} halt] habt O \textbf{15} iuwerem] ewern I \textbf{16} tavelrunder] Tavelruͯnde L  $\cdot$ hâts] [ha*]: hats G habent sin I hat O L M Q Z Fr40 \textbf{17} manz im] man im I man imz O (L) (Z) man osz M  $\cdot$ enzît] bezite Z \textbf{18} sîn] er sin I  $\cdot$ ellen] eren Q \textbf{19} gebt] Nun gebt Q (Fr40) \textbf{20} wæren] waren G (L)  $\cdot$ alle] \textit{om.} G \textbf{21} soltz im] solde imz O (L) solde om M  $\cdot$ weren] sturen M  $\cdot$ des] daz L \textbf{22} erloubte] erlaubet I (O) (Q) (Fr40)  $\cdot$ im] keyn O Z Keýen L keien M kay Q Fr40  $\cdot$ den] \textit{om.} Z \textbf{23} der] de Z  $\cdot$ seneschalt] sinschalt G shinisshalt I sinetschalt O sinetshalt L tschenescalt M senechalt Q Sinehtschalt Z \textbf{24} dô] Da M Z  $\cdot$ walt] [gewalt]: walt O \textbf{25} disen] disem I den O Q Fr40  $\cdot$ küenen] chomenden O (L) (M) (Q) (Z) (Fr40) \textbf{26} minne] minnen I Z  $\cdot$ swæren] \textit{om.} Z \textbf{27} vuogte] vugt I (O) (L) (Z) (Fr40) \textbf{28} es] ez I (O) (L) (M) (Z) Fr40  $\cdot$ swer] wer L M Q Fr40  $\cdot$ im] immer Z  $\cdot$ nû mêre] iht mer I nv iht mer O mer nu Q Fr40 nv Z \newline
\end{minipage}
\hspace{0.5cm}
\begin{minipage}[t]{0.5\linewidth}
\small
\begin{center}*T
\end{center}
\begin{tabular}{rl}
 & der \textbf{dort ûze noch} tjoste gert.\\ 
 & sîn lîp ist \textbf{ouch} wol prîses wert."\\ 
 & \begin{large}K\end{large}ey, der küene man,\\ 
 & brâhte \textbf{daz} mære vür den künec sân,\\ 
5 & Segremors wære gestochen abe\\ 
 & unde dort ûze hielt ein strenger knabe,\\ 
 & der gerte tjoste als ê.\\ 
 & er sprach: "mir tuot \textbf{iemer} wê,\\ 
 & sol er\textbf{z} genozzen scheiden hin.\\ 
10 & ob ich iu \textbf{nû} sô \textbf{wert} bin,\\ 
 & lât mich versuochen, wes er ger,\\ 
 & sît er mit ûf gerihtem sper\\ 
 & dort \textbf{haltet} vor iuwerm wîbe.\\ 
 & niemer ich blîbe\\ 
15 & in iuwerm dienste mêre;\\ 
 & tavelrunder hât \textbf{e\textit{s}} unêre,\\ 
 & ob man\textbf{s} \textbf{enzît im niht} wert.\\ 
 & ûf unsern prîs sîn ellen zert.\\ 
 & gebt mir strîtes urloup.\\ 
20 & wære wi\textit{r b}lint oder toup,\\ 
 & \textbf{man} soltz im wern, \textbf{daz} wære zît."\\ 
 & Artus erloubete \textbf{Key} \textbf{den} strît.\\ 
 & Gewâpent wart der seneschalt.\\ 
 & dô wolter swenden den walt\\ 
25 & mit tjost ûf disen \textbf{komenden} gast.\\ 
 & der truoc der minne \textbf{swæren} last.\\ 
 & daz vuogtim snê unde bluot.\\ 
 & \textbf{ez} ist sünde, swer im \textbf{nû mêr} tuot.\\ 
 & \multicolumn{1}{l}{ - - - }\\ 
30 & \multicolumn{1}{l}{ - - - }\\ 
\end{tabular}
\scriptsize
\line(1,0){75} \newline
T U V W \newline
\line(1,0){75} \newline
\textbf{3} \textit{Initiale} T U W  \textbf{23} \textit{Majuskel} T  \newline
\line(1,0){75} \newline
\textbf{1} ûze] aussen W  $\cdot$ noch] nach U \textbf{3} Key] Kein V KEy W \textbf{4} brâhte] Der brahte V  $\cdot$ daz] dise U diz V \textbf{5} Segremors] Segremos U [S*]: Sagremors V  $\cdot$ wære] der were V was W \textbf{6} unde] \textit{om.} W  $\cdot$ ûze] aussen W  $\cdot$ hielt] hielte U \textbf{7} gerte] gert U V W  $\cdot$ tjoste] yostes recht U tiost reht V  $\cdot$ als ê] [also]: alsê T als sam ee W \textbf{8} iemer] [*]: herre V \textbf{9} sol erz] Solt er W  $\cdot$ scheiden] seheiden U \textbf{10} nû] \textit{om.} W  $\cdot$ wert] werder W \textbf{16} tavelrunder] Die tauelrunder W  $\cdot$ hât es] hatez T hantz V hat W \textbf{17} mans] man iz U man W  $\cdot$ enzît im niht] [*]: enzît im niht T in zit nit U im niht bi zite V ims nicht in zeite W \textbf{19} gebt] Gip U \textbf{20} wir blint] wir niht blint T wir alle blint W \textbf{21} man soltz im] Jr [soͤltent]: soͤlten ez im V Man solt im es W  $\cdot$ daz] dez V \textbf{22} Key] key T U [*]: keien V in W \textbf{23} der seneschalt] seneschalt U schinischalt W \textbf{24} den] gar den V \textbf{28} ist] \textit{om.} U  $\cdot$ swer] wer U W  $\cdot$ im nû] dem icht W \textbf{29} \textit{Die Verse 290.29-30 sind am Rand nachgetragen und später radiert:} och hatz die minne krencken pris / Di: stiez vf in ir :::e ris V  \newline
\end{minipage}
\end{table}
\end{document}
