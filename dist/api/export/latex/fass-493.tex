\documentclass[8pt,a4paper,notitlepage]{article}
\usepackage{fullpage}
\usepackage{ulem}
\usepackage{xltxtra}
\usepackage{datetime}
\renewcommand{\dateseparator}{.}
\dmyyyydate
\usepackage{fancyhdr}
\usepackage{ifthen}
\pagestyle{fancy}
\fancyhf{}
\renewcommand{\headrulewidth}{0pt}
\fancyfoot[L]{\ifthenelse{\value{page}=1}{\today, \currenttime{} Uhr}{}}
\begin{document}
\begin{table}[ht]
\begin{minipage}[t]{0.5\linewidth}
\small
\begin{center}*D
\end{center}
\begin{tabular}{rl}
\textbf{493} & \begin{large}S\end{large}aturnus loufet sô \textbf{hôhe} enbor,\\ 
 & daz ez diu wunde wesse vor,\\ 
 & ê der ander vrost \textbf{kome} her nâch.\\ 
 & dem snê was ninder als gâch,\\ 
5 & \textbf{er} viel alrêst \textbf{an} der andern naht\\ 
 & in der sumerlîchen maht.\\ 
 & dô mans küneges \textbf{vrost} sus werte,\\ 
 & die diet ez vreuden herte."\\ 
 & \textbf{dô} sprach der kiusche Trevrizent:\\ 
10 & "si \textbf{enpfiengen} jâmers soldiment;\\ 
 & daz sper in vreude enpfuorte,\\ 
 & daz ir \textbf{herzen} verch sus ruorte.\\ 
 & dô machte ir jâmers triwe\\ 
 & des toufes \textbf{lêre} al niwe."\\ 
15 & Parzival zem wirte sprach:\\ 
 & "vünf unt zweinzec meide \textbf{ich} \textbf{dâ} sach,\\ 
 & die vor dem \textbf{künege} stuonden\\ 
 & unt wol mit zühten kunden."\\ 
 & der wirt sprach: "\textbf{es} suln meide pflegen\\ 
20 & - des hât sich got gein \textbf{im} bewegen -,\\ 
 & des Grâles, \textbf{dem} si dâ \textbf{dienden} vür.\\ 
 & der Grâl ist mit hôher kür.\\ 
 & sô suln sîn rîter hüeten\\ 
 & mit kiuscheclîchen güeten.\\ 
25 & \textbf{der hôhen} sterne komendiu zît\\ 
 & der diet al dâ \textbf{grôz} \textbf{jâmer} gît,\\ 
 & den jungen unt den alten.\\ 
 & got hât zorn behalten\\ 
 & gein in al ze lange dâ.\\ 
30 & wenne suln \textbf{si} \textbf{vreude sprechen} 'jâ'?\\ 
\end{tabular}
\scriptsize
\line(1,0){75} \newline
D Fr11 Fr31 \newline
\line(1,0){75} \newline
\textbf{1} \textit{Initiale} D Fr11 Fr31  \newline
\line(1,0){75} \newline
\textbf{1} Saturnus] Sa:::nyvs Fr31 \textbf{2} diu] die Fr31  $\cdot$ wunde] wuͯndn Fr11 \textbf{3} ê] \textit{om.} Fr31  $\cdot$ kome] cham Fr11 (Fr31) \textbf{4} Der sne waz in der asse gach Fr31  $\cdot$ snê was] ist Fr11 \textbf{7} sus] so Fr31 \textbf{9} Trevrizent] Trevrizzent D Trefrezzent Fr11 trevrezent Fr31 \textbf{13} machte] macht Fr11 \textbf{14} toufes] ruͯfes Fr11 :::fes Fr31 \textbf{15} Parzival] Parcifal D Partzival Fr11 Parzifal Fr31 \textbf{19} es] ez Fr11 dez Fr31 \textbf{20} gein] von Fr31 \textbf{21} dâ] \textit{om.} Fr31 \textbf{25} hôhen] hoch Fr11  $\cdot$ sterne] sternen Fr31  $\cdot$ komendiu] chomen div D chom divͯ Fr11 \textbf{30} vreude sprechen jâ] fravͯ sprechen da Fr11 \newline
\end{minipage}
\hspace{0.5cm}
\begin{minipage}[t]{0.5\linewidth}
\small
\begin{center}*m
\end{center}
\begin{tabular}{rl}
 & Saturnus loufet sô \textbf{nâhe} enbor,\\ 
 & daz ez diu wunde wesse vor,\\ 
 & ê der ander vrost \textbf{kæme} her nâch.\\ 
 & dem snê was nindert alsô gâch,\\ 
5 & \textbf{er} viel alle\textit{r}êrst \textbf{an} der andern naht\\ 
 & in der sumerlîchen maht.\\ 
 & dô mans küniges \textbf{vrost} sus werte,\\ 
 & die d\textit{i}et ez vröuden herte."\\ 
 & \textbf{dô} sprach der kiusche Trevrizent:\\ 
10 & "si \textbf{enpfienc} jâmers soldiment;\\ 
 & daz sper in vröude enpfuorte,\\ 
 & daz ir \textbf{hêrren} verch sus ruorte.\\ 
 & \multicolumn{1}{l}{ - - - }\\ 
 & des toufes \textbf{lêre} alniuwe."\\ 
15 & Parcifal zuom wirte sprach:\\ 
 & "vünf und zweinzic megde \textbf{d\textit{â}} \dag saz\dag ,\\ 
 & die vor dem \textbf{künige} stuonden\\ 
 & und wol mit zühten kunden."\\ 
 & der wirt sprach: "\textbf{es} sullen megde pflegen\\ 
20 & - des het sich got gegen \textbf{in} bewegen -,\\ 
 & des Grâles, \textbf{den} si d\textit{â} \textbf{br\textit{â}hten} vür.\\ 
 & der Grâl ist mit hôher kür.\\ 
 & sô sullen sî\textit{n} ritter hüeten\\ 
 & mit kiuschlîchen güeten.\\ 
25 & \textbf{der hôhe\textit{n}} sternen k\textit{o}mendiu zît\\ 
 & der diet aldâ \textbf{jâmers} gît,\\ 
 & den jungen und den alten.\\ 
 & got het zorn behalten\\ 
 & gegen in al ze lange d\textit{â}.\\ 
30 & wan sullen \textbf{si} \textbf{vröude sprechen} 'jâ'?\\ 
\end{tabular}
\scriptsize
\line(1,0){75} \newline
m n o \newline
\line(1,0){75} \newline
\newline
\line(1,0){75} \newline
\textbf{1} Saturnus] Satturnuͯs o \textbf{5} allerêrst] [anerer]: allerer erst m  $\cdot$ andern] ander o \textbf{7} mans] man o  $\cdot$ vrost] [ros]: frost m \textbf{8} diet] tet m (o) \textbf{9} Trevrizent] [tre*]: trevrizent m trefrizent n treurizent o \textbf{10} enpfienc] enpfingent n o \textbf{11} enpfuorte] enpfunde o \textbf{13} \textit{Vers 493.13 fehlt} m n o  \textbf{14} \textit{Vers 493.14 fehlt} n   $\cdot$ des] Das o \textbf{15} sprach] sprach das n \textbf{16} dâ] do m n o \textbf{17} dem] \textit{om.} o  $\cdot$ stuonden] [suͯnden]: stuͯnden o \textbf{20} in] yme o \textbf{21} den] duͯn o  $\cdot$ dâ] do m n \textit{om.} o  $\cdot$ brâhten] brechtten m \textbf{22} Grâl] grole n \textbf{23} sô] Sie o  $\cdot$ sîn] sẏ m (o) \textbf{25} hôhen] hohenden m  $\cdot$ komendiu] kemendie m (n) kemende o \textbf{26} jâmers] grosses jomer n grosz jamer o \textbf{29} dâ] do m n o \newline
\end{minipage}
\end{table}
\newpage
\begin{table}[ht]
\begin{minipage}[t]{0.5\linewidth}
\small
\begin{center}*G
\end{center}
\begin{tabular}{rl}
 & \begin{large}S\end{large}aturnus loufet sô \textbf{hôhe} enbor,\\ 
 & daz ez diu wunde wesse vor,\\ 
 & ê der ander vrost \textbf{kom} her nâch.\\ 
 & dem snê was ninder als gâch,\\ 
5 & \textbf{er} viel alrêrst \textbf{an} der andern naht\\ 
 & in der sumerlîchen maht.\\ 
 & dô mans küniges \textbf{vrost} sus werte,\\ 
 & die diet ez vröuden herte."\\ 
 & \textbf{dô} sprach der kiusche Trevrizzent:\\ 
10 & "si \textbf{enpfiengen} \textit{jâmers} soldiment;\\ 
 & daz sper in vröude enpfuorte,\\ 
 & daz ir \textbf{herzen} verch sus ruorte.\\ 
 & dô machete ir jâmers triuwe\\ 
 & des toufes \textbf{lêre} al niuwe."\\ 
15 & Parzival ze dem wirte sprach:\\ 
 & "vünf unde zweinzic meide \textbf{ich} \textbf{dâ} sach,\\ 
 & die vor dem \textbf{künige} stuonden\\ 
 & unde wol mit zühten kunden."\\ 
 & der wirt sprach: "\textbf{ez} suln meide pflegen\\ 
20 & - des hât sich got gein \textbf{in} bewegen -\\ 
 & des Grâles, \textbf{dem} si dâ \textbf{dienden} vür.\\ 
 & der Grâl ist mit hôher kür.\\ 
 & sô suln sîn rîter hüeten\\ 
 & mit kiuschlîchen güeten.\\ 
25 & \textbf{der hôhen} sterne komendiu zît\\ 
 & der diet al dâ \textbf{grôz} \textbf{jâmer} gît,\\ 
 & den jungen unde den alten.\\ 
 & got hât zorn behalten\\ 
 & gein in al ze lange dâ.\\ 
30 & wenne suln \textbf{si} \textbf{vröude sprechen} 'jâ'?\\ 
\end{tabular}
\scriptsize
\line(1,0){75} \newline
G I L M Z Fr49 \newline
\line(1,0){75} \newline
\textbf{1} \textit{Initiale} G I L Z  \textbf{15} \textit{Initiale} I  \newline
\line(1,0){75} \newline
\textbf{1} Saturnus] Satvrnuͯs L \textbf{2} ez] \textit{om.} M Z \textbf{3} her nâch] darnoch M \textbf{4} ninder] nirgen M  $\cdot$ als] so I Z Fr49 \textbf{5} viel] vil M \textbf{7} dô] Da M Z  $\cdot$ sus] so Z \textbf{8} ez] ist M  $\cdot$ vröuden] vreude I (Fr49) \textbf{9} dô] Da M  $\cdot$ Trevrizzent] trefrizent G Treuerezent I Trevriscent L trefrescent M Trevenzzent Z Trêuitzent Fr49 \textbf{10} jâmers] \textit{om.} G \textbf{11} vröude] vrouden M \textbf{12} daz] Da Z  $\cdot$ herzen] hertze Z \textbf{13} dô] Da M Z  $\cdot$ machete] macht I (L) (Z) Fr49 \textbf{15} Parzival] Parziual G Parzifal I M Parcifal Z Fr49 \textbf{16} vünf unde zweinzic] Funffvnndzcweczig M  $\cdot$ dâ] \textit{om.} Fr49 \textbf{19} ez] ez sprach ez Fr49 \textbf{20} got] \textit{om.} Fr49  $\cdot$ in] im I L (M) Fr49 \textbf{21} dem] den M  $\cdot$ vür] fuͯre L \textbf{23} sîn] Sine M \textbf{24} kiuschlîchen güeten] chuschlicher guͦte I \textbf{25} hôhen sterne] hohe stern M \textbf{26} al dâ grôz] da groszen L algrosz M \textbf{29} al] allen I \newline
\end{minipage}
\hspace{0.5cm}
\begin{minipage}[t]{0.5\linewidth}
\small
\begin{center}*T
\end{center}
\begin{tabular}{rl}
 & \textit{\begin{large}S\end{large}}aturnus loufet sô \textbf{hôhe} enbor,\\ 
 & daz ez d\textit{iu} wunde wiste vor,\\ 
 & ê der ander vrost \textbf{kæme} her nâch.\\ 
 & dem snê was niender alse gâch,\\ 
5 & \textbf{der} viel alrêst der andern naht\\ 
 & in der sumerlîchen maht.\\ 
 & dô man des küneges \textbf{vroste} sus werte,\\ 
 & die diet ez vröuden herte",\\ 
 & sprach der kiusche Trefrizent,\\ 
10 & "si \textbf{enpfiengen} jâmers soldiment;\\ 
 & daz sper in vröude enpfuorte,\\ 
 & daz ir \textbf{herzen} verch sus \textit{r}uorte.\\ 
 & dô mahte ir jâmers triuwe\\ 
 & des toufes \textbf{êre} alniuwe."\\ 
15 & Parcifal zem wirte sprach:\\ 
 & "vünf unde zwêncic megede \textbf{ich} sach,\\ 
 & die vor dem \textbf{wirte} stuonden\\ 
 & unde \textbf{ez} wol mit zühten kunden."\\ 
 & Der wirt sprach: "\textbf{sîn} suln megde pflegen\\ 
20 & - des hât sich got gegen \textbf{in} bewegen -,\\ 
 & des Grâles, \textbf{dem} si dâ \textbf{dienten} vür.\\ 
 & der Grâl ist mit \textbf{sô} hôher kür.\\ 
 & sô suln sîn rîter hüeten\\ 
 & mit kiuscheclîchen güeten.\\ 
25 & \textbf{etslîcher} sternen komendiu zît\\ 
 & der diet aldâ \textbf{grôz} \textbf{jâmer} gît,\\ 
 & den jungen unde den alten.\\ 
 & got hât \textbf{noch} zorn behalten\\ 
 & gegen in alze lange dâ.\\ 
30 & wenne suln \textbf{wir} \textbf{sprechen vröude} 'jâ'?\\ 
\end{tabular}
\scriptsize
\line(1,0){75} \newline
T U V W O Q R Fr40 \newline
\line(1,0){75} \newline
\textbf{1} \textit{Initiale} T V O Q Fr40  \textbf{15} \textit{Initiale} W  \textbf{19} \textit{Majuskel} T  \newline
\line(1,0){75} \newline
\textbf{1} \textit{Die Verse 453.1-502.30 fehlen} U   $\cdot$ Saturnus] KAtvrnvs T ÷atvrnvs O  $\cdot$ loufet] heűfet Q \textbf{2} diu] die T  $\cdot$ wunde] wunden R Fr40 \textbf{3} kæme] kam Q  $\cdot$ her] dar V R er Q \textbf{4} alse] so Q  $\cdot$ gâch] iach Q \textbf{5} der] [E*]: Er V  $\cdot$ alrêst] erst W R  $\cdot$ der] [*]: an der V an der W (O) (R) Fr40 in die Q \textbf{7} vroste] frost W O Q R (Fr40)  $\cdot$ sus] ausz Q \textbf{9} \textit{Die Verse 493.9-494.18 fehlen} R   $\cdot$ sprach] [*]: Do sprach V  $\cdot$ Trefrizent] trefizent V trefrisent W \textbf{12} ir] [*]: irz V (W) (Q)  $\cdot$ herzen] herren V W O Fr40  $\cdot$ sus] es Q  $\cdot$ ruorte] [*inte]: vuͦrte T \textbf{13} mahte] maht V (O) (Q) Fr40  $\cdot$ ir] in Q Fr40  $\cdot$ triuwe] newe Q \textbf{14} toufes] tewfels Q  $\cdot$ êre] [*]: ler V sere W lere O Q :ere Fr40 \textbf{15} Parcifal] Parzifal V Fr40 PArtzifal W (Q) \textbf{16} sach] [*]: do sach V da sach O Fr40 do sach Q \textbf{18} ez] \textit{om.} V W O Q Fr40 \textbf{19} sîn] \textit{om.} W \textbf{20} bewegen] gewegen Q \textbf{21} dem] [dem]: den V  $\cdot$ dienten] [*]: brohtent V dienen O (Q) (Fr40) \textbf{22} mit sô] so mit W Q \textbf{25} sternen] sterne W O Q sterne: Fr40  $\cdot$ komendiu] chomende O \textbf{26} diet] die Q  $\cdot$ aldâ] da O \textbf{29} alze] also W  $\cdot$ dâ] do V W [do]: da Q \textbf{30} wir] sv́ V (W) (O) (Q) (Fr40)  $\cdot$ sprechen vröude] sprechen mit vroͤiden V frevde sprechen O (Fr40) die frewe sprechen Q \newline
\end{minipage}
\end{table}
\end{document}
