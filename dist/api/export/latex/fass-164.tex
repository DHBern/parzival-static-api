\documentclass[8pt,a4paper,notitlepage]{article}
\usepackage{fullpage}
\usepackage{ulem}
\usepackage{xltxtra}
\usepackage{datetime}
\renewcommand{\dateseparator}{.}
\dmyyyydate
\usepackage{fancyhdr}
\usepackage{ifthen}
\pagestyle{fancy}
\fancyhf{}
\renewcommand{\headrulewidth}{0pt}
\fancyfoot[L]{\ifthenelse{\value{page}=1}{\today, \currenttime{} Uhr}{}}
\begin{document}
\begin{table}[ht]
\begin{minipage}[t]{0.5\linewidth}
\small
\begin{center}*D
\end{center}
\begin{tabular}{rl}
\textbf{164} & in eine kemenâten.\\ 
 & si begunden \textbf{im} alle râten:\\ 
 & "lât den harnasch von iu bringen\\ 
 & unt iweren liden ringen."\\ 
5 & \textbf{\begin{large}S\end{large}chiere er muose entwâpent sîn}.\\ 
 & \textbf{dô} si diu rûhen ribbalîn\\ 
 & \textbf{unt diu tôren kleit gesâhen},\\ 
 & \textbf{dô} erschrâken \textbf{die sîn pflâgen}.\\ 
 & vil \textbf{blûwec} ez wart ze hove gesagt.\\ 
10 & \textbf{der wirt} vor schame was \textbf{nâch} verzagt.\\ 
 & Ein ritter sprach durch sîne zuht:\\ 
 & "deiswâr, sô werdeclîche vruht\\ 
 & erkôs nie mîner ougen sehe.\\ 
 & an im lît der sælden spehe\\ 
15 & \textbf{mit} \textbf{reiner}, \textbf{süezen, hôhen} art.\\ 
 & wie ist der minnen blic \textbf{alsus} bewart?\\ 
 & mich jâmert immer, daz ich vant\\ 
 & an der \textbf{werlde} vröude \textbf{al}sölh gewant.\\ 
 & wol doch der muoter, diu in truoc,\\ 
20 & an \textbf{dem} \textbf{des wunsches lît} genuoc.\\ 
 & Sîn zimierde ist rîche,\\ 
 & \textbf{daz} harnasch stuont ritterlîche,\\ 
 & \textbf{ê ez} \textbf{kœme} von dem gehiuren.\\ 
 & von einer quaschiuren\\ 
25 & bluotige amesiere\\ 
 & kôs ich an im schiere."\\ 
 & \textbf{Der wirt} sprach z\textbf{em} ritter sân:\\ 
 & "\textbf{daz} ist durch \textbf{wîbe} gebôt getân."\\ 
 & "nein, hêrre, \textbf{er} ist \textbf{mit} sölhen siten,\\ 
30 & er\textbf{n} kunde nimer wîp gebiten,\\ 
\end{tabular}
\scriptsize
\line(1,0){75} \newline
D \newline
\line(1,0){75} \newline
\textbf{5} \textit{Initiale} D  \textbf{11} \textit{Majuskel} D  \textbf{21} \textit{Majuskel} D  \textbf{27} \textit{Majuskel} D  \newline
\line(1,0){75} \newline
\newline
\end{minipage}
\hspace{0.5cm}
\begin{minipage}[t]{0.5\linewidth}
\small
\begin{center}*m
\end{center}
\begin{tabular}{rl}
 & in eine kemenâten.\\ 
 & si begunden alle râten:\\ 
 & "lât daz harnasch von iu bringen\\ 
 & und iuweren liden ringen."\\ 
5 & \textbf{mit der rede entwâpent\textit{en si} in}.\\ 
 & \textbf{nû} si diu rûhen r\textit{i}bbalîn\\ 
 & \textbf{und diu tôren kleit ersâhen},\\ 
 & \textbf{si} erschrâken, \textbf{die sîn pflâgen}.\\ 
 & vil \textbf{schier} e\textit{z} war\textit{t} ze hove gesaget.\\ 
10 & \textbf{der wirt} vor schame was verzaget.\\ 
 & \begin{large}E\end{large}in ritter sprach durch sîne zuht:\\ 
 & "dês wâr, sô werdeclîch vruht\\ 
 & erkôs nie mîner \dag ouge spehen\dag .\\ 
 & an ime lît der sælden spehe\\ 
15 & \textbf{mit} \textbf{reiner}, \textbf{hôher, süezer} art.\\ 
 & wie ist de\textit{r} minnen blic \textbf{alsus} bewart?\\ 
 & mich jâmert iemer, daz ich vant\\ 
 & an der \textbf{werden} vröude solich gewant.\\ 
 & wol doch der muoter, diu in truoc,\\ 
20 & an \textbf{dem} \textbf{des wunsches lît} genuoc.\\ 
 & sîn zimierde ist rîche,\\ 
 & \textbf{daz} harnasch stuont ritterlîche,\\ 
 & \textbf{ê ez} \textbf{kœme} von dem gehiuren.\\ 
 & von einer \textit{qu}a\textit{sch}iure\textit{n}\\ 
25 & \textit{bluotic am}e\textit{siere}\\ 
 & k\textit{ô}s ich an im schiere."\\ 
 & \textbf{der wir\textit{t}} sprach zuo\textbf{m} ritter sân:\\ 
 & "\textbf{daz} ist durch \textbf{wîb\textit{e}} gebôt getân."\\ 
 & "nein, hêrre, \textbf{ez} ist \textbf{mit} solichen siten,\\ 
30 & er \textbf{en}kunde niemer wîp gebiten,\\ 
\end{tabular}
\scriptsize
\line(1,0){75} \newline
m n o \newline
\line(1,0){75} \newline
\textbf{11} \textit{Initiale} m   $\cdot$ \textit{Capitulumzeichen} n  \newline
\line(1,0){75} \newline
\textbf{2} begunden] begúnde o \textbf{4} iuweren] iren m uwer n (o)  $\cdot$ liden ringen] liechte lide ringen n liehte [ringen lide]: lide ringen o \textbf{5} entwâpenten si] entwappent es m enwoppent su n \textbf{6} ribbalîn] rabbalin m \textbf{7} ersâhen] gesohen n o \textbf{9} schier ez wart] schierest war \textit{(krit. Text emendiert nach V#'* ͫ)} m schiere wart n (o) \textbf{10} vor] von n an o  $\cdot$ was] was noch n was noher o \textbf{13} mîner ouge spehen] min ouge me n mynn auͯge nie me o \textbf{15} hôher süezer] hohen suͯssen n (o) \textbf{16} der] den m  $\cdot$ minnen] mynnenclichen n  $\cdot$ alsus] \textit{om.} n o \textbf{18} vröude] frouwen n (o) \textbf{22} stuont] stat jme n (o) \textbf{23} kœme] kome m kume n (o) \textbf{24} quaschiuren] scoatvͯre m \textbf{25} \textit{Vers 164.25 fehlt} m   $\cdot$ amesiere] amiser o \textbf{26} kôs] Kas m \textbf{27} wirt] wir m \textbf{28} wîbe] wibot m \textbf{29} solichen] sollichem n (o) \newline
\end{minipage}
\end{table}
\newpage
\begin{table}[ht]
\begin{minipage}[t]{0.5\linewidth}
\small
\begin{center}*G
\end{center}
\begin{tabular}{rl}
 & in eine kemenâten.\\ 
 & si begunden \textbf{im} alle râten:\\ 
 & "lât daz harnasch von iu bringen\\ 
 & unde iweren liden ringen."\\ 
5 & \textbf{vil schierer muose entwâpent sîn}.\\ 
 & \textbf{dô} si diu rûhen ribalîn\\ 
 & \textbf{\begin{large}U\end{large}nde diu tôren kleit ersâhen},\\ 
 & \textbf{si} erschrâken, \textbf{die sîn pflâgen}.\\ 
 & vil \textbf{blûge}z wart ze hove gesaget.\\ 
10 & \textbf{der wirt} vor schame was \textbf{nâch} verzaget.\\ 
 & ein rîter sprach durch sîne zuht:\\ 
 & "dês wâr, sô werdiclîche vruht\\ 
 & erkôs nie mîner ougen sehe.\\ 
 & an im lît der sælden spehe\\ 
15 & \textbf{von} \textbf{reine}, \textbf{süeze, hôher} art.\\ 
 & wie ist der minnen blic \textbf{alsus} bewart?\\ 
 & mich jâmert imer, daz ich vant\\ 
 & an der \textbf{werlde} vröude \textit{s}olch gewant.\\ 
 & wol doch der muoter, diu in truoc.\\ 
20 & an \textbf{im} \textbf{des wunsches lît} genuoc.\\ 
 & sîn zimierde ist rîche,\\ 
 & \textbf{daz} harnasch stuont rîterlîche;\\ 
 & \textbf{ez} \textbf{kom} von dem gehiuren.\\ 
 & von einer quatschiuren\\ 
25 & bluotic amesier\\ 
 & kôs ich an im schier."\\ 
 & \textbf{der wirt} sprach z\textbf{em} rîter sân:\\ 
 & "\textbf{ez} ist \textbf{lîhte} durch \textbf{wîbe} gebôt getân."\\ 
 & "nein, hêrre, \textbf{er}st \textbf{in} solhen siten,\\ 
30 & er kunde nimer wîp gebiten,\\ 
\end{tabular}
\scriptsize
\line(1,0){75} \newline
G I O L M Q R Z Fr17 Fr47 \newline
\line(1,0){75} \newline
\textbf{5} \textit{Initiale} I O L R Z Fr17 Fr47  \textbf{7} \textit{Initiale} G  \textbf{15} \textit{Initiale} M  \textbf{25} \textit{Initiale} O  \textbf{27} \textit{Initiale} I  \newline
\line(1,0){75} \newline
\textbf{4} iweren liden] úwer lider R Jvren glidern Fr47 \textbf{5} vil] ÷lsch O \textit{om.} L ÷Jl Fr47  $\cdot$ schierer muose] schir er musz Q schier muͦst er R (Fr47) \textbf{6} dô] Da M Z  $\cdot$ rûhen] rvͦhe O rechtin M \textbf{7} tôren] roten I Fr17 \textbf{8} si] Die M  $\cdot$ erschrâken] er schrachten O (Z) \textbf{9} blûgez] balde ez O (Q) blodeclich ez M  $\cdot$ hove] houwen M \textbf{10} vor] \textit{om.} M von Q  $\cdot$ was nâch] nach was I O waz R \textbf{12} dês wâr] Entzwar Q Zwar Z  $\cdot$ werdiclîche] Ritterliche R \textbf{13} erkôs] Chos O (R) Kons Q  $\cdot$ nie] nie mer R \textbf{14} lît] [j*]: lieht Q  $\cdot$ sælden] selbe R \textbf{15} von] Mit O L M Q R Z  $\cdot$ reine süeze] reiner svͦzen O (L) (Z) reiner suszer M (Q) (R)  $\cdot$ hôher] \textit{om.} O hohen L M Z \textbf{16} wie] Wir M  $\cdot$ minnen blic] minne blic I (L) (R) minnechlich O  $\cdot$ alsus] \textit{om.} I Fr17 svs O (L) also Q R  $\cdot$ bewart] gewart Q \textbf{17} imer] myner M \textbf{18} vröude] voͮde Fr17  $\cdot$ solch] alsolch G \textbf{19} doch] [ovch]: doch G \textit{om.} R Z \textbf{20} im] dem Z  $\cdot$ des wunsches lît] doch lit wunsches I lit des wunsches R \textbf{21} sîn] Sy R  $\cdot$ zimierde] zimier daz L zunirde Q geczimber R \textbf{22} daz] Sin L (Q)  $\cdot$ stuont] stuͤnt auch I stat R \textbf{23} ez] E ez Z  $\cdot$ kom] choͤm I (Z) chome Fr17  $\cdot$ dem] den L  $\cdot$ gehiuren] gehiűrem Q \textbf{24} quatschiuren] quascure I quatschivrem Q \textbf{25} bluotic] ÷lvtich O Blutet M \textbf{26} an] han M \textbf{27} zem rîter] zcu den rittern M \textbf{28} ez] Daz Z  $\cdot$ lîhte] \textit{om.} L Z lich R  $\cdot$ wîbe] wibes L M Z  $\cdot$ gebôt] bot R bet Z \textbf{29} hêrre] \textit{om.} O  $\cdot$ erst] es ist Q  $\cdot$ in] ninder in I mit O M Q Z von L nit R  $\cdot$ solhen] soͯlicher R \textbf{30} er] ern I (L) (M) (R)  $\cdot$ nimer] myner M \newline
\end{minipage}
\hspace{0.5cm}
\begin{minipage}[t]{0.5\linewidth}
\small
\begin{center}*T
\end{center}
\begin{tabular}{rl}
 & in eine kemenâten.\\ 
 & Si begunden \textbf{im} alle râten:\\ 
 & "lât den harnasch von iu bringen\\ 
 & unde iuwern liden ringen."\\ 
5 & \textbf{vil schiere muoser entwâpent sîn}.\\ 
 & \textbf{dô} si di\textit{u} rûhen ribbalîn\\ 
 & \textbf{ersâhen unde diu tôren kleit},\\ 
 & \textbf{si} erschrâken, \textbf{unde was in leit}.\\ 
 & vil \textbf{bl\textit{œ}de} ez wart ze hove gesaget.\\ 
10 & vor schame was \textbf{nâch} verzaget\\ 
 & ein rîter, \textbf{der} sprach durch sîne zuht:\\ 
 & "deiswâr, sô werdeclîche vruht\\ 
 & erkôs nie mîner ougen sehe.\\ 
 & an im lît der sælden spehe\\ 
15 & \textbf{mit} \textbf{reiner}, \textbf{süezen, hôhen} art.\\ 
 & wie ist der minnen blic bewart?\\ 
 & mich jâmert iemer, daz ich vant\\ 
 & an der \textbf{werlte} vröude sölch gewant.\\ 
 & Wol doch der muoter, diu in truoc.\\ 
20 & an \textbf{im} \textbf{liget wunsches} genuoc.\\ 
 & sîn zimier ist rîche,\\ 
 & \textbf{sîn} harnasch stuont rîterlîche.\\ 
 & \textbf{Ê ich} \textbf{kæme} von dem gehiuren,\\ 
 & von einer quaschiuren\\ 
25 & bluotic amesiere\\ 
 & kôs ich an im schiere."\\ 
 & \textbf{Dô} sprach zuo \textbf{zim ein} rîter sân:\\ 
 & "\textbf{ez} ist \textbf{lîhte} durch \textbf{wîbes} gebôt getân."\\ 
 & "Nein, hêrre, \textbf{er} ist \textbf{mit} solhen siten,\\ 
30 & er\textbf{n} kunde niemer wîp gebiten,\\ 
\end{tabular}
\scriptsize
\line(1,0){75} \newline
T U V W \newline
\line(1,0){75} \newline
\textbf{2} \textit{Majuskel} T  \textbf{17} \textit{Initiale} W  \textbf{19} \textit{Majuskel} T  \textbf{23} \textit{Majuskel} T  \textbf{27} \textit{Majuskel} T  \textbf{29} \textit{Majuskel} T  \newline
\line(1,0){75} \newline
\textbf{5} muoser] er muͦze U er mvͦste V \textbf{6} diu] die T \textbf{7} diu] der U V \textbf{9} blœde ez] blvde ez T bluͦde er iz U [*]: schier ez V blúge echt W \textbf{10} [*]: Der wurt vor scham waz noch verzaget V · Der wirt schampt sich vor zaghait W  $\cdot$ verzaget] verschamet U \textbf{11} der] do V \textit{om.} W \textbf{12} werdeclîche] minnigliche W \textbf{15} süezen hôhen] suͦzer hoher U (W) svͤzen hoher V \textbf{16} bewart] [*]: alsus bewart V also bewart W \textbf{17} iemer] ymter W \textbf{18} sölch gewant] selig gewant U also bewand W \textbf{19} in] dich W \textbf{20} liget wunsches] wunsches liget W \textbf{22} sîn] [S*]: Daz V  $\cdot$ stuont] [*]: stvͦnt V stat W \textbf{23} Ê ich kæme] Eich keine U E [*]: ez kam V Ee er keme W \textbf{27} [D*]: Der wurt sprach zvome ritter san V  $\cdot$ Er sprach zuͦ dem ritter san W  $\cdot$ zim] im U \textbf{28} durch wîbes] duͦrch wibe U (V) vmb weibes W \textbf{30} ern] Er W \newline
\end{minipage}
\end{table}
\end{document}
