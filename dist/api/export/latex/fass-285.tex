\documentclass[8pt,a4paper,notitlepage]{article}
\usepackage{fullpage}
\usepackage{ulem}
\usepackage{xltxtra}
\usepackage{datetime}
\renewcommand{\dateseparator}{.}
\dmyyyydate
\usepackage{fancyhdr}
\usepackage{ifthen}
\pagestyle{fancy}
\fancyhf{}
\renewcommand{\headrulewidth}{0pt}
\fancyfoot[L]{\ifthenelse{\value{page}=1}{\today, \currenttime{} Uhr}{}}
\begin{document}
\begin{table}[ht]
\begin{minipage}[t]{0.5\linewidth}
\small
\begin{center}*D
\end{center}
\begin{tabular}{rl}
\textbf{285} & \begin{large}B\end{large}eide lief unde spranc\\ 
 & Segramors, der \textbf{ie nâch strîte} ranc.\\ 
 & swâ der vehten wânde vinden,\\ 
 & dâ muose man in binden,\\ 
5 & oder er wolde dâr mite sîn.\\ 
 & \textbf{ninder} ist sô breit der Rîn,\\ 
 & sæher strîten am andern stade,\\ 
 & dâ würde wênec nâch dem bade\\ 
 & \textbf{getastet}, ez wære warm oder kalt:\\ 
10 & \textbf{er} viel \textbf{sus} dran, der \textbf{degen} balt.\\ 
 & \textbf{Snellîche} kom der jungelinc\\ 
 & ze hove an Artuses rinc.\\ 
 & der \textbf{werde} künec vaste slief.\\ 
 & Segramors \textbf{im} durch die snüere lief,\\ 
15 & zer poulûnes tür drang er în.\\ 
 & ein declachen zobelîn\\ 
 & zucter ab \textbf{in}, \textbf{diu} lâgen\\ 
 & unt süezes \textbf{slâfes} pflâgen,\\ 
 & sô daz si muosen wachen\\ 
20 & unt sîner \textbf{unvuoge} lachen.\\ 
 & Dô sprach er zuo der niftel sîn:\\ 
 & "Gynover, vrouwe künegîn,\\ 
 & unser sippe ist \textbf{des} \textbf{bekant}\\ 
 & - man weiz wol über \textbf{mengiu} lant -,\\ 
25 & daz ich genâden wart an dich.\\ 
 & nû hilf mir, vrouwe, und sprich\\ 
 & gein Artuse, dînem man,\\ 
 & daz ich von im müeze hân\\ 
 & - ein âventiure ist hie bî -,\\ 
30 & daz ich zer tjost der êrste sî."\\ 
\end{tabular}
\scriptsize
\line(1,0){75} \newline
D \newline
\line(1,0){75} \newline
\textbf{1} \textit{Initiale} D  \textbf{11} \textit{Majuskel} D  \textbf{21} \textit{Majuskel} D  \newline
\line(1,0){75} \newline
\textbf{12} Artuses] Artvss D \newline
\end{minipage}
\hspace{0.5cm}
\begin{minipage}[t]{0.5\linewidth}
\small
\begin{center}*m
\end{center}
\begin{tabular}{rl}
 & beide lief und spranc\\ 
 & Segramors, der \textbf{ie nâch strîte} ranc.\\ 
 & wâ der vehten wânde vinden,\\ 
 & d\textit{â} muose man in binden,\\ 
5 & oder er wolte dâr mite sîn.\\ 
 & \textbf{niender} ist sô breit der R\textit{î}n,\\ 
 & \textit{s}æhe er strîten an dem andern s\textit{t}ade,\\ 
 & dâ würde wênic nâch dem bade\\ 
 & \textbf{getast\textit{e}t}, ez wære warm oder kalt:\\ 
10 & \textbf{er} viel \textbf{sus} dran, der \textbf{degen} balt.\\ 
 & \textbf{snelleclîch} kam der jungelinc\\ 
 & ze hove an Artuses rinc.\\ 
 & der \textbf{werde} künic vaste slief.\\ 
 & Segramors \textit{\textbf{ime}} durch die snüere lief,\\ 
15 & zer pavel\textit{ûn}e tür dranc er în.\\ 
 & ein deckelachen zobelîn\\ 
 & zuckete er ab \textbf{den}, \textbf{die dâ} lâgen\\ 
 & und süezes \textbf{slâfens} pflâgen,\\ 
 & sô daz si muose\textit{n} \textit{w}achen\\ 
20 & und sîner \textbf{ungevüege} lachen.\\ 
 & dô sprach er zuo der nifteln sîn:\\ 
 & "Ginov\textit{e}r, vrouwe künigîn,\\ 
 & unser sippe ist \textbf{alsô} \textbf{bekant}\\ 
 & - man weiz wol über \textbf{manigiu} lant -,\\ 
25 & daz ich gnâden warte an dich.\\ 
 & nû hilf mir, vrouwe, und spr\textit{i}ch\\ 
 & gegen Artuse, dînem man,\\ 
 & daz ich von ime müez\textit{e} hân\\ 
 & - ein âventiure ist hie bî -,\\ 
30 & daz ich zer just der êrste sî."\\ 
\end{tabular}
\scriptsize
\line(1,0){75} \newline
m n o Fr8 \newline
\line(1,0){75} \newline
\textbf{11} \textit{Versal} Fr8  \textbf{21} \textit{Versal} Fr8  \newline
\line(1,0){75} \newline
\textbf{1} beide] Wender Fr8 \textbf{2} Segramors] Segramurs n Segramures o Segremuͦrs Fr8  $\cdot$ ie] \textit{om.} Fr8 \textbf{3} wâ] Swa Fr8  $\cdot$ wânde vinden] wunden snẏden n \textbf{4} dâ] Do m n o  $\cdot$ muose] musse m muͯste n o \textbf{6} \textit{nach 285.6:} So ne mocht ers nicht erbiten / Ob er da sach striten Fr8   $\cdot$ ist sô breit] so breit ist n Fr8  $\cdot$ Rîn] rein m rẏn Fr8 \textbf{7} sæhe er strîten] Zehe er stritten m Ander sit Fr8  $\cdot$ stade] schade m \textbf{8} dâ] So n Do o \textbf{9} getastet] getastat m Gestattet o Gedacht Fr8  $\cdot$ ez] er o \textbf{10} viel] wolte Fr8  $\cdot$ sus] \textit{om.} n o Fr8  $\cdot$ dran] da hin Fr8 \textbf{12} Artuses] artusses m Arthuses Fr8 \textbf{14} Segramors] Segremursz n Geragumurs o Segremors Fr8  $\cdot$ ime] sin m in n o  $\cdot$ snüere] snur o \textbf{15} zer pavelûne tür] Zuͦ pauelún tor o \textbf{17} zuckete] Zucket n o  $\cdot$ den] in n o Fr8  $\cdot$ die] \textit{om.} o  $\cdot$ dâ] do n o \textbf{18} süezes] sussen o  $\cdot$ slâfens] sloffes n (Fr8) sloffen o \textbf{19} sô] Do o  $\cdot$ muosen wachen] mussen lachen vnd wachen m muͯsten wachen n muͦsen lachen Fr8 \textbf{20} Von siner vnuoge wachen Fr8 \textbf{21} nifteln] nẏfftelin n (Fr8) [niftel]: nifteln  o \textbf{22} Ginover] Ginovar m Genofer n o Gẏnouer Fr8 \textbf{23} alsô] des Fr8 \textbf{24} manigiu] manig n o (Fr8) \textbf{25} daz] Do o  $\cdot$ gnâden] genode n (o) (Fr8) \textbf{26} sprich] sprach m \textbf{27} Artuse] Arthuse Fr8  $\cdot$ dînem] sinem n dinen Fr8 \textbf{28} müeze] muͯssen m \textbf{30} zer] [der]: zer Fr8  $\cdot$ êrste] selbe erste n \newline
\end{minipage}
\end{table}
\newpage
\begin{table}[ht]
\begin{minipage}[t]{0.5\linewidth}
\small
\begin{center}*G
\end{center}
\begin{tabular}{rl}
 & beidiu lief unde spranc\\ 
 & Segremors, der \textbf{nâch strîte ie} ranc.\\ 
 & s\textit{w}â der vehten wânde vinden,\\ 
 & dâ muose man in binden,\\ 
5 & oder er wolt dâr mite sîn.\\ 
 & \textbf{ninder} ist sô breit der Rîn,\\ 
 & sæher strîten an dem anderen stade,\\ 
 & dâ würde wênic nâch dem bade\\ 
 & \textbf{\begin{large}G\end{large}erastet}, ez wære warm oder kalt:\\ 
10 & \textbf{er} viele \textbf{sus} dran, der \textbf{helt} balt.\\ 
 & \textbf{sus} kom der \textbf{snelle} jungelinc\\ 
 & ze hove an Artuses rinc.\\ 
 & der \textbf{werde} künec vaste slief.\\ 
 & Segremors \textbf{im} durch die snüere lief,\\ 
15 & zer pavelûnes tür drang er \textit{î}n.\\ 
 & ein declachen zobelîn\\ 
 & zucter abe \textbf{in}, \textbf{die dâ} lâgen\\ 
 & unde süezes \textbf{slâfes} pflâgen,\\ 
 & sô daz si muosen wachen\\ 
20 & unde sîner \textbf{unvuoge} lachen.\\ 
 & dô sprach er zuo der niftelen sîn:\\ 
 & "Schinover, vrouwe künigîn,\\ 
 & unser sippe ist \textbf{des} \textbf{erkant}\\ 
 & - man weiz wol über \textbf{manic} lant -,\\ 
25 & daz ich genâden warte an dich.\\ 
 & \textit{nû} hilf mir, vrouwe, unde sprich\\ 
 & gein Artuse, dînem man,\\ 
 & daz ich von im müe\textit{z}e hân\\ 
 & - ein âventiure, \textbf{diu} ist hie bî -,\\ 
30 & daz ich zer tjost der êrste sî."\\ 
\end{tabular}
\scriptsize
\line(1,0){75} \newline
G I O L M Q R Z Fr40 Fr60 \newline
\line(1,0){75} \newline
\textbf{1} \textit{Initiale} O R Fr40 Fr60  \textbf{3} \textit{Initiale} Z  \textbf{9} \textit{Initiale} G  \textbf{11} \textit{Initiale} I M  \newline
\line(1,0){75} \newline
\textbf{1} beidiu] ÷æidiv O Beide er L Beide R Baider Fr60 \textbf{2} Segremors] seigremors I Saigrimors L Sigremors M  $\cdot$ nâch strîte ie] îe nach strite O (L) (Q) (R) (Z) (Fr40) (Fr60) noch strite M \textbf{3} swâ] sa G Wo L (M) (Q) (R)  $\cdot$ vehten wânde] wande vechten L \textbf{4} dâ] Do Q R  $\cdot$ muose] muͤst I mves Fr60 \textbf{5} oder] olde G Ob O (Fr60) \textbf{6} ninder] Nirgen M  $\cdot$ Rîn] ryn M rein Fr40 Fr60 \textbf{7} sæher] Sach er M  $\cdot$ dem anderen stade] den andern staden L dem andere state Q \textbf{8} dâ] Do Q R  $\cdot$ bade] pfade Q \textbf{9} Gerastet] Getast L Q R Z Gestat M Gefragt Fr60  $\cdot$ ez] ob ez Z  $\cdot$ warm oder kalt] kalt oder warm alt R \textbf{10} er] Ez L  $\cdot$ viele] fuͦr R  $\cdot$ sus] us Q  $\cdot$ dran] darin R \textbf{12} An artus houe den Ring R  $\cdot$ Artuses] artus G M Q (Fr40) \textbf{13} vaste] vil vaste Z  $\cdot$ slief] lieff Q \textbf{14} Segremors] Saigremors L Sigemors Fr60  $\cdot$ lief] rief L [rief]: lief Fr60 \textbf{15} zer] Zcu M  $\cdot$ pavelûnes] pavilvne L  $\cdot$ tür] durch M  $\cdot$ drang er în] dranger hin G dranch er dar in L er in [lieff]: trang R \textbf{16} zobelîn] zoͯbelin lang R \textbf{17} zucter] zucht er I (O) (Q) (Z) (Fr40) (Fr60) Zukter er R  $\cdot$ dâ] \textit{om.} M Q Z Fr40 \textbf{19} muosen] muͤstan I \textbf{20} unvuoge] vngefuͤge I (O) (Fr60) \textbf{21} dô sprach er] Da sprach O Er sprach L Do sprach Fr60  $\cdot$ der] \textit{om.} L  $\cdot$ niftelen] niftel I O Fr60 \textbf{22} Schinover] Ginouer I Kynover O Gýnoviere L Ginofer M Seiner Q Gynower R Gynover Z (Fr40) Fr60  $\cdot$ vrouwe] frawen Q \textbf{23} sippe] \textit{om.} Z  $\cdot$ ist] was O  $\cdot$ des] so R \textbf{24} weiz] wil M weist R  $\cdot$ wol] \textit{om.} Q  $\cdot$ manic] \textit{om.} O Fr60 \textbf{25} genâden] genade I (M) \textbf{26} nû] vnde G  $\cdot$ mir] \textit{om.} O Fr60 \textbf{27} Artuse] artus Q Fr40 artusen R  $\cdot$ dînem] dinen O \textbf{28} müeze] moͮse G (R) muͯszen L \textbf{29} diu] \textit{om.} R Z \textbf{30} Das ich der erste zem strit muͯs sin R  $\cdot$ daz] da Fr40  $\cdot$ zer] zv Z  $\cdot$ êrste] este Fr60 \newline
\end{minipage}
\hspace{0.5cm}
\begin{minipage}[t]{0.5\linewidth}
\small
\begin{center}*T
\end{center}
\begin{tabular}{rl}
 & beide lief unde spranc\\ 
 & Segremors, der \textbf{ie nâch strîte} ranc.\\ 
 & swâ der vehten wânde vinden,\\ 
 & dâ muose man in binden,\\ 
5 & oder er wolte dâr mite sîn.\\ 
 & \textbf{niergen} ist sô breit der Rîn,\\ 
 & sæher strîten an dem andern stade,\\ 
 & dâ würde wênic nâch dem bade\\ 
 & \textbf{getast}, ez wære warm oder kalt:\\ 
10 & \textbf{ez} viele dran der \textbf{degen} balt.\\ 
 & \textbf{\begin{large}S\end{large}us} kom der \textbf{snelle} jungelinc\\ 
 & ze hove an Artuses rinc.\\ 
 & der künec \textbf{dannoch} vaste slief.\\ 
 & Segremors durch die snüere lief,\\ 
15 & zer pavelûn tür dranc er \textbf{dar} în.\\ 
 & ein deckelachen zobelîn\\ 
 & zuhter abe \textbf{in}, \textbf{dâ diu} \textbf{zwei} lâgen\\ 
 & unde süezes \textbf{slâfes} pflâgen,\\ 
 & sô daz si muosen wachen\\ 
20 & unde sîner \textbf{unvuoge} lachen.\\ 
 & Dô sprach er zuo der niftel\textit{e}n sîn:\\ 
 & "Gynover, vrou künegîn,\\ 
 & Unser sippe ist \textbf{des} \textbf{bekant}\\ 
 & - man weiz wol über \textbf{manec} lant -,\\ 
25 & daz ich gnâden warte an dich.\\ 
 & nû hilf mir, vrouwe, unde sprich\\ 
 & gegen Artuse, dînem man,\\ 
 & daz ich von im müeze hân\\ 
 & - ein âventiure, \textbf{diu} ist hie bî -,\\ 
30 & daz i\textit{ch} zer tjost der êrste sî."\\ 
\end{tabular}
\scriptsize
\line(1,0){75} \newline
T U V W \newline
\line(1,0){75} \newline
\textbf{11} \textit{Initiale} T U V  \textbf{21} \textit{Initiale} W   $\cdot$ \textit{Majuskel} T  \textbf{23} \textit{Majuskel} T  \newline
\line(1,0){75} \newline
\textbf{2} Segremors] Seigremuͦrs U [S*]: Sagremors V \textbf{3} swâ] Wa U (W) \textbf{4} dâ] Do U V W \textbf{6} niergen] Niendert W  $\cdot$ breit] braite W  $\cdot$ Rîn] in U \textbf{7} sæher] Sach er W \textbf{8} dâ] Do U W  $\cdot$ dem] den W \textbf{10} viele dran] enviele drin V viele sus dran W \textbf{13} künec dannoch] werde kúnig W \textbf{14} Segremors] [S*gremors]: Sagremors V  $\cdot$ durch] im durch W \textbf{15} zer pavelûn] [*z]: Zvͦ dez gezeltes V Zuͦ der pauiluns W  $\cdot$ dar] \textit{om.} W \textbf{17} zuhter abe in] Zoch von in U Zoch er V Zoch er ab im W  $\cdot$ dâ] do U V W \textbf{18} slâfes] [slafe*]: slafes V \textbf{19} muosen] mvͤsten V \textbf{21} niftelen] niftelin T \textbf{22} Gynover] Genover T Ginouer V Schinouer W \textbf{23} bekant] erkant U V \textbf{25} ich] \textit{om.} U \textbf{27} man] lieben man W \textbf{28} müeze] muͦsse W \textbf{30} ich] ist T  $\cdot$ êrste] [*]: erste T \newline
\end{minipage}
\end{table}
\end{document}
