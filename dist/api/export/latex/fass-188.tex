\documentclass[8pt,a4paper,notitlepage]{article}
\usepackage{fullpage}
\usepackage{ulem}
\usepackage{xltxtra}
\usepackage{datetime}
\renewcommand{\dateseparator}{.}
\dmyyyydate
\usepackage{fancyhdr}
\usepackage{ifthen}
\pagestyle{fancy}
\fancyhf{}
\renewcommand{\headrulewidth}{0pt}
\fancyfoot[L]{\ifthenelse{\value{page}=1}{\today, \currenttime{} Uhr}{}}
\begin{document}
\begin{table}[ht]
\begin{minipage}[t]{0.5\linewidth}
\small
\begin{center}*D
\end{center}
\begin{tabular}{rl}
\textbf{188} & \begin{large}D\end{large}er gast gedâhte, ich sage \textbf{iu} wie:\\ 
 & "Liaze ist dort, Liaze ist hie.\\ 
 & \textbf{mir wil got sorge} mâzen.\\ 
 & nû sihe ich Liazen,\\ 
5 & des werden Gurnemanzes kint."\\ 
 & Liazen schœne was ein wint\\ 
 & gein \textbf{der} meide, diu \textbf{hie} saz,\\ 
 & an der got wunsches niht vergaz\\ 
 & - \textbf{Diu} was des landes vrouwe -,\\ 
10 & \textbf{alsô} von dem \textbf{süezem} touwe\\ 
 & diu rôse ûz ir bälgelîn\\ 
 & \textbf{blecket} niwen werden schîn,\\ 
 & der beidiu wîze ist unt rôt.\\ 
 & \textbf{daz} vuogte ir gaste grôze nôt.\\ 
15 & Sîn manlîch zuht was \textbf{im} sô ganz,\\ 
 & sît in der werde Gurnamanz\\ 
 & von sîner tumpheit \textbf{geschiet}\\ 
 & unt im vrâgen widerriet,\\ 
 & \textbf{ez enwære} bescheidenlîche.\\ 
20 & bî der küneginne rîche\\ 
 & saz sîn mu\textit{nt} gar âne wort,\\ 
 & nâhe aldâ, niht verre dort.\\ 
 & maneger kan noch rede sparn,\\ 
 & der mêr \textbf{gein} vrouwen ist gevarn.\\ 
25 & Diu küneginne \textbf{gedâhte} sân:\\ 
 & "ich wæne, mich smæhet dirre man,\\ 
 & durch daz mîn lîp vertwâlt ist.\\ 
 & nein, er tuotz durch einen list.\\ 
 & er ist gast, ich bin wirtîn.\\ 
30 & diu êrste rede \textbf{wære} mîn.\\ 
\end{tabular}
\scriptsize
\line(1,0){75} \newline
D \newline
\line(1,0){75} \newline
\textbf{1} \textit{Initiale} D  \textbf{9} \textit{Majuskel} D  \textbf{15} \textit{Majuskel} D  \textbf{25} \textit{Majuskel} D  \newline
\line(1,0){75} \newline
\textbf{21} munt] mvͦtr D \newline
\end{minipage}
\hspace{0.5cm}
\begin{minipage}[t]{0.5\linewidth}
\small
\begin{center}*m
\end{center}
\begin{tabular}{rl}
 & \begin{large}D\end{large}er gast gedâhte, ich sage wie:\\ 
 & "Liaze ist dort, Liaze ist hie.\\ 
 & \textbf{mir wil got sorge} mâzen.\\ 
 & nû sihe ich Liazen,\\ 
5 & des werden Gurnemanzes kint."\\ 
 & Liazen schœne was ein \textit{w}int\\ 
 & gegen \textbf{dirre} megde, diu \textbf{hie} saz,\\ 
 & an der got wunsches niht vergaz\\ 
 & - \textbf{diu} was des landes vrouwe -,\\ 
10 & \textbf{als} von dem \textbf{süezen} touwe\\ 
 & diu rôse ûz ir bälgelîn\\ 
 & \textbf{blecket} niwen werden schîn,\\ 
 & der beidiu wîz ist und rôt.\\ 
 & \textbf{daz} vuogete ir gast grôze nôt.\\ 
15 & sîn manlîch zuht was \textbf{im} sô ganz,\\ 
 & sît in der werde Gurnemanz\\ 
 & von sîner tumpheit \textbf{geschiet}\\ 
 & und i\textit{m} vrâgen widerriet,\\ 
 & \textbf{ez enwære} bescheidenlîch.\\ 
20 & bî der küniginn\textit{e} rîch\\ 
 & saz sîn munt gar âne wort,\\ 
 & nâhe aldâ, niht verre dort.\\ 
 & maniger kan noch rede sparn,\\ 
 & der mêr \textbf{gê\textit{n}} \textit{v}rouwen ist gevarn.\\ 
25 & \begin{large}D\end{large}iu künigîn \textbf{gedâhte} sân:\\ 
 & "ich wæne, mich smâhet dirre man,\\ 
 & durch daz mîn lîp vertwâlet ist.\\ 
 & nein, er tuot ez durch einen list.\\ 
 & er ist gast, ich bin wirtîn.\\ 
30 & diu êrste rede \textbf{wær} \textbf{billîche} mîn.\\ 
\end{tabular}
\scriptsize
\line(1,0){75} \newline
m n o Fr69 \newline
\line(1,0){75} \newline
\textbf{1} \textit{Initiale} m   $\cdot$ \textit{Capitulumzeichen} n  \textbf{25} \textit{Initiale} m Fr69   $\cdot$ \textit{Capitulumzeichen} n  \newline
\line(1,0){75} \newline
\textbf{2} Liaze (Lyas n Las o ) ist dort liaze (lyas n lasz o ) ist hie m (n) (o) \textbf{3} mâzen] lossen n (o) \textbf{4} Liazen] liossen n laizen o \textbf{5} des] Das o  $\cdot$ Gurnemanzes] [gurnemanczen]: gurnemanczes m gurnemantzen n gurnem anczlicz o \textbf{6} Liazen] Liantzen n Laszen o  $\cdot$ wint] kint m \textbf{8} niht] nẏe n (o) \textbf{10} als] Al n \textbf{12} niwen] nit wenne n nit wan o  $\cdot$ werden] werder o \textbf{14} daz] Der n  $\cdot$ ir gast] jren gasten o \textbf{15} sô] also o \textbf{16} Gurnemanz] gurnemancz m gurnemantz n gurmenancz o \textbf{18} im] in m n o  $\cdot$ widerriet] wider geriet n o \textbf{20} küniginne] kuniginnen m \textbf{24} mêr gên vrouwen] mergent zuͯm frouwen m nẏergent zuͯ den frouwen n nẏndert zuͯn frowen o \textbf{25} gedâhte] dochte n \textbf{26} smâhet] smacket o \textbf{27} vertwâlet] verswalet o \newline
\end{minipage}
\end{table}
\newpage
\begin{table}[ht]
\begin{minipage}[t]{0.5\linewidth}
\small
\begin{center}*G
\end{center}
\begin{tabular}{rl}
 & \begin{large}D\end{large}er gast gedâht, ich sage \textbf{iu} wie:\\ 
 & "Liaze ist dort, Liaze ist hie.\\ 
 & \textbf{mich \textit{wil} got leides} mâzen.\\ 
 & nû sihe ich Liazen,\\ 
5 & des werden Gurnomanzes kint."\\ 
 & Liazen schœne was ein wint\\ 
 & gein \textbf{dirre} meide, diu \textbf{hie} saz,\\ 
 & an der got wunsches niht vergaz:\\ 
 & \textbf{dô} was des landes vrouwe,\\ 
10 & \textbf{als} von dem \textbf{süezen} touwe\\ 
 & diu rôse ûz ir bälgelîn\\ 
 & \textbf{enblecket} niwen werden schîn,\\ 
 & der beidiu wîz ist unde rôt.\\ 
 & \textbf{daz} vuogete ir gaste grôze nôt.\\ 
15 & sîn manlîch zuht was sô ganz,\\ 
 & sît in der werde Gurnomanz\\ 
 & von sîner tumpheite \textbf{schiet}\\ 
 & unde im vrâgen widerriet\\ 
 & \textbf{niwan} bescheidenlîche.\\ 
20 & bî der küniginne rîche\\ 
 & saz sîn munt gar âne wort,\\ 
 & nâhe al dâ, niht verre dort.\\ 
 & maniger kan noch rede sparn,\\ 
 & der mêr \textbf{ze} vrouwen ist gevarn.\\ 
25 & diu küniginne \textbf{dâhte} sân:\\ 
 & "ich wæne, mich smæht dirre man,\\ 
 & dur daz mîn lîp vertwâlt ist.\\ 
 & nein, er tuotz durch einen list.\\ 
 & er ist gast, ich bin wirtîn.\\ 
30 & diu êrste rede \textbf{solte wesen} mîn.\\ 
\end{tabular}
\scriptsize
\line(1,0){75} \newline
G I O L M Q R Z Fr47 \newline
\line(1,0){75} \newline
\textbf{1} \textit{Initiale} G  \textbf{9} \textit{Initiale} R  \textbf{15} \textit{Initiale} I  \textbf{19} \textit{Initiale} L Z Fr47  \textbf{25} \textit{Initiale} O Q  \newline
\line(1,0){75} \newline
\textbf{1} Der] Des Q  $\cdot$ ich sage iu] \textit{om.} M \textbf{2} liaz (Lýaze L Lyassen Q Lyaze R ) ist dort liaz (Lýaze L liassen Q Lyaze R ) ist hie I (L) (M) (Q) (R) (Z) \textbf{3} Got wil mir sorge maszen L  $\cdot$ mich] auch I Mir M Q R Z  $\cdot$ wil] \textit{om.} G  $\cdot$ leides] sorgen O (M) (Q) R sorge Z \textbf{4} Liazen] Lýaszen L liaszin M lyassen Q lyazen R liazzen Z \textbf{5} Gurnomanzes] Gurnomzes G [kurnanzes]: kurnmanzes I kvrnemanzes O gammuretisz M gurnomantzes Q Gurnamaurs R gvrnemantzes Z \textbf{6} Liazen] Lýazen L Liaze M Lẏ:ssen Q Lyazen R Liazzen Z \textbf{7} dirre] der L  $\cdot$ hie] da L \textbf{8} wunsches] [wuncches]: wunsches O  $\cdot$ niht] nie I (Q) \textbf{9} dô] Da O M Die Z \textbf{10} dem] den M  $\cdot$ süezen] suͤzzem I \textbf{12} enblecket] endechet I \textbf{13} der] Wer M  $\cdot$ beidiu wîz ist] ist beidiu wiz I bey der weysz Q  $\cdot$ unde] vnder Q \textbf{14} daz] Da R  $\cdot$ vuogete] vuͤget I (Z) (Fr47)  $\cdot$ grôze] groziv O hertze L \textbf{15} was] ie was I was im O (M) Z (Fr47) \textbf{16} in] ir Q  $\cdot$ Gurnomanz] churnomanz G kurnemanz I (O) Gvrnomantz L (Q) gurnemancz M Gurnamancz R gvrnemantz Z Gurnem::: Fr47 \textbf{17} schiet] geschiet O M (R) Z (Fr47) geschigt Q \textbf{18} vrâgen] vrage L  $\cdot$ widerriet] widerreit I \textbf{19} niwan] wan I (O) (L) (M) (Q) (R) Ez wer Z ÷ann Fr47 \textbf{20} küniginne] kunginnen R  $\cdot$ rîche] richen I \textbf{21} saz] Das M A as Q  $\cdot$ sîn] seine Q  $\cdot$ gar] \textit{om.} Fr47 \textbf{22} al dâ niht] nih Gar I da niht Z \textbf{23} noch rede] noch nicht M nach rede Z  $\cdot$ sparn] Gesparn I \textbf{24} der] De R  $\cdot$ ze vrouwen] zefrævden O (L) (Fr47) zu frúmen Q \textbf{25} diu] ÷iv O  $\cdot$ dâhte] Gedahte I (O) (L) (M) (Q) (R) (Z) (Fr47) \textbf{26} smæht] smehe Z smech Fr47  $\cdot$ dirre] dise M \textbf{27} vertwâlt] verwalet L \textbf{28} er] ein Q  $\cdot$ einen] ein I eyne M \textbf{29} \textit{Versfolge 188.30-29} O Fr47   $\cdot$ ist] \textit{om.} I  $\cdot$ ich bin] vnde ich ich div O \textbf{30} êrste] erstev O  $\cdot$ solte wesen] were I (O) L M Q R Z ist Fr47 \newline
\end{minipage}
\hspace{0.5cm}
\begin{minipage}[t]{0.5\linewidth}
\small
\begin{center}*T
\end{center}
\begin{tabular}{rl}
 & Der gast gedâhte, ich sag\textbf{iu} wie:\\ 
 & "Lyaze ist dort, Lyaze ist hie.\\ 
 & \textbf{got wil mich sorgen} mâzen.\\ 
 & nû sihich Lyazen,\\ 
5 & des werden Gurnemanzes kint."\\ 
 & Lyazen schœne was ein wint\\ 
 & gegen \textbf{der} megede, diu \textbf{dâ} saz,\\ 
 & an der got wunsches niht vergaz:\\ 
 & \textbf{dô} was des landes vrouwe,\\ 
10 & \textbf{sam} von dem touwe\\ 
 & di\textit{u} rôse ûz ir bälgelîn\\ 
 & \textbf{enblecket} niuwen werden schîn,\\ 
 & der beid\textit{iu} wîz ist unde rôt.\\ 
 & \textbf{dô} vuogte ir gaste grôze nôt.\\ 
15 & sîn manlîch zuht was \textbf{wol} sô ganz,\\ 
 & sît in der werde Gurnemanz\\ 
 & von sîner tumpheit \textbf{gar} \textbf{geschiet}\\ 
 & unde im vrâgen widerriet,\\ 
 & \textbf{ez enwære} bescheidenlîche.\\ 
20 & bî der küneginne rîche\\ 
 & saz sîn munt gar âne wort,\\ 
 & nâhe aldâ, niht verre dort.\\ 
 & Maneger kan noch rede sparn,\\ 
 & der mê \textbf{ze} vrouwen ist gevarn.\\ 
25 & \begin{large}D\end{large}iu künegîn \textbf{gedâhte} sân:\\ 
 & "ich wæne, mich \textit{s}mæhet dirre man,\\ 
 & durch daz mîn lîp vertwâlet ist.\\ 
 & Nein, er tuot ez durch einen list.\\ 
 & er ist gast, ich bin wirtîn.\\ 
30 & diu êrste rede \textbf{sol wesen} mîn.\\ 
\end{tabular}
\scriptsize
\line(1,0){75} \newline
T U V W \newline
\line(1,0){75} \newline
\textbf{1} \textit{Initiale} W   $\cdot$ \textit{Majuskel} T  \textbf{23} \textit{Majuskel} T  \textbf{25} \textit{Initiale} T U V  \textbf{28} \textit{Majuskel} T  \newline
\line(1,0){75} \newline
\textbf{1} sagiu] sage U V \textbf{2} Liaze (Lyas W ) ist dort liaze (lias W ) ist hie V (W) \textbf{3} mich] [mich]: mir V \textbf{4} sihich] sich ich hie V  $\cdot$ Lyazen] Lŷazen T liazen V lyassen W \textbf{5} Gurnemanzes] Guͦrnemanzes U [Gvrnam*]: Gvrnamanzes V \textbf{6} Lyazen] Liazen V Lyassen W \textbf{7} der] [*]: dirre V  $\cdot$ dâ] hie V ee do W \textbf{9} dô] Die V \textbf{10} sam] [S*m]: Alse V  $\cdot$ touwe] [*]: svͤzzen towe V suͤssen tauwe W \textbf{11} diu] die T  $\cdot$ ûz] [*]: vs V \textbf{12} enblecket] Erblecket W \textbf{13} beidiu] beide T  $\cdot$ wîz ist] ist weiß W \textbf{14} dô vuogte] Des vuchte U Daz fvͦgete V (W)  $\cdot$ grôze nôt] herze not U (V) (W) \textbf{16} Gurnemanz] gurnemantz W \textbf{21} saz] Waz V \textbf{23} noch] nicht W \textbf{24} der mê ze] der nie zuͦ U [D*]: Der mer zvͦ V Der nie gen W \textbf{25} gedâhte] gedacht ir W \textbf{26} smæhet] machet T versmahet V \textbf{28} Nein] \textit{om.} U \textbf{30} sol wesen] wer W \newline
\end{minipage}
\end{table}
\end{document}
