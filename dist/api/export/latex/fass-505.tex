\documentclass[8pt,a4paper,notitlepage]{article}
\usepackage{fullpage}
\usepackage{ulem}
\usepackage{xltxtra}
\usepackage{datetime}
\renewcommand{\dateseparator}{.}
\dmyyyydate
\usepackage{fancyhdr}
\usepackage{ifthen}
\pagestyle{fancy}
\fancyhf{}
\renewcommand{\headrulewidth}{0pt}
\fancyfoot[L]{\ifthenelse{\value{page}=1}{\today, \currenttime{} Uhr}{}}
\begin{document}
\begin{table}[ht]
\begin{minipage}[t]{0.5\linewidth}
\small
\begin{center}*D
\end{center}
\begin{tabular}{rl}
\textbf{505} & \begin{large}D\end{large}er schilt was \textbf{ouch} verhouwen.\\ 
 & Gawan begunde in schouwen,\\ 
 & dô er dar zuo \textbf{was} geriten.\\ 
 & der tjoste venster was \textbf{gesniten}\\ 
5 & mit der glavîne wît;\\ 
 & alsus mâlt si der strît.\\ 
 & wer gülte\textbf{s} den schiltæren,\\ 
 & ob ir varwe alsus wæren?\\ 
 & der linden grôz was \textbf{der} stam.\\ 
10 & ouch saz ein vrouwe, \textbf{an} vreuden lam,\\ 
 & dâr hinder ûf \textbf{grüenem} klê.\\ 
 & der tet grôz jâmer als wê,\\ 
 & daz si der vreude gar vergaz.\\ 
 & er reit hin umbe gein ir baz.\\ 
15 & \textbf{ir} lac ein rîter in \textbf{der} schôz,\\ 
 & dâ von ir jâmer was sô grôz.\\ 
 & Gawan \textbf{sîn} grüezen niht versweic;\\ 
 & diu vrouwe \textbf{im} dancte und neic.\\ 
 & er vant ir stimme heise,\\ 
20 & verschrît durch ir vreise.\\ 
 & dô erbeizte mîn hêr Gawan.\\ 
 & dâ lac durchstochen ein man,\\ 
 & dem gienc daz bluot in den lîp.\\ 
 & dô vrâgter des heldes wîp,\\ 
25 & ob der rîter lebte\\ 
 & ode mit dem tôde strebte.\\ 
 & dô sprach si: "hêrre, er lebt noch -\\ 
 & ich wæne, \textbf{daz} ist \textbf{unlenge} doch.\\ 
 & got sande iuch mir ze trôste her;\\ 
30 & nû râtet \textbf{nâch} \textbf{iwerre} triwen ger.\\ 
\end{tabular}
\scriptsize
\line(1,0){75} \newline
D \newline
\line(1,0){75} \newline
\textbf{1} \textit{Initiale} D  \newline
\line(1,0){75} \newline
\newline
\end{minipage}
\hspace{0.5cm}
\begin{minipage}[t]{0.5\linewidth}
\small
\begin{center}*m
\end{center}
\begin{tabular}{rl}
 & der schilt was verhouwen.\\ 
 & Gawan begunde in schouwen,\\ 
 & dô er dâ zuo \textbf{kam} geriten.\\ 
 & der juste venster was \textbf{gesniten}\\ 
5 & mit der glevîn wît;\\ 
 & alsus mâlet si der strît.\\ 
 & wer gülte \textbf{si} den schiltæren,\\ 
 & ob ir varwe alsus wæren?\\ 
 & der linden grôz was \textbf{der} stam.\\ 
10 & ouch saz ein vrouwe, \textbf{an} vröuden lam,\\ 
 & dâr hinder ûf \textbf{dem grüenen} klê.\\ 
 & der tet grô\textit{z} jâmer alsô wê,\\ 
 & daz si der vröuden gar vergaz.\\ 
 & er reit hin umb gegen ir baz.\\ 
15 & \textbf{d\textit{â}} lac ein ritter in \textbf{ir} schôz,\\ 
 & dâ von ir jâmer was sô grôz.\\ 
 & Gawan \textbf{im} grüezen niht versweic;\\ 
 & diu vrouwe dankete und neic.\\ 
 & er vant ir stimme heise,\\ 
20 & verschrîet durch ir vreise.\\ 
 & dô erbeizte mîn hêrre Gawan.\\ 
 & d\textit{â} lac durchstochen ein man,\\ 
 & dem gienc daz bluot in den lîp.\\ 
 & dô vrâgte er des heldes wîp,\\ 
25 & ob der ritter lebete\\ 
 & oder mit dem tôde strebete.\\ 
 & dô sprach si: "hêrre, er lebet noch -\\ 
 & ich wæne, \textit{\textbf{ez}} ist \textbf{âne lenge} doch.\\ 
 & got sant iuch mir zuo trôste her;\\ 
30 & nû râtet \textbf{durch} \textbf{iuwer} triuwe ger.\\ 
\end{tabular}
\scriptsize
\line(1,0){75} \newline
m n o \newline
\line(1,0){75} \newline
\newline
\line(1,0){75} \newline
\textbf{3} dâ zuo] zuͯ dar n \textbf{7} schiltæren] schilten o \textbf{8} varwe] farwen n \textbf{9} grôz] grosze o \textbf{12} grôz] gro m grosses n \textbf{14} baz] has o \textbf{15} dâ] Do m n o  $\cdot$ ritter] werder ritter n  $\cdot$ ir] \textit{om.} o \textbf{17} Gawan] [Gawa*n]: Gawan m \textbf{18} dankete] jme dancket n im danckte o \textbf{22} dâ] Do m n o \textbf{26} mit] mir n \textbf{27} dô] Doch o \textbf{28} ez] \textit{om.} m es es n \newline
\end{minipage}
\end{table}
\newpage
\begin{table}[ht]
\begin{minipage}[t]{0.5\linewidth}
\small
\begin{center}*G
\end{center}
\begin{tabular}{rl}
 & \begin{large}D\end{large}er schilt was \textbf{ouch} verhouwen.\\ 
 & Gawan begunde in schouwen,\\ 
 & dô er dar zuo \textbf{kom} geriten.\\ 
 & der tjoste venster was \textbf{gesniten}\\ 
5 & mit der glavîne wî\textit{t};\\ 
 & alsus mâlte si der strît.\\ 
 & wer gülte\textbf{s} den schiltæren,\\ 
 & ob ir varwe alsus w\textit{æ}ren?\\ 
 & der linden grôz was \textbf{der} stam.\\ 
10 & ouch saz ein vrouwe, \textbf{an} vröuden lam,\\ 
 & dâr hinder ûf \textbf{grüenem} klê.\\ 
 & der tet grôz jâmer als wê,\\ 
 & daz si der vröude gar vergaz.\\ 
 & er reit hin umbe gein ir baz.\\ 
15 & \textbf{ir} lac ein rîter in \textbf{\textit{de}r} schôz,\\ 
 & dâ von ir jâmer was sô grôz.\\ 
 & G\textit{a}wan \textbf{sîn} grüezen niht versweic;\\ 
 & diu vrouwe \textbf{im} danket unde neic.\\ 
 & er vant ir stimme heise,\\ 
20 & verschrîet durch ir vreise.\\ 
 & dô erbeizte mîn hêrre Gawan.\\ 
 & dâ lac durchstochen ein man,\\ 
 & dem gienc daz bluot in den lîp.\\ 
 & dô vrâget er des heldes wîp,\\ 
25 & op der rîter lebete\\ 
 & ode mit dem tôde strebete.\\ 
 & dô sprach si: "hêrre, er lebet noch -\\ 
 & ich wæne, \textbf{daz} ist \textbf{unlenge} doch.\\ 
 & got sande iuch mir ze trôste her;\\ 
30 & nû râtet \textbf{nâch} \textbf{iuwer} triuwen ger.\\ 
\end{tabular}
\scriptsize
\line(1,0){75} \newline
G I L M Z Fr57 \newline
\line(1,0){75} \newline
\textbf{1} \textit{Initiale} G I L M Z Fr57  \textbf{17} \textit{Initiale} I  \newline
\line(1,0){75} \newline
\textbf{2} in] \textit{om.} M \textbf{3} dô] Dar M (Z) \textbf{4} gesniten] vorsnyten M \textbf{5} glavîne] geleueinen M \textbf{6} mâlte si] malet sich I malet sie M (Z) \textbf{7} gültes] gulde sie M  $\cdot$ den] danne Z \textbf{8} alsus] suͯs L  $\cdot$ wæren] waren G L \textbf{9} der stam] ir Stam M \textbf{10} ouch saz] Durch das M \textbf{11} dâr hinder] Herhinder L  $\cdot$ grüenem] grunen M \textbf{12} grôz] groszir M  $\cdot$ als] \textit{om.} M \textbf{13} si] er I  $\cdot$ der] \textit{om.} Z  $\cdot$ vröude] froͮiden I (M) \textbf{14} umbe] \textit{om.} L  $\cdot$ gein] zv Z \textbf{15} in der] in ir G an der I \textbf{17} Gawan] gewan G Gawa::: M  $\cdot$ sîn] Sie M \textbf{18} vrouwe] \textit{om.} Z  $\cdot$ danket] dankite M \textbf{20} verschrîet] [gesrit]: geshrit I  $\cdot$ ir] die Z  $\cdot$ vreise] reise L \textbf{21} dô] Da M Z  $\cdot$ erbeizte] der baizte I erbeiszet L (Z)  $\cdot$ mîn] \textit{om.} L  $\cdot$ hêrre] er M  $\cdot$ Gawan] gawein G \textbf{22} dâ] do I  $\cdot$ durchstochen] irstochin M \textbf{24} dô] Da M Z  $\cdot$ vrâget] vragite M \textbf{27} dô] Da M \textbf{28} unlenge] vnlange I M \textbf{30} triuwen] truͯwe L \newline
\end{minipage}
\hspace{0.5cm}
\begin{minipage}[t]{0.5\linewidth}
\small
\begin{center}*T
\end{center}
\begin{tabular}{rl}
 & \begin{large}D\end{large}er schilt was \textbf{ouch} verhouwen.\\ 
 & Gawan begunde in schouwen,\\ 
 & dô er dar zuo \textbf{kom} geriten.\\ 
 & der tjoste venster was \textbf{versniten}\\ 
5 & Mit der glevîne wît;\\ 
 & Alsus mâlt si der strît.\\ 
 & wer gülte\textbf{z} den schiltæren,\\ 
 & Ob ir varwe alsus wæren?\\ 
 & der linden grôz was \textbf{ir} stam.\\ 
10 & Ouch saz ein vrouwe vreuden lam\\ 
 & dâr hinder ûf \textbf{grüenem} klê.\\ 
 & der tet grôz jâmer als wê,\\ 
 & daz si der vreuden gar vergaz.\\ 
 & Er reit hin umbe gein ir baz.\\ 
15 & \textbf{ir} lac ein ritter in \textbf{der} schôz,\\ 
 & dâ von ir jâmer was sô grôz.\\ 
 & Gawan \textbf{si} grüezen niht versweic;\\ 
 & diu vrouwe \textbf{im} danket unde neic.\\ 
 & Er vant ir stimme heise,\\ 
20 & verschrît durch ir vreise.\\ 
 & dô erbeizte mîn hêr Gawan.\\ 
 & dâ lac durchstochen ein man,\\ 
 & dem gie daz bluot in den lîp.\\ 
 & dô \textit{v}râget er de\textit{s} heldes wîp,\\ 
25 & Ob der ritter lebte\\ 
 & Oder mit dem tôde strebte.\\ 
 & dô sprach si: "hêrre, er lebet noch -\\ 
 & Ich wæne, \textbf{daz} ist \textbf{unlange} doch.\\ 
 & Got sande iuch mir ze trôste her;\\ 
30 & Nû râtet \textbf{nâch} \textbf{iuwer} triuwen ger.\\ 
\end{tabular}
\scriptsize
\line(1,0){75} \newline
T U V W O Q R Fr39 \newline
\line(1,0){75} \newline
\textbf{1} \textit{Initiale} T U O Q Fr39  \textbf{5} \textit{Majuskel} T  \textbf{6} \textit{Majuskel} T  \textbf{8} \textit{Majuskel} T  \textbf{10} \textit{Majuskel} T  \textbf{14} \textit{Majuskel} T  \textbf{17} \textit{Initiale} W  \textbf{19} \textit{Majuskel} T  \textbf{25} \textit{Majuskel} T  \textbf{26} \textit{Majuskel} T  \textbf{28} \textit{Majuskel} T  \textbf{29} \textit{Majuskel} T  \textbf{30} \textit{Majuskel} T  \newline
\line(1,0){75} \newline
\textbf{1} Der] [*er]: ÷er O  $\cdot$ verhouwen] zerhowen Fr39 \textbf{2} Gawan] Gawain R  $\cdot$ in] [*]: in V \textit{om.} O \textbf{3} dô] Da R  $\cdot$ zuo kom] kam zu Q \textbf{4} versniten] vermitten W \textbf{5} der] einer R \textbf{6} mâlt] malete V  $\cdot$ si] sich O \textbf{7} gültez] vergvltes Fr39 \textbf{8} varwe] varben W frawe Q \textbf{9} ir] der V \textbf{10} vreuden] an vreiden U (V) (W) (O) (Q) (R) \textbf{11} hinder] hinden O \textbf{12} als] \textit{om.} O R \textbf{14} Er reit [g*]: hin vmbe gegen ir bas V  $\cdot$ Er reit gen Jr hinumb bas R \textbf{15} lac] lag ir W  $\cdot$ der] dem U ir O \textbf{16} jâmer] iarmer Q \textbf{17} Gawan] Gawain R  $\cdot$ si] sin V  $\cdot$ grüezen] grussens Q  $\cdot$ versweic] ver meit O \textbf{18} unde] vnd im Fr39 \textbf{21} erbeizte] erbeicze R  $\cdot$ Gawan] her Gawain R \textbf{22} dâ] Do U V W Q (Fr39) \textbf{24} dô vrâget er] do wraget er T Er fragt R  $\cdot$ des] der T  $\cdot$ heldes] selben Fr39 \textbf{26} dem] \textit{om.} Q \textbf{28} unlange] vnlenge U vnlavgen O \textbf{30} nâch] mir dvrch O  $\cdot$ ger] \textit{om.} U \newline
\end{minipage}
\end{table}
\end{document}
