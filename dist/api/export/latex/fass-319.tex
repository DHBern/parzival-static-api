\documentclass[8pt,a4paper,notitlepage]{article}
\usepackage{fullpage}
\usepackage{ulem}
\usepackage{xltxtra}
\usepackage{datetime}
\renewcommand{\dateseparator}{.}
\dmyyyydate
\usepackage{fancyhdr}
\usepackage{ifthen}
\pagestyle{fancy}
\fancyhf{}
\renewcommand{\headrulewidth}{0pt}
\fancyfoot[L]{\ifthenelse{\value{page}=1}{\today, \currenttime{} Uhr}{}}
\begin{document}
\begin{table}[ht]
\begin{minipage}[t]{0.5\linewidth}
\small
\begin{center}*D
\end{center}
\begin{tabular}{rl}
\textbf{319} & \begin{large}C\end{large}undrie la surziere,\\ 
 & diu unsüeze unt doch diu fiere,\\ 
 & den Waleis \textbf{si} beswæret hât.\\ 
 & waz half in küenes herzen rât\\ 
5 & unt \textbf{wâriu} zuht bî manheit?\\ 
 & \textbf{unt} dennoch mêr im was \textbf{bereit}:\\ 
 & scham ob allen sînen siten.\\ 
 & den rehten valsch het er vermiten,\\ 
 & wan scham gît prîs ze lône\\ 
10 & unt ist doch der sêle krône.\\ 
 & scham ist ob siten \textbf{ein geüebet} uop.\\ 
 & Cunneware daz êrste weinen huop,\\ 
 & daz Parzival, den degen balt,\\ 
 & Cundrie \textbf{surziere} \textbf{sus} beschalt,\\ 
15 & \textbf{ein} \textbf{alsô} wunderlîch geschaf.\\ 
 & herzen jâmer ougen saf\\ 
 & gap maneger werden vrouwen,\\ 
 & \textbf{die man weinende muose} schouwen.\\ 
 & Cundrie was \textbf{ir} trûrens wer.\\ 
20 & \textbf{diu} reit \textbf{enwec}. nû reit dort her\\ 
 & ein rîter, der truoc hôhen muot.\\ 
 & al sîn harnasch was \textbf{sô} guot\\ 
 & von \textbf{den vüezen} unz \textbf{an des houptes dach},\\ 
 & daz mans vür \textbf{grôze} koste jach.\\ 
25 & Sîn zimierde was rîche.\\ 
 & gewâpent rîterlîche\\ 
 & was \textbf{daz} ors unt sîn \textbf{selbes} lîp.\\ 
 & \textbf{nû vander} magt, \textbf{man} unt wîp\\ 
 & trûrec ame ringe hie.\\ 
30 & dâ reit er zuo, nû hœret wie:\\ 
\end{tabular}
\scriptsize
\line(1,0){75} \newline
D \newline
\line(1,0){75} \newline
\textbf{1} \textit{Initiale} D  \textbf{25} \textit{Majuskel} D  \newline
\line(1,0){75} \newline
\textbf{1} Cvndrîe lasvrzîere D \textbf{14} Cundrie surziere] Cvndrîe surziere D \textbf{19} Cundrie] Cvndrîe D \newline
\end{minipage}
\hspace{0.5cm}
\begin{minipage}[t]{0.5\linewidth}
\small
\begin{center}*m
\end{center}
\begin{tabular}{rl}
 & \textit{\begin{large}C\end{large}o}ndrie la\textit{s}urz\textit{i}ere,\\ 
 & diu unsüeze und doch diu fiere,\\ 
 & den Waleis \textbf{si} beswær\textit{e}t hât.\\ 
 & waz half in küenes herzen rât\\ 
5 & und \textbf{wârheit}, zuht bî manheit?\\ 
 & \textbf{und} dannoch mêre im was \textbf{gereit}:\\ 
 & schame ob allen sînen siten.\\ 
 & den rehten valsch het er vermiten,\\ 
 & wanne schame gît prîs ze lône\\ 
10 & und ist doch der sêle krône.\\ 
 & scham ist ob siten \textbf{ein geüebet} \dag vluot\dag .\\ 
 & Cu\textit{nne}w\textit{a}re \dag daz es si\dag  weinen huop,\\ 
 & daz Parcifaln, den degen balt,\\ 
 & Condrie \textbf{surziere} \textbf{sus} beschalt,\\ 
15 & \textbf{ein} \textbf{al\textit{s}ô} wunderlîch geschaf.\\ 
 & herzen jâmer ougen saf\\ 
 & gap maniger werden vrouwen,\\ 
 & \textbf{die man weinende muose} schouwen.\\ 
 & \begin{large}C\end{large}ondrie was \textbf{dô} trûrens wer.\\ 
20 & \textbf{diu} reit \textbf{enwec}. nû reit dort her\\ 
 & ein ritter, der truoc hôhen muot.\\ 
 & allez sîn harnasch was \textbf{sô} guot\\ 
 & von \textbf{de\textit{m} vuoze} unz \textbf{an daz houbetdach},\\ 
 & daz mans vür \textbf{grôze} koste jach.\\ 
25 & sîn zimierde was rîche.\\ 
 & gewâpent ritterlîche\\ 
 & was \textbf{daz} ros und sîn \textbf{selbes} lîp.\\ 
 & \textbf{nû vant er} magt, \textbf{man} und wîp\\ 
 & trûric \textit{am}e ringe hie.\\ 
30 & dâ \textit{reit} er zuo, nû hœret wie:\\ 
\end{tabular}
\scriptsize
\line(1,0){75} \newline
m n o \newline
\line(1,0){75} \newline
\textbf{1} \textit{Initiale} m  \textbf{19} \textit{Illustration mit Überschrift:} Also kingramors (konigramors o  ) von Artuse vnd der masenẏe gawanen kanppfes an sprach n (o)   $\cdot$ \textit{Überschrift:} Wie kingrinvrsel von artuse vnd der massenie Gawanen kampfes ansprach m   $\cdot$ \textit{Initiale} m n o  \newline
\line(1,0){75} \newline
\textbf{1} Sundrie lazurzirere m  $\cdot$ Condri lazurtziere n  $\cdot$ Gundri Lazurziere o \textbf{2} fiere] fuͯre n \textbf{3} Waleis] valeise o  $\cdot$ beswæret] beswerent m \textbf{5} wârheit] woheit o \textbf{6} im] \textit{om.} o  $\cdot$ gereit] bereit n o \textbf{7} allen sînen] allem sime n (o) \textbf{8} den] Der o  $\cdot$ rehten] rechte n \textbf{10} sêle] selen n o \textbf{11} Scham ist [eyn]: ob allen eyn gauͯbet fluͯht o \textbf{12} Cunneware] Cumuwere m Conware n Conne waren o  $\cdot$ es] ist o  $\cdot$ huop] duͦt n dot o \textbf{13} Parcifaln] parcifalen n parcifal o \textbf{14} Condrie surziere] Contri surtzier n Condri surczier o  $\cdot$ sus] sasz n (o) \textbf{15} alsô] allo m  $\cdot$ geschaf] geschafft n o \textbf{18} muose] muͯsse m muͯste n o \textbf{19} Condrie] SOndrie o  $\cdot$ was] das o \textbf{20} enwec] hin weg n \textbf{22} guot] gros o \textbf{23} dem] den m \textbf{24} grôze] grosser o \textbf{25} zimierde] zimde o \textbf{27} sîn] sins o \textbf{29} ame] one m \textbf{30} reit] \textit{om.} m \newline
\end{minipage}
\end{table}
\newpage
\begin{table}[ht]
\begin{minipage}[t]{0.5\linewidth}
\small
\begin{center}*G
\end{center}
\begin{tabular}{rl}
 & Gundrie lasurziere,\\ 
 & diu unsüeze unde doch diu fiere,\\ 
 & den Waleis \textbf{si} beswært hât.\\ 
 & waz half in küenes herzen rât\\ 
5 & unt \textbf{wâriu} zuht bî manheit?\\ 
 & \textit{\textbf{und}} dannoch mêr im was \textbf{bereit}:\\ 
 & scham obe allen sînen siten.\\ 
 & den rehten valsch het er vermiten,\\ 
 & \textit{wan} schame gît prîs ze lône\\ 
10 & unde ist doch der sêle krône.\\ 
 & schame ist \textit{ob} siten \textbf{rehter} uop.\\ 
 & Kuneware daz êrste weinen huop,\\ 
 & daz Parzivalen, den degen balt,\\ 
 & Gundrie \textbf{alsus} beschalt\\ 
15 & \textbf{umbe} \textbf{alsus} wunderlîch geschaf.\\ 
 & herzen jâmer ougen saf\\ 
 & gap maniger werden vrouwen.\\ 
 & \textbf{man muose hie weinen} schouwen.\\ 
 & Gundrie was \textbf{ir} trûrens wer.\\ 
20 & \textbf{si} reit \textbf{den wec}. nû reit dort her\\ 
 & ein rîter, der truoc hôhen muot.\\ 
 & al sîn harnasc\textit{h} \textit{w}as guot\\ 
 & von \textbf{vuoze} unze \textbf{an des houbtes dach},\\ 
 & daz mans vür \textbf{grôze} koste jach.\\ 
25 & sîn zimier, \textbf{daz} was rîche.\\ 
 & gewâpent rîterlîche\\ 
 & was \textbf{sîn} ors unde \textbf{ouch} sîn \textbf{selbes} lîp.\\ 
 & \textbf{manic} maget unde wîp\\ 
 & \textbf{was} trûric an dem ringe hie.\\ 
30 & dâ reit er zuo, nû hœret wie:\\ 
\end{tabular}
\scriptsize
\line(1,0){75} \newline
G I O L M Q R Z Fr22 Fr39 Fr40 \newline
\line(1,0){75} \newline
\textbf{7} \textit{Initiale} O  \textbf{9} \textit{Initiale} G L Fr22  \textbf{19} \textit{Initiale} Z  \textbf{20} \textit{Capitulumzeichen} L  \textbf{21} \textit{Überschrift:} Aventiwer van Gawan wie in Gingrimursel vmb seins herren tot sprach an vmb champhes not Vergulacht im daz gebot I   $\cdot$ \textit{Initiale} I R  \textbf{25} \textit{Initiale} Fr40  \newline
\line(1,0){75} \newline
\textbf{1} gundrie la surziere G  $\cdot$ Gundrie lasuszier I  $\cdot$ Gvndrie de svrzier O  $\cdot$ Gvndrýe Lasuͯrzier L  $\cdot$ Kondrie lachsurzier M  $\cdot$ Kundrie lazurziere Q (Fr40)  $\cdot$ Kundrie lazurzarie R  $\cdot$ Kvndrie Lasurziere Z (Fr39)  $\cdot$ Gvndrie de svrzîer Fr22 \textbf{2} diu] di Fr40  $\cdot$ doch] och R  $\cdot$ diu fiere] fiere Fr40 \textbf{3} Waleis] waleys O wales L \textbf{4} waz] nu waz I (O) (L) (M) (Q) (R) (Fr22) (Fr39) (Fr40)  $\cdot$ half] hilfet O  $\cdot$ in] \textit{om.} Q  $\cdot$ küenes] keines R \textbf{5} bî] gein O (Q) (R) Fr39 (Fr40) \textbf{6} und] \textit{om.} G  $\cdot$ im was] was M waz im R \textbf{7} scham] ÷cham O  $\cdot$ obe] uff M \textbf{8} den] der Fr39  $\cdot$ het er] het O er hatte M \textbf{9} wan] \textit{om.} G \textbf{10} unde] Ze lone Vnd R  $\cdot$ ist doch] ist I R Fr39 ich auch Q  $\cdot$ der sêle] div selbe O \textbf{11} ob] an G vobin allen M  $\cdot$ rehter] ein gvͦbet O (L) (Q) (R) (Z) (Fr22) (Fr39) (Fr40) gevbit M \textbf{12} Kuneware] Gunwar I Kvnware O (Q) Geneware L Kunwaren M Cuͦnware R Kondwar Fr22 Cuͦneware Fr39 cunware Fr40  $\cdot$ daz êrste] des ersten R  $\cdot$ weinen] weine Q \textbf{13} Parzivalen] parzifaln I (Fr22) Fr39 Fr40 Barcifaln O parzifal L M partzifalen Q parczifal R parcifaln Z  $\cdot$ den] dem R  $\cdot$ balt] bal O \textbf{14} Gundrie] Gvndrie svrzir L [Kundri*]: Kundrilach surzier M Kundrie lazurziere Q (Fr40) Kundrie lasurcziere R Kundrie Svrtziere Z Gvndrie svrzîer Fr22 Kundrie lasurziere Fr39  $\cdot$ alsus] also Z \textbf{15} umbe] Ein Z  $\cdot$ alsus] also O L M Q R Z Fr22 Fr39 Fr40  $\cdot$ geschaf] geschafft Q beschaf Z \textbf{16} saf] schaff M \textbf{17} werden] werder O \textbf{18} Die man weinde muste schowen Z  $\cdot$ weinen] iamer I \textbf{19} Gundrie] Kvndrie O (Q) (R) Z Kondrie M  $\cdot$ ir] irre Q  $\cdot$ wer] ver Q \textbf{20} si] Dᵫ R (Z)  $\cdot$ reit den wec] reit ein wench O reiten weg L reit en wec M (Q) (R) (Z) (Fr22) (Fr40) \textbf{21} ein] Er Q \textbf{22} al] Als Q R (Fr40)  $\cdot$ harnasch was] harnasch daz was G harnaisch was so I (O) (L) (M) (Q) (R) (Z) (Fr40) \textbf{23} vuoze] den fvzzen Z  $\cdot$ des houbtes] das hobet R \textbf{24} mans] man R  $\cdot$ vür] von Q  $\cdot$ grôze] \textit{om.} O grosser Q  $\cdot$ jach] sach R \textbf{25} daz] \textit{om.} Q R Fr40 \textbf{27} sîn] \textit{om.} L Fr39  $\cdot$ ouch] \textit{om.} L M Q R Fr39 Fr40  $\cdot$ selbes] \textit{om.} I Q R Fr40 \textbf{28} manic] Do vant er O L (Q) (Fr39) Da vant er mannigen M Da vand er R Z (Fr40)  $\cdot$ maget] magt man O L (Q) Z Fr39 Fr40 man M  $\cdot$ unde] noch L Fr39 \textbf{29} was] \textit{om.} O L M Q R Z Fr39 Fr40  $\cdot$ hie] alhie Q \textbf{30} dâ] do I (O) (Q) (R) Fr39 \newline
\end{minipage}
\hspace{0.5cm}
\begin{minipage}[t]{0.5\linewidth}
\small
\begin{center}*T
\end{center}
\begin{tabular}{rl}
 & Kundrie Lasurziere,\\ 
 & diu unsüeze unde doch diu fiere,\\ 
 & den Waleis \textbf{sus} beswæret hât.\\ 
 & waz half in küenes herzen rât\\ 
5 & unde \textbf{wâriu} zuht bî manheit?\\ 
 & dannoch mêr im was \textbf{bereit}:\\ 
 & schame ob allen sînen siten.\\ 
 & den rehten valsch heter vermiten,\\ 
 & wan schame gît prîs ze lône\\ 
10 & unde ist doch der sêle krône.\\ 
 & scham ist ob siten \textbf{ein geüebet} uop.\\ 
 & Cunnewar daz êrste weinen huop,\\ 
 & daz Parcifaln, den degen balt,\\ 
 & Kundrie \textbf{alsus} beschalt\\ 
15 & \textbf{umb} \textbf{alsô} wunderlîch geschaf.\\ 
 & herzen jâmer ougen saf\\ 
 & gab maneger werden vrouwen.\\ 
 & \textbf{man muose hie weinen} schouwen.\\ 
 & \begin{large}K\end{large}undrie was \textbf{ir} trûrens wer.\\ 
20 & \textbf{diu} reit \textbf{enwec}. nû reit dort her\\ 
 & ein rîter, der truoc hôhen muot.\\ 
 & al sîn harnasch was \textbf{sô} guot\\ 
 & von\textbf{me vuoze} unz \textbf{ûf des houbetes dach},\\ 
 & daz mans vür \textbf{rîche} koste jach.\\ 
25 & sîn zimierde, \textbf{daz} was rîche.\\ 
 & gewâpent rîterlîche\\ 
 & was \textbf{sîn} ors unde sîn lîp.\\ 
 & \textbf{dô vander} maget unde wîp\\ 
 & trûric an dem ringe hie.\\ 
30 & dâ reit er zuo, nû hœret wie:\\ 
\end{tabular}
\scriptsize
\line(1,0){75} \newline
T U V W \newline
\line(1,0){75} \newline
\textbf{18} \textit{Initiale (319.18¹)} V  \textbf{19} \textit{Überschrift:} Hie kam ein rittere vnd beschalt her gawan vnd fordert in kampfs W   $\cdot$ \textit{Platz für Illustration ausgespart} W   $\cdot$ \textit{Initiale} T U W  \newline
\line(1,0){75} \newline
\textbf{1} Kvndrie Lasvrzîere T  $\cdot$ Kuͦndrie lasuͦrziere U  $\cdot$ Kundrie die sußier W \textbf{3} sus beswæret] sv́ besweret V sy suß beschworen W \textbf{6} mêr im] meretim W \textbf{7} schame] Schone W \textbf{8} den rehten valsch] Die rechte valscheit U  $\cdot$ heter] hat er W \textbf{10} doch] oͮch V \textbf{11} Schame ist ein U  $\cdot$ ob] ob allen W \textbf{12} Cunnewar] Cvnnewâr T Kuͦnnewar U Kvnneware V Kunnewar W \textbf{13} Parcifaln] parzifaln T parzifal V partzifal W \textbf{14} Kundrie] Kuͦndrie surziere U Kvndrie lasurziere V Kundrie surßier W  $\cdot$ alsus] sus W \textbf{15} geschaf] geschafft W \textbf{17} gab] Gab do W \textbf{18} \textit{nach 319.18:} Kvndrie sprach aber hie / Her kv́nig gehortent ir ie / Von kastel orgeluse sagen / Die mere wil ich nv́t verdagen / Do wonet drv́hvndert ritter guͦt / Vnde sehse vnde sehzig wol gemuͦt / Do hat bi im oͮch herre min / Jegelicher oͮch sine frv́ndin / Die edel ist vnde wunnebere / Do von sage ich v́ch die mere / Oͮch sint iegelichen svnder wan / Zwenzig ritter vndertan / Do velet nieman der dar ritet / Wil er er iustieret oder stritet / Wer ritterschaft suͦchen wil / Kvmet er dar er vindet ir vil / Gyflet do es svn sprach do zestvnt / Jch rite dar blip ich gesvnt (Einschub entspr. 'Conte du Graal', V. 4688-4700 und 4721-4722, und 'Élucidation', V. 419-425) V   $\cdot$ Die man weinende muͤste schowen V  $\cdot$ muose] mvese T  $\cdot$ weinen] gros wainen W \textbf{19} ir] do V \textbf{22} al sîn] Als in U Aller sein W \textbf{23} vonme] Von V  $\cdot$ unz] \textit{om.} U  $\cdot$ ûf des houbetes] an das haubt W \textbf{24} daz] Do W  $\cdot$ mans] man V \textbf{27} unde] vnd auch U (V) \textbf{28} maget] maget manne W \textbf{30} dâ] Do U V W \newline
\end{minipage}
\end{table}
\end{document}
