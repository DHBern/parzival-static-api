\documentclass[8pt,a4paper,notitlepage]{article}
\usepackage{fullpage}
\usepackage{ulem}
\usepackage{xltxtra}
\usepackage{datetime}
\renewcommand{\dateseparator}{.}
\dmyyyydate
\usepackage{fancyhdr}
\usepackage{ifthen}
\pagestyle{fancy}
\fancyhf{}
\renewcommand{\headrulewidth}{0pt}
\fancyfoot[L]{\ifthenelse{\value{page}=1}{\today, \currenttime{} Uhr}{}}
\begin{document}
\begin{table}[ht]
\begin{minipage}[t]{0.5\linewidth}
\small
\begin{center}*D
\end{center}
\begin{tabular}{rl}
\textbf{182} & wander vorhte des \textbf{orses} val.\\ 
 & \textbf{dô} \textbf{lasch} ouch anderhalp der schal.\\ 
 & die ritter truogen wider în\\ 
 & helme, schilde, \textbf{ir} \textbf{swerte} schîn\\ 
5 & unt sluzzen zuo \textbf{ir} porten.\\ 
 & \textbf{grœzer her} si vorhten.\\ 
 & Sus zôch hin \textbf{über} Parzival\\ 
 & unt kom geriten an ein wal,\\ 
 & dâ maneger sînen tôt \textbf{erkôs},\\ 
10 & \textbf{der} durch ritters prîs den lîp verlôs\\ 
 & vor der porte gein dem palas,\\ 
 & der hôch unt \textbf{wol} \textbf{gehêrt} was.\\ 
 & Einen rinc er an der porte vant,\\ 
 & \textbf{den ruort er vaste} mit der hant.\\ 
15 & sînes \textbf{rüefens} nam \textbf{dâ} niemen war,\\ 
 & wan ein juncvrouwe wol gevar.\\ 
 & \textbf{ûz} einem venster sach diu magt\\ 
 & den helt halden unverzagt.\\ 
 & \begin{large}D\end{large}iu schœne zühte rîche\\ 
20 & sprach: "sît ir vîentlîche\\ 
 & her komen, \textbf{hêrre}? daz ist ân nôt.\\ 
 & ân iuch man uns vil \textbf{hazzens} bôt\\ 
 & vome lande unt ûf dem mer,\\ 
 & zornec \textbf{unt} ellenthaftez her."\\ 
25 & \textbf{dô sprach er}: "vrouwe, hie \textbf{habt} ein man,\\ 
 & der iu dienet, ob \textbf{ich} kan.\\ 
 & iwer gruoz sol sîn \textbf{mîn} solt.\\ 
 & \textbf{ich bin} iu dienstlîche holt."\\ 
 & Dô gienc diu magt mit sinne\\ 
30 & vür die küneginne\\ 
\end{tabular}
\scriptsize
\line(1,0){75} \newline
D Fr15 \newline
\line(1,0){75} \newline
\textbf{7} \textit{Majuskel} D  \textbf{13} \textit{Majuskel} D  \textbf{19} \textit{Initiale} D  \textbf{29} \textit{Initiale} Fr15   $\cdot$ \textit{Majuskel} D  \newline
\line(1,0){75} \newline
\textbf{4} schilde] vnd Fr15 \textbf{5} sluzzen] slvͦgen Fr15 \textbf{7} Parzival] Partsival Fr15 \textbf{11} porte] porten Fr15  $\cdot$ dem] den Fr15 \textbf{12} hôch unt] \textit{om.} Fr15 \textbf{13} porte] porten Fr15 \textbf{24} zornec] zornlich Fr15 \newline
\end{minipage}
\hspace{0.5cm}
\begin{minipage}[t]{0.5\linewidth}
\small
\begin{center}*m
\end{center}
\begin{tabular}{rl}
 & wand er vorhte des \textbf{tiuvels} val.\\ 
 & \textbf{dô} \textbf{lasch} ouch anderhalp der schal.\\ 
 & die ritter truogen wider în\\ 
 & helme, schilte \textbf{und} \textbf{swerte} schîn\\ 
5 & und \dag vlühtic\dag  zuo \textbf{ir} porten.\\ 
 & \textbf{grôzer huote} si vorhten.\\ 
 & \dag in nâch\dag  hin \textbf{wider} Parcifal\\ 
 & und kam geriten an ein wal,\\ 
 & dâ maniger sînen tôt \textbf{erkôs},\\ 
10 & \textbf{der} durch ritters prîs den lîp verlôs\\ 
 & vor der porte gegen dem palas,\\ 
 & der hôch und \textbf{wol} \textbf{gezieret} was.\\ 
 & einen rinc er an der porten vant,\\ 
 & \textbf{den ruort er vaste} mit der hant.\\ 
15 & sînes \textbf{ruofes} nam niemen war,\\ 
 & wan ein juncvrouwe wol gevar.\\ 
 & \textbf{ûz} einem venster sach diu maget\\ 
 & den helt \textbf{dâ} halte\textit{n} unverzaget.\\ 
 & diu schœne zühte rîche\\ 
20 & sprach: "sît \textit{ir} vîentlîche\\ 
 & her komen, \textbf{hêr\textit{r}e}? d\textit{az} ist âne nôt.\\ 
 & âne iuch man uns vil \textbf{hazzes} bôt\\ 
 & von dem lande und ûf dem mer,\\ 
 & zornic ellenthaftez her."\\ 
25 & \textbf{er sprach}: "vrouwe, hie \textbf{habet} ein man,\\ 
 & der iu dienet, ob \textbf{er} kan.\\ 
 & iuwer gruoz sol sîn \textbf{mîn} solt.\\ 
 & \textbf{ich bin} iu dienestlîchen holt."\\ 
 & \begin{large}D\end{large}ô gie diu maget mit sinne\\ 
30 & vür die küniginne\\ 
\end{tabular}
\scriptsize
\line(1,0){75} \newline
m n o Fr69 \newline
\line(1,0){75} \newline
\textbf{29} \textit{Initiale} m n o  \newline
\line(1,0){75} \newline
\textbf{1} tiuvels] orses Fr69 \textbf{2} lasch] lac Fr69 \textbf{7} wider] v́ber n (o) \textbf{9} dâ] Do m n o \textbf{10} den] sin n o \textbf{12} gezieret] geeret n geheret o \textbf{13} porten] porte o \textbf{14} den] Der o \textbf{15} ruofes] ruͯffens n (o)  $\cdot$ nam] nam do n (o) \textbf{16} gevar] [gewar]: gefar o \textbf{18} dâ] do n o  $\cdot$ halten] haltent m \textbf{19} zühte] zuchten n \textbf{20} ir] \textit{om.} m  $\cdot$ vîentlîche] winecliche o \textbf{21} hêrre] herte m  $\cdot$ daz] do m \textbf{22} uns] vncz o \textbf{24} zornic] Zornig vnd Fr69 \textbf{25} habet] hap o babt Fr69 \textbf{29} gie] \textit{om.} n \newline
\end{minipage}
\end{table}
\newpage
\begin{table}[ht]
\begin{minipage}[t]{0.5\linewidth}
\small
\begin{center}*G
\end{center}
\begin{tabular}{rl}
 & wan er vorhte des \textbf{orses} val.\\ 
 & \textbf{dô} \textbf{erlasch} ouch anderhalp der schal.\\ 
 & die rîter truogen wider în\\ 
 & helm, schilte, \textbf{swertes} schîn\\ 
5 & unde sluzzen zuo \textbf{die} porten.\\ 
 & \textbf{grœzer her} si vorhten.\\ 
 & sus zôch hin \textbf{über} Parcival\\ 
 & unde kom geriten an ein wal,\\ 
 & dâ maniger sînen \textit{tôt} \textbf{\textit{e}r\textit{k}ôs}\\ 
10 & \textbf{unt} durch rîters prîs den lîp verlôs\\ 
 & vor der porte gein dem palas,\\ 
 & der hôch unde \textbf{gehêrt} was.\\ 
 & einen rinc er an der porten vant,\\ 
 & \textbf{den ruorter vaste} mit der hant.\\ 
15 & sînes \textbf{ruofenes} nam \textbf{dâ} niemen war,\\ 
 & wan ein juncvrouwe wol gevar.\\ 
 & \textbf{von} einem venster sach diu maget\\ 
 & den helt halden unverzaget.\\ 
 & diu schœne zühte rîche\\ 
20 & sprach: "sît ir vîentlîche\\ 
 & her komen? deist âne nôt.\\ 
 & âne iuch man uns vil \textbf{hazzes} bôt\\ 
 & vome lande unde ûf dem mer,\\ 
 & zornec ellenthaftez her."\\ 
25 & \textbf{dô sprach er}: "vrouwe, hie \textbf{halt} ein man,\\ 
 & der iu dienet, obe \textbf{ich} kan.\\ 
 & iwer gruoz sol sîn \textbf{mîn} solt.\\ 
 & \textbf{ich bin} iu dienstlîchen holt."\\ 
 & dô gie diu maget mit sinne\\ 
30 & vür die küniginne\\ 
\end{tabular}
\scriptsize
\line(1,0){75} \newline
G I O L M Q R Z Fr40 \newline
\line(1,0){75} \newline
\textbf{3} \textit{Initiale} I  \textbf{13} \textit{Initiale} M  \textbf{19} \textit{Initiale} I O L Q R Z Fr40  \newline
\line(1,0){75} \newline
\textbf{1} vorhte] vorht O (R) (Fr40) \textbf{2} dô] Da M Z  $\cdot$ ouch anderhalp] andehalb O \textbf{3} truogen] trúngeen Q \textbf{4} swertes] vnd swertes I (L) ir swertes O (M) vnd ir swerte Q (R) (Fr40) \textbf{5} sluzzen] slugen M slvzzen die Z \textbf{6} grœzer] Groz I \textbf{7} sus] Lust Q  $\cdot$ Parcival] parzival G (I) Parcifal O (L) (Z) parzeval M patzifal Q parczifal R parzifal Fr40 \textbf{8} an ein] in ein L am Z \textbf{9} dâ] Do O Q  $\cdot$ tôt erkôs] lip verlos G tot chos O (L) ende kosz Q (R) (Fr40) \textbf{10} den] der R  $\cdot$ lîp] leyt Q  $\cdot$ verlôs] [velos]: verlos G \textbf{11} Vorne engeyn deme palas M  $\cdot$ porte] porten I R Z \textbf{12} gehêrt] wol geheret O Z vil geherret L wol ge erit M (Fr40) wol gehebet Q wol geczieret R \textbf{13} er] \textit{om.} Q  $\cdot$ porten] porte R \textbf{14} den] Er rief vnd L  $\cdot$ ruorter] rvͦrt er O (R) (Z) (Fr40) ruͯrte L  $\cdot$ vaste] \textit{om.} L  $\cdot$ der] siner Z \textbf{15} ruofenes] ruffes Q  $\cdot$ dâ] do Q \textit{om.} R \textbf{16} \textit{Vers 182.16 fehlt} Q   $\cdot$ wan] Danne Z \textbf{17} von] Vz Z \textbf{18} halden] handeln Q \textbf{19} diu] ÷iv O \textbf{20} vîentlîche] wicliche I \textbf{21} deist] herre dest O (L) (M) Q (R) (Z) (Fr40)  $\cdot$ âne] ein O (M) vn L \textbf{22} âne] Nu M  $\cdot$ uns] \textit{om.} M  $\cdot$ hazzes] hazzens O \textbf{23} vome lande] Vff dem land R Von den landen O  $\cdot$ ûf] von O (L) M \textbf{24} zornec] Zornet Q \textbf{25} dô] Da M Z \textbf{26} der iu] Die M  $\cdot$ ich] er O L (M) R \textbf{27} sîn] sy M \textbf{29} dô] Da Z  $\cdot$ mit sinne] mýnne L mit sinden Q \newline
\end{minipage}
\hspace{0.5cm}
\begin{minipage}[t]{0.5\linewidth}
\small
\begin{center}*T
\end{center}
\begin{tabular}{rl}
 & wander vorhte des \textbf{orses} val.\\ 
 & \textbf{Nû} \textbf{erlasch} ouch anderhalp der schal.\\ 
 & die rîter truogen wider în\\ 
 & helme, schilte, \textbf{swerte} schîn\\ 
5 & unde sluzzen zuo \textbf{ir} porten.\\ 
 & \textbf{grœzer her} si vorhten.\\ 
 & Sus zôch hin \textbf{über} Parcifal\\ 
 & unde kom geriten an einen wal,\\ 
 & dâ maneger sînen tôt \textbf{kôs},\\ 
10 & \textbf{der} durch rîters prîs den lîp verlôs\\ 
 & vor der porte gegen dem palas,\\ 
 & der hôch unde \textbf{wol} \textbf{gehêret} was.\\ 
 & einen rinc er an der porten vant.\\ 
 & \textbf{er rief unde ruortin} mit der hant.\\ 
15 & Sînes \textbf{ruofens} nam \textbf{dâ} niemen war,\\ 
 & wan ein juncvrouwe wol gevar.\\ 
 & \textbf{ûz} einem venster sach diu maget\\ 
 & den helt halten unverzaget.\\ 
 & \begin{large}D\end{large}iu schœne zühte rîche\\ 
20 & sprach: "sît ir vîentlîche\\ 
 & her komen? dêst âne nôt.\\ 
 & âne iuch man uns vil \textbf{\textit{h}azzes} bôt\\ 
 & Vonme lande unde ûf dem mer,\\ 
 & zornic \textbf{unde} ellent\textit{ha}ftez her."\\ 
25 & \textbf{Dô sprach er}: "vrouwe, hie \textbf{haltet} ein man,\\ 
 & der iu dienet, ob \textbf{er} kan.\\ 
 & iuwer gruoz sol sîn \textbf{sîn} solt.\\ 
 & \textbf{er ist} iu dienstlîchen holt."\\ 
 & Dô gie diu maget mit sinne\\ 
30 & vür die küneginne\\ 
\end{tabular}
\scriptsize
\line(1,0){75} \newline
T U V W \newline
\line(1,0){75} \newline
\textbf{2} \textit{Majuskel} T  \textbf{7} \textit{Majuskel} T  \textbf{15} \textit{Majuskel} T  \textbf{19} \textit{Initiale} T U V W  \textbf{23} \textit{Majuskel} T  \textbf{25} \textit{Majuskel} T  \textbf{29} \textit{Majuskel} T  \newline
\line(1,0){75} \newline
\textbf{1} wander] Wan der U \textbf{2} Nû erlasch] Do [*]: lasch V \textbf{4} Helm schilt vnd schwertes schein W \textbf{6} vorhten] vuͦrten U \textbf{7} zôch] sich zoch W  $\cdot$ Parcifal] parzifal V partzifal W \textbf{8} einen] eine V \textbf{9} dâ] Do U V W  $\cdot$ sînen tôt] sein ende W \textbf{10} durch rîters prîs] vmb ritterschafft W \textbf{11} porte] porten U W \textbf{14} rief] rief [*]: vaste V \textbf{15} ruofens] rvͦfes V rieffens W  $\cdot$ dâ] do U V W \textbf{18} halten] [*]: do halten V \textbf{19} zühte] zv́hten V \textbf{21} dêst] duͦst U [*]: herre dast V \textbf{22} iuch] îv T  $\cdot$ hazzes] hahazzes T \textbf{23} ûf] von W \textbf{24} unde ellenthaftez] vnde ellentahftez T nothafftes W \textbf{25} er] die W \textbf{27} sîn solt] [m*]: min solt V mein solt W \textbf{28} er ist] [E*]: Jch bin V Ich bin W \newline
\end{minipage}
\end{table}
\end{document}
