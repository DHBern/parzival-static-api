\documentclass[8pt,a4paper,notitlepage]{article}
\usepackage{fullpage}
\usepackage{ulem}
\usepackage{xltxtra}
\usepackage{datetime}
\renewcommand{\dateseparator}{.}
\dmyyyydate
\usepackage{fancyhdr}
\usepackage{ifthen}
\pagestyle{fancy}
\fancyhf{}
\renewcommand{\headrulewidth}{0pt}
\fancyfoot[L]{\ifthenelse{\value{page}=1}{\today, \currenttime{} Uhr}{}}
\begin{document}
\begin{table}[ht]
\begin{minipage}[t]{0.5\linewidth}
\small
\begin{center}*D
\end{center}
\begin{tabular}{rl}
\textbf{430} & \begin{large}S\end{large}i wâren im \textbf{durch} sippe holt\\ 
 & unt dienden im ûf sînen solt.\\ 
 & werdecheit gab er ze lône\\ 
 & unt pflag ir anders schône.\\ 
5 & Gawan sprach zen kindelîn:\\ 
 & "wol iu, \textbf{süezen} mâge mîn,\\ 
 & mich dunket des, ir wolt mich klagen,\\ 
 & ob ich wære alhie erslagen."\\ 
 & Man moht in klage \textbf{trûwen} wol;\\ 
10 & si wâren \textbf{halt} sus \textbf{in} jâmers \textbf{dol}.\\ 
 & er sprach: "mir was umbe iuch vil leit.\\ 
 & wâ wârt ir, dô man mit mir streit?"\\ 
 & si sagtenz im, ir keiner louc:\\ 
 & "ein mûzersprinzelîn enpflouc\\ 
15 & uns, dô ir bî der künegîn\\ 
 & \textbf{sâzet. dâ liefe wir elliu hin.}"\\ 
 & Die dâ stuonden unde sâzen,\\ 
 & die merkens niht vergâzen,\\ 
 & die prüeveten, daz hêr Gawan\\ 
20 & wære ein \textbf{manlîch}, \textbf{höfsch} man.\\ 
 & urloubes er dô gerte,\\ 
 & des in der künec gewerte\\ 
 & unt daz volc al gemeine,\\ 
 & wan der lantgrâve al eine.\\ 
25 & die zwêne nam diu künegîn\\ 
 & unt Gawans junchêrrelîn.\\ 
 & si vuorte si, dâ ir pflâgen\\ 
 & juncvrouwen âne bâgen.\\ 
 & dô nam ir wol mit zühten war\\ 
30 & manec \textbf{juncvrouwe} wol gevar.\\ 
\end{tabular}
\scriptsize
\line(1,0){75} \newline
D Fr68 \newline
\line(1,0){75} \newline
\textbf{1} \textit{Initiale} D Fr68  \textbf{9} \textit{Majuskel} D  \textbf{17} \textit{Majuskel} D  \newline
\line(1,0){75} \newline
\textbf{5} zen] den Fr68 \textbf{8} wære alhie] hie were Fr68 \textbf{10} halt] ioh Fr68 \textbf{18} die merkens] merkens si Fr68 \textbf{19} die] si Fr68 \textbf{20} manlîch höfsch] houisch manlih Fr68 \textbf{22} gewerte] werte Fr68 \textbf{24} al] \textit{om.} Fr68 \newline
\end{minipage}
\hspace{0.5cm}
\begin{minipage}[t]{0.5\linewidth}
\small
\begin{center}*m
\end{center}
\begin{tabular}{rl}
 & si wâre\textit{n} ime \textbf{durch} sippe holt\\ 
 & und dienden ime ûf sînen solt.\\ 
 & werdicheit gap er ze lône\\ 
 & und pflac ir anders schône.\\ 
5 & \begin{large}G\end{large}awan sprach zen kindelîn:\\ 
 & "wol iu, \textbf{süezen} mâg\textit{e} mîn,\\ 
 & mich dunket des, ir wolt mich klagen,\\ 
 & obe ich wære a\textit{lhie erslag}en."\\ 
 & man m\textit{o}hte in klage \textbf{getriuwen} wol;\\ 
10 & si wâren \textbf{joch} sus \textbf{i\textit{n}} \textit{j}âmers \textbf{dol}.\\ 
 & er sprach: "mir was umb iuch vil leit.\\ 
 & wâ wâret ir, dô man mit mir streit?"\\ 
 & si sagetenz ime, ir keiner louc:\\ 
 & "ein mûzersprinzelîn enpflouc\\ 
15 & un\textit{s}, dô \textit{i}r bî der künigîn\\ 
 & \textbf{sâzet. d\textit{â} \textit{l}iefen wir alliu hie hin.}"\\ 
 & die dâ stuonden unde \textit{s}âzen,\\ 
 & die merkens niht vergâzen,\\ 
 & die brüefeten, daz hêr Gawan\\ 
20 & wære ei\textit{n} \textbf{\textit{m}anlîch} \textbf{hoves} man.\\ 
 & urloubes er dô gerte,\\ 
 & des in der künic gewerte\\ 
 & und daz volc al gemeine,\\ 
 & wanne der lantgrâve aleine.\\ 
25 & die zwêne nam diu künigîn\\ 
 & und Gawanes junchêrrelîn.\\ 
 & si vuorte \textit{si}, dâ \textit{ir} pflâgen\\ 
 & juncvrouwen âne bâgen.\\ 
 & dô nam ir wol mit zühten war\\ 
30 & manic \textbf{juncvrouw\textit{e}} wol gevar.\\ 
\end{tabular}
\scriptsize
\line(1,0){75} \newline
m n o \newline
\line(1,0){75} \newline
\textbf{5} \textit{Initiale} m   $\cdot$ \textit{Capitulumzeichen} n  \newline
\line(1,0){75} \newline
\textbf{1} wâren] worem m \textbf{2} und] Vnd sú n \textbf{3} er] er in o \textbf{5} zen] zuͯ dem n (o) \textbf{6} süezen mâge] suͯssen maget m suͯsse maget n sússe mage o \textbf{7} des] das o \textbf{8} alhie erslagen] ahczehen jar wol miden m \textbf{9} mohte] moͯhte m (n) \textbf{10} joch] ouch n (o)  $\cdot$ in jâmers] in in iamers m \textbf{11} umb] ẏm o \textbf{14} mûzersprinzelîn] músser sprintzeln n (o) \textbf{15} uns] Vnd m o  $\cdot$ ir] er m \textbf{16} sâzet] Sossen n o  $\cdot$ dâ liefen] do ir bẏ lieffen m do liessen n o  $\cdot$ alliu hie hin] alhin n o \textbf{17} dâ] do n o  $\cdot$ sâzen] assen m \textbf{18} merkens] marcketens n (o) \textbf{20} ein manlîch] ein hofflich vnd manlich m  $\cdot$ hoves man] hoff man n (o) \textbf{21} urloubes] Vrlop n (o) \textbf{22} des] Das n \textbf{25} zwêne] zwey o  $\cdot$ nam] nand o \textbf{26} Gawanes] gawans n o \textbf{27} vuorte si] furte ẏn m fuͦrte n  $\cdot$ dâ] do n o  $\cdot$ ir] sẏ m (n) \textbf{30} juncvrouwe] jungfrouwen m \newline
\end{minipage}
\end{table}
\newpage
\begin{table}[ht]
\begin{minipage}[t]{0.5\linewidth}
\small
\begin{center}*G
\end{center}
\begin{tabular}{rl}
 & si wâren im \textbf{umbe} sippe holt\\ 
 & unt dienden im ûf sînen solt.\\ 
 & werdicheit gap er ze lône\\ 
 & unde pflac ir anders schône.\\ 
5 & Gawan sprach zen kindelîn:\\ 
 & "\textbf{ô} wol iu, \textbf{lieben} mâge mîn,\\ 
 & mich \textit{dunket} des, ir wolt mich klagen,\\ 
 & obe ich wære al hie erslagen."\\ 
 & man mohte in klage \textbf{getrûwen} wol;\\ 
10 & si wâren \textbf{doch} sus \textbf{in} jâmers \textbf{dol}.\\ 
 & er sprach: "mir was umbe iuch vil leit.\\ 
 & wâ wârt ir, dô man mit mir streit?"\\ 
 & si sagetenz im, ir deheiner louc:\\ 
 & "ein mûzersprinzelîn enpflouc\\ 
15 & uns, dô ir bî der künigîn\\ 
 & \textbf{sâzet. dâ liefen wir elliu hin}."\\ 
 & \begin{large}D\end{large}ie dâ stuonden unde sâzen,\\ 
 & die merkens niht vergâzen,\\ 
 & d\textit{ie} prüeveten, daz hêr Gawan\\ 
20 & wære ein \textbf{manlîch}, \textbf{höfsch} man.\\ 
 & urloubes er dô gerte,\\ 
 & des in der künic gewerte\\ 
 & unt daz volc algemeine,\\ 
 & wan der lantgrâve \textit{al}eine.\\ 
25 & die zwêne \textit{n}a\textit{m d}iu künigîn\\ 
 & unde Gawanes junchêrrelîn.\\ 
 & si vuorte si, dâ ir pflâgen\\ 
 & juncvrouwen âne bâgen.\\ 
 & dô nam ir wol mit zühten war\\ 
30 & manic \textbf{vrouwe} wol gevar.\\ 
\end{tabular}
\scriptsize
\line(1,0){75} \newline
G I O L M Q R Z Fr21 \newline
\line(1,0){75} \newline
\textbf{1} \textit{Initiale} I O L M Fr21   $\cdot$ \textit{Capitulumzeichen} R  \textbf{15} \textit{Initiale} I  \textbf{17} \textit{Initiale} G  \newline
\line(1,0){75} \newline
\textbf{1} si] ÷i O  $\cdot$ umbe] dvrch O (L) (M) (Q) (R) (Z) (Fr21) \textbf{2} ûf] vmb I  $\cdot$ solt] sol Z \textbf{3} er] er in I R er im Z \textbf{5} Gawan] Gawain R  $\cdot$ zen] zdem O (Q) \textbf{6} ô] \textit{om.} O L M Q R Z Fr21  $\cdot$ lieben] liebe I svͦze O (L) (Q) Fr21 suszin M (R) (Z) \textbf{7} dunket] \textit{om.} G  $\cdot$ des] daz R  $\cdot$ wolt mich] woldet mich I mich woͯlten R  $\cdot$ klagen] clag Q \textbf{8} wære al hie] hie wer R \textbf{9} in] ev I  $\cdot$ klage] clagen M clagen in R  $\cdot$ getrûwen] [trowen]: trvwen O truwen M (R) (Fr21) \textbf{10} doch sus] halt svs O (M) Fr21 suͯs halt L halt ausz Q sust R (Z)  $\cdot$ jâmers] iammer M  $\cdot$ dol] [vol]: dol G vol L \textbf{11} umbe iuch] ouch L  $\cdot$ vil] \textit{om.} M \textbf{12} dô] da I O M Z \textbf{13} sagetenz im] sagten yms Q  $\cdot$ deheiner] dehainz I \textbf{14} mûzersprinzelîn] gemvͦztez sprinzelin O mvͯszer srintzelin L mussensprincelin Q músser sperwer R  $\cdot$ enpflouc] in enphlauͦc I (R) ern flouc M \textbf{15} dô] da M Z \textbf{16} sâzet] Sassen R  $\cdot$ dâ] do O L Q R Fr21  $\cdot$ liefen] lifft Q lief Z  $\cdot$ elliu] alle do Q alle R \textbf{17} Die] \textit{om.} L  $\cdot$ dâ] do Q R \textit{om.} Z \textbf{18} die merkens] Merkens sy R \textbf{19} die] do G  $\cdot$ prüeveten] brvͦvent O  $\cdot$ hêr] er M \textbf{20} manlîch höfsch man] harte hvbscher man O (Fr21) harte hovisch man L (M) harte hubsman Q hart hofflich man R \textbf{21} urloubes] Vrlob R  $\cdot$ dô] \textit{om.} M da Z \textbf{22} des in der künic] Der kung in des R \textbf{24} lantgrâve] margraff R  $\cdot$ aleine] eine G \textbf{25} nam diu] man vnde diu G [man]: nam die L \textbf{26} Gawanes] Gawans I O (M) (Q) (Z) Fr21 Gawansz L die Iungen R \textbf{27} vuorte si] furtes Q fvrtens Fr21  $\cdot$ dâ] do O Q \textbf{28} juncvrouwen] juncherren I  $\cdot$ bâgen] wagen Q \textbf{29} dô] Da M Z  $\cdot$ nam] nam man R  $\cdot$ ir] er Z  $\cdot$ mit] \textit{om.} M \textbf{30} vrouwe] ivnchfrowe O (L) (M) (Q) (R) ivncfrowen Z \newline
\end{minipage}
\hspace{0.5cm}
\begin{minipage}[t]{0.5\linewidth}
\small
\begin{center}*T
\end{center}
\begin{tabular}{rl}
 & si wâren im \textbf{durch} sippe holt\\ 
 & unde dienden im ûf sînen solt.\\ 
 & werdecheit gap er \textbf{in} ze lône\\ 
 & unde pflac ir anders schône.\\ 
5 & \begin{large}G\end{large}awan sprach zen kindelîn:\\ 
 & "wol iu, \textbf{süezen} mâge mîn,\\ 
 & mich dunket des, ir wolt mich klagen,\\ 
 & ob ich wære alhie erslagen."\\ 
 & Man moht in klage \textbf{getriuwen} wol;\\ 
10 & si wâren \textbf{halt} sus jâmers \textbf{vol}.\\ 
 & Er sprach: "mir was umbe iuch vil leit.\\ 
 & wâ wâret ir, dô man \textit{mit} mir streit?"\\ 
 & Si sagetenz im, ir deheiner louc:\\ 
 & "ein mûzersprinz\textit{e}lîn enpflouc\\ 
15 & uns, dô ir \textbf{wâret} bî der künegîn."\\ 
 & "\textbf{Wol ir süezen mâge mîn!}"\\ 
 & Die dâ stuonden unde sâzen,\\ 
 & die merkens niht vergâzen,\\ 
 & die  prüeveten, daz hêr Gawan\\ 
20 & wære ein \textbf{harte} \textbf{hövescher} man.\\ 
 & Urloubes er dô gerte,\\ 
 & des in der künec gewerte\\ 
 & unde daz volc algemeine,\\ 
 & wan der lantgrâve aleine.\\ 
25 & die zwêne nam diu künegîn\\ 
 & unde Gawans junchêrrelîn.\\ 
 & si vuorte si, dâ ir pflâgen\\ 
 & juncvrouwen âne bâgen.\\ 
 & Dô nam ir wol mit zühten war\\ 
30 & manec \textbf{juncvrouwe} wol gevar.\\ 
\end{tabular}
\scriptsize
\line(1,0){75} \newline
T U V W \newline
\line(1,0){75} \newline
\textbf{5} \textit{Initiale} T U  \textbf{9} \textit{Majuskel} T  \textbf{11} \textit{Initiale} W   $\cdot$ \textit{Majuskel} T  \textbf{13} \textit{Majuskel} T  \textbf{16} \textit{Majuskel} T  \textbf{17} \textit{Majuskel} T  \textbf{21} \textit{Majuskel} T  \textbf{29} \textit{Majuskel} T  \newline
\line(1,0){75} \newline
\textbf{3} in] \textit{om.} V W \textbf{5} zen] [ze*]: zen V \textbf{6} iu] \textit{om.} U  $\cdot$ süezen] suͤsse W \textbf{7} ir] ich U  $\cdot$ wolt] woltent V \textbf{9} moht] moͤcht W \textbf{10} halt sus] alt suͦs U [*]: ioch svs V suß hart W  $\cdot$ jâmers vol] in iamers dol U (V) W \textbf{11} iuch] îv T  $\cdot$ vil] \textit{om.} W \textbf{12} mit] \textit{om.} T \textbf{13} ir] \textit{om.} W \textbf{14} mûzersprinzerlîn] mvzer sprinzerlin T [mv́]: mv́sser sprinzerlin V \textbf{15} ir wâret] wart ir U ir V W \textbf{16} Sazent do liefent wir alle hin V (W) \textbf{17} Die dâ] [*]: [D* d*e]: Die do V Die do W \textbf{22} künec] kv́nigin do V \textbf{23} algemeine] alle gemeine U do algemeine V \textbf{25} \textit{Versfolge 430.26-25} W  \textbf{27} vuorte] vuͦrten U  $\cdot$ si dâ] es do W \textbf{29} zühten] zuchte W \newline
\end{minipage}
\end{table}
\end{document}
