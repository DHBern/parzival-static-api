\documentclass[8pt,a4paper,notitlepage]{article}
\usepackage{fullpage}
\usepackage{ulem}
\usepackage{xltxtra}
\usepackage{datetime}
\renewcommand{\dateseparator}{.}
\dmyyyydate
\usepackage{fancyhdr}
\usepackage{ifthen}
\pagestyle{fancy}
\fancyhf{}
\renewcommand{\headrulewidth}{0pt}
\fancyfoot[L]{\ifthenelse{\value{page}=1}{\today, \currenttime{} Uhr}{}}
\begin{document}
\begin{table}[ht]
\begin{minipage}[t]{0.5\linewidth}
\small
\begin{center}*D
\end{center}
\begin{tabular}{rl}
\textbf{367} & \begin{large}S\end{large}us bin ich ûf der strâzen\\ 
 & oder ich muoz den lîp dâ lâzen."\\ 
 & daz was Lyppauten ein herzeleit.\\ 
 & er sprach: "hêrre, durch iwer werdecheit\\ 
5 & unt durch iwerre zühte hulde\\ 
 & \textbf{sô} vernemt mîn unschulde.\\ 
 & Ich hân zwô tohter, die mir sint\\ 
 & liep, wan si sint mîniu kint.\\ 
 & swaz mir got \textbf{hât an den} gegeben,\\ 
10 & dâ wil ich \textbf{bî} mit vreuden leben.\\ 
 & \textbf{ôwol} mich, daz ich ie gewan\\ 
 & kumber, den ich von in hân!\\ 
 & den treit iedoch diu eine\\ 
 & mit mir al gemeine.\\ 
15 & ungelîch ist diu gesellecheit.\\ 
 & \textbf{mîn} \textbf{hêrre}, ir tuot mit minnen leit\\ 
 & unt mir mit unminne,\\ 
 & als ich \textbf{michs} versinne,\\ 
 & mîn \textbf{hêrre} mir gewalt wil tuon,\\ 
20 & durch daz ich hân decheinen sun.\\ 
 & mir sulen \textbf{ouch} tohter lieber sîn.\\ 
 & waz \textbf{denne}, ob ich\textbf{s} \textbf{nû} lîde pîn?\\ 
 & den wil ich mir ze sælden zeln.\\ 
 & swer sol mit \textbf{sîner tohter} weln,\\ 
25 & swie ir verboten sî daz swert,\\ 
 & ir wer ist anders als wert:\\ 
 & si erwirbet im kiuscheclîche\\ 
 & einen sun \textbf{vil} ellens rîche.\\ 
 & des selben ich gedingen hân."\\ 
30 & "nû \textbf{gewers} iuch got", sprach Gawan.\\ 
\end{tabular}
\scriptsize
\line(1,0){75} \newline
D Fr4 \newline
\line(1,0){75} \newline
\textbf{1} \textit{Initiale} D Fr4  \textbf{7} \textit{Majuskel} D  \newline
\line(1,0){75} \newline
\textbf{3} Lyppauten] Lyppaoten D lippaothe Fr4 \textbf{18} als ich michs] alsich mich \textit{nachträglich korrigiert zu:} alsichs mich Fr4 \textbf{19} mir gewalt wil] wil mir gewalt Fr4 \textbf{20} wan ich ne han nicheinin sun Fr4 \textbf{27} im] ir Fr4 \textbf{28} vil ellens] ellins vil Fr4 \textbf{29} gedingen] gedinge Fr4 \newline
\end{minipage}
\hspace{0.5cm}
\begin{minipage}[t]{0.5\linewidth}
\small
\begin{center}*m
\end{center}
\begin{tabular}{rl}
 & sus bin ich ûf der strâzen\\ 
 & oder ich muoz den lîp dâ lâzen."\\ 
 & daz was Lippo\textit{u}t ein herzeleit.\\ 
 & er sprach: "hêrre, durch iuwer werdicheit\\ 
5 & und durch iuwer zühte hulde\\ 
 & \textbf{sô} vernemet mîne unschulde.\\ 
 & ich hân zwô tohter, die mi\textit{r} sint\\ 
 & lie\textit{p}, wanne si sint mîniu kint.\\ 
 & waz mir got \textbf{an in hât} ge\textit{b}en,\\ 
10 & d\textit{â} wil ich mit vröuden leben.\\ 
 & \textbf{ouch} \textbf{wol} mich, daz ich ie gewan\\ 
 & kumber, den ich von in hân!\\ 
 & den treit iedoch diu eine\\ 
 & mit mi\textit{r} al gemeine.\\ 
15 & ungelîch ist diu gesellecheit.\\ 
 & \textbf{ir}, \textbf{hêrre}, ir tuot mit minne le\textit{i}t\\ 
 & und mir mit unminne,\\ 
 & als ich\textbf{s mich} versinne,\\ 
 & mîn \textbf{hêrre} mir gewalt wil tuon,\\ 
20 & durch daz \textit{ich} hân de\textit{k}einen sun.\\ 
 & mir \textit{s}ullen \textbf{ouch} tohter lieber sîn.\\ 
 & waz \textbf{danne}, ob ich \textbf{nû} lîde pîn?\\ 
 & den wil ich mir ze sælden zeln.\\ 
 & wer sol mit \textbf{sînen tohteren} weln,\\ 
25 & wie ir verboten sî daz swert,\\ 
 & ir wer ist anders alsô wert:\\ 
 & si erwirbet ime kiuscheclîche\\ 
 & einen sun \textbf{vil} ellens rîche.\\ 
 & des selben ich gedingen hân."\\ 
30 & "nû \textbf{gewers} iuch got", sprach Gawan.\\ 
\end{tabular}
\scriptsize
\line(1,0){75} \newline
m n o \newline
\line(1,0){75} \newline
\newline
\line(1,0){75} \newline
\textbf{2} dâ] do n o \textbf{3} Lippout] lippoat m lippaote n lipoote o \textbf{5} zühte] \textit{om.} n \textbf{6} sô] Sie o \textbf{7} mir] min m \textbf{8} liep] Lie m \textbf{9} in] den n o  $\cdot$ geben] gegen m \textbf{10} dâ] Do m n o  $\cdot$ mit] bẏ mit o \textbf{13} treit] reit n streit o  $\cdot$ diu] nie o \textbf{14} mir] mit m \textbf{15} diu gesellecheit] ir selickeit n (o) \textbf{16} \textit{Versdoppelung (²o); Lesarten des vorausgehenden Verses mit ¹o bezeichnet} o   $\cdot$ ir] Mir n o  $\cdot$ mit] mir \textsuperscript{2}\hspace{-1.3mm} o  $\cdot$ leit] let m \textbf{17} mit] herre min n (o) \textbf{18} ichs mich] ich michs n o \textbf{20} ich] nie m  $\cdot$ dekeinen] decke einen m do keinen n \textbf{21} sullen] ssüllen m \textbf{22} ob] >ob< o \textbf{23} sælden] selde o  $\cdot$ zeln] [pflegen]: zeln n \textbf{25} verboten] verbatten o \textbf{30} iuch got sprach] sprach got o \newline
\end{minipage}
\end{table}
\newpage
\begin{table}[ht]
\begin{minipage}[t]{0.5\linewidth}
\small
\begin{center}*G
\end{center}
\begin{tabular}{rl}
 & \hspace*{-.7em}\big| ode ich muoz den lîp dâ lâzen,\\ 
 & \hspace*{-.7em}\big| sus bin ich ûf der strâzen."\\ 
 & daz was Libaute ein herzeleit.\\ 
 & er sprach: "hêrre, durch iwer werdicheit\\ 
5 & unt durch iwer zühte hulde\\ 
 & vernemet mîn unschulde.\\ 
 & \begin{large}I\end{large}ch hân zwô tohter, die mir sint\\ 
 & liep, wan si sint mîniu kint.\\ 
 & swaz mir got \textbf{hât an in} gegeben,\\ 
10 & dâ wil ich \textbf{bî} mit vröuden leben.\\ 
 & \textbf{wol} mich, daz ich ie gewan\\ 
 & kumber, den ich von in hân!\\ 
 & den treit iedoch diu eine\\ 
 & mit mir algemeine.\\ 
15 & ungelîch ist diu gesellicheit.\\ 
 & \textbf{mîn} \textbf{hêrre}, ir tuot mit minnen leit\\ 
 & unde mir mit unminne,\\ 
 & als ich \textbf{mich} versinne,\\ 
 & mîn \textbf{hêrre} mir gewalt wil tuon,\\ 
20 & durch daz ich hân deheinen sun.\\ 
 & mir sulen \textbf{doch} tohter lieber sîn.\\ 
 & waz \textbf{dâr umbe}, obe ich \textbf{es} \textbf{nû} lîde pîn?\\ 
 & den wil ich mir ze sælden zelen.\\ 
 & swer sol mit \textbf{sîner tohter} welen,\\ 
25 & swie ir verboten sî daz swert,\\ 
 & ir wer ist anders als wert:\\ 
 & si erwirbt im kiuschlîche\\ 
 & einen sun \textbf{vil} ellens rîche.\\ 
 & des selben ich gedingen hân."\\ 
30 & "nû \textbf{wers} iuch got", sprach Gawan.\\ 
\end{tabular}
\scriptsize
\line(1,0){75} \newline
G I O L M Q R Z Fr21 Fr38 \newline
\line(1,0){75} \newline
\textbf{3} \textit{Initiale} I L Z Fr21   $\cdot$ \textit{Capitulumzeichen} R  \textbf{7} \textit{Initiale} G  \textbf{15} \textit{Initiale} I  \newline
\line(1,0){75} \newline
\textbf{2} \textit{Versfolge 367.1-2} O L Q R Z Fr21   $\cdot$ dâ] \textit{om.} M do Q \textbf{1} der] den M \textbf{3} Libaute] liabut I Lybavt O (Z) Libavt L (M) lybaut Q Lybant R Libovt Fr21 \textbf{6} vernemet] So vernemt O (L) (M) (Q) (R) Z (Fr21) :::mit Fr38 \textbf{8} mîniu] mine R \textbf{9} swaz] Waz L (M) (Q) (R)  $\cdot$ got hât] hat got Fr38  $\cdot$ an in] anden O (Z) (Fr21) an dem L ande\%- M (Q) (Fr38)  $\cdot$ gegeben] geben R \textbf{10} dâ] Do Q  $\cdot$ bî mit] mit In in R \textbf{11} wol] Owol L (Q) (R) (Z) So wol M Ow:l Fr21  $\cdot$ ich ie] ich sy y M ye R (Z) \textbf{12} kumber] Den Kumber R \textbf{13} iedoch] doch R \textbf{15} diu] die Fr21  $\cdot$ gesellicheit] geselheit R \textbf{16} ir tuot] tuͦt Jr R  $\cdot$ mit] mir Z  $\cdot$ minnen] minne O (L) (Q) R Fr21 (Fr38) \textbf{19} mîn] mir I  $\cdot$ mir gewalt wil] wil mir gewalt I O L Fr38 gewalt wil R gewalt mir wil Z \textbf{20} durch daz ich] Dar durch R \textbf{21} Mir sond die tochtren lieber sin R  $\cdot$ mir] wir I  $\cdot$ doch] \textit{om.} I \textbf{22} Was darumb Liden ich pin R  $\cdot$ es nû] sin I nv O L (M) (Q) (Z) Fr38 \textbf{24} swer] Wer L M Q R \textbf{25} swie] Wie L (M) Q R \textbf{26} anders] ander R  $\cdot$ als] aber I so M \textbf{27} im] ir I O \textit{om.} L Jn R  $\cdot$ kiuschlîche] kúnstlicher R \textbf{28} vil] \textit{om.} M  $\cdot$ ellens] \textit{om.} I eren Q ellen Fr21 \textbf{29} selben] selbe Q  $\cdot$ gedingen] gedinge M Q gedigen Fr21 \textbf{30} wers] gewers I L M Q gewer sin Z  $\cdot$ Gawan] gaban Q her Gawan Fr21 \newline
\end{minipage}
\hspace{0.5cm}
\begin{minipage}[t]{0.5\linewidth}
\small
\begin{center}*T
\end{center}
\begin{tabular}{rl}
 & \hspace*{-.7em}\big| oder ich muoz den lîp dâ lâzen,\\ 
 & \hspace*{-.7em}\big| sus bin ich ûf der strâzen."\\ 
 & \begin{large}D\end{large}az was Lybaut ein herzeleit.\\ 
 & er sprach: "hêrre, durch iuwer werdecheit\\ 
5 & unde durch iuwer zühte hulde\\ 
 & \textbf{sô} vernemet mîne unschulde.\\ 
 & ich hân zwô tohtere, die mir sint\\ 
 & liep, wande si sint mîniu kint.\\ 
 & \multicolumn{1}{l}{ - - - }\\ 
10 & \multicolumn{1}{l}{ - - - }\\ 
 & \textbf{ôwol} mich, daz ich ie gewan\\ 
 & kumber, den ich von in hân!\\ 
 & den treit iedoch diu eine\\ 
 & mit mir algemeine.\\ 
15 & unglîch ist di\textit{u} gesellecheit.\\ 
 & \textbf{ir} \textbf{herze} ir tuot mit mi\textit{nn}e leit\\ 
 & unde mir mit unminne,\\ 
 & als ich \textbf{mich} versinne,\\ 
 & mîn \textbf{herze} mir gewalt wil tuon,\\ 
20 & durch \textit{daz} ich hân deheinen suon.\\ 
 & mir suln \textbf{doch} tohtere lieber sîn.\\ 
 & waz \textbf{dâr umbe}, ob ich\textbf{s} lîde pîn?\\ 
 & den wil ich mir ze sælden zeln.\\ 
 & swer sol mit \textbf{sîner tohter} weln,\\ 
25 & swie ir verboten sî daz swert,\\ 
 & ir wer ist anders alse wert:\\ 
 & si erwirbet im kiuschlîche\\ 
 & einen sun ellens rîche.\\ 
 & des selben ich gedinge hân."\\ 
30 & "Nû \textbf{gewers} iuch got", sprach Gawan.\\ 
\end{tabular}
\scriptsize
\line(1,0){75} \newline
T V W \newline
\line(1,0){75} \newline
\textbf{3} \textit{Initiale} T  \textbf{30} \textit{Majuskel} T  \newline
\line(1,0){75} \newline
\textbf{2} dâ] do W \textbf{3} Lybaut] lẏbaut V lybout W \textbf{4} durch] vmb W \textbf{5} iuwer] v́werre V \textbf{6} vernemet] nement W \textbf{8} mîniu] minne W \textbf{9} \textit{Die Verse 367.9-10 fehlen} T W   $\cdot$ Swaz mir got hat an den gegeben V \textbf{10} Da wil ich bi mit vroͤiden leben V \textbf{11} mich] mir V \textbf{13} iedoch] doch W \textbf{14} mir] ir W \textbf{15} ist] ist doch W  $\cdot$ diu] die T \textbf{16} ir herze] Min herre V Ir herre W  $\cdot$ minne] mime T \textbf{20} daz] \textit{om.} T  $\cdot$ hân] [en*]: enhan V \textbf{21} doch] oͮch V die W \textbf{22} dâr umbe] [*]: danne V  $\cdot$ ichs] [*]: ich ez nv V ich des W \textbf{25} swie] We W \textbf{26} alse] alleß W \textbf{28} ellens] vil ellens V der ist ellendß W \textbf{30} gewers iuch] gewers iv T gewer úchs W \newline
\end{minipage}
\end{table}
\end{document}
