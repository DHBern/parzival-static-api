\documentclass[8pt,a4paper,notitlepage]{article}
\usepackage{fullpage}
\usepackage{ulem}
\usepackage{xltxtra}
\usepackage{datetime}
\renewcommand{\dateseparator}{.}
\dmyyyydate
\usepackage{fancyhdr}
\usepackage{ifthen}
\pagestyle{fancy}
\fancyhf{}
\renewcommand{\headrulewidth}{0pt}
\fancyfoot[L]{\ifthenelse{\value{page}=1}{\today, \currenttime{} Uhr}{}}
\begin{document}
\begin{table}[ht]
\begin{minipage}[t]{0.5\linewidth}
\small
\begin{center}*D
\end{center}
\begin{tabular}{rl}
\textbf{557} & \begin{large}D\end{large}er wirt sprach mit triwen:\\ 
 & "hêrre, sô muoz mich riwen,\\ 
 & daz iuch des vrâgens niht bevilt.\\ 
 & ich wil iu lîhen einen schilt;\\ 
5 & nû wâpent iuch ûf einen strît.\\ 
 & \textbf{ze} Terre Marvale ir sît.\\ 
 & Lit Marvale ist hie.\\ 
 & hêrre, \textit{ez} wart versuochet nie\\ 
 & \textbf{ûf} Schastel Marvale diu nôt.\\ 
10 & iwer leben wil in den tôt.\\ 
 & Ist iu âventiure bekant,\\ 
 & swaz ie gestreit iwer hant,\\ 
 & daz was noch gar ein kindes spil.\\ 
 & \textbf{nû} \textbf{næhent iu} riubæriu zil."\\ 
15 & Gawan sprach: "mir wære leit,\\ 
 & ob \textbf{mîn} gemach ân arbeit\\ 
 & von disen vrouwen hinnen rite,\\ 
 & ich \textbf{en}versuocht ê baz ir site.\\ 
 & ich hân \textbf{ouch} ê von in vernomen.\\ 
20 & sît ich sô nâhe \textbf{nû} bin komen,\\ 
 & \textbf{mich ensol des} niht betrâgen,\\ 
 & ich \textbf{en}\textbf{welle}z durch si wâgen."\\ 
 & Der wirt mit triwen klagete.\\ 
 & sîme gaste er \textbf{dô} sagete:\\ 
25 & "aller kumber ist ein \textbf{niht},\\ 
 & wan dem ze \textbf{lîdene} geschiht\\ 
 & disiu âventiure,\\ 
 & \textbf{diu} \textbf{ist} \textbf{scharpf unt} ungehiure,\\ 
 & vür wâr \textbf{und} âne liegen;\\ 
30 & hêrre, i\textbf{ne} kan niht triegen."\\ 
\end{tabular}
\scriptsize
\line(1,0){75} \newline
D \newline
\line(1,0){75} \newline
\textbf{1} \textit{Initiale} D  \textbf{11} \textit{Majuskel} D  \textbf{23} \textit{Majuskel} D  \newline
\line(1,0){75} \newline
\textbf{7} Lit Marvale] [*]: Lit Marvale D \textbf{8} ez] \textit{om.} D \textbf{9} Schastel Marvale] Scastel marvale D \newline
\end{minipage}
\hspace{0.5cm}
\begin{minipage}[t]{0.5\linewidth}
\small
\begin{center}*m
\end{center}
\begin{tabular}{rl}
 & \begin{large}D\end{large}er wirt sprach mit triuwen:\\ 
 & "hêrre, sô muoz mich riuwen,\\ 
 & daz iuch des vrâgens niht bevilt.\\ 
 & ich wil iu lîhen einen schilt;\\ 
5 & nû wâpent iuch ûf einen strît.\\ 
 & \textbf{zuo} Terre Mar\textit{ve}ile ir sît.\\ 
 & L\textit{e}t Mar\textit{ve}ile ist \textbf{ouch} hie.\\ 
 & hêrre, ez wart versuochet nie\\ 
 & \textbf{ûf} S\textit{ch}ah\textit{t}el Mar\textit{ve}ile diu nôt.\\ 
10 & iuwer leben wil in de\textit{n} tôt.\\ 
 & ist iu âventiur bekant,\\ 
 & waz ie gestreit iuwer hant,\\ 
 & daz was noch gar ein kindes spil.\\ 
 & \textbf{nû} \textbf{nâhe\textit{n}t iu} riuwe\textit{bæ}r\textit{iu} zil."\\ 
15 & Gawan sprach: "mir wær leit,\\ 
 & ob \textbf{mîn} gemach âne arbeit\\ 
 & von disen vrouwen hinnen rite,\\ 
 & i\textit{ch} versuochte ê baz ir site.\\ 
 & ich hab \textbf{ouch} ê von i\textit{n} vernomen.\\ 
20 & sît ich sô nâhen bin komen,\\ 
 & \textbf{sô ensol mich} niht betrâgen,\\ 
 & ich \textbf{wi\textit{l}} \textit{e}z durch si wâgen."\\ 
 & der wirt mit triuwen klagte.\\ 
 & sînem gast er \textbf{dô} sagte:\\ 
25 & "alle\textit{r} kumber ist ein \textbf{wiht},\\ 
 & wan dem zuo \textbf{lî\textit{d}ende} geschiht\\ 
 & disiu âventiure\\ 
 & \textbf{ist} ungehiure,\\ 
 & vür wâr \textbf{und} âne liegen;\\ 
30 & hêrre, ich kan niht triegen."\\ 
\end{tabular}
\scriptsize
\line(1,0){75} \newline
m n o \newline
\line(1,0){75} \newline
\textbf{1} \textit{Initiale} m   $\cdot$ \textit{Capitulumzeichen} n  \newline
\line(1,0){75} \newline
\textbf{4} einen] eẏne o \textbf{6} Terre] tirre ir o  $\cdot$ Marveile] marnaile m n o \textbf{7} Let Marveile] Lot marnaile m n Lot marnale o \textbf{9} Schahtel Marveile] stahel marnaile m scahel marnaile n o \textbf{10} den] dent m \textbf{13} noch] ouch n \textbf{14} nâhent] nahet m (n) o  $\cdot$ riuwebæriu] ruwerei m ruweberi n o \textbf{18} ich] Jr m \textbf{19} in] ẏm m \textbf{20} bin] nuͯ bin n (o) \textbf{22} wil ez] wil wi es m \textbf{25} aller] Allen m o \textbf{26} lîdende] libende m \textbf{27} \textit{Verse 557.27-28 kontrahiert zu:} Dise aufentuͯr ist vngehure m (o)   $\cdot$ Dise grosse Offentúre n \textbf{28} Die ist gar vngehúre n \newline
\end{minipage}
\end{table}
\newpage
\begin{table}[ht]
\begin{minipage}[t]{0.5\linewidth}
\small
\begin{center}*G
\end{center}
\begin{tabular}{rl}
 & \begin{large}D\end{large}er wirt sprach mit triuwen:\\ 
 & "hêrre, sô muoz mich riuwen,\\ 
 & daz iuch des vrâgens niht bevilt.\\ 
 & ich wil iu lîhen einen schilt;\\ 
5 & nû wâpent iuch ûf einen strît.\\ 
 & \textbf{ze} Terre Marveile ir sît.\\ 
 & Let Marveile ist hie.\\ 
 & hêrre, ez wart versuochet nie\\ 
 & \textbf{ûf} Tschastel Marveile diu nôt.\\ 
10 & iuwer leben wil in den tôt.\\ 
 & ist iu âventiure bekant,\\ 
 & swaz ie gestreit iuwer hant,\\ 
 & daz was noch gar ein kindes spil.\\ 
 & \textbf{nû} \textbf{nâhent iu} riuwebæriu zil."\\ 
15 & Gawan sprach: "mir wære leit,\\ 
 & ob \textbf{mîn} gemach âne arbeit\\ 
 & von disen vrouwen hin rite,\\ 
 & ich \textbf{en}versuohte \textit{ê} baz ir site.\\ 
 & ich hân \textbf{ouch} ê von in vernomen.\\ 
20 & sît ich sô nâhen \textbf{nû} bin komen,\\ 
 & \textbf{mich ensol des} niht betrâgen,\\ 
 & ich \textbf{en}\textbf{welle}z durch si wâgen."\\ 
 & der wirt mit triuwen klagete.\\ 
 & sînem gaste er sagete:\\ 
25 & "aller kumber ist ein \textbf{niht},\\ 
 & wan dem ze \textbf{lîden} geschiht\\ 
 & disiu âventiure,\\ 
 & \textbf{diu} \textbf{ist} \textbf{scharpf unde} ungehiure,\\ 
 & vür wâr \textbf{unde} âne liegen;\\ 
30 & hêrre, ich\textbf{ne} kan niht triegen."\\ 
\end{tabular}
\scriptsize
\line(1,0){75} \newline
G I L M Z Fr23 Fr62 \newline
\line(1,0){75} \newline
\textbf{1} \textit{Initiale} G I L M Z Fr23 Fr62  \textbf{15} \textit{Initiale} I  \newline
\line(1,0){75} \newline
\textbf{2} muoz] mirz Fr62 \textbf{3} des] \textit{om.} L \textbf{6} ze Terre Marveile] zeterre maveile G zeterre malveil I Zeterre marviel Fr23 ce terre marueile Fr62 \textbf{7} Let Marveile] letmarveile I Lit Marvale Z Liet marviel Fr23 leite marueile Fr62 \textbf{8} wart] en wart M (Fr23)  $\cdot$ versuochet] versusehet Fr23 \textbf{9} ûf] Vsz M  $\cdot$ Tschastel Marveile] shatimorueile I Thfastel marveile L Sastel Marveile M Tschahtel marvale Z scatel marviel Fr23 tsaster marueile Fr62 \textbf{10} wil] we Fr23 \textbf{12} swaz] Waz L (M) \textbf{13} was] ist M  $\cdot$ noch gar ein] liht I \textbf{14} nâhent] hat I nahet M \textbf{16} gemach] lýp L (M) (Fr23) \textbf{18} enversuohte] versuchte I (M) (Fr23)  $\cdot$ ê] \textit{om.} G \textbf{19} ê] \textit{om.} I \textbf{21} ensol] sol I Fr23 Fr62  $\cdot$ niht] Nu nicht M \textbf{22} ich enwellez] ichn welle I  $\cdot$ wâgen] fragen Fr23 \textbf{23} klagete] chlage Fr23 \textbf{24} sînem] Sinest Fr23  $\cdot$ sagete] chlagte Fr23 \textbf{25} ist] der ist I  $\cdot$ ein niht] enwiht Z \textbf{26} ze lîden] zeliden hie I \textbf{28} diu] \textit{om.} I \textbf{30} ichne] ich I  $\cdot$ niht] uͯch niht L \newline
\end{minipage}
\hspace{0.5cm}
\begin{minipage}[t]{0.5\linewidth}
\small
\begin{center}*T
\end{center}
\begin{tabular}{rl}
 & Der wirt sprach mit triuwen:\\ 
 & "\textbf{âmen}, hêrre, sô muoz mich riuwen,\\ 
 & daz iuch des vrâgens niht bevilt.\\ 
 & ich wil iu lîhen einen schilt;\\ 
5 & nû wâpent iuch ûf einen strît.\\ 
 & \textbf{in} Terre Marvele ir sît.\\ 
 & \hspace*{-.7em}\big| hêrre, ez \textbf{en}wart versuochet nie,\\ 
 & \hspace*{-.7em}\big| Let Marvele ist hie.\\ 
 & Tschahtel Marvele \textbf{ist} diu nôt.\\ 
10 & iuwer leben wil in den tôt.\\ 
 & ist iu âventiure bekant,\\ 
 & swaz ie gestreit iuwer hant,\\ 
 & daz was noch gar ein kindes spil.\\ 
 & \textbf{iu nâhent} riuwebæriu zil."\\ 
15 & Gawan sprach: "mir wære leit,\\ 
 & ob \textbf{mich} gemach âne arbeit\\ 
 & von disen vrouwen hinnen rite,\\ 
 & ich \textbf{en}versuochte ê baz ir site.\\ 
 & ich hân ê von in vernomen.\\ 
20 & sît ich sô nâhe \textbf{nû} bin komen,\\ 
 & \textbf{mich ensol des} niht betrâgen,\\ 
 & i\textbf{ne} \textbf{welle}z durch si wâgen."\\ 
 & \textit{\begin{large}D\end{large}}er wirt \textbf{ez} mit triuwen klagete.\\ 
 & sînem gaste er sagete:\\ 
25 & "aller kumber ist ein \textbf{niht},\\ 
 & wan dem ze \textbf{ladenne} geschiht\\ 
 & disiu âventiure,\\ 
 & \textbf{diu} \textbf{scharpf unde} ungehiure,\\ 
 & vür wâr âne liegen;\\ 
30 & hêrre, i\textbf{n} kan \textbf{iuch} niht triegen."\\ 
\end{tabular}
\scriptsize
\line(1,0){75} \newline
T U V W Q R Fr39 Fr40 \newline
\line(1,0){75} \newline
\textbf{1} \textit{Überschrift:} Hie saite der wirt her gawan die gelegenhait vmb kastel marfeile W   $\cdot$ \textit{Initiale} W Q R Fr40   $\cdot$ \textit{Majuskel} T  \textbf{23} \textit{Initiale} T W  \newline
\line(1,0){75} \newline
\textbf{1} \textit{Die Verse 553.1-599.30 fehlen} U  \textbf{2} âmen] \textit{om.} V W Q R Fr40 \textbf{3} iuch] iv T \textbf{5} iuch] îv T \textbf{6} in Terre] Jnter Q  $\cdot$ Marvele] [*]: marfeile V W marveile Q R (Fr39) (Fr40) \textbf{8} enwart] wart Q R Fr40 \textbf{7} Let] zer Fr40  $\cdot$ Marvele] [*]:  marueile V marfeile W marveile Q R Fr39 Fr40  $\cdot$ ist] \textit{om.} R \textbf{9} Tschahtel Marvele] [*veile]: Vf kastelmarveile V Tschatel marfeile W Tschachtelmarveile Q Schatel marveile R Tschahtel marveile Fr39 Fr40 \textbf{10} wil] daz wil V (W) (Q) (R) (Fr39) Fr40 \textbf{12} swaz] Was W Q R \textbf{13} ein] \textit{om.} Q \textbf{14} iu] Nv V (W) (Q) (R) (Fr39) (Fr40)  $\cdot$ nâhent] nahet v́ch V (W) (Q) (R) (Fr39) (Fr40)  $\cdot$ riuwebæriu] rúwberes R \textbf{15} Gawan] Gawin R \textbf{16} mich] mein W Q (R) (Fr39) Fr40 \textbf{18} enversuochte] versuͦchte W (R) Fr39 (Fr40)  $\cdot$ ê] \textit{om.} W  $\cdot$ ir] irn T R \textbf{19} hân] han oͮch V (W) (Q) (R) (Fr40) \textbf{20} sô nâhe nû] nu so nahen Fr40 \textbf{21} ensol] sol Fr40 \textbf{22} ine] ich Fr40 \textbf{23} Der] ÷er T OEr W  $\cdot$ ez] \textit{om.} V W Q R Fr39 Fr40 \textbf{24} er] er do V W \textbf{26} dem] das R  $\cdot$ ze ladenne] zvͦ lidenne V (W) (Q) (R) (Fr39) (Fr40)  $\cdot$ geschiht] uch geschiht R \textbf{28} diu] So R ist Fr40  $\cdot$ ungehiure] die vngehv́re V \textbf{29} âne] an allez Fr40 \textbf{30} in kan iuch] ich kan îv T ich enkan V W Fr39 ich kon Q (R) (Fr40) \newline
\end{minipage}
\end{table}
\end{document}
