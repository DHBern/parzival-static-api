\documentclass[8pt,a4paper,notitlepage]{article}
\usepackage{fullpage}
\usepackage{ulem}
\usepackage{xltxtra}
\usepackage{datetime}
\renewcommand{\dateseparator}{.}
\dmyyyydate
\usepackage{fancyhdr}
\usepackage{ifthen}
\pagestyle{fancy}
\fancyhf{}
\renewcommand{\headrulewidth}{0pt}
\fancyfoot[L]{\ifthenelse{\value{page}=1}{\today, \currenttime{} Uhr}{}}
\begin{document}
\begin{table}[ht]
\begin{minipage}[t]{0.5\linewidth}
\small
\begin{center}*D
\end{center}
\begin{tabular}{rl}
\textbf{496} & \begin{large}S\end{large}us pflac ich\textbf{s} durch die werden\\ 
 & ûf den drîn teilen der erden,\\ 
 & ze \textbf{Europa} unt in Asia\\ 
 & unt verre in Affrica.\\ 
 & \multicolumn{1}{l}{ - - - }\\ 
 & \multicolumn{1}{l}{ - - - }\\ 
5 & sô ich rîche tjoste wolde tuon,\\ 
 & sô reit ich vür \textbf{Gauriuon}.\\ 
 & ich hân ouch manege tjost getân\\ 
 & vor dem berc \textbf{ze} Famorgan.\\ 
 & ich tet vil \textbf{rîcher} tjoste schîn\\ 
10 & \textbf{vor} dem berc ze Agremontin.\\ 
 & swer einhalp wil ir tjoste hân,\\ 
 & dâ komen\textit{t} ûz \textbf{viurige} man;\\ 
 & anderhalp si brinnent niht,\\ 
 & swaz man \textbf{dâ} tjostiure siht,\\ 
15 & \textbf{unt dô} ich vür den Rohas\\ 
 & durch âventiure gestrichen was,\\ 
 & dâ kom ein werdiu windisch diet\\ 
 & ûz durch tjoste geinbiet.\\ 
 & \textbf{ich} vuor von Sibilje\\ 
20 & daz mer \textbf{al}umbe gein \textbf{Zilje},\\ 
 & durch \textbf{Friul} \textbf{ûz} \textbf{vür} Aglei.\\ 
 & owê unde heiâ hei,\\ 
 & daz ich dînen vater ie gesach,\\ 
 & der mir ze sehen al dâ geschach,\\ 
25 & dô ich ze Sibilje zogte în!\\ 
 & dô het der werde Anschevin\\ 
 & vor mir geherberget ê.\\ 
 & sîn vart tuot mir iemer wê,\\ 
 & die er vuor ze Baldac;\\ 
30 & ze tjostieren er \textbf{dâ} tôt \textbf{lac}.\\ 
\end{tabular}
\scriptsize
\line(1,0){75} \newline
D Fr11 Fr31 \newline
\line(1,0){75} \newline
\textbf{1} \textit{Initiale} D Fr11 Fr31  \textbf{25} \textit{Initiale} Fr11  \newline
\line(1,0){75} \newline
\textbf{1} Sus] Nv Fr31  $\cdot$ ichs] ich Fr11 Fr31  $\cdot$ die] divͯ Fr11 \textbf{2} teilen] tail Fr31 \textbf{3} Europa] Evropa D Evropie Fr11 anropye Fr31  $\cdot$ Asia] Asya Fr11 Fr31 \textbf{4} Affrica] Afrika Fr31 \textbf{6} vür] gein Fr31  $\cdot$ Gauriuon] Gavrivͦn D Gevrivͯn Fr11 garivͦn Fr31 \textbf{7} ouch] \textit{om.} Fr31 \textbf{8} ze Famorgan] zefemorgan Fr11 \textbf{10} ze Agremontin] ze Agremontîn D ze Agremontein Fr11 zegelmo:::in Fr31 \textbf{12} dâ] do D  $\cdot$ koment] chomen D Fr11 \textbf{13} brinnent] brennent Fr31 \textbf{14} dâ] \textit{om.} Fr11 Fr31  $\cdot$ tjostiure] tyostyrn Fr11 \textbf{15} dô] \textit{om.} Fr31  $\cdot$ Rohas] Roas Fr11 \textbf{17} dâ] Do Fr11 \textbf{19} Sibilje] Sibilie D Sybilie Fr11 \textbf{20} Zilje] Zilie D (Fr11) \textbf{21} Friul] Friv́le Fr11  $\cdot$ vür] dvrch Fr11  $\cdot$ Aglei] Aglêi D \textbf{22} owê] aẅe Fr11 \textbf{25} Sibilje] Sibilie D Sybilye Fr11 \textbf{26} dô] da Fr11  $\cdot$ Anschevin] Anshevein Fr11 \textbf{29} die er] Divͯ ich Fr11  $\cdot$ Baldac] Baldach D Fr11 \newline
\end{minipage}
\hspace{0.5cm}
\begin{minipage}[t]{0.5\linewidth}
\small
\begin{center}*m
\end{center}
\begin{tabular}{rl}
 & sus pflac ich durch die werden\\ 
 & ûf den drîn teilen der erden,\\ 
 & zuo \textbf{Erupie} und in Asia\\ 
 & und \dag wer\dag  \textbf{hin} in Affrica\\ 
 & ritterschaft zuo tuon\\ 
 & sunder vröude und âne suon.\\ 
5 & \textit{sô ich rîche juste wolte tuon,}\\ 
 & sô reit ich vür \textbf{Gauriu\textit{o}n}.\\ 
 & ich hab ouch manig\textit{e} \textit{j}ust getân\\ 
 & vor dem berge Famorgan.\\ 
 & ich tet vil \textbf{rîcher} juste schîn\\ 
10 & \textbf{zuo} dem berge zuo Agr\textit{e}montin.\\ 
 & wer ei\textit{n}hal\textit{p} wil ir juste hân,\\ 
 & dâ koment ûz \textbf{viurige} man;\\ 
 & anderhalp si brinne\textit{n}t niht,\\ 
 & waz man \textbf{d\textit{â}} justiere siht,\\ 
15 & \textbf{und dô} ich vür den Ro\textit{h}as\\ 
 & durch âventiur gestrichen was,\\ 
 & dô kam ein werdiu windesch diet\\ 
 & ûz durch juste gegenbiet,\\ 
 & \textbf{und} vuor von Sibilj\textit{e}\\ 
20 & daz mer \textbf{al} umbe gegen \textbf{Zilje},\\ 
 & durch \textbf{Friol} \textbf{ûz} \textbf{vür} Agle\textit{i}.\\ 
 & ouwê und heiâ hei,\\ 
 & daz ich dînen vater ie ge\textit{s}ach,\\ 
 & der mir zuo sehen aldâ geschach,\\ 
25 & dô ich zuo Sibilje z\textit{o}gte în!\\ 
 & dô het der werde A\textit{n}schevin\\ 
 & vor mir \dag geherget\dag  ê.\\ 
 & sîn vart tuot mir iemer wê,\\ 
 & die er vuor zuo \textit{B}aldac;\\ 
30 & zuo justieren er tôt \textbf{dâ} \textbf{lac}.\\ 
\end{tabular}
\scriptsize
\line(1,0){75} \newline
m n o \newline
\line(1,0){75} \newline
\newline
\line(1,0){75} \newline
\textbf{1} ich] ich sús o \textbf{2} der] uͯff der o \textbf{3} Erupie] erupe n o  $\cdot$ Asia] assia o \textbf{4} in Affrica] maffrica o \textbf{4} vröude] fride n o  $\cdot$ suon] suͯn m n sone o \textbf{5} \textit{Vers 496.5 fehlt} m  \textbf{6} Gauriuon] gaurivn m ganriuͦn n ganriturn o \textbf{7} manige just] mange tuͯgent vnd just m \textbf{10} zuo] \textit{om.} n  $\cdot$ Agremontin] agromontin m agramotinn o \textbf{11} einhalp] einphalt m  $\cdot$ ir] er n \textbf{13} brinnent] brinnet m bringent o \textbf{14} dâ] do m n o  $\cdot$ siht] [va]: siht m \textbf{15} den] [der]: den n  $\cdot$ Rohas] roas m n o \textbf{17} werdiu] [werder]: werden o  $\cdot$ windesch] wunsch o \textbf{19} Sibilje] sibilia m n sibillia o \textbf{20} Zilje] zilie m n zilia o \textbf{21} Friol] frẏol n  $\cdot$ Aglei] agleie m agleẏ n aglie o \textbf{23} gesach] geschah m \textbf{25} zuo] so o  $\cdot$ Sibilje] sibilie m n zibillie o  $\cdot$ zogte] zougtte m zúchte o \textbf{26} Anschevin] auscevin m n ansce vin o \textbf{29} vuor] vor o  $\cdot$ Baldac] raldag m n o \textbf{30} tôt] >dot< o  $\cdot$ dâ lac] dolag n (o) \newline
\end{minipage}
\end{table}
\newpage
\begin{table}[ht]
\begin{minipage}[t]{0.5\linewidth}
\small
\begin{center}*G
\end{center}
\begin{tabular}{rl}
 & \begin{large}S\end{large}us pflac ich \textbf{es} durch die werden\\ 
 & ûf den drîn teil\textit{en} der erden,\\ 
 & ze \textbf{Aropie} unde in Asia\\ 
 & unt verre in Affrica.\\ 
 & \multicolumn{1}{l}{ - - - }\\ 
 & \multicolumn{1}{l}{ - - - }\\ 
5 & sô ich rîche tjoste wolte tuon,\\ 
 & sô reit ich vür \textbf{Gaurian}.\\ 
 & ich hân ouch manige tjoste getân\\ 
 & vor dem berge \textbf{ze} Phimurgan.\\ 
 & ich tet vil \textbf{rîcher} tjoste schîn\\ 
10 & \textbf{vor} dem berc ze Agremontin.\\ 
 & swer einhalp wil ir tjoste hân,\\ 
 & dâ koment ûz \textbf{viurige} man;\\ 
 & anderhalp si brinnent niht,\\ 
 & swaz man \textbf{dâ} tjostiure siht,\\ 
15 & \textbf{unde} ich vür den Roas\\ 
 & durch âventiure gestrichen was,\\ 
 & dô kom ein werdiu windesch diet\\ 
 & ûz durch tjoste gegenbiet.\\ 
 & \textbf{ich} vuor von Sibilje\\ 
20 & daz mer \textbf{al} umbe gein \textbf{Zilje},\\ 
 & durch \textbf{Vriul} \textbf{ûz} \textbf{durch} Aglei.\\ 
 & owê unde heiâ hei,\\ 
 & daz ich dînen vater ie gesach,\\ 
 & der mir ze sehenne al dâ geschach,\\ 
25 & dâ ich ze Sibilje zogte în!\\ 
 & dô het der werde Anschevin\\ 
 & vor mir geherberget ê.\\ 
 & sîn vart tuot mir immer wê,\\ 
 & die er vuor ze Baldac;\\ 
30 & ze tjostieren er \textbf{dâ} tôt \textbf{lac}.\\ 
\end{tabular}
\scriptsize
\line(1,0){75} \newline
G I L M Z Fr61 \newline
\line(1,0){75} \newline
\textbf{1} \textit{Initiale} G I L M Z  \textbf{15} \textit{Initiale} I  \newline
\line(1,0){75} \newline
\textbf{1} es] \textit{om.} L Fr61 \textbf{2} teilen] teil G  $\cdot$ der] ander M \textbf{3} Aropie] [aropię]: aropie G Arabie L êvropie M evropia Z Evropa Fr61  $\cdot$ in Asia] inasia G Asya Z \textbf{4} in Affrica] inaffrica G Africa I \textbf{5} rîche] hohe I \textbf{6} vür] gegen Fr61  $\cdot$ Gaurian] Cauruͤni I Gavrivͯn L gavrivn M Gaurivn Z Gargarvͤn Fr61 \textbf{8} dem berge] den bergen I  $\cdot$ ze Phimurgan] zefamorgan G faimurgan I Famvrgan L zcu famorgan M (Z) ze Feimvrgan Fr61 \textbf{9} \textit{Die Verse 486.9-10 fehlen} Fr61   $\cdot$ rîcher] riche L \textbf{10} ze Agremontin] zeAgremontin G zcu agremonzcin M zv Agromontin Z \textbf{11} swer] Wer L M  $\cdot$ einhalp wil ir] wolt einhalp die I \textbf{12} ûz] vf L  $\cdot$ viurige] fewerein Fr61 \textbf{14} swaz] Waz L (M)  $\cdot$ tjostiure] Tiostiern I (L) tiost tiure M ẏoste Fr61 \textbf{15} unde] Vnd da L (M) Z Do Fr61  $\cdot$ Roas] Rohas Fr61 \textbf{16} gestrichen] geriten L (M) \textbf{17} Miͤr widerrait ein werden diet Fr61  $\cdot$ dô] Da L M Z  $\cdot$ werdiu] werdir M \textbf{18} durch] [der]: dvrch G \textbf{19} von] duͯrch L  $\cdot$ Sibilje] sibilie G M (Z) (Fr61) Sibille I Sýbilie L \textbf{20} Zilje] zilie G I L Z sicilie M Cecilie Fr61 \textbf{21} Vriul] friul I (L) (Z) vriol M virgaul Fr61  $\cdot$ ûz durch] vnd Gein I vnz gein L usz vor M vz fvͤr Z fuer Fr61  $\cdot$ Aglei] Agelei I agoley M Agleẏ Fr61 \textbf{22} Heia heia vnd hei L  $\cdot$ owê] Awe I Owe heya M Auwe Fr61  $\cdot$ hei] hoy M \textbf{24} sehenne] bekennenne L  $\cdot$ al] \textit{om.} Fr61 \textbf{25} dâ] do I (L) (Fr61)  $\cdot$ ich] [iz]: ih G  $\cdot$ ze Sibilje] zesibilie G zezilie I zuͯ Sýbilie L zcu sibilie M (Z) (Fr61)  $\cdot$ zogte] zcoch M \textbf{26} dô] Da M Z  $\cdot$ Anschevin] enthseuin I ansevin M anshevin Z Aschewein Fr61 \textbf{27} mir] \textit{om.} I \textbf{29} er vuor] vuͤr er I  $\cdot$ ze Baldac] [zeb]: ze baldach G (L) (M) (Fr61) zebaldac I \textbf{30} ze tjostieren] ander Tiost I Zuͯ tiost L Von eyner tiost M (Fr61)  $\cdot$ dâ tôt] tot I M tot do Fr61  $\cdot$ lac] gelach L (Z) \newline
\end{minipage}
\hspace{0.5cm}
\begin{minipage}[t]{0.5\linewidth}
\small
\begin{center}*T
\end{center}
\begin{tabular}{rl}
 & Sus pflac ich\textbf{s} durch die werden\\ 
 & ûf den drîn teilen der erden,\\ 
 & ze \textbf{Arabie} unde in Asya\\ 
 & unde verre in Affryca.\\ 
 & \multicolumn{1}{l}{ - - - }\\ 
 & \multicolumn{1}{l}{ - - - }\\ 
5 & sô ich rîche tjoste wolte tuon,\\ 
 & sô reit ich vür \textbf{die} \textbf{Covriuon}.\\ 
 & \multicolumn{1}{l}{ - - - }\\ 
 & \multicolumn{1}{l}{ - - - }\\ 
 & ich tet vil \textbf{rîche} tjoste schîn\\ 
10 & \textbf{vor} dem berge ze Agremontin.\\ 
 & swer einhalp wil ir tjost hân,\\ 
 & dâ koment ûz \textbf{viurîne} man;\\ 
 & anderhalp si brinnent niht,\\ 
 & swaz man tjostiure siht.\\ 
15 & \textbf{\begin{large}D\end{large}ô} ich vür den Roas\\ 
 & durch âventiure gestrichen was,\\ 
 & dô kom ein wert windesch diet\\ 
 & ûz durch tjoste gegenbiet.\\ 
 & \textbf{ich} vuor von Sibylie\\ 
20 & daz mer umbe gegen \textbf{Cecilie},\\ 
 & durch \textbf{Fryuz} \textbf{vür} Aglei.\\ 
 & ouwê unde heiâ hei,\\ 
 & daz ich dînen vater ie gesach,\\ 
 & der mir ze sehenne aldâ geschach,\\ 
25 & Dô ich ze Sibylie zogete în!\\ 
 & dô hete der werde Anschevin\\ 
 & vor mir geherberget ê.\\ 
 & sîn vart tuot mir iemer wê,\\ 
 & die er vuor ze Baldac;\\ 
30 & ze tjostierne er tôt \textbf{gelac}.\\ 
\end{tabular}
\scriptsize
\line(1,0){75} \newline
T U V W O Q R \newline
\line(1,0){75} \newline
\textbf{1} \textit{Initiale} O   $\cdot$ \textit{Majuskel} T  \textbf{3} \textit{Initiale} W  \textbf{15} \textit{Initiale} T  \textbf{25} \textit{Majuskel} T  \newline
\line(1,0){75} \newline
\textbf{1} \textit{Die Verse 453.1-502.30 fehlen} U   $\cdot$ Sus] ÷vs O Als Q  $\cdot$ ichs] ich Q R  $\cdot$ die] [d*ie]: die T \textbf{2} den] \textit{om.} R  $\cdot$ drîn teilen] drittailen W \textbf{3} Arabie] arabia W (R) Arabîæ O araby Q  $\cdot$ in] ze R  $\cdot$ Asya] asŷa T asia W R \textbf{4} Affryca] africa V affrica W (O) Q R \textbf{4} \textit{Die Verse 496.4¹-4² sind am Rand nachgetragen und später radiert:} R:schaft : tvne / svnder vride vnde ane svne V  \textbf{5} rîche] \textit{om.} O  $\cdot$ tjoste] strit R \textbf{6} die] den V W Q R \textit{om.} O  $\cdot$ Covriuon] Covrivͦn T [*]: ganrivn V kouertun W covrivn O coortun Q Covrwͦn R \textbf{7} \textit{Die Verse 496.7-8 fehlen} T O Q R W   $\cdot$ \textit{Die Verse 496.7-8 sind am Rand nachgetragen und später radiert:} ::: han oͮch manige :st getan / von dem berge famorgan V  \textbf{9} rîche] reicher W (O) Q (R) \textbf{10} dem berge] den bergen W  $\cdot$ Agremontin] agremontein W agramantin O agremonyn Q Agremonin R \textbf{12} dâ] So W  $\cdot$ viurîne] [fúrrenn]: fúrreny R \textbf{14} swaz] Was W Q R  $\cdot$ man] man do V W Q man da O  $\cdot$ tjostiure] tiostiren Q das strittes R \textbf{15} Roas] rohas O \textbf{17} wert] [w*]: werde V  $\cdot$ windesch] windich O \textbf{18} durch] gegen R  $\cdot$ tjoste gegenbiet] gegen tyost biet W (Q) tyoste biet O \textbf{19} vuor] fuͦr fúr R  $\cdot$ Sibylie] sibilie V W (R) sibilîe O sibille Q \textbf{20} Cecilie] Cecilŷe T cilie W Q (R) cilîe O \textbf{21} Fryuz] Fryvz T [*]: vriol V friul W (O) R friűl Q  $\cdot$ vür] [*]: vz fv́r V auß fúr W (O) vz vor Q vnd fúr R  $\cdot$ Aglei] Agley T (V) (Q) \textbf{22} ouwê] Owi V (Q) R Awi O \textbf{23} gesach] [gesach]: gestach Q \textbf{24} So leide mir von im geschach W  $\cdot$ geschach] sachach Q \textbf{25} Sibylie] Sybile T sibilie V W (O) R sibillen Q \textbf{26} Anschevin] Anscevin T antscheuein W Anshevin O (Q) (R) \textbf{28} tuot] die tuͦt R \textbf{29} Baldac] baldag V W baldack Q Baldach O \textbf{30} ze tjostierne] Ze stritten R  $\cdot$ tôt] do tot V R  $\cdot$ gelac] lach O (Q) \newline
\end{minipage}
\end{table}
\end{document}
