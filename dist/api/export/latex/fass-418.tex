\documentclass[8pt,a4paper,notitlepage]{article}
\usepackage{fullpage}
\usepackage{ulem}
\usepackage{xltxtra}
\usepackage{datetime}
\renewcommand{\dateseparator}{.}
\dmyyyydate
\usepackage{fancyhdr}
\usepackage{ifthen}
\pagestyle{fancy}
\fancyhf{}
\renewcommand{\headrulewidth}{0pt}
\fancyfoot[L]{\ifthenelse{\value{page}=1}{\today, \currenttime{} Uhr}{}}
\begin{document}
\begin{table}[ht]
\begin{minipage}[t]{0.5\linewidth}
\small
\begin{center}*D
\end{center}
\begin{tabular}{rl}
\textbf{418} & \textbf{\begin{large}D\end{large}â} wære von mînen handen\\ 
 & in kreize bestanden\\ 
 & Gawan, der ellenthafte degen.\\ 
 & des het ich mich gein im bewegen,\\ 
5 & \textbf{daz} der kampf wære alhie getân,\\ 
 & woltes mîn hêrre gestattet hân.\\ 
 & der treit mit sünden mînen haz.\\ 
 & ich \textbf{trûwete} \textbf{im} anderre dinge baz.\\ 
 & Hêr Gawan, lobt mir her vür wâr,\\ 
10 & daz ir von hiute über ein jâr\\ 
 & mir ze gegenrede stêt\\ 
 & \textbf{in} kampfe, ob ez \textbf{sô} hie ergêt,\\ 
 & daz iu mîn hêrre læt daz leben.\\ 
 & dâ wirt iu kampf von mir \textbf{gegeben}.\\ 
15 & Ich sprach iuch an zem Plimizœl.\\ 
 & nû sî der kampf ze Parbigœl\\ 
 & vor dem künege Melianze.\\ 
 & der sorgen zeime kranze\\ 
 & trag ich unz ûf daz teidinc,\\ 
20 & daz ich \textbf{gein iu kum} in den rinc.\\ 
 & dâ sol mir sorge tuon bekant\\ 
 & iwer \textbf{manlîchiu} hant."\\ 
 & Gawan, der ellens rîche,\\ 
 & bôt gezogenlîche\\ 
25 & nâch dirre bet sicherheit.\\ 
 & dô was mit rede \textbf{al dâ} bereit\\ 
 & der herzoge Liddamus\\ 
 & begunde \textbf{ouch} sîner rede alsus\\ 
 & mit spæhlîchen worten,\\ 
30 & al dâ siz alle hôrten.\\ 
\end{tabular}
\scriptsize
\line(1,0){75} \newline
D Fr5 \newline
\line(1,0){75} \newline
\textbf{1} \textit{Initiale} D Fr5  \textbf{9} \textit{Majuskel} D  \textbf{15} \textit{Majuskel} Fr5  \textbf{22} \textit{Capitulumzeichen} Fr5  \newline
\line(1,0){75} \newline
\textbf{3} Gawan] Gauwan Fr5  $\cdot$ ellenthafte] ellinthaftir Fr5 \textbf{5} kampf wære alhie] vride alhie were Fr5 \textbf{6} gestattet] des gistattit Fr5 \textbf{8} trûwete] gitrirete:: Fr5  $\cdot$ anderre] ander Fr5 \textbf{9} Gawan] Gauwan Fr5  $\cdot$ mir her] her mir Fr5 \textbf{15} an zem] an andem Fr5  $\cdot$ Plimizœl] Plimizoͤl D plimizol Fr5 \textbf{16} Parbigœl] Parbigoͤl D berbigol Fr5 \textbf{20} Wie da :rgen rainiv dinc Fr5 \textbf{21} \textit{Die Verse 418.21-22 fehlen} Fr5  \textbf{23} Gawan] Gauwan Fr5 \textbf{26} mit rede al dâ bereit] da mit rede al bireit Fr5 \textbf{29} \textit{Die Verse 418.29-30 fehlen} Fr5  \newline
\end{minipage}
\hspace{0.5cm}
\begin{minipage}[t]{0.5\linewidth}
\small
\begin{center}*m
\end{center}
\begin{tabular}{rl}
 & \textbf{jâ}, wære von mîne\textit{n} handen\\ 
 & in kreize bestanden\\ 
 & Gawan, der ellenthafte degen.\\ 
 & des hette ich mich gegen ime bewegen,\\ 
5 & \textbf{daz} der kamp\textit{f} wære alhie getân,\\ 
 & wolte es mîn hêrre gestattet hân.\\ 
 & der treit mit sü\textit{n}den mînen haz.\\ 
 & ich \textbf{trûwete} \textbf{ime} andere dinge baz.\\ 
 & hêr Gawan, lobt mir her vür wâr,\\ 
10 & daz ir von hiute über ein jâr\\ 
 & mir \textit{ze} gegenrede stât\\ 
 & \textbf{in} kampfe, ob ez  hie ergât,\\ 
 & daz iu mîn hêrre lât daz leben.\\ 
 & dâ wirt i\textit{u} \textit{k}ampf von mir \textbf{geben}.\\ 
15 & ich sprach \textit{iuch} an zem Plimizol.\\ 
 & nû sî der kampf ze Barbigol\\ 
 & vo\textit{r} dem künige Melianze.\\ 
 & der sorgen ze einem kranze\\ 
 & trage ich unz ûf daz tegedinc,\\ 
20 & daz ich \textbf{gegen iuch kome} in den rinc.\\ 
 & d\textit{â} sol mir sorge tuon bekant\\ 
 & iuwer \textbf{manlîche} hant."\\ 
 & \begin{large}G\end{large}awan, der ellens rîche,\\ 
 & bôt gezogenlîche\\ 
25 & nâch dirre bete sicherheit.\\ 
 & dô was mit rede \textbf{aldâ} bereit\\ 
 & der herzoge Liddamus.\\ 
 & \textbf{er} begunde sîner rede alsus\\ 
 & mit spæhelîchen worten,\\ 
30 & aldâ siz alle hôrten.\\ 
\end{tabular}
\scriptsize
\line(1,0){75} \newline
m n o \newline
\line(1,0){75} \newline
\textbf{23} \textit{Initiale} m  \newline
\line(1,0){75} \newline
\textbf{1} jâ] So n  $\cdot$ mînen] minem m \textbf{3} degen] \sout{gegen} tegen o \textbf{4} des] Do o \textbf{5} kampf] kamp m \textbf{6} gestattet] gestatten o \textbf{7} treit] reit n  $\cdot$ sünden] svmden m sunder n  $\cdot$ mînen] minem n \textbf{8} trûwete] getruwete n  $\cdot$ andere] ander n o \textbf{9} lobt] lop o \textbf{10} ir] er n \textbf{11} mir] Mit n  $\cdot$ ze] \textit{om.} m \textbf{12} hie] so hie n o \textbf{14} dâ] Do n o  $\cdot$ kampf] von kampfe m \textbf{15} iuch] \textit{om.} m  $\cdot$ Plimizol] plúmzol n pluͯmzol o \textbf{16} kampf] j kamff o  $\cdot$ Barbigol] parbigol n o \textbf{17} vor] Von m  $\cdot$ Melianze] meliantz n meliancz o \textbf{18} sorgen] sorge n o \textbf{19} unz] \textit{om.} n  $\cdot$ tegedinc] gedingk n o \textbf{20} ich gegen iuch kome] ich kom gegen úch n \textbf{21} dâ] Do m n o \textbf{22} manlîche] manlich o \textbf{24} bôt] Bat o \textbf{25} sicherheit] sicherlich o \textbf{26} aldâ] aldo n \textbf{27} Liddamus] liddinus n lidemuͯs o \textbf{30} aldâ] Aldo n [Ada]: Alda o  $\cdot$ siz] do sú n sie o \newline
\end{minipage}
\end{table}
\newpage
\begin{table}[ht]
\begin{minipage}[t]{0.5\linewidth}
\small
\begin{center}*G
\end{center}
\begin{tabular}{rl}
 & \textbf{dâ} wære von mînen handen\\ 
 & in kreize bestanden\\ 
 & Gawan, der ellenthafte degen.\\ 
 & des hete ich mich gein im bewegen.\\ 
5 & der kampf wære al hie getân,\\ 
 & wolt es mîn hêrre gestatet hân.\\ 
 & der treit mit sünden mînen haz.\\ 
 & ich \textbf{getrûwte} \textbf{im} ander dinge baz.\\ 
 & hêr Gawan, lobet mir her vür wâr,\\ 
10 & daz ir von hiute über ein jâr\\ 
 & mir ze gegenrede stêt\\ 
 & \textbf{mit} kampfe, obe ez \textbf{sô} hie ergêt,\\ 
 & daz iu mîn hêrre lât dez leben.\\ 
 & dâ wirt iu kampf von mir \textbf{gegeben}.\\ 
15 & ich sprach iuch an zem Blimzol.\\ 
 & nû sî der kampf ze Barbigol\\ 
 & vor dem künege Melianze.\\ 
 & der sorgen zeinem kranze\\ 
 & \begin{large}T\end{large}rage ich unze ûf daz teidinc,\\ 
20 & daz ich \textbf{kum gein iu} in den rinc.\\ 
 & dâ sol mir sorge tuon bekant\\ 
 & iwer \textbf{werlîchiu} hant."\\ 
 & Gawan, der ellens rîche,\\ 
 & bôt gezogenlîche\\ 
25 & nâch dirre bet sicherheit.\\ 
 & dô was mit rede \textbf{al dâ} bereit\\ 
 & der herzoge Lidamus\\ 
 & begunde \textbf{ouch} sîner rede alsus\\ 
 & mit spæhlîchen worten,\\ 
30 & al dâ siz alle hôrten.\\ 
\end{tabular}
\scriptsize
\line(1,0){75} \newline
G I O L M Q R Z \newline
\line(1,0){75} \newline
\textbf{1} \textit{Initiale} I O L Z   $\cdot$ \textit{Capitulumzeichen} R  \textbf{17} \textit{Initiale} I  \textbf{19} \textit{Initiale} G  \newline
\line(1,0){75} \newline
\textbf{1} dâ] Hie I ÷a O Ia L (M) (Q) (Z) Do R \textbf{2} in] Ein R \textbf{3} Gawan] Gawin R  $\cdot$ ellenthafte] ellinthaffter M \textbf{4} bewegen] erwegen R \textbf{5} der] daz der I (O) (L) (M) (Q) (R) (Z)  $\cdot$ wære al hie] were also M alhie were R \textbf{6} es] \textit{om.} I des O (L) (M) Z  $\cdot$ mîn hêrre gestatet] sin der kunc gestaret I \textbf{7} der] Des R  $\cdot$ mit sünden] mẏne svnde L myn sunde M min sunden R  $\cdot$ mînen] mit R \textbf{8} getrûwte] trewet Z  $\cdot$ ander] an ander Q anderre Z \textbf{9} Gawan] Gawin R  $\cdot$ lobet] gilobite M  $\cdot$ her] \textit{om.} R \textbf{10} daz ir] Daz er O Z Har R \textbf{11} stêt] gestet I \textbf{12} mit] [*]: Ze I Jn O L M Q R Z  $\cdot$ ez] er Q  $\cdot$ hie] \textit{om.} M \textbf{13} lât] Jm lat R \textbf{14} dâ] So Q Do R  $\cdot$ wirt] wart O \textbf{15} Blimzol] plimizol I L Q R Z plymizol O blimizol M \textbf{16} der] \textit{om.} I  $\cdot$ ze] zűm Q \textbf{17} vor] Von O  $\cdot$ Melianze] Melyanze O meliancze R Meliantze Z \textbf{18} zeinem] zemen R \textbf{19} unze ûf] vf uͯch L bisz vff Q vff R (Z)  $\cdot$ teidinc] gedinge R \textbf{20} kum gein iu] gein ev chum I (O) (M) (Q) (Z) gein kvm L kom zu úch R  $\cdot$ in] an Z \textbf{21} dâ sol] Do solt Q  $\cdot$ mir] min R  $\cdot$ sorge] sorgen I \textit{om.} Q  $\cdot$ bekant] kekant R \textbf{22} werlîchiu] manlichiv O (L) (M) (Q) (R) (Z) \textbf{23} Gawan] Gawin R  $\cdot$ ellens rîche] errenreiche Q \textbf{24} bôt] Bat Q  $\cdot$ gezogenlîche] gezogenchliche O \textbf{26} dô] Da O M Z Das R  $\cdot$ mit rede] do mit reden Q  $\cdot$ al dâ] also Q aldo R \textbf{27} Lidamus] Liddamvs O (L) (Q) (Z) litdamus M liddanus R \textbf{28} ouch] \textit{om.} O L Q R  $\cdot$ alsus] sus I \textbf{29} \textit{Die Verse 418.29-30 fehlen} R  \textbf{30} al dâ] Aldo Q  $\cdot$ alle] alles Q \newline
\end{minipage}
\hspace{0.5cm}
\begin{minipage}[t]{0.5\linewidth}
\small
\begin{center}*T
\end{center}
\begin{tabular}{rl}
 & \textbf{dâ} wære von mînen handen\\ 
 & in kreize bestanden\\ 
 & Gawan, der ellenthafte degen.\\ 
 & des hetich mich gegen im bewegen,\\ 
5 & \textbf{daz} der kampf wære alhie getân,\\ 
 & woltes mîn hêre gestatet hân.\\ 
 & der treit mit sünden mînen haz.\\ 
 & ich \textbf{getriuwe} anderre dinge baz.\\ 
 & hêr Gawan, lobet mir her vür wâr,\\ 
10 & daz ir von hiute über ein jâr\\ 
 & mir ze gegenrede stêt\\ 
 & \textbf{in} kampfe, ob ez \textbf{sô} hie ergêt,\\ 
 & daz iu mîn hêrre lât daz leben.\\ 
 & dâ wirt iu kampf von mir \textbf{gegeben}.\\ 
15 & ich sprach iu an zem Plymizol.\\ 
 & nû sî der kampf ze Barbigol\\ 
 & vor dem künege Melyanze.\\ 
 & der sorgen zeinem kranze\\ 
 & tragich unz ûf daz tegedinc,\\ 
20 & daz ich \textbf{gegen iu kum} in den rinc.\\ 
 & dâ sol mir sorge tuon bekant\\ 
 & iuwer \textbf{manlîche} hant."\\ 
 & \begin{large}G\end{large}awan, der ellens rîche,\\ 
 & bôt gezogenlîche\\ 
25 & nâch dirre bete sicherheit.\\ 
 & Dô was mit rede \textbf{alhie} bereit\\ 
 & der herzoge Lyddamus\\ 
 & begunde sîner rede alsus\\ 
 & mit spæh\textit{e}lîchen worten,\\ 
30 & aldâ siz alle hôrten.\\ 
\end{tabular}
\scriptsize
\line(1,0){75} \newline
T U V W \newline
\line(1,0){75} \newline
\textbf{23} \textit{Initiale} T U V W  \newline
\line(1,0){75} \newline
\textbf{1} dâ] Do U Ja V (W)  $\cdot$ wære] wer were W \textbf{6} woltes] Molt es W  $\cdot$ gestatet] gestat U W \textbf{8} getriuwe anderre] getruͦwete im an diseme U getruwe im anderre V getrauwet im anders W  $\cdot$ dinge] dings W \textbf{9} lobet] lobete U gelobt W \textbf{11} ze] \textit{om.} W \textbf{12} in] Mit W \textbf{14} dâ] Do U V W \textbf{15} Plymizol] [plimizol]: plinyzol T plimizol V W \textbf{16} Barbigol] Barbygol T \textbf{17} Melyanze] Melianze U (W) \textbf{19} unz ûf daz] mit of U \textbf{21} dâ] Do U W \textbf{26} Dô] Das W  $\cdot$ alhie] aldo W \textbf{27} Lyddamus] littamus V lidamus W \textbf{28} begunde] Beguͦn U \textbf{29} spæhelîchen] spehclichen T \textbf{30} aldâ] Aldo W \newline
\end{minipage}
\end{table}
\end{document}
