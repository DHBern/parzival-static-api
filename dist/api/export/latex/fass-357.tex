\documentclass[8pt,a4paper,notitlepage]{article}
\usepackage{fullpage}
\usepackage{ulem}
\usepackage{xltxtra}
\usepackage{datetime}
\renewcommand{\dateseparator}{.}
\dmyyyydate
\usepackage{fancyhdr}
\usepackage{ifthen}
\pagestyle{fancy}
\fancyhf{}
\renewcommand{\headrulewidth}{0pt}
\fancyfoot[L]{\ifthenelse{\value{page}=1}{\today, \currenttime{} Uhr}{}}
\begin{document}
\begin{table}[ht]
\begin{minipage}[t]{0.5\linewidth}
\small
\begin{center}*D
\end{center}
\begin{tabular}{rl}
\textbf{357} & \begin{large}H\end{large}ie ein tjost, \textbf{diu} ander dort.\\ 
 & daz her begunde \textbf{ouch} trecken vort\\ 
 & her gein der stat durch hôhen muot.\\ 
 & ir vesperîe wart vil guot.\\ 
5 & ze bêder sîte \textbf{rotte} ungezalt,\\ 
 & garzûne, krîe manecvalt,\\ 
 & bêde Schottesch unt Walsch\\ 
 & wart dâ gerüefet sunder valsch.\\ 
 & der ritter tât was âne vride,\\ 
10 & die helde erswungen dâ die lide.\\ 
 & \textbf{ez wâren doch allez meistec} kint,\\ 
 & die ûzem her dar komen sint.\\ 
 & die \textbf{begunden} \textbf{dâ} \textbf{vil} \textbf{werde} tât,\\ 
 & die burgære pfanten si ûf der \textbf{sât}.\\ 
15 & der \textbf{nie} gediende \textbf{an} wîbe\\ 
 & \textbf{kleinôt}, der \textbf{m\textit{ö}hte} an \textbf{sîme} lîbe\\ 
 & niemer bezzer wât getragen.\\ 
 & \textbf{von Melyanze} \textbf{hôrt} ich sagen,\\ 
 & sîn zimierde wære guot.\\ 
20 & er hete ouch selbe hôhen muot\\ 
 & \textbf{unt} reit ein schœne kastelân,\\ 
 & daz Meljacanz dort gewan,\\ 
 & dô er Keien sô \textbf{hôhe} dar \textbf{hinder} stach,\\ 
 & daz mann \textbf{am} aste hangen sach.\\ 
25 & dô ez Melyacanz dort erstreit,\\ 
 & Melyanz von Liz ez \textbf{hie} wol reit.\\ 
 & sîn tât was vor ûz \textbf{sô} \textbf{bekant},\\ 
 & alle sîne tjoste in ir ougen vant\\ 
 & Obie dort ûf dem palas,\\ 
30 & dar si durch warten komen was.\\ 
\end{tabular}
\scriptsize
\line(1,0){75} \newline
D \newline
\line(1,0){75} \newline
\textbf{1} \textit{Initiale} D  \newline
\line(1,0){75} \newline
\textbf{7} Schottesch] scottesc D  $\cdot$ Walsch] walsc D \textbf{16} möhte] mohte D \textbf{22} Meljacanz] Meliacanz D \textbf{23} Keien] keyen D \textbf{26} Liz] Lŷz D \textbf{29} Obie] Obŷe D \newline
\end{minipage}
\hspace{0.5cm}
\begin{minipage}[t]{0.5\linewidth}
\small
\begin{center}*m
\end{center}
\begin{tabular}{rl}
 & hie \textbf{wart} ein just, \textbf{diu} ander dort.\\ 
 & daz her begunde \textbf{ouch} \dag strecken\dag  vort\\ 
 & her gegen der stat durch hôhen muot.\\ 
 & ir vesper\textit{î}e wart vil guot.\\ 
5 & ze beider sîte \textbf{rot} un\textit{g}ezalt,\\ 
 & garzûne, krîe manicvalt,\\ 
 & beidiu Schotte\textit{s}ch und Welsch\\ 
 & wart dâ gerüefet sunder velsch.\\ 
 & der ritter t\textit{â}t was âne vride,\\ 
10 & die helde erswungen d\textit{â} die lide.\\ 
 & \textbf{ez wâren doch allez daz mêrteil} kint,\\ 
 & die ûzem her dar komen sint.\\ 
 & die \textbf{begiengen} \textbf{dâ} \textbf{vil} \textbf{werde} tât,\\ 
 & die burgære pfante\textit{n}s û\textit{f} der \textbf{stat}.\\ 
15 & der \textbf{nie} gediende \textbf{an} wîbe\\ 
 & \textbf{kleinôt}, der \textbf{en}\textbf{m\textit{ö}hte} an lîbe\\ 
 & niemer bezzer w\textit{â}t getragen.\\ 
 & \textbf{von Melianze} \textbf{hôrt} ich sagen,\\ 
 & sîn zimierde wære guot.\\ 
20 & er hete ouch selbe hôhen muot\\ 
 & \textit{\textbf{und} reit ein schœn kastelân,}\\ 
 & \textit{daz \dag Melianz\dag  dort gewan,}\\ 
 & \textit{dô er \dag küene\dag  sô \textbf{hôhe} dar \textbf{nider} stach,}\\ 
 & \textit{daz man in \textbf{am} ast hangen sach.}\\ 
25 & dô ez Mel\textit{ia}ganz dort erstreit,\\ 
 & Mel\textit{i}anz von Li\textit{z} ez wol r\textit{ei}t.\\ 
 & sîn tât was vor ûz \textbf{sô} \textbf{bekant},\\ 
 & \textbf{daz} alle sîne juste in ir ougen vant\\ 
 & O\textit{b}ie dort ûf dem palas,\\ 
30 & dar si durch warten komen was.\\ 
\end{tabular}
\scriptsize
\line(1,0){75} \newline
m n o \newline
\line(1,0){75} \newline
\newline
\line(1,0){75} \newline
\textbf{3} her] \textit{om.} n o \textbf{4} vesperîe] vespere m \textbf{5} rot] rate o  $\cdot$ ungezalt] vnbezalt m \textbf{6} garzûne] Gariczẏm o \textbf{7} schottesch] scottech m \textbf{8} dâ] do n o \textbf{9} tât] >tot< m der n \textbf{10} dâ] do m n [d*]: do o  $\cdot$ lide] gelide n \textbf{11} wâren] was n o  $\cdot$ allez daz] alles des m das n \textbf{12} ûzem] vssern n (o) \textbf{13} begiengen] begunden o  $\cdot$ dâ] do n \textbf{14} pfantens ûf] pfantes vs m \textbf{16} enmöhte] enmohte m (o) moͯchte n \textbf{17} wât] wart m o \textbf{18} Melianze] meliantz n meliancz o \textbf{19} zimierde] zumede o \textbf{21} \textit{Die Verse 357.21-24 fehlen} m   $\cdot$ kastelân] castalan o \textbf{22} Melianz] meliantz n meliancz o  $\cdot$ gewan] gawan n \textbf{24} in] im o  $\cdot$ hangen] gangen o \textbf{25} Meliaganz] melegancz m meliantz n [kúne so]: meliancz o \textbf{26} Melianz] Melancz m Meliantz n Me ligant o  $\cdot$ Liz] lir m lire n ler o  $\cdot$ ez] es hie n o  $\cdot$ reit] riet m \textbf{27} ûz] vns n \textbf{28} ougen] burgen n \textbf{29} Obie] Ob die m  $\cdot$ ûf] an o \textbf{30} dar] Do o \newline
\end{minipage}
\end{table}
\newpage
\begin{table}[ht]
\begin{minipage}[t]{0.5\linewidth}
\small
\begin{center}*G
\end{center}
\begin{tabular}{rl}
 & hie ein tjost, \textbf{diu} andere dort.\\ 
 & daz her begunde trecken vort\\ 
 & her gein der stat durch hôhen muot.\\ 
 & ir vesperîe wart vil guot.\\ 
5 & ze beider sîte \textbf{roten} ungezalt,\\ 
 & garzûne, krîe manicvalt,\\ 
 & beidiu Schotsch unde Walsch\\ 
 & wart dâ geruofen sunder valsch.\\ 
10 & \hspace*{-.7em}\big| die helde erswungen dâ die lide,\\ 
 & \hspace*{-.7em}\big| der rîter tât was âne vride.\\ 
 & \textbf{wol tâtenz ouch diu selben} kint,\\ 
 & diu ûz dem her dâ komen sint.\\ 
 & die \textbf{begiengen} \textbf{werdiclî\textit{ch}e} tât,\\ 
 & die burgær pfanten si ûf der \textbf{sât}.\\ 
15 & \begin{large}D\end{large}er \textbf{nie} gediente \textbf{an} wîbe\\ 
 & \textbf{kleinœde}, der \textbf{en}\textbf{dorfte} an lîbe\\ 
 & nimer bezzer wât getragen.\\ 
 & \textbf{von Melianze} \textbf{hœre} ich sagen,\\ 
 & sîn zimiere wær guot.\\ 
20 & er het ouch selbe hôhen muot.\\ 
 & \textbf{er} reit ein schœne kastelân,\\ 
 & daz Meliahganz dort gewan,\\ 
 & dôr Kayn sô \textbf{verre} dar \textbf{hinder} stach,\\ 
 & daz man in \textbf{ame} aste hangen sach.\\ 
25 & dâz Meliahganz dort erstreit,\\ 
 & Melianz von Liz ez \textbf{hie} wol reit.\\ 
 & sîn tât was vor ûz \textbf{sô} \textbf{bekant},\\ 
 & alle sîne tjost in ir ougen vant\\ 
 & Obie dort ûf dem palas,\\ 
30 & dar si durch warten komen was.\\ 
\end{tabular}
\scriptsize
\line(1,0){75} \newline
G I O L M Q R Z Fr39 \newline
\line(1,0){75} \newline
\textbf{10} \textit{Initiale} I L Fr39  \textbf{9} \textit{Initiale} O R Z  \textbf{15} \textit{Initiale} G  \textbf{23} \textit{Initiale} I  \newline
\line(1,0){75} \newline
\textbf{1} ein tjost] ist ein tiost L (Fr39) eyn her M  $\cdot$ diu andere] ein ander I (O) (L) (M) (Q) (R) (Fr39) \textbf{2} her] er Q  $\cdot$ begunde] begvnde ovch O (L) (M) (Q) (R) (Z) (Fr39)  $\cdot$ trecken] treche O strechen L (R) crechen Z \textbf{3} her gein] Jn keyn M Her den R \textbf{5} beider sîte] beiden siten I (L) (Fr39) beidir siten M (Q)  $\cdot$ roten] rotte L Fr39  $\cdot$ ungezalt] mancualt I \textbf{6} krîe] Grien I æht ie O kýrie L (Fr39)  $\cdot$ manicvalt] vngezalt I \textbf{7} Schotsch] shottisch I schotech O schottisch L Fr39 Schottes M schothisz Q schottesch R Z  $\cdot$ Walsch] welsch I O L M Q R Fr39 \textbf{8} dâ] do Q Fr39  $\cdot$ geruofen] geruͤfet I (O) (L) (Q) (R) (Fr39)  $\cdot$ valsch] velwesh I velsch Fr39 \textbf{10} \textit{Versfolge 359.9-10} O Q R Z   $\cdot$ helde] helden R  $\cdot$ erswungen] swungen I erswuͯnden L (Fr39)  $\cdot$ dâ] do Q Fr39 \textit{om.} R  $\cdot$ die lide] diu lide I ierú gelidee R \textbf{9} der] er I ÷er O  $\cdot$ tât] tet Q rott Z \textbf{11} tâtenz] tanczin M \textbf{12} diu] die Fr39  $\cdot$ dâ] \textit{om.} O do Q \textbf{13} die] Div Fr39  $\cdot$ werdiclîche] werdchlie G werdiglichen Q  $\cdot$ tât] getat I O \textbf{14} pfanten si] enpfandes O phantes M (Q)  $\cdot$ der sât] dem [stade]: sat I der stat Q R \textbf{16} der endorfte] deern bedorft I der dort O dort dorft L Fr39 dorn dorfft R \textbf{17} wât] wart O M \textbf{18} Melianze] Melianz I Melyanz O melyanse Q Milianze R Meliantze Z  $\cdot$ hœre] hort Z \textbf{19} wær guot] wært gvͦte O \textbf{20} het] truͤc I  $\cdot$ selbe] selben M selber Q \textbf{21} ein] auch selbe ein I \textbf{22} daz] Da R  $\cdot$ Meliahganz] meliaganz I Melyanz O Meliahkanz L R Meliachkanz M melianz Q Meliahkantz Z meliaka:: Fr39 \textbf{23} Dôr] Da he M (Z)  $\cdot$ Kayn] kain G I kêyn O keýen L keyn M (Z) kein Q keyen R keien Fr39  $\cdot$ verre] hoch O (L) (M) (Q) (R) Z (Fr39)  $\cdot$ dar hinder] drab I hindir M \textbf{24} daz] Do L (Fr39) \textbf{25} dâz] da ez Z  $\cdot$ Meliahganz] meliahkanz G (L) Meliaganz I Melyakanz O Meliachkancz M meliahkans Q Maliahkancz R meliahkantz Z meliakanz Fr39 \textbf{26} Melianz] Melyanz O Meliancz R  $\cdot$ Liz ez] Lyez O Lizes L lisz M liesz es Q lis es R ::zes Fr39 \textbf{27} vor ûz sô] vns vor so O so vor uͯsz L (Fr39) vor so vsz M \textbf{28} alle] allev I Aller O  $\cdot$ in] \textit{om.} O M \textbf{29} Obie] Obŷe O Oblie Q Obye R Z  $\cdot$ dem] den Q \textbf{30} dar] Do Q das Fr39  $\cdot$ was] sach R \newline
\end{minipage}
\hspace{0.5cm}
\begin{minipage}[t]{0.5\linewidth}
\small
\begin{center}*T
\end{center}
\begin{tabular}{rl}
 & Hie \textbf{ist} ein tjost, \textbf{ein} anderiu dort.\\ 
 & daz her begunde \textbf{ouch} trecken vort\\ 
 & her gegen der stat durch hôhen muot.\\ 
 & ir vesperîe wart vil guot.\\ 
5 & ze beider sît \textbf{rotten} ungezalt,\\ 
 & garzûne, krîe manecvalt,\\ 
 & beidiu Schotesch unde Walsch\\ 
 & wart dâ gerüefet sunder valsch.\\ 
10 & \hspace*{-.7em}\big| die helde erswungen dâ die lide,\\ 
 & \hspace*{-.7em}\big| der rîter tât was âne vride.\\ 
 & \textbf{wol tâtenz dâ diu selben} kint,\\ 
 & die ûz dem her dar komen sint.\\ 
 & die \textbf{begiengen} \textbf{werdeclîche} tât,\\ 
 & die burgære pfantens ûf der \textbf{stat}.\\ 
15 & Der \textbf{ie} gediende wîbe\\ 
 & \textbf{kleinôt}, der \textbf{en}\textbf{dorft} an lîbe\\ 
 & niemer bezzer wât getragen\\ 
 & \textbf{danne Melyanz}, \textbf{hœre} ich sagen.\\ 
 & \multicolumn{1}{l}{ - - - }\\ 
20 & \multicolumn{1}{l}{ - - - }\\ 
 & \textbf{er} reit ein schœne kastelân,\\ 
 & daz Melyahganz dort gewan,\\ 
 & dô er Key sô \textbf{hôhe} dar \textbf{hinder} stach,\\ 
 & daz man \textit{in} \textbf{an einem} aste hangen sach.\\ 
25 & dô ez Melyahganz dort erstreit,\\ 
 & Melyanz von Lyz ez \textbf{hie} wol reit.\\ 
 & sîn tât was vor ûz \textbf{erkant}.\\ 
 & alle sîne tjost in ir ougen vant\\ 
 & Obye dort ûf dem palas,\\ 
30 & dar si durch warten komen was.\\ 
\end{tabular}
\scriptsize
\line(1,0){75} \newline
T V W \newline
\line(1,0){75} \newline
\textbf{1} \textit{Majuskel} T  \textbf{15} \textit{Majuskel} T   $\cdot$ \textit{Initiale} W  \newline
\line(1,0){75} \newline
\textbf{1} ist] wart V  $\cdot$ ein anderiu] die ander V W \textbf{2} trecken] ziehen V \textbf{5} beider sît] beidensiten V (W)  $\cdot$ ungezalt] gezalt W \textbf{7} Schotesch] schoͤttest W \textbf{8} dâ] do V W \textbf{10} dâ die] do die V do ir W \textbf{9} tât] getat W \textbf{11} dâ] do V W \textbf{13} werdeclîche] do vil werde V \textbf{15} ie] nie V  $\cdot$ wîbe] an wibe V \textbf{16} der endorft an] der moͤht an V in seinem W \textbf{17} Dorfft er bessers nie getragen W \textbf{18} danne] Von V W  $\cdot$ Melyanz] melianze V W  $\cdot$ hœre] hort V W \textbf{19} \textit{Die Verse 357.19-20 fehlen} T W   $\cdot$ Sin zimierde were guͦt V \textbf{20} Er hette oͮch selbe hohen muͦt V \textbf{22} Melyahganz] meliaganz V W \textbf{23} Key] keygin V gayen W  $\cdot$ sô hôhe] \textit{om.} W  $\cdot$ dar hinder] der nider V daruon W \textbf{24} in] \textit{om.} T  $\cdot$ aste] haste W \textbf{25} Melyahganz] meliaganz V W \textbf{26} Melyanz] Melianz V W  $\cdot$ Lyz] lys V lis W \textbf{27} erkant] so bekant V \textbf{28} alle] Das alle V  $\cdot$ ir] vor W \newline
\end{minipage}
\end{table}
\end{document}
