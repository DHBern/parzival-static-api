\documentclass[8pt,a4paper,notitlepage]{article}
\usepackage{fullpage}
\usepackage{ulem}
\usepackage{xltxtra}
\usepackage{datetime}
\renewcommand{\dateseparator}{.}
\dmyyyydate
\usepackage{fancyhdr}
\usepackage{ifthen}
\pagestyle{fancy}
\fancyhf{}
\renewcommand{\headrulewidth}{0pt}
\fancyfoot[L]{\ifthenelse{\value{page}=1}{\today, \currenttime{} Uhr}{}}
\begin{document}
\begin{table}[ht]
\begin{minipage}[t]{0.5\linewidth}
\small
\begin{center}*D
\end{center}
\begin{tabular}{rl}
\textbf{574} & \textit{\begin{large}M\end{large}}it vorhten luogete \textbf{oben} în;\\ 
 & des wart vil bleich ir \textbf{liehter} schîn.\\ 
 & diu \textbf{junge} sô verzagete,\\ 
 & daz ez diu alte klagete,\\ 
5 & Arnive, diu wîse.\\ 
 & dar umbe ich si noch prîse,\\ 
 & daz si den rîter nerte\\ 
 & unt im dô sterben werte.\\ 
 & Si gie ouch dar durch schouwen.\\ 
10 & dô wart von der vrouwen\\ 
 & zem venster \textbf{ob} în gesehen,\\ 
 & daz si \textbf{neweders} mohte jehen:\\ 
 & ir künfteclîcher vreuden tage\\ 
 & oder immer \textbf{herzenlîcher} klage.\\ 
15 & \textbf{si} vorhte, der rîter wære tôt;\\ 
 & des \textbf{lêrten} si \textbf{gedanke} nôt,\\ 
 & wander sus ûf dem lewen lac\\ 
 & unt anders deheines bettes pflac.\\ 
 & Si sprach: "mir ist von herzen leit,\\ 
20 & ob dîn getriwiu manheit\\ 
 & dîn werdez leben hât verlorn.\\ 
 & hâstû den tôt alhie erkorn\\ 
 & durch \textbf{uns} vil ellenden diet,\\ 
 & sît \textbf{dir} dîn triwe daz geriet,\\ 
25 & mich erbarmet immer dîn tugent,\\ 
 & dû habest alter oder jugent."\\ 
 & \textbf{Hin} zalden vrouwen si \textbf{dô} sprach,\\ 
 & wand si den helt \textbf{sus} ligen sach:\\ 
 & "ir vrouwen, die des \textbf{toufes} pflegen,\\ 
30 & rüefet alle \textbf{an got umbe sînen} segen."\\ 
\end{tabular}
\scriptsize
\line(1,0){75} \newline
D Fr7 Fr59 \newline
\line(1,0){75} \newline
\textbf{1} \textit{Initiale} D  \textbf{3} \textit{Initiale} Fr7  \textbf{9} \textit{Majuskel} D  \textbf{19} \textit{Majuskel} D  \textbf{27} \textit{Majuskel} D  \newline
\line(1,0){75} \newline
\textbf{1} Mit] ÷it D  $\cdot$ luogete] luͦget Fr7 \textbf{5} Arnive] Arnîve D \textbf{8} im dô] in da Fr7 \textbf{10} der] den Fr7 \textbf{11} ob] obene Fr7  $\cdot$ gesehen] gesehene Fr7 \textbf{12} mohte] mohten Fr7  $\cdot$ jehen] iehene Fr7 \textbf{13} künfteclîcher vreuden] kunfteclichen freude Fr7 \textbf{14} herzenlîcher] herzeliche Fr7 \textbf{17} sus ûf dem lewen] uf dem lewen sus Fr7 \textbf{18} bettes] bette Fr7 \textbf{19} von] vmb Fr7 \textbf{20} ob] ab Fr7  $\cdot$ getriwiu] getriwe Fr7 \textbf{23} ellenden] ellendiv Fr7 \textbf{25} immer] iemer me Fr59 \textbf{27} zalden] zalen Fr7 \textbf{30} rufet got an vmbe sin leben Fr7 \newline
\end{minipage}
\hspace{0.5cm}
\begin{minipage}[t]{0.5\linewidth}
\small
\begin{center}*m
\end{center}
\begin{tabular}{rl}
 & mit vorhten luog\textit{e}t \textbf{oben} în;\\ 
 & d\textit{e}s wart \textit{v}il bleich ir schîn.\\ 
 & diu \textbf{süeze} sô verzagete,\\ 
 & daz ez diu alte klagete,\\ 
5 & Ar\textit{niv}e, diu wîse.\\ 
 & dar umb ich si noch prîse,\\ 
 & daz si den ritter nerte\\ 
 & und ime dô sterben werte.\\ 
 & si gienc ouch d\textit{â} durch schouwen.\\ 
10 & dô wart von der vrouwen\\ 
 & zem venster \textbf{oben} în gesehen,\\ 
 & daz si \textbf{iet\textit{w}eders} moht jehen:\\ 
 & ir künftlîcher vröuden tage\\ 
 & oder iemer \textbf{herzelîcher} klage.\\ 
15 & \textbf{si} vorht, der ritter wær tôt;\\ 
 & des \textbf{lêrte} si \textbf{dankes} nôt,\\ 
 & wan er sus ûf dem lewen lac\\ 
 & und anders keines bettes pflac.\\ 
 & si sprach: "mir ist von herzen leit,\\ 
20 & ob dîn getriuwiu manheit\\ 
 & \dag sîn\dag  werdez leben het verlorn.\\ 
 & hâstû den tôt alhie erkorn\\ 
 & durch vil ellenden diet,\\ 
 & sît dîn triuwe daz geriet,\\ 
25 & mich erbarmet iemer \textit{\textbf{mê}} dîn tugent,\\ 
 & dû habest alter oder jugent."\\ 
 & zuo allen den vrouwen si \textbf{dô} sprach,\\ 
 & wan si den helt \textbf{d\textit{â}} ligen sach:\\ 
 & "ir vrouwen, die des pfl\textit{e}gen,\\ 
30 & rüefet alle \textbf{got an um\textit{b s}înen} segen."\\ 
\end{tabular}
\scriptsize
\line(1,0){75} \newline
m n o \newline
\line(1,0){75} \newline
\newline
\line(1,0){75} \newline
\textbf{1} luoget] lugent m \textbf{2} des wart vil] Das wart wil m \textbf{5} Arnive] Aruwe m Arniwe n Ariwe o \textbf{8} sterben] sterbe o \textbf{9} dâ] do m n o \textbf{12} ietweders] yetteders m  $\cdot$ moht] moͯchte n \textbf{14} herzelîcher] hertzecliche n \textbf{15} vorht] worcht o \textbf{16} des] Do o  $\cdot$ dankes] gedanckes n o \textbf{18} keines] komers o \textbf{21} leben] loben o \textbf{22} erkorn] verkorn o \textbf{23} ellenden] ellende n ellendes o \textbf{25} erbarmet] erbarmen o  $\cdot$ mê] nie m \textit{om.} n \textbf{27} sprach] [sprang]: sprach o \textbf{28} dâ] do m n \textbf{29} pflegen] pflagen m \textbf{30} an] \textit{om.} n  $\cdot$ umb sînen] vmb den sinen m \newline
\end{minipage}
\end{table}
\newpage
\begin{table}[ht]
\begin{minipage}[t]{0.5\linewidth}
\small
\begin{center}*G
\end{center}
\begin{tabular}{rl}
 & \begin{large}M\end{large}it vorhten luogete \textbf{ob\textit{en}} în;\\ 
 & des wart vil bleich ir schîn.\\ 
 & diu \textbf{junge} sô verzaget\textit{e},\\ 
 & daz ez diu alte klagete,\\ 
5 & Arnive, diu wîse.\\ 
 & dar umbe ich si noch brîse,\\ 
 & daz si den rîter nerte\\ 
 & unde im dô sterben werte.\\ 
 & si gienc ouch dar durch schouwen.\\ 
10 & dô wart von der vrouwen\\ 
 & zem venster \textbf{oben} în gesehen,\\ 
 & daz si \textbf{neweders} mohte jehen:\\ 
 & ir künfticlî\textit{ch}en vröuden tage\\ 
 & ode immer \textbf{herzenlîcher} klage.\\ 
15 & \textbf{si} vorhte, der rîter wære tôt;\\ 
 & des \textbf{lêrten} si \textbf{gedanke} nôt,\\ 
 & wand er sus ûf dem lewen lac\\ 
 & unde anders deheines bettes pflac.\\ 
 & si sprach: "mir ist von herzen leit,\\ 
20 & op dîn getriuwiu manheit\\ 
 & dîn werdez leben hât verlorn.\\ 
 & hâstû den tôt al hie erkorn\\ 
 & durch \textbf{uns} vil ellendiu diet,\\ 
 & sît \textbf{dir} dîn triuwe daz geriet,\\ 
25 & mich erbarmet immer \textbf{mê} dîn tugent,\\ 
 & dû habest alter ode jugent."\\ 
 & \textbf{hin} ze allen den vrouwen si sprach,\\ 
 & wan si den helt \textbf{sus} ligen sach:\\ 
 & "ir vrouwen, die des \textbf{toufes} pflegen,\\ 
30 & rüefet alle \textbf{an den gotes} segen."\\ 
\end{tabular}
\scriptsize
\line(1,0){75} \newline
G I L M Z Fr23 \newline
\line(1,0){75} \newline
\textbf{1} \textit{Initiale} G I L Z  \newline
\line(1,0){75} \newline
\textbf{1} oben] ob G \textbf{2} bleich ir] blaicher I  $\cdot$ schîn] lýchter schin L liehter schin Z \textbf{3} verzagete] uirzagit G \textbf{5} Arnive] Arnife L Arnuwe M \textbf{8} dô] so I daz L da M Z \textit{om.} Fr23 \textbf{9} si] Do Fr23 \textbf{10} dô] Da L M Z  $\cdot$ der] den L \textbf{11} oben în] ob in I \textbf{12} si neweders] sine weders G si dewederz I (Fr23) entweders L sie entwedirs M entweders sie Z  $\cdot$ mohte] mohten I L \textbf{13} künfticlîchen] kvnftichlien G chunftechlichez I kvniflichen L kuͤnstecliche Z  $\cdot$ tage] tagen I \textbf{14} immer] miner Fr23  $\cdot$ herzenlîcher] herczeliche M  $\cdot$ klage] chlagen I \textbf{16} gedanke] gedanken Z \textbf{18} deheines bettes] anders bettes nien Fr23 \textbf{20} dîn] diu I \textbf{22} hâstû] Hastu dv Fr23 \textbf{23} ellendiu] ellende I \textbf{25} mê] \textit{om.} L Fr23 \textbf{27} allen] \textit{om.} L  $\cdot$ si] diu frowe I \textbf{30} den gotes segen] got vmbe sinen segen L (Z) got dvrch syn segen M got vmb sin segen Fr23 \newline
\end{minipage}
\hspace{0.5cm}
\begin{minipage}[t]{0.5\linewidth}
\small
\begin{center}*T
\end{center}
\begin{tabular}{rl}
 & mit vorhten luoget\textit{e} \textbf{ob} în;\\ 
 & des wart vil bleich ir \textbf{liehter} schîn.\\ 
 & diu \textbf{junge} sô verzagte,\\ 
 & daz ez diu alte klagte,\\ 
5 & Arnyve, diu wîse.\\ 
 & dar umb ich si noch prîse,\\ 
 & daz si den ritter nerte\\ 
 & und im dô sterben werte.\\ 
 & si gienc ouch dar durch schouwen.\\ 
10 & dô wart von der vrouwen\\ 
 & zuom venster \textbf{oben} în gesehen,\\ 
 & daz s\textbf{deweders} mohte jehen:\\ 
 & ir künfticlîchen vreuden tage\\ 
 & oder immer \textbf{herzelîche} klage.\\ 
15 & \textbf{diu} vorhte, der ritter wære tôt;\\ 
 & des \textbf{lêrte} si \textbf{gedanken} nôt,\\ 
 & wan er sus ûf dem lewen lac\\ 
 & und anders keines bettes pflac.\\ 
 & si sprach: "mir ist von herzen leit,\\ 
20 & ob dîn getriuwiu manheit\\ 
 & dîn werdez leben hât verlorn.\\ 
 & hâstû den tôt alhie erkorn\\ 
 & durch \textbf{uns} vil ellendiu diet,\\ 
 & s\textit{ît} \textbf{dir} dîn triuwe daz geriet,\\ 
25 & \textit{m}ich erbarmet immer \textbf{mê} dîn tugent,\\ 
 & dû habest alter oder jugent."\\ 
 & \textbf{hin}z al den vrouwen \textit{si} sprach,\\ 
 & wan si den helt \textbf{sus} ligen sach:\\ 
 & "ir vrouwen, die des \textbf{toufes} pflegen,\\ 
30 & rüeft al \textbf{an got umb sînen} segen."\\ 
\end{tabular}
\scriptsize
\line(1,0){75} \newline
Q R W V U \newline
\line(1,0){75} \newline
\textbf{3} \textit{Capitulumzeichen} R  \textbf{9} \textit{Initiale} V  \newline
\line(1,0){75} \newline
\textbf{1} \textit{Die Verse 553.1-599.30 fehlen} U   $\cdot$ luogete] luchten Q  $\cdot$ ob] obnen R oben W [*]: oben V \textbf{2} liehter] lichter Q \textbf{4} ez] \textit{om.} R \textbf{7} ritter nerte] rirter ernerte W \textbf{8} \textit{Vers 574.8 ist am Rand nachgetragen und später radiert:} v: im sin s:e V   $\cdot$ dô] sin V \textbf{9} durch] nach R \textbf{10} der] den R \textbf{11} oben în] obnan R obenan in V \textbf{12} mohte] mochten R moͤchte W (V) \textbf{13} vreuden] froͤide V \textbf{14} immer] Jamer R \textbf{15} diu] Sy R W (V)  $\cdot$ vorhte] [forchte]: forchten R \textbf{16} Dez lerten sv́ [gedanken *]: gedanke not V  $\cdot$ des] Das W  $\cdot$ gedanken] gedenke R gedencken W \textbf{17} sus ûf] als vff Q vf R (V) \textbf{21} hât] hab R \textbf{22} den tôt] dein leib W  $\cdot$ alhie] hie R \textbf{23} Durch uns] [D*]: Durch vnz V \textbf{24} sît] Sein Q \textbf{25} mich] Jch Q  $\cdot$ mê] \textit{om.} W \textbf{27} hinz al] Hin zallen V  $\cdot$ si] \textit{om.} Q \textbf{28} sus] als Q \textbf{29} des] \textit{om.} W \newline
\end{minipage}
\end{table}
\end{document}
