\documentclass[8pt,a4paper,notitlepage]{article}
\usepackage{fullpage}
\usepackage{ulem}
\usepackage{xltxtra}
\usepackage{datetime}
\renewcommand{\dateseparator}{.}
\dmyyyydate
\usepackage{fancyhdr}
\usepackage{ifthen}
\pagestyle{fancy}
\fancyhf{}
\renewcommand{\headrulewidth}{0pt}
\fancyfoot[L]{\ifthenelse{\value{page}=1}{\today, \currenttime{} Uhr}{}}
\begin{document}
\begin{table}[ht]
\begin{minipage}[t]{0.5\linewidth}
\small
\begin{center}*D
\end{center}
\begin{tabular}{rl}
\textbf{722} & \begin{large}M\end{large}it Beacurse komen sint\\ 
 & \textbf{mêr danne} vünfzec \textbf{clâriu} kint,\\ 
 & die von \textbf{ir} art gâben liehten schîn,\\ 
 & herzogen und grævelîn.\\ 
5 & dâ reit ouch \textbf{etslîch} küneges sun.\\ 
 & \textbf{dô sach man} grôz enpfâhen tuon\\ 
 & von den kinden ze bêder sît;\\ 
 & \textbf{si} enpfiengen ein ander âne nît.\\ 
 & Beacurs \textbf{pflac} varwe lieht.\\ 
10 & der künec sich \textbf{vrâgens} sûmte niht.\\ 
 & \textbf{Bene im sagete} mære,\\ 
 & \textbf{wer} der clâre rîter wære:\\ 
 & "\textbf{ez ist} Beacurs, Lotes kint."\\ 
 & \textbf{dô dâhter: "herze, nû} vint\\ 
15 & si, diu dem gelîche,\\ 
 & der hie rîtet \textbf{sô} minneclîche.\\ 
 & si ist \textbf{vür wâr} sîn swester,\\ 
 & diu, geworht in Sinzester,\\ 
 & mit ir spærwære sande \textbf{mir} \textbf{den} huot.\\ 
20 & ob si mir \textbf{mêr} \textbf{genâde} tuot,\\ 
 & \textbf{al} \textbf{irdischiu} rîcheit\\ 
 & \textbf{ob derde wære noch als} breit,\\ 
 & dâ vür næme ich si einen.\\ 
 & si solz mit triwen \textbf{meinen}.\\ 
25 & ûf ir genâde kum ich hie.\\ 
 & si hât mich \textbf{sô} getrœstet ie,\\ 
 & ich getrûwe \textbf{ir} wol, \textbf{daz} si mir tuot,\\ 
 & dâ von sich \textbf{hœhert} baz mîn muot."\\ 
 & Er nam ir clâren bruoder hant\\ 
30 & in die sîne. diu \textbf{was ouch} lieht erkant.\\ 
\end{tabular}
\scriptsize
\line(1,0){75} \newline
D \newline
\line(1,0){75} \newline
\textbf{1} \textit{Initiale} D  \textbf{29} \textit{Majuskel} D  \newline
\line(1,0){75} \newline
\textbf{1} Beacurse] Beahcvrse D \textbf{9} Beacurs] Beahcvrs D \textbf{13} Beacurs] Beahvcrs D  $\cdot$ Lotes] Lots D \newline
\end{minipage}
\hspace{0.5cm}
\begin{minipage}[t]{0.5\linewidth}
\small
\begin{center}*m
\end{center}
\begin{tabular}{rl}
 & \begin{large}M\end{large}it Beakurs komen sint\\ 
 & \textbf{mê dan} vünfzic kint,\\ 
 & die von \textbf{ir} art gâben liehten schîn,\\ 
 & herzogen und grævelîn.\\ 
5 & d\textit{â} reit ouch \textbf{etlîches} küniges sun.\\ 
 & \textbf{dô sach man} grôz enpfâhen tuon\\ 
 & von den kinden zuo beider sît;\\ 
 & \textbf{si} enpfiengen ein ander âne nît.\\ 
 & Beakurs \textbf{pflac} varwe lieht.\\ 
10 & der künic sich \textbf{vrâgens} sûmte niht.\\ 
 & \textbf{Bene im sagte} mære,\\ 
 & \textbf{daz} der clâre ritter wære\\ 
 & Beacurs, Lotes kint.\\ 
 & \textbf{dô gedâht er: "herz, nû} vint\\ 
15 & si, \textit{diu} dem glîch,\\ 
 & der hie rîtet minneclîch.\\ 
 & si ist \textbf{vür wâr} sîn swester,\\ 
 & diu, geworht in Sinzester,\\ 
 & mit ir sperwer sant \textbf{mir} \textbf{den} huot.\\ 
20 & ob si mir \textbf{mê} \textbf{genâden} tuot,\\ 
 & \textbf{alle} \textbf{irdensch} rîcheit,\\ 
 & \textbf{wær diu erd\textit{e n}och alsô} breit,\\ 
 & d\textit{â} vür næme ich si einen.\\ 
 & si solz mit triuwen \textbf{scheinen}.\\ 
25 & ûf ir gnâde kome ich hie.\\ 
 & si het mich \textbf{sô} getrœstet ie,\\ 
 & ich getrûwe \textbf{ir} wol, \textbf{waz} si mir tuot,\\ 
 & d\textit{â} von sich \dag hœher\dag  baz \textit{m}î\textit{n m}uot."\\ 
 & er nam ir clâren bruoder hant\\ 
30 & in die sîne. diu \textbf{was ouch} lieht erkant.\\ 
\end{tabular}
\scriptsize
\line(1,0){75} \newline
m n o Fr69 \newline
\line(1,0){75} \newline
\textbf{1} \textit{Initiale} m   $\cdot$ \textit{Capitulumzeichen} n  \newline
\line(1,0){75} \newline
\textbf{1} Beakurs] beatuͯrs o \textbf{3} liehten] lichten Fr69 \textbf{4} herzogen] Herczogin o \textbf{5} dâ] Do m n o  $\cdot$ etlîches] etlich n (o) \textbf{9} Beakurs] Beakúrs n Beatuͯrs o \textbf{11} mære] \sout{lag} mere o \textbf{13} Beacurs] Beakúrs n Beatus o  $\cdot$ Lotes] lotzs m lots n lottes o \textbf{15} diu] \textit{om.} m \textbf{16} hie] do hie n  $\cdot$ minneclîch] so mẏnneclich n o \textbf{18} Sinzester] sin zester m n o \textbf{19} mit] Mir o \textbf{22} noch] nit [a]: noch m \textbf{23} dâ] Do m n o \textbf{24} scheinen] scheine o \textbf{25} hie] [eẏne]: hie o \textbf{27} waz] das n o \textbf{28} dâ] Do m n o  $\cdot$ mîn muot] sin muͯs muͯt m \newline
\end{minipage}
\end{table}
\newpage
\begin{table}[ht]
\begin{minipage}[t]{0.5\linewidth}
\small
\begin{center}*G
\end{center}
\begin{tabular}{rl}
 & \begin{large}M\end{large}it Beakurs komen sint\\ 
 & \textbf{mê dane} vünf\textit{ze}c \textbf{clâriu} kint,\\ 
 & die von arde gâben liehten schîn,\\ 
 & herzogen unde grævelîn.\\ 
5 & dâ reit ouch \textbf{etslîch} küniges sun.\\ 
 & \textbf{man sach dâ} grôz enpfâhen tuon\\ 
 & von den kinden ze bêder sît;\\ 
 & \textbf{die} enpfiengen ein ander \textit{âne} nît.\\ 
 & Beakurs \textbf{truog} varwe lieht.\\ 
10 & der künic sich \textbf{vrâge} sûmte niht.\\ 
 & \textbf{im sagte Bene} mære,\\ 
 & \textbf{wer} der clâre rîter wære:\\ 
 & "\textbf{ez ist} Beakurs, Lotes kint."\\ 
 & \textbf{dô dâht er: "herze, nû} vint\\ 
15 & si, diu dem glîche,\\ 
 & der hie rîtet \textbf{sô} minniclîche.\\ 
 & si ist \textbf{benamen} sîn swester,\\ 
 & diu, geworht in Sincester,\\ 
 & mit ir sparwære sande \textbf{ir} huot.\\ 
20 & op si mir \textbf{gnâde} tuot,\\ 
 & \hspace*{-.7em}\big| \textbf{op diu erde wære sô breit},\\ 
 & \hspace*{-.7em}\big| \textbf{alle} \textbf{irdeschen} rîcheit,\\ 
 & dâ vür næme ich si einen.\\ 
 & si solz mit triwen \textbf{meinen}.\\ 
25 & ûf ir gnâde kum ich hie.\\ 
 & si hât mich \textbf{doch} getrœstet ie,\\ 
 & ich getrûwe \textbf{ir} wol, \textbf{daz} si mir tuot,\\ 
 & dâ von sich \textbf{hœhet} baz mîn muot."\\ 
 & er nam ir clâren bruoder hant\\ 
30 & in die sîne. diu \textbf{ouch was} lieht erkant.\\ 
\end{tabular}
\scriptsize
\line(1,0){75} \newline
G I L M Z Fr20 Fr45 \newline
\line(1,0){75} \newline
\textbf{1} \textit{Initiale} G L Z Fr20  \textbf{11} \textit{Initiale} I  \newline
\line(1,0){75} \newline
\textbf{1} Mit] ÷it Fr20  $\cdot$ Beakurs] beacurse I M beakuͯrs L beachur Fr20 beakuͦrs Fr45 \textbf{2} vünfzec] fvnfch G  $\cdot$ clâriu] \textit{om.} I M cleiniv Fr20 (Fr45) \textbf{3} liehten] lýchten L (M) (Fr45) \textbf{5} etslîch küniges] ettisliches greven M (Fr45) \textbf{6} sach] mac M  $\cdot$ dâ] \textit{om.} Fr45  $\cdot$ enpfâhen] ein pfahen L \textbf{8} enpfiengen] einpfingen L  $\cdot$ âne] \textit{om.} G Z \textbf{9} Beakurs] beacurs I Beakuͯrs L Bakurs M beachurs Fr20 Beakuͦrs Fr45  $\cdot$ truog] mit Fr45  $\cdot$ lieht] lýcht L (M) \textbf{10} vrâge] fragens I fragen Z varwe Fr20  $\cdot$ sûmte] svmet L (M) Z svnde Fr20 \textbf{11} im] nv Fr20 Vnd Fr45  $\cdot$ sagte] seit I (L) (Z) vragte Fr45  $\cdot$ Bene] benen Fr45 \textbf{13} Beakurs] beacurs I (M) Beakuͯrs L beachurs Fr20 beakuͦrs Fr45  $\cdot$ Lotes] lots G lotis Fr20 lothes Fr45 \textbf{14} dô] Da L M \textbf{15} diu] \textit{om.} Fr45 \textbf{18} diu] \textit{om.} L  $\cdot$ geworht] wor::: Fr45  $\cdot$ in] uon Fr20  $\cdot$ Sincester] zincester I sin zester L sin Zcester M sincestir Fr20 \textbf{19} mit] Die mit L  $\cdot$ sande] ::: sante Fr45  $\cdot$ ir huot] er ir einen huͤt I mir [*]: ir hvt Z \textbf{22} sô] noch so M nach also Z (Fr45) \textbf{21} alle irdeschen] aller diser I Vnd alle irdische M an erdescher Fr45 \textbf{23} einen] eine Z \textbf{24} meinen] meine Z \textbf{28} hœhet baz] hohet gar M hohet daz Z hohen sol Fr45 \textbf{30} die] \textit{om.} Fr45  $\cdot$ sîne] sinen Z  $\cdot$ ouch] \textit{om.} L M Fr45  $\cdot$ lieht] licht L \newline
\end{minipage}
\hspace{0.5cm}
\begin{minipage}[t]{0.5\linewidth}
\small
\begin{center}*T
\end{center}
\begin{tabular}{rl}
 & mit Beakurse komen sint\\ 
 & \textbf{wol} vünfzic \textbf{clâriu} kint,\\ 
 & die von arte gâben liehten schîn,\\ 
 & herzogen und grævelîn.\\ 
5 & d\textit{â} reit ouch \textbf{etlîch} küneges sun.\\ 
 & \textbf{man sach d\textit{â}} grôz entvâhen tuon\\ 
 & von den kinden zuo beider sît;\\ 
 & \textbf{die} entviengen ein ander âne nît.\\ 
 & \begin{large}B\end{large}eakurs \textbf{truoc} varwe lieht.\\ 
10 & der künec sich \textbf{vrâge} sûmete niht.\\ 
 & \textbf{im sagete Bene} mære,\\ 
 & \textbf{wer} der clâre rîter wære:\\ 
 & "\textbf{ez ist} Beakurs, Lotes kint."\\ 
 & \textbf{"nû herze", dâhter}, "vint\\ 
15 & si, diu dem \textbf{sî} gelîch,\\ 
 & der hie rîtet \textbf{sô} minneclîch.\\ 
 & si ist \textbf{bînamen} sîn swester,\\ 
 & diu, geworht in Sincester,\\ 
 & mit ir sperwære sante \textbf{mir} \textbf{ir} huot.\\ 
20 & o\textit{b} si mir \textbf{gnâde} tuot,\\ 
 & \hspace*{-.7em}\big| \textbf{o\textit{b} die erde wære alsô breit},\\ 
 & \hspace*{-.7em}\big| \textbf{alle} \textbf{irdische} rîcheit,\\ 
 & d\textit{â} vür næme ich si einen.\\ 
 & si sol ez mit triuwen \textbf{meinen}.\\ 
25 & ûf ir gnâde kome ich hie.\\ 
 & si hât mich \textbf{sô} getrœste\textit{t} ie,\\ 
 & ich getrûwe wol, \textbf{daz} si mir tuot,\\ 
 & d\textit{â} von sich \textbf{hebet} baz mîn muot."\\ 
 & er nam i\textit{r} clâren bruoder hant\\ 
30 & in die sîne. diu \textbf{was ouch} lieht erkant.\\ 
\end{tabular}
\scriptsize
\line(1,0){75} \newline
U V W Q R \newline
\line(1,0){75} \newline
\textbf{1} \textit{Initiale} Q  \textbf{9} \textit{Initiale} U V W  \newline
\line(1,0){75} \newline
\textbf{1} Beakurse] Beakvrs V (W) (Q) Beakursen R \textbf{2} wol] Mer danne V (W) (Q) (R)  $\cdot$ clâriu] cleine R \textbf{3} \textit{Versdoppelung 723.22 nach 722.3} R   $\cdot$ liehten] lichten Q \textbf{5} dâ] Do U V W Q  $\cdot$ etlîch] etteliches V (Q) \textbf{6} dâ] do U V W Q \textbf{7} von] Zu Q \textbf{8} nît] streit W \textbf{9} Beakurs] Beakuͦrs U bEakurß W beakursz Q Beakuͯrs R  $\cdot$ lieht] licht Q \textbf{10} sich] \textit{om.} W  $\cdot$ vrâge] fragens V fragte Q  $\cdot$ sûmete] saumet W sewmen Q schwinde R \textbf{13} Beakurs] [*akvrs]: Beakvrs V beakursz Q beakúrs R  $\cdot$ Lotes] lottes W \textbf{14} nû herze dâhter] Do dahte er herze nv V (W) (Q) (R)  $\cdot$ vint] muͯt Q \textbf{15} diu] \textit{om.} Q  $\cdot$ sî] \textit{om.} R \textbf{17} bînamen] fúrwar W \textbf{18} Die worcht in senitester Q \textbf{19} ir sperwære] dem sperber W  $\cdot$ sante] santtent R  $\cdot$ ir huot] [*]: den hvͦt V \textbf{20} ob] Oder U  $\cdot$ gnâde] [*]: mer genaden V \textbf{22} ob] Oder U  $\cdot$ alsô] noch also V W (Q) noch so R \textbf{21} alle irdische] Alle irdensche V Als erdesche R \textbf{23} dâ] Do U V W Q \textbf{24} si] Du Q So R  $\cdot$ sol ez] [solt]: sol es V solt Q  $\cdot$ mit] mir R \textbf{26} sô getrœstet] so getroste U also getroͤstet W getrostet so Q getroͯstet R \textbf{27} wol] [*]: ir wol V ir wol Q (R)  $\cdot$ mir] mirs V \textbf{28} dâ] Do U V W Q  $\cdot$ hebet] hoͤhet V W (Q) (R)  $\cdot$ baz] \textit{om.} Q \textbf{29} nam] man Q  $\cdot$ ir] irre U irs W \textbf{30} die] \textit{om.} W  $\cdot$ lieht] [*]: lieht V licht Q liecht gnuͯg R \newline
\end{minipage}
\end{table}
\end{document}
