\documentclass[8pt,a4paper,notitlepage]{article}
\usepackage{fullpage}
\usepackage{ulem}
\usepackage{xltxtra}
\usepackage{datetime}
\renewcommand{\dateseparator}{.}
\dmyyyydate
\usepackage{fancyhdr}
\usepackage{ifthen}
\pagestyle{fancy}
\fancyhf{}
\renewcommand{\headrulewidth}{0pt}
\fancyfoot[L]{\ifthenelse{\value{page}=1}{\today, \currenttime{} Uhr}{}}
\begin{document}
\begin{table}[ht]
\begin{minipage}[t]{0.5\linewidth}
\small
\begin{center}*D
\end{center}
\begin{tabular}{rl}
\textbf{637} & \begin{large}M\end{large}în kunst mir des niht \textbf{halbes} giht,\\ 
 & i\textbf{ne} bin solch küchen meister niht,\\ 
 & daz ich die spîse kunne \textbf{sagen},\\ 
 & diu dâ mit \textbf{zuht} wart vür getragen.\\ 
5 & dem wirte unt den vrouwen gar\\ 
 & dienden meide wol gevar.\\ 
 & Anderhalp den rîtern an \textbf{ir} want,\\ 
 & \textbf{den} diende manec sarjant.\\ 
 & Ein \textbf{vorhtlîch} zuht si des \textbf{betwanc},\\ 
10 & daz \textbf{sich} der knappen keiner \textbf{dranc}\\ 
 & mit den juncvrouwen.\\ 
 & man muoste si sunder schouwen,\\ 
 & si trüegen spîse oder wîn;\\ 
 & sus muosen si mit zühten sîn.\\ 
15 & Si mohten dô wol wirtschaft jehen.\\ 
 & ez \textbf{was in} selten ê \textbf{geschehen},\\ 
 & de\textit{n} vrouwen unt der rîterschaft,\\ 
 & sît si Clinschors kraft\\ 
 & mit sînen listen überwant.\\ 
20 & si wâren ein ander unbekant\\ 
 & unt beslôz si doch ein porte,\\ 
 & daz si ze gegenworte\\ 
 & nie kômen, vrouwen \textbf{noch} die man.\\ 
 & Dô schuof mîn hêr Gawan,\\ 
25 & daz \textbf{diz} volc ein ander sach,\\ 
 & dâr \textbf{an} in liebes vil \textbf{geschach}.\\ 
 & Gawane was ouch \textbf{liep} \textbf{geschehen},\\ 
 & doch muoser \textbf{tougenlîchen} sehen\\ 
 & an die clâren herzoginne;\\ 
30 & \textbf{diu} twanc sînes herzen \textbf{sinne}.\\ 
\end{tabular}
\scriptsize
\line(1,0){75} \newline
D Z Fr1 \newline
\line(1,0){75} \newline
\textbf{1} \textit{Initiale} D Z Fr1  \textbf{7} \textit{Majuskel} D  \textbf{9} \textit{Majuskel} D  \textbf{15} \textit{Majuskel} D  \textbf{24} \textit{Majuskel} D  \newline
\line(1,0){75} \newline
\textbf{4} zuht] zvhten Z \textbf{8} den] \textit{om.} Z \textbf{9} zuht] sorge Z \textbf{12} muoste] sust Z \textbf{13} trüegen] trvgen Z \textbf{15} dô] da Z  $\cdot$ jehen] sehen Z \textbf{17} den] der D \textbf{18} Clinschors] Clinscors D Clingesores Z Clẏnscors Fr1 \textbf{21} si] \textit{om.} Z  $\cdot$ porte] borte Fr1 \textbf{23} vrouwen] die frowen Z \textbf{24} Dô] Da Z \textbf{27} Gawane] Gawan Z \textbf{30} sinne] sin Z \newline
\end{minipage}
\hspace{0.5cm}
\begin{minipage}[t]{0.5\linewidth}
\small
\begin{center}*m
\end{center}
\begin{tabular}{rl}
 & mîn kunst mir des niht \textbf{halben} giht,\\ 
 & ich bin solich küchen meister niht,\\ 
 & daz ich die spîse kunne \textbf{sagen},\\ 
 & diu d\textit{â} mit \textbf{zuht} wart v\textit{ü}r \textit{ge}tragen.\\ 
5 & dem wirt und den vrowen gar\\ 
 & dienden megde wol gevar.\\ 
 & anderhalp den rittern an \textbf{der} want\\ 
 & diende manic sarjant.\\ 
 & ein \textbf{vorhtlîch} zuht si des \textbf{betwanc},\\ 
10 & daz \textbf{sich} der knappen keiner \textbf{dranc}\\ 
 & mit den juncvrouwen.\\ 
 & man muoste si sunder schouwen,\\ 
 & si trüegen spîse oder wîn;\\ 
 & \textit{su}s muoste\textit{n} si mit zühten sîn.\\ 
15 & si mohten dô wol wirtschaft jehen.\\ 
 & ez \textbf{was in} selten ê \textbf{beschehen},\\ 
 & den vrowen und der ritterschaft,\\ 
 & sît si Clinsors kraft\\ 
 & mit sînen listen überwant.\\ 
20 & si wâren ein ander unbekant\\ 
 & und beslô\textit{z} si doch ein porte,\\ 
 & daz si ze gegenworte\\ 
 & nie kômen, \textit{vrowen} \textbf{und} die \textit{man}.\\ 
 & dô schuof mîn hêr Gawan,\\ 
25 & daz \textbf{diz} volc ein ander sach,\\ 
 & dâr \textbf{an} in liebes vil \textbf{beschach}.\\ 
 & Gawan was ouch \textbf{liep} \textbf{beschehen},\\ 
 & doch muost er \textbf{tugentlîchen} sehen\\ 
 & an die clâren herzoginne;\\ 
30 & \textbf{diu} twanc sînes herzen \textbf{minne}.\\ 
\end{tabular}
\scriptsize
\line(1,0){75} \newline
m n o \newline
\line(1,0){75} \newline
\newline
\line(1,0){75} \newline
\textbf{1} mir des] des mir n o  $\cdot$ halben] halbes n o \textbf{3} kunne] kuͯnde o \textbf{4} dâ mit] do mit m n >do mit< o  $\cdot$ vür getragen] vertragen m \textbf{5} und] vnd ouch n \textbf{7} der] \textit{om.} n ir o \textbf{8} diende] Do ende n \textbf{9} betwanc] twang n \textbf{10} daz] Des o \textbf{13} trüegen] tragen o \textbf{14} sus] Das m  $\cdot$ muosten] muͯstte m (n) (o) \textbf{15} mohten] moͯchten n \textbf{20} unbekant] wol bekant o \textbf{21} beslôz] beslossen m beslosse n o  $\cdot$ doch] durch o \textbf{23} Nie komen man vnd die frowen m \textbf{24} hêr] herre her n \textbf{26} Dar an vil liebes in do geschach n \textbf{27} beschehen] [g]: beschehen o \textbf{28} doch] Do n \textbf{29} an] \textit{om.} n \textbf{30} minne] sinne n sin o \newline
\end{minipage}
\end{table}
\newpage
\begin{table}[ht]
\begin{minipage}[t]{0.5\linewidth}
\small
\begin{center}*G
\end{center}
\begin{tabular}{rl}
 & \begin{large}M\end{large}în kunst mir des niht \textbf{halbes} giht,\\ 
 & ich \textbf{en}bin sol\textit{ch} küchen meister niht,\\ 
 & daz ich die spîse kunne \textbf{sagen},\\ 
 & diu dâ mit \textbf{zühten} wart vür getragen.\\ 
5 & dem wirte unde den vrouwen gar\\ 
 & dienden meide wol gevar.\\ 
 & anderhalben den rîtern an \textbf{der} want\\ 
 & diende manic sarjant.\\ 
 & ein \textbf{wertlîch} zuht si des \textbf{betwanc},\\ 
10 & daz der knappen deheiner \textbf{dranc}\\ 
 & mit den juncvrouwen.\\ 
 & man muose si sunder schouwen,\\ 
 & si trüegen spîse oder wîn;\\ 
 & sus muosen si mit zühten sîn.\\ 
15 & si mohten dâ wol wirtschaft jehen.\\ 
 & \textit{ez} \textit{\textbf{ist}} selten ê \textbf{geschehen}\\ 
 & den vrouwen unde der rîterschaft,\\ 
 & sît \textbf{daz} si Clinsor\textit{s} kraft\\ 
 & mit sînen listen überwant.\\ 
20 & si wâren ein ander unbekant\\ 
 & unde beslôz si doch ein borte,\\ 
 & daz si ze gegenworte\\ 
 & nie kômen, vrouwen \textbf{noch} die man.\\ 
 & dô schuof mîn hêrre Gawan,\\ 
25 & daz \textbf{diz} volc ein ander sach,\\ 
 & dâr \textbf{an} in liebes vil \textbf{geschach}.\\ 
 & Gawan was ouch \textbf{liep} \textbf{geschehen},\\ 
 & doch muos er \textbf{tougenlîchen} sehen\\ 
 & an die clâren herzogîn;\\ 
30 & \textbf{diu} twanc sînes herzen \textbf{sin}.\\ 
\end{tabular}
\scriptsize
\line(1,0){75} \newline
G I L M Z Fr18 \newline
\line(1,0){75} \newline
\textbf{1} \textit{Initiale} G L Z Fr18  \textbf{21} \textit{Initiale} I  \newline
\line(1,0){75} \newline
\textbf{2} enbin] bin I Fr18  $\cdot$ solch küchen] solhe chuchen G solher chunste I \textbf{4} zühten] zuͯcht L \textbf{7} \textit{Die Verse 637.7-8 fehlen} L   $\cdot$ der] ir M Z \textbf{9} wertlîch zuht] werclich zuͯcht L forhtlich sorge Z \textbf{10} daz] Daz sich Z  $\cdot$ dranc] tranc I \textbf{12} man] ma I  $\cdot$ muose] sust Z  $\cdot$ si] \textit{om.} M \textbf{13} trüegen] trvgen Z \textbf{14} sus muosen si] sus musten I Si muͯsten suͯsz L \textbf{15} dâ wol wirtschaft] wol wirtschaft da L  $\cdot$ jehen] sehen Z \textbf{16} ez ist] \textit{om.} G Ez waz in L (M) (Z) (Fr18)  $\cdot$ ê] \textit{om.} L \textbf{18} daz] \textit{om.} L M Z Fr18  $\cdot$ Clinsors] chlinshorf G Clinisors L klinsors M Clingesores Z Clinshors Fr18 \textbf{21} si] \textit{om.} Z  $\cdot$ borte] porte I L Fr18 phorte M \textbf{23} vrouwen] die vrowen L (Z) \textbf{24} dô] Da M Z  $\cdot$ hêrre Gawan] ergawan M \textbf{25} diz volc] [dirvolch]: dizvolch G daz volc I (L) (M) Fr18 \textbf{26} in liebes vil] in beiden liep I vil liebe in L \textbf{27} Gawan] Gawane M \textbf{28} muos er] muͤster I \newline
\end{minipage}
\hspace{0.5cm}
\begin{minipage}[t]{0.5\linewidth}
\small
\begin{center}*T
\end{center}
\begin{tabular}{rl}
 & mîne kunst mir des niht \textbf{halbez} giht,\\ 
 & ich \textbf{en}bin solicher küchen meister niht,\\ 
 & daz ich die spîse kunne \textbf{gesagen},\\ 
 & diu d\textit{â} mit \textbf{zuht} wart vür getragen.\\ 
5 & dem wirte und den vrouwen gar\\ 
 & dieneten megde wol gevar.\\ 
 & anderhalp den rîtern an \textbf{ir} want\\ 
 & dienete manec sarjant.\\ 
 & eine \textbf{volleclîchiu} zuht si des \textbf{twanc},\\ 
10 & daz der knappen dekeiner \textbf{tranc}\\ 
 & mit den juncvrouwen.\\ 
 & man muose si sunder schouwen,\\ 
 & si trüegen spîse oder wîn;\\ 
 & sus muosen si mit zühten sîn.\\ 
15 & si mohten dô wol wirtschaft jehen.\\ 
 & ez \textbf{was in} selten ê \textbf{geschehen},\\ 
 & den vrouwen und der rîterschaft,\\ 
 & sît si Clynsors kraft\\ 
 & mit sînen listen überwant.\\ 
20 & si wâren ein ander unbekant\\ 
 & und beslôz si doch eine porte,\\ 
 & daz si zuo gegenwort\textit{e}\\ 
 & nie kâmen, vrouwen \textbf{noch} die man.\\ 
 & dô schuof mîn hêr Gawan,\\ 
25 & daz \textbf{daz} volc ein ander sach,\\ 
 & dâ in liebes vil \textbf{geschach}.\\ 
 & Gawan was ouch \textbf{liebe} \textbf{geschehen},\\ 
 & doch muos er \textbf{tugentlîche} sehen\\ 
 & an die clâren herzoginne;\\ 
30 & \textbf{des} twanc \textbf{in} sînes herzen \textbf{sinne}.\\ 
\end{tabular}
\scriptsize
\line(1,0){75} \newline
U V W Q R Fr40 \newline
\line(1,0){75} \newline
\textbf{1} \textit{Initiale} Q Fr40   $\cdot$ \textit{Capitulumzeichen} R  \textbf{27} \textit{Initiale} W  \newline
\line(1,0){75} \newline
\textbf{1} mîne] Sin R  $\cdot$ mir] \textit{om.} W  $\cdot$ niht] \textit{om.} Q R Fr40  $\cdot$ halbez] halben R halbes Fr40 \textbf{2} enbin] bin V W Q R (Fr40)  $\cdot$ solicher] ein solch W soͯlichs R (Fr40)  $\cdot$ küchen meister] kuchenmeisters Fr40 \textbf{3} gesagen] sagen W Fr40 \textbf{4} dâ] do U V W Q  $\cdot$ zuht] zv́hten V  $\cdot$ vür] dar W \textbf{7} anderhalp] Innerhalb W  $\cdot$ ir] der V (W) \textbf{8} dienete] Dienten V \textbf{9} volleclîchiu] vorhtlich V (W) (Q) (Fr40) froͯlich R  $\cdot$ twanc] betzwanck Q (R) (Fr40) \textbf{10} daz] [D*]: Daz sich V Das sich W  $\cdot$ tranc] drang V W \textbf{12} muose] mvͤste V \textbf{13} trüegen] [*gen]: trvͤgen V trugen Fr40 \textbf{14} sus] Als Q  $\cdot$ muosen] mvͤsten V (Fr40) \textbf{15} mohten] moͤhten V (W) (R)  $\cdot$ dô] da Fr40 \textbf{16} ê] ie Fr40  $\cdot$ geschehen] beschehen W \textbf{18} Clynsors] [Cy*]: Clynsors U clinsors V klynshors W clinshors Q Clingshors R klinshors Fr40 \textbf{21} porte] [p*]: porte U pforte Q \textbf{22} gegenworte] gein worten U \textbf{24} Gawan] gawann Q \textbf{25} daz volc] dis volck W  $\cdot$ ander] andren R \textbf{26} dâ] Dar an V W Q (R) (Fr40)  $\cdot$ in] Im R \textbf{27} Gawan] Gawane V (W) Q (Fr40) Gawine R  $\cdot$ geschehen] beschehen W \textbf{28} doch] Do W  $\cdot$ muos] mvͤst V  $\cdot$ tugentlîche] toͮgenliche V (R) (Fr40) \textbf{30} des] Das W Die Q (R) (Fr40)  $\cdot$ in] \textit{om.} W Q R Fr40  $\cdot$ sinne] [*]: sin V sin Q R Fr40 \newline
\end{minipage}
\end{table}
\end{document}
