\documentclass[8pt,a4paper,notitlepage]{article}
\usepackage{fullpage}
\usepackage{ulem}
\usepackage{xltxtra}
\usepackage{datetime}
\renewcommand{\dateseparator}{.}
\dmyyyydate
\usepackage{fancyhdr}
\usepackage{ifthen}
\pagestyle{fancy}
\fancyhf{}
\renewcommand{\headrulewidth}{0pt}
\fancyfoot[L]{\ifthenelse{\value{page}=1}{\today, \currenttime{} Uhr}{}}
\begin{document}
\begin{table}[ht]
\begin{minipage}[t]{0.5\linewidth}
\small
\begin{center}*D
\end{center}
\begin{tabular}{rl}
\textbf{56} & \textbf{er} ist \textbf{erborn} von Anschouwe.\\ 
 & diu minne wirt sîn vrouwe.\\ 
 & sô wirt \textbf{aber} er an strîte ein schûr,\\ 
 & den vîenden herter nâchgebûr.\\ 
5 & Wizzen sol der sun mîn:\\ 
 & sîn an, der hiez Gandin,\\ 
 & \textbf{der} lac an rîterschefte tôt.\\ 
 & des \textit{v}ater leit die selben nôt,\\ 
 & der was geheizen Addanz\\ 
10 & - sîn schilt beleip vil selten ganz -,\\ 
 & \textbf{der} was von arde ein Bertûn.\\ 
 & er unde Utepandragun\\ 
 & w\textit{â}ren zweier \textbf{gebruoder} kint,\\ 
 & die \textbf{bêde} alhie geschriben sint:\\ 
15 & \textbf{daz was} \textbf{einer} Lazaliez,\\ 
 & \textbf{Brickus} der ander hiez.\\ 
 & der \textbf{zweier} vater hiez Mazadan.\\ 
 & den vuort ein \textbf{feie in Morgan},\\ 
 & \textbf{diu hiez} Terredelaschoye.\\ 
20 & er was ir herzen boye.\\ 
 & von \textbf{in zwein} kom geslehte mîn,\\ 
 & daz immer mêr gît liehten schîn.\\ 
 & ieslîcher \textbf{sider} krône truoc\\ 
 & unt \textbf{heten} werdecheit genuoc.\\ 
25 & vrouwe, wil dû toufen dich,\\ 
 & dû maht \textbf{ouch} noch erwerben mich."\\ 
 & \textit{\begin{large}D\end{large}}es \textbf{en}gerte si kein wa\textit{n}del niht.\\ 
 & "ouwê, wie \textbf{balde} daz geschiht,\\ 
 & wil er wider wenden,\\ 
30 & \textbf{schiere} \textbf{sol} ichz enden.\\ 
\end{tabular}
\scriptsize
\line(1,0){75} \newline
D \newline
\line(1,0){75} \newline
\textbf{5} \textit{Majuskel} D  \textbf{27} \textit{Initiale} D  \newline
\line(1,0){75} \newline
\textbf{1} Anschouwe] Anschoͮwe D \textbf{8} vater] watr D \textbf{12} Utepandragun] Vtepandragv̂n D \textbf{13} wâren] wæren D \textbf{15} Lazaliez] Lazalîez D \textbf{19} Terredelaschoye] Terre de lascôye D \textbf{27} Des] ÷es D  $\cdot$ wandel] waldel D \newline
\end{minipage}
\hspace{0.5cm}
\begin{minipage}[t]{0.5\linewidth}
\small
\begin{center}*m
\end{center}
\begin{tabular}{rl}
 & \textbf{er} ist \textbf{geborn} von Anschouwe.\\ 
 & diu minne w\textit{i}rt sîn vrouwe.\\ 
 & sô wirt \textbf{aber} er an strîte ein schûr,\\ 
 & den vîenden herter nâchgebûr.\\ 
5 & wi\textit{zz}en sol der sun mîn:\\ 
 & sîn ane, der hiez Gandin\\ 
 & \textbf{und} lac an ritterschafte tôt.\\ 
 & des vater leit die selben nôt,\\ 
 & der was geheizen Adanz\\ 
10 & - sîn schilt bleip vil selten ganz -,\\ 
 & \textbf{der} was von arde ein Br\textit{i}tûn.\\ 
 & er und Utrapandragun\\ 
 & wâren zweier \textbf{gebruoder} kint,\\ 
 & die alhie geschriben sint:\\ 
15 & \textbf{der hiez} \textbf{einer} Lazaliez,\\ 
 & \textbf{Brickus} der ander hiez.\\ 
 & der \textbf{zweier} vater hiez Ma\textit{z}adan.\\ 
 & den vuorte ein \textbf{feie, hiez Morgan},\\ 
 & \textbf{in} Terredelaschoie.\\ 
20 & er was ir herzen boie.\\ 
 & von \textbf{in zwên} kam geslehte mîn,\\ 
 & daz iemer mêr gît liehten schîn.\\ 
 & iegelîcher \textbf{sît her ein} krône truoc\\ 
 & und \textbf{hette} wirdicheit genuoc.\\ 
25 & vrowe, wiltû toufen dich,\\ 
 & dû maht \textbf{ouch} noch erwerben mich."\\ 
 & des \textbf{en}gerte si deheinen wandel niht.\\ 
 & "owê, wie \textbf{schiere} daz geschiht,\\ 
 & wil er wider wenden,\\ 
30 & \textbf{schiere} \textbf{\textit{so}l} ich \textit{ez} enden.\\ 
\end{tabular}
\scriptsize
\line(1,0){75} \newline
m n o \newline
\line(1,0){75} \newline
\newline
\line(1,0){75} \newline
\textbf{1} Anschouwe] anschoͧwe n anschowe o \textbf{2} wirt] wart m \textbf{5} wizzen] wisen m \textbf{6} hiez] hiesse n  $\cdot$ Gandin] gandein o \textbf{8} selben] selbe n o \textbf{9} Adanz] adantz m addantz n addancz o \textbf{11} der] Er n o  $\cdot$ Britun] brvtun m barun n baruͯn o \textbf{12} er und] Vnd der von n o  $\cdot$ Utrapandragun] vtrapangun n vtrapangon o \textbf{13} gebruoder] bruͯder n (o) \textbf{14} die] Die beide n o \textbf{15} Lazaliez] lazalies m o lazaliesz n \textbf{16} Brickus] Brickuͯs o \textbf{17} Mazadan] maradan m maranden n maradin o \textbf{18} feie] feigo o  $\cdot$ Morgan] morgen n \textbf{19} Jn terre dela scoie m  $\cdot$ jnterdelascoŷe n  $\cdot$ Jn terdolascorie o \textbf{20} ir] irs m (n) o  $\cdot$ boie] boͯle o \textbf{21} in zwên] [seit]: in [wein]: zwein o  $\cdot$ geslehte] geslechte \textit{nachträglich korrigiert zu:} daz geslechte m das geslechte n \textbf{23} sît her] \textit{om.} n sicher o  $\cdot$ krône] kruͯne o \textbf{25} vrowe wiltû] Wilt du frouwe n \textbf{27} des] Das o  $\cdot$ engerte] gert n o  $\cdot$ deheinen] do keine n decken o \textbf{30} sol ich ez] isenlich m \newline
\end{minipage}
\end{table}
\newpage
\begin{table}[ht]
\begin{minipage}[t]{0.5\linewidth}
\small
\begin{center}*G
\end{center}
\begin{tabular}{rl}
 & \textit{\textbf{und}} \textit{ist} \textbf{geboren} von Anschouwe.\\ 
 & diu minne wirt sîn vrouwe.\\ 
 & sô wirt \textbf{aber} er an strîte ein schûr,\\ 
 & den vînden \textbf{ein} herter nâchgebûr.\\ 
5 & wizzen sol der sun mîn:\\ 
 & sîn ene, der hiez Gandin,\\ 
 & \textbf{der} lac an rîterschefte tôt.\\ 
 & des vater leit die selben nôt,\\ 
 & der was geheizen Adanz\\ 
10 & - sîn schilt beleip vil selten ganz -\\ 
 & \textbf{unde} was von arde ein Britun.\\ 
 & er und Utpandragun\\ 
 & wâren zweier \textbf{bruoder} kint,\\ 
 & die \textbf{bêde} al hie geschriben sint:\\ 
15 & \textbf{daz was} \textbf{einer} Lazaliez,\\ 
 & \textbf{Pricurs} der ander hiez.\\ 
 & der \textit{\textbf{zweier}} vater hie\textit{z} \textit{M}azadan.\\ 
 & den vuorte ein \textbf{Phimurgan},\\ 
 & \textbf{diu hiez} Terdilatschoie.\\ 
20 & er was ir herzen boie.\\ 
 & von \textbf{den zwein} kom \textit{\textbf{daz}} geslähte mîn,\\ 
 & daz imer mê gît liehten schîn.\\ 
 & ieslîcher \textbf{sunder} krône truoc\\ 
 & unde \textbf{heten} werdecheit genuoc.\\ 
25 & vrouwe, wil dû toufen dich,\\ 
 & dû maht noch \textbf{wol} erwerben mich."\\ 
 & \begin{large}D\end{large}es gerte si d\textit{eheinen} wandel niht.\\ 
 & "owê, wie \textbf{schiere} daz geschiht,\\ 
 & wil er wider wenden,\\ 
30 & \textbf{vil} \textbf{balde} \textbf{sol} ich ez enden.\\ 
\end{tabular}
\scriptsize
\line(1,0){75} \newline
G I O L M Q R Z Fr21 Fr37 \newline
\line(1,0){75} \newline
\textbf{1} \textit{Initiale} O M  \textbf{15} \textit{Initiale} I  \textbf{27} \textit{Initiale} G L M R Z Fr21 Fr37  \newline
\line(1,0){75} \newline
\textbf{1} und ist] \textit{om.} G ÷r ist O Der ist M  $\cdot$ geboren] gebort M  $\cdot$ von] \textit{om.} R  $\cdot$ Anschouwe] anschoͮwe G antswauͤ I anschawe O Fr37 Anschowe L (M) Fr21 [anschwove]: anschwowe R antschowe Z \textbf{3} sô] Vnt Fr37  $\cdot$ aber] auch I \textit{om.} Fr37  $\cdot$ er] \textit{om.} O Z Fr37  $\cdot$ an strîte ein schûr] [anschir]: anschur M \textbf{4} den] Der M  $\cdot$ ein] \textit{om.} L M Q R Z Fr21 [einer]: ein Fr37 \textbf{6} ene der] ený der L name M  $\cdot$ Gandin] chandin I gandein Q Fr37 [d*]: gaudin R \textbf{8} des] Sein Q  $\cdot$ leit] [reit]: leit Q  $\cdot$ selben] selbe L selbigen M \textbf{9} Adanz] âdanz I adancz M R adansz Q adantz Z adans Fr37 \textbf{10} sîn] Des M Q R  $\cdot$ vil] \textit{om.} R \textbf{11} unde] Der Z Vo Fr21  $\cdot$ Britun] brituͤn I britvͦn O brittvn L Z (Fr21) britton Q briton R \textbf{12} und] \textit{om.} Q  $\cdot$ Utpandragun] vpandragun G vterpandracuͤn I utpandragv̂n O vterpandragrym M vtpandragún Q vrpandragvn Z vtrapandagun Fr37 \textbf{14} bêde] beidú R  $\cdot$ al] \textit{om.} R Fr37 \textbf{15} was einer] eine was Q einer Fr21  $\cdot$ Lazaliez] zazalies L lasaliens Q lazalies R Zazaliez Fr37 \textbf{16} Pricurs] brikus I Bricvrs O Brikuͯrs L Brikuͦrs R Brickurs M Krikurs Q Brikvrs Z Brichurs Fr21 Fr37  $\cdot$ der] den M \textbf{17} zweier] \textit{om.} G  $\cdot$ hiez] hiez oͮch G  $\cdot$ Mazadan] mazadân I mazandan M Maradon R madan Fr37 \textbf{18} den] Der Z  $\cdot$ vuorte] vuͤrt I fvr Z  $\cdot$ Phimurgan] vaimurgan I femvrgan O L (Fr21) frie morgan M feimúrgan Q feimorgon R fein mvrgan Z fermurgan Fr37 \textbf{19} diu] Der M Q  $\cdot$ Terdilatschoie] der da latschoy I div [Dalas]: Dalahsoy O terrderlaschoýe L der lashoie M derdelazhoy͑ Q der dalashoẏ R derdelashoie Z derdalashoye Fr21 derdalashoy Fr37 \textbf{20} \textit{Vers 56.20 fehlt} R   $\cdot$ was] wart I  $\cdot$ ir herzen] bý ir herzen L geheiszin M iresz herszen Q \textbf{21} den] in O L (M) (R) Fr21 Fr37  $\cdot$ daz] \textit{om.} G \textbf{22} mê] \textit{om.} O M R mir L Fr37  $\cdot$ gît] giht Z  $\cdot$ liehten] lychten L (M) (Q) (Fr21) \textbf{23} sunder] sin O (Q) (Fr37) sit L Fr21 siden M sider R Z  $\cdot$ krône] kronen M \textbf{26} dû maht] so maht du I (Fr37)  $\cdot$ noch wol] wol I Fr37 avch noch O (M) (Z) (Fr21) noch L auch wol Q \textbf{27} gerte] engert I O Z Fr21 on gerte M gert R Fr37  $\cdot$ deheinen wandel] do wandel G wandels L \textbf{28} owê] Awi I Awe O Q Owi L Fr21  $\cdot$ schiere] balde O L M Q (R) Z Fr21 Fr37 \textbf{29} er] er mir R \textbf{30} vil balde] Vil schier O (L) (M) (Q) R Fr21 (Fr37) Schier Z  $\cdot$ sol] \textit{om.} M  $\cdot$ enden] wenden Q \newline
\end{minipage}
\hspace{0.5cm}
\begin{minipage}[t]{0.5\linewidth}
\small
\begin{center}*T (U)
\end{center}
\begin{tabular}{rl}
 & \textbf{er} ist \textbf{geborn} von Anschouwe.\\ 
 & diu mi\textit{nn}e wir\textit{t} sî\textit{n} vrouwe.\\ 
 & sô wirt \textbf{iu} er an strîte ein schûr,\\ 
 & den vînden \textbf{ein} herte nâchgebûr.\\ 
5 & wizzen sol der sun mîn:\\ 
 & sîn an, der hiez Gandin\\ 
 & \textbf{und} lac an ritterschefte tôt.\\ 
 & des vater leit die selbe nôt,\\ 
 & der was ge\textit{h}eizen Andanz\\ 
10 & - sîn schilt beleip vil selten ganz -\\ 
 & \textbf{und} was von art ein Britun.\\ 
 & er und Utpandragun\\ 
 & wâren zweier \textbf{bruoder} kint,\\ 
 & die \textbf{beide} alhie geschriben sint:\\ 
15 & \textbf{daz was} \textbf{ein} Lazaliez,\\ 
 & \textbf{Pricus} der ander hiez.\\ 
 & der vater, \textbf{der} hiez Mazadan.\\ 
 & den vuorte ein \textbf{F\textit{e}i\textit{m}organ},\\ 
 & \textbf{diu hiez} Terredelaschoie.\\ 
20 & er was ir herzen bo\textit{i}e.\\ 
 & von \textbf{im} kom \textbf{daz} geslehte mîn,\\ 
 & daz iemer mê gît liehten schîn.\\ 
 & ieglîcher \textbf{sît} krône truoc\\ 
 & und \textbf{hete} wirdecheite genuoc.\\ 
25 & vrouwe, wiltû toufen dich,\\ 
 & dû maht \textbf{ouch} noch erwerben mich."\\ 
 & \begin{large}D\end{large}es \textbf{en}gert si dekeinen wandel niht.\\ 
 & "ouwê, wie \textbf{balde} daz geschiht,\\ 
 & wil er wider wenden,\\ 
30 & \textbf{vil} \textbf{schiere} \textbf{solt} ich ez enden.\\ 
\end{tabular}
\scriptsize
\line(1,0){75} \newline
U V W T \newline
\line(1,0){75} \newline
\textbf{27} \textit{Initiale} U V W T  \newline
\line(1,0){75} \newline
\textbf{1} Anschouwe] Anschowe U V antschowe W Anschoͮwe T \textbf{2} minne wirt sîn] mime wirte sine U \textbf{3} vnd wirt an strîte ein werlich wer T  $\cdot$ iu er] er [*]: aber V er W  $\cdot$ strîte] ritterschefft W \textbf{4} sin vater ist comen von kvnegen her T \textbf{5} wizzen sol] dar zvͦ wizze T \textbf{6} der] \textit{om.} W  $\cdot$ Gandin] Gaudin U (W) \textbf{7} und] der T \textbf{8} selbe] selben W T \textbf{9} geheizen] gebeizen U  $\cdot$ Andanz] Andantz V audantz W \textbf{10} selten] wenic T \textbf{11} Britun] Brituͦn U Brittun V \textbf{12} Utpandragun] vtpandraguͦn U Vterpandragun V (W) \textbf{13} bruoder] gebruͦder V \textbf{14} alhie] hie W T \textbf{15} daz] da T  $\cdot$ ein] einer V W T  $\cdot$ Lazaliez] lioließ W \textbf{16} Pricus] Prikuͦs U Prikus V Prichur W \textbf{17} der vater] Der [*]: zweier vatter V Der zweyer vatter W (T)  $\cdot$ der] \textit{om.} W T \textbf{18} den] der T  $\cdot$ ein] frauwe W  $\cdot$ Feimorgan] vin norgan U feimurgan V W \textbf{19} Terredelaschoie] Terdel Adschoie U [Terdel*]: Terdelaschoie V terdel adschonye W terredelascôie T \textbf{20} Die minne was sein boye W  $\cdot$ ir] irs V  $\cdot$ boie] bole U \textbf{21} im] in zwein V T \textbf{22} liehten] lichten U \textbf{23} sît] [*si*]: sider V \textbf{24} hete] hetten V (T) \textbf{25} vrouwe] Erauw W \textbf{26} ouch noch] vil wol W T \textbf{27} engert] gerte W  $\cdot$ dekeinen] do W  $\cdot$ wandel] wandels W \textbf{30} solt ich ez] sol ich es W solichs T \newline
\end{minipage}
\end{table}
\end{document}
