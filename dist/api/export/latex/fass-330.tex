\documentclass[8pt,a4paper,notitlepage]{article}
\usepackage{fullpage}
\usepackage{ulem}
\usepackage{xltxtra}
\usepackage{datetime}
\renewcommand{\dateseparator}{.}
\dmyyyydate
\usepackage{fancyhdr}
\usepackage{ifthen}
\pagestyle{fancy}
\fancyhf{}
\renewcommand{\headrulewidth}{0pt}
\fancyfoot[L]{\ifthenelse{\value{page}=1}{\today, \currenttime{} Uhr}{}}
\begin{document}
\begin{table}[ht]
\begin{minipage}[t]{0.5\linewidth}
\small
\begin{center}*D
\end{center}
\begin{tabular}{rl}
\textbf{330} & \begin{large}S\end{large}ol ich durch mîner zuht gebot\\ 
 & hœren nû der werlde spot,\\ 
 & sô mac sîn râten niht sîn ganz:\\ 
 & mir riet der werde Gurnamanz,\\ 
5 & daz ich vrevellîche vrâge mite\\ 
 & unt \textbf{immer} gein unvuogen strite.\\ 
 & Vil werder rîter sihe ich hie.\\ 
 & durch iwer zuht nû râtet \textbf{mir}, wie\\ 
 & \textbf{daz} ich \textbf{iwern} hulden \textbf{næhe} mich.\\ 
10 & ez ist ein strenge, scherpf gerich\\ 
 & \textbf{gein mir mit worten hie} getân.\\ 
 & swes hulde ich drumbe verlorn hân,\\ 
 & daz wil ich wênec \textbf{wîzen} im.\\ 
 & swenne ich hernâch prîs genim,\\ 
15 & sô habt mich aber \textbf{denne} dar nâch.\\ 
 & mir ist ze scheiden von iu gâch.\\ 
 & Ir \textbf{gâbt} mir alle geselleschaft,\\ 
 & \textbf{die wîle} ich \textbf{stuont} \textbf{in} prîses kraft.\\ 
 & \textbf{der} sît \textbf{nû} ledec, unz ich bezal,\\ 
20 & dâ von mîn grüeniu vreude ist val.\\ 
 & mîn sol grôz jâmer alsô pflegen,\\ 
 & daz herze geb den ougen regen,\\ 
 & sît ich ûf Munsalvæsche liez,\\ 
 & daz mich von \textbf{wâren} vreuden stiez.\\ 
25 & ohteiz, wie \textbf{manege clâre} magt!\\ 
 & swaz iemen wunders hât gesagt,\\ 
 & dennoch pflît\textbf{s} mêr der Grâl.\\ 
 & der wirt hât siufzebæren twâl.\\ 
 & \textbf{ay}, helfelôser Anfortas,\\ 
30 & waz half dich, daz ich bî dir was?"\\ 
\end{tabular}
\scriptsize
\line(1,0){75} \newline
D \newline
\line(1,0){75} \newline
\textbf{1} \textit{Initiale} D  \textbf{7} \textit{Majuskel} D  \textbf{17} \textit{Majuskel} D  \newline
\line(1,0){75} \newline
\textbf{23} Munsalvæsche] Mvnsalvæsce D \newline
\end{minipage}
\hspace{0.5cm}
\begin{minipage}[t]{0.5\linewidth}
\small
\begin{center}*m
\end{center}
\begin{tabular}{rl}
 & sol ich durch mîner zuht gebot\\ 
 & hœre\textit{n} nû der werlte spot,\\ 
 & sô \textbf{en}mac sîn râten niht sî\textit{n} ganz:\\ 
 & mir r\textit{ie}t der werde Gurnemanz,\\ 
5 & daz ic\textit{h v}revellîch vrâg\textit{e} mite\\ 
 & und \textbf{iemer} gegen unvuoge strite.\\ 
 & vil werder ritter sich ich hie.\\ 
 & durch iuwere zuht nû râtet, wie\\ 
 & ich \textbf{ir} h\textit{u}lden \textbf{genæhe} mich.\\ 
10 & ez ist ein strenge, scharf gerich\\ 
 & \textbf{gegen mir mit worten hie} getân.\\ 
 & wes hulde ich d\textit{r}umbe verlorn hân,\\ 
 & daz wil ich wênic \textbf{wizzen} ime.\\ 
 & wenne ich hernâch prîs genime,\\ 
15 & sô habt mich aber \textbf{danne} dar nâch.\\ 
 & mir ist ze scheiden von iu gâch.\\ 
 & ir \textbf{gâbet} mir alle geselleschaft,\\ 
 & \textbf{die wîle} ich \textbf{stuont} \textbf{in} prîses kraft.\\ 
 & \textbf{der} sît \textbf{i\textit{r}} ledic, unz \textbf{daz} ich bezal,\\ 
20 & dâ von mîniu grüene vröude ist val.\\ 
 & mîn sol grôz jâmer alsô pflegen,\\ 
 & daz herze gebe den ougen regen,\\ 
 & sît ich ûf Mun\textit{t}salvasche liez,\\ 
 & daz mich von \textbf{wâren} vröuden stiez.\\ 
25 & ohtei\textit{z}, wie \textbf{manige clâre} maget!\\ 
 & waz iemen wunders hât gesaget,\\ 
 & dennoch pfligt \textbf{es} mêr der Grâl.\\ 
 & der wirt hât siuftebæren twâl.\\ 
 & \textbf{dû} helfelôser Anfortas,\\ 
30 & waz half dich, daz ich bî dir was?"\\ 
\end{tabular}
\scriptsize
\line(1,0){75} \newline
m n o \newline
\line(1,0){75} \newline
\newline
\line(1,0){75} \newline
\textbf{2} hœren nû] hoͯrent nuͯ m Nuͯ hoͯren n \textbf{3} enmac] mag n o  $\cdot$ sîn] sẏ m \textbf{4} riet] reit m  $\cdot$ Gurnemanz] gurnemancz m o gurnemantz n \textbf{5} Das ich fragette vnd frefenlich fragette mite m  $\cdot$ vrevellîch vrâge] froge freuelich o \textbf{7} sich ich hie] sich alhie n o \textbf{8} iuwere] ire m \textbf{9} ir] iren m uweren n (o)  $\cdot$ hulden] helden m hulde o \textbf{10} strenge] stenge o  $\cdot$ scharf] scharppfe n \textbf{12} drumbe] dumbe m \textbf{14} genime] gewin n \textbf{15} habt] halp o \textbf{19} ir] ym m in mir o  $\cdot$ unz daz ich] vntze ich n [noch]: vncz ich o \textbf{20} grüene] gnuͯne o \textbf{21} grôz] grosses n grosse o \textbf{23} Muntsalvasche] mvnsaluasce m montsaluasce n munt saluasce o \textbf{25} ohteiz] Ohteir m Ochten n (o) \textbf{27} es] sin n o \newline
\end{minipage}
\end{table}
\newpage
\begin{table}[ht]
\begin{minipage}[t]{0.5\linewidth}
\small
\begin{center}*G
\end{center}
\begin{tabular}{rl}
 & sol ich durch mîner zuht gebot\\ 
 & \textit{hœr}en nû der werlte spot,\\ 
 & sô\textbf{ne} mac sîn râten niht sîn ganz:\\ 
 & mir riet der werde Gurnomanz,\\ 
5 & daz ich vrevellîche vrâge mite\\ 
 & unde \textbf{immer} gein unvuoge strite.\\ 
 & vil werder rîter sihe ich hie.\\ 
 & durch iuwer zuht nû râtet, wie\\ 
 & \textbf{daz} ich \textbf{iuweren} hulden \textbf{næhe} mich.\\ 
10 & ez ist ein strenge, scharf gerich\\ 
 & \textbf{gein mir mit worten hie} getân.\\ 
 & swes hulde ich drumbe verloren hân,\\ 
 & daz wil ich wênic \textbf{wîzen} im.\\ 
 & swenne ich hernâch prîs genim,\\ 
15 & sô habet mich aber \textbf{danne} dar nâch.\\ 
 & mir ist ze scheidene von iu gâch.\\ 
 & ir \textbf{gâbet} mir alle geselleschaft,\\ 
 & \textbf{die wîle} ich \textbf{stuont} \textbf{an} brîses kraft.\\ 
 & \textbf{des} sît \textbf{nû} ledic, unze ich bezal,\\ 
20 & dâ von mîn grüeniu vröude ist val.\\ 
 & mîn sol grôz jâmer alsô pflegen,\\ 
 & daz herze gebe den ougen regen,\\ 
 & sît ich ûf Muntsalvatsche liez,\\ 
 & daz mich von \textbf{wâren} vröuden stiez.\\ 
25 & \textit{ohteiz, wie \textbf{manic clâriu} maget}!\\ 
 & \textit{swaz ieman wunders hât gesaget},\\ 
 & \textit{dannoch pfliget \textbf{sîn} mêre der Grâl}.\\ 
 & \textit{der wirt hât siufzebær twâl}.\\ 
 & \textit{\textbf{hei}, helfelôser Anfortas},\\ 
30 & \textit{waz half dich, daz ich bî dir was}?"\\ 
\end{tabular}
\scriptsize
\line(1,0){75} \newline
G I O L M Q R Z Fr21 Fr27 Fr39 \newline
\line(1,0){75} \newline
\textbf{1} \textit{Initiale} Q  \textbf{3} \textit{Capitulumzeichen} R  \textbf{5} \textit{Initiale} I  \textbf{13} \textit{Initiale} O M  \textbf{21} \textit{Initiale} I  \textbf{25} \textit{Initiale} Z  \newline
\line(1,0){75} \newline
\textbf{2} hœren] dvlten G  $\cdot$ werlte] lute I \textbf{3} sône] So O R An deme M  $\cdot$ râten] rat L M Q (R) Fr21  $\cdot$ sîn] \textit{om.} Fr21 \textbf{4} mir] Hie M  $\cdot$ Gurnomanz] kurnomanz G Gurnemanz I kvrnemanz O Gvrnemantz L gurnemancz M gurnomantz Q Z Guͯrnomancz R G:::manz Fr21 ::ernemanz Fr27 \textbf{5} daz] Dach Q  $\cdot$ vrâge] fragen R \textbf{6} immer] iommer M  $\cdot$ unvuoge] vnfugen Q \textbf{7} sihe] sy M \textbf{8} iuwer] ewer zwer Z  $\cdot$ nû] so L \textit{om.} Q  $\cdot$ râtet] ratet mir I O (Q) (R) (Z) Fr21 \textbf{9} ich] \textit{om.} O Z  $\cdot$ iuweren hulden] ewrer hulde I uwern hulde R  $\cdot$ næhe] genahin M \textbf{10} strenge scharf] so sherpher I strenger scharfh L scharff strenge M (R) starcke strenge Q \textbf{12} swes] Wez L (M) (Q) (R)  $\cdot$ ich drumbe] darvmbe ich Z \textbf{13} daz] ÷az O (M)  $\cdot$ wîzen] wizzen O (M) (Q) (R) \textbf{14} swenne] Wenne L (M) (Q) R Swenne der Fr27  $\cdot$ hernâch] er nach I \textbf{15} mich] \textit{om.} Fr27  $\cdot$ aber] \textit{om.} L Fr39  $\cdot$ danne] \textit{om.} I \textbf{16} von] \textit{om.} R \textbf{17} alle] gar I \textbf{18} stuont] stunde Q  $\cdot$ an] in I \textbf{19} des] Der L R Fr39  $\cdot$ unze] biz Fr27  $\cdot$ ich] \textit{om.} Q  $\cdot$ bezal] bezalt I \textbf{20} dâ von] Das von Q \textbf{21} sol] soͤlt Z \textbf{22} gebe] geben Q  $\cdot$ den] di O den ob Fr27 \textbf{23} ich] \textit{om.} Fr21  $\cdot$ ûf Muntsalvatsche] vf muntschaluasche I uff Munsalvatsche M uffmuntsalvasche Q vff Munsaluashe R vf montsalvatsche Z uf monsauasch::: Fr27 vf Muntsaluasche Fr39 \textbf{25} \textit{Die Verse 330.25-30 fehlen} G   $\cdot$ ohteiz] Auch des Q Ohteriz R  $\cdot$ manic clâriu] menge klare R \textbf{26} Was wunders yemet hat gesaget R  $\cdot$ swaz] Waz L (M) (Q) \textbf{28} siufzebær] sevftweren O (L) (M) (Q) (Z) (Fr27) (Fr39) \textbf{29} hei] Eý L (M) (Q) (R) (Z) (Fr21) (Fr27) (Fr39)  $\cdot$ helfelôser] helfelose O  $\cdot$ Anfortas] anfrotas I Ampfortas O Amfortas L \newline
\end{minipage}
\hspace{0.5cm}
\begin{minipage}[t]{0.5\linewidth}
\small
\begin{center}*T
\end{center}
\begin{tabular}{rl}
 & sol ich durch mîner zuht gebot\\ 
 & hœren nû der werlte spot,\\ 
 & sô\textbf{ne} mac sîn râten niht sîn ganz:\\ 
 & mir riet der werde Gurnemanz,\\ 
5 & daz ich vrevellîche vrâge mite\\ 
 & unde \textbf{niemer} gegen unvuoge strite.\\ 
 & vil werder rîter sihe ich hie.\\ 
 & durch iuwer zuht nû râtet \textbf{mir}, wie\\ 
 & \textbf{daz} ich \textbf{ir} hulden \textbf{næhe} mich.\\ 
10 & ez ist ein strenge, scharpf gerich\\ 
 & \textbf{mit worten hie gegen mir} getân.\\ 
 & swes hulde ich drumbe verlorn hân,\\ 
 & daz wil ich wênic \textbf{wîzen} ime.\\ 
 & swennich hernâch prîs genime,\\ 
15 & sô habe\textit{t} mich aber dar nâch.\\ 
 & mirst ze scheiden von iu gâch.\\ 
 & ir \textbf{habt} mir alle geselleschaft\\ 
 & \textbf{getân}, \textbf{dô} ich \textbf{in} prîses kraft\\ 
 & \textbf{stuont}. \textbf{des} sît \textbf{nû} ledic, unz ich bezal,\\ 
20 & dâ von mîn grüene vröude ist \textit{val}.\\ 
 & mîn sol grôz jâmer alsô pflegen,\\ 
 & \textbf{daz} daz herze gebe den ougen regen,\\ 
 & sît ich ûf Munsalvasche liez,\\ 
 & daz mich von \textbf{werden} vröuden stiez.\\ 
25 & o\textit{ht}eiz, wie \textbf{manec clâre} maget!\\ 
 & swaz ieman wunders hât gesaget,\\ 
 & dannoch pfliget \textbf{sîn} mêr der Grâl.\\ 
 & der wirt hât siufzebær\textit{e} twâl.\\ 
 & \textbf{ei}, helfelôser Anfortas,\\ 
30 & waz half dich, daz ich bî dir was?"\\ 
\end{tabular}
\scriptsize
\line(1,0){75} \newline
T U V W \newline
\line(1,0){75} \newline
\newline
\line(1,0){75} \newline
\textbf{3} sône] So W  $\cdot$ râten] rat V reiten W \textbf{4} Gurnemanz] gurnemantz W \textbf{5} ich] \textit{om.} W  $\cdot$ vrâge] [vragen]: vrage V \textbf{6} niemer gegen unvuoge] nuͦmer Gein vngevuͦge U [niemer]: iemer gegen vnfvͦge V gen vnfuͦge nymmer W \textbf{7} sihe] sehe U [*]: sihe V \textbf{8} mir] \textit{om.} W \textbf{9} ich ir hulden] [*]: ich uwern hvlden V eúwer hulde W \textbf{10} strenge] strenges W  $\cdot$ scharpf] scharpez U (W) \textbf{12} swes] Wes U W \textbf{14} swennich] Wan ich U Wem ich W  $\cdot$ genime] [*]: geneme U \textbf{15} habet] haben T  $\cdot$ mich aber] aber mich U \textbf{17} habt] gabent W \textbf{18} getân dô ich] Die weil ich stuͦnd W \textbf{19} stuont des sît] Die seint W  $\cdot$ unz ich] mit U \textbf{20} mîn] mir W  $\cdot$ val] \textit{om.} T \textbf{22} daz daz] Daz U (W) \textbf{23} Munsalvasche] mvnsalvatsce T muͦntsalvatsche U mvntsaluasche V montsaluatz W \textbf{25} ohteiz] otheiz T (U) Gotteweis V Achteis W \textbf{26} swaz] Waz U (W) \textbf{27} pfliget] plege U \textbf{28} siufzebære] svfceberiv T sufteberes U sv́fzeberendes V \textbf{29} ei] Ahy W \newline
\end{minipage}
\end{table}
\end{document}
