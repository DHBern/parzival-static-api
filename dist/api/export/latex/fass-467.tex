\documentclass[8pt,a4paper,notitlepage]{article}
\usepackage{fullpage}
\usepackage{ulem}
\usepackage{xltxtra}
\usepackage{datetime}
\renewcommand{\dateseparator}{.}
\dmyyyydate
\usepackage{fancyhdr}
\usepackage{ifthen}
\pagestyle{fancy}
\fancyhf{}
\renewcommand{\headrulewidth}{0pt}
\fancyfoot[L]{\ifthenelse{\value{page}=1}{\today, \currenttime{} Uhr}{}}
\begin{document}
\begin{table}[ht]
\begin{minipage}[t]{0.5\linewidth}
\small
\begin{center}*D
\end{center}
\begin{tabular}{rl}
\textbf{467} & \textbf{\begin{large}S\end{large}wâ} werc \textbf{verwürkent} sînen gruoz,\\ 
 & daz gotheit sich schamen muoz,\\ 
 & \textbf{wem} lât \textbf{den} menschlîchiu zuht?\\ 
 & wa\textit{r} hât diu arme sêle vluht?\\ 
5 & welt ir \textbf{nû} gote vüegen leit,\\ 
 & der ze bêden sîten ist bereit,\\ 
 & zer minne unt \textbf{gein} dem zorne,\\ 
 & sô sît ir der verlorne.\\ 
 & nû kêrt iwer gemüete,\\ 
10 & daz er iu \textbf{danke} güete."\\ 
 & Parzival sprach zim dô:\\ 
 & "hêrre, ich bin des immer vrô,\\ 
 & daz ir mich von dem bescheiden hât,\\ 
 & d\textit{er} nihtes ungelônt lât,\\ 
15 & der missewende noch der tugent.\\ 
 & ich hân mit sorgen mîne jugent\\ 
 & alsus brâht an disen tac,\\ 
 & daz ich durch triwe \textbf{kumbers} pflac."\\ 
 & der wirt sprach \textbf{aber wider} zim:\\ 
20 & "\textbf{nimt}s iuch niht hæle, gern ich vernim,\\ 
 & waz ir kumbers und \textbf{sünden} hât.\\ 
 & ob ir mich \textbf{diu} \textbf{prüeven} lât,\\ 
 & dar zuo gib ich iu \textbf{lîhte} rât,\\ 
 & des ir selbe niht enhât."\\ 
25 & Dô sprach aber Parzival:\\ 
 & "mîn hœhstiu nôt ist umben Grâl,\\ 
 & dâ nâch umb mîn selbes wîp.\\ 
 & ûf erde nie schœner lîp\\ 
 & \textbf{gesouc} \textbf{an deheiner} muoter brust.\\ 
30 & nâch den beiden sent sich mîn gelust."\\ 
\end{tabular}
\scriptsize
\line(1,0){75} \newline
D \newline
\line(1,0){75} \newline
\textbf{1} \textit{Initiale} D  \textbf{25} \textit{Majuskel} D  \newline
\line(1,0){75} \newline
\textbf{4} war] warte D \textbf{11} Parzival] Parcifal D \textbf{14} der] daz D \textbf{25} Parzival] Parcifal D \newline
\end{minipage}
\hspace{0.5cm}
\begin{minipage}[t]{0.5\linewidth}
\small
\begin{center}*m
\end{center}
\begin{tabular}{rl}
 & \textbf{sîn} werc \textbf{verwürket} sînen gruoz,\\ 
 & daz gotheit sich schamen muoz.\\ 
 & \textbf{wem} l\textit{â}t \textbf{den} menschlîch zuht?\\ 
 & war het diu arme sêle vluht?\\ 
5 & wolt ir \textbf{nû} got vüegen leit,\\ 
 & der zuo beiden sîten ist bereit,\\ 
 & zuor minne und \textbf{gegen} dem zorne,\\ 
 & sô sît ir der verlorne.\\ 
 & nû kêret iuwer gemüete,\\ 
10 & daz er iu \textbf{danke} güete."\\ 
 & Parcifal sprach zuo im d\textit{ô}:\\ 
 & "hêrre, ich bin des iemer vrô,\\ 
 & daz ir mich von  bescheiden hât,\\ 
 & der nihtes ungelônet lât,\\ 
15 & der missewende noch der tugent.\\ 
 & ich hân mit sorgen mîne jugent\\ 
 & alsus brâht an disen tac,\\ 
 & daz ich durch triuwe \textbf{kumbers} pflac."\\ 
 & der wirt sprach \textbf{aber} zuo im:\\ 
20 & "\textbf{nemt} es iuch niht hæle! gerne ich vernim,\\ 
 & waz \textit{ir} kumbers und \textbf{\textit{sü}nde} hât.\\ 
 & ob ir mich \textbf{diu} \textbf{wi\textit{zz}en} lât,\\ 
 & dar zuo gip ich iu \textbf{vil liehten} rât,\\ 
 & des ir selbe niht enhât."\\ 
25 & dô sprach aber Parcifal:\\ 
 & "mîn hœhstiu nôt ist umb den Grâl,\\ 
 & dar nâch umb mîn selbes wîp.\\ 
 & ûf erden nie \textbf{sô} schœner lîp\\ 
 & \textbf{gelac} \textbf{an enkeiner} muoter brust.\\ 
30 & nâch den beiden sent sich mîn gelust."\\ 
\end{tabular}
\scriptsize
\line(1,0){75} \newline
m n o \newline
\line(1,0){75} \newline
\newline
\line(1,0){75} \newline
\textbf{3} lât] lont m \textit{om.} n  $\cdot$ den] denne n \textbf{7} dem] \textit{om.} n \textbf{10} daz] Dar o \textbf{11} dô] dar m \textbf{13} bescheiden] bescheidenheit n \textbf{15} der tugent] den tugent o \textbf{16} sorgen] sorge n \textbf{19} aber] aber wider n o \textbf{20} vernim] vernem o \textbf{21} ir] \textit{om.} m  $\cdot$ sünde] gesinde m \textbf{22} \textit{Versfolge 467.23-22} n   $\cdot$ wizzen] wisen m \textbf{23} vil liehten] vil lichtten m lichte n lihten o \textbf{24} des] Das o \textbf{27} dar nâch] Dannach o  $\cdot$ umb] \textit{om.} n \newline
\end{minipage}
\end{table}
\newpage
\begin{table}[ht]
\begin{minipage}[t]{0.5\linewidth}
\small
\begin{center}*G
\end{center}
\begin{tabular}{rl}
 & \textbf{\begin{large}S\end{large}wâ} werke \textbf{verwürkent} sînen gruoz,\\ 
 & daz \textbf{diu} gotheit sich schamen muoz,\\ 
 & \textbf{wem} lât \textbf{den} menschlîchiu zuht?\\ 
 & war hât diu arme sêle vluht?\\ 
5 & welt ir \textbf{nû} gote vüegen leit,\\ 
 & der ze bêden sîten ist bereit,\\ 
 & ze der minne unde \textbf{gein} dem zorne,\\ 
 & sô sît ir der verlorne.\\ 
 & nû kêrt iuwer gemüete,\\ 
10 & daz er iu \textbf{danke} güete."\\ 
 & Parzival sprach zim dô:\\ 
 & "hêrre, ich bin des immer vrô,\\ 
 & daz ir mich von dem bescheiden hât,\\ 
 & de\textit{r} nihtes ungelônet lât,\\ 
15 & der missewende noch der tugent.\\ 
 & ich hân mit sorgen mîne jugent\\ 
 & alsus brâht an disen tac,\\ 
 & daz ich durch triuwe \textbf{jâmers} pflac."\\ 
 & der wirt sprach \textbf{aber} zim:\\ 
20 & "\textbf{nemet}s iuch niht hæle! gerne ich vernim,\\ 
 & waz i\textit{r} kumbers unde \textbf{sünden} hât.\\ 
 & ob ir mich \textbf{di\textit{e}} \textbf{prüeven} lât,\\ 
 & dar zuo gib ich iu \textbf{lîhte} rât,\\ 
 & des ir selbe niht enhât."\\ 
25 & dô sprach aber Parzival:\\ 
 & "mîn hœhestiu nôt ist umb den Grâl,\\ 
 & dar nâch umbe mîn selbes wîp.\\ 
 & ûf erde nie schœner lîp\\ 
 & \textbf{gesouc} \textbf{an deheiner} muoter brust.\\ 
30 & nâch den beiden senet sich mîn gelust."\\ 
\end{tabular}
\scriptsize
\line(1,0){75} \newline
G I O L M Z Fr18 Fr22 \newline
\line(1,0){75} \newline
\textbf{1} \textit{Initiale} G I O L M Z  \textbf{11} \textit{Initiale} I  \newline
\line(1,0){75} \newline
\textbf{1} Swâ] ÷wa O Wa L  $\cdot$ werke] werkin M \textbf{2} diu] \textit{om.} L M \textbf{3} wem] Wen Z  $\cdot$ den] div O (Z) \textbf{4} war] wa I \textbf{5} welt] Wan Z  $\cdot$ nû gote] got nv O L  $\cdot$ vüegen] [f:gen]: fugen L \textbf{6} der ze] Zcu den M \textbf{7} minne] libe M  $\cdot$ unde] \textit{om.} I \textbf{11} Parzival] Parziual G Parzifal I L M Barcifal O Parcifal Z  $\cdot$ dô] da M \textbf{14} der] Des G Daz Z  $\cdot$ nihtes] nih I \textbf{15} tugent] [*vgent]: tvgent G \textbf{18} jâmers] chvmbers O (L) (M) (Z) \textbf{19} zim] wider zim O (L) (M) (Z) \textbf{20} nemets iuch] [m:]: nems evch I Nimt es ivch O (L) (Z) Nimt ivchs Fr18  $\cdot$ vernim] verneme I \textbf{21} ir] ich G  $\cdot$ sünden] sunde M \textbf{22} mich] iuch Fr18  $\cdot$ die] diu G \textbf{23} lîhte] lihten O (M) \textbf{24} selbe] selben M  $\cdot$ niht enhât] niene hat Fr18 \textbf{25} dô] Da O M  $\cdot$ Parzival] parziual G parzifal I L M Barcifal O parcifal Z (Fr18) \textbf{28} erde] erdin M  $\cdot$ schœner] so schoner L \textbf{29} gesouc] Gesuͤf I  $\cdot$ an] nie I \textit{om.} O Fr18 \textbf{30} beiden] henden I  $\cdot$ senet] schiet M  $\cdot$ gelust] glut M \newline
\end{minipage}
\hspace{0.5cm}
\begin{minipage}[t]{0.5\linewidth}
\small
\begin{center}*T
\end{center}
\begin{tabular}{rl}
 & \textbf{swâ} werc \textbf{verwirkent} sînen gruoz,\\ 
 & daz \textbf{diu} goteheit sich schamen muoz,\\ 
 & \textbf{wan} lât \textbf{diu} menschlîchiu zuht?\\ 
 & war hât diu arme sêle vluht?\\ 
5 & welt ir gote vüegen leit,\\ 
 & der ze beiden sîten ist bereit,\\ 
 & zer minne unde \textbf{z}em zorne,\\ 
 & sô sît ir der verlorne.\\ 
 & nû kêret iuwer gemüete,\\ 
10 & daz er iu \textbf{danne} güete."\\ 
 & \begin{large}P\end{large}arcifal sprach zim dô:\\ 
 & "hêrre, ich bin des iemer vrô,\\ 
 & daz ir mich von dem bescheiden hât,\\ 
 & \textit{der nihtes ungelônet} lât,\\ 
15 & der missewende noch der tugent.\\ 
 & Ich hân mit sorgen \textit{m}îne jugent\\ 
 & alsus brâht an disen tac,\\ 
 & daz ich durch triuwe \textbf{kumbers} pflac."\\ 
 & Der wirt sprach \textbf{wider} zim:\\ 
20 & "\textbf{nimt}s iuch niht hæle, gerne ich vernim,\\ 
 & waz ir kumbers unde \textbf{sünden} hât.\\ 
 & ob ir mich \textbf{die} \textbf{prüeven} lât,\\ 
 & dar zuo gibich iu \textbf{lîhte} rât,\\ 
 & des ir selbe niht enhât."\\ 
25 & Dô sprach aber Parcifal:\\ 
 & "mîn hœhest\textit{iu} nôt ist umben Grâl,\\ 
 & dar nâch umbe mîn selbes wîp.\\ 
 & ûf erde nie schœner lîp\\ 
 & \textbf{gesouc} \textbf{nie} muoter brust.\\ 
30 & nâch den beiden sent sich mîn gelust."\\ 
\end{tabular}
\scriptsize
\line(1,0){75} \newline
T U V W Q R Fr42 \newline
\line(1,0){75} \newline
\textbf{1} \textit{Initiale} R  \textbf{11} \textit{Initiale} T V W R Fr42  \textbf{16} \textit{Majuskel} T  \textbf{19} \textit{Capitulumzeichen} R   $\cdot$ \textit{Majuskel} T  \textbf{25} \textit{Capitulumzeichen} R   $\cdot$ \textit{Majuskel} T  \newline
\line(1,0){75} \newline
\textbf{1} \textit{Die Verse 453.1-502.30 fehlen} U   $\cdot$ swâ] Wo W Q Wan R  $\cdot$ werc] wer R  $\cdot$ verwirkent] [*]: verwirkent V ver wurcket Q Ruͯchett R \textbf{2} diu] \textit{om.} W Q R \textbf{3} wan] Wem V W R  $\cdot$ menschlîchiu] menscliche R \textbf{4} war] Wo Q \textbf{5} vüegen] nun fuͤgen W (Q) (R) \textbf{7} zer] Zer der R  $\cdot$ zem] zvͦ V gen dem W gen Q \textbf{10} danne] danke V (W) (Q) R \textbf{11} Parcifal] PArzifal V PArtzifal W (Q) Parczifal R P::: Fr42 \textbf{13} bescheiden] bescheidet Q \textbf{14} ob ir mich die prveven lat \textit{(Versdoppelung 467.22)} T  $\cdot$ ungelônet] vngelon Q \textbf{16} \textit{Vers 467.16 fehlt} Q   $\cdot$ mîne] sine T \textbf{19} wider] aber wider W aber Q \textbf{20} Núttes nicht ich helle ger vernim R  $\cdot$ iuch] iv T \textbf{21} kumbers unde sünden] sv́nde vnde kvmbers V \textbf{22} mich die] die mich Q  $\cdot$ prüeven] wissen R \textbf{23} dar] [Daz]: Da V \textbf{24} selbe] selber W selbs R  $\cdot$ niht enhât] inne hat W \textbf{25} Parcifal] parzifal V partzifal W Q parczifal R \textbf{26} hœhestiu] hoheste T hoͤchte W \textbf{27} wîp] wil R \textbf{28} ûf erde] Es ward R \textbf{29} gesouc] Gesoͯge R  $\cdot$ nie] deheiner V (W) (Q) \textbf{30} sich] ich Q  $\cdot$ gelust] lust W \newline
\end{minipage}
\end{table}
\end{document}
