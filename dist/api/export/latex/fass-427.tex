\documentclass[8pt,a4paper,notitlepage]{article}
\usepackage{fullpage}
\usepackage{ulem}
\usepackage{xltxtra}
\usepackage{datetime}
\renewcommand{\dateseparator}{.}
\dmyyyydate
\usepackage{fancyhdr}
\usepackage{ifthen}
\pagestyle{fancy}
\fancyhf{}
\renewcommand{\headrulewidth}{0pt}
\fancyfoot[L]{\ifthenelse{\value{page}=1}{\today, \currenttime{} Uhr}{}}
\begin{document}
\begin{table}[ht]
\begin{minipage}[t]{0.5\linewidth}
\small
\begin{center}*D
\end{center}
\begin{tabular}{rl}
\textbf{427} & stuont ninder \textbf{decheiniu} alsô rôt.\\ 
 & swem si güetlîche ir küssen bôt,\\ 
 & des muose swenden sich der walt\\ 
 & mit maneger tjost ungezalt.\\ 
5 & Mit lobe wir \textbf{solden} grüezen\\ 
 & die kiuschen unt die süezen\\ 
 & \textbf{vroun} Antikonien,\\ 
 & vor \textbf{valscheit} die vrîen,\\ 
 & wan si lebte in \textbf{solhen} siten,\\ 
10 & daz ninder was underriten\\ 
 & ir prîs mit valschen worten.\\ 
 & alle, die ir prîs gehôrten,\\ 
 & ieslîch munt ir wunschte dô,\\ 
 & daz ir prîs bestüende alsô\\ 
15 & bewart vor valscher trüeben jehe.\\ 
 & lûter, \textbf{virrec} als \textbf{ein valkensehe}\\ 
 & was balsemmæzec \textbf{stæte} an ir.\\ 
 & daz riet ir werdeclîchiu gir.\\ 
 & Diu süeze, sælden rîche\\ 
20 & sprach gezogenlîche:\\ 
 & "bruoder, hie bringe ich den degen,\\ 
 & des dû mich selbe hieze pflegen.\\ 
 & nû lâz in mîn geniezen.\\ 
 & \textbf{des} \textbf{en}sol dich niht verdriezen.\\ 
25 & denke an brüederlîche triwe\\ 
 & unt tuo daz âne riwe.\\ 
 & dir stêt manlîchiu triwe baz,\\ 
 & \textbf{denne} daz dû \textbf{dolst} der \textbf{werlde} haz\\ 
 & und mînen, künde ich hazzen.\\ 
30 & den lêre mich gein dir mâzen."\\ 
\end{tabular}
\scriptsize
\line(1,0){75} \newline
D Fr1 Fr5 Fr68 \newline
\line(1,0){75} \newline
\textbf{1} \textit{Initiale} Fr5  \textbf{5} \textit{Majuskel} D  \textbf{19} \textit{Majuskel} D  \newline
\line(1,0){75} \newline
\textbf{1} stuont ninder] DA stuͦnt an niendir Fr5 stunt niergen Fr68 \textbf{2} ir] ein Fr1 \textbf{3} des] da Fr68 \textbf{4} mit] von Fr1  $\cdot$ ungezalt] [manech valt]: vngezalt Fr1 gizalt Fr5 \textbf{7} vroun] di kvnegin Fr1 \textit{om.} Fr5 Fr68  $\cdot$ Antikonien] Antikonîen D Antẏconîen Fr1 Anthychonyen Fr5 anthýconien Fr68 \textbf{8} valscheit die] valsche dri Fr68 \textbf{9} solhen] so reinen Fr1 \textbf{10} ninder] niergen Fr68 \textbf{11} ir pris mit [valscer]: valscen \sout{:::ben iehe} Fr1 \textbf{12} \textit{Die Verse 427.12-15 fehlen, wurden offensichtlich über der Spalte nachgetragen, sind jedoch bis auf den Schluss dem Schnitt zum Opfer gefallen:} also bewart vor valscer trvͤbe Fr1   $\cdot$ prîs] lob Fr68 \textbf{13} ieslîch] der iegelih Fr68 \textbf{14} bestüende] gistuͦende Fr5 \textbf{15} vor] von Fr5  $\cdot$ trüeben] truͦebe Fr5 (Fr68)  $\cdot$ jehe] iehin Fr5 \textbf{16} virrec] \textit{om.} Fr68  $\cdot$ ein valkensehe] eins valchin sehin Fr5 \textbf{18} werdeclîchiu] werdichlichez Fr68 \textbf{19} Diu] di Fr68 \textbf{22} hieze] bæte Fr1 \textbf{24} ensol] sol Fr68  $\cdot$ verdriezen] bedriezen Fr68 \textbf{25} denke an] gedenche in Fr1 \textbf{28} daz] \textit{om.} Fr68  $\cdot$ dolst] dultest Fr5 (Fr68) \textbf{30} den] Der Fr5  $\cdot$ mâzen] lazzen Fr1 \newline
\end{minipage}
\hspace{0.5cm}
\begin{minipage}[t]{0.5\linewidth}
\small
\begin{center}*m
\end{center}
\begin{tabular}{rl}
 & stuont niender \textbf{dekeiniu} alsô rôt.\\ 
 & wem si güetlîch ir küssen bôt,\\ 
 & des muose \textit{s}wenden sich der walt\\ 
 & mit maniger juste ungezalt.\\ 
5 & mit lobe wir \textbf{sullen} grüezen\\ 
 & die kiuschen und die süezen\\ 
 & \textbf{maget} Anticonien,\\ 
 & vor \textbf{valschen} die vrîen,\\ 
 & wand si lebete in \textbf{solichem} siten,\\ 
10 & daz niende\textit{r w}as underriten\\ 
 & ir prîs mit valschen worten.\\ 
 & alle, die ir prîs gehôrten,\\ 
 & ieglîc\textit{h} munt ir \textit{wu}nsche\textit{t} dô,\\ 
 & daz ir prîs bestüende alsô\\ 
15 & bewart vor valscher trüeber jeh\textit{e}.\\ 
 & lûter, \textbf{v\textit{e}r\textit{t}ic} als \textbf{ein valkenseh\textit{e}}\\ 
 & was balsammæzic \textbf{stæte} an ir.\\ 
 & daz riet ir werdeclîchiu gir.\\ 
 & diu süez\textit{e}, sælden rîche\\ 
20 & sprach gezogenlîche:\\ 
 & "bruoder, hie bringe ich den degen,\\ 
 & des dû mich selbe hieze pflegen.\\ 
 & nû lâz in mîn geniezen.\\ 
 & \textbf{daz} sol dich niht verdriezen.\\ 
25 & denke an brüederlîche triuwe\\ 
 & und tuo daz âne riuwe.\\ 
 & dir stât manlîchiu triuwe baz,\\ 
 & \textit{\textbf{wenne}} daz dû \textbf{dolest} der \textbf{werlte} haz\\ 
 & und mînen, künde \textit{ich} hazzen.\\ 
30 & den lêre mich gegen dir mâzen."\\ 
\end{tabular}
\scriptsize
\line(1,0){75} \newline
m n o \newline
\line(1,0){75} \newline
\newline
\line(1,0){75} \newline
\textbf{1} niender] niergent n  $\cdot$ dekeiniu] do keiner n dekeiner o \textbf{2} küssen] kúschen o \textbf{3} muose] musse m muͯste n  $\cdot$ swenden] wenden m \textbf{5} sullen] solten n (o) \textbf{7} Anticonien] antitonien n (o) \textbf{8} valschen] falsch n o \textbf{10} niender was] niender pris was m (o) nẏergent prisz was n \textbf{11} ir] Mit n o  $\cdot$ mit] nit o \textbf{13} ieglîch] Ẏegliche m  $\cdot$ ir] er n  $\cdot$ wünsche] vunsche m wunschet n wuͯnsch o  $\cdot$ dô] da o \textbf{15} vor] bẏ n  $\cdot$ valscher] falcher o  $\cdot$ trüeber] trúbe n (o)  $\cdot$ jehe] iehen m (o) \textbf{16} vertic] fuͯreig m  $\cdot$ valkensehe] volken sehen m \textbf{17} ir] [in]: ir o \textbf{19} süeze] sussen m  $\cdot$ sælden] selde o \textbf{20} gezogenlîche] gezúgnissz o \textbf{22} des] Dasz o \textbf{25} denke] Gedencke n (o) \textbf{28} wenne] \textit{om.} m  $\cdot$ dolest] doltest n dúltest o \textbf{29} mînen] mine n  $\cdot$ künde] kúnden o  $\cdot$ ich] \textit{om.} m \newline
\end{minipage}
\end{table}
\newpage
\begin{table}[ht]
\begin{minipage}[t]{0.5\linewidth}
\small
\begin{center}*G
\end{center}
\begin{tabular}{rl}
 & stuont ninder \textbf{deheiniu} als rôt.\\ 
 & swem si güetlîche ir küssen bôt,\\ 
 & des muose swenden sich der walt\\ 
 & mit maniger tjost ungezalt.\\ 
5 & mit lobe wir \textbf{solten} grüezen\\ 
 & die kiuschen unde die süezen,\\ 
 & \textbf{die} \textbf{maget} Antikonien,\\ 
 & vor \textbf{valscheit} die vrîen,\\ 
 & wan si lebet in \textbf{solhe\textit{n}} siten,\\ 
10 & daz ninder was underriten\\ 
 & \begin{large}I\end{large}r brîs mit valschen worten.\\ 
 & alle, die ir brîs gehôrten,\\ 
 & ieslîch munt ir wunschte dô,\\ 
 & daz ir brîs bestüende alsô\\ 
15 & bewart vor \textit{valscher trüebe} jehe.\\ 
 & lûter, \textbf{virrec} als \textbf{ein valkensehe}\\ 
 & was balsemmæzic \textbf{stæte} an ir.\\ 
 & daz riet ir werdiclîchiu gir.\\ 
 & diu süeze, sælden rîche\\ 
20 & sprach gezogenlîche:\\ 
 & "bruoder, hie bringe ich den degen,\\ 
 & des dû mich selbe hieze pflegen.\\ 
 & nû lâze in mîn gen\textit{ie}zen.\\ 
 & \textbf{des} \textbf{en}sol dich niht verdriezen.\\ 
25 & denke an brüederlîche triwe\\ 
 & unt tuo daz âne riwe.\\ 
 & dir stêt manlîchiu triwe baz,\\ 
 & \textbf{dane} daz dû \textbf{dolest} der \textbf{werlde} haz\\ 
 & unt \textbf{den} mînen, künde ich hazzen.\\ 
30 & den lêre mich gein dir mâzen."\\ 
\end{tabular}
\scriptsize
\line(1,0){75} \newline
G I O L M Q R Z Fr21 \newline
\line(1,0){75} \newline
\textbf{1} \textit{Initiale} I O L Z Fr21  \textbf{11} \textit{Initiale} G  \textbf{17} \textit{Initiale} I  \textbf{19} \textit{Initiale} R  \newline
\line(1,0){75} \newline
\textbf{1} stuont] ÷tvͦnt O Stuͦnde R  $\cdot$ ninder deheiniu] da bluͦmen I ninder einiv O (L) nirgen keyn M nidert eine Q nyendert eine R  $\cdot$ als] so [grosz]: rot M \textbf{2} swem si] Wem sie L (M) Wann Q Wem die R  $\cdot$ güetlîche] geliche L  $\cdot$ küssen] kusse M R \textbf{3} muose] muͤste I mynne M  $\cdot$ swenden sich] wendet sich M sich schwenden R \textbf{5} lobe] loben R  $\cdot$ wir solten] mir solde Q \textbf{6} unde] vnde ouch M \textbf{7} die maget] \textit{om.} I O Z Fr21 Die schonen L Vnde die werdin M Die reinen Q (R)  $\cdot$ Antikonien] Anticonien I Antykonîen O anthonien M Antikonyen R ANtigonieN Fr21 \textbf{8} vor valscheit die] die >vor< valsheit I Var valscheit div O Die suͯszen vnd die R \textbf{9} wan] \textit{om.} I  $\cdot$ lebet] lebete I (M) (R) [bebte]: lebte  Q leben Z  $\cdot$ in solhen] in solher G mit soͯlichen R \textbf{10} ninder] nirgen M nider Fr21 \textbf{12} ir] \textit{om.} I  $\cdot$ gehôrten] horten L (Q) \textbf{13} ir] \textit{om.} R  $\cdot$ wunschte] wuͯste L  $\cdot$ dô] dar M \textbf{14} bestüende] belibe I stunde M bestund Q (R) \textbf{15} vor valscher trüebe] vor aller valschen G vor falscher trvben Z  $\cdot$ jehe] gehen I speche R \textbf{16} virrec] virrecht M wirrig R  $\cdot$ ein] \textit{om.} I eynen M es Q  $\cdot$ valkensehe] valchen sehen I falke seche R \textbf{17} balsemmæzic] balschem maz ich I balsem mazichich L balsam niesent R \textbf{18} werdiclîchiu] werdighe Q wirdeklicher R \textbf{19} diu süeze] disev suͤzzev I Die suͦze Fr21  $\cdot$ sælden rîche] selden reichen Q \textbf{20} gezogenlîche] gezogenchliche O (Q) \textbf{22} selbe hieze] selbin hiesze M hissest selbe Q \textbf{23} lâze] lan Q  $\cdot$ geniezen] [genzien]: genzizen G \textbf{24} ensol] sol O Q las R \textbf{25} denke] Gedenche L Bedenck Q  $\cdot$ brüederlîche] bruͯderlicher L bruͦdiche R \textbf{26} tuo daz] tun das Q daz dv Z \textbf{27} dir] [der]: dir G Dy M \textbf{28} dane daz] Denne L Wan das M  $\cdot$ dolest] dvltest O (L) (R) Fr21 geduldest M erberbest Q  $\cdot$ der werlde] der [werde]: werlde G der luͯte L \textit{om.} Q \textbf{29} den] \textit{om.} O L M Q R Z Fr21  $\cdot$ künde ich] kvnden L kundic Q (R)  $\cdot$ hazzen] kaszen L \textbf{30} dir] mir Q \newline
\end{minipage}
\hspace{0.5cm}
\begin{minipage}[t]{0.5\linewidth}
\small
\begin{center}*T
\end{center}
\begin{tabular}{rl}
 & stuont niender \textbf{ein\textit{iu}} alse rôt.\\ 
 & swem si güetlîche ir küssen bôt,\\ 
 & des muose swenden sich der walt\\ 
 & mi\textit{t} \textit{m}aneger tjost ungezalt.\\ 
5 & mit lobe wir \textbf{solten} grüezen\\ 
 & die kiuschen unde die süezen,\\ 
 & \textbf{die} \textbf{reine\textit{n}} Antickonien,\\ 
 & vor \textbf{valscheite} di\textit{e} vrîen,\\ 
 & wan si lebete in \textbf{solhen} siten,\\ 
10 & daz niender was underriten\\ 
 & ir prîs mit valschen worten.\\ 
 & alle, die ir prîs gehôrten,\\ 
 & iegeslîch munt ir wunschete dô,\\ 
 & daz ir prîs bestüende alsô\\ 
15 & bewart vor valscher trüebe jehe.\\ 
 & Lûter, \textbf{wiric} als \textbf{eines valken sehe}\\ 
 & was balsemmæzic \textbf{stat} an ir.\\ 
 & daz riet ir werdeclîch\textit{iu} gir.\\ 
 & \begin{large}D\end{large}iu süeze sældenrîche\\ 
20 & sprach gezogenlîche:\\ 
 & "bruoder, hie bringich den degen,\\ 
 & des dû mich selbe hieze pflegen.\\ 
 & nû lâ in mîn geniezen.\\ 
 & \textbf{des} sol dich \textit{niht} verdriezen.\\ 
25 & denke an brüederlîche triuwe\\ 
 & unde tuo daz âne riuwe.\\ 
 & dir stât manlîch\textit{iu} triuwe baz,\\ 
 & \textbf{denne} daz dû \textbf{dultes} der \textbf{werden} haz\\ 
 & unde mînen, kündich hazzen.\\ 
30 & den lêre mi\textit{ch} gegen di\textit{r} mâzen."\\ 
\end{tabular}
\scriptsize
\line(1,0){75} \newline
T U V W \newline
\line(1,0){75} \newline
\textbf{16} \textit{Majuskel} T  \textbf{19} \textit{Initiale} T U  \newline
\line(1,0){75} \newline
\textbf{1} niender] nider U niergent V  $\cdot$ einiu] eine T deheine V \textbf{2} swem] Wem U W Swen V  $\cdot$ küssen] lachen W \textbf{3} muose swenden] muͦze wenden U mvͤste swenden V (W) \textbf{4} mit maneger] mit mit maneger T \textbf{7} reinen] reine T  $\cdot$ Antickonien] Antikonien U antyconien V antikoneyen W \textbf{8} die] div T \textbf{10} niender] nider U niergent V  $\cdot$ underriten] vndersnitten V \textbf{12} ir] irn U \textbf{15} trüebe jehe] trvͤben iehe V treúwe iehen W \textbf{16} wiric] virric U (V) (W)  $\cdot$ eines] ein U W \textbf{17} stat] stete V \textbf{18} werdeclîchiu] werdecliche T virdecliche U \textbf{19} süeze] svͤzen V \textbf{22} selbe hieze] hieze selber U [*]: selbe hiesse V selber hiessest W \textbf{23} lâ] [*]: los V \textbf{24} [*]: Dez sol dich niht verdriessen V  $\cdot$ niht] \textit{om.} T \textbf{25} denke] Denken U [*]: Denke V \textbf{26} [*]: vnde tuͦ daz one ruwe V \textbf{27} dir] [*]: Dir V  $\cdot$ manlîchiu] manliche T  $\cdot$ baz] \textit{om.} U \textbf{28} [*]: Den daz du tultest der welte haz V  $\cdot$ dultes der werden] dultest der werde U doldest weltlichen W \textbf{29} kündich] kuͦndigen U [*]: kvnde ich V \textbf{30} den] [*]: Den V  $\cdot$ lêre] leite W  $\cdot$ mich] >mir< T  $\cdot$ dir] dirre T mir W \newline
\end{minipage}
\end{table}
\end{document}
