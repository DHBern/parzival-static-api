\documentclass[8pt,a4paper,notitlepage]{article}
\usepackage{fullpage}
\usepackage{ulem}
\usepackage{xltxtra}
\usepackage{datetime}
\renewcommand{\dateseparator}{.}
\dmyyyydate
\usepackage{fancyhdr}
\usepackage{ifthen}
\pagestyle{fancy}
\fancyhf{}
\renewcommand{\headrulewidth}{0pt}
\fancyfoot[L]{\ifthenelse{\value{page}=1}{\today, \currenttime{} Uhr}{}}
\begin{document}
\begin{table}[ht]
\begin{minipage}[t]{0.5\linewidth}
\small
\begin{center}*D
\end{center}
\begin{tabular}{rl}
\textbf{475} & \begin{large}S\end{large}ult ir in nemen in iwer gebet.\\ 
 & mîn vater, \textbf{der} hiez Gahmuret;\\ 
 & \textbf{er} was von arde ein Anschevin.\\ 
 & hêrre, ich \textbf{en}bin \textbf{ez} niht, Læhelin.\\ 
5 & genam ich ie den rêroup,\\ 
 & sô was ich an den witzen toup.\\ 
 & ez ist iedoch \textbf{von} mir \textbf{geschehen}.\\ 
 & der selben \textbf{sünde} muoz ich jehen:\\ 
 & Ithern von Kukumerlant,\\ 
10 & den sluoc mîn sündebæriu hant.\\ 
 & ich leite in tôten ûffez gras\\ 
 & unt nam, swaz \textbf{dâ} ze nemen was."\\ 
 & "Owê, werlt, wie tuostû sô?",\\ 
 & sprach der wirt; \textbf{der} was \textbf{des mæres} unvrô.\\ 
15 & "dû gîst den liuten herzesêr\\ 
 & unt riwebæres mêr\\ 
 & denne der vreude. wie stêt dîn lôn?\\ 
 & sus endet sich dînes mæres dôn."\\ 
 & dô sprach er: "lieber swester sun,\\ 
20 & waz râtes möht ich dir \textbf{nû} \textbf{tuon}?\\ 
 & dû hâst dîn eigen verch erslagen.\\ 
 & wil dû vür got die schulde tragen,\\ 
 & \textbf{sît} daz ir bêde wâret ein bluot,\\ 
 & ob got dâ reht gerihte tuot,\\ 
25 & sô gilt im dîn eigen leben.\\ 
 & waz wildû im dâ ze gelte geben,\\ 
 & Ithern von Gaheviez?\\ 
 & der rehten werdecheit geniez,\\ 
 & des diu werlt was gereinet,\\ 
30 & \textbf{het got} an im erscheinet.\\ 
\end{tabular}
\scriptsize
\line(1,0){75} \newline
D \newline
\line(1,0){75} \newline
\textbf{1} \textit{Initiale} D  \textbf{13} \textit{Majuskel} D  \newline
\line(1,0){75} \newline
\textbf{3} Anschevin] Anshevin D \textbf{9} Ithern] Jthern D  $\cdot$ Kukumerlant] Cvnchvmerlant D \textbf{27} Ithern] Jthern D  $\cdot$ Gaheviez] kahevîez D \newline
\end{minipage}
\hspace{0.5cm}
\begin{minipage}[t]{0.5\linewidth}
\small
\begin{center}*m
\end{center}
\begin{tabular}{rl}
 & solt ir  nemen in iuwer gebe\textit{t}.\\ 
 & mîn vater hiez Ga\textit{h}mure\textit{t};\\ 
 & \textbf{er} was von art ein A\textit{n}schevin.\\ 
 & hêrre, ich bin niht Lehelin.\\ 
5 & gen\textit{am} ich ie den rêroup,\\ 
 & sô was ich an den witzen toup.\\ 
 & ez ist iedoch \textbf{von} mir \textbf{beschehen}.\\ 
 & der selben \textbf{sün\textit{d}e\textit{n}} muoz ich jehen:\\ 
 & I\textit{t}h\textit{ern} von Kukumerlant,\\ 
10 & den sluoc mîn sündebæriu hant.\\ 
 & ich leite in tôten ûf daz gras\\ 
 & und nam, waz ze nemenne was."\\ 
 & "ouwê, werlt, wie tuostû sô?",\\ 
 & sprach der wirt; \textbf{der} was \textbf{der mære} unvrô.\\ 
15 & "dû gîst den liuten herzen sêre\\ 
 & und riuwebæres \textbf{\textit{k}umber\textit{s}} mêre\\ 
 & \textit{den}n\textit{e} der vröude. \textit{wie} stât dîn lôn?\\ 
 & sus endet sich dînes mæres dôn."\\ 
 & dô sprach er: "lieber swester sun,\\ 
20 & waz râtes möht ich dir \textbf{getuon}?\\ 
 & d\textit{û} hâst dîn eigen verch erslagen.\\ 
 & wiltû vür got die schulde tragen,\\ 
 & \textbf{sît} daz ir beide wâ\textit{rt} ein bluot,\\ 
 & ob got d\textit{â} rehte \textit{ge}rih\textit{t}e tuot,\\ 
25 & sô gilte\textit{t} im dîn eigen leben.\\ 
 & waz wiltû ime dâr zuo gelte g\textit{e}ben,\\ 
 & Ither von Gaheviez,\\ 
 & der rehten wir\textit{di}cheit gen\textit{i}ez?\\ 
30 & \hspace*{-.7em}\big| \textbf{got het} an ime ersch\textit{e}inet,\\ 
 & \hspace*{-.7em}\big| des diu werlt was gereinet.\\ 
\end{tabular}
\scriptsize
\line(1,0){75} \newline
m n o \newline
\line(1,0){75} \newline
\textbf{13} \textit{Initiale} \textsuperscript{2}\hspace{-1.3mm} m   $\cdot$ \textit{Capitulumzeichen} n  \newline
\line(1,0){75} \newline
\textbf{1} gebet] [b]: gebette m gebette o \textbf{2} hiez] der hiesz n (o)  $\cdot$ Gahmuret] gahamurette m gamuret n gamúret o \textbf{3} ein] \textit{om.} o  $\cdot$ Anschevin] ausceuin m auscavin n antovin o \textbf{4} bin] bins n o \textbf{5} genam] Genien m  $\cdot$ rêroup] rauͯp o \textbf{7} iedoch] edoch o \textbf{8} sünden] suͯnne m \textbf{9} Ithern] Jch \textit{nachträglich korrigiert zu:} Jther m Jtel n Jthern o  $\cdot$ Kukumerlant] kukumer lant m kacuͯmerlant o \textbf{11} tôten] dot n \textbf{12} ze] do zuͯ n (o) \textbf{13} \textit{Versdoppelung 475.13-14 (²m) nach 475.13; Lesarten der vorausgehenden Verse mit ¹m bezeichnet} m   $\cdot$ tuostû] túschú o \textbf{14} mære] meren o \textbf{16} riuwebæres] rueberes o  $\cdot$ kumbers] thomer m \textbf{17} denne] Von m  $\cdot$ vröude] freiden n  $\cdot$ wie] \textit{om.} m \textbf{20} möht] mocht o  $\cdot$ getuon] nuͯ duͯn n (o) \textbf{21} dû] Do m \textbf{22} vür] vor m \textbf{23} wârt] was m \textbf{24} dâ] do m n o  $\cdot$ rehte gerihte] rehtte rihe m >recht< gerechte o \textbf{25} giltet] giltte m \textbf{26} dâr] do n o  $\cdot$ geben] [gege*]: gegben m \textbf{27} Ither] Jther m n Jthier o  $\cdot$ Gaheviez] gahe vies m gahaviesz n gahavies o \textbf{28} wirdicheit geniez] wirkeit geniyes m \textbf{30} het] hat n  $\cdot$ erscheinet] erschinet m \newline
\end{minipage}
\end{table}
\newpage
\begin{table}[ht]
\begin{minipage}[t]{0.5\linewidth}
\small
\begin{center}*G
\end{center}
\begin{tabular}{rl}
 & \begin{large}S\end{large}ult ir in nemen in iuwer gebet.\\ 
 & mîn vater, \textbf{der} hiez Gahmuret;\\ 
 & \textbf{er} was von arde ein Anschevin.\\ 
 & hêrre, ich \textbf{en}bin\textbf{z} niht, Lehelin.\\ 
5 & genam ich \textit{i}e den rêroup,\\ 
 & sô was ich an den witzen toup.\\ 
 & ez ist iedoch \textbf{von} mir \textbf{geschehen}.\\ 
 & de\textit{r} selben \textbf{sünden} muoz ich jehen:\\ 
 & Itheren von Kukumerlant,\\ 
10 & den sluoc mîn sündebæ\textit{r}iu hant.\\ 
 & ich leit in tôten ûf daz gras\\ 
 & unt nam, swaz \textbf{dâ} ze nemen was."\\ 
 & "owê, werlt, wie tuostû sô?",\\ 
 & sprach der wirt; \textbf{der} was \textbf{des mæres} unvrô.\\ 
15 & "dû gîst den liuten herzesêr\\ 
 & unde riuwebæres \textbf{kumbers} mêr\\ 
 & danne der vröude. wie stêt dîn lôn?\\ 
 & sus endet sich dînes mæres dôn."\\ 
 & dô sprach er: "lieber swester sun,\\ 
20 & waz râtes möht i\textit{ch} dir \textbf{nû} \textbf{tuon}?\\ 
 & dû hâst dîn eigen verch erslagen.\\ 
 & wil dû vür got die schulde tragen,\\ 
 & \textbf{sît} daz ir bêde wârt ein bluot,\\ 
 & ob got dâ rehte gerihte tuot,\\ 
25 & sô giltet im dîn eigen leben.\\ 
 & waz wild\textit{û} im dâ ze gelte geben,\\ 
 & Ithern von Kahaviez?\\ 
 & der rehten werdecheit geniez,\\ 
 & des diu werlt was gereinet,\\ 
30 & \textbf{\textit{het} got} an im erscheinet.\\ 
\end{tabular}
\scriptsize
\line(1,0){75} \newline
G I O L M Z Fr18 Fr49 \newline
\line(1,0){75} \newline
\textbf{1} \textit{Initiale} G I O L Z Fr18  \textbf{13} \textit{Initiale} I  \textbf{19} \textit{Initiale} M  \newline
\line(1,0){75} \newline
\textbf{1} Sult] ÷vlt O  $\cdot$ in nemen in iuwer] nennen in uwirme M \textbf{2} der] \textit{om.} I L Fr49  $\cdot$ Gahmuret] gahmvret G (L) (Fr18) Gamvret O gamuͯret M Gamuret Z \textbf{3} Anschevin] antsheuin I anshevin O Z (Fr18) ansevin M \textbf{4} hêrre] \textit{om.} O Fr18  $\cdot$ enbinz] pinz I (L) (M) (Fr49)  $\cdot$ Lehelin] Læhelin O lehelein Fr49 \textbf{5} ie] hîe G \textit{om.} M \textbf{6} den] \textit{om.} I Fr49 \textbf{8} der] Des G  $\cdot$ sünden] svnde O (M) Z Fr18 (Fr49)  $\cdot$ muoz ich] ich mvͦz O (M) Fr18 \textbf{9} Itheren] Jthern I O M Jethern L Fr18 Jchern Z Jkern Fr49  $\cdot$ Kukumerlant] kamvrlant G cukumerlant I kvcvmerlant O Fr18 (Fr49) Cvcvmerlant L kukumber lant M Chvncumerlant Z \textbf{10} sündebæriu] svndebærhiv G sundigiu I (Fr49) \textbf{11} tôten] tot M \textbf{12} swaz] waz L (M) Z Fr49  $\cdot$ dâ ze nemen] an im I Fr49 daz zuͯ nemene L \textbf{14} der was des mæres] vnde wart O (Fr18) der waz der mere L der wart dez mers Fr49  $\cdot$ unvrô] vro M \textbf{15} herzesêr] herzen ser I (M) (Z) (Fr49) \textbf{16} riuwebæres] riwers O ruveris M \textbf{17} vröude] frevden O (L) (Fr18) \textbf{18} dînes mæres] des iamers I (Fr49) der mær O din mær Fr18 \textbf{20} ich] ir G M  $\cdot$ tuon] getun Z \textbf{21} verch] verc M \textbf{23} ein] eynen M \textbf{24} dâ] \textit{om.} I Fr49 \textbf{25} eigen] augen I selbes L  $\cdot$ leben] [geben]: leben I \textbf{26} wildû] wilde G  $\cdot$ dâ] \textit{om.} O L Fr18 \textbf{27} Ithern] Jthern I (O) (M) (Fr18) (Fr49) Jehtern L Jchern Z  $\cdot$ von] vo I  $\cdot$ Kahaviez] chachueis I Kaheviez L (Z) kahevisz M Cahauis Fr49 \textbf{29} diu] du I Fr49 \textbf{30} het] Eh G Hat M  $\cdot$ an] [von]: an O  $\cdot$ erscheinet] gescheinet O Fr18 \newline
\end{minipage}
\hspace{0.5cm}
\begin{minipage}[t]{0.5\linewidth}
\small
\begin{center}*T
\end{center}
\begin{tabular}{rl}
 & sult irn nemen in iuwer gebet.\\ 
 & mîn vater, \textbf{der} hiez Gahmuret\\ 
 & \textbf{unde} was von art ein Anschevin.\\ 
 & hêrre, ich bin\textbf{z} niht, Lehelin.\\ 
5 & genam ich ie den rêroup,\\ 
 & sô was ich an den witzen toup.\\ 
 & ez ist iedoch mir \textbf{geschehen}.\\ 
 & der selben \textbf{sünden} muoz ich jehen:\\ 
 & Ithern von Kukumerlant,\\ 
10 & den sluoc mîn sündebæriu hant.\\ 
 & ich leitin tôten ûffez gras\\ 
 & unde nam, swaz \textbf{dâ} ze nemenne was."\\ 
 & "Ouwê, werlt, wie tuostû sô?",\\ 
 & sprach der wirt \textbf{unde} was \textbf{des mæres} unvrô.\\ 
15 & "dû gîst den liuten herzesêr\\ 
 & unde riuwebæres \textbf{kumbers} mêr\\ 
 & danne der vröuden. wie stât dîn lôn?\\ 
 & sus endet sich dînes mæres dôn."\\ 
 & Dô sprach er: "lieber swester sun,\\ 
20 & waz râtes m\textit{ö}hte ich dir \textbf{nû} \textbf{tuon}?\\ 
 & dû hâst dîn eigen verch erslagen.\\ 
 & wiltû vür got die schulde tragen,\\ 
 & \textbf{sô} daz ir beide wârt ein bluot,\\ 
 & ob got dâ reht gerihte tuot,\\ 
25 & sô giltet im dîn eigen leben.\\ 
 & waz wiltû im dâ ze \textit{ge}lte geben,\\ 
 & Ithern von Kaheviez?\\ 
 & der rehten werdecheit geniez,\\ 
 & des diu werlt was gereinet,\\ 
30 & \textbf{hete got} an im erscheinet.\\ 
\end{tabular}
\scriptsize
\line(1,0){75} \newline
T U V W Q R \newline
\line(1,0){75} \newline
\textbf{1} \textit{Initiale} W  \textbf{13} \textit{Majuskel} T  \textbf{19} \textit{Initiale} W R   $\cdot$ \textit{Majuskel} T  \newline
\line(1,0){75} \newline
\textbf{1} \textit{Die Verse 453.1-502.30 fehlen} U   $\cdot$ sult irn] Den soͤllent ir nemen V SVlt ir nennen W  $\cdot$ iuwer] eúwerm W (R) \textbf{2} der] \textit{om.} W R  $\cdot$ Gahmuret] Gamvret V (W) gaműret Q Gamurett R \textbf{3} Anschevin] anscevin T antscheuin W anscheűin Q anshevin R \textbf{4} binz] enbins V enbin W bin R  $\cdot$ Lehelin] lechelin R \textbf{7} iedoch] ie W  $\cdot$ mir] von mir V W Q R \textbf{8} muoz ich] ich mvͦz V (W) (Q) \textbf{9} Ithern] Jthern T Ytern V Ythern W Jchern Q Jhtern R  $\cdot$ Kukumerlant] Cvkvmerlant T Kukűmerlant Q kumnerlant R \textbf{10} den] Der Q  $\cdot$ mîn] im Q  $\cdot$ sündebæriu] [*]: svndebere V sunderbere W R \textbf{11} leitin] in valte V  $\cdot$ tôten] tot V  $\cdot$ gras] land R \textbf{12} Vnd nam das ich da fand R  $\cdot$ swaz dâ] do swaz V do was W was do Q \textbf{14} mæres] \textit{om.} W  $\cdot$ unvrô] gar vnfro R \textbf{15} herzesêr] hertzen ser W herten ser Q herre sere R \textbf{16} riuwebæres] rúweres W  $\cdot$ kumbers] Rvmmers Q \textbf{17} wie] nu Q \textbf{18} sus] Es Q  $\cdot$ dînes] din V seines W des R \textbf{19} er] der wirt R \textbf{20} möhte] mohte T (Q) (R)  $\cdot$ ich dir] ir R  $\cdot$ nû] \textit{om.} W  $\cdot$ tuon] getvn V \textbf{21} verch] werck Q werch R \textbf{22} tragen] nú tragen Q \textbf{23} sô] Sit V (W) (Q) (R)  $\cdot$ beide] beidu R \textbf{24} dâ] do V W Q \textbf{25} dîn] [*]: din V sein W dine R \textbf{26} dâ] denne R  $\cdot$ ze gelte] zelte T zuͦ gelten W \textbf{27} Ithern] Ytern V Ythern W Jchern Q  $\cdot$ Kaheviez] kahevies V kahafies W kahevicz Q kachevies R \textbf{30} hete got] [*z d*]: Got hette V hat W  $\cdot$ erscheinet] bescheinet W \newline
\end{minipage}
\end{table}
\end{document}
