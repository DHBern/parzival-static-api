\documentclass[8pt,a4paper,notitlepage]{article}
\usepackage{fullpage}
\usepackage{ulem}
\usepackage{xltxtra}
\usepackage{datetime}
\renewcommand{\dateseparator}{.}
\dmyyyydate
\usepackage{fancyhdr}
\usepackage{ifthen}
\pagestyle{fancy}
\fancyhf{}
\renewcommand{\headrulewidth}{0pt}
\fancyfoot[L]{\ifthenelse{\value{page}=1}{\today, \currenttime{} Uhr}{}}
\begin{document}
\begin{table}[ht]
\begin{minipage}[t]{0.5\linewidth}
\small
\begin{center}*D
\end{center}
\begin{tabular}{rl}
\textbf{800} & \textit{\begin{large}D\end{large}}es tages blic was dennoch grâ.\\ 
 & Kyot \textbf{iedoch} erkant \textbf{al} dâ\\ 
 & des Grâles wâpen an der schar.\\ 
 & si vuorten turteltûben gar.\\ 
5 & dô ersiufte sîn alter lîp,\\ 
 & wan Schoysianen, sîn \textbf{kiusche} wîp,\\ 
 & ze Munsalvæsche im sælde erwarp,\\ 
 & \textbf{diu} von Sigunen gebürte erstarp.\\ 
 & Kyot gein Parzivale gienc,\\ 
10 & in unt die sîne er wol enpfienc.\\ 
 & Er sant ein junchêrrelîn\\ 
 & nâch dem marschalke der künegîn\\ 
 & \textbf{unt} bat \textbf{in} schaffen guot gemach,\\ 
 & swaz er dâ rîter halden sach.\\ 
15 & Er vuort in \textbf{selben} \textbf{mit} der hant,\\ 
 & dâ er der küneginne kamern vant,\\ 
 & ein kleine gezelt von buckeram.\\ 
 & daz harnasch man \textbf{gar} von im \textbf{dâ} nam.\\ 
 & diu künegîn des \textbf{noch} niht \textbf{en}weiz.\\ 
20 & Loherangrin unt Kardeiz\\ 
 & vant Parzival bî ir ligen.\\ 
 & dô muose vreude an im gesigen.\\ 
 & in eime gezelt hôch unt wît,\\ 
 & \textbf{dâ} her unt \textbf{dâ} \textbf{in} alle sît\\ 
25 & clârer vrouwen lac genuoc.\\ 
 & Kyot ûfez \textbf{declachen} sluoc.\\ 
 & er bat die küneginne wachen\\ 
 & unt vrœlîche lachen.\\ 
 & \textit{\begin{large}S\end{large}}i blicte ûf unt sach \textbf{ir man}.\\ 
30 & si hete niht wanz hemde an.\\ 
\end{tabular}
\scriptsize
\line(1,0){75} \newline
D \newline
\line(1,0){75} \newline
\textbf{1} \textit{Initiale} D  \textbf{11} \textit{Majuskel} D  \textbf{15} \textit{Majuskel} D  \textbf{29} \textit{Initiale} D  \newline
\line(1,0){75} \newline
\textbf{1} Des] ÷es D \textbf{6} Schoysianen] Scoysianen D \textbf{7} Munsalvæsche] Mvnsalvæsce D \textbf{9} Parzivale] Parcifale D \textbf{21} Parzival] [Parcifaln]: Parcifal D \textbf{29} Si] ÷i D \newline
\end{minipage}
\hspace{0.5cm}
\begin{minipage}[t]{0.5\linewidth}
\small
\begin{center}*m
\end{center}
\begin{tabular}{rl}
 & des tages blic was dannoch grâ.\\ 
 & Kyot erkante d\textit{â}\\ 
 & des Grâles wâpen an der schar.\\ 
 & si vuorten turteltûben gar.\\ 
5 & dô ersiufzet sîn alter lîp,\\ 
 & wan Schoisianen, sîn \textbf{süeze\textit{z}} wîp,\\ 
 & zuo Muntsalvasche im sælde erwarp,\\ 
 & \textbf{diu} von Sigunen geburt erstarp.\\ 
 & Kyot gegen Parcifal gienc,\\ 
10 & in und die sîn er wol enpfienc.\\ 
 & er sant ein junchêrrelîn\\ 
 & nâch dem marscha\textit{l}k der künigîn\\ 
 & \textbf{und} bat \textbf{in} schaffen guot gemach,\\ 
 & waz er d\textit{â} ritter halten sach.\\ 
15 & er vuorte in \textbf{selben} \textbf{mit} der hant,\\ 
 & d\textit{â} er der künigîn kamern vant,\\ 
 & ein klein gezelt von buckeram.\\ 
 & daz harnasch man von im \textbf{d\textit{â}} nam.\\ 
 & diu künigîn des \textbf{noch} niht weiz.\\ 
20 & Lohelangrin und C\textit{a}rdeiz\\ 
 & vant Parcifal bî ir ligen.\\ 
 & dô muos vröude an im gesigen.\\ 
 & in ein\textit{em} gezelt hôch und wît,\\ 
 & \textbf{d\textit{â}} her und \textbf{in} all\textit{e} sît\\ 
25 & clârer vrowen lac genuoc.\\ 
 & Kyot ûf daz \textbf{declachen} sluoc.\\ 
 & er bat die künigîn wachen\\ 
 & und vrœlîchen lachen.\\ 
 & si blicket ûf und sach \textbf{ir man}.\\ 
30 & si het niht wan daz hemde an.\\ 
\end{tabular}
\scriptsize
\line(1,0){75} \newline
m n o V V' W \newline
\line(1,0){75} \newline
\textbf{1} \textit{Initiale} V  \textbf{9} \textit{Initiale} W  \newline
\line(1,0){75} \newline
\textbf{1} tages] tagez tages o  $\cdot$ blic] \textit{om.} n blike V (V')  $\cdot$ was dannoch grâ] wasz dannoch [fro]: gro o [erkante do]: waz dannoch gro V' \textbf{2} Kyot] Kẏot m Kiot n  $\cdot$ erkante] iedoch erkante V (V')  $\cdot$ dâ] do m n o V V' \textbf{4} turteltûben] durczel tuben o \textbf{5} dô] Der o  $\cdot$ ersiufzet] ersúfzete V (V') (W)  $\cdot$ alter] alten o \textbf{6} Schoisianen] Scoisianen m n scoisanen o [*]: sociane V sociane V' tschosiane W  $\cdot$ süezez] suͯssen m \textbf{7} Muntsalvasche] muntsaluasce m n (o) montsalfasche V mvnschalsche V' montsaluatschs W  $\cdot$ im] \textit{om.} V' \textbf{8} Sigunen] siguͯnen m sigune V \textit{om.} V' sygunen W  $\cdot$ geburt] edilme geburte V' \textbf{9} Kyot] Kẏot m  $\cdot$ Parcifal] parzefale V parzifale V' partzifale W \textbf{10} sîn] sinen V (W)  $\cdot$ er] \textit{om.} o \textbf{11} ein] in n \textbf{12} marschalk] marschag m \textbf{14} waz] Swaz V  $\cdot$ dâ ritter] do ritter m n o (V) W ritter do V'  $\cdot$ halten] haben V' \textbf{15} selben] selber n W selbe o V V' \textbf{16} dâ] Do m n o V V' W  $\cdot$ künigîn] konig o  $\cdot$ kamern] kamerer o kamere V (V') (W) \textbf{17} buckeram] buckorren o bruckeram V' \textbf{18} dâ] do m n o V \textit{om.} V' W \textbf{19} des noch niht] nach nit daz V'  $\cdot$ weiz] enweiz V \textbf{20} Lohelangrin] Loelangrin o  $\cdot$ Cardeiz] cordeis m cardeis n o V' cardeiß W \textbf{21} Parcifal] Parzefal V partzifal W \textbf{22} muos] mvͤste V \textbf{23} \textit{Die Verse 800.23-25 fehlen} V'   $\cdot$ einem] ein m n o V \textbf{24} dâ] Do m n o V \textit{om.} W  $\cdot$ und] vnd dar W  $\cdot$ alle] allen m \textbf{25} lac] [*]: lag V \textbf{26} \textit{Verse 800.26-27 kontrahiert zu:} Kyot bat die kvnigin wachen V'   $\cdot$ Kyot] Kẏot m n  $\cdot$ ûf daz] vffens V \textbf{28} Vnd zucte sy sy begonde lachen V' \textbf{29} blicket] blickete V sach W  $\cdot$ ir] den V' \textbf{30} het] enhette V' hat W \newline
\end{minipage}
\end{table}
\newpage
\begin{table}[ht]
\begin{minipage}[t]{0.5\linewidth}
\small
\begin{center}*G
\end{center}
\begin{tabular}{rl}
 & \begin{large}D\end{large}es tages blic was dannoch grâ.\\ 
 & Kiot \textbf{iedoch} erkande \textbf{al} dâ\\ 
 & des Grâles wâpen an der schar.\\ 
 & si vuorten türteltûben gar.\\ 
5 & dô ersiufte sîn alter lîp,\\ 
 & wan Schoysiane, sîn \textbf{kiusche} wîp,\\ 
 & ze Muntsalfatsche im sælde erwarp,\\ 
 & \textbf{dâ} von Sigunen geburt erstarp.\\ 
 & Kiot gein Parzivale gienc,\\ 
10 & in unde die sîne er wol enpfienc.\\ 
 & er sande ein junchêrrelîn\\ 
 & nâch dem marschalke der künigîn\\ 
 & \textbf{unde} bat \textbf{in} schaffen guot gemach,\\ 
 & swaz er dâ rîter halden sach.\\ 
15 & er vuorte in \textbf{selben} \textbf{bî} der hant,\\ 
 & dâ er der künigîn kamer vant,\\ 
 & ein kleine gezelt von buggeram.\\ 
 & daz harnasch man \textbf{gar} von im nam.\\ 
 & diu künigîn des niht \textbf{en}weiz.\\ 
20 & Loherangrin unde Kardeiz\\ 
 & vant Parzival bî ir ligen.\\ 
 & dô muose vröude an im gesigen.\\ 
 & in ein\textit{em} gezelt hôch unde wît,\\ 
 & her unde \textbf{dâ} \textbf{in} alle sît\\ 
25 & clâre\textit{r} vrouwen lac genuoc.\\ 
 & Kiot ûffez \textbf{declachen} sluoc.\\ 
 & er bat die künigîn wachen\\ 
 & unde vrœlîchen lachen.\\ 
 & si blicte ûf unde sach \textbf{ir man}.\\ 
30 & si\textbf{ne} hete niht wanez hemde an.\\ 
\end{tabular}
\scriptsize
\line(1,0){75} \newline
G I L M Z \newline
\line(1,0){75} \newline
\textbf{1} \textit{Initiale} G L Z  \textbf{3} \textit{Initiale} I  \textbf{23} \textit{Initiale} I  \newline
\line(1,0){75} \newline
\textbf{1} tages] morgens M \textbf{2} Kiot] Kýot L Kyot Z  $\cdot$ al] \textit{om.} L M \textbf{4} si] die I \textbf{5} dô] Da M Z  $\cdot$ ersiufte] er svnftzte L  $\cdot$ alter lîp] alte [wip]: lip M \textbf{6} Schoysiane] scoẏsian G schoisian I schosiane L scosian M tschoisiane Z  $\cdot$ sîn] sine Z \textbf{7} Muntsalfatsche] mvntsalvatsche G (L) muntsaluasce I Munsalvatsche M montsalvatsche Z  $\cdot$ sælde] solt I selbe L \textbf{8} dâ] Die Z \textbf{9} Kiot] kẏot G (L) Kyot Z  $\cdot$ Parzivale] parcivale G parzifal I M parzifale L parcifaln Z \textbf{10} sîne] sinen Z \textbf{13} bat] \textit{om.} M \textbf{14} swaz] Waz L (M) \textbf{15} vuorte] vuͤrt I (Z)  $\cdot$ selben] selbe L Z \textbf{16} Zuͯ der kvnigýnne ob der er vant L \textbf{18} daz] Dar M  $\cdot$ man gar] man do L man M gar man Z  $\cdot$ nam] da nam M (Z) \textbf{19} niht enweiz] noch [nihtenweiz]: niht enweiz L noch nicht weisz M (Z) \textbf{20} Loherangrin] Laheragrim I Joherangrin L Johangryn M Loagrin Z  $\cdot$ Kardeiz] karedeiz L kardeisz M \textbf{21} Parzival] parcifal G Z parzifal I M parazifal L \textbf{22} dô] Da M Des Z \textbf{23} einem] ein G I (M) \textbf{24} alle] allen I L Z \textbf{25} clârer] Clare G \textbf{26} Kiot] kẏot G Kýot L Kyot Z \textbf{27} wachen] [lachen]: wachen I \textbf{29} \textit{Vers 800.29 fehlt} L   $\cdot$ blicte] blicht I (Z) \textbf{30} \textit{nach 800.30:} Do sẏ wachte der werde man L   $\cdot$ sine] Si I (M)  $\cdot$ niht wanez] niht wan ir I \newline
\end{minipage}
\hspace{0.5cm}
\begin{minipage}[t]{0.5\linewidth}
\small
\begin{center}*T
\end{center}
\begin{tabular}{rl}
 & des tages blic was dannoch grâ.\\ 
 & Kyot \textbf{iedoch} erk\textit{a}nte \textbf{al}dâ\\ 
 & des Grâles wâpen an der schar.\\ 
 & si vuorten tur\textit{tel}tûben gar.\\ 
5 & dô ersiufzete sîn alter lîp,\\ 
 & wan Schosiane, sîn \textbf{kiuschez} wîp,\\ 
 & zuo Munsalvasche im sælde erwarp,\\ 
 & \textbf{diu} von Sygunen geburt erstarp.\\ 
 & Kyot gein Parcifale gienc,\\ 
10 & in und die sîne er wol enpfienc.\\ 
 & er sante ein junchêrrelîn\\ 
 & nâch dem marschalke der künegîn.\\ 
 & \textbf{den} bat \textbf{er} schaffen guot gemach,\\ 
 & waz er dâ rîter halten sach.\\ 
15 & er vuorte in \textbf{selber} \textbf{bî} der hant,\\ 
 & dâ er der küneginne kamer vant,\\ 
 & ein kleine gezelt von buckeram.\\ 
 & \begin{large}D\end{large}en harnasch man von im \textbf{d\textit{â}} nam.\\ 
 & diu künegîn des \textbf{noch} niht \textbf{en}weiz.\\ 
20 & Lohrangrin und Kardeiz\\ 
 & vant Parcifal bî ir ligen.\\ 
 & dô muose vreude an im gesigen.\\ 
 & in eime gezelt hôch und wît,\\ 
 & her und \textbf{dâ} alle sît\\ 
25 & clârer vrouwen lac genuoc.\\ 
 & Kyot ûf daz \textbf{bette} sluoc.\\ 
 & er bat die küneginne wachen\\ 
 & und vrœlîche lachen.\\ 
 & si blicte ûf und sach \textbf{in an}.\\ 
30 & si \textbf{en}hâte niht wan daz hemede an.\\ 
\end{tabular}
\scriptsize
\line(1,0){75} \newline
U Q R \newline
\line(1,0){75} \newline
\textbf{1} \textit{Initiale} R  \textbf{18} \textit{Initiale} U  \newline
\line(1,0){75} \newline
\textbf{2} erkante] erkente U \textbf{3} der] dem Q \textbf{4} turteltûben] turtuben U turteltublin R \textbf{5} ersiufzete] er sunfftte R \textbf{6} Schosiane] tschosian Q Schoisiane R \textbf{7} Munsalvasche] muͦntsalvatsche U muntsaluasche Q Munshaͯlesche R \textbf{8} Sygunen] Syguͦnen U sigúnen Q  $\cdot$ geburt] burck Q purt R \textbf{9} Parcifale] Parzifale U partzifalen Q parczifaln R \textbf{10} sîne] sinen R \textbf{13} den bat er] Vnd bat in Q Er bat Jn R \textbf{14} dâ] do Q \textbf{16} dâ] Do Q  $\cdot$ küneginne] kúnginnen R \textbf{17} \textit{Versfolge 800.18-17} U  \textbf{18} Den] Das Q R  $\cdot$ von im dâ] von im do U do von im Q (R) \textbf{19} noch niht] nicht noch Q  $\cdot$ enweiz] weisz Q (R) \textbf{20} Lohrangrin] Lorangrin U Lohrangin R  $\cdot$ Kardeiz] Cardeiz U kardeis Q R \textbf{21} Parcifal] Parzifal U partzifal Q parczifal R \textbf{22} gesigen] versigen Q \textbf{24} dâ] do Q R  $\cdot$ alle] allen U in allen Q in allem R  $\cdot$ sît] zit sit R \textbf{25} clârer] Claren Q (R)  $\cdot$ lac] lagent R \textbf{26} bette] declachen Q (R) \textbf{27} küneginne] kunginnen R \textbf{29} blicte] bliktt R  $\cdot$ in an] iren man Q (R) \textbf{30} enhâte] het R \newline
\end{minipage}
\end{table}
\end{document}
