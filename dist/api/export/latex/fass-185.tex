\documentclass[8pt,a4paper,notitlepage]{article}
\usepackage{fullpage}
\usepackage{ulem}
\usepackage{xltxtra}
\usepackage{datetime}
\renewcommand{\dateseparator}{.}
\dmyyyydate
\usepackage{fancyhdr}
\usepackage{ifthen}
\pagestyle{fancy}
\fancyhf{}
\renewcommand{\headrulewidth}{0pt}
\fancyfoot[L]{\ifthenelse{\value{page}=1}{\today, \currenttime{} Uhr}{}}
\begin{document}
\begin{table}[ht]
\begin{minipage}[t]{0.5\linewidth}
\small
\begin{center}*D
\end{center}
\begin{tabular}{rl}
\textbf{185} & dâ heime in mîn selbes hûs,\\ 
 & dâ wirt gevreut \textbf{vil} selten mûs,\\ 
 & wan \textbf{diu} müese ir spîse steln.\\ 
 & \textbf{die} dorfte niemen vor mir heln.\\ 
5 & i\textbf{ne} vinde ir offenlîche niht.\\ 
 & al ze dicke daz geschiht\\ 
 & mir, Wolfram von Eschenbach,\\ 
 & daz ich dolte \textbf{al} solch gemach.\\ 
 & Mîner klage ist vil vernomen.\\ 
10 & nû \textbf{sol} \textbf{diz} mære wider komen,\\ 
 & wie Pelrapeire stuont jâmers vol.\\ 
 & dâ gap diu diet \textbf{von} vreuden zol.\\ 
 & die helde \textbf{triwen} rîche\\ 
 & lebten kumberlîche.\\ 
15 & ir wâriu manheit daz gebôt.\\ 
 & \textbf{nû} \textbf{lât} erbarmen iuch \textbf{ir} nôt!\\ 
 & \multicolumn{1}{l}{ - - - }\\ 
 & \multicolumn{1}{l}{ - - - }\\ 
 & \textbf{Nû} hœret mêre von den armen,\\ 
20 & \textbf{die} solten iuch erbarmen.\\ 
 & si enpfiengen schemlîche\\ 
 & \textbf{ir} gast ellens rîche.\\ 
 & \textbf{der} dûhtes anders wol sô wert,\\ 
 & daz er niht dorfte hân gegert\\ 
25 & ir herberge, als ez \textbf{in} stuont.\\ 
 & ir grôziu nôt was im unkunt.\\ 
 & \begin{large}M\end{large}an leite einen teppech ûfez gras,\\ 
 & dâ vermûret und geleitet was\\ 
 & durch \textbf{den} schaten ein linde.\\ 
30 & dô entwâpent inz gesinde.\\ 
\end{tabular}
\scriptsize
\line(1,0){75} \newline
D Fr15 \newline
\line(1,0){75} \newline
\textbf{9} \textit{Majuskel} D  \textbf{19} \textit{Majuskel} D  \textbf{27} \textit{Initiale} D  \newline
\line(1,0){75} \newline
\textbf{7} Eschenbach] Esscenbach D \textbf{17} \textit{Die Verse 185.17-18 fehlen} D  \newline
\end{minipage}
\hspace{0.5cm}
\begin{minipage}[t]{0.5\linewidth}
\small
\begin{center}*m
\end{center}
\begin{tabular}{rl}
 & dâ heime in mîn selbes hûs,\\ 
 & d\textit{â} wirt gevr\textit{öuw}et  selten mûs,\\ 
 & wan \textbf{diu} müese ir spîse steln.\\ 
 & \textbf{die} dorfte niemen vo\textit{r} mir heln.\\ 
5 & ich \textbf{en}vinde ir offenlîche niht.\\ 
 & al ze dicke daz geschiht\\ 
 & mir, Wolfra\textit{m}e von \textit{E}schenbach,\\ 
 & daz ich dulte \textbf{al} solich gemach.\\ 
 & mîner klage ist \textbf{ze} vil vernomen.\\ 
10 & nû \textbf{sol} \textbf{diz} mære wider kome\textit{n},\\ 
 & wie Pelraperie stuont jâmers vol.\\ 
 & d\textit{â} gap diu diet \textbf{von} vröuden zol.\\ 
 & die helde \textbf{triuwen} rîche\\ 
 & lebeten kumberlîche.\\ 
15 & ir wâriu manheit daz gebôt.\\ 
 & \textbf{des} \textbf{solte} erbarmen iuch \textbf{ir} nôt.\\ 
 & \multicolumn{1}{l}{ - - - }\\ 
 & \multicolumn{1}{l}{ - - - }\\ 
 & \textbf{nû} hœret mê von den armen,\\ 
20 & \textbf{die} solten iuch erbarmen.\\ 
 & si enpfiengen schamelîche\\ 
 & \textbf{ir} gast ellens rîche.\\ 
 & \textbf{der} dûhte sanders wol sô wert,\\ 
 & daz er niht dorfte hân gegert\\ 
25 & ir herberge, als ez \textbf{in} \textbf{d\textit{ô}} \textit{s}tuont.\\ 
 & ir grôziu nôt was ime unkunt.\\ 
 & man leit ein teppich ûf daz gras,\\ 
 & d\textit{â} vermûret und \dag geleistet\dag  was\\ 
 & durch \textbf{den} schaten ein linde.\\ 
30 & dô entwâpen\textit{t} in daz gesinde.\\ 
\end{tabular}
\scriptsize
\line(1,0){75} \newline
m n o Fr69 \newline
\line(1,0){75} \newline
\newline
\line(1,0){75} \newline
\textbf{1} in] im n \textbf{2} dâ] Do m n  $\cdot$ gevröuwet] gefroget m  $\cdot$ selten] vil selten n o Fr69 \textbf{3} müese] muͦs Fr69  $\cdot$ steln] verstel::: Fr69 \textbf{4} Sin dorft es niemer heln Fr69  $\cdot$ vor] von m \textbf{5} ich envinde ir] Jch enpfinde ir n Jn wirt da Fr69 \textbf{7} Wolframe] wolfrane m wolffram n o wolfram Fr69  $\cdot$ Eschenbach] enschenbach m eschebach n o Eschlibach Fr69 \textbf{8} dulte] dolte n o \textbf{9} ze] so o \textbf{10} diz] dise n das o  $\cdot$ komen] komem m \textbf{11} Pelraperie] pelrapeire m pelrapier n o \textbf{12} dâ] Do m n o \textbf{13} helde] helden m \textbf{14} \textit{Vers 185.14 fehlt} o  \textbf{15} \textit{Verse 185.13 und 185.15 kontrahiert zu:} Die wore manheit das gebot o  \textbf{16} des] Das o \textbf{17} \textit{Die Verse 185.17-18 fehlen} m n o  \textbf{19} den] [dem]: den m \textbf{20} erbarmen] erbarnen m \textbf{23} sanders] sonders o \textbf{24} dorfte] dúrffte n  $\cdot$ gegert] begert n o \textbf{25} dô stuont] do noch stunt m \textbf{28} dâ] Do m n o  $\cdot$ geleistet] verlestet n o \textbf{30} entwâpent] entwoppentet m \newline
\end{minipage}
\end{table}
\newpage
\begin{table}[ht]
\begin{minipage}[t]{0.5\linewidth}
\small
\begin{center}*G
\end{center}
\begin{tabular}{rl}
 & dâ heime in mîn selbes hûs,\\ 
 & dâ wirt gevröut \textbf{vil} selten mûs,\\ 
 & wan \textbf{diu} m\textit{üe}se ir spîse stelen.\\ 
 & \textbf{si}\textbf{ne} dorfte niemen vor mir helen.\\ 
5 & ich vinde ir offenlîchen niht.\\ 
 & al ze dicke daz geschiht\\ 
 & mir, Wolfram von Eschenbach,\\ 
 & daz ich dulte solch gemach.\\ 
 & mîner klage ist vil vernomen.\\ 
10 & nû \textbf{sol} \textbf{diz} mære wider komen,\\ 
 & wie Pelrapeire stuont jâmers vol.\\ 
 & dâ gap diu diet \textbf{von} vröuden zol.\\ 
 & die helde \textbf{jâmers} rîche\\ 
 & lebten kumberlîche.\\ 
15 & ir wâriu manheit daz gebôt.\\ 
 & \textbf{nû} \textbf{solde} erbarmen iuch \textbf{ir} nôt.\\ 
 & ir lîp ist \textbf{nû} benennet pfant,\\ 
 & sine lœse drûz diu hœheste hant.\\ 
 & hœret mêre von den armen,\\ 
20 & \textbf{die} solten iuch erbarmen.\\ 
 & si enpfiengen schamlîche\\ 
 & \textbf{ir} gast ellenes rîche.\\ 
 & \textbf{er} dûhte si anders wol sô wert,\\ 
 & daz er niht dorfte hân gegert\\ 
25 & ir herberge, als ez \textbf{im} stuont.\\ 
 & ir grôziu nôt was im unkuont.\\ 
 & man leit einen tepch ûf dez gras,\\ 
 & dâ vermûret unde geleit was\\ 
 & durch \textbf{den} schate ein linde.\\ 
30 & dô entwâpen\textit{t} inz gesinde.\\ 
\end{tabular}
\scriptsize
\line(1,0){75} \newline
G I O L M Q R Z \newline
\line(1,0){75} \newline
\textbf{7} \textit{Initiale} I  \textbf{9} \textit{Initiale} L R  \textbf{19} \textit{Initiale} Z  \textbf{27} \textit{Initiale} I Q  \newline
\line(1,0){75} \newline
\textbf{1} heime] heimi R  $\cdot$ mîn] myns M meinem Q \textbf{2} dâ] Do Q  $\cdot$ vil] \textit{om.} I Z \textbf{3} wan] \textit{om.} I [Dan]: Wan O  $\cdot$ diu] die I  $\cdot$ müese] muͦse muzzen I mvͯze L (R) Muse M musz Q  $\cdot$ stelen] [selten]: stelen Q \textbf{4} sine dorfte] Sich dorfte O Sich endorfte L Sy bedoͯrfftte R  $\cdot$ niemen] nimmer I  $\cdot$ vor] var O  $\cdot$ helen] verheln I \textbf{5} ich] Jch en L M (Q) (R) (Z) \textbf{6} al ze] Als Q  $\cdot$ geschiht] beschicht R \textbf{7} mir] ir \textit{nachträglich korrigiert zu:} Her I Dir O  $\cdot$ Wolfram] wolffram M R wolfran Q  $\cdot$ Eschenbach] eskenbac I eschenpach O Eschelbach L essinbach M [esce]: eschenbach Q esschenbach Z \textbf{8} dulte] lide M  $\cdot$ solch] al solhen I O alsulich M (Q) (Z) also soͯmlich R \textbf{9} mîner] min I  $\cdot$ vil] gnvch L  $\cdot$ vernomen] vnuernomen I \textbf{11} Pelrapeire] pailrapier I Pelrapair O palrapeẏr Q [pel*]: pelrapeire R  $\cdot$ stuont] stvnde O \textit{om.} R \textbf{12} dâ] Do O Q R  $\cdot$ vröuden] iamer O \textbf{13} helde] holde R  $\cdot$ rîche] [riten]: richen Q \textbf{15} manheit] wipheit O manhet R \textbf{16} erbarmen iuch] ivch erbarmen O (L) (M) \textbf{18} sine] Si O (Q)  $\cdot$ lœse] loͤse danne O  $\cdot$ drûz] daz vs R  $\cdot$ hœheste] hohestev I \textbf{19} den] dem Q \textbf{20} die] si I  $\cdot$ erbarmen] [erbarb]: erbarmen G \textbf{21} si] Die O L Q R Z  $\cdot$ enpfiengen] enpfiengen in Z \textbf{22} ellenes rîche] ellensrichen I (O) (R) eren reichen Q \textbf{23} dûhte] ducktt R  $\cdot$ sô] \textit{om.} R \textbf{24} er] \sout{er} O  $\cdot$ hân] hant O \textbf{25} im] in I O L M R Z \textbf{26} grôziu] grose R \textbf{27} leit] leite M  $\cdot$ einen] yn M  $\cdot$ ûf dez] andas M (Z) aus Q \textbf{28} dâ] Do Q  $\cdot$ geleit] verleitet Z \textbf{29} schate] schatten L (Q) R \textbf{30} dô] Da L M R Z  $\cdot$ entwâpent] entwapende G  $\cdot$ inz] sein Q \newline
\end{minipage}
\hspace{0.5cm}
\begin{minipage}[t]{0.5\linewidth}
\small
\begin{center}*T
\end{center}
\begin{tabular}{rl}
 & dâ heime in mîn selbes hûs,\\ 
 & dâ wirt gevröut \textbf{vil} selten mûs,\\ 
 & wan \textbf{si} müese ir spîse steln.\\ 
 & \textbf{die}\textbf{n} dorf\textit{t}e niemen vor mir heln.\\ 
5 & i\textbf{ne} vindir offenlîche niht.\\ 
 & alze dicke daz geschiht\\ 
 & mir, Wolframe von Eschebach,\\ 
 & daz ich dulte solich gemach.\\ 
 & Mîner klage ist vil vernomen.\\ 
10 & nû \textbf{lât} \textbf{daz} mære wider komen,\\ 
 & wie Peilrapere stuont jâmers vol.\\ 
 & dâ gap diu diet vröuden zol.\\ 
 & die helde \textbf{triuwen} rîche\\ 
 & lebeten kumberlîche.\\ 
15 & ir wâre manheit daz gebôt.\\ 
 & \textbf{nû} \textbf{solte} erbarmen iuch \textbf{die} nôt.\\ 
 & ir lîp \textbf{muoz sîn} benennet pfant,\\ 
 & sine lœse drûz di\textit{u} hœheste hant.\\ 
 & \textbf{Nû} hœret mê von den armen.\\ 
20 & \textbf{si} solten iuch erbarmen.\\ 
 & si enpfiengen schemelîche\\ 
 & \textbf{den} gast ellens rîche.\\ 
 & \textbf{der} dûhte si anders wol sô wert,\\ 
 & daz er niht dorfte hân gegert\\ 
25 & ir herberge, alsez \textbf{in} stuont.\\ 
 & ir grôze nôt was im unkunt.\\ 
 & Man leite ein teppich ûf daz gras,\\ 
 & dâ vermûret unde geleitet was\\ 
 & durch schate ein linde.\\ 
30 & dâ entwâpentin daz gesinde.\\ 
\end{tabular}
\scriptsize
\line(1,0){75} \newline
T U V W \newline
\line(1,0){75} \newline
\textbf{9} \textit{Majuskel} T  \textbf{19} \textit{Initiale} W   $\cdot$ \textit{Majuskel} T  \textbf{27} \textit{Majuskel} T  \newline
\line(1,0){75} \newline
\textbf{1} dâ] Do U W  $\cdot$ mîn selbes] mime selbes U meinem W \textbf{2} dâ] Do U V W \textbf{3} si müese] sie muͦz U die mvͤste V \textbf{4} dien dorfte] dien dorfe T Die dorffte W \textbf{5} ine] Ich W \textbf{6} alze] Also W \textbf{7} Wolframe] wolfram W  $\cdot$ Eschebach] eschibach V eschenbach W \textbf{8} ich dulte] mich duͦchte U \textbf{9} klage] clagen U \textbf{10} lât daz] laz die U lant dis W \textbf{11} Peilrapere] belrepere V pelrapier W \textbf{12} dâ] Do U V W  $\cdot$ vröuden] [*]: von frv́nden V von froͤden W \textbf{13} triuwen] [*]: iomers V \textbf{16} erbarmen iuch] erbarmen iv T eúch erbarmen W  $\cdot$ die] [*]: ir V dise W \textbf{17} benennet] bey namen W \textbf{18} sine] Seinen W  $\cdot$ diu] die T \textbf{19} den] [dem]: den T \textbf{20} iuch] iv T \textbf{21} enpfiengen] entfingin yn U (W) \textbf{22} den] Jr U (W) [J*]: Jrn  V \textbf{25} in] [*]: in V \textbf{26} Sy taten als dicke leúte thuͦnd W \textbf{27} ein] in W \textbf{28} dâ] [Daz]: Do V Das W  $\cdot$ geleitet] [*]: geleitet V \textbf{29} schate] den schaden U den schatte V den schetten W \textbf{30} dâ] Do W \newline
\end{minipage}
\end{table}
\end{document}
