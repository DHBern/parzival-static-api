\documentclass[8pt,a4paper,notitlepage]{article}
\usepackage{fullpage}
\usepackage{ulem}
\usepackage{xltxtra}
\usepackage{datetime}
\renewcommand{\dateseparator}{.}
\dmyyyydate
\usepackage{fancyhdr}
\usepackage{ifthen}
\pagestyle{fancy}
\fancyhf{}
\renewcommand{\headrulewidth}{0pt}
\fancyfoot[L]{\ifthenelse{\value{page}=1}{\today, \currenttime{} Uhr}{}}
\begin{document}
\begin{table}[ht]
\begin{minipage}[t]{0.5\linewidth}
\small
\begin{center}*D
\end{center}
\begin{tabular}{rl}
\textbf{745} & waz prîses bejagte ich danne an dir?\\ 
 & stant stille und sage mir,\\ 
 & werlîcher helt, wer dû sîs.\\ 
 & vür wâr, dû hetes mînen prîs\\ 
5 & behabt, der lange ist mich gewert,\\ 
 & wære dir \textbf{zerbrosten} niht dîn swert.\\ 
 & Nû sî von uns bêden vride,\\ 
 & unz uns \textbf{geruowen} baz diu \textbf{lide}."\\ 
 & si sâzen nider ûfez gras.\\ 
10 & manheit bî \textbf{zuht} \textbf{an} beiden was\\ 
 & unt ir bêder jâr von solher zît,\\ 
 & z\textbf{alt} noch ze \textbf{junc} si bêde ûf strît.\\ 
 & Der heiden zem getouften sprach:\\ 
 & "nû geloube, helt, daz ich \textbf{gesach}\\ 
15 & bî \textbf{mînen zîten} noch nie man,\\ 
 & der baz den \textit{prîs} m\textit{ö}hte hân,\\ 
 & den man in strîte \textbf{sol} bejagen.\\ 
 & nû \textbf{ruoche}, helt, mir \textbf{beidiu} sagen,\\ 
 & dînen namen unt dînen art;\\ 
20 & sô ist wol bewendet her mîn vart."\\ 
 & Dô sprach Herzeloyden sun:\\ 
 & "sol ich daz durch vorhte tuon,\\ 
 & sô\textbf{ne} darf \textbf{es} ni\textit{emen} an mich gern,\\ 
 & sol ichs betwungenlîche \textbf{wern}."\\ 
25 & \textbf{Der heiden} von Thasme\\ 
 & \textbf{sprach}: "\textbf{ich wil} mich nennen ê,\\ 
 & unt lâ daz laster wesen mîn.\\ 
 & ich bin Feirefiz Anschevin,\\ 
 & sô rîche wol, daz mîner hant\\ 
30 & mit zinse dienet manec lant."\\ 
\end{tabular}
\scriptsize
\line(1,0){75} \newline
D \newline
\line(1,0){75} \newline
\textbf{7} \textit{Majuskel} D  \textbf{13} \textit{Majuskel} D  \textbf{21} \textit{Majuskel} D  \textbf{25} \textit{Majuskel} D  \newline
\line(1,0){75} \newline
\textbf{16} prîs] strit D  $\cdot$ möhte] mohte D \textbf{21} Herzeloyden] Herceloyden D \textbf{23} niemen] nimmer D \textbf{25} Thasme] Thasmê D \textbf{28} Anschevin] Anscivin D \newline
\end{minipage}
\hspace{0.5cm}
\begin{minipage}[t]{0.5\linewidth}
\small
\begin{center}*m
\end{center}
\begin{tabular}{rl}
 & waz prîses bejagete \textit{ich denne} an dir?\\ 
 & stant stille und sage mir,\\ 
 & we\textit{r}lîcher helt, wer dû sîs.\\ 
 & vür wâr, dû hettest mînen prîs\\ 
5 & behabt, der lange ist mich gewert,\\ 
 & wær dir \textbf{zerbrochen} niht dîn swert.\\ 
 & nû sî von uns beiden vride,\\ 
 & unz uns \textbf{geruowen} baz diu \textbf{glide}."\\ 
 & si sâzen nider ûf daz gras.\\ 
10 & manheit bî \textbf{zuht} \textbf{in} beiden was\\ 
 & und ir beider jâr vo\textit{n} solicher zît,\\ 
 & zuo \textbf{alt} noch zuo \textbf{jun\textit{c}} si beide ûf strît.\\ 
 & der heiden zuo dem getouften sprach:\\ 
 & "nû gloube, helt, daz ich \textbf{sach}\\ 
15 & bî \textbf{mînen zîten} noch nieman,\\ 
 & der baz den prîs m\textit{ö}hte hân,\\ 
 & den man in strîte \textbf{sol} bejagen.\\ 
 & nû \textbf{geruoche}, helt, mir sagen,\\ 
 & dînen namen und dîn art;\\ 
20 & sô ist wol bewendet her mîn vart."\\ 
 & dô sprach \textbf{der} Hercz\textit{e}loiden sun:\\ 
 & "sol ich daz durch vorhte tuon,\\ 
 & sô darf \textbf{es} niemen an mich gern,\\ 
 & sol ichs betwungenlîchen \textbf{wern}."\\ 
25 & \textbf{dô sprach der} von Thas\textit{m}e:\\ 
 & "\textbf{sô wil ich} mich n\textit{e}nnen ê,\\ 
 & und lâ \textit{d}a\textit{z} laster wesen mîn.\\ 
 & ich bin Fer\textit{e}fi\textit{z} Anschevin,\\ 
 & sô rîch wol, daz mîner hant\\ 
30 & mit zinse dienet manic lant."\\ 
\end{tabular}
\scriptsize
\line(1,0){75} \newline
m n o V V' Fr69 \newline
\line(1,0){75} \newline
\newline
\line(1,0){75} \newline
\textbf{1} \textit{Die Verse 744.28-745.17 fehlen} V'   $\cdot$ ich denne] die m \textbf{3} werlîcher] Welicher m \textbf{4} hettest] hestest o \textbf{5} behabt] Behalt o  $\cdot$ mich] myn o \textbf{8} diu glide] gelide n die lid o (V) \textbf{9} ûf daz] vffens V \textbf{10} zuht] zúchten o  $\cdot$ in] an V \textbf{11} von] vor m \textbf{12} alt] alte n  $\cdot$ junc] juͯnge m \textbf{14} sach] gesach n o V \sout{dir} [sage]: gesach Fr69 \textbf{15} \textit{Versdoppelung (mit Anteil aus Vers 745.14):} Bẏ minen zitten noch nie gesach / Bẏ minen zitten noch nẏeman m  \textbf{16} möhte] mohtte m (o) \textbf{17} sol] solte n \textbf{18} \textit{statt 745.18-19:} Vnd sprach sage mir helt dinen nomen V'   $\cdot$ Nv ruͦche [*]: helt mir beide sagen V \textbf{20} \textit{statt 745.20-24:} Dv bist ein helt vz erkorn / Do sprach der getoufte wol geborn / Sol ich daz von forchte du / So endurfent ir mirs nit mvten zu V'  \textbf{21} Herczeloiden] hertzloiden m hertzoleiden n herczeleide o herzelauden V \textbf{23} darf] endarf V \textbf{24} ichs] ich iches o \textbf{25} Thasme] thasine m n o thassine V' \textbf{26} nennen] ninnen m \textbf{27} daz] wasser m \textbf{28} Ferefiz] ferifir m ferifúr n ferifer o ferevis V V'  $\cdot$ Anschevin] anscevin m ansce vin n ansce win o \textbf{30} dienet manic] dienent manige V' \newline
\end{minipage}
\end{table}
\newpage
\begin{table}[ht]
\begin{minipage}[t]{0.5\linewidth}
\small
\begin{center}*G
\end{center}
\begin{tabular}{rl}
 & \begin{large}W\end{large}az prîses bejagte ich danne an dir?\\ 
 & stant stille unde sage mir,\\ 
 & werlîcher helt, wer dû sîs.\\ 
 & vür wâr, dû heist mînen brîs\\ 
5 & behabet, der lange ist mich gewert,\\ 
 & wære dir \textbf{zerbrochen} niht dîn swert.\\ 
 & nû sî von uns beiden vride,\\ 
 & unze uns \textbf{geruowen} baz diu \textbf{lide}."\\ 
 & si sâzen nider ûf daz gras.\\ 
10 & manheit bî \textbf{zuht} \textbf{an} beiden was\\ 
 & unde ir beider jâr von solher zît,\\ 
 & ze \textbf{junc} noch ze \textbf{alt} si bêde ûf strît.\\ 
 & der heiden zuo dem getouften sprach:\\ 
 & "nû gloube, helt, daz ich \textbf{gesach}\\ 
15 & bî \textbf{mîner zît} noch nie \textbf{den} man,\\ 
 & der baz den brîs möhte hân,\\ 
 & den man in strîte \textbf{sol} bejagen.\\ 
 & nû \textbf{ruoche}, helt, mir \textbf{beidiu} sagen,\\ 
 & dînen namen unde dînen art;\\ 
20 & sô ist wol bewendet her mîn vart."\\ 
 & dô sprac\textit{h} \textit{H}erzeloide\textit{n} sun:\\ 
 & "sol ich daz durch vorhte tuon,\\ 
 & sô\textbf{ne} darf \textbf{es} niemen an mich gern,\\ 
 & sol ich es betwungenlîchen \textbf{wern}."\\ 
25 & \textbf{der heiden} von Tasme\\ 
 & \textbf{sprach}: "\textbf{ich wil} mich nennen ê,\\ 
 & unde lâ daz laster wesen mîn.\\ 
 & ich bin Feirafiz Antschevin,\\ 
 & sô rîche wol, daz mîner hant\\ 
30 & mit zinse dient manic lant."\\ 
\end{tabular}
\scriptsize
\line(1,0){75} \newline
G I L M Z Fr24 Fr48 Fr50 \newline
\line(1,0){75} \newline
\textbf{1} \textit{Initiale} G L Z  \textbf{3} \textit{Initiale} Fr50  \textbf{5} \textit{Initiale} I  \textbf{25} \textit{Initiale} I  \newline
\line(1,0){75} \newline
\textbf{1} danne] \textit{om.} M Fr50 \textbf{5} behabet] Gehat M  $\cdot$ ist mich] mich hat I \textbf{6} zerbrochen] zerbrosten I L \textbf{8} unze uns] vnz Fr50  $\cdot$ geruowen baz] Geruwen wol I baz gervwen Fr50 \textbf{10} bî] mit Z  $\cdot$ zuht] zuhten I (L) (M)  $\cdot$ an] an in I (M) bi Z \textbf{11} beider] \textit{om.} L \textbf{12} si] \textit{om.} Fr50  $\cdot$ bêde] beider M  $\cdot$ ûf strît] an strite Fr50 \textbf{14} gloube] glovbet L volge Fr50  $\cdot$ gesach] sach I \textbf{15} noch] \textit{om.} I Fr50 \textbf{16} möhte] mohte I (L) (M) Z (Fr24) (Fr48) Fr50 \textbf{18} ruoche helt] ruͤchet helt I geruch helt M [rvchelt]: rvch helt Z  $\cdot$ sagen] ze sagen Fr48 \textbf{19} dînen art] dine art L (M) (Z) (Fr48) (Fr50) \textbf{20} mîn] mit M \textbf{21} dô] Da M  $\cdot$ sprach] sprah der G  $\cdot$ Herzeloiden] herzeloyde G herzenlauden I Hertzelovden L herczelodin M herzenlovden Z hertzelouden Fr48 herzelovden Fr50 \textbf{23} mich] ivch Fr50  $\cdot$ gern] begern Fr48 \textbf{24} es] sin I ichz L (M) Z (Fr48) o\textit{m. } Fr50 \textbf{25} heiden] heide M  $\cdot$ von] sprach von I  $\cdot$ Tasme] Tasine L Thasme Fr48 \textbf{26} sprach ich wil] so wil ich I \textbf{27} daz laster] daster L \textbf{28} Feirafiz] ferefis L ferefisz M ferefiz Z ferrefiez Fr48 ferrefiz Fr50  $\cdot$ Antschevin] anschevin G antscheuin I Anschovin L ansevin M anshevin Z (Fr48) anschivin Fr50 \textbf{29} mîner hant] min lant Fr50 \textbf{30} dient manic] dienent manige Z (Fr48)  $\cdot$ lant] hant Fr50 \newline
\end{minipage}
\hspace{0.5cm}
\begin{minipage}[t]{0.5\linewidth}
\small
\begin{center}*T
\end{center}
\begin{tabular}{rl}
 & waz prîses bejagete ich dan an dir?\\ 
 & stant stille und sage mir,\\ 
 & werlîcher helt, wer dû sîs.\\ 
 & vür wâr, dû hetest mînen prîs\\ 
5 & behabet, der lange ist mich gewert,\\ 
 & wære dir \textbf{zerbrochen} niht dîn swert.\\ 
 & nû sî von uns beiden vride,\\ 
 & unz uns \textbf{geruowent} baz diu \textbf{lide}."\\ 
 & \begin{large}S\end{large}i sâzen nider ûf daz gras.\\ 
10 & manheit bî \textbf{zühten} \textbf{an in} beiden was\\ 
 & und ir beider jâr von solicher zît,\\ 
 & zuo \textbf{junc} noch zuo \textbf{alt} si beide \textit{ûf} strît.\\ 
 & der heiden zuo dem getouften sprach:\\ 
 & "nû geloube, helt, daz ich \textbf{gesach}\\ 
15 & bî \textbf{mîner zît} noch \textit{nie} \textbf{den} man,\\ 
 & der baz den prîs m\textit{ö}hte hân,\\ 
 & den man in strîte \textbf{solte} bejagen.\\ 
 & nû \textbf{geruoche}, helt, mir \textbf{beidiu} sagen,\\ 
 & dînen namen und dînen art;\\ 
20 & sô ist wol bewendet her mîn vart."\\ 
 & dô sprach Herzeloyden \textit{sun}:\\ 
 & "sol ich daz durch vorhte tuon,\\ 
 & sô\textbf{ne} \textit{darf} nieman an mich gern,\\ 
 & sol ich es betwungenlîche \textbf{gewern}."\\ 
25 & \textbf{der heiden} von Tasme\\ 
 & \textbf{sprach}: "\textbf{ich wil} mich nennen ê,\\ 
 & und lâz daz laster wesen mîn.\\ 
 & ich bin Ferefis \textbf{von} Anschevin,\\ 
 & sô rîche wol, daz mîner hant\\ 
30 & mit zinse dienet manec lant."\\ 
\end{tabular}
\scriptsize
\line(1,0){75} \newline
U W Q R \newline
\line(1,0){75} \newline
\textbf{9} \textit{Initiale} U W  \textbf{21} \textit{Initiale} R  \newline
\line(1,0){75} \newline
\textbf{6} zerbrochen niht] nicht zerbrochen W \textbf{8} unz] mit U  $\cdot$ geruowent] geruͦgen W gerúwen Q (R) \textbf{10} zühten] zucht W Q R  $\cdot$ an in] in W an Q R \textbf{11} ir] \textit{om.} Q \textbf{12} ûf] \textit{om.} U \textbf{15} nie] \textit{om.} U \textbf{16} möhte] mochte U Q \textbf{17} solte] sol W Q so R \textbf{18} geruoche] ruͦche W (Q) (R)  $\cdot$ helt mir] mir helt Q mir held mir R \textbf{19} dînen art] dein art Q (R) \textbf{20} bewendet her] geendet her W bewennet do er Q \textbf{21} Herzeloyden] herzeleyden U hertzeloyden W hertzeloúden Q herczelauden R  $\cdot$ sun] kint U \textbf{23} sône darf] So in ist U So endarff es W (Q) So darff es R  $\cdot$ gern] gerne U begern R \textbf{24} gewern] gewerne U wern W (Q) R \textbf{25} Tasme] thasme W Q Thasine R \textbf{26} wil] sol W \textbf{28} bin] bins R  $\cdot$ Ferefis] ferafis W feirefiz Q feirefis R  $\cdot$ von] \textit{om.} W Q  $\cdot$ Anschevin] antscheuein W anshevin Q \textbf{30} zinse] Reise R \newline
\end{minipage}
\end{table}
\end{document}
