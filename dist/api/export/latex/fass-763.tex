\documentclass[8pt,a4paper,notitlepage]{article}
\usepackage{fullpage}
\usepackage{ulem}
\usepackage{xltxtra}
\usepackage{datetime}
\renewcommand{\dateseparator}{.}
\dmyyyydate
\usepackage{fancyhdr}
\usepackage{ifthen}
\pagestyle{fancy}
\fancyhf{}
\renewcommand{\headrulewidth}{0pt}
\fancyfoot[L]{\ifthenelse{\value{page}=1}{\today, \currenttime{} Uhr}{}}
\begin{document}
\begin{table}[ht]
\begin{minipage}[t]{0.5\linewidth}
\small
\begin{center}*D
\end{center}
\begin{tabular}{rl}
\textbf{763} & \begin{large}M\end{large}it der küneginne Arniven az.\\ 
 & \textbf{ir enwederiu} dâ niht vergaz\\ 
 & \textbf{ir} gesellecheite\\ 
 & \textbf{wâren si} ein ander \textbf{vil} bereite.\\ 
5 & bî Gawane saz sîn ane,\\ 
 & Orgeluse ûzerhalp her dane.\\ 
 & dâ erzeigete diu reht unzuht\\ 
 & \textbf{von} dem ringe ir snellen vluht.\\ 
 & Man truoc bescheidenlîche dar\\ 
10 & den rîtern unt \textbf{den} vrouwen \textbf{gar}\\ 
 & ir spîse zühteclîche.\\ 
 & Feirefiz, der rîche,\\ 
 & sprach ze Parzivale, dem bruoder sîn:\\ 
 & "Jupiter \textbf{die} reise mîn\\ 
15 & mir ze sælden het \textbf{erdâht},\\ 
 & daz mich sîn helfe her \textbf{hât} brâht,\\ 
 & \textbf{dâ} ich mîne werden mâge sihe.\\ 
 & von rehter schult ich prîses gihe\\ 
 & mînem vater, den ich hân verlorn;\\ 
20 & \textbf{der} was ûz rehtem prîse \textbf{erborn}."\\ 
 & \textbf{Parzival} sprach: "ir sult noch sehen\\ 
 & liute, den ir prîses müezet jehen,\\ 
 & bî Artuse, dem houbtman,\\ 
 & manegen rîter \textbf{manlîch} getân.\\ 
25 & swie schiere \textbf{diz} ezzen nû \textbf{zergêt},\\ 
 & unlangez dâ nâch gestêt,\\ 
 & unz ir die werden seht komen,\\ 
 & an den vil prîses ist vernomen.\\ 
 & swaz tavelrunde \textbf{kreft ist} bî,\\ 
30 & der \textbf{en}\textbf{sitzet} hie \textbf{niwan} rîter drî:\\ 
\end{tabular}
\scriptsize
\line(1,0){75} \newline
D Fr12 \newline
\line(1,0){75} \newline
\textbf{1} \textit{Initiale} D  \textbf{9} \textit{Majuskel} D  \newline
\line(1,0){75} \newline
\textbf{1} küneginne] herzogin Fr12  $\cdot$ Arniven] Arnîven D ar:::en Fr12 \textbf{4} vil] \textit{om.} Fr12 \textbf{5} Gawane] :::ne Fr12 \textbf{6} Orgeluse] Orgelvͦse D :::e Fr12 \textbf{8} snellen] s:::e Fr12 \textbf{13} Parzivale] Parcifale D :::al Fr12 \textbf{14} Jupiter] :::r Fr12 \textbf{15} het erdâht] daz mich bedaht Fr12 \textbf{21} Parzival] Parcifal D \newline
\end{minipage}
\hspace{0.5cm}
\begin{minipage}[t]{0.5\linewidth}
\small
\begin{center}*m
\end{center}
\begin{tabular}{rl}
 & mit der künigîn Ar\textit{niv}e az.\\ 
 & \textbf{ir iewederiu} dô niht vergaz,\\ 
 & \textbf{si wâren} gesellicheit\\ 
 & ein ander \textbf{vil} bereit.\\ 
5 & bî Gawan saz sîn ane,\\ 
 & Urgeluse ûzerhalp herdane\\ 
 & dô e\textit{r}z\textit{ei}gte diu reht unzuht\\ 
 & \textbf{vor} dem ringe ir snellen vluht.\\ 
 & man truoc bescheidenlîchen dar\\ 
10 & den rittern und \textbf{den} vrowen \textbf{gar}\\ 
 & ir spîse zühteclîch.\\ 
 & \textit{F}erefiz, der rîch,\\ 
 & sprach ze Parcifal, dem bruoder sîn:\\ 
 & "Jupiter \textbf{der} reise mîn\\ 
15 & mir ze sæl\textit{d}en het \textbf{erdâht},\\ 
 & daz mich sîn helfe her \textbf{het} brâht,\\ 
 & \textbf{d\textit{â}} ich mîn werden mâge sihe.\\ 
 & von rehter schult ich prîses gihe\\ 
 & mîne\textit{m} vater, den ich hân verlorn;\\ 
20 & \textbf{er} was ûz rehtem prîs \textbf{erkorn}."\\ 
 & \textbf{der Waleis} sprach: "ir solt noch sehen\\ 
 & liute, den ir prîses müezt jehen,\\ 
 & bî Artuse, dem houbtman,\\ 
 & manigen ritter \textbf{manlîch} getân.\\ 
25 & wie schier \textbf{diz} ezzen nû \textbf{zergât},\\ 
 & unlangez dâ nâch gestât,\\ 
 & unz ir die werden sehet komen,\\ 
 & an den vil prîses ist vernomen.\\ 
 & waz tavelrunde \textbf{ist krefte} bî,\\ 
30 & der \textbf{sitzent} hie \textbf{nû} ritter drî:\\ 
\end{tabular}
\scriptsize
\line(1,0){75} \newline
m n o V V' W \newline
\line(1,0){75} \newline
\textbf{21} \textit{Initiale} W  \newline
\line(1,0){75} \newline
\textbf{1} \textit{Die Verse 762.22-764.27 fehlen} o   $\cdot$ \textit{Die Verse 762.29-764.22 fehlen} V'   $\cdot$ Arnive] aruͯne m arniwe n arniven V arniue W  $\cdot$ az] saß W \textbf{2} iewederiu] [enweder*]: enwederú V \textbf{3} wâren] enwere V \textbf{5} Gawan] Gawane V \textbf{6} Urgeluse] Orgaluse V Orgeluse W \textbf{7} erzeigte] enzogtte m erzoͮegete V  $\cdot$ diu reht] \textit{om.} V \textbf{8} vor] Von V  $\cdot$ ir snellen] [iren snellen]: ire snelle V ir schnelle W \textbf{10} den rittern] \textit{om.} W \textbf{11} zühteclîch] gar zúchticleiche W \textbf{12} Ferefiz] Jferefis m Ferrefis n Fereuis V Ferafis W \textbf{13} Parcifal] parzefal V partzifal W \textbf{14} Jupiter] Juppiter n (V) Iupiter W  $\cdot$ der] die V W \textbf{15} ze sælden] zeselben m  $\cdot$ het] hat n W \textbf{17} dâ] Do m n Daz V (W) \textbf{18} prîses] preiß W \textbf{19} mînem] Minen m Mein W \textbf{20} erkorn] erborn n W \textbf{21} Waleis] walleis V waleiß W  $\cdot$ noch] \textit{om.} W \textbf{22} prîses] pris m (n) (W) \textbf{23} Artuse] artus W \textbf{25} wie] Swie V  $\cdot$ diz ezzen] das W \textbf{26} gestât] bestat W \textbf{27} werden] werdent W  $\cdot$ sehet] sehen n W \textbf{29} waz] Swaz V  $\cdot$ tavelrunde] [taluelrund*]: taluelrunder V tauelrunder W  $\cdot$ ist krefte] krafft ist W \textbf{30} nû] nuwen V \newline
\end{minipage}
\end{table}
\newpage
\begin{table}[ht]
\begin{minipage}[t]{0.5\linewidth}
\small
\begin{center}*G
\end{center}
\begin{tabular}{rl}
 & \begin{large}M\end{large}it der künigîn Arnive az.\\ 
 & \textbf{Itonie} dô niht vergaz\\ 
 & \textbf{ir} gesellecheit\\ 
 & \textbf{wâren si} ein ander bereit.\\ 
5 & bî Gawan sa\textit{z} \textit{s}în ane,\\ 
 & Orgeluse ûzerhalp her dane.\\ 
 & dô erzeigte diu rehte unzuht\\ 
 & \textbf{von} dem ringe ir snelle vluht.\\ 
 & man truoc bescheidenlîche dar\\ 
10 & den rîtern unde \textbf{der} vrouwen \textbf{schar}\\ 
 & ir spîse zühteclîche.\\ 
 & Feirafiz, der rîche,\\ 
 & sprach ze Parcival, dem bruoder sîn:\\ 
 & "Juppiter \textbf{die} reise mîn\\ 
15 & mir ze sælden hete \textbf{\textit{ged}âht},\\ 
 & daz mich sîn helfe her \textbf{hât} brâht,\\ 
 & \textbf{daz} ich mîne werde mâge sihe.\\ 
 & von rehter schult ich brîses gihe\\ 
 & mînem vater, den ich hân verlorn;\\ 
20 & \textbf{der} was ûz rehtem brîs \textbf{erkorn}."\\ 
 & \textbf{der Waleis} sprach: "ir sült noch sehen\\ 
 & liute, den ir brîses muozet jehen,\\ 
 & bî Artus, dem houbetman,\\ 
 & manigen rîter \textbf{manlîch} getân.\\ 
25 & swie schiere \textbf{ditze} ezzen nû \textbf{ergêt},\\ 
 & unlange ez dâr nâch gestêt,\\ 
 & unze ir die werden sehet komen,\\ 
 & an den vil prîses ist vernomen.\\ 
 & swaz tavelrunder \textbf{kraft ist} bî,\\ 
30 & der \textbf{sitzent} hie \textbf{niuwan} rîter drî."\\ 
\end{tabular}
\scriptsize
\line(1,0){75} \newline
G I L M Z Fr45 \newline
\line(1,0){75} \newline
\textbf{1} \textit{Initiale} G L Z  \textbf{5} \textit{Initiale} I  \textbf{21} \textit{Initiale} I  \newline
\line(1,0){75} \newline
\textbf{1} Arnive] arniue I Arniven L (M) (Z) arnẏuen Fr45 \textbf{2} Itonie] Jtonie G I L Jthonie M Jr ietwedere Z ir newedere Fr45  $\cdot$ dô] do da I \textit{om.} L Fr45 da M Z  $\cdot$ niht] \textit{om.} Z \textbf{3} ir] ir vil suzen I \textbf{4} bereit] vil bereit I (M) Z \textbf{5} Gawan] Gawane M  $\cdot$ saz sîn] saz si sin G \textbf{6} Orgeluse] Orgillv̂sie G Orguluse I Orgelýse L orgiliuse Fr45  $\cdot$ her dane] hindane Fr45 \textbf{7} dô] Da M Z Der Fr45  $\cdot$ erzeigte] erzovgte L tzeigete Fr45 \textbf{8} ir] ein I  $\cdot$ snelle] snellen Fr45 \textbf{10} den] Der M \textbf{11} spîse] spiset Fr45  $\cdot$ zühteclîche] vil zuhtechliche I \textbf{12} Feirafiz] Feirefiz G Z Ferefiz L Feirefisz M feẏrafẏz Fr45  $\cdot$ der] der haiden I \textbf{13} Parcival] parcifal G Z parzifal I M \textit{om.} L persciual Fr45 \textbf{14} Juppiter] Jupiter L M Z (Fr45)  $\cdot$ die] der I (L) (Z) (Fr45) \textbf{15} mir] Mit Fr45  $\cdot$ ze] zegrozzen I \textit{om.} M Fr45  $\cdot$ hete] hat L Z  $\cdot$ gedâht] braht G erdacht L (M) (Z) (Fr45) \textbf{16} helfe] \textit{om.} L \textbf{17} mîne werde] die minen werden I mine werden Fr45 \textbf{19} mînem] Min L  $\cdot$ hân] ha M \textbf{20} erkorn] irborn Fr45 \textbf{21} Waleis] waleẏs Fr45 \textbf{22} ir] ich I  $\cdot$ brîses] prises Genuͤc I  $\cdot$ muozet] muͤz I moͮgt Fr45 \textbf{23} Artus] Artuse L (M) \textbf{25} swie] Wie L (M) Fr45  $\cdot$ ditze] das M (Fr45)  $\cdot$ nû ergêt] ergen Z \textbf{26} dâr] danne dar M (Fr45)  $\cdot$ gestêt] stet L \textbf{29} swaz] Waz L (M) (Fr45)  $\cdot$ tavelrunder] Tavelrvnde L (Z) \textbf{30} der] die I dern Fr45  $\cdot$ sitzent] ensitzent L sizzet Fr45  $\cdot$ niuwan] wan M nicht me wen Fr45  $\cdot$ rîter] \textit{om.} L M Z Fr45 \newline
\end{minipage}
\hspace{0.5cm}
\begin{minipage}[t]{0.5\linewidth}
\small
\begin{center}*T
\end{center}
\begin{tabular}{rl}
 & mit der künegîn Arnyven az.\\ 
 & \textbf{ir ietwederiu} dô niht vergaz\\ 
 & \textbf{ir alten} gesellecheit\\ 
 & \textbf{wâren si} ein ander \textbf{vil} bereit.\\ 
5 & bî Gawane saz sîn an,\\ 
 & Orgeluse ûzerhalp herdan.\\ 
 & dô erzeigete diu rehte unzuht\\ 
 & \textbf{von} dem ringe ir snelle vluht.\\ 
 & \begin{large}M\end{large}an truoc bescheidenlîche dar\\ 
10 & den rîtern und \textbf{den} vrouwen \textbf{gar}\\ 
 & ir spîse zühteclîche.\\ 
 & Ferefis, der rîche,\\ 
 & sprach zuo Parcifal, dem bruoder sîn:\\ 
 & "Jupiter \textbf{die} reise mîn\\ 
15 & mir zuo sælden hete \textbf{erdâht},\\ 
 & daz mich sîn \textit{helfe} her \textbf{hete} brâht,\\ 
 & \textbf{daz} ich mîne werde mâge sihe.\\ 
 & von rehter schult ich prîses gihe\\ 
 & mîne\textit{m} vater, den ich hân verlorn;\\ 
20 & \textbf{der} was ûz rehtem prîse \textbf{erkorn}."\\ 
 & \textbf{der Waleis} sprach: "ir sult noch sehen\\ 
 & liute, den ir prîses müezet jehen,\\ 
 & bî Artuse, dem ho\textit{ubetm}an,\\ 
 & manegen rîter \textbf{wol}getân.\\ 
25 & wie schiere \textbf{daz} ezzen nû \textbf{ergêt},\\ 
 & unlange ez \textbf{dan} dâ nâch gestêt,\\ 
 & unz \textbf{daz} ir die werden sehet komen,\\ 
 & an den vil prîses ist vernomen.\\ 
 & waz tavelrunder \textbf{kraft ist} bî,\\ 
30 & der \textbf{en}\textbf{sitzet} hie \textbf{niht wan} rîter drî:\\ 
\end{tabular}
\scriptsize
\line(1,0){75} \newline
U Q R \newline
\line(1,0){75} \newline
\textbf{9} \textit{Initiale} U  \newline
\line(1,0){75} \newline
\textbf{1} Arnyven] arniue Q \textbf{2} ir] \textit{om.} Q \textbf{3} gesellecheit] gesellenheit R \textbf{5} Gawane] gawan Q Gawin R \textbf{6} Orgeluse] Orgelusze Q Orguluse R  $\cdot$ herdan] erdann Q dran R \textbf{8} vluht] [frucht]: flucht R \textbf{12} Ferefis] feirefisz Q Feirefis R \textbf{13} Parcifal] partzifal Q parczifal R \textbf{16} helfe] \textit{om.} U  $\cdot$ her] \textit{om.} Q \textbf{17} daz] Da R  $\cdot$ werde] rechte R \textbf{18} prîses] \textit{om.} R \textbf{19} mînem] Minen U (R) \textbf{20} rehtem] rechtten R  $\cdot$ erkorn] erboren Q (R) \textbf{22} müezet] muͯszen R \textbf{23} Artuse] artus Q (R)  $\cdot$ houbetman] honpinan U \textbf{24} manegen] Menig R  $\cdot$ wolgetân] menlich gethan Q \textbf{25} nû] \textit{om.} R \textbf{26} dan dâ nâch] denne R  $\cdot$ gestêt] stet Q \textbf{27} unz] Mit U  $\cdot$ daz] \textit{om.} Q R \textbf{28} prîses] pris R  $\cdot$ vernomen] veromen R \textbf{29} waz] Sus R \textbf{30} ensitzet] siczen R \newline
\end{minipage}
\end{table}
\end{document}
