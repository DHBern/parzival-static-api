\documentclass[8pt,a4paper,notitlepage]{article}
\usepackage{fullpage}
\usepackage{ulem}
\usepackage{xltxtra}
\usepackage{datetime}
\renewcommand{\dateseparator}{.}
\dmyyyydate
\usepackage{fancyhdr}
\usepackage{ifthen}
\pagestyle{fancy}
\fancyhf{}
\renewcommand{\headrulewidth}{0pt}
\fancyfoot[L]{\ifthenelse{\value{page}=1}{\today, \currenttime{} Uhr}{}}
\begin{document}
\begin{table}[ht]
\begin{minipage}[t]{0.5\linewidth}
\small
\begin{center}*D
\end{center}
\begin{tabular}{rl}
\textbf{570} & \textbf{\begin{large}D\end{large}er} was \textbf{vreissam} getân.\\ 
 & von \textbf{visches} \textbf{hiute} truoc er an\\ 
 & ein surkôt unt ein bônît\\ 
 & unt des selben zwô hosen wît.\\ 
5 & Einen kolben er in der hende truoc,\\ 
 & des \textbf{kiule} grœzer denne ein kruoc\\ 
 & \textbf{was}. er gienc gein Gawane her.\\ 
 & daz \textbf{en}was \textbf{doch} ninder sîn \textbf{ger},\\ 
 & wande in sînes kumens dâ verdrôz.\\ 
10 & Gawan \textbf{dâhte}: "dirre ist blôz;\\ 
 & sîn wer ist gein mir harte laz."\\ 
 & er rihte sich ûf unde saz,\\ 
 & als ob \textbf{in} \textbf{swære ninder lit}.\\ 
 & Jener trat hinder einen trit,\\ 
15 & als ob er wolde entwîchen,\\ 
 & und sprach doch zornlîchen:\\ 
 & "ir \textbf{en}durfet \textbf{mich} entsitzen niht;\\ 
 & ich \textbf{vüege} aber wol, daz \textbf{iu} geschiht,\\ 
 & dâ von ir den lîp ze pfande gebt.\\ 
20 & von \textbf{des} tiuvels \textbf{kreften} ir noch lebt;\\ 
 & sol iuch der hie hân ernert,\\ 
 & ir sît doch sterbens unerwert.\\ 
 & \textbf{des} bringe ich iuch wol innen,\\ 
 & als ich nû scheide hinnen."\\ 
25 & Der vilân trat wider în.\\ 
 & Gawan mit dem swerte sîn\\ 
 & vome schilde sluoc die zeine.\\ 
 & die pfîle algemeine\\ 
 & wâren hin durch gedrungen,\\ 
30 & daz si in den ringen klungen.\\ 
\end{tabular}
\scriptsize
\line(1,0){75} \newline
D \newline
\line(1,0){75} \newline
\textbf{1} \textit{Initiale} D  \textbf{5} \textit{Majuskel} D  \textbf{14} \textit{Majuskel} D  \textbf{25} \textit{Majuskel} D  \newline
\line(1,0){75} \newline
\textbf{1} Der] ÷er \textit{nachträglich korrigiert zu:} der D \newline
\end{minipage}
\hspace{0.5cm}
\begin{minipage}[t]{0.5\linewidth}
\small
\begin{center}*m
\end{center}
\begin{tabular}{rl}
 & \textbf{der} was \textbf{vreislîch} getân.\\ 
 & von \textbf{vische} \textbf{hiuten} truoc er an\\ 
 & ein surk\textit{ô}t und ein bônît\\ 
 & und des selben z\textit{wô} hosen wît.\\ 
5 & einen kolben er in der hende truoc,\\ 
 & des \textbf{kiule} grœzer dan ein kruoc.\\ 
 & er gienc gegen Gawane her.\\ 
 & daz was ninder sîn \textbf{beger},\\ 
 & wan in sînes komens d\textit{â} verdrôz.\\ 
10 & Gawan \textbf{gedâhte}: "dirre ist blôz;\\ 
 & sîn wer ist gegen mir harte laz."\\ 
 & er richte sich ûf und saz,\\ 
 & als ob \textbf{im} \textbf{nindert wære} \dag leit\dag .\\ 
 & jener trat hinder einen trit,\\ 
15 & als ob er wolt entwîchen,\\ 
 & und sprach doch zorneclîchen:\\ 
 & "ir durft \textbf{mich} entsitzen niht;\\ 
 & ich \textbf{vüege} aber wol, daz \textbf{iu} geschiht,\\ 
 & dâ von ir den lîp ze pfande gebt.\\ 
20 & von \textbf{des} tiuvels \textbf{\textit{k}reften} ir noch lebt;\\ 
 & sol iuch der hie hân ernert,\\ 
 & ir sît doch sterbens unerwert:\\ 
 & \textbf{den} bringe ich iuch wol innen,\\ 
 & als ich nû scheide hinnen."\\ 
25 & der vilân trat wider în.\\ 
 & Gawan mit dem swerte sîn\\ 
 & von dem schilt sluoc die \dag zuo ein\dag ,\\ 
 & \textbf{wan} die pfîl algemein\\ 
 & wâren hin durch gedrungen,\\ 
30 & daz si in den ringen klungen.\\ 
\end{tabular}
\scriptsize
\line(1,0){75} \newline
m n o \newline
\line(1,0){75} \newline
\newline
\line(1,0){75} \newline
\textbf{1} der] Des o \textbf{2} an] dan o \textbf{3} surkôt] sorkat m o sarkot n \textbf{4} zwô] ze m \textbf{6} grœzer] grosses o \textbf{8} ninder] nẏergent doch n nuͯdert doch o  $\cdot$ beger] ger n o \textbf{9} dâ] do m n o \textbf{13} leit] lit n o \textbf{17} durft] bedúrfft n \textbf{20} kreften] reftten m \textbf{27} dem] \textit{om.} n \textbf{28} algemein] alle gemein n \textbf{29} gedrungen] gedringen n \textbf{30} ringen] ringel o  $\cdot$ klungen] clingen n \newline
\end{minipage}
\end{table}
\newpage
\begin{table}[ht]
\begin{minipage}[t]{0.5\linewidth}
\small
\begin{center}*G
\end{center}
\begin{tabular}{rl}
 & \textbf{\begin{large}E\end{large}r} was \textbf{vreislîch} getân.\\ 
 & von \textbf{vischen} \textbf{hiute} truo\textit{c} er an\\ 
 & ein surkôt unde ein bô\textit{n}ît\\ 
 & unde des selben zwuo hosen wît.\\ 
5 & einen kolben \textit{er} in der hende truoc,\\ 
 & des \textbf{kiule} grœzer danne ein kruoc.\\ 
 & er gienc gein Gawan her.\\ 
 & daz was \textbf{doch} ninder sîn \textbf{ger},\\ 
 & wande in sînes komens dar verdrôz.\\ 
10 & Gawan \textbf{dâhte}: "dirre ist blôz;\\ 
 & sîn wer ist gein mir harte laz."\\ 
 & er rihte sich ûf unde saz,\\ 
 & als ob \textbf{in} \textbf{swære ninder lit}.\\ 
 & jener trat hinder einen trit,\\ 
15 & als ob er wolde entwîchen,\\ 
 & unde sprach doch zornlîchen:\\ 
 & "ir\textbf{n} durfet \textbf{mich} entsitzen niht;\\ 
 & ich \textbf{vüege} aber wol, daz \textbf{iu} geschiht,\\ 
 & dâ von ir den lîp ze pfande gebet.\\ 
20 & von tiuvels \textbf{kreften} ir noch lebet;\\ 
 & sol iuch der hie hân ernert,\\ 
 & ir sît doch sterbens unerwert.\\ 
 & \textbf{des} bringe ich iuch wol innen,\\ 
 & als ich nû scheide hinnen."\\ 
25 & der vilân trat wider în.\\ 
 & Gawan mit dem swerte sîn\\ 
 & vome schilte sluoc die zeine.\\ 
 & die pfîle algemeine\\ 
 & wâren hin durch gedrungen,\\ 
30 & daz si in den ringen klungen.\\ 
\end{tabular}
\scriptsize
\line(1,0){75} \newline
G I L M Z Fr23 \newline
\line(1,0){75} \newline
\textbf{1} \textit{Initiale} G L Z  \textbf{5} \textit{Initiale} I  \textbf{25} \textit{Initiale} I  \newline
\line(1,0){75} \newline
\textbf{1} Er] Der L \textbf{2} vischen hiute] vishuten I visches huͯten L visches huͤt M fisches hvͤte Z  $\cdot$ truoc] truͦge G \textbf{3} Eyne surkot vnde eyne ponit M  $\cdot$ ein bônît] ein boit G \textbf{5} kolben] clolben Fr23  $\cdot$ er in der hende] si in der hende G inder hende er Fr23 \textbf{6} kiule] chugel I  $\cdot$ grœzer] was grozzer I (L) (Fr23) \textbf{7} Gawan] Gawane L (M) (Z) \textbf{8} was] en was I (M) (Fr23) en L  $\cdot$ ninder] nirgen M \textbf{9} sînes] sin Z Fr23  $\cdot$ komens] chunbers I  $\cdot$ dar] Gar I  $\cdot$ verdrôz] bedroz Fr23 \textbf{10} dâhte] gedahte I \textbf{11} gein mir harte] harte gein mir Z \textbf{13} ob] >ob< G L uff M  $\cdot$ in] im L Z Fr23  $\cdot$ swære] swern I swͦr Fr23  $\cdot$ lit] lide Fr23 \textbf{14} hinder] hinder sich I \textbf{15} ob] uff M \textbf{16} doch] \textit{om.} M \textbf{17} irn] Jr M Fr23 \textbf{18} wol] \textit{om.} M Fr23 \textbf{20} von] von des I (M)  $\cdot$ kreften] keften Z  $\cdot$ noch] nol Fr23 \textbf{21} iuch der] der evch Z \textbf{22} unerwert] vn eruert M \textbf{25} wider] hinder L \textbf{27} zeine] steine I \newline
\end{minipage}
\hspace{0.5cm}
\begin{minipage}[t]{0.5\linewidth}
\small
\begin{center}*T
\end{center}
\begin{tabular}{rl}
 & \textbf{der} was \textbf{vreislîch} getân.\\ 
 & von \textbf{vische} \textbf{hiuten} truoc er an.\\ 
 & \multicolumn{1}{l}{ - - - }\\ 
 & \multicolumn{1}{l}{ - - - }\\ 
5 & einen kolben er in der hende truoc,\\ 
 & Des \textbf{kolbe} \textbf{was} grœzer danne ein kruoc.\\ 
 & er gie gein Gawane her.\\ 
 & daz \textbf{en}was \textbf{dannoch} niender sîn \textbf{ger},\\ 
 & wand in sînes komens dâ verdrôz.\\ 
10 & Gawan \textbf{dâhte}: "dirre ist blôz;\\ 
 & sîn wer ist gein mir harte laz."\\ 
 & er rihte sich ûf unde saz,\\ 
 & als ob \textbf{im} \textbf{swære niender lit}.\\ 
 & Jener trat hinder \textbf{sich} einen trit,\\ 
15 & als ob er wolte entwîchen,\\ 
 & unde sprach doch zornlîchen:\\ 
 & "ir\textbf{n} durfet \textbf{mîn} entsitzen niht;\\ 
 & ich \textbf{gevüege} aber wol, daz geschiht,\\ 
 & dâ von ir den lîp ze pfande gebet.\\ 
20 & von \textbf{des} tiuvels \textbf{krefte} ir noch lebet;\\ 
 & sol iuch der hie hân ernert,\\ 
 & ir sît doch sterbens unerwert.\\ 
 & \textbf{des} bring ich iuch wol innen,\\ 
 & als ich nû scheide hinnen."\\ 
25 & \textit{\begin{large}D\end{large}}er vilân trat wider în.\\ 
 & Gawan mit dem swerte sîn\\ 
 & von dem schilte sluoc die zeine.\\ 
 & die pfîle algemeine\\ 
 & w\textit{âren} hin durch gedrungen,\\ 
30 & daz sin den ringen klungen.\\ 
\end{tabular}
\scriptsize
\line(1,0){75} \newline
T U V W Q R Fr39 \newline
\line(1,0){75} \newline
\textbf{1} \textit{Initiale} Q Fr39   $\cdot$ \textit{Capitulumzeichen} R  \textbf{6} \textit{Majuskel} T  \textbf{14} \textit{Majuskel} T  \textbf{25} \textit{Überschrift:} Hie strait her gawan mit dem loͤwen in der auenteúre W   $\cdot$ \textit{Initiale} T W  \newline
\line(1,0){75} \newline
\textbf{1} \textit{Die Verse 553.1-599.30 fehlen} U  \textbf{2} vische hiuten] fúschen heúten W fischeheute Q fische hútte R \textbf{3} \textit{Die Verse 570.3-4 fehlen} T   $\cdot$ Ein svrkoyt vnde ein bonit (bomeit W ) V (W) (Q) (R) \textbf{4} Vnde dez selben [z*o]: zwo hosen wit V · Vnd (Von R ) des selben zwuͦ (zw Q ) hosen weit W (Q) (R) \textbf{6} Des] Eres W Der Q  $\cdot$ kolbe] kv́le V (W) (Q) (R) \textbf{7} er] D W  $\cdot$ Gawane] gawan W Gawinen R :::e Fr39 \textbf{8} daz] Des R  $\cdot$ dannoch] doch V W Q R  $\cdot$ niender] nyeman R  $\cdot$ ger] wer R \textbf{9} dâ] do V W Q Fr39 \textit{om.} R \textbf{10} Gawan] Gawin R  $\cdot$ dâhte] dach R \textbf{11} wer] gewer W \textbf{13} im] in R  $\cdot$ swære] were W swer were Q \textit{om.} R  $\cdot$ lit] lide Q werre leit R \textbf{14} trit] schrit R \textbf{15} ob] \textit{om.} Q \textbf{16} zornlîchen] zornntleichen Q \textbf{17} mîn] mich W Q R  $\cdot$ entsitzen] entsichen R \textbf{18} gevüege] fvͤge V (W) (Q) (R)  $\cdot$ daz] v́ch V das úch W (Q) (R) :::iv Fr39 \textbf{20} noch] [*]: noch V doch R \textbf{21} iuch] îv T  $\cdot$ hie] noch Q \textbf{22} doch] noch V  $\cdot$ unerwert] vnerneret R \textbf{23} iuch] îv T euch des Q \textbf{24} hinnen] von hinnen W (Q) (R) \textbf{25} Der] ÷er T  $\cdot$ vilân] gebawr Q  $\cdot$ trat] trat hin W \textbf{26} Gawan] Gawin R \textbf{27} zeine] steine Q \textbf{28} die] [*]: Wande die V  $\cdot$ algemeine] algemiene V \textbf{29} wâren] w::: T \textbf{30} sin] in Q  $\cdot$ klungen] erklvngen V klúnden Q \newline
\end{minipage}
\end{table}
\end{document}
