\documentclass[8pt,a4paper,notitlepage]{article}
\usepackage{fullpage}
\usepackage{ulem}
\usepackage{xltxtra}
\usepackage{datetime}
\renewcommand{\dateseparator}{.}
\dmyyyydate
\usepackage{fancyhdr}
\usepackage{ifthen}
\pagestyle{fancy}
\fancyhf{}
\renewcommand{\headrulewidth}{0pt}
\fancyfoot[L]{\ifthenelse{\value{page}=1}{\today, \currenttime{} Uhr}{}}
\begin{document}
\begin{table}[ht]
\begin{minipage}[t]{0.5\linewidth}
\small
\begin{center}*D
\end{center}
\begin{tabular}{rl}
\textbf{98} & ich \textbf{wære des} trûric oder geil,\\ 
 & \textbf{mich behabt hie} rîters urteil.\\ 
 & vart wider, sagt ir dienest mîn.\\ 
 & ich \textbf{sul} iedoch ir ritter sîn.\\ 
5 & ob mir alle krône wæren bereit,\\ 
 & ich hân nâch ir mîn hœhste leit."\\ 
 & Er bôt in sîne grôze habe.\\ 
 & sîner \textbf{gebe} tâten si sich abe.\\ 
 & die boten vuoren ze lande\\ 
10 & gar ân ir vrouwen schande.\\ 
 & sine gerten urloubes niht,\\ 
 & \textbf{als} \textbf{lîhte} in zorne noch geschiht.\\ 
 & \textbf{ir} \textbf{knappen - vürsten}, \textbf{disiu} kint -\\ 
 & wâren von weinen vil nâch blint.\\ 
15 & \begin{large}D\end{large}ie den schilt verkêrt \textbf{dâ} hânt getragen,\\ 
 & den \textbf{begunde} \textbf{ir vriwent} ze velde sagen:\\ 
 & "vrou Herzeloyde, diu künegîn,\\ 
 & \textbf{diu} hât behabt den Anschevin."\\ 
 & "wer was von Anschouwe dâ?\\ 
20 & unser hêrre ist leider anderswâ,\\ 
 & durch rîters prîs ze\textbf{n} Sarrazin.\\ 
 & daz ist \textbf{nû} unser hœhster pîn."\\ 
 & "Der \textbf{hie} den prîs hât bezalt\\ 
 & unt \textbf{sô} manegen ritter ab gevalt\\ 
25 & \textbf{unt} der sô stach \textbf{und} sluoc\\ 
 & unt der \textbf{den} tiwern anker truoc\\ 
 & ûf dem helme lieht gesteinet,\\ 
 & daz ist, den ir dâ meinet.\\ 
 & \textbf{Mir} sagt der künec Kaylet,\\ 
30 & der Anschevin wære Gahmuret.\\ 
\end{tabular}
\scriptsize
\line(1,0){75} \newline
D \newline
\line(1,0){75} \newline
\textbf{7} \textit{Majuskel} D  \textbf{15} \textit{Initiale} D  \textbf{23} \textit{Majuskel} D  \textbf{29} \textit{Majuskel} D  \newline
\line(1,0){75} \newline
\textbf{18} Anschevin] Anscevin D \textbf{19} Anschouwe] Anscoͮwe D \textbf{30} Anschevin] Anscevin D  $\cdot$ Gahmuret] Gahmvret D \newline
\end{minipage}
\hspace{0.5cm}
\begin{minipage}[t]{0.5\linewidth}
\small
\begin{center}*m
\end{center}
\begin{tabular}{rl}
 & ich \textbf{wære des} trûric oder geil,\\ 
 & \textbf{mich behabet hie} ritters urteil.\\ 
 & vart wider, saget ir dienest mîn.\\ 
 & ich \textbf{sulle} iedoch ir ritter sîn.\\ 
5 & ob mir alle krône wæren bereit,\\ 
 & ich hân nâch ir mîn hœheste leit."\\ 
 & er bôt in sîne grôze habe.\\ 
 & sîner \textbf{gâben} tâten \textit{si} sich abe.\\ 
 & die boten vuoren ze lande\\ 
10 & gar âne ir vrouwen schande.\\ 
 & sine gerten urloubes niht,\\ 
 & \textbf{als} \textbf{lîhte} \textit{i}n zorne noch geschiht.\\ 
 & \textbf{ir} \textbf{knebelîn, der vürsten} kint,\\ 
 & wâren von weinen \textit{vil} nâch blint.\\ 
15 & \begin{large}D\end{large}ie den schilt verkêret hânt getragen,\\ 
 & den \textbf{begunde} \textbf{man} ze velde sagen:\\ 
 & "vrouwe Herczeloid\textit{e}, diu künigîn,\\ 
 & h\textit{â}t behabet den A\textit{n}schevin."\\ 
 & "wer was von Anschouwe d\textit{â}?\\ 
20 & unser hêrre ist leider anderswâ,\\ 
 & durch ritters prîs ze\textbf{n} Sarrazin.\\ 
 & daz ist \textbf{nû} unser hœhester pîn."\\ 
 & "der \textbf{hie} den prîs hât bezalt\\ 
 & und \textbf{sô} manigen ritter \textbf{hât} ab gevalt\\ 
25 & \textbf{und} der sô stach \textbf{oder} sluoc\\ 
 & und der tiuren anker truoc\\ 
 & ûf dem helm lieht gesteinet,\\ 
 & daz ist, den ir d\textit{â} meinet.\\ 
 & \textbf{mir} sagete der künic Kailet,\\ 
30 & der A\textit{n}schevin wære Gahmuret.\\ 
\end{tabular}
\scriptsize
\line(1,0){75} \newline
m n o \newline
\line(1,0){75} \newline
\textbf{15} \textit{Initiale} m   $\cdot$ \textit{Capitulumzeichen} n  \newline
\line(1,0){75} \newline
\textbf{2} behabet] behaltet n o \textbf{6} hœheste] hoͯhestes n o \textbf{7} grôze] groste o \textbf{8} gâben] gobe n (o)  $\cdot$ si] \textit{om.} m o \textbf{9} vuoren] fuͯren n o \textbf{11} sine] Sú n (o) \textbf{12} in] ein m \textbf{14} vil] \textit{om.} m wil o \textbf{15} Die] Die die o \textbf{16} sagen] [jagen]: sagen o \textbf{17} Herczeloide] herczeloiden m hertzeloide n herczeleide o \textbf{18} hât] Habet m  $\cdot$ Anschevin] ausceuin m anscevin n (o) \textbf{19} Anschouwe] anschowe o  $\cdot$ dâ] do m n \textbf{21} prîs] \textit{om.} n  $\cdot$ Sarrazin] sarazzin m sarrezin n farrezin o \textbf{22} hœhester] hoͯheste n (o) \textbf{24} hât] \textit{om.} n o \textbf{25} oder] vnd n o \textbf{26} tiuren] den túren n (o) \textbf{28} daz] Den o  $\cdot$ dâ] do m n o \textbf{29} sagete] saget n (o) \textbf{30} der] Vnd n o  $\cdot$ Anschevin] auscevin m [auscenin]: ausceuin n anscevin o  $\cdot$ Gahmuret] gamúret n gamuͯret o \newline
\end{minipage}
\end{table}
\newpage
\begin{table}[ht]
\begin{minipage}[t]{0.5\linewidth}
\small
\begin{center}*G
\end{center}
\begin{tabular}{rl}
 & ich \textbf{werdes} trûrec oder geil,\\ 
 & \textbf{mich behabt hie} rîters urteil.\\ 
 & vart wider, saget ir dienst mîn.\\ 
 & ich \textbf{sol} iedoch ir rîter sîn.\\ 
5 & \begin{large}O\end{large}be mir alle krône w\textit{æ}ren bereit,\\ 
 & ich hân nâch ir mîn hœhest leit."\\ 
 & er bôt in sîne grôze habe.\\ 
 & sîner \textbf{gâbe} tâten si sich abe.\\ 
 & die boten vuoren ze lande\\ 
10 & gar âne ir vrouwen schande.\\ 
 & sine gerten urloubes niht,\\ 
 & \textbf{als} \textbf{dicke} in zorne noch geschiht.\\ 
 & \textbf{ir} \textbf{knappen - vürsten}, \textbf{disiu} kint -\\ 
 & wâren von weinen vil nâch blint.\\ 
15 & die den schilt verkêrt hânt getragen,\\ 
 & den \textbf{begunde} \textbf{ir vriunt} ze velde sagen:\\ 
 & "vrô Herzeloide, diu künigîn,\\ 
 & hât behabet den Antschevin."\\ 
 & "wer was von Anschouwe dâ?\\ 
20 & unser hêrre ist leider anderswâ,\\ 
 & durch rîters prîs ze Sarrazin.\\ 
 & daz ist \textbf{nû} unser hœhester pîn."\\ 
 & "der \textbf{dâ} den prîs hât bezalt\\ 
 & unde manigen rîter abe gevalt,\\ 
25 & der sô stach \textbf{unt} \textbf{der} \textbf{sô} sluoc\\ 
 & unt der \textbf{den} tiweren anker truoc\\ 
 & ûf dem helme lieht gesteinet,\\ 
 & daz ist, den ir dâ meinet.\\ 
 & \textbf{uns} saget der künic Kailet,\\ 
30 & der Antschevin wære Gahmuret,\\ 
\end{tabular}
\scriptsize
\line(1,0){75} \newline
G I O L M Q R Z Fr36 \newline
\line(1,0){75} \newline
\textbf{3} \textit{Initiale} I O  \textbf{5} \textit{Initiale} G  \textbf{9} \textit{Capitulumzeichen} L  \textbf{15} \textit{Initiale} L Q R Z  \textbf{23} \textit{Initiale} I  \newline
\line(1,0){75} \newline
\textbf{1} werdes] werde sin I (O) werde es L were des M werd Q werde des Z  $\cdot$ trûrec] vro I \textbf{2} \textit{Vers 98.2 fehlt} O  \textbf{3} vart] Wart Z  $\cdot$ ir] \textit{om.} O [hie]: ir Z  $\cdot$ dienst] den dienst I (O) (L) (M) (Q) (R) \textbf{4} iedoch] doch L \textbf{5} krône] cronen M R  $\cdot$ wæren] waren G \textbf{6} mîn] minne I (Q)  $\cdot$ hœhest] hohstev O (L) (Q) groste M \textbf{7} grôze] grozev O \textbf{8} gâbe] gaben L R  $\cdot$ tâten] tet Q \textbf{10} gar] \textit{om.} L  $\cdot$ ir] irr Z  $\cdot$ schande] schanden L (Q) \textbf{12} dicke] liht O (L) (M) (Q) (R) Z  $\cdot$ in] an L \textbf{14} wâren] Wordin M  $\cdot$ von weinen vil nâch] nach von iamer I von weinenne nach L vor weinen vil nach Q vor weinende nach R \textbf{15} verkêrt] vercherten O (Q)  $\cdot$ hânt] hant her I da heten O do han Q da hant R (Z) \textbf{16} begunde] begonden M  $\cdot$ vriunt] frivnde O (L) (Fr36) \textbf{17} Herzeloide] herzenlau I herzenlavde O hertzelauͯde L herczeloide M herzeloude Q herczelaude R herzelovde Z \textbf{18} Antschevin] anschevin G antscheuin I anshevin O Z Anschovin L anschevyn M anshewin Q Anschwevin R :::schewin Fr36 \textbf{19} wer] owe wer I We wer O L So wer M We war Q Wo wer R  $\cdot$ Anschouwe] anschoͮwe G antschoͮe I anschawe O Anschowe L (R) (Z) anscowe M anshowe Q :::howe Fr36 \textbf{20} ist] was Z \textbf{21} durch rîters prîs] ze ritershaft I  $\cdot$ ze Sarrazin] zesarazin G ze den sarrazin O (Q) zuͯ den sarszin L zun sarrasin R zv sarrazzin Z \textbf{22} daz] da Fr36  $\cdot$ hœhester] hoheste L (M) (R) (Z) \textbf{23} dâ den prîs] da pris M den preisz do Q  $\cdot$ bezalt] gezalt Q \textbf{24} unde] \textit{om.} O R \textbf{25} der sô stach] vnde der so stach I (L) (Z) Vnde der da stach M  $\cdot$ der sô sluoc] so sloͮch O der sluͯch L der da sluc M do so schluͦg R slvc Z \textbf{26} unt] \textit{om.} O Q  $\cdot$ den] \textit{om.} Z  $\cdot$ tiweren] hohen I tewr Q \textbf{27} lieht] lẏcht L (M) (Q) \textbf{28} den] der den I (L) denne R daz Z  $\cdot$ ir dâ] ir I R da Q [ist]: ir da Z \textbf{29} uns] Vnde O (Z) Jr mir Q  $\cdot$ saget] sagte L sag R sagt vns Z  $\cdot$ Kailet] Gahilet I kaylet O M Q R kaẏlet L gailet Z \textbf{30} der] Das R  $\cdot$ Antschevin] anschevin G antscheuin I anshevin O (L) Z aschevin M anszhewein Q anschwevin R  $\cdot$ wære] sẏ L  $\cdot$ Gahmuret] Gamvret O (M) (Q) (Z) Gahmuͯret L \newline
\end{minipage}
\hspace{0.5cm}
\begin{minipage}[t]{0.5\linewidth}
\small
\begin{center}*T (U)
\end{center}
\begin{tabular}{rl}
 & ich \textbf{werdes} trûric oder geil,\\ 
 & \textbf{ich behabete ie} ritters urteil.\\ 
 & vart wider \textbf{und} saget ir \textbf{den} dienst mîn.\\ 
 & ich \textbf{wil} iedoch ir ritter sîn.\\ 
5 & ob mir alle krônen wæren bereit,\\ 
 & ich hân nâch \textit{ir} mî\textit{n} hœhest\textit{e} leit."\\ 
 & er bôt in sîne grôze habe.\\ 
 & sîner \textbf{gâbe} tâten si sich abe.\\ 
 & die boten vuoren zuo lande\\ 
10 & gar âne ir vrouwen schande.\\ 
 & si engerten urloubes niht,\\ 
 & \textbf{daz} \textbf{lîhte} in zorne noch geschiht.\\ 
 & \textbf{die} \textbf{vürsten - knappen}, \textbf{disiu} kint -\\ 
 & wâren von weinen vil nâch blint.\\ 
15 & \begin{large}D\end{large}ie den schilt verkêret hânt getragen,\\ 
 & den \textbf{begunden} \textbf{ir vriunt} zuo velde sagen:\\ 
 & "vrou Herzeloyde, diu künegîn,\\ 
 & hât behabt den Anschevin."\\ 
 & "\textbf{wê}, wer was von Anschouwe dâ?\\ 
20 & unser hêrre ist leider anderswâ,\\ 
 & durch ritters prîs zuo Sarrazin.\\ 
 & daz ist \textbf{noch} unser hœheste pîn."\\ 
 & "der \textbf{d\textit{â}} den prîs hât bezalt\\ 
 & und manegen ritter abe gevalt\\ 
25 & \textbf{und} der sô stach \textbf{und} sluoc\\ 
 & und der \textbf{den} tiuren anker truoc\\ 
 & ûf deme helme lieht gesteine\textit{t},\\ 
 & daz ist, den ir dâ meinet.\\ 
 & \textbf{uns} sagete der künec Kaylet,\\ 
30 & der Anschevin wære Gahmuret,\\ 
\end{tabular}
\scriptsize
\line(1,0){75} \newline
U V W T \newline
\line(1,0){75} \newline
\textbf{3} \textit{Majuskel} T  \textbf{5} \textit{Majuskel} T  \textbf{8} \textit{Majuskel} T  \textbf{13} \textit{Initiale} W  \textbf{15} \textit{Initiale} U V T  \textbf{23} \textit{Majuskel} T  \newline
\line(1,0){75} \newline
\textbf{1} werdes] wúrd es W werdez T \textbf{2} ich behabete ie] ich behalte ie V Ich behabte ir W mich behabt hie T  $\cdot$ urteil] tail W \textbf{3} und] \textit{om.} V T \textbf{4} wil] sol T  $\cdot$ iedoch] auch ymmer W  $\cdot$ ritter] dienst T \textbf{5} krônen] crône T \textbf{6} ir mîn hœheste] mime hohesten U mein hoͤchste W \textbf{8} gâbe] haben V \textbf{11} engerten] gerten W \textbf{12} als noch in zorne gesciht T  $\cdot$ in] von W  $\cdot$ geschiht] beschicht V \textbf{13} die vürsten knappen] DIe knappen fúrsten W ir knappen vursten T \textbf{14} \textit{nach 98.14:} Das sy vor layde taten / Auch was es dannoch nit zuͦ spaten W   $\cdot$ von] vor T  $\cdot$ weinen] weinende V  $\cdot$ vil nâch] also W \textbf{16} den begunden] den begunde V begvnde T \textbf{17} Herzeloyde] Herzeleide U herzelaude V hertzeloyde W \textbf{18} Anschevin] Anscheuin V antscheuin W \textbf{19} wê] Wei V  $\cdot$ Anschouwe] Anschowe U V antschowe W Anschôuwe T  $\cdot$ dâ] do W \textbf{21} zuo Sarrazin] zuͦ [*]: den sarrasin V zuͦ den sarazin W zen riters T \textbf{22} noch] [*]: nv V nv T \textbf{23} dâ] do U W \textbf{24} ritter] riters T \textbf{25} sô] do W  $\cdot$ sluoc] so slvͦc T \textbf{26} der] \textit{om.} W \textbf{27} lieht] wol W  $\cdot$ gesteinet] gesteine U \textbf{28} Das der in do meinet W  $\cdot$ den] der den T \textbf{29} sagete] saget W  $\cdot$ Kaylet] kaẏlet V gaylet W \textbf{30} Anschevin] Anscheuin V antscheuin W  $\cdot$ Gahmuret] [Gahnuͦ*]: Gah muͦret U Gamuret V (W) [gemvret]: gamvret  T \newline
\end{minipage}
\end{table}
\end{document}
