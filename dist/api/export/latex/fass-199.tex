\documentclass[8pt,a4paper,notitlepage]{article}
\usepackage{fullpage}
\usepackage{ulem}
\usepackage{xltxtra}
\usepackage{datetime}
\renewcommand{\dateseparator}{.}
\dmyyyydate
\usepackage{fancyhdr}
\usepackage{ifthen}
\pagestyle{fancy}
\fancyhf{}
\renewcommand{\headrulewidth}{0pt}
\fancyfoot[L]{\ifthenelse{\value{page}=1}{\today, \currenttime{} Uhr}{}}
\begin{document}
\begin{table}[ht]
\begin{minipage}[t]{0.5\linewidth}
\small
\begin{center}*D
\end{center}
\begin{tabular}{rl}
\textbf{199} & ê daz ich si gereche,\\ 
 & al dâ ich schilt durchsteche.\\ 
 & Sage \textbf{Artuse unt dem wîbe sîn},\\ 
 & \textbf{in beiden, von mir dienest mîn},\\ 
5 & dar zuo der massenîe gar,\\ 
 & \textbf{unt} daz ich \textbf{nimmer kum} dar,\\ 
 & ê ich lasters mich entsage,\\ 
 & daz ich geselleclîchen trage\\ 
 & mit ir, diu mir lachen bôt.\\ 
10 & des \textbf{kom ir lîp} in grôze nôt.\\ 
 & sag ir, ich sî ir dienstman,\\ 
 & \textbf{dienstlîcher dienste} undertân."\\ 
 & Der rede ein volge dâ geschach.\\ 
 & die helde man sich scheiden sach.\\ 
15 & Hin wider kom gegangen,\\ 
 & dâ sîn ors was gevangen,\\ 
 & der burgære kampfes trôst.\\ 
 & si wurden sît von im erlôst.\\ 
 & zwîvels pflac daz ûzer her,\\ 
20 & daz Kingrun an sîner wer\\ 
 & \textbf{was} enschumpfieret.\\ 
 & \textbf{nû} wart gecondwieret\\ 
 & \textit{\begin{large}P\end{large}}arzival zer künegîn.\\ 
 & \textbf{diu} tet \textbf{im umbevâhens} schîn.\\ 
25 & \textbf{si} druct in vaste an ir lîp.\\ 
 & si sprach: "ine wirde niemer wîp\\ 
 & ûf erde decheines man,\\ 
 & wan den ich umbevangen hân."\\ 
 & Si half, daz er entwâpent wart.\\ 
30 & ir dienst was vil ungespart.\\ 
\end{tabular}
\scriptsize
\line(1,0){75} \newline
D \newline
\line(1,0){75} \newline
\textbf{3} \textit{Majuskel} D  \textbf{13} \textit{Majuskel} D  \textbf{15} \textit{Majuskel} D  \textbf{23} \textit{Initiale} D  \textbf{29} \textit{Majuskel} D  \newline
\line(1,0){75} \newline
\textbf{23} Parzival] ÷arzival \textit{nachträglich korrigiert zu:} Parzival D \newline
\end{minipage}
\hspace{0.5cm}
\begin{minipage}[t]{0.5\linewidth}
\small
\begin{center}*m
\end{center}
\begin{tabular}{rl}
 & ê daz ich si gereche,\\ 
 & aldâ ich schilt durchsteche.\\ 
 & sage \textbf{Artuse und \textit{d}em wîbe sîn},\\ 
 & \textbf{in bêden, von mir den dienest mîn},\\ 
5 & dar zuo der massenîe gar,\\ 
 & \textbf{und} daz ich \textbf{niemer kume} dar,\\ 
 & ê \textbf{daz} ich lasters mich entsage,\\ 
 & daz ich geselleclîch trage\\ 
 & mit ir, diu mir lachen bôt,\\ 
10 & des \textbf{ir lîp kom} in grôze nôt.\\ 
 & sage ir, ich sî ir dienestman,\\ 
 & \textbf{dienstlîcher dieneste} undertân."\\ 
 & \begin{large}D\end{large}er rede ein volge d\textit{â} geschach.\\ 
 & die helde man sich scheiden sach.\\ 
15 & hin wider kam gegangen,\\ 
 & d\textit{â} sîn ros was gevangen,\\ 
 & der burgære kampfes trôst.\\ 
 & si wurden sît von ime erlôst.\\ 
 & zwîvels pflac daz ûzer her,\\ 
20 & daz Kingr\textit{un} an \textit{s}î\textit{n}e\textit{r} \textit{w}er\\ 
 & \textbf{was} entschumpfieret.\\ 
 & \textbf{nû} wart gecondwieret\\ 
 & Parcifal zer künigîn.\\ 
 & \textbf{diu} tet \textbf{umbevâhen im} schîn.\\ 
25 & \textbf{si} druht in vaste an ir lîp.\\ 
 & si sprach: "ine wirde niemer wîp\\ 
 & ûf \textbf{der} erden keines man,\\ 
 & wan den ich umbevangen hân."\\ 
 & si half, daz er entwâpent wart.\\ 
30 & ir dienest was vil ungespart.\\ 
\end{tabular}
\scriptsize
\line(1,0){75} \newline
m n o Fr69 \newline
\line(1,0){75} \newline
\textbf{13} \textit{Illustration mit Überschrift:} Also parcifal die (der o  ) burger vnd die konigin erlost von dem strit vnd die kv́niginne parcifalen (parcifaln o  ) halff sinen harnesche vs duͦn n (o)   $\cdot$ \textit{Initiale} m n o Fr69  \newline
\line(1,0){75} \newline
\textbf{1} daz] dasz das o \textbf{2} aldâ] Alt da Fr69  $\cdot$ ich] ir n o \textbf{3} dem] diem m \textbf{4} in bêden von mir] Den gruͯsz vnd ouch n Den grusz vnd o \textbf{6} ich] \textit{om.} o \textbf{7} mich] me o \textbf{8} daz] Vnd n o \textbf{9} diu mir] min Fr69 \textbf{11} ich sî] sẏhe eyn o ich bin Fr69  $\cdot$ ir] \textit{om.} o  $\cdot$ dienestman] dienste mann o \textbf{12} dienstlîcher] Dienstlichem n Dienstlichen o \textbf{13} dâ] do m n o \textbf{16} dâ] Do m n o \textbf{17} der] Die n o \textbf{18} ime] mir n \textbf{20} Kingrun] kingrim m kingrun n konigrim o  $\cdot$ sîner wer] inmer hant wer m \textbf{21} entschumpfieret] entschiempfieret o \textbf{22} nû] Mú o \textbf{23} Parcifal] Parcifalen n \textbf{25} druht] truckt o \textbf{26} ine wirde] ich wúrde n (o) \textbf{27} erden] erde Fr69 \textbf{30} \textit{Vers 199.30 fehlt} o  \newline
\end{minipage}
\end{table}
\newpage
\begin{table}[ht]
\begin{minipage}[t]{0.5\linewidth}
\small
\begin{center}*G
\end{center}
\begin{tabular}{rl}
 & ê daz ich si gereche,\\ 
 & al dâ ich schilt durchsteche,\\ 
 & \textbf{unde} sage \textbf{von mînem lîbe}\\ 
 & \textbf{Artuse unde sînem wîbe},\\ 
5 & dar zuo der massenîe gar,\\ 
 & daz ich \textbf{wil nimer komen} dar,\\ 
 & \begin{large}Ê\end{large} \textbf{daz} ich lasters mich entsage,\\ 
 & daz ich geselliclîchen trage\\ 
 & mit ir, diu mir lachen bôt.\\ 
10 & des \textbf{kom ir lîp} in grôze nôt.\\ 
 & sage ir, ich sî ir dienstman,\\ 
 & \textbf{dienstlîcher dienst} undertân."\\ 
 & der rede ein volge dâ geschach.\\ 
 & die helde man sich scheiden sach.\\ 
15 & hin widere kom gegangen,\\ 
 & dâ sîn ors was gevangen,\\ 
 & der bürgære kampfes trôst.\\ 
 & si wurden sît von im erlôst.\\ 
 & zwîvels pflac daz ûzer her,\\ 
20 & daz Kingrun an sîner wer\\ 
 & \textbf{was} entschumpfieret.\\ 
 & \textbf{dô} wart gecondwieret\\ 
 & Parzival zer künigîn.\\ 
 & \textbf{si} tet \textbf{im umbevâhen} schîn.\\ 
25 & \textbf{si} druhte in vaste an ir lîp.\\ 
 & si sprach: "ichne wirde nimer wîp\\ 
 & ûf \textbf{dirre} erde deheines man,\\ 
 & wan den ich umbevangen hân."\\ 
 & si half, daz er entwâpent wart.\\ 
30 & ir dienst was vil ungespart.\\ 
\end{tabular}
\scriptsize
\line(1,0){75} \newline
G I O L M Q R Z \newline
\line(1,0){75} \newline
\textbf{3} \textit{Initiale} I  \textbf{7} \textit{Initiale} G  \textbf{11} \textit{Initiale} M  \textbf{13} \textit{Initiale} O  \textbf{15} \textit{Initiale} L  \textbf{19} \textit{Initiale} Z  \textbf{25} \textit{Initiale} I  \textbf{29} \textit{Überschrift:} Hie strit parczifal mit kygrun vnd v́ber wand Jm mit gwalt R   $\cdot$ \textit{Initiale} M R  \newline
\line(1,0){75} \newline
\textbf{1} si] sei O \textbf{2} al dâ] Al daz L \textbf{3} Sag artus (artusen Z ) vnd dem weybe sein Q (Z)  $\cdot$ unde sage] Vnde sage ir M Sag R \textbf{4} Yn beyden von mir den (\textit{om.} Z ) dinst meyn Q (Z)  $\cdot$ Artuse] artus I  $\cdot$ unde] von M \textbf{5} dar zuo] vnd I (Z) \textbf{6} daz] Vnd das Q  $\cdot$ wil nimer] nimmer wil I (O) (M) (Q) (Z) niemer welle L niemer R \textbf{7} ich lasters] lasters ich R \textbf{8} ich] \textit{om.} O  $\cdot$ trage] clage Q \textbf{9} mir] mÿ M \textbf{10} des] Do Q R  $\cdot$ grôze] groziv O \textbf{13} der] ÷er O  $\cdot$ dâ geschach] [wart]: do geschach O do geschag Q \textbf{14} helde] helden R Z  $\cdot$ man sich scheiden] scheiden man sich Z \textbf{15} hin] Ejn L \textbf{16} dâ] Do L Dann Q  $\cdot$ was] wart Q \textbf{17} trôst] tiost Z \textbf{19} zwîvels] Zweiuel Q (R) \textbf{20} Kingrun] kyngrvn O (M) (R) kýngrvn L kingrún Q \textbf{22} dô] Da O M Z \textbf{23} Parzival] Parzifal I L M Parcifal O Z Partzifal Q Parczifal R \textbf{24} umbevâhen] vmbeuahens I (Z) vnwachen Q \textbf{25} druhte] [druch]: drucht I  $\cdot$ in] in in I \textbf{26} si] vnd I (O) (M) (Q) (R) (Z)  $\cdot$ ichne] ich I O L R  $\cdot$ wirde] wurd Q (R) \textbf{27} dirre] der I (O) (L) (M) (Q) (R) (Z)  $\cdot$ erde] erden L (M) Q \newline
\end{minipage}
\hspace{0.5cm}
\begin{minipage}[t]{0.5\linewidth}
\small
\begin{center}*T
\end{center}
\begin{tabular}{rl}
 & ê daz ich si gereche,\\ 
 & aldâ ich schilt durchsteche.\\ 
 & sage \textbf{Artuse unde dem wîbe sîn},\\ 
 & \textbf{in beiden, von mir den dienst mîn},\\ 
5 & dar zuo der massenîe gar,\\ 
 & \textbf{unde} daz ich \textbf{niemer kome} dar,\\ 
 & ê \textbf{daz} ich lasters mich entsage,\\ 
 & daz ich geselleclîche trage\\ 
 & mit ir, diu mir lachen bôt.\\ 
10 & des \textbf{kom ir lîp} in grôze nôt.\\ 
 & sagir, ich sî ir dienstman,\\ 
 & \textbf{dienstlîches dienstes} undertân."\\ 
 & \begin{large}D\end{large}er rede ein volge dâ geschach.\\ 
 & die helde man sich scheiden sach.\\ 
15 & hin wider kom gegangen,\\ 
 & dâ sîn ors was gevangen,\\ 
 & der bürgære kampfes trôst.\\ 
 & si wurden sît von im erlôst.\\ 
 & Zwîvels pflac daz ûzer her,\\ 
20 & daz Kyngrun an sîner wer\\ 
 & \textbf{sus} \textbf{wart} entschumpfieret.\\ 
 & \textbf{dâ} wart gecundewieret\\ 
 & Parcifal zer künegîn.\\ 
 & \textbf{diu} tet \textbf{im umbevâhens} schîn\\ 
25 & \textbf{unde} druhtin vaste an ir lîp.\\ 
 & si sprach: "ine wirde niemer wîp\\ 
 & ûf \textbf{der} erde keines man,\\ 
 & wan den ich umbevangen hân."\\ 
30 & \hspace*{-.7em}\big| ir dienst was vil ungespart.\\ 
 & \hspace*{-.7em}\big| si half \textbf{im}, daz er entwâpent wart.\\ 
\end{tabular}
\scriptsize
\line(1,0){75} \newline
T U V W \newline
\line(1,0){75} \newline
\textbf{13} \textit{Initiale} T U V W  \textbf{19} \textit{Majuskel} T  \newline
\line(1,0){75} \newline
\textbf{1} daz] \textit{om.} V \textbf{2} durchsteche] dvrch breche V \textbf{3} dem] \textit{om.} W \textbf{6} [*]: Vnde daz ich niemer kvmme dar V  $\cdot$ kome] queme U \textbf{7} lasters] laster U [*]: lasters V \textbf{8} daz] [*]: Daz V \textbf{9} diu mir] [*]: die mir V \textbf{10} [*]: Dez kam ir lip in grosse not V \textbf{11} ich sî] [*]: ich si V \textbf{12} [*]: Dienestliches dienstes vndertan V · Dienstlicher dienst vndertan W \textbf{13} dâ geschach] do geschach V ward getan W \textbf{14} man sich scheiden sach] schieden sich do san W \textbf{16} dâ] Do W \textbf{19} Zwîvels] Zwoͤlff W \textbf{20} Kyngrun] kyngruͦn U kingrun W \textbf{21} wart] was U W \textbf{22} dâ] Do U V W \textbf{23} Parcifal] Parzifal V Partzifal W \textbf{24} umbevâhens] vmb vahen U (V) \textbf{25} unde] [*]: Sv́ V Sy W \textbf{26} ine] ich W  $\cdot$ wirde] worde U \textbf{27} der erde] erden W \textbf{28} den] des den W \textbf{30} ungespart] gespart W \textbf{29} im] \textit{om.} W \newline
\end{minipage}
\end{table}
\end{document}
