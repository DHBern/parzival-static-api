\documentclass[8pt,a4paper,notitlepage]{article}
\usepackage{fullpage}
\usepackage{ulem}
\usepackage{xltxtra}
\usepackage{datetime}
\renewcommand{\dateseparator}{.}
\dmyyyydate
\usepackage{fancyhdr}
\usepackage{ifthen}
\pagestyle{fancy}
\fancyhf{}
\renewcommand{\headrulewidth}{0pt}
\fancyfoot[L]{\ifthenelse{\value{page}=1}{\today, \currenttime{} Uhr}{}}
\begin{document}
\begin{table}[ht]
\begin{minipage}[t]{0.5\linewidth}
\small
\begin{center}*D
\end{center}
\begin{tabular}{rl}
\textbf{402} & \begin{large}D\end{large}och vernemt durch iwer güete,\\ 
 & wie ein lûter gemüete\\ 
 & vremder valsch \textbf{gevrumte} trüebe.\\ 
 & ob ich \textbf{iu} vürbaz üebe\\ 
5 & diz mære mit \textbf{rehter} sage,\\ 
 & sô kumt irs mit mir in klage.\\ 
 & dô sprach der künec Vergulaht:\\ 
 & "hêrre, ich hân mich des bedâht,\\ 
 & ir sult rîten dort hin în.\\ 
10 & mag ez mit iweren hulden sîn,\\ 
 & ich \textbf{briche iu} nû gesellecheit.\\ 
 & ist ab iu \textbf{mîn} vürbaz rîten leit,\\ 
 & ich lâze, swaz ich ze schaffen hân."\\ 
 & dô sprach der werde Gawan:\\ 
15 & "hêrre, swaz ir gebietet,\\ 
 & billîche ir iuch des nietet.\\ 
 & daz ist ouch âne mînen zorn\\ 
 & mit guotem willen gar verkorn."\\ 
 & Dô sprach der künec von Ascalun:\\ 
20 & "hêrre, ir seht wol Schamfanzun.\\ 
 & dâ ist mîn swester ûf, ein magt.\\ 
 & swaz munt von schœne hât gesagt,\\ 
 & des hât si volleclîchen teil.\\ 
 & welt ir\textbf{z} \textbf{iu} prüeven vür ein heil,\\ 
25 & deiswâr, \textbf{sô} muoz si sich bewegen,\\ 
 & daz si iwer unz an mich sol pflegen.\\ 
 & ich kum \textbf{iu} \textbf{schierre}, \textbf{denn} ich sol.\\ 
 & ouch \textbf{erbeit} ir mîn \textbf{vil} wol,\\ 
 & geseht ir die swester mîn.\\ 
30 & ir \textbf{enruochtet}, wolt ich \textbf{noch} lenger sîn."\\ 
\end{tabular}
\scriptsize
\line(1,0){75} \newline
D \newline
\line(1,0){75} \newline
\textbf{1} \textit{Initiale} D  \textbf{19} \textit{Majuskel} D  \newline
\line(1,0){75} \newline
\textbf{7} Vergulaht] Vergvlaht D \textbf{20} Schamfanzun] Scamfanzvn D \newline
\end{minipage}
\hspace{0.5cm}
\begin{minipage}[t]{0.5\linewidth}
\small
\begin{center}*m
\end{center}
\begin{tabular}{rl}
 & doch vernemt durch iuwer güete,\\ 
 & wie ein lûter gemüete\\ 
 & vrömder valsch \textbf{gevrumte} trüebe.\\ 
 & ob ich \textbf{iu} vürbaz üebe\\ 
5 & diz mære mit \textbf{rehter} sage,\\ 
 & sô komet irs mit mir in \textbf{die} klage.\\ 
 & \begin{large}D\end{large}ô sprach der künic Vergula\textit{h}t:\\ 
 & "hêrre, ich hân mich des bedâht,\\ 
 & ir sult rîten dort hin în.\\ 
10 & mac ez mit iuwern hulden sîn,\\ 
 & ich \textbf{briche iu} nû gesellicheit.\\ 
 & ist aber iu \textbf{nû} vürbaz rîten leit,\\ 
 & ich lâz \textbf{ez}, waz ich ze schaffen hân."\\ 
 & dô sprach der werde Gawan:\\ 
15 & "hêrre, waz ir gebietet,\\ 
 & billîch ir iuch des nietet.\\ 
 & daz ist ouch âne mînen zorn\\ 
 & mit guotem willen gar verkorn."\\ 
 & dô sprach der künic von Ascalun:\\ 
20 & "hêrre, ir seht wol Schanfanzun.\\ 
 & dâ ist mîn swester ûf, ein maget.\\ 
 & waz munt von schœne hât gesaget,\\ 
 & des hât si volleclîchen teil.\\ 
 & wellet ir\textbf{z} \textbf{iu} brüefen vür ein heil,\\ 
25 & deiswâr, \textbf{dô} muoz si sich bewegen,\\ 
 & daz \textit{si} iuwer unz an mich sol pflegen.\\ 
 & ich kome \textbf{ouch} \textbf{schiere}, \textbf{ob} ich sol.\\ 
 & ouch \textbf{erb\textit{ei}tet} ir mîn wol,\\ 
 & gesehet ir die swester mîn.\\ 
30 & ir \textbf{enruochet}, wolt ich lenger sîn."\\ 
\end{tabular}
\scriptsize
\line(1,0){75} \newline
m n o \newline
\line(1,0){75} \newline
\textbf{7} \textit{Initiale} m   $\cdot$ \textit{Capitulumzeichen} n  \newline
\line(1,0){75} \newline
\textbf{1} doch] Duch o \textbf{3} gevrumte] gefremter o \textbf{5} diz] Dise n \textbf{6} irs] ir n o \textbf{7} Vergulaht] vergulat m (o) vergulacht n \textbf{12} nû] min n (o) \textbf{13} ez] \textit{om.} n o \textbf{16} nietet] genẏtet n nieten o \textbf{19} Ascalun] ascelun n astalim o \textbf{20} Schanfanzun] Scanfanzún m scanpfanzun n stampfanczẏm o \textbf{21} dâ] Das m \textbf{22} von schœne hât gesaget] schon het gesagent o \textbf{23} des] Dasz o \textbf{24} iu brüefen] auch prisen o \textbf{25} dô] so n o  $\cdot$ muoz] mús m muͯsz n o \textbf{26} si] úch m \textbf{27} ob] denne n (o) \textbf{28} erbeitet] erbiettet m (n) \textbf{30} ich] ir o \newline
\end{minipage}
\end{table}
\newpage
\begin{table}[ht]
\begin{minipage}[t]{0.5\linewidth}
\small
\begin{center}*G
\end{center}
\begin{tabular}{rl}
 & doch vernemet durch iwer güete,\\ 
 & wie ein lûter gemüete\\ 
 & vrömder valsch \textbf{gevrumte} trüebe.\\ 
 & obe ich \textbf{iu} vürbaz üebe\\ 
5 & diz mære mit \textbf{rehter} sage,\\ 
 & sô komt irs mit mir in klage.\\ 
 & dô sprach der künic Vergulaht:\\ 
 & "hêrre, ich hân mich des bedâht,\\ 
 & ir sult rîten dort hin în.\\ 
10 & mag ez mit iwern hulden sîn,\\ 
 & ich \textbf{briche iu} nû gesellicheit.\\ 
 & ist abe iu \textbf{mîn} vürbaz rîten leit,\\ 
 & ich lâze, swaz ich ze schaffene hân."\\ 
 & dô sprach der werde Gawan:\\ 
15 & "hêrre, swaz ir gebiet,\\ 
 & billîche ir iuch des niet.\\ 
 & daz ist ouch âne mînen zorn\\ 
 & mit guotem willen gar verkorn."\\ 
 & dô sprach der künic von Aschalun:\\ 
20 & "hêrre, ir seht wol Tschanfenzun.\\ 
 & dâ ist mîn swester ûf, ein maget.\\ 
 & \begin{large}S\end{large}waz munt von schœne hât gesaget,\\ 
 & des hât si volliclîchen teil.\\ 
 & welt ir\textbf{z} \textbf{nû} prüeven vür ein heil,\\ 
25 & dêswâr, \textbf{sô} muoz si sich bewegen,\\ 
 & daz si iwer unze ane mich sol pflegen.\\ 
 & ich kum \textbf{iu} \textbf{schier}, \textbf{dane} ich sol.\\ 
 & ouch \textbf{erbeit} ir mîn \textbf{vil} wol,\\ 
 & gesehet ir die swester mîn.\\ 
30 & ir \textbf{enruocht}, wolt ich \textbf{noch} lenger sîn."\\ 
\end{tabular}
\scriptsize
\line(1,0){75} \newline
G I O L M Q R Z \newline
\line(1,0){75} \newline
\textbf{1} \textit{Initiale} I O L Z   $\cdot$ \textit{Capitulumzeichen} R  \textbf{15} \textit{Initiale} I  \textbf{22} \textit{Initiale} G  \newline
\line(1,0){75} \newline
\textbf{1} \textit{Die Verse 370.13-412.12 fehlen} Q   $\cdot$ doch] ÷och O \textbf{3} valsch] valchs I  $\cdot$ gevrumte] gefrumter I frumte R braht Z \textbf{5} diz] Daz O  $\cdot$ rehter] rehte L \textbf{6} irs] ir sin I ir O L R  $\cdot$ in] indie O (M) (R) (Z) \textbf{7} dô] Da O M  $\cdot$ Vergulaht] vergvlaht G O L virgulath I vergulacht M R \textbf{11} briche] sprich I (M) bericht R  $\cdot$ nû] hie I  $\cdot$ gesellicheit] geselheit R \textbf{12} vürbaz rîten] riten fvrbaz O \textbf{13} Jch laz es [tzeschafen swaz]:  swaz tzeschafen ich han O  $\cdot$ lâze] sag ev I lasze esz M (Z)  $\cdot$ swaz] waz L (M) (R) \textbf{14} dô] Da M \textbf{15} swaz] waz L (M) (R) \textbf{17} ist] \textit{om.} R \textbf{19} dô] Da O M  $\cdot$ Aschalun] ascalun I (L) (M) (R) (Z) Aschalv̂n O \textbf{20} Tschanfenzun] tschanphenzun G sanphanzuͤn I schampfozvn O Tsanfenzvn L scanpfanzcuͯn M schanfenzun R Tschanfanzvn Z \textbf{21} swester] tohter Z \textbf{22} Swaz] Waz L (M) (R)  $\cdot$ munt] yemen R  $\cdot$ hât] noch hat I \textbf{23} des] Das R \textbf{24} nû] ev Z \textbf{25} dêswâr] Jst ez war M Zwar Z \textbf{26} unze] fusz M  $\cdot$ sol pflegen] nu phlege I \textbf{27} iu] och R  $\cdot$ schier] schirer O ehir M \textbf{28} erbeit] erbit I (O) erbietet L  $\cdot$ vil] \textit{om.} Z \textbf{29} gesehet] Gesicht R \textbf{30} enruocht] en ruchtet M ruͦchent R  $\cdot$ wolt] wol ob R  $\cdot$ noch] \textit{om.} M R  $\cdot$ sîn] bin R \newline
\end{minipage}
\hspace{0.5cm}
\begin{minipage}[t]{0.5\linewidth}
\small
\begin{center}*T
\end{center}
\begin{tabular}{rl}
 & Doch vernemt durch iuwer güete,\\ 
 & wie ein lûter gemüete\\ 
 & vremder valsch \textbf{vrumte} trüebe.\\ 
 & ob ich vürbaz üebe,\\ 
5 & diz mære mit \textbf{rehte} sage,\\ 
 & sô komet irs mit mir in klage.\\ 
 & \begin{large}D\end{large}ô sprach der künec Vergulaht:\\ 
 & "hêrre, ich hân mich des bedâht,\\ 
 & ir sult rîten dort hin în.\\ 
10 & mag ez mit iuwern hulden sîn,\\ 
 & ich \textbf{bringiu} nû gesellecheit.\\ 
 & ist aber iu \textbf{mîn} vürbaz rîten leit,\\ 
 & ich lâze\textbf{z}, swaz ich ze schaffenne hân."\\ 
 & Dô sprach der werde Gawan:\\ 
15 & "hêrre, swaz ir gebietet,\\ 
 & billîch ir iuch des nietet.\\ 
 & daz ist ouch âne mînen zorn\\ 
 & mit guotem willen gar verkorn."\\ 
 & Dô sprach der künec von Ascalun:\\ 
20 & "hêrre, ir seht wol Tschampfenzun.\\ 
 & dâ ist mîn swester ûf, ein maget.\\ 
 & swaz munt von schœne hât gesaget,\\ 
 & des hât si volleclîchen teil.\\ 
 & welt ir \textbf{nû} prüeven vür ein heil,\\ 
25 & dêswâr, \textbf{sô} muoz si sich bewegen,\\ 
 & daz si iuwer unz an mich sol pflegen.\\ 
 & ich kume \textbf{i\textit{u}} \textbf{schierer}, \textbf{danne} ich sol.\\ 
 & ouch \textbf{erbît} ir mîn wol,\\ 
 & geseht ir die swester mîn.\\ 
30 & ir\textbf{n ruochtet}, wolt ich langer sîn."\\ 
\end{tabular}
\scriptsize
\line(1,0){75} \newline
T U V W \newline
\line(1,0){75} \newline
\textbf{1} \textit{Majuskel} T  \textbf{7} \textit{Initiale} T U W  \textbf{14} \textit{Majuskel} T  \textbf{19} \textit{Majuskel} T  \newline
\line(1,0){75} \newline
\textbf{3} vremder] Vremede U Eroͤmder W  $\cdot$ vrumte] gefroͤmete V frúnt W \textbf{4} ich] ich vch U (V) (W) \textbf{5} diz] Dise U  $\cdot$ rehte] rechter U (V) W \textbf{7} künec] kuͤne W  $\cdot$ Vergulaht] vergvlaht T Vergulacht U (W) virgulaht V \textbf{11} bringiu] briche úch V (W) \textbf{12} iu] vch nuͦ U  $\cdot$ vürbaz rîten] reiten fúrbas W \textbf{13} lâzez] lase V  $\cdot$ swaz] waz U \textbf{15} swaz] waz U (V) (W) \textbf{16} billîch] Willigleiche W  $\cdot$ iuch] îv T \textbf{17} ouch âne] on allen W \textbf{18} guotem] guͦten V \textbf{19} Ascalun] ascalv̂n T Aschalon U astalún W \textbf{20} Tschampfenzun] scampfenzvn T Tschamfenzon U schanpfanzvn V \textbf{22} swaz] Waz U (W) \textbf{24} ir] ir iz U (V)  $\cdot$ vür] fút W \textbf{26} si] \textit{om.} W  $\cdot$ unz] mit U \textbf{27} iu schierer] ivch scierer T oͮch schierre V eúch schiere W \textbf{28} erbît] erbeit U W erbitent V  $\cdot$ wol] vil wol U W \textbf{30} ruochtet] ruͦchent V  $\cdot$ langer] noch lenger V W \newline
\end{minipage}
\end{table}
\end{document}
