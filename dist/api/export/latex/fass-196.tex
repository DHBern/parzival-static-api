\documentclass[8pt,a4paper,notitlepage]{article}
\usepackage{fullpage}
\usepackage{ulem}
\usepackage{xltxtra}
\usepackage{datetime}
\renewcommand{\dateseparator}{.}
\dmyyyydate
\usepackage{fancyhdr}
\usepackage{ifthen}
\pagestyle{fancy}
\fancyhf{}
\renewcommand{\headrulewidth}{0pt}
\fancyfoot[L]{\ifthenelse{\value{page}=1}{\today, \currenttime{} Uhr}{}}
\begin{document}
\begin{table}[ht]
\begin{minipage}[t]{0.5\linewidth}
\small
\begin{center}*D
\end{center}
\begin{tabular}{rl}
\textbf{196} & als ez \textbf{mîn} lîp volbringen mac."\\ 
 & diu naht het ende unt kom der tac.\\ 
 & Diu \textbf{vrouwe} stuont ûf unt neic.\\ 
 & ir grôzen danc si niht versweic.\\ 
5 & \textbf{dô} sleich \textbf{si} wider lîse.\\ 
 & niemen was dâ sô wîse,\\ 
 & der würde ir \textbf{gêns} dâ gewar,\\ 
 & wan Parzival, der \textbf{lieht} gevar.\\ 
 & \textbf{der} slief niht langer dô dar nâch.\\ 
10 & der sunnen was gein \textbf{der} hœhe gâch.\\ 
 & \textbf{ir glesten} durch die wolken dranc.\\ 
 & dô \textbf{erhôrt} er maneger glocken klanc.\\ 
 & kirchen, münster suochte \textbf{diu} diet,\\ 
 & die Clamide von vreuden schiet.\\ 
15 & \begin{large}Û\end{large}f \textbf{rihte} \textbf{sich} der junge man.\\ 
 & der küneginne kappelân\\ 
 & sanc gote unt sîner vrouwen.\\ 
 & ir gast si \textbf{muose} schouwen,\\ 
 & \textbf{unz} \textbf{daz} der benedi\textit{z} geschach.\\ 
20 & nâch sînem harnasch er sprach.\\ 
 & dâ wart er wol gewâpent în.\\ 
 & er tet \textbf{ouch} ritters ellen schîn\\ 
 & mit \textbf{rehter} manlîcher wer.\\ 
 & dô kom Clamides her\\ 
25 & mit maneger baniere.\\ 
 & Kingrun kom schiere\\ 
 & vor den anderen verre\\ 
 & ûf einem orse von Iserterre.\\ 
 & als \textbf{irz} mære \textbf{hânt} vernomen,\\ 
30 & \textbf{dô} was ouch vür die porten komen\\ 
\end{tabular}
\scriptsize
\line(1,0){75} \newline
D Fr15 \newline
\line(1,0){75} \newline
\textbf{3} \textit{Majuskel} D  \textbf{15} \textit{Initiale} D  \newline
\line(1,0){75} \newline
\textbf{1} \textit{Die Verse 196.1-2 sind auf drei Zeilen aufgeteilt, von denen nur noch die Versanfänge oder Versenden zu lesen sind:} Also ::: / mach ::: / nv cho::: Fr15  \textbf{3} vrouwe] vruͦ::: Fr15 \textbf{14} Clamide] Chlammide D \textbf{19} benediz] bendizt D \textbf{28} Iserterre] :::rre Fr15 \newline
\end{minipage}
\hspace{0.5cm}
\begin{minipage}[t]{0.5\linewidth}
\small
\begin{center}*m
\end{center}
\begin{tabular}{rl}
 & als ez \textbf{mîn} lîp vollebringen mac."\\ 
 & diu naht hât ende und kam der tac.\\ 
 & \begin{large}D\end{large}iu \textbf{vrouwe} stuont ûf und neic.\\ 
 & ir grôzen danc si niht versweic\\ 
5 & \textbf{und} sleich \textbf{dô} wider lîse.\\ 
 & niemen was d\textit{â} sô wîse,\\ 
 & der würde i\textit{r} \textbf{\textit{gê}ns} d\textit{â} gewar,\\ 
 & wan Parcifal, der \textbf{lieht} gevar.\\ 
 & \textbf{der} slief niht langer dô dar nâch.\\ 
10 & der sunnen was gegen hœhe gâch.\\ 
 & \textbf{ir glesten} durch die \textit{wolk}e\textit{n} dranc.\\ 
 & dô \textbf{hôrt} er maniger glocken klanc.\\ 
 & kirchen, münster suohte \textbf{diu} diet,\\ 
 & die Clamide von vröuden schiet.\\ 
15 & ûf \textbf{rih\textit{t}e} \textbf{ouch} \textbf{sich} der junge man.\\ 
 & der küniginne kappelân\\ 
 & sanc got und sîner vrouwen.\\ 
 & ir gast si \textbf{muose} schouwen,\\ 
 & \textbf{unz} \textbf{daz} der benedi\textit{z} geschach.\\ 
20 & nâch sînem harnasch er sprach.\\ 
 & dâ wart er wol gewâpent în.\\ 
 & er tet \textbf{ouch} ritters ellen schîn\\ 
 & mit \textbf{rehter} manlîcher wer.\\ 
 & dô kam Clamides her\\ 
25 & mit maniger baniere.\\ 
 & Kingr\textit{un} kam \textbf{ouch} schiere\\ 
 & vor den anderen verre\\ 
 & ûf einem rosse von Iserterre.\\ 
 & \begin{large}A\end{large}ls \textbf{ich es} mære \textbf{hân} vernomen,\\ 
30 & \textbf{dô} was ouch vür die porten komen\\ 
\end{tabular}
\scriptsize
\line(1,0){75} \newline
m n o Fr69 \newline
\line(1,0){75} \newline
\textbf{3} \textit{Illustration mit Überschrift:} Also die kv́nnigin kam geslichen v́ber parcifals bette do er an lag vnd slieff n (o)   $\cdot$ \textit{Initiale} m n o  \textbf{29} \textit{Initiale} m Fr69   $\cdot$ \textit{Capitulumzeichen} n  \newline
\line(1,0){75} \newline
\textbf{1} mac] kan vnd mag n \textbf{2} hât] hette o \textbf{6} dâ] do m n o  $\cdot$ sô] zuͦ o \textbf{7} ir gêns] ir ge geuͯnes m irs gundez o  $\cdot$ dâ] do m n o \textbf{9} dô dar nâch] do noch n darnoch o \textbf{11} wolken] sunne m volken o \textbf{13} suohte] suͯchet n (Fr69) \textbf{15} rihte] riche m \textbf{16} küniginne] kv́nniginnen n (o) \textbf{17} sîner] ir n \textbf{18} si] \textit{om.} n  $\cdot$ muose] muͯse m (n) muͯstent o \textbf{19} daz] \textit{om.} Fr69  $\cdot$ benediz] beneditt m benedig n (o)  $\cdot$ geschach] gesach o \textbf{22} tet] wardert o  $\cdot$ ellen] allen o \textbf{24} Clamides] clamidez o \textbf{26} Kingrun] Kingrim m Konigrim o \textbf{27} vor] Von n o \textbf{28} ûf] Von o  $\cdot$ Iserterre] yserterre Fr69 \textbf{29} ich es] ich dise n ich disz o ichs Fr69 \textbf{30} porten] porte n (o) \newline
\end{minipage}
\end{table}
\newpage
\begin{table}[ht]
\begin{minipage}[t]{0.5\linewidth}
\small
\begin{center}*G
\end{center}
\begin{tabular}{rl}
 & als ez \textbf{mîn} lîp volbringen mac."\\ 
 & diu naht het ende unde kom der tac.\\ 
 & diu \textbf{vrouwe} stuont ûf unde neic.\\ 
 & ir grôzen danc si niht versweic.\\ 
5 & \textbf{dô} sleich \textbf{si} wider lîse.\\ 
 & niemen was dâ sô wîse,\\ 
 & der würde ir \textbf{gênes} dâr gewar,\\ 
 & wan Parcival, der \textbf{wol} gevar.\\ 
 & \textbf{er} slief niht lenger dô dar nâch.\\ 
10 & der sunnen was gein hœhe gâch.\\ 
 & \textbf{ir glesten} dur die wolken dranc.\\ 
 & dô \textbf{hôrt}er maniger glocken klanc.\\ 
 & \begin{large}K\end{large}irchen, münster suoht \textbf{diu} diet,\\ 
 & die Clamide von vröuden schiet.\\ 
15 & ûf \textbf{rihte} \textbf{sich} der junge man.\\ 
 & der küniginne kappelân\\ 
 & sanc got unde sîner vrouwen.\\ 
 & ir gast si \textbf{wolte} schouwen,\\ 
 & \textbf{biz} der benediz geschach.\\ 
20 & nâch sînem harnasch er sprach.\\ 
 & dâ wart er wol gewâpent în.\\ 
 & er tet \textbf{dô} rîters ellen schîn\\ 
 & mit \textbf{rehte} manlîcher wer.\\ 
 & dô kom Clamides her\\ 
25 & mit maniger baniere.\\ 
 & Kingrun kom schiere\\ 
 & vor den andern verre\\ 
 & ûf einem orse von Yserterre.\\ 
 & als \textbf{ich daz} mære \textbf{hân} vernomen,\\ 
30 & \textbf{dô} was ouch vür die borte komen\\ 
\end{tabular}
\scriptsize
\line(1,0){75} \newline
G I O L M Q R Z \newline
\line(1,0){75} \newline
\textbf{13} \textit{Initiale} G M  \textbf{15} \textit{Initiale} I  \textbf{21} \textit{Initiale} R Z  \textbf{27} \textit{Initiale} Q  \textbf{29} \textit{Initiale} L  \newline
\line(1,0){75} \newline
\textbf{1} als] al I  $\cdot$ volbringen] wol bringen O von enden Z \textbf{2} naht] nach R  $\cdot$ het] had M (Q)  $\cdot$ unde] do I da M \textbf{3} \textit{Versfolge 196.4-3} M   $\cdot$ vrouwe] kunginne I (O) (L) (M) (Q) (R) (Z)  $\cdot$ neic] gienk R \textbf{4} grôzen] grosser R  $\cdot$ versweic] verschek R \textbf{5} dô] Da M Z  $\cdot$ lîse] liuse R \textbf{6} don was hie nieman so wise I  $\cdot$ niemen] Daz nieman L  $\cdot$ dâ] \textit{om.} O L R do Q \textbf{7} Da der ir gens wuͯrde gewar L  $\cdot$ Der ir ganges do wurd gewar R  $\cdot$ gênes] gensh I (O) (L) (M) (Z) \textbf{8} Parcival] parzival G parzifal I L M Parcifal O (Z) partzifal Q parczifal R \textbf{9} er] ern I (M) (Z) Der O L R Dern Q  $\cdot$ lenger] lange L  $\cdot$ dô] \textit{om.} I M R \textbf{10} der sunnen] Der svnne O Die svnne M Den súnnen Q (Z)  $\cdot$ hœhe] hohen Q huͦe R \textbf{11} glesten] glast O L Q geleiste R \textbf{12} dô] Da M Z  $\cdot$ hôrter] hort er I Q Z hort man O  $\cdot$ maniger] mangen I O L (M) (Q) (R) (Z) \textbf{13} suoht] suͯchte L suchten M (Q) suͦchent R  $\cdot$ diu] \textit{om.} M \textbf{14} Clamide] chlamide I Glamide O Clamides L  $\cdot$ von] vor Q \textbf{15} rihte] riht I O  $\cdot$ junge] Iungen R \textbf{16} küniginne] kvnige M \textbf{17} sanc] Sant R \textbf{18} ir] Jrn L Z  $\cdot$ gast si] geist so R  $\cdot$ wolte] wolden Z \textbf{19} biz der] biz daz I Biz daz der O (M) (Q) (R) Z Vntz daz der L  $\cdot$ benediz] benedicite O benedicte L \textbf{20} nâch] Nam Q  $\cdot$ er] er do I L R \textbf{22} dô] avch O (L) (Q) doch M R Z  $\cdot$ ellen] eren Q \textbf{23} rehte] rehter I O (L) (M) (Q) (R) (Z) \textbf{24} dô] Da M  $\cdot$ kom] kom auch L  $\cdot$ Clamides] Glamides O  $\cdot$ her] [wer]: her M \textbf{25} maniger] manchem Q  $\cdot$ baniere] banieren R \textbf{26} Kingrun] chingrun I Kyngrvͯn L Kyngrún Q [Kyngun]: Kyngrun R \textbf{27} vor] von I  $\cdot$ den] der M \textbf{28} Yserterre] isenterre I ysen terre O iserterre L M ysentere Q [*]: Jserterre R [voniser]: von iser terre Z \textbf{29} ich daz] ich die L ich Q \textbf{30} dô was ouch] Ouch waz L Nu was auch Q Nun was R Da was ovch Z  $\cdot$ vür] [dvr]: fvr O  $\cdot$ borte] porten I O L R phorten M (Q) \newline
\end{minipage}
\hspace{0.5cm}
\begin{minipage}[t]{0.5\linewidth}
\small
\begin{center}*T
\end{center}
\begin{tabular}{rl}
 & alsez \textbf{der} lîp vollebringen mac."\\ 
 & di\textit{u} naht hete ende unde kom der tac.\\ 
 & Diu \textbf{künegîn} stuont ûf unde neic.\\ 
 & ir grôzen danc si niht versweic\\ 
5 & \textbf{unde} sleich \textbf{hin} wider lîse.\\ 
 & niemen was dâ sô wîse,\\ 
 & der würde ir \textbf{ganges} dâ gewar,\\ 
 & wan Parcifal, der \textbf{lieht} gevar.\\ 
 & \textbf{der} slief niht langer dô dar nâch.\\ 
10 & der sunnen was gegen hœhe gâch.\\ 
 & \textbf{der glast} durch die wolken dranc.\\ 
 & dô \textbf{erhôrt}er maneger glocken klanc.\\ 
 & kirchen, münster suochte \textbf{ein} diet,\\ 
 & die Clamide von vröuden schiet.\\ 
15 & ûf \textbf{stuont} \textbf{dô} der junge man.\\ 
 & Der küneginne kappelân\\ 
 & sanc gote unde sîner vrouwen.\\ 
 & ir gast si \textbf{muose} schouwen,\\ 
 & \textbf{unze} \textbf{daz} der benediz geschach.\\ 
20 & nâch sînem harnasch er \textbf{dô} sprach.\\ 
 & dâ wart er wol gewâpent în.\\ 
 & er tet \textbf{ouch} rîters ellen schîn\\ 
 & mit manlîcher wer.\\ 
 & dô kom Clamides her\\ 
25 & mit maneger baniere.\\ 
 & Kyngrun kom schiere\\ 
 & vor den andern verre\\ 
 & ûf einem orse von Isenterre.\\ 
 & \begin{large}A\end{large}ls \textbf{ich daz} mære \textbf{hân} vernomen,\\ 
30 & \textbf{nû} was ouch vür die porten komen\\ 
\end{tabular}
\scriptsize
\line(1,0){75} \newline
T U V W \newline
\line(1,0){75} \newline
\textbf{3} \textit{Majuskel} T  \textbf{16} \textit{Majuskel} T  \textbf{29} \textit{Initiale} T U V W  \newline
\line(1,0){75} \newline
\textbf{1} der] [*]: min V  $\cdot$ vollebringen] verbringen W \textbf{2} diu] die T \textbf{6} dâ] do U V \textit{om.} W \textbf{7} würde ir ganges dâ] wuͦrde irs ganes do U (V) irs ganges wúrd W \textbf{8} Parcifal] parzifal V partzifal W \textbf{9} dar nâch] er lag W \textbf{10} Von sunnen was es liechter tag W  $\cdot$ hœhe] hohen U \textbf{11} der] Ir W \textbf{12} erhôrter maneger] horter manegen U (W) hort er manger V \textbf{13} suochte ein] svͦchten die V suͦchte gar die W \textbf{14} die] [D*]: Do U  $\cdot$ Clamide] klamide W \textbf{17} sanc] Der sang W \textbf{18} ir] Jrn V (W)  $\cdot$ muose] mvese T [*]: wolte V \textbf{19} unze] Mit U Biß W  $\cdot$ benediz geschach] benedic ward getan W \textbf{20} er dô sprach] hies er gan W \textbf{23} manlîcher] rechter manlicher U W manger manlicher V \textbf{24} kom] kam oͮch V (W)  $\cdot$ Clamides] klamides W \textbf{26} Kyngrun] Kyngruͦn U Kẏngrun V Kingrun W  $\cdot$ kom] kam oͮch V (W) \textbf{28} Isenterre] Jsenterrre T U ysenterre V W \textbf{29} daz] die U dis W \textbf{30} was] ist W  $\cdot$ porten] porte U V \newline
\end{minipage}
\end{table}
\end{document}
