\documentclass[8pt,a4paper,notitlepage]{article}
\usepackage{fullpage}
\usepackage{ulem}
\usepackage{xltxtra}
\usepackage{datetime}
\renewcommand{\dateseparator}{.}
\dmyyyydate
\usepackage{fancyhdr}
\usepackage{ifthen}
\pagestyle{fancy}
\fancyhf{}
\renewcommand{\headrulewidth}{0pt}
\fancyfoot[L]{\ifthenelse{\value{page}=1}{\today, \currenttime{} Uhr}{}}
\begin{document}
\begin{table}[ht]
\begin{minipage}[t]{0.5\linewidth}
\small
\begin{center}*D
\end{center}
\begin{tabular}{rl}
\textbf{793} & \textit{\begin{large}S\end{large}}ît uns der jâmerstric beslôz.\\ 
 & habt \textbf{stille}, uns næhet vreude grôz."\\ 
 & Feirefiz Anschevin\\ 
 & mante Parzivale\textit{n}, den bruoder sîn,\\ 
5 & \textbf{an} der selben zîte.\\ 
 & \textbf{er} gâhete geime strîte.\\ 
 & Cundrie \textbf{in mit dem zoume} vienc.\\ 
 & sîner tjost \textbf{dâ} niht ergienc.\\ 
 & Dô sprach diu \textbf{magt} \textbf{rûch} gemâl\\ 
10 & balde zir hêrren Parzival:\\ 
 & "schilde und baniere\\ 
 & \textbf{möht ir} \textbf{erkennen} schiere.\\ 
 & \textbf{dort} habt niht wans Grâles schar.\\ 
 & die sint vil diensthaft iu gar."\\ 
15 & Dô sprach der werde heiden:\\ 
 & "sô \textbf{sî} der strît gescheiden."\\ 
 & Parzival Cundrien bat\\ 
 & gein \textbf{in} rîten ûf den pfat.\\ 
 & diu reit und sagete in mære,\\ 
20 & waz in vreuden komen wære.\\ 
 & Swaz dâ templeise was,\\ 
 & die erbeizten nider ûfez gras.\\ 
 & an den selben stunden\\ 
 & manec helm wart ab gebunden.\\ 
25 & \textbf{Parzivaln} enpfiengen si ze vuoz.\\ 
 & ein segen dûhte si sîn gruoz.\\ 
 & si enpfiengen ouch Feirefizen,\\ 
 & den swarzen unt den wîzen.\\ 
 & ûf Munsalvæsche \textbf{wart} geriten\\ 
30 & al weinende unt \textbf{doch} mit vreude siten.\\ 
\end{tabular}
\scriptsize
\line(1,0){75} \newline
D \newline
\line(1,0){75} \newline
\textbf{1} \textit{Initiale} D  \textbf{9} \textit{Majuskel} D  \textbf{15} \textit{Majuskel} D  \textbf{21} \textit{Majuskel} D  \newline
\line(1,0){75} \newline
\textbf{1} Sît] ÷it D \textbf{3} Anschevin] Anscevin D \textbf{4} Parzivalen] Parcivale D \textbf{10} Parzival] Parcifal D \textbf{17} Parzival] Parcifal D \textbf{25} Parzivaln] Parcifaln D \textbf{29} Munsalvæsche] Mvnsalvæsce D \newline
\end{minipage}
\hspace{0.5cm}
\begin{minipage}[t]{0.5\linewidth}
\small
\begin{center}*m
\end{center}
\begin{tabular}{rl}
 & sît un\textit{s} der jâmerstric beslôz.\\ 
 & habet \textbf{stille}, uns nâhet vröude grôz."\\ 
 & Ferefiz A\textit{n}schevin\\ 
 & \textit{m}ante Parcifaln, den bruoder sîn,\\ 
5 & \textbf{in} der selben zîte\\ 
 & \textbf{und} gâhete gegen dem strîte.\\ 
 & Condri\textit{e} \textbf{\textit{m}it dem z\textit{o}um in} vienc,\\ 
 & \textbf{daz} sîner just \textbf{d\textit{â}} niht ergienc.\\ 
 & dô sprach diu \textbf{rîch} gemâl\\ 
10 & balde zuo ir hêrren Parcifal:\\ 
 & \hspace*{-.7em}\big| "\textbf{ir mögt} \textbf{er\textit{k}e\textit{nn}en} schier\\ 
 & \hspace*{-.7em}\big| schilt und banier.\\ 
 & \textbf{dort} habt \textit{n}i\textit{ht} wan des Grâles schar.\\ 
 & die sint vil diensthaft iu gar."\\ 
15 & dô sprach der werde \textit{h}e\textit{id}en:\\ 
 & "sô \textbf{sî} der strît gescheid\textit{e}n."\\ 
 & Parcifal Condrien bat\\ 
 & gegen rîten ûf den pfat.\\ 
 & diu reit und sagte in mære,\\ 
20 & waz in vröuden komen wære.\\ 
 & waz d\textit{â} templ\textit{ei}se was,\\ 
 & die erbeizten nider ûf daz gras.\\ 
 & an den selben stunden\\ 
 & manic helm wart ab gebunden.\\ 
25 & \textbf{Parcifaln} enpfiengen si zuo vuoz.\\ 
 & ein segen dûhte si \textit{s}în gruoz.\\ 
 & si enpfiengen ouch Ferefizen,\\ 
 & den swarzen und den wîzen.\\ 
 & ûf Muntsalvasche \textbf{dô wart} geriten\\ 
30 & al weinende und \textbf{doch} mit vröuden siten.\\ 
\end{tabular}
\scriptsize
\line(1,0){75} \newline
m n o V V' W \newline
\line(1,0){75} \newline
\textbf{3} \textit{Initiale} V V'  \textbf{12} \textit{Initiale} W  \textbf{29} \textit{Überschrift:} Hie kummet parzefal vnde sin bruͦder feruis anscheuin vnde kv́nig artus vnde die Tauelrunder alle zvͦ Muntsalfasche zvͦ dem Grole V   $\cdot$ \textit{Initiale} V  \newline
\line(1,0){75} \newline
\textbf{1} uns] vnd m  $\cdot$ der jâmerstric] dez iomers strig V' \textbf{2} stille] \textit{om.} V' \textbf{3} Ferefiz] Ferefis m o Ferrefis o Fereuis V Ferevis V' Ferafis W  $\cdot$ Anschevin] auscevin m n ansce vin o anschefin V antscheuein W \textbf{4} mante] Nante m  $\cdot$ Parcifaln] parcifalen n parcifal o artusen [*]: vnd parzefal V artusen vnd V' partzifaln W  $\cdot$ den] dem o \textbf{5} in] An V V'  $\cdot$ der] dem V' \textbf{6} gâhete] gahet V'  $\cdot$ dem] \textit{om.} o \textbf{7} Condrie mit] Condrie vnd mit m Cundrie mit o Kvndrie in mit V V' Kundrie mit W  $\cdot$ zoum] zum m  $\cdot$ in] sie o \textit{om.} V V' \textbf{8} sîner] sin V' (W)  $\cdot$ just dâ] just do m (n) o (V) (W) [*]: strit  V' \textbf{9} diu] die magt W  $\cdot$ rîch gemâl] richgemale V (V') \textbf{10} ir] jren o  $\cdot$ Parcifal] parzefale V parzifale V' partzifal W \textbf{12} mögt] moͤhtent V mochten V'  $\cdot$ erkennen] erschemen m \textbf{11} schilt] Jr schilt V' \textbf{13} niht] mir m \textbf{15} dô] Der o  $\cdot$ heiden] tegen m heidin o \textbf{16} gescheiden] [gescheidet]: gescheidetn m \textbf{17} Parcifal] Parzefal V Parzifal V' Herr partzifal W  $\cdot$ Condrien] Cuͯndrien o kvndrien V V' (W) \textbf{18} gegen] Gegen in V (V') W  $\cdot$ den] dem o \textbf{19} sagte] sagt W \textbf{20} in vröuden] freuden in V' \textbf{21} \textit{Vers 793.21 nach 793.23:} Was do templeis kunden n   $\cdot$ was] Swaz V  $\cdot$ dâ] do m o V V' W  $\cdot$ templeise] templies m \textbf{22} die] \textit{om.} V V'  $\cdot$ ûf daz] uffens V in das W \textbf{23} stunden] stunden was n \textbf{24} ab] do abe n \textbf{25} Parcifaln] Parcifalen n Parzefaln V Parzifaln V' Partzifaln W \textbf{26} ein] Sein W  $\cdot$ sîn] ein m \textbf{27} Ferefizen] ferrefisen n fereficzes o artusen vnd oͮch [ferevi*]: ferevissen V artusen vnd ferevisen V' ferafissen W \textbf{28} \textit{nach 793.28:} Vnde die touelrunder alle gar / Die mit artuse worent kummen dar V (V')  \textbf{29} Muntsalvasche] muntsaluasce m o montsoluasce n Munschalfasche V (V') montsaluatsch W  $\cdot$ dô wart] wart do V wart V' \textbf{30} doch] ouch V'  $\cdot$ vröuden] guten V' \newline
\end{minipage}
\end{table}
\newpage
\begin{table}[ht]
\begin{minipage}[t]{0.5\linewidth}
\small
\begin{center}*G
\end{center}
\begin{tabular}{rl}
 & \begin{large}S\end{large}ît uns der jâmers stric beslôz.\\ 
 & habet \textbf{ûf}, uns nâhet vröude grôz."\\ 
 & Feirafiz Anschevin\\ 
 & mante Parzivalen, den bruoder sîn,\\ 
5 & \textbf{an} der selben zîte\\ 
 & \textbf{unde} gâhte \textbf{ouch} gein dem strîte.\\ 
 & Gundrie \textbf{in mit dem zo\textit{um}e} vienc,\\ 
 & \textbf{daz} sîner tjoste niht ergienc.\\ 
 & dô sprach diu \textbf{maget} \textbf{rûch} gemâl\\ 
10 & balde zir hêrren Parzival:\\ 
 & "schilt unde baniere\\ 
 & \textbf{m\textit{ö}ht ir} \textbf{erkennen} schiere.\\ 
 & \textbf{hie}\textbf{ne} habt niht wan des Grâles schar.\\ 
 & die sint vil diensthaft iu gar."\\ 
15 & dô sprach der werde heiden:\\ 
 & "sô \textbf{sî} der strît gescheiden."\\ 
 & Parzival Gundrien bat\\ 
 & gein \textbf{in} rîten ûf den pfat.\\ 
 & diu reit unde seit in mære,\\ 
20 & waz in vröude komen wære.\\ 
 & \multicolumn{1}{l}{ - - - }\\ 
 & \multicolumn{1}{l}{ - - - }\\ 
 & an den selben stunden\\ 
 & manic helm wart abe gebunden.\\ 
25 & \textbf{ir hêrren} enpfiengen si ze vuoz.\\ 
 & ein segen dûhte si sîn gruoz.\\ 
 & si enpfiengen ouch Feirafizzen,\\ 
 & den swarzen unde den wîzen.\\ 
 & ûf Muntsalfatsche \textbf{wart dô} geriten\\ 
30 & alweinde unde \textbf{doch} mit vröuden siten.\\ 
\end{tabular}
\scriptsize
\line(1,0){75} \newline
G I L M Z \newline
\line(1,0){75} \newline
\textbf{1} \textit{Initiale} G I L Z  \textbf{3} \textit{Initiale} M  \textbf{15} \textit{Initiale} L  \textbf{17} \textit{Initiale} I  \newline
\line(1,0){75} \newline
\textbf{1} jâmers stric] iamerstric I (L) (M) iamers strit Z \textbf{2} habet] hap I  $\cdot$ nâhet] nahent I \textbf{3} Feirafiz] Ferefiz L Ferefisz M Feirefiz Z  $\cdot$ Anschevin] entsheuin I Anshevin L Z Ansevin M \textbf{4} mante] mant I (Z)  $\cdot$ Parzivalen] parcifalen G parzifaln I L parzifal M parcifal Z \textbf{6} gâhte] Gaht I (Z) sprach M \textbf{7} Gundrie] Kvndrie L Z Kundri M  $\cdot$ in] \textit{om.} Z  $\cdot$ zoume] zorne G zovm in Z \textbf{8} niht] da niht L (M) Z \textbf{9} dô] Da M Z \textbf{10} balde] Salde M  $\cdot$ Parzival] parcifal G Z parzifal I L M \textbf{12} möht] moht G L (M) Z  $\cdot$ erkennen] bechennen I (L) (M) \textbf{13} hiene] Hie L \textbf{14} die sint] diu ist I Die sin L  $\cdot$ vil diensthaft iu] ev vil diensthaft I \textbf{15} dô] Da M Z \textbf{16} sô sî] Do ist M \textbf{17} \textit{Die Verse 793.17-24 fehlen} L   $\cdot$ Parzival] parcifal G (Z) Parzifal I M  $\cdot$ Gundrien] kundrien M (Z) \textbf{19} seit] seite M \textbf{20} vröude] vrouden M \textbf{21} \textit{Die Verse 793.21-22 fehlen} G I L M Z  \textbf{24} wart] in I \textit{om.} M \textbf{25} ir] wart irn I \textbf{26} ein] [sin]: ein I  $\cdot$ dûhte] duͯrch L \textbf{27} Feirafizzen] firafizzen G Feirafizen I ferefizen L feirefizzen M feirefizen Z \textbf{28} unde] nih I \textbf{29} Muntsalfatsche] muntshaluasche I Mvntsalvatsche L Musalvatsch M montsalvatsch Z  $\cdot$ dô] da M Z \textbf{30} alweinde] Weinde L  $\cdot$ doch] \textit{om.} L  $\cdot$ vröuden] frovde L (Z) \newline
\end{minipage}
\hspace{0.5cm}
\begin{minipage}[t]{0.5\linewidth}
\small
\begin{center}*T
\end{center}
\begin{tabular}{rl}
 & sît uns der jâmers stric beslôz.\\ 
 & habet \textbf{ûf}, uns \textit{nâhet} vreude grôz."\\ 
 & \begin{large}F\end{large}erefis Anschevin\\ 
 & mante Parcifaln, den bruoder sîn,\\ 
5 & \textbf{an} der selben zîte\\ 
 & \textbf{und} gâhte \textbf{ouch} gein dem strîte.\\ 
 & Kundrie \textbf{in mit dem zoume} vienc,\\ 
 & \textbf{daz} sîner jost \textbf{dâ} niht ergienc.\\ 
 & dô sprach diu \textbf{maget} \textbf{\textit{rû}ch} gemâl\\ 
10 & balde zuo ir hêrre\textit{n} Parcifal:\\ 
 & "s\textit{c}hilte und baniere\\ 
 & \textbf{m\textit{ö}ht ir} \textbf{bekennen} schiere.\\ 
 & \textbf{hie} \textbf{en}habet niht wan des Grâles schar.\\ 
 & die sint vil diensthaft iu gar."\\ 
15 & dô sprach der werde heiden:\\ 
 & "sô \textbf{ist} der strît gescheiden."\\ 
 & Parcifal Kundrien bat\\ 
 & gein \textbf{in} rîten ûf den pfat.\\ 
 & diu reit und sagte in mære,\\ 
20 & waz in vreuden komen wære.\\ 
 & waz dâ templeise was,\\ 
 & die erbeizeten nider ûf daz gras.\\ 
 & an den selben stunden\\ 
 & manec helm wart abe gebunden.\\ 
25 & \textbf{ir hêrren} enpfiengen si zuo vuoz.\\ 
 & ein segen dûhte si \textit{s}în gruoz.\\ 
 & si enpfiengen ouch Ferefisen,\\ 
 & den swarzen und den wîzen.\\ 
 & ûf Munsalvasche \textbf{wart dô} geriten\\ 
30 & a\textit{l} weinde und \textbf{ouch} mit vreude siten.\\ 
\end{tabular}
\scriptsize
\line(1,0){75} \newline
U Q R \newline
\line(1,0){75} \newline
\textbf{3} \textit{Initiale} U  \textbf{9} \textit{Initiale} R  \newline
\line(1,0){75} \newline
\textbf{1} der jâmers stric] des iamers streit Q der Jamer stric R  $\cdot$ beslôz] blos R \textbf{2} ûf] \textit{om.} Q  $\cdot$ nâhet] \textit{om.} U  $\cdot$ vreude] froͯrde R \textbf{3} Ferefis] feirefisz Q Feriefis R  $\cdot$ Anschevin] anshevin Q [as*]: anscheuin R \textbf{4} Parcifaln] Parzifaln U partzifaln Q parczifal R  $\cdot$ den bruoder sîn] \textit{om.} R \textbf{5} selben] selbe R \textbf{7} Kundrie] Kuͦndrie U Kúndrie Q Kondire R  $\cdot$ zoume] zeme Q \textbf{8} sîner] sine R  $\cdot$ dâ] do Q R \textbf{9} rûch] hoch U \textbf{10} zuo ir] zur ym Q  $\cdot$ hêrren] herre U  $\cdot$ Parcifal] Parzifal U partzifal Q Barczifal R \textbf{11} schilte] So hilte U \textbf{12} Mich erkennent schiere R  $\cdot$ möht] Mocht U Q \textbf{13} enhabet] habt R  $\cdot$ des] \textit{om.} R \textbf{14} sint] sine R  $\cdot$ iu] auch Q \textbf{17} Parcifal] Parzifal U Partzifal Q Parczifal R  $\cdot$ Kundrien] kuͦndrien U kundrian Q kundrye R \textbf{18} den] daz R \textbf{19} sagte] seit Q (R) \textbf{20} vreuden] frewde Q (R) \textbf{21} dâ] do Q R  $\cdot$ templeise] templeiser Q \textbf{22} die] Do R  $\cdot$ erbeizeten] erbeiste Q enbeiczent R \textbf{26} sîn] ein U \textbf{27} Ferefisen] feirefizzen Q feirefizen R \textbf{29} Munsalvasche] muͦntsalvatschen U muntsaluasche Q Munsaluasche R  $\cdot$ dô] \textit{om.} R \textbf{30} al] Alle U  $\cdot$ ouch] doch Q  $\cdot$ vreude] [vende]: veinde Q froͯden R \newline
\end{minipage}
\end{table}
\end{document}
