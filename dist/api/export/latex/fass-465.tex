\documentclass[8pt,a4paper,notitlepage]{article}
\usepackage{fullpage}
\usepackage{ulem}
\usepackage{xltxtra}
\usepackage{datetime}
\renewcommand{\dateseparator}{.}
\dmyyyydate
\usepackage{fancyhdr}
\usepackage{ifthen}
\pagestyle{fancy}
\fancyhf{}
\renewcommand{\headrulewidth}{0pt}
\fancyfoot[L]{\ifthenelse{\value{page}=1}{\today, \currenttime{} Uhr}{}}
\begin{document}
\begin{table}[ht]
\begin{minipage}[t]{0.5\linewidth}
\small
\begin{center}*D
\end{center}
\begin{tabular}{rl}
\textbf{465} & \begin{large}V\end{large}on Adames künne\\ 
 & huop sich \textbf{riwe} und wünne,\\ 
 & sît er uns sippe lougent niht,\\ 
 & den ieslîch engel ob im siht,\\ 
5 & unt \textbf{daz} diu sippe ist \textbf{sünden} wagen,\\ 
 & \textbf{sô} daz wir \textbf{sünde} müezen tragen.\\ 
 & dar über erbarme sich \textbf{des} kraft,\\ 
 & dem \textbf{erbarme} gît geselleschaft,\\ 
 & sît sîn getriuwiu menscheit\\ 
10 & mit triwen gein untriwe streit.\\ 
 & ir sult ûf in verkiesen,\\ 
 & welt ir sælde niht verliesen.\\ 
 & lât wandel iu vür \textbf{sünde} bî.\\ 
 & sît rede unt werke niht sô vrî.\\ 
15 & wan \textbf{der} sîn leit sô rîchet,\\ 
 & daz er unkiusche sprichet,\\ 
 & von des lône tuon ich iu kunt,\\ 
 & \textbf{in} \textbf{urteilt} sîn se\textit{l}bes munt.\\ 
 & nemt altiu mære vür niwe,\\ 
20 & ob si iuch \textbf{lêren} triwe.\\ 
 & Der pareliure Plato\\ 
 & sprach bî sînen zîten \textbf{dô}\\ 
 & unt Sibille, diu prophêtisse,\\ 
 & sunder fâlierens misse.\\ 
25 & si sagten dâ vor \textbf{manec} jâr,\\ 
 & uns solde komen al vür wâr\\ 
 & vür die hœhsten schulde pfant.\\ 
 & ze\textbf{r} helle uns nam diu hœhste hant\\ 
 & mit der götlîchen minne.\\ 
30 & die unkiuschen liez er dinne.\\ 
\end{tabular}
\scriptsize
\line(1,0){75} \newline
D \newline
\line(1,0){75} \newline
\textbf{1} \textit{Initiale} D  \textbf{21} \textit{Majuskel} D  \newline
\line(1,0){75} \newline
\textbf{18} selbes] sebes D \newline
\end{minipage}
\hspace{0.5cm}
\begin{minipage}[t]{0.5\linewidth}
\small
\begin{center}*m
\end{center}
\begin{tabular}{rl}
 & von Adames k\textit{ünn}e\\ 
 & huop sich \textbf{riuwe} und wünne,\\ 
 & sît er uns sippe lougent niht,\\ 
 & den ieglîch engel ob im siht,\\ 
5 & und \textbf{daz} diu sippe ist \textbf{sünden} wagen,\\ 
 & \textbf{sus} daz wir \textbf{sünde} müezen tragen.\\ 
 & dar über \textbf{sô} erbarme sich \textbf{sîn} kraft,\\ 
 & dem \textbf{erbermde} gît geselleschaft,\\ 
 & sît sîn getriuwiu menscheit\\ 
10 & mit triuwen gegen \textit{untriuw}e\textit{n} streit.\\ 
 & ir solt ûf in verkiesen,\\ 
 & wolt ir sælde niht verliesen.\\ 
 & lât wandel iu v\textit{ür} \textbf{sünden} \textit{b}î.\\ 
 & sît rede und werke niht sô vrî.\\ 
15 & wan \textbf{wer} sîn leit sô rîchet,\\ 
 & daz er unkiusche \textit{s}prichet,\\ 
 & von des lôn tuon ich iu kunt,\\ 
 & \textbf{in} \textbf{urteilet} sîn selbes munt.\\ 
 & nemet altiu mær vür niuwe,\\ 
20 & ob si iuch \textbf{lêren} triuwe.\\ 
 & der pa\textit{r}e\textit{l}iu\textit{r}e Plato\\ 
 & sprach bî sînen zîten \textbf{dô}\\ 
 & und Sybille, diu prophêt\textit{i}s\textit{s}e,\\ 
 & sunder fâl\textit{i}er\textit{en}s \textit{m}is\textit{s}e.\\ 
25 & s\textit{i} sageten dô vo\textit{r} \textbf{manegem} jâr,\\ 
 & uns solte komen al vür wâr\\ 
 & vür die hœhsten schulde pfant.\\ 
 & zuo\textbf{r} helle uns nam diu hœhst\textit{e} hant\\ 
 & mit der götlîche\textit{n} minne.\\ 
30 & die unkiuschen liez er dâr inne.\\ 
\end{tabular}
\scriptsize
\line(1,0){75} \newline
m n o \newline
\line(1,0){75} \newline
\newline
\line(1,0){75} \newline
\textbf{1} Adames] adams m n o  $\cdot$ künne] kome m \textbf{3} uns] \sout{sit} vns o \textbf{6} sus] So n o  $\cdot$ müezen] muͯsse o \textbf{7} sô] \textit{om.} n o  $\cdot$ sîn] des n o \textbf{8} gît] giht o \textbf{9} menscheit] [menheit]: menscheit o \textbf{10} untriuwen] yme m \textbf{13} vür] von m  $\cdot$ sünden] súnde n (o)  $\cdot$ bî] frẏ m \textbf{16} sprichet] prichet m \textbf{19} mær] were n  $\cdot$ niuwe] nuͯwen o \textbf{20} triuwe] trúwen o \textbf{21} pareliure] pauelriue m pauelure n panelúre o  $\cdot$ Plato] plane o \textbf{23} Sybille] sibille n o  $\cdot$ prophêtisse] broffitichsie m [phitiste]: philiste o \textbf{24} fâlierens] valerius m walierens o  $\cdot$ misse] wise m \textbf{25} si] So m Zú n  $\cdot$ vor] von m  $\cdot$ manegem] manig n o \textbf{26} uns] Vnd n  $\cdot$ solte komen] solten keinen o  $\cdot$ al] also n (o) \textbf{28} zuor] Zú o  $\cdot$ hœhste] hohsten m (n) (o) \textbf{29} götlîchen] gottlicher m \newline
\end{minipage}
\end{table}
\newpage
\begin{table}[ht]
\begin{minipage}[t]{0.5\linewidth}
\small
\begin{center}*G
\end{center}
\begin{tabular}{rl}
 & \begin{large}V\end{large}on Adames künne\\ 
 & huop sich \textbf{riuwe} unde wünne,\\ 
 & sît er uns sippe lougent niht,\\ 
 & den ieslîch engel ob im siht,\\ 
5 & unt \textbf{daz} diu sippe ist \textbf{sünden} wagen,\\ 
 & \textbf{sô} daz wir \textbf{sünden} müezen tragen.\\ 
 & dar über erbarme sich \textbf{sîn} kraft,\\ 
 & dem \textbf{erbermde} gît ge\textit{s}elleschaft,\\ 
 & sît sîniu getriuwiu menscheit\\ 
10 & mit triuwen gein untriuwen streit.\\ 
 & ir sult ûf in verkiesen,\\ 
 & welt ir s\textit{æ}l\textit{de} niht verliesen.\\ 
 & lât wandel iu vür \textbf{sünde} bî.\\ 
 & sît rede unde werke niht sô vrî.\\ 
15 & wan \textbf{swer} sîn leit sô rîchet,\\ 
 & daz er unkiusche sprichet,\\ 
 & von des lône tuon ich iu kunt,\\ 
 & \textbf{in} \textbf{verteilt} sî\textit{n} selbes munt.\\ 
 & nemet altiu mær vür niuwe,\\ 
20 & op si iuch \textbf{lêren} triuwe.\\ 
 & der pareliure Plato\\ 
 & sprach bî sînen zîten \textbf{sô}\\ 
 & unde Sibille, diu prophêtisse,\\ 
 & sunder fâlierens misse.\\ 
25 & si sageten dâ vor \textbf{manic} jâr,\\ 
 & uns solde \textit{kom}en al vür wâr\\ 
 & vür die hœhesten schulde pfant.\\ 
 & ze \textbf{der} helle uns nam di\textit{u} hœheste hant\\ 
 & mit der götelîchen minne.\\ 
30 & die unkiuschen liez er drinne.\\ 
\end{tabular}
\scriptsize
\line(1,0){75} \newline
G I O L M Z Fr18 Fr22 Fr61 \newline
\line(1,0){75} \newline
\textbf{1} \textit{Initiale} G I O L Fr22  \textbf{13} \textit{Initiale} I  \newline
\line(1,0){75} \newline
\textbf{1} Von] ÷on O  $\cdot$ Adames] adams I adamis M (Fr22) \textbf{2} Huͤb sich vnwunne Fr61  $\cdot$ riuwe] triwe O (L) (Z) (Fr22) \textbf{3} \textit{Vers 465.3 fehlt} L  \textbf{4} den] dem ein Fr61 \textbf{5} Daz deu svnde ist der sunden wagen Fr61  $\cdot$ unt daz] Vndaz Fr22  $\cdot$ sünden] svnder O L (Fr22)  $\cdot$ wagen] clage M \textbf{6} sô] Vnd Z  $\cdot$ sünden] sunde I (O) (L) (Fr22) (Fr61) die svnde Z  $\cdot$ müezen] mazze Fr61  $\cdot$ tragen] trage M \textbf{7} dar über] Darvmme M (Fr61)  $\cdot$ sîn] des O L M Fr22 Fr61 die Z \textbf{8} erbermde] irbarmen M erbarme Z  $\cdot$ geselleschaft] geschelleschaft G \textbf{9} sîniu] daz sein Fr61 \textbf{10} untriuwen] vntrewe Z Fr61 \textbf{11} sult] \textit{om.} Z \textbf{12} sælde] solt G selbe L \textbf{13} So lat wandel sein da bei Fr61  $\cdot$ lât] Hat I  $\cdot$ sünde] sunden I (M) \textbf{14} rede unde werke] ewer rede Fr61 \textbf{15} swer] wer L M Fr61 [s*er]: wer Z \textbf{18} in] im I (M) Fr61  $\cdot$ sîn] sins G \textbf{19} nemet] Nem L  $\cdot$ mær] \textit{om.} L  $\cdot$ niuwe] niͮwev G \textbf{22} sô] do O L Z Fr61 da M \textbf{23} Sibille] Sibilla I Sýbille L \textbf{24} sunder fâlierens] sunderns valiern I \textbf{25} dâ] [dem]: do G  $\cdot$ manic] mange L \textbf{26} uns] Wie osz M  $\cdot$ komen] geben G  $\cdot$ al] \textit{om.} L Fr61 \textbf{27} die hœhesten schulde] der hohste schulden L  $\cdot$ pfant] ein pfant Fr61 \textbf{28} ze der] von der I Ze O (L) (M) (Fr18) (Fr61)  $\cdot$ uns nam] nam vns Fr61  $\cdot$ diu] die G \textbf{29} der götelîchen] goͤtleicher Fr61 \textbf{30} liez] liezze Fr61 \newline
\end{minipage}
\hspace{0.5cm}
\begin{minipage}[t]{0.5\linewidth}
\small
\begin{center}*T
\end{center}
\begin{tabular}{rl}
 & \begin{large}V\end{large}o\textit{n} Adames künne\\ 
 & huop sich \textbf{triuwe} unde wünne,\\ 
 & sît er uns sippe lougent niht,\\ 
 & den ieglîch engel ob im siht,\\ 
5 & unde diu sippe ist \textbf{unser} wagen,\\ 
 & \textbf{sô} daz wir \textbf{schulde} müezen tragen.\\ 
 & dar über erbarme sich \textbf{des} kraft,\\ 
 & dem \textbf{erbermede} gît geselleschaft,\\ 
 & sît sîn\textit{iu} getriuw\textit{iu} menscheit\\ 
10 & mit triuwen gen untriuwen streit.\\ 
 & ir sult ûf in verkiesen,\\ 
 & welt ir sælde niht verliesen.\\ 
 & lât wandel iu vür \textbf{sünde} bî.\\ 
 & sît rede unde werke niht sô vrî.\\ 
15 & wan \textbf{swer} sîn leit sô rîchet,\\ 
 & daz er unkiusche sprichet,\\ 
 & von des lône tuon ich iu kunt,\\ 
 & \textbf{im} \textbf{urteilet} sîn selbes munt.\\ 
 & Nemet altiu mære vür niuwe,\\ 
20 & ob si iuch \textbf{lêrent} triuwe.\\ 
 & der parliure Plato\\ 
 & sprach bî sînen zîten \textbf{dô}\\ 
 & unde Sibille, diu prophêtisse,\\ 
 & sunder fallierens misse.\\ 
25 & Si sageten dâ vor \textbf{manegiu} jâr,\\ 
 & uns solte komen alvür wâr\\ 
 & vür die hœhsten schulde pfant.\\ 
 & ze helle uns nam diu hœhste hant\\ 
 & mit der götelîchen minne.\\ 
30 & die unkiuschen liez er drinne.\\ 
\end{tabular}
\scriptsize
\line(1,0){75} \newline
T U V W Q R Fr42 \newline
\line(1,0){75} \newline
\textbf{1} \textit{Initiale} T W Fr42   $\cdot$ \textit{Capitulumzeichen} R  \textbf{19} \textit{Majuskel} T  \textbf{25} \textit{Majuskel} T  \newline
\line(1,0){75} \newline
\textbf{1} \textit{Die Verse 453.1-502.30 fehlen} U   $\cdot$ Von] Vov T \textbf{2} sich] sich sit V  $\cdot$ triuwe unde wünne] [*rv́we]: rv́we vnde wúnne V froͤde vnd wunne W trúw vnd vnminne R \textbf{3} er uns sippe] vnser sippe er W \textbf{5} unde] [*]: vnde daz V  $\cdot$ unser] [*]: sv́nden V sunder W (Q) R \textbf{6} schulde] sv́nde V (W) (Q) (R) \textbf{9} sîniu getriuwiu] sine getriuwe T sin getrúwe R \textbf{10} untriuwen] vntrúwe R \textbf{13} sünde] súnden W (Q) \textbf{15} swer] der W wer Q R  $\cdot$ sô] also R \textbf{16} sprichet] brichet Q \textbf{17} von] Won R \textbf{18} im] Jn V R  $\cdot$ urteilet] verteilet V (Q) \textbf{20} iuch] îv T  $\cdot$ lêrent] lere Q \textbf{21} parliure] meister V parlwe Q \textbf{25} sageten] sagt Q  $\cdot$ manegiu] [man*]: manig V manig W (Q) mengem R \textbf{26} solte] [solten]: solte V \textbf{27} hœhsten] [hochse]: hochsten Q \textbf{28} ze] Vz der V \textbf{30} unkiuschen] vnkewsche Q \newline
\end{minipage}
\end{table}
\end{document}
