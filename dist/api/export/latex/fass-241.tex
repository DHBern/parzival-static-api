\documentclass[8pt,a4paper,notitlepage]{article}
\usepackage{fullpage}
\usepackage{ulem}
\usepackage{xltxtra}
\usepackage{datetime}
\renewcommand{\dateseparator}{.}
\dmyyyydate
\usepackage{fancyhdr}
\usepackage{ifthen}
\pagestyle{fancy}
\fancyhf{}
\renewcommand{\headrulewidth}{0pt}
\fancyfoot[L]{\ifthenelse{\value{page}=1}{\today, \currenttime{} Uhr}{}}
\begin{document}
\begin{table}[ht]
\begin{minipage}[t]{0.5\linewidth}
\small
\begin{center}*D
\end{center}
\begin{tabular}{rl}
\textbf{241} & \begin{large}W\end{large}er der selbe wære,\\ 
 & des vreischet her nâch mære;\\ 
 & dar zuo der wirt, \textbf{sîn} burc, \textbf{sîn} lant,\\ 
 & \textbf{die} \textbf{werden} iu von mir genant\\ 
5 & her nâch, sô des wirdet zît,\\ 
 & bescheidenlîchen, âne strît\\ 
 & unt \textbf{ân} allez vür \textbf{zogen}.\\ 
 & ich sage die senewen \textbf{âne bogen}.\\ 
 & diu senewe ist ein bîspel.\\ 
10 & \textbf{nû} dunket \textbf{iuch} der boge snel;\\ 
 & \textbf{doch} ist sneller, daz diu senewe jaget.\\ 
 & ob ich iu rehte hân gesaget,\\ 
 & diu senewe gelîchet mæren sleht,\\ 
 & \textbf{die dunkent} ouch die liute reht.\\ 
15 & swer iu saget von der krumbe,\\ 
 & der wil iuch leiten umbe.\\ 
 & Swer den bogen gespannen siht,\\ 
 & der senewen \textbf{er} der slehte giht,\\ 
 & \textbf{man} welle \textbf{si} \textbf{zer biuge} \textbf{erdenen},\\ 
20 & sô si den schuz \textbf{muoz} \textbf{menen}.\\ 
 & swer \textbf{aber} dem sîn mære schiuzet,\\ 
 & \textbf{des} in durch nôt verdriuzet\\ 
 & - wan \textbf{daz} hât dâ ninder \textbf{stat}\\ 
 & \textbf{unt vil} \textbf{gerûmeclîchen pfat}:\\ 
25 & zeinem ôren în, zem ander\textit{n} vür -,\\ 
 & mîn arbeit ich gar verlür,\\ 
 & ob den mîn mære drünge,\\ 
 & ich sagete oder sünge,\\ 
 & daz ez noch baz vernæme ein boc\\ 
30 & oder ein ulmiger stoc.\\ 
\end{tabular}
\scriptsize
\line(1,0){75} \newline
D \newline
\line(1,0){75} \newline
\textbf{1} \textit{Initiale} D  \textbf{17} \textit{Majuskel} D  \newline
\line(1,0){75} \newline
\textbf{25} andern] ander D \newline
\end{minipage}
\hspace{0.5cm}
\begin{minipage}[t]{0.5\linewidth}
\small
\begin{center}*m
\end{center}
\begin{tabular}{rl}
 & \textit{\begin{large}W\end{large}}er der selbe \textit{w}ære,\\ 
 & des vreischet her nâch mære;\\ 
 & dar zuo der wirt, \textbf{diu} burc, \textbf{daz} lant,\\ 
 & \textbf{die} \textbf{werdent} iu von mir genant\\ 
5 & her nâch, sô des wirt zît,\\ 
 & bescheidenlîch \textbf{und} âne strît\\ 
 & und allez vür \textbf{zogen}.\\ 
 & ich sage die senewen \textbf{âne bogen}.\\ 
 & diu senewe ist ein bîspel.\\ 
10 & \textbf{nû} dunket \textbf{mich} der boge snel;\\ 
 & \textbf{doch} ist sneller, daz diu senewe jaget.\\ 
 & ob ich iu rehte hân gesaget,\\ 
 & diu senewe gelîche\textit{t} mæren sleht,\\ 
 & \textbf{diu dunket} ouch \textit{d}i\textit{e} liute reht.\\ 
15 & we\textit{r} \textit{i}u saget von der krumbe,\\ 
 & der wil i\textit{u}ch leiten umbe.\\ 
 & wer den bogen gespannen sih\textit{t},\\ 
 & der senwen \textbf{er} der slihte \textit{giht},\\ 
 & \textbf{man} \textbf{en}welle \textbf{si} \textbf{zer biuge} \textbf{erdenen},\\ 
20 & sô si den schuz \textbf{müeze} \textbf{menen}.\\ 
 & wer \textbf{aber} dem sîn mære schiuzet,\\ 
 & \textbf{des} in durch nôt verdriuzet\\ 
 & - wan \textbf{daz} hât d\textit{â} niender \textbf{stat}\\ 
 & \textbf{und vil} \textbf{grimmeclîchen pfat}:\\ 
25 & ze einem ôren în, zuo dem andern vür -,\\ 
 & mîn arbeit ich \textbf{dô} gar verlür,\\ 
 & ob den mîn mære drünge,\\ 
 & ich sagete oder sünge,\\ 
 & daz ez n\textit{o}c\textit{h} \textit{b}az vernæme \textit{ein} boc\\ 
30 & oder \textit{e}in ulmiger stoc.\\ 
\end{tabular}
\scriptsize
\line(1,0){75} \newline
m n o Fr69 \newline
\line(1,0){75} \newline
\textbf{1} \textit{Initiale} m  \newline
\line(1,0){75} \newline
\textbf{1} Wer] Der m  $\cdot$ wære] mere m \textbf{2} her] er m n o \textbf{4} die] Du Fr69  $\cdot$ iu] auch o  $\cdot$ genant] erkant Fr69 \textbf{5} \textit{Vers 241.5 fehlt} o   $\cdot$ des] das n \textbf{7} und] One n o  $\cdot$ vür zogen] verzogen n \textbf{8} sage] sach n o  $\cdot$ senewen] senewe n o \textbf{10} boge] boͯgen o \textbf{13} senewe] [senen]: senewe o  $\cdot$ gelîchet] gelicher m \textbf{14} dunket] dunckel o  $\cdot$ die] ier m \textbf{15} wer iu] Wer ich uͯch m \textbf{16} iuch] ich m \textbf{17} siht] [siht]: sihte m \textbf{18} der slihte] die slihte o  $\cdot$ giht] \textit{om.} m \textbf{19} enwelle] welle n o  $\cdot$ biuge] bogen o  $\cdot$ erdenen] denen n o \textbf{20} müeze menen] muͯsse menen m muͦsz wenen n muͯssent wenen o \textbf{21} wer] Swer Fr69 \textbf{22} des] Das o \textbf{23} dâ] do m n durch o \textbf{27} den] denne n \textbf{28} ich] Oder ich n  $\cdot$ sünge] ich singe n \textbf{29} Das es nach dem bas verneme bog m \textbf{30} ein] in m  $\cdot$ ulmiger] vnluͯnger o wlender Fr69 \newline
\end{minipage}
\end{table}
\newpage
\begin{table}[ht]
\begin{minipage}[t]{0.5\linewidth}
\small
\begin{center}*G
\end{center}
\begin{tabular}{rl}
 & \begin{large}W\end{large}er der selbe wære,\\ 
 & des vreischet \textbf{ir} her nâch mære;\\ 
 & dar zuo der wirt, \textbf{diu} burc, \textbf{sîn} lant,\\ 
 & \textbf{diu} \textbf{werdent} iu von mir genant\\ 
5 & her nâch, sô des wirt zît,\\ 
 & bescheidenlîchen, âne strît\\ 
 & unde alle\textit{z} \textit{v}ür \textbf{gezogen}.\\ 
 & ich sage die senwe \textbf{ungelogen}.\\ 
 & diu senwe ist ein bîspel.\\ 
10 & \textbf{ouch} dunket \textbf{iuch} der boge snel;\\ 
 & \textbf{noch} ist sneller, daz diu senwe jaget.\\ 
 & obe ich iu rehte hân gesaget,\\ 
 & diu senwe gelîchet mæren sleht,\\ 
 & \textbf{diu dunkent} \textit{ouch di}e liute reht.\\ 
15 & \textbf{wan} swer iu seit von der krumbe,\\ 
 & der wil iuch \textit{leit}en umbe.\\ 
 & swer den bogen \textit{ge}spannen siht,\\ 
 & der senwe \textbf{man} der slihte giht,\\ 
 & \textbf{si}\textbf{ne} welle \textbf{sich} \textbf{zer biuge} \textbf{denen},\\ 
20 & sô si den schuz \textbf{muoz} \textbf{nemen}.\\ 
 & swer dem sîn mære schiuzet,\\ 
 & \textbf{dâ} in\textbf{s} durch nôt verdriuzet\\ 
 & - wan \textbf{ez} \textbf{en}hât dâ niender \textbf{stat}\\ 
 & \textbf{noch} \textbf{gerûmigez pfat}:\\ 
25 & zeinem ôren în, zem andern vür -,\\ 
 & mîn arbeit ich gar verlür,\\ 
 & op den mîn mære drünge,\\ 
 & ich sagte oder sünge,\\ 
 & daz ez noch baz vernæme ein boc\\ 
30 & oder ein \textit{u}lmiger stoc.\\ 
\end{tabular}
\scriptsize
\line(1,0){75} \newline
G I O L M Q R Z \newline
\line(1,0){75} \newline
\textbf{1} \textit{Initiale} G I O L M Z  \newline
\line(1,0){75} \newline
\textbf{1} Wer] ÷er O \textbf{2} vreischet] vernemt R  $\cdot$ ir] \textit{om.} L Z \textbf{3} der wirt] \textit{om.} R  $\cdot$ diu burc] sin burc I (Q) (Z) der barc M sin bruch R \textbf{4} diu] die I  $\cdot$ genant] be kant Q \textbf{5} wirt] wircz R \textbf{7} vnde allez rehte vur gezogen G Vnd ane allez fuͯrzogen L (Z)  $\cdot$ vür gezogen] fᵫrczogen R \textbf{8} die] dir M dise Z  $\cdot$ ungelogen] ane bogen O (L) (M) (Q) R Z \textbf{9} ein] \textit{om.} O \textbf{10} ouch] Nu Q (R)  $\cdot$ dunket] dunke R  $\cdot$ snel] zv snel Z \textbf{11} ist sneller] ich sneller I snellir ist M \textbf{12} rehte] war R \textbf{13} diu] die I  $\cdot$ gelîchet] Gelichent I di glichet Q dunket R  $\cdot$ mæren] mere Q  $\cdot$ sleht] reht Z \textbf{14} diu] die I  $\cdot$ dunkent] dvnchet O L (M) (R) (Z)  $\cdot$ ouch die] alle G  $\cdot$ reht] [sleht]: reht O sleht Z \textbf{15} swer] wer L M Q R Z  $\cdot$ iu] nun R  $\cdot$ von der] die L \textbf{16} leiten] foͮren G \textbf{17} swer] Wan swer O Wan wer L (M) R Wer Q  $\cdot$ gespannen] spannen G gespannet O L Q (Z) \textbf{18} senwe] senwen I (L) (M) Q Z  $\cdot$ man] er I L Z  $\cdot$ slihte] slehte O \textbf{19} sine] Man L R Man en Q  $\cdot$ sich] sie L Q (R)  $\cdot$ zer] zu L R  $\cdot$ biuge] lvge O  $\cdot$ denen] den M \textbf{20} schuz] shuͤz I schuͯtz L sucz R  $\cdot$ muoz] musszen Q  $\cdot$ nemen] menen L (R) nennen Q \textbf{21} swer] Wer L M Q R  $\cdot$ dem] denne I (L)  $\cdot$ sîn] sine R  $\cdot$ schiuzet] [er]: engivzet O ergeúset Q en gúczet R \textbf{22} dâ ins] des vns I is en M Do ins Q Das in R  $\cdot$ durch] mit R \textbf{23} wan] Ob Z  $\cdot$ ez enhât dâ] ez hat da I L eszn hatt Q ez da hat Z  $\cdot$ niender] nirgen M \textbf{24} noch] Nach O R Vnd Z  $\cdot$ gerûmigez] vil Gerumez I gervmlichen O vil gervmeclichen L (M) (R) wil gerawmelichen Q wil gervmeclichen Z \textbf{25} zem] daz dem I  $\cdot$ vür] swr Q \textbf{26} mîn] Mitt Q \textbf{27} op] Vff M  $\cdot$ den] dem O L \textbf{28} sagte] sage M  $\cdot$ sünge] ich sunge I \textbf{29} ein] \textit{om.} R \textbf{30} ulmiger] fulmiger G vil vuler I vlmýner L olmidir M \newline
\end{minipage}
\hspace{0.5cm}
\begin{minipage}[t]{0.5\linewidth}
\small
\begin{center}*T
\end{center}
\begin{tabular}{rl}
 & Wer der selbe wære,\\ 
 & des vreischet \textbf{ir} her nâch mære;\\ 
 & dar zuo der wirt, \textbf{sîn} burc, \textbf{sîn} lant,\\ 
 & \textbf{diu} \textbf{werdent} iu von mir genant\\ 
5 & her nâch, sô des wirt zît,\\ 
 & bescheidenlîchen, âne strît\\ 
 & unde \textbf{âne} allez vür \textbf{zogen}.\\ 
 & ich sage \textbf{iu} die senewen \textbf{âne bogen}.\\ 
 & diu senewe ist ein bîspel.\\ 
10 & \textbf{ouch} dunket \textbf{iuch} der boge snel;\\ 
 & \textbf{noch} ist sneller, daz di\textit{u} senewe jaget.\\ 
 & obich iu rehte hân gesaget,\\ 
 & diu senewe glîchet mæren sleht,\\ 
 & \textbf{di\textit{u} dunket} ouch die liute reht.\\ 
15 & \textbf{wan} swer iu saget von der krumbe,\\ 
 & der wil iuch leiten umbe.\\ 
 & \textbf{wan} swer den bogen gespannen siht,\\ 
 & der senewen \textbf{er} der slihte giht,\\ 
 & \textbf{man} \textbf{en}welle \textbf{si} \textbf{zerbrochen} \textbf{denen},\\ 
20 & sô si den schuz \textbf{muoz} \textbf{ze vaste} \textbf{menen}.\\ 
 & swer dem sîn mære schiuzet,\\ 
 & \textbf{d\textit{es}} \textit{i}n durch nôt verdriuzet\\ 
 & - wan \textbf{daz} \textbf{en}hât dâ niender \textbf{pfat}\\ 
 & \textbf{noch} \textbf{gerûmeclîche stat}:\\ 
25 & zeinem ôren în, zem andern vür -,\\ 
 & mîn arbeit ich gar verlür,\\ 
 & \hspace*{-.7em}\big| ich sagete oder sünge,\\ 
 & \hspace*{-.7em}\big| ob den mîn mære drünge,\\ 
 & daz ez noch baz vernæme ein boc\\ 
30 & oder ein ulmiger stoc.\\ 
\end{tabular}
\scriptsize
\line(1,0){75} \newline
T U V W \newline
\line(1,0){75} \newline
\textbf{1} \textit{Initiale} W   $\cdot$ \textit{Majuskel} T  \newline
\line(1,0){75} \newline
\textbf{2} vreischet ir] veirschent W  $\cdot$ her nâch] dannoch U \textbf{3} dar zuo] Wer sey W  $\cdot$ sîn burc] [d*]: sin burg V  $\cdot$ sîn lant] [daz]: sin lant V \textbf{4} genant] [genat]: genant T benant W \textbf{5} des] das W \textbf{6} âne] [*]: vnde ane V \textbf{7} Vnd alles recht fúr gezogen W  $\cdot$ vür zogen] vor gezogen U [*]: fúr gezogen V \textbf{8} iu] \textit{om.} W  $\cdot$ senewen] senewe U \textbf{10} ouch] [*]: Nv́ V  $\cdot$ iuch] îv T [d]: vch U [*]: mich V mich W  $\cdot$ snel] sul U \textbf{11} noch] [Doch]: Noch V  $\cdot$ diu] di T \textbf{14} diu] die T  $\cdot$ dunket] \textit{om.} W  $\cdot$ ouch] al U  $\cdot$ reht] hant fúr recht W \textbf{15} wan swer] wan wer U Wer W \textbf{16} iuch] iv T \textbf{17} swer] wer U W \textbf{18} senewen] senewe W \textbf{19} enwelle] welle U (V) (W)  $\cdot$ zerbrochen denen] [z*]: zer bv́ge erdenen V zuͦ krumbe denen W \textbf{20} muoz ze vaste menen] von ir muͦß nenen W \textbf{21} swer] Wer U W [S*]: Swer aber V  $\cdot$ schiuzet] [*]: engússet V \textbf{22} des in] daz sin T Daz sie U [D*]: Dez in V Das ins W \textbf{23} wan] Man W  $\cdot$ daz] iz U [*]: ez V \textit{om.} W  $\cdot$ dâ niender] do niden U do niergent V  $\cdot$ pfat] stat V W \textbf{24} [N*]: Noch vil gerumeclichen phat V  $\cdot$ Noch vil gerumlichen pfat W \textbf{26} gar] [*]: Da gar V \textbf{27} ob] of U  $\cdot$ mîn] mine U \textbf{29} noch] \textit{om.} W \textbf{30} ulmiger] milwiger W \newline
\end{minipage}
\end{table}
\end{document}
