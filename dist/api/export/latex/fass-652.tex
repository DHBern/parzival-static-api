\documentclass[8pt,a4paper,notitlepage]{article}
\usepackage{fullpage}
\usepackage{ulem}
\usepackage{xltxtra}
\usepackage{datetime}
\renewcommand{\dateseparator}{.}
\dmyyyydate
\usepackage{fancyhdr}
\usepackage{ifthen}
\pagestyle{fancy}
\fancyhf{}
\renewcommand{\headrulewidth}{0pt}
\fancyfoot[L]{\ifthenelse{\value{page}=1}{\today, \currenttime{} Uhr}{}}
\begin{document}
\begin{table}[ht]
\begin{minipage}[t]{0.5\linewidth}
\small
\begin{center}*D
\end{center}
\begin{tabular}{rl}
\textbf{652} & \begin{large}N\end{large}û \textbf{warp} der künec sîne vart.\\ 
 & \textbf{des} \textbf{wart} der tavelrunden art\\ 
 & des tages \textbf{al} dâ \textbf{volrecket}.\\ 
 & ez het in vreude \textbf{erwecket},\\ 
5 & daz der werde Gawan\\ 
 & dennoch sîn leben solte hân;\\ 
 & des wâren si innen worden.\\ 
 & der tavelrunden orden\\ 
 & wart dâ begangen âne haz.\\ 
10 & der künec ob tavelrunden \textbf{az}\\ 
 & unt die dâ sitzen solten,\\ 
 & die prîs mit arbeit holten.\\ 
 & al die tavelrun\textit{d}ære\\ 
 & genuzzen dirre mære.\\ 
15 & Nû lât den knappen wider komen,\\ 
 & von dem diu botschaft sî vernomen.\\ 
 & der huop sich dan \textbf{ze} rehter zît.\\ 
 & der küneginne kamerære im gît\\ 
 & pfantlôse, ors unt ander kleit.\\ 
20 & der knappe dan mit vreuden reit,\\ 
 & wander \textbf{an} Artuse \textbf{erwarp},\\ 
 & dâ von sînes \textbf{hêrren} sorge \textbf{erstarp}.\\ 
 & Er kom wider in solhen tagen,\\ 
 & des ich vür wâr niht kan gesagen,\\ 
25 & \textbf{ûf} Schastel Marveile.\\ 
 & Arnive wart diu geile,\\ 
 & wand ir der portenære enbôt,\\ 
 & der \textbf{knappe} wære mit\textbf{s} orses nôt\\ 
 & balde wider gestrichen.\\ 
30 & \textbf{gein} dem \textbf{si kom} geslichen,\\ 
\end{tabular}
\scriptsize
\line(1,0){75} \newline
D \newline
\line(1,0){75} \newline
\textbf{1} \textit{Initiale} D  \textbf{15} \textit{Majuskel} D  \textbf{23} \textit{Majuskel} D  \newline
\line(1,0){75} \newline
\textbf{13} tavelrundære] Tavelrvnderære D \textbf{25} v̂f Scastel marvêile D \textbf{26} Arnive] Arniwe D \newline
\end{minipage}
\hspace{0.5cm}
\begin{minipage}[t]{0.5\linewidth}
\small
\begin{center}*m
\end{center}
\begin{tabular}{rl}
 & nû \textbf{warp} der künic \dag sîner\dag  vart.\\ 
 & \textbf{des} \textbf{wart} der tavelrunder art\\ 
 & des tage\textit{s} d\textit{â} \textbf{volrecket}.\\ 
 & ez het in vröude \textbf{erwecket},\\ 
5 & daz der werde Gawan\\ 
 & dannoch sîn leben solte hân;\\ 
 & des wâren si innen worden.\\ 
 & der tavelrunder orden\\ 
 & wart d\textit{â} begangen âne haz.\\ 
10 & der künic ob \textbf{de\textit{r}} \textit{t}avelrunder \textbf{az}\\ 
 & und die d\textit{â} sitzen solten,\\ 
 & die prîs mit arbeit holten.\\ 
 & alle die tavelrundære\\ 
 & genuzzen diser mære.\\ 
15 & nû lât den knappen wider komen,\\ 
 & von dem diu botschaft sî vernomen.\\ 
 & der huop sich dan \textbf{ze} rehter zît.\\ 
 & der künigîn kamerer im gît\\ 
 & pfantlôse, ros und ander kleit.\\ 
20 & der knappe dan mit vröuden reit,\\ 
 & wan er \textbf{an} Artuse \textbf{erwarp},\\ 
 & dâ von s\textit{îne}s \textbf{hêrren} sorge \textbf{erstarp}.\\ 
 & er kam wider in solichen tagen,\\ 
 & d\textit{e}s ich vür wâr niht kan gesagen,\\ 
25 & \textbf{ûf} Schahtel Marv\textit{e}ile.\\ 
 & Ar\textit{niv}e wart diu geile,\\ 
 & want ir der portener enbôt,\\ 
 & der \textbf{bote} wær mit rosses nôt\\ 
 & balde wider gestrichen.\\ 
30 & \textbf{gegen} dem \textbf{kam si} geslichen,\\ 
\end{tabular}
\scriptsize
\line(1,0){75} \newline
m n o \newline
\line(1,0){75} \newline
\newline
\line(1,0){75} \newline
\textbf{3} tages] tage m  $\cdot$ dâ] do m n o \textbf{8} tavelrunder] tafelrunden n \textbf{9} dâ] do m n o \textbf{10} der tavelrunder] der der tafelrunder m der tafelrunden n \textbf{11} dâ] do m n o  $\cdot$ sitzen solten] solten siczen o \textbf{13} tavelrundære] tafelluͯnder o \textbf{14} genuzzen] Genuͯsser o \textbf{17} ze rehter] zuͯ der rechten n zúr rechter o \textbf{21} Artuse] artuͯse o \textbf{22} sînes] sus m \textbf{24} des] Das m \textbf{25} Vff schahttel maruaile m  $\cdot$ Vff schathel marueile n  $\cdot$ Vff schattel marueẏle o \textbf{26} Arnive] Arune m Arniwe n \textbf{28} mit] by mit n \newline
\end{minipage}
\end{table}
\newpage
\begin{table}[ht]
\begin{minipage}[t]{0.5\linewidth}
\small
\begin{center}*G
\end{center}
\begin{tabular}{rl}
 & \begin{large}N\end{large}û \textbf{schuof} der künic sîne vart.\\ 
 & \textbf{ouch} der tave\textit{l}runder art\\ 
 & des tages dâ \textbf{vol reichet}.\\ 
 & ez het in vröude \textbf{erweichet},\\ 
5 & daz der werde Gawan\\ 
 & dannoch sîn leben solde hân;\\ 
 & des wâren si innen worden.\\ 
 & der tavelrunder orden\\ 
 & wart dâ begangen âne haz.\\ 
10 & der künic ob \textit{\textbf{der}} tave\textit{l}runder \textbf{saz}\\ 
 & unde die dâ sitzen solden,\\ 
 & die brîs mit arbeit holden.\\ 
 & al d\textit{ie} tavelrundære\\ 
 & genuzzen dirre mære.\\ 
15 & nû lât den knappen wider komen,\\ 
 & von dem diu botschaft sî vernomen.\\ 
 & der huop sich dan \textbf{an} rehter zît.\\ 
 & der künegîn kamerære im gît\\ 
 & pfantlôse, ors unde ander kleit.\\ 
20 & der knappe dan mit vröuden reit,\\ 
 & wan e\textit{r} \textbf{dâ ze} Artuse \textbf{erwarp},\\ 
 & dâ von sînes \textbf{herzen} sorge \textbf{erstarp}.\\ 
 & er kom wider in solhen tagen,\\ 
 & des ich vür wâr niht kan gesagen,\\ 
25 & \textbf{ze} Tschastel Marveile.\\ 
 & Arnive \textit{wart} diu geile,\\ 
 & wan ir der bortenære enbôt,\\ 
 & der \textbf{knappe} wære mit\textbf{s} orses nôt\\ 
 & balde wider \textit{ge}strichen.\\ 
30 & \textbf{zuo} dem \textbf{si kom} geslichen,\\ 
\end{tabular}
\scriptsize
\line(1,0){75} \newline
G I L M Z Fr48 \newline
\line(1,0){75} \newline
\textbf{1} \textit{Initiale} G I L Z Fr48  \textbf{15} \textit{Initiale} I  \newline
\line(1,0){75} \newline
\textbf{2} ouch] vnde auch I Ouch wart L M Fr48 Ouch was Z  $\cdot$ tavelrunder] tauelunrunder G \textbf{3} vol reichet] uolreichet G vol rechet L (Z) Fr48 wol gereckit M \textbf{4} erweichet] erwecket L (M) (Z) :::cket Fr48 \textbf{9} dâ] do Fr48 \textbf{10} der tavelrunder] [tauelun*]: tauelunrunder G der Tauelrunde I (L) \textbf{11} die] die die I \textbf{13} die] der G \textbf{16} sî] ist L \textbf{17} an] zcu M \textbf{20} dan mit vröuden] mit freuden dannan I \textbf{21} er] erz G  $\cdot$ dâ] do Fr48  $\cdot$ Artuse] Artus I (Z) Fr48 Artuͯse L \textbf{22} herzen] herren I (M) (Fr48)  $\cdot$ erstarp] starp I M \textbf{24} des] Daz L \textbf{25} Tschastel Marveile] shattemorveile I kastel Marveile L schastil Marveile M tschahtel marveile Z schahtel Marueile Fr48 \textbf{26} Arnive] Arniue I  $\cdot$ wart] \textit{om.} G wart do I  $\cdot$ diu] \textit{om.} I \textbf{27} wan] Do Fr48 \textbf{29} gestrichen] strichen G \newline
\end{minipage}
\hspace{0.5cm}
\begin{minipage}[t]{0.5\linewidth}
\small
\begin{center}*T
\end{center}
\begin{tabular}{rl}
 & nû \textbf{warp} der künic sîne vart.\\ 
 & \textbf{ouch} \textbf{wart} der tavelrunder art\\ 
 & des tages d\textit{â} \textbf{volrecket}.\\ 
 & ez het in vreude \textbf{erwecket},\\ 
5 & daz der werde Gawan\\ 
 & dannoch sîn leben solde hân;\\ 
 & des wâren si innen worden.\\ 
 & der tavelrunder orden\\ 
 & wart dô begangen âne haz.\\ 
10 & der künic ob \textbf{der} tavelrunder \textbf{az}\\ 
 & und die d\textit{â} sitzen solten,\\ 
 & die prîs mit arbeit holten.\\ 
 & al die tavelrundære\\ 
 & genuzzen diser mære.\\ 
15 & nû lât den knaben wider komen,\\ 
 & von dem diu botschaft sî vernomen.\\ 
 & der huop sich dan \textbf{zuo} rehter zît.\\ 
 & der künigîn kamerer im gît\\ 
 & pfantlôse, ros und ander kleit.\\ 
20 & der knabe danne mit vreuden reit,\\ 
 & wan er \textbf{d\textit{â} zuo} Artuse \textbf{warp},\\ 
 & dâ von sînes \textbf{hêrren} sorge \textbf{starp}.\\ 
 & e\textit{r} kam wider in solhen tagen,\\ 
 & des ich vür wâr niht kan gesagen,\\ 
25 & \textbf{ûf} Tschahtel Marveile.\\ 
 & Arnyve war\textit{t d}iu geile,\\ 
 & wan ir der pfortener enbôt,\\ 
 & der \textbf{knabe} wære mit \textbf{des} rosses nôt\\ 
 & balde wider gestrichen.\\ 
30 & \textbf{gên} dem \textbf{si kam} geslichen,\\ 
\end{tabular}
\scriptsize
\line(1,0){75} \newline
Q R W V Fr40 \newline
\line(1,0){75} \newline
\textbf{1} \textit{Initiale} W V Fr40  \newline
\line(1,0){75} \newline
\textbf{1} warp] [*]: warp V  $\cdot$ sîne] [*]: sine V \textbf{3} dâ] do Q R W [d*]: do V  $\cdot$ volrecket] [*]: volle gerecket V \textbf{4} in] [im]: in V im Fr40 \textbf{9} âne] an allen W \textbf{10} tavelrunder] tauelrunde R (V)  $\cdot$ az] [*]: as V \textbf{11} dâ] do Q W V \textbf{13} tavelrundære] tauelrunde R \textbf{16} botschaft sî] boschaft ist V  $\cdot$ vernomen] [k*]: vernomen Q \textbf{19} ander] andere W \textbf{21} er dâ] er do Q W do er R er [*]: an V  $\cdot$ zuo] \textit{om.} V  $\cdot$ warp] erwarb R W (V) (Fr40) \textbf{22} Do von [sin* trvr*]: sins herren trvren erstarb V  $\cdot$ sînes] sein Fr40  $\cdot$ hêrren] hercze R hertzen W  $\cdot$ sorge] froͤde W  $\cdot$ starp] erstarb R W (Fr40) \textbf{23} er] Es Q  $\cdot$ solhen] solhem Fr40 \textbf{24} ich] ich eúch W  $\cdot$ gesagen] sagen Fr40 \textbf{25} Vff tschachtel maueile Q  $\cdot$ Vff Schahtel marveile R  $\cdot$ Auff kastel marfeyle W  $\cdot$ Vf schahtelmarveile V \textbf{26} Arnyve] Arniue Q Arnẏue R Arnyue W Arnive V Fr40  $\cdot$ wart] wart in Q \textbf{27} pfortener] portner R (W) (V) (Fr40) \newline
\end{minipage}
\end{table}
\end{document}
