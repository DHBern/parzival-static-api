\documentclass[8pt,a4paper,notitlepage]{article}
\usepackage{fullpage}
\usepackage{ulem}
\usepackage{xltxtra}
\usepackage{datetime}
\renewcommand{\dateseparator}{.}
\dmyyyydate
\usepackage{fancyhdr}
\usepackage{ifthen}
\pagestyle{fancy}
\fancyhf{}
\renewcommand{\headrulewidth}{0pt}
\fancyfoot[L]{\ifthenelse{\value{page}=1}{\today, \currenttime{} Uhr}{}}
\begin{document}
\begin{table}[ht]
\begin{minipage}[t]{0.5\linewidth}
\small
\begin{center}*D
\end{center}
\begin{tabular}{rl}
\textbf{653} & \begin{large}A\end{large}l dâ \textbf{der} în \textbf{verlâzen} wart.\\ 
 & \textbf{si} vrâgete in umbe sîne vart,\\ 
 & war nâch er \textbf{ûz wære} geriten.\\ 
 & der knappe sprach: "daz wirt vermiten,\\ 
5 & vrouwe, i\textbf{ne} \textbf{tar}s iu niht \textbf{gesagen},\\ 
 & ich muoz \textbf{ez} durch mînen eit verdagen.\\ 
 & \textbf{ez wære ouch} mîme hêrren leit,\\ 
 & bræche ich mit mæren mînen eit;\\ 
 & des diuhte ich \textbf{iuch} der tumbe.\\ 
10 & vrouwe, vrâget in \textbf{selben} drumbe."\\ 
 & Si spiltz mit vrâge an \textbf{manegen} ort.\\ 
 & der knappe sprach \textbf{êt} disiu wort:\\ 
 & "vrouwe, ir sûmet mich ân nôt;\\ 
 & ich leiste, daz mir der eit gebôt."\\ 
15 & Er gienc, dâ er \textbf{sînen hêrren} vant.\\ 
 & der Turkote Florant\\ 
 & unt der herzoge von Gowerzin\\ 
 & \textbf{unt} von Logroys diu herzogîn\\ 
 & \textbf{saz dâ} \textbf{mit grôzer vrouwen} schar.\\ 
20 & der knappe gie ouch zuo \textbf{z}in dar.\\ 
 & Ûf stuont \textbf{mîn} hêr Gawan.\\ 
 & er nam den knappen sunder dan\\ 
 & \textbf{unt bat} in willekomen sîn.\\ 
 & er sprach: "sag \textbf{an}, geselle mîn,\\ 
25 & eintweder vreude oder nôt\\ 
 & \textbf{oder} swaz man \textbf{mir} von hove enbôt.\\ 
 & vünde dû den künec dâ?"\\ 
 & der knappe sprach: "hêrre, jâ.\\ 
 & ich vant den künec unt \textbf{des} wîp\\ 
30 & und manegen \textbf{werdeclîchen} lîp.\\ 
\end{tabular}
\scriptsize
\line(1,0){75} \newline
D \newline
\line(1,0){75} \newline
\textbf{1} \textit{Initiale} D  \textbf{11} \textit{Majuskel} D  \textbf{15} \textit{Majuskel} D  \textbf{21} \textit{Majuskel} D  \newline
\line(1,0){75} \newline
\newline
\end{minipage}
\hspace{0.5cm}
\begin{minipage}[t]{0.5\linewidth}
\small
\begin{center}*m
\end{center}
\begin{tabular}{rl}
 & aldâ \textbf{der} în \textbf{verlâzen} wart.\\ 
 & \textbf{si} vrâgte in umb sîn vart,\\ 
 & wâ nâch er \textbf{ûz wær} geriten.\\ 
 & der knappe sprach: "daz wirt vermiten,\\ 
5 & vrowe, ich \textbf{getar}s iu niht \textbf{sagen},\\ 
 & ich muoz durch mînen eit ver\textit{dag}en.\\ 
 & \textbf{ez wær} mînem hêrren leit,\\ 
 & bræch ich mit mæren mînen eit;\\ 
 & \dag daz\dag  d\textit{i}uht ich \textbf{iuch} der tumbe.\\ 
10 & vrowe, vrâg\textit{et} in \textbf{selbe} dâr umbe."\\ 
 & si spilt ez mit vrâge an \textbf{manigen} ort.\\ 
 & der knappe sprach \textbf{ouch} disiu wort:\\ 
 & "vrowe, ir s\textit{ûmet} mich âne nôt;\\ 
 & ich leiste, daz mir der eit gebôt."\\ 
15 & er gienc, d\textit{â} er \textbf{sînen hêrren} vant.\\ 
 & der Turkoite Florant\\ 
 & und der herzoge von Gowertzin,\\ 
 & von Logrois diu herzogîn\\ 
 & \textbf{sâzen d\textit{â}} \textbf{mit starker} schar.\\ 
20 & der knappe gienc ouch zuo in dar.\\ 
 & ûf stuont \textbf{mîn} hêr Gawan.\\ 
 & er nam den knappen sunder dan\\ 
 & \textbf{und bat} in willekum sîn.\\ 
 & er sprach: "\textbf{nû} sage, geselle mîn,\\ 
25 & eintweder vröude \textit{oder} nôt\\ 
 & \textbf{oder} waz man \textbf{mir} von hove enbôt.\\ 
 & vünde dû den künic dâ?"\\ 
 & der knappe sprach: "hêrre, jâ.\\ 
 & ich vant den künic und \textbf{ouch} \textbf{daz} wîp\\ 
30 & und manig\textit{en} \textbf{werdeclîche\textit{n}} lîp.\\ 
\end{tabular}
\scriptsize
\line(1,0){75} \newline
m n o Fr69 \newline
\line(1,0){75} \newline
\newline
\line(1,0){75} \newline
\textbf{5} sagen] gesagen n o \textbf{6} ich] Here ich o  $\cdot$ mînen] vwern o  $\cdot$ verdagen] vertruken m \textbf{9} daz diuht] Das tuht m (n) (o) ::: Fr69 \textbf{10} vrâget] fragtte m (o)  $\cdot$ selbe] selbez n \textbf{11} manigen] manig Fr69 \textbf{12} ouch] echt n o ::: Fr69 \textbf{13} sûmet] sin m \textbf{15} dâ] do m n o \textbf{16} Turkoite] turkoitte m turkoie o \textbf{17} und der] Vnder m  $\cdot$ Gowertzin] gowerczin o \textbf{18} von Logrois] Vnd logras o \textbf{19} dâ] do m n o  $\cdot$ starker] grosser n o \textbf{20} ouch] \textit{om.} n \textbf{21} hêr] herre her n \textbf{23} in] ẏm o ::: Fr69  $\cdot$ willekum] wilkomen n ::: Fr69 \textbf{25} oder] vnd m \textbf{26} man] \textit{om.} o \textbf{27} dâ] do n \textbf{29} den] d:: o  $\cdot$ ouch] \textit{om.} n o \textbf{30} manigen] manig m o  $\cdot$ werdeclîchen] werdeclicher m \newline
\end{minipage}
\end{table}
\newpage
\begin{table}[ht]
\begin{minipage}[t]{0.5\linewidth}
\small
\begin{center}*G
\end{center}
\begin{tabular}{rl}
 & \begin{large}A\end{large}l dâ \textbf{er} în \textbf{verlâzen} wart\\ 
 & \textbf{unt} vrâget in umbe sîne vart,\\ 
 & war nâch er \textbf{ûz wære} geriten.\\ 
 & der knappe sprach: "daz wirt vermiten,\\ 
5 & vrowe, ich \textbf{en}\textbf{getar} es iu niht \textbf{gesagen},\\ 
 & ich muoz durch mînen ei\textit{t v}erdagen.\\ 
 & \textbf{ouch wære} mînem hêrren leit,\\ 
 & bræche ich mit mæren mînen eit;\\ 
 & des d\textit{i}uhte ich \textbf{in} der tumbe.\\ 
10 & vrouwe, vrâget in \textbf{selbe} drumbe."\\ 
 & \multicolumn{1}{l}{ - - - }\\ 
 & \multicolumn{1}{l}{ - - - }\\ 
 & \multicolumn{1}{l}{ - - - }\\ 
 & \multicolumn{1}{l}{ - - - }\\ 
15 & er gienc, dâ er \textbf{Gawanen} vant.\\ 
 & der Turkoite Florant\\ 
 & unde der herzoge von Gowerzin,\\ 
 & von Logroys diu herzogîn\\ 
 & \textbf{dâ saz} \textbf{unde ander vrouwen} schar.\\ 
20 & der knappe gienc ouch zuo in dar.\\ 
 & ûf stuont hêr Gawan.\\ 
 & er nam den knappen sunder dan.\\ 
 & \textbf{er hiez} in willekomen sîn.\\ 
 & er sprach: "\textbf{nû} sage, geselle mîn,\\ 
25 & eintweder vröude oder nôt\\ 
 & \textbf{unde} swaz man \textbf{mir} von hove enbôt.\\ 
 & vünde dû den künic dâ?"\\ 
 & der knappe sprach: "hêrre, jâ.\\ 
 & ich vant den künic unde \textbf{sîn} wîp\\ 
30 & unde \textbf{dar zuo} manigen \textbf{werden} lîp.\\ 
\end{tabular}
\scriptsize
\line(1,0){75} \newline
G I L M Z Fr48 \newline
\line(1,0){75} \newline
\textbf{1} \textit{Initiale} G I L M Z Fr48  \textbf{25} \textit{Initiale} I  \newline
\line(1,0){75} \newline
\textbf{1} verlâzen] gelazzen Z \textbf{2} vrâget] vragite M \textbf{3} ûz wære] were vz L \textbf{5} engetar es iu] [*]: engetar es iv G getar ev I en tars uch M entar evchz Z  $\cdot$ gesagen] sagen L (M) \textbf{6} muoz] muͯsz ez L  $\cdot$ eit verdagen] eit uil uirdagin G \textbf{7} \textit{Versfolge 653.8-7} L   $\cdot$ ouch] Daz L  $\cdot$ wære] wer ez I \textbf{9} diuhte] dv̂hte G (I) (L) (M)  $\cdot$ in] evch Z \textbf{10} selbe] \textit{om.} M selber Z \textbf{11} \textit{Die Verse 653.11-14 fehlen} G I L M   $\cdot$ Sie spilten mit frage an manigen ort Z \textbf{12} Der knappe sprach ot dise wort Z \textbf{13} Frowe ir svͤmet mich ane not Z \textbf{14} Jch leist daz mir der eit gebot Z \textbf{15} er Gawanen] er Gawan I ergawan M \textbf{16} Turkoite] tv̂rkoite G Turchoyde I Tuͯrkoite \sout{vnd} L  $\cdot$ Florant] floriant G I \textbf{17} Gowerzin] gouerzin I gowerczin M \textbf{18} von] Vnd von Z  $\cdot$ Logroys] lorgrois G logrois M (Z) \textbf{19} unde ander] mit grozzer Z \textbf{20} ouch] \textit{om.} I \textbf{24} sage] sage an Z \textbf{25} eintweder] Ietdwederz I  $\cdot$ oder] vnd I \textbf{26} unde] oder I (M) (Z)  $\cdot$ swaz] waz L (M) Z \textbf{28} hêrre] \textit{om.} I \newline
\end{minipage}
\hspace{0.5cm}
\begin{minipage}[t]{0.5\linewidth}
\small
\begin{center}*T
\end{center}
\begin{tabular}{rl}
 & aldâ \textit{\textbf{er}} în \textbf{gelâzen} wart\\ 
 & \textbf{und} vrâgte in umb sîn vart,\\ 
 & war nâch er \textbf{wær ûz} geriten.\\ 
 & der knabe sprach: "daz wirt vermiten,\\ 
5 & vrou, ich \textbf{en}\textbf{darf}s iu niht \textbf{sagen},\\ 
 & ich muoz \textbf{ez} durch mînen eit verdagen.\\ 
 & \textbf{ouch wær ez} mînem hêrren leit,\\ 
 & bræch ich mit mæren mînen eit;\\ 
 & des d\textit{i}uht ich \textbf{in} der tumbe.\\ 
10 & vrou, vrâget in \textbf{selber} drumbe."\\ 
 & si spilt ez mit vrâge an \textbf{manege} ort.\\ 
 & der knabe sprach \textbf{ouch} disiu wort:\\ 
 & "vrou, ir sûmet mich âne nôt;\\ 
 & ich leiste, daz mir der eit gebôt."\\ 
15 & er gienc, d\textit{â} er \textbf{sînen hêrren} vant.\\ 
 & der Turkoyte Florant\\ 
 & und der herzoge von Gowerzin\\ 
 & \textbf{und} von Logrois diu herzogîn\\ 
 & \textbf{d\textit{â} saz} \textbf{mit grôzer vr\textit{ouw}en} schar.\\ 
20 & der knabe gienc ouch zuo in dar.\\ 
 & ûf stuont hêr Gawan.\\ 
 & er nam den knaben sunder dan.\\ 
 & \textbf{er hiez} \textit{in} willekomen sîn.\\ 
 & er sprach: "\textbf{nû} sag, geselle mîn,\\ 
25 & eintweder vreude oder nôt\\ 
 & \textbf{oder} waz man \textbf{dir} von hove enbôt.\\ 
 & vünde dû den künic d\textit{â}?"\\ 
 & der knabe sprach: "hêrre, jâ.\\ 
 & ich vant den künic und \textbf{sîn} wîp\\ 
30 & und \textbf{dâ zuo} manegen \textbf{werden} lîp.\\ 
\end{tabular}
\scriptsize
\line(1,0){75} \newline
Q R W V Fr40 \newline
\line(1,0){75} \newline
\textbf{1} \textit{Initiale} R Fr40  \textbf{15} \textit{Initiale} W  \textbf{21} \textit{Initiale} V  \newline
\line(1,0){75} \newline
\textbf{1} er] \textit{om.} Q  $\cdot$ gelâzen] verlassen R W (V) (Fr40) \textbf{2} vrâgte] fraget R (V) (Fr40) \textbf{3} er] \textit{om.} Fr40  $\cdot$ wær ûz] auß were W (V) (Fr40) \textbf{5} endarfs iu] getar úchs R tar es eúch W getar sv́ch V getar euch Fr40  $\cdot$ sagen] gesagen R V Fr40 \textbf{6} ez] úchs W  $\cdot$ verdagen] vertragen R \textbf{9} diuht] to͑ucht Q (R) (W) (V) (Fr40) \textbf{10} vrou] \textit{om.} W  $\cdot$ selber] selben Fr40 \textbf{11} an] den Fr40  $\cdot$ manege] mangen W (Fr40) manig V \textbf{12} ouch] echt R W (Fr40) o\textit{m. } V \textbf{14} der] [mi*]: min V \textbf{15} dâ] do Q R W V \textbf{16} Turkoyte] turkoite Q (V) Turkotte R \textbf{17} der herzoge] den herczogen R  $\cdot$ Gowerzin] kawerzin Q Gowerczin R gawerzin Fr40 \textbf{18} Logrois] logroys W \textbf{19} dâ] Do Q R W [D*]: Do V  $\cdot$ saz] [*]: sazen V  $\cdot$ vrouwen] frewden Q \textbf{20} zuo in] zvͦzin V (Fr40) \textbf{21} hêr] mein herre W \textbf{23} in] \textit{om.} Q  $\cdot$ willekomen] wilkome W (V) \textbf{24} er] Vnd W V  $\cdot$ nû sag] sag dar W \textbf{25} eintweder] Ein wedre R Enweder W [Enweder]: Einweder V \textbf{26} waz] swaz Fr40  $\cdot$ dir] mir R W V Fr40  $\cdot$ enbôt] bot W \textbf{27} dâ] do Q \textbf{28} knabe] knappe der V \newline
\end{minipage}
\end{table}
\end{document}
