\documentclass[8pt,a4paper,notitlepage]{article}
\usepackage{fullpage}
\usepackage{ulem}
\usepackage{xltxtra}
\usepackage{datetime}
\renewcommand{\dateseparator}{.}
\dmyyyydate
\usepackage{fancyhdr}
\usepackage{ifthen}
\pagestyle{fancy}
\fancyhf{}
\renewcommand{\headrulewidth}{0pt}
\fancyfoot[L]{\ifthenelse{\value{page}=1}{\today, \currenttime{} Uhr}{}}
\begin{document}
\begin{table}[ht]
\begin{minipage}[t]{0.5\linewidth}
\small
\begin{center}*D
\end{center}
\begin{tabular}{rl}
\textbf{267} & hât \textbf{werdecheit an mir} bezalt.\\ 
 & nû erlâze mich, küener degen balt,\\ 
 & suone gein disem wîbe\\ 
 & \textbf{unt} gebiute mîme lîbe\\ 
5 & anders, swaz dîn êre sîn.\\ 
 & gein der geunêrten herzogîn\\ 
 & mag ich \textbf{suone gepflegen} niht,\\ 
 & swaz \textbf{halt} anders mir geschiht."\\ 
 & Parzival, der hôch gemuot,\\ 
10 & sprach: "liute, lant \textbf{noch} varende guot,\\ 
 & der decheinez mac gehelfen dir,\\ 
 & dûne tuost des \textbf{sicherheit} gein mir,\\ 
 & daz dû gein \textbf{Bertane} varst\\ 
 & unt \textbf{die} reise niht langer sparst\\ 
15 & \textbf{z}einer magt, die blou durch mich\\ 
 & ein man, gein dem ist mîn gerich\\ 
 & \textbf{âne} ir bete niht verkorn.\\ 
 & dû solt der meide wol geborn\\ 
 & sichern unt mîn dienest sagen\\ 
20 & oder \textbf{wirde} alhie erslagen.\\ 
 & sage Artuse unt dem wîbe sîn,\\ 
 & in beiden, von mir dienest mîn,\\ 
 & \textbf{daz} si \textbf{mîn} dienst \textbf{sus} letzen\\ 
 & \textbf{unt} die magt ir slege ergetzen.\\ 
25 & dar zuo wil ich \textbf{schouwen}\\ 
 & in dînen hulden \textbf{dise vrouwen}\\ 
 & mit suone âne vâre\\ 
 & oder dû muost eine bâre\\ 
 & tôt hinnen rîten,\\ 
30 & wiltû mich\textbf{s} widerstrîten.\\ 
\end{tabular}
\scriptsize
\line(1,0){75} \newline
D \newline
\line(1,0){75} \newline
\newline
\line(1,0){75} \newline
\newline
\end{minipage}
\hspace{0.5cm}
\begin{minipage}[t]{0.5\linewidth}
\small
\begin{center}*m
\end{center}
\begin{tabular}{rl}
 & hât \textbf{an mir werdecheit} bezalt.\\ 
 & nû erlâz mich, küener degen balt,\\ 
 & suone gegen \textit{d}isem wîbe\\ 
 & \textbf{und} gebiut mînem lîbe\\ 
5 & anders, waz dîn êre sî\textit{n}.\\ 
 & gegen der geunêrten herzogîn\\ 
 & mac ich \textbf{gepflegen suone} niht,\\ 
 & waz \textbf{halt} anders mir geschiht."\\ 
 & \begin{large}P\end{large}arcifal, der hôchgemuot,\\ 
10 & sprach: "liute, lant \textbf{noch} varende guot,\\ 
 & der dekeinez mac gehelfen dir,\\ 
 & dû entuost des \textbf{sicherheit} gegen mir,\\ 
 & daz dû gegen \textbf{Britanie} varst\\ 
 & und \textbf{die} reise niht langer sparst\\ 
15 & \textbf{ze} einer maget, die blou durch mich\\ 
 & ein man, gegen dem ist mîn gerich\\ 
 & \textbf{âne} ir bete niht verkorn.\\ 
 & dû solt der megde wol geborn\\ 
 & sicher\textit{n} und mînen dienst sagen\\ 
20 & oder \textbf{wirt dû} alhie erslagen.\\ 
 & sage Artuse und dem wîbe sîn,\\ 
 & in beiden, von mir \textbf{den} dienest mîn,\\ 
 & \textbf{daz} si \textbf{mir} dienest \textbf{sus} letzen\\ 
 & \textbf{und} die maget ir sl\textit{e}ge ergetzen.\\ 
25 & dar zuo wil ich \textbf{dise vrouwen}\\ 
 & in dînen hulden \textbf{schouwen}\\ 
 & mit suone âne vâre\\ 
 & oder dû muost eine bâre\\ 
 & tôt \textbf{von} hinnen rîten,\\ 
30 & wilt dû mich\textbf{s} widerstrîten.\\ 
\end{tabular}
\scriptsize
\line(1,0){75} \newline
m n o Fr69 \newline
\line(1,0){75} \newline
\textbf{9} \textit{Illustration mit Überschrift:} Also parcifal den ritter twang das er zuͯ dem kv́nige (konigin o  ) von pritanie (britanie o  ) muͯste faren vnd ẏme parcifals dienste múste sagen n (o)   $\cdot$ \textit{Initiale} m n o  \newline
\line(1,0){75} \newline
\textbf{3} disem] deissem m disen o dem Fr69 \textbf{4} gebiut] gebúte du n (o) \textbf{5} sîn] sẏ m \textbf{6} geunêrten] zemierten n ganureten o \textbf{7} suone] \textit{om.} Fr69 \textbf{8} waz] Swaz Fr69 \textbf{11} dekeinez] do keines n \textbf{12} des] das o \textbf{13} daz] Dast o  $\cdot$ Britanie] brittanie m brẏtanie n britanie o \textbf{15} blou] bla o \textbf{19} sichern] Sichere m \textbf{20} wirt dû] wurde aber du n wirt dú aber o  $\cdot$ erslagen] geslagen o \textbf{22} in] Den n o \textbf{24} slege] slage m \textbf{27} suone] sonne o \textbf{28} oder] Odú o \newline
\end{minipage}
\end{table}
\newpage
\begin{table}[ht]
\begin{minipage}[t]{0.5\linewidth}
\small
\begin{center}*G
\end{center}
\begin{tabular}{rl}
 & hât \textbf{an mir werdecheit} bezalt.\\ 
 & nû erlâ mich, küener degen balt,\\ 
 & suone gein disem wîbe.\\ 
 & gebiut mînem lîbe\\ 
5 & anders, swaz dîn êre sîn.\\ 
 & gein der geunêrten herzogîn\\ 
 & mag ich \textbf{suone gepflegen} niht,\\ 
 & swaz \textbf{halt} anders mir geschiht."\\ 
 & Parzival, der hôchgemuot,\\ 
10 & sprach: "liute, lant \textbf{noch} varnde guot,\\ 
 & der deheinez mac gehelfen dir,\\ 
 & dûne tuost des \textbf{sicherheit} gein mir,\\ 
 & daz dû gein \textbf{Britanie} varst\\ 
 & unt \textbf{die} reise niht langer sparst\\ 
15 & \textbf{gein} einer meide, die blou durch mich\\ 
 & ein man, gein dem ist mîn gerich\\ 
 & \textbf{âne} ir bete niht verkoren.\\ 
 & dû solt der meide wolgeboren\\ 
 & sicheren unde mîn dienst sagen\\ 
20 & oder \textbf{dû wirst} al hie erslagen.\\ 
 & sage Artuse unde dem wîbe sîn,\\ 
 & in beiden, von mir dienst mîn,\\ 
 & \textbf{daz} si \textbf{mîn} dienst \textbf{sol} letzen,\\ 
 & die maget ir slege ergetzen.\\ 
25 & dar zuo wil ich \textbf{beschouwen}\\ 
 & in dînen hulden \textbf{dise vrouwen}\\ 
 & mit suone âne vâre\\ 
 & ode dû muost eine bâre\\ 
 & tôter hinnen rîten,\\ 
30 & wil dû mich \textbf{es} widerstrîten.\\ 
\end{tabular}
\scriptsize
\line(1,0){75} \newline
G I O L M Q R Z Fr21 \newline
\line(1,0){75} \newline
\textbf{9} \textit{Initiale} I M R  \textbf{27} \textit{Initiale} O Q Z Fr21  \newline
\line(1,0){75} \newline
\textbf{1} bezalt] bescalt Q \textbf{2} küener] [kúmer]: kúnner Q  $\cdot$ degen] tagen L \textbf{3} suone] Als Q \textbf{4} gebiut] Vnd gebevt Z \textbf{5} swaz] waz L (M) (Q) (R) Z  $\cdot$ dîn êre] dem eren Q  $\cdot$ sîn] sy R \textbf{6} gein] \textit{om.} Z  $\cdot$ der] dirr I  $\cdot$ geunêrten] vnferten Q vngeertten R  $\cdot$ herzogîn] chungin I \textbf{7} suone gepflegen] gephlegen suͤn I (O) (M) (Z) (Fr21) \textbf{8} swaz] Waz L (Q) (R)  $\cdot$ anders mir] mir da von I myr anders M \textbf{9} Parzival] Parzifal I L M Parcifal O Z Fr21 Partzifal Q PArczifal R \textbf{10} liute lant] levt noch lant O (Fr21) lant levt Z  $\cdot$ noch varnde] vnd varndes L noch varndez Z noch R \textbf{11} deheinez] icheyn M \textbf{12} dûne tuost] Du dust Q  $\cdot$ des] die I L den M denne R \textbf{13} Britanie] pritanne I Brittanien L britania M britangen Q \textbf{14} die] din I R  $\cdot$ langer] lange R \textbf{15} blou] slvͦch O [blov]: slug M strech R \textbf{16} ein] Von einem R  $\cdot$ ist mîn] han ich I  $\cdot$ gerich] gericht R \textbf{17} ir] \textit{om.} O  $\cdot$ bete] het Q bitte R  $\cdot$ verkoren] erkoren Q \textbf{19} sagen] [gebe]: sagen O \textbf{20} oder] vnd I  $\cdot$ dû wirst] du werst I wirt Z  $\cdot$ al] \textit{om.} L R \textbf{21} Artuse] Artus I (Q) (R) (Fr21) \textbf{22} in] \textit{om.} I O L M Q Fr21  $\cdot$ von mir] sampt I vnd von mir Q  $\cdot$ dienst] den dienst I (L) (M) (Q) (R) Z Fr21 \textbf{23} mîn] minen O (L) Z  $\cdot$ sol] so O L M Q R Fr21 sust Z \textbf{24} die] Vnde die O (L) (Q) (R) (Z) (Fr21)  $\cdot$ ir] der O M (Fr21) o\textit{m. } R  $\cdot$ slege] streich R \textbf{25} beschouwen] sauwen I beschwen R \textbf{26} Dine hulde diser frowen R \textbf{27} mit] ÷it O  $\cdot$ suone] suͯsze R  $\cdot$ vâre] geuare R \textbf{30} Wilte ichttes wider stritten R  $\cdot$ es] sin I des L \newline
\end{minipage}
\hspace{0.5cm}
\begin{minipage}[t]{0.5\linewidth}
\small
\begin{center}*T
\end{center}
\begin{tabular}{rl}
 & hât \textbf{werdecheit an mir} bezalt.\\ 
 & nû erlâ mich, küener degen balt,\\ 
 & suone gegen disem wîbe\\ 
 & \textbf{unde} gebiut mînem lîbe\\ 
5 & anders, swaz dîn êre sîn.\\ 
 & gegen der geunêrten herzogîn\\ 
 & mag ich \textbf{suone gepflegen} niht,\\ 
 & swaz \textbf{joch} anders mir geschiht."\\ 
 & \begin{large}P\end{large}arcifal, der hôchgemuot,\\ 
10 & sprach: "liute, lant, varnde guot,\\ 
 & der deheine\textit{z} mac gehelfen dir,\\ 
 & dûne tuo\textit{st} des \textbf{sicher} gegen mir,\\ 
 & daz dû gegen \textbf{Britanie} varst\\ 
 & unde \textbf{dîne} reise niht langer sparst\\ 
15 & \textbf{gegen} einer maget, die blou durch mich\\ 
 & ein man, gegen dem ist mîn gerich\\ 
 & \textbf{unde mit} ir bete niht verkorn.\\ 
 & dû solt der megde wol geborn\\ 
 & sichern unde mînen dienst sagen\\ 
20 & oder \textbf{wirt} alhie erslagen.\\ 
 & sage Artuse unde dem wîbe sîn,\\ 
 & in beiden, von mir \textbf{den} dienst mîn.\\ 
 & \textbf{bit} si \textbf{mîn} dienst \textbf{sus} letzen,\\ 
 & die maget ir slege ergetzen.\\ 
25 & dar zuo wil ich \textbf{schouwen}\\ 
 & in dînen hulden \textbf{dise vrouwen}\\ 
 & mit suone âne vâre\\ 
 & oder dû muost eine bâre\\ 
 & tôt \textbf{von} hinnen rîten,\\ 
30 & wilt dû mich widerstrîten.\\ 
\end{tabular}
\scriptsize
\line(1,0){75} \newline
T U V W \newline
\line(1,0){75} \newline
\textbf{9} \textit{Initiale} T U V W  \newline
\line(1,0){75} \newline
\textbf{3} disem] dem W \textbf{4} gebiut] beút W \textbf{5} swaz] waz U (W) \textbf{8} swaz] Waz U (W)  $\cdot$ joch] \textit{om.} U aber V \textbf{9} Parcifal] Parzifal V PArtzifal W \textbf{10} lant] vnd lant U  $\cdot$ varnde] [*]: noch varnde V \textbf{11} deheinez] deheines T V \textbf{12} dûne] Du W  $\cdot$ tuost] tvͦz T  $\cdot$ sicher] sicherheit U V (W) \textbf{13} Britanie] pritanie V \textbf{14} dîne] die U V \textbf{17} Die weil sy die schuld nit hat erkoren W  $\cdot$ unde mit ir] Mit irre U [*t]: ane ir V \textbf{20} wirt] duͦ wirst U (V) (W) \textbf{22} beiden] buͦt U \textbf{23} mîn] [min]: mir V \textbf{24} die] [*]: Vnde die V  $\cdot$ slege] slagens U \textbf{25} schouwen] beschowen V (W) \textbf{26} dise] die W \textbf{30} mich] mich is U (V) \newline
\end{minipage}
\end{table}
\end{document}
