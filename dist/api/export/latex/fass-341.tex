\documentclass[8pt,a4paper,notitlepage]{article}
\usepackage{fullpage}
\usepackage{ulem}
\usepackage{xltxtra}
\usepackage{datetime}
\renewcommand{\dateseparator}{.}
\dmyyyydate
\usepackage{fancyhdr}
\usepackage{ifthen}
\pagestyle{fancy}
\fancyhf{}
\renewcommand{\headrulewidth}{0pt}
\fancyfoot[L]{\ifthenelse{\value{page}=1}{\today, \currenttime{} Uhr}{}}
\begin{document}
\begin{table}[ht]
\begin{minipage}[t]{0.5\linewidth}
\small
\begin{center}*D
\end{center}
\begin{tabular}{rl}
\textbf{341} & gein strîte was zer tjost brâht.\\ 
 & des wart ouch \textbf{dâ} hin zim gedâht.\\ 
 & \begin{large}G\end{large}awan \textbf{sach} \textbf{geflôrieret}\\ 
 & unt wol \textbf{gezimieret}\\ 
5 & von rîcher koste helme vil.\\ 
 & si vuorten gein ir nîtspil\\ 
 & \textbf{wîz} niwer sper ein wunder,\\ 
 & diu \textbf{gemâlt} wâren besunder\\ 
 & junchêrren gegeben in die hant,\\ 
10 & ir hêrren wâpen dâr an \textbf{bekant}.\\ 
 & Gawan fillu roy Lot\\ 
 & sach von gedrenge grôze nôt,\\ 
 & mûle, die harnasch muosen tragen,\\ 
 & unt manegen wol geladen wagen;\\ 
15 & den was \textbf{gein} herbergen gâch.\\ 
 & \textbf{ouch} vuor der market hinden nâch\\ 
 & mit wunderlîcher pârât.\\ 
 & des \textbf{en}was êt \textbf{doch} \textbf{dehein} ander rât.\\ 
 & \textbf{ouch was} der vrouwen dâ genuoc.\\ 
20 & etslîchiu den zwelften gürtel truoc\\ 
 & ze pfande nâch ir minne.\\ 
 & ez wâren niht küneginne:\\ 
 & die selben trippâniersen\\ 
 & hiezen soldiersen.\\ 
25 & hie der junge, dort der alde,\\ 
 & dâ \textbf{vuor} vil ribalde.\\ 
 & \textbf{ir loufen machte in} müediu lide.\\ 
 & etslîcher zæme baz an der wide,\\ 
 & denne er daz \textbf{her} dâ mêrte\\ 
30 & unt werdez volc unêrte.\\ 
\end{tabular}
\scriptsize
\line(1,0){75} \newline
D \newline
\line(1,0){75} \newline
\textbf{3} \textit{Initiale} D  \newline
\line(1,0){75} \newline
\textbf{11} Lot] lôt D \newline
\end{minipage}
\hspace{0.5cm}
\begin{minipage}[t]{0.5\linewidth}
\small
\begin{center}*m
\end{center}
\begin{tabular}{rl}
 & gegen strîte was zer juste brâht.\\ 
 & des wart ouch \textbf{dâ} hin zuo ime gedâht.\\ 
 & Gawan \textbf{sach} \textbf{geflôrieret}\\ 
 & und wol \textbf{gezimieret}\\ 
5 & von rîcher \textit{k}o\textit{s}te helme vil.\\ 
 & si vuorten gegen ir nîtspil\\ 
 & \textbf{\textit{v}i\textit{l}} \textit{n}iuwer sper ein wunder,\\ 
 & diu \textbf{gemâlet} wâren besunder\\ 
 & junchêrren gegeben in die hant,\\ 
10 & ir hêrren wâpen dran \textbf{erkant}.\\ 
 & Gawan fili roi L\textit{o}t\\ 
 & sach von gedrenge grôze nôt,\\ 
 & mûle, die harnasch muosen tragen,\\ 
 & und manigen wolgeladenen wagen;\\ 
15 & den was \textbf{ze} herbergen gâch.\\ 
 & \textbf{ouch} vuor der market hinden nâch\\ 
 & mit wunderlîchem pârât.\\ 
 & des \textbf{en}was eht \textbf{niht} \textbf{dô} ander rât.\\ 
 & \textbf{ouch was} der vrouwen dâ genuoc.\\ 
20 & etslîchiu den zwelften gürtel truoc\\ 
 & ze pfande nâch ir minne.\\ 
 & ez wâren niht küniginne:\\ 
 & die selben trippâniersen\\ 
 & hiezen soldie\textit{r}sen.\\ 
25 & hie der junge, dort der alte,\\ 
 & d\textit{â} \textbf{vuoren} vil ribalte.\\ 
 & \textbf{ir loufen macht in} müede lide.\\ 
 & etslîcher zæme baz an der wid\textit{e},\\ 
 & danne erz \textbf{her} dâ mêrt\textit{e}\\ 
30 & und werdez vol\textit{c} unêrt\textit{e}.\\ 
\end{tabular}
\scriptsize
\line(1,0){75} \newline
m n o \newline
\line(1,0){75} \newline
\newline
\line(1,0){75} \newline
\textbf{2} des] Das o  $\cdot$ dâ hin zuo] gegen n o \textbf{3} geflôrieret] florieret o \textbf{5} koste] scote m \textbf{7} vil niuwer] Wir uͯwer m \textbf{9} gegeben] geben n gebent o  $\cdot$ die] ir n o \textbf{11} Gawan] Gewan o  $\cdot$ fili] vili m  $\cdot$ roi Lot] roilet m roilot n o \textbf{13} die] die do n  $\cdot$ muosen] muͯssen m músten n \textbf{15} herbergen] herberge n o \textbf{18} eht niht dô] ouch do kein n (o) \textbf{19} dâ] do n o \textbf{23} trippâniersen] tripponiersen m krippomoẏsen n kripomorsen o \textbf{24} soldiersen] soldiesen m o soldoysen n \textbf{26} dâ] Do m n o \textbf{27} ir] Das n o  $\cdot$ macht] machte n  $\cdot$ lide] ir lide n o \textbf{28} wide] widen m \textbf{29} dâ] do n o  $\cdot$ mêrte] merten m \textbf{30} werdez] wer dasz o  $\cdot$ volc unêrte] vol vnerten m \newline
\end{minipage}
\end{table}
\newpage
\begin{table}[ht]
\begin{minipage}[t]{0.5\linewidth}
\small
\begin{center}*G
\end{center}
\begin{tabular}{rl}
 & gein strîte was zer tjoste brâht.\\ 
 & des wart ouch \textbf{dâ} hin ze im gedâht.\\ 
 & \begin{large}G\end{large}awan \textbf{sach} \textbf{geflô\textit{r}ieret}\\ 
 & unde wol \textbf{gezimieret}\\ 
5 & von rîcher koste helm vil.\\ 
 & si vuorten gein ir nîtspil\\ 
 & \textbf{wîz} niwer sper ein wunder,\\ 
 & diu \textbf{gemâlten} wâren besunder\\ 
 & junchêrren gegeben in die hant,\\ 
10 & ir hêrren wâpen dran \textbf{erkant}.\\ 
 & Gawan filiroys Lot\\ 
 & sach von gedrenge grôze nôt,\\ 
 & \textbf{vil} mûle, die harnasch muosen tragen,\\ 
 & unde manigen wol geladenen wagen;\\ 
15 & den was \textbf{gein} herbergen gâch.\\ 
 & \textbf{dâ} vuor der market hinden nâch\\ 
 & mit wunderlîcher pârât.\\ 
 & des was êt \textbf{dô} \textbf{dehein} ander rât.\\ 
 & \textbf{er sach} der vrouwen dâ genuoc.\\ 
20 & etslîchiu den zwelften gürtel truoc\\ 
 & ze pfande nâch ir minne.\\ 
 & ez wâren niht küniginne:\\ 
 & die selben trippâniersen\\ 
 & hiezen soldiersen.\\ 
25 & hie der junge, dort der alde,\\ 
 & dâ \textbf{vuor} vil ribalde.\\ 
 & \textbf{den machet ir loufen} müede lide.\\ 
 & etslîcher zæme baz an der wide,\\ 
 & dane er daz \textbf{her} dâ mêrte\\ 
30 & unde werdez volc unêrte.\\ 
\end{tabular}
\scriptsize
\line(1,0){75} \newline
G I O L M Q R Z Fr22 Fr39 Fr40 \newline
\line(1,0){75} \newline
\textbf{3} \textit{Initiale} G I O L Z Fr22 Fr39 Fr40   $\cdot$ \textit{Capitulumzeichen} R  \textbf{23} \textit{Initiale} I  \newline
\line(1,0){75} \newline
\textbf{1} zer tjoste] zestrite O zuͯ týost L (M) (Fr39) \textbf{2} des] Der O  $\cdot$ wart] was O Q R  $\cdot$ dâ] do Q Fr39  $\cdot$ hin ze im] zu im Q hie zu R \textbf{3} Gawan] ÷Awan O G::: Fr40  $\cdot$ sach geflôrieret] sach gefloieret G sich florrirte L sich geflorirte M (Fr22) [*]: sach geflorierte  Fr39 \textbf{4} wol gezimieret] vil wol zimierte L Fr39 wol gezcymierte M (Fr22) \textbf{5} von] mit I \textbf{6} vuorten] furter M  $\cdot$ ir] im O  $\cdot$ nîtspil] mit spil M niter spil Z \textbf{7} wîz] vnde wiz O Wîzzir Fr22  $\cdot$ niwer] niwiu I \textbf{8} gemâlten] gemalet I  $\cdot$ besunder] bisundern M \textbf{9} gegeben] gaben O \textbf{11} Gawan] :::wan Fr39 g::: Fr40  $\cdot$ filiroys Lot] filli roylot O sýz lvroý lot L fiszli roilot M fiez lyroi lot Z fiz luroy lot Fr39 Fr22 \textbf{13} mûle] Muler M  $\cdot$ die] den O \textit{om.} L Fr39  $\cdot$ muosen] muͤsen I (Fr39) mvͦse O \textbf{14} unde] vnd dar zuͤ I  $\cdot$ wol] \textit{om.} O \textbf{15} herbergen] herberge M R herberich Q \textbf{16} dâ] den I \textbf{17} wunderlîcher] wunderlichem Q \textbf{18} was] enwaz L (M) (Z) (Fr39)  $\cdot$ êt] \textit{om.} L Z e Q  $\cdot$ dô] da M R Z \textbf{19} sach] \textit{om.} M  $\cdot$ dâ] \textit{om.} O do Q Fr39 och R \textbf{20} etslîchiu] Ettlicher R  $\cdot$ den] \textit{om.} O \textbf{22} wâren] enwaren L (M) Z Fr39 \textbf{23} trippâniersen] troponeisen I tvppeniersen O triben irn sen R terppeniersen Z trvmpinîersin Fr22 \textbf{24} die sint in solhen reisen I  $\cdot$ Hie zwen soldenesen R  $\cdot$ soldiersen] soldrisen Q \textbf{25} hie] Die Fr22  $\cdot$ junge] ivngen O  $\cdot$ dort] die Fr22  $\cdot$ alde] alden O \textbf{26} dâ] Do Q (Fr39)  $\cdot$ vuor] furen Z  $\cdot$ vil] \textit{om.} M  $\cdot$ ribalde] ribalden O \textbf{27} den] dem I (Q) Der Z  $\cdot$ machet] machte O Fr22 machtin M  $\cdot$ loufen] lofe R  $\cdot$ müede] muͤdiu I (O)  $\cdot$ lide] liden M \textbf{28} etslîcher] Ettisliche M Etslichim Fr22  $\cdot$ zæme] hienge O  $\cdot$ der] einer I (O) \textbf{29} dane] Denne das R  $\cdot$ her] er Q  $\cdot$ dâ] do Fr39 \textbf{30} werdez] widirs M de wedres R  $\cdot$ volc] her O  $\cdot$ unêrte] gevnerte Z \newline
\end{minipage}
\hspace{0.5cm}
\begin{minipage}[t]{0.5\linewidth}
\small
\begin{center}*T
\end{center}
\begin{tabular}{rl}
 & gegen strîte was zer tjost brâht.\\ 
 & des wart ouch hin zim gedâht.\\ 
 & \begin{large}G\end{large}awan \textbf{was} \textbf{gezimieret}\\ 
 & unde wol \textbf{geflôrieret}\\ 
5 & von rîcher koste helme vil.\\ 
 & si vuorten gegen ir nîtspil\\ 
 & \textbf{wîz} niuwer sper ein wunder,\\ 
 & diu \textbf{gemâlet} wâren besunder\\ 
 & junchêrren gegeben in die hant,\\ 
10 & ir hêrren wâpen dran \textbf{erkant}.\\ 
 & Gawan filliroys Lot\\ 
 & sach von gedrenge grôze nôt,\\ 
 & \textbf{vil} mûle, die harnasch muosen tragen,\\ 
 & unde manegen wol geladen wagen;\\ 
15 & den was \textbf{gegen} herbergen gâch.\\ 
 & \textbf{dâ} vuor der market hinden nâch\\ 
 & mit wunderlîcher pârât.\\ 
 & des was eht \textbf{dô} \textbf{kein} ander rât.\\ 
 & \textbf{er sach} der vrouwen dâ genuoc.\\ 
20 & etslîchiu den zwelften gürtel truoc\\ 
 & ze pfande nâch ir minne.\\ 
 & ez wâren niht küneginne:\\ 
 & die selben trippâniersen\\ 
 & hiezen soldiersen.\\ 
25 & hie der junge, dort der alde,\\ 
 & dâ \textbf{vuoren} vil ribalde.\\ 
 & \textbf{den machte ir loufen} müediu lide.\\ 
 & etslîcher zæme baz an der wide,\\ 
 & danner daz \textbf{volc} dâ mêrte\\ 
30 & unde werdez volc unêrte.\\ 
\end{tabular}
\scriptsize
\line(1,0){75} \newline
T V W \newline
\line(1,0){75} \newline
\textbf{3} \textit{Initiale} T W  \newline
\line(1,0){75} \newline
\textbf{2} des wart ouch] Des selben waz W \textbf{3} was] sach V \textbf{5} koste] kost sy W \textbf{6} si vuorten gegen] Fuͦrten auff W \textbf{11} filliroys] des kv́niges svn V  $\cdot$ Lot] lôt T \textbf{16} dâ] Do V W  $\cdot$ der] das W \textbf{17} wunderlîcher] wirdiglicher W \textbf{18} ander] \textit{om.} W \textbf{19} dâ] do V W \textbf{23} trippâniersen] Trimpenirsen T \textbf{26} dâ vuoren vil] Do fuͦr vil V Do lieffen W \textbf{28} Etlicher zaume was ein wide W \textbf{29} volc dâ] her do V W \textbf{30} werdez] [w*s]: werdes V \newline
\end{minipage}
\end{table}
\end{document}
