\documentclass[8pt,a4paper,notitlepage]{article}
\usepackage{fullpage}
\usepackage{ulem}
\usepackage{xltxtra}
\usepackage{datetime}
\renewcommand{\dateseparator}{.}
\dmyyyydate
\usepackage{fancyhdr}
\usepackage{ifthen}
\pagestyle{fancy}
\fancyhf{}
\renewcommand{\headrulewidth}{0pt}
\fancyfoot[L]{\ifthenelse{\value{page}=1}{\today, \currenttime{} Uhr}{}}
\begin{document}
\begin{table}[ht]
\begin{minipage}[t]{0.5\linewidth}
\small
\begin{center}*D
\end{center}
\begin{tabular}{rl}
\textbf{601} & \begin{large}S\end{large}waz \textbf{dâ} stuonden bluomen lieht,\\ 
 & die wâren gein dirre varwe \textbf{ein} nieht,\\ 
 & die Orgeluse brâhte.\\ 
 & Gawan an si \textbf{dâhte},\\ 
5 & sô daz sîn êrst ungemach\\ 
 & im deheines kumbers jach.\\ 
 & Sus reit si mit ir gaste\\ 
 & von der burc wol eine raste\\ 
 & eine strâzen wît und sleht\\ 
10 & vür ein \textbf{clârez} fôreht.\\ 
 & \textbf{der art des boume\textit{s} muosen} sîn\\ 
 & tämris unt prisîn;\\ 
 & \textbf{daz} was \textbf{der} \textbf{Clinschors} walt.\\ 
 & Gawan, der degen balt,\\ 
15 & sprach: "\textbf{vrouwe}, wâ brich ich den kranz,\\ 
 & \textbf{des} mîn dürkel \textbf{vreude} \textbf{werde} ganz?"\\ 
 & Er \textbf{solt}s \textbf{êt} hân \textbf{gedûhet} nider,\\ 
 & als dicke ist geschehen sider\\ 
 & maneger clâren vrouwen.\\ 
20 & si sprach: "ich lâze iuch schouwen,\\ 
 & \textbf{al dâ} ir prîs megt behaben."\\ 
 & über velt gein eime graben\\ 
 & riten si \textbf{sô} nâhen,\\ 
 & des kranzes boum si sâhen.\\ 
25 & Dô sprach si: "hêrre, \textbf{jenen} stam,\\ 
 & \textbf{den heiet}, der mir vreude nam.\\ 
 & bringet \textbf{ir} mir dar ab ein rîs,\\ 
 & nie rîter alsô hôhen prîs\\ 
 & mit dienste erwarp \textbf{durch} minne."\\ 
30 & \textbf{sus} sprach diu herzoginne:\\ 
\end{tabular}
\scriptsize
\line(1,0){75} \newline
D Z Fr7 \newline
\line(1,0){75} \newline
\textbf{1} \textit{Initiale} D Z  \textbf{7} \textit{Majuskel} D  \textbf{17} \textit{Majuskel} D  \textbf{25} \textit{Majuskel} D  \newline
\line(1,0){75} \newline
\textbf{2} ein] \textit{om.} Z \textbf{4} dâhte] gedahte Z \textbf{5} êrst] erster Z \textbf{9} strâzen] strazze Z \textbf{11} des boumes] des boͮme D die bovme Z \textbf{12} tämris] Tæmris D Tempris Z \textbf{13} der] den Z  $\cdot$ Clinschors] Clinscors D Clingores Z \textbf{15} vrouwe] \textit{om.} Z \textbf{16} vreude] hertze Z \textbf{17} solts êt] soldez Z \textbf{19} clâren] karen Z \textbf{20} schouwen] schowe Z \textbf{22} gein] bi Z \textbf{25} jenen] einen Fr7 \textbf{26} heiet] heget Fr7  $\cdot$ nam] benam Z \textbf{29} durch] nach Z \textbf{30} sus] So Z \newline
\end{minipage}
\hspace{0.5cm}
\begin{minipage}[t]{0.5\linewidth}
\small
\begin{center}*m
\end{center}
\begin{tabular}{rl}
 & waz \textbf{d\textit{â}} stuonden bluomen lieht,\\ 
 & die wâren gegen diser varwe niht,\\ 
 & die Urgeluse brâhte.\\ 
 & Gawan an si \textbf{gedâhte},\\ 
5 & sô daz sîn êrst ungemach\\ 
 & im dekeines kumbers jach.\\ 
 & sus reit si mit ir gast\\ 
 & von der burc wol ein rast\\ 
 & ein strâze wît und sleht\\ 
10 & vür ein \textbf{clârez} fôreht.\\ 
 & \textbf{der ar\textit{t} des boumes muoste} sîn\\ 
 & tä\textit{m}ris \textit{un}d prisîn;\\ 
 & \textbf{daz} was \textbf{des} \textbf{Clinsors} walt.\\ 
 & Gawan, der degen balt,\\ 
15 & sprach: "\textbf{vrouwe}, wâ brich ich den kranz,\\ 
 & \textbf{des} mîn dürkel \textbf{vröude} \textbf{wirt} ganz?"\\ 
 & er \textbf{sol} si hân \textbf{gedûhet} nider,\\ 
 & alsô dicke ist geschehen sider\\ 
 & maniger clâren vrouwen.\\ 
20 & si sprach: "ich lâze iuch schouwen,\\ 
 & \textbf{aldâ} ir prîs muget behaben."\\ 
 & über velt gegen einem graben\\ 
 & riten si \textbf{sô} nâhen,\\ 
 & des kranzes boum si sâhen.\\ 
25 & dô sprach si: "hêrre, \textbf{jene\textit{n}} \textit{s}tam,\\ 
 & \textbf{den \textit{h}eiet}, \textit{d}er mir vröude nam.\\ 
 & bringet mir d\textit{ar a}b ein rîs,\\ 
 & nie ritter alsô hôhen prîs\\ 
 & mit dienst erwarp \textbf{durch} minne."\\ 
30 & \textbf{sus} sprach diu herzoginne:\\ 
\end{tabular}
\scriptsize
\line(1,0){75} \newline
m n o \newline
\line(1,0){75} \newline
\newline
\line(1,0){75} \newline
\textbf{1} dâ] do m n o \textbf{5} êrst] erstes n \textbf{6} dekeines] do keines n \textbf{11} art] arte m o  $\cdot$ boumes] boume n (o) \textbf{12} tämris und] Teniris pud m Teniris vnd n Teneris vnd o \textbf{13} des] das o \textbf{16} des] Das o \textbf{17} sol] solt n o \textbf{21} muget] múgen n \textbf{24} kranzes boum si] kranckes bauͯm so o \textbf{25} jenen stam] jenen stein vnd stam m Jnnen stam o \textbf{26} heiet der] teiet er m \textbf{27} dar ab] den pris ab m das ab o \textbf{30} sus] Su n \newline
\end{minipage}
\end{table}
\newpage
\begin{table}[ht]
\begin{minipage}[t]{0.5\linewidth}
\small
\begin{center}*G
\end{center}
\begin{tabular}{rl}
 & swaz stuon\textit{den} bluomen lieht,\\ 
 & die wâren gein dirre varwe \textbf{ein} niht,\\ 
 & die Orgeluse brâhte.\\ 
 & Gawan an si \textbf{gedâhte},\\ 
5 & sô daz sîn êrster ungemach\\ 
 & ime deheines kumbers jach.\\ 
 & sus reit si mit ir gaste\\ 
 & von der bürge wol ein raste\\ 
 & ein strâze wît unde sleht\\ 
10 & vür ein \textbf{clârez} fôreht,\\ 
 & \textbf{d\textit{â} die boume muosen} sîn\\ 
 & tä\textit{m}ris unde prisîn;\\ 
 & \textbf{sus} was \textbf{der} \textbf{kleine} walt.\\ 
 & Gawan, der \textit{d}e\textit{gen} balt,\\ 
15 & sprach: "wâ briche ich den kranz,\\ 
 & \textbf{des} mîn dürkel \textbf{herze} \textbf{werde} ganz?"\\ 
 & er \textbf{solde} si hân \textbf{gedûht} nider,\\ 
 & als dicke ist geschehen sider\\ 
 & maniger clâren vrouwen.\\ 
20 & si sprach: "ich lâz iuch schouwen,\\ 
 & \textbf{wâ} ir prîs muget behaben."\\ 
 & über \textbf{daz} velt gein einem graben\\ 
 & riten si \textbf{dô} nâhen.\\ 
 & des kranzes boum s\textit{i s}âhen.\\ 
25 & dô sprach si: "hêrre, \textbf{jener} stam\\ 
 & \textbf{\textit{ist d}e\textit{s}}, der mir vröude nam.\\ 
 & bringet \textbf{ir} mir dar abe ein rîs,\\ 
 & nie rîter alsô hôhen prîs\\ 
 & mit dienst erwarp \textbf{nâch} minne."\\ 
30 & \textbf{dô} sprach diu herzoginne:\\ 
\end{tabular}
\scriptsize
\line(1,0){75} \newline
G I L M Z Fr51 \newline
\line(1,0){75} \newline
\textbf{1} \textit{Initiale} L Z  \textbf{15} \textit{Initiale} I  \textbf{25} \textit{Initiale} M  \newline
\line(1,0){75} \newline
\textbf{1} swaz] swa I Waz L (M)  $\cdot$ stuonden] stuͦnt G da stuͯnden L (M) (Z)  $\cdot$ lieht] lýcht L (M) \textbf{2} wâren] en waren M  $\cdot$ ein] \textit{om.} L M Z \textbf{3} Orgeluse] Orguluse I Orgelýse L \textbf{5} êrster] erste L M \textbf{10} ein clârez] einen chlaren I ein reinz L \textbf{11} dâ] do G Der art L M  $\cdot$ boume] bluͦmen I \textbf{12} tämris] Ten pris G den pris I Tampris L Tempris M \textbf{13} Daz was den Clingores walt Z  $\cdot$ der kleine] der klare I \textbf{14} degen] helt G \textbf{15} sprach] srach I \textbf{16} herze] vrode L \textbf{17} solde si] solte sich L soldez Z  $\cdot$ gedûht] genehit M \textbf{19} clâren] schonen L karen Z \textbf{20} schouwen] schowe Z \textbf{21} wâ] Al da Z Dar Fr51  $\cdot$ muget] mogen Fr51  $\cdot$ behaben] gehaben I beiagen L (M) Fr51 \textbf{22} daz] \textit{om.} L M Z iz Fr51  $\cdot$ gein] bi Z  $\cdot$ einem] ienem I \textbf{23} dô] so L M Z Fr51 \textbf{24} si sâhen] si do sahen G \textbf{25} dô] Da L  $\cdot$ jener] ienen Z (Fr51) \textbf{26} ist des] heizet G (L) (M) Fr51 Den heyet Z  $\cdot$ der] ir Fr51  $\cdot$ nam] benam Z \textbf{27} ir] \textit{om.} Fr51 \textbf{28} hôhen] sohen Fr51 \textbf{29} erwarp] warp I (Fr51) \textbf{30} dô] So M Z Dus Fr51  $\cdot$ herzoginne] chunginne I \newline
\end{minipage}
\hspace{0.5cm}
\begin{minipage}[t]{0.5\linewidth}
\small
\begin{center}*T
\end{center}
\begin{tabular}{rl}
 & waz \textbf{dâ} stuonden bluomen lieht,\\ 
 & die wâren gein dirre varwe \textbf{ein} niht,\\ 
 & die Or\textit{ge}luse brâhte.\\ 
 & Gawan an si \textbf{gedâhte},\\ 
5 & sô daz sîn êrstez ungemach\\ 
 & im dekeines kumbers jach.\\ 
 & sus reit si mit ir gaste\\ 
 & von der burc wol ein raste\\ 
 & eine strâze wît und sleht\\ 
10 & vür ein \textbf{kleinez} fôreht.\\ 
 & \textbf{der art die boume muosen} sîn\\ 
 & tä\textit{m}ris und prisîn;\\ 
 & \textbf{daz} was \textbf{Clynsors} walt.\\ 
 & Gawan, der degen balt,\\ 
15 & sprach: "\textbf{vrouwe}, wâ brich ich den kranz,\\ 
 & \textbf{daz} mîn dürkel \textbf{herze} \textbf{würde} ganz?"\\ 
 & \begin{large}E\end{large}r \textbf{solte} si hân \textbf{gedrücket} nider,\\ 
 & als dicke ist geschehen sider\\ 
 & maneger clâren vrouwen.\\ 
20 & si sprach: "ich lâz iuch schouwen,\\ 
 & \textbf{wâ} ir prîs moget behaben."\\ 
 & über velt gein eime graben\\ 
 & riten si \textbf{sô} nâhen,\\ 
 & des kranzes boum si sâhen.\\ 
25 & dô sprach si: "hêrre, \textbf{jener} stam,\\ 
 & \textbf{den heiet}, der mir vreude nam.\\ 
 & bringet \textbf{ir} mir drabe ein rîs,\\ 
 & nie rîter als hôhen prîs\\ 
 & mit dienste erwarp \textbf{nâch} minne."\\ 
30 & \textbf{sô} sprach diu herzoginne:\\ 
\end{tabular}
\scriptsize
\line(1,0){75} \newline
U V W Q R \newline
\line(1,0){75} \newline
\textbf{17} \textit{Überschrift:} Hie kvmet gawan zvͦ gramaflanz do er den cranz in sime garten brechen wil V  · Initiale U V  \newline
\line(1,0){75} \newline
\textbf{1} waz] Swa V  $\cdot$ dâ] do V W Q  $\cdot$ lieht] licht Q \textbf{2} varwe] varwen U [frawen]: farwe Q  $\cdot$ ein niht] niecht V entnicht W gar en nicht R \textbf{3} Orgeluse] oriluse U origeluse W orgelusze Q orguluse R \textbf{4} Gawan] Gawin R \textbf{5} êrstez] erster Q erste R \textbf{7} si] er R \textbf{10} kleinez] [clo*]: clores V kleine W  $\cdot$ fôreht] forcht forecht R \textbf{11} die] der W  $\cdot$ muosen] [m*en]: mvͤsten V muͦße W \textbf{12} [Denn]: Den pris vnd pris in U  $\cdot$ [Tem*]: Tempris vnde prisin V  $\cdot$ Tempreiß vnd prisein W  $\cdot$ Tembrisz vnd prisin Q  $\cdot$ Tembris vnd prisin R \textbf{13} Clynsors] clinsors V klinshors W klinszhors Q clinshors R \textbf{14} der] der teg der R \textbf{15} vrouwe] \textit{om.} Q  $\cdot$ den] den úwern R \textbf{16} dürkel] dunkel V trurig R  $\cdot$ herze] froͤide V  $\cdot$ würde] werde V (W) R \textbf{17} gedrücket] gedauchet W (Q) \textbf{19} maneger] Manchen Q \textbf{21} moget] mugen R  $\cdot$ behaben] beiagen Q behaltten R \textbf{22} graben] graben alten R \textbf{25} jener] ienen W Q Iena R \textbf{26} heiet] heizt Q het R  $\cdot$ mir] minne W  $\cdot$ vreude] freúden Q (R) \textbf{29} nâch] [*]: durch V \textbf{30} sô] Do Q \newline
\end{minipage}
\end{table}
\end{document}
