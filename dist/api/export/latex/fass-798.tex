\documentclass[8pt,a4paper,notitlepage]{article}
\usepackage{fullpage}
\usepackage{ulem}
\usepackage{xltxtra}
\usepackage{datetime}
\renewcommand{\dateseparator}{.}
\dmyyyydate
\usepackage{fancyhdr}
\usepackage{ifthen}
\pagestyle{fancy}
\fancyhf{}
\renewcommand{\headrulewidth}{0pt}
\fancyfoot[L]{\ifthenelse{\value{page}=1}{\today, \currenttime{} Uhr}{}}
\begin{document}
\begin{table}[ht]
\begin{minipage}[t]{0.5\linewidth}
\small
\begin{center}*D
\end{center}
\begin{tabular}{rl}
\textbf{798} & \begin{large}T\end{large}revrizent ze Parzivale sprach:\\ 
 & "grœzer wunder selten ie geschach,\\ 
 & sît ir \textbf{ab} \textbf{got} erzürnet hât,\\ 
 & daz sîn endelôsiu trinitât\\ 
5 & iwers willen \textbf{werschaft} worden ist.\\ 
 & ich louc durch ableitens list\\ 
 & vome Grâle, wie ez umb in stüende.\\ 
 & gebt mir wandel vür die sünde.\\ 
 & Ich sol gehôrsam iu nû sîn,\\ 
10 & swestersun unt \textbf{der} hêrre mîn.\\ 
 & daz die vertribene geiste\\ 
 & mit der gotes volleiste\\ 
 & bî dem Grâle wæren,\\ 
 & \textbf{kom} \textbf{iu von} mir ze mæren,\\ 
15 & unze daz si hulde dâ gebiten.\\ 
 & got ist stæte mit sölhen siten,\\ 
 & er strîtet iemer \textbf{wider} sie,\\ 
 & die ich \textbf{iu} ze \textbf{hulden} nante hie.\\ 
 & Swer sînes lônes iht wil tragen,\\ 
20 & der muoz den selben widersagen.\\ 
 & êweclîch \textbf{sint si} verlorn.\\ 
 & die vlust si selbe hânt erkorn.\\ 
 & Mich müet \textbf{êt} iwer arbeit.\\ 
 & ez was ie ungewonheit,\\ 
25 & daz den Grâl ze keinen zîten\\ 
 & iemen mohte erstrîten.\\ 
 & ich het iuch gern dâ von genomen.\\ 
 & nû ist ez anders umb iuch komen.\\ 
 & sich hât gehœhet iwer gewin.\\ 
30 & nû kêrt an \textbf{diemuot} iwern sin."\\ 
\end{tabular}
\scriptsize
\line(1,0){75} \newline
D \newline
\line(1,0){75} \newline
\textbf{1} \textit{Initiale} D  \textbf{9} \textit{Majuskel} D  \textbf{19} \textit{Majuskel} D  \textbf{23} \textit{Majuskel} D  \newline
\line(1,0){75} \newline
\textbf{1} Parzivale] Parcifale D \newline
\end{minipage}
\hspace{0.5cm}
\begin{minipage}[t]{0.5\linewidth}
\small
\begin{center}*m
\end{center}
\begin{tabular}{rl}
 & \textit{Trevrizent} zuo Parcifal sprach:\\ 
 & "grœzer wunde\textit{r} selten ie geschach,\\ 
 & sît ir \textbf{ab} \textbf{got} erzürnet hât,\\ 
 & daz sîn endelôsiu trinitât\\ 
5 & iuwers willen \textbf{werhaft} worden ist.\\ 
 & ich louc \textit{durch} ableite\textit{n}s list\\ 
 & von dem Grâl, wie ez umb in stüende.\\ 
 & gebt mir wandel vür die sünde.\\ 
 & ich sol gehôrsam iu nû sîn,\\ 
10 & swestersun und hêrre mîn.\\ 
 & daz die vertriben geist\\ 
 & mit der gotes volleist\\ 
 & bî dem Grâl w\textit{æ}ren,\\ 
 & \textbf{kam} \textbf{iu von} mir ze mæren,\\ 
15 & unz daz si hulde d\textit{â} gebiten.\\ 
 & got ist stæte mit soliche\textit{n} siten,\\ 
 & er strîtet iemer \textbf{wider} sie,\\ 
 & die ich \textbf{iu} ze \textbf{hulden} nante hie.\\ 
 & wer sînes lônes iht wil tragen,\\ 
20 & der muoz den selben widersagen.\\ 
 & êweclîch \textbf{sint si} verlorn.\\ 
 & die vlust si \textbf{alle} selbe hânt erkorn.\\ 
 & mich müejet \textbf{eht} iuwer arbeit.\\ 
 & ez was ie ungewonheit,\\ 
25 & daz den Grâl zuo keinen zîten\\ 
 & ieman mohte erstrîten.\\ 
 & ich het iuch gern dâ von genomen.\\ 
 & nû ist ez anders umb iuch komen.\\ 
 & si\textit{ch} het gehœhet iuwer gewin.\\ 
30 & nû kêrt an \textbf{diemuot} iuwern sin."\\ 
\end{tabular}
\scriptsize
\line(1,0){75} \newline
m n o V V' W \newline
\line(1,0){75} \newline
\textbf{1} \textit{Initiale} V  \newline
\line(1,0){75} \newline
\textbf{1} \textit{Die Verse 797.13-798.30 fehlen} V'   $\cdot$ Trevrizent] Ferefis m Ferrefize n Fereficze o Treuerrzent V Treuerrissent W  $\cdot$ Parcifal] parzefale V partzifale W \textbf{2} wunder] wunden m  $\cdot$ ie] ee W \textbf{3} hât] [wasz]: hat o \textbf{5} werhaft] worhafft n werschaft V  $\cdot$ worden] \textit{om.} W \textbf{6} durch] \textit{om.} m  $\cdot$ ableitens] ableittes m \textbf{11} vertriben] vertribenen W \textbf{13} wæren] woren m \textbf{15} dâ] do m n o V W \textbf{16} solichen] sollichem m n (o) [soliche*]: solichen  V \textbf{18} ze hulden] zehulde V \textbf{19} wer] Wasz wer o Swer V \textbf{21} sint si] sv́ sint V \textbf{22} vlust] verflust n  $\cdot$ alle] \textit{om.} n o V W  $\cdot$ selbe] selber W \textbf{23} mich] \textit{om.} o  $\cdot$ müejet] muͦte V \textbf{24} ie] nie W \textbf{26} mohte] moͯchte n (V) (W) \textbf{29} sich] Sit m \textbf{30} sin] [sin gewin]: sin o \newline
\end{minipage}
\end{table}
\newpage
\begin{table}[ht]
\begin{minipage}[t]{0.5\linewidth}
\small
\begin{center}*G
\end{center}
\begin{tabular}{rl}
 & \begin{large}T\end{large}revrizzent ze Parzival sprach:\\ 
 & "grœzer wunder selten ie geschach,\\ 
 & sît ir \textbf{aber} erzürnet hât,\\ 
 & daz sîn endelôs trinitât\\ 
5 & iwers willen \textbf{werhaft} worden ist.\\ 
 & ich louc durch ableitens list\\ 
 & vonme Grâle, wie ez umb in stüende.\\ 
 & gebt mir wandel vür die sünde.\\ 
 & ich sol gehôrsam iu nû sîn,\\ 
10 & swestersun unde \textbf{der} hêrre mîn.\\ 
 & daz die vertribenen geiste\\ 
 & mit der gotes volleiste\\ 
 & bî dem Grâle wæren\\ 
 & \textbf{komen} mir ze mæren,\\ 
15 & unze daz si hulde dâ gebiten.\\ 
 & got ist stæte mit sölhen siten,\\ 
 & er strît imer \textbf{an} sie,\\ 
 & die ic\textit{h} \textit{z}e \textbf{hulden} nande hie.\\ 
 & swer sînes lônes iht wil tragen,\\ 
20 & der muoz den selben widersagen.\\ 
 & êwiclîch \textbf{si sint} verlorn.\\ 
 & die vlust si selbe hânt erkorn.\\ 
 & mich müet \textbf{êt} iwer arbeit.\\ 
 & ez was ie ungewonheit,\\ 
25 & daz den Grâl ze deheinen zîten\\ 
 & iemen mohte erstrîten.\\ 
 & ich het iuch gerne dâr von genomen.\\ 
 & nû ist ez anders umbe iuch komen.\\ 
 & sich hât gehœhet iwer gewin.\\ 
30 & nû kêrt an \textbf{diemuot} iwern sin."\\ 
\end{tabular}
\scriptsize
\line(1,0){75} \newline
G I L M Z Fr48 \newline
\line(1,0){75} \newline
\textbf{1} \textit{Initiale} G I Z Fr48  \textbf{15} \textit{Initiale} I  \newline
\line(1,0){75} \newline
\textbf{1} \textit{Die Verse 798.1-30 fehlen} L   $\cdot$ Trevrizzent] Trevrizent G Treuerezent I Trefrezent M Trefrizzent Z TReurizzent Fr48  $\cdot$ Parzival] parcifal G Z Parzifaln I parzifal M partzifal Fr48 \textbf{3} ir aber] aber er I ir abir got M ir ab got Z Fr48  $\cdot$ erzürnet] Gezurnet I \textbf{5} werhaft] Gewer I \textbf{7} stüende] stuͦnt Fr48 \textbf{8} vür] vmb Z \textbf{11} vertribenen] vortribene M \textbf{13} wæren] waren M \textbf{14} komen] chomen von I (M) Qvam ev von Z (Fr48) \textbf{15} unze daz] Bisz das M Vntz Z \textbf{18} ich ze] ich nv ze G uch zcu M (Z) (Fr48) \textbf{19} swer] Wer M \textbf{21} si sint] sint hie M \textbf{22} selbe] selben M  $\cdot$ erkorn] vorkorn M \textbf{23} êt] \textit{om.} M Z Fr48 \textbf{25} ze] an Z Fr48 \textbf{26} mohte] [moht]: moͤht Fr48 \textbf{30} diemuot] den muͤt I (M)  $\cdot$ iwern] uwer M \newline
\end{minipage}
\hspace{0.5cm}
\begin{minipage}[t]{0.5\linewidth}
\small
\begin{center}*T
\end{center}
\begin{tabular}{rl}
 & \begin{large}T\end{large}refrizent zuo Parcifale sprach:\\ 
 & "grœzer wunder selten ie geschach,\\ 
 & sît \textit{ir} \textbf{aber} \textbf{go\textit{t}} erz\textit{ür}net hât,\\ 
 & daz sîn endelôsiu trinitât\\ 
5 & iuwers willen \textbf{werhaft} worden ist.\\ 
 & ich \textit{louc} durch abeleitens list\\ 
 & von dem Grâle, wie ez umb in stüende.\\ 
 & gebet mir wandel vür die sünde.\\ 
 & ich sol gehôrsam iu nû sîn,\\ 
10 & swestersun und hêrre mîn.\\ 
 & daz die vertribenen geiste\\ 
 & mit der gotes volleiste\\ 
 & bî dem Grâle wæren,\\ 
 & \textbf{kam} \textbf{iu von} mir zuo mæren,\\ 
15 & unz daz si hulde dâ gebiten.\\ 
 & got ist stæte mit solichen siten,\\ 
 & er strîtet imer \textbf{an} sie,\\ 
 & die \textit{ich} \textbf{iu} zuo \textbf{hulde} nante hie.\\ 
 & wer sînes lônes iht wil tragen,\\ 
20 & der muoz den selben widersagen.\\ 
 & êweclîche \textbf{sint si} verlorn.\\ 
 & die verlust si selber hânt \textit{e}rkorn.\\ 
 & mich müewet iuwer arbeit.\\ 
 & ez was ie ungewonheit,\\ 
25 & daz den Grâl zuo dekeinen zîten\\ 
 & ieman mohte erstrîten.\\ 
 & ich hete \textit{i}uch gerne dâr von genomen.\\ 
 & nû ist ez anders umb iuch komen.\\ 
 & sich hât gehœhet iuwer gewin.\\ 
30 & nû \textit{kêrt} an \textbf{den muot} iuwern sin."\\ 
\end{tabular}
\scriptsize
\line(1,0){75} \newline
U Q R \newline
\line(1,0){75} \newline
\textbf{1} \textit{Initiale} U  \newline
\line(1,0){75} \newline
\textbf{1} Trefrizent] Preffrizent Q Treifriczent R  $\cdot$ zuo] vnd R  $\cdot$ Parcifale] Parzifale U partzifale Q barczifal R \textbf{2} ie] nye Q \textbf{3} ir aber got] abir go U aber daz ir got R  $\cdot$ erzürnet] erzinnet U \textbf{4} endelôsiu] endlose R \textbf{5} werhaft] warhafftt R \textbf{6} louc] \textit{om.} U  $\cdot$ abeleitens] ableittes R \textbf{7} von dem] Vmb den Q \textbf{9} sol] so Q \textbf{10} hêrre] der herre Q R \textbf{11} vertribenen] vertriben Q (R) \textbf{13} wæren] wauren R \textbf{15} unz] Mit U  $\cdot$ dâ] do Q R \textbf{18} ich] \textit{om.} U  $\cdot$ hulde] hulden Q (R) \textbf{21} sint si] sie seint Q (R) \textbf{22} die] Den R  $\cdot$ erkorn] verkorn U \textbf{23} müewet] múet auch Q muͯt echt R \textbf{24} ez] Er Q  $\cdot$ ungewonheit] von gewonheit Q \textbf{25} den] dem R  $\cdot$ dekeinen] [deine*]: demheinen R \textbf{26} mohte] moͯchtte R \textbf{27} iuch] auch U  $\cdot$ gerne] gernen Q \textbf{28} ez] er Q  $\cdot$ iuch] in R \textbf{29} sich] Sicht Q \textbf{30} kêrt] \textit{om.} U  $\cdot$ den muot] demút Q (R) \newline
\end{minipage}
\end{table}
\end{document}
