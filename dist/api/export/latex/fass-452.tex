\documentclass[8pt,a4paper,notitlepage]{article}
\usepackage{fullpage}
\usepackage{ulem}
\usepackage{xltxtra}
\usepackage{datetime}
\renewcommand{\dateseparator}{.}
\dmyyyydate
\usepackage{fancyhdr}
\usepackage{ifthen}
\pagestyle{fancy}
\fancyhf{}
\renewcommand{\headrulewidth}{0pt}
\fancyfoot[L]{\ifthenelse{\value{page}=1}{\today, \currenttime{} Uhr}{}}
\begin{document}
\begin{table}[ht]
\begin{minipage}[t]{0.5\linewidth}
\small
\begin{center}*D
\end{center}
\begin{tabular}{rl}
\textbf{452} & \begin{large}E\end{large}r sprach: "ist gotes kraft sô fier,\\ 
 & daz si beidiu ors unde tier\\ 
 & unt die liute mac wîsen,\\ 
 & sîne kraft wil ich im prîsen.\\ 
5 & mac gotes \textbf{kunst} die helfe hân,\\ 
 & diu wîse mir diz kastelân\\ 
 & daz wægest umbe die reise mîn.\\ 
 & sô tuot sîn \textbf{güete helfe} schîn.\\ 
 & \textbf{nû} \textbf{genc} nâch der gotes kür."\\ 
10 & den zügel gein den ôren vür\\ 
 & er dem orse legete.\\ 
 & mit den sporn erz vaste regete.\\ 
 & gein Fontane lasalvatsche ez gienc,\\ 
 & dâ Orilus den eit enpfienc.\\ 
15 & der kiusche Trevrizent dâ saz,\\ 
 & der manegen mântac übel geaz,\\ 
 & als tet er gar die wochen.\\ 
 & er het gar versprochen\\ 
 & Môraz, wîn unt \textbf{ouchz} brôt.\\ 
20 & sîn kiusche \textbf{im} \textbf{dannoch mêr} gebôt:\\ 
 & der spîse het er decheinen muot,\\ 
 & \textbf{vische noch vleisch}, swaz trüege bluot.\\ 
 & sus stuont sîn \textbf{heileclîchez} leben.\\ 
 & got het im den muot gegeben:\\ 
25 & der hêrre sich bereite gar\\ 
 & gein der himelischen schar.\\ 
 & mit \textbf{vaste} er grôzen kumber leit.\\ 
 & sîn kiusche gein dem tievel \textbf{streit}.\\ 
 & an dem ervert nû Parzival\\ 
30 & diu verholniu mære umben Grâl.\\ 
\end{tabular}
\scriptsize
\line(1,0){75} \newline
D Fr5 \newline
\line(1,0){75} \newline
\textbf{1} \textit{Initiale} D Fr5  \textbf{19} \textit{Majuskel} D  \newline
\line(1,0){75} \newline
\textbf{4} im] \textit{om.} Fr5 \textbf{6} diz] daz Fr5 \textbf{13} Fontane lasalvatsche] funtane saluasche Fr5  $\cdot$ ez] en Fr5 \textbf{15} Trevrizent] Trevrizzent D treuresent Fr5 \textbf{18} gar] [gar vnt]: gar D \textbf{22} noch vleisch] fleisch noch Fr5  $\cdot$ trüege] treit Fr5 \textbf{23} Sus herticliche stuͦnt sin lebin Fr5 \textbf{24} den] dein Fr5 \textbf{29} an] Ab Fr5  $\cdot$ Parzival] Parcifal D (Fr5) \textbf{30} diu verholniu] Daz verholne Fr5 \newline
\end{minipage}
\hspace{0.5cm}
\begin{minipage}[t]{0.5\linewidth}
\small
\begin{center}*m
\end{center}
\begin{tabular}{rl}
 & er sprach: "ist gotes kraft sô fier,\\ 
 & daz si beidiu ros und tier\\ 
 & und die liute mac wîsen,\\ 
 & sîn kraft wil ich im prîsen.\\ 
5 & ma\textit{c} g\textit{o}tes \textbf{gunst} die helfe hân,\\ 
 & diu wîse mir diz kastelân\\ 
 & daz wægest umb \textit{die} reise mîn.\\ 
 & sô tuot sîn \textbf{güete helfe} schîn.\\ 
 & \textbf{nû} \textbf{ganc} nâch der gotes kür."\\ 
10 & den zügel gegen den ôren vür\\ 
 & er dem rosse leget.\\ 
 & mit den sporn er ez vast reget.\\ 
 & gegen Funtaine lasalvasche ez gienc,\\ 
 & d\textit{â} Orilus den eit enpfienc.\\ 
15 & der kiusch Tre\textit{v}r\textit{i}zent d\textit{â} saz,\\ 
 & der manigen mântac übel geaz,\\ 
 & alsô tet er gar die wochen.\\ 
 & er het gar versprochen\\ 
 & môraz, wîn und \textbf{ouch daz} brôt.\\ 
20 & sîn kiusch \textbf{i\textit{m}} \textbf{dannoch \textit{m}ê} gebôt:\\ 
 & der spîse \textbf{en}het er keinen muot,\\ 
 & \textbf{vleisch noch visch}, waz trüege b\textit{l}uot.\\ 
 & sus stuont sîn \textbf{heiligez} leben.\\ 
 & got het im den muot gegeben:\\ 
25 & der hêrre sich bereitet gar\\ 
 & gegen der himelschen schar.\\ 
 & mit \textbf{vasten} er grôzen kumber leit.\\ 
 & sîn kiusch gegen dem tiuvel \textbf{reit}.\\ 
 & an dem er\textit{v}e\textit{r}t nû Parcifal\\ 
30 & diu verholnen mære umb den Grâl.\\ 
\end{tabular}
\scriptsize
\line(1,0){75} \newline
m n o \newline
\line(1,0){75} \newline
\newline
\line(1,0){75} \newline
\textbf{4} wil ich im] vil ich ẏmb o \textbf{5} mac gotes] Magttes m \textbf{6} diz] des o \textbf{7} die] \textit{om.} m \textbf{10} den] Zuͦ den o \textbf{12} \textit{Die Verse 452.12-30 fehlen} o   $\cdot$ reget] weget n \textbf{13} Funtaine lasalvasche] funtaine lasaluasce m funtanie salauasce n \textbf{14} dâ] Do m n \textbf{15} Trevrizent] trerurzent m treurizent n  $\cdot$ dâ] do m n \textbf{20} im] in m  $\cdot$ mê] nie m \textbf{22} bluot] buͯt m \textbf{23} heiligez] heilecliches n \textbf{24} het] hat n  $\cdot$ gegeben] geben n \textbf{28} reit] streit n \textbf{29} ervert] erneret m ernert n \textbf{30} diu] Denne die n \newline
\end{minipage}
\end{table}
\newpage
\begin{table}[ht]
\begin{minipage}[t]{0.5\linewidth}
\small
\begin{center}*G
\end{center}
\begin{tabular}{rl}
 & \begin{large}E\end{large}r sprach: "ist gotes kraft sô fier,\\ 
 & daz si beidiu ors unde tier\\ 
 & unde die liute mac wîsen,\\ 
 & sîne krafte wil ich im brîsen.\\ 
5 & mac gotes \textbf{kunst} die helfe hân,\\ 
 & diu wîse mir ditze kastelân\\ 
 & daz wægest umbe die reise mîn.\\ 
 & sô tuot sîn \textbf{güete helfe} schîn.\\ 
 & \textbf{nû} \textbf{genc} nâch der gotes kür."\\ 
10 & den zügel gein den ôren vür\\ 
 & er dem orse legete.\\ 
 & mit den sporn erz vaste regete.\\ 
 & gein Funtane lasalvatsche ez gienc,\\ 
 & dâ Orillus den eit enpfienc.\\ 
15 & der kiusche Trevrizzent dâ saz,\\ 
 & der manigen mântac übel geaz,\\ 
 & als tet er gar die wochen.\\ 
 & er hete gar versprochen\\ 
 & môraz, wîn unde \textbf{ouch daz} brôt.\\ 
20 & sîn kiusche \textbf{im} \textbf{dannoch mêr} gebôt:\\ 
 & der spîse het er deheinen muot,\\ 
 & \textbf{vische noch vleisch}, swaz trüege bluot.\\ 
 & sus stuont sîn \textbf{heileclîchez} leben.\\ 
 & got het im de\textit{n} muot gegeben:\\ 
25 & der hêrre sich bereite gar\\ 
 & gein der himelischen schar.\\ 
 & mit \textbf{vasten} er grôzen kumber leit.\\ 
 & sîn kiusche gein dem \textit{ti}evel \textbf{streit}.\\ 
 & an dem ervert nû Parzival\\ 
30 & diu verholnen mære umb den Grâl.\\ 
\end{tabular}
\scriptsize
\line(1,0){75} \newline
G I O L M Z \newline
\line(1,0){75} \newline
\textbf{1} \textit{Initiale} G I O L Z  \newline
\line(1,0){75} \newline
\textbf{1} Er] ÷r O  $\cdot$ gotes kraft sô] got fo O \textbf{4} sîne krafte] sin craft I (O) (Z)  $\cdot$ ich im] ich L ich yn M \textbf{5} kunst] chraft O \textbf{6} ditze] [daz]: dýz L \textbf{8} so tuͤt got sin helfe shin I  $\cdot$ So tuͯt sin helfe mir guͯte schin L  $\cdot$ sîn] sie M  $\cdot$ güete] Gyͦte O \textbf{9} genc nâch] giench O ging nach L  $\cdot$ der] des Z \textbf{10} den] [Der]: Den O \textbf{12} den] \textit{om.} I O \textbf{13} Funtane lasalvatsche] fundane laseuasche I fontanie Mvntsalvatsche O Fontanie salvatsche L funkane la salvatsche M fontane Lasalvatsch Z \textbf{14} Orillus] Orilus I (O) L M (Z) \textbf{15} Trevrizzent] drevrizent I Trefrizent O Trefriszent L trefrenzent M \textbf{17} \textit{Versfolge 452.18-17} O   $\cdot$ gar die] aldy M \textbf{19} daz] \textit{om.} O M \textbf{20} sîn] Dy M  $\cdot$ gebôt] bot Z \textbf{22} vische noch vleisch] vish vleis vnd I  $\cdot$ swaz] waz L (M) \textbf{23} sus] alsus I  $\cdot$ heileclîchez] heiligez I (L) (M) \textbf{24} het] hat M  $\cdot$ den] dem G \textbf{26} himelischen] himlichen I (O) heymelichen M \textbf{28} dem] \textit{om.} Z  $\cdot$ tievel] nefel G \textbf{29} ervert] erwarp L dervert Z  $\cdot$ Parzival] parziual G Parzifal I L (M) Barcifal O parcifal Z \newline
\end{minipage}
\hspace{0.5cm}
\begin{minipage}[t]{0.5\linewidth}
\small
\begin{center}*T
\end{center}
\begin{tabular}{rl}
 & er sprach: "ist gotes kraft sô fier,\\ 
 & daz si beidiu ors unde tier\\ 
 & unde die liute mac wîsen,\\ 
 & sîne kraft wil ich im prîsen.\\ 
5 & mac gotes \textbf{kunst} die helfe hân,\\ 
 & di\textit{u} wîse mir diz kastelân\\ 
 & daz wægeste umbe die reise mîn.\\ 
 & sô tuot sîn \textbf{helfe güete} schîn.\\ 
 & \textbf{\begin{large}D\end{large}ô} \textbf{ergiengez} nâch der gotes kür."\\ 
10 & den zügel gegen den ôren vür\\ 
 & er dem orse legete.\\ 
 & mit den sporn erz vaste regete.\\ 
 & gegen Fontane lasalvatsche ez gienc,\\ 
 & dâ Orilus den eit enpfienc.\\ 
15 & der kiusche Trefrizent dâ saz,\\ 
 & der manegen mântac übel geaz,\\ 
 & als tet er gar die wochen.\\ 
 & er hete gar versprochen\\ 
 & môraz, wîn unde brôt.\\ 
20 & sîn kiusche \textbf{mêr dannoch} gebôt:\\ 
 & der spîse heter deheinen muot,\\ 
 & \textbf{vische noch vleisch}, swaz trüege bluot.\\ 
 & sus stuont sîn \textbf{heileclîchez} leben.\\ 
 & got hetim den muot gegeben:\\ 
25 & der hêrre sich bereite gar\\ 
 & gegen der himelschen schar.\\ 
 & mit \textbf{vastene} er grôzen kumber leit.\\ 
 & sîn kiusche gegen dem tiuvele \textbf{streit}.\\ 
 & an dem ervert nû Parcifal\\ 
30 & die verholnen mære umbe den Grâl.\\ 
\end{tabular}
\scriptsize
\line(1,0){75} \newline
T U V W Q R \newline
\line(1,0){75} \newline
\textbf{1} \textit{Initiale} W Q R  \textbf{9} \textit{Initiale} T U V  \newline
\line(1,0){75} \newline
\textbf{3} wîsen] gewisen V \textbf{4} wil] sol W  $\cdot$ im] in Q \textbf{5} kunst] krafft W \textbf{6} diu] die T So R  $\cdot$ wîse] zeige W  $\cdot$ kastelân] castilon R \textbf{7} daz] Vnd Q  $\cdot$ wægeste] beste W \textbf{8} helfe güete] helfe guͦten U guͤte hilffe W (Q) \textbf{9} NV [*]: gang noch der gottez kv́r V  $\cdot$ Dô] Nuͦ U (W) (Q) (R)  $\cdot$ ergiengez] gieng W (Q) (R) \textbf{10} den zügel gegen] Der zigel gieng R  $\cdot$ ôren] [or*n]: oren V \textbf{12} den sporn] [dem]: den sporn V dem spor R  $\cdot$ erz] er U  $\cdot$ regete] wegete U \textbf{13} Fontane lasalvatsche] fontange de Salvasce T fontange de salvasch U [font*]: fontanie la saluasche V fontage de saluatsche W funtangen von saluasche Q montanie desaluasche R \textbf{14} dâ] Do U V W Q  $\cdot$ den eit] [de* vieng vnd]: den eẏd R \textbf{15} kiusche] konick Q  $\cdot$ Trefrizent] Trefricent T trefizent V treuerissent W grefissent Q  $\cdot$ dâ] do V W Q R \textbf{16} geaz] aß W (Q) \textbf{17} tet] [*et]: tet V \textbf{18} hete] hat R \textbf{19} wîn] [win]: wein Q \textbf{20} sîn] Die W  $\cdot$ mêr dannoch] im dan mer U im dannoch me V W (Q) (R) \textbf{22} vische] [*wisce]: visce T  $\cdot$ swaz] waz U V (W) (Q) (R) \textbf{23} heileclîchez] [*]: heilicliches V hellichiches R \textbf{24} hetim] get im U hat im R  $\cdot$ den] [dem]: den V \textbf{25} hêrre] hette Q \textbf{29} ervert] erfuͦr R  $\cdot$ Parcifal] parzifal V partzifal W Q parczifal R \textbf{30} verholnen] verholne V W Rechtten R \newline
\end{minipage}
\end{table}
\end{document}
