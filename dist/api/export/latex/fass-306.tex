\documentclass[8pt,a4paper,notitlepage]{article}
\usepackage{fullpage}
\usepackage{ulem}
\usepackage{xltxtra}
\usepackage{datetime}
\renewcommand{\dateseparator}{.}
\dmyyyydate
\usepackage{fancyhdr}
\usepackage{ifthen}
\pagestyle{fancy}
\fancyhf{}
\renewcommand{\headrulewidth}{0pt}
\fancyfoot[L]{\ifthenelse{\value{page}=1}{\today, \currenttime{} Uhr}{}}
\begin{document}
\begin{table}[ht]
\begin{minipage}[t]{0.5\linewidth}
\small
\begin{center}*D
\end{center}
\begin{tabular}{rl}
\textbf{306} & \textbf{unz} iuch mîn herze erkande,\\ 
 & \textbf{dô} mich an vröuden pfande\\ 
 & Keie, der mich \textbf{dô} sô sluoc.\\ 
 & daz habt \textbf{gerochen ir} genuoc.\\ 
5 & Ich kust iuch, wære ich kusses wert."\\ 
 & "des het ich hiute sân gegert",\\ 
 & sprach Parzival, "get\textit{ö}rst ich sô,\\ 
 & wand ich bin iwers enpfâhens vrô."\\ 
 & Si kusten unt \textbf{sâzen} nider.\\ 
10 & eine juncvrouwen si sande wider\\ 
 & unt hiez \textbf{ir} bringen rîchiu kleit.\\ 
 & diu wâren gesniten al \textbf{gereit}\\ 
 & ûz pfelle von Ninnive.\\ 
 & si solde der künec Clamide,\\ 
15 & ir gevangen, hân getragen.\\ 
 & diu magt si brâhte unt begunde klagen,\\ 
 & der mantel wære âne snuor.\\ 
 & Cunneware \textbf{sus} gevuor:\\ 
 & \textbf{von} blanker \textbf{sîte} ein snüerelîn\\ 
20 & \textbf{si zucte} unt zôch \textbf{ez im} dar în.\\ 
 & Mit urloube er \textbf{sich} dô twuoc\\ 
 & den râm von im. der junge truoc\\ 
 & bî rôtem munde liehtez vel.\\ 
 & gekleit wart der degen snel.\\ 
25 & dô was er fier und clâr.\\ 
 & swer in sach, der jach vür wâr,\\ 
 & \textbf{er wære} geblüemet \textbf{vür} alle man.\\ 
 & \textbf{diz} lop sîn varwe muose hân.\\ 
 & \begin{large}P\end{large}arzivale stuont wol sîn wât.\\ 
30 & einen \textbf{grüenen} smarât\\ 
\end{tabular}
\scriptsize
\line(1,0){75} \newline
D \newline
\line(1,0){75} \newline
\textbf{5} \textit{Majuskel} D  \textbf{9} \textit{Majuskel} D  \textbf{21} \textit{Majuskel} D  \textbf{29} \textit{Initiale} D  \newline
\line(1,0){75} \newline
\textbf{7} getörst] getorst D \textbf{13} Ninnive] Ninnivê D \textbf{14} Clamide] Chlamidê D \newline
\end{minipage}
\hspace{0.5cm}
\begin{minipage}[t]{0.5\linewidth}
\small
\begin{center}*m
\end{center}
\begin{tabular}{rl}
 & \textbf{unz} iuch mîn herze erkante,\\ 
 & \textbf{dô} mich an vröuden pfante\\ 
 & Keie, der mich \textbf{dô} sô sluoc.\\ 
 & daz habt \textbf{ir gerochen} genuoc.\\ 
5 & ich kuste iuch, wær ich kusses wert."\\ 
 & "des hete ich hiute sân gegert",\\ 
 & sprach Parcifal, "getörst ich sô,\\ 
 & wand ich bin iuwers enpfâhens vrô."\\ 
 & si kuste in und \textbf{sazte in} nider.\\ 
10 & ein juncvrouwen si sante wider\\ 
 & und hiez \textbf{ir} bringen rîchiu kleit.\\ 
 & diu wâren gesniten al \textbf{breit}\\ 
 & ûz pfelle von N\textit{ini}ve.\\ 
 & si solte der künic Clamide,\\ 
15 & ir gevangen, hân getragen.\\ 
 & diu magt si brâhte und begunde klagen,\\ 
 & der mantel, \textbf{der} wær âne snuor.\\ 
 & Cu\textit{nne}w\textit{a}re \textbf{sus} gevuor:\\ 
 & \textbf{von} blanker \textbf{sîte} ein snüerlîn\\ 
20 & \textbf{si zuckete} und zôch \textbf{im daz} dar în.\\ 
 & mit urloube er dô twuoc\\ 
 & den râm von ime. der junge truoc\\ 
 & bî rôtem munde liehtez vel.\\ 
 & gekleit wart der degen snel.\\ 
25 & dô was er fier und clâr.\\ 
 & wer in sach, der jach vür wâr,\\ 
 & \textbf{er wære} gebl\textit{üe}met \textbf{über} alle man.\\ 
 & \textbf{daz} lop sîn varwe muose hân.\\ 
 & Parcifal stuont wol sîn wât.\\ 
30 & einen \textbf{grüenen} smarât\\ 
\end{tabular}
\scriptsize
\line(1,0){75} \newline
m n o \newline
\line(1,0){75} \newline
\newline
\line(1,0){75} \newline
\textbf{1} herze] herzen m \textbf{3} Keie] Keẏe n Keẏ o \textbf{6} des] Das o  $\cdot$ gegert] begert n o \textbf{7} getörst] gedurste n (o) \textbf{10} juncvrouwen] jungfrouwe n (o) \textbf{12} wâren] werent o  $\cdot$ al] alzuͯ n also o \textbf{13} Ninive] [Nuwe]: Nuͯve m nuwede n nuwe o \textbf{14} künic] kv́nigin n \textbf{15} hân] han ich o \textbf{18} Cunneware] Cumuwere m Caneware n Conne ware o  $\cdot$ gevuor] fuͦr n \textbf{19} sîte] sẏden n (o)  $\cdot$ snüerlîn] sẏmelin o \textbf{20} zuckete] zackte o  $\cdot$ im] \textit{om.} o \textbf{21} twuoc] truͦg o \textbf{23} vel] folke o \textbf{24} gekleit] Becleit n o \textbf{27} geblüemet] gebleͯmet m  $\cdot$ über] fúr n (o) \textbf{28} muose] musse m muͯste n \newline
\end{minipage}
\end{table}
\newpage
\begin{table}[ht]
\begin{minipage}[t]{0.5\linewidth}
\small
\begin{center}*G
\end{center}
\begin{tabular}{rl}
 & \textbf{ê} iuch mîn herze erkande,\\ 
 & \textbf{sît} mich an vröuden pfande\\ 
 & Kay, der mich \textbf{dô} sô sluoc.\\ 
 & daz habet \textbf{gerochen ir} genuoc.\\ 
5 & ich kust iuch, wære ich kusses wert."\\ 
 & "des hete ich hiute sân gegert",\\ 
 & sprach Parzival, "get\textit{ö}rste ich sô,\\ 
 & wan ich bin iuwers \textit{enpfâhen}s vrô."\\ 
 & si kustin unde \textbf{sazte in} nider.\\ 
10 & eine juncvrouwen si sande wider\\ 
 & unde hiez \textbf{ir} bringen rîchiu kleit.\\ 
 & diu wâren gesniten al\textbf{bereit}\\ 
 & ûz pfelle von Ninve.\\ 
 & si solte der künic Clamide,\\ 
15 & ir gevangen, hân getragen.\\ 
 & diu maget si brâhte unde begunde klagen,\\ 
 & der mandel wære âne snuor.\\ 
 & Kuneware \textbf{alsus} gevuor:\\ 
 & \textbf{ûz} blanker \textbf{sîte} ein snüerelîn\\ 
20 & \textbf{si zucte} unde zôch \textbf{im daz} dar în.\\ 
 & mit urloube er \textbf{sich} dô twuoc\\ 
 & den râm von im. der junge truoc\\ 
 & bî rôtem munde liehtez vel.\\ 
 & gekleit wart der degen snel.\\ 
25 & dô was er fier unde klâr.\\ 
 & swer in sach, der jach vür wâr\\ 
 & geblüeme\textit{t} \textbf{\textit{v}ür} alle man.\\ 
 & \textbf{diz} lop sîn varwe muose hân.\\ 
 & Parzivale stuont wol sîn wât.\\ 
30 & einen \textbf{tiuren} smarât\\ 
\end{tabular}
\scriptsize
\line(1,0){75} \newline
G I O L M Q R Z \newline
\line(1,0){75} \newline
\textbf{3} \textit{Initiale} L  \textbf{7} \textit{Überschrift:} Hie vor ein wenic vindet man geschriben wie her gawan parcifaln brahte vnder artus gesinde Vnd wie er da erlich von der massenie enpfangen wart Z   $\cdot$ \textit{Initiale} Z  \textbf{13} \textit{Initiale} O  \textbf{15} \textit{Initiale} I  \textbf{29} \textit{Initiale} L M  \newline
\line(1,0){75} \newline
\textbf{1} ê] \textit{om.} L M Q  $\cdot$ iuch] Ouch M \textbf{2} sît] Do Q R Da Z \textbf{3} Kay] kaẏ G kain I Key O Q R Z Keie M  $\cdot$ der] \textit{om.} L  $\cdot$ dô] \textit{om.} L da R Z  $\cdot$ sô] \textit{om.} Q R \textbf{4} gerochen ir] ir girochen M (R) \textbf{5} ich kusses] ir kussens M ir kusses Z \textbf{6} hete] hat M  $\cdot$ ich] \textit{om.} O Z  $\cdot$ sân] lang R \textbf{7} Partzifal sprach getrost ich so Q  $\cdot$ Parzival] parzifal I L M Barcifal O barczifal R parcifal Z  $\cdot$ getörste] getorste G (I) (O) L (M) (Z) getort R \textbf{8} enpfâhens] chusses G enphahes L \textbf{9} sazte in] saszent R sazzen Z \textbf{10} eine] Jr Q Eie Z  $\cdot$ si sande] sante sie M (Q) \textbf{11} ir] im O (L) (M) Q (R) Z \textbf{12} diu] Da Z  $\cdot$ albereit] algereit I (L) (M) (Q) al gemæit O als gereit Z \textbf{13} ûz] ÷z O  $\cdot$ Ninve] niniue I Q ninive O (M) Nynive L Nẏniue R ninnive Z \textbf{14} si] Jr L  $\cdot$ Clamide] Glamide O \textbf{16} si] daz L \textit{om.} M sie sie Z  $\cdot$ brâhte unde] \textit{om.} L M brachtte R \textbf{18} Kuneware] Gunuwar I Kvnwar O Cvneware L Z Kinowar M Conware Q Cuͦnwarte R \textbf{19} ûz blanker] vzer blanchen I Von blancker Q Von wiser R  $\cdot$ sîte] siden I (L) (M) (R) (Z) \textbf{20} zucte] lvht O slangk M  $\cdot$ im] \textit{om.} O  $\cdot$ daz] ez I si L \textbf{21} sich] \textit{om.} Q R  $\cdot$ dô] da M Z \textbf{22} von im der junge] der iunge von im I \textbf{23} liehtez] lieht ez I  $\cdot$ vel] vil M \textbf{24} wart] war R  $\cdot$ degen] helt L  $\cdot$ snel] wert M \textbf{25} dô] Da Z  $\cdot$ fier] vyn M finer R \textbf{26} swer] Wer L M Q R  $\cdot$ jach] dich M \textbf{27} geblüemet vür] gebloͮmet in vur G er wer gebluͤmt vur I (L) (M) (Q) (R) (Z) \textbf{28} diz] daz I (O) (R)  $\cdot$ varwe] lip I fraw Q (Z) \textbf{29} Parzivale] Parzifal I M Barcifal O PArcifal L (Z) Partzifal Q Parczifal R  $\cdot$ wât] gewant Q \textbf{30} tiuren] grunen Q (R)  $\cdot$ smarât] smarant Q \newline
\end{minipage}
\hspace{0.5cm}
\begin{minipage}[t]{0.5\linewidth}
\small
\begin{center}*T
\end{center}
\begin{tabular}{rl}
 & \textbf{ê} iuch mîn herze erkande,\\ 
 & \textbf{sît} mich an vröuden pfande\\ 
 & Key, der mich sô sluoc.\\ 
 & daz habt \textbf{ir gerochen} genuoc.\\ 
5 & ich kust iuch, wær \textit{ich} kusses wert."\\ 
 & "Des hetich hiute sân gegert",\\ 
 & sprach Parcifal, "get\textit{ö}rst ich sô,\\ 
 & wandich bin iuwer\textit{s} enpfâhens vrô."\\ 
 & Si kustin unde \textbf{satin} nider.\\ 
10 & eine juncvrouwen si sante wider\\ 
 & unde hiez \textbf{im} bringen rîchiu kleit.\\ 
 & di\textit{u} wâren gesniten al\textbf{bereit}\\ 
 & ûz pfelle von Ninive.\\ 
 & si solte der künec Clamide,\\ 
15 & ir gevangene, hân getragen.\\ 
 & diu maget si brâhte unde begunde klagen,\\ 
 & der mantel, \textbf{der} wære âne s\textit{n}uor.\\ 
 & Cunnewar \textbf{alsus} gevuor:\\ 
 & \textbf{ûz} blanker \textbf{sîden} ein snüerlîn\\ 
20 & \textbf{zucte si} unde zôch \textbf{es} drîn.\\ 
 & Mit urloube er \textbf{sich} dô twuoc\\ 
 & den râm von im. der junge truoc\\ 
 & bî rôtem munde liehtez vel.\\ 
 & gekleidet wart der degen snel.\\ 
25 & dô was er fier unde clâr.\\ 
 & swer in sach, der jach vür wâr,\\ 
 & \textbf{er wære} geblüemet \textbf{vür} alle man.\\ 
 & \textbf{diz} lop sîn varwe muose hân.\\ 
 & Parcifale stuont wol sîn wât.\\ 
30 & einen \textbf{tiuren} smarât\\ 
\end{tabular}
\scriptsize
\line(1,0){75} \newline
T U V W \newline
\line(1,0){75} \newline
\textbf{6} \textit{Majuskel} T  \textbf{9} \textit{Initiale} W   $\cdot$ \textit{Majuskel} T  \textbf{21} \textit{Majuskel} T  \textbf{29} \textit{Majuskel} T  \newline
\line(1,0){75} \newline
\textbf{1} ê] [*]: Vnz V  $\cdot$ iuch] îv T \textbf{2} sît] [*]: Do V \textbf{3} Key] Keẏn V  $\cdot$ der] [*]: der T  $\cdot$ sô] do W \textbf{5} kust] kost U  $\cdot$ iuch] îv T  $\cdot$ ich kusses] kvsses T (W) ich kuͦssens U \textbf{7} Parcifal] parzifal T V partzifal W  $\cdot$ getörst] getorst T U got troͤste W  $\cdot$ ich] [*]: ich V eúch W \textbf{8} Als ich enpfahens von eúch bin W  $\cdot$ iuwers] îuwer T \textbf{9} kustin] kost in U  $\cdot$ satin] [*]: sazen V \textbf{10} juncvrouwen] iuͦncvreuͦwe U \textbf{11} im] [im]: ir V in W  $\cdot$ rîchiu] rich U \textbf{12} diu] die T \textbf{13} pfelle] pfellen W  $\cdot$ Ninive] niniue V W \textbf{14} der] den U  $\cdot$ Clamide] klamide W \textbf{15} gevangene] geuangner W \textbf{16} begunde] [begonde]: begoude V \textbf{17} der] \textit{om.} W  $\cdot$ snuor] [*]: suͦr T \textbf{18} Cunnewar] Kuͦnnewar U Kvnneware V Kunnewar W \textbf{19} sîden] [side]: siden U [*]: sẏde V  $\cdot$ snüerlîn] [*]: snvͤrlin V \textbf{20} zucte] Worhte V  $\cdot$ es] ez [*]: im V ims W \textbf{22} truoc] kluͦg W \textbf{26} swer] Wer U W \textbf{27} geblüemet] gelobt W \textbf{28} muose] mvese T muͤste V \textbf{29} Parcifale] Parzifale T Parzifal V Partzifali W \textbf{30} einen] Eine U Einem W  $\cdot$ tiuren] [*]: gruͤne V  $\cdot$ smarât] schmarat W \newline
\end{minipage}
\end{table}
\end{document}
