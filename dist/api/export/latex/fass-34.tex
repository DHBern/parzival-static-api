\documentclass[8pt,a4paper,notitlepage]{article}
\usepackage{fullpage}
\usepackage{ulem}
\usepackage{xltxtra}
\usepackage{datetime}
\renewcommand{\dateseparator}{.}
\dmyyyydate
\usepackage{fancyhdr}
\usepackage{ifthen}
\pagestyle{fancy}
\fancyhf{}
\renewcommand{\headrulewidth}{0pt}
\fancyfoot[L]{\ifthenelse{\value{page}=1}{\today, \currenttime{} Uhr}{}}
\begin{document}
\begin{table}[ht]
\begin{minipage}[t]{0.5\linewidth}
\small
\begin{center}*D
\end{center}
\begin{tabular}{rl}
\textbf{34} & sine wolt \textbf{ouch} des niht lâzen,\\ 
 & dâ sîniu kinder sâzen,\\ 
 & \textbf{diu} \textbf{bat} si ezzen vaste.\\ 
 & ditze bôt si zêren ir gaste.\\ 
5 & gar disiu junchêrrelîn\\ 
 & wâren holt der künegîn.\\ 
 & \textbf{dar nâch diu vrouwe} ni\textit{ht} vergaz,\\ 
 & si \textbf{gie} \textbf{ouch}, dâ der wirt saz\\ 
 & unt \textbf{des} \textbf{wîp}, diu burcgrævîn.\\ 
10 & \textbf{den} becher huop diu künegîn.\\ 
 & si sprach: "lâ dir bevolhen sîn\\ 
 & unseren gast. diu êre ist dîn.\\ 
 & dâr umbe ich iuch beide man."\\ 
 & si nam urloup. \textbf{dô} \textbf{gie} \textbf{si} dan\\ 
15 & aber \textbf{hin} wider vür \textbf{ir} gast.\\ 
 & des herze truoc ir minnen last.\\ 
 & daz selbe ouch \textbf{ir von im} geschach,\\ 
 & \textbf{des} ir \textbf{herze unt ir ougen} jach.\\ 
 & die muosens mit ir pflihte hân.\\ 
20 & mit zühten sprach diu vrouwe sân:\\ 
 & "Gebietet, hêrre, swes ir gert,\\ 
 & daz schaf ich, wand ir sît es wert,\\ 
 & \textbf{unt} lât mich iwer urloup hân.\\ 
 & wirt iu guot gemach getân,\\ 
25 & des vröuwen wir uns überal."\\ 
 & guldîn wâren ir kerzestal.\\ 
 & \textbf{vier} lieht man \textbf{vor} \textbf{ir} \textbf{drûfe} truoc.\\ 
 & si reit ouch, dâ si vant genuoc.\\ 
 & \textbf{\textit{\begin{large}S\end{large}}i}\textbf{ne} \textbf{âzen} ouch niht \textbf{langer} dô.\\ 
30 & \textbf{der hêrre} \textbf{was} trûric unt vrô.\\ 
\end{tabular}
\scriptsize
\line(1,0){75} \newline
D \newline
\line(1,0){75} \newline
\textbf{21} \textit{Majuskel} D  \textbf{29} \textit{Initiale} D  \newline
\line(1,0){75} \newline
\textbf{7} niht] nith D \textbf{29} Sine] ÷ine D \newline
\end{minipage}
\hspace{0.5cm}
\begin{minipage}[t]{0.5\linewidth}
\small
\begin{center}*m
\end{center}
\begin{tabular}{rl}
 & \textit{\begin{large}S\end{large}}in\textit{e} wolte \textbf{eht} des niht lâzen,\\ 
 & dâ sîniu kinder sâzen,\\ 
 & \textbf{diu} \textbf{bat} si ezzen vaste.\\ 
 & diz bôt si zêren ir gaste.\\ 
5 & gar disiu junchêrrelîn\\ 
 & wâren holt der künigîn.\\ 
 & \textbf{dar nâch diu vrouwe} niht vergaz,\\ 
 & si \textbf{gienge} \textbf{ouch}, d\textit{â} der wirt saz\\ 
 & und \textbf{daz} \textbf{wîp}, diu burcgrævîn.\\ 
10 & \textbf{den} becher huop diu künigîn.\\ 
 & si sprach: "lâ dir bevolhen sîn\\ 
 & unsern gast. diu êre ist dîn.\\ 
 & dâr umbe ich iuch beide mane."\\ 
 & si nam urloup. \textbf{dô} \textbf{gienc} \textbf{si} dane\\ 
15 & aber \textbf{hin} wider vür \textbf{ir} gast.\\ 
 & des herz\textit{e} truo\textit{c} ir minnen last.\\ 
 & daz selbe ouch \textbf{ir von im} geschach,\\ 
 & \textbf{des} ir \textbf{herze und ir ouge} jach.\\ 
 & die muosens mit ir pflihte hân.\\ 
20 & mit zühten sprach diu vrowe sân:\\ 
 & "gebiete\textit{t}, hêrre, wes ir gert,\\ 
 & daz schaf ich, wand ir sît es wert,\\ 
 & \textbf{und} lât mich iuwern urloup hân.\\ 
 & wirt iu \textbf{hie} guot gemach getân,\\ 
25 & des vröuwen wir uns überal."\\ 
 & guldîn wâren ir kerzestal.\\ 
 & \textbf{vier} lieht man \textbf{vor} \textbf{in} \textbf{drûfe} truoc.\\ 
 & si reit ouch, d\textit{â} si \textbf{den} vant genuoc.\\ 
 & \textbf{\begin{large}M\end{large}an} \textbf{az} ouch \textbf{hie} niht \textbf{lange} dô.\\ 
30 & \textbf{Gahmuret} \textbf{was} trûric und vrô.\\ 
\end{tabular}
\scriptsize
\line(1,0){75} \newline
m n o W \newline
\line(1,0){75} \newline
\textbf{1} \textit{Initiale} m   $\cdot$ \textit{Capitulumzeichen} n  \textbf{15} \textit{Initiale} W  \textbf{29} \textit{Initiale} m   $\cdot$ \textit{Capitulumzeichen} n  \newline
\line(1,0){75} \newline
\textbf{1} Sine] Einer m Jch n o Sy W  $\cdot$ eht] ouch n (o) (W) \textbf{2} dâ] Des n Das o Do W \textbf{4} diz] Das W  $\cdot$ ir] jeren o \textbf{5} \textit{Versdoppelung 34.5-15 (²o) nach 34.15, 34.8 (²o) auch nach 34.8 verdoppelt; Lesarten der vorausgehenden Verse mit ¹o bezeichnet} o   $\cdot$ disiu] dise m n o \textbf{8} \textit{Versdoppelung} o   $\cdot$ gienge] ging n \textsuperscript{1}\hspace{-1.3mm} o W  $\cdot$ dâ] do m \textsuperscript{1}\hspace{-1.3mm} o W der do n \textbf{9} daz] des n \textsuperscript{1}\hspace{-1.3mm} o W  $\cdot$ burcgrævîn] burgrafen \textsuperscript{2}\hspace{-1.3mm} o \textbf{10} becher] becher \textit{nachträglich korrigiert zu:} bet er m \textbf{14} dô gienc si] vnd ging n \textsuperscript{1}\hspace{-1.3mm} o vnd gieng von W \textbf{15} wider] \textit{om.} W \textbf{16} herze] herczen m (n) (o)  $\cdot$ truoc] trut m (o) drat n  $\cdot$ minnen] minne W \textbf{17} geschach] beschach o W \textbf{18} des] Das o W  $\cdot$ jach] sach n veriach W \textbf{19} die] Sú n (o) (W)  $\cdot$ muosens] mússent W  $\cdot$ ir] ie o  $\cdot$ pflihte] pfliegete o \textbf{20} vrowe] frowen m \textbf{21} gebietet] Gebietten m  $\cdot$ wes] was o \textbf{22} schaf ich] schafft W  $\cdot$ ir] ie o  $\cdot$ es] \textit{om.} n o sein W  $\cdot$ wert] gewert n o W \textbf{23} iuwern] úwer W \textbf{25} uns] wol o \textbf{27} in] ir o W  $\cdot$ truoc] [war]: truͦg o \textbf{28} ouch dâ si den] ouch do sÿ den m ouch do sú n (o) do man auch W \textbf{29} az] asse n \textit{om.} o  $\cdot$ lange] langer n (o) (W) \textbf{30} Gahmuret] Gamiret n Gamuret o W \newline
\end{minipage}
\end{table}
\newpage
\begin{table}[ht]
\begin{minipage}[t]{0.5\linewidth}
\small
\begin{center}*G
\end{center}
\begin{tabular}{rl}
 & sine wolt \textbf{ouch} des niht lâzen,\\ 
 & dâ sîniu kinder sâzen,\\ 
 & \textbf{si}\textbf{ne} \textbf{bæ\textit{t}e} si ezzen vaste.\\ 
 & diz bôt si zêren ir gaste.\\ 
5 & gar disiu junchêrrelîn\\ 
 & wâren holt der künegîn.\\ 
 & \textbf{dar nâch diu vrouwe} niht vergaz,\\ 
 & si \textbf{gieng} \textbf{och}, dâ der wirt saz\\ 
 & unde \textbf{sîn} \textbf{wîp}, diu burcgrævîn.\\ 
10 & \textbf{ir} becher huob diu künigîn.\\ 
 & si sprach: "lâ dir bevolhen sîn\\ 
 & unseren gast. diu êre ist dîn.\\ 
 & dâr umbe ich iuch beidiu man."\\ 
 & si nam urloup \textbf{und} \textbf{vuor} \textbf{von} dan\\ 
15 & aber wider vür \textbf{ir} gast.\\ 
 & des herze truoc ir minnen last.\\ 
 & daz selbe ouch \textbf{ir von im} geschach,\\ 
 & \textbf{als} ir \textbf{herze und ir ouge} jach.\\ 
 & diu muosens mit ir pflihte hân.\\ 
20 & mit zühten sprach diu vrouwe sân:\\ 
 & "gebiet, hêrre, swes ir gert,\\ 
 & daz schaffe ich, wan ir sît es wert.\\ 
 & lât mich iwer urloup hân.\\ 
 & wirt iu \textbf{hie} guot gemach getân,\\ 
25 & des vröun wir uns überal."\\ 
 & guldîn wâren ir kerzestal.\\ 
 & \textbf{vil} lieht man \textbf{vor} \textbf{ir} \textbf{ûf} truoc.\\ 
 & si reit ouch, dâ si vant genuoc.\\ 
 & \textbf{si} \textbf{âzen} ouch niht \textbf{lenger} dô.\\ 
30 & \textbf{der helt} \textbf{wart} trûrec und vrô.\\ 
\end{tabular}
\scriptsize
\line(1,0){75} \newline
G O L M Q R Z Fr21 Fr32 \newline
\line(1,0){75} \newline
\textbf{1} \textit{Initiale} O L M Q R Z Fr32  \textbf{11} \textit{Versal} Fr32  \textbf{29} \textit{Initiale} L Q R Z Fr21  \newline
\line(1,0){75} \newline
\textbf{1} sine] ÷i O SJ L (M)  $\cdot$ des] \textit{om.} R \textbf{2} dâ] Do Q \textbf{3} sine] Si O (L) (Q) (R) (Z) (Fr32)  $\cdot$ bæte] babte G bat Q R Fr32 hiez Z  $\cdot$ si] siv O \textbf{4} diz] Daz O (Q) Des R  $\cdot$ bôt] tet L  $\cdot$ ir] dem O \textbf{5} gar] \textit{om.} Q  $\cdot$ disiu] die L  $\cdot$ junchêrrelîn] Jungforlichin M \textbf{8} gieng] gienge O gingen M  $\cdot$ dâ] do O Q  $\cdot$ saz] [wasz]: saz M \textbf{9} sîn] des O L (M) Q R Z Fr21 Fr32 \textbf{10} \textit{Versfolge 34.11-10} Q   $\cdot$ ir] Den Z  $\cdot$ huob] [huͦbt]: huͦb O \textbf{11} bevolhen] enpholhen O (R) (Z) \textbf{12} unseren] Vnser Q \textbf{13} iuch] ivch gerne O \textbf{14} und vuor von] da fvͦr si O do fvr sie L (Q) (Fr32) die frow fuͦr R da gie sie Z vnde fuͦr Fr21 \textbf{15} wider] hin wider Z \textbf{16} des] Das L  $\cdot$ herze] herczen R  $\cdot$ ir] \textit{om.} M  $\cdot$ minnen] minne O (L) (Q) R Z \textbf{17} selbe] selben R  $\cdot$ ir von im] im von ir Q \textbf{18} als] Als ob Q Des Z  $\cdot$ herze] avge O (L) (Q) (Z) (Fr21) (Fr32) ougen M (R)  $\cdot$ ouge] herze O (L) (M) (Q) (R) (Z) Fr21 (Fr32) \textbf{19} muosens] mvͦsen O (L) (M) (R) (Fr21) mústes Q muste Z  $\cdot$ mit] \textit{om.} Z  $\cdot$ pflihte] pflichten Q \textbf{21} swes] wes L (Q) Z was R \textbf{22} sît] sin Z  $\cdot$ es] des L R \textit{om.} Z  $\cdot$ wert] gewert O Q Z \textbf{23} lât] Nv lat O Vnd lat L (M) (Q) (R) Z (Fr21) (Fr32)  $\cdot$ mich iwer] ovch mich Z \textbf{24} hie] \textit{om.} O  $\cdot$ guot] \textit{om.} L \textbf{25} uns] vns hy Q \textbf{26} kerzestal] cherzen stal O (Z) \textbf{27} vil] Vier O L M R Z (Fr21) (Fr32)  $\cdot$ vor ir ûf] vor ir dar vͦffe O (L) (Z) (Fr21) (Fr32) vor on M vor ir Q dar uff R \textbf{28} dâ] do Q \textbf{29} si] Sie on M (R) Dine Fr21  $\cdot$ ouch] \textit{om.} R  $\cdot$ dô] da O M \textbf{30} helt] herre Z  $\cdot$ wart] was O (L) (M) (Q) (R) Z Fr21 \newline
\end{minipage}
\hspace{0.5cm}
\begin{minipage}[t]{0.5\linewidth}
\small
\begin{center}*T
\end{center}
\begin{tabular}{rl}
 & Sine wolte des niht lâzen,\\ 
 & dâ sîn\textit{iu} kinder sâzen,\\ 
 & \textbf{si}\textbf{ne} \textbf{bæte} si ezzen vaste.\\ 
 & diz bôt si zêren ir gaste.\\ 
5 & Gar dis\textit{iu} junchêrrelîn\\ 
 & wâren holt der künegîn.\\ 
 & \textbf{Diu vrouwe ouch des} niht vergaz,\\ 
 & si \textbf{en}\textbf{gienge}, dâ der wirt saz\\ 
 & und \textbf{des} \textbf{wirtîn}, diu burcgrævîn.\\ 
10 & \textbf{ir} becher huop diu künegîn.\\ 
 & si sprach: "lâ dir bevolhen sîn\\ 
 & unsern gast. diu êre ist dîn.\\ 
 & dâr umb ich iuch beidiu man."\\ 
 & Si nam urloup. \textbf{dô} \textbf{vuor} \textbf{si} dan\\ 
15 & aber wider vür \textbf{den} gast.\\ 
 & des herze truoc ir minnen last.\\ 
 & daz selbe ouch \textbf{im von ir} geschach,\\ 
 & \textbf{als} ir \textbf{ouge dem herzen} jach.\\ 
 & die muosens mit ir pflihte hân.\\ 
20 & mit zühten sprach diu vrouwe sân:\\ 
 & "gebiet, hêrre, swes ir gert,\\ 
 & daz schaff ich, wan ir sît e\textit{s} wert,\\ 
 & \textbf{und} lât mich iuwern urloup hân.\\ 
 & wirt iu \textbf{hie} guot gemach getân,\\ 
25 & des vröuwen wir uns überal."\\ 
 & guldîn wâren ir kerzestal.\\ 
 & \textbf{vier} lieht man \textbf{drûffe} truoc.\\ 
 & si reit ouch, dâ si vant genuoc.\\ 
 & \textbf{\begin{large}S\end{large}i}\textbf{ne} \textbf{âzen} ouch niht \textbf{lenger} dô.\\ 
30 & \textbf{der helt} \textbf{was} trûric und vrô.\\ 
\end{tabular}
\scriptsize
\line(1,0){75} \newline
T U V \newline
\line(1,0){75} \newline
\textbf{1} \textit{Majuskel} T  \textbf{5} \textit{Majuskel} T  \textbf{7} \textit{Majuskel} T  \textbf{14} \textit{Majuskel} T  \textbf{29} \textit{Initiale} T V  \newline
\line(1,0){75} \newline
\textbf{1} des] auch des U (V) \textbf{2} sîniu] sine T \textbf{4} bôt] tet V \textbf{5} disiu] dise T \textbf{8} engienge] gienge V  $\cdot$ dâ] do U V \textbf{9} des] die V \textbf{11} lâ] Ja U \textbf{13} iuch] îv T \textbf{14} dô vuor si] do [buͦr]: vuͦr si T vnd vuͦr U (V) \textbf{16} ir] der U \textbf{17} ouch im] im oͮch V \textbf{18} ouge] oͮgen V \textbf{19} muosens] mvesens T (V) \textbf{21} swes] wes U \textbf{22} es] ez T  $\cdot$ wert] gewert U V \textbf{27} drûffe] dar U [*]: vor ir dar vffe V \textbf{28} dâ] do U \textbf{29} Sine] Sie U \newline
\end{minipage}
\end{table}
\end{document}
