\documentclass[8pt,a4paper,notitlepage]{article}
\usepackage{fullpage}
\usepackage{ulem}
\usepackage{xltxtra}
\usepackage{datetime}
\renewcommand{\dateseparator}{.}
\dmyyyydate
\usepackage{fancyhdr}
\usepackage{ifthen}
\pagestyle{fancy}
\fancyhf{}
\renewcommand{\headrulewidth}{0pt}
\fancyfoot[L]{\ifthenelse{\value{page}=1}{\today, \currenttime{} Uhr}{}}
\begin{document}
\begin{table}[ht]
\begin{minipage}[t]{0.5\linewidth}
\small
\begin{center}*D
\end{center}
\begin{tabular}{rl}
\textbf{18} & zen venstern unde sâhen dar;\\ 
 & si \textbf{næmen} \textbf{des} \textbf{vil} rehte war,\\ 
 & sîne knappen unde sîn harnas,\\ 
 & wie daz gefe\textit{i}tieret wa\textit{s}.\\ 
5 & \textbf{dô truoc} der helt milte\\ 
 & ûf einem \textbf{hermînem} schilte,\\ 
 & \textbf{ine weiz, wie} manegen zobelbalc.\\ 
 & der küneginne marschalc\\ 
 & hetez vür einen anker grôz.\\ 
10 & ze sehen in \textbf{wênic} \textbf{dâr} verd\textit{rô}z.\\ 
 & \textbf{dô} muosen sîniu ouge jehen,\\ 
 & daz er hete gesehen\\ 
 & disen ritter oder sînen schîn.\\ 
 & daz muose ze Alexandrie sîn,\\ 
15 & dô der bâruc \textbf{dâr} vor lac.\\ 
 & sînen prîs dâ niemen widerwac.\\ 
 & \begin{large}S\end{large}us vuor der muotes rîche\\ 
 & in die stat behagenlîche.\\ 
 & zehen soumære hiez er vazzen.\\ 
20 & die zogeten \textbf{hin} die gazzen.\\ 
 & \textbf{dâ} riten zwênzic knappen nâch.\\ 
 & sîn bovel man dort \textbf{vor ersach}.\\ 
 & \textbf{garzûne}, koche unde \textbf{ir} knaben\\ 
 & heten sich \textbf{hin} vür erhaben.\\ 
25 & stolz was sîn gesinde.\\ 
 & zwelf wol geborner kinde\\ 
 & dô \textbf{hinden} nâch den knappen riten,\\ 
 & \textbf{an} \textbf{guoter} zuht, \textbf{mit} süezen siten.\\ 
 & etslîcher wa\textit{s} ein Sarrazin.\\ 
30 & dâr nâch muose \textbf{ouch getrecket} sîn\\ 
\end{tabular}
\scriptsize
\line(1,0){75} \newline
D Fr9 \newline
\line(1,0){75} \newline
\textbf{17} \textit{Überschrift:} Hie tuͦt diz mere v̂ kvnt aventivre von patelamvnt Fr9   $\cdot$ \textit{Großinitiale} Fr9   $\cdot$ \textit{Initiale} D  \newline
\line(1,0){75} \newline
\textbf{1} sâhen] sagen Fr9 \textbf{2} si] [*i]: si D  $\cdot$ næmen] namen Fr9  $\cdot$ rehte] guͦte Fr9 \textbf{3} sîne] Vmme de Fr9 \textbf{4} gefeitieret] gefettirt D gefaẏteret Fr9  $\cdot$ was] waz D \textbf{5} dô] Doch Fr9 \textbf{6} hermînem] hermẏnen Fr9 \textbf{10} verdrôz] [vrdorz]: verdorz D \textbf{14} Alexandrie] alexandrẏe Fr9 \textbf{15} dâr vor] [dar]: da vuͦre Fr9 \textbf{17} der] des Fr9 \textbf{22} vor ersach] vuͦr [im]: ir sach Fr9 \textbf{24} erhaben] gehaben Fr9 \textbf{29} was] waz D \textbf{30} muose] mvsen Fr9 \newline
\end{minipage}
\hspace{0.5cm}
\begin{minipage}[t]{0.5\linewidth}
\small
\begin{center}*m
\end{center}
\begin{tabular}{rl}
 & \textbf{an} den vensteren und sâhen dar;\\ 
 & si \textbf{\textit{nâm}en} \textbf{des} \textbf{vil} rehte war,\\ 
 & sîne knappen und sîn harna\textit{s},\\ 
 & wie daz gef\textit{ei}tie\textit{re}t was.\\ 
5 & \textbf{dô truoc} der helt milte\\ 
 & ûf einem \textbf{hürnînen} schilte\\ 
 & \textbf{sô} menigen z\textit{o}belbalc.\\ 
 & der küniginne marschalc\\ 
 & hette ez vür eine\textit{n} anker grôz.\\ 
10 & zuo sehen \textit{i}n \textbf{wênic} \textbf{d\textit{e}s} verdrôz.\\ 
 & \textbf{dô} muosen sîniu ougen jehen,\\ 
 & daz er \textbf{dâ vor} hette gesehen\\ 
 & disen ritter oder sînen schîn.\\ 
 & daz muos zAlexandrie sîn,\\ 
15 & dô der bâruc \textbf{dâr} vor lac.\\ 
 & sînen prîs d\textit{â} niemen widerwac.\\ 
 & sus vuor der muotes rîche\\ 
 & in die stat behagenlîche.\\ 
 & zehen s\textit{oumer} hiez er vazzen.\\ 
20 & die zog\textit{e}ten \textbf{hin} die gazzen.\\ 
 & \textbf{dâ} riten zwênzic knappen nâch.\\ 
 & sîn povel man dort \textbf{vor ersach}\\ 
 & \textbf{gar schône}. k\textit{o}che und \textbf{ir} knaben\\ 
 & hetten sich \textbf{her} vür \textit{e}rhaben.\\ 
25 & stolz was sîn gesinde.\\ 
 & zwelf wol geborner kinde\\ 
 & dô \textbf{hinden} nâch den knappen riten,\\ 
 & \textbf{\textit{v}on} \textbf{guoter} zuht, \textbf{von} süezen siten.\\ 
 & etlîcher was ein Sarrazin.\\ 
30 & dâr nâch muos \dag \textbf{gestrecket}\dag  sîn\\ 
\end{tabular}
\scriptsize
\line(1,0){75} \newline
m n o \newline
\line(1,0){75} \newline
\newline
\line(1,0){75} \newline
\textbf{1} sâhen] [sehen]: sehent o \textbf{2} nâmen] m::end \textit{nachträglich korrigiert zu:} nanend m \textbf{3} Sîn harnesch vnd sin knappen n (o)  $\cdot$ harnas] harnach \textit{nachträglich korrigiert zu:} harnasch m \textbf{4} gefeitieret] gevanteriert \textit{nachträglich korrigiert zu:} gevassiret m gefanttiert n (o)  $\cdot$ was] hetten n (o) \textbf{5} helt] hilt n (o) \textbf{6} hürnînen] hermynen n heren mẏnen o \textbf{7} menigen] maniger o  $\cdot$ zobelbalc] zebel balk m \textbf{8} küniginne] konig o \textbf{9} einen] einem m \textbf{10} in] eÿn m  $\cdot$ des] das m o \textbf{14} zAlexandrie] allexandrie n allexandrige o \textbf{15} bâruc] barig o \textbf{16} dâ niemen] do nÿmen m nyeman do n nẏmann o \textbf{19} soumer] suffiren m  $\cdot$ hiez] hiesse n \textbf{20} zogeten] zogentend m zeigeten n zoiͯgeten o \textbf{22} dort] [det]: dert o \textbf{23} koche] kuche \textit{nachträglich korrigiert zu:} knechte m kúsch n sich kuͯch o \textbf{24} vür erhaben] vorher haben m fúr gehaben n (o) \textbf{25} stolz] Solcz o \textbf{28} von guoter] Won guter m  $\cdot$ von süezen] ansuͯssen n (o) \textbf{29} Sarrazin] sarrasin n \textbf{30} muos] muͦsz ouch n auch mynns o  $\cdot$ gestrecket] gestrecket \textit{nachträglich korrigiert zu:} gedecket m \newline
\end{minipage}
\end{table}
\newpage
\begin{table}[ht]
\begin{minipage}[t]{0.5\linewidth}
\small
\begin{center}*G
\end{center}
\begin{tabular}{rl}
 & \textbf{in} den vensteren und sâhen dar;\\ 
 & si \textbf{nâmen} \textbf{ouch} \textbf{des} rehte war,\\ 
 & sîne knappen und sîn harnas,\\ 
 & wie daz gefeitieret was.\\ 
5 & \textbf{dô truoc} der helt milte\\ 
 & ûf einem \textbf{hermînem} schilte,\\ 
 & \textbf{ichne weiz, wie} manigen zobeles balc.\\ 
 & der küniginne marschalc\\ 
 & het ez vür einen anker grôz.\\ 
10 & ze sehene in \textbf{lützel} \textbf{des} verdrôz.\\ 
 & \textbf{im} muosen sîniu ougen jehen,\\ 
 & daz er het \textbf{ê} gesehen\\ 
 & disen rîter oder sînen schîn.\\ 
 & daz muose ze Alexandrie sîn,\\ 
15 & dâ der bâruc vor lac.\\ 
 & sînen brîs dâ niemen widerwac.\\ 
 & sus vuor der muotes rîche\\ 
 & in die stat behânlîche.\\ 
 & zehen soumære hiez er vazzen.\\ 
20 & die zogeten \textbf{hin} die gazzen.\\ 
 & \textbf{\begin{large}D\end{large}en} riten zweinzic knappen nâch.\\ 
 & sîne\textit{n} bovel man dort \textbf{vor ersach}.\\ 
 & \textbf{garzûne}, koche und \textbf{ir} knaben\\ 
 & heten sich \textbf{hin} vür erhaben.\\ 
25 & stolz was sîn gesinde.\\ 
 & zwelf wol geborner kinde\\ 
 & dâ \textbf{hinden} nâch den knappen riten,\\ 
 & \textbf{an} \textbf{ganzer} zuht, \textbf{mit} süezen siten.\\ 
 & etelîcher was ein Sarrazin.\\ 
30 & dâr nâch muose \textbf{er geprüevet} sîn,\\ 
\end{tabular}
\scriptsize
\line(1,0){75} \newline
G O L M Q R W Z Fr29 Fr32 Fr36 Fr71 \newline
\line(1,0){75} \newline
\textbf{1} \textit{Initiale} O M  \textbf{2} \textit{Versal} Fr32  \textbf{17} \textit{Initiale} L Q R W Z Fr32 Fr71  \textbf{21} \textit{Initiale} G  \textbf{27} \textit{Versal} Fr32  \newline
\line(1,0){75} \newline
\textbf{1} in den] ÷e den O Zu den M (R) (Z) (Fr32) (Fr36) \textbf{2} nâmen] nam Fr71  $\cdot$ ouch des] avch des vil O (M) (R) (Fr32) (Fr36) ouch daz L des auch vil Q des vil W Z \textbf{3} sîne] Siner Fr71  $\cdot$ knappen und] knechtten R  $\cdot$ sîn] sines Fr71 \textbf{4} gefeitieret] gefartiret L gefeit M gefeget Q gefurieret W gevertier Z \textbf{5} dô] Avch O (M) (Z)  $\cdot$ truoc] fvͦrte O (M) (Z)  $\cdot$ helt] [gelt]: helt M herre W \textbf{6} hermînem] herminen R Z Fr32 \textbf{7} zobeles balc] zoͮbel palch O (L) (R) (W) \textbf{9} anker] wunder R \textbf{10} sehene] sechent R  $\cdot$ in] [*]: des Fr32  $\cdot$ lützel] wennigk M nutzel \textit{nachträglich korrigiert zu:} lutzel Q  $\cdot$ des] in R dar Z Fr71  $\cdot$ verdrôz] bedrosz L \textbf{11} im] Do Q W (Fr32) Da Z  $\cdot$ muosen sîniu] muͤsen si ire W  $\cdot$ jehen] sehen Q \textbf{12} \textit{Vers 18.12 fehlt} Q   $\cdot$ ê] \textit{om.} O L M R Z Fr29 Fr32 Fr71 \textbf{14} Alexandrie] allexandria M allexandrie Q (R) Alexandria W Alexan::: Fr29 Alexandrî Fr71 \textbf{15} dâ] Do O Q R W (Fr32)  $\cdot$ bâruc] Saruͦc R garuck \textit{nachträglich korrigiert zu:} baruck Q  $\cdot$ vor] der vor O (M) (Q) R (Z) Fr32 der ::: Fr29 \textbf{16} sînen] Sinem O Sinē M Q  $\cdot$ brîs] prisen M  $\cdot$ dâ niemen] do niemen O (W) nymant da M er nẏman Q  $\cdot$ widerwac] verwach O (Fr71) \textbf{18} behânlîche] behangenliche O behendicliche M behan [geliche]: gelich Fr71 \textbf{19} soumære] eymer \textit{nachträglich korrigiert zu:} seymer Q semer R \textbf{20} zogeten] cohin M  $\cdot$ hin] in O hin \textit{nachträglich korrigiert zu:} in Q hin durch W hin in Fr32  $\cdot$ gazzen] gassze M \textbf{21} Den] Da O M R Z Fr29 (Fr32) Do L Q  $\cdot$ knappen] knechtte R \textbf{22} sînen] sinenen G Sein* \textit{nachträglich korrigiert zu:} Seine Q Sine Z  $\cdot$ bovel] bufeln M pouel \textit{nachträglich korrigiert zu:} pouer Q  $\cdot$ dort] da Fr32 Fr71  $\cdot$ ersach] [sach]: ersach L sach W \textbf{23} garzûne] Garczin M Barzunne Q  $\cdot$ koche] kochin M  $\cdot$ ir] \textit{om.} W \textbf{24} heten] Die heten O (L) (M) (Q) (R) (W) Z (Fr29) (Fr32)  $\cdot$ hin vür] fur hin Q dar auff W \textbf{25} stolz] Vil stoltz L W Solchs Q \textbf{26} geborner] geborn M geborne Q erborner R \textbf{27} dâ] Die O R W \textit{om.} L  $\cdot$ hinden nâch] Hin nach L  $\cdot$ den] die Fr32  $\cdot$ knappen] knechtten R \textbf{28} ganzer] guͦter O (L) (M) (Q) (R) W (Z) (Fr29) Fr32  $\cdot$ süezen] svszem L (Q) \textbf{30} muose] mvͦsen O (Z)  $\cdot$ er] avch O (L) (M) (Q) (R) (Z) (Fr32) (Fr71)  $\cdot$ geprüevet] gedechet O (Z) (Fr71) bereitet L gerichtet M getrecket \textit{nachträglich korrigiert zu:} gedecket Q getrecket R (Fr32) \newline
\end{minipage}
\hspace{0.5cm}
\begin{minipage}[t]{0.5\linewidth}
\small
\begin{center}*T
\end{center}
\begin{tabular}{rl}
 & \textbf{z}en vensteren und sâhen dar;\\ 
 & si \textbf{nâmen} \textbf{sîn} \textbf{vil} rehte war,\\ 
 & sîne knappen und sîn harnas,\\ 
 & wie daz gefeitieret was.\\ 
5 & \textbf{ouch vuorte} der helt milte\\ 
 & ûf einem \textbf{härmînen} schilte,\\ 
 & \textbf{ine weiz, wie} manegen zobels balc.\\ 
 & Der küneginne marschalc\\ 
 & het ez vür einen anker grôz.\\ 
10 & ze sehene in \textbf{lützel} \textbf{dâr} verdrôz.\\ 
 & \textbf{im} muosen sîniu ougen jehen,\\ 
 & daz er hete gesehen\\ 
 & disen rîter oder sînen schîn.\\ 
 & daz muose ze Alexandrie sîn,\\ 
15 & dô der bâruc \textbf{dâr} vor lac.\\ 
 & sînen prîs dâ nieman widerwac.\\ 
 & \begin{large}S\end{large}us vuor der muotes rîche\\ 
 & in die stat behagenlîche.\\ 
 & zehen soumer hiez er vazzen.\\ 
20 & die zogeten die gazzen.\\ 
 & \textbf{dâ} riten zwênzic knappen nâch.\\ 
 & sînen povel man dort \textbf{vorne sach}.\\ 
 & \textbf{garzûne}, koche und \textbf{der} knaben\\ 
 & heten sich \textbf{hin} vür erhaben.\\ 
25 & stolz was sîn gesinde.\\ 
 & zwelf wol geborner kinde\\ 
 & dâ \textbf{bî} nâch den knappen riten,\\ 
 & \textbf{an} \textbf{guoter} zuht, \textbf{mit} süezen siten.\\ 
 & etslîcher was ein Sarrazin.\\ 
30 & dâr nâch muose \textbf{ouch getrecket} sîn\\ 
\end{tabular}
\scriptsize
\line(1,0){75} \newline
T U V \newline
\line(1,0){75} \newline
\textbf{8} \textit{Majuskel} T  \textbf{17} \textit{Initiale} T U V  \newline
\line(1,0){75} \newline
\textbf{1} zen vensteren] Jn ein venster U [*]: Jn den venstern V \textbf{2} sîn] des V \textbf{3} harnas] harnaschs V \textbf{6} härmînen] [hermi*]: hermimen V \textbf{7} manegen] maniges U V \textbf{9} het] Herre U Hies V \textbf{10} verdrôz] bedroz U \textbf{11} muosen] mvesen T (V) \textbf{14} muose] mvese T (V)  $\cdot$ Alexandrie] alexander U \textbf{16} dâ] do V \textbf{18} behagenlîche] behegeliche U \textbf{19} soumer] suͦmete U \textbf{20} zogeten] zageten in U  $\cdot$ die gazzen] [*]: hin durch die gassen V \textbf{21} dâ] [D*]: Do V  $\cdot$ zwênzic] zwene U V \textbf{22} dort] do U  $\cdot$ vorne] vornan V \textbf{23} der] ir U V \textbf{27} bî nâch] \sout{b} nach U \textbf{30} muose] mvese T muͦzen U muͤsten V  $\cdot$ getrecket sîn] trechen in U bereit sin V \newline
\end{minipage}
\end{table}
\end{document}
