\documentclass[8pt,a4paper,notitlepage]{article}
\usepackage{fullpage}
\usepackage{ulem}
\usepackage{xltxtra}
\usepackage{datetime}
\renewcommand{\dateseparator}{.}
\dmyyyydate
\usepackage{fancyhdr}
\usepackage{ifthen}
\pagestyle{fancy}
\fancyhf{}
\renewcommand{\headrulewidth}{0pt}
\fancyfoot[L]{\ifthenelse{\value{page}=1}{\today, \currenttime{} Uhr}{}}
\begin{document}
\begin{table}[ht]
\begin{minipage}[t]{0.5\linewidth}
\small
\begin{center}*D
\end{center}
\begin{tabular}{rl}
\textbf{545} & \begin{large}M\end{large}ir diz ors erworben,\\ 
 & mit prîse al unverdorben,\\ 
 & wand iwer hant in nider stach,\\ 
 & dem al diu werlt ie prîses jach\\ 
5 & mit wârheit unz an disen tac.\\ 
 & iwer prîs - sînhalp der gotes slac -\\ 
 & im vreude hât enpfüeret;\\ 
 & grôz sælde iuch hât gerüeret."\\ 
 & Gawan sprach: "er stach mich nider;\\ 
10 & des erholte ich mich sider.\\ 
 & Sît man iu tjost \textbf{verzinsen} sol,\\ 
 & er mac iu zins \textbf{geleisten} wol.\\ 
 & hêrre, dort stêt ein runzît:\\ 
 & daz erwarp an mir sîn strît.\\ 
15 & daz nemt, ob ir gebiet.\\ 
 & der sich \textbf{dises} \textbf{orses} niet,\\ 
 & daz bin ich; ez muoz mich hinnen tragen,\\ 
 & solt \textbf{halt ir} nimer ors bejagen.\\ 
 & Ir nennet reht; welt ir daz nemen,\\ 
20 & \textbf{sô}ne darf iuch \textbf{nimer} des gezemen,\\ 
 & daz ich ze \textbf{vuoz} hinnen gê.\\ 
 & wan daz tæte mir ze wê,\\ 
 & \textbf{Solt} diz ors iwer sîn;\\ 
 & daz \textbf{was} sô ledeclîche mîn\\ 
25 & dennoch hiute \textbf{morgen} vruo.\\ 
 & welt ir gemaches grîfen zuo,\\ 
 & sô \textbf{rîtet ir} \textbf{sanfter} \textbf{einen} stap.\\ 
 & diz ors mir ledeclîchen gap\\ 
 & Orilus der Burgunjoys.\\ 
30 & \textbf{Urjans}, der \textbf{vürste ûz} Punturtoys,\\ 
\end{tabular}
\scriptsize
\line(1,0){75} \newline
D Fr7 \newline
\line(1,0){75} \newline
\textbf{1} \textit{Initiale} D  \textbf{11} \textit{Majuskel} D  \textbf{19} \textit{Majuskel} D  \textbf{23} \textit{Majuskel} D  \newline
\line(1,0){75} \newline
\textbf{4} al] alle Fr7 \textbf{5} unz] wen Fr7 \textbf{10} erholte] er holt Fr7 \textbf{12} zins] eins Fr7 \textbf{18} halt] ioch Fr7 \textbf{20} sône] sonen Fr7  $\cdot$ iuch] vͦch Fr7 \textbf{21} vuoz] fuͦzzen Fr7 \textbf{24} daz sol lediclichen sin min Fr7 \textbf{25} Dannoch waz es hivten morgen fruͦ Fr7 \textbf{29} Orilus] Orilius Fr7  $\cdot$ Burgunjoys] bvrgvnscoys D [burgundois]: burgundiois Fr7 \textbf{30} Urjans] Vrians D vrian Fr7  $\cdot$ Punturtoys] punturtois Fr7 \newline
\end{minipage}
\hspace{0.5cm}
\begin{minipage}[t]{0.5\linewidth}
\small
\begin{center}*m
\end{center}
\begin{tabular}{rl}
 & mir diz ros erworben,\\ 
 & mit prîse al \dag umb\dag  verdorben,\\ 
 & wan iuwer hant in nider stach,\\ 
 & dem alliu diu we\textit{r}l\textit{t} ie prîses jach\\ 
5 & mit wârheit unz an disen tac.\\ 
 & iuwer prîs - sînhalp der gotes slac -\\ 
 & im vröude het enpfüeret;\\ 
 & grôz sælde iuch het gerüeret."\\ 
 & Gawan sprach: "er stach mich nider;\\ 
10 & des erholt \textit{ich} mich sider.\\ 
 & sît man iu juste \textbf{zinsen} sol,\\ 
 & er mac iu zins \textbf{gehelfen} wol.\\ 
 & hêrre, dort stât ein runzît:\\ 
 & daz erwarp an mir sîn strît.\\ 
15 & daz nemt, ob ir gebietet.\\ 
 & der sich \textbf{des} \textbf{rosses} nietet,\\ 
 & daz bin ich; ez muoz mich hinnen tragen,\\ 
 & solt \textbf{ir halt} nimmer ros bejagen.\\ 
 & ir ne\textit{nn}et reht; wel\textit{t} ir daz ne\textit{m}en,\\ 
20 & \textbf{dô} endarf iuch \textbf{nimmer} des gezemen,\\ 
 & daz ich \textit{zuo} \textbf{vuoz} \textbf{von} hinnen gê.\\ 
 & wa\textit{n} daz tæte mir zuo wê,\\ 
 & \textbf{solt} diz ros iuwer sîn,\\ 
 & daz sô ledeclîchen mîn\\ 
25 & \textbf{was} dannoch hiute vruo.\\ 
 & wel\textit{t} ir gemaches grîfen zuo,\\ 
 & sô \textbf{rîtet ir} \textbf{sanfter} \textbf{ûf einem} stap.\\ 
 & diz ros mi\textit{r} ledeclîchen gap\\ 
 & Orilus der Burgunschois.\\ 
30 & \textbf{Vrians}, der Ponturtois,\\ 
\end{tabular}
\scriptsize
\line(1,0){75} \newline
m n o \newline
\line(1,0){75} \newline
\newline
\line(1,0){75} \newline
\textbf{3} hant] \textit{om.} o \textbf{4} werlt] wel m \textbf{7} enpfüeret] [enpfret]: enpf:ret o \textbf{10} ich] \textit{om.} m \textbf{13} runzît] rinzit n (o) \textbf{14} sîn] den n \textbf{15} nemt] nemtp o \textbf{17} muoz] muͯs m (o) \textbf{19} nennet] nement m  $\cdot$ welt] wol m  $\cdot$ nemen] nennen m \textbf{20} nimmer des] des niemer des o  $\cdot$ gezemen] geschemen n \textbf{21} zuo] \textit{om.} m \textbf{22} wan] Was m \textbf{26} welt] Wol m \textbf{27} ûf einem] einen n (o) \textbf{28} mir] mich m \textbf{29} Orilus] Oriluͯs o  $\cdot$ Burgunschois] burgunscois m burgunscois n buͯrguͯnscois o \textbf{30} der] de n  $\cdot$ Ponturtois] pontortois n ponturteis o \newline
\end{minipage}
\end{table}
\newpage
\begin{table}[ht]
\begin{minipage}[t]{0.5\linewidth}
\small
\begin{center}*G
\end{center}
\begin{tabular}{rl}
 & \begin{large}M\end{large}ir diz ors erworben,\\ 
 & mit brîse al u\textit{n}verdorben,\\ 
 & wan iuwer hant in nider stach,\\ 
 & dem al diu werlt ie brîses jach\\ 
5 & mit wârheit unze an disen tac.\\ 
 & iuwer brîs - sînhalp der gotes slac -\\ 
 & im vröude hât enpfüeret;\\ 
 & grôz sælde iuch hât gerüeret."\\ 
 & Gawan sprach: "er stach mich nider;\\ 
10 & des erholt ich mich sider.\\ 
 & sît man iu tjost \textbf{verzinsen} sol,\\ 
 & er mac iu zins \textbf{geleisten} wol.\\ 
 & hêrre, dort stêt ein runzît:\\ 
 & daz erwarp an mir sîn strît.\\ 
15 & daz nemet, ob ir gebiet.\\ 
 & der sich \textbf{dises} \textbf{orses} \textit{n}iet,\\ 
 & daz bin ich; ez muoz mich hinnen tragen,\\ 
 & solt \textbf{halt ir} niemer ors bejagen.\\ 
 & ir \textit{nenne}t reht; welt ir daz nemen,\\ 
20 & \textbf{sô}ne darf iuch \textbf{nimmer} des gezemen,\\ 
 & daz ich ze \textbf{vüezen} hinne gê.\\ 
 & wan daz tæte mir ze wê,\\ 
 & \textbf{sol} diz ors iuwer sîn,\\ 
 & daz sô lediclîchen mîn\\ 
25 & \textbf{was} dannoch hiute \textbf{morgen} vruo.\\ 
 & welt ir gemaches grîfen zuo,\\ 
 & sô \textbf{rîtet ir} \textbf{sanfter} \textbf{einen} stap.\\ 
 & diz ors mir lediclîchen gap\\ 
 & \textbf{der herzoge} Orillus der \textbf{von} Burgonois.\\ 
30 & \textbf{Vrians}, der \textbf{vürste ûz} Ponturtois,\\ 
\end{tabular}
\scriptsize
\line(1,0){75} \newline
G I L M Z \newline
\line(1,0){75} \newline
\textbf{1} \textit{Initiale} G L Z  \textbf{5} \textit{Initiale} I  \textbf{19} \textit{Initiale} I  \newline
\line(1,0){75} \newline
\textbf{1} Mir] Musz M  $\cdot$ diz] daz I  $\cdot$ erworben] erwerbin M \textbf{2} unverdorben] [umb]: umverdorbin G vnuorterben M \textbf{4} ie] \textit{om.} I \textbf{6} gotes] Gote I \textbf{8} iuch hât] in hat I hat uch M \textbf{10} sider] sit wider I \textbf{11} iu] \textit{om.} I  $\cdot$ verzinsen] zinsen Z \textbf{12} iu] \textit{om.} I \textbf{15} ir] irz I \textbf{16} niet] geniet G \textbf{17} muoz mich] mich musz M \textbf{18} halt] aber I \textit{om.} M \textbf{19} nennet] tuͦt G \textbf{20} sône] So M \textbf{21} vüezen] fuͯz L (M) \textbf{22} tæte] tuͤt I \textbf{23} sol] solt I (Z) \textbf{24} daz] Daz was Z \textbf{25} was] \textit{om.} Z  $\cdot$ morgen] en morgen I (L) \textbf{27} sanfter] sanft I \textbf{28} lediclîchen] ledicliclichen Z \textbf{29} Orillus] orilus G Z (I) (M) Orilluͯs L  $\cdot$ der von] de L Z von M  $\cdot$ Burgonois] purgunoys I Bvrgvnioysz L bvrgoniois Z \textbf{30} Vrians] vrianz I Vriancz M  $\cdot$ ûz] vz von I  $\cdot$ Ponturtois] pvntorteis G puntortoys I Pvntuͯrtoýsz L punturtois M (Z) \newline
\end{minipage}
\hspace{0.5cm}
\begin{minipage}[t]{0.5\linewidth}
\small
\begin{center}*T
\end{center}
\begin{tabular}{rl}
 & mir diz ors erworben,\\ 
 & mit prîse al unverdorben,\\ 
 & wand iuwer hant in nider stach,\\ 
 & dem aldiu werlt ie prîses jach\\ 
5 & mit wârheit unz an disen tac.\\ 
 & iuwer prîs - sînhalp der gotes slac -\\ 
 & im vröude hât enpfüeret;\\ 
 & grôz sælde iuch hât gerüeret."\\ 
 & Gawan sprach: "er stach mich nider;\\ 
10 & des erholt ich mich sider.\\ 
 & sît man iu tjost \textbf{verzinsen} sol,\\ 
 & er mac iu zins \textbf{geleisten} wol.\\ 
 & hêrre, dort stât ein runzît:\\ 
 & daz erwarp an mir sîn strît.\\ 
15 & daz nemet, ob ir gebietet.\\ 
 & der sich \textbf{disses} nietet,\\ 
 & daz bin ich; ez muoz mich hinnen tragen,\\ 
 & solt \textbf{halt ir} niemer ors bejagen.\\ 
 & ir nennet reht; welt ir daz nemen,\\ 
20 & \textbf{sô}ne darf iuch des \textbf{niht} gezemen,\\ 
 & daz ich ze \textbf{vuoz} hinnen gê.\\ 
 & wan daz tæte mir ze wê,\\ 
 & \textbf{solte} diz ors iuwer sîn,\\ 
 & daz sô ledeclîche mîn\\ 
25 & \textbf{was} dannoch hiute vruo.\\ 
 & welt ir gemaches grîfen zuo,\\ 
 & sô \textbf{ritich} \textbf{lieber} \textbf{einen} stap.\\ 
 & diz ors mir ledeclîche gap\\ 
 & \textbf{der herzoge} Orilus der Burgenoys.\\ 
30 & \textbf{Vrianz}, der \textbf{vürste ûz} Puntertoys,\\ 
\end{tabular}
\scriptsize
\line(1,0){75} \newline
T U V W O Q R Fr40 \newline
\line(1,0){75} \newline
\textbf{1} \textit{Initiale} O Q Fr40  \newline
\line(1,0){75} \newline
\textbf{1} mir] ÷ir O  $\cdot$ erworben] erwerben Q \textbf{2} al] alle U als Q  $\cdot$ unverdorben] verdorben U \textbf{3} \textit{Die Verse 545.3-8 fehlen} O  \textbf{4} aldiu] alle R  $\cdot$ ie] \textit{om.} Fr40  $\cdot$ prîses] pris U (R) \textbf{5} unz] mit U  $\cdot$ disen] disen disen U \textbf{6} slac] lac U \textbf{7} im] Jn Q  $\cdot$ vröude] froͯden R \textbf{8} sælde] solde R  $\cdot$ iuch] iv T \textit{om.} Fr40 \textbf{9} Gawan] Gawin R \textbf{10} des] Dez V  $\cdot$ erholt] erhorte U  $\cdot$ sider] wider V \textbf{11} tjost verzinsen] strit verzollen R \textbf{12} zins] zol R  $\cdot$ wol] sol R \textbf{13} runzît] runsit W \textbf{15} ir] irs V irz O \textbf{16} disses] diz orses V (Q) des roßes W (R) der orses O \textbf{17} muoz] \textit{om.} W  $\cdot$ hinnen] hunnnen Q \textbf{18} halt] halp V \textit{om.} O  $\cdot$ ir niemer] nymmer ir W ir minner Q \textbf{19} reht] ir recht W \textbf{20} sône] So O Q R  $\cdot$ iuch] iv T  $\cdot$ des niht] nuͦmer des U (W) (O) niemer daz V minner desz Q daz niemer R \textbf{22} mir] mit W \textbf{23} solte] Sol O \textbf{24} sô] waz so V (W) (O) (Q) (R)  $\cdot$ ledeclîche] ledenlich R \textbf{25} was] \textit{om.} V W O Q R  $\cdot$ hiute] hv́tte morgen V (W) (O) (R) morgen hewt Q \textbf{27} ritich lieber] riten ich lichte U rittent ir lihter V (W) (O) (Q) Rittent Jr lichtenklicher R \textbf{28} diz] Das R \textbf{29} der herzoge] \textit{om.} O  $\cdot$ Orilus] arilus W Orylvs O  $\cdot$ der] de Q von R  $\cdot$ Burgenoys] buͦrgenoys U burgonoys V burgunnoys W bvrgoͤnoẏs O burgonioys Q Burguniois R \textbf{30} Vrianz] Vrians U W R Vriantz V Vryans O Frians Q  $\cdot$ ûz] von R  $\cdot$ Puntertoys] Puͦntertoys U punthertoys V ponturtoys W Pvntortoẏs O punturtoys Q puͯntrortois R \newline
\end{minipage}
\end{table}
\end{document}
