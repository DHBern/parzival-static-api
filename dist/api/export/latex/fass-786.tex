\documentclass[8pt,a4paper,notitlepage]{article}
\usepackage{fullpage}
\usepackage{ulem}
\usepackage{xltxtra}
\usepackage{datetime}
\renewcommand{\dateseparator}{.}
\dmyyyydate
\usepackage{fancyhdr}
\usepackage{ifthen}
\pagestyle{fancy}
\fancyhf{}
\renewcommand{\headrulewidth}{0pt}
\fancyfoot[L]{\ifthenelse{\value{page}=1}{\today, \currenttime{} Uhr}{}}
\begin{document}
\begin{table}[ht]
\begin{minipage}[t]{0.5\linewidth}
\small
\begin{center}*D
\end{center}
\begin{tabular}{rl}
\textbf{786} & \begin{large}D\end{large}ie boten \textbf{vuoren endehafte} dan.\\ 
 & Parzival sîne rede alsus huob an.\\ 
 & \textbf{en franzoys} er zin allen sprach,\\ 
 & als Trevrizent dor\textit{t} \textbf{vorn} jach,\\ 
5 & daz den Grâl ze keinen zîten\\ 
 & niemen m\textit{ö}ht erstrîten,\\ 
 & wan der \textbf{vor} gote \textbf{ist} \textbf{dâr bekant}.\\ 
 & \textbf{diz} mære kom über elliu lant:\\ 
 & dehein strît \textbf{m\textit{ö}ht in} erwerben.\\ 
10 & vil liute \textbf{liez dô} verderben\\ 
 & nâch dem Grâle gewerbes list,\\ 
 & dâ von er noch verborgen ist.\\ 
 & Parzival unt Feirefiz\\ 
 & diu wîp lêrten jâmers vlîz.\\ 
15 & si heten\textbf{z} ungern vermiten,\\ 
 & in diu vier stücke shers si riten.\\ 
 & si nâmen urloup zal der diet.\\ 
 & ieweder dan mit vreuden schiet,\\ 
 & gewâpent wol \textbf{gein} strîtes wer.\\ 
20 & Ame \textbf{dritten} tage ûz\textbf{es} heidens her\\ 
 & wart ze Joflanze brâht,\\ 
 & \textbf{sô grôzer} gâbe wart \textbf{nie gedâht}.\\ 
 & Swelch künec dâ \textbf{sîne} gâbe enpfant,\\ 
 & daz half immer mêr \textbf{des} lant.\\ 
25 & ieslîchem man nâch mâze sîn\\ 
 & wart nie \textbf{sô tiuriu gâbe} schîn,\\ 
 & \textbf{Al} den vrouwen rîchiu prêsente\\ 
 & von Triande unt von Nouriente.\\ 
 & ine weiz, wie d\textit{az} her \textbf{sich} schiede hie.\\ 
30 & Cundrie unt \textbf{dise} zwên, hin riten sie.\\ 
\end{tabular}
\scriptsize
\line(1,0){75} \newline
D \newline
\line(1,0){75} \newline
\textbf{1} \textit{Initiale} D  \textbf{20} \textit{Majuskel} D  \textbf{23} \textit{Majuskel} D  \textbf{27} \textit{Majuskel} D  \newline
\line(1,0){75} \newline
\textbf{2} Parzival] Parcifal D \textbf{4} dort] dor D \textbf{6} möht] moht D \textbf{7} wan der] [wand*]: wander D \textbf{9} möht] moht D \textbf{13} Parzival] Parcifal D \textbf{28} Nouriente] Noͮriente D \textbf{29} daz] des D \textbf{30} Cundrie] Cvndrîe D \newline
\end{minipage}
\hspace{0.5cm}
\begin{minipage}[t]{0.5\linewidth}
\small
\begin{center}*m
\end{center}
\begin{tabular}{rl}
 & die boten \textbf{vuoren endehaft} dan.\\ 
 & Parcifal sîn rede alsus huop an.\\ 
 & \textbf{in franzois} er zuo in allen sprach,\\ 
 & als Trevrizent dort \textbf{vornân} jach,\\ 
5 & daz den Grâl zuo keinen zîten\\ 
 & nieman m\textit{ö}ht erstrîten,\\ 
 & wan der \textbf{von} got \textbf{wær} \textbf{dar benant}.\\ 
 & \textbf{daz} mær kam über alliu lant,\\ 
 & \textbf{daz} kein strît \textbf{in m\textit{ö}ht} erwerben.\\ 
10 & vil liute \textbf{lie dô} verderben\\ 
 & nâch dem Grâl gewerbes list,\\ 
 & dâ von er noch verborgen ist.\\ 
 & Parcifal und Ferefiz\\ 
 & diu wîp lêrten jâmers vlîz.\\ 
15 & si heten\textbf{z} ungern vermiten,\\ 
 & in diu \textit{vi}er stück des hers si riten.\\ 
 & si nâmen urloup zuo aller der diet.\\ 
 & ietweder dan mit vröuden schiet,\\ 
 & gewâpent wol \textbf{ûf} strîte\textit{s w}er.\\ 
20 & an dem \textbf{dritten} tage ûz heidens her\\ 
 & wart zuo Joflanze brâht,\\ 
 & \textbf{sô grôzer} gâbe wart \textbf{nie gedâht}.\\ 
 & welich künic dâ \textbf{sîner} gâbe enpfant,\\ 
 & daz half iemer mêr \textbf{sîn} lant.\\ 
25 & ieglîch\textit{em} man nâch mâze sîn\\ 
 & wart nie \textbf{sô gâbe tiur} schîn,\\ 
 & \textbf{al} den vrowen rîch prêsente\\ 
 & von Triande und von Nori\textit{e}n\textit{t}e.\\ 
 & ich \textit{en}weiz, wie daz her schiede hie.\\ 
30 & Condrie und \textbf{dise} zwên, hin riten sie.\\ 
\end{tabular}
\scriptsize
\line(1,0){75} \newline
m n o V V' W Fr6 \newline
\line(1,0){75} \newline
\textbf{1} \textit{Majuskel} Fr6  \textbf{8} \textit{Majuskel} Fr6  \textbf{13} \textit{Großinitiale} Fr6   $\cdot$ \textit{Initiale} V  \textbf{29} \textit{Initiale} W   $\cdot$ \textit{Majuskel} Fr6  \newline
\line(1,0){75} \newline
\textbf{1} vuoren] fuͦren do W \textbf{2} Parcifal] Parzefal V Parzifal V' Partzifal W  $\cdot$ alsus huop] sus vieng W \textbf{3} franzois] francois m frantzois n franczois o frantzoys W \textbf{4} als] Al o  $\cdot$ Trevrizent] treurizent m tranrizent n tronrizent o Trefrizent V (V') trefrissent W  $\cdot$ vornân] vorne Fr6 \textbf{5} daz] [Dar]: Das n \textbf{6} möht] moht m (o) (V') (Fr6) \textbf{7} von] vor V V' Fr6  $\cdot$ wær dar] dar were V (W) wer V'  $\cdot$ benant] genant V' \textbf{8} daz] Diz Fr6 \textbf{9} möht] moht m (o) (V') (Fr6) \textbf{11} gewerbes] gewerbens V [ger*]: gewerbens V' \textbf{13} Parcifal] PArzefal V Parzifal V' Herr partzifal W  $\cdot$ Ferefiz] ferefis m o ferrefis n ferevis V V' ferafiß W \textbf{15} hetenz] hetten W \textbf{16} in] An W  $\cdot$ vier] mer m (n) o  $\cdot$ stück] stúcken V \textbf{17} aller] alle o (V) (V') \textbf{18} dan mit vröuden] mit freuden dannen V' \textbf{19} ûf] gegen V (V) (Fr6)  $\cdot$ strîtes wer] strittes her wer m strite wer V' \textbf{20} heidens] heiden n Fr6 dez [*]: heidens V dez heiden V' \textbf{21} Joflanze] joflantz m n joflancz o jofflanze V jofflantze V' tschoflantze W \textbf{22} sô] Das W  $\cdot$ wart nie] nie ward W \textbf{23} welich] Swelich V V' (Fr6)  $\cdot$ dâ] do n o [*h]: do V \textit{om.} V' W  $\cdot$ sîner gâbe] seine gabe do W  $\cdot$ enpfant] [enpfing]: enpfant o enphing zvhant V' do enpfant W \textbf{24} lant] hant Fr6 \textbf{25} \textit{Die Verse 786.25-28 fehlen} V'   $\cdot$ ieglîchem] Jeglich m (n) o  $\cdot$ mâze] [*]: der mose V \textbf{26} gâbe tiur] túre gobe n (o) (V) (W) (Fr6) \textbf{28} Triande] driande V  $\cdot$ Noriente] noriande m Noferiente V oriente W novriente Fr6 \textbf{29} enweiz] weis m n (W) wasz o  $\cdot$ daz her] das das her n sich daz her V V' daz her sich Fr6  $\cdot$ schiede] scheide o schiet W \textbf{30} \textit{nach 786.30:} Kv́nig artus vnde die [to*]: tovelrunder (tauelrunder V'  ) alle / Mit in (riten mit in V'  ) mit groszeme schalle V (V')   $\cdot$ Condrie] Cundrie o (Fr6) Kvndrie V V' Knndrie W  $\cdot$ dise] die V' sy W  $\cdot$ riten] rittenz V \newline
\end{minipage}
\end{table}
\newpage
\begin{table}[ht]
\begin{minipage}[t]{0.5\linewidth}
\small
\begin{center}*G
\end{center}
\begin{tabular}{rl}
 & \begin{large}D\end{large}ie boten \textbf{vuoren mit ende} dan.\\ 
 & Parcival sîn rede alsus huop an.\\ 
 & \textbf{mit zühten} er zin allen sprach,\\ 
 & alse Trevrizzent dort \textbf{vorne} jach,\\ 
5 & daz den Grâl zenheinen zîten\\ 
 & niemen möhte erstrîten,\\ 
 & wan der \textbf{von} got \textbf{ist} \textbf{dar benant}.\\ 
 & \textbf{daz} mære kom über elliu lant:\\ 
 & dehein strît \textbf{m\textit{ö}hte in} erwerben.\\ 
10 & vil liute \textbf{lie dô} verderben\\ 
 & nâch dem Grâle gewerbes list,\\ 
 & dâ von er noch verborgen ist.\\ 
 & Parcival unde Feirafiz\\ 
 & diu wîp \textbf{dâ} lêrte\textit{n} jâmers vlîz.\\ 
15 & si heten ungerne vermiten,\\ 
 & in diu vier stücke des hers si riten.\\ 
 & si nâmen urloup ze al der diet.\\ 
 & ietweder dan mit vröuden schiet,\\ 
 & gewâpent wol \textbf{gein} strîtes wer.\\ 
20 & anme \textbf{vierden} tage ûz \textbf{des} heidens her\\ 
 & wart ze Tschofflanze brâht,\\ 
 & \textbf{daz nie grœzer} gâbe wart \textbf{erdâht}.\\ 
 & swelch künic dâ \textbf{sîner} gâbe enpfant,\\ 
 & daz half imer mêre \textbf{daz} lant.\\ 
25 & ieslîchem man nâch mâze sîn\\ 
 & wart nie \textbf{grœzer gâbe} schîn,\\ 
 & den vrouwen rîche prêsente\\ 
 & von Triand unde von Novriente.\\ 
 & ichne weiz, wie daz her \textbf{sich} schiede hie.\\ 
30 & Gundrie unde \textbf{die} zwêne, hin riten sie.\\ 
\end{tabular}
\scriptsize
\line(1,0){75} \newline
G I L M Z \newline
\line(1,0){75} \newline
\textbf{1} \textit{Initiale} G I Z  \textbf{13} \textit{Initiale} I  \textbf{29} \textit{Überschrift:} Aventiwer wie Parzifal van got besezen hat den gral des er herlichen wielt vnd in vnz an sein ende behielt I   $\cdot$ \textit{Initiale} I  \newline
\line(1,0){75} \newline
\textbf{1} mit ende] endehafte L M Z \textbf{2} Sin rede huͯp parzifal do ane L  $\cdot$ Parcival] parcifal G (Z) Parzifal I M  $\cdot$ alsus huop] huͤp al sus I hub M \textbf{4} Trevrizzent] Treuereschent I Trevrizent L trefrezent M Trefrezzent Z  $\cdot$ dort vorne jach] da vornen iach I dort veriach L \textbf{6} möhte] moht I (L) (M) Z \textbf{8} daz] Die L  $\cdot$ mære kom] chom mere I \textbf{9} Jn mochte dehein strit erwerben L  $\cdot$ möhte] mohte G (I) (M) (Z) \textbf{10} lie] lieszin M  $\cdot$ dô] da I M Z \textbf{12} verborgen] [verd]: verborgen I \textbf{13} Parcival] parcifal G (Z) Parzifal I L M  $\cdot$ Feirafiz] feirefiz G Z ferefiz L Feirafisz M \textbf{14} lêrten] lerte G  $\cdot$ vlîz] [list]: fliz I \textbf{17} al der] alle der M aller Z \textbf{19} gein] nach Z  $\cdot$ strîtes] prises L \textbf{20} vierden] vierdem I  $\cdot$ ûz des heidens her] das here M \textbf{21} Tschofflanze] shoffanze I Tschoflanze L scoflanze M tschoflantze Z \textbf{22} grœzer gâbe wart] wart groszer gabe L \textbf{23} swelch] Welch L (M) \textbf{24} daz lant] des lant Z \textbf{25} mâze] der maze I (M) \textbf{27} rîche] riche riche I \textbf{28} Triand] triend G trient I Triant L (M) Z  $\cdot$ Novriente] novrigente I Norient L Noriente M Z \textbf{29} Jch weiz wie her sich schiet hie Z  $\cdot$ sich schiede] sich shaide I sceide M \textbf{30} Gundrie] kvndrie G (Z) Kvndrien L Kundri M  $\cdot$ die] si I  $\cdot$ hin riten sie] man riten lie L riten sie M \newline
\end{minipage}
\hspace{0.5cm}
\begin{minipage}[t]{0.5\linewidth}
\small
\begin{center}*T
\end{center}
\begin{tabular}{rl}
 & die boten \textbf{endehaft vuoren} dan.\\ 
 & Parcifal sîn rede alsus huop an.\\ 
 & \textbf{in franzois} er zuo in allen sprach,\\ 
 & als Trefrizent dort \textbf{vo\textit{r}n} jach,\\ 
5 & daz den G\textit{râl} zuo dekeinen zîten\\ 
 & nieman m\textit{ö}hte erstrîten,\\ 
 & wan der \textbf{von} gote \textbf{ist} \textbf{dar benant}.\\ 
 & \textbf{daz} mære kam über alliu lant,\\ 
 & \textbf{daz} dekein strît \textbf{m\textit{ö}hte \textit{in}} erwerben.\\ 
10 & vil liute \textbf{dô liez} verderben\\ 
 & nâch dem Grâle gewerbes list,\\ 
 & dâ von er noch verborgen ist.\\ 
 & Parcifal und Ferefis\\ 
 & diu wîp \textbf{d\textit{â}} lerten jâmers vlîz.\\ 
15 & si heten ungerne vermiten,\\ 
 & in diu viere \textit{stü}ck\textit{e} des hers si riten.\\ 
 & si nâmen urloup zuo al der diet.\\ 
 & ietweder dan mit vreuden schiet,\\ 
 & gewâpent wol \textbf{gein} strîtes wer.\\ 
20 & an dem \textbf{vierten} tage ûz \textbf{des} heidens her\\ 
 & wart zuo Tschoflanze brâht,\\ 
 & \textbf{daz nie grœzer} gâbe wart \textbf{erdâht}.\\ 
 & welch künec dâ \textbf{sîner} gâbe enpfant,\\ 
 & daz half immer mê \textbf{daz} lant.\\ 
25 & ieclîchem manne nâch \textbf{der} mâze sîn\\ 
 & wart nie \textbf{grœzer gâbe} schîn,\\ 
 & den vrouwen rîche prêsente\\ 
 & von Triande und von Noriente.\\ 
 & ich enweiz, wie daz \textit{h}er schiede hie.\\ 
30 & Kundrie und \textbf{die} zwêne, hin riten sie.\\ 
\end{tabular}
\scriptsize
\line(1,0){75} \newline
U Q R \newline
\line(1,0){75} \newline
\textbf{1} \textit{Illustration mit Überschrift:} Hie sol man kundrien machen mit der schwarczen kappen als da vornen sy stat mit den zen vnd mit dem mund vnd vff ir [k*]: kleid turttúblin nach des grals vappen vnd zu ir parczifaln mit dem sy reit in botschafft wis vom gral Hie rittent die botten hin mit dem brieff zu feirefis habe R   $\cdot$ \textit{Initiale} R  \newline
\line(1,0){75} \newline
\textbf{1} \textit{Die Verse 784.9-789.19 fehlen} Q   $\cdot$ endehaft vuoren] fuͦrent endhafftt R \textbf{2} Parcifal] Parzifal U Parczifal R \textbf{3} franzois] franczois R \textbf{4} Trefrizent] trefrizet U terfrizent R  $\cdot$ vorn] vor in U vornen R  $\cdot$ jach] sprach R \textbf{5} Grâl] gar U \textbf{6} möhte] mochte U \textbf{9} daz] \textit{om.} R  $\cdot$ möhte] mochte U moͯch R  $\cdot$ in] \textit{om.} U \textbf{10} dô liez] lie da R \textbf{13} Parcifal] Parzifal U Parczifal R  $\cdot$ Ferefis] feirefis R \textbf{14} dâ] do U R \textbf{16} stücke] kocken U \textbf{17} urloup] vrlol R \textbf{18} vreuden] froͯwen R \textbf{20} heidens] heides R \textbf{21} Tschoflanze] schofflancze R \textbf{23} Welich kúng do sine gab empfieng R \textbf{24} mê] \textit{om.} R \textbf{27} vrouwen] frowe R \textbf{28} Noriente] Novriende U \textbf{29} enweiz] weis R  $\cdot$ her] er U  $\cdot$ schiede] sich schiede R \textbf{30} Kundrie] Kuͦndrie U Kondrie R \newline
\end{minipage}
\end{table}
\end{document}
