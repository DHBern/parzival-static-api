\documentclass[8pt,a4paper,notitlepage]{article}
\usepackage{fullpage}
\usepackage{ulem}
\usepackage{xltxtra}
\usepackage{datetime}
\renewcommand{\dateseparator}{.}
\dmyyyydate
\usepackage{fancyhdr}
\usepackage{ifthen}
\pagestyle{fancy}
\fancyhf{}
\renewcommand{\headrulewidth}{0pt}
\fancyfoot[L]{\ifthenelse{\value{page}=1}{\today, \currenttime{} Uhr}{}}
\begin{document}
\begin{table}[ht]
\begin{minipage}[t]{0.5\linewidth}
\small
\begin{center}*D
\end{center}
\begin{tabular}{rl}
\textbf{717} & \begin{large}A\end{large}rtus, der wîse, höfsche man,\\ 
 & gienc \textbf{her} ûz zuo den kinden sân.\\ 
 & er gruozte si, dô er si \textbf{sach}.\\ 
 & der kinde einez zim \textbf{dô} sprach:\\ 
5 & "hêrre, der künec Gramoflanz\\ 
 & iuch \textbf{bittet}, daz ir machet ganz\\ 
 & gelübde, diu dâ sî getân\\ 
 & zwischen im unt Gawan,\\ 
 & durch iwer selbes êre.\\ 
10 & hêrre, er \textbf{bittet} iuch mêre,\\ 
 & daz dehein ander man \textbf{im vüere strît}.\\ 
 & iwer her ist sô wît,\\ 
 & solt er si \textbf{alle} übervehten,\\ 
 & daz \textbf{en}glîchte niht dem rehten.\\ 
15 & ir sult Gawanen \textbf{lâzen} komen,\\ 
 & gein dem der kampf \textbf{dâ} \textbf{sî} genomen."\\ 
 & Der künec sprach zen kinden:\\ 
 & "ich wil uns des enbinden.\\ 
 & mîme neven geschach nie grœzer leit,\\ 
20 & daz er \textbf{selbe dâ niht} streit.\\ 
 & der mit iwerm hêrren vaht,\\ 
 & \textbf{dem was der sig} wol geslaht.\\ 
 & \textbf{ez} ist Gahmuretes kint.\\ 
 & alle, \textbf{die} in drîen hern sint\\ 
25 & komen von allen sîten,\\ 
 & die\textbf{ne} \textbf{vrieschen} nie \textbf{gein} strîten\\ 
 & deheinen \textbf{man} sô manlîch.\\ 
 & \textbf{sîn} tât dem prîse ist \textbf{gar} gelîch.\\ 
 & ez ist mîn neve Parzival.\\ 
30 & ir sult in sehen, den lieht gemâl.\\ 
\end{tabular}
\scriptsize
\line(1,0){75} \newline
D \newline
\line(1,0){75} \newline
\textbf{1} \textit{Initiale} D  \textbf{17} \textit{Majuskel} D  \newline
\line(1,0){75} \newline
\textbf{15} Gawanen] Gawann D \textbf{23} Gahmuretes] Gahmvrets D \textbf{29} Parzival] Parcifal D \newline
\end{minipage}
\hspace{0.5cm}
\begin{minipage}[t]{0.5\linewidth}
\small
\begin{center}*m
\end{center}
\begin{tabular}{rl}
 & \begin{large}A\end{large}rtus, der wîse, höfsch man,\\ 
 & gienc \textbf{her} ûz zuo den kinden sân.\\ 
 & er gruozte si, dô er si \textbf{sach}.\\ 
 & der kinde einez zuo im \textbf{dô} sprach:\\ 
5 & "hêrre, der künic Gramolantz\\ 
 & iuch \textbf{bittet}, daz ir mache\textit{t} ganz\\ 
 & gelübde, diu dâ sî getân\\ 
 & zwischen im und Gawan,\\ 
 & durch iuwer selbes êre.\\ 
10 & hêrre, er \textbf{bittet} iuch mêre,\\ 
 & daz kein ander \textit{man} \textbf{\textit{im vüer} strît}.\\ 
 & iuwer her, \textbf{daz} ist sô wît,\\ 
 & solt ers \textbf{alle} übervehten,\\ 
 & daz glîchet niht dem rehten.\\ 
15 & ir solt Gawanen \textbf{heizen} komen,\\ 
 & gegen dem der kampf \textbf{d\textit{â}} \textbf{ist} genomen."\\ 
 & der künic sprach zuo den kinden:\\ 
 & "ich wil uns des enbinden.\\ 
 & mînem neven geschach nie grœzer leit,\\ 
20 & daz er \textbf{selbe d\textit{â} niht} streit.\\ 
 & der mit iuwerm hêrren \textbf{d\textit{â}} vaht,\\ 
 & \textbf{dem was der sic} wo\textit{l g}eslaht.\\ 
 & \textbf{er} ist Gahmurettes kint.\\ 
 & alle in \textbf{den} drîn heren sint\\ 
25 & komen von allen sîten,\\ 
 & die \textbf{gevr\textit{ie}schen} nie \textbf{gegen} strîten\\ 
 & dekeine\textit{n} \textbf{helt} sô manlîch,\\ 
 & \textbf{sît} tât dem prîse ist gelîch.\\ 
 & ez ist mîn neve Parcifal.\\ 
30 & ir solt in sehen, den lieht gemâl.\\ 
\end{tabular}
\scriptsize
\line(1,0){75} \newline
m n o Fr69 \newline
\line(1,0){75} \newline
\textbf{1} \textit{Initiale} m Fr69   $\cdot$ \textit{Capitulumzeichen} n  \newline
\line(1,0){75} \newline
\textbf{1} Artus] A::: Fr69 \textbf{2} kinden] konigen o \textbf{3} gruozte] kuste n  $\cdot$ dô er] do der n \textbf{4} dô] >da< o \textbf{5} Gramolantz] gramolancz o Gramoflanz Fr69 \textbf{6} ir] \textit{om.} Fr69  $\cdot$ machet] machens m \textbf{7} dâ] do n o \textbf{8} Gawan] gawann o \textbf{11} man im vüer] \textit{om.} m \textbf{13} ers] ir o \textbf{14} niht] mit o \textbf{15} heizen] lossen n (o) (Fr69) \textbf{16} dâ] do m n o \textbf{20} selbe] selbes n  $\cdot$ dâ] do m n o \textbf{21} dâ] do m n o \textbf{22} geslaht] gefar geslaht m \textbf{23} Gahmurettes] gamúretes n gamuͯretes o \textbf{24} alle in den] Allo in n All: :dem o \textbf{26} gevrieschen] gefreischen m \textbf{27} dekeinen] Dekeinem m Do keinen n \textbf{28} sît] Sit das n \textbf{30} lieht] liechten o \newline
\end{minipage}
\end{table}
\newpage
\begin{table}[ht]
\begin{minipage}[t]{0.5\linewidth}
\small
\begin{center}*G
\end{center}
\begin{tabular}{rl}
 & \begin{large}A\end{large}rtus, der wîse, höfsche man,\\ 
 & gienc ûz zuo den kinden sân.\\ 
 & er gruozt si, dô er si \textbf{sach}.\\ 
 & der kinde einez zim \textbf{dô} sprach:\\ 
5 & "hêrre, der künec Gramoflanz\\ 
 & iuch \textbf{bite}, daz \textit{i}r machet ganz\\ 
 & gelübde, diu dâ sî getân\\ 
 & zwischen im unde Gawan,\\ 
 & durch iwer selbes êre.\\ 
10 & hêrre, er \textbf{bit} iuch mêre,\\ 
 & daz dehein ander man \textbf{vür in strîte}.\\ 
 & iwer her ist sô wîte,\\ 
 & sold ers übervehten,\\ 
 & daz glîchet niht dem rehten.\\ 
15 & ir sült Gawanen \textbf{lâzen} komen,\\ 
 & gein dem der kampf \textbf{sî} genomen."\\ 
 & der künec sprach ze den kinden:\\ 
 & "ich wil uns des enbinden.\\ 
 & mînem neven geschach nie grôzer leit,\\ 
20 & \textbf{dane} daz er \textbf{selbe niht} streit.\\ 
 & der mit iwerm hêrren vaht,\\ 
 & \textbf{dem was des siges} wol geslaht.\\ 
 & \textbf{er} ist Gahmuretes kint.\\ 
 & alle, \textbf{die} in drîn hern sint\\ 
25 & komen von allen sîten,\\ 
 & die \textbf{gevrieschen} nie \textbf{von} strîten\\ 
 & deheinen \textbf{helt} sô manlîch.\\ 
 & \textbf{sîn} tât dem brîse ist \textbf{gar} gelîch.\\ 
 & ez ist mîn neve Parcival.\\ 
30 & ir sült in sehen, den lieht gemâl.\\ 
\end{tabular}
\scriptsize
\line(1,0){75} \newline
G I L M Z Fr20 Fr24 \newline
\line(1,0){75} \newline
\textbf{1} \textit{Initiale} G I L Z Fr24  \textbf{17} \textit{Initiale} I  \newline
\line(1,0){75} \newline
\textbf{2} gienc] Gienc er M  $\cdot$ sân] dan L \textbf{3} dô] da M Z  $\cdot$ sach] gesach M Fr24 \textbf{4} einez] eyn M  $\cdot$ dô] \textit{om.} L M \textbf{5} Gramoflanz] gramorflanz M gramoflantz Z \textbf{6} bite] bitet I M Z (Fr24)  $\cdot$ ir machet] er macht G \textbf{7} dâ] \textit{om.} I  $\cdot$ sî] sin I Fr24 sint L M Z \textbf{10} bit] enbiut I (L) \textbf{11} dehein] icheyn M kein Z deheine Fr24  $\cdot$ vür in] yme gebe M im fvre Z \textbf{13} ers] er si alle I (L) (M) (Z) (Fr24) \textbf{15} Gawanen] Gawan I (M) (Z) Fr24 \textbf{16} sî] ist L Z \textbf{18} enbinden] enplinden M \textbf{19} grôzer] so I \textbf{20} dane] Wan Z  $\cdot$ selbe] selben M  $\cdot$ niht] da nicht M (Z) (Fr24)  $\cdot$ streit] enstreit I L Fr24 \textbf{22} was des siges] sint die sige I waz siges L was der sig Z der sige: Fr20 \textbf{23} Gahmuretes] Gahmvretes G Fr24 Gamvretes L Gamuͯretis M gamuretes Z Gahmuretis Fr20 \textbf{24} in drîn] inden dren M \textbf{26} gevrieschen] frieschen L \textbf{27} deheinen] Jchein M Keinen Z \textbf{28} gar] \textit{om.} L Fr20 \textbf{29} Parcival] parcifal G Z Fr20 Fr24 parzifal I L M \textbf{30} den] er ist L  $\cdot$ lieht] licht M helt Fr20 \newline
\end{minipage}
\hspace{0.5cm}
\begin{minipage}[t]{0.5\linewidth}
\small
\begin{center}*T
\end{center}
\begin{tabular}{rl}
 & \begin{large}A\end{large}rtus, der wîse, hövesch man,\\ 
 & gienc \textbf{her} ûz zuo den kinden sân.\\ 
 & er gruozte si, dô er si \textbf{gesach}.\\ 
 & der kinde einez zuo im sprach:\\ 
5 & "hêrre, der künec Gramoflanz\\ 
 & iuch \textbf{bitet}, daz ir machet ganz\\ 
 & gelübede, diu d\textit{â} s\textit{î} getân\\ 
 & zwischen im und \textbf{hêrn} Gawan,\\ 
 & durch iuwer selbes êre.\\ 
10 & hêrre, er \textbf{bitet} iuch mêre,\\ 
 & daz dekein ander man \textbf{im gebe strît}.\\ 
 & iuwer her ist sô wît,\\ 
 & solte er si \textbf{alle} übervehten,\\ 
 & daz gelîchet niht dem rehten.\\ 
15 & ir solt Gawan \textbf{lâzen} komen,\\ 
 & gein dem der kampf \textbf{sî} genomen."\\ 
 & der künec sprach zuo den kinden:\\ 
 & "ich wil u\textit{ns} des enbinden.\\ 
 & mîme neven geschach nie grôzer leit,\\ 
20 & \textbf{dan} daz er \textbf{niht selber} streit.\\ 
 & der mit iuwerme hêrren vaht,\\ 
 & \textbf{der ist dem sige} wol geslaht.\\ 
 & \textbf{er} ist Gahmuretes kint.\\ 
 & alle, \textbf{die} \textit{in} drîn hern sint\\ 
25 & k\textit{o}men von allen sîten,\\ 
 & die \textbf{gevrieschen} nie \textbf{gein} strîten\\ 
 & dekeinen \textbf{helt} sô manlîch.\\ 
 & \textbf{sîn} tât dem prîse ist \textbf{gar} glîch.\\ 
 & ez ist mîn neve Parcifal.\\ 
30 & ir solt in sehen, den lieht gemâl.\\ 
\end{tabular}
\scriptsize
\line(1,0){75} \newline
U V W Q R \newline
\line(1,0){75} \newline
\textbf{1} \textit{Initiale} U V W Q R  \newline
\line(1,0){75} \newline
\textbf{1} hövesch] hofliche kunstenriche R \textbf{2} kinden] Zwein knaben R \textbf{3} gesach] sach W Q \textbf{5} Gramoflanz] gramaflanz V gramoflantz W Q Gramoflancz R \textbf{6} bitet] bietet U \textbf{7} dâ] do U V W Q  $\cdot$ sî] sint U \textbf{8} hêrn] herr W \textit{om.} Q R  $\cdot$ Gawan] Gawon V \textbf{9} selbes] selbers Q \textbf{10} bitet] bietet U  $\cdot$ iuch] \textit{om.} W \textbf{11} dekein] dehein V R kein W Q \textbf{13} er si alle] ers alles R \textbf{14} gelîchet] gleichte sich Q gelichet sich R  $\cdot$ rehten] rechte Q \textbf{15} Gawan] Gawanen V (Q) Gawannen R \textbf{16} sî] werd R \textbf{17} zuo den] zúm Q \textbf{18} uns] vch U  $\cdot$ des] den R \textbf{20} dan] Wan Q  $\cdot$ niht selber] do nv́t selber V selber do nicht W selbe da nicht Q selb nit R \textbf{21} hêrren] herre Q \textbf{22} der] Dem Q R  $\cdot$ dem sige] des syges W (Q) (R) \textbf{23} Gahmuretes] Gahmuͦretes U Gameretes V gamuretes W gamúretes Q Gahmurtes R \textbf{24} in] \textit{om.} U \textbf{25} komen] Quamen U  $\cdot$ von] an Q \textbf{26} gevrieschen] gefreischten W gefreischen R \textbf{27} dekeinen] Deheinen V R Keinen W Q \textbf{28} dem] \textit{om.} Q  $\cdot$ ist gar glîch] gleichet sich W ist gleich Q \textbf{29} Parcifal] parzefal V partzifal W Q Parczifal R \textbf{30} lieht] nicht R \newline
\end{minipage}
\end{table}
\end{document}
