\documentclass[8pt,a4paper,notitlepage]{article}
\usepackage{fullpage}
\usepackage{ulem}
\usepackage{xltxtra}
\usepackage{datetime}
\renewcommand{\dateseparator}{.}
\dmyyyydate
\usepackage{fancyhdr}
\usepackage{ifthen}
\pagestyle{fancy}
\fancyhf{}
\renewcommand{\headrulewidth}{0pt}
\fancyfoot[L]{\ifthenelse{\value{page}=1}{\today, \currenttime{} Uhr}{}}
\begin{document}
\begin{table}[ht]
\begin{minipage}[t]{0.5\linewidth}
\small
\begin{center}*D
\end{center}
\begin{tabular}{rl}
\textbf{315} & \begin{large}K\end{large}ünec Artus, dû stüende ze lobe\\ 
 & hôhe dînen genôzen obe.\\ 
 & dîn stîgender prîs nû sinket,\\ 
 & dîn snelliu \textbf{wirde} hinket,\\ 
5 & dîn hôhez lop sich neiget,\\ 
 & dîn prîs hât valsch erzeiget.\\ 
 & Tavelrunde prîses kraft\\ 
 & hât erlemt ein geselleschaft,\\ 
 & die drüber gap hêr Parzival,\\ 
10 & der \textbf{ouch} dort treit diu rîters mâl.\\ 
 & ir nennet in '\textbf{der} ritter rôt',\\ 
 & nâch dem, der lac vor Nantes tôt.\\ 
 & ungelîch ir zweier leben was,\\ 
 & \textbf{wan} munt von rîter nie gelas,\\ 
15 & der pflæge sô ganzer werdecheit."\\ 
 & vome künege \textbf{si vür den Waleis} reit.\\ 
 & Si sprach: "ir tuot mir site buoz,\\ 
 & daz ich versage mînen gruoz\\ 
 & \textbf{Artuse} unt der massenîe sîn.\\ 
20 & geunêrt sî iwer liehter schîn\\ 
 & unt iwer manlîchen lide.\\ 
 & het ich suone oder vride,\\ 
 & \textbf{die} wæren iu \textbf{beidiu} tiure.\\ 
 & ich dunke iuch ungehiure\\ 
25 & unt bin gehiuwerer \textbf{doch} danne ir.\\ 
 & hêr Parzival, wan sagt ir mir\\ 
 & \textbf{unt} bescheidet mich \textbf{einer} mære,\\ 
 & dô der trûrige vischære\\ 
 & saz âne \textbf{vreude} unt âne trôst,\\ 
30 & warumbe iren \textbf{niht siufzens} hât er\textit{l}ôst?\\ 
\end{tabular}
\scriptsize
\line(1,0){75} \newline
D \newline
\line(1,0){75} \newline
\textbf{1} \textit{Initiale} D  \textbf{7} \textit{Majuskel} D  \textbf{17} \textit{Majuskel} D  \newline
\line(1,0){75} \newline
\textbf{30} erlôst] erost D \newline
\end{minipage}
\hspace{0.5cm}
\begin{minipage}[t]{0.5\linewidth}
\small
\begin{center}*m
\end{center}
\begin{tabular}{rl}
 & künic Artus, dû stüende ze lobe\\ 
 & hôhe dînen genôzen obe.\\ 
 & dîn stîgende\textit{r} prîs nû sinket,\\ 
 & dîn snelliu \textbf{vröude} \textbf{nû} hinket,\\ 
5 & dîn hôhez lop sich neiget,\\ 
 & dîn prîs het valsch erz\textit{ei}get.\\ 
 & \textbf{der} tave\textit{l}runde prîses kraft\\ 
 & hât erlemt eine geselleschaft,\\ 
 & die \textit{d}rüber gap hêr Parcifal,\\ 
10 & der \textbf{ouch} dort treit diu ritte\textit{r}s mâl.\\ 
 & i\textit{r} nennet in '\textbf{der} ritter rôt',\\ 
 & nâch dem, der lac vor Nantes tôt.\\ 
 & ungelîch ir zweier leben was,\\ 
 & \textbf{wanne} munt von ritte\textit{r} nie gelas,\\ 
15 & der pflæge sô ganzer werdicheit."\\ 
 & vonme künige \textbf{vür den Waleis si} reit.\\ 
 & si sprach: "ir tuot mi\textit{r} site buoz,\\ 
 & daz ich versage mînen gruoz\\ 
 & \textbf{Artuses} und der massenîe sîn.\\ 
20 & geunêret sî iuwer liehter schîn\\ 
 & und iuwer manlîchen lide.\\ 
 & hete ich suone oder vride,\\ 
 & \textbf{diu} wæren iu \textbf{beiden} tiure.\\ 
 & ich dunke iuch ungehiure\\ 
25 & und bin ge\textit{h}iure\textit{r} \textbf{doch} danne ir.\\ 
 & hêr Parcifal, wanne saget ir mir\\ 
 & \textbf{und} bescheidet mich \textbf{einer} mære,\\ 
 & dô der trûrige vischære\\ 
 & saz âne \textbf{vröude} und âne trôst,\\ 
30 & warumbe iren \textbf{siuftens niht} habt erlôst?\\ 
\end{tabular}
\scriptsize
\line(1,0){75} \newline
m n o \newline
\line(1,0){75} \newline
\newline
\line(1,0){75} \newline
\textbf{1} dû] der o \textbf{2} hôhe] Halp o \textbf{3} stîgender] stigendes m sagener o \textbf{5} sich neiget] ist geneiget n \textbf{6} erzeiget] erzouget m (o) \textbf{7} tavelrunde] taffen runde m \textbf{9} drüber] truber m \textbf{10} ritters] rittes m \textbf{11} ir] Jn m  $\cdot$ der] den n o \textbf{12} lac] do lag n  $\cdot$ Nantes] nannetez o \textbf{13} zweier] zwir o \textbf{14} ritter] ritters m \textbf{16} vür] \textit{om.} o  $\cdot$ Waleis] faleisz n \textbf{17} mir] mit m \textbf{19} Artuses] Artusez o \textbf{21} manlîchen] manlich n o \textbf{23} beiden] beide n o \textbf{25} gehiurer] geburren m \textbf{30} iren] ir n o  $\cdot$ habt] han o \newline
\end{minipage}
\end{table}
\newpage
\begin{table}[ht]
\begin{minipage}[t]{0.5\linewidth}
\small
\begin{center}*G
\end{center}
\begin{tabular}{rl}
 & künic Artus, dû stüende ze lobe\\ 
 & hôhe dînen genôzen obe.\\ 
 & dîn stîgender brîs nû sinket,\\ 
 & dîn snelliu \textbf{wirde} hinket,\\ 
5 & dîn hôhez lop sich neiget,\\ 
 & dîn prîs hât valsch erzeiget.\\ 
 & \textbf{der} tavelrunder brîses kraft\\ 
 & hât erlemet ein geselleschaft,\\ 
 & die drüber gap hêr Parzival,\\ 
10 & der dort treit diu rîters mâ\textit{l}.\\ 
 & ir nennet in \textbf{den} rîter rôt,\\ 
 & nâch dem, der lac vor Nantis tôt.\\ 
 & ungelîch ir zweier leben was:\\ 
 & munt von rîter nie gelas,\\ 
15 & der pflæge \textit{sô} g\textit{an}zer werdicheit."\\ 
 & von dem künige \textbf{si vür den Waleis} reit.\\ 
 & si sprach: "ir tuot mir site buoz,\\ 
 & daz ich versage mînen gruoz\\ 
 & \textbf{dem künige} unde der messenîe sîn.\\ 
20 & geunêrt sî iuwer liehter schîn\\ 
 & unde iuwer manlîche lide.\\ 
 & hete ich suone oder vride,\\ 
 & \textbf{diu} wæren iu \textbf{beidiu} tiure.\\ 
 & ich dunke iuch ungehiure\\ 
25 & unde bin gehiurer \textbf{doch} danne ir.\\ 
 & hêr Parzival, wan saget ir mir\\ 
 & \textbf{unde} bescheidet mich \textbf{der} mære,\\ 
 & dô der trûrige vischære\\ 
 & saz âne \textbf{helfe} unde âne trôst,\\ 
30 & warumbe ir in \textbf{niht \textit{siufzens}} habt erlôst?\\ 
\end{tabular}
\scriptsize
\line(1,0){75} \newline
G I O L M Q R Z Fr39 Fr64 \newline
\line(1,0){75} \newline
\textbf{5} \textit{Initiale} O L  \textbf{17} \textit{Initiale} I R Z Fr39  \newline
\line(1,0){75} \newline
\textbf{1} dû] do O dy M (Fr64) \textbf{2} dînen] dyneme M \textbf{3} stîgender] sigender I \textbf{4} dîn] Div O Eyn M  $\cdot$ hinket] nv hinchet O (M) (Z) \textbf{5} dîn] ÷in O \textbf{6} valsch] falsches Q \textbf{7} der] \textit{om.} Z  $\cdot$ tavelrunder] tavelrvnde O (L)  $\cdot$ brîses] prise R \textbf{9} drüber gap] drvbet gar Z  $\cdot$ hêr] er M Q dr R  $\cdot$ Parzival] parzifal I M Barcifal O parcifal L Z Fr39 partzifal Q parczifal R \textbf{10} der] Der ouch Z  $\cdot$ dort treit] tot trit M  $\cdot$ diu] \textit{om.} O des R  $\cdot$ mâl] man G \textbf{12} Nantis tôt] nantes tot I (L) nantistot M natis tot R nantys tot Fr39 \textbf{13} ungelîch] Eyn glich M \textbf{14} munt] Wan munt Z  $\cdot$ rîter] rittern L \textbf{15} pflæge] pflach O  $\cdot$ sô ganzer] grozer G \textbf{16} von dem] Wonne R  $\cdot$ vür] wider fvr Z  $\cdot$ Waleis reit] waleýs reit L witreit M \textbf{17} si] [i*]: si G  $\cdot$ mir] mit R  $\cdot$ site buoz] side basz M \textbf{18} ich] ir L \textbf{19} dem künige] Dem chvnge Artvs O (M) (Q) (R) (Fr39) Artus Z \textbf{20} liehter] lihter O (L) (M) (Q) \textbf{21} manlîche] manlichiu I (O) menlicher M (Q) minnenclichen Z \textbf{22} ich] ir O \textbf{23} iu] \textit{om.} O  $\cdot$ beidiu] beidin M (R) \textbf{24} dunke] danken R \textbf{25} unde] ich I (M) (Z)  $\cdot$ doch] \textit{om.} O vil Fr39  $\cdot$ danne] an Q \textbf{26} Parzival] Parzifal I (M) (Fr39) Barcifal O parcifal L Z partzifal Q parczifal R \textbf{27} der] einer O L (M) Q (R) Z Fr39 \textbf{28} dô] Da L M Z  $\cdot$ trûrige] truͯriger L (Q) \textbf{30} siufzens] \textit{om.} G trovrens O (Q) (R) (Fr39) \newline
\end{minipage}
\hspace{0.5cm}
\begin{minipage}[t]{0.5\linewidth}
\small
\begin{center}*T
\end{center}
\begin{tabular}{rl}
 & künec Artus, dû stüende ze lobe\\ 
 & hôhe dînen genôzen obe.\\ 
 & dîn stîgende prîs nû sinket,\\ 
 & dîn snell\textit{iu} \textbf{wirde} hinket,\\ 
5 & dîn hôhez lop sich neiget,\\ 
 & dîn prîs hât valsch erzeiget.\\ 
 & tavelrunder prîses kraft\\ 
 & hât erlemt ein geselleschaft,\\ 
 & die drüber gap hêr Parcifal,\\ 
10 & der dort treit diu rîters mâl.\\ 
 & ir nennet in \textbf{den} rîter rôt,\\ 
 & nâch dem, der lac vor Nantes tôt.\\ 
 & unglîch ir zweier leben was:\\ 
 & munt von rîter nie gelas,\\ 
15 & der pflæge sô ganzer werdecheit."\\ 
 & Von dem künege \textbf{si vür den Waleis} reit.\\ 
 & si sprach: "ir tuot mir site buoz,\\ 
 & daz ich versage mînen gruoz\\ 
 & \textbf{Artuse} unde der massenîe sîn.\\ 
20 & geunêrt sî iuwer liehter schîn\\ 
 & unde iuwer manlîchen lide.\\ 
 & hetich suone oder vride,\\ 
 & \textbf{diu} wæren iu \textbf{beidiu} tiure.\\ 
 & ich dunk iuch ungehiure\\ 
25 & unde bin gehiurer \textbf{iedoch} danne ir.\\ 
 & hêr Parcifal, wan saget ir mir,\\ 
 & bescheidet mich \textbf{einer} mære,\\ 
 & dô der trûrige vis\textit{c}hære\\ 
 & saz âne \textbf{h\textit{e}lf} unde âne trôst,\\ 
30 & warumb irn \textbf{siufzens niht} habt erlôst?\\ 
\end{tabular}
\scriptsize
\line(1,0){75} \newline
T U V W \newline
\line(1,0){75} \newline
\textbf{1} \textit{Initiale} W  \textbf{9} \textit{Initiale} V  \textbf{16} \textit{Majuskel} T  \newline
\line(1,0){75} \newline
\textbf{2} Seint hochgeding dein gnos obe W \textbf{3} stîgende] stigender V strenger W \textbf{4} dîn] Vnd dein W  $\cdot$ snelliu] snelle T \textbf{6} erzeiget] erzoͤget V \textbf{7} tavelrunder] Davelruͦnne U (W) [*]: Der tavelrunder  V \textbf{9} die drüber] Der treúwe W  $\cdot$ Parcifal] parzifal T V partzifal W \textbf{11} in] im U  $\cdot$ rîter] ritetr W \textbf{12} der] der do W  $\cdot$ Nantes] nantis W \textbf{16} Waleis] waleisen W \textbf{19} Artuse] Artus W \textbf{21} manlîchen] manliche U V manlich W  $\cdot$ lide] liden U \textbf{22} vride] vriden U \textbf{23} beidiu] beiden U [*]: beide V \textbf{24} ich dunk] mich duͦnke U  $\cdot$ iuch] îv T \textbf{25} unde] Ich W  $\cdot$ iedoch] \textit{om.} V doch W \textbf{26} Parcifal] parzifal T V partzifal W  $\cdot$ ir] \textit{om.} W \textbf{27} bescheidet] [*escheident]: V́nde bescheident V \textbf{28} vischære] viscehere T \textbf{29} helf] half T froͤide V  $\cdot$ trôst] trâst T \textbf{30} habt] han W \newline
\end{minipage}
\end{table}
\end{document}
