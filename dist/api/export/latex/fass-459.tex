\documentclass[8pt,a4paper,notitlepage]{article}
\usepackage{fullpage}
\usepackage{ulem}
\usepackage{xltxtra}
\usepackage{datetime}
\renewcommand{\dateseparator}{.}
\dmyyyydate
\usepackage{fancyhdr}
\usepackage{ifthen}
\pagestyle{fancy}
\fancyhf{}
\renewcommand{\headrulewidth}{0pt}
\fancyfoot[L]{\ifthenelse{\value{page}=1}{\today, \currenttime{} Uhr}{}}
\begin{document}
\begin{table}[ht]
\begin{minipage}[t]{0.5\linewidth}
\small
\begin{center}*D
\end{center}
\begin{tabular}{rl}
\textbf{459} & \begin{large}P\end{large}arzival stuont ûffem snê.\\ 
 & ez tæte einem kranken manne wê,\\ 
 & ob er harnasch trüege,\\ 
 & dâ der vrost sus an \textbf{in} slüege.\\ 
5 & der wirt in vuorte in eine gruft,\\ 
 & dar selten kom des windes luft.\\ 
 & dâ lâgen \textbf{glüendic} koln.\\ 
 & \textbf{die} mohte der gast vil gerne doln.\\ 
 & eine kerzen \textbf{zunde} des wirtes hant.\\ 
10 & dô entwâpent sich der wîgant.\\ 
 & under im lac ramschoup unt \textit{v}arm.\\ 
 & al sîne \textbf{lide} \textbf{im} wurden warm,\\ 
 & sô daz sîn vel liehten schîn\\ 
 & \textbf{gap}. er mohte wol waltmüede sîn,\\ 
15 & wand er het der strâzen \textbf{wênec} g\textit{e}riten,\\ 
 & âne dach die naht des tages erbiten.\\ 
 & als het er manege ander.\\ 
 & getriwen wirt dâ vander.\\ 
 & dâ lac ein roc. den lêch im an\\ 
20 & der wirt unt vuorten \textbf{mit im} dan\\ 
 & zeiner andern gruft. dâ inne was\\ 
 & \textbf{sîniu} buoch, dâr an der kiusche las.\\ 
 & nâch des tages \textbf{site} ein alterstein\\ 
 & dâ stuont al blôz. dâr ûf \textbf{erschein}\\ 
25 & ein kefse. diu wart schiere erkant:\\ 
 & dar ûffe Parzivals hant\\ 
 & \textbf{swuor} einen ungevelschten eit,\\ 
 & dâ von vroun Jeschuten leit\\ 
 & ze liebe wart verkêret\\ 
30 & unt ir vröude gemêret.\\ 
\end{tabular}
\scriptsize
\line(1,0){75} \newline
D \newline
\line(1,0){75} \newline
\textbf{1} \textit{Initiale} D  \newline
\line(1,0){75} \newline
\textbf{1} Parzival] Parcifal D \textbf{11} varm] warm D \textbf{15} geriten] getriten D \textbf{26} Parzivals] Parcifals D \textbf{28} Jeschuten] Jescvten D \newline
\end{minipage}
\hspace{0.5cm}
\begin{minipage}[t]{0.5\linewidth}
\small
\begin{center}*m
\end{center}
\begin{tabular}{rl}
 & Parcifal stuont ûf dem snê.\\ 
 & ez tæte einem kranken man wê,\\ 
 & ob er harnasch trüege,\\ 
 & d\textit{â} der vrost sus an slüege.\\ 
5 & der wirt in vuorte in eine kruft,\\ 
 & dar selten kam des windes luft.\\ 
 & d\textit{â} lâgen \textbf{glüejende} koln.\\ 
 & \textbf{die} mohte der gast vil gerne doln.\\ 
 & eine \textit{k}erze \textbf{enzündet} des wirtes hant.\\ 
10 & dô entwâpent sich der wîgant.\\ 
 & under im lac ramschoup und varm.\\ 
 & alle sîn \textbf{lide} \textbf{ime} wurden warm,\\ 
 & sô daz sîn \textit{v}el \textbf{gap} liehten schîn.\\ 
 & er m\textit{o}hte wol waltmüede sîn,\\ 
15 & wan er het der strâze \textbf{niht mêr} geriten,\\ 
 & âne dach die naht des tag\textit{e}s erbiten.\\ 
 & als het er manige ander.\\ 
 & getriuwen wirt d\textit{â} vander.\\ 
 & d\textit{â} lac ein roc. den lêch im an\\ 
20 & der wirt und vuorte in dan\\ 
 & zuo einer andern kruft. dâr in was\\ 
 & \textbf{sîn} buoch, dâr an der kiusche las.\\ 
 & nâch des tages \textbf{sit} ein alterstein\\ 
 & d\textit{â} stuont \dag abelâz\dag . dâr ûf \textbf{erschein}\\ 
25 & ein k\textit{a}f\textit{s}e. diu wart schier \textit{erk}ant:\\ 
 & dar ûf Parcifals hant\\ 
 & \textbf{swuor} einen ungevelscheten eit,\\ 
 & dâ von vrouwen Jeschuten leit\\ 
 & zuo li\textit{e}be wart verkêret\\ 
30 & und ir vröude gemêret.\\ 
\end{tabular}
\scriptsize
\line(1,0){75} \newline
m n o \newline
\line(1,0){75} \newline
\textbf{1} \textit{Capitulumzeichen} n  \newline
\line(1,0){75} \newline
\textbf{4} dâ] Do m n o  $\cdot$ vrost] furst m  $\cdot$ an] an in n o \textbf{5} kruft] crafft n \textbf{7} dâ] Do m n o  $\cdot$ glüejende] gluͯgenden n \textbf{9} kerze] hercz m \textbf{11} ramschoup] rouͯm scoup n \textbf{12} lide] gelide n \textit{om.} o  $\cdot$ wurden] wuͯrdent o \textbf{13} vel] pfelle m felle n o \textbf{14} mohte] moͯhte m (n) (o) \textbf{15} strâze niht mêr] strossen wenig n (o) \textbf{16} tages] [taget]: tagets m \textbf{18} dâ] do m n o \textbf{19} dâ] Do m n o  $\cdot$ den lêch im] lecht nuͯ o \textbf{20} in] in mit jme n (o) \textbf{21} andern] ander o \textbf{22} las] lag o \textbf{23} sit] sint o \textbf{24} dâ] Do m n o \textbf{25} ein kafse] Ein kouffe m Ein koffe n En koffe o  $\cdot$ erkant] getant m \textbf{26} Parcifals] parcifales n o \textbf{27} ungevelscheten] vngefelschen n \textbf{28} vrouwen] frouwe m (o) o\textit{m. } n  $\cdot$ Jeschuten] jescuten m (n) jescuͯten o \textbf{29} liebe] libe m \newline
\end{minipage}
\end{table}
\newpage
\begin{table}[ht]
\begin{minipage}[t]{0.5\linewidth}
\small
\begin{center}*G
\end{center}
\begin{tabular}{rl}
 & \begin{large}P\end{large}arzival stuont ûf dem snê. \\ 
 & ez tæte eine\textit{m} kranken manne wê,\\ 
 & ob er harnasch trüege,\\ 
 & dâ der vrost sus an slüege.\\ 
5 & der wirt in vuorte in eine gruft,\\ 
 & dar selten kom des windes luft.\\ 
 & dâ lâgen \textbf{glüende} kolen.\\ 
 & \textbf{daz} mohte der gast vil gerne dolen.\\ 
 & eine kerzen \textbf{zunte} des wirtes hant.\\ 
10 & dô entwâpente sich der wîgant.\\ 
 & under im lac ramschoup unde varm.\\ 
 & al sîn \textbf{lide} \textbf{i\textit{m}} wurden warm,\\ 
 & sô daz sîn vel \textbf{gap} liehten schîn.\\ 
 & er moht wol waltmüede sîn,\\ 
15 & wan er het der strâze \textbf{wênic} geriten,\\ 
 & âne da\textit{ch} die naht des tages erbiten.\\ 
 & als het er manege ander.\\ 
 & getriuwen wirt dâ vander.\\ 
 & dâ lac ein roc. den lê\textit{ch} im an\\ 
20 & der wirt unde vuort in \textbf{mit im} dan\\ 
 & ze einer andern gruft. dâ inne was\\ 
 & \textbf{sîniu} buoch, dâr an der kiusche las.\\ 
 & nâch des tages \textbf{site} ein alterstein\\ 
 & dâ stuont alblôz. dâr ûf \textbf{erschein}\\ 
25 & ein ke\textit{fs}e. diu wart schier erkant:\\ 
 & dar ûffe Parzivales hant\\ 
 & \textbf{swuor} einen ungevelscheten eit,\\ 
 & dâ von vrôn Jeschuten leit\\ 
 & ze liebe wart \textit{v}erkêr\textit{et}\\ 
30 & unde ir vröude gemêret.\\ 
\end{tabular}
\scriptsize
\line(1,0){75} \newline
G I O L M Z \newline
\line(1,0){75} \newline
\newline
\line(1,0){75} \newline
\textbf{1} Parzival] Parzifal I L M ÷Arcifal O Parcifal Z \textbf{2} einem] einen G  $\cdot$ kranken] chranchem I O \textbf{4} dâ] Do O  $\cdot$ an] an in O L Z an uch M \textbf{5} in vuorte] in fuerte G vuͤrt in I \textbf{7} lâgen] lugen M  $\cdot$ glüende] gluͦndiu I genvͦge O glvndige Z \textbf{8} daz] Die O L M Z \textbf{9} zunte] inzcunde M zvͤnt Z \textbf{10} dô] Da M Z  $\cdot$ entwâpente] entwapent I (O) (L) Z \textbf{11} \textit{Versdoppelung 459.11-12 (²Z) nach 457.28; Lesarten der vorausgehenden Verse mit ¹Z bezeichnet} Z   $\cdot$ ramschoup] reyne schoip M  $\cdot$ varm] warm L \textbf{12} al] Alle O  $\cdot$ lide] ledeme M  $\cdot$ im] in G \textit{om.} O L M die \textsuperscript{2}\hspace{-1.3mm} Z \textbf{13} gap] \textit{om.} M  $\cdot$ liehten] lýchten L (M) \textbf{14} wol] \textit{om.} L \textbf{15} het] \textit{om.} Z  $\cdot$ strâze] strazin G  $\cdot$ wênic] wenic het Z \textbf{16} dach] danc G  $\cdot$ tages] \textit{om.} L \textbf{17} er] \textit{om.} M \textbf{18} dâ] \textit{om.} Z \textbf{19} dâ] Do L  $\cdot$ lêch] leit G \textbf{20} vuort] fuert G \textbf{21} ze] Ezn M  $\cdot$ andern gruft] andir crafft M \textbf{22} sîniu] Sin M \textbf{23} des] \textit{om.} L  $\cdot$ ein] einen Z  $\cdot$ alterstein] altir [man]: steyn M \textbf{24} dâ] der I  $\cdot$ erschein] schein O L \textbf{25} kefse] chesfe G \textbf{26} Parzivales] parzifales I Barcifals O parzifalz L Parzifals M parcifals Z \textbf{27} swuor] So wor M  $\cdot$ ungevelscheten] [vngeschaffen]: vngefalischten L \textbf{28} von] \textit{om.} L  $\cdot$ vrôn] vrow L (M)  $\cdot$ Jeschuten] ieschvten G iescuten I Z Jescuͯten L jescuten M \textbf{29} verkêret] wercherte G gekeret Z \newline
\end{minipage}
\hspace{0.5cm}
\begin{minipage}[t]{0.5\linewidth}
\small
\begin{center}*T
\end{center}
\begin{tabular}{rl}
 & \begin{large}P\end{large}arcifal stuont ûf dem snê.\\ 
 & ez tæte einem kranken manne wê,\\ 
 & ob er harnasch trüege,\\ 
 & dâ der vrost sus an \textbf{in} slüege.\\ 
5 & Der wirt in vuorte in \textit{eine} gruft,\\ 
 & d\textit{ar} selten kom des windes \textit{l}u\textit{f}t.\\ 
 & dâ lâgen \textbf{glüende} koln.\\ 
 & \textbf{die} mohte der gast vil gerne doln.\\ 
10 & \hspace*{-.7em}\big| dô entwâpente sich der wîgant.\\ 
 & \hspace*{-.7em}\big| eine kerze \textbf{enzunte} de\textit{s} wirtes hant.\\ 
 & under im lac ramschoup unde varm.\\ 
 & alle sîne \textbf{gelide} wurden warm,\\ 
 & sô daz sîn vel \textbf{gap} liehten schîn.\\ 
 & er mohte wol waltmüede sîn,\\ 
15 & wan er hete der strâze \textbf{wênic} geriten,\\ 
 & âne dach die naht des tages erbiten.\\ 
 & alse het er manege ander.\\ 
 & getriuwen wirt dâ vander.\\ 
 & \textit{dâ lac ein roc. den lêch im an}\\ 
20 & \textit{der wirt und vuorte in \textbf{mit im} dan}\\ 
 & zeiner andern gruft. dâ inne was\\ 
 & \textbf{sîn} buoch, dâr an der kiusche las.\\ 
 & nâch des tages \textbf{zît} ein alterstein\\ 
 & dâ stuont alblôz. dâr ûffe \textbf{schein}\\ 
25 & ein kafse. di\textit{u} war\textit{t} schiere erkant:\\ 
 & dar ûffe \textbf{swuor} Parcifales hant\\ 
 & einen ungevelscheten eit,\\ 
 & dâ von vroun Jeschuten leit\\ 
 & ze liebe wart verkêret\\ 
30 & unde ir vröude gemêret.\\ 
\end{tabular}
\scriptsize
\line(1,0){75} \newline
T U V W Q R \newline
\line(1,0){75} \newline
\textbf{1} \textit{Initiale} T V W Q   $\cdot$ \textit{Capitulumzeichen} R  \textbf{5} \textit{Majuskel} T  \newline
\line(1,0){75} \newline
\textbf{1} \textit{Die Verse 453.1-502.30 fehlen} U   $\cdot$ Parcifal] Parzifal V PArtzifal W (Q) Parczifal R  $\cdot$ dem] den W R \textbf{2} manne] \textit{om.} Q \textbf{3} harnasch] an harnesch V \textbf{4} dâ] Daz V Do W Q  $\cdot$ sus] also R \textbf{5} vuorte] furt sunst Q  $\cdot$ eine] \textit{om.} T \textbf{6} dar] den T [D*]: Dar V Do Q  $\cdot$ luft] vlvht T \textbf{7} dâ] Do W Q  $\cdot$ glüende] [k]: gluende Q \textbf{8} die] [D*]: Die V  $\cdot$ mohte] moͤcht W  $\cdot$ vil] \textit{om.} R \textbf{10} \textit{Versfolge 459.9-10} Q R   $\cdot$ dô] Da V  $\cdot$ entwâpente] [entwapente*]: entwapente: T entwappent Q (R) \textbf{9} eine kerze] Fin kertze W Ein kerczen R  $\cdot$ enzunte] enzűnt Q (R)  $\cdot$ des] der T \textbf{11} ramschoup] gras schoͮp V rein schoup W \textbf{12} gelide] lide im V lide Q (R) \textbf{14} mohte] moͤchte W (R) \textbf{15} hete der strâze wênic] der straze wening hatte V  $\cdot$ geriten] gerit::: Q \textbf{17} het] hat R \textbf{18} getriuwen] Getrúwer R  $\cdot$ wirt] ward R  $\cdot$ dâ] do V W Q R \textbf{19} \textit{Die Verse 459.19-20 fehlen (Zeilen ausgespart)} T   $\cdot$ dâ] do V W Q  $\cdot$ den] dem Q \textbf{22} sîn] Seine Q Sinú R \textbf{23} zît] sitte V W (Q) R \textbf{24} dâ] Do V W Q  $\cdot$ alblôz] blos V ablas W R  $\cdot$ schein] erschein V \textbf{25} kafse] kapsce W  $\cdot$ diu] die T do V  $\cdot$ wart] war T \textbf{26} Parcifales] Parcifals T parzifales V partzifals W Q parczifals R \textbf{28} vroun] fraw W \textit{om.} R  $\cdot$ Jeschuten] Jescuten T (V) (Q) iestuten W Jescutten R \textbf{29} verkêret] gekeret Q \newline
\end{minipage}
\end{table}
\end{document}
