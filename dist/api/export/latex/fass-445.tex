\documentclass[8pt,a4paper,notitlepage]{article}
\usepackage{fullpage}
\usepackage{ulem}
\usepackage{xltxtra}
\usepackage{datetime}
\renewcommand{\dateseparator}{.}
\dmyyyydate
\usepackage{fancyhdr}
\usepackage{ifthen}
\pagestyle{fancy}
\fancyhf{}
\renewcommand{\headrulewidth}{0pt}
\fancyfoot[L]{\ifthenelse{\value{page}=1}{\today, \currenttime{} Uhr}{}}
\begin{document}
\begin{table}[ht]
\begin{minipage}[t]{0.5\linewidth}
\small
\begin{center}*D
\end{center}
\begin{tabular}{rl}
\textbf{445} & begreif mit sînen handen.\\ 
 & nû \textbf{jeh\textit{ts}} im niht ze schanden,\\ 
 & daz er sich âne schergen hienc.\\ 
 & mit den vüezen er gevienc\\ 
5 & under im des velses herte.\\ 
 & in grôzem ungeverte\\ 
 & lac daz ors \textbf{dort} niden tôt.\\ 
 & der ritter gâhete von der nôt\\ 
 & anderhalben ûf \textbf{die} halden hin.\\ 
10 & wolt er teilen den gewin,\\ 
 & den er erwarp an Parzival,\\ 
 & sô half im baz dâ heime der Grâl.\\ 
 & Parzival her wider steic.\\ 
 & der zügel gein der erden seic.\\ 
15 & dâ hete daz ors \textbf{durch} getreten,\\ 
 & als ob ez bîtens wære gebeten,\\ 
 & \textbf{des} jener ritter dâ vergaz.\\ 
 & dô Parzival dar ûf gesaz,\\ 
 & dô\textbf{ne} was niht wan sîn sper verlorn.\\ 
20 & \textbf{diu vlust} gein vînden was verkorn.\\ 
 & ich wæne, der \textbf{starke} Læhelin\\ 
 & noch der stolze Kingrisin\\ 
 & noch roy Gramoflanz\\ 
 & noch cons \textbf{Lascoyt fiz} Gurnemanz,\\ 
25 & nie bezzer \textbf{tjost} geriten,\\ 
 & denne als \textbf{diz ors wart} erstriten.\\ 
 & dô reit er, erne wiste \textbf{selbe} war,\\ 
 & sô daz \textbf{der Munsalvæscher} schar\\ 
 & in mit strîte gar vermeit.\\ 
30 & des Grâles vremde was im leit.\\ 
\end{tabular}
\scriptsize
\line(1,0){75} \newline
D Fr5 Fr31 \newline
\line(1,0){75} \newline
\textbf{1} \textit{Initiale} Fr31  \newline
\line(1,0){75} \newline
\textbf{2} jehts im] iehst im D iehent imes Fr31  $\cdot$ schanden] schande Fr5 \textbf{5} velses] velsen Fr31 \textbf{7} dort] da Fr5 Fr31  $\cdot$ niden] nidinin Fr5 \textbf{9} die] der Fr31 \textbf{11} Parzival] Parcifal D (Fr5) Parzifal Fr31 \textbf{12} im] in Fr31 \textbf{13} Parzival] Parcifal D Fr5 Parzifal Fr31  $\cdot$ her wider] her nidir Fr5 (Fr31) \textbf{14} zügel] zvͦhel Fr31  $\cdot$ erden] erde Fr31 \textbf{16} ob] \textit{om.} Fr5  $\cdot$ bîtens wære] were bîtens Fr5 war bitens Fr31 \textbf{18} Parzival] Parcifal D (Fr5) Parzifal Fr31 \textbf{19} dône] Da ne Fr5 Do Fr31 \textbf{20} gein vînden was] gein vigindin Fr5 si gein vienden Fr31 \textbf{21} Læhelin] lehilin Fr5 \textbf{22} Kingrisin] Kyngrisin D (Fr5) \textbf{23} roy] roys Fr5 Fr31  $\cdot$ Gramoflanz] Gramovlanz D gramuͦlanz Fr5 gramvlanz Fr31 \textbf{24} cons Lascoyt] Cons Lascôyt D ouns laschuͦt Fr5 ovnslaschvͦt Fr31  $\cdot$ Gurnemanz] gurnimanz Fr5 \textbf{26} diz ors wart] was diz ors Fr5 (Fr31) \textbf{27} selbe] \textit{om.} Fr5 Fr31 \textbf{28} sô] Do Fr31  $\cdot$ der Munsalvæscher] der Mvnsælvæscer D div muntsaluasch Fr5 div mvntschalvasche Fr31 \newline
\end{minipage}
\hspace{0.5cm}
\begin{minipage}[t]{0.5\linewidth}
\small
\begin{center}*m
\end{center}
\begin{tabular}{rl}
 & begrei\textit{f} mit sînen handen.\\ 
 & nû \textbf{jeh\textit{ts}} ime niht zuo scha\textit{n}den,\\ 
 & daz er sich âne sch\textit{er}jen hienc.\\ 
 & mit den vüezen er gevienc\\ 
5 & under ime des velses herte.\\ 
 & in grôzem ungeverte\\ 
 & lac daz ros \textbf{dort} \textit{ni}den tôt.\\ 
 & der ritter gâhete von der nôt\\ 
 & anderhalben ûf \textbf{die} halden hin.\\ 
10 & wolte er teilen den gewin,\\ 
 & den er \textit{er}warp an Parcifal,\\ 
 & sô half ime baz dâ heime der Grâl.\\ 
 & Parcifal her wider steic.\\ 
 & der zügel gegen der erden seic.\\ 
15 & dâ hete daz ros \textbf{durch} getreten,\\ 
 & als ob ez bîtens wære gebeten,\\ 
 & \textbf{daz} jener ritter d\textit{â} vergaz.\\ 
 & dô Parcifal dar ûf gesaz,\\ 
 & dô \textbf{en}was niht wenne sîn sper verlorn,\\ 
20 & \textbf{die wîle} \textbf{v\textit{lu}st} gegen vînden was verkorn.\\ 
 & ich wæne, de\textit{r} \textbf{starke} Lehelin\\ 
 & noch der stolze Kingrisi\textit{n}\\ 
 & noch rois Gram\textit{o}lanz\\ 
 & noch cuns \textbf{Lasc\textit{o}t fi\textit{z}} Gurnemanz,\\ 
25 & nie bezzer \textbf{just} geriten,\\ 
 & danne als \textbf{diz ros wart} erstriten.\\ 
 & dô reit er, er enwuste war,\\ 
 & sô daz \textbf{diu Mun\textit{t}salvasche} schar\\ 
 & in mit strîte gar vermeit.\\ 
30 & des Grâles vre\textit{m}de was im leit.\\ 
\end{tabular}
\scriptsize
\line(1,0){75} \newline
m n o \newline
\line(1,0){75} \newline
\newline
\line(1,0){75} \newline
\textbf{1} begreif] Begreis m \textbf{2} jehts] gehst m get es n genchtz o  $\cdot$ schanden] schaden m \textbf{3} scherjen] schreien m  $\cdot$ hienc] ving o \textbf{5} ime] \textit{om.} n o  $\cdot$ velses] falsches n (o) \textbf{7} niden] in den m nider o \textbf{11} erwarp] warp m \textbf{13} steic] [streit]: streig o \textbf{15} getreten] tretten n \textbf{17} daz jener] Der jmer o  $\cdot$ dâ] do m n o \textbf{19} wenne] wenic n \textbf{20} vlust] fulst m  $\cdot$ vînden] vinde n winde o \textbf{21} wæne der] [were]: wene de m \textbf{22} Kingrisin] kingrisim m kingresin n konigresin o \textbf{23} Gramolanz] Gramanlanz m gramolans n gramolancz o \textbf{24} cuns Lascot] kuns lascoit m conslascoit n glasca [consla*]: conslascvit o  $\cdot$ fiz] fir m o fúr n  $\cdot$ Gurnemanz] gurnemancz m gúrnemans n guͯrmenancz o \textbf{27} reit] [er]: reit er m  $\cdot$ enwuste] wuste n o \textbf{28} daz] das das o  $\cdot$ Muntsalvasche] munsaluasce m monsoluasce n montsaluasce o \textbf{30} vremde] urende m \newline
\end{minipage}
\end{table}
\newpage
\begin{table}[ht]
\begin{minipage}[t]{0.5\linewidth}
\small
\begin{center}*G
\end{center}
\begin{tabular}{rl}
 & begreif mit sînen handen.\\ 
 & nû \textbf{gebets} im niht ze schanden,\\ 
 & daz er sich ân scherigen hienc.\\ 
 & mit den vüezen er gevienc\\ 
5 & under  de\textit{s} velse\textit{s} hert\textit{e}.\\ 
 & in grôzem ungeverte\\ 
 & lac daz ors \textbf{dort} niden tôt.\\ 
 & der rîter gâhte von der nôt\\ 
 & anderhalp ûf \textbf{die} halden hin.\\ 
10 & wolder teilen den gewin,\\ 
 & den er erwa\textit{r}p an Parzival,\\ 
 & sô half i\textit{m} baz dâ heime der Grâl.\\ 
 & Parzival her wider steic.\\ 
 & der zügel gein de\textit{r} erden seic.\\ 
15 & dâ het daz ors \textbf{durch} getreten,\\ 
 & als ob ez bîtens wære gebeten,\\ 
 & \textbf{des} jener rîter dâ vergaz.\\ 
 & dô Parzival drûf gesaz,\\ 
 & dô\textbf{ne} was niht wan sîn sper verlorn.\\ 
20 & \textbf{der schade} gein vînden was verkorn.\\ 
 & ich wæne, der \textbf{starke} Lehelin\\ 
 & noch der stolze Kingrisin\\ 
 & noch roys Gramoflanz\\ 
 & noch cons \textbf{fiz Lascheit} Gurnomanz\\ 
25 & nie bezzer \textbf{tjost} geriten,\\ 
 & danne als \textbf{wart daz ors} erstriten.\\ 
 & dô reit er, erne wesse war,\\ 
 & sô daz \textbf{diu Muntsalvatscher} schar\\ 
 & in mit strîte gar vermeit.\\ 
30 & des Grâles vrömede was im leit.\\ 
\end{tabular}
\scriptsize
\line(1,0){75} \newline
G I O L M Z \newline
\line(1,0){75} \newline
\textbf{1} \textit{Initiale} O L Z  \textbf{19} \textit{Initiale} I  \newline
\line(1,0){75} \newline
\textbf{1} begreif] ÷egreif O \textbf{2} nu iehet sin nih zeshanden I · Nv ieht yms nihtzeschanden O · Nv ieht ez im >ir< iht zuͯ schanden L · Nu set vns nicht zcu schanden M · Nv geht imz niht zv schanden Z \textbf{3} scherigen] strengen M \textbf{5} Vnder dem felse herten G  $\cdot$ under] Vnder im O L (M) Z \textbf{6} in] mit I \textbf{7} dort] da I  $\cdot$ niden] nider I inden O \textbf{8} gâhte] gahet O sprach M \textbf{9} anderhalp] Ander [hap]: halp G  $\cdot$ hin] [sin]: hin G \textbf{11} den] dan I  $\cdot$ er erwarp] er erwap G er gewan I er warp O erwarp L M  $\cdot$ an] von Z  $\cdot$ Parzival] parziual G parzifal I M Parcifal O (Z) parzifale L \textbf{12} sô] Da O  $\cdot$ half] hulf I  $\cdot$ im] in G \textbf{13} Parzival] Parziual G parzifal I (M) (L) Parcifal O Z \textbf{14} der] [de*]: den G  $\cdot$ erden] erde O L \textbf{15} het] hat M \textbf{16} als ob] als I Also uff M Sam ob Z  $\cdot$ bîtens] bindens I \textbf{17} \textit{Versfolge 445.18-17} Z  \textbf{18} dô] Dv O Da M Z  $\cdot$ Parzival] parziual G Parzifal I (L) (M) Parcifal O (Z) \textbf{19} dône was] Do was O (L) Da enwas M Z \textbf{20} der schade] Div flvst O (L) (M) (Z)  $\cdot$ gein vînden was] was geyn vienden M \textbf{21} Lehelin] læhelin O \textbf{22} stolze] starche I  $\cdot$ Kingrisin] kyngrisin O \textbf{23} Gramoflanz] Gramov Lanz L gramoflancz M \textbf{24} cons fiz Lascheit] conz fiz lashoit I cons lascort fiz O kvnglischot fiz L kons lascoit fisz M (Z)  $\cdot$ Gurnomanz] Garnemanz I Gvrnemanz O (M) (Z) Gvrnomantz L \textbf{25} nie] nie hetn I \textbf{26} wart daz ors] diz ors was I daz ors wart da O wart diz rosz L (M) ditz orss wart Z \textbf{27} dô] Da O M Z \textbf{28} diu] in M  $\cdot$ Muntsalvatscher] muntshalvasche I mvntsalvatsche O Mutsalvatsche M montsalvatscher Z \textbf{30} Jm was des grals fremde leit Z  $\cdot$ was im] yme was M \newline
\end{minipage}
\hspace{0.5cm}
\begin{minipage}[t]{0.5\linewidth}
\small
\begin{center}*T
\end{center}
\begin{tabular}{rl}
 & begreif mit sînen handen.\\ 
 & Nû \textbf{jehet\textit{s}} im niht ze schanden,\\ 
 & daz er sich âne schergen hienc.\\ 
 & mit den vüezen er gevienc\\ 
5 & under im des velses herte.\\ 
 & in grôzem ungeverte\\ 
 & lac daz ors \textbf{dâ} nidene tôt.\\ 
 & der rîter gâhete von der nôt\\ 
 & anderhalp ûf \textbf{der} halden hin.\\ 
10 & wolter teilen den gewin,\\ 
 & den er erwarp an Parcifal,\\ 
 & sô half im baz dâ heime der Grâl.\\ 
 & \begin{large}P\end{large}arcifal her wider steic.\\ 
 & der zügel gegen der erden seic.\\ 
15 & dâ hete daz ors \textbf{in} getreten,\\ 
 & als ob ez bîtens wære gebeten,\\ 
 & \textbf{des} jener rîter dâ vergaz.\\ 
 & dô Parcifal dar ûf gesaz,\\ 
 & dô was niht wan sîn sper verlorn.\\ 
20 & \textbf{diu vlust} gegen vîenden was verkorn.\\ 
 & ich wæne, der \textbf{stolze} Lehelin\\ 
 & noch der stolze Kyngrisin\\ 
 & \hspace*{-.7em}\big| noch cons \textbf{Lascot fiz} Gurnemanz,\\ 
 & \hspace*{-.7em}\big| noch rois Gramoflanz\\ 
25 & nie bezzer \textbf{ors} geriten,\\ 
 & danne alse \textbf{diz ors wart} erstriten.\\ 
 & Dô reit er, ern wiste war,\\ 
 & sô daz \textbf{diu Munsalvasche} schar\\ 
 & in mit strîte gar vermeit.\\ 
30 & des Grâles vremde was im leit.\\ 
\end{tabular}
\scriptsize
\line(1,0){75} \newline
T U V W Q R \newline
\line(1,0){75} \newline
\textbf{2} \textit{Majuskel} T  \textbf{13} \textit{Initiale} T U V  \textbf{27} \textit{Majuskel} T  \newline
\line(1,0){75} \newline
\textbf{2} jehets im] iehent im is U gecht ims W (Q) iehet vnsz Q \textbf{3} schergen] [*]: scherien V schregen R \textbf{4} gevienc] geinck Q \textbf{5} velses] felsches Q \textbf{6} in] [im]: in T \textbf{7} lac] Lat Q  $\cdot$ dâ nidene] [d* i*]: dort niden V dort in den Q (R) \textbf{8} rîter] Rittenter R \textbf{9} anderhalp] Ander habp Q  $\cdot$ der] [*]: die V die W Q R  $\cdot$ halden] haiden W  $\cdot$ hin] [syͯn]: :n hin Q \textbf{11} Parcifal] Parzifal U (V) partzifal W Q parczifal R \textbf{12} sô] [*o]: So V Ja Q  $\cdot$ dâ heime] dekeime U  $\cdot$ der] >der< U \textbf{13} Parcifal] Parzifal V Partzifal W Q Barczifal R  $\cdot$ her] er Q  $\cdot$ steic] streit Q \textbf{14} erden] erde R \textbf{15} hete] hat R  $\cdot$ in] durch W Q R \textbf{17} des] Daz V  $\cdot$ jener] einer U ienre V  $\cdot$ dâ vergaz] do vergas V (W) [dovergasz]: do vergasz  Q \textbf{18} Parcifal] parzifal V partzifal W Q parczifal R \textbf{20} vlust] flucht W \textbf{21} stolze] [st*]: starke V starcke W Q R  $\cdot$ Lehelin] lechelin R \textbf{22} Kyngrisin] Kŷnrisin T kingrisin W \textbf{24} \textit{Versfolge 445.23-24} W Q R   $\cdot$ cons Lascot] konslaschot U graue [*]: laschoit V conflastot W Conflascot Q Konslascot R  $\cdot$ fiz] svn V  $\cdot$ Gurnemanz] gurnemantz Q guͯrnomancz R \textbf{23} rois] kv́nig V  $\cdot$ Gramoflanz] gramaflanz V gramaflantz W gramoflans Q gramoflancz R \textbf{25} ors] [*] tiost V tyost W (Q) (R) \textbf{27} ern wiste] ern [feste]: weste Q en weste R \textbf{28} Munsalvasche] mvnsalvasce T muͦntsalvatscher U [mvnscha*]: mvnschalvasche V montsaluatscher W muntsalvasche Q \textbf{29} vermeit] vermitten R \textbf{30} Vnd enruͯchttent war er Ritte R \newline
\end{minipage}
\end{table}
\end{document}
