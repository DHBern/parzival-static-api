\documentclass[8pt,a4paper,notitlepage]{article}
\usepackage{fullpage}
\usepackage{ulem}
\usepackage{xltxtra}
\usepackage{datetime}
\renewcommand{\dateseparator}{.}
\dmyyyydate
\usepackage{fancyhdr}
\usepackage{ifthen}
\pagestyle{fancy}
\fancyhf{}
\renewcommand{\headrulewidth}{0pt}
\fancyfoot[L]{\ifthenelse{\value{page}=1}{\today, \currenttime{} Uhr}{}}
\begin{document}
\begin{table}[ht]
\begin{minipage}[t]{0.5\linewidth}
\small
\begin{center}*D
\end{center}
\begin{tabular}{rl}
\textbf{46} & "\textbf{gêt} \textbf{nâher}, \textbf{mîn hêr} Razalic.\\ 
 & \multicolumn{1}{l}{ - - - }\\ 
 & \textbf{ir sult küssen} mîn wîp.\\ 
 & \multicolumn{1}{l}{ - - - }\\ 
 & \textbf{\textit{\begin{large}A\end{large}}lsô} tuot \textbf{ouch} ir, hêr Gaschier."\\ 
 & Hiutegern den \textbf{Schotten} fier\\ 
5 & bat er si küssen an \textbf{ir} munt.\\ 
 & \textbf{der} was von sîner tjost wunt.\\ 
 & er bat si alle sitzen.\\ 
 & \textbf{al stênde} \textbf{sprach er} mit witzen:\\ 
 & "ich sæhe \textbf{ouch} gerne den neven mîn,\\ 
10 & m\textit{ö}ht ez mit sînen hulden sîn,\\ 
 & der in hie gevangen hât.\\ 
 & ine hâns \textbf{von} sippe decheinen rât,\\ 
 & i\textbf{ne} muoz in ledec machen."\\ 
 & diu künegîn begunde lachen.\\ 
15 & Si hiez \textbf{balde} \textbf{nâch im springen}.\\ 
 & dort her begunde dringen\\ 
 & der minneclîche bêâ kunt.\\ 
 & \textbf{der} was von \textbf{rîterschefte} wunt\\ 
 & unt \textbf{het ez} \textbf{ouch} dâ vil \textbf{guot} getân.\\ 
20 & Gaschier der \textbf{Oriman}\\ 
 & \textbf{in dar brâhte}. er was kurtoys.\\ 
 & sîn vater was ein Franzoys.\\ 
 & \textbf{er} was Kayletes swester barn.\\ 
 & in wîbes dienste er was gevarn.\\ 
25 & er hiez Killirjacac.\\ 
 & \textbf{aller manne schœne er} widerwac.\\ 
 & \textbf{Dô} in Gahmuret \textbf{gesach},\\ 
 & ir antlütze sippe jach.\\ 
 & diu wâren ein ander vil gelîch.\\ 
30 & er bat die küneginne rîch\\ 
\end{tabular}
\scriptsize
\line(1,0){75} \newline
D Fr14 \newline
\line(1,0){75} \newline
\textbf{3} \textit{Initiale} D Fr14  \textbf{15} \textit{Majuskel} D  \textbf{27} \textit{Majuskel} D  \newline
\line(1,0){75} \newline
\textbf{1} Razalic] Razalik D Razalich Fr14 \textbf{3} Alsô] ÷lso D  $\cdot$ Gaschier] Gaschîer D Gascier Fr14 \textbf{4} Hiutegern] Hvtegern D hvͦtegern Fr14  $\cdot$ Schotten] Scotten D (Fr14) \textbf{8} sprach er] er sprach Fr14 \textbf{10} möht] moht D \textbf{11} gevangen] gevangen Fr14 \textbf{19} ez] \textit{om.} Fr14 \textbf{20} Gaschier] Gascier D Fr14  $\cdot$ Oriman] orman Fr14 \textbf{22} Franzoys] Franzôys D franzois Fr14 \textbf{23} Kayletes] kailets Fr14 \textbf{25} Killirjacac] killiriacach D killiriakak Fr14 \textbf{27} Gahmuret] Gahmvret D Gahm:ret Fr14  $\cdot$ gesach] er sach Fr14 \newline
\end{minipage}
\hspace{0.5cm}
\begin{minipage}[t]{0.5\linewidth}
\small
\begin{center}*m
\end{center}
\begin{tabular}{rl}
 & \textbf{er} \textbf{sprach}: "\textbf{mîn hêrre} Razali\textit{c},\\ 
 & \multicolumn{1}{l}{ - - - }\\ 
 & \textbf{gât nâher und} \textbf{küsset} mîn wîp.\\ 
 & \multicolumn{1}{l}{ - - - }\\ 
 & \textbf{alsô} tuot \textbf{ouch} ir, hêr Gaschier."\\ 
 & Hutegeren de\textit{n} \textbf{Schotten} fier\\ 
5 & bat er si küssen an \textbf{den} munt.\\ 
 & \textbf{der} was von sîner juste wunt.\\ 
 & er \textit{b}at si alle sitzen.\\ 
 & \textbf{al stênde} \textbf{sprach er} mit witzen:\\ 
 & "ich sæhe \textbf{ouch} gerne den neven mîn,\\ 
10 & m\textit{ö}h\textit{t} e\textit{z} mit sînen hulden sîn,\\ 
 & der in hie gevangen hât.\\ 
 & \textit{in}e hâ\textit{n}s \textbf{von} sippe keinen rât,\\ 
 & \textit{i}\textbf{\textit{n}e} muoze in ledic machen."\\ 
 & diu küniginne begunde lachen.\\ 
15 & si hiez \textbf{balde} \textbf{nâch ime springen}.\\ 
 & dort her begunde dringen\\ 
 & der minneclîch bêâ kunt.\\ 
 & \textbf{der} was von \textbf{ritterschafte} wunt\\ 
 & und \textbf{het ez} \textbf{ouch} d\textit{â} vil \textbf{guot} getân.\\ 
20 & Gaschier der \textbf{\textit{O}r\textit{i}man}\\ 
 & \textbf{in dar brâhte}. er was kurtois.\\ 
 & sîn vater was ein Franzois.\\ 
 & \textbf{er} was Kailetes swester barn.\\ 
 & in wîbes dienste er was gevarn.\\ 
25 & er hiez Kiliria\textit{c}ac.\\ 
 & \textbf{er aller manne schœne} widerwac.\\ 
 & \textbf{dô} in Gahmuret \textbf{gesach},\\ 
 & ir an\textit{t}litz sipp\textit{e} \textit{j}ach.\\ 
 & diu wâren ein ander vil gelîch.\\ 
30 & er bat \textit{die} küniginne rîch\\ 
\end{tabular}
\scriptsize
\line(1,0){75} \newline
m n o \newline
\line(1,0){75} \newline
\newline
\line(1,0){75} \newline
\textbf{1} Razalic] razalip m n o \textbf{3} ir] min n  $\cdot$ Gaschier] gascier m n gastaret o \textbf{4} Hutegeren] Huͯttegeren m Húttigeren n Huͯttigeren o  $\cdot$ den] [den]: der m \textbf{7} bat] hat m \textbf{8} Mit hoffelichen witzen n (o) \textbf{9} neven] nefe o \textbf{10} möht ez] Moch er \textit{nachträglich korrigiert zu:} Mochtes m Mocht es o  $\cdot$ sînen] siner o \textbf{12} ine] Me m Jch n o  $\cdot$ hâns] hands m han n o \textbf{13} ine] Me m Jch n o  $\cdot$ muoze] muͯsse n (o)  $\cdot$ ledic] leidig m n o \textbf{19} dâ] do m n o  $\cdot$ guot] \textit{om.} n o \textbf{20} Gaschier] Gascier m n o  $\cdot$ Oriman] ere man \textit{nachträglich korrigiert zu:} eren man m arme man n ariman o \textbf{22} Franzois] franczos \textit{nachträglich korrigiert zu:} franczoys m frantzois n franczois o \textbf{23} Kailetes] kailittes m kaẏlet n kalet o  $\cdot$ swester] swestern n \textbf{25} Kiliriacac] Kiliria tag m kili riatag n kẏli riatag o \textbf{26} er] Der n o \textbf{27} Gahmuret] gamiret \textit{nachträglich korrigiert zu:} gamuret m gamiret n gamuret o \textbf{28} Jr antzlicz sippe was vnd iach m  $\cdot$ antlitz] anczelit o \textbf{29} diu] Sú n (o) \textbf{30} die küniginne] kunigine \textit{nachträglich korrigiert zu:} die kunigine m \newline
\end{minipage}
\end{table}
\newpage
\begin{table}[ht]
\begin{minipage}[t]{0.5\linewidth}
\small
\begin{center}*G
\end{center}
\begin{tabular}{rl}
 & "\textbf{gêt} \textbf{her}, \textbf{mîn hêr} Razalic.\\ 
 & \multicolumn{1}{l}{ - - - }\\ 
 & \textbf{ir sult küssen} mîn wîp.\\ 
 & \multicolumn{1}{l}{ - - - }\\ 
 & \textbf{sam} tuot ir, \textbf{mîn} hêr Gatschier."\\ 
 & Hutegeren den \textbf{Schotten} fier\\ 
5 & bat er si küssen an \textbf{ir} munt.\\ 
 & \textbf{der} was von sîner tjoste wunt.\\ 
 & er bat si alle sitzen.\\ 
 & \textbf{alstênde}\textbf{r sprach} mit witzen:\\ 
 & "ich sæhe \textbf{ouch} gerne den neven mîn,\\ 
10 & m\textit{ö}ht ez mit sînen hulden sîn,\\ 
 & der in hie gevangen hât.\\ 
 & ichne hâns \textbf{vor} sippe neheinen rât,\\ 
 & ich\textbf{ne} muoze in ledic machen."\\ 
 & diu künigîn begunde lachen.\\ 
15 & si hiez \textbf{in} \textbf{balde} \textbf{bringen}.\\ 
 & dort her begunde dringen\\ 
 & der minniclîche bêâ kunt.\\ 
 & \textbf{der} was von \textbf{einer tjoste} wunt\\ 
 & unde \textbf{hetz} \textbf{ouch} dâ vil \textbf{guot} getân.\\ 
20 & Gatschier der \textbf{Norman}\\ 
 & \textbf{brâhtin}. er was kurtois.\\ 
 & sîn vater was ein Franzois\\ 
 & \textbf{unde} was Kailetes swester barn.\\ 
 & in wîbes dienster was gevarn.\\ 
25 & er hiez Kiliriakac.\\ 
 & \textbf{aller manne schœne er} widerwac.\\ 
 & \textbf{\begin{large}A\end{large}ls} in Gahmuret \textbf{ersach},\\ 
 & ir antlütze sippe jach.\\ 
 & diu wâren ein ander vil gelîch.\\ 
30 & er bat die küniginne rîch\\ 
\end{tabular}
\scriptsize
\line(1,0){75} \newline
G I O L M Q R Z Fr21 \newline
\line(1,0){75} \newline
\textbf{1} \textit{Initiale} O  \textbf{15} \textit{Initiale} M  \textbf{27} \textit{Initiale} G I L Q R Fr21  \newline
\line(1,0){75} \newline
\textbf{1} \textit{Die Verse 44.7-51.12 fehlen} Z   $\cdot$ gêt] ÷et O Git M  $\cdot$ her] naher O L Q (R) Fr21 nehir her M  $\cdot$ hêr] [R]: Her O  $\cdot$ Razalic] razalich G Razalip L kasalic M rasalip Q kazalich Fr21 \textbf{2} küssen] kuschin M \textbf{3} sam] Sam alse M Also R  $\cdot$ ir] ouch ir M [schir]: ir Q  $\cdot$ mîn hêr] her I (L) o\textit{m. } M  $\cdot$ Gatschier] chatschier I kathasir M gaschir Q \textbf{4} Hutegeren] huͤtgern I Hvͦteger O Huͯttegeren L Hutigern M Hutegern Q Hútte gern R Hvͦtigeren Fr21  $\cdot$ den] [ir]: den M der R  $\cdot$ Schotten] schoten G shotten I schotin M  $\cdot$ fier] schier Q \textbf{5} ir] den I Q orin M (R) \textbf{6} der] Do Q \textbf{8} alstênder sprach] Stende sprach er O Al steyne her [bat]: sprach M \textbf{9} sæhe] sehc O sih Fr21 sehe euch \textit{nachträglich korrigiert zu:} sehe auch Q \textbf{10} möht] moht G O (L) (M) (Q) (R) Fr21  $\cdot$ sînen] uwern M  $\cdot$ sîn] gesein Q \textbf{11} der] Swer O Fr21 Wer L M Q R \textbf{12} ichne] Jch L M Q  $\cdot$ hâns] hon Q (R)  $\cdot$ vor] von O Q  $\cdot$ neheinen] sein keinen Q \textbf{13} \textit{Versfolge 46.14-13} Fr21   $\cdot$ ichne] ichene G Ich I (O) (L) (M) (Q)  $\cdot$ muoze] mvͦz O (M) (Q) Fr21  $\cdot$ ledic] leidig M \textbf{14} diu] die I  $\cdot$ lachen] zu lachen Q \textbf{15} in balde bringen] balde nach im springen O L (M) (Q) (R) (Fr21) \textbf{17} minniclîche] mundigliche Q manliche Fr21  $\cdot$ bêâ kunt] gralkűnt Q \textbf{18} einer tjoste] riterschefte O (L) (M) (Q) (R) (Fr21) \textbf{19} hetz] hysz Q  $\cdot$ dâ vil] da I do vil O Q vil L M \textbf{20} Gatschier] Gatschiͤr I Gatischer M Gatschir Q Batschier R  $\cdot$ der] vnd L dyn M  $\cdot$ Norman] arme man M \textbf{21} brâhtin] Brachten in L  $\cdot$ er] die M  $\cdot$ kurtois] ein kurtoẏs Q \textbf{22} Franzois] franzoẏs G fronzoys I franzoys O R franczeis M frantzoysz Q franzeis Fr21 \textbf{23} Kailetes] chailetes G Gahiletes I kayletes O R Fr21 kaýletes L kayletes M haẏletes Q  $\cdot$ swester barn] barn I swester [svͦn]: barn O \textbf{24} wîbes] wibe I \textbf{25} Kiliriakac] killirriakach G chlariacac I kyliriacac O kallirakach L [bilia]: kiliriakac M killiriack Q kalliriakac R killiriakac Fr21 \textbf{26} aller] alle I  $\cdot$ manne schœne] schoͯne er der manen R  $\cdot$ er widerwac] enwider wac M \textbf{27} Als] Do O L M Q (R) Fr21  $\cdot$ Gahmuret] Gamvret O gahmvret L gamuret M gamúret Q Gahmoret Fr21 \textbf{28} antlütze] antlit zu R \textbf{29} ein] an Fr21 \textbf{30} küniginne] kúnginnen R \newline
\end{minipage}
\hspace{0.5cm}
\begin{minipage}[t]{0.5\linewidth}
\small
\begin{center}*T (U)
\end{center}
\begin{tabular}{rl}
 & "\textbf{gât}, \textbf{hêrre mîn}, Razalic,\\ 
 & tretet an der sælden stîc.\\ 
 & \textbf{ir sult küssen} mîn wîp,\\ 
 & diu mir ist als der lîp.\\ 
 & \textbf{sam} tuot ir, hêrre Gatschier."\\ 
 & Hutegern, den \textbf{degen} fier,\\ 
5 & bat er si küssen an \textbf{sînen} munt.\\ 
 & \textbf{er} was von sîner jost wunt.\\ 
 & er bat si alle sitzen.\\ 
 & \textbf{alle stunde} \textbf{er sprach} mit witzen:\\ 
 & "ich sæhe gerne den neven mîn,\\ 
10 & m\textit{ö}hte ez mit sînen hulden sîn,\\ 
 & der in hie gevangen hât.\\ 
 & ich enhâns \textbf{von} sippe dekeinen rât,\\ 
 & ich muoz in ledic machen."\\ 
 & diu künegîn begunde lachen.\\ 
15 & si hiez \textbf{balde} \textbf{nâch im springen}.\\ 
 & dort her begunde dringen\\ 
 & der minneclîche bêâ kunt.\\ 
 & \textbf{er} was von \textbf{ritterschaft} wunt\\ 
 & und \textbf{hât er} dâ vil \textbf{wol} getân.\\ 
20 & Gatschier der \textbf{Norman}\\ 
 & \textbf{brâht in}. er was kurtoys.\\ 
 & sîn vater was ein Franzoys.\\ 
 & \textbf{er} was Kayletes swester barn.\\ 
 & in wîbes dienste er was gevarn.\\ 
25 & er hiez Kylliriakac.\\ 
 & \textbf{aller manne schœne er} widerwac.\\ 
 & \textbf{\begin{large}D\end{large}ô} in Gahmuret \textbf{ersach},\\ 
 & ir antlütze sippe jach.\\ 
 & diu wâren ein ander vil gelîch.\\ 
30 & er bat die küneginne rîch\\ 
\end{tabular}
\scriptsize
\line(1,0){75} \newline
U V W T \newline
\line(1,0){75} \newline
\textbf{9} \textit{Majuskel} T  \textbf{14} \textit{Majuskel} T  \textbf{27} \textit{Initiale} U V W T  \textbf{28} \textit{Majuskel} T  \newline
\line(1,0){75} \newline
\textbf{1} Gant har min [*R*]: herre Razalig V Get her naher mein herre razzalick W er sprach her Razalic gêt her T \textbf{1} \textit{Vers 46.1\textasciicircum1 fehlt} T  \textbf{2} kvsset min wip daz ist min ger T  $\cdot$ ir sult küssen] Kússen solt ir W \textbf{2} \textit{Vers 46.2\textasciicircum1 fehlt} T  \textbf{3} tuot ir] tuͦ oͮch V  $\cdot$ Gatschier] Gathschier V Gatscier T \textbf{4} Hutegern] Huͦtegern U Hutigern V Hútiger W  $\cdot$ degen] [*]: schotten V schoten T \textbf{5} bat er si] [D*]: Den bat er sv́ V  $\cdot$ küssen] oͮch kússen V  $\cdot$ sînen] irn V den W ir T \textbf{6} er] der T  $\cdot$ wunt] worden wunt W \textbf{8} alle stunde] al stande V (W) vnde stânde T  $\cdot$ er sprach] sprach er W \textbf{9} sæhe] [sehen]: sehe U  $\cdot$ gerne] ôvch gerne T \textbf{10} möhte ez] Mochte ez U [mah ez]: mahtez T  $\cdot$ mit] wol mit W \textbf{12} enhâns] han es W (T)  $\cdot$ von] vor T \textbf{13} ich muoz] ine mvez T Ich muͤße W  $\cdot$ ledic] lidig V \textbf{15} balde] \textit{om.} T \textbf{16} begunde] begund er W \textbf{18} von riterscefte was er wunt T \textbf{19} hât er] hat es V (W) hatez T  $\cdot$ dâ] do W  $\cdot$ wol] guͦt W (T) \textbf{20} Gatschier] Gatscier T \textbf{22} Franzoys] franzoẏs V frantzoys W \textbf{23} er] vnd T  $\cdot$ Kayletes] kylites U kaẏletes V kyeletes W  $\cdot$ swester] sweter V \textbf{24} wîbes] vrôvwen T  $\cdot$ er was] was er W \textbf{25} er] vnd T  $\cdot$ Kylliriakac] kyliriakac U kẏlliarakag V kilriatag W \textbf{27} Gahmuret] Gahmuͦret U Gamuret V (W) \textbf{28} sippe] der sippe W \newline
\end{minipage}
\end{table}
\end{document}
