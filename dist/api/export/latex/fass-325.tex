\documentclass[8pt,a4paper,notitlepage]{article}
\usepackage{fullpage}
\usepackage{ulem}
\usepackage{xltxtra}
\usepackage{datetime}
\renewcommand{\dateseparator}{.}
\dmyyyydate
\usepackage{fancyhdr}
\usepackage{ifthen}
\pagestyle{fancy}
\fancyhf{}
\renewcommand{\headrulewidth}{0pt}
\fancyfoot[L]{\ifthenelse{\value{page}=1}{\today, \currenttime{} Uhr}{}}
\begin{document}
\begin{table}[ht]
\begin{minipage}[t]{0.5\linewidth}
\small
\begin{center}*D
\end{center}
\begin{tabular}{rl}
\textbf{325} & \begin{large}S\end{large}us schiet der wol gelobte man\\ 
 & von dem Plimizœls plân.\\ 
 & dô Kingrimursel wart genant,\\ 
 & ohteiz, dô wart er schiere erkant!\\ 
5 & werden, virrigen prîs\\ 
 & het an im der vürste wîs.\\ 
 & si jâhen, daz hêr Gawan\\ 
 & des kampfes \textbf{sorge müese} hân\\ 
 & \textbf{gein} sîner \textbf{wâren} manheit,\\ 
10 & des vürsten, der \textbf{dâ von in} reit.\\ 
 & ouch \textbf{wante} manegen trûrens nôt,\\ 
 & daz man im dâ niht \textbf{êre} bôt.\\ 
 & \textbf{dar wâren} solhiu mære komen,\\ 
 & als ir \textbf{wol ê} hât vernomen,\\ 
15 & die lîhte \textbf{erwanden} \textbf{einem} gast,\\ 
 & daz wirtes \textbf{gruozes} im gebrast.\\ 
 & \textbf{Von} Cundrien man ouch innen wart\\ 
 & \textbf{Parzivales} namen unt \textbf{sîner} art,\\ 
 & daz in gebar ein künegîn\\ 
20 & unt wie die erwarp \textbf{der} Anschevin.\\ 
 & Maneger sprach: "\textbf{vil} wol ich\textbf{z} weiz,\\ 
 & daz er si vor Kanvoleiz\\ 
 & gediende hurteclîche\\ 
 & mit manegem poynder rîche\\ 
25 & unt daz sîn ellen unverzagt\\ 
 & erwarp die \textbf{sældebæren} magt.\\ 
 & Ampflise, diu gehêrte,\\ 
 & ouch Gahmureten lêrte,\\ 
 & dâ von der helt wart kurtoys.\\ 
30 & nû sol \textbf{ein} ieslîch Bertenoys\\ 
\end{tabular}
\scriptsize
\line(1,0){75} \newline
D \newline
\line(1,0){75} \newline
\textbf{1} \textit{Initiale} D  \textbf{17} \textit{Majuskel} D  \textbf{21} \textit{Majuskel} D  \newline
\line(1,0){75} \newline
\textbf{2} Plimizœls] Primizols D \textbf{17} Cundrien] Cvndrîen D \textbf{20} Anschevin] Anscevin D \textbf{27} Ampflise] Amphlise D \textbf{28} Gahmureten] Gahmvreten D \textbf{30} Bertenoys] bertenoẏs D \newline
\end{minipage}
\hspace{0.5cm}
\begin{minipage}[t]{0.5\linewidth}
\small
\begin{center}*m
\end{center}
\begin{tabular}{rl}
 & sus schiet der wolgelobte man\\ 
 & von dem Plimizols plân.\\ 
 & \begin{large}D\end{large}ô K\textit{i}ngr\textit{im}u\textit{r}sel wart genant,\\ 
 & ohte\textit{iz}, dô wart er schiere erkant,\\ 
5 & \textbf{wand} werden, virrigen prîs\\ 
 & hete an ime der vürste wîs.\\ 
 & si jâh\textit{en}, daz hêr Gawan\\ 
 & des kampfes \textbf{sorge müese} hân\\ 
 & \textbf{gegen} sîner \textbf{wâren} manheit,\\ 
10 & des vürsten, der \textbf{d\textit{â} von in} reit.\\ 
 & ouch \textbf{mante} manigen trûrens nôt,\\ 
 & daz man ime d\textit{â} niht \textbf{êren} bôt.\\ 
 & \textbf{dô} \textbf{wâren da\textit{r}} \textit{s}oliche mær \textit{komen},\\ 
 & als ir \textbf{ê wol} habet vernomen,\\ 
15 & die lîhte \textbf{erwanten} \textbf{einen} gast,\\ 
 & d\textit{az} wirtes \textbf{gruozes} ime gebrast.\\ 
 & \textbf{\begin{large}V\end{large}on} Condrien man ouch \textit{i}nnen wart\\ 
 & \textbf{des Wâleises} namen \textit{u}nd \textbf{\textit{sîn}er} art,\\ 
 & daz in gebar ein künigîn\\ 
20 & und wie die erwarp \textbf{de\textit{r}} A\textit{n}schevin.\\ 
 & maniger sprach: "\textbf{wie} wol ich weiz,\\ 
 & daz er si vor Kanvoleiz\\ 
 & gediente h\textit{u}rteclîche\\ 
 & mit manigem poinder rîche\\ 
25 & und daz sîn ellen unverzagt\\ 
 & erwarp die \textbf{sældebæren} magt.\\ 
 & Ampflise, diu gehêrte,\\ 
 & ouch Gahmureten lêrte,\\ 
 & dâ von der helt wart kurt\textit{o}is.\\ 
30 & nû sol \textbf{sich} ieglîch Brit\textit{u}nois\\ 
\end{tabular}
\scriptsize
\line(1,0){75} \newline
m n o \newline
\line(1,0){75} \newline
\textbf{3} \textit{Initiale} m   $\cdot$ \textit{Capitulumzeichen} n  \textbf{17} \textit{Initiale} m n  \newline
\line(1,0){75} \newline
\textbf{1} wolgelobte] wol gelepte o \textbf{2} Plimizols] plamizols n planuͯcels o \textbf{3} Kingrimursel] kungrvnsel m kingrimúrsel n konungrimursel o \textbf{4} ohteiz] Ohtten m (o) Achten n \textbf{5} wand] \textit{om.} n  $\cdot$ virrigen] verigen n o \textbf{7} jâhen] ioh m  $\cdot$ Gawan] gewan o \textbf{8} des] Es o  $\cdot$ müese] muͯsse m \textbf{9} sîner] sinen n \textbf{10} dâ] do m n o \textbf{11} trûrens] truzens n [truczen]: truczens o \textbf{12} dâ] Do m n o \textbf{13} soliche mær komen] komen solliche mer m \textbf{14} habet] haben n \textbf{16} daz] Des m n (o)  $\cdot$ gruozes] grusse o \textbf{17} Condrien] kondrien m  $\cdot$ innen] minen m \textbf{18} des] Das n o  $\cdot$ Wâleises] waleis m n o  $\cdot$ und sîner] ander m \textbf{20} der] den m de n  $\cdot$ Anschevin] auscevin m n anscevin o \textbf{21} sprach] sproche n \textbf{22} vor] fuͯr o  $\cdot$ Kanvoleiz] kanvoleis m kanfoleis n o \textbf{23} hurteclîche] hertekliche m (n) (o) \textbf{28} Gahmureten] [gahn]: gahmuretten m gamureten n gamuͯreten o  $\cdot$ lêrte] lert o \textbf{29} kurtois] cortuis m \textbf{30} Britunois] britonois m britonis n britanisz o \newline
\end{minipage}
\end{table}
\newpage
\begin{table}[ht]
\begin{minipage}[t]{0.5\linewidth}
\small
\begin{center}*G
\end{center}
\begin{tabular}{rl}
 & sus schiet der wolgelobte man\\ 
 & von dem Blimzoles plân.\\ 
 & dô Kingrimursel wart genant,\\ 
 & ohteiz, dô wart er schiere erkant!\\ 
5 & werden, virrigen prîs\\ 
 & het an im der vürste wîs.\\ 
 & si jâhen, daz hêr Gawan\\ 
 & des kampfes \textbf{sorge m\textit{üe}se} hân\\ 
 & \textbf{von} sîner \textbf{werden} manheit,\\ 
10 & des vürsten, der \textbf{dâ von in} reit.\\ 
 & ouch \textbf{wante} manigen trûrens nôt,\\ 
 & daz man im dâ niht \textbf{êren} bôt.\\ 
 & \textbf{dâ wâren} solhiu mære komen,\\ 
 & als ir \textbf{ê} habet vernomen,\\ 
15 & die lîhte \textbf{erwanden} \textbf{einen} gast,\\ 
 & daz wirtes \textbf{gruozes} im gebrast.\\ 
 & \textbf{an} Gundrien man ouch innen wart\\ 
 & \textbf{Parzivales} namen unde \textbf{sîner} art,\\ 
 & daz in gebar ein künigîn\\ 
20 & unt wie die erwarp \textbf{der} Anschevin.\\ 
 & \textbf{vil} maniger sprach: "\textbf{\textit{w}i\textit{e}} wol ich weiz,\\ 
 & daz er si vor Kanvoleiz\\ 
 & gediente hurticlîche\\ 
 & mit maniger ponder rîche\\ 
25 & \begin{large}U\end{large}nde daz sîn ellen \textit{un}verzaget\\ 
 & erwarp die \textbf{sældebe\textit{r}nden} maget.\\ 
 & Anphlise, diu gehêrte,\\ 
 & ouch Gahmureten lêrte,\\ 
 & dâ von der helt wart kurtois.\\ 
30 & nû sol \textbf{\textit{s}i\textit{ch}} ieslîch Britanois\\ 
\end{tabular}
\scriptsize
\line(1,0){75} \newline
G I O L M Q R Z Fr40 \newline
\line(1,0){75} \newline
\textbf{11} \textit{Initiale} O L  \textbf{13} \textit{Initiale} I  \textbf{21} \textit{Initiale} Z  \textbf{25} \textit{Initiale} G  \textbf{27} \textit{Initiale} Fr40  \textbf{29} \textit{Initiale} I  \newline
\line(1,0){75} \newline
\textbf{1} wolgelobte] wol gemvͦte O (Z) [v*]: hoch gelopte R \textbf{2} dem] dein L  $\cdot$ Blimzoles] plimizol I Brimizols O plýmizolz L plimizcol M plimzols Q plumzols R plimizols Z Fr40 \textbf{3} dô] Da Z  $\cdot$ Kingrimursel] kyngrimvrsel O (Fr40) kýngrimuͯrsel L kyngrymuͯrsel M kingrun múrsel Q kúngrimursel R \textbf{4} ohteiz] Got weiz O Och des Q Ocheis R  $\cdot$ dô] dv O da M Z \textbf{5} werden] wer den G werder Fr40  $\cdot$ virrigen] vrrigen Q wirrigen Fr40 \textbf{6} het an im] An im her L Hat an Jm R (Z)  $\cdot$ vürste] fvͦrte O \textbf{7} jâhen] sahin M  $\cdot$ hêr] er Q Z \textbf{8} sorge] sorgen M  $\cdot$ müese] moͮse G (M) (Q) (Z) (Fr40) \textbf{9} von] Gein Z  $\cdot$ werden] waren O L (M) (Q) (R) Z Fr40  $\cdot$ manheit] warheit Fr40 \textbf{10} dâ] do Q R  $\cdot$ in] im I O (M) Q Fr40 \textbf{11} ouch] ÷vch O  $\cdot$ manigen] manec I (L) (M) manche Q \textbf{12} dâ] do Q \textbf{13} dâ] Do I \textbf{14} habet] wol habt Z \textbf{15} die] Du R  $\cdot$ erwanden] erwande I  $\cdot$ einen] eren Z \textbf{16} daz] wan daz I Daz des O (Q) (R) o\textit{m. } L M  $\cdot$ wirtes] winters Q  $\cdot$ gruozes] gruͤz I (R) [grvͦz]: grvͦzes  O \textbf{17} an Gundrien] Angundrien I An Gvndrýen L An kondrien M Von kundrien Q R Z (Fr40)  $\cdot$ man ouch] man O ouch man M (Q) Fr40 \textbf{18} Parzivales] [parzifals]: Parzifals I Barcifals O Parcifals L Z Parzifals M (Fr40) Partzifals Q Parczifals R  $\cdot$ sîner] sein Q (R) \textbf{20} die] \textit{om.} O  $\cdot$ Anschevin] ensheuin I ansehvin O anshewin L ansevin M ansheuin Q Anshefin R anshevin Z anschwein Fr40 \textbf{21} vil maniger] Manger O L (M) (Q) (R) (Z) (Fr40)  $\cdot$ wie] vil G  $\cdot$ ich] ich z I (L) (M) (R) \textbf{22} daz] Dar O  $\cdot$ si] sey Q (R)  $\cdot$ vor] von L M Q  $\cdot$ Kanvoleiz] kanpholeiz I convaleiz O Camvoleiz L kanvoleis M kanfoleys Q kanuoleis R kamfoleiz Z \textbf{24} maniger] mangem L (M) (Z) manchen Q (Fr40)  $\cdot$ ponder] paniere R  $\cdot$ rîche] richer O (R) \textbf{25} ellen] eren Q  $\cdot$ unverzaget] verzaget G \textbf{26} sældebernden] sældebenden G seldebarn I saldenbernden L seldenbernde Q selden beren R seldeberen Z (Fr40) \textbf{27} Anphlise] anflise G (R) (Z) (Fr40) Anfise I Ampflize O Amflise L Anfilise M An fleysz Q  $\cdot$ diu] di Fr40  $\cdot$ gehêrte] gehúrre geherte R \textbf{28} Gahmureten] gahmvreten G (L) Gamvreten O gamuren M gaműreten Q Gahmuretten R gamureten Z \textbf{29} der helt wart] ward der held R \textbf{30} nû] Vnd Z  $\cdot$ sich] ein G  $\cdot$ ieslîch] iesglicher I  $\cdot$ Britanois] britoynois G pritonois I Britonys O Brittanois L britaneis M britoneis Q brytonis R britunois Z briteneis Fr40 \newline
\end{minipage}
\hspace{0.5cm}
\begin{minipage}[t]{0.5\linewidth}
\small
\begin{center}*T
\end{center}
\begin{tabular}{rl}
 & sus schiet der wol gelobete man\\ 
 & von dem Plymizols plân.\\ 
 & \begin{large}D\end{large}ô Kyngrimursel wart genant,\\ 
 & otheiz, dô wart er schiere erkant!\\ 
5 & werden, virrigen prîs\\ 
 & het an im der vürste wîs.\\ 
 & si jâhen, daz hêr Gawan\\ 
 & des kampfes \textbf{müese sorge} hân\\ 
 & \textbf{von} sîner \textbf{grôzen} manheit,\\ 
10 & des vürsten, der \textbf{von in dâ} reit.\\ 
 & ouch \textbf{wante} manegen trûrens nôt,\\ 
 & daz man im dâ niht \textbf{êre} bôt.\\ 
 & \textbf{dar wâren} solhe mære komen,\\ 
 & als ir \textbf{ê} hât vernomen,\\ 
15 & die lîhte \textbf{wanten} \textbf{einen} gast,\\ 
 & daz wirtes \textbf{gruoz} im gebrast.\\ 
 & \textbf{An} Kundrien man ouch innen wart\\ 
 & \textbf{Parcifales} namen unde \textbf{sînen} art,\\ 
 & daz in gebar ein künegîn\\ 
20 & unde wie di\textit{e} erwarp \textbf{ein} \textit{Ansch}e\textit{v}in.\\ 
 & \textbf{Vil} maneger sprach: "\textbf{wie} wol ich weiz,\\ 
 & daz er si vo\textit{r} Kanvoleiz\\ 
 & gediende hurteclîche\\ 
 & mit manegem poynder rîche\\ 
25 & unde daz sîn ellen unverzaget\\ 
 & erwarp die \textbf{sældebæren} maget.\\ 
 & Anflise, diu gehêrte,\\ 
 & ouch Gahmureten lêrte,\\ 
 & dâ von der helt wart kurteis.\\ 
30 & nû sol \textbf{sich} ieslîch Brituneis\\ 
\end{tabular}
\scriptsize
\line(1,0){75} \newline
T U V W \newline
\line(1,0){75} \newline
\textbf{3} \textit{Initiale} T U W  \textbf{17} \textit{Majuskel} T  \textbf{21} \textit{Majuskel} T  \newline
\line(1,0){75} \newline
\textbf{1} schiet] kam W  $\cdot$ wol gelobete] wol [gelo*]: gelobete T \textbf{2} dem] des W  $\cdot$ Plymizols] plimizols V W \textbf{3} Kyngrimursel] kuͦngrimors U kẏngrimursel V kingrimursel W  $\cdot$ wart] was U W [*]: wart V  $\cdot$ genant] gemant U \textbf{4} otheiz] Gotteweis V Ehteis W  $\cdot$ erkant] gemant U \textbf{5} werden virrigen] Wande [w*]: werden wirigen V \textbf{6} vürste] vursten U \textbf{8} müese sorge] sorgen muͦzen U sorge mvͤste V sorge muͦße W \textbf{9} grôzen] grozer U \textbf{10} der] do er U W [*]: der do V  $\cdot$ dâ] do U V W \textbf{12} dâ niht êre] do nit eren U (V) do ere W \textbf{15} lîhte wanten] [*]: lihte [er*]: erwante V \textbf{16} daz] Des U W [D*]: Daz V  $\cdot$ gruoz] gruͤssens W  $\cdot$ im] an im V \textbf{17} Kundrien] kvndrîen T kuͦndrien U kundrie W  $\cdot$ man ouch] \textit{om.} U oͮch men V \textbf{18} Parcifales] PArzifales T (V) Parcifals U Partzifals W  $\cdot$ sînen] sin U siner V (W) \textbf{20} die] div T  $\cdot$ ein Anschevin] ein heidenin T ein Anschevin U [*]: der anschevin V ein antscheuin W \textbf{21} Vil] Wie W  $\cdot$ wie] \textit{om.} W \textbf{22} vor] von T  $\cdot$ Kanvoleiz] kanvoleis V kanuoleis W \textbf{23} hurteclîche] hertecliche U \textbf{26} sældebæren] [seldenb*]: seldenbere V selben bernden W \textbf{27} Anflise] An flize U Anflize V Vor kanuoleis W \textbf{28} Gahmureten] Gahmuͦreten U Gamvreten V gamureten W \textbf{30} ieslîch] iechich U  $\cdot$ Brituneis] Brituͦneys U [brit*]: britunoẏs V britunis W \newline
\end{minipage}
\end{table}
\end{document}
