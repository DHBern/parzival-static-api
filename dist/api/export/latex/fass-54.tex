\documentclass[8pt,a4paper,notitlepage]{article}
\usepackage{fullpage}
\usepackage{ulem}
\usepackage{xltxtra}
\usepackage{datetime}
\renewcommand{\dateseparator}{.}
\dmyyyydate
\usepackage{fancyhdr}
\usepackage{ifthen}
\pagestyle{fancy}
\fancyhf{}
\renewcommand{\headrulewidth}{0pt}
\fancyfoot[L]{\ifthenelse{\value{page}=1}{\today, \currenttime{} Uhr}{}}
\begin{document}
\begin{table}[ht]
\begin{minipage}[t]{0.5\linewidth}
\small
\begin{center}*D
\end{center}
\begin{tabular}{rl}
\textbf{54} & swaz der \textbf{gelten} mohte ein jâr,\\ 
 & den selben liezen si dâ gar.\\ 
 & daz tâten si umb ir \textbf{selber} muot.\\ 
 & Gahmuret daz grôze guot\\ 
5 & sîn volc hiez behalden.\\ 
 & die muosen\textbf{s} sunder walden.\\ 
 & smorgens vor der veste\\ 
 & rû\textit{m}den\textit{z} \textbf{gar} die geste.\\ 
 & sich schieden, die dâ wâren\\ 
10 & und vuorten manege bâren.\\ 
 & \textbf{daz velt} \textbf{herberge stuont} \textbf{al} blôz\\ 
 & wan ein gezelt, daz was vil grôz.\\ 
 & daz hiez de\textit{r} künec ze schiffe tragen.\\ 
 & \textbf{dô begunder dem volke} sagen,\\ 
15 & er wolde\textbf{z vüeren} \textbf{in} Azagouc.\\ 
 & mit der rede er si betrouc.\\ 
 & \textbf{dâ} was der \textbf{stolze, küene} man,\\ 
 & unz er sich \textbf{vaste} \textbf{senen} began.\\ 
 & daz er niht rîterschefte vant,\\ 
20 & \textbf{des} was sîn vreude sorgen pfant.\\ 
 & \textbf{\textit{\begin{large}D\end{large}}och} was im daz swarze wîp\\ 
 & \textbf{noch} lieber denne sîn selbes lîp.\\ 
 & ez \textbf{en}wart nie wîp geschicket baz.\\ 
 & der vrouwen herze \textbf{nie} vergaz,\\ 
25 & \textbf{im} envüere \textbf{ein} \textbf{werdiu} \textbf{volge} mite,\\ 
 & an \textbf{rehter kiusche} wîplîch site.\\ 
 & \textbf{Von} Sibilje \textbf{ûz}er stat\\ 
 & was geborn, den er dô bat\\ 
 & \textbf{dannen kêrens} \textbf{zeiner} wîle.\\ 
30 & \textbf{der} hete in manege mîle\\ 
\end{tabular}
\scriptsize
\line(1,0){75} \newline
D \newline
\line(1,0){75} \newline
\textbf{21} \textit{Initiale} D  \textbf{27} \textit{Majuskel} D  \newline
\line(1,0){75} \newline
\textbf{8} rûmdenz] rvmendens D \textbf{13} der] de D \textbf{15} Azagouc] Azagovch D \textbf{21} Doch] ÷och D \textbf{27} Sibilje] Sybilie D \newline
\end{minipage}
\hspace{0.5cm}
\begin{minipage}[t]{0.5\linewidth}
\small
\begin{center}*m
\end{center}
\begin{tabular}{rl}
 & waz der \textbf{gelte\textit{n}} m\textit{o}hte ein jâr,\\ 
 & den selben liezen si dâ gar.\\ 
 & daz \textit{tâten si} umb ir \textbf{selber} muot.\\ 
 & Gahmuret daz grôze guot\\ 
5 & sîn volc \dag muos\dag  behalten.\\ 
 & die muosen sunder walten.\\ 
 & \begin{large}D\end{large}es morgens vor der veste\\ 
 & rûm\textit{t}e\textit{n} ez \textbf{gar} die geste.\\ 
 & sich schieden, die dâ wâren\\ 
10 & und vuorten manige bâren.\\ 
 & \textbf{daz volc} \textbf{stuont herberge} blôz\\ 
 & wanne ein gezelt, daz was vil grôz.\\ 
 & daz hiez der künic ze schiffe tragen.\\ 
 & \textbf{dô begunde er dem volke} \textit{sag}en,\\ 
15 & er wolte \textbf{varen} \textbf{in} Azagouc.\\ 
 & mit der rede er si betr\textit{ou}c.\\ 
 & \textbf{daz} was der \textbf{stolze, küene} man,\\ 
 & unz er sich \textbf{vaste} \textbf{gesetzen} began.\\ 
 & daz er niht ritterschefte vant,\\ 
20 & \textbf{daz} was sîn vröude sorgen pfant.\\ 
 & \textbf{doch} was ime daz swarze wîp\\ 
 & lieber danne sîn selbes lîp.\\ 
 & ez \textbf{en}wart nie wîp geschicket baz.\\ 
 & der vrowen herze \textbf{nie} vergaz,\\ 
25 & \textbf{im} envüere \textbf{ein} \textbf{werder} \textbf{vogel} mite,\\ 
 & an \textbf{reh\textit{t}er kiusche} wîplîcher site.\\ 
 & \textbf{\begin{large}V\end{large}on} Sibilie \textbf{ûz} der stat\\ 
 & was geborn, den er dô bat\\ 
 & \textbf{dannen kêren\textit{s}} \textbf{ze einer} wîle.\\ 
30 & \textbf{der} hete in manige mîle\\ 
\end{tabular}
\scriptsize
\line(1,0){75} \newline
m n o \newline
\line(1,0){75} \newline
\textbf{7} \textit{Initiale} m  \textbf{27} \textit{Initiale} m   $\cdot$ \textit{Capitulumzeichen} n  \newline
\line(1,0){75} \newline
\textbf{1} gelten] geltes \textit{nachträglich korrigiert zu:} gelten m  $\cdot$ mohte] moͯchte m n \textbf{2} dâ] do n o \textbf{3} tâten si] \textit{om.} m  $\cdot$ selber] selbes n o \textbf{4} Gahmuret] Gamiret n Gamuͯret o \textbf{6} muosen] muͯssent n o \textbf{7} vor] von n o \textbf{8} rûmten] Rument m \textbf{9} die dâ] die do n do die o \textbf{12} vil] \textit{om.} m n \textbf{14} sagen] iehen \textit{nachträglich korrigiert zu:} Sagen m \textbf{15} varen] fuͯren n o  $\cdot$ Azagouc] azagovg m azagoug n azaguͯng o \textbf{16} betrouc] betruog \textit{nachträglich korrigiert zu:} betroͧg m \textbf{18} gesetzen] setzen n siczen o \textbf{20} sîn] siner n o  $\cdot$ vröude] freiden o \textbf{21} doch] Das o  $\cdot$ daz] das das n \textbf{22} selbes lîp] eigenlip n \textbf{23} enwart] wart n o \textbf{25} envüere] fuͯre n o \textbf{26} rehter] recher m \textbf{29} kêrens] kerentz m o \newline
\end{minipage}
\end{table}
\newpage
\begin{table}[ht]
\begin{minipage}[t]{0.5\linewidth}
\small
\begin{center}*G
\end{center}
\begin{tabular}{rl}
 & swaz der \textbf{vergelten} moht ein jâr,\\ 
 & den selben liezen si \textbf{im} dâ gar.\\ 
 & daz tâten si umbe ir \textbf{selber} muot.\\ 
 & Gahmuret daz grôze guot\\ 
5 & sîn volc hiez behalden.\\ 
 & die muosen\textbf{s} sunder walden.\\ 
 & des morgens vor der veste\\ 
 & \textbf{dô} rûmdenz \textbf{dâ} die geste.\\ 
 & sich schieden, die dâ wâren\\ 
10 & unde vuorten manige bâren.\\ 
 & \textbf{daz velt} \textbf{herberge wart} \textbf{al} blôz\\ 
 & wan ein gezelt, daz was vil grôz.\\ 
 & daz hiez der künic ze schiffe tragen.\\ 
 & \textbf{\begin{large}S\end{large}înem volc er dô begunde} sagen,\\ 
15 & er wolte\textbf{z vüeren} \textbf{ze} Azagouc.\\ 
 & mit der rede er si betrouc.\\ 
 & \textbf{dô} was \textbf{al dâ} der \textbf{küene} man,\\ 
 & unzer sich \textbf{sêre} \textbf{senen} began.\\ 
 & daz er niht rîterschefte vant,\\ 
20 & \textbf{des} was sîn vröude sorgen pfant.\\ 
 & \textbf{doch} was im daz swarze wîp\\ 
 & lieber dane sîn selbes lîp.\\ 
 & ez wart nie wîp geschicket baz.\\ 
 & der vrouwen herze \textbf{niht} vergaz,\\ 
25 & \textbf{ir} envüere \textbf{rehtiu} \textbf{mâze} mite,\\ 
 & an \textbf{reiner zühte} wîplîch site.\\ 
 & \textbf{ze} Sibilie \textbf{ûz} der stat\\ 
 & was geboren, den er dâ bat\\ 
 & \textbf{dane kêrenes} \textbf{eine} wîle.\\ 
30 & \textbf{er} het in manige mîle\\ 
\end{tabular}
\scriptsize
\line(1,0){75} \newline
G I O L M Q R Z Fr21 Fr37 \newline
\line(1,0){75} \newline
\textbf{3} \textit{Initiale} M  \textbf{14} \textit{Initiale} G  \textbf{19} \textit{Initiale} I  \textbf{27} \textit{Überschrift:} Hie zucht Gahmuret úber mer Von siner frowen die was schwanger worden vnd wie im ward der adamas en schwert zwo hossen R  Wie gamuret von der moͤrin schiet Z  Aventever wie Gahmuret von Belakanen schiet Fr37   $\cdot$ \textit{Initiale} L Q R Z Fr21 Fr37  \newline
\line(1,0){75} \newline
\textbf{1} \textit{Die Verse 48.21-54.6 fehlen} R   $\cdot$ swaz] Waz L (M) (Q)  $\cdot$ vergelten] gelde M \textbf{2} liezen si] suessen \textit{nachträglich korrigiert zu:} liesszen Q  $\cdot$ im] on M \textit{om.} Z  $\cdot$ dâ] do L Q \textbf{3} umbe] durch I myt M  $\cdot$ ir] orm M  $\cdot$ selber] selbe L selbis M (Z) selben Q \textbf{4} Gahmuret] Gamvret O Gahmuͯret L Gamuret M Q Z \textbf{6} die] Sie L Des Q  $\cdot$ muosens] muͤstens I musten M munstens Z  $\cdot$ sunder] selber Q \textbf{7} veste] festin M \textbf{8} dô rûmdenz] Rovmptenz O (L) (Q) (Z) Rumten M (R)  $\cdot$ dâ] \textit{om.} I do Q da gar Z \textbf{9} dâ] do Q \textbf{10} vuorten] furte Q  $\cdot$ manige] manchen Q (R) (Z) \textbf{11} herberge wart] stvͦnt herberge O (L) (Fr21) herberge stunt M Q (Z) herbergen stuͦnd R  $\cdot$ al] \textit{om.} L \textbf{12} vil] \textit{om.} L R Z \textbf{13} hiez der künic] der kvnich hiez L  $\cdot$ ze schiffe] zeseffe I \textbf{14} sînem] Dem O L (M) Q R Z Fr21  $\cdot$ er] \textit{om.} M  $\cdot$ dô] \textit{om.} I L Z da M R  $\cdot$ sagen] zu sagen Q \textbf{15} woltez] wolde Fr21  $\cdot$ vüeren] \textit{om.} O  $\cdot$ ze] in O L M Q R Z Fr21  $\cdot$ Azagouc] azagoͮc G I azagovch O Azagoͮch L azagoyc M azagouck Q azagovc Z Fr21 \textbf{16} \textit{Vers 54.16 fehlt} Q   $\cdot$ der rede] den redin M \textbf{17} dô] Daz L (R) Da M Z Fr21  $\cdot$ al dâ] \textit{om.} O L M Q R Z Fr21  $\cdot$ küene] chvͦne stolze O (L) (M) (R) (Z) stoltze kúne Q kuͦne stoze Fr21 \textbf{18} unzer] Vnde her M  $\cdot$ sêre senen] vaste senen O L (M) (R) Z (Fr21) senen fast Q \textbf{20} sîn] \textit{om.} Q  $\cdot$ sorgen] worden I sorge Q \textbf{21} doch] Jedoch O L (M) (Q) (R) Z (Fr21)  $\cdot$ swarze] \textit{om.} Q \textbf{22} selbes] eigen M \textbf{23} ez] ezn I (L) (M) (Q) (R) (Z) (Fr37)  $\cdot$ wîp] leip Fr37 \textbf{24} niht] nie I O (R) Z \textbf{25} Jm (Jmen Q ) fvͤr (fuͦr R ) ein werdev (rechte L werde R ) volge mit O (L) (M) (Q) (R)  $\cdot$ envüere] f:or Fr37  $\cdot$ rehtiu mâze] rehtiuͤ vuͤr I ein werdiv uolge Fr37 \textbf{26} reiner zühte] rehter chevsche O (L) (M) (Q) (R) (Z) (Fr21) (Fr37)  $\cdot$ wîplîch] wiplicher I (L) (M) \textbf{27} ze] Von L Q (Z) ÷e Fr37  $\cdot$ Sibilie] sibille I (L) Fr37 sibilen Q Sẏbile R sẏbilie Fr21 \textbf{28} den] der I  $\cdot$ dâ] \textit{om.} O M R Fr21 Fr37 do Q \textbf{29} kêrenes] kerten si I (L) keren Z Fr21  $\cdot$ eine] ze einer I (O) (L) (M) (Q) R (Z) (Fr21) \textbf{30} het] hat O (M) (Q) (R)  $\cdot$ in manige] im manger I vil menge R \newline
\end{minipage}
\hspace{0.5cm}
\begin{minipage}[t]{0.5\linewidth}
\small
\begin{center}*T (U)
\end{center}
\begin{tabular}{rl}
 & waz der \textbf{vergelten} mohte ein jâr,\\ 
 & den selben liezen si dâ gar.\\ 
 & daz tâten si umb ir \textbf{selbes} muot.\\ 
 & Gahmuret daz grôze guot\\ 
5 & sîn volc hiez \textbf{er} behalten.\\ 
 & die muosen\textbf{s} sunder walten.\\ 
 & des morgens vor der vest\textit{e}\\ 
 & rûmden\textit{z} \textbf{dâ} die geste.\\ 
 & sich schieden, die dâ wâren\\ 
10 & und vuorten manigen bâren.\\ 
 & \textbf{diu} \textbf{herberge stuont} \textbf{dâ} blôz\\ 
 & wan ein gezelt, daz was vil grôz.\\ 
 & daz hiez der künec zuo schiffe tragen.\\ 
 & \textbf{dem volke er dô begunde} sagen,\\ 
15 & er wolt \textbf{ez vüeren} \textbf{in} Azagouc.\\ 
 & mit der rede er si betrouc.\\ 
 & \textbf{dô} was der \textbf{stolze, küene} man,\\ 
 & un\textit{z} er sich \textbf{vaste} \textbf{senen} began.\\ 
 & daz er niht ritterschefte vant,\\ 
20 & \textbf{des} was sîn vreude \textbf{der} sorgen pfant.\\ 
 & \textbf{iedoch} was im daz swarze wîp\\ 
 & lieber dan sîn selbes lîp.\\ 
 & ez \textbf{en}wart nie wîp geschicket baz.\\ 
 & der vrouwen herze \textbf{niht} vergaz,\\ 
25 & \textbf{im} envüere \textbf{ein} \textbf{werde} \textbf{volg\textit{e}} mite,\\ 
 & an \textbf{rehter} \textbf{kiuscheheite} \textbf{ein} wîplîch \textit{s}i\textit{te}.\\ 
 & \textbf{\begin{large}V\end{large}on} Sybilie der stat\\ 
 & was geborn, den er d\textit{â} bat\\ 
 & \textbf{dankêr} \textbf{zuo einer} wîle.\\ 
30 & \textbf{er} hete in manige mîle\\ 
\end{tabular}
\scriptsize
\line(1,0){75} \newline
U V W T \newline
\line(1,0){75} \newline
\textbf{1} \textit{Majuskel} T  \textbf{13} \textit{Majuskel} T  \textbf{27} \textit{Überschrift:} Hie floch gamuret von zazamang vnd kam auf den turney gen kanuoleiß. in dem lande zuͦ valeiß erwarb er die kúnigin W   $\cdot$ \textit{Platz für Illustration ausgespart} W   $\cdot$ \textit{Initiale} U V W T  \newline
\line(1,0){75} \newline
\textbf{1} waz] Swas V (T)  $\cdot$ mohte] moͤhte V \textbf{2} liezen si] liessen [si]: sv́ im V liezens im T  $\cdot$ dâ] [*]: do V do W \textbf{3} selbes] selber T \textbf{4} Gahmuret] Gahmuͦret U Gamuret V W \textbf{5} sîn] Seinem W  $\cdot$ hiez er] [hie*]: hies V hieß W (T) \textbf{6} muosens sunder] [muͤsten*]: muͤstent es selber V  $\cdot$ die] Sy W \textbf{7} veste] vesten U \textbf{8} rûmdenz] Ruͦmeden U [rumente*]: rumentent V  $\cdot$ dâ] do V \textit{om.} W \textbf{9} dâ] do W \textbf{10} vuorten] fuͦrte W  $\cdot$ manigen] [*]: uf den V mange W (T) \textbf{11} [*]: Siechen daz daz velt herbergen stuͦnt do blos V  $\cdot$ daz velt stvnt herbergen blôz T  $\cdot$ dâ] all W \textbf{12} vil] \textit{om.} T \textbf{14} er dô begunde] begvnder T \textbf{15} Azagouc] Azaguc U azagoͮg V Azagôvc T \textbf{16} do mitte er daz volg betroͮg V \textbf{17} dô] Das W da T  $\cdot$ stolze küene] kvene stolze T \textbf{18} unz] Vnd U vntze daz V  $\cdot$ vaste] \textit{om.} V sêre T \textbf{20} was] stvͦnt T  $\cdot$ der sorgen] gar sein W sorgen T \textbf{23} ez enwart] Auch ward W  $\cdot$ geschicket] gepriset T \textbf{24} niht] [*]: nie V nie W \textbf{25} im] Ir W  $\cdot$ envüere] fuͦre W (T)  $\cdot$ volge] volgen U \textbf{26} kiuscheheite] kúsche V (W) (T)  $\cdot$ ein] \textit{om.} T  $\cdot$ site] wip U \textbf{27} Von] Ze T  $\cdot$ Sybilie] Sẏbilie V sibillie W  $\cdot$ der] auß der W (T) \textbf{28} dâ] do U V W \textbf{29} dankêr] Dannan keren V (W)  $\cdot$ wîle] wilen W \textbf{30} hete] hat W  $\cdot$ manige mîle] manger milen W \newline
\end{minipage}
\end{table}
\end{document}
