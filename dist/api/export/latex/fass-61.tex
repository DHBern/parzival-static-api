\documentclass[8pt,a4paper,notitlepage]{article}
\usepackage{fullpage}
\usepackage{ulem}
\usepackage{xltxtra}
\usepackage{datetime}
\renewcommand{\dateseparator}{.}
\dmyyyydate
\usepackage{fancyhdr}
\usepackage{ifthen}
\pagestyle{fancy}
\fancyhf{}
\renewcommand{\headrulewidth}{0pt}
\fancyfoot[L]{\ifthenelse{\value{page}=1}{\today, \currenttime{} Uhr}{}}
\begin{document}
\begin{table}[ht]
\begin{minipage}[t]{0.5\linewidth}
\small
\begin{center}*D
\end{center}
\begin{tabular}{rl}
\textbf{61} & tet ez ûf, als im ze muote was.\\ 
 & dâr \textbf{ob} stuont der palas.\\ 
 & \textbf{ouch} \textbf{saz} diu küneginne\\ 
 & ze\textbf{n venstern} dâr inne\\ 
5 & mit maneger \textbf{werden vrouwen}.\\ 
 & die \textbf{begunden} \textbf{dâ} schouwen,\\ 
 & \textbf{waz} dise knappen tâten.\\ 
 & \textbf{die} heten sich berâten\\ 
 & \textbf{unt} sluogen ûf \textbf{ein} gezelt.\\ 
10 & umb unvergolten minnen gelt\\ 
 & wart es ein künec âne.\\ 
 & des twang in Belakane.\\ 
 & mit arbeit \textbf{wart} ûf geslagen,\\ 
 & daz \textbf{drîzec} soumære \textbf{muosen} tragen,\\ 
15 & ein gezelt. daz \textbf{zeigete} rîcheit.\\ 
 & \textbf{ouch} was der \textbf{plân} wol \textbf{sô} breit,\\ 
 & daz sich die snüere stracten dran.\\ 
 & Gahmuret, der werde man,\\ 
 & \textbf{die selben zît} dort ûze enbeiz.\\ 
20 & dar nâch er sich mit vlîze vleiz,\\ 
 & wie er \textbf{höfschlîche} kœme geriten.\\ 
 & \textbf{des} \textbf{en}wart \textbf{niht langer dô} gebiten.\\ 
 & sîne k\textit{n}appen an den stunden\\ 
 & sîniu sper zesamne bunden,\\ 
25 & ieslîcher vünfiu an ein bant.\\ 
 & daz sehste vuort\textbf{er an der hant}\\ 
 & \begin{large}M\end{large}it einer baniere.\\ 
 & sus kom gevarn der fiere.\\ 
 & \textbf{von} der künegîn wart vernomen,\\ 
30 & \textbf{daz ein gast dâ} \textbf{solte} komen\\ 
\end{tabular}
\scriptsize
\line(1,0){75} \newline
D \newline
\line(1,0){75} \newline
\textbf{27} \textit{Initiale} D  \newline
\line(1,0){75} \newline
\textbf{12} Belakane] Belachane D \textbf{18} Gahmuret] Gahmvret D \textbf{23} knappen] chappen D \newline
\end{minipage}
\hspace{0.5cm}
\begin{minipage}[t]{0.5\linewidth}
\small
\begin{center}*m
\end{center}
\begin{tabular}{rl}
 & tet ez ûf, als ime ze muote was.\\ 
 & dâr \textbf{obe} stuont der palas.\\ 
 & \textbf{ouch} \textbf{saz} diu küniginne\\ 
 & ze\textbf{n venstern} dâr inne\\ 
5 & mit maniger \textbf{werden vrouwen}.\\ 
 & die \textbf{begunden} schouwen,\\ 
 & \textbf{waz} dise knappen tâten.\\ 
 & \textbf{die} heten sich berâten\\ 
 & \textbf{und} sluogen ûf \textbf{ein} gezelt.\\ 
10 & umb unvergolten minnen gelt\\ 
 & wart es ein künic âne.\\ 
 & des twanc in Belakane.\\ 
 & mit arbeit \textbf{was} ûf geslagen,\\ 
 & da\textit{z} \textit{s}ou\textit{m}ære \textbf{solten} tragen,\\ 
15 & ein gezelt. daz \textbf{zeigete} rîcheit.\\ 
 & \textbf{ouch} was der \textbf{plân} \textbf{dâ} wol \textbf{sô} breit,\\ 
 & daz sich die snüere strahten dran.\\ 
 & Gahmuret, der werde man,\\ 
 & \textbf{die selben zît} dort ûze enbeiz.\\ 
20 & dar nâch er sich mit vlîze \textit{v}leiz,\\ 
 & wie er \textbf{hövelîch} k\textit{œ}me geriten.\\ 
 & \textbf{des} \textbf{en}wart \textbf{niht langer dô} gebiten.\\ 
 & sîne knappen an den stunden\\ 
 & sîniu sper z\textit{e}samen bunden,\\ 
25 & ieglîcher vünfiu an ein bant.\\ 
 & daz sehste vuort \textbf{er an der hant}\\ 
 & mit einer baniere.\\ 
 & sus kam gevarn der fiere.\\ 
 & \textbf{vor} de\textit{r} künigîn wart vernomen,\\ 
30 & \textbf{daz ein gast d\textit{â}} \textbf{solte} komen\\ 
\end{tabular}
\scriptsize
\line(1,0){75} \newline
m n o \newline
\line(1,0){75} \newline
\newline
\line(1,0){75} \newline
\textbf{6} begunden] begúnde o \textbf{10} minnen] mynne n (o) \textbf{12} Belakane] belekane n belacane o \textbf{13} was] wart n o \textbf{14} daz soumære] Das truͯg soumnere m Das sammer n Das somer o  $\cdot$ solten] solte n o \textbf{15} zeigete] zeiget n o \textbf{16} dâ wol sô breit] so wol bereit n o \textbf{17} snüere] summer n somer o \textbf{18} Gahmuret] Gamiret n Gamuͯret o \textbf{19} selben] selbe n o  $\cdot$ enbeiz] enbis o \textbf{20} vleiz] schleis m \textbf{21} kœme] kome m \textbf{22} des] Der o  $\cdot$ niht] des nit o  $\cdot$ dô] \textit{om.} n o \textbf{23} an] and o \textbf{24} zesamen] [zan*nen]: zasamen m \textbf{26} vuort] fuͯrt m \textbf{29} der] den m \textbf{30} dâ] do m \textit{om.} n o \newline
\end{minipage}
\end{table}
\newpage
\begin{table}[ht]
\begin{minipage}[t]{0.5\linewidth}
\small
\begin{center}*G
\end{center}
\begin{tabular}{rl}
 & tet ez ûf, als im ze muote was.\\ 
 & dâr \textbf{obe} stuont der palas.\\ 
 & \textbf{dâ} \textbf{was} diu küniginne\\ 
 & ze\textbf{n vensteren} dâr inne\\ 
5 & mit maniger \textbf{juncvrouwen}.\\ 
 & die \textbf{begunden} \textbf{alle} schouwen,\\ 
 & \textbf{waz} dise knappen tâten.\\ 
 & \textbf{die} heten sich berâten.\\ 
 & \textbf{si} sluogen ûf \textbf{ein} gezelt.\\ 
10 & umbe unvergolten minnen gelt\\ 
 & \textit{\begin{large}W\end{large}}art e\textit{s} ein künic âne.\\ 
 & des twanc in Belacane.\\ 
 & mit arbeit \textbf{wart} ûf geslagen,\\ 
 & daz \textbf{drîzic} soumære \textbf{muosen} tragen,\\ 
15 & ein gezelt. daz \textbf{zeigte} rîcheit.\\ 
 & \textbf{dô} was der \textbf{anger} wol \textbf{sô} breit,\\ 
 & daz sich die snüere stracten dran.\\ 
 & Gahmuret, der werde man,\\ 
 & \textbf{al die wîle} dort ûze enbeiz.\\ 
20 & dar nâch er sich mit vlîze vleiz,\\ 
 & wier \textbf{stolzlîche} k\textit{œ}me geriten.\\ 
 & \textbf{des} wart \textbf{dô langer niht} gebiten.\\ 
 & sîne knappen an den stunden\\ 
 & sîniu sper zesamene bunden,\\ 
25 & iegelîcher vünfiu an ein bant.\\ 
 & daz sehste vuort\textbf{er in der hant}\\ 
 & mit einer baniere.\\ 
 & sus kom gevaren der fiere.\\ 
 & \textbf{vor} der künigîn wart vernomen,\\ 
30 & \textbf{wie dâ ein rîter} \textbf{solte} komen\\ 
\end{tabular}
\scriptsize
\line(1,0){75} \newline
G I O L M Q R Z Fr37 Fr44 \newline
\line(1,0){75} \newline
\textbf{1} \textit{Initiale} O  \textbf{11} \textit{Initiale} G  \textbf{13} \textit{Initiale} I  \textbf{29} \textit{Überschrift:} Wie gamuret zu der kvnigin zv waleis qvam Z   $\cdot$ \textit{Initiale} L M Q Z Fr37 Fr44  \newline
\line(1,0){75} \newline
\textbf{1} \textit{Die Verse 58.9-63.24 fehlen (Blattverlust)} R   $\cdot$ tet ez] Setz O Tet her osz M \textbf{2} obe] ubir M (Fr44) oben Q \textbf{3} dâ] Vnd O Do Q Fr37 Ovch Z  $\cdot$ was] stuͤnt I saz O L (M) (Q) Z Fr37 Fr44 \textbf{4} zen vensteren] zuͤ dem venster I (Fr37) Zuͯ einem venster L An den fenstern M (Q) \textbf{5} maniger] mangen O  $\cdot$ juncvrouwen] schonen frawen O werden frouͯwen L (M) (Z) (Fr37) (Fr44) frawen werden Q \textbf{6} alle] \textit{om.} O L M Q Z Fr37 Fr44 \textbf{7} dise] die L (Q)  $\cdot$ tâten] tæten O totin M \textbf{8} die] Si O (L) (M) (Q) (Z) (Fr37) (Fr44) \textbf{9} si] Vnd O L (M) Z (Fr37) (Fr44) Vff Q  $\cdot$ gezelt] geczeilt M \textbf{10} minnen] minne O (L) (M) (Q) mine Fr37 \textbf{11} Wart] Vwart G  $\cdot$ es] ez G sin I  $\cdot$ künic] chunne nih I \textbf{12} twanc] betwanc M  $\cdot$ Belacane] bellicân I Belecane L Belachane Fr44 \textbf{13} arbeit wart] arbeit war I arbeiten wart es L \textbf{14} \textit{Vers 61.14 fehlt} Q   $\cdot$ muosen] muͤsen I \textbf{15} zeigte] bezæigt O czeiget M er zeigete eyn Q erzeigt Fr37 \textbf{16} dô] Da L M Z  $\cdot$ anger] plan O L M Q Z Fr37 Fr44  $\cdot$ wol] uol Fr37  $\cdot$ sô] \textit{om.} Q \textbf{17} die] \textit{om.} L  $\cdot$ stracten] stracken Q wol stracten Fr44  $\cdot$ dran] dan Fr37 \textbf{18} Gahmuret] Gamvret O Gahmuͯret L Gamuret M Z Fr44 Gamuert Q \textbf{19} al die wîle] Die selben zit O L (M) (Q) Z (Fr37) Die selbe zit Fr44  $\cdot$ dort ûze enbeiz] da [*z]: ez enbeiz I vz er beizt O \textbf{20} er sich mit vlîze] er sich mit Q uil sere er sich Fr44 \textbf{21} stolzlîche] hvbslich O (L) (M) (Q) (Z) (Fr37) boueliche Fr44  $\cdot$ kœme] chome G O (L) (Q) Fr37 \textbf{22} des] Da L  $\cdot$ wart] en wart O M (Q) (Z) (Fr37) (Fr44)  $\cdot$ dô] da I M Z Fr37 \textit{om.} L  $\cdot$ langer] \textit{om.} Q  $\cdot$ gebiten] vermiten I \textbf{23} an den] da zeden I \textbf{24} sîniu] Jr Fr44 \textbf{25} vünfiu] vunfen I  $\cdot$ ein] eine L \textbf{26} sehste] sheste I  $\cdot$ vuorter] vuͤrt er I  $\cdot$ in] an O L (M) Q Z Fr44 \textbf{27} einer] eyme M (Q) \textbf{28} sus] alsus I  $\cdot$ kom gevaren] chom I gevarn chom O \textbf{29} vor] von I (Q) (Fr44)  $\cdot$ künigîn] borg M \textbf{30} dâ ein rîter] ein gast dar O L (Q) (Z) eyn gast M (Fr37) ein gast do Fr44 \newline
\end{minipage}
\hspace{0.5cm}
\begin{minipage}[t]{0.5\linewidth}
\small
\begin{center}*T (U)
\end{center}
\begin{tabular}{rl}
 & tet ez ûf, als im zuo muote was.\\ 
 & d\textit{â} \textbf{oben} stuont der palas.\\ 
 & \textbf{d\textit{â}} \textbf{saz} diu küneginne\\ 
 & zuo \textbf{eim venster} dâr inne\\ 
5 & mit maniger \textbf{werden vrouwen}.\\ 
 & die \textbf{gerne wolten} schouwen,\\ 
 & \textbf{daz} dise knappen t\textit{â}ten.\\ 
 & \textbf{si} heten sich berâten\\ 
 & \textbf{und} sluogen ûf \textbf{ir} gezelt.\\ 
10 & umb unvergolten minnen gelt\\ 
 & wart es ein künec âne.\\ 
 & des twanc in Belacane.\\ 
 & mit arbeit \textbf{wart} ûf geslagen,\\ 
 & daz \textbf{drîzic} soumær \textbf{muosen} tragen,\\ 
15 & ein gezelte. daz \textbf{erzeigete} rîcheit.\\ 
 & \textbf{dô} was der \textbf{plân} wol \textit{b}reit,\\ 
 & daz sich die snüere stracten dran.\\ 
 & Gahmuret, der werde man,\\ 
 & \textbf{die selbe zît} dor\textit{t} ûze enbeiz.\\ 
20 & dar nâch er sich mit vlîze vleiz,\\ 
 & wie er \textbf{höveschlîch} kæme geriten.\\ 
 & \textbf{dô} wart \textbf{dâ langer niht} gebiten.\\ 
 & sîne knappen an den stunden\\ 
 & sîniu sper zesamen bunden,\\ 
25 & ieclîcher vünfiu an ein bant.\\ 
 & daz sehste vuorte \textbf{der wîgant}\\ 
 & mit einer baniere.\\ 
 & sus kam gevaren der fiere.\\ 
 & \textbf{\begin{large}V\end{large}or} der küniginne wart vernomen,\\ 
30 & \textbf{wie dâ ein ritter} \textbf{wære} komen\\ 
\end{tabular}
\scriptsize
\line(1,0){75} \newline
U V W T \newline
\line(1,0){75} \newline
\textbf{3} \textit{Majuskel} T  \textbf{10} \textit{Majuskel} T  \textbf{13} \textit{Initiale} T  \textbf{23} \textit{Majuskel} T  \textbf{29} \textit{Initiale} U W T  \newline
\line(1,0){75} \newline
\textbf{2} dâ] Do U V W \textbf{3} Dâ saz] Do saz U (W) T [*]: Oͮch las V \textbf{4} eim venster] den venstern V (W) (T) \textbf{5} \textit{Versfolge 61.6-5} W  \textbf{6} gerne wolten] begundent alle V (T) \textbf{7} daz] Waz V (T)  $\cdot$ tâten] deden U \textbf{8} si] Die V \textbf{9} ir] ein V T \textbf{11} es] ez T \textbf{12} in] ein W  $\cdot$ Belacane] Belakane V belikane W belacâne T \textbf{13} arbeit] arbaiten W \textbf{14} muosen] muͦzen U soltent V moͤchten W mvͤsen T \textbf{15} erzeigete] zoͮgete V zaiget W zeigete T \textbf{16} dô] Das W da T  $\cdot$ breit] bereit U so breit V (T) so berait W \textbf{17} stracten] straifften W \textbf{18} Gahmuret] Gahmuͦret U Gamuret V W \textbf{19} selbe] selben V W T  $\cdot$ dort] dor U  $\cdot$ ûze] auch W \textbf{21} er höveschlîch kæme] hoͤueliche erkeme V \textbf{22} dô] des T  $\cdot$ wart] enwart V  $\cdot$ dâ langer niht] nv́t langer V lenger nit W do lenger niht T \textbf{23} \textit{Versfolge 61.24-23} V   $\cdot$ an den stunden] sa zestvnden T \textbf{24} sîniu] sine T \textbf{25} an ein] zuͦ sammen W \textbf{26} vuorte der wîgant] fuͦrte er an der (seiner W ) hant V (W) (T) \textbf{29} Vor] VOn W \textbf{30} wie dâ ein ritter] Daz ein ritter V Wie ein ritter W wie ein gast da T \newline
\end{minipage}
\end{table}
\end{document}
