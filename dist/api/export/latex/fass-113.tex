\documentclass[8pt,a4paper,notitlepage]{article}
\usepackage{fullpage}
\usepackage{ulem}
\usepackage{xltxtra}
\usepackage{datetime}
\renewcommand{\dateseparator}{.}
\dmyyyydate
\usepackage{fancyhdr}
\usepackage{ifthen}
\pagestyle{fancy}
\fancyhf{}
\renewcommand{\headrulewidth}{0pt}
\fancyfoot[L]{\ifthenelse{\value{page}=1}{\today, \currenttime{} Uhr}{}}
\begin{document}
\begin{table}[ht]
\begin{minipage}[t]{0.5\linewidth}
\small
\begin{center}*D
\end{center}
\begin{tabular}{rl}
\textbf{113} & \textbf{die küneginne} des geluste,\\ 
 & daz si in \textbf{vil dicke} kuste.\\ 
 & \multicolumn{1}{l}{ - - - }\\ 
 & \textbf{si} sprach \textbf{hin zim} in allen vlîz:\\ 
 & "bon fiz, \textbf{scher fiz}, bêâfiz."\\ 
5 & \textbf{\begin{large}D\end{large}iu künegîn nam dô} sunder twâl\\ 
 & \textbf{diu} rôten \textbf{velwelohten} mâl;\\ 
 & ich meine ir \textbf{tütelînes gränselîn},\\ 
 & \textbf{daz} \textbf{schoup si} im in sîn \textbf{vlänselîn}.\\ 
 & \textbf{selbe was} sîn amme,\\ 
10 & diu in truoc in ir wamme.\\ 
 & an ir brüste si in zôch,\\ 
 & di\textit{u} wîbes missewende vlôch.\\ 
 & si \textbf{dûhte}, \textbf{si hete} Gahmureten\\ 
 & wider an ir arm erbeten.\\ 
15 & \multicolumn{1}{l}{ - - - }\\ 
 & \multicolumn{1}{l}{ - - - }\\ 
 & \textbf{Vrou} Herzeloyde sprach mit sinne:\\ 
 & "diu hœhste küneginne\\ 
 & Jesus ir brüste bôt,\\ 
20 & der sît durch uns \textbf{vil} scharpfen tôt\\ 
 & ame kriuze menschlîche enpfienc\\ 
 & unt sîne triwe an uns begienc.\\ 
 & swes \textit{lîp} \textbf{\textit{sîn} zürnen} \textbf{ringet},\\ 
 & \textbf{des} sêle unsamfte dinget,\\ 
25 & swie kiusche er sî unt wære.\\ 
 & des weiz ich wâriu mære."\\ 
 & Sich begôz des landes vrouwe\\ 
 & mit ir \textbf{herzen jâmers touwe}.\\ 
 & ir ougen \textbf{regenden} ûf den knaben.\\ 
30 & si kunde wîbes triwe haben.\\ 
\end{tabular}
\scriptsize
\line(1,0){75} \newline
D Fr33 \newline
\line(1,0){75} \newline
\textbf{5} \textit{Initiale} D Fr33  \textbf{17} \textit{Majuskel} D  \textbf{27} \textit{Majuskel} D  \newline
\line(1,0){75} \newline
\textbf{3} Die kuningin sprach in allen vliz Fr33 \textbf{4} scher fiz] Girofiz Fr33 \textbf{5} künegîn] vrouwe Fr33 \textbf{6} diu] ir Fr33  $\cdot$ velwelohten] verwehten Fr33 \textbf{7} tütelînes] tuttels ::: Fr33 \textbf{8} daz] Vnd Fr33 \textbf{12} diu] die D  $\cdot$ vlôch] ie vloch Fr33 \textbf{13} Gahmureten] Gahmvreten D Gamureten Fr33 \textbf{15} \textit{Die Verse 113.15-16 fehlen} D   $\cdot$ Sine k::te sich niht an :os:eit Fr33 \textbf{16} die demut w:::r bereit Fr33 \textbf{17} Herzeloyde] herzeloide Fr33  $\cdot$ sinne] sinnen Fr33 \textbf{18} küneginne] kuninginnen Fr33 \textbf{19} Jesus] iesvs D Jhu: Fr33 \textbf{20} scharpfen tôt] groze not Fr33 \textbf{23} lîp sîn] sin lip D \textbf{24} dinget] clinget Fr33 \textbf{26} wâriu] war div Fr33 \textbf{28} jâmers] iamer Fr33 \textbf{30} si kunde] Si ::: kunde Fr33 \newline
\end{minipage}
\hspace{0.5cm}
\begin{minipage}[t]{0.5\linewidth}
\small
\begin{center}*m
\end{center}
\begin{tabular}{rl}
 & \textbf{die küniginne} des geluste,\\ 
 & daz si in \textbf{vil dicke} kuste.\\ 
 & \multicolumn{1}{l}{ - - - }\\ 
 & \textbf{si} sprach \textbf{hin zuo ime} in allen vlîz:\\ 
 & "bonfiz, \textbf{gerafiz}, bêâfiz."\\ 
5 & \textbf{dô nam si ouch dicke} sunder twâl\\ 
 & \textbf{diu} r\textit{ô}ten \textbf{v\textit{e}lwel\textit{o}hte\textit{n}} mâl;\\ 
 & i\textit{ch} meine ir \textbf{tüttels gränselîn},\\ 
 & \textbf{daz} \textbf{schoup si} ime in sîn \textbf{vlänselîn}.\\ 
 & \textbf{si was selber} sîn amme,\\ 
10 & diu in truoc in ir wamme.\\ 
 & an ir brüste si in zôch,\\ 
 & diu wîbes missewende vlôch.\\ 
 & si \textbf{dûhte}, \textbf{si hete} Gahmureten\\ 
 & wider an ir arm erbeten.\\ 
15 & si kêrte sich niht an lôsheit.\\ 
 & \textbf{ir was diu diemuot} bereit.\\ 
 & \textbf{vrouwe} Herczeloid\textit{e} sprach mit sinne:\\ 
 & "diu hœheste küniginne\\ 
 & Jhesus ir brüste bôt,\\ 
20 & der sît durch uns \textbf{vil} scharfen tôt\\ 
 & an dem kriuze menschlîch enpfienc\\ 
 & und sîne triuwe an uns begienc.\\ 
 & wes lîp \textbf{sînen zorn} \textbf{erringet},\\ 
 & \textbf{des} sê\textit{le} unsanfte dinget,\\ 
25 & wie kiusche er sî und wære.\\ 
 & des weiz ich wâriu mære."\\ 
 & sich begôz des landes vrouwe\\ 
 & mit ir \textbf{herzen jâmers touwe}.\\ 
 & ir ougen \dag regende\dag  ûf de\textit{n} kna\textit{b}en.\\ 
30 & si kunde wîbes triuwe haben.\\ 
\end{tabular}
\scriptsize
\line(1,0){75} \newline
m n o \newline
\line(1,0){75} \newline
\textbf{19} \textit{Capitulumzeichen} n  \newline
\line(1,0){75} \newline
\textbf{1} des] suͯ des n \textbf{3} in allen] mit n o \textbf{4} gerafiz bêâfiz] gerasis beasis n \textbf{5} dicke] \textit{om.} o  $\cdot$ twâl] twale m \textbf{6} rôten] ratten m  $\cdot$ velwelohten] valwelchtem m falbelechten n balbeletten o  $\cdot$ mâl] mole m \textbf{7} ich] Jn m  $\cdot$ tüttels] tettels n o \textbf{8} vlänselîn] frenselin n \textbf{9} selber] selbe o \textbf{10} ir] sin n \textbf{12} missewende] muͯsse wendes o \textbf{13} Gahmureten] gahmuretten m gamireten n gamuͯreten o \textbf{14} erbeten] gebetten n \textbf{17} vrouwe] Froͧwen n  $\cdot$ Herczeloide] herczeloiden m hertezloide n herczeleide o \textbf{19} bôt] bat o \textbf{20} uns] \textit{om.} n \textbf{21} menschlîch] menlich o \textbf{22} an] gar na an o \textbf{24} sêle] selben m \textbf{28} ir] ires n (o) \textbf{29} regende] regene o  $\cdot$ den knaben] dem knaken m \textbf{30} si] Sin o \newline
\end{minipage}
\end{table}
\newpage
\begin{table}[ht]
\begin{minipage}[t]{0.5\linewidth}
\small
\begin{center}*G
\end{center}
\begin{tabular}{rl}
 & \textbf{sîne muoter} des geluste,\\ 
 & daz sin \textbf{vil dicke} kuste.\\ 
 & \multicolumn{1}{l}{ - - - }\\ 
 & \textbf{diu künigîn} sprac\textit{h} in allen vlîz:\\ 
 & "bon fiz, \textbf{tschier fiz}, beanfiz."\\ 
5 & \textbf{diu künigîn nam dô} sunder twâl\\ 
 & \textbf{iriu} rôten \textbf{velwelohten} mâl;\\ 
 & ich meine ir \textbf{tütelînes gränsel}.\\ 
 & \textbf{si} \textbf{schoup si}m in sîn \textbf{vlänsel}.\\ 
 & \textbf{selbe was} sîn amme,\\ 
10 & diu in truoc in ir wamme.\\ 
 & an ir brüste si in zôch,\\ 
 & di\textit{u} wîbes missewende vlôch.\\ 
 & si \textbf{dûhte}, \textbf{si hete} Gahmureten\\ 
 & wider an ir arm erbeten.\\ 
15 & si kêrte sich niht an lôsheit.\\ 
 & \textbf{diemuot was ir} bereit.\\ 
 & \textbf{vrô} Herzeloide sprach mit sinne:\\ 
 & "diu hœheste küniginne\\ 
 & Jesus ir brüste bôt,\\ 
20 & der sît durch uns \textbf{vil} scharpfen tôt\\ 
 & ame kriuze menschlîche enpfie\\ 
 & unde sîne triwe an uns begie.\\ 
 & swes lîp \textbf{sînen zorn} \textbf{\textit{r}inget},\\ 
 & \textbf{diu} sêle unsanfte dinget,\\ 
25 & wie kiuscher sî und wære.\\ 
 & des weiz ich wâriu mære."\\ 
 & sich begôz des landes vrouwe\\ 
 & mit ir \textbf{herzen jâmers touwe}.\\ 
 & ir ougen \textbf{regenden} ûf den knaben.\\ 
30 & si kunde wîbes triwe haben.\\ 
\end{tabular}
\scriptsize
\line(1,0){75} \newline
G I O L M Q R Z \newline
\line(1,0){75} \newline
\textbf{1} \textit{Initiale} O  \textbf{5} \textit{Initiale} L R  \textbf{7} \textit{Initiale} Q  \textbf{15} \textit{Initiale} I  \newline
\line(1,0){75} \newline
\textbf{1} sîne] ÷ine O Siner M \textbf{2} sin] sim I in Q  $\cdot$ dicke kuste] gekust I \textbf{3} künigîn] muͤter I  $\cdot$ sprach] sprac G  $\cdot$ in allen] in allem I Z in allē L M Q nit R  $\cdot$ vlîz] wis O \textbf{4} bon fiz] Benficz Q  $\cdot$ tschier fiz] [verafiz]: veirafiz I tschie fiz O \textbf{5} \textit{vor 113.5:} \sout{Die konigin sprach} Q   $\cdot$ dô] \textit{om.} O da M Z  $\cdot$ sunder] sund R \textbf{6} rôten] rotiv O (L) (M) (Q) (Z) Rote R  $\cdot$ velwelohten] valwohten I velbelohtiv O (R) velwelchtin M velweleche Q  $\cdot$ mâl] har mal Z \textbf{7} \textit{Die Verse 113.7-8 fehlen} O   $\cdot$ ir] irs Q R Z  $\cdot$ tütelînes] [roͤten]: roten I tutules M (Q) trúttes R tvttels Z  $\cdot$ gränsel] [was]: Grans I flensil Q (R) \textbf{8} si] Die L M Q Das R (Z)  $\cdot$ sim] imz I si Q  $\cdot$ sîn] sinen I  $\cdot$ vlänsel] vlans I grensel Q R \textbf{9} was sîn] si was I waz sie sin L (Q) (Z) was sy R \textbf{10} in] Jm R  $\cdot$ ir wamme] ieren wamen R \textbf{12} diu] die G  $\cdot$ missewende] missewendes R \textbf{13} dûhte] daht I  $\cdot$ Gahmureten] Gamvreten O gamureten M (Z) Gahmuͯreten L gamúreten Q \textbf{14} wider] Wunder O  $\cdot$ ir] irn L  $\cdot$ arm] arme O Z  $\cdot$ erbeten] gebeten O M Q (R) \textbf{15} si] Sin I (M) (Z)  $\cdot$ kêrte] chert I (O) (Z)  $\cdot$ lôsheit] sein lotheit Q \textbf{16} diemuot] Die muter Q Die diemut Z  $\cdot$ ir] im Q \textbf{17} vrô] Fow R  $\cdot$ Herzeloide] herzelaude I herzenlavde O hertzelauͯde L herczeloide M herzeloude Q herczelaude R herzelovde Z  $\cdot$ sinne] sinde Q \textbf{19} Jesus] iesus G Jê svs O Jesuͯs L Jhesus M Z \textbf{20} sît] ist Q  $\cdot$ durch uns vil] och ein L dur vns den R  $\cdot$ scharpfen tôt] sharphe not I \textbf{21} Menschlich an dem chrevze enphiench O  $\cdot$ menschlîche] menlich M mensch Q \textbf{22} sîne] si O \textbf{23} swes] Wez L (M) (Q) (R)  $\cdot$ sînen] sin O Z in L  $\cdot$ zorn] zvͤrnen O (M) (Q) (R) (Z)  $\cdot$ ringet] erringet G \textbf{24} diu] des I (O) (L) (M) (Q) (Z) \textbf{25} wie] Wir M Swie Z  $\cdot$ kiuscher] kusse er M \textbf{26} \textit{Vers 113.26 fehlt} R  \textbf{28} ir herzen jâmers] herzen iamers L (M) irs hertzensz iamer Q \textbf{29} regenden] regen O (M) \newline
\end{minipage}
\hspace{0.5cm}
\begin{minipage}[t]{0.5\linewidth}
\small
\begin{center}*T (U)
\end{center}
\begin{tabular}{rl}
 & \textbf{sîn muoter} des geluste,\\ 
 & daz sin \textbf{ofte} kuste\\ 
 & - er was rôserôt und snêwîz -\\ 
 & \multicolumn{1}{l}{ - - - }\\ 
 & \textbf{und sprach}: "bonfiz, befiz."\\ 
5 & \textbf{diu künegîn nam dô} sunder twâl\\ 
 & \textbf{ir} rôten \textbf{valw\textit{e}n} mâl\\ 
 & - ich meine ir \textbf{tuten gränsel} -\\ 
 & \textbf{und} \textbf{slo\textit{u}f ez} im in sî\textit{n} \textbf{vlänsel}.\\ 
 & \textbf{selbe was si} sîn amme,\\ 
10 & diu in truoc in ir wamme.\\ 
 & \multicolumn{1}{l}{ - - - }\\ 
 & \multicolumn{1}{l}{ - - - }\\ 
 & si \textbf{wânde} \textbf{haben} Gahmuret\textit{en}\\ 
 & wider an ir a\textit{r}m erbeten.\\ 
15 & si \textbf{en}kêrte sich niht an lôsheit.\\ 
 & \textbf{di\textit{e}mu\textit{o}t was ir} \textbf{vil} bereit.\\ 
 & Herzeloyde sprach mit sinne:\\ 
 & "diu hœheste küniginne\\ 
 & Jesuse ir brüste bôt,\\ 
20 & der sît durch uns \textbf{den} scharpfen tôt\\ 
 & anme kriuze menschlîche entvie\\ 
 & und sîne triuwe an uns begie.\\ 
 & wes lîp \textbf{sînen zorn} \textbf{erringet},\\ 
 & \textbf{des} sêle unsanfte dinget,\\ 
25 & wie kiusche er sî und wære.\\ 
 & des weiz ich wâriu mære."\\ 
 & sich begôz des landes vrouwe\\ 
 & mit ir \textbf{herzen touwe}.\\ 
 & ir ougen \textbf{truogen regen} ûf den knaben.\\ 
30 & si kunde \textbf{wol} wîbes triuwe haben.\\ 
\end{tabular}
\scriptsize
\line(1,0){75} \newline
U V W T \newline
\line(1,0){75} \newline
\textbf{1} \textit{Majuskel} T  \textbf{3} \textit{Majuskel} T  \textbf{4} \textit{Majuskel} T  \textbf{5} \textit{Initiale} V W T  \textbf{23} \textit{Majuskel} T  \textbf{27} \textit{Majuskel} T  \newline
\line(1,0){75} \newline
\textbf{2} \textit{Vers 113.2¹ fehlt} T  \textbf{3} \textit{Vers 113.3 fehlt} U V W   $\cdot$ Sin mvͦter sprach in allen vliz T \textbf{4} [*]: Sv́ iach dicke bonfis beafis schierfis V  $\cdot$ Sy sprach bonfis beafis befis W  $\cdot$ Bônfiz Beafiz Befiz T \textbf{5} nam dô] gab im W \textbf{6} \textit{Vers nachträglich weitgehend radiert} T   $\cdot$ valwen] valweten U (T) val wißen W \textbf{7} \textit{Versteile nachträglich weitgehend radiert} T   $\cdot$ tuten] titen U titel V T tútten W \textbf{8} und slouf ez im] Vnd slof iz im U [*]: Die schoͮp sv́ im V Die schob sy im W die stiez sim T  $\cdot$ sîn] sinen U \textbf{11} \textit{Die Verse 113.11-12 fehlen} U W   $\cdot$ [S*]: An ir brúste sú in zoch V  $\cdot$ an ir brvsten si in zoch T \textbf{12} Die (div T ) wibes messewende floch V (T) \textbf{13} wânde haben] duhte sú hette V (T)  $\cdot$ Gahmureten] Gahmuͦret U Gamuretten V gamureten W \textbf{14} arm] ararm U  $\cdot$ erbeten] gebetten V (T) \textbf{15} enkêrte] kerte W T \textbf{16} diemuot] Der muͦnt U  $\cdot$ vil] \textit{om.} V W wol T \textbf{17} Herzeloyde] Herzeleide U Froͮ herzelaude V Frawe hertzeloyde W Vrôv Herzeloyde T \textbf{19} Jesuse] Jesuͦze U Jesus V Ihesu W \textbf{20} den scharpfen] vil scharpfen V den bittern W scharpfen T \textbf{23} wes] Swes V (T) \textbf{26} wâriu] rechte W \textbf{28} ir herzen touwe] irm herzin dauwe U irs hertzen iamers toͮwe V irs hertzen tauwe W ir herzen iamers toͮwe T \textbf{29} truogen regen] [regen*]: regentent V [regentten]: regeneten T \textbf{30} wol] \textit{om.} T \newline
\end{minipage}
\end{table}
\end{document}
