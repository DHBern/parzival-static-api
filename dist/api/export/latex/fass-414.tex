\documentclass[8pt,a4paper,notitlepage]{article}
\usepackage{fullpage}
\usepackage{ulem}
\usepackage{xltxtra}
\usepackage{datetime}
\renewcommand{\dateseparator}{.}
\dmyyyydate
\usepackage{fancyhdr}
\usepackage{ifthen}
\pagestyle{fancy}
\fancyhf{}
\renewcommand{\headrulewidth}{0pt}
\fancyfoot[L]{\ifthenelse{\value{page}=1}{\today, \currenttime{} Uhr}{}}
\begin{document}
\begin{table}[ht]
\begin{minipage}[t]{0.5\linewidth}
\small
\begin{center}*D
\end{center}
\begin{tabular}{rl}
\textbf{414} & \begin{large}W\end{large}elt ir nû hœren, ich tuon iu kunt,\\ 
 & wâ von ê sprach mîn munt,\\ 
 & daz lûter gemüete \textbf{trüebe} wart.\\ 
 & geunêrt sî \textbf{diu} strîtes vart,\\ 
5 & die ze Schampfanzun tet Vergulaht,\\ 
 & wan \textbf{daz} was im \textbf{niht geslaht}\\ 
 & von vater \textbf{noch} von muoter.\\ 
 & der junge man vil guoter\\ 
 & von schame leit vil grôzen pîn,\\ 
10 & \textbf{dô} sîn swester, diu künegîn,\\ 
 & in begunde vêhen.\\ 
 & man hôrt in sêre vlêhen.\\ 
 & Dô sprach diu juncvrouwe wert:\\ 
 & "hêr Vergulaht, trüege ich\textbf{z} swert\\ 
15 & unt wære von gotes gebot ein man,\\ 
 & daz ich schildes ambet solde hân,\\ 
 & iwer strîten wære \textbf{hie} gar \textbf{verzagt}.\\ 
 & dô was ich âne wer ein magt,\\ 
 & wan daz ich truoc \textbf{doch} einen schilt,\\ 
20 & ûf den ist werdecheit gezilt.\\ 
 & des wâpen \textbf{sol} ich nennen,\\ 
 & ob ir ruochet diu \textbf{bekennen}:\\ 
 & guot gebærde unt kiuscher site,\\ 
 & den zwein \textbf{wont} vil stæte mite.\\ 
25 & den bôt ich vür den ritter mîn,\\ 
 & den ir mir sandet dâ her în.\\ 
 & \begin{large}A\end{large}nders schermes \textbf{het} ich niht.\\ 
 & swâ man iuch nû bî wandel siht,\\ 
 & ir habt doch an mir missetân,\\ 
30 & ob wîplîch prîs \textbf{sîn} reht sol hân.\\ 
\end{tabular}
\scriptsize
\line(1,0){75} \newline
D \newline
\line(1,0){75} \newline
\textbf{1} \textit{Initiale} D  \textbf{13} \textit{Initiale} D  \textbf{27} \textit{Initiale} D  \newline
\line(1,0){75} \newline
\textbf{5} Vergulaht] Vergvlaht D \textbf{14} Vergulaht] Vergvlaht D \newline
\end{minipage}
\hspace{0.5cm}
\begin{minipage}[t]{0.5\linewidth}
\small
\begin{center}*m
\end{center}
\begin{tabular}{rl}
 & \begin{large}W\end{large}elt ir nû hœren, ich tuon iu kunt,\\ 
 & wâ von ê sprach mîn munt,\\ 
 & daz lûter gemüete \textbf{trüebe} wart.\\ 
 & g\textit{eu}n\textit{ê}ret sî \textbf{diu} strîtes vart,\\ 
5 & die ze Schanfanzun tet Verg\textit{u}laht,\\ 
 & wanne \textbf{daz} was ime \textbf{ungeslaht}\\ 
 & von vater \textbf{und} von muoter.\\ 
 & der junge man vil guoter\\ 
 & von scheme leit vil grôzen pîn,\\ 
10 & \textbf{dô} sîn swester, diu künigîn,\\ 
 & in begunde \textit{v}êhen.\\ 
 & man hôrt in sêre vlêhen.\\ 
 & dô sprach diu juncvrouwe wert:\\ 
 & "hêr Vergul\textit{a}ht, trüege ich swert\\ 
15 & und wære von gotes gebote ein man,\\ 
 & daz ich schiltes ambet solte hân,\\ 
 & iuwer strîten wære \textbf{mich} gar \textbf{verdaget}.\\ 
 & dô was ich âne wer ein maget,\\ 
 & wanne daz ich truoc \textbf{dô} einen schilt,\\ 
20 & ûf den ist werdicheit gezilt.\\ 
 & des wâpen \textbf{sol} ich nennen,\\ 
 & ob ir ruochet diu \textbf{bekennen}:\\ 
 & guot gebærde und kiusche site,\\ 
 & den zwein \textbf{wont} vil stæte mite.\\ 
25 & den bôt ich vür den ritter mîn,\\ 
 & den ir mir sandet dâ her în.\\ 
 & anders schirmes \textbf{hete} ich niht.\\ 
 & wâ man iuch nû bî wandel siht,\\ 
 & ir habet doch an mir missetân,\\ 
30 & ob wîplîch prîs \textbf{sîn} reht sol hân.\\ 
\end{tabular}
\scriptsize
\line(1,0){75} \newline
m n o \newline
\line(1,0){75} \newline
\textbf{1} \textit{Initiale} m   $\cdot$ \textit{Capitulumzeichen} n  \newline
\line(1,0){75} \newline
\textbf{1} nû] \textit{om.} n \textbf{4} geunêret] Ganuͯret m \textbf{5} Schanfanzun] scanfanzun m n tanfanczẏm o  $\cdot$ Vergulaht] [verh]: verguhlaht m vergulacht n \textbf{6} was] \textit{om.} o \textbf{9} scheme] schemde n o  $\cdot$ leit] liet o  $\cdot$ grôzen] grosse n o \textbf{11} Jme begunde nehen n (o) \textbf{14} Vergulaht] verguleht m vergulacht n \textbf{16} ambet] amptes n \textbf{20} gezilt] ist gezilt n \textbf{23} guot] [*]: Guͯter n  $\cdot$ kiusche] kúscher n (o) \textbf{24} wont] wonte n o \textbf{26} sandet] do santent n santen o  $\cdot$ dâ] \textit{om.} n do o \textbf{28} nû] \textit{om.} n o \newline
\end{minipage}
\end{table}
\newpage
\begin{table}[ht]
\begin{minipage}[t]{0.5\linewidth}
\small
\begin{center}*G
\end{center}
\begin{tabular}{rl}
 & welt ir nû hœren, ich tuon iu kunt,\\ 
 & wâ von ê sprach mîn munt,\\ 
 & daz lûter gemüete \textbf{getrüebet} wart.\\ 
 & geunêrt sî \textbf{diu} strîtes vart,\\ 
5 & \begin{large}D\end{large}ie ze Tschanfenzun tet Vergulaht,\\ 
 & wan \textbf{daz} was im \textbf{niht geslaht}\\ 
 & von vater \textbf{noch} von muoter.\\ 
 & der junge man vil guoter\\ 
 & von schem leit vil grôzen pîn,\\ 
10 & \textbf{dô} sîn swester, diu künigîn,\\ 
 & in begunde vêhen.\\ 
 & man hôrte in sêre vlêhen.\\ 
 & dô sprach diu juncvrouwe wert:\\ 
 & "hêr Vergulaht, trüege ich \textbf{daz} swert\\ 
15 & unde wære von gotes gebot ein man,\\ 
 & daz ich schiltes ambet solte hân,\\ 
 & iwer strîten wære \textbf{hie} gar \textbf{verdaget}.\\ 
 & dô was ich âne wer ein maget,\\ 
 & wan daz ich truoc \textbf{doch} einen schilt,\\ 
20 & ûf den ist werdicheit gezilt.\\ 
 & des wâpen \textbf{wil} ich nennen,\\ 
 & obe ir ruochet diu \textbf{bekennen}:\\ 
 & guot gebærde unde kiuscher site,\\ 
 & den zwein \textbf{wont} vil stæte mite.\\ 
25 & den bôt ich vür den rîter mîn,\\ 
 & den ir mir sandet dâ her în.\\ 
 & anders schermes \textbf{het} ich niht.\\ 
 & swâ man iuch nû bî wandel siht,\\ 
 & ir habt doch an mir missetân,\\ 
30 & obe wîplîch prîs \textbf{sîn} reht sol hân.\\ 
\end{tabular}
\scriptsize
\line(1,0){75} \newline
G I O L M Q R Z \newline
\line(1,0){75} \newline
\textbf{1} \textit{Initiale} I R  \textbf{3} \textit{Initiale} O L Z  \textbf{5} \textit{Initiale} G  \textbf{21} \textit{Initiale} I  \newline
\line(1,0){75} \newline
\textbf{1} nû] \textit{om.} O  $\cdot$ hœren] horet I \textbf{3} \textit{Die Verse 414.3-4 fehlen} R   $\cdot$ daz] ÷az O  $\cdot$ lûter] luczet Q  $\cdot$ getrüebet] trvͦbe O (L) (M) (Q) (Z) \textbf{4} geunêrt] Gevnerter M  $\cdot$ diu] des I (O) des dy M \textbf{5} ze Tschanfenzun] zetschanfenzvn G zeshanpanzuͤn I zeschampfazvn O zcu schanfinzcun M zu tschanpfenzűn Q ze schanfenzun R zv Tschanfanzvn Z  $\cdot$ tet] [tot]: tet O  $\cdot$ Vergulaht] virgulaht I [vergulabt]: vergvlaht O vergvlaht L vergulacht M Q (R) \textbf{6} daz] es Q  $\cdot$ geslaht] geslah I \textbf{7} noch von] vnde von O noch Q \textbf{9} schem] sehen O  $\cdot$ vil] \textit{om.} I R  $\cdot$ grôzen] grosse Q (R) \textbf{11} begunde] begunden Q \textbf{13} dô] Da M Z  $\cdot$ juncvrouwe] iunge frowe I \textbf{14} Vergulaht] virgulaht I vergvlaht O L Z virgulacht M vergulacht Q R  $\cdot$ trüege] vnde trvͦge O  $\cdot$ daz] \textit{om.} O L M R ewer Q \textbf{15} gebot] amp I gabe O gebort M \textbf{17} iwer] V́we R  $\cdot$ strîten] strit I M (Q) R  $\cdot$ gar verdaget] gar vertzagt Q vertagt R \textbf{18} dô] Da M  $\cdot$ âne wer ein] ein arme R  $\cdot$ maget] mag Q \textbf{19} wan] an I Was R  $\cdot$ truoc doch] doch truͤc I (R) \textbf{20} werdicheit] werlich R \textbf{21} wil] sol O L (M) Q Z \textbf{22} ruochet diu] die ruͯchet L  $\cdot$ bekennen] erchennen O bekennet Q \textbf{23} guot] Guͤte I  $\cdot$ kiuscher] chiusche I (L) (Q) (R)  $\cdot$ site] sitten R \textbf{24} wont] wonten I volgt O (L) (R) volgt got Q  $\cdot$ vil] \textit{om.} Z \textbf{25} den] Die L \textbf{26} sandet] da sandet Z \textbf{27} schermes] schermel O  $\cdot$ het] hat Q \textbf{28} swâ] Wa L M (Q) R Wan Z  $\cdot$ iuch nû] nun úch R \textbf{30} prîs sîn] kuͤsch ir I \newline
\end{minipage}
\hspace{0.5cm}
\begin{minipage}[t]{0.5\linewidth}
\small
\begin{center}*T
\end{center}
\begin{tabular}{rl}
 & \begin{large}W\end{large}elt ir nû hœren, ich tuon iu kunt,\\ 
 & wâ von ê sprach mîn munt,\\ 
 & daz lûter gemüete \textbf{t\textit{rü}ebe} wart.\\ 
 & geunêrt sî \textbf{des} strîtes vart,\\ 
5 & die ze Tschampfenzun tet Vergulaht,\\ 
 & wande\textit{\textbf{z}} was im \textbf{niht geslaht}\\ 
 & von vater \textbf{noch} von muoter.\\ 
 & der junge man vil guoter\\ 
 & von schame leit vil grôzen pîn.\\ 
10 & sîn swester, diu künegîn,\\ 
 & in begunde vêhen.\\ 
 & man hôrtin sêre vlêhen.\\ 
 & \begin{large}D\end{large}ô sprach diu juncvrouwe wert:\\ 
 & "hêr Vergulaht, trüegich swert\\ 
15 & unde wære von gotes gebote ein man,\\ 
 & daz ich schiltes ambet solte hân,\\ 
 & iuwer strîten wære \textbf{hie} gar \textbf{verdaget}.\\ 
 & dô was ich âne wer ein maget,\\ 
 & wan daz ich truoc \textbf{doch} einen schilt,\\ 
20 & ûf den ist werdecheit gezilt.\\ 
 & des wâpen \textbf{sol} ich nennen,\\ 
 & ob ir ruochet di\textit{u} \textbf{erkennen}:\\ 
 & guot gebærde unde kiusche site,\\ 
 & den zwein \textbf{volget} vil stæte mite.\\ 
25 & den bôt ich vür den rîter mîn,\\ 
 & den ir mir santet dâ her în.\\ 
 & anders schirmes \textbf{hân} ich niht.\\ 
 & swâ man iuch nû bî wandel siht,\\ 
 & ir hânt doch an mir missetân,\\ 
30 & ob wîplîch prîs \textbf{ir} reht sol hân.\\ 
\end{tabular}
\scriptsize
\line(1,0){75} \newline
T U V W \newline
\line(1,0){75} \newline
\textbf{1} \textit{Initiale} T U  \textbf{3} \textit{Initiale} W  \textbf{13} \textit{Initiale} T U  \newline
\line(1,0){75} \newline
\textbf{1} nû] \textit{om.} W \textbf{3} trüebe] tvrebe T \textbf{4} des] die V \textbf{5} Tschampfenzun] Tscampfenzvn T Tschamfenzuͦn U schanpfanzvn V  $\cdot$ Vergulaht] Vergvlaht T vergulacht U W virgulaht V \textbf{6} wandez] wandes T wan das V  $\cdot$ niht geslaht] vngeslaht V \textbf{8} junge] iunger W \textbf{9} schame] schamen V  $\cdot$ vil grôzen] [*]: vil gorssen V vil grosser W \textbf{10} sîn] Die sin U (W) Do sin V \textbf{11} in] [*]: Jn V \textbf{13} sprach] spracht W \textbf{14} Vergulaht] Vergvlaht T vergulacht U (W) virgulaht V  $\cdot$ trüegich] truͦg ichs W \textbf{17} iuwer] Eúwerm W  $\cdot$ hie] [*]: hie V \textbf{18} ich] auch W \textbf{20} den] dem W \textbf{22} ruochet diu] rvͦchet die T die ruͦchet W \textbf{26} dâ] do U V W \textbf{27} hân] hat U V het W \textbf{28} swâ] Wo U W  $\cdot$ iuch] îv T \textbf{30} ir] sein W \newline
\end{minipage}
\end{table}
\end{document}
