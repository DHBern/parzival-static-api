\documentclass[8pt,a4paper,notitlepage]{article}
\usepackage{fullpage}
\usepackage{ulem}
\usepackage{xltxtra}
\usepackage{datetime}
\renewcommand{\dateseparator}{.}
\dmyyyydate
\usepackage{fancyhdr}
\usepackage{ifthen}
\pagestyle{fancy}
\fancyhf{}
\renewcommand{\headrulewidth}{0pt}
\fancyfoot[L]{\ifthenelse{\value{page}=1}{\today, \currenttime{} Uhr}{}}
\begin{document}
\begin{table}[ht]
\begin{minipage}[t]{0.5\linewidth}
\small
\begin{center}*D
\end{center}
\begin{tabular}{rl}
\textbf{803} & \textbf{\begin{large}D\end{large}es gezeltes winden nam man} abe.\\ 
 & der künec sprach: "wederz ist der knabe,\\ 
 & der künec sol sîn über iwer lant?"\\ 
 & \textbf{al} den vürsten tet er \textbf{dâ} bekant:\\ 
5 & Waleis unt Norgals,\\ 
 & Kanvoleiz unt Kingrivals\\ 
 & \textbf{der selbe sol mit rehte} hân,\\ 
 & \textbf{z}Anschouwe unt \textbf{in} Bealzenan.\\ 
 & \textbf{kom} er immer \textbf{an} mannes kraft,\\ 
10 & dâr \textbf{leistet} im geselleschaft.\\ 
 & Gahmuret mîn vater hiez,\\ 
 & der mirz mit rehtem erbe liez.\\ 
 & Mit sælde ich geerbet hân den Grâl.\\ 
 & nû enpfâhet ir an disem mâl\\ 
15 & \textbf{iweriu} lêhen von mîme kinde,\\ 
 & ob ich an iu triwe vinde."\\ 
 & Mit guotem willen daz geschach.\\ 
 & vil vanen man \textbf{dort} vüeren sach.\\ 
 & dâ lihen zwô kleine hende\\ 
20 & wîter lande manec ende.\\ 
 & Gekrœnt wart dô Kardeiz.\\ 
 & der betwanc ouch sider Kanvoleiz\\ 
 & unt vil \textbf{des} Gahmuretes was.\\ 
 & bî dem Plimizœl ûf \textbf{ein} gras\\ 
25 & wart gesidel \textbf{unt} wîter rinc genomen,\\ 
 & dâ si zem brôte solden komen.\\ 
 & Snellîche dâ enbizzen wart.\\ 
 & \textbf{daz her} kêrte an die \textbf{heimvart}.\\ 
 & diu gezelt nam man elliu nider.\\ 
30 & mit dem jungen künege si \textbf{vuoren} wider.\\ 
\end{tabular}
\scriptsize
\line(1,0){75} \newline
D \newline
\line(1,0){75} \newline
\textbf{1} \textit{Initiale} D  \textbf{13} \textit{Majuskel} D  \textbf{17} \textit{Majuskel} D  \textbf{21} \textit{Majuskel} D  \textbf{27} \textit{Majuskel} D  \newline
\line(1,0){75} \newline
\textbf{5} Waleis] Wâls D \textbf{6} Kingrivals] kyngrivals D \textbf{8} zAnschouwe] zAnscowe D \textbf{11} Gahmuret] Gahmvret D \textbf{21} Kardeiz] kardeyz D \textbf{22} Kanvoleiz] kanvoleyz D \textbf{23} Gahmuretes] Gahmvretes D \textbf{24} Plimizœl] Plimizol D \newline
\end{minipage}
\hspace{0.5cm}
\begin{minipage}[t]{0.5\linewidth}
\small
\begin{center}*m
\end{center}
\begin{tabular}{rl}
 & \textbf{des gezelte\textit{s} winden nam man} ab.\\ 
 & der künîc sprach: "wederz ist der knab,\\ 
 & der künîc sol sîn über iuwer lant?"\\ 
 & \textbf{al} den vürsten tet er bekant,\\ 
5 & \textbf{daz er} Wals und Norgals,\\ 
 & Kanvoleiz und Kingrivals,\\ 
 & \hspace*{-.7em}\big| Anscho\textit{u}we und Bealtzenan\\ 
 & \hspace*{-.7em}\big| \textbf{von rehte solte} hân.\\ 
 & \textbf{k\textit{o}m} er iemer \textbf{an} mannes kraft,\\ 
10 & dâr \textbf{leisten} im geselleschaft.\\ 
 & \textit{G}a\textit{hmure}t mîn vater hiez,\\ 
 & der mirz mit rehtem erbe liez.\\ 
 & mit sælde ich geerbet hân den Grâl.\\ 
 & nû enpfâhet ir an disem mâl\\ 
15 & \textbf{iuwer} lêhen von mî\textit{n}e\textit{m} kinde,\\ 
 & ob ich an iu triuwe vinde."\\ 
 & mit guotem willen daz geschach.\\ 
 & vil vane\textit{n} man \textbf{dort} vüeren sach.\\ 
 & d\textit{â} lihen zwô klein\textit{e h}ende\\ 
20 & wîter l\textit{a}nde \textbf{vil} manic ende.\\ 
 & gekrœnet wart dô Cardeiz.\\ 
 & der betwanc ouch sider Kanvoleiz\\ 
 & und vil \textbf{des} Gahmuretes was.\\ 
 & bî dem Plimizol ûf \textbf{daz} gras\\ 
25 & wart gesidel wîter rinc genomen,\\ 
 & d\textit{â} si zuo dem brôte solten komen.\\ 
 & snelleclîch d\textit{â} enbizzen wart.\\ 
 & \textbf{daz her} kêrte an die \textbf{hinvart}.\\ 
 & diu gezelt nam \textit{ma}n alli\textit{u n}ider.\\ 
30 & mit dem jungen künige si \textbf{vuoren} wider.\\ 
\end{tabular}
\scriptsize
\line(1,0){75} \newline
m n o V V' W \newline
\line(1,0){75} \newline
\textbf{11} \textit{Initiale} W  \textbf{21} \textit{Initiale} V  \newline
\line(1,0){75} \newline
\textbf{1} \textit{Vers 803.1 fehlt} V'   $\cdot$ gezeltes] gezeltten m (o)  $\cdot$ winden] wunden o winde V \textbf{2} \textit{Versdoppelung} m   $\cdot$ \textit{statt 803.2-8:} Der kvnic tet den [de]: fursten bekant / Daz sin erbe wer daz lant (vgl. 803.12: erbe) V'  \textbf{3} sol sîn] sein sol W \textbf{4} er] der W \textbf{5} Wals] mals o [*]: waleiz V waleiß W \textbf{6} Kanvoleiz] Canvoleis m n Comvoleis o Canfoleis V Kanuoleis W  $\cdot$ Kingrivals] kingriwals n kingriuals W \textbf{8} Anschouwe] Anschowe m V Anschauwe n Beschowe o Antschowe W  $\cdot$ Bealtzenan] ebalczenan o [*]: bealsenan V balsanan W \textbf{9} \textit{Die Verse 803.9-10 fehlen} V'   $\cdot$ kom] Kam m n  $\cdot$ er] \textit{om.} o \textbf{10} leisten] leistent V \textbf{11} Er sprach gameret min vatter hies \textit{(vgl. 803.2:} sprach\textit{)} V'  $\cdot$ Gahmuret] Canvert \textit{nachträglich korrigiert zu:} Gamuret m Gamuret n (W) Gamuͯret o [Gam*]: Gamuret V \textbf{12} mit] zu V'  $\cdot$ erbe] erben V \textbf{13} sælde] selden V' \textbf{14} ir] \textit{om.} W  $\cdot$ disem] diesen V' \textbf{15} mînem] mẏnnen m mẏnne o \textbf{16} an iu triuwe] truwe an uch V' \textbf{17} guotem] guten V' \textbf{18} vanen] fanenen m \textbf{19} dâ] Do m n o V V' W  $\cdot$ lihen] luhen V V'  $\cdot$ kleine hende] cleine man hende m \textbf{20} wîter] Vil witer V' Weite W  $\cdot$ lande] lende m  $\cdot$ vil] \textit{om.} V V' \textbf{21} Cardeiz] cardeis m n o W Kardeiz V kardeis V' \textbf{22} ouch sider] dar nach V'  $\cdot$ Kanvoleiz] canuoleis m n camuoleis o kanfoleiz V konfoleis V' kanuoleis W \textbf{23} des] das W  $\cdot$ Gahmuretes] gahmurettes m gamiretes n gamuretes o W Gammerehtes V gamerethen V' \textbf{24} Plimizol] plimzol n plimenzale V plimezale V' plymizol W  $\cdot$ daz] ein V V' \textbf{25} wîter] wite V' vnd weiter W \textbf{26} dâ] Do m n o V V' W  $\cdot$ dem brôte] der spise V' den bretten W \textbf{27} dâ] do m n o V V' W  $\cdot$ enbizzen] erbissen W \textbf{28} die] \textit{om.} o  $\cdot$ hinvart] heimfart V (V') \textbf{29} nam man] nomen m  $\cdot$ alliu nider] alle wider vnd nẏder m \textbf{30} si] \textit{om.} m n o W \newline
\end{minipage}
\end{table}
\newpage
\begin{table}[ht]
\begin{minipage}[t]{0.5\linewidth}
\small
\begin{center}*G
\end{center}
\begin{tabular}{rl}
 & \textbf{man nam des gezeltes winden} abe.\\ 
 & \begin{large}D\end{large}er künic sprach: "wederz ist der knabe,\\ 
 & der künic sol sîn über iwer lant?"\\ 
 & den vürsten tet er \textbf{dâ} bekant:\\ 
5 & "Waleis unde Nurgals,\\ 
 & Kanvoleis unde Kinrivals\\ 
 & \textbf{der selbe sol von rehte} hân,\\ 
 & Antschowe unde Belzanan.\\ 
 & \textbf{kum} er iemer \textbf{in} mannes kraft,\\ 
10 & dâr \textbf{leist ich} im geselleschaft.\\ 
 & Gahmuret mîn vater hiez,\\ 
 & der mirz mit rehtem erbe liez.\\ 
 & mit sælde ich geerbet hân den Grâl.\\ 
 & nû enpfâhet ir an dise\textit{m} mâl\\ 
15 & \textbf{iwer} lêhen vo\textit{n} \textit{mîne}m kinde,\\ 
 & obe ich an iu triwe vinde."\\ 
 & mit guotem willen daz geschach.\\ 
 & vil vanen man \textbf{dâr} vüeren sach.\\ 
 & dâ lihen zwô kleine hende\\ 
20 & wîter lande manig ende.\\ 
 & gekrœnet wart dô Kardeiz.\\ 
 & der betwanc ouch sider Kanvoleiz\\ 
 & unde vil, \textbf{d\textit{az}} Gahmuretes was.\\ 
 & bî dem Blimzol ûf \textbf{daz} gras\\ 
25 & wart gesidel \textbf{unde} wîter rinc genomen,\\ 
 & dâ si zem brôte solden komen.\\ 
 & snellîch dâ enbizzen wart.\\ 
 & \textbf{er} kêrt an die \textbf{heimvart}.\\ 
 & diu gezelt nam man elliu nider.\\ 
30 & mit dem jungen künige si \textbf{kêrten} wider.\\ 
\end{tabular}
\scriptsize
\line(1,0){75} \newline
G I L M Z Fr48 \newline
\line(1,0){75} \newline
\textbf{1} \textit{Initiale} L Z  \textbf{2} \textit{Initiale} G  \textbf{11} \textit{Initiale} I  \newline
\line(1,0){75} \newline
\textbf{1} winden] wende M \textbf{2} künic] knape M  $\cdot$ wederz ist] weder ist I kardeiz L widir hiesz ist M \textbf{3} der künic sol] Sol kvnig L \textbf{4} dâ] daz L \textbf{5} Waleis] vvaleis G  $\cdot$ Nurgals] norgelals I Nvrglas L \textbf{6} Kanvoleis] kanvoleiz G kanpholeis I Kamfoleis Z  $\cdot$ Kinrivals] kinkrivals G kingriuals I kingrivals L Z k:ngrivals M \textbf{8} Antschowe] Anschoͮwe G [Antshowe]: Antshawe I Anshowe L Z  $\cdot$ Belzanan] bealzenan G (Z) bealzanan I (L) \textbf{9} kum] chumt I \textbf{10} leist ich] leistet Z \textbf{11} \textit{Die Verse 803.11-20 fehlen} L   $\cdot$ Gahmuret] Gamuret Z Gachmuret Fr48 \textbf{12} mit rehtem] zerehtem I \textbf{13} mit sælde] mit selden I Mir selbe Z \textbf{14} disem] disen G \textbf{15} von mînem] vom G \textbf{17} willen] willem Fr48 \textbf{18} man dâr] dar man Fr48  $\cdot$ vüeren sach] Gesach I \textbf{19} dâ] Die Z  $\cdot$ kleine] clare Z \textbf{20} witev lant in mangen ende I \textbf{21} dô] da Z \textbf{22} der] der chunc I  $\cdot$ ouch sider] do I och sit L  $\cdot$ Kanvoleiz] chanfoleiz I kamvoleiz L kamfoleiz Z kamuoleiz Fr48 \textbf{23} daz] des G  $\cdot$ Gahmuretes] gamuretes Z Gahmuͦretes Fr48 \textbf{24} Blimzol] plimizol I L Z Fr48  $\cdot$ daz] ein I L Fr48 dem Z \textbf{26} dâ] Do Fr48  $\cdot$ zem] zuͯ L  $\cdot$ solden] wolten L \textbf{27} dâ] \textit{om.} I do L \textbf{28} er] Daz her L (Z) Fr48 \textbf{29} diu] Den Fr48 \textbf{30} si kêrten] kertens L \newline
\end{minipage}
\hspace{0.5cm}
\begin{minipage}[t]{0.5\linewidth}
\small
\begin{center}*T
\end{center}
\begin{tabular}{rl}
 & \textbf{\begin{large}M\end{large}an nam des gezeltes winden} abe.\\ 
 & der künec sprach: "\textit{wederz} ist der knabe,\\ 
 & der künec sol sîn über iuwer lant?"\\ 
 & den vürsten tet er \textbf{dâ} bekant:\\ 
5 & "Waleis und Nurgals,\\ 
 & Kanvoleiz und Kingrivals\\ 
 & \textbf{der selbe \textit{sol} von rehte} hân,\\ 
 & \textbf{zuo} Anschouwe und \textbf{in} Bealzenan.\\ 
 & \textbf{kumt} er imer \textbf{an} mannes kraft,\\ 
10 & dâr \textbf{leiste ich} im geselleschaft.\\ 
 & Gahmuret mîn vater hiez,\\ 
 & der mir ez mit rehtem erbe liez.\\ 
 & mit sælden ich geerbet hân den Grâl.\\ 
 & nû enpfâhet ir an disem mâl\\ 
15 & \textbf{iuwer} lêhen von mîme kinde,\\ 
 & ob ich an iu triuwe vinde."\\ 
 & mit guotem willen daz geschach.\\ 
 & vil vanen man \textbf{dâr} vüeren sach.\\ 
 & dâ lihen zwô kleine hende\\ 
20 & wîter lande manec ende.\\ 
 & gekrœnet wart dô Kardeiz.\\ 
 & der betwanc ouch sider Kanvoleiz\\ 
 & und vil, \textbf{daz} Gahmuretes was.\\ 
 & bî dem Plymizol ûf \textbf{ein} gras\\ 
25 & wart gesidel \textbf{und} wîter rinc genomen,\\ 
 & d\textit{â} si zuo dem brôte solten komen.\\ 
 & snellîche dâ enbizzen wart.\\ 
 & \textbf{daz volc} kêrte an die \textbf{heimvart}.\\ 
 & diu gezelt nam man alliu nider.\\ 
30 & mit dem jungen künege si \textbf{kêrten} wider.\\ 
\end{tabular}
\scriptsize
\line(1,0){75} \newline
U Q R \newline
\line(1,0){75} \newline
\textbf{1} \textit{Initiale} U Q R  \newline
\line(1,0){75} \newline
\textbf{1} gezeltes] [gedeyltes]: gedeltes Q \textbf{2} wederz] \textit{om.} U deweders R \textbf{4} dâ] do Q R \textbf{5} Nurgals] nuͦrgals U \textbf{6} Kanvoleiz] Kanvoleis U kanúoleis Q Kanuoleis R  $\cdot$ Kingrivals] nurgals Q \textbf{7} sol] \textit{om.} U \textbf{8} Anschouwe] Anschowe U (R) anszhowe Q  $\cdot$ in] \textit{om.} Q R  $\cdot$ Bealzenan] Beazenan U bealzanan Q (R) \textbf{11} Gahmuret] Gahmuͦret U Gamuret Q  $\cdot$ hiez] lies R \textbf{12} der mir ez] Dersz mir Q \textbf{13} sælden] selde Q R \textbf{17} Mit tuͯtten trúwen das beschach R \textbf{18} dâr] do Q R \textbf{19} dâ] Do Q R  $\cdot$ lihen] luchen R  $\cdot$ zwô] zu Q \textbf{20} manec] an mengem R \textbf{21} Kardeiz] kardeisz Q Cardeiz R \textbf{22} Kanvoleiz] kanvoleis Q kanuoleis R \textbf{23} Gahmuretes] gamúretes Q Gahmuurtes R \textbf{24} Plymizol] plimizol U Q R  $\cdot$ ein] dem R \textbf{25} gesidel] gesidelt Q gesidlet R  $\cdot$ wîter] ward R \textbf{26} dâ] Do U Q Das R  $\cdot$ zuo dem brôte solten] zemborte solde Q \textbf{27} Schellenklichen do enbissen vnd gosen ward R  $\cdot$ enbizzen] enbisset Q \textbf{28} kêrte] kert R \textbf{29} nam man] man nam R  $\cdot$ nider] wider Q \textbf{30} jungen] iunge Q  $\cdot$ si kêrten] kert man Q \newline
\end{minipage}
\end{table}
\end{document}
