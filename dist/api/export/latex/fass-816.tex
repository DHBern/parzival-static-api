\documentclass[8pt,a4paper,notitlepage]{article}
\usepackage{fullpage}
\usepackage{ulem}
\usepackage{xltxtra}
\usepackage{datetime}
\renewcommand{\dateseparator}{.}
\dmyyyydate
\usepackage{fancyhdr}
\usepackage{ifthen}
\pagestyle{fancy}
\fancyhf{}
\renewcommand{\headrulewidth}{0pt}
\fancyfoot[L]{\ifthenelse{\value{page}=1}{\today, \currenttime{} Uhr}{}}
\begin{document}
\begin{table}[ht]
\begin{minipage}[t]{0.5\linewidth}
\small
\begin{center}*D
\end{center}
\begin{tabular}{rl}
\textbf{816} & \begin{large}W\end{large}ie diu wirtîn \textbf{selbe} dan \textbf{gegienc}\\ 
 & unt wie manz dâ nâch an \textbf{gevienc},\\ 
 & daz man sîn wol mit \textbf{betten} pflac,\\ 
 & der doch durch minne unsanfte lac,\\ 
5 & Wie al der templeise diet\\ 
 & mit \textbf{senfte} \textbf{unsenfte} von \textbf{in} schiet,\\ 
 & \textbf{dâ von} würde \textbf{ein} langiu sage.\\ 
 & ich wil iu künden von dem tage,\\ 
 & dô \textbf{der} des morgens \textbf{vruo} erschein,\\ 
10 & Parzival wart des enein\\ 
 & unt Anfortas, der guote,\\ 
 & mit endehaftem muote\\ 
 & si bâten den von Zazamanc\\ 
 & komen, den diu minne twanc,\\ 
15 & in den tempel vür den Grâl.\\ 
 & er gebôt ouch an dem selbem mâl\\ 
 & \textbf{dem} wîsem \textbf{templeise} dar.\\ 
 & scharjande, rîter, grôziu schar\\ 
 & \textbf{dâ} stuont. \textbf{nû} gienc der heiden în.\\ 
20 & der toufnapf was ein rubîn,\\ 
 & von jaspis ein \textbf{grêde} sinwel,\\ 
 & dâr ûf er stuont. Titurel\\ 
 & het in mit kost erziuget sô.\\ 
 & Parzival zuo sînem bruoder dô\\ 
25 & sprach: "wiltû die muomen mîn\\ 
 & haben, al die gote dîn\\ 
 & muostû durch si versprechen\\ 
 & unt immer gerne rechen\\ 
 & den widersaz des hœhsten gotes\\ 
30 & unt mit triwen schônen sînes gebotes."\\ 
\end{tabular}
\scriptsize
\line(1,0){75} \newline
D \newline
\line(1,0){75} \newline
\textbf{1} \textit{Initiale} D  \textbf{5} \textit{Majuskel} D  \newline
\line(1,0){75} \newline
\textbf{10} Parzival] Parcifal D \textbf{13} Zazamanc] Zazamanch D \textbf{20} rubîn] Rvbbin D \textbf{21} jaspis] Jaspes D \textbf{24} Parzival] Parcifal D \newline
\end{minipage}
\hspace{0.5cm}
\begin{minipage}[t]{0.5\linewidth}
\small
\begin{center}*m
\end{center}
\begin{tabular}{rl}
 & wie diu wirtîn \textbf{selben} dan \textbf{gienc}\\ 
 & und wie manz dâ nâch an \textbf{vienc},\\ 
 & daz man sîn wol mit \textbf{b\textit{e}tten} pflac,\\ 
 & der doch durch minne unsanft lac,\\ 
5 & wie alle de\textit{r} \textit{te}mpleise diet\\ 
 & mit \textbf{vröuden} \textbf{unvröude} von \textbf{in} schiet,\\ 
 & \textbf{dâ von} würde \textbf{ein} langiu sage.\\ 
 & ich wil iu künden von dem tage,\\ 
 & dô des morgens \textbf{lieht} erschein,\\ 
10 & Parcifal wart des in ein\\ 
 & und Anfortas, der guote,\\ 
 & mit endehaftem muote\\ 
 & si bâten den von Zazamanc\\ 
 & komen, den diu minne twanc,\\ 
15 & in de\textit{n} \textit{t}empel vür den Grâl.\\ 
 & er gebôt ouch an dem selben mâl\\ 
 & \textbf{dem} wîsen \textbf{templeise} dar.\\ 
 & sarjande, ritter, grôziu schar\\ 
 & \textbf{d\textit{â}} stuont. \textit{\textbf{nû}} gienc der heiden în.\\ 
20 & der toufnap\textit{f} \textit{w}as ein rubîn,\\ 
 & von jaspis ein \textbf{grêde} sinwel,\\ 
 & dâr ûf \textit{er} stuont. Titurel\\ 
 & het in mit koste erz\textit{iu}get sô.\\ 
 & Parcifal zuo sînem bruoder dô\\ 
25 & sprach: "wiltû die muomen mîn\\ 
 & haben, alle die gote dîn\\ 
 & muostû durch si versprechen\\ 
 & und iemer gerne rechen\\ 
 & den widersaz des hœhsten gotes\\ 
30 & und mit triuwen schônen sînes gebotes."\\ 
\end{tabular}
\scriptsize
\line(1,0){75} \newline
m n V V' W \newline
\line(1,0){75} \newline
\textbf{1} \textit{Initiale} V  \textbf{25} \textit{Initiale} W  \newline
\line(1,0){75} \newline
\textbf{1} \textit{Die Verse 808.12-816.5 fehlen} V'   $\cdot$ selben] selb n (V) W  $\cdot$ gienc] gegieng V \textbf{2} vienc] gevieng V (W) \textbf{3} betten] beitten m (n) \textbf{4} \textit{nach 816.4:} [*]: Oͮch liez man nv́t vnder wegen / Artuses wart wol gepflegen / Vnde der tavelrunder schar / Wart oͮch herlich genomen war V   $\cdot$ durch] mit W \textbf{5} der templeise] der j tumpleise m der tampleise n der templeisen V \textbf{6} \textit{statt 816.6-10:} Da hatten sie freude wider strit (vgl. 815.16: erstreit) / Vnd waz sie solten habin zu eren / Daz konde in anfortas wol meren (vgl. 815.2: Anfortas, mêre) / Sie en hatten aller gebresten dekein / Dar nach wart parzifal in eyn V'   $\cdot$ vröuden] froͤide V  $\cdot$ in] im W \textbf{8} von] \textit{om.} n \textbf{10} Parcifal] Parzefal V Partzifal W \textbf{11} und] Artus vnde V (V')  $\cdot$ der guote] die guͦten V (V') \textbf{12} endehaftem muote] endehaften mvͦten V (V') \textbf{13} Zazamanc] zazamang m n V' W zasamang V \textbf{15} den tempel] den j tempel m dem tempel n  $\cdot$ den Grâl] dem gral n \textbf{17} dem] [De]: Den V (V') (W)  $\cdot$ templeise] templeisen n V V' (W) \textbf{19} dâ] Do m n V V' W  $\cdot$ nû] ẏm in m do V' W \textbf{20} toufnapf was] touff napf e was m  $\cdot$ rubîn] robin n V rubein W \textbf{21} jaspis] jaspez V iaspis W \textbf{22} er] \textit{om.} m n V'  $\cdot$ Titurel] titturel n tẏturel V her tyturel V' tyturel W \textbf{23} erziuget] erzeiget m n \textbf{24} Parcifal] Parzefal V Parzifal V' Partzifal W \textbf{25} muomen] muͦme V W \textbf{27} Musten fur sie fur sprechen V' \textbf{29} hœhsten] hohen W \textbf{30} gebotes] bottes W \newline
\end{minipage}
\end{table}
\newpage
\begin{table}[ht]
\begin{minipage}[t]{0.5\linewidth}
\small
\begin{center}*G
\end{center}
\begin{tabular}{rl}
 & \begin{large}S\end{large}wie diu wirtîn \textbf{selbe} dan \textbf{gienc}\\ 
 & unde wie manz dar nâch an \textbf{vienc},\\ 
 & daz man sîn wol mit \textbf{triwen} p\textit{f}lac,\\ 
 & der doch durch minne unsanft lac,\\ 
5 & \textbf{unde} wie al der templeis diet\\ 
 & mit \textbf{senft} \textbf{unsenft} von \textbf{i\textit{m}} schiet,\\ 
 & \textbf{daz} würde \textbf{ein} \textbf{al ze} langiu sage.\\ 
 & ich wil iu künden von dem tage,\\ 
 & dô \textbf{der}s morgens \textbf{lieht} erschein,\\ 
10 & Parcival wart des in ein\\ 
 & unde Anfortas, der guote,\\ 
 & mit endehaftem muote\\ 
 & si bâten den von Zazamanc\\ 
 & komen, den diu minne twanc,\\ 
15 & in den tempel vür den Grâl.\\ 
 & er gebôt ouch an dem selben mâl\\ 
 & \textbf{den} wîsen \textbf{templeisen} dar.\\ 
 & sarjande, rîter, grôziu schar\\ 
 & \textbf{hie} stuont. \textbf{dâ} gienc der heiden în.\\ 
20 & der toufnapf was ein rubîn,\\ 
 & von jaspis ein \textbf{grêde} sinwel,\\ 
 & dâr ûfe er stuont. Titurel\\ 
 & het in mit koste erziuget sô.\\ 
 & Parcival ze sînem bruoder dô\\ 
25 & sprach: "wil dû die muomen mîn\\ 
 & haben, al die gote dîn\\ 
 & muostû durch si versprechen\\ 
 & unde imer gerne rechen\\ 
 & den widersaz des hœhesten gotes\\ 
30 & unde mit triwen schônen sînes gebotes."\\ 
\end{tabular}
\scriptsize
\line(1,0){75} \newline
G I L Z \newline
\line(1,0){75} \newline
\textbf{1} \textit{Initiale} G L Z  \textbf{13} \textit{Initiale} I  \newline
\line(1,0){75} \newline
\textbf{1} Swie] Wie L Z \textbf{2} vienc] gevienc Z \textbf{3} pflac] plach G \textbf{5} templeis] Tenpleisen I \textbf{6} im] in G \textbf{7} sage] klage Z \textbf{9} dô ders morgens] oder des morgens I Do der morgen L Da des morgens Z  $\cdot$ lieht] lýcht L \textbf{10} Parcival] Parcifal G Z parzifal I (L) \textbf{11} Anfortas] Amfortas L \textbf{13} Zazamanc] zazamanch G L \textbf{15} in] Fvr in Z  $\cdot$ vür den] zv dem Z \textbf{16} selben] selbe Z \textbf{18} sarjande] Sariander I  $\cdot$ grôziu] Groze I \textbf{19} hie stuont dâ] hie do stuͤnden do I Hie stuͯnt do L Da stunt hie Z \textbf{21} jaspis] Iaspes G iaspide L \textbf{22} er stuont] er stvnt vnd L erstunde Z  $\cdot$ Titurel] titvrel G Týtuͯrel L Tyturel Z \textbf{24} Parcival] Parcifal G Z parzifal I (L)  $\cdot$ sînem] sinen I \textbf{27} durch si] \textit{om.} I \textbf{29} widersaz] widersazt I \textbf{30} mit triwen schônen] mit triwen huͤten I schonen mit truͯwen L \newline
\end{minipage}
\hspace{0.5cm}
\begin{minipage}[t]{0.5\linewidth}
\small
\begin{center}*T
\end{center}
\begin{tabular}{rl}
 & \textit{\begin{large}W\end{large}}ie diu wirtinne \textbf{selber} dan \textbf{gienc}\\ 
 & und wie manz dar nâch an \textbf{vienc},\\ 
 & daz man sîn wol mit \textbf{betten} pflac,\\ 
 & der doch durch minne unsanfte lac,\\ 
5 & \textbf{und} wie al der templeise diet\\ 
 & mit \textbf{senfte} \textbf{unsenfte} von \textbf{im} schiet,\\ 
 & \textbf{daz} würde \textbf{al zuo} langiu sage.\\ 
 & ich wil iu künden von dem tage,\\ 
 & dô \textbf{der} des morgens \textbf{lieht} erschein,\\ 
10 & Parcifal wart des enein\\ 
 & und Anfortas, der guote,\\ 
 & mit endehafte\textit{m} muote\\ 
 & si bâten den von Zazamanc\\ 
 & komen, den diu minne twanc,\\ 
15 & in den temp\textit{el} vür den Grâl.\\ 
 & er gebôt ouch an dem selben mâl\\ 
 & \textbf{den} wîsen \textbf{templeisen} dar.\\ 
 & sarjande, rîter, grôziu schar\\ 
 & \textbf{dâ} stuont. \textbf{hie} gienc der heiden în.\\ 
20 & der toufnapf was ein rubîn,\\ 
 & von jaspis ein \textbf{grât} sinewel,\\ 
 & dâr ûf er stuont. Tyturel\\ 
 & hât in mit kost erziuget sô.\\ 
 & Parcifal zuo sîme bruoder dô\\ 
25 & sprach: "wiltû die muomen mîn\\ 
 & haben, al die gote dîn\\ 
 & muostû durch si versprechen\\ 
 & und imer gerne rechen\\ 
 & den widersaz des hœhesten gotes\\ 
30 & und mit triuwen schônen sînes gebotes."\\ 
\end{tabular}
\scriptsize
\line(1,0){75} \newline
U Q R \newline
\line(1,0){75} \newline
\textbf{1} \textit{Initiale} U R  \newline
\line(1,0){75} \newline
\textbf{1} Wie] Vye U  $\cdot$ selber] selbe Q \textbf{2} dar nâch] dann noch Q \textit{om.} R  $\cdot$ an vienc] angefieng R \textbf{5} templeise] tempelise R \textbf{6} senfte] senfften R  $\cdot$ im] yn Q im do R \textbf{7} \textit{Versfolge 816.8-7} U   $\cdot$ al zuo langiu sage] ein alzu lange sagen Q alze lang ein sage R \textbf{9} lieht erschein] lichten schein Q \textbf{10} Parcifal] Parzifal U Partzifal Q Parczifal R \textbf{11} Anfortas] Anfortes R \textbf{12} endehaftem] endehaften U endemhaffte Q \textbf{13} Zazamanc] zasamanc R \textbf{15} \textit{Versfolge 816.16-15} R   $\cdot$ den tempel] den templein U dem tempel R  $\cdot$ den Grâl] dem grale Q \textbf{16} gebôt] gebt R \textbf{17} wîsen] [we*]: wesen Q  $\cdot$ templeisen] tempelisen R \textbf{18} grôziu] grose R \textbf{19} dâ] Do Q  $\cdot$ hie] do R \textbf{20} rubîn] Ruͦbin U rubein Q \textbf{21} jaspis] iaspis Q  $\cdot$ grât] grede Q grade R \textbf{22} Tyturel] Tytuͦrel U titurel Q \textbf{23} hât] Het Q R  $\cdot$ mit kost erziuget] ertzewget mit kost Q \textbf{24} Parcifal] Parzifal U Partzifal Q Parczifal R \newline
\end{minipage}
\end{table}
\end{document}
