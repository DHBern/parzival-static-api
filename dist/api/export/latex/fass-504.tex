\documentclass[8pt,a4paper,notitlepage]{article}
\usepackage{fullpage}
\usepackage{ulem}
\usepackage{xltxtra}
\usepackage{datetime}
\renewcommand{\dateseparator}{.}
\dmyyyydate
\usepackage{fancyhdr}
\usepackage{ifthen}
\pagestyle{fancy}
\fancyhf{}
\renewcommand{\headrulewidth}{0pt}
\fancyfoot[L]{\ifthenelse{\value{page}=1}{\today, \currenttime{} Uhr}{}}
\begin{document}
\begin{table}[ht]
\begin{minipage}[t]{0.5\linewidth}
\small
\begin{center}*D
\end{center}
\begin{tabular}{rl}
\textbf{504} & \begin{large}W\end{large}iez Gawane komen sî,\\ 
 & der ie was missewende vrî,\\ 
 & sît er von Schanpfanzun \textbf{geschiet},\\ 
 & ob sîn reise ûf strît geriet,\\ 
5 & \textbf{des} jehen, die ez dâ sâhen.\\ 
 & \textbf{er} muoz nû strîte nâhen.\\ 
 & Eines morgens kom \textbf{mîn} hêr Gawan\\ 
 & geriten ûf einen \textbf{grüenen} plân.\\ 
 & dâ sach er \textbf{blicken} einen schilt\\ 
10 & - dâ was \textbf{ein} tjoste durch gezilt -\\ 
 & unt ein pfert, daz vrouwen gereite truoc;\\ 
 & \textbf{des zoum unt satel was} tiur genuoc.\\ 
 & ez was gebunden vaste\\ 
 & \textbf{zuo}me schilte \textbf{an} einem aste.\\ 
15 & dô \textbf{dâhte}r: "wer mac sîn \textbf{diz} wîp,\\ 
 & diu alsus werlîchen lîp\\ 
 & hât, daz si schildes pfligt?\\ 
 & ob si sich strîtes gein mir bewigt,\\ 
 & wie sol ich mich \textbf{ir wern}?\\ 
20 & ze vuoze \textbf{trûwe} ich mich wol \textbf{ernern}.\\ 
 & wil si \textbf{die lenge} ringen,\\ 
 & si mac mich nider bringen;\\ 
 & ich erwerbe\textbf{s} haz ode gruoz,\\ 
 & sol \textbf{dâ} \textbf{ein tjost ergên} ze vuoz.\\ 
25 & ob ez halt vrou Camille wære,\\ 
 & diu mit \textbf{rîterlîchem} mære\\ 
 & vor Laurente prîs erstreit,\\ 
 & wære si \textbf{gesunt}, als si dort reit,\\ 
 & ez würde iedoch versuochet an sie,\\ 
30 & ob si mir strîten büte \textbf{al hie}."\\ 
\end{tabular}
\scriptsize
\line(1,0){75} \newline
D \newline
\line(1,0){75} \newline
\textbf{1} \textit{Großinitiale} D  \textbf{7} \textit{Majuskel} D  \newline
\line(1,0){75} \newline
\textbf{3} Schanpfanzun] Tschanfanzvn D \textbf{25} Camille] kamille D \textbf{27} Laurente] Lavrente D \newline
\end{minipage}
\hspace{0.5cm}
\begin{minipage}[t]{0.5\linewidth}
\small
\begin{center}*m
\end{center}
\begin{tabular}{rl}
 & wie ez Gawan komen sî,\\ 
 & der ie was missewende vrî,\\ 
 & sît er von Schanfanz\textit{un} \textbf{schiet},\\ 
 & ob sîn reise ûf strît geriet,\\ 
5 & \textbf{daz} jehen, die ez d\textit{â} sâhen.\\ 
 & \textbf{er} muoz nû strîte nâhen.\\ 
 & \textit{\begin{large}E\end{large}}ines morgens kam hêr Gawan\\ 
 & geriten ûf einen \textbf{grüenen} plân.\\ 
 & d\textit{â} sach er \textbf{blicken} einen schilt\\ 
10 & - dâ was \textbf{ein} juste durch gezilt -\\ 
 & und ein pfert, daz vrouwen g\textit{e}reit truoc;\\ 
 & \textbf{des z\textit{o}um und satel was} \textit{t}iur genuoc.\\ 
 & ez was gebunden vast\\ 
 & \textbf{bî} de\textit{m} schilt \textbf{zuo} einem ast.\\ 
15 & dô \textbf{gedâht} er: "wer mac sîn \textbf{diz} wîp,\\ 
 & diu alsus werlîchen lîp\\ 
 & het, daz si schiltes pfliget?\\ 
 & ob si sich strîtes gegen mir bewiget,\\ 
 & wie sol ich mich \textbf{ir dan erwern}?\\ 
20 & zuo vuoze \textbf{getriuwe} ich mich wol \textbf{\textit{n}ern}.\\ 
 & wil \textbf{aber} si \textbf{dan} ringen,\\ 
 & s\textit{i} mac mich nider bringen;\\ 
 & ich erwerbe \textbf{sîn} haz oder \dag guot\dag ,\\ 
 & sol \textbf{d\textit{â}} \textbf{ein juste ergân} zuo vuoz.\\ 
25 & ob ez halt vrouwe C\textit{a}mille wære,\\ 
 & diu mit \textbf{ritterlîchem} mære\\ 
 & vor L\textit{au}re\textit{n}te prîs erstreit,\\ 
 & wær si \textbf{lebende}, als si dort reit,\\ 
 & ez würde iedoch versuochet an sie,\\ 
30 & ob si mir strîten büt\textit{e} \textbf{hie}."\\ 
\end{tabular}
\scriptsize
\line(1,0){75} \newline
m n o \newline
\line(1,0){75} \newline
\textbf{7} \textit{Initiale} m   $\cdot$ \textit{Capitulumzeichen} n  \newline
\line(1,0){75} \newline
\textbf{1} Gawan] gawanen n \textbf{3} Schanfanzun] scanfancz ẏm m scanfantz nuͯ n scanfancz ÿm o \textbf{4} ob] Vnd o \textbf{5} dâ] do m o \textit{om.} n  $\cdot$ sâhen] johen n \textbf{6} muoz] muͯs m \textbf{7} Eines] Eeines m  $\cdot$ morgens] morgans o \textbf{9} dâ] Do m n o  $\cdot$ blicken] blecken n \textbf{11} ein] einen n  $\cdot$ gereit] getreit m \textbf{12} zoum] zuͯm m  $\cdot$ tiur] getuͯr m \textbf{14} dem] den m n \textbf{19} erwern] wern n o \textbf{20} nern] wern m \textbf{22} si] So m \textbf{23} guot] guͦten gruͦsz n \textbf{24} dâ] do m n nuͯ o \textbf{25} Camille] comille m n Comillie o \textbf{27} Laurente] loreitte m laureite n laúrente o \textbf{30} Ob mir strite buwen hie o  $\cdot$ büte] buͯttet m bitte n \newline
\end{minipage}
\end{table}
\newpage
\begin{table}[ht]
\begin{minipage}[t]{0.5\linewidth}
\small
\begin{center}*G
\end{center}
\begin{tabular}{rl}
 & \textit{\begin{large}W\end{large}}iez Gawane komen sî,\\ 
 & der ie was missewende vrî,\\ 
 & sît er von Tschanfenzune \textbf{schiet},\\ 
 & op sîn reise ûf strît geriet,\\ 
5 & \textbf{des} jehen, diez dâ sâhen.\\ 
 & \textbf{er} muoz nû strîte nâhen.\\ 
 & eines morgens kom \textbf{mîn} hêr Gawan\\ 
 & geriten ûf einen \textbf{grüenen} plân.\\ 
 & dâ sach er \textbf{blîchen} einen schilt\\ 
10 & - dâ was tjoste durch gezilt -\\ 
 & unde ein pfert, daz vrouwen gereit truoc;\\ 
 & \textit{\textbf{daz was der zoum unde satel}} \textit{tiur genuoc}.\\ 
 & ez was gebunden vaste\\ 
 & \textbf{zuo}me schilte \textbf{an} einem aste.\\ 
15 & dô \textbf{dâht} er: "wer mac sîn \textbf{daz} wîp,\\ 
 & diu alsus werlîchen lîp\\ 
 & hât, daz si schildes pfliget?\\ 
 & op si sich strîtes gein mir bewiget,\\ 
 & wie sol ich mich \textbf{danne ir wern}?\\ 
20 & ze vuoze \textbf{trûwe} ich mich wol \textbf{ernern}.\\ 
 & wil si \textbf{die lenge} ringen,\\ 
 & si mac mich nider bringen;\\ 
 & ich erwerbe\textbf{s} haz ode gruoz,\\ 
 & sol \textbf{dâ} \textbf{ein tjost \textit{er}gên} ze vuoz.\\ 
25 & ob ez halt vrouwe Kamille wære,\\ 
 & diu mit \textbf{rîterlîchem} mære\\ 
 & vor Lorente prîs erstreit,\\ 
 & wære si \textbf{gesunt}, als si dort reit,\\ 
 & ez würde iedoch versuochet an sie,\\ 
30 & op si mir strîten büte \textbf{al hie}."\\ 
\end{tabular}
\scriptsize
\line(1,0){75} \newline
G I L M Z Fr57 \newline
\line(1,0){75} \newline
\textbf{1} \textit{Initiale} G I L Z  \textbf{13} \textit{Initiale} I  \newline
\line(1,0){75} \newline
\textbf{1} Wiez] Siez G  $\cdot$ Gawane] Gawan I Gawanen L (Z) \textbf{2} ie was] was ie Z \textbf{3} er] der I  $\cdot$ Tschanfenzune] tschanfanzvne G shanfanzune I Tscanfenzvn L shanphenzcun M tschanfenzvn Z  $\cdot$ schiet] geschiet Z \textbf{4} strît] striten Z \textbf{5} jehen] sprechin M \textbf{6} er] Jr M Ez Z \textbf{7} mîn] \textit{om.} M Z  $\cdot$ hêr Gawan] her Gauwan I ergawan M \textbf{8} grüenen] \textit{om.} I Z \textbf{9} dâ] Do Z  $\cdot$ blîchen] blicken M (Z) \textbf{10} dâ] Daz L  $\cdot$ tjoste] eine tiost L (M) (Z)  $\cdot$ gezilt] gezist I \textbf{11} ein] \textit{om.} I  $\cdot$ gereit] geriten M \textbf{12} \textit{Vers 504.12 fehlt} G   $\cdot$ Dez zovm vnd (des Z ) satel waz tuͯre genuͯch L (M) (Z)  $\cdot$ Des zo::: Fr57 \textbf{14} \textit{Versdoppelung:} \sout{Zvm schilte an ein gebunden vaste} Zvm schilte an einem aste Z  \textbf{15} dô dâht er] Do gedaht er L Er dachte M  $\cdot$ daz] dize I (M) (Z) \textbf{16} alsus] also Z  $\cdot$ werlîchen] warlichen L wertlichin M \textbf{18} ob si strites Gein mir sich bewigt I \textbf{19} wie] Swie L  $\cdot$ mich danne ir] mich ir danne I L mich danne M mir danne Z  $\cdot$ wern] erwerren M \textbf{20} ze vuoze] zevuͤzen I  $\cdot$ ernern] erwern L \textbf{23} erwerbes] erwerbe I M \textbf{24} ergên] gen G \textbf{25} halt] oͮch I \textit{om.} L  $\cdot$ vrouwe] \textit{om.} M  $\cdot$ Kamille] kanille I gamille M Camille Z \textbf{26} rîterlîchem mære] riterlicher gebere I redelichen mere M \textbf{27} vor] vol I  $\cdot$ Lorente] laurente I (L) M (Z) \textbf{28} wære] Wart M  $\cdot$ gesunt] nu gesunt I  $\cdot$ si dort] si [dol]: dort G si do I L dort M \newline
\end{minipage}
\hspace{0.5cm}
\begin{minipage}[t]{0.5\linewidth}
\small
\begin{center}*T
\end{center}
\begin{tabular}{rl}
 & \begin{large}W\end{large}iez Gawane komen sî,\\ 
 & der ie was missewende vrî,\\ 
 & sît er von Tschampfenzun \textbf{geschiet},\\ 
 & ob sîn reise ûf strît geriet,\\ 
5 & \textbf{des} jehen, diez dâ sâhen.\\ 
 & \textbf{ez} muoz nû strîte nâhen.\\ 
 & \begin{large}E\end{large}ines morgens kom hêr Gawan\\ 
 & geriten ûf einen plân.\\ 
 & dâ sach er \textbf{blicken} einen schilt\\ 
10 & - dâ was \textbf{ein} tjost durch gezilt -\\ 
 & unde ein pfert, daz vrouwen gereite truoc;\\ 
 & \textbf{des satel unde zoum was} tiure genuoc.\\ 
 & ez was gebunden vaste\\ 
 & \textbf{ze}m schilte \textbf{an} einem aste.\\ 
15 & Dô \textbf{dâhte}r: "wer mac sîn \textbf{diz} wîp,\\ 
 & diu alsus werlîchen lîp\\ 
 & hât, \textbf{sô} daz si schiltes pfliget?\\ 
 & ob si sich strîtes gegen mir bewiget,\\ 
 & wie sol ich mich \textbf{ir danne erwern}?\\ 
20 & ze vuoz \textbf{triuwe} ich mich wol \textbf{ernern}.\\ 
 & wil si \textbf{die lenge} ringen,\\ 
 & si mac mich nider bringen;\\ 
 & ich erwerbe\textbf{s} haz oder gruoz,\\ 
 & sol \textbf{hie} \textbf{ergân ein tjost} ze vuoz.\\ 
25 & ob ez halt vrou Camylle wære,\\ 
 & diu mit \textbf{redelîchem} mære\\ 
 & vor Laurente prîs erstreit,\\ 
 & wære si \textbf{gesunt}, als si dort reit,\\ 
 & Ez würde iedoch versuocht an sie,\\ 
30 & ob si mir strîten büt \textbf{al hie}."\\ 
\end{tabular}
\scriptsize
\line(1,0){75} \newline
T U V W O Q R Fr39 \newline
\line(1,0){75} \newline
\textbf{1} \textit{Initiale} T U V W O R Fr39  \textbf{7} \textit{Überschrift:} Hie kam her gawan czuͦ einem gewundten ritter den ernert er von dem tode W  · Initiale T W  \textbf{15} \textit{Majuskel} T  \textbf{29} \textit{Majuskel} T  \newline
\line(1,0){75} \newline
\textbf{1} Wiez] ÷ie ez O SJt das Fr39  $\cdot$ Gawane] Gawanen O (Q) Gawaine R gawan Fr39 \textbf{3} Tschampfenzun] Tschanfenzuͦn U schanpfenzvne V tschampfensun W tschanphazvn O schanpfenzún Q [schanfencze]: schanfenczun R Tscanfenzvn Fr39  $\cdot$ geschiet] schiet V W Q (R) Fr39 \textbf{5} jehen] iahen W iæhen O Jechent R  $\cdot$ dâ] do U W Q R Fr39  $\cdot$ sâhen] sæhen O \textbf{6} ez] Er U O R (Fr39) [Ez]: Er  V \textbf{7} Eines] [En]: Eines T  $\cdot$ Gawan] Gawain R \textbf{8} einen] einen gruͦnen U (V) (W) (O) (Q) (R) Fr39 \textbf{10} tjost] sper R \textbf{11} ein] [*]: ein V  $\cdot$ vrouwen] ein frowen R vrowe Fr39  $\cdot$ gereite] gerete U [gerite]: gereiten Q \textbf{12} satel unde zoum] sattel zoͮm V zaum vnd sattel W (O) (Q) (R) (Fr39) \textbf{13} ez was] Jz vaste U \textbf{14} einem] einen W \textbf{15} Dô] Da O \textit{om.} Q  $\cdot$ dâhter] Gedacht er Q  $\cdot$ diz] daz V O (Q) (R) \textbf{16} diu] Das W \textbf{17} sô] \textit{om.} W O Q R Fr39  $\cdot$ schiltes] schiltes ambt V schilt W \textbf{18} sich] \textit{om.} Q R \textbf{19} mich ir] [mir]: mich O mich R [*]: ir mich Fr39  $\cdot$ erwern] weren W (O) Q R (Fr39) \textbf{21} si] aber sv́ V  $\cdot$ die lenge] [denne*]: denne V \textbf{23} erwerbes] erwerben U erwurbe ez V erwerb W (Q) werbes O erwirb R \textbf{24} Sol [h*]: do ein jvst ergon zefvͦz V  $\cdot$ ergân ein tjost] ein tiost ergen Q (R) (Fr39) ein tiost gen O \textbf{25} halt] alt U ioch R  $\cdot$ vrou] von Fr39  $\cdot$ Camylle] Camille U (V) (Q) (R) Fr39 kamille W [clamille]: camille O \textbf{26} redelîchem] redelicher U ritterlichem V Redlichen R (Fr39) \textbf{27} Laurente] Lavrente U Lavrenti O \textbf{28} gesunt] [*t]: lebende V \textbf{29} sie] ir Q \newline
\end{minipage}
\end{table}
\end{document}
