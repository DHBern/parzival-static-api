\documentclass[8pt,a4paper,notitlepage]{article}
\usepackage{fullpage}
\usepackage{ulem}
\usepackage{xltxtra}
\usepackage{datetime}
\renewcommand{\dateseparator}{.}
\dmyyyydate
\usepackage{fancyhdr}
\usepackage{ifthen}
\pagestyle{fancy}
\fancyhf{}
\renewcommand{\headrulewidth}{0pt}
\fancyfoot[L]{\ifthenelse{\value{page}=1}{\today, \currenttime{} Uhr}{}}
\begin{document}
\begin{table}[ht]
\begin{minipage}[t]{0.5\linewidth}
\small
\begin{center}*D
\end{center}
\begin{tabular}{rl}
\textbf{173} & Man unt wîp, die sint al ein\\ 
 & \textbf{als} diu sunne, diu hiute schein,\\ 
 & unt \textbf{ouch} der \textbf{name}, der \textbf{heizet} tac.\\ 
 & der enwederz sich gescheiden mac.\\ 
5 & \textbf{si} blüent ûz eime kerne gar,\\ 
 & des nemt kunstec\textit{l}îche war."\\ 
 & Der gast dem wirte durch \textbf{râten} neic.\\ 
 & sîner muoter er gesweic\\ 
 & mit rede unt in dem herzen niht,\\ 
10 & als noch \textbf{getriwem} man geschiht.\\ 
 & \textit{\begin{large}D\end{large}}er wirt sprach sîn êre:\\ 
 & "noch sult ir lernen mêre\\ 
 & kunst an rîterlîchen siten.\\ 
 & wie kâmet ir zuo mir geriten!\\ 
15 & ich hân \textbf{beschouwet} manege want,\\ 
 & dâ ich den schilt baz hangen vant,\\ 
 & denn er iu ze halse tæte.\\ 
 & ez ist uns niht ze spæte,\\ 
 & wir sulen ze velde gâhen,\\ 
20 & dâ sult ir künste nâhen.\\ 
 & Bringet im sîn ors unt mirz mîn\\ 
 & unt ieslîchem ritterz sîn.\\ 
 & \textbf{junchêrren sulen ouch dar} komen,\\ 
 & \textbf{der} \textbf{ieslîcher} habe genomen\\ 
25 & \textbf{einen} \textbf{starken} schaft unt bringen dar,\\ 
 & der nâch der niwe sî gevar."\\ 
 & Sus kom der vürste ûf den plân.\\ 
 & dâ wart \textbf{mit rîten} kunst getân.\\ 
 & sîme gaste er \textbf{râten} gap,\\ 
30 & wie erz ors \textbf{ûzem} walap\\ 
\end{tabular}
\scriptsize
\line(1,0){75} \newline
D \newline
\line(1,0){75} \newline
\textbf{1} \textit{Majuskel} D  \textbf{7} \textit{Majuskel} D  \textbf{11} \textit{Initiale} D  \textbf{21} \textit{Majuskel} D  \textbf{27} \textit{Majuskel} D  \newline
\line(1,0){75} \newline
\textbf{6} kunsteclîche] chvnstechiche \textit{nachträglich korrigiert zu:} chvnstechliche D \textbf{11} Der] ÷eR \textit{nachträglich korrigiert zu:} DeR D \newline
\end{minipage}
\hspace{0.5cm}
\begin{minipage}[t]{0.5\linewidth}
\small
\begin{center}*m
\end{center}
\begin{tabular}{rl}
 & man und wîp, die sint alein\\ 
 & \textbf{als} diu sunne, diu hiute sch\textit{e}in,\\ 
 & und \textbf{ouch} der \textbf{mân}, der \textbf{heize} tac.\\ 
 & der e\textit{n}wederz sich gescheiden mac.\\ 
5 & \textbf{si} blüegent ûz einem kernen gar,\\ 
 & des nemt künsteclîchen war."\\ 
 & der gast dem wirt durch \textbf{râten} neic.\\ 
 & sîner muoter er gesweic\\ 
 & mit rede und in dem herzen niht,\\ 
10 & als noch \textbf{getriuwen} man geschiht.\\ 
 & \begin{large}D\end{large}er wirt sprach \textbf{durch} sîn êre:\\ 
 & "noch sullet ir lernen mêre\\ 
 & kunst an ritterlîchen siten.\\ 
 & wie kômet ir ze mir geriten!\\ 
15 & ich hân \textbf{geschouwet} manige \dag yant\dag ,\\ 
 & d\textit{â} ich den schilt baz hangen vant,\\ 
 & danne er iu ze halse tæte.\\ 
 & e\textit{z} ist uns niht ze spæte,\\ 
 & wir sullen ze velde gâhen,\\ 
20 & d\textit{â} sullet ir kü\textit{nst}e nâhen.\\ 
 & bringet im sîn ros und mir daz mîn\\ 
 & und ieglîchem ritter daz sîn.\\ 
 & \textbf{junchêrren sullen ouch dar} komen,\\ 
 & \textbf{der} \textbf{ieglîche\textit{r}} \textit{h}abe genomen\\ 
25 & \textbf{sînen} \textbf{starken} schaft und bringen dar,\\ 
 & d\textit{e}r nâch der niuwe sî gevar."\\ 
 & sus kam der vürste ûf den plân.\\ 
 & dâ wart \textbf{mit rîten} kunst getân.\\ 
 & sînem gaste er \textbf{den rât} gap,\\ 
30 & wie er daz ros \textbf{ûz dem} walap\\ 
\end{tabular}
\scriptsize
\line(1,0){75} \newline
m n o Fr69 \newline
\line(1,0){75} \newline
\textbf{11} \textit{Initiale} m o Fr69  \newline
\line(1,0){75} \newline
\textbf{2} schein] schin m \textbf{3} heize] heisset n o \textbf{4} enwederz] entweders m n (o) \textbf{5} si] Die o  $\cdot$ kernen] kerne n o \textbf{6} des] Das o \textbf{9} und] \textit{om.} n o \textbf{10} getriuwen man] getruwem manne n (o) \textbf{12} ir] \textit{om.} Fr69  $\cdot$ lernen] leren n o \textbf{13} ritterlîchen] ritterlichem n o \textbf{15} geschouwet] beschouwet n (o)  $\cdot$ manige yant] an úch gewant n (o) \textbf{16} dâ] Do m n o \textbf{18} ez] Er m n o \textbf{19} sullen] solt o \textbf{20} dâ] Do m n o  $\cdot$ künste nâhen] kusche nohen m kunst enpfohen n o \textbf{24} der] Die o  $\cdot$ ieglîcher habe] yeglicher ha habe m  $\cdot$ genomen] vernomen n \textbf{25} sînen] Einen n o \textbf{26} der] Dar m \textbf{27} den] dem n \textbf{28} dâ] Do n o \newline
\end{minipage}
\end{table}
\newpage
\begin{table}[ht]
\begin{minipage}[t]{0.5\linewidth}
\small
\begin{center}*G
\end{center}
\begin{tabular}{rl}
 & man unde wîp, diu sint al ein\\ 
 & \textbf{\textit{al}sam} diu sunne, diu hiute \textbf{dâ} schein,\\ 
 & unt \textit{\textbf{ouch}} der \textbf{mân}, de\textit{r} \textbf{\textit{h}eizet} tac.\\ 
 & der dewederz sich gescheiden mac.\\ 
5 & \textbf{si} blüent ûz einem kerne gar,\\ 
 & des nemet künsticlîchen war."\\ 
 & der gast dem wirte durch \textbf{râten} neic.\\ 
 & sîner muoter er gesweic\\ 
 & mit rede unde in dem herzen niht,\\ 
10 & \begin{large}A\end{large}ls noch \textbf{getriwem} man geschiht.\\ 
 & der wirt sprach sîn êre:\\ 
 & "noch sult ir lernen mêre\\ 
 & kunst an rîterlîchen siten.\\ 
 & wie kômet ir zuo mir geriten!\\ 
15 & ich hân \textbf{beschouwet} manige want,\\ 
 & dâ ich den schilt baz hangen vant,\\ 
 & danne er iu ze halse tæte.\\ 
 & ez ist uns niht ze spæte,\\ 
 & wir sulen ze velde gâhen,\\ 
20 & dâ sult ir künste nâhen.\\ 
 & bringet im sîn ors unde mir daz mîne\\ 
 & unde ieslîchem rîter daz sîne.\\ 
 & \textbf{dar sulen ouch junchêrren} komen,\\ 
 & \textbf{der} \textbf{ieslîcher} habe genomen\\ 
25 & \textbf{einen} \textbf{starken} schaft unde bringe in dar,\\ 
 & der nâch der niwe sî gevar."\\ 
 & sus kom der vürste ûf den plân.\\ 
 & dâ wart \textbf{mit rîtene} kunst getân.\\ 
 & sînem gaste\textit{r} \textbf{\textit{r}âten} gap,\\ 
30 & wierz ors \textbf{ûf den} walap\\ 
\end{tabular}
\scriptsize
\line(1,0){75} \newline
G I O L M Q R Z Fr21 Fr47 \newline
\line(1,0){75} \newline
\textbf{1} \textit{Initiale} I Q  \textbf{7} \textit{Initiale} O R Z Fr21  \textbf{10} \textit{Initiale} G  \textbf{11} \textit{Initiale} L  \textbf{15} \textit{Initiale} Fr47  \textbf{21} \textit{Initiale} I  \newline
\line(1,0){75} \newline
\textbf{1} wîp] wil R  $\cdot$ diu] \textit{om.} O Q Fr21 Fr47  $\cdot$ sint] sin Z \textbf{2} Als am die hútte die sunne schine R  $\cdot$ alsam] sam G Alz L Fr47  $\cdot$ diu] der L  $\cdot$ dâ] \textit{om.} O L M Q R Z Fr21 Fr47  $\cdot$ schein] scheine Z \textbf{3} ouch] \textit{om.} G  $\cdot$ mân] nam R (Z)  $\cdot$ der heizet] der de hezet G hæizet O (L) (M) (Q) heize Fr21 (Fr47) \textbf{4} der] \textit{om.} R  $\cdot$ dewederz] enwederz L (Z) iwedirs M  $\cdot$ mac] kan R \textbf{5} blüent] bluͯmt L bluet Q bluende Fr47  $\cdot$ einem] einen Fr47 \textbf{6} künsticlîchen] kuschlichen M [keuschchen]: kvnstlichen Q \textbf{7} Der gast durch raten dem wirte neic Z  $\cdot$ der] ÷er O  $\cdot$ râten] raute R taten Fr21 \textbf{8} er] er [v]: Gar I \textbf{10} getriwem man] getriwen man O (R) dem getruͯwen man L getruwen manne M getrewē man Q \textbf{12} noch] Nv Fr21 \textbf{13} rîterlîchen] ritterlicher Q Ritterlichem R \textbf{15} ich hân] Jo han ich M ÷ch han Fr47  $\cdot$ beschouwet] beswawet I \textbf{16} dâ] Das Q  $\cdot$ baz] bar M  $\cdot$ hangen] hangende Z (Fr21) \textbf{17} iu] auch Q \textbf{18} ist] enist L (M) (Z) (Fr21)  $\cdot$ niht] noch [nih]: niht Fr47 \textbf{19} gâhen] gehin M \textbf{20} nâhen] fachen R \textbf{21} sîn] das M  $\cdot$ unde] \textit{om.} R \textbf{22} ieslîchem] iegli::: I idem Fr47  $\cdot$ rîter] \textit{om.} L \textbf{23} dar] Das Q \textbf{26} der niwe] [Miwe]: Niwe M \textbf{27} sus] Ausz Q So Fr47  $\cdot$ kom der vürste] komen die fᵫrsten R \textbf{28} dâ] Do Q R  $\cdot$ mit] \textit{om.} Fr21  $\cdot$ rîtene kunst] rittern chvnst O (M) ritterkúnst Q (Fr21) \textbf{29} sînem] Sine M  $\cdot$ gaster râten] gaster do raten G gaste er ze raten O (Fr21) gaste er rat L M gast er rate Q (Z) (Fr47) \textbf{30} ûf den] vf dem I (O) (R) (Fr47) indem M vssen Q vz dem Z \newline
\end{minipage}
\hspace{0.5cm}
\begin{minipage}[t]{0.5\linewidth}
\small
\begin{center}*T
\end{center}
\begin{tabular}{rl}
 & man unde wîp, die sint al ein\\ 
 & \textbf{alsam} der sunne, der hiute schein,\\ 
 & unde der \textbf{name}, der \textbf{heizet} tac.\\ 
 & der dewederz sich gescheiden mac.\\ 
5 & \textbf{die} blüent ûz eime kernen gar,\\ 
 & des nemet künsteclîche war."\\ 
 & \begin{large}D\end{large}er gast dem wirte durch \textbf{vrâgen} neic.\\ 
 & sîner muoter er \textbf{gar} gesweic\\ 
 & mit rede unde imme herzen niht,\\ 
10 & alse noch \textbf{getriuwen} man geschiht.\\ 
 & Der wirt sprach sîn êre:\\ 
 & "noch sult ir lernen mêre\\ 
 & kunst an rîterlîchen siten.\\ 
 & wie kâmet ir ze mir geriten!\\ 
15 & ich hân \textbf{beschouwet} manege want,\\ 
 & dâ ich den schilt baz hangen vant,\\ 
 & danner iu ze halse tæte.\\ 
 & ez ist uns niht ze spæte,\\ 
 & wir suln ze velde gâhen,\\ 
20 & dâ sult ir künste nâhen.\\ 
 & Bringet im sîn ors unde mir daz mîn\\ 
 & unde iegelîchem rîter daz sîn.\\ 
 & \textbf{dar suln ouch junchêrren} komen,\\ 
 & \textbf{unde daz} \textbf{iegelîcher} habe genomen\\ 
25 & \textbf{einen} schaft unde bringin dar,\\ 
 & der nâch der niuwe sî gevar."\\ 
 & \begin{large}S\end{large}us kom der vürste ûf den plân.\\ 
 & dâ wart \textbf{von rîtern} kunst getân.\\ 
 & sînem gaste er \textbf{den rât} gap,\\ 
30 & wier daz ors \textbf{ûz dem} walap\\ 
\end{tabular}
\scriptsize
\line(1,0){75} \newline
T U V W \newline
\line(1,0){75} \newline
\textbf{7} \textit{Initiale} T U W  \textbf{11} \textit{Majuskel} T  \textbf{21} \textit{Majuskel} T  \textbf{27} \textit{Initiale} T U V  \newline
\line(1,0){75} \newline
\textbf{1} die sint] sein W \textbf{2} alsam der sunne der] Als an der suͦnne die U Alsam [d*]: die svnne die V All sam die sunn die W \textbf{3} der name der] [*]: oͮch der namme der V der mone W \textbf{4} dewederz] iewederz U \textbf{5} blüent] bluͦme W \textbf{7} vrâgen] [*]: raten V rat W \textbf{10} getriuwen] getrv́wem V getreúwē W \textbf{13} Kúnstigliche mit ritterscheffte sitten W \textbf{14} kâmet] kuͦnet U kamen W \textbf{15} manege want] [*ge]: manige want V \textbf{16} dâ] Do U W \textbf{20} dâ] Do U W  $\cdot$ nâhen] vahen U enpfahen W \textbf{21} sîn] daz W \textbf{26} der] Dar U  $\cdot$ gevar] gewar U (W) \textbf{28} dâ wart von] Do wart von U [*]: Do wart von V Do ward mit W \textbf{29} den] \textit{om.} W \textbf{30} ûz] [*]: vf V \newline
\end{minipage}
\end{table}
\end{document}
