\documentclass[8pt,a4paper,notitlepage]{article}
\usepackage{fullpage}
\usepackage{ulem}
\usepackage{xltxtra}
\usepackage{datetime}
\renewcommand{\dateseparator}{.}
\dmyyyydate
\usepackage{fancyhdr}
\usepackage{ifthen}
\pagestyle{fancy}
\fancyhf{}
\renewcommand{\headrulewidth}{0pt}
\fancyfoot[L]{\ifthenelse{\value{page}=1}{\today, \currenttime{} Uhr}{}}
\begin{document}
\begin{table}[ht]
\begin{minipage}[t]{0.5\linewidth}
\small
\begin{center}*D
\end{center}
\begin{tabular}{rl}
\textbf{99} & dem \textbf{ist hie} wol gelungen."\\ 
 & \textbf{nâch den} orsen si dô sprungen.\\ 
 & ir wât wart von \textbf{den} ougen naz,\\ 
 & dô si kômen, dâ ir hêrre \textbf{saz}.\\ 
5 & si enpfiengen in, er enpfienc \textbf{ouch} sie.\\ 
 & vreude unt jâmer, \textbf{daz} was \textbf{hie}.\\ 
 & Dô kuster die getriwen.\\ 
 & er sprach: "iuch \textbf{en}sol niht riwen\\ 
 & zunmâzer wîs der bruoder mîn.\\ 
10 & ich mag iuch wol ergetzen sîn.\\ 
 & kêrt ûf den schilt nâch sîner art,\\ 
 & \textbf{gehabt} iuch an der vreuden vart!\\ 
 & ich \textbf{sol} mînes vater wâpen tragen.\\ 
 & sîn lant mîn anker hât beslagen.\\ 
15 & der anker ist ein recken zil,\\ 
 & den \textbf{trage und nem} \textbf{nû}, swer der wil.\\ 
 & \begin{large}I\end{large}ch muoz nû \textbf{lebelîche}\\ 
 & gebâren. ich bin rîche.\\ 
 & wan \textbf{solt} ich volkes hêrre sîn?\\ 
20 & \textbf{den} tæte wê der \textbf{jâmer} mîn.\\ 
 & vrou Herzeloyde, helfet mir,\\ 
 & daz wir \textbf{bitten}, ich unt ir,\\ 
 & künege unt vürsten, die hie sîn,\\ 
 & daz si durch den dienest mîn\\ 
25 & belîben, unz ir mich gewert,\\ 
 & des minnen \textbf{werc} ze\textbf{r} minnen gert."\\ 
 & Die bete warb ir beider munt.\\ 
 & die werden lobtenz \textbf{sâ} zestunt.\\ 
 & ieslîcher vuor an sîn gemach.\\ 
30 & diu künegîn z\textbf{ir vriunde} sprach:\\ 
\end{tabular}
\scriptsize
\line(1,0){75} \newline
D \newline
\line(1,0){75} \newline
\textbf{7} \textit{Majuskel} D  \textbf{17} \textit{Initiale} D  \textbf{27} \textit{Majuskel} D  \newline
\line(1,0){75} \newline
\newline
\end{minipage}
\hspace{0.5cm}
\begin{minipage}[t]{0.5\linewidth}
\small
\begin{center}*m
\end{center}
\begin{tabular}{rl}
 & dem \textbf{hie ist} wol gelungen."\\ 
 & \textbf{nâch de\textit{n}} rossen si dô sprungen.\\ 
 & ir wât wart von \textbf{\textit{i}r} \textit{ou}gen naz,\\ 
 & dô si kômen, d\textit{â} ir hêrre \textbf{saz}.\\ 
5 & si enpfiengen in, er enpfienc sie.\\ 
 & vröude und jâmer, \textbf{daz} was \textbf{hie}.\\ 
 & dô kust er d\textit{ie} getriuwen.\\ 
 & er sprach: "iuch sol niht riuwen\\ 
 & zunmâzer wîs der bruoder mîn.\\ 
10 & ich mac iuch wol erg\textit{e}tzen sîn.\\ 
 & kêret ûf den schilt nâch sîner art,\\ 
 & \textbf{gehabet} iuch an de\textit{r} vröuden vart!\\ 
 & ich \textbf{sol} mînes vater wâpen tragen.\\ 
 & sîn lant mî\textit{n} anke\textit{r} hât beslagen.\\ 
15 & der anker ist ein recken zil,\\ 
 & den \textbf{trage und neme} \textit{\textbf{nû}}, wer der wil.\\ 
 & ich muoz \textit{nû} \textbf{lobelîche}\\ 
 & gebârn. ich bin rîche.\\ 
 & wanne \textbf{solt} ich volkes hêrre sîn?\\ 
20 & \textbf{den} tæte wê der \textbf{jâmer} mîn.\\ 
 & vrouwe Herczeloid\textit{e}, hel\textit{f}et mir,\\ 
 & daz wir \textbf{gebieten}, ich und ir,\\ 
 & \dag küniginne\dag  und vürsten, die hie sî\textit{n},\\ 
 & daz si durch den dienst mîn\\ 
25 & belîben, unz ir mich gewert,\\ 
 & des minnen \textbf{wert} \dag er\dag  minnen gert."\\ 
 & die bete \dag wart\dag  ir beider munt.\\ 
 & die werden lobetenz zestunt.\\ 
 & iegelîcher vuor an sîn gemach.\\ 
30 & diu künigîn zuo \dag iren vriunden\dag  sprach:\\ 
\end{tabular}
\scriptsize
\line(1,0){75} \newline
m n o \newline
\line(1,0){75} \newline
\newline
\line(1,0){75} \newline
\textbf{1} gelungen] gelingen o \textbf{2} den] dem m  $\cdot$ sprungen] springen o \textbf{3} ir ougen] trungen m \textbf{4} dâ] do m n duͯrch do o \textbf{5} enpfiengen] enpfing n o  $\cdot$ in] in vnd n  $\cdot$ enpfienc] enpfing ouch n (o) \textbf{6} daz was] worent n >daz< was o \textbf{7} die] den m \textbf{9} zunmâzer] Zuͯ mossen n (o)  $\cdot$ wîs] wisse n \textbf{10} ergetzen] ergenczen m \textbf{12} gehabet] Behabt n o  $\cdot$ der] den m \textbf{14} mîn anker] mit ancken m \textbf{16} nû] im m  $\cdot$ der] do n o \textbf{17} nû] im m \textbf{20} den] Die o  $\cdot$ der] den o \textbf{21} Herczeloide] herczeloiden m hertzeloide n herczeleide o  $\cdot$ helfet] helset m \textbf{22} \textit{Die Verse 99.22-100.2 fehlen} o  \textbf{23} vürsten] fúrste n  $\cdot$ sîn] sint m \newline
\end{minipage}
\end{table}
\newpage
\begin{table}[ht]
\begin{minipage}[t]{0.5\linewidth}
\small
\begin{center}*G
\end{center}
\begin{tabular}{rl}
 & dem \textbf{sî dâ} wol gelungen."\\ 
 & \textbf{ze} orsen si dô sprungen.\\ 
 & ir wât wart von \textbf{den} ougen naz,\\ 
 & dô si kômen, dâ ir hêrre \textbf{saz}.\\ 
5 & si enpfiengen in, er enpfien\textit{c} \textbf{ouch} si.\\ 
 & vröude und jâmer was \textbf{dâ bî}.\\ 
 & dô kuster die getriwen.\\ 
 & er sprach: "iuch sol niht riwen\\ 
 & ze u\textit{n}mâzer wîse der bruoder mîn.\\ 
10 & ich mac iuch wol ergetzen sîn.\\ 
 & \textit{\begin{large}K\end{large}}êrt ûf den schilt nâch sîner art\\ 
 & \textit{\textbf{und}} \textbf{habet} iuch an der vröuden vart!\\ 
 & ich \textbf{wil} mînes vater wâpen tragen.\\ 
 & sîn lant mîn anker hât beslagen.\\ 
15 & der anker ist ein recken zil,\\ 
 & den \textbf{\textit{trag}e und \textit{neme}}, \textit{s}wer der wil.\\ 
 & ich muoz nû \textbf{lebelîche}\\ 
 & gebâren. ich bin rîche.\\ 
 & wan \textbf{sol} ich volkes hêrre sîn?\\ 
20 & \textbf{dem} tæte wê der \textbf{kumber} mîn.\\ 
 & vrouwe \textit{Herzeloide} helfet mir,\\ 
 & daz wir \textbf{biten}, ich und ir,\\ 
 & künige und vürsten, die hie sîn,\\ 
 & daz si durch den dienst mîn\\ 
25 & belîben, \textit{un}z ir mich gewert,\\ 
 & des minnen \textbf{werc} ze minnen gert."\\ 
 & die bete warb ir beider munt.\\ 
 & die werden lobten ez zestunt.\\ 
 & iegelîcher vuor an sîn gemach.\\ 
30 & diu künigîn z\textbf{ir vriunde} sprach:\\ 
\end{tabular}
\scriptsize
\line(1,0){75} \newline
G I O L M Q R Z Fr36 \newline
\line(1,0){75} \newline
\textbf{7} \textit{Initiale} L  \textbf{11} \textit{Initiale} G I  \textbf{17} \textit{Initiale} L Q R Z  \newline
\line(1,0){75} \newline
\textbf{1} dâ] do Q \textbf{2} ze] Ze den O (L) (Q) (Z) Von den R  $\cdot$ orsen] ors I (Q)  $\cdot$ si dô] si da O (Z) dasie M  $\cdot$ sprungen] sprúngen Q \textbf{3} wât] want Q  $\cdot$ von] do von Q  $\cdot$ den] irn M \textbf{4} dô] Da M Z  $\cdot$ kômen] koment R  $\cdot$ dâ] do L Q  $\cdot$ saz] was I (L) \textbf{5} enpfiengen] giengen I  $\cdot$ enpfienc] enphien G  $\cdot$ ouch] \textit{om.} M R \textbf{6} was] daz was O (L) (M) (R) Z  $\cdot$ dâ bî] hîe O (L) (M) (R) (Z) auch hie Q \textbf{7} dô] Da M Z  $\cdot$ kuster] kust er I (O) R Z  $\cdot$ getriwen] truwin M \textbf{8} sol] en sal M (R) \textbf{9} ze unmâzer wîse] zevmazer wise G zunmazzen I Ze vnmaze O (Q) Zvm maszer wisz L Zcu maszer wise M Zu mausze R \textbf{11} Kêrt] Echert G Ker R \textbf{12} und] \textit{om.} G  $\cdot$ habet] halt M  $\cdot$ an] nach L (M)  $\cdot$ vart] art L R \textbf{14} lant] hand R  $\cdot$ beslagen] geslagen Q \textbf{15} ein] eyns e M \textbf{16} trage und neme swer] neme vnde trage nv swer G trage vnde neme in swer O trage vnd neme wer L (M) (Z) trag vnd neme euch wer Q trag vnd nem in wer R  $\cdot$ der] dir I da M R Z do Q \textbf{17} lebelîche] lobeliche M \textbf{18} gebâren] Geúaren Q Gebarn wan Z  $\cdot$ bin] ben nuͯ L \textbf{19} sol] sold O (L) (M) (Q) (R) Z  $\cdot$ volkes] landes I valschir M des volkes Q \textbf{20} kumber] iamer O (L) (M) Q (R) Z \textbf{21} Herzeloide] chungin G herzenlaude I (O) Hertzelauͯde L herczeloide M herzeloude Q herczelaude R herzenlovde Z  $\cdot$ helfet] heffet R nu helf Z \textbf{22} wir] ir wir M \textbf{23} künige] Konnigin M  $\cdot$ hie sîn] hieszin M \textbf{25} unz] biz G \textbf{26} des minnen] Des mýn L Des mynne M Des minem R die minen Fr36  $\cdot$ werc] wert M were Q  $\cdot$ ze minnen] ze minne O zuͯ der mynne L (M) (Z) ze minem R \textbf{27} bete] \textit{om.} O  $\cdot$ warb] erwarb O (L) \textbf{28} werden lobten ez] werde labetin M  $\cdot$ zestunt] sa ze stvnt O (L) (M) (Z) (Fr36) [sa]: al zestúnd Q da zustund R \textbf{29} iegelîcher] Zeslicher Fr36  $\cdot$ sîn] sinen O (Q) Z \textbf{30} zir] zun R  $\cdot$ vriunde] vrivnden O (R) (Z) \newline
\end{minipage}
\hspace{0.5cm}
\begin{minipage}[t]{0.5\linewidth}
\small
\begin{center}*T (U)
\end{center}
\begin{tabular}{rl}
 & dem \textbf{sî dâ} wol gelungen."\\ 
 & \textbf{zuo} orsen si dô sprungen.\\ 
 & ir wât wart von \textbf{den} ougen naz,\\ 
 & dô si kâmen, d\textit{â} ir hêrre \textbf{was}.\\ 
5 & si entviengen in, er entvienc \textbf{ouch} sie.\\ 
 & vreude und jâmer, \textbf{daz} was \textbf{hie}.\\ 
 & dô kuster die getriuwen.\\ 
 & er sprach: "\textbf{ouch} iuch sol niht riuwen\\ 
 & zuo unmâzer wîs der bruoder mîn.\\ 
10 & ich mac iuch wol ergetzen sîn.\\ 
 & kêret ûf den schilt nâch sîner art\\ 
 & \textbf{und} \textbf{habt} iuch an der vreude\textit{n} \textit{v}art!\\ 
 & ich \textbf{wil} mînes vater wâpen tragen.\\ 
 & sîn lant mîn anker hât beslagen.\\ 
15 & der anker ist ein r\textit{e}c\textit{k}en zil,\\ 
 & den \textbf{n\textit{e}m und trage} \textbf{nû}, wer der wil.\\ 
 & ich muoz nû \textbf{le\textit{be}lîche}\\ 
 & gebâren. ich bin rîche.\\ 
 & wan \textbf{solt} ich volkes hêrre sîn?\\ 
20 & \textbf{dem} tæte wê der \textbf{jâmer} mîn.\\ 
 & \begin{large}V\end{large}rou Herzeloyde, \textbf{nû} helfet mir,\\ 
 & daz wir \textbf{biten}, ich und ir,\\ 
 & künege und vürsten, die hie sîn,\\ 
 & daz si durch den dienst mîn\\ 
25 & blîben, u\textit{nz} ir mich gewert,\\ 
 & des minne \textbf{wert} ze\textbf{r} minne gert."\\ 
 & die bete warp ir beider munt.\\ 
 & die werden lobeten\textit{z} \textbf{sô} zuostunt.\\ 
 & ieclîcher vuor an sîn gemach.\\ 
30 & diu küniginne zuo\textbf{me gaste} sprach:\\ 
\end{tabular}
\scriptsize
\line(1,0){75} \newline
U V W T \newline
\line(1,0){75} \newline
\textbf{2} \textit{Majuskel} T  \textbf{5} \textit{Majuskel} T  \textbf{7} \textit{Majuskel} T  \textbf{17} \textit{Initiale} W  \textbf{21} \textit{Initiale} U V  \textbf{27} \textit{Initiale} T  \textbf{28} \textit{Majuskel} T  \textbf{29} \textit{Majuskel} T  \textbf{30} \textit{Majuskel} T  \newline
\line(1,0){75} \newline
\textbf{1} dâ] do W \textbf{2} zuo] [*]: von den V Zen T  $\cdot$ orsen] rosse W \textbf{3} jr [ôu*]: oûgen wurden von regene naz T \textbf{4} dâ] do U V W  $\cdot$ was] saz T \textbf{5} er entvienc ouch] sam tet er T \textbf{7} kuster] kust er V W \textbf{8} ouch] \textit{om.} V W T  $\cdot$ sol] ensol V \textbf{10} mac] wil V sol W \textbf{12} habt] heben W  $\cdot$ vreuden vart] vreide hart U froͤden [*art]: vart V \textbf{14} hât] han W \textbf{15} ein] eins T  $\cdot$ recken] reichen U \textbf{16} nem und trage] nim vnd trage U tragen vnde neme T  $\cdot$ nû] \textit{om.} W  $\cdot$ wer] swer V (T)  $\cdot$ der] do W \textbf{17} lebelîche] leliche U \textbf{19} wan] \textit{om.} T  $\cdot$ volkes] vockes W minres volkes T \textbf{20} dem] den V (W) T \textbf{21} Herzeloyde] Herzeleide U Hertzelaude V hertzeloyde W \textbf{22} wir] beidiv T \textbf{23} und] \textit{om.} V W  $\cdot$ die] vnd die W \textbf{25} unz] vnd zuͦ U \textbf{26} minne wert] minne werg V minnen werck W (T)  $\cdot$ gert] gernt V \textbf{27} die] Div T  $\cdot$ warp] [*]: erwarp V \textbf{28} lobetenz] lobeten U \textbf{30} zuome gaste] zuͦ ir [*]: frúnde V zir vrivnde T \newline
\end{minipage}
\end{table}
\end{document}
