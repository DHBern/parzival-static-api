\documentclass[8pt,a4paper,notitlepage]{article}
\usepackage{fullpage}
\usepackage{ulem}
\usepackage{xltxtra}
\usepackage{datetime}
\renewcommand{\dateseparator}{.}
\dmyyyydate
\usepackage{fancyhdr}
\usepackage{ifthen}
\pagestyle{fancy}
\fancyhf{}
\renewcommand{\headrulewidth}{0pt}
\fancyfoot[L]{\ifthenelse{\value{page}=1}{\today, \currenttime{} Uhr}{}}
\begin{document}
\begin{table}[ht]
\begin{minipage}[t]{0.5\linewidth}
\small
\begin{center}*D
\end{center}
\begin{tabular}{rl}
\textbf{336} & \begin{large}E\end{large}ckuba, diu junge,\\ 
 & vuor gein ir schiffunge,\\ 
 & ich meine \textbf{die rîchen} heidenîn.\\ 
 & dô kêrte manegen \textbf{ende} hin\\ 
5 & daz volc von \textbf{dem} Plimizœl.\\ 
 & Artus vuor gein Karidol.\\ 
 & Cunneware unt Clamide,\\ 
 & \textbf{die} nâmen ouch sînen urloub ê.\\ 
 & Orilus, der vürste erkant,\\ 
10 & unt vrou Jeschute von Karnant,\\ 
 & \textbf{die} nâmen ouch sînen urloub sân,\\ 
 & \textbf{doch} beliben si ûf dem plân\\ 
 & \textbf{bî} Clamide den dritten tac,\\ 
 & wander \textbf{der} brûtloufte pflac,\\ 
15 & niht mit benanter hôchgezît.\\ 
 & si wart dâ heime grœzer sît,\\ 
 & wand im sîn milte daz geriet.\\ 
 & vil ritter, \textbf{kumberhaftiu} diet,\\ 
 & \textbf{beleib} \textbf{in} Cl\textit{a}mides schar\\ 
20 & unt ouch daz varende volc \textbf{vil} gar,\\ 
 & die vuorter heim ze lande.\\ 
 & mit êren, âne schande\\ 
 & wart in geteilet dâ sîn habe,\\ 
 & mit valsche niht \textbf{gewîset} abe.\\ 
25 & \textbf{Dô} vuor vrou Jeschute\\ 
 & mit Orilus, ir trûte,\\ 
 & durch Clamiden ze Brandigan.\\ 
 & daz wart z\textbf{einen} êren getân\\ 
 & \textbf{vroun} Cunnewaren, der \textbf{künegîn}:\\ 
30 & dâ krônte man die \textbf{swester sîn}.\\ 
\end{tabular}
\scriptsize
\line(1,0){75} \newline
D \newline
\line(1,0){75} \newline
\textbf{1} \textit{Initiale} D  \textbf{25} \textit{Majuskel} D  \newline
\line(1,0){75} \newline
\textbf{1} Eckuba] Ekvba D \textbf{5} Plimizœl] Plimizol D \textbf{7} Clamide] Chlamidê D \textbf{10} Jeschute] Jescvte D \textbf{13} Clamide] Chlamide D \textbf{19} Clamides] Chlmides D \textbf{25} Jeschute] Jescv̂te D \textbf{27} Clamiden] Chlamiden D \newline
\end{minipage}
\hspace{0.5cm}
\begin{minipage}[t]{0.5\linewidth}
\small
\begin{center}*m
\end{center}
\begin{tabular}{rl}
 & E\textit{c}uba, diu junge,\\ 
 & vuor ge\textit{g}en ir schiffunge,\\ 
 & ich meine \textbf{die rîchen} heidenîn.\\ 
 & dô kêrte \textbf{an} manigen \textbf{enden} hin\\ 
5 & daz volc von Plimizol.\\ 
 & Artus vuor gegen Karid\textit{o}l.\\ 
 & C\textit{unn}ew\textit{a}re und Cla\textit{m}ide,\\ 
 & \textbf{die} nâmen ouch sîn urloup ê.\\ 
 & Orilus, der vürste erkant,\\ 
10 & und vrouwe Jeschute von Karnant,\\ 
 & \textbf{die} nâmen ouch sîn urloup sân,\\ 
 & \textbf{iedoch} beliben si ûf dem plân\\ 
 & \textbf{bî} Clamide den dritten tac,\\ 
 & want er \textbf{der} brûtlouf pflac,\\ 
15 & niht mit benanter hôchgezît.\\ 
 & si wart dâ heime grœzer sît,\\ 
 & want ime sîn milte daz geriet.\\ 
 & vil ritter \textbf{und} \textbf{kumberhafter} diet\\ 
 & \textbf{beleip} \textbf{an} Clamides schar\\ 
20 & und ouch daz varende volc gar,\\ 
 & die vuort er heim ze lande.\\ 
 & mit êren, âne schande\\ 
 & wart in geteilet d\textit{â} sîn habe,\\ 
 & mit valsche niht \textbf{gewîset} abe.\\ 
25 & \textbf{dô} vuor vrouwe Jeschute\\ 
 & mit Oriluse, ir trûte,\\ 
 & durch Clamiden ze Brandigan.\\ 
 & daz wart ze \textbf{einen} êren getân\\ 
 & \textbf{vrouwen} Cu\textit{nne}w\textit{a}ren, der \textbf{künigîn}:\\ 
30 & d\textit{â} krônde man die \textbf{swester sîn}.\\ 
\end{tabular}
\scriptsize
\line(1,0){75} \newline
m n o \newline
\line(1,0){75} \newline
\newline
\line(1,0){75} \newline
\textbf{1} Ecuba] Etuͯba m \textbf{2} gegen] geben m \textbf{3} rîchen] riche n o \textbf{5} Plimizol] dem plumzol n dem [plimil]: plimizol o \textbf{6} gegen] gon n [*]: gan o  $\cdot$ Karidol] caridal m caridol n karidal o \textbf{7} Cunneware] Comewere m Conneware n Conne waren o  $\cdot$ Clamide] clanide m \textbf{8} sîn] ir n den o  $\cdot$ ê] hie o \textbf{10} Jeschute] jescutte m jescute n o  $\cdot$ Karnant] kornant n o \textbf{11} ouch] uch o  $\cdot$ sîn] den n o \textbf{15} hôchgezît] hochzit n \textbf{17} geriet] geriht o \textbf{18} ritter] rihter o  $\cdot$ kumberhafter] kumberhaffte n (o) \textbf{23} dâ] do m n o \textbf{25} Jeschute] jescute m n o \textbf{26} Oriluse] orilus n o \textbf{27} Clamiden] clamide n o \textbf{29} vrouwen] Frouwe m (n) (o)  $\cdot$ Cunnewaren] cumuweren m conwaren n Conne waren o  $\cdot$ der] die o \textbf{30} dâ] Do m n o \newline
\end{minipage}
\end{table}
\newpage
\begin{table}[ht]
\begin{minipage}[t]{0.5\linewidth}
\small
\begin{center}*G
\end{center}
\begin{tabular}{rl}
 & \multicolumn{1}{l}{ - - - }\\ 
 & \multicolumn{1}{l}{ - - - }\\ 
 & \multicolumn{1}{l}{ - - - }\\ 
 & \multicolumn{1}{l}{ - - - }\\ 
5 & \multicolumn{1}{l}{ - - - }\\ 
 & \multicolumn{1}{l}{ - - - }\\ 
 & \multicolumn{1}{l}{ - - - }\\ 
 & \multicolumn{1}{l}{ - - - }\\ 
 & \multicolumn{1}{l}{ - - - }\\ 
10 & \multicolumn{1}{l}{ - - - }\\ 
 & \multicolumn{1}{l}{ - - - }\\ 
 & \multicolumn{1}{l}{ - - - }\\ 
 & \multicolumn{1}{l}{ - - - }\\ 
 & \multicolumn{1}{l}{ - - - }\\ 
15 & \multicolumn{1}{l}{ - - - }\\ 
 & \multicolumn{1}{l}{ - - - }\\ 
 & \multicolumn{1}{l}{ - - - }\\ 
 & \multicolumn{1}{l}{ - - - }\\ 
 & \multicolumn{1}{l}{ - - - }\\ 
20 & \multicolumn{1}{l}{ - - - }\\ 
 & \multicolumn{1}{l}{ - - - }\\ 
 & \multicolumn{1}{l}{ - - - }\\ 
 & \multicolumn{1}{l}{ - - - }\\ 
 & \multicolumn{1}{l}{ - - - }\\ 
25 & \multicolumn{1}{l}{ - - - }\\ 
 & \multicolumn{1}{l}{ - - - }\\ 
 & \multicolumn{1}{l}{ - - - }\\ 
 & \multicolumn{1}{l}{ - - - }\\ 
 & \multicolumn{1}{l}{ - - - }\\ 
30 & \multicolumn{1}{l}{ - - - }\\ 
\end{tabular}
\scriptsize
\line(1,0){75} \newline
G I O L M Q R Z Fr39 \newline
\line(1,0){75} \newline
\textbf{1} \textit{Initiale} R Z  \newline
\line(1,0){75} \newline
\textbf{1} \textit{Die Verse 336.1-337.30 fehlen} G I O L M Fr39   $\cdot$ Ekuba die iûnge Q (R) (Z) \textbf{2} Fur gen ir schiffunge (stiffunge Z ) Q (R) (Z) \textbf{3} Jch mein die rechten (Richen R riche Z ) heidenin (hedin R ) Q (R) (Z) \textbf{4} Do (Da Z ) kerten (kerte Z ) manchen enden (menge ende R ) hin Q (R) (Z) \textbf{5} Das volck von dem plimizol (plimuzol R ) Q (R) (Z) \textbf{6} Artus fur gen karidol (kardol R ) Q (R) (Z) \textbf{7} Conware (Cuͦwarte R Cvnneware Z ) vnd clamide Q (R) (Z) \textbf{8} Die (\textit{om.} R Z ) namen auch seinen (ir R ) vrlaup e Q (R) (Z) \textbf{9} Orilus der furste erkant Q (R) Z \textbf{10} Vnd fraw iescute (Jscute R Jescute Z ) dekarnant (von karnant R Z ) Q (R) (Z) \textbf{11} Die namen seinen (och ir R ouch sinen Z ) vrlaup san Q (R) (Z) \textbf{12} Ydoch (Doch R ) plibens (blibent R ) vff dem plan Q (R) (Z) \textbf{13} Bey dem (\textit{om.} R Z ) clamide den tritten dack Q (R) (Z) \textbf{14} Wann er der (da Z ) brutelufte pflac Q (R) (Z) \textbf{15} Nicht mit genanter (benanter R Z ) hochtzeit Q (R) (Z) \textbf{16} Sie wart do heime (heiman R ) groszer (\textit{om.} R ) sit Q (R) (Z) \textbf{17} Wann in (im R Z ) sein milte das geriet Q (R) (Z) \textbf{18} Vil ritter kummerhaffte diet Q (Z)  $\cdot$ Sin kumerhaff:e Ritterliche diet R \textbf{19} Bleib an clamides schar Q (R) (Z) \textbf{20} Vnd auch das (des R ) varnde (werden R ) volk vil gar Q (R) (Z) \textbf{21} Die furt er heim zu lande Q (R) (Z) \textbf{22} Mit eren ane schande Q (R) Z \textbf{23} Wart in (Im do R ) geteilt do (\textit{om.} R da Z ) sein habe Q (R) (Z) \textbf{24} Mit falsche nicht gewiset (gewisset R ) abe Q (R) (Z) \textbf{25} Do (Da Z ) fur fraw jescute (Jscute R ) Q (R) (Z) \textbf{26} Mit orilus ir trawte Q (R) (Z) \textbf{27} Durch clamide (Clamiden R [ Z ]) gen brandian (ze Brandigan R [ Z ]) Q (R) (Z) \textbf{28} Das wart zu einen eren gethan Q (Z)  $\cdot$ Das ward Jnen zu erre getan R \textbf{29} Frawen conwaren (kvnewaren Z ) der konigin Q (Z)  $\cdot$ Frow Cuͦnartten die schwester sin R \textbf{30} Do (Da Z ) cronte man die swester sein Q (R) (Z) \newline
\end{minipage}
\hspace{0.5cm}
\begin{minipage}[t]{0.5\linewidth}
\small
\begin{center}*T
\end{center}
\begin{tabular}{rl}
 & \begin{large}E\end{large}ckuba, diu junge,\\ 
 & vuor gegen ir schiffunge,\\ 
 & ich meine, \textbf{di\textit{u} rîche} heidenîn.\\ 
 & dô kêrte \textbf{in} manegen \textbf{enden} hin\\ 
5 & daz volc von \textbf{dem} Plymizol.\\ 
 & Artus vuor gegen Karidol.\\ 
 & Cunnewar unde Clamide\\ 
 & nâmen ouch sînen urloub ê.\\ 
 & Orilus, der vürste erkant,\\ 
10 & unde vrou Jeschute von Garnant\\ 
 & nâmen ouch sînen urloup sân,\\ 
 & \textbf{doch} blibens ûf dem plân\\ 
 & \textbf{durch} Clamide den dritten tac,\\ 
 & wand er \textbf{dâ} brûtloufte pflac,\\ 
15 & niht mit benanter hôchgezît.\\ 
 & si wart dâ heime grœzer sît,\\ 
 & wand im sîn milte daz geriet.\\ 
 & vil rîter, \textbf{kumberhaft\textit{iu}} diet,\\ 
20 & \hspace*{-.7em}\big| unde ouch daz varnde volc gar\\ 
 & \hspace*{-.7em}\big| \textbf{bliben} \textbf{an} Clamides schar,\\ 
 & die vuort er heim ze lande.\\ 
 & mit êren, âne schande\\ 
 & Wart in geteilet dâ sîn habe,\\ 
 & mit valsche niht \textbf{gelenet} abe.\\ 
25 & \textbf{ouch} vuor vrou Jeschute\\ 
 & mit Orilus, ir trûte,\\ 
 & durch Clamide ze Brandigan.\\ 
 & daz wart zêren getân\\ 
 & Cunnewarn, der \textbf{swester sîn}:\\ 
30 & dâ krônde man die \textbf{künegîn}.\\ 
\end{tabular}
\scriptsize
\line(1,0){75} \newline
T U V W \newline
\line(1,0){75} \newline
\textbf{1} \textit{Initiale} T U W  \textbf{23} \textit{Majuskel} T  \newline
\line(1,0){75} \newline
\textbf{1} Eckuba] Eccuba U [Eckvb]: Eckvba V HEckuba W \textbf{3} diu rîche] die riche T die reichen W  $\cdot$ heidenîn] heidemin U \textbf{4} in manegen enden] an mangem ende W \textbf{5} Plymizol] plẏmizol V plimizol W \textbf{7} Cunnewar] kvnnewar T V (W) Kuͦnnewar U  $\cdot$ Clamide] Clamidê T klamide W \textbf{8} sînen] ir W \textbf{10} Jeschute] Jescvte T (U) [*]: Jescute  V iestute W  $\cdot$ Garnant] [Gar]: Garnant V \textbf{11} sînen] sein W \textbf{12} blibens] bliben V \textbf{13} Clamide] Clamiden U (V) klamide W \textbf{14} er] der U  $\cdot$ dâ] do U V W \textbf{15} hôchgezît] hochzeit W \textbf{16} dâ heime] dekeime U keine W \textbf{18} kumberhaftiu] kvmberhafte T kvmberhafter V (W) \textbf{20} gar] vil gar W \textbf{19} Clamides] klamides W \textbf{21} die] Do W \textbf{23} dâ] do V W \textbf{24} gelenet] geleidet U (W) gewiset V \textbf{25} ouch] Vnd W  $\cdot$ vuor] vuͦrte U  $\cdot$ Jeschute] Jescvte T (U) iescute V iestute W \textbf{26} Orilus] Oriluse U orilo W \textbf{27} Clamide] klamiden W  $\cdot$ Brandigan] Brandigân T [brandi*]: brandigan V \textbf{28} zêren] zuͦ einen eren W \textbf{29} Cunnewarn] kvnnewarn T (W) Kuͦnnewaren U Cvnnewaren V  $\cdot$ der] die U \textbf{30} dâ] Do U W  $\cdot$ krônde] kraoͤte W \newline
\end{minipage}
\end{table}
\end{document}
