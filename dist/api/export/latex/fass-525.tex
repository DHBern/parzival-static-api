\documentclass[8pt,a4paper,notitlepage]{article}
\usepackage{fullpage}
\usepackage{ulem}
\usepackage{xltxtra}
\usepackage{datetime}
\renewcommand{\dateseparator}{.}
\dmyyyydate
\usepackage{fancyhdr}
\usepackage{ifthen}
\pagestyle{fancy}
\fancyhf{}
\renewcommand{\headrulewidth}{0pt}
\fancyfoot[L]{\ifthenelse{\value{page}=1}{\today, \currenttime{} Uhr}{}}
\begin{document}
\begin{table}[ht]
\begin{minipage}[t]{0.5\linewidth}
\small
\begin{center}*D
\end{center}
\begin{tabular}{rl}
\textbf{525} & "\begin{large}S\end{large}waz dort geschach - dû stêst nû hie.\\ 
 & \textbf{dû hœrest} \textbf{ouch} \textbf{vor dir} sprechen ie,\\ 
 & swer dem andern half, daz er genas,\\ 
 & daz er \textbf{sîn vîent dâ nâch} was.\\ 
5 & ich tuon als die bî witzen sint.\\ 
 & sich vüeget baz, ob weinet ein kint\\ 
 & denn ein bartohter man.\\ 
 & ich wil diz ors al eine hân."\\ 
 & Mit sporn er\textbf{z} vaste von im reit.\\ 
10 & daz was \textbf{doch} Gawane leit.\\ 
 & \textbf{Der} sprach \textbf{zer vrouwen}: "ez kom \textbf{alsô}:\\ 
 & der künec Artus, \textbf{der} was dô\\ 
 & in der stat ze Dianazdrun,\\ 
 & \textbf{mit} im dâ \textbf{manec} Bertun.\\ 
15 & \textbf{dem} \textbf{was} ein \textbf{vrouwe} dar gesant\\ 
 & durch botschaft in sîn lant.\\ 
 & Ouch was \textbf{der} ungehiure\\ 
 & ûz komen durch âventiure.\\ 
 & er was gast unt si gestîn.\\ 
20 & dô \textbf{geriet} im sîn kranker sin,\\ 
 & daz er mit der vrouwen ranc\\ 
 & nâch sînem willen, ân ir danc.\\ 
 & hin ze hove kom daz geschrei.\\ 
 & der künec rief \textbf{lûte}: 'heiâ hei!'\\ 
25 & \textbf{diz} geschach vor einem walde;\\ 
 & dar \textbf{gâhten} wir \textbf{alle} balde.\\ 
 & ich vuor den andern verre vor\\ 
 & unt \textbf{begreif} des schuldehaften spor.\\ 
 & gevangen vuort ich \textbf{wider dan}\\ 
30 & vür den künec \textbf{disen man}.\\ 
\end{tabular}
\scriptsize
\line(1,0){75} \newline
D Fr11 \newline
\line(1,0){75} \newline
\textbf{1} \textit{Initiale} D Fr11  \textbf{9} \textit{Majuskel} D  \textbf{11} \textit{Majuskel} D  \textbf{17} \textit{Majuskel} D  \newline
\line(1,0){75} \newline
\textbf{2} hœrest] hortz ez Fr11 \textbf{13} Dianazdrun] Dyanazdrvn D \newline
\end{minipage}
\hspace{0.5cm}
\begin{minipage}[t]{0.5\linewidth}
\small
\begin{center}*m
\end{center}
\begin{tabular}{rl}
 & "waz dort geschach - d\textit{û} stâst nû hie.\\ 
 & \textbf{nû hôrtestû} \textbf{doch} sprech\textit{en} ie,\\ 
 & wer dem andern hal\textit{f}, daz er genas,\\ 
 & daz er \textbf{dâ nâch sîn vîent} was.\\ 
5 & ich tuon als die bî witzen sint.\\ 
 & sich vüeget baz, ob weinet ein kint\\ 
 & dan ein bart\textit{o}hter man.\\ 
 & ich wil diz ros alein hân."\\ 
 & mit sporn er vaste von im r\textit{ei}t.\\ 
10 & daz was \textbf{doch} Gawanen leit.\\ 
 & \textbf{\begin{large}G\end{large}awan} sprach: "ez kam \textbf{alsô}:\\ 
 & der künic Artus, \textbf{der} was dô\\ 
 & in der stat zuo Dianazdr\textit{un},\\ 
 & \textbf{mit} im d\textit{â} \textbf{manigen} Britu\textit{n}.\\ 
15 & \textbf{dô} \textbf{wart} ein \textbf{vrouwe} dar gesant\\ 
 & durch botschaft in sîn lant.\\ 
 & ouch was \textbf{diser} ungehiure\\ 
 & ûz komen durch âventiure.\\ 
 & er was gast und si gestîn.\\ 
20 & dô \textbf{geriet} im sîn kranker sin,\\ 
 & daz er mit der vrouwen ranc\\ 
 & nâch sînem willen, ân\textit{e} ir danc.\\ 
 & hin zuo hove kam daz geschrei.\\ 
 & der künic rie\textit{f}: 'heiâ hei!'\\ 
25 & \textbf{diz} geschach vor einem walde;\\ 
 & dar \textbf{gâheten} wir \textbf{al} balde.\\ 
 & ich vuor den andern verre vor\\ 
 & und \textbf{begreif} des schuldehaften spor.\\ 
 & gevangen vuorte ich \textbf{wider dan}\\ 
30 & vür den künic \textbf{disen man}.\\ 
\end{tabular}
\scriptsize
\line(1,0){75} \newline
m n o \newline
\line(1,0){75} \newline
\textbf{11} \textit{Initiale} m  \newline
\line(1,0){75} \newline
\textbf{1} dû] do m \textbf{2} sprechen] sprech m \textbf{3} half] halft m \textbf{4} vîent] wint o \textbf{7} bartohter] barthehtter m bartherter n o \textbf{9} er] ers o  $\cdot$ reit] riet m \textbf{12} Artus] artuͯs o \textbf{13} Dianazdrun] [dianazdrum]: dianazdrẏm m diananczẏm o \textbf{14} dâ] do m n o  $\cdot$ Britun] brittuͯm m britẏm o \textbf{19} si] sin o \textbf{20} kranker] krancken o \textbf{22} âne] onen m \textbf{24} rief] ries m  $\cdot$ heiâ] hin o \textbf{25} diz] Das n o \textbf{26} al] alle n \textbf{27} vuor] fuͯre n \textbf{28} des schuldehaften] das schuͯlde haffte o \newline
\end{minipage}
\end{table}
\newpage
\begin{table}[ht]
\begin{minipage}[t]{0.5\linewidth}
\small
\begin{center}*G
\end{center}
\begin{tabular}{rl}
 & "\textit{\begin{large}S\end{large}}waz dort geschach - dû stêst nû hie.\\ 
 & \textbf{dû hôrtest} \textbf{ouch} \textbf{vor dir} sprechen ie,\\ 
 & swer dem andern half, daz er genas,\\ 
 & daz er \textbf{sîn vîent dar nâch} was.\\ 
5 & ich tuon als die bî witzen sint.\\ 
 & sich vüeget baz, ob weint ein kint\\ 
 & danne ein ba\textit{r}tohter man.\\ 
 & ich wil ditze ors al eine hân."\\ 
 & mit sporn er\textbf{z} vaste von im reit.\\ 
10 & daz was \textbf{doch} Gawanen leit.\\ 
 & \textbf{er} sprach \textbf{zer vrouwen}: "ez kom \textbf{alsô}:\\ 
 & der künic Artus, \textbf{der} was dô\\ 
 & in der stat ze Dianazdrun,\\ 
 & \textbf{mit} im dâ \textbf{manic} Britun.\\ 
15 & \textbf{dem} \textbf{was} ein \textbf{vrouwe} dar gesant\\ 
 & durch botschaft in sîn lant.\\ 
 & ouch was \textbf{dirre} ungehiure\\ 
 & ûz komen durch âventiure.\\ 
 & er was gast unde si gestîn.\\ 
20 & dô \textbf{geriet} im sîn kranker sin,\\ 
 & daz er mit de\textit{r} vrouwen ranc\\ 
 & nâch sîne\textit{m} willen, ân ir danc.\\ 
 & hin ze hove ko\textit{m} daz geschrei.\\ 
 & der künic rief \textbf{lûte}: 'heiâ hei!'\\ 
25 & \textbf{daz} geschach vor einem walde;\\ 
 & dar \textbf{gâhten} wir \textbf{alle} balde.\\ 
 & ich vuor den andern verre vor\\ 
 & \textit{und} \textbf{begreif} des schuldehaften spor.\\ 
 & gevangen \textit{vuorte ich} \textbf{wider dan}\\ 
30 & vür den künic \textbf{disen man}.\\ 
\end{tabular}
\scriptsize
\line(1,0){75} \newline
G I L M Z Fr28 Fr62 \newline
\line(1,0){75} \newline
\textbf{1} \textit{Initiale} G I L Z Fr62  \textbf{5} \textit{Initiale} M  \textbf{17} \textit{Initiale} I  \newline
\line(1,0){75} \newline
\textbf{1} Swaz] Wwaz G Waz L (M)  $\cdot$ stêst] bist Fr62 \textbf{2} vor dir] \textit{om.} I L wol Fr62 \textbf{3} swer] Wer L M \textbf{4} Das her dar noch sin vient was M  $\cdot$ daz er dar ::: Fr28 \textbf{6} vüeget] vugte I \textbf{7} bartohter] barhtohter G berherter L berthechter M barthafter Fr28 \textbf{9} von] vor I \textit{om.} M \textbf{10} Gawanen] Gawan I (Z) Gawane L (M) \textbf{12} Artus] Artuͯs L  $\cdot$ der was dô] was alda M \textbf{13} ze Dianazdrun] zedianazdrvn G zedenazdrun I zcu dianzcedruͯn M zv Dyanazdrvn Z :::dianansdrvͦn Fr28 \textbf{14} mit im dâ] vnd mit im I  $\cdot$ Britun] pritun I [Britt*n]: Brittvn L [bricvn]: britvn M \textbf{19} si] sý waz L \textbf{20} dô] Da M Z  $\cdot$ geriet] rýet L \textbf{21} der] den G \textbf{22} sînem] sinen G \textbf{23} kom] chome G \textbf{24} rief lûte] lute rief I  $\cdot$ hei] ey M \textbf{25} daz] Disz L M (Z)  $\cdot$ geschach] Gesach I \textbf{26} dar] do I  $\cdot$ gâhten] gahte Z  $\cdot$ alle balde] albalde M \textbf{28} und] Ih G  $\cdot$ des] den M \textbf{29} vuorte ich] ih foͮrte G  $\cdot$ wider] in wider I \textbf{30} [vor]: vvr den selben kunc dem man I \newline
\end{minipage}
\hspace{0.5cm}
\begin{minipage}[t]{0.5\linewidth}
\small
\begin{center}*T
\end{center}
\begin{tabular}{rl}
 & "Swaz dort geschach - dû stêst nû hie.\\ 
 & \textbf{dû hôrtes} \textbf{vor dir} sprechen ie,\\ 
 & swer dem andern half, daz er genas,\\ 
 & daz er \textbf{sîn vîent dâ nâch} was.\\ 
5 & Ich tuon als die bî \textit{wi}tzen sint.\\ 
 & sich vüeget baz, ob weinet ein kint\\ 
 & danne ein \textit{b}artohter man.\\ 
 & ich wil diz ors aleine hân."\\ 
 & mit sporn er\textbf{z} vaste von im reit.\\ 
10 & daz was \textbf{iedoch} Gawane leit.\\ 
 & \textbf{\textit{\begin{large}E\end{large}}r} sprach \textbf{zer vrouwen}: "ez kom \textbf{sô},\\ 
 & \textbf{daz} der künec Artus was dô\\ 
 & in der stat ze Dyanarsun,\\ 
 & \textbf{bî} im dâ \textbf{manec} Britun.\\ 
15 & \textbf{dem} \textbf{was} ein \textbf{maget} dar gesant\\ 
 & durch botschaft in sîn lant.\\ 
 & ouch was \textbf{dirre} ungehiure\\ 
 & ûz komen durch âventiure.\\ 
 & er wa\textit{s} gast unde si gestîn.\\ 
20 & dô \textbf{riet} im sîn kranker sin,\\ 
 & daz er mit der vrouwen ranc\\ 
 & nâch sînem willen, \textbf{gar} ân ir danc.\\ 
 & hin ze hove kom daz geschrei.\\ 
 & der künec rief \textbf{lûte}: 'heiâ hei!'\\ 
25 & \textbf{diz} geschach vor eine\textit{m} walde;\\ 
 & dar \textbf{kêrte} wir \textbf{al}balde.\\ 
 & ich vuor den andern verre vor\\ 
 & unde \textbf{ergreif} des schuldehaften spor.\\ 
 & gevangen vuortich \textbf{disen man}\\ 
30 & vür den künec \textbf{under dan}.\\ 
\end{tabular}
\scriptsize
\line(1,0){75} \newline
T U V W O Q R Fr40 \newline
\line(1,0){75} \newline
\textbf{1} \textit{Initiale} O Fr40   $\cdot$ \textit{Majuskel} T  \textbf{5} \textit{Majuskel} T  \textbf{11} \textit{Initiale} T U V W  \textbf{13} \textit{Überschrift:} [*]: Hie wart gawan betrogen vmb sin ros V  \newline
\line(1,0){75} \newline
\textbf{1} Swaz] Waz U (W) (Q) (R) ÷waz O  $\cdot$ dû stêst] da bistu R \textbf{2} [*]: Nv hortest dv doch sprechen ie V  $\cdot$ hôrtes] horest Q \textbf{3} swer] Wer U W Q R \textbf{4} Daz er dar nach sin vint was O \textbf{5} die] die dy Q  $\cdot$ bî witzen] bitzen T \textbf{6} sich] Jch Q  $\cdot$ ob weinet] ob [weine ei:i*]: weinet V das weine W \textbf{7} bartohter] gehartohter T gebarteter V \textbf{8} wil] \textit{om.} O  $\cdot$ diz] das R \textbf{9} vaste von im] von im vaste V \textbf{10} iedoch] sere W doch O Q R Fr40  $\cdot$ Gawane] gawanen W Gawans O Gawainen R \textbf{11} Div frowe [sr]: sprach wie chom ez so O  $\cdot$ Er] ÷r T  $\cdot$ ez] er W  $\cdot$ sô] [s*]: also V \textbf{12} daz] \textit{om.} W O Q R Fr40 \textbf{13} ze] \textit{om.} R  $\cdot$ Dyanarsun] Dyanarzuͦn U dyabarzvn V dyanazarun W Dyanazrvn O (R) (Fr40) dianazton Q \textbf{14} bî] Mit W O Q R (Fr40)  $\cdot$ dâ] do U V Q R \textit{om.} W O  $\cdot$ manec] mang werder W  $\cdot$ Britun] brituͦn U brittvn V brittuͯm Q Brytun R \textbf{15} dem was] Do wart V \textbf{17} dirre] der W  $\cdot$ ungehiure] gehewre Q \textbf{19} was] wast T \textbf{20} riet] geriet W O Q R (Fr40)  $\cdot$ kranker] bester O tumber R \textbf{21} vrouwen] maget R \textbf{22} V́ber irn willen Im gelank R  $\cdot$ willen] \textit{om.} U  $\cdot$ gar] \textit{om.} W O Q Fr40  $\cdot$ danc] dranc U \textbf{23} hove] hoe Q  $\cdot$ daz] dis W \textbf{24} lûte] leute Q \textbf{25} geschach] gesach O  $\cdot$ einem] einen T \textbf{26} dar] [D*]: Dar V Dann W (O) Q (R) (Fr40)  $\cdot$ albalde] alle balde W O Q R \textbf{28} ergreif] begreif Fr40  $\cdot$ des] die U des reht O  $\cdot$ schuldehaften] schvldigen O schuldehafen Fr40 \textbf{29} vuortich] fuͦrt In R  $\cdot$ disen man] wider dan W O (Q) R Fr40 \textbf{30} under dan] wider dan V disen man W O (Q) R Fr40 \newline
\end{minipage}
\end{table}
\end{document}
