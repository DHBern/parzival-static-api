\documentclass[8pt,a4paper,notitlepage]{article}
\usepackage{fullpage}
\usepackage{ulem}
\usepackage{xltxtra}
\usepackage{datetime}
\renewcommand{\dateseparator}{.}
\dmyyyydate
\usepackage{fancyhdr}
\usepackage{ifthen}
\pagestyle{fancy}
\fancyhf{}
\renewcommand{\headrulewidth}{0pt}
\fancyfoot[L]{\ifthenelse{\value{page}=1}{\today, \currenttime{} Uhr}{}}
\begin{document}
\begin{table}[ht]
\begin{minipage}[t]{0.5\linewidth}
\small
\begin{center}*D
\end{center}
\begin{tabular}{rl}
\textbf{406} & "\begin{large}H\end{large}êrre, sît ir anders kluoc,\\ 
 & sô mag es dunken iuch genuoc.\\ 
 & ich erbiutez \textbf{iu} durch mînes bruoder bete,\\ 
 & daz e\textit{z} Ampflise Gahmurete,\\ 
5 & mînem œheime, nie baz \textbf{erbôt},\\ 
 & âne bî ligen. \textbf{mîn} triwe ein lôt\\ 
 & an dem orte vürbaz wæge,\\ 
 & der uns wegens ze rehte pflæge.\\ 
 & unt enweiz doch, hêrre, wer ir sît,\\ 
10 & \textbf{doch} ir an sô kurzer zît\\ 
 & welt mîne minne hân."\\ 
 & dô sprach der werde Gawan:\\ 
 & "Mich lêret \textbf{mîner} künde sin,\\ 
 & ich sag \textbf{iu}, vrouwe, \textbf{daz} ich bin\\ 
15 & mîner basen bruoder sun.\\ 
 & welt ir mir genâde tuon,\\ 
 & daz enlât niht durch mînen art.\\ 
 & der ist gein iwerm sô bewart,\\ 
 & daz si bêde \textbf{al} gelîche stênt\\ 
20 & unt in rehter mâze gênt."\\ 
 & Ein magt begunde in schenken,\\ 
 & dâr nâch schiere von in wenken.\\ 
 & mêre vrouwen dennoch dâ sâzen,\\ 
 & die ouch des niht vergâzen,\\ 
25 & si giengen \textbf{unt} \textbf{schuofen} umbe ir pflege.\\ 
 & ouch was der ritter von dem wege,\\ 
 & der in dar brâhte.\\ 
 & Gawan des gedâhte,\\ 
 & dô si alle von \textbf{im} kômen ûz,\\ 
30 & daz dicke den grôzen strûz\\ 
\end{tabular}
\scriptsize
\line(1,0){75} \newline
D Fr5 \newline
\line(1,0){75} \newline
\textbf{1} \textit{Initiale} D Fr5  \textbf{13} \textit{Capitulumzeichen} Fr5   $\cdot$ \textit{Majuskel} D  \textbf{21} \textit{Capitulumzeichen} Fr5   $\cdot$ \textit{Majuskel} D  \newline
\line(1,0){75} \newline
\textbf{4} ez] er D  $\cdot$ Gahmurete] Gahmvrete D \textbf{28} Gawan] Ga::: Fr5 \textbf{29} dô] So Fr5 \newline
\end{minipage}
\hspace{0.5cm}
\begin{minipage}[t]{0.5\linewidth}
\small
\begin{center}*m
\end{center}
\begin{tabular}{rl}
 & "hêrre, sît ir anders kluoc,\\ 
 & sô mac es dunken iuch genuoc.\\ 
 & ich erbiut ez \textbf{iu} durch mînes bruoder bete,\\ 
 & daz ez \textit{Amp}flise Gahmurete,\\ 
5 & mînem œheime, nie baz \textbf{erbôt},\\ 
 & âne bî ligen. \textbf{mîn} triuwe ein lôt\\ 
 & an dem orte vürbaz wæg\textit{e},\\ 
 & der uns wegens ze rehte pflæg\textit{e}.\\ 
 & und enweiz doch, hêrre, wer ir sît,\\ 
10 & \textbf{daz} ir an sô kurzer zît\\ 
 & wellet mîne minne hân."\\ 
 & dô sprach der werde Gawan:\\ 
 & "mich lêret \textbf{mîner} künde sin,\\ 
 & ich sage, vrouwe, \textbf{daz} ich bin\\ 
15 & mîner basen bruoder sun.\\ 
 & welt ir mir gnâde tuon,\\ 
 & daz enlât niht durch mînen art.\\ 
 & der ist gegen iuwerm sô bewart,\\ 
 & daz si beide \textbf{alle} gelîche stânt\\ 
20 & und in rehter mâze gânt."\\ 
 & \begin{large}E\end{large}in magt begunde in schenken,\\ 
 & dâr nâch schiere von in wenken.\\ 
 & mêre vrouwen dennoch dâ sâzen,\\ 
 & die ouch des niht vergâzen,\\ 
25 & si giengen \textbf{schaffen} umb ir pflege.\\ 
 & ouch was der ritter von dem wege,\\ 
 & der in dar brâhte.\\ 
 & Gawan des gedâhte,\\ 
 & dô si alle von \textbf{in} kômen ûz,\\ 
30 & daz dicke den grôzen strûz\\ 
\end{tabular}
\scriptsize
\line(1,0){75} \newline
m n o \newline
\line(1,0){75} \newline
\textbf{21} \textit{Initiale} m n  \newline
\line(1,0){75} \newline
\textbf{3} erbiut] enbuͯt o \textbf{4} Ampflise] entflise m anflisse n aneflise o  $\cdot$ Gahmurete] gamúrete n gahumurete o \textbf{5} Mẏnnen ohemm nie [has]: bas erbort o \textbf{6} lôt] bot o \textbf{7} wæge] wegen m \textbf{8} pflæge] pflegen m \textbf{12} werde] [selbe]: werde o \textbf{14} daz ich] das o \textbf{17} enlât] lont n (o) \textbf{18} der] Er n o  $\cdot$ sô] also n o \textbf{19} alle] wol n o \textbf{20} mâze] mossen n \textbf{22} in] \textit{om.} o \textbf{23} mêre] Nie o  $\cdot$ dâ] do n o \textbf{27} brâhte] hette bracht n (o) \newline
\end{minipage}
\end{table}
\newpage
\begin{table}[ht]
\begin{minipage}[t]{0.5\linewidth}
\small
\begin{center}*G
\end{center}
\begin{tabular}{rl}
 & "\begin{large}H\end{large}êrre, sît ir anders kluoc,\\ 
 & sô mag es dunken iuch genuoc.\\ 
 & ich erbiutz \textbf{iu} durch mînes bruoder bete,\\ 
 & daz ez Anphlise Gahmuret,\\ 
5 & mîne\textit{m} œheime, nie baz \textbf{erbôt},\\ 
 & âne bî ligen. \textbf{mîn} triwe ein lôt\\ 
 & ame orte vürbaz wæge,\\ 
 & der uns wegens \textit{ze} rehte pflæge,\\ 
 & unt \textbf{ich}ne weiz doch, hêrre, wer ir sît,\\ 
10 & \textit{\textbf{doch}} ir an sô kurzer zît\\ 
 & welt mîne minne hân."\\ 
 & dô sprach der werde Gawan:\\ 
 & "mich lêrt \textbf{muoter} künde sin,\\ 
 & ich sage \textbf{iu}, vrouwe, \textbf{wer} ich bin:\\ 
15 & mîner basen bruoder sun.\\ 
 & welt ir \textit{mir genâde} tuon,\\ 
 & daz enlât niht durch mînen art.\\ 
 & derst gein \textbf{dem} iweren sô bewart,\\ 
 & daz si \textit{bêde} \textit{\textbf{al}}gelîche \textit{st}ênt\\ 
20 & unt in \textit{reht}er mâze \textit{g}ênt."\\ 
 & ein maget begunde in schenken,\\ 
 & dâr nâch schiere von in wenken.\\ 
 & mêr vrouwen dannoch dâ sâzen,\\ 
 & die ouch des niht vergâzen,\\ 
25 & si giengen \textbf{unde} \textbf{schuofen} umbe ir pflege.\\ 
 & ouch was der rîter von dem wege,\\ 
 & der in dar brâhte.\\ 
 & Gawan des gedâhte,\\ 
 & dô si alle von \textbf{im} kômen ûz,\\ 
30 & daz dicke den grôzen strûz\\ 
\end{tabular}
\scriptsize
\line(1,0){75} \newline
G I O L M Q R Z Fr22 \newline
\line(1,0){75} \newline
\textbf{1} \textit{Initiale} G I O L R Z Fr22  \textbf{15} \textit{Initiale} I  \newline
\line(1,0){75} \newline
\textbf{1} \textit{Die Verse 370.13-412.12 fehlen} Q   $\cdot$ Hêrre] ÷erre O  $\cdot$ ir] \textit{om.} M \textbf{2} es] \textit{om.} I sin O  $\cdot$ dunken iuch] uch duncken M (R) (Fr22) \textbf{3} erbiutz iu] erbiut evz I (O) (L) irbotes uch M \textbf{4} ez] ichz O  $\cdot$ Anphlise] anpholiza I ampflise O Anflise L anfilise M anfliesse R amflise Z :::se Fr22  $\cdot$ Gahmuret] Gamuret O M (Z) Gahmuͯrete L :a:::et Fr22 \textbf{5} mînem] minen G (L)  $\cdot$ baz] so wol I \textbf{6} triwe] tru: truwe M  $\cdot$ lôt] [bot]: lot R \textbf{7} vürbaz] rehte I \textbf{8} ze] \textit{om.} G \textbf{9} unt] \textit{om.} L  $\cdot$ ichne] ich I R \textit{om.} M \textbf{10} doch] vnt G Daz L \textbf{12} dô] Da O M \textbf{13} lêrt] lerte I  $\cdot$ muoter] min muͤter I mine O mýner L (R) (Z)  $\cdot$ künde] chundic I \textbf{14} wer] daz Z \textbf{16} mir genâde] genade an mir G \textbf{17} enlât] lazet I (L) (M) lazzet durch Z  $\cdot$ mînen] mine Z \textbf{18} derst] diu ist I  $\cdot$ dem] der I (M) o\textit{m. } O Z  $\cdot$ iweren] ivrem O (L) (Z) uwir M \textbf{19} daz si vil nach gelîche gênt G  $\cdot$ bêde algelîche] beidú alle glichen R \textbf{20} in rehter] in einer G mir rehter O  $\cdot$ gênt] stent G \textbf{21} in schenken] senkin M \textbf{22} schiere] \textit{om.} R  $\cdot$ in] yme M \textbf{23} dannoch dâ] da dannoch L da R \textbf{25} si giengen] sin giengen I (L)  $\cdot$ unde schuofen umbe] vnd schuͯfen L schlaffen vmb R vmb Z \textbf{28} des] do R \textbf{29} dô] Da M Z  $\cdot$ von im kômen] komen L quamen von im Z \textbf{30} den] deme M \newline
\end{minipage}
\hspace{0.5cm}
\begin{minipage}[t]{0.5\linewidth}
\small
\begin{center}*T
\end{center}
\begin{tabular}{rl}
 & "\begin{large}H\end{large}êrre, sît ir anders kluoc,\\ 
 & sô mag es dunken iuch genuoc.\\ 
 & ich erbiut ez durch mînes bruoder bete,\\ 
 & daz ez Anflise Gahmurete,\\ 
5 & mînem œheime, nie baz \textbf{gebôt},\\ 
 & âne bî ligen. \textbf{ein} triuwe ein lôt\\ 
 & an dem orte vürbaz wæge,\\ 
 & der uns wegens ze rehte pflæge.\\ 
 & unde enweiz doch, hêrre, wer ir sît,\\ 
10 & \textbf{daz} ir an sô kurzer zît\\ 
 & welt mîne minne hân."\\ 
 & Dô sprach der werde Gawan:\\ 
 & "mich lêret \textit{\textbf{mîner}} künde sin,\\ 
 & ich sag\textbf{iu}, vrouwe, \textbf{wer} ich bin:\\ 
15 & mîner basen bruoder suon.\\ 
 & welt ir mir gnâde tuon,\\ 
 & daz enlât niht durch mînen art.\\ 
 & der ist gegen \textbf{dem} iuwern sô bewart,\\ 
 & daz si beide glîche stênt\\ 
20 & unde in rehter mâze gênt."\\ 
 & Ein maget begundin schenken,\\ 
 & dâr nâch schiere von in wenken.\\ 
 & mê vrouwen dannoch dâ sâzen,\\ 
 & die ouch des niht vergâzen,\\ 
25 & si\textbf{ne} giengen \textbf{schaffen} umbir pflege.\\ 
 & ouch was der rîter von dem wege,\\ 
 & der in dar brâhte.\\ 
 & Gawan des gedâhte,\\ 
 & dô salle von \textbf{in} kômen ûz,\\ 
30 & daz dicke den grôzen strûz\\ 
\end{tabular}
\scriptsize
\line(1,0){75} \newline
T U V W \newline
\line(1,0){75} \newline
\textbf{1} \textit{Initiale} T U  \textbf{12} \textit{Majuskel} T  \textbf{21} \textit{Majuskel} T  \newline
\line(1,0){75} \newline
\textbf{2} iuch] iv T \textbf{3} erbiut ez] erbietenz vch U (V) erbút úch W \textbf{4} ez] \textit{om.} W  $\cdot$ Anflise] anfelyse W  $\cdot$ Gahmurete] Gahmvrete T Gahmuͦrete U gamurette V gamuret W \textbf{5} gebôt] erbot W \textbf{6} ein triuwe] min truwe U (V) niem treúw W \textbf{9} unde] Vnd ich W \textbf{10} Doch ir also kurtzer bit W  $\cdot$ an sô] also U \textbf{13} mîner] \textit{om.} \textit{(Platz ausgespart)} T \textit{om.} U \textbf{14} wer] das V \textit{om.} W \textbf{17} enlât] enlant ir V \textbf{18} dem] \textit{om.} W  $\cdot$ sô] \textit{om.} U \textbf{19} glîche] all gleiche W \textbf{21} schenken] sencken W \textbf{22} in] im W \textbf{23} mê] Nie U  $\cdot$ dâ] do V W \textbf{25} sine] Sie U \textbf{29} von in kômen] quemen von in U komen vor im V kamen von im W \newline
\end{minipage}
\end{table}
\end{document}
