\documentclass[8pt,a4paper,notitlepage]{article}
\usepackage{fullpage}
\usepackage{ulem}
\usepackage{xltxtra}
\usepackage{datetime}
\renewcommand{\dateseparator}{.}
\dmyyyydate
\usepackage{fancyhdr}
\usepackage{ifthen}
\pagestyle{fancy}
\fancyhf{}
\renewcommand{\headrulewidth}{0pt}
\fancyfoot[L]{\ifthenelse{\value{page}=1}{\today, \currenttime{} Uhr}{}}
\begin{document}
\begin{table}[ht]
\begin{minipage}[t]{0.5\linewidth}
\small
\begin{center}*D
\end{center}
\begin{tabular}{rl}
\textbf{28} & die er tet ûf einen küenen man,\\ 
 & \textbf{der ouch} sîn ende \textbf{al}dâ gewan.\\ 
 & daz was mîn vriunt Isenhart.\\ 
 & ir ieweder innen wart\\ 
5 & eines spers durch schilt unt \textbf{durch} \textbf{den} lîp.\\ 
 & daz klag ich noch, \textbf{vil} armez wîp.\\ 
 & ir bêder tôt mich immer müet.\\ 
 & ûf \textbf{mîner triwe} jâmer blüet.\\ 
 & ich enwart nie wîp decheines man."\\ 
10 & Gahmureten dûhte sân,\\ 
 & swie si wære ein heidenîn,\\ 
 & mit triwen wîplîcher sin\\ 
 & in wîbes herze nie geslouf.\\ 
 & ir kiusche was ein reiner touf\\ 
15 & unt \textbf{ouch} der \textbf{regen}, der si begôz,\\ 
 & der \textbf{wâc}, der von ir ougen vlôz\\ 
 & ûf \textbf{ir} zobel unt \textbf{a\textit{n}} \textit{i}r brust.\\ 
 & riwen \textbf{pflege} was ir gelust\\ 
 & unt rehtiu jâmers lêre.\\ 
20 & si seit im vürbaz mêre:\\ 
 & "Dô suohte mich von über mer\\ 
 & der Schotten künec mit sînem her.\\ 
 & der was sînes œheimes sun.\\ 
 & \textbf{si}\textbf{ne} \textbf{mohten} mir niht mêr getuon\\ 
25 & schaden, denne mir was geschehen\\ 
 & an Isenharte, \textbf{ich muoz es} jehen."\\ 
 & \begin{large}D\end{large}iu vrouwe ersûfte dicke.\\ 
 & durch die zeher \textbf{manege} blicke\\ 
 & si schamende gastlîchen sach\\ 
30 & an Gahmureten. dô verjach\\ 
\end{tabular}
\scriptsize
\line(1,0){75} \newline
D \newline
\line(1,0){75} \newline
\textbf{21} \textit{Majuskel} D  \textbf{27} \textit{Initiale} D  \newline
\line(1,0){75} \newline
\textbf{3} Isenhart] Jsenhart D \textbf{10} Gahmureten] Gahmvreten D \textbf{17} an ir] an an ir D \textbf{22} Schotten] Scotten D \textbf{26} Isenharte] Jsenharte D \textbf{30} Gahmureten] Gahmvreten D \newline
\end{minipage}
\hspace{0.5cm}
\begin{minipage}[t]{0.5\linewidth}
\small
\begin{center}*m
\end{center}
\begin{tabular}{rl}
 & die er tet ûf einen küenen man,\\ 
 & \textbf{der ouch} sîn ende \textbf{al} dâ gewan.\\ 
 & daz was mîn vriunt Ysenhart.\\ 
 & ir ietweder innen wart\\ 
5 & eines spers durch schilt und \textbf{durch} \textbf{den} lîp.\\ 
 & daz klage ich noch, armez wîp.\\ 
 & ir beider tôt mich iemer müet.\\ 
 & ûf \textbf{mîne triuwe} jâmer blüet.\\ 
 & \textit{in}e wart nie wîp decheines man."\\ 
10 & Gahmureten dûhte sân,\\ 
 & wie si wær ein heidenîn,\\ 
 & mit triuwen wîplîcher sin\\ 
 & in wîbes herze nie geslouf.\\ 
 & ir kiusche was ein reiner touf\\ 
15 & und \textbf{ouch} der \textbf{regen}, der si begôz,\\ 
 & der \dag was\dag , der \dag nû ir ougen verlôz\dag ,\\ 
 & ûf \textbf{ir} zobel und \textbf{an} ir brust.\\ 
 & riuwen \textbf{pflegen} was ir gelust\\ 
 & und rehtiu \textbf{triuwe} jâmers lêre.\\ 
20 & si seite ime vürbaz mêre:\\ 
 & "dô suochte mich von über mer\\ 
 & der Schotten künic mit sînem her.\\ 
 & der was sînes œheims sun.\\ 
 & \textbf{si} \textbf{mohten} mir niht mê getuon\\ 
25 & schaden, danne mir was geschehen\\ 
 & an Ysenhart, \textbf{ich muoz es} jehen."\\ 
 & \begin{large}D\end{large}iu vrouwe ersûfze\textit{t}  dicke.\\ 
 & durch die zeher \textbf{manige} blicke\\ 
 & si schamende gastlîche sac\textit{h}\\ 
30 & an Gahmureten. dô verjach\\ 
\end{tabular}
\scriptsize
\line(1,0){75} \newline
m n o W \newline
\line(1,0){75} \newline
\textbf{27} \textit{Initiale} m n  \newline
\line(1,0){75} \newline
\textbf{1} \textit{Die Verse 26.4-29.1 fehlen} o   $\cdot$ die] Denne die n  $\cdot$ küenen] kúner n \textbf{3} Ysenhart] jsenhart m \textbf{5} eines spers] Ein spers m Ein sper n  $\cdot$ den] \textit{om.} n \textbf{6} armez] vil armes n \textbf{9} ine] Me m Jch n  $\cdot$ decheines] do kenies n \textbf{10} Gahmureten] Gahmuretten m Gamireten n \textbf{13} in] [Jr]: Jn n \textbf{18} riuwen pflegen] Ruwe pflege n \textbf{20} seite] seit n \textbf{22} Schotten künic] konnig schotten n \textbf{24} mohten] moͯchten n \textbf{26} Ysenhart] ÿsenhart m jsenhart n \textbf{27} ersûfzet] ersuffcze m \textbf{28} manige] manigen n \textbf{29} gastlîche] gaistlich W  $\cdot$ sach] sache m \textbf{30} Gahmureten] gahmuretten m gamúreten n gamureten W \newline
\end{minipage}
\end{table}
\newpage
\begin{table}[ht]
\begin{minipage}[t]{0.5\linewidth}
\small
\begin{center}*G
\end{center}
\begin{tabular}{rl}
 & die er tet ûf einen küenen man.\\ 
 & sînen ende \textbf{er} dâ gewan.\\ 
 & daz was mîn vriunt Ysenhart.\\ 
 & ir ietweder innen wart\\ 
5 & eines spers durch schilt und \textbf{durch} lîp.\\ 
 & daz klage ich noch, \textbf{vil} armez wîp.\\ 
 & ir beider tôt mich imer müet.\\ 
 & ûf \textbf{mînen triwen} jâmer blüet.\\ 
 & \begin{large}I\end{large}chne wart nie wîp deheines man."\\ 
10 & Gahmureten dûhte sân,\\ 
 & swie si wære ein heidenîn,\\ 
 & mit triwen wîplîcher sin\\ 
 & in wîbes herze nie geslouf.\\ 
 & ir kiusche was ein reiner touf\\ 
15 & unt der \textbf{wâc}, der si begôz,\\ 
 & der \textbf{regen}, der von ir ougen vlôz\\ 
 & ûf \textbf{ir} zobel und \textbf{ûf} ir brust.\\ 
 & riwen \textbf{pflege} was ir gelust\\ 
 & unde rehtiu jâmers lêre.\\ 
20 & si seite im vürbaz mêre:\\ 
 & "dô suohte mich von über mer\\ 
 & der Schotten künic mit sînem her.\\ 
 & der was sînes œheimes sun.\\ 
 & \textbf{er} \textbf{mohte} mir niht mê getuon\\ 
25 & schaden, dane mir was geschehen\\ 
 & an Ysenharte, \textbf{des muoz ich} jehen."\\ 
 & diu vrouwe ersûfte dicke.\\ 
 & durch die zahere \textbf{maniger} blicke\\ 
 & si schamende gastlîchen sach\\ 
30 & an Gahmureten. dô verjach\\ 
\end{tabular}
\scriptsize
\line(1,0){75} \newline
G O L M Q R W Z Fr29 Fr32 Fr71 \newline
\line(1,0){75} \newline
\textbf{1} \textit{Initiale} O M Fr29  \textbf{9} \textit{Initiale} G  \textbf{15} \textit{Initiale} Fr71   $\cdot$ \textit{Versal} Fr32  \textbf{27} \textit{Initiale} O L Q R Z Fr32  \newline
\line(1,0){75} \newline
\textbf{1} die] ÷ie O  $\cdot$ er] \textit{om.} W  $\cdot$ einen küenen] in ein kuͤne W \textbf{2} Do er avch sinen ende genam O  $\cdot$ Der ouch sin ende da genam L  $\cdot$ Da her ouch syn ende gewan M  $\cdot$ Der auch sein ende do (alda Z da Fr32 ) gewan Q (R) (W) (Z) (Fr32)  $\cdot$ Da er o::: Fr29 \textbf{3} was] \textit{om.} Z  $\cdot$ mîn] ein L  $\cdot$ Ysenhart] ẏsenhart G Fr32 isenhart O Jsenhart L R eysenhart Q \textbf{4} ir ietwedere] Jetweder O Er y eyner der M \textbf{5} eines] Eyn M (Q)  $\cdot$ spers] sper Q  $\cdot$ schilt] schilde M den schilt Fr71  $\cdot$ durch] durch den O (M) (Z) (Fr32) (Fr71) ouch den L den R W \textbf{7} beider] bruͦder Fr32 \textbf{8} mînen triwen] miner triwe O (L) (M) (Q) (R) (Z) (Fr32) \textbf{9} Ichne] Jch O  $\cdot$ wart nie] ny wart M wart me Fr32 \textbf{10} Gahmureten] Gahmvreten G Fr71 Gamvreten O Gahmuͯreten L Gamuͯreten M Gamureten Q (W) (Z) Gamurten R Gahmvͦr::: Fr29 gamvͦreten Fr32  $\cdot$ sân] sam O \textbf{11} swie] Wie O L (Q) R Z \textbf{13} wîbes] ræiner Fr71  $\cdot$ herze] herczin M (R) (W) \textbf{14} kiusche] kussze M  $\cdot$ reiner] rainiv O \textbf{15} unt] Vnd ovch Z  $\cdot$ wâc] regen O L (M) (Q) (R) (W) Z Fr32 \textbf{16} regen] waͦch O (L) (Q) (R) (W) (Z) (Fr29) (Fr32) flosz M  $\cdot$ von ir] von iren Q der von irn W von in Z \textbf{17} zobel] hubel Q  $\cdot$ und ûf] an O Q vnd an L (M) (R) (W) (Fr32) \textbf{18} riwen] Riwe O Truwen L (R)  $\cdot$ gelust] [lust]: gelust M \textbf{19} unde rehtiu] Vnd rehtes L (W) vnrehte Fr32 Vnd reht Fr71 \textbf{20} seite] sagt O L (R) Z Fr29 (Fr32) (Fr71) \textbf{21} dô] Da M R Z  $\cdot$ suohte] svht Z  $\cdot$ von] her Q \textbf{22} Schotten] schoten G O sotten M  $\cdot$ her] hern M \textbf{23} der] Do Q  $\cdot$ was] ist Fr71  $\cdot$ œheimes] ohemen M \textbf{24} er mohte] Ern moht O Sin mochten Q (Z) (Fr29) (Fr32) Sy mochttent R  $\cdot$ niht mê] nimmer M (Z) \textbf{25} schaden] Leydes W (Fr29) Zeschaden Fr71  $\cdot$ dane] der Q \textbf{26} Ysenharte] ẏsenharte G isenharten O Jsenharte L isenarte M eyszenharte Q Jsenhart R Fr29 isenhart Z ysinhart Fr71  $\cdot$ des muoz ich] ich mvͦz sin O ich muͯsz es L (M) (R) (W) (Z) (Fr29) \textbf{27} diu] ÷iv O  $\cdot$ ersûfte] ersewftze Q (R) ersovftet Fr71 \textbf{28} maniger] mannige M (R)  $\cdot$ blicke] blicthe Fr71 \textbf{29} gastlîchen] geistlichen M Q  $\cdot$ sach] sprach L Fr71 \textbf{30} Gahmureten] Gamvreten O Gahmuͯreten L gamuretin M gamuerten Q gamureten W Z Gahmvͦreten Fr29 gamvͦreten Fr32 Gahmvreten Fr71  $\cdot$ dô] da R Z \newline
\end{minipage}
\hspace{0.5cm}
\begin{minipage}[t]{0.5\linewidth}
\small
\begin{center}*T
\end{center}
\begin{tabular}{rl}
 & dier tet ûf einen küenen man,\\ 
 & \textbf{der ouch} sîn ende dâ gewan.\\ 
 & daz was mîn vriunt Isenhart.\\ 
 & ir ietwederre innen wart\\ 
5 & eines \textit{s}pers durch schilt und \textbf{den} lîp.\\ 
 & daz klagich noch, \textbf{vil} arme\textit{z} wîp.\\ 
 & ir beider tôt mich iemer müet.\\ 
 & ûf \textbf{mîner triuwe} jâmer blüet.\\ 
 & ine wart nie wîp deheines man."\\ 
10 & Gahmureten dûhte sân,\\ 
 & swie si wære ein heidenîn,\\ 
 & mit triuwen wîplîcher sin\\ 
 & in wîbes herze nie geslouf.\\ 
 & ir kiusche was ein reiner touf\\ 
15 & und der \textbf{regen}, der si begôz,\\ 
 & der \textbf{wâc}, der von ir ougen vlôz\\ 
 & ûf \textbf{den} zobel und \textbf{an} ir brust.\\ 
 & riuwen \textbf{pflege} was ir gelust\\ 
 & und reht\textit{iu} jâmers lêre.\\ 
20 & si sagetim vürbaz mêre:\\ 
 & "dô suochte mich von über mer\\ 
 & der Schotten künec mit sînem her.\\ 
 & der was sînes œheimes sun.\\ 
 & \textbf{er}\textbf{n} \textbf{mohte} mir niht mê getuon\\ 
25 & schaden, denne mir was geschehen\\ 
 & an Isenharte, \textbf{ich muoz es} jehen."\\ 
 & \begin{large}D\end{large}iu vrouwe ersûfte dicke.\\ 
 & durch die zehere \textbf{manege} blicke\\ 
 & si schamende gastlîchen sach\\ 
30 & an Gahmureten. dô verjach\\ 
\end{tabular}
\scriptsize
\line(1,0){75} \newline
T U V \newline
\line(1,0){75} \newline
\textbf{27} \textit{Initiale} T U V  \newline
\line(1,0){75} \newline
\textbf{1} dier] Der U Die V  $\cdot$ einen] [*]: in ein V  $\cdot$ küenen] kuͤne V \textbf{2} dâ] do U V \textbf{3} Isenhart] Jsenhart T U Jsinhart V \textbf{4} innen] man U \textbf{5} eines spers] eins dspers T Ein sper U  $\cdot$ und den] vnd duͦrch den U [*]: vnde durch den V \textbf{6} armez] armes T \textbf{7} müet] ruwet V \textbf{8} mîner] mine U \textbf{9} ine] Jch U \textbf{10} Gahmureten] Gahmvreten T Gahmuͦrethen U Gamureten V \textbf{11} swie] Wie U \textbf{14} kiusche] kuͦscheit U  $\cdot$ reiner] reine V \textbf{17} den] ir U V \textbf{19} rehtiu] rehte T rethe in U \textbf{20} sagetim] saget im U V \textbf{22} Schotten] schoten T \textbf{24} ern mohte] er enmoͤhte V \textbf{26} Isenharte] Jsenharte T U V  $\cdot$ ich muoz es] des muͦz ich U (V) \textbf{28} zehere] zuͦ here U \textbf{29} Sie schamete gastliche sache U \textbf{30} Gahmureten] Gahmvreten T Gahmuͦreten U Gamureten V \newline
\end{minipage}
\end{table}
\end{document}
