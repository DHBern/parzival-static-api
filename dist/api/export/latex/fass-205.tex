\documentclass[8pt,a4paper,notitlepage]{article}
\usepackage{fullpage}
\usepackage{ulem}
\usepackage{xltxtra}
\usepackage{datetime}
\renewcommand{\dateseparator}{.}
\dmyyyydate
\usepackage{fancyhdr}
\usepackage{ifthen}
\pagestyle{fancy}
\fancyhf{}
\renewcommand{\headrulewidth}{0pt}
\fancyfoot[L]{\ifthenelse{\value{page}=1}{\today, \currenttime{} Uhr}{}}
\begin{document}
\begin{table}[ht]
\begin{minipage}[t]{0.5\linewidth}
\small
\begin{center}*D
\end{center}
\begin{tabular}{rl}
\textbf{205} & wir geben in noch strî\textit{t}es vil\\ 
 & unt bringenz ûz ir vreuden zil.\\ 
 & \textbf{\begin{large}M\end{large}an unt mâge sult ir} manen\\ 
 & unt \textbf{suochet} die stat mit zwein vanen.\\ 
5 & wir mugen an \textbf{der} lîten\\ 
 & wol ze \textbf{orse} zuo \textbf{z}in rîten.\\ 
 & die porten \textbf{suochen wir} ze vuoz.\\ 
 & \textbf{deiswâr}, wir tuon in schimpfes buoz."\\ 
 & den rât gap Galogandres,\\ 
10 & der herzog \textbf{von} \textbf{Gippones}.\\ 
 & der brâht die burgære in nôt.\\ 
 & \textbf{er} \textbf{holt} \textbf{ouch} an ir letze den tôt.\\ 
 & \textbf{als} tet der grâve Narant,\\ 
 & ein vürste \textbf{ûz} Ukerlant,\\ 
15 & unt \textbf{manec wert armer man},\\ 
 & \textbf{den} man \textbf{tôten truoc} \textbf{her dan}.\\ 
 & Nû hœret ein ander mære,\\ 
 & wie die burgære\\ 
 & ir letze tâten goume.\\ 
20 & si nâmen lange boume\\ 
 & unt \textbf{stiezen} starke stecken drîn\\ 
 & - daz gap den suochæren pîn -,\\ 
 & \textbf{mit seilen} si \textbf{die} hiengen.\\ 
 & die ronen in redern giengen.\\ 
25 & daz was geprüevet allez, ê\\ 
 & si \textbf{suochte sturmes} Clamide,\\ 
 & nâch Kingruns schumpfentiwer.\\ 
 & \textbf{ouch} \textbf{kom} in heidensch wilde viwer\\ 
 & mit \textbf{der} spîse in daz lant.\\ 
30 & daz ûzer antwerc wart verbrant.\\ 
\end{tabular}
\scriptsize
\line(1,0){75} \newline
D \newline
\line(1,0){75} \newline
\textbf{3} \textit{Initiale} D  \textbf{17} \textit{Majuskel} D  \newline
\line(1,0){75} \newline
\textbf{1} strîtes] stris D \textbf{13} Narant] Nerant D \textbf{14} Ukerlant] Vcher lant D \textbf{27} Kingruns] kingrvͦns D \newline
\end{minipage}
\hspace{0.5cm}
\begin{minipage}[t]{0.5\linewidth}
\small
\begin{center}*m
\end{center}
\begin{tabular}{rl}
 & wir geben in noch strîtes vil\\ 
 & und bringen\textit{z} ûz ir vröude zil.\\ 
 & \textbf{man und mâge sullet ir} \textit{m}a\textit{n}en\\ 
 & und \textbf{suochet} die stat mit zwein vanen.\\ 
5 & wir mügen an \textbf{der} lîten\\ 
 & wol ze \textbf{rosse} zuo in rîten.\\ 
 & die porten \textbf{suochen wir} ze vuoz.\\ 
 & \textbf{vür wâr}, wi\textit{r} tuon in schimpfes buoz."\\ 
 & den rât gap Galogand\textit{re}s,\\ 
10 & der herzoge \textbf{von} \textbf{Gippones}.\\ 
 & der brâhte die burgære in nôt.\\ 
 & \textbf{er} \textbf{holte} \textbf{ouch} an ir letze den tôt.\\ 
 & \textbf{als} det der grâve Narant,\\ 
 & ein vürste \textbf{ûz} Ucherlant,\\ 
15 & und \textbf{manic wert armman},\\ 
 & \textbf{den} man \textbf{tôten truoc} \textbf{hin dan}.\\ 
 & nû hœret ein ander mære,\\ 
 & wie die burgære\\ 
 & ir letze tâten goume.\\ 
20 & si nâmen lange boume\\ 
 & und \textbf{stiezen} starke stecken drîn\\ 
 & - daz gap den suochern pîn -,\\ 
 & \textbf{mit seilen} si \textbf{die} hiengen.\\ 
 & die ronen in rederen giengen.\\ 
25 & daz was gebrüefet allez, ê\\ 
 & si \textbf{suohte sturm\textit{e}s} Clamide,\\ 
 & nâch Kingrunes schumpfentiure.\\ 
 & \textbf{ouch} \textbf{kam} in heidensch wilde viure\\ 
 & mit \textbf{der} spîse in daz lant.\\ 
30 & daz ûzer antw\textit{e}rc wart verbrant.\\ 
\end{tabular}
\scriptsize
\line(1,0){75} \newline
m n o Fr69 \newline
\line(1,0){75} \newline
\newline
\line(1,0){75} \newline
\textbf{2} bringenz] bringent m n (o)  $\cdot$ ûz] \textit{om.} o  $\cdot$ vröude] freiden n o (Fr69)  $\cdot$ zil] spil n \textbf{3} man und mâge] Mag vnd man Fr69  $\cdot$ manen] namen m \textbf{4} stat] [zuht]: stat o \textbf{5} an der lîten] anderluͯten o \textbf{8} wir] wirt m \textbf{9} Galogandres] galoganders m galogandrez o galagandres Fr69 \textbf{10} der herzoge] Die hirczogin o  $\cdot$ Gippones] gẏpponez o gipenes Fr69 \textbf{12} holte] holt n o  $\cdot$ ir] \textit{om.} n \textbf{13} Narant] narrant m \textbf{14} Ucherlant] vcherlant m v́cher lant n ucher lant o \textbf{16} hin dan] her dan n (o) \textbf{19} goume] goime n goͯnne o \textbf{20} nâmen] mament o \textbf{25} allez] also n \textbf{26} suohte] suchten o  $\cdot$ sturmes] sturmens m mit sturme Fr69  $\cdot$ Clamide] klamade o \textbf{27} Kingrunes] [kingrunen]: kingrunes m konigruͯnez o kingruns Fr69  $\cdot$ schumpfentiure] scuͯnferture o \textbf{28} in] ein Fr69  $\cdot$ heidensch] hẏdensch o \textbf{29} \textit{nach 205.29:} Vnd fant den konig artus o  \textbf{30} ûzer] vs n o  $\cdot$ antwerc] anttwederg m \newline
\end{minipage}
\end{table}
\newpage
\begin{table}[ht]
\begin{minipage}[t]{0.5\linewidth}
\small
\begin{center}*G
\end{center}
\begin{tabular}{rl}
 & wir geben in noch strîtes vil\\ 
 & unde bringenz ûz ir vröuden zil.\\ 
 & \textbf{man unde mâge sult ir} manen\\ 
 & unde \textbf{suochen} die stat mit zwein vanen.\\ 
5 & wir mugen an \textbf{einer} lîten\\ 
 & wol ze \textbf{orsen} zuo in rîten.\\ 
 & die porte \textbf{suoche man} ze vuoz.\\ 
 & \textbf{des} wir tuon in schimpfes buoz."\\ 
 & den rât gap Galogandres,\\ 
10 & der herzoge \textbf{Tschinpones}.\\ 
 & der brâht die burgære in nôt\\ 
 & \textbf{unde} \textbf{holt} \textbf{ouch} an ir letze \textit{d}en tôt.\\ 
 & \textbf{sam} tet der grâve Narrant,\\ 
 & ein vürste \textbf{ûz} Ukerlant,\\ 
15 & unde \textbf{manic wert armman},\\ 
 & \textbf{den} man \textbf{truoc tôten} \textbf{her dan}.\\ 
 & nû hœrt ein ander mære,\\ 
 & wie die burgære\\ 
 & ir letze tâten goume.\\ 
20 & si nâmen lange boume\\ 
 & unde \textbf{stiezen} starke stecken drîn\\ 
 & - daz gap den suochæren pîn -,\\ 
 & \textbf{mit seilen} si \textbf{si} hiengen.\\ 
 & die ronen in rederen giengen.\\ 
25 & daz was geprüevet allez, ê\\ 
 & si \textbf{ze sturme suohte} Clamide,\\ 
 & nâch Kingruns schumpfentiur.\\ 
 & \textbf{ouch} \textbf{was} in heidensch wilde viur\\ 
 & mit \textbf{der} spîse \textbf{brâht} in daz lant.\\ 
30 & daz ûzer antwerc wart verbrant.\\ 
\end{tabular}
\scriptsize
\line(1,0){75} \newline
G I O L M Q R Z Fr21 \newline
\line(1,0){75} \newline
\textbf{9} \textit{Initiale} I R  \textbf{13} \textit{Initiale} O Q Fr21  \textbf{17} \textit{Initiale} L Z  \textbf{29} \textit{Initiale} R  \newline
\line(1,0){75} \newline
\textbf{2} bringenz] bringe ez L  $\cdot$ ûz] vff R  $\cdot$ zil] spil O R \textbf{3} man unde mâge] man vnd magen I Mag vnd man Q (R) \textbf{4} suochen] svͦcht O (L) (M) (Q) (R) (Z) (Fr21)  $\cdot$ stat] \textit{om.} I \textbf{5} einer] \textit{om.} I der Z  $\cdot$ lîten] syten L \textbf{6} wol ze orsen] ze orsen wol I Wol ze rose O (L) (R) (Z) (Fr21) Wol mit rossin M Wol vff rossen Q  $\cdot$ zuo in] zu zin Q mit in Fr21  $\cdot$ rîten] geritten L [striten]: riten Fr21 \textbf{7} porte] porten I O Z phortin M pforte Q  $\cdot$ suoche] suchte M  $\cdot$ vuoz] vnsz M \textbf{8} \textit{Vers 205.8 fehlt} M   $\cdot$ des] des war I (O) (L) (R) (Fr21) Entswar Q Zwar Z  $\cdot$ schimpfes] strites R \textbf{9} gap] \textit{om.} I den gab M  $\cdot$ Galogandres] kalocandres O Gagolandres L kalograndres M kalograndes Q kalogandres Z Fr21 \textbf{10} herzoge] herzoge von O (L) (M) (Q) (R) (Z) (Fr21)  $\cdot$ Tschinpones] shinpones I schipones O (R) (Fr21) Thispodres L tscipones M gippones Q tschopones Z \textbf{12} holt] lac I \textit{om.} O M holte L nam Fr21  $\cdot$ den tôt] entot G \textbf{13} sam] ÷am O  $\cdot$ der] \textit{om.} L oͮch der Fr21  $\cdot$ grâve] herczoge M  $\cdot$ Narrant] Narant L M (Q) R (Z) \textbf{14} ûz] von L  $\cdot$ Ukerlant] vcherlant G L Q akerlant I Vckerlant R vckurlant Z \textbf{15} manic wert] anders manig werder L wenig R  $\cdot$ armman] [arman]: armman G man L armer man M (R) \textbf{16} truoc tôten] toten truͯch L (M) (Q) (R) (Fr21)  $\cdot$ her dan] von dan O (Q) hindan R \textbf{17} ein] an I \textbf{21} stiezen] spilten L saszen M  $\cdot$ starke] lange I L scharffe M grose R \textbf{22} suochæren] suftebern I suͯchere L \textbf{23} mit] Jn L  $\cdot$ seilen] seyl L (Q)  $\cdot$ si si] si O (M) Fr21 sie die L \textbf{24} \textit{Vers 205.24 fehlt} I  \textbf{25} daz] Ditze O (L) (M) (Q) (Z) (Fr21)  $\cdot$ geprüevet allez] geproͮvet [alz]: allz G geprwuet [ee]: elles Q allez geprvfet Z geprvͦfet Fr21 \textbf{26} si] Sit O Fr21  $\cdot$ sturme] stúrmen R  $\cdot$ suohte] seucht I (O) (Fr21) suchten M  $\cdot$ Clamide] klamide I Glamide O \textbf{27} Kingruns] kyngrvnes O (M) kýngrvnes L kingrúns Q [kyngurn]: kyngurns R \textbf{28} heidensch] hadnisch I hæidenis O heildins M heydenisz Q heidech R heidens Z  $\cdot$ wilde] wildes Q \textbf{30} antwerc] werc M hantwerck Q  $\cdot$ verbrant] verbant L \newline
\end{minipage}
\hspace{0.5cm}
\begin{minipage}[t]{0.5\linewidth}
\small
\begin{center}*T
\end{center}
\begin{tabular}{rl}
 & wir geben in noch strîtes vil\\ 
 & unde bringenz ûz ir vröuden zil.\\ 
 & \textbf{ir sult mâge unde man} manen.\\ 
 & \textbf{suochet} die stat mit zwein vanen!\\ 
5 & wir mugen an \textbf{einer} lîten\\ 
 & wol ze \textbf{orse} zuo in rîten.\\ 
 & die porten \textbf{suoche man} ze vuoz.\\ 
 & \textbf{deiswâr}, wir tuon in schimpfes buoz."\\ 
 & Den rât gap Galoganders,\\ 
10 & der herzoge \textbf{von} \textbf{Schampones}.\\ 
 & der brâhte die burgære in nôt.\\ 
 & \textbf{der} \textbf{dolte} an ir letze den tôt.\\ 
 & \textbf{als} tet \textbf{ouch} der grâve Narant,\\ 
 & ein vürste \textbf{von} Uckerlant,\\ 
15 & unde \textbf{anders manec werder man},\\ 
 & \textbf{die} man \textbf{tôt truoc} \textbf{dâr v\textit{o}n}.\\ 
 & \begin{large}N\end{large}û hœret ein ander mære,\\ 
 & wie die burgære\\ 
 & ir letze tâten goume.\\ 
20 & si nâmen lange boume\\ 
 & unde \textbf{spizten} starke stecken drîn\\ 
 & - daz gap den suochæren pîn -,\\ 
 & \textbf{in seil} si \textbf{die} hiengen.\\ 
 & die ronen in redern giengen.\\ 
25 & daz was geprüevet allez ê,\\ 
 & \textbf{ê} si \textbf{suochte sturmes} Clamide.\\ 
 & nâch Kyngrunes schumpfentiur\\ 
 & \textbf{kom} in heidensch wildez viur\\ 
 & mit \textbf{ir} spîse in daz lant.\\ 
30 & daz ûzer antwerc wart verbrant.\\ 
\end{tabular}
\scriptsize
\line(1,0){75} \newline
T U V W \newline
\line(1,0){75} \newline
\textbf{9} \textit{Majuskel} T  \textbf{17} \textit{Initiale} T U V W  \newline
\line(1,0){75} \newline
\textbf{2} ir] \textit{om.} W \textbf{3} sult] súllen W  $\cdot$ mâge unde man] man vnd mage U (V)  $\cdot$ manen] des manen W \textbf{5} lîten] seiten W \textbf{6} zuo in rîten] zvͦzin riten V streiten W \textbf{7} porten] porte W  $\cdot$ suoche] suͦchet U \textbf{9} Galoganders] Galogandres U (V) galogandes W \textbf{10} Schampones] Tscampones U schimpones W \textbf{12} der dolte] Er hulte U Er holte V W  $\cdot$ ir] der W \textbf{13} grâve] graue von W  $\cdot$ Narant] Harant T (V) harnant W \textbf{14} Uckerlant] vker lant W \textbf{15} anders] ander W \textbf{16} die] [D*]: Den V  $\cdot$ dâr van] der van T [*]: her dan V von dan W \textbf{21} spizten starke] [*]: stiessent scharpfe V spichten lange W \textbf{22} pîn] grosse pein W \textbf{24} ronen] rouen W \textbf{26} suochte sturmes] sturmes suͦchte W  $\cdot$ Clamide] klamide W \textbf{27} Kyngrunes] kyngruͦnes U kingrunes W \textbf{28} kom] [*]: Oͮch kam V \textbf{29} ir] [*]: der V  $\cdot$ daz] dis W \newline
\end{minipage}
\end{table}
\end{document}
