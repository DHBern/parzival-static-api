\documentclass[8pt,a4paper,notitlepage]{article}
\usepackage{fullpage}
\usepackage{ulem}
\usepackage{xltxtra}
\usepackage{datetime}
\renewcommand{\dateseparator}{.}
\dmyyyydate
\usepackage{fancyhdr}
\usepackage{ifthen}
\pagestyle{fancy}
\fancyhf{}
\renewcommand{\headrulewidth}{0pt}
\fancyfoot[L]{\ifthenelse{\value{page}=1}{\today, \currenttime{} Uhr}{}}
\begin{document}
\begin{table}[ht]
\begin{minipage}[t]{0.5\linewidth}
\small
\begin{center}*D
\end{center}
\begin{tabular}{rl}
\textbf{48} & von der küneginne rîch.\\ 
 & \textbf{si kuste den degen} minneclîch.\\ 
 & \multicolumn{1}{l}{ - - - }\\ 
 & \multicolumn{1}{l}{ - - - }\\ 
 & \hspace*{-.7em}\big| er was ir mannes muomen sun\\ 
 & \hspace*{-.7em}\big| - si moht ez wol mit êren tuon -\\ 
5 & \textbf{\textit{\begin{large}U\end{large}}nde was von} arde ein künic hêr.\\ 
 & der wirt sprach lachende mêr:\\ 
 & "Got weiz, hêr Kaylet,\\ 
 & ob ich iu n\textit{æ}me Dolet\\ 
 & unt iwer lant \textbf{ze} Spane\\ 
10 & durch den künec \textbf{Gascane},\\ 
 & der iu \textbf{dicke tuot mit} zornes gir,\\ 
 & daz wære ein untriwe an mir,\\ 
 & \textbf{wan} ir sît mîner muomen kint.\\ 
 & die besten gar mit iu hie sint,\\ 
15 & der rîterschefte herte.\\ 
 & \textbf{wer} twang iuch dirre verte?"\\ 
 & \textbf{sô} sprach der stolze degen junc:\\ 
 & "mir gebôt mîn veter Schiltunc,\\ 
 & des tohter Vridebrant dâ hât,\\ 
20 & daz ich im diende. ez wære sîn rât.\\ 
 & \textbf{der} hât von sîme wîbe\\ 
 & hie von \textbf{mîn eines} lîbe\\ 
 & sehs tûsent rîter wol bekant.\\ 
 & die tragent \textbf{werlîche hant}.\\ 
25 & \textbf{ich brâht} ouch rîter mêr durch in;\\ 
 & der ist ein teil gescheiden hin.\\ 
 & hie wâren durch \textbf{die} Schotten\\ 
 & die werlîche rotten.\\ 
 & \textbf{im kom} von Gruonlanden\\ 
30 & helde ze\textbf{n} handen,\\ 
\end{tabular}
\scriptsize
\line(1,0){75} \newline
D Fr9 \newline
\line(1,0){75} \newline
\textbf{5} \textit{Initiale} D  \textbf{7} \textit{Majuskel} D  \newline
\line(1,0){75} \newline
\textbf{5} Unde] ÷nde D \textbf{8} næme] neme D \textbf{18} Schiltunc] Sciltvnch D \textbf{27} die] den Fr9  $\cdot$ Schotten] Scotten D (Fr9) \textbf{28} werlîche] werlichen Fr9 \textbf{29} kom] quamen Fr9  $\cdot$ Gruonlanden] Grvͦnlanden D grvnlanden Fr9 \newline
\end{minipage}
\hspace{0.5cm}
\begin{minipage}[t]{0.5\linewidth}
\small
\begin{center}*m
\end{center}
\begin{tabular}{rl}
 & von der küniginne rîch.\\ 
 & \textbf{si kuste den degen} minneclîch.\\ 
 & \multicolumn{1}{l}{ - - - }\\ 
 & \multicolumn{1}{l}{ - - - }\\ 
 & \hspace*{-.7em}\big| er was ir mannes muomen sun\\ 
 & \hspace*{-.7em}\big| - si mohte ez wol mit êren tuon -\\ 
5 & \textbf{und was von} art ein künic hêr.\\ 
 & der wirt sprach lachende mêr:\\ 
 & "goteweiz, hêr Kailet,\\ 
 & obe \textit{ich} iu n\textit{æ}m\textit{e} Dolet\\ 
 & und iuwer lant \textbf{ze} Spane\\ 
10 & durch den künic \textbf{von} \textbf{Saltane},\\ 
 & der iu \textbf{dicke tuot mit} zornes gir,\\ 
 & daz wære ein untriuwe an mir,\\ 
 & \textbf{wanne} ir sît mîner muomen kint.\\ 
 & die besten gar mit iu hie sint,\\ 
15 & der ritterschefte herte.\\ 
 & \textbf{wer} twanc iuch dirre verte?"\\ 
 & \textbf{\begin{large}D\end{large}ô} sprach der stolze degen junc:\\ 
 & "mir gebôt mîn veter Schildunc,\\ 
 & des tohter Fridebrant dâ hât,\\ 
20 & daz ich im diente. ez wær \textit{s}în rât.\\ 
 & \textbf{der} \textit{het} von sînem wîbe\\ 
 & hie von \textbf{mîn eines} lîbe\\ 
 & sehs tûsent ritter wolbekant.\\ 
 & die tragent \textbf{werlîche hant}.\\ 
25 & \textbf{ich brâhte} ouch ritter mêr durch in;\\ 
 & der ist ein teil gescheiden hin.\\ 
 & hie wâren \textbf{ouch} durch \textbf{die} Schotten\\ 
 & die werlîchen rotten.\\ 
 & \textbf{im kamen} von Grunlanden\\ 
30 & helde ze\textbf{n} handen,\\ 
\end{tabular}
\scriptsize
\line(1,0){75} \newline
m n o \newline
\line(1,0){75} \newline
\textbf{17} \textit{Initiale} m o   $\cdot$ \textit{Capitulumzeichen} n  \newline
\line(1,0){75} \newline
\textbf{2} degen] tougen n \textbf{4} ir] ires n (o) \textbf{3} mohte] moͯchte n \textbf{7} Kailet] kaẏlet n kaͯlet o \textbf{8} Obe uch niemen dolet m  $\cdot$ iu] do o \textbf{10} von] zuͯ n (o)  $\cdot$ Saltane] galtane n o \textbf{12} ein] an ein o  $\cdot$ untriuwe] vngetruwe n \textbf{18} Schildunc] schildung m n schuldung o \textbf{19} Fridebrant] vride brant m  $\cdot$ dâ] do n o \textbf{20} ez] das n o  $\cdot$ wær] ist n es wer o  $\cdot$ sîn] min m n \textbf{21} het] \textit{om.} m \textbf{22} lîbe] [libes]: libe m \textbf{27} die] die \textit{nachträglich korrigiert zu:} dien m den n o \textbf{29} kamen] komet o  $\cdot$ Grunlanden] grúnlanden o \newline
\end{minipage}
\end{table}
\newpage
\begin{table}[ht]
\begin{minipage}[t]{0.5\linewidth}
\small
\begin{center}*G
\end{center}
\begin{tabular}{rl}
 & von der küniginne rîch.\\ 
 & \textbf{diu kuste den \textit{degen}} minneclîch.\\ 
 & \multicolumn{1}{l}{ - - - }\\ 
 & \multicolumn{1}{l}{ - - - }\\ 
 & \begin{large}S\end{large}i mahtz wol mit êren tuon,\\ 
 & er was ir mannes muomen sun\\ 
5 & \textbf{und was von} arde ein künic hêr.\\ 
 & der wirt sprach lachende mêr:\\ 
 & "got weiz, hêr Kailet,\\ 
 & obe ich iu næme Dolet\\ 
 & unde iwer lant Spaninge\\ 
10 & durch den künic \textbf{von} \textbf{Gasconinge},\\ 
 & der iu \textbf{dicke tuot mit} zornes gir,\\ 
 & daz wære ein untriwe an mir.\\ 
 & ir sît \textbf{doch} mîner muomen kint.\\ 
 & die besten gar mit iu hie sint,\\ 
15 & der rîterschefte herte.\\ 
 & \textbf{waz} twang iuch dirre verte?"\\ 
 & \textbf{dô} sprach der stolze degen junc:\\ 
 & "mir gebôt mîn veter Schiltunc,\\ 
 & des tohter Fridebrant dâ hât,\\ 
20 & daz ich im diente. ez wære sîn rât.\\ 
 & \textbf{er} hât von sînem wîbe\\ 
 & hie von \textbf{mîn eines} lîbe\\ 
 & sehs tûsent rîter wol bekant.\\ 
 & die tragent \textbf{werlîche hant}.\\ 
25 & \multicolumn{1}{l}{ - - - }\\ 
 & \multicolumn{1}{l}{ - - - }\\ 
 & hie wâren durch \textbf{die} Schotten\\ 
 & die we\textit{r}lîchen rotten.\\ 
 & \textbf{hie was} von Gruonlanden\\ 
30 & helde z\textbf{ir} handen,\\ 
\end{tabular}
\scriptsize
\line(1,0){75} \newline
G I O L M Q R Z \newline
\line(1,0){75} \newline
\textbf{1} \textit{Initiale} O M  \textbf{3} \textit{Initiale} G  \textbf{7} \textit{Initiale} I  \textbf{27} \textit{Initiale} L Q  \textbf{29} \textit{Initiale} I  \newline
\line(1,0){75} \newline
\textbf{1} \textit{Die Verse 44.7-51.12 fehlen} Z   $\cdot$ von] ÷on O  $\cdot$ küniginne] kúnginnen R \textbf{2} diu] do I Sy R  $\cdot$ kuste] chust I (Q)  $\cdot$ den degen] den G in L  $\cdot$ minneclîch] mundigleich Q wunnenklich R \textbf{3} mahtz] mochen es Q \textbf{4} ir] irs L Q \textbf{5} was] \textit{om.} O \textbf{7} Kailet] Gahilet I kaylet O L R kayleck Q \textbf{8} Dolet] delet R \textbf{9} Spaninge] zeyspanien I spanîe O hyspanie L spanie M spange Q (R) \textbf{10} von] \textit{om.} R  $\cdot$ Gasconinge] washonien I Gatsanîe O Gascanie L gaschanie M gatschange Q Sascange R \textbf{11} dicke tuot] tuͯt dicke L tuͦt R  $\cdot$ mit zornes] zorns I zuͯ dem rosze L \textbf{12} untriwe] vngefvͦge O vnfuge Q (R) \textbf{13} muomen kint] muͤmensun I \textbf{14} gar mit iu hie] gar hye mit euch Q hie mit úch R \textbf{15} der] Die M \textbf{16} waz] Wer O L M Q R  $\cdot$ twang] tawck Q  $\cdot$ dirre] [mi*]: dirre I \textbf{17} dô] \textit{om.} I  $\cdot$ stolze] selbe O \textbf{18} Schiltunc] schiltunch G (O) (L) shiltunc I schilt M \textbf{19} Fridebrant] vridbrant I  $\cdot$ dâ] do O R \textbf{20} ez] daz I O \textbf{21} \textit{Die Verse 48.21-54.6 fehlen} R  \textbf{22} mîn eines] minem eins I mynesz [w]: eynesz M \textbf{25} \textit{Die Verse 48.25-26 fehlen} G I   $\cdot$ Hie was avch ritter mer durch in O (L) (M) (Q) \textbf{26} Der ist ein (\textit{om.} Q ) teil gescheiden hin O (L) (M) (Q) \textbf{27} hie] Sje L  $\cdot$ durch] durc in I  $\cdot$ die] den O L M Q  $\cdot$ Schotten] schoten G O shotten I schottin M \textbf{28} werlîchen] welichen G werlich I O (L) wertlichin M \textbf{29} Gruonlanden] groͮnlanden G gruͤnlanden I grvͦne landen O Grvnlanden L grune landin M grúnlanden Q \textbf{30} zir] ze den O (L) (M) sten Q  $\cdot$ handen] landen I [land]: handen O \newline
\end{minipage}
\hspace{0.5cm}
\begin{minipage}[t]{0.5\linewidth}
\small
\begin{center}*T (U)
\end{center}
\begin{tabular}{rl}
 & von der küneginne rîch.\\ 
 & \textbf{ir kus, der was} minneclîch,\\ 
 & den si dem degen bôt\\ 
 & mit i\textit{r} munde rôsenrôt.\\ 
 & \hspace*{-.7em}\big| er was ir mannes muomen sun\\ 
 & \hspace*{-.7em}\big| - si moht ez wol mit êren tuon -,\\ 
5 & \textbf{von küneges} art ein künec hêr.\\ 
 & der wirt sprach lachende mêr:\\ 
 & "got weiz, hêr Kaylet,\\ 
 & ob ich iu næme Dolet\\ 
 & und iuwer lant Spanie\\ 
10 & durch den künic \textbf{von} \textbf{Kasganie},\\ 
 & der iu \textbf{tet dicke} zornes gir,\\ 
 & daz wære ein untriuwe an mir.\\ 
 & ir sît \textbf{doch} mîner muomen kint.\\ 
 & die besten gar mit iu hie sint,\\ 
15 & der ritterschefte herte.\\ 
 & \textbf{wer} twanc iuch \textbf{zuo} dirre verte?"\\ 
 & \textbf{dô} sprach der stolze degen junc:\\ 
 & "mir gebôt mîn veter Schiltunc,\\ 
 & des tohter Fridebrant dâ hât,\\ 
20 & daz ich im diente. ez wære sîn rât.\\ 
 & \textbf{er} hâ\textit{t} von sînem wîbe\\ 
 & hie von \textbf{mîme einigen} lîbe\\ 
 & sehs tûsent ritter wol bekant.\\ 
 & die tragent \textbf{werlîchiu bant}.\\ 
25 & \textbf{hie was} ouch ritter mêr durch in;\\ 
 & der ist ein teil gescheiden hin.\\ 
 & hie wâren durch \textbf{den} Schotten\\ 
 & die werlîchen rotten.\\ 
 & \textbf{hie was} von Gruonlanden\\ 
30 & helde zuo handen,\\ 
\end{tabular}
\scriptsize
\line(1,0){75} \newline
U V W T \newline
\line(1,0){75} \newline
\textbf{6} \textit{Majuskel} T  \textbf{7} \textit{Majuskel} T  \textbf{17} \textit{Initiale} T  \textbf{27} \textit{Initiale} W  \newline
\line(1,0){75} \newline
\textbf{2} div kvste den degen minnencliche T  $\cdot$ was] was so W \textbf{2} \textit{Die Verse 48.2\textasciicircum1-2\textasciicircum2 fehlen} T  \textbf{2} ir] irn U  $\cdot$ munde rôsenrôt] rote rosen munde rot W \textbf{4} \textit{Versfolge 48.3-4} T   $\cdot$ ir] irs U V W \textbf{3} moht ez] moͤht es V (W) mohtes T \textbf{5} vnd was von art ein kvnec riche T \textbf{6} Der sprach abr gvetliche T  $\cdot$ wirt] kúnig W \textbf{7} got weiz] Gotweiz wol T  $\cdot$ Kaylet] kylet U [K*ẏlet]: Kaẏlet V kyelet W \textbf{9} Spanie] yspanie U ẏspanie V hyspanie W \textbf{10} Kasganie] Gosganie U [*]: gascanie V gaßgonie W \textbf{11} iu tet dicke] tet dike [*]: v́ch V eúch thuͦt dick W tvͦt mir dicke T  $\cdot$ zornes] zoͤrner W \textbf{13} doch] \textit{om.} W \textbf{14} mit iu hie] hie mit eúch W \textbf{15} der] Die W \textbf{16} zuo] \textit{om.} T \textbf{18} veter] [*]: vatter V  $\cdot$ Schiltunc] schiliuͦnc U Schiltung V (W) \textbf{19} dâ] do V W \textbf{20} ich] \textit{om.} W \textbf{21} hât] hattte U \textbf{22} mîme einigen] min eines V (W) T \textbf{24} bant] hant V (W) T \textbf{27} hie] [*ie]: Die V SY W  $\cdot$ Schotten] Schoten T \textbf{29} hie was von] [*]: Jm kam von V hie waren T  $\cdot$ Gruonlanden] gruͦnlanden U [*]: grvͦnlanden V \textbf{30} zuo] [*]: zvͦ den V zuͦ ir W wol ze T \newline
\end{minipage}
\end{table}
\end{document}
