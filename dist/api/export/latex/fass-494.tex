\documentclass[8pt,a4paper,notitlepage]{article}
\usepackage{fullpage}
\usepackage{ulem}
\usepackage{xltxtra}
\usepackage{datetime}
\renewcommand{\dateseparator}{.}
\dmyyyydate
\usepackage{fancyhdr}
\usepackage{ifthen}
\pagestyle{fancy}
\fancyhf{}
\renewcommand{\headrulewidth}{0pt}
\fancyfoot[L]{\ifthenelse{\value{page}=1}{\today, \currenttime{} Uhr}{}}
\begin{document}
\begin{table}[ht]
\begin{minipage}[t]{0.5\linewidth}
\small
\begin{center}*D
\end{center}
\begin{tabular}{rl}
\textbf{494} & \begin{large}N\end{large}eve, nû wil ich sagen dir,\\ 
 & \textbf{daz} \textbf{dû maht wol} gelouben mir:\\ 
 & ein schanze dicke \textbf{stêt} vo\textit{r} i\textit{n},\\ 
 & si \textbf{gebent} unde \textbf{nement} gewin.\\ 
5 & si \textbf{enpfâhent} kleiniu kinder dar\\ 
 & von hôher art unt wol gevar.\\ 
 & wirt \textbf{iender} hêrrenlôs ein lant,\\ 
 & \textbf{erkennent} \textbf{si} dâ \textbf{die} gotes hant,\\ 
 & sô \textbf{daz} diu diet \textbf{eines} hêrren gert\\ 
10 & von\textbf{s} Grâles schar, \textbf{die} \textbf{sîn} gewert.\\ 
 & \textbf{des} muozen \textbf{ouch si} mit \textbf{zühten} pflegen;\\ 
 & sîn hüetet al dâ der gotes segen.\\ 
 & got schaffet verholne dan die man,\\ 
 & offenlîche gît man meide dan.\\ 
15 & dû solt des sîn vil gewis,\\ 
 & \textbf{daz} der künec Castis\\ 
 & Herzeloyden gerte,\\ 
 & der man in schône werte.\\ 
 & dîne muoter gap man im ze konen;\\ 
20 & er solde aber niht ir minne wonen,\\ 
 & der tôt in \textbf{ê} leite inz grap.\\ 
 & dâ vor er dîner muoter gap\\ 
 & Waleis unt Norgals,\\ 
 & Kanvoleiz und Kingrivals;\\ 
25 & daz ir mit \textbf{sale} wart gegeben.\\ 
 & der künec niht \textbf{lenger} solde leben.\\ 
 & \textbf{diz} was ûf sîner reise wider:\\ 
 & der \textbf{künec} \textbf{sich leite} \textbf{sterbens} nider.\\ 
 & dô truoc si krône über zwei lant.\\ 
30 & \textbf{dâ} erwarp \textbf{si} Gahmuretes hant.\\ 
\end{tabular}
\scriptsize
\line(1,0){75} \newline
D Fr11 Fr31 \newline
\line(1,0){75} \newline
\textbf{1} \textit{Initiale} D Fr11 Fr31  \newline
\line(1,0){75} \newline
\textbf{2} maht wol] wol macht Fr11 \textbf{3} vor in] von ir D \textbf{4} \textit{nach 494.4 am Rand nachgetragen:} So einer stirbet vnder in D  \textbf{8} dâ] daz Fr31  $\cdot$ die] divͯ Fr11 \textbf{9} daz] \textit{om.} Fr31 \textbf{10} vons] voner Fr11  $\cdot$ die] divͯ Fr11  $\cdot$ sîn] sint Fr11 Fr31 \textbf{11} muozen] mvͦz Fr31  $\cdot$ zühten] zvhte Fr31 \textbf{13} verholne] verholne in Fr11 \textbf{14} gît] gert Fr11  $\cdot$ meide dan] die mægeda::: Fr31 \textbf{15} des sîn] sein dez Fr11 \textbf{16} Castis] Kastis Fr31 \textbf{17} Herzeloyden] Herceloyden D Hertznlavden Fr11 Herzelavden Fr31  $\cdot$ gerte] gert Fr11 \textbf{18} werte] wert Fr11 \textbf{19} ze] \textit{om.} Fr31 \textbf{20} niht] \textit{om.} Fr11 Fr31  $\cdot$ minne] zeminne Fr31 \textbf{21} ê] \textit{om.} Fr31  $\cdot$ leite] lait Fr11 \textbf{23} Waleis] Walêis D Waleys Fr11 \textbf{24} Kanvoleiz] Chanvoleis Fr11 Kanvolays Fr31  $\cdot$ Kingrivals] Kyngrivals Fr11 kinrivals Fr31 \textbf{25} sale] \textit{om.} \textit{(Platz für das Wort ausgespart)} Fr31 \textbf{27} diz] Daz Fr31  $\cdot$ ûf] auch Fr11 \textbf{28} sich leite] legt sich Fr11  $\cdot$ sterbens nider] sterben::: Fr11 sterben nid::: Fr31 \textbf{30} dâ] der Fr11 Do Fr31  $\cdot$ Gahmuretes] Gahmvretes D Fr31 Gamuͯretes Fr11 \newline
\end{minipage}
\hspace{0.5cm}
\begin{minipage}[t]{0.5\linewidth}
\small
\begin{center}*m
\end{center}
\begin{tabular}{rl}
 & neve, nû wil ich sagen dir,\\ 
 & \textbf{des} \textbf{dû wol maht} glouben mir:\\ 
 & ein schanz dicke \textbf{stât} vor in,\\ 
 & si \textbf{gebent} und \textbf{nement} gewin.\\ 
5 & si \textbf{enpfâhent} kleiniu kint dar\\ 
 & von hôher art und wol gevar.\\ 
 & wirt \textbf{i\textit{e}n\textit{d}e\textit{rt}} hêrrenlôs ein lant,\\ 
 & \textbf{erkennet} \textbf{si} dâ \textbf{diu} gotes hant,\\ 
 & sô \textbf{daz} diu diet \textbf{einen} hêrren gert\\ 
10 & von Grâles schar, \textbf{die} \textbf{sint} gewert.\\ 
 & \textbf{des} muozen \textbf{ouch si} mit \textbf{zühten} pflegen;\\ 
 & sîn hüetet aldâ der gotes segen.\\ 
 & got schaffet verholn \textbf{in} dan die man,\\ 
 & offenlîch gît man \textbf{die} megde dan.\\ 
15 & dû solt des sîn vil gewis,\\ 
 & \textbf{daz} der künic Castis\\ 
 & Herczeloiden gerte,\\ 
 & der man in schône werte.\\ 
 & dîn muoter gap man im zuo k\textit{o}nen;\\ 
20 & er solt aber niht ir minne wonen,\\ 
 & der tôt in \textbf{ê} le\textit{i}te in daz grap.\\ 
 & dâ vor er dîner muoter gap\\ 
 & Waleis und Norgals,\\ 
 & Kanvoleiz und Kingrivals;\\ 
25 & daz ir mit \textbf{sale} wart geben.\\ 
 & der künic niht \textbf{langer} solte leben.\\ 
 & \textbf{diz} was ûf sîner reise wider:\\ 
 & der \textbf{künic} \textbf{sich leite} \textbf{sterbens} nider.\\ 
 & dô truoc si krône über zwei lant.\\ 
30 & \textbf{dô} erwarp \textbf{si} Gahmuretes hant.\\ 
\end{tabular}
\scriptsize
\line(1,0){75} \newline
m n o \newline
\line(1,0){75} \newline
\newline
\line(1,0){75} \newline
\textbf{2} des] Das o \textbf{5} kleiniu] cleide o  $\cdot$ kint] kinde m n o \textbf{7} iendert] jnnen m  $\cdot$ hêrrenlôs] herrem losz n \textbf{8} erkennet] Er kennent n (o)  $\cdot$ dâ] do n \textbf{9} diu] \textit{om.} o \textbf{11} des] Das o \textbf{17} Herczeloiden] Hertzoloiden n Herczeleiden o  $\cdot$ gerte] gert o \textbf{18} in schône werte] ẏm schone wert o \textbf{19} im] in n (o)  $\cdot$ konen] kennen m \textbf{20} niht] mit o \textbf{21} tôt] det n  $\cdot$ leite] lette m leit n  $\cdot$ grap] [gras]: grab m \textbf{24} Kanvoleiz] Kanfoleis m n o  $\cdot$ Kingrivals] kingruwals n konigrwals o \textbf{26} \textit{Versdoppelung 494.26-27 nach 494.27} o  \textbf{30} Gahmuretes] gahmuͯretes m gamuretes n gahúmuuͯretes o \newline
\end{minipage}
\end{table}
\newpage
\begin{table}[ht]
\begin{minipage}[t]{0.5\linewidth}
\small
\begin{center}*G
\end{center}
\begin{tabular}{rl}
 & \textit{\begin{large}N\end{large}}eve, nû wil ich sagen dir,\\ 
 & \textbf{daz} \textbf{dû wol maht} gelouben mir:\\ 
 & ein schanze dicke \textbf{stêt} vor in,\\ 
 & si \textbf{gebent} unde \textbf{nement} gewin.\\ 
5 & si \textbf{enpfâhent} kleiniu kinder dar\\ 
 & von hôher art unde wol gevar.\\ 
 & wirt \textbf{iender} hêrrenlôs ein lant,\\ 
 & \textbf{erkennet} \textbf{si} d\textit{â} \textbf{di\textit{u}} gotes hant,\\ 
 & sô \textbf{daz} diu diet \textbf{eines} hêrren gert\\ 
10 & von\textbf{es} Grâles schar, \textbf{die} \textbf{sint} gewert.\\ 
 & \textbf{des} müezen \textbf{ouch si} mit \textbf{zühten} pflegen;\\ 
 & sîn hüetet al dâ der gotes segen.\\ 
 & got schaffet verholn dan die man,\\ 
 & offenlîchen gît man \textbf{die} meide dan.\\ 
15 & dû solt des sîn vil gewis,\\ 
 & \textbf{daz} der künic Castis\\ 
 & Herzeloide gerte,\\ 
 & der man in schône werte.\\ 
 & dîne muoter gap man im ze konen;\\ 
20 & er solt aber niht ir minne \textit{w}onen,\\ 
 & der tôt in \textbf{ê} leit in daz grap.\\ 
 & dâ vor er dîner muoter gap\\ 
 & Waleis unde Nurgals,\\ 
 & Kanvoleiz unde Kinrivals;\\ 
25 & daz ir mit \textbf{sale} wart gegeben.\\ 
 & der künic niht \textbf{lenger} solde leben.\\ 
 & \textbf{daz} was ûf sîner reise wider:\\ 
 & der \textbf{künic} \textbf{\textit{leit} sich} \textbf{sterbens} nider.\\ 
 & dô truoc si krôn über zwei lant.\\ 
30 & \textbf{dâ} erwarp \textbf{si} Gahmuretes hant.\\ 
\end{tabular}
\scriptsize
\line(1,0){75} \newline
G I L M Z Fr49 \newline
\line(1,0){75} \newline
\textbf{1} \textit{Initiale} G I L M Z  \textbf{15} \textit{Initiale} I  \newline
\line(1,0){75} \newline
\textbf{1} Neve] Eeve \textit{(Initialbuchstabe} n\textit{ vorgeschrieben)} G \textbf{2} dû wol maht] mahtu wol I duͯ macht wol L (M) (Z) \textbf{5} kleiniu] chlaine Fr49  $\cdot$ kinder] kindel I (Fr49) kindelin L \textbf{6} von] \textit{om.} L \textbf{7} iender] irgen M in der Fr49 \textbf{8} erkennet] Erkennent Z  $\cdot$ dâ diu] daz die G da L \textbf{9} diet] \textit{om.} Z \textbf{10} vones] Von M  $\cdot$ die sint] si ist I si sint L ist si Fr49 \textbf{11} des] Das M  $\cdot$ ouch si] sie ouch L M \textbf{12} al] \textit{om.} L  $\cdot$ der] des Z \textbf{13} schaffet] schephet M  $\cdot$ dan] in da L daryn M  $\cdot$ die] den M \textbf{14} die meide offenlichen dan I (Fr49)  $\cdot$ Offenlich die meide git man dan Z \textbf{15} dû solt des] Dez soltu L \textbf{16} Castis] kostis L kastis M \textbf{17} Herzeloide] Herzeloyde G herzenlauden I Hertzeleuden L Herczelouden M Hertzenlovden Z Herzenlaude Fr49  $\cdot$ gerte] gert G \textbf{18} in schône werte] schone in gewerte L \textbf{20} er] ern I  $\cdot$ niht] \textit{om.} M mit Fr49  $\cdot$ wonen] gewonen G \textbf{21} ê] \textit{om.} I M Fr49  $\cdot$ leit] leite M Z \textbf{22} dâ vor] Davon L \textbf{23} Nurgals] Norgals G I Z Norglas L [ne*]: norgals Fr49 \textbf{24} Kanvoleiz] kanfolaiz I Kamvoleis L Z Kanvoleis M kanfoleis Fr49  $\cdot$ Kinrivals] kanriuals G kiriuals I kingrivals L M (Z) [niri]: kirinals  Fr49 \textbf{25} sale] selde I (Fr49) \textbf{27} \textit{Die Verse 494.27-28 fehlen} Z   $\cdot$ daz] Disz L M \textbf{28} leit sich] sih G sich leite sich L sich leite M  $\cdot$ sterbens] sterben I (L) (Fr49) \textbf{29} dô] Da M Z \textbf{30} dâ] do I Fr49  $\cdot$ si] \textit{om.} Z  $\cdot$ Gahmuretes] gahmures G Gahmversz L gamuretes M Z Gamuretez Fr49 \newline
\end{minipage}
\hspace{0.5cm}
\begin{minipage}[t]{0.5\linewidth}
\small
\begin{center}*T
\end{center}
\begin{tabular}{rl}
 & \textit{\begin{large}N\end{large}}eve, nû wil ich sagen dir,\\ 
 & \textbf{daz} \textbf{maht dû wol} gelouben mir:\\ 
 & ein schanze dicke \textbf{liget} vor in,\\ 
 & si \textbf{enpfâhent} unde \textbf{gebent} gewin.\\ 
5 & Si \textbf{nement} kleiniu kint dar\\ 
 & von hôher art unde wol gevar.\\ 
 & wirt \textbf{aber} \textbf{iemer} hêrrenlôs ein lant,\\ 
 & \textbf{erkennet} \textbf{man} dâ \textbf{die} gotes hant,\\ 
 & sô diu diet \textbf{eines} hêrren gert\\ 
10 & von \textbf{des} Grâles schar, \textbf{si} \textbf{sint} gewert.\\ 
 & \textbf{die} müezen \textbf{sîn} mit \textbf{triuwen} pflegen;\\ 
 & sîn hüetet aldâ der gotes segen.\\ 
 & got schaffet verholn dan die man,\\ 
 & offenlîche gît man \textbf{die} megde dan.\\ 
15 & dû solt des sîn vil gewis,\\ 
 & \textbf{dô} der künec Castis\\ 
 & Herzeloyden gerte,\\ 
 & der man in schône werte.\\ 
 & dîne muoter gap man im ze ko\textit{n}en;\\ 
20 & er\textbf{n} solte aber niht \textbf{in} ir minne wonen,\\ 
 & der tôt in leite in daz grap.\\ 
 & dâ vo\textit{r} er dîner muoter gap\\ 
 & Waleis unde Norgals,\\ 
 & Kanvoleis unde Kingrivals;\\ 
25 & daz ir mit \textbf{zal} wart gegeben,\\ 
 & \textbf{dô} der künec niht \textbf{mêr} solte leben.\\ 
 & \textbf{daz} wa\textit{s û}f sîner reise wider:\\ 
 & der \textbf{sich leite} \textbf{sterben} nider.\\ 
 & dô truoc si krône über zwei lant.\\ 
30 & \textbf{daz} erwarp Gahmuretes hant.\\ 
\end{tabular}
\scriptsize
\line(1,0){75} \newline
T U V W O Q R Fr40 \newline
\line(1,0){75} \newline
\textbf{1} \textit{Initiale} T V O Fr40  \textbf{5} \textit{Majuskel} T  \newline
\line(1,0){75} \newline
\textbf{1} \textit{Die Verse 453.1-502.30 fehlen} U   $\cdot$ \textit{Die Verse 493.9-494.18 fehlen} R   $\cdot$ Neve] Leve T ÷eve O  $\cdot$ nû] \textit{om.} V  $\cdot$ wil ich] ich wil V \textbf{2} daz] Des W  $\cdot$ maht dû] du magst W (O) Q (Fr40) \textbf{3} dicke liget] ligt diche O \textbf{7} iemer] iender W O (Fr40) nindert Q \textbf{8} erkennet man] [Erkenne*]: Erkennent sv́ V  $\cdot$ dâ] do V W Q \textbf{9} sô] [*]: So daz V \textbf{11} [D*]: Die mvͤzent [*]: sin mit truwen pflegen V \textbf{12} aldâ] do W  $\cdot$ der] des Q \textbf{13} dan] [*]: in V  $\cdot$ die] [*]: die V den Fr40 \textbf{16} dô] Das W (O) Q (Fr40)  $\cdot$ Castis] kastis V W O \textbf{17} Herzeloyden] Herzelauden V Hertzeloyden W Herzelovden O Herzelouden Q (Fr40) \textbf{18} der] Das W  $\cdot$ werte] :erte Fr40 \textbf{19} ze konen] ze kômen T [z*]: zekonen V \textbf{20} ern] Er V W O R er: Fr40  $\cdot$ in] \textit{om.} V W O Q R Fr40 \textbf{21} der tôt] [D*]: Der tot e V  $\cdot$ leite] leit O \textbf{22} vor] von T V R \textbf{23} Waleis] Walleis V Waleys O Q  $\cdot$ Norgals] nyrgals Q Nurgals R (Fr40) \textbf{24} Kanvoleis] Kanuoleys W Kanvoleys Q kanwoleis Fr40  $\cdot$ Kingrivals] kyngrvals T kyingrivals V kingriuals W kyngriuals Q \textbf{25} ir] \textit{om.} Q Fr40  $\cdot$ zal] [*l]: zale V sal W O (Q) R (Fr40) \textbf{26} dô] \textit{om.} W O Q R Fr40  $\cdot$ mêr] lenger W (O) Q R Fr40 \textbf{27} was ûf] was was vf T was [vf]: avf Fr40 \textbf{28} der] Do er V Der kúnig W (O) (Q) (Fr40)  $\cdot$ leite] legt O \textbf{29} dô] da Fr40  $\cdot$ krône] corn Q \textbf{30} daz] Do V W Q R (Fr40) Da O  $\cdot$ erwarp] er warp T (Q) erwarp sv́ V (W) (O) (R) (Fr40)  $\cdot$ Gahmuretes] gamvretes V (O) gamurettes W gamuretes Q \newline
\end{minipage}
\end{table}
\end{document}
