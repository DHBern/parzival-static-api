\documentclass[8pt,a4paper,notitlepage]{article}
\usepackage{fullpage}
\usepackage{ulem}
\usepackage{xltxtra}
\usepackage{datetime}
\renewcommand{\dateseparator}{.}
\dmyyyydate
\usepackage{fancyhdr}
\usepackage{ifthen}
\pagestyle{fancy}
\fancyhf{}
\renewcommand{\headrulewidth}{0pt}
\fancyfoot[L]{\ifthenelse{\value{page}=1}{\today, \currenttime{} Uhr}{}}
\begin{document}
\begin{table}[ht]
\begin{minipage}[t]{0.5\linewidth}
\small
\begin{center}*D
\end{center}
\begin{tabular}{rl}
\textbf{67} & hie \textbf{sint} die Waleise,\\ 
 & \textbf{daz} si \textbf{behabent} \textbf{ir} reise\\ 
 & durch \textbf{den} poinder, swâ sis gernt.\\ 
 & von \textbf{der} kraft ir landes si des wernt.\\ 
5 & hie ist manec ritter durch diu wîp,\\ 
 & \textbf{des} niht erkennen \textbf{mac} mîn lîp.\\ 
 & \textbf{al} die ich \textbf{hie} \textbf{benennet} hân,\\ 
 & wir ligen mit wârheit sunder wân\\ 
 & mit grôzer vuore in der stat,\\ 
10 & als uns diu küneginne bat.\\ 
 & Ich sage dir, wer ze velde ligt,\\ 
 & \textbf{die} unser \textbf{wer} vil ringe wigt:\\ 
 & der werde künec von Ascalun\\ 
 & unt der \textbf{stolze} künec von Arragun,\\ 
15 & Cidegast \textbf{von} \textbf{Logroys}\\ 
 & unt der künec von Punturtoys,\\ 
 & der \textbf{heizet} Brandelidelin.\\ 
 & \textbf{dâ} ist \textbf{ouch} der küene Læhelin.\\ 
 & \textbf{dâ} ist Morholt von Yrlant,\\ 
20 & der brichet \textbf{ab uns} \textbf{beidiu} pfant.\\ 
 & \textbf{Dâ} ligent ûf \textbf{dem plâne}\\ 
 & die stolzen Alemane.\\ 
 & der herzoge von Brabant\\ 
 & ist gestrichen in ditze lant\\ 
25 & durch den künec Hardizen.\\ 
 & sîne swester Alizen\\ 
 & gap i\textit{m} der künec von Gascon.\\ 
 & sîn dienest hât vor enpfangen lôn.\\ 
 & \textbf{\begin{large}D\end{large}ie} sint mit zorne \textbf{hie} gein mir.\\ 
30 & nû sol ich wol getrûwen dir.\\ 
\end{tabular}
\scriptsize
\line(1,0){75} \newline
D Fr9 Fr33 \newline
\line(1,0){75} \newline
\textbf{11} \textit{Majuskel} D  \textbf{21} \textit{Majuskel} D  \textbf{29} \textit{Initiale} D  \newline
\line(1,0){75} \newline
\textbf{4} der kraft ir landes] ir lant kraft Fr9  $\cdot$ wernt] wern Fr9 \textbf{8} wir ligen] die ligent Fr9 \textbf{15} Cidegast] Sidgast D  $\cdot$ von] die Fr9  $\cdot$ Logroys] logrois Fr9 \textbf{16} Punturtoys] pvntertoẏs Fr9 \textbf{18} dâ] Hie Fr9  $\cdot$ Læhelin] Lehelin D \textbf{19} dâ ist] Vnde Fr9  $\cdot$ Morholt] morolt Fr9  $\cdot$ Yrlant] ẏralant Fr9 \textbf{20} Der ist von grozer kraft ir kant Fr9 \textbf{21} ûf] ouch of Fr9 \textbf{22} Alemane] Alemâne D \textbf{24} ist] Der ist Fr9 \textbf{25} Hardizen] Hardŷsen D \textbf{26} Alizen] Alîzen D \textbf{27} im] in D \newline
\end{minipage}
\hspace{0.5cm}
\begin{minipage}[t]{0.5\linewidth}
\small
\begin{center}*m
\end{center}
\begin{tabular}{rl}
 & hie \textbf{sint} die Waleise,\\ 
 & \textbf{daz} si \textbf{behaben\textit{t}} \textbf{ir} reise\\ 
 & durch \textbf{die} poinder, wâ si es gernt.\\ 
 & von \textbf{der} kraft ir landes si des wernt.\\ 
5 & \textit{h}ie ist manic ritter durch diu wîp,\\ 
 & \textbf{des} niht erkennen \textbf{mac} mîn lîp.\\ 
 & \textbf{al} die ich \textbf{hie} \textbf{benennet} hân,\\ 
 & wir ligen mit wârheit sunder wân\\ 
 & mit grôzer v\textit{uo}re in der stat,\\ 
10 & als uns diu künigîn bat.\\ 
 & ich sage dir, wer ze velde liget,\\ 
 & \textbf{die} unser \textbf{wer} vil ringe wiget:\\ 
 & der werde künic von Ascalun\\ 
 & und der \textbf{stolze} künic von Aragun,\\ 
15 & Zidegast \textbf{de} \textbf{Logrois}\\ 
 & und der künic von Ponturtois,\\ 
 & der \textbf{herzoge} Brandelide\textit{l}in.\\ 
 & \textbf{dâ} ist \textbf{ouch} der küene Lehelin.\\ 
 & \textbf{d\textit{â}} ist Morolt von Irlant,\\ 
20 & der brichet \textbf{ab uns} \textbf{gæben} pfant.\\ 
 & \textbf{d\textit{â}} ligent ûf \textbf{dem plâne}\\ 
 & die stolzen Alemane.\\ 
 & der herzoge von Br\textit{ab}ant\\ 
 & ist gestrichen in diz lant\\ 
25 & durch den künic Hardizen.\\ 
 & sîne swester Alizen\\ 
 & gap ime der künic von Gascon.\\ 
 & sîn dienst hât vor enpfangen lôn.\\ 
 & \textbf{die} sint mit zorne \textbf{hie} gegen mir.\\ 
30 & nû sol ich wol getrûwen dir.\\ 
\end{tabular}
\scriptsize
\line(1,0){75} \newline
m n o \newline
\line(1,0){75} \newline
\newline
\line(1,0){75} \newline
\textbf{2} behabent] behaben m \textbf{3} gernt] gern m \textbf{4} des] das n o \textbf{5} hie] Nie \textit{nachträglich korrigiert zu:} Hie m \textbf{9} vuore] fare m \textbf{13} Ascalun] ascaluͯn m ascaluͦn n ascalon o \textbf{14} Aragun] araguͯn m arraguͦn n arraguͯn o \textbf{15} Zidegast] Cide gast m Cidigast n o  $\cdot$ de] von de n  $\cdot$ Logrois] logoris n loͯgrois o \textbf{16} \textit{Vers 67.16 fehlt} n   $\cdot$ Ponturtois] [pantotis]: pantertois o \textbf{17} Brandelidelin] brandelidein m brandelin n brandolin o \textbf{18} dâ] Do n o  $\cdot$ Lehelin] [leor]: lehelin o \textbf{19} dâ] Do m n o  $\cdot$ Morolt] morlt o  $\cdot$ Irlant] ẏrlant o \textbf{21} dâ] Das m Do n o \textbf{22} stolzen] stolcze o  $\cdot$ Alemane] allemone n allamone o \textbf{23} Brabant] brobrant m n o \textbf{25} Hardizen] hardiczen m o harditzen n \textbf{26} Alizen] aliczen m o alitzen n \textbf{27} ime der] vor dem o \textbf{30} wol] [gegen]: wol o \newline
\end{minipage}
\end{table}
\newpage
\begin{table}[ht]
\begin{minipage}[t]{0.5\linewidth}
\small
\begin{center}*G
\end{center}
\begin{tabular}{rl}
 & hie \textbf{sint} die Waleise,\\ 
 & \textbf{daz} si \textbf{behabent} \textbf{die} reise\\ 
 & durch \textbf{den} ponder, swâ sis gernt.\\ 
 & von \textbf{der} kraft ir landes si des wernt.\\ 
5 & hie ist manic rîter durch diu wîp,\\ 
 & \textbf{des} niht erkennen \textbf{mac} mîn lîp,\\ 
 & \textbf{wan} die ich \textbf{dir} \textbf{benennet} hân.\\ 
 & wir ligen mit wârheit sunder wân\\ 
 & mit grôzer vuore in der stat,\\ 
10 & als uns diu küniginne bat.\\ 
 & ich sage dir, wer ze velde liget,\\ 
 & \textbf{die} unser \textbf{strît} vil ringe wiget:\\ 
 & der werde künic von Aschalun\\ 
 & unt der \textbf{vreche} künic von Arragun.\\ 
15 & \textbf{d\textit{â}} \textbf{ist} Zidegast \textbf{von} \textbf{Orileis}\\ 
 & unt der künic von Ponturteis,\\ 
 & der \textbf{heizet} Brandelidelin.\\ 
 & \textbf{hie} ist der küene Lehelin.\\ 
 & \textbf{\textit{\begin{large}D\end{large}â}} ist Morolt von Yrlant,\\ 
20 & der brichet \textbf{abe uns} \textbf{gæbiu} pfant.\\ 
 & \textbf{hie} ligent ûf \textbf{der plânje}\\ 
 & die stolzen Almanje.\\ 
 & der herzoge von Brabant,\\ 
 & \textbf{der} ist gestrichen in diz lant\\ 
25 & durch den künic Hardizen.\\ 
 & sîne swester Alizen\\ 
 & gap im der künic von Gascon.\\ 
 & sîn dienst hât vor enpfangen lôn.\\ 
 & \textbf{die} sint mit zorne \textbf{hie} gein mir.\\ 
30 & nû sol ich wol getrûwen dir.\\ 
\end{tabular}
\scriptsize
\line(1,0){75} \newline
G I O L M Q R Z Fr21 Fr44 \newline
\line(1,0){75} \newline
\textbf{2} \textit{Initiale} O  \textbf{5} \textit{Initiale} M Z Fr44   $\cdot$ \textit{Capitulumzeichen} L  \textbf{9} \textit{Initiale} I  \textbf{19} \textit{Initiale} G  \textbf{25} \textit{Initiale} I  \newline
\line(1,0){75} \newline
\textbf{1} \textit{Versfolge 67.2-1} O   $\cdot$ Waleise] walleise I M waleyse Q \textbf{2} daz] ÷az O  $\cdot$ behabent] behabten O  $\cdot$ die] ir O L M Q R Z Fr44 \textbf{3} Durch die poynder wan si sin gerne Z  $\cdot$ swâ] \textit{om.} O wa L M (Q) R \textbf{4} si des] \textit{om.} O sis L  $\cdot$ wernt] werne Z \textbf{5} diu] \textit{om.} L \textbf{6} erkennen] bekennen R  $\cdot$ mîn] sin R \textbf{7} wan die] Wan Q Alle die Z  $\cdot$ benennet hân] genennet han I (L) hie benennen chan O hier benomet han M die hẏ gennet han Q hie genomet han R (Z) \textbf{8} sunder] ane L \textbf{9} grôzer vuore] grosze fure M grozen furen Fr44 \textbf{11} wer] swer Fr44 \textbf{12} unser] vnsz R  $\cdot$ strît] wer O L Q R (Z) Fr44 \textit{om.} M  $\cdot$ ringe] cleyne M (Fr44) ringer R \textbf{13} der] des I  $\cdot$ Aschalun] ascalun G (M) (Z) (Fr44) ascholún Q ascalon R \textbf{14} vreche] werde O M stoltze Z  $\cdot$ künic] \textit{om.} Fr44  $\cdot$ Arragun] arragún Q aragon R \textbf{15} dâ] daz G Do O Q Fr44  $\cdot$ Zidegast] Cistegast L citegast M cidegast Q (R) Cydegast Z Cithegast Fr44  $\cdot$ Orileis] oruleis G orilais L Orielis M Orligeis R longroys Z Logroẏs Fr44 \textbf{16} Ponturteis] pontorteis I pvntvreis O Pvntuͯrtais L pvndurteis M púnturtoys Q puͦntuͦrteis R pvntvrteis Z Ponturtoẏs Fr44 \textbf{17} Brandelidelin] prandelidelin G Brandlidelin O (Q) (Z) Fr44 Brantlidelin L (M) \textbf{18} hie] Do O Q Fr44 Da L M R Z  $\cdot$ ist] ist avch O (L) (M) (Q) (R) (Z)  $\cdot$ küene] kunc I (M) (Q) (R)  $\cdot$ Lehelin] lechelin R [leh*lin]: lehelin Z \textbf{19} Dâ] Hie G Do O Q  $\cdot$ ist] ist ouch L  $\cdot$ Morolt] morholt I (O) (L) M (R) Z (Fr21) [*orolt]: Morolt  Fr44  $\cdot$ von] vnd Q vo Fr21  $\cdot$ Yrlant] ẏrlant G irlant I (M) Q Jerlant O jrlant L (Z) (Fr44) Jersant Fr21 \textbf{20} brichet] briche Q  $\cdot$ abe uns] vns ab I (L) Fr21 (Fr44)  $\cdot$ gæbiu] div geben O garbi L der gæben Fr21 \textbf{21} hie] Da O M R Z Fr21 Fr44 Die L Do Q  $\cdot$ ligent] legin M  $\cdot$ der plânje] dem plane L (Q) (Z) (Fr44) den plange R \textbf{22} die] Der L  $\cdot$ Almanje] almanige G almange I almanîe O Almane L (M) (Fr21) almigange Q alimange R alamane Z Alimâne Fr44 \textbf{23} Brabant] brabrange Q pravant Z Prauant Fr44 \textbf{24} der] \textit{om.} I Z  $\cdot$ diz] das R \textbf{25} Hardizen] hardiezen O (Fr21) hardisen L hardysen R hardizin Z \textbf{26} Alizen] aliezen O Q (Fr21) Alysen L alyzen R aliezin Z \textbf{27} von] >von< O \textit{om.} Fr21  $\cdot$ Gascon] [cascon]: caschon G Gatschon I ascon O Gaschon L aschon M gaston R \textbf{28} hât] her R \textbf{29} gein] gitan M ge Z \textbf{30} wol] vil wol O \newline
\end{minipage}
\hspace{0.5cm}
\begin{minipage}[t]{0.5\linewidth}
\small
\begin{center}*T (U)
\end{center}
\begin{tabular}{rl}
 & hie \textbf{hânt} die Waleise\\ 
 & \textbf{durch} si \textbf{behabet} \textbf{ir} reise\\ 
 & durch \textbf{den} poynder, wâ si e\textit{s} gernt.\\ 
 & von \textbf{ir} kraft ir landes si des wer\textit{n}t.\\ 
5 & hie ist manec ritter durch diu wîp,\\ 
 & \textbf{der} niht erkennen \textbf{kan} mîn lîp,\\ 
 & \textbf{wan} die ich \textbf{dir} \textbf{hie} \textbf{genennet} hân.\\ 
 & wir ligen mit wârheit sunder wân\\ 
 & mit grôzer vuore in der stat,\\ 
10 & als uns diu küneginne bat.\\ 
 & ich sage dir, wer zuo velde liget,\\ 
 & \textbf{den} unser \textbf{vuore} vil ringe wiget:\\ 
 & der werde künec von Ascalun\\ 
 & und der \textbf{vreche} künec von Arragun.\\ 
15 & \textbf{hie} \textbf{ist} Cydegast \textbf{von} \textbf{Logroys}\\ 
 & und der künec von Puntertoys,\\ 
 & der \textbf{heizet} Brandelidelin,\\ 
 & \textbf{hie} ist \textbf{ouch} der küene Lehelin.\\ 
 & \textbf{hie} ist Morolt von Irlant,\\ 
20 & der brichet \textbf{uns abe} \textbf{gæbiu} pfant.\\ 
 & \textbf{hie} ligent ûf \textbf{dem plâne}\\ 
 & die stolzen Alamane.\\ 
 & der herzoge von Brabant\\ 
 & ist gestrichen in diz lant\\ 
25 & durch den künec Hardysen.\\ 
 & sîne swester Alysen\\ 
 & gab i\textit{m} der künec von Gasgon.\\ 
 & sîn dienst hât vor entvangen lôn.\\ 
 & \textbf{si} sint mit zorne gein mir.\\ 
30 & nû sol ich wol getrûwen dir.\\ 
\end{tabular}
\scriptsize
\line(1,0){75} \newline
U V W T \newline
\line(1,0){75} \newline
\textbf{1} \textit{Majuskel} T  \textbf{4} \textit{Majuskel} T  \textbf{5} \textit{Initiale} T  \textbf{13} \textit{Majuskel} T  \textbf{19} \textit{Majuskel} T  \textbf{29} \textit{Initiale} W  \newline
\line(1,0){75} \newline
\textbf{1} hânt] sint V (W) T  $\cdot$ die] die von W  $\cdot$ Waleise] walleẏse V \textbf{2} durch] Daz V (W) (T)  $\cdot$ behabet] behabent V (W) (T)  $\cdot$ ir] die V \textbf{3} poynder] pvneiz T  $\cdot$ wâ] swo V (T)  $\cdot$ es] iz U \textbf{4} Jr stichgenôze si crefte wernt T  $\cdot$ ir kraft] craft V  $\cdot$ wernt] wert U \textbf{6} der] Des V W (T)  $\cdot$ kan] mag W \textbf{7} dir] \textit{om.} T  $\cdot$ hie] \textit{om.} W  $\cdot$ genennet] [*an]: benennet V benennet W \textbf{12} den unser vuore] Die [*sser]: vsser wer V Den vnser W die vnser wer T  $\cdot$ ringe] lútzel W \textbf{13} Ascalun] Ascaluͦn U astalun W \textbf{14} vreche] \textit{om.} T  $\cdot$ Arragun] aragun U W Arragv̂n T \textbf{15} Cydegast] Gidegast V zytegast W  $\cdot$ von] vnd W  $\cdot$ Logroys] Logroẏs V oriles W Logrôeiz T \textbf{16} Puntertoys] Puͦntercoys U Puntroẏs V pontertes W Pvnterteiz T \textbf{18} vnd der kvnec Lehelin T  $\cdot$ küene] kúnig W  $\cdot$ Lehelin] Loͤhelin V \textbf{19} hie] Da T  $\cdot$ ist] ist oͮch V  $\cdot$ Morolt] morholt W  $\cdot$ Irlant] Jrlant U (T) ẏrlant V yrland W \textbf{20} gæbiu] vil gebe W \textbf{21} hie] da T \textbf{22} Alamane] alemane W alamâne T \textbf{23} Brabant] braband W \textbf{24} ist] der ist T \textbf{25} Hardysen] Hardisen V (W) \textbf{26} Alysen] Alisen V (W) \textbf{27} gab im] Gabin U  $\cdot$ Gasgon] gasgane W Gasgôn T \textbf{28} Mit dem ist er hie nach minnen lane W  $\cdot$ hât] [*]: hatte V \textbf{29} si] DYe W (T)  $\cdot$ gein] hie gegn T \newline
\end{minipage}
\end{table}
\end{document}
