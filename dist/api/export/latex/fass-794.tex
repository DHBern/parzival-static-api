\documentclass[8pt,a4paper,notitlepage]{article}
\usepackage{fullpage}
\usepackage{ulem}
\usepackage{xltxtra}
\usepackage{datetime}
\renewcommand{\dateseparator}{.}
\dmyyyydate
\usepackage{fancyhdr}
\usepackage{ifthen}
\pagestyle{fancy}
\fancyhf{}
\renewcommand{\headrulewidth}{0pt}
\fancyfoot[L]{\ifthenelse{\value{page}=1}{\today, \currenttime{} Uhr}{}}
\begin{document}
\begin{table}[ht]
\begin{minipage}[t]{0.5\linewidth}
\small
\begin{center}*D
\end{center}
\begin{tabular}{rl}
\textbf{794} & \textbf{\textit{S}i} vunden volkes ungezalt,\\ 
 & manegen \textbf{wünneclîchen} rîter alt,\\ 
 & edeliu kint, vil scharjante.\\ 
 & die \textbf{trûrigen} mahinante\\ 
5 & dirre \textbf{künfte} \textbf{vrô wol} \textbf{mohten} sîn.\\ 
 & Feirefiz Anschevin\\ 
 & unt Parzival, si bêde,\\ 
 & vor dem palase \textbf{an} der \textbf{grêde}\\ 
 & \textbf{si} wurden wol enpfangen.\\ 
10 & \textbf{in} den palas wart gegangen.\\ 
 & Dâ lac nâch ir gewonheit\\ 
 & \textbf{hundert sinwelle} teppech breit,\\ 
 & ûf ieslîchem ein pflûmît\\ 
 & unt ein kulter \textbf{lanc} von samît.\\ 
15 & Vuoren die zwêne mit witzen,\\ 
 & si mohten \textbf{etswâ} \textbf{dâ} \textbf{sitzen},\\ 
 & unz manz harnasch von in enpfienc.\\ 
 & ein kamerære dar nâher gienc.\\ 
 & der brâht in kleider rîche,\\ 
20 & den \textbf{beiden} \textbf{al} gelîche.\\ 
 & Si sâzen, swaz dâ rîter was.\\ 
 & man truoc von golde - ez was niht glas -\\ 
 & vür si manegen tiwern schâl.\\ 
 & Feirefiz unt Parzival\\ 
25 & trunken und giengen dan\\ 
 & zAnfortase, dem trûrigem man.\\ 
 & Ir habt \textbf{ê} \textbf{vernomen}, daz\\ 
 & \textbf{der} lente unt \textbf{daz er} \textbf{selten} saz\\ 
 & unt wie sîn bette gehêrt was.\\ 
30 & \textbf{dise} zwêne enpfienc dô Anfortas\\ 
\end{tabular}
\scriptsize
\line(1,0){75} \newline
D \newline
\line(1,0){75} \newline
\textbf{1} \textit{Initiale} D  \textbf{11} \textit{Majuskel} D  \textbf{15} \textit{Majuskel} D  \textbf{21} \textit{Majuskel} D  \textbf{27} \textit{Majuskel} D  \newline
\line(1,0){75} \newline
\textbf{1} Si] ÷i D \textbf{6} Anschevin] Anscivin D \textbf{7} Parzival] Parcifal D \textbf{17} manz] [mans]: manz D \textbf{24} Parzival] Parcifal D \newline
\end{minipage}
\hspace{0.5cm}
\begin{minipage}[t]{0.5\linewidth}
\small
\begin{center}*m
\end{center}
\begin{tabular}{rl}
 & \textbf{si} vunden volkes ungezalt,\\ 
 & manig\textit{en} \textbf{wünneclîchen} ritter alt,\\ 
 & edeliu kint, vil sar\textit{ja}n\textit{t}e.\\ 
 & die \textbf{trûrigen} mahinante\\ 
5 & diser \textbf{geste} \textbf{vrô} \textbf{mohten} sîn.\\ 
 & Ferefiz A\textit{n}schevin\\ 
 & und Parcifal, si bêde,\\ 
 & vor dem palas \textbf{vor} der \textbf{grêde}\\ 
 & wurden wol enpfangen.\\ 
10 & \textbf{in} den palas wart gegangen.\\ 
 & d\textit{â} lac nâch ir gewonheit\\ 
 & \textbf{hundert sinewel} teppich breit,\\ 
 & ûf ieglîchem ein plûmît\\ 
 & und ein kulter \textbf{lanc} von samît.\\ 
15 & vuoren die zwêne mit witzen,\\ 
 & si mohten \textbf{etwâ} \textbf{sitzen},\\ 
 & unz man daz harnasch von in enpfienc.\\ 
 & ein kamerer dâ nâher gienc.\\ 
 & der brâht i\textit{n} kleider rîch,\\ 
20 & den \textbf{beiden} \textbf{al}gelîch.\\ 
 & si sâzen, waz d\textit{â} ritter was.\\ 
 & man truoc von golde - ez was niht glas -\\ 
 & vür si manigen tiuren schâl.\\ 
 & Ferefiz und Parcifal\\ 
25 & trunken und giengen dan\\ 
 & zuo Anfortasse, dem trûrigen man.\\ 
 & ir habt \textbf{wol ê} \textbf{vernomen}, daz\\ 
 & \textbf{der} lente und \textbf{selten} saz\\ 
 & und wie sîn bette gehêret was.\\ 
30 & \textbf{dise} zwên enpfienc dô Anfortas\\ 
\end{tabular}
\scriptsize
\line(1,0){75} \newline
m n o V V' W \newline
\line(1,0){75} \newline
\newline
\line(1,0){75} \newline
\textbf{2} manigen] Manig m o  $\cdot$ wünneclîchen] manlichen V' \textbf{3} \textit{Die Verse 794.3-4 fehlen} V'   $\cdot$ vil] vnd W  $\cdot$ sarjante] sarrazine m n (o) \textbf{4} trûrigen] [trugen]: trurigen o \textbf{5} diser] Die ir V V'  $\cdot$ vrô] fro wol V wol fro V'  $\cdot$ mohten] moͯchten n (V) \textbf{6} Ferefiz] Ferefis m o Ferrevis n Artus fereuis V Artus ferevis V' Eerafiß W  $\cdot$ Anschevin] auscevin m n anscevin o antschevin V' antscheuein W \textbf{7} Parzefal vnd die touelrunder stete V (V')  $\cdot$ Parcifal] [par*]: parcipal n Parzifal V' herr partzifal W \textbf{8} vor] an V V' W \textbf{9} wurden] Wordin sie V' \textbf{10} den] dem o \textbf{11} dâ] Do m n o V V' W  $\cdot$ gewonheit] wonheit V' \textbf{12} teppich] [tepite]: teppte V' \textbf{14} lanc] [*]: lag V do lac V' \textbf{15} \textit{Die Verse 794.15-20 fehlen} V'   $\cdot$ zwêne] geste V \textbf{16} mohten] moͯchten n (V)  $\cdot$ etwâ] ezwo do V \textbf{17} man daz] daz man W \textbf{18} dâ] do W \textbf{19} in] im m \textbf{20} beiden algelîch] gesten allen gliche V \textbf{21} \textit{statt 794.21-23:} Sie saszen nider do mit schal V'   $\cdot$ waz] swaz V  $\cdot$ dâ] do m n o V W \textbf{23} manigen tiuren] mange teúre W \textbf{24} Ferefis vnd parcifal m o Ferrefis vnd parcifal n Artus fereuis vnd parzefal V Artus [parzifal]: ferevis vnd parzifal V' Ferafis vnd partzifal W \textbf{25} trunken] Trincken n Vnde die tovelrunder trunkent V (V')  $\cdot$ dan] [dar]: dan m \textbf{26} Anfortasse] anfortasze V anfortas V' W \textbf{27} \textit{Die Verse 794.27-29 fehlen} V'   $\cdot$ vernomen] genommen o \textbf{28} \textit{Versfolge 794.29-30-28} m   $\cdot$ der] [Der]: Daz er V Der do W  $\cdot$ lente] leinte W \textbf{29} wie] [we]: wie n  $\cdot$ bette] [here]: bette o \textbf{30} \textit{statt 794.30-795.1:} Er enphinc sie alle mit siten V'   $\cdot$ zwên] rittere alle V  $\cdot$ Anfortas] anfortaz V \newline
\end{minipage}
\end{table}
\newpage
\begin{table}[ht]
\begin{minipage}[t]{0.5\linewidth}
\small
\begin{center}*G
\end{center}
\begin{tabular}{rl}
 & \textbf{dâ} vunden \textbf{si} volkes ungezalt,\\ 
 & manigen \textbf{junclîchen} rîter alt,\\ 
 & edeliu kint, vil sarjande,\\ 
 & die \textbf{truogen} mahinande.\\ 
5 & \begin{large}D\end{large}irre \textbf{künfte} \textbf{wol vrô} \textbf{mohte} sîn\\ 
 & Feirafiz Anschevin\\ 
 & unde Parzival, si bêde.\\ 
 & vor dem palas \textbf{an} der \textbf{grêde}\\ 
 & \textbf{si} wurden wol enpfangen.\\ 
10 & \textit{\textbf{ûf} den palas wart gegangen.}\\ 
 & dâ lac nâch ir gewonheit\\ 
 & \textbf{sinwel hundert} tepech breit,\\ 
 & ûf ieslîchem ein pf\textit{l}ûmît\\ 
 & unde ein kulter \textbf{lanc} von samît.\\ 
15 & vuoren die zwêne mit witzen,\\ 
 & si mohten \textbf{wol} \textbf{gesitzen},\\ 
 & unze manz harnasch von in enpfienc.\\ 
 & ein kamerære dar nâher gienc.\\ 
 & der brâhte in kleider rîche,\\ 
20 & den \textbf{zwein} \textbf{al}gelîche.\\ 
 & si sâzen, swaz dâ rîter was.\\ 
 & man truoc von golde - ez \textbf{en}was niht glas -\\ 
 & vür si manige tiure schâl.\\ 
 & Feirafiz unde Parzival\\ 
25 & trunken unde giengen dan\\ 
 & ze Anfortas, dem trûrigen man.\\ 
 & ir habt \textbf{wol} \textbf{gehôrt} daz,\\ 
 & \textbf{daz er} lente unde \textbf{niht} \textbf{en}saz\\ 
 & unde wie sîn bette gehêrt was.\\ 
30 & \textbf{die} zwêne enpfienc dô Anfortas\\ 
\end{tabular}
\scriptsize
\line(1,0){75} \newline
G I L M Z \newline
\line(1,0){75} \newline
\textbf{5} \textit{Initiale} G I Z  \textbf{15} \textit{Initiale} I  \newline
\line(1,0){75} \newline
\textbf{1} vunden] von Z  $\cdot$ volkes] volc I valsches M \textbf{2} junclîchen] \textit{om.} I wvͯnneclichen L \textbf{3} vil] vnde vil I \textbf{4} truogen] trvrige L trurigen Z  $\cdot$ mahinande] mahenandir M \textbf{5} Dirre] Dirre iungelinc I  $\cdot$ künfte] kanipft L kunst M  $\cdot$ mohte] mac I mochten L (Z) \textbf{6} Feirafiz] Ferefiz L Feirefisz M Feirefiz Z  $\cdot$ Anschevin] [Antheuein]: Anthseuein I Anshevin L Z Ansevin M \textbf{7} Parzival] parcifal G Z parzifal I L M \textbf{8} an] vf I \textbf{10} \textit{Vers 794.10 fehlt} G  \textbf{11} lac] lagen L  $\cdot$ ir] \textit{om.} L \textbf{13} ieslîchem] iegelichz I  $\cdot$ pflûmît] phvmit G pulmit I \textbf{14} lanc] \textit{om.} L lac M \textbf{15} vuoren] Fuͤrten I Sý fuͯren L Fvͤren Z  $\cdot$ die zwêne] beide L \textbf{16} wol gesitzen] etswa wol sitzen L (M) etteswa da sitzen Z \textbf{18} nâher] nach M \textbf{19} in] \textit{om.} L \textbf{21} si] Do L  $\cdot$ swaz] ouch waz L (M) Z  $\cdot$ dâ] der M \textbf{22} von golde] \textit{om.} Z  $\cdot$ ez enwas] ez was I Z \textit{om.} L  $\cdot$ glas] von glaz L (Z) \textbf{24} Feirafiz] Ferefiz L Feirefisz M Feirefiz Z  $\cdot$ Parzival] parcifal G Z [parzifal]: Parzifal I parzifal L M \textbf{25} trunken] Trvͯcken L \textbf{26} Anfortas] Amfortas L \textbf{27} wol] auch I ouch wol L M Z \textbf{28} lente] leinde I lenit L (Z)  $\cdot$ ensaz] saz L (M) \textbf{30} dô] \textit{om.} I da M Z  $\cdot$ Anfortas] Anforta: G Amfortas L \newline
\end{minipage}
\hspace{0.5cm}
\begin{minipage}[t]{0.5\linewidth}
\small
\begin{center}*T
\end{center}
\begin{tabular}{rl}
 & \textbf{d\textit{â}} vunden \textbf{si} volkes ungezalt,\\ 
 & manegen \textbf{wünneclîchen} rîter alt,\\ 
 & edeliu kint, vil sarjande.\\ 
 & die \textbf{trûrigen} mahinande\\ 
5 & dirre \textbf{kün\textit{f}te} \textbf{wol vrô} \textbf{mohten} sîn.\\ 
 & Ferefis Anschevin\\ 
 & und Parcifal, si beide,\\ 
 & vor dem palas \textbf{an} der \textbf{heide}\\ 
 & \textbf{si} wurden wol enpfangen.\\ 
10 & \textbf{ûf} den palas wart gegangen.\\ 
 & d\textit{â} lac nâch ir gewonheit\\ 
 & \textbf{sinewel hundert} teppich breit.\\ 
 & ûf ieclîchem ein plûmît\\ 
 & \textbf{lac} und ein kulter von samît.\\ 
15 & vuoren die zwêne mit witzen,\\ 
 & si mohten \textbf{etswâ} \textbf{sitzen},\\ 
 & unz man daz harnasch von in enpfienc.\\ 
 & ein kamerære dar nâher gienc.\\ 
 & der brâhte in kleider rîche,\\ 
20 & den \textbf{zwein} \textbf{allen} glîche.\\ 
 & \begin{large}S\end{large}i sâzen \textbf{ouch}, waz d\textit{â} rîte\textit{r} was.\\ 
 & man truoc von golde - ez \textbf{en}was niht glas -\\ 
 & \textit{v}ür si manege tiure schâl.\\ 
 & Ferefis und Parcifal\\ 
25 & trunken und giengen dan\\ 
 & zuo Anfortas, dem trûrigen man.\\ 
 & ir habet \textbf{doch wol} \textbf{gehœret} daz,\\ 
 & \textbf{daz er} lente und \textbf{selten} saz\\ 
 & und wie sîn bette ge\textit{hê}ret was.\\ 
30 & \textbf{die} zwêne enpfienc dô Anfortas\\ 
\end{tabular}
\scriptsize
\line(1,0){75} \newline
U Q R \newline
\line(1,0){75} \newline
\textbf{11} \textit{Initiale} R  \textbf{21} \textit{Initiale} U  \newline
\line(1,0){75} \newline
\textbf{1} dâ] Do U Q R \textbf{5} künfte] kuͦnste U \textbf{6} Ferefis] feirefisz Q Ferrefis R  $\cdot$ Anschevin] anshevin Q \textbf{7} Parcifal] Parzifal U partzifal Q parczifal R \textbf{8} an der heide] an der grede Q andem grade R \textbf{11} dâ] Do U Q \textbf{13} ieclîchem] yeclich R  $\cdot$ plûmît] pulmit R \textbf{16} sitzen] wol sitzen Q (R) \textbf{17} unz] Mit U  $\cdot$ daz] den R  $\cdot$ in] im Q \textbf{18} dar] do Q \textbf{20} allen] al Q (R) \textbf{21} ouch] alle R  $\cdot$ dâ] do U Q  $\cdot$ rîter] rite U \textbf{23} \textit{Vers 794.23 fehlt} R   $\cdot$ vür si] Bursi U \textbf{24} Ferefis] feirefisz Q Feirefis R  $\cdot$ Parcifal] Parzifal U partzifal Q parczifal R \textbf{25} trunken] Trunken ausen R \textbf{26} Anfortas] anfortes R \textbf{28} daz] \textit{om.} R  $\cdot$ lente] leinte Q R \textbf{29} gehêret] geret U gehoͯret R \textbf{30} dô] \textit{om.} R \newline
\end{minipage}
\end{table}
\end{document}
