\documentclass[8pt,a4paper,notitlepage]{article}
\usepackage{fullpage}
\usepackage{ulem}
\usepackage{xltxtra}
\usepackage{datetime}
\renewcommand{\dateseparator}{.}
\dmyyyydate
\usepackage{fancyhdr}
\usepackage{ifthen}
\pagestyle{fancy}
\fancyhf{}
\renewcommand{\headrulewidth}{0pt}
\fancyfoot[L]{\ifthenelse{\value{page}=1}{\today, \currenttime{} Uhr}{}}
\begin{document}
\begin{table}[ht]
\begin{minipage}[t]{0.5\linewidth}
\small
\begin{center}*D
\end{center}
\begin{tabular}{rl}
\textbf{264} & \textit{\begin{large}I\end{large}}ch \textbf{sag iu} des einen zorn:\\ 
 & daz sîn wîp wol geborn\\ 
 & dâ vor was genôtzogt.\\ 
 & er was iedoch ir rehter vogt,\\ 
5 & sô daz si schermes wart \textbf{an} in.\\ 
 & er wânde, ir wîplîcher sin\\ 
 & wære gein im \textbf{verkêret}\\ 
 & unt daz si geunêret\\ 
 & het ir kiusche unt ir prîs\\ 
10 & mit einem andern âmîs.\\ 
 & Des lasters nam er pflihte.\\ 
 & \textbf{ouch} ergie sîn gerihte\\ 
 & über si, daz grœzer nôt\\ 
 & \textbf{wîp nie gedolte} âne tôt\\ 
15 & unt ân alle \textbf{ir} \textbf{schulde}.\\ 
 & er \textbf{moht} ir sîne hulde\\ 
 & versagen, swenn er wolde;\\ 
 & niemen daz wenden solde,\\ 
 & \textbf{ob} \textbf{der} man des wîbes \textbf{hât} gewalt.\\ 
20 & Parzival, der degen balt,\\ 
 & Oriluses hulde gerte\\ 
 & vroun Jeschuten mit dem swerte.\\ 
 & \textbf{des} hôrt ich ie güetlîche biten,\\ 
 & ez kom \textbf{dâ} gar \textbf{von} \textbf{schimpfes} siten.\\ 
25 & mich dunket, si haben bêde reht.\\ 
 & der beidiu krump unt sleht\\ 
 & \textbf{geschuof}, künner scheiden,\\ 
 & sô wende\textbf{r} daz an beiden,\\ 
 & deiz âne sterben dâ ergê.\\ 
30 & si tuont \textbf{doch} sus ein ander wê.\\ 
\end{tabular}
\scriptsize
\line(1,0){75} \newline
D \newline
\line(1,0){75} \newline
\textbf{1} \textit{Initiale} D  \textbf{11} \textit{Majuskel} D  \newline
\line(1,0){75} \newline
\textbf{1} Ich] ÷ch \textit{nachträglich korrigiert zu:} Jch D \textbf{21} Oriluses] Ôrilvs D \textbf{22} Jeschuten] Jescvten D \newline
\end{minipage}
\hspace{0.5cm}
\begin{minipage}[t]{0.5\linewidth}
\small
\begin{center}*m
\end{center}
\begin{tabular}{rl}
 & \begin{large}I\end{large}\textit{c}h \textbf{wil iu sagen} des einen zorn:\\ 
 & daz sîn wîp wol geborn\\ 
 & dâ vor was genô\textit{t}zoget.\\ 
 & er was iedoch ir rehter voget,\\ 
5 & sô da\textit{z} si s\textit{ch}er\textit{me}s wart in.\\ 
 & er wânde, ir wîplîcher sin\\ 
 & wære gegen im \textbf{erkêret}\\ 
 & und daz si geunêret\\ 
 & hete ir kiusche und ir prîs\\ 
10 & mit einem andern âmîs.\\ 
 & des lasters nam er pflihte.\\ 
 & \textbf{ouch} ergienc sîn gerihte\\ 
 & über si, daz grœzer nôt\\ 
 & \textbf{\textit{ni}e wîp gedolte} \textit{ân} tôt\\ 
15 & und âne alle \textbf{schulden}.\\ 
 & er \textbf{möht} ir sîne hulden\\ 
 & versagen, wenner wold\textit{e};\\ 
 & niemen daz wenden sold\textit{e},\\ 
 & \textbf{wanne} \textbf{der} man des wîbes \textbf{het} gewalt.\\ 
20 & Parcifal, de\textit{r} degen balt,\\ 
 & Oriluses hulde gerte\\ 
 & vrouwen Jeschuten mit dem swerte.\\ 
 & \textbf{des} hôrt ich ie güetlîch biten,\\ 
 & ez kam \textbf{d\textit{â}} gar \textbf{von} \textbf{schimpfes} siten.\\ 
25 & mich dunket, si haben beide reht.\\ 
 & der beidiu krump und sleht\\ 
 & \textbf{geschuof}, \dag kumber\dag  scheiden,\\ 
 & sô wend \textbf{er} daz an beiden,\\ 
 & daz e\textit{z} âne sterben dâ ergê.\\ 
30 & si tuont \textbf{doch} sus ein ander wê.\\ 
\end{tabular}
\scriptsize
\line(1,0){75} \newline
m n o Fr69 \newline
\line(1,0){75} \newline
\textbf{1} \textit{Initiale} m   $\cdot$ \textit{Capitulumzeichen} n  \newline
\line(1,0){75} \newline
\textbf{1} Ich] ICch m \textbf{3} genôtzoget] genozoget m \textbf{5} daz] da m  $\cdot$ schermes] sterbens m stormes n \textbf{7} erkêret] verkeret n o \textbf{14} nie] Me m  $\cdot$ ân] \textit{om.} m \textbf{15} schulden] ir schulde n o \textbf{16} möht] mochte o  $\cdot$ ir sîne hulden] ir sin hulde n sin ir hulde o \textbf{17} wolde] wolden m \textbf{18} solde] solden m \textbf{19} het] hat n \textbf{20} der] den m \textbf{21} Oriluses] Oreluses o \textbf{22} vrouwen] Frouwe m n (o)  $\cdot$ Jeschuten] jescutten m jescuten n jescuͯten o iesculten Fr69 \textbf{23} ich] \textit{om.} o \textbf{24} dâ] do m n o \textbf{25} haben] habe o \textbf{27} kumber] kromber n kramer o kume Fr69 \textbf{28} an] in n o \textbf{29} ez] er m  $\cdot$ dâ] do n o \textbf{30} doch] duch o  $\cdot$ ein ander] eẏander o \newline
\end{minipage}
\end{table}
\newpage
\begin{table}[ht]
\begin{minipage}[t]{0.5\linewidth}
\small
\begin{center}*G
\end{center}
\begin{tabular}{rl}
 & ich \textbf{wil iu sagen} des einen zorn:\\ 
 & daz sîn wîp wolgeborn\\ 
 & dâ vor was genôtzoget.\\ 
 & er was iedoch ir rehter voget,\\ 
5 & sô daz si schermes wart \textbf{an} in.\\ 
 & er wânde, ir wîplîcher sin\\ 
 & wære gein im \textbf{verkêret}\\ 
 & unde daz si geunêret\\ 
 & het ir kiusche unde ir prîs\\ 
10 & mit einem anderen âmîs.\\ 
 & des lasters nam er pflihte,\\ 
 & \textbf{doch} ergie sîn gerihte\\ 
 & über si, daz grœzer nôt\\ 
 & \textbf{nie wîp erleit} âne \textbf{den} tôt\\ 
15 & unde âne alle \textbf{ir} \textbf{schulde}.\\ 
 & er \textbf{moht} ir sîne hulde\\ 
 & versagen, swenner wolde;\\ 
 & niemen daz wenden solde,\\ 
 & \textbf{obe} man des wîbes \textbf{h\textit{et}} gewalt.\\ 
20 & Parzival, der degen balt,\\ 
 & Orilluses hulde gerte\\ 
 & vroun Jeschuten mit dem swerte.\\ 
 & \textbf{des} hôrt ich ie güetlîchen biten,\\ 
 & ez kom \textbf{hie} gar \textbf{von} \textbf{smeiches} siten.\\ 
25 & mich dunket, si hân bêde reht.\\ 
 & der beidiu krump unde sleht\\ 
 & \textbf{geschuof}, künner scheiden,\\ 
 & sô wende daz an beiden,\\ 
 & deiz âne sterben dâ ergê.\\ 
30 & si tuont \textbf{doch} sus ein ander wê.\\ 
\end{tabular}
\scriptsize
\line(1,0){75} \newline
G I O L M Q R Z Fr21 \newline
\line(1,0){75} \newline
\textbf{1} \textit{Initiale} I  \textbf{25} \textit{Initiale} I L  \textbf{27} \textit{Initiale} O Z Fr21  \newline
\line(1,0){75} \newline
\textbf{3} genôtzoget] genofflogt O \textbf{4} iedoch] doch O L (M) Q Z Fr21  $\cdot$ rehter] rechte Q \textbf{5} si] \textit{om.} R  $\cdot$ schermes] beschermens M  $\cdot$ wart] warte L M R  $\cdot$ an] gewert [*]: an O \textbf{6} wânde] wante an in Q \textbf{8} unde] [vns]: vnd I  $\cdot$ geunêret] gen vneret Q \textbf{10} anderen] andrem I (Q) (Fr21) \textbf{11} lasters] lastes M laster Fr21 \textbf{12} doch] do I Auch Q (Z) \textbf{13} si] \textit{om.} R \textbf{14} nie wîp] Wip nie Z Hie wip Fr21  $\cdot$ erleit] der leit Z  $\cdot$ âne den] anden Fr21 \textbf{15} alle] \textit{om.} Z \textbf{16} er] Ern O \textbf{17} swenner] wenne er L (M) (Q) (R) Z \textbf{18} wenden] wende Q \textbf{19} man] ein man L der man Q (Z)  $\cdot$ des wîbes het] des wibes habe G sines wibes hat L hott des weybs Q \textbf{20} Parzival] parzifal I (L) (M) Parcifal O Q Z Fr21 Parczifal R  $\cdot$ balt] bal Z \textbf{21} Orilluses] orillus G (L) orilus I (O) (M) (Q) (Z) (Fr21) Orlius R  $\cdot$ hulde] hulden Z \textbf{22} vroun] Vrouw L (M) (Q) (R)  $\cdot$ Jeschuten] ieschuten G ieskuten I Jeschvͦten O Jescuͯten L iescuten M (Z) Jescuten Q R \textbf{23} hôrt ich] [hortes]: horte L hort Q  $\cdot$ ie] \textit{om.} I ir M In R  $\cdot$ güetlîchen] gern Fr21 \textbf{24} hie] doch I [do]: da O do L Q R da M Z Fr21  $\cdot$ von smeiches] von swachen I uͯsz smeichensz L ausz sweigens Q vs schmaͯchen R  $\cdot$ siten] site Fr21 \textbf{25} hân] haten L habe R  $\cdot$ bêde] beiden M \textit{om.} R \textbf{26} beidiu] bede O \textbf{27} geschuof] ÷eschvͦfe O Geschuͯffe L (M) (R) (Fr21)  $\cdot$ künner] chunber I (Q) \textbf{28} wende] wendet I wender O (L) M (R) Fr21 wendt er Q (Z)  $\cdot$ an] an in Q \textbf{29} deiz] Das M Es Q Ditz Z  $\cdot$ âne] an ein Q  $\cdot$ dâ] do Q R \textbf{30} doch] \textit{om.} L  $\cdot$ ein ander] an ein ander O ein andren R \newline
\end{minipage}
\hspace{0.5cm}
\begin{minipage}[t]{0.5\linewidth}
\small
\begin{center}*T
\end{center}
\begin{tabular}{rl}
 & Ich \textbf{wil iu sagen} des eines zorn:\\ 
 & daz \textbf{im} sîn wîp wol geborn\\ 
 & dâ vor was genôtzoget.\\ 
 & er was iedoch ir rehter voget,\\ 
5 & sô daz si schirmes wart \textbf{an} in.\\ 
 & er wânde, ir wîplîcher sin\\ 
 & wære gegen im \textbf{verkêret}\\ 
 & unde daz si geunêret\\ 
 & hete ir kiusche unde ir prîs\\ 
10 & mit einem andern âmîs.\\ 
 & des lasters nam er pflihte.\\ 
 & \textbf{ouch} ergienc sîn gerihte\\ 
 & über si, daz grœzer nôt\\ 
 & \textbf{nie wîp gedolte} âne \textbf{den} tôt\\ 
15 & unde âne alle \textbf{ir} \textbf{schulde}.\\ 
 & er \textbf{moht}ir sîne hulde\\ 
 & versagen, swenner wolte;\\ 
 & niemen daz wenden solte,\\ 
 & \textbf{ob} man des wîbes \textbf{hât} gewalt.\\ 
20 & Parzifal, der degen balt,\\ 
 & Oriluses hulde gerte\\ 
 & vroun Jeschuten mit dem swerte.\\ 
 & \textbf{der} hôrtich ie güetlîche biten,\\ 
 & ez kom \textbf{dâ} gar \textbf{ûz} \textbf{smeichens} siten.\\ 
25 & mich dunk\textit{e}t, si haben beide reht.\\ 
 & der beidiu krumb unde sleht\\ 
 & \textbf{geschüefe}, künner scheiden,\\ 
 & sô wende\textbf{r} daz an beiden,\\ 
 & daz ez âne sterben dâ ergê.\\ 
30 & si tuont sus \textbf{beide} ein ander wê.\\ 
\end{tabular}
\scriptsize
\line(1,0){75} \newline
T U V W \newline
\line(1,0){75} \newline
\textbf{1} \textit{Initiale} W   $\cdot$ \textit{Majuskel} T  \newline
\line(1,0){75} \newline
\textbf{1} eines] einen V W \textbf{2} wîp] weih W \textbf{5} si] \textit{om.} W  $\cdot$ wart] wartet W \textbf{8} si] sy gegen im W \textbf{12} sîn] im W \textbf{14} gedolte] gedulte U V erdolte W \textbf{16} mohtir] moͤht ir V  $\cdot$ sîne] seiner W \textbf{17} swenner] wan er er U wenn er W \textbf{19} ob] [*]: Wan der V \textbf{20} Parzifal] Parcifal U Partzifal W \textbf{21} Oriluses] Orilvs T (U) (W) [Ori*]: Oriluses  V \textbf{22} vroun] Vreuͦwe U (W)  $\cdot$ Jeschuten] Jescvten T (U) iescuten V iestuten W \textbf{23} der] Do U [D*]: Dez V Des W \textbf{24} dâ] do V W  $\cdot$ ûz smeichens] von smeichens U von [*]: schinphes V von schmehen W \textbf{25} dunket] dv́nkent T \textbf{27} künner] kuͦmer U [kv́*]: kv́nne er V kund ers W \textbf{28} an] \textit{om.} U [*]: an V in W \textbf{29} ez] \textit{om.} W  $\cdot$ dâ] do W \textbf{30} sus beide] [*]: doch sus V doch sus W \newline
\end{minipage}
\end{table}
\end{document}
