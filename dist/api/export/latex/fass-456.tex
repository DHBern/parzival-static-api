\documentclass[8pt,a4paper,notitlepage]{article}
\usepackage{fullpage}
\usepackage{ulem}
\usepackage{xltxtra}
\usepackage{datetime}
\renewcommand{\dateseparator}{.}
\dmyyyydate
\usepackage{fancyhdr}
\usepackage{ifthen}
\pagestyle{fancy}
\fancyhf{}
\renewcommand{\headrulewidth}{0pt}
\fancyfoot[L]{\ifthenelse{\value{page}=1}{\today, \currenttime{} Uhr}{}}
\begin{document}
\begin{table}[ht]
\begin{minipage}[t]{0.5\linewidth}
\small
\begin{center}*D
\end{center}
\begin{tabular}{rl}
\textbf{456} & \begin{large}D\end{large}iu slâ in dâ niht halden liez.\\ 
 & Fontane lasalvatsche hiez\\ 
 & ein wesen, dar sîn reise gienc.\\ 
 & er vant den wirt, der in enpfienc.\\ 
5 & der einsidel zim sprach:\\ 
 & "\textbf{ouwê}, hêrre, daz iu geschach\\ 
 & in dirre \textbf{heileclîchen} zît!\\ 
 & hât iuch angestlîcher strît\\ 
 & in diz harnasch getriben\\ 
10 & ode sît ir âne strît beliben?\\ 
 & sô stüende iu baz ein ander wât,\\ 
 & \textbf{lieze} iuch hôchverte rât.\\ 
 & nû ruochet erbeizen, hêrre,\\ 
 & - ich wæne, iu daz iht werre -\\ 
15 & unt erwarmet bî einem viure.\\ 
 & hât iuch âventiure\\ 
 & ûz gesant durch minnen solt?\\ 
 & sît ir rehter minne holt,\\ 
 & sô minnet, als \textbf{nû diu minne} gêt,\\ 
20 & als disses tages minne stêt.\\ 
 & dient her nâch umbe \textbf{wîbe} gruoz.\\ 
 & ruocht erbeizen, ob ichs bitten muoz."\\ 
 & Parzival, der wîgant,\\ 
 & erbeizte nider al zehant.\\ 
25 & mit grôzer zuht er vor im stuont.\\ 
 & er tet im von den liuten kunt,\\ 
 & die in dar \textbf{wîsten},\\ 
 & wie die \textbf{sîn râten} prîsten.\\ 
 & dô sprach er: "hêrre, \textbf{nû} gebt mir rât.\\ 
30 & ich bin ein man, der sünde hât."\\ 
\end{tabular}
\scriptsize
\line(1,0){75} \newline
D \newline
\line(1,0){75} \newline
\textbf{1} \textit{Initiale} D  \newline
\line(1,0){75} \newline
\textbf{21} dient her] [die*]: dient her D \textbf{23} Parzival] Parcival D \newline
\end{minipage}
\hspace{0.5cm}
\begin{minipage}[t]{0.5\linewidth}
\small
\begin{center}*m
\end{center}
\begin{tabular}{rl}
 & diu slâ in d\textit{â} niht halten liez.\\ 
 & Funtaine lasalvasche hiez\\ 
 & ein w\textit{e}sen, dar sîn reise gienc.\\ 
 & er vant den wirt, der in enpfienc.\\ 
5 & der einsidel zuo im sprach:\\ 
 & "\textbf{ouwê}, hêrre, daz iu \textbf{sus} geschach\\ 
 & in diser \textbf{heili\textit{clî}che\textit{n}} zît!\\ 
 & het iuch angestlîcher strît\\ 
 & in diz harnasch getriben\\ 
10 & oder sît ir âne strît bliben?\\ 
 & sô stüende iu baz ein ander wât,\\ 
 & \textbf{lâze} iuch hôchverte rât.\\ 
 & nû ruochet erbeizen, hêrre,\\ 
 & - ich wæne, iu daz iht werre -\\ 
15 & und erwarmet bî einem viur.\\ 
 & het iuch âventiur\\ 
 & ûz gesant durch min\textit{n}en solt?\\ 
 & sît ir rehter minne holt,\\ 
 & sô minnet, als \textbf{diu minne} gât,\\ 
20 & als dises tages minne stât.\\ 
 & dienet hernâch umb \textbf{wîbes} gruoz.\\ 
 & ruocht erbeizen, ob ich es bitten muoz."\\ 
 & Parcifal, der wîgant,\\ 
 & erbeizte nider al zehant.\\ 
25 & mit grôzer zuht er vor im stuont.\\ 
 & er tet im von den liuten kunt,\\ 
 & die in \textbf{al} \textit{dar} \textbf{\textit{h}eten gewisen},\\ 
 & wie die \textbf{sîn râten} prisen.\\ 
 & dô sprach er: "hêrre, \textbf{nû} gebt mir rât.\\ 
30 & ich bin ein man, der sünde hât."\\ 
\end{tabular}
\scriptsize
\line(1,0){75} \newline
m n o \newline
\line(1,0){75} \newline
\newline
\line(1,0){75} \newline
\textbf{1} dâ] do m n o \textbf{2} Funtaine lasalvasche] Funtaine lasaluasce m Funtanie lasaluasce n Funtanie >la< saluasce o \textbf{3} wesen] wisen m n \textbf{7} heiliclîchen] heilische m \textbf{8} angestlîcher] an glucher o \textbf{10} sît ir] sint sie o \textbf{11} sô] >so< o \textbf{12} lâze] Las m (o) \textbf{17} minnen] minen m \textbf{22} es] \textit{om.} n \textbf{24} al zehant] vff das lant n \textbf{27} al dar heten] aller luͯtten hetten m \textbf{28} die] sú n \newline
\end{minipage}
\end{table}
\newpage
\begin{table}[ht]
\begin{minipage}[t]{0.5\linewidth}
\small
\begin{center}*G
\end{center}
\begin{tabular}{rl}
 & \begin{large}D\end{large}iu slâ in dâ niht halden \textit{l}iez.\\ 
 & Funtane lasalvatsche hiez\\ 
 & ein wesen, dar sîn reise gienc.\\ 
 & er vant den wirt, der in enpfienc.\\ 
5 & der einsidel zim sprach:\\ 
 & "\textbf{wê}, hêrre, daz iu \textbf{sus} geschach\\ 
 & in dirre \textbf{heiliclîchen} zît!\\ 
 & hât iuch angestlîcher strît\\ 
 & in ditze harnasch getriben\\ 
10 & oder sît ir ân strît beliben?\\ 
 & sô stüende iu baz ein ander wât,\\ 
 & \textbf{lieze} iuch hôchverte rât.\\ 
 & nû ruochet erbeizen, hêrre,\\ 
 & - ich wæne, iu daz iht werre -\\ 
15 & unde erwarmet bî einem viure.\\ 
 & hât iuch âventiure\\ 
 & ûz gesant durch minnen solt?\\ 
 & sît ir rehter minne holt,\\ 
 & sô minnet, als \textbf{nû diu minne} gêt,\\ 
20 & als disses tages minne stêt.\\ 
 & dient her nâch umbe \textbf{wîbe} gruoz.\\ 
 & ruochet erbeizen, ob ichs biten muoz."\\ 
 & Parzival, der wîgant,\\ 
 & erbeizet nider alzehant.\\ 
25 & mit grôzer zuht er vo\textit{r} im stuont.\\ 
 & er tet im von den liuten kunt,\\ 
 & die in dar \textbf{wîsten},\\ 
 & wie die \textbf{sîn râten} brîsten.\\ 
 & dô sprach er: "hêrre, gebet mir rât.\\ 
30 & ich bin ein man, der sünde hât."\\ 
\end{tabular}
\scriptsize
\line(1,0){75} \newline
G I O L M Z \newline
\line(1,0){75} \newline
\textbf{1} \textit{Initiale} G I L M Z  \newline
\line(1,0){75} \newline
\textbf{1} liez] hiez G \textbf{2} Funtane lasalvatsche] Fvntane lasalvasche G funtane lasaluasche I Fontanie Mvntsalvatsche O Fontanie La savaische L Fontanie lasalvatsche M (Z) \textbf{3} reise] [wesin]: reise G \textbf{6} wê] Awe O Owe L Z (M)  $\cdot$ hêrre] \textit{om.} O  $\cdot$ iu] ev ie I \textbf{7} heiliclîchen] hailigen I (O) (L) \textbf{9} ditze] daz I disen O (L) (M) \textbf{12} hôchverte] hoch vertiger I \textbf{14} iu daz] daz iv O (L) (Z)  $\cdot$ iht] nih I  $\cdot$ werre] gewerre Z \textbf{15} einem] einen I \textbf{17} minnen] mynne M \textbf{18} minne] mýnnen L \textbf{19} nû diu] ev nu I iv div O \textbf{20} als] vnd als I  $\cdot$ disses] ditse O \textbf{21} hernâch] harnasch O  $\cdot$ wîbe] wip O wibes L (M) \textbf{22} ruochet erbeizen] Rvͦchet er beizet O Erbeiszet L  $\cdot$ ichs] ich O  $\cdot$ muoz] muͤz I (M) \textbf{23} Parzival] Parzifal I L M Parcifal O Z \textbf{24} \textit{Versdoppelung} M   $\cdot$ erbeizet] erbaizte I (O) (L) (M)  $\cdot$ alzehant] sa zuͯhant L \textbf{25} vor im] uon [in]: im G \textbf{26} von] vor I  $\cdot$ den] dem Z \textbf{28} sîn râten] sin rat I (O) (M) sine rat L \textbf{29} gebet] nv gebt O (Z) \newline
\end{minipage}
\hspace{0.5cm}
\begin{minipage}[t]{0.5\linewidth}
\small
\begin{center}*T
\end{center}
\begin{tabular}{rl}
 & \begin{large}D\end{large}iu slâ in dâ niht halten liez.\\ 
 & Fontane lasalvatsche hiez\\ 
 & ein w\textit{e}sen, dâ sîn reise gienc.\\ 
 & er vant den wirt, der in enpfienc.\\ 
5 & Der einsidel zim sprach:\\ 
 & "\textbf{ouwê}, hêrre, daz iu \textbf{sus} geschach\\ 
 & in dirre \textbf{heiligen} zît!\\ 
 & hât iuch angestlîcher strît\\ 
 & in disen harnasch getriben\\ 
10 & oder sît ir âne strît beliben?\\ 
 & sô stüende iu baz ein ander wât,\\ 
 & \textbf{liez}iuch hôchverte rât.\\ 
 & nû ruochet erbeizen, hêrre,\\ 
 & - ich wæniu daz iht werre -\\ 
15 & unde erwarmet bî einem viure.\\ 
 & hât iuch âventiure\\ 
 & ûz gesant durch minnen solt?\\ 
 & sît ir rehter minne holt,\\ 
 & sô minnet, alse \textbf{diu minne nû} gêt,\\ 
20 & als disses tages minne stêt.\\ 
 & dient her nâch umbe \textbf{wîbe} gruoz.\\ 
 & ruochet erbeizen, ob ichs biten muoz."\\ 
 & Parcifal, der wîgant,\\ 
 & erbeizete nider alzehant.\\ 
25 & mit grôzer zuht er vor im stuont.\\ 
 & er tet im von den liuten kunt,\\ 
 & die in dar \textbf{wîsten},\\ 
 & wie die \textbf{sînen rât} prîsten.\\ 
 & \begin{large}D\end{large}ô sprach er: "hêrre, gebt mir rât.\\ 
30 & ich bin ein man, der sünde hât."\\ 
\end{tabular}
\scriptsize
\line(1,0){75} \newline
T U V W Q R \newline
\line(1,0){75} \newline
\textbf{1} \textit{Initiale} T W Q  \textbf{5} \textit{Majuskel} T  \textbf{29} \textit{Initiale} T  \newline
\line(1,0){75} \newline
\textbf{1} \textit{Die Verse 453.1-502.30 fehlen} U   $\cdot$ dâ] do W R \textbf{2} Fontane lasalvatsche] Fontange de Salvasce T [*]: Vontanie la salvasche V Fontage de saluatsche W Fontange von (de R ) salvasche Q (R) \textbf{3} ein wesen] ein weisen T Q Ich enwen W  $\cdot$ dâ] do W Q \textbf{6} iu sus geschach] ich úch suß gesach W \textbf{8} iuch] îv T \textbf{9} in] An Q \textbf{11} stüende iu] stund auch Q  $\cdot$ wât] [wârt]: wâ:t T \textbf{12} lieziuch] lieziv T  $\cdot$ hôchverte] hochuerie W hochfertige R \textbf{14} wæniu] wene daz v́ch V wend Q  $\cdot$ iht] euch Q ich R \textbf{16} iuch] iv T \textbf{17} ûz gesant] Zu [gesagt]: gesant Q  $\cdot$ minnen] meinen Q minn R \textbf{18} rehter] [reht*]: rehter T \textbf{19} diu minne nû] nv die minne V (W) (Q) uch die minne R \textbf{20} stêt] gestet W Q (R) \textbf{21} wîbe] weibes W \textbf{22} ichs] ich R  $\cdot$ biten] v́ch bitten V \textbf{23} Parcifal] Parzifal V Partzifal W Q Parczifal R \textbf{29} hêrre] herberge W \newline
\end{minipage}
\end{table}
\end{document}
