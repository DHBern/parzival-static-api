\documentclass[8pt,a4paper,notitlepage]{article}
\usepackage{fullpage}
\usepackage{ulem}
\usepackage{xltxtra}
\usepackage{datetime}
\renewcommand{\dateseparator}{.}
\dmyyyydate
\usepackage{fancyhdr}
\usepackage{ifthen}
\pagestyle{fancy}
\fancyhf{}
\renewcommand{\headrulewidth}{0pt}
\fancyfoot[L]{\ifthenelse{\value{page}=1}{\today, \currenttime{} Uhr}{}}
\begin{document}
\begin{table}[ht]
\begin{minipage}[t]{0.5\linewidth}
\small
\begin{center}*D
\end{center}
\begin{tabular}{rl}
\textbf{632} & \begin{large}D\end{large}ô sprach diu magt mit sinnen:\\ 
 & "hêrre, wen solt ich minnen?\\ 
 & sît mir mîn êrster tac erschein,\\ 
 & sô wart rîter nie dechein,\\ 
5 & ze dem ich ie \textbf{gespræche} wort,\\ 
 & wan als ir hiute hât gehôrt."\\ 
 & "Sô \textbf{m\textit{ö}hten} iu doch mære komen,\\ 
 & wâ ir mit \textbf{manheit} hât vernomen\\ 
 & \textbf{bejagten} prîs mit rîterschaft\\ 
10 & unt wer mit herzenlîcher kraft\\ 
 & nâch minnen dienst bieten kan",\\ 
 & sus sprach mîn hêr Gawan.\\ 
 & Des antwurte im diu clâre magt:\\ 
 & "nâch minne ist dienst mich verdagt,\\ 
15 & wan der herzoginne von Logroys\\ 
 & dient manec rîter kurtoys\\ 
 & beidiu nâch minne unt umb ir solt.\\ 
 & der hât maneger hie \textbf{geholt}\\ 
 & tjustieren, \textbf{dâ} wirz sâhen.\\ 
20 & ir decheiner nie sô nâhen\\ 
 & kom, als ir uns komen sît.\\ 
 & \textbf{den} prîs ûf hœhet \textbf{iwer} strît."\\ 
 & Er sprach zer meide wol gevar:\\ 
 & "war krieget der herzoginne schar\\ 
25 & sus manec rîter ûzerkorn?\\ 
 & wer hât ir hulde verlorn?"\\ 
 & Si sprach: "daz hât \textbf{der künec} Gramoflanz,\\ 
 & der der werdecheite kranz\\ 
 & treit, als \textbf{im} diu volge giht.\\ 
30 & hêrre, \textbf{des erkenne ich} anders niht."\\ 
\end{tabular}
\scriptsize
\line(1,0){75} \newline
D Z Fr63 \newline
\line(1,0){75} \newline
\textbf{1} \textit{Initiale} D Z Fr63  \textbf{7} \textit{Majuskel} D  \textbf{13} \textit{Majuskel} D  \textbf{23} \textit{Majuskel} D  \textbf{27} \textit{Majuskel} D  \newline
\line(1,0){75} \newline
\textbf{4} wart] enwart Z \textbf{7} möhten] mohten D (Fr63) moht Z \textbf{8} manheit] warheit Z \textbf{9} bejagten] Beiagen Z \textbf{11} minnen] minne Z  $\cdot$ bieten] ich bieten Z \textbf{13} antwurte] antwurt Z Fr63 \textbf{15} wan] Wan nach Z  $\cdot$ Logroys] Logrois Z Fr63 \textbf{19} dâ] daz Z \textbf{20} decheiner] keinen Z \textbf{22} den prief vf hohet iwrn strit Fr63  $\cdot$ den] Der Z  $\cdot$ iwer] ewern Z \textbf{26} ir] iwer Fr63 \textbf{27} der künec Gramoflanz] Gramolanz Fr63 \textbf{28} der der] der dir Fr63 \newline
\end{minipage}
\hspace{0.5cm}
\begin{minipage}[t]{0.5\linewidth}
\small
\begin{center}*m
\end{center}
\begin{tabular}{rl}
 & dô sprach diu maget mit sinnen:\\ 
 & "hêrre, wen solte ich minnen?\\ 
 & sît mir mîn êrster tac erschein,\\ 
 & sô wart ritter nie dekein,\\ 
5 & zuo dem \textit{ich} i\textit{e} \textbf{\textit{s}præche} wort,\\ 
 & wan als ir hiute habt gehôrt."\\ 
 & "sô \textbf{m\textit{ö}hten} iu doch mær komen,\\ 
 & wâ ir mit \textbf{manheit} het vernomen\\ 
 & \textbf{bejagten} prîs mit ritterschaft\\ 
10 & und wer mit herzelîcher \textit{kr}aft\\ 
 & nâch minne dienst bieten kan",\\ 
 & sus sprach mîn hêr Gawan.\\ 
 & des antwurt im diu clâre maget:\\ 
 & "nâch minne ist dienst mich verdaget,\\ 
15 & wan der herzogîn von Logrois\\ 
 & dienet manic ritter kurtois\\ 
 & beidiu nâch minne und umb ir solt.\\ 
 & der het maniger hie \textbf{geholt}\\ 
 & justieren, \textbf{dâ} wirz sâhen.\\ 
20 & ir dekeiner nie sô nâhen\\ 
 & kam, als ir uns komen sît.\\ 
 & \textbf{den} prîs ûf hœhe\textit{t} \textbf{iuwer} strît."\\ 
 & er sprach zer megde wol gevar:\\ 
 & "war krieget der herzogîn schar\\ 
25 & sus manic ritter ûzerkorn?\\ 
 & wer het ir hulde verlorn?"\\ 
 & si sprach: "daz het Gram\textit{o}lanz,\\ 
 & der der wirdecheite kranz\\ 
 & treit, als \textbf{im} diu volge giht.\\ 
30 & hêrre, \textbf{des erkenn ich} anders niht."\\ 
\end{tabular}
\scriptsize
\line(1,0){75} \newline
m n o \newline
\line(1,0){75} \newline
\newline
\line(1,0){75} \newline
\textbf{1} maget] frouwe n  $\cdot$ mit sinnen] mẏn súnde o \textbf{2} solte] solten o \textbf{4} sô] Do o  $\cdot$ dekein] do kein n \textbf{5} ich ie spræche] ẏe gescheh vnd spreh m ich ye gesprechen n ich e sprache o \textbf{6} hiute] zucht n \textbf{7} möhten] mohtten m (o) \textbf{10} herzelîcher] herczeclichen o  $\cdot$ kraft] ritterschaft m \textit{om.} o \textbf{12} sus] So o  $\cdot$ hêr] herreher n \textbf{13} des] Das o \textbf{14} dienst] [dich]: dienst n \textbf{19} dâ] do n o \textbf{20} ir] Dir o  $\cdot$ dekeiner nie] do keiner me n \textbf{22} hœhet] hoher m n o \textbf{24} war] Wer n \textbf{27} Gramolanz] gramalancz m gramanlantz n gramolancz o \newline
\end{minipage}
\end{table}
\newpage
\begin{table}[ht]
\begin{minipage}[t]{0.5\linewidth}
\small
\begin{center}*G
\end{center}
\begin{tabular}{rl}
 & \begin{large}D\end{large}ô sprach diu maget mit sinnen:\\ 
 & "hêrre, wen solde ich minnen?\\ 
 & sît mir mîn êrster tac erschein,\\ 
 & sô\textbf{ne} wart rîter nie dehein,\\ 
5 & zuo dem ich ie \textbf{gespræche} wort,\\ 
 & wan als ir hiut habet gehôrt."\\ 
 & "sô \textbf{m\textit{e}ht} iu doch mære komen,\\ 
 & wâ ir mit \textbf{wârheit} habet vernomen\\ 
 & \textbf{bejagen} prîs mit rîterschaft\\ 
10 & unde wer mit herzenlîcher kraft\\ 
 & nâch minne dienst bieten kan",\\ 
 & sus sprach mîn hêr Gawan.\\ 
 & des antwurte im diu clâre maget:\\ 
 & "nâch minne ist dienst mich verdaget,\\ 
15 & wan der herzogîn von Logrois\\ 
 & dient manic rîter kurtois\\ 
 & beidiu nâch minne unde umbe ir solt.\\ 
 & der hât maniger hie \textbf{gedolt}\\ 
 & tjostieren, \textbf{daz} wir ez sâhen.\\ 
20 & ir deheine\textit{r} nie sô nâhen\\ 
 & kom, als ir uns komen sît.\\ 
 & \textbf{der} prîs ûf hœhet \textbf{iuwern} strît."\\ 
 & er sprach zer meide wol gevar:\\ 
 & "war krieget der herzoginne schar\\ 
25 & sus manic rîter ûzerkorn?\\ 
 & wer hât ir hulde verlorn?"\\ 
 & si sprach: "daz hât Gramoflanz,\\ 
 & der der werdecheit kranz\\ 
 & treit, alsô diu volge giht.\\ 
30 & hêrre, \textbf{des erkenne ich} anders niht."\\ 
\end{tabular}
\scriptsize
\line(1,0){75} \newline
G I L M Z Fr51 \newline
\line(1,0){75} \newline
\textbf{1} \textit{Initiale} G I L Z Fr51  \textbf{23} \textit{Initiale} Fr51  \textbf{27} \textit{Initiale} I  \newline
\line(1,0){75} \newline
\textbf{1} Dô] Da M \textbf{3} mir] \textit{om.} L  $\cdot$ êrster] erste Fr51 \textbf{4} sône] So M  $\cdot$ rîter nie] nie ritter L \textbf{5} ie] \textit{om.} I L  $\cdot$ gespræche] gesproche L \textbf{6} wan] Mern Fr51  $\cdot$ habet] haben Fr51 \textbf{7} meht] maht G (I) mochten L (M) (Fr51) \textbf{9} bejagen] [beiagit]: beiagin G Beiagten Fr51 \textbf{10} wer] we Fr51  $\cdot$ herzenlîcher] ritterlicher L \textbf{11} minne] myme M  $\cdot$ dienst] dienste G  $\cdot$ bieten] ich bieten Z beide Fr51 \textbf{12} sus] Dus Fr51  $\cdot$ mîn] \textit{om.} Fr51  $\cdot$ hêr Gawan] ergawan M \textbf{13} antwurte] antwurt I (Fr51)  $\cdot$ clâre] shone I \textbf{14} ist dienst] dienst ist M  $\cdot$ mich] mir Fr51 \textbf{15} wan] Wan nach Z  $\cdot$ Logrois] logroys I Logroýs L l:grois Fr51 \textbf{17} unde umbe ir] vmbe L vnde vmbe Fr51 \textbf{18} gedolt] geholt I Z \textbf{19} daz] dar Fr51  $\cdot$ ez] daz I  $\cdot$ sâhen] sagen M (Fr51) \textbf{20} deheiner] deheine G keinen Z nekein Fr51  $\cdot$ nâhen] iahen M \textbf{22} ûf] us I \textit{om.} L Fr51 \textbf{24} krieget] krech Fr51 \textbf{25} manic] mangen Fr51 \textbf{27} hât] ist L hat der kunic Z  $\cdot$ Gramoflanz] gromorflanz M \textbf{29} alsô] des I als im Z \textbf{30} anders] \textit{om.} I L M Fr51 \newline
\end{minipage}
\hspace{0.5cm}
\begin{minipage}[t]{0.5\linewidth}
\small
\begin{center}*T
\end{center}
\begin{tabular}{rl}
 & \begin{large}D\end{large}ô sprach diu maget mit sinnen:\\ 
 & "hêrre, wen solt ich minnen?\\ 
 & sît mir mîn êrster tac erschein,\\ 
 & sô \textbf{en}wart rîter nie dekein,\\ 
5 & zuo dem ich ie \textbf{gespræche} wort,\\ 
 & wan als ir hiute hât gehôrt."\\ 
 & "sô \textbf{m\textit{ö}hten} iu doch mære komen,\\ 
 & wâ ir mit \textbf{wârheit} hât vernomen\\ 
 & \textbf{bejageten} prîs mit rîterschaft\\ 
10 & und wer mit herzeclîcher kraft\\ 
 & nâch minne dienst bieten kan",\\ 
 & sus sprach mîn hêr Gawan.\\ 
 & des antwurte im diu clâre maget:\\ 
 & "nâch minne ist dienst mich verdaget,\\ 
15 & wan der herzoginne von Logrois\\ 
 & dienet manec rîter kurtois\\ 
 & beidiu nâch minne und umb ir solt.\\ 
 & der hât maneger hie \textbf{gedolt}\\ 
 & jostieren, \textbf{daz} wir ez sâhen.\\ 
20 & ir dekeiner nie sô nâhen\\ 
 & kam, als ir uns komen sît.\\ 
 & \textbf{der} prîs ûf hœhet \textbf{iuwern} strît."\\ 
 & er sprach zuo der megde wol gevar:\\ 
 & "war krieget der herzoginne schar\\ 
25 & sus manec rîter ûzerkorn?\\ 
 & wer hât ir hulde verlorn?"\\ 
 & si sprach: "daz hât \textbf{der künec} Gramoflanz,\\ 
 & der der wirdecheit kranz\\ 
 & treit, als diu volge giht.\\ 
30 & hêrre, \textbf{ich erkenne sîn} anders niht."\\ 
\end{tabular}
\scriptsize
\line(1,0){75} \newline
U V W Q R \newline
\line(1,0){75} \newline
\textbf{1} \textit{Initiale} U W   $\cdot$ \textit{Capitulumzeichen} R  \textbf{13} \textit{Capitulumzeichen} R  \newline
\line(1,0){75} \newline
\textbf{2} wen] wan Q \textbf{4} rîter nie] nie ritter V \textbf{5} gespræche] gesproche Q \textbf{6} als] als vil \textit{(abweichende Reklamante:} als\textit{)} R  $\cdot$ hât] haben \textit{(abweichende Reklamante)} R \textbf{7} möhten] mochten U (Q) \textbf{8} wâ] [Waz]: Wa V \textbf{9} bejageten] Beiagen Q \textbf{12} sus] Als Q  $\cdot$ Gawan] gawann Q \textbf{13} antwurte] antwúrt V (R) atwort Q \textbf{14} Nach minne [*]: ist dienest mich verdaget V  $\cdot$ dienst] dienstes W \textbf{15} der] die W R  $\cdot$ Logrois] logroẏs V logroys W ligrois Q \textbf{18} gedolt] geholt V Q R \textbf{19} wir ez] wirs hie V \textbf{22} der prîs ûf] [D* *f]: Der pris vf V  $\cdot$ iuwern strît] úwer sind R \textbf{25} sus] Als Q \textbf{27} der] \textit{om.} W  $\cdot$ Gramoflanz] gramaflanz V gramoflantz W Q [Gam]: Gramoflancz R \textbf{29} giht] [*]: im giht V \textbf{30} erkenne sîn] erkenne es W (R) erkendes Q \newline
\end{minipage}
\end{table}
\end{document}
