\documentclass[8pt,a4paper,notitlepage]{article}
\usepackage{fullpage}
\usepackage{ulem}
\usepackage{xltxtra}
\usepackage{datetime}
\renewcommand{\dateseparator}{.}
\dmyyyydate
\usepackage{fancyhdr}
\usepackage{ifthen}
\pagestyle{fancy}
\fancyhf{}
\renewcommand{\headrulewidth}{0pt}
\fancyfoot[L]{\ifthenelse{\value{page}=1}{\today, \currenttime{} Uhr}{}}
\begin{document}
\begin{table}[ht]
\begin{minipage}[t]{0.5\linewidth}
\small
\begin{center}*D
\end{center}
\begin{tabular}{rl}
\textbf{180} & er \textbf{en}mag es vor jâmer niht \textbf{enthaben},\\ 
 & ez welle \textbf{springen} oder draben.\\ 
 & \textbf{kriuze} unt stûden stric,\\ 
 & dar zuo \textbf{der} wagenleisen \textbf{bic}\\ 
5 & sîne waltstrâzen meit.\\ 
 & vil ungevertes er dô reit,\\ 
 & dâ \textbf{wênic} wegerîches stuont.\\ 
 & \textbf{tal unt berc} \textbf{wâren} im unkunt.\\ 
 & Genuoge hânt des einen site\\ 
10 & \textbf{unt} \textbf{sprechent} \textbf{sus}: swer irre rite,\\ 
 & daz \textbf{der} den slegel vünde.\\ 
 & slegels urkünde\\ 
 & lac dâ âne mâze vil,\\ 
 & sulen grôze ronen \textbf{sîn} slegels zil.\\ 
15 & \textbf{\begin{large}D\end{large}och} reit er \textbf{wênec} \textbf{irre},\\ 
 & \textbf{wan} die slihte \textbf{an} der \textbf{virre}\\ 
 & \textbf{kom} er des tages von Graharz\\ 
 & in daz künecrîche ze Brobarz\\ 
 & durch wilde gebirge hôch.\\ 
20 & der tac gein dem âbende zôch.\\ 
 & \textbf{Dô kom er} an ein wazzer snel.\\ 
 & daz was von sîme \textbf{duzze} hel.\\ 
 & \textbf{ez} gâben die velse ein ander.\\ 
 & daz reit er \textbf{nider}. dô vander\\ 
25 & die stat ze Pelrapeire.\\ 
 & der künec Tampenteire\\ 
 & het si geerbet ûf sîn kint,\\ 
 & bî der vil liute in kumber sint.\\ 
 & daz wazzer vuor nâch \textbf{bolze} siten,\\ 
30 & die wol gevidert unt gesniten\\ 
\end{tabular}
\scriptsize
\line(1,0){75} \newline
D Fr15 \newline
\line(1,0){75} \newline
\textbf{9} \textit{Majuskel} D  \textbf{15} \textit{Initiale} D Fr15  \textbf{21} \textit{Majuskel} D  \newline
\line(1,0){75} \newline
\textbf{26} Tampenteire] Tampenteîre D \newline
\end{minipage}
\hspace{0.5cm}
\begin{minipage}[t]{0.5\linewidth}
\small
\begin{center}*m
\end{center}
\begin{tabular}{rl}
 & er mac es vor jâmer niht \textbf{erhaben},\\ 
 & ez welle \textbf{springen} oder traben.\\ 
 & \textbf{grüete} und stûden stric,\\ 
 & dar zuo \textbf{den} wagenleisen \textbf{bic}\\ 
5 & sîne waltstrâzen meit.\\ 
 & vil ungevertes er dô reit,\\ 
 & d\textit{â} \textbf{wênic} wegerîches stuont.\\ 
 & \textbf{berge und tal} \textbf{wâren} ime unkunt.\\ 
 & genuoge hânt des einen site,\\ 
10 & \textbf{si} \textbf{jehen\textit{t}} \textbf{sus}: wer irre rite,\\ 
 & daz \textbf{der} den slegel vünde.\\ 
 & \textit{s}lege\textit{l}s urkünde\\ 
 & lac d\textit{â} âne mâze vil,\\ 
 & \textit{s}uln grôze ronen \textbf{sîn} slegels zil.\\ 
15 & \textbf{doch} reit er \textbf{wênic} \textbf{strâze},\\ 
 & \textbf{wanne} die slihte \textbf{an} der \textbf{mâze}\\ 
 & \textbf{kam} er des tages von Graha\textit{r}z\\ 
 & in daz künicrîch ze Bro\textit{b}a\textit{r}z\\ 
 & durch wilde gebirge hôch.\\ 
20 & der tac \textbf{sich} gegen dem âbende zôch.\\ 
 & \textbf{dô kam er} an ein wazzer snel.\\ 
 & daz was von sînem \textbf{duzze} hel.\\ 
 & \textbf{ez} gâben die velse einander.\\ 
 & daz reit er \textbf{in}. dô vant er\\ 
25 & die stat ze P\textit{e}l\textit{r}ape\textit{ri}e.\\ 
 & der künic \textit{T}ampent\textit{erie}\\ 
 & hete si geerb\textit{e}t ûf sîn kint,\\ 
 & bî der vil liute in kumber sint.\\ 
 & daz wazzer vuor nâch \textbf{bolze} siten,\\ 
30 & die wol gevidert und gesniten\\ 
\end{tabular}
\scriptsize
\line(1,0){75} \newline
m n o Fr69 \newline
\line(1,0){75} \newline
\newline
\line(1,0){75} \newline
\textbf{1} vor] von n o \textbf{3} grüete] Trute n o \textbf{5} waltstrâzen] walstrossen n \textbf{6} ungevertes] vn gefortes n (o) \textbf{7} dâ] Do m n o \textbf{9} des] das o \textbf{10} si] So o  $\cdot$ jehent] iehen m johent n (o)  $\cdot$ wer] were n  $\cdot$ rite] sitte n \textbf{11} der] er o \textbf{12} slegels] Vrkunde sleges m n o \textbf{13} dâ] do m n o  $\cdot$ âne mâze] vnmossen n o \textbf{14} suln] fvln m (n) (o)  $\cdot$ sîn] sins n o  $\cdot$ slegels] [*]: sleges o \textbf{15} strâze] irre Fr69 \textbf{16} slihte] slecht n (o)  $\cdot$ mâze] virre Fr69 \textbf{17} kam] Kan o  $\cdot$ Graharz] grahancz m grahantz n grahars Fr69 \textbf{18} Brobarz] brobrancz m brobrantz n brabrancz o brobars Fr69 \textbf{19} wilde] vilde n \textbf{23} velse] folsch n felsch o \textbf{24} reit er] reit n \textbf{25} die] [Dis]: Die n  $\cdot$ Pelraperie] palapeire m pelapeir n pelapier o \textbf{26} Tampenterie] kampentirer m tampanteir n tampenteir o \textbf{27} geerbet] gerbot m \textbf{29} bolze] boltzes n (o) \textbf{30} die] Do o \newline
\end{minipage}
\end{table}
\newpage
\begin{table}[ht]
\begin{minipage}[t]{0.5\linewidth}
\small
\begin{center}*G
\end{center}
\begin{tabular}{rl}
 & er \textbf{en}mag es vor jâmer niht \textbf{enthaben},\\ 
 & ez welle \textbf{schiuften} oder draben.\\ 
 & \textbf{kriuze} unde stûden stric,\\ 
 & dar zuo \textbf{der} wagenleisen \textbf{blic}\\ 
5 & sîne waltstrâze meit.\\ 
 & vil ungevertes er dô reit,\\ 
 & dâ \textbf{lützel} we\textit{ge}rîches stuont.\\ 
 & \textbf{tal unde berc} \textbf{was} im unkunt.\\ 
 & genuoge habent des einen site\\ 
10 & \textbf{unde} \textbf{gehent}, swer irre rite,\\ 
 & daz \textbf{der} den slegel vünde.\\ 
 & slegeles urkünde\\ 
 & lac dâ âne mâze vil,\\ 
 & sulen grôze ronen \textbf{sîn} slegels zil.\\ 
15 & \textbf{doch} reit er \textbf{lützel} \textbf{irre},\\ 
 & \textbf{wan} die slihte \textbf{an} der \textbf{virre}\\ 
 & \textbf{kom} er des tages von Graharz\\ 
 & in daz künicrîche ze Briubarz\\ 
 & durch wilde gebirge hôch.\\ 
20 & der tac \textbf{dô} gein dem âbende zôch.\\ 
 & \textbf{er kom} an ein wazzer snel.\\ 
 & daz was von sînem \textbf{duzze} hel.\\ 
 & \textbf{ez} gâben die velse ein ander.\\ 
 & daz reit er \textbf{nider}. dô vand er\\ 
25 & die stat ze Pelrapeire.\\ 
 & der künic Tampunteire\\ 
 & het si geerbet ûf sîn kint,\\ 
 & bî der vil liute in kumber sint.\\ 
 & daz wazzer vuor nâch \textbf{bolzes} siten,\\ 
30 & die wol gevidert unde gesniten\\ 
\end{tabular}
\scriptsize
\line(1,0){75} \newline
G I O L M Q R Z Fr40 Fr47 \newline
\line(1,0){75} \newline
\textbf{7} \textit{Überschrift:} Aventiwer wie Parzifal frovn [Gundwiramus]: Gundwiramurs gewan die liecht gemal I   $\cdot$ \textit{Initiale} I  \textbf{13} \textit{Initiale} M  \textbf{15} \textit{Initiale} O L Z Fr40  \textbf{25} \textit{Initiale} I  \newline
\line(1,0){75} \newline
\textbf{1} \sout{Er enmochcz vor iamer nit gehaben} Er enmochcz vor Jamer nit mere gehaben R  $\cdot$ er enmag es] er mags I Ern mach sin O (L) Er mag sein Fr47  $\cdot$ vor] von O  $\cdot$ enthaben] gehaben M Q (R) (Fr40) (Fr47) \textbf{2} ez] ezn I  $\cdot$ welle] woltt R  $\cdot$ schiuften] springen I O L (M) (Q) R Z (Fr40) (Fr47) \textbf{3} kriuze] Groz O Cruͯte L Cruter Q Kvrtze Z Strauch Fr47  $\cdot$ stûden] chleine stvden O stunden Z  $\cdot$ stric] strit R \textbf{4} dar] Das Q  $\cdot$ wagenleisen] waleisen I wagenleise O L (M) (R) (Z) (Fr40) (Fr47)  $\cdot$ blic] bich O L (M) (Z) (Fr47) bit Q wit R \textbf{6} dô] da M Z  $\cdot$ reit] [meit]: reit O \textbf{7} dâ] Daz I Do Q  $\cdot$ lützel] wenig L  $\cdot$ wegerîches] weriches G vngereyses Q wegerich Fr47 \textbf{8} berc] berge O M Q Z Fr40  $\cdot$ was] warn I (O) (L) (Q) (R) (Z) Fr40 Fr47 sin M \textbf{9} des] \textit{om.} Q R Fr40  $\cdot$ einen] ein I  $\cdot$ site] siten Q (R) [siten]: site  Z \textbf{10} swer] des swer O M des wer L Q wer R des wes Z \textbf{11} daz der] daz er I Wer L  $\cdot$ slegel vünde] slegen súnde R \textbf{12} slegeles] Ylegels Q Schleges R \textbf{13} \textit{Die Verse 180.13-14 fehlen} I   $\cdot$ lac] Lat Q  $\cdot$ dâ] do Q  $\cdot$ âne mâze] vnemasze R \textbf{14} slegels zil] slegil zcil M [selegels]: slegels stil Q \textbf{15} doch] do I ÷och O  $\cdot$ er] der Q \textbf{16} Von der slechte andir virre M  $\cdot$ slihte] schilte L schlechte R  $\cdot$ an der] an den R nach der Z  $\cdot$ virre] wirre Fr40 \textbf{17} von] \textit{om.} R  $\cdot$ Graharz] grahorsz M grahars Q Graharcz R \textbf{18} ze] \textit{om.} R  $\cdot$ Briubarz] briebarz I Brvbarz O (L) (Z) brubarsz M bribartz Q Brubarcz R brubartz Fr40 \textbf{19} hôch] hohc O \textbf{20} dô] \textit{om.} L Z da M \textbf{21} er kom] do (Da Z ) chom er I (O) (L) (M) (Q) (R) (Z) (Fr40) \textbf{22} duzze] dvssen L also Q \textbf{23} gâben] gab I  $\cdot$ velse] velsche M felsen R \textbf{24} dô] da R Z \textbf{25} Pelrapeire] [pelrap*ire]: pelrapeire G pailrapaier I Pelrapeir O peylrapeire Q \textbf{26} Tampunteire] tanpantaier I Tampvnteir O Tampvntaiere L túmpenteire Q Tampuͦnteire R \textbf{27} si] sich I  $\cdot$ sîn] siniu I (M)  $\cdot$ kint] kin Q \textbf{28} liute] lútten R \textbf{29} daz] Dize O (L) (M) (Q) (R) (Fr40)  $\cdot$ wazzer] wassers R  $\cdot$ nâch] in I  $\cdot$ bolzes] boltze Z \textbf{30} die] Der L  $\cdot$ wol] wl R \newline
\end{minipage}
\hspace{0.5cm}
\begin{minipage}[t]{0.5\linewidth}
\small
\begin{center}*T
\end{center}
\begin{tabular}{rl}
 & er\textbf{n} mag\textit{s} vor jâmer niht \textbf{enthaben},\\ 
 & e\textit{z} welle \textbf{springen} oder traben.\\ 
 & \textbf{Kriuze} unde \textbf{der} stûden stric,\\ 
 & dar zuo \textbf{der} wagenleisen \textbf{bic},\\ 
5 & sîne waltstrâzen \textbf{er} meit.\\ 
 & vil ungevertes er dô reit,\\ 
 & dâ \textbf{lützel} wegerîches stuont.\\ 
 & \textbf{tal unde berc} \textbf{wâren} im unkunt.\\ 
 & Genuoge hânt des einen site\\ 
10 & \textbf{unde} \textbf{jehent} \textbf{des}: swer irre rite,\\ 
 & daz \textbf{er} den slegel vünde.\\ 
 & slegels urkünde\\ 
 & lac dâ âne mâze vil,\\ 
 & suln grôze ronen \textbf{he\textit{i}zen} slegels zil.\\ 
15 & \textbf{\begin{large}D\end{large}ô} reit er \textbf{wênic} \textbf{irre}.\\ 
 & die slihte \textbf{nâch} der \textbf{virre}\\ 
 & \textbf{kôs} er des tages von Greharz\\ 
 & in daz künecrîche ze Brebarz\\ 
 & durch wilde gebirge hôch.\\ 
20 & der tac \textbf{sich} gegen dem âbende zôch.\\ 
 & \textbf{Dô kom er} an ein wazzer snel.\\ 
 & daz was von sînem \textbf{guzze} hel.\\ 
 & \textbf{daz} gâben die velse einander.\\ 
 & daz reit er \textbf{nider}. dô vander\\ 
25 & die stat ze Peilrapere.\\ 
 & der künec Tampuntere\\ 
 & hete si geerbet ûf sîn kint,\\ 
 & bî der vil liute in kumber sint.\\ 
 & daz wazzer vuor nâch \textbf{bolze} siten,\\ 
30 & die wol gevidert unde gesniten\\ 
\end{tabular}
\scriptsize
\line(1,0){75} \newline
T U V W \newline
\line(1,0){75} \newline
\textbf{3} \textit{Majuskel} T  \textbf{9} \textit{Majuskel} T  \textbf{15} \textit{Überschrift:} Hie kvmet parzifal zem ersten male zvͦ pelrepere V  Hie kam parzifal in daz kúnigrich zuͦ brebars in die stat pelrapier vnd erstrait aldo die kúnigin gundwiramurs W   $\cdot$ \textit{Initiale} T U V W  \textbf{21} \textit{Majuskel} T  \newline
\line(1,0){75} \newline
\textbf{1} ern] Er W  $\cdot$ mags] magz T  $\cdot$ vor] [von]: vor V \textbf{2} ez] es T (W)  $\cdot$ welle] enwelle V \textbf{3} kriuze] Krv́ter V  $\cdot$ der] \textit{om.} V W \textbf{4} der] [den]: der V  $\cdot$ wagenleisen] wegelaisen W  $\cdot$ bic] blick W \textbf{5} waltstrâzen] walt straze U guͦten stras W  $\cdot$ meit] vermeit V \textbf{6} ungevertes] vngenertes V \textbf{7} dâ] Do U V W \textbf{8} Als vnkunde leúte thuͦnd W \textbf{9} des] auch W \textbf{10} swer] wer U W \textbf{12} slegels] Slegens U Schlegelns W \textbf{13} dâ] do U W \textbf{14} heizen] herzen T [*]: sin V sein W \textbf{15} Dô] [D*]: DOch V DOch W \textbf{16} die] [*]: wande die V  $\cdot$ slihte nâch] stifte an U [*]: slihte vz V schlichte an W  $\cdot$ virre] verre U [*]: wirre V \textbf{17} kôs] [K*]: Kam V  $\cdot$ Greharz] grehartz V grahars W \textbf{18} Brebarz] Breharz T U brobarz V brebars W \textbf{19} wilde] wildes W \textbf{20} dem âbende] der nacht W \textbf{21} ein] \textit{om.} W \textbf{22} guzze] [*]: dvsse V fleis W \textbf{23} daz] Jz U (V) (W) \textbf{24} daz] Die U [D*]: Daz V \textbf{25} Peilrapere] pelrepere V pelrapier W \textbf{26} Tampuntere] Tampuͦntere U Tamputere V tampentier W \textbf{27} sîn] seine W \textbf{28} der] dir U \textbf{29} bolze] bolzes V (W) \newline
\end{minipage}
\end{table}
\end{document}
