\documentclass[8pt,a4paper,notitlepage]{article}
\usepackage{fullpage}
\usepackage{ulem}
\usepackage{xltxtra}
\usepackage{datetime}
\renewcommand{\dateseparator}{.}
\dmyyyydate
\usepackage{fancyhdr}
\usepackage{ifthen}
\pagestyle{fancy}
\fancyhf{}
\renewcommand{\headrulewidth}{0pt}
\fancyfoot[L]{\ifthenelse{\value{page}=1}{\today, \currenttime{} Uhr}{}}
\begin{document}
\begin{table}[ht]
\begin{minipage}[t]{0.5\linewidth}
\small
\begin{center}*D
\end{center}
\begin{tabular}{rl}
\textbf{273} & \textbf{\textit{\begin{large}S\end{large}}us begunde im ein rîter} sagen:\\ 
 & "ich sach ûf \textbf{einen} plân geslagen\\ 
 & tûsent poulûn oder mêr.\\ 
 & Artus, der rîche künec hêr,\\ 
5 & der \textbf{Bertenoyse} hêrre,\\ 
 & lît uns \textbf{hie} niht verre\\ 
 & mit wünneclîcher vrouwen schar.\\ 
 & ungevertes ist ein mîle dar.\\ 
 & dâ ist \textbf{ouch} von rîtern \textbf{grœzlîch} schal.\\ 
10 & bî dem Plimizœle ze tal\\ 
 & ligents an iewederm stade."\\ 
 & dô gâhte \textbf{vaste} ûzem bade\\ 
 & der herzoge Orilus.\\ 
 & Jeschute unt er gewurben sus.\\ 
15 & Diu senfte, süeze, wolgetân\\ 
 & gie \textbf{ouch} ûz ir bade sân\\ 
 & \textbf{an} \textbf{sîn} bette; dâ wart trûrens rât.\\ 
 & ir lide \textbf{gedienden} bezzer wât,\\ 
 & den si dâ vor truoc lange.\\ 
20 & mit nâhem umbevange\\ 
 & behiel\textit{t} ir minne vreuden prîs,\\ 
 & der vürstinne unt des vürsten wîs.\\ 
 & juncvrouwen kleideten ir vrouwen sân,\\ 
 & sîn harnasch truoc man dar dem man.\\ 
25 & Jeschuten wât man muose loben.\\ 
 & vogele gevangen ûf \textbf{dem} kloben\\ 
 & si mit vreuden âzen,\\ 
 & dô si an ir bette sâzen,\\ 
 & \textbf{vrou} Jeschute etslîchen kus\\ 
30 & enpfienc, den gap ir Orilus.\\ 
\end{tabular}
\scriptsize
\line(1,0){75} \newline
D \newline
\line(1,0){75} \newline
\textbf{1} \textit{Initiale} D  \textbf{15} \textit{Majuskel} D  \newline
\line(1,0){75} \newline
\textbf{1} Sus] ÷vs \textit{nachträglich korrigiert zu:} Svs D \textbf{10} Plimizœle] Primizoͤle D \textbf{14} Jeschute] Jescvte D \textbf{21} behielt] behîel D \textbf{25} Jeschuten] Jescvten D \textbf{29} Jeschute] Jescvte D \newline
\end{minipage}
\hspace{0.5cm}
\begin{minipage}[t]{0.5\linewidth}
\small
\begin{center}*m
\end{center}
\begin{tabular}{rl}
 & \textbf{\begin{large}E\end{large}in ritter kam und begunde} sagen:\\ 
 & "ich sach ûf \textbf{einem} plân geslagen\\ 
 & \textbf{wol} tûsent pavelûne oder mêr.\\ 
 & Artus, der rîche künic hêr,\\ 
5 & \textit{der \textbf{Brituneiser} hêrre},\\ 
 & lît uns \textbf{hie} niht \textit{v}er\textit{r}e\\ 
 & mit wünneclîcher vrouwen schar.\\ 
 & ungevertes ist \textbf{wol} ein mîle dar.\\ 
 & d\textit{â} ist \textbf{ouch} von rittern \textbf{michel} schal.\\ 
10 & bî dem Pl\textit{i}m\textit{i}zol ze tal\\ 
 & ligent si an ietwederem stade."\\ 
 & dô gâhete \textbf{vaste} ûz dem bade\\ 
 & der herzoge Orilus.\\ 
 & Jeschute unt er gewurben sus.\\ 
15 & diu senfte, süeze, wol getân\\ 
 & gienc \textbf{ouch} ûz ir bade sân\\ 
 & \textbf{in} \textbf{sîn} bette; d\textit{â} wart trûrens rât.\\ 
 & ir lide \textbf{gedienden} bezzer wât,\\ 
 & danne si dâ vor truoc lange.\\ 
20 & mit nâhem umbevange\\ 
 & behielt ir minne vröuden prîs,\\ 
 & der vürstinne unt des vürsten wîs.\\ 
 & juncvrouwen kleiten ir vrouwen sân,\\ 
 & sîn harna\textit{s}ch truoc man dar dem man.\\ 
25 & Jeschuten wât ma\textit{n} muose loben.\\ 
 & vogele gevangen ûf \textbf{dem} kloben\\ 
 & si mit vröuden âzen,\\ 
 & dô si an ir bette sâzen,\\ 
 & \textbf{vrouwe} Jeschute etslîchen kus\\ 
30 & enpfienc, den gap ir Orilus.\\ 
\end{tabular}
\scriptsize
\line(1,0){75} \newline
m n o Fr69 \newline
\line(1,0){75} \newline
\textbf{1} \textit{Initiale} m n o  \newline
\line(1,0){75} \newline
\textbf{5} \textit{Vers 273.5 fehlt} m   $\cdot$ Brituneiser] britaneiser o \textbf{6} verre] mere m \textbf{7} wünneclîcher] mẏnneclicher n (o) \textbf{8} ungevertes] Vngefers o \textbf{9} dâ] Do m n o  $\cdot$ schal] [schar]: schal o \textbf{10} Plimizol] plvm zol m plumzol n \textbf{12} gâhete] gehoͯhete o \textbf{14} Jeschute] Jescutte m Jescute n o  $\cdot$ gewurben] die wurben n sie worben o \textbf{16} ûz] uch o \textbf{17} dâ] do m n o \textbf{18} gedienden bezzer] besser gedienten besser o \textbf{19} danne] Das n  $\cdot$ truoc] truͯge n (o) \textbf{20} nâhem] fohen n fahen o \textbf{24} harnasch] harnach m harnersch o \textbf{25} Jeschuten] Jescutten m Jescuten n Juscuten o  $\cdot$ man muose] mam musse m man muͯste n \textbf{26} vogele gevangen] Vogel gesange n o  $\cdot$ dem] den n \textbf{28} dô] Die o \textbf{29} Jeschute] jescutte m jescute n (Fr69) jescuͯten o  $\cdot$ etslîchen] etlicher o \textbf{30} ir] er o  $\cdot$ Orilus] [orilur]: orilus n orilis o \newline
\end{minipage}
\end{table}
\newpage
\begin{table}[ht]
\begin{minipage}[t]{0.5\linewidth}
\small
\begin{center}*G
\end{center}
\begin{tabular}{rl}
 & \textbf{sus begunde im ein rîter} sagen:\\ 
 & "ich sach ûf \textbf{einen} plân geslagen\\ 
 & \textbf{wol} tûsent pavelûn oder mêr.\\ 
 & Artus, der rîche künic hêr,\\ 
5 & \begin{large}D\end{large}er \textbf{britânische} hêrre,\\ 
 & lît uns \textbf{hie} niht \textbf{ze} verre\\ 
 & mit wünniclîcher vrouwen schar.\\ 
 & ungevertes ist ein mîle dar.\\ 
 & dâ ist von rîteren \textbf{grôzer} schal.\\ 
10 & bî dem Blimzol ze tal\\ 
 & ligent si an ietwederm stade."\\ 
 & dô gâhte \textbf{balde} ûz dem bade\\ 
 & der herzoge Orillus.\\ 
 & Jeschute unde er gewurben sus.\\ 
15 & diu senfte, süeze, wolgetân\\ 
 & gienc \textbf{ouch} ûz ir bade sân\\ 
 & \textbf{an} \textbf{ir} bette; dâ wart trûrens rât.\\ 
 & ir lide \textbf{dienten} bezzer wât,\\ 
 & danne si dâ vor truoc lange.\\ 
20 & mit nâhem umbevange\\ 
 & behielt ir minne vröuden prîs,\\ 
 & der vürstîn unde des vürsten wîs.\\ 
 & juncvrouwen kleiten ir vrouwen sân,\\ 
 & sîn harnasch truoc man dar dem man.\\ 
25 & Jeschuten wât man muose loben.\\ 
 & vogele gevangen ûf \textbf{dem} kloben\\ 
 & si mit vröuden âzen,\\ 
 & dâ si an ir bette sâzen,\\ 
 & Jeschute etslîchen kus\\ 
30 & enpfie, den gap ir Orillus.\\ 
\end{tabular}
\scriptsize
\line(1,0){75} \newline
G I O L M Q R Z Fr36 \newline
\line(1,0){75} \newline
\textbf{1} \textit{Initiale} R  \textbf{5} \textit{Initiale} G  \textbf{13} \textit{Initiale} I  \textbf{25} \textit{Initiale} M Z  \textbf{27} \textit{Initiale} R  \textbf{29} \textit{Initiale} O L  \newline
\line(1,0){75} \newline
\textbf{1} im] in Fr36  $\cdot$ rîter] knappe Fr36 \textbf{2} einen] einem O Fr36  $\cdot$ plân] paln Q \textbf{3} wol] \textit{om.} I O L M Q R Z  $\cdot$ pavelûn] geczelt R \textbf{4} Artus] Artuͯs L  $\cdot$ rîche] \textit{om.} I \textbf{5} britânische] britansche G brittanisce I Britaneisen O Brýttaneise L britaneiser M britoneysze Q briteneiser R briteneise Z \textbf{6} uns hie] von vns I  $\cdot$ ze] \textit{om.} I O L M Q Z  $\cdot$ verre] were R \textbf{7} vrouwen] vroudin M \textbf{8} ungevertes] Vngevestes M Vngefer des R  $\cdot$ mîle] milde Q \textbf{9} dâ] Do O Q  $\cdot$ ist] ist auch I Q (R) (Z) \textbf{10} Blimzol] blimizol I M Fr36 Blimvlzol O plimizol L Z plinűtzol Q Bimizol R \textbf{11} ligent si] legen sie M Ligend Fr36  $\cdot$ an ietwederm] ietwederthalp an dem I \textbf{12} dô] Da M Z  $\cdot$ gâhte] gaht O Fr36 iagete M  $\cdot$ ûz] zcu M \textbf{13} Orillus] Orilus I (O) M (Q) R (Z) (Fr36) \textbf{14} er vnd ieskute wurben sus I  $\cdot$ Jeschute] ieschute G Jeschvͦte O Jescuͯte L Jescute M Q Z Jscute R iescute Fr36  $\cdot$ er gewurben] er di wurben O erworben L her gidachten M er gewrbe Q  $\cdot$ sus] alsus M \textbf{15} senfte süeze] suͯsze senfte L (M) \textbf{16} ûz ir] ausz dem Q vsser R \textbf{17} ir] sin O L (M) (Q) R Z (Fr36)  $\cdot$ dâ] do Q R  $\cdot$ trûrens] trofrins Q \textbf{18} lide] libe I M leyde Q  $\cdot$ dienten] diente ein I diente M gedienten Fr36  $\cdot$ wât] tat L \textbf{19} danne] Do Q  $\cdot$ truoc] truge I Q (R) trugen M \textbf{20} nâhem] nahen Fr36  $\cdot$ umbevange] vmme fangen Q \textbf{21} vröuden] selden I vrouwen M werden Fr36 \textbf{22} Die vorstÿnne vnde der vursten wis M  $\cdot$ des vürsten] der furste R \textbf{23} juncvrouwen] Jvngfrauwe L  $\cdot$ kleiten] leitten R \textbf{24} dar] \textit{om.} I  $\cdot$ dem] der M \textbf{25} Jeschuten] ieschuten G ieskuten I Jeschvͦten O Jescuͯten L Jescuten M (Q) (Z) Jscuten R  $\cdot$ wât] wert L was M  $\cdot$ muose] sie Q \textbf{26} dem] den M einen Q \textbf{27} vröuden] ein ander L \textbf{28} Am bett da sy sazen R  $\cdot$ dâ] Do Q \textbf{29} Jeschute] ieschute G ieskute I ÷eschvte O Jescuͯte L Jescuten M Jescute Q Z Frow Iscuten R \textbf{30} Orillus] Orilus I (O) (M) (Q) R (Z) \newline
\end{minipage}
\hspace{0.5cm}
\begin{minipage}[t]{0.5\linewidth}
\small
\begin{center}*T
\end{center}
\begin{tabular}{rl}
 & \textbf{sus begundim ein rîter} sagen:\\ 
 & "ich sach ûf \textbf{einen} plân geslagen\\ 
 & tûsent pavelûn oder mêr.\\ 
 & Artus, der rîche künec hêr,\\ 
5 & der \textbf{Brituneise} hêrre,\\ 
 & lît uns \textbf{bî} niht verre\\ 
 & mit wünneclîcher vrouwen schar.\\ 
 & ungevertes ist ein mîle dar.\\ 
 & dâ ist \textbf{ouch} von rîtern \textbf{grôzer} schal.\\ 
10 & bî dem Plymizol ze tal\\ 
 & ligent si a\textit{n} ietwederme stade."\\ 
 & Dô gâhte \textbf{der vürste} ûz dem bade,\\ 
 & der herzoge Orilus.\\ 
 & Jeschute unde er gewurben sus.\\ 
15 & \begin{large}D\end{large}iu senfte, süeze, wol getân\\ 
 & gie ûz ir bade sân\\ 
 & \textbf{an} \textbf{ir} bette; dâ wart trûrens rât.\\ 
 & ir \textit{lide} \textbf{gedienden} bezzer wât,\\ 
 & danne si dâ vor truoc lange.\\ 
20 & mit nâhem umbevange\\ 
 & behielt ir minne vröuden prîs,\\ 
 & der vürstîn unde des vürsten wîs.\\ 
 & Juncvrouwen kleideten ir vrouwen sân,\\ 
 & sîn harnasch truog man dar dem man.\\ 
25 & Jeschuten wât man muose loben.\\ 
 & vogele gevangen ûf \textbf{den} kloben\\ 
 & si mit vröuden âzen,\\ 
 & dâ si an ir bette sâzen,\\ 
 & \textbf{vrou} Jeschute etslîchen kus\\ 
30 & enpfienc, den gab ir Orilus.\\ 
\end{tabular}
\scriptsize
\line(1,0){75} \newline
T U V W \newline
\line(1,0){75} \newline
\textbf{12} \textit{Majuskel} T  \textbf{15} \textit{Initiale} T U W  \textbf{23} \textit{Majuskel} T  \newline
\line(1,0){75} \newline
\textbf{1} Ein ritter kam vnd begonde sagen V  $\cdot$ begundim] begund W \textbf{2} einen] eime U \textbf{3} tûsent pavelûn] [*]: Wol tusent gezelt V \textbf{5} Brituneise] [britv*]: britvnesce T Brituͦneise U brittvneise V brituneise W \textbf{6} bî] hie bi U V hie W \textbf{7} vrouwen] \textit{om.} U \textbf{8} ist] ist [*]: wol V \textbf{9} dâ] Do U V W  $\cdot$ ouch] \textit{om.} W \textbf{10} Plymizol] plimizol V W \textbf{11} an ietwederme] andietwederme T an letwederme U \textbf{12} gâhte] gedachte U \textbf{14} Jeschute] Jescvte T (U) [Jescuten]: Jescute  V Iestute W  $\cdot$ er] der U  $\cdot$ gewurben] gevuͦren U [*]: gewurben V wurben W \textbf{15} senfte süeze] suͤsse senffte W \textbf{16} ûz] oͮch vz V (W)  $\cdot$ ir] dem U \textbf{17} an ir] An irm U [A*]: An sin V An ein W  $\cdot$ dâ] [*]: do V do W  $\cdot$ wart] was U \textbf{18} ir lide] [*]: ir ::: T Jr liden U Ir leib vnd lide W  $\cdot$ gedienden] gedietende U  $\cdot$ bezzer] besserv́ V \textbf{20} nâhem] nassem W \textbf{23} kleideten] [cleiden]: cleideten T \textbf{24} dar] daz W \textbf{25} Jeschuten] Jescvten T (V) Jescute U Iestuten W  $\cdot$ muose] mvͤste V \textbf{26} den] dem V W \textbf{28} dâ] Do V W \textbf{29} Jeschute] Jescvte T Jescuͦte U [jescute*]: jescute V iestute W \textbf{30} gab ir] ir gab W \newline
\end{minipage}
\end{table}
\end{document}
