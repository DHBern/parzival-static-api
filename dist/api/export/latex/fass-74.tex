\documentclass[8pt,a4paper,notitlepage]{article}
\usepackage{fullpage}
\usepackage{ulem}
\usepackage{xltxtra}
\usepackage{datetime}
\renewcommand{\dateseparator}{.}
\dmyyyydate
\usepackage{fancyhdr}
\usepackage{ifthen}
\pagestyle{fancy}
\fancyhf{}
\renewcommand{\headrulewidth}{0pt}
\fancyfoot[L]{\ifthenelse{\value{page}=1}{\today, \currenttime{} Uhr}{}}
\begin{document}
\begin{table}[ht]
\begin{minipage}[t]{0.5\linewidth}
\small
\begin{center}*D
\end{center}
\begin{tabular}{rl}
\textbf{74} & der vürste hiez Lambekin.\\ 
 & waz \textbf{tæten} die sîn?\\ 
 & \textbf{die} beschutten in mit swerten.\\ 
 & die helde strîtes gerten.\\ 
5 & \begin{large}D\end{large}ô stach der künec von Arragun\\ 
 & den alten Utepandragun\\ 
 & hinder\textit{z} ors ûf die plâne,\\ 
 & den künec von Bertane.\\ 
 & \textbf{ez} \textbf{stuont} dâ \textbf{bluomen vil} umb in.\\ 
10 & \textbf{wê}, wie gevüege ich doch bin,\\ 
 & daz ich den werden Berteneis\\ 
 & sô schône lege vür Kanvoleis,\\ 
 & dâ nie getrat vilânes vuoz\\ 
 & - ob ich\textbf{z} \textbf{iu rehte sagen} muoz -\\ 
15 & noch lîhte nimmer dâ geschiht.\\ 
 & er\textbf{n} dorfte sîn \textbf{besezzen} niht\\ 
 & ûf dem orse, \textbf{aldâ} er saz.\\ 
 & niht langer man sîn dô vergaz;\\ 
 & in beschutten, die \textbf{ob im} dâ \textbf{striten}.\\ 
20 & dâ wart grôz \textbf{hurt} niht vermiten.\\ 
 & Dô kom der künec von Punturteis.\\ 
 & \textbf{der} wart al \textbf{hie} vor Kanvoleis\\ 
 & gevellet ûf sînes orses slâ,\\ 
 & daz er dâr hinder lag aldâ.\\ 
25 & daz tet der stolze Gahmuret.\\ 
 & wettâ, hêrre, wettâ wet!\\ 
 & mit strîte \textbf{vunden} si geweten.\\ 
 & sîner muomen sun Kayleten,\\ 
 & \textbf{den} viengen Punturteise.\\ 
30 & dâ wart vil \textbf{rûch} diu reise.\\ 
\end{tabular}
\scriptsize
\line(1,0){75} \newline
D Fr33 \newline
\line(1,0){75} \newline
\textbf{5} \textit{Initiale} D  \textbf{21} \textit{Majuskel} D  \newline
\line(1,0){75} \newline
\textbf{5} Arragun] Arragvͦn D \textbf{6} Utepandragun] Vͦtepandragvͦn D \textbf{7} hinderz] hindrs D \textbf{25} Gahmuret] Gahmvret D \textbf{29} Punturteise] Pvntvrteyse D \newline
\end{minipage}
\hspace{0.5cm}
\begin{minipage}[t]{0.5\linewidth}
\small
\begin{center}*m
\end{center}
\begin{tabular}{rl}
 & der vürste hiez Lambekin.\\ 
 & waz \textbf{dô tæten} die sîn?\\ 
 & \textbf{si} beschutten in mit swerten.\\ 
 & die helde strîtes gerten.\\ 
5 & \begin{large}D\end{large}ô stach der künic von Aragun\\ 
 & den alten Ut\textit{ra}pandragun\\ 
 & Hinder daz ros ûf die plânîe,\\ 
 & den künic von Britanie.\\ 
 & \textbf{ez} \textbf{stuont} d\textit{â} \textbf{bluomen vil} umb in.\\ 
10 & \textbf{wê}, wie gevüege ich doch bin,\\ 
 & daz ich den werden Brit\textit{u}neis\\ 
 & sô schône lege vür Kanvoleis,\\ 
 & dâ nie getrat villânes vuoz\\ 
 & - ob ich \textbf{daz} \textbf{iu rehte sagen} muoz -\\ 
15 & noch lîhte niemer dâ geschiht.\\ 
 & er \textbf{en}dorfte sîn \textbf{besezzen} niht\\ 
 & ûf dem rosse, \textbf{aldâ} er saz.\\ 
 & niht langer man sîn dô vergaz;\\ 
 & in beschutten, die \textbf{ob ime} d\textit{â} \textbf{striten}.\\ 
20 & d\textit{â} wart grôz \textbf{hurten} niht vermiten.\\ 
 & dô kam der künic von P\textit{on}turteis.\\ 
 & \textbf{der} wart al\textbf{hie} vor Kanvo\textit{l}eis\\ 
 & gevellet ûf sînes rosses slâ,\\ 
 & daz er dâr hinder lac aldâ.\\ 
25 & daz tet der stolze Gahmuret.\\ 
 & wetâ, hêr, wet\textit{â} \textit{w}e\textit{t}!\\ 
 & mit strîte \textbf{vunden} si geweten.\\ 
 & sîner muomen sun Kaileten,\\ 
 & \textbf{den} vien\textit{g}en Ponturteise.\\ 
30 & d\textit{â} wart vil \textbf{rîch} diu reise.\\ 
\end{tabular}
\scriptsize
\line(1,0){75} \newline
m n o \newline
\line(1,0){75} \newline
\textbf{5} \textit{Illustration mit Überschrift:} Also der koͯnnig von arragun den koͯnig von britanyen stach vnder das rosz n   $\cdot$ \textit{Initiale} m n  \newline
\line(1,0){75} \newline
\textbf{1} hiez] der hiesz n (o) \textbf{2} dô] da o  $\cdot$ tæten] doten n (o) \textbf{5} \textit{Die Verse 74.5-26 fehlen} o   $\cdot$ Aragun] arragun n \textbf{6} Utrapandragun] vtarbandragun m vtre pandragun n \textbf{7} hinder daz] Hinder \textit{nachträglich korrigiert zu:} Hindersz m \textbf{8} Britanie] brittane m \textbf{9} dâ] do m n \textbf{11} Brituneis] brittneis m britaneis n \textbf{12} vür] von n  $\cdot$ Kanvoleis] canuoleis m kanaleis n \textbf{13} dâ] Do n \textbf{14} daz iu] úch das n \textbf{15} niemer dâ geschiht] in einer tageschicht n \textbf{16} endorfte] enbedorfft n \textbf{19} dâ] do m n \textbf{20} dâ] Do m n \textbf{21} Ponturteis] puliturteis m prilitorteis n \textbf{22} wart] [was]: wart m  $\cdot$ Kanvoleis] [tannloeis]: canvoloeis m kanfoleis n \textbf{25} daz] Des n  $\cdot$ Gahmuret] gamiret n \textbf{26} wetâ wet] weta h weta m \textbf{27} vunden] fuͯnden o \textbf{28} Kaileten] kailetten m kaẏleten o \textbf{29} viengen] vienden m  $\cdot$ Ponturteise] punturteise m ponturteisen n puͯrtoteẏsen o \textbf{30} dâ] Do m n o  $\cdot$ reise] reisen n [weise]: reise o \newline
\end{minipage}
\end{table}
\newpage
\begin{table}[ht]
\begin{minipage}[t]{0.5\linewidth}
\small
\begin{center}*G
\end{center}
\begin{tabular}{rl}
 & der \textit{vürste} hiez Lambikine.\\ 
 & waz \textbf{tâten dô} die sîne?\\ 
 & \textbf{si} beschutten in mit swerten.\\ 
 & die helde strîtes gerten.\\ 
5 & dô stach der künic von Arragun\\ 
 & den alten Utpandragun\\ 
 & hinderz ors ûf die plânîe,\\ 
 & den künic von Britanie.\\ 
 & \textbf{dô} \textbf{stuont} dâ \textbf{bluomen vil} umbe in.\\ 
10 & \textbf{wê}, wie gevüege ich doch bin,\\ 
 & daz ich den werden Britaneis\\ 
 & sô schône lege vor Kanvoleis,\\ 
 & dâ nie getrat vilânes vuoz\\ 
 & - obe ich \textbf{ez} \textbf{iu rehte sagen} muoz -\\ 
15 & noch lîhte nimer dâ geschiht.\\ 
 & er dorfte sîn \textbf{gesezzen} niht\\ 
 & ûf dem orse, \textbf{dâ} er saz.\\ 
 & niht langer man sîn dâ vergaz;\\ 
 & in beschutten, die \textbf{umbe in} dâ \textbf{riten}.\\ 
20 & dâ\textbf{ne} wart grôz \textbf{hurten} niht vermiten.\\ 
 & dô kom der künic von Ponturteis.\\ 
 & \textbf{der} wart al \textbf{dâ} vor Kanvoleis\\ 
 & gevalt ûf sînes orses slâ,\\ 
 & daz er dâr hinder lac al dâ.\\ 
25 & daz tet der stolze Gahmuret.\\ 
 & wetâ, hêrre, wetâ wet!\\ 
 & mit strîte \textbf{wurden} si geweten.\\ 
 & sîner muomen sun Kaileten\\ 
 & viengen Ponturteise.\\ 
30 & dâ wart vil \textbf{rûch} diu reise.\\ 
\end{tabular}
\scriptsize
\line(1,0){75} \newline
G I O L M Q R Z Fr21 Fr56 \newline
\line(1,0){75} \newline
\textbf{1} \textit{Initiale} O  \newline
\line(1,0){75} \newline
\textbf{1} \textit{Versfolge 74.2-1} R   $\cdot$ der vürste] der G ÷er fvrste O  $\cdot$ Lambikine] [lambi*ine]: lambikine G lamizin I Lamechin O (Fr21) Lammekin L (R) lampikin M lammechine Q lemmekin Z Lammekine Fr56 \textbf{2} tâten dô] do taten I (O) (Q) Fr21 (Fr56) da taten L Z tatin M (R) \textbf{3} si] Die L  $\cdot$ beschutten] besucztin M \textbf{4} helde] heldes L \textbf{5} dô] Da M Z  $\cdot$ stach] sach R  $\cdot$ künic] herre O (Fr21) Fr56  $\cdot$ Arragun] arragon M arragún Q Aragun R arragvͦn Fr21 Fr56 \textbf{6} den alten] der alt I  $\cdot$ Utpandragun] vtrepandagun I vterpandragon M vtpandraguͯn Q Vtpandragvͦn Fr21 Fr56 \textbf{7} die plânîe] deme plane M den plan R \textbf{8} den] der I (Z)  $\cdot$ Britanie] britanige G britanien I Brittanie L britange Q Britigan R britane Z \textbf{9} dô] Da M Z  $\cdot$ stuont] stunden Q  $\cdot$ dâ] do O \textit{om.} Q Z \textbf{10} wê] awe I Wy M \textit{om.} Z  $\cdot$ wie] \textit{om.} Fr56  $\cdot$ doch] \textit{om.} R \textbf{11} Britaneis] [pr*aneis]: pritaneis G Britenais L britaneis M britoneis Q [Ba]: Britanẏs R britvneis Z \textbf{12} Kanvoleis] kanvoleiz G champoneis I canvolais O kanvolais L kanuoleis M kan roleis Q kanvoloẏs R kamfoleis Z \textbf{13} dâ] Do O Q  $\cdot$ getrat] getrat do Q \textbf{14} ez] \textit{om.} O R  $\cdot$ iu] \textit{om.} I L zcu M \textbf{15} noch lîhte nimer] niemen me ez I Noch niemer lýchte L  $\cdot$ dâ] do Q daz Z \textbf{16} er] ern I (O) (L) (Q) (R) (Z)  $\cdot$ dorfte] torste Q  $\cdot$ gesezzen] gesetze M \textbf{17} dem] daz O einem Z  $\cdot$ orse] \textit{om.} Z  $\cdot$ dâ] do Q al da Z  $\cdot$ saz] ê da shaz I ê sas O (L) (Q) (R) (Z) vor uffe sasz M \textbf{18} niht langer] Nit [ne]: lenger R  $\cdot$ man sîn dâ] da man sin M man sein Q \textbf{19} in] Ome M  $\cdot$ beschutten] besutzen M  $\cdot$ umbe in dâ] ob im O ob in da L obir om da M ob ym da Q (Z) da ob im R  $\cdot$ riten] striten O (L) (M) (Q) (R) Z \textbf{20} dâne] da I (O) (L) (R) (Z) Fr56 Do Q  $\cdot$ hurten] [hurdm]: hurdn Q \textbf{21} dô] Da O M Z  $\cdot$ von] \textit{om.} R  $\cdot$ Ponturteis] pontturteis I pvntvrtaýs L Punterteis M punturteis Q (Z) puͦntuͦrteis R pvntorteis Fr56 \textbf{22} der] da I  $\cdot$ dâ] hie O (L) (M) (Q) (R) (Z) Fr56  $\cdot$ Kanvoleis] kanvoleiz G chanpoleis I canvoleis O kanvolaiz L kanuoleis M kanúoleis Q Ranuoleis R kamfoleis Z \textbf{23} sînes orses] sein rosz Q sin Roses R \textbf{24} hinder] nider I  $\cdot$ lac al dâ] da gelag R \textbf{25} Gahmuret] Gamvret O Gahmuͯret L gamuret M gaműuret Q gahmeret R gamurete Z \textbf{26} wetâ hêrre] Wetta herra O Wetten herre Q \textbf{27} strîte] streiten Q  $\cdot$ wurden] werden M \textbf{28} Kaileten] Gahileten I kayleten O L Q Fr56 kayletten R gaileten Z \textbf{29} viengen] Den viengen O L (Q) (Z) (Fr56) Den viengen dy von M Die viengen R  $\cdot$ Ponturteise] punturteise I (Z) (Fr56) [pv*]: pvnttvrteise  O pvmtvrtaise L punterteise M putureisen Q puͦnpuͦrteise R \textbf{30} dâ] do I (Q) (R)  $\cdot$ rûch] ruche Q  $\cdot$ reise] weise Q \newline
\end{minipage}
\hspace{0.5cm}
\begin{minipage}[t]{0.5\linewidth}
\small
\begin{center}*T (U)
\end{center}
\begin{tabular}{rl}
 & der vürste, \textbf{der} hiez Lamekin.\\ 
 & waz \textbf{tâten dô} die sîn?\\ 
 & \textbf{si} beschutten in mit swerten.\\ 
 & die helde strîtes gerten.\\ 
5 & dô stach der künec von Arragun\\ 
 & den alten Utpandragun,\\ 
 & \hspace*{-.7em}\big| den künec von Britanie,\\ 
 & \hspace*{-.7em}\big| hinder\textit{z} ors ûf d\textit{ie} plânîe.\\ 
 & \textbf{dô} \textbf{stuonden} d\textit{â} \textbf{vil bluomen} umb in.\\ 
10 & wie gevüege ich doch bin,\\ 
 & daz ich den werden Brituneis\\ 
 & sô schône lege vor Kanvoleis,\\ 
 & dâ nie getrat vilânes vuoz\\ 
 & - ob ich \textbf{der wârheit} \textbf{jehen} muoz -\\ 
15 & noch lîht niemer dâ geschiht.\\ 
 & er\textbf{n} dorfte sîn \textbf{gesezzen} niht\\ 
 & ûf dem orse, \textbf{dâ} er \textbf{ûffe} saz.\\ 
 & niht langer man sîn dâ vergaz;\\ 
 & in beschutten, die \textbf{ob im} dâ \textbf{striten}.\\ 
20 & dâ\textbf{n} wart grôz \textbf{hurten} niht vermiten.\\ 
 & dô kam der künec von Punterteis.\\ 
 & \textbf{dô} wart al \textbf{dâ} vor Kanvoleis\\ 
 & gevellet ûffe sînes orses slâ,\\ 
 & daz er dâr hinder lac al dâ.\\ 
25 & daz tet der stolze Gahmuret.\\ 
 & wettâ, hêrre, wetâ wet!\\ 
 & mit strîte \textbf{wurden} si geweten.\\ 
 & sîner muomen sun Kayleten,\\ 
 & \textbf{den} viengen Punterteise.\\ 
30 & dô wart vil \textbf{rûch} diu reise.\\ 
\end{tabular}
\scriptsize
\line(1,0){75} \newline
U V W T \newline
\line(1,0){75} \newline
\textbf{5} \textit{Initiale} T  \textbf{21} \textit{Initiale} V  \newline
\line(1,0){75} \newline
\textbf{1} der hiez] hieß W (T)  $\cdot$ Lamekin] Lamikin V lamechin W \textbf{2} sîn] gardie sein W \textbf{3} beschutten] bestunden W behvͦten T \textbf{5} Arragun] arogon U Arragv̂n T \textbf{6} Utpandragun] vtpandraguͦn U Vterpandragun V (W) \textbf{8} \textit{Versfolge 74.7-8} W T   $\cdot$ den] Der W  $\cdot$ Britanie] Pritanie V \textbf{7} hinderz] Hinder U  $\cdot$ die] den U \textbf{9} stuonden] stunt V (T)  $\cdot$ dâ vil bluomen] do vil bluͦmen U da bluͦmen vil V auch vil bluͦmen W [bl*]: blvͦmen vil T \textbf{10} doch] do T \textbf{11} den] dem W  $\cdot$ Brituneis] Prittonoẏs V britaneis W \textbf{12} lege] lag W  $\cdot$ Kanvoleis] kanuoloẏs V kanuoleis W \textbf{13} dâ] Do V W  $\cdot$ vilânes] filians W \textbf{14} ich der wârheit jehen] ichs eúchs von recht sagen W \textbf{15} dâ] me W \textit{om.} T \textbf{16} ern] Er W  $\cdot$ dorfte] doͤrffte W \textbf{17} \textit{Die Verse 74.17-30 fehlen} T   $\cdot$ dâ] do V W  $\cdot$ ûffe saz] [*]: e sas V ee auff saß W \textbf{18} man sîn] sein man W  $\cdot$ dâ] do V (W) \textbf{19} beschutten] besuͦchten W  $\cdot$ dâ] do V \textit{om.} W \textbf{20} dân] do V (W)  $\cdot$ wart] enwart V \textbf{21} Punterteis] Puͦntercos U Puntertoẏs V ponterteis W \textbf{22} dô] der V (W)  $\cdot$ Kanvoleis] kanvolos U Kanuoloẏs V kanuoleis W \textbf{25} Gahmuret] Gahmuͦret U Gamuret V (W) \textbf{26} wettâ] Werta W  $\cdot$ wetâ] werta W \textbf{27} geweten] wetten V gebeten W \textbf{28} Kayleten] kyleten U kaẏletten V gayleten W \textbf{29} Punterteise] Puͦnterteyse U Ponterteise V ponturceise W \newline
\end{minipage}
\end{table}
\end{document}
