\documentclass[8pt,a4paper,notitlepage]{article}
\usepackage{fullpage}
\usepackage{ulem}
\usepackage{xltxtra}
\usepackage{datetime}
\renewcommand{\dateseparator}{.}
\dmyyyydate
\usepackage{fancyhdr}
\usepackage{ifthen}
\pagestyle{fancy}
\fancyhf{}
\renewcommand{\headrulewidth}{0pt}
\fancyfoot[L]{\ifthenelse{\value{page}=1}{\today, \currenttime{} Uhr}{}}
\begin{document}
\begin{table}[ht]
\begin{minipage}[t]{0.5\linewidth}
\small
\begin{center}*D
\end{center}
\begin{tabular}{rl}
\textbf{713} & \textit{\begin{large}D\end{large}}ô erkante \textbf{wol} vrou Bene\\ 
 & \textbf{dise} knappen zwêne,\\ 
 & des \textbf{künec} Gramoflanzes kint,\\ 
 & die nâch Artuse komen sint.\\ 
5 & Si sprach: "hie \textbf{solte} niemen stên.\\ 
 & \textbf{welt} ir, \textbf{ich heize} \textbf{vürder} gên\\ 
 & daz volc ûzen snüeren.\\ 
 & wil mîne vrouwen rüeren\\ 
 & solch ungenâde umb ir trût,\\ 
10 & daz mære kumt schiere über lût."\\ 
 & Vrou Bene her ûz wart gesant.\\ 
 & der kinde einez \textbf{in} ir hant\\ 
 & \textbf{smucte} den brief untz vingerlîn.\\ 
 & si heten ouch den hôhen pîn\\ 
15 & von \textbf{ir vrouwen} wol vernomen\\ 
 & und jâhen, \textbf{des} si wæren komen,\\ 
 & und wolten Artusen \textbf{sprechen},\\ 
 & ob si daz \textbf{ruohte} zechen.\\ 
 & Si sprach: "stêt verre dort hin dan,\\ 
20 & unz ich iuch \textbf{gêns zuo mir} \textbf{man}."\\ 
 & \textbf{Von} Benen, der süezen magt,\\ 
 & ime gezelde wart gesagt,\\ 
 & \textbf{daz} Gramoflanzes boten dâ\\ 
 & \textbf{wæren} und vrâgten, wâ\\ 
25 & Artus, der künec, wære.\\ 
 & "daz dûhte mich ungebære,\\ 
 & ob ich in \textbf{zeigete} an \textbf{diz} gespræche.\\ 
 & seht \textbf{denne}, waz ich ræche\\ 
 & an mîner vrouwen, ob \textbf{si} sie\\ 
30 & \textbf{alsus sæhen} weinen hie."\\ 
\end{tabular}
\scriptsize
\line(1,0){75} \newline
D \newline
\line(1,0){75} \newline
\textbf{1} \textit{Initiale} D  \textbf{5} \textit{Majuskel} D  \textbf{11} \textit{Majuskel} D  \textbf{19} \textit{Majuskel} D  \textbf{21} \textit{Majuskel} D  \newline
\line(1,0){75} \newline
\textbf{1} Dô] ÷o D \textbf{3} Gramoflanzes] Gramoflanzs D \textbf{23} Gramoflanzes] Gramoflanzs D \newline
\end{minipage}
\hspace{0.5cm}
\begin{minipage}[t]{0.5\linewidth}
\small
\begin{center}*m
\end{center}
\begin{tabular}{rl}
 & dô erkante vrowe Bene\\ 
 & \textbf{diser} knappen zwêne,\\ 
 & des \textbf{künic} Gramolantzes kint,\\ 
 & die nâch Artuse komen sint.\\ 
5 & si sprach: "hie \textbf{solte} niemen stên.\\ 
 & \textbf{wolt} ir, \textbf{ich heize} \textbf{vürbaz} gên\\ 
 & daz volc ûz den snüeren.\\ 
 & wil mîn vrowe rüeren\\ 
 & solich ungen\textit{âde} umb \textit{ir} t\textit{r}ût,\\ 
10 & daz mære kumt schier über lût."\\ 
 & vrowe Bene her ûz wart gesant.\\ 
 & der kinde einez \textbf{an} ir hant\\ 
 & \textbf{schoup} den brief und daz vingerlîn.\\ 
 & si hetten ouch die hôhe pîn\\ 
15 & von \textbf{Ithonien} wol vernomen\\ 
 & und jâhen, \textbf{daz} si \dag wære\dag  komen,\\ 
 & und wolten Artusen \textbf{sprechen},\\ 
 & ob si daz \textbf{ruohte} zechen.\\ 
 & si sprach: "stât verre dort hin dan,\\ 
20 & unz ich iuch \textbf{gâns zuo mir} \textbf{man}."\\ 
 & \textbf{von} Benen, der süezen maget,\\ 
 & im gezelt wart gesaget,\\ 
 & Gramolantzes boten \textbf{wâren} dâ\\ 
 & und vrâgten, wâ\\ 
25 & Artus, der künic, wære.\\ 
 & "daz dûhte mich ungebære,\\ 
 & ob ich in \textbf{zougte} an \textbf{diz} gespræche.\\ 
 & seht \textbf{dan}, waz ich ræche\\ 
 & an mîner vrowen, ob sie\\ 
30 & \textbf{si} \textbf{s\textit{e}hent alsus} weinen hie."\\ 
\end{tabular}
\scriptsize
\line(1,0){75} \newline
m n o \newline
\line(1,0){75} \newline
\newline
\line(1,0){75} \newline
\textbf{1} erkante] erkankante o \textbf{3} künic] kv́niges n (o)  $\cdot$ Gramolantzes] gramolanczes o \textbf{6} vürbaz] fúrter n (o) \textbf{9} ungenâde] vngenig m  $\cdot$ ir trût] tuͯt m \textbf{15} Ithonien] jthonien o \textbf{18} ruohte] rechte o \textbf{22} wart] hett wart o \textbf{23} Gramolantzes] [Ga]: Gramolantz m Gramonlantz n Gramolancz o  $\cdot$ dâ] do n \textbf{24} wâ] ouch wo n \textbf{27} zougte] zeigete n \textbf{28} dan] das o  $\cdot$ ræche] rechte o \textbf{30} sehent] sohent m  $\cdot$ weinen] \textit{om.} n \newline
\end{minipage}
\end{table}
\newpage
\begin{table}[ht]
\begin{minipage}[t]{0.5\linewidth}
\small
\begin{center}*G
\end{center}
\begin{tabular}{rl}
 & \begin{large}D\end{large}ô erkande \textbf{wol} vrou Bene\\ 
 & \textbf{dise} knappen \textit{zw}ê\textit{n}e,\\ 
 & des \textbf{künic} Gramoflanzes kint,\\ 
 & die nâch Artus komen sint.\\ 
5 & si sprach: "hie \textbf{sol} niemen stân,\\ 
 & \textbf{welt} ir \textbf{heizen} \textbf{vürder} gân\\ 
 & daz volc ûz den snüeren.\\ 
 & wil mîn vrouwe rüeren\\ 
 & \textit{s}olc\textit{h} ungnâde umbe ir trût,\\ 
10 & daz mære kumt schiere über lût."\\ 
 & vrou Bene her ûz wart gesant.\\ 
 & der kinde einez \textbf{in} ir hant\\ 
 & \textbf{stiez} den brief unde daz vingerlîn.\\ 
 & si heten ouch den hôhen pîn\\ 
15 & von \textbf{ir vrouwen} wol vernomen\\ 
 & unde jâhen, \textbf{des} si wæren komen,\\ 
 & unde wolden Artusen \textbf{sprechen},\\ 
 & op si daz \textbf{ruohte} zechen.\\ 
 & si sprach: "stêt verre dort hin dan,\\ 
20 & unze ich iuch \textbf{gêns zuo mir} \textbf{man}."\\ 
 & \textbf{vroun} Benen, der süezen maget,\\ 
 & in dem gezelte wart gesaget,\\ 
 & Gramoflanzes boten dâ\\ 
 & \textbf{wæren} unde vrâgten, wâ\\ 
25 & Artus, der künic, wære.\\ 
 & "daz dûhte mich ungebære,\\ 
 & ob ich i\textit{n} \textbf{zeigte} an \textbf{ditze} gespræche.\\ 
 & seht \textbf{dane}, waz ich ræche\\ 
 & an mîner vrouwen, ob \textbf{si} sie\\ 
30 & \textbf{sehent alsus} weinen hie."\\ 
\end{tabular}
\scriptsize
\line(1,0){75} \newline
G I L M Z Fr18 Fr22 \newline
\line(1,0){75} \newline
\textbf{1} \textit{Initiale} G I L Z  \newline
\line(1,0){75} \newline
\textbf{1} Dô] Da M \textbf{2} zwêne] bede G \textbf{3} künic] chunges I (Fr18)  $\cdot$ Gramoflanzes] gramorflanzes M gramoflantzes Z (Fr18) \textbf{4} Artus] Artuse I L (M) (Fr18) \textbf{5} sprach] [sprahen]: sprachen I  $\cdot$ sol] solte L (M) (Z) (Fr18) \textbf{6} welt] Mvget Z  $\cdot$ heizen] heiszet L (Fr18) \textbf{8} mîn vrouwe] myne vrouwen M (Fr18) \textbf{9} solch] [sc*]: scholc G \textbf{10} schiere] sere M \textbf{11} her ûz wart] wart her vz L \textbf{14} heten] het I \textbf{16} jâhen] sprachen M  $\cdot$ wæren] [*]: waren L \textbf{17} \textit{Versfolge 713.18-17} L   $\cdot$ unde] \textit{om.} L  $\cdot$ Artusen] artus Z  $\cdot$ sprechen] gesprechen I Fr18 \textbf{18} ruohte] geruchte M \textbf{19} verre dort] dort verre I \textbf{20} Vntz ich [z]: evch zv mir gendes gan Z  $\cdot$ unze] Bisz M  $\cdot$ gêns zuo mir] zuͤ Gensh I zuͦ mir gens L (M) (Fr18) \textbf{21} vroun] Von L Vrouwe M  $\cdot$ Benen] bene M  $\cdot$ der] von der M Fr18 (Fr22) \textbf{23} Gramoflanzes] Das gramorflanzes M Gramoflantzes Z Daz Gramoflantzes Fr18 \textbf{24} wæren unde] \textit{om.} I Waren vnd L (Z) \textbf{26} dûhte] duhten I \textbf{27} in zeigte] iv zeigte G zeigte in Fr18  $\cdot$ ditze] daz I (M) Fr18 \textbf{29} si sie] sie I \textbf{30} sehent] sehe I Sahen L  $\cdot$ weinen] weinende Z Fr18 \newline
\end{minipage}
\hspace{0.5cm}
\begin{minipage}[t]{0.5\linewidth}
\small
\begin{center}*T
\end{center}
\begin{tabular}{rl}
 & \begin{large}D\end{large}ô erkante \textbf{wol} vrou Bene\\ 
 & \textbf{dise} knappen zwêne,\\ 
 & des \textbf{küneges} Gramoflanzes kint,\\ 
 & die nâch Artuse komen sint.\\ 
5 & si sprach: "hie \textbf{sol} nieman stên,\\ 
 & \textbf{moget} ir \textbf{heizen} \textbf{vürder} gên\\ 
 & daz volc ûz den snüeren.\\ 
 & wil mîne vrouwen rüeren\\ 
 & soliche ungenâde umb ir trût,\\ 
10 & diu mære kumet schiere über lût."\\ 
 & vrou Bene her ûz wart gesant.\\ 
 & der kinde einez \textbf{in} ir hant\\ 
 & \textbf{stiez} den brief und daz vingerlîn.\\ 
 & si heten ouch die hôhen pîn\\ 
15 & von \textbf{ir vrouwen} wol vernomen\\ 
 & und jâhen, \textbf{daz} si wæren komen,\\ 
 & und wolten Artusen \textbf{gesprechen},\\ 
 & o\textit{b} si daz \textbf{wolte}\textbf{n} zechen.\\ 
 & si sprach: "stêt verre dort hin dan,\\ 
20 & \textit{unz} ich iuch \textbf{zuo mir gênnes} \textbf{gan}."\\ 
 & \textbf{von} Benen, der süezen maget,\\ 
 & in dem gezelte wart gesaget,\\ 
 & \textbf{daz} Gramoflanzes boten dâ\\ 
 & \textbf{wæren} und vrâgeten, wâ\\ 
25 & Artus, der künec, wære.\\ 
 & "daz dûhte mich ungebære,\\ 
 & ob ich in \textbf{zeigete} an \textbf{daz} gespræche.\\ 
 & sehet, waz ich \textbf{dan} ræche\\ 
 & an mîner vrouwen, ob \textbf{si} sie\\ 
30 & \textbf{sæhen alsus} weinen hie."\\ 
\end{tabular}
\scriptsize
\line(1,0){75} \newline
U V W Q R \newline
\line(1,0){75} \newline
\textbf{1} \textit{Initiale} U V W Q  \newline
\line(1,0){75} \newline
\textbf{3} Gramoflanzes] Gramaflanzes V gramoflantzes W Q Gramoflanczes R \textbf{4} Artuse] Artus R \textbf{5} sprach] [*]: sprach V sprachen Q  $\cdot$ sol] [*]: sol V solt Q \textbf{6} [*]: Mv́get ir heissen fúrder gen V  $\cdot$ vürder] fúrbas R \textbf{7} den] \textit{om.} W \textbf{8} vrouwen] frawe W (Q) (R) \textbf{10} diu] Daz V (W) (Q) (R)  $\cdot$ kumet] kunt R \textbf{11} her ûz] vs her V her vnsz Q \textbf{14} die] den V Q R \textbf{15} ir vrouwen] [*]: itonien V \textbf{17} gesprechen] besprechen W (Q) sprechen R \textbf{18} ob] Oder U  $\cdot$ daz] des R  $\cdot$ wolten] [*]: rvͦchte V ruͦchten W (R) ruchte Q \textbf{19} verre dort] [*]: verre dort V \textbf{20} unz] Mit U  $\cdot$ zuo mir gênnes gan] zvͦ mir gendes [man]: mane V genesz zu mir man Q zu mir heis gan R \textbf{21} von] [V*]: Von V \textbf{22} gezelte] zeltlin R \textbf{23} Gramoflanzes] Gramaflanzes V gramoflantzes W gramaflantzes Q Gramoflanczes R  $\cdot$ dâ] do W Q \textbf{27} zeigete] geriete W  $\cdot$ daz gespræche] des gesproche Q \textbf{30} [Sehe *]: Sehen alsus weinen hie V  $\cdot$ sæhen] \textit{om.} R \newline
\end{minipage}
\end{table}
\end{document}
