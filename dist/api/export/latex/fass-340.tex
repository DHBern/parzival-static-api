\documentclass[8pt,a4paper,notitlepage]{article}
\usepackage{fullpage}
\usepackage{ulem}
\usepackage{xltxtra}
\usepackage{datetime}
\renewcommand{\dateseparator}{.}
\dmyyyydate
\usepackage{fancyhdr}
\usepackage{ifthen}
\pagestyle{fancy}
\fancyhf{}
\renewcommand{\headrulewidth}{0pt}
\fancyfoot[L]{\ifthenelse{\value{page}=1}{\today, \currenttime{} Uhr}{}}
\begin{document}
\begin{table}[ht]
\begin{minipage}[t]{0.5\linewidth}
\small
\begin{center}*D
\end{center}
\begin{tabular}{rl}
\textbf{340} & \begin{large}E\end{large}z was von Munsalvæsche komen\\ 
 & unt het\textbf{z} Læhelin genomen\\ 
 & ze Brumbane bîme sê.\\ 
 & eime rîter tet sîn tjost wê,\\ 
5 & den er tôt \textbf{dar} hinder stach,\\ 
 & des \textbf{sider} Trevrizent verjach.\\ 
 & Gawan dâhte: "swer verzagt,\\ 
 & \textbf{sô} daz er vliuht, ê man jagt,\\ 
 & daz ist sîme prîse gar ze vruo.\\ 
10 & ich \textbf{wil} \textbf{in} nâher stapfen zuo,\\ 
 & swaz mir \textbf{dâ von nû mac} geschehen,\\ 
 & ir hât michz mêre teil gesehen.\\ 
 & \textbf{des} sol doch guot rât werden."\\ 
 & dô erbeizt er \textbf{ze der} erden,\\ 
15 & reht als er habete einen stal.\\ 
 & die rotte wâren âne zal,\\ 
 & die dâ mit kumpânîe riten.\\ 
 & er sach vil kleider wol gesniten\\ 
 & unt manegen schilt \textbf{wol} gevar,\\ 
20 & daz er niht \textbf{bekande} gar\\ 
 & noch deheine baniere under in.\\ 
 & "disem her ein gast ich bin",\\ 
 & \textbf{sus} sprach der werde Gawan.\\ 
 & "sît ich \textbf{ir} deheine künde hân,\\ 
25 & wellent siz in übel wenden,\\ 
 & eine tjost sol ich in senden,\\ 
 & deiswâr, mit mîn selbes hant,\\ 
 & ê daz ich von in sî gewant."\\ 
 & \textbf{dô} was ouch Gringuljeten gegurt,\\ 
30 & daz in manegen angestlîchen vurt\\ 
\end{tabular}
\scriptsize
\line(1,0){75} \newline
D \newline
\line(1,0){75} \newline
\textbf{1} \textit{Initiale} D  \newline
\line(1,0){75} \newline
\textbf{1} Munsalvæsche] Mvntsalvasce D \textbf{29} Gringuljeten] Gringvlieten D \newline
\end{minipage}
\hspace{0.5cm}
\begin{minipage}[t]{0.5\linewidth}
\small
\begin{center}*m
\end{center}
\begin{tabular}{rl}
 & ez was von Mun\textit{t}salvasche komen\\ 
 & und hete Lehelin genomen\\ 
 & ze Bru\textit{n}ban\textit{e} bî dem sê.\\ 
 & einem ritter tet sîn juste wê,\\ 
5 & den er tôt \textbf{her} hinder stach,\\ 
 & des \textbf{sider} Trevrizent verjach.\\ 
 & Gawan dâhte: "wer \textbf{sô} verzaget,\\ 
 & daz er vliuhet, ê man \textbf{in} \textit{j}aget,\\ 
 & daz ist sînem prîse gar zuo vruo.\\ 
10 & ich \textbf{wil} \textbf{in} nâher s\textit{t}a\textit{p}fen zuo,\\ 
 & waz mir \textbf{dâ von nû mac} geschehen,\\ 
 & ir hât mich daz mêre teil gesehen.\\ 
 & \textbf{des} sol doch guot rât werden."\\ 
 & dô erbeizeter \textbf{zuo der} erden,\\ 
15 & rehte als er habete ein stal.\\ 
 & die rote wâren âne zal,\\ 
 & die dâ \textbf{nâch} mit k\textit{o}mpânîe riten.\\ 
 & er sach vil kleider wol gesniten\\ 
 & und manigen schilt \textbf{sô} gevar,\\ 
20 & daz er niht \textbf{bekante} gar\\ 
 & noch dekeine banier under in.\\ 
 & "disem her ein gast ich bin",\\ 
 & \textbf{sus} sprach der werde Gawan.\\ 
 & "sît ich\textbf{s} enkeine \textit{k}ü\textit{n}de hân,\\ 
25 & wellent siz in übel wenden,\\ 
 & eine just sol ich in senden,\\ 
 & deiswâr, mit mîn selbes hant,\\ 
 & ê daz ich von in sî gewant."\\ 
 & \textbf{dô} was ouch Gringulete gegurt,\\ 
30 & daz in manigen angestlîchen \textit{v}urt\\ 
\end{tabular}
\scriptsize
\line(1,0){75} \newline
m n o \newline
\line(1,0){75} \newline
\newline
\line(1,0){75} \newline
\textbf{1} Muntsalvasche] munsaluasce m montsaluasce n mont saluasce o \textbf{3} ze Brunbane] Zebrvngbang m Zuͯ brunibang n Zuͦ brunbang o \textbf{4} einem] Gnen o \textbf{5} her] dar n o \textbf{6} Trevrizent] treirierzen n trieriezens o \textbf{7} dâhte] gedochte n (o) \textbf{8} jaget] saget m \textbf{9} ist] wer n o \textbf{10} stapfen] schaffen m \textbf{11} dâ] [gegen]: do o  $\cdot$ nû] \textit{om.} n o  $\cdot$ geschehen] beschehen n o \textbf{12} daz] des n \textbf{14} erbeizeter] erbeisset er n (o) \textbf{15} habete] habt o  $\cdot$ ein] einen n (o) \textbf{16} die] Do o  $\cdot$ wâren] was n (o) \textbf{17} dâ nâch] do n o  $\cdot$ kompânîe] campanie m Capanie o \textbf{20} niht] ir nit n o \textbf{21} dekeine] do kein n \textbf{23} werde] wede o \textbf{24} enkeine] nuͯ kein n nie keyn o  $\cdot$ künde hân] hulde [lan]: han m \textbf{25} in] nit n o \textbf{27} mîn] mẏnns o \textbf{29} Gringulete] gringulet n gruͯn gullat o  $\cdot$ gegurt] begúrte n (o) \textbf{30} vurt] wurt m \newline
\end{minipage}
\end{table}
\newpage
\begin{table}[ht]
\begin{minipage}[t]{0.5\linewidth}
\small
\begin{center}*G
\end{center}
\begin{tabular}{rl}
 & ez was von Muntsalvatsche komen\\ 
 & unt hete Lehelin genomen\\ 
 & ze Brunbanie bî dem sê.\\ 
 & einem rîter tet sîn tjos\textit{t w}ê,\\ 
5 & den er tôt \textbf{dar} hinder stach,\\ 
 & des \textbf{sider} Trevrizzent verjach.\\ 
 & Gawan dâhte: "swer verzaget,\\ 
 & \textbf{sô} daz er vliuhet, ê man jaget,\\ 
 & dês sînem brîse gar ze vruo.\\ 
10 & ich \textbf{sol} \textbf{hin} nâher stapfen zuo,\\ 
 & swaz mir \textbf{dâ von nû mac} geschehen,\\ 
 & ir hât mich daz mêre teil gesehen.\\ 
 & \textbf{de\textit{s}} sol doch guot rât werden."\\ 
 & dô erbeizter \textbf{ûf die} erden,\\ 
15 & reht alser hete einen stal.\\ 
 & die rote wâren âne zal,\\ 
 & die dâ mit kompânîe riten.\\ 
 & er sach vil kleider wol gesniten\\ 
 & unde manigen schilt \textbf{sô} gevar,\\ 
20 & daz er \textbf{ir} niht \textbf{erkande} gar\\ 
 & \textit{noch} \textit{d}ehein baniere under in.\\ 
 & "disem her ein gast ich bin",\\ 
 & sprach der werde Gawan.\\ 
 & "sît ich \textbf{ir} deheine künde hân,\\ 
25 & wellent siz in übel wenden,\\ 
 & eine tjost sol ich in senden,\\ 
 & dêswâr, mit mîn selbes hant,\\ 
 & ê daz ich von in sî gewant."\\ 
 & \textbf{nû} was ouch Gringuliet gegurt,\\ 
30 & daz in manigen angestlîchen vurt\\ 
\end{tabular}
\scriptsize
\line(1,0){75} \newline
G I O L M Q R Z Fr22 Fr39 Fr40 \newline
\line(1,0){75} \newline
\textbf{1} \textit{Initiale} O L Z Fr39   $\cdot$ \textit{Capitulumzeichen} R  \textbf{23} \textit{Initiale} M  \newline
\line(1,0){75} \newline
\textbf{1} ez] ÷z O Er M  $\cdot$ Muntsalvatsche] muntshaluasch I mvntschalvatsche O Muntsalfatsche M múntsalvasche Q Munsaluashe R monsalvatsche Z mvntsaluatsche Fr39 munsalvashe Fr40 \textbf{2} hete] het ez I (R) o\textit{m. } O  $\cdot$ Lehelin] Lechelin R lehelein Fr40 \textbf{3} ze Brunbanie] zebrunbanie G ze brumbame I Zebrvmbale O Zuͯ brvmbanie L (M) (Z) (Fr39) Zu brubange Q Ze Brumgange R zebrunbange Fr40  $\cdot$ dem] den R \textbf{4} tet] dem Q dem tet R  $\cdot$ tjost wê] tiost so we G \textbf{5} den] Tet den Q  $\cdot$ er] \textit{om.} I  $\cdot$ hinder] nidar L (Fr22) (Fr39)  $\cdot$ stach] ab stach I \textbf{6} sider] sit I (L) M Fr39  $\cdot$ Trevrizzent] trevrezent G treuerezente I trevezzent O Trevriszent L treffrezcent M trefrizzent Q Trefrissent R Trevrrizzen Z Trefrîzzent Fr22 trefrizent Fr40 \textbf{7} Gawan] Gawann Q  $\cdot$ dâhte] gidachte M  $\cdot$ swer] wer L M Q R Z \textbf{8} E das man Jn Iagt R  $\cdot$ sô daz er] So er O Do das er Q Daz er Fr22  $\cdot$ ê man] e man in O (M) (Q) Z ê daz man in Fr22 \textbf{10} hin] in I Z îv Fr22 \textbf{11} \textit{Versfolge 340.12-11} I   $\cdot$ swaz] Waz L (M) (Q) (R)  $\cdot$ dâ von nû] nv da O da von Jo R da von Fr22 \textbf{12} ich han lute ovch ê gesehen I  $\cdot$ Sie habent michz mir teil gesehen L  $\cdot$ gesehen] gesesen Q \textbf{13} des] de G  $\cdot$ sol doch] sol I sol nv L doch sol R \textbf{14} dô] Da M  $\cdot$ erbeizter] erbaizt er I (O) (L) (Q) (Z) (Fr40)  $\cdot$ die] der Z \textbf{15} alser] als Z \textbf{16} rote] rotten I  $\cdot$ âne] alle M \textbf{17} dâ] do Q R Fr39  $\cdot$ kompânîe] der [*]: kompanie R \textbf{19} sô] wol I O L M Z Fr39 \textbf{20} er ir] erre Fr22 \textbf{21} noch] vnde G  $\cdot$ dehein] neheine G (Fr22) icheiner M deheinen R \textbf{23} werde] herre I \textbf{24} deheine] keinen Z \textbf{25} in] mit I mir R \textbf{26} eine] Einen Q  $\cdot$ sol] wil R \textbf{27} dêswâr] Enzwar Q Zwar Z \textbf{28} daz] \textit{om.} R  $\cdot$ von in sî] si von in I von iv si O von yme sy M \textbf{29} ouch] \textit{om.} O  $\cdot$ Gringuliet] kringuliet G Q (Fr22) kingruliet I kyngrvlet O Gringuͯliet L kingrulet M Z kringulet R gringulget Fr39 \textbf{30} daz] da I  $\cdot$ in] yme M (Z)  $\cdot$ manigen angestlîchen] manc [anges*]: angeslicher I \newline
\end{minipage}
\hspace{0.5cm}
\begin{minipage}[t]{0.5\linewidth}
\small
\begin{center}*T
\end{center}
\begin{tabular}{rl}
 & ez was von Munsalvasche komen\\ 
 & unde het \textbf{ez} Lehelin genomen.\\ 
 & Ze Brumbanie bî dem sê\\ 
 & einem rîter tet sîn tjost wê,\\ 
5 & den er tôt \textbf{dar} hinder stach,\\ 
 & des \textbf{sît} Trefrizent verjach.\\ 
 & \begin{large}G\end{large}awan dâhte: "swer verzaget,\\ 
 & \textbf{sô} daz er vliuhet, ê man \textbf{in} jaget,\\ 
 & daz ist sînem prîse gar ze vruo.\\ 
10 & ich \textbf{sol} \textbf{in} nâher stapfen zuo,\\ 
 & swaz mir \textbf{nû mac dar von} geschehen,\\ 
 & ir hât mich daz mêre teil gesehen.\\ 
 & \textbf{ez} sol doch guot rât werden."\\ 
 & dô erbeizeter \textbf{ûf die} erden,\\ 
15 & rehte alser hete einen stal.\\ 
 & die rotten wâren âne zal,\\ 
 & die dâ mit kumpânîe riten.\\ 
 & er sach vil kleider wol gesniten\\ 
 & unde manegen schilt \textbf{wol} gevar,\\ 
20 & daz er \textbf{ir} niht \textbf{erkante} gar\\ 
 & noch deheine banier under in.\\ 
 & "disem her ein gast ich bin",\\ 
 & sprach der werde Gawan.\\ 
 & "sît ich \textbf{ir} deheine künde hân,\\ 
25 & wellent siz in übel wenden,\\ 
 & eine tjost sol ich in senden,\\ 
 & deist wâr, mit mîn selbes hant,\\ 
 & ê daz ich von in sî gewant."\\ 
 & \textbf{Nû} was ouch Krynguliet gegurt,\\ 
30 & daz in manegen angestlîchen vurt\\ 
\end{tabular}
\scriptsize
\line(1,0){75} \newline
T V W \newline
\line(1,0){75} \newline
\textbf{1} \textit{Initiale} W  \textbf{3} \textit{Majuskel} T  \textbf{7} \textit{Initiale} T  \textbf{21} \textit{Initiale} V  \textbf{29} \textit{Majuskel} T  \newline
\line(1,0){75} \newline
\textbf{1} Munsalvasche] mvnsalvasce T mvntsalvasche V montsaluatz W \textbf{2} Lehelin] lehalein W \textbf{3} Brumbanie] brvnbanie V brunnlyan W \textbf{5} dar hinder] darnider V \textbf{6} des sît] Das seit der W  $\cdot$ Trefrizent] trefizeut V treuerissent W \textbf{7} swer] er wer W \textbf{10} in] hin V \textbf{11} Was do von mag geschehen W  $\cdot$ geschehen] beschehen V \textbf{12} mêre] merer W  $\cdot$ gesehen] ersehen V \textbf{14} ûf die] zvͦ der V \textbf{15} An der stette sundertwal W \textbf{17} dâ] do V W \textbf{18} kleider] [kleiner]: kleider T \textbf{19} wol] so V so wol W \textbf{20} ir] \textit{om.} W \textbf{28} von in sî] sei von in W \textbf{29} Krynguliet] kryngvliet T gringvlet V kringulet W \textbf{30} manegen] maniger V  $\cdot$ vurt] wúrt W \newline
\end{minipage}
\end{table}
\end{document}
