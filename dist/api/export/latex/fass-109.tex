\documentclass[8pt,a4paper,notitlepage]{article}
\usepackage{fullpage}
\usepackage{ulem}
\usepackage{xltxtra}
\usepackage{datetime}
\renewcommand{\dateseparator}{.}
\dmyyyydate
\usepackage{fancyhdr}
\usepackage{ifthen}
\pagestyle{fancy}
\fancyhf{}
\renewcommand{\headrulewidth}{0pt}
\fancyfoot[L]{\ifthenelse{\value{page}=1}{\today, \currenttime{} Uhr}{}}
\begin{document}
\begin{table}[ht]
\begin{minipage}[t]{0.5\linewidth}
\small
\begin{center}*D
\end{center}
\begin{tabular}{rl}
\textbf{109} & \textbf{\begin{large}D\end{large}ie} muosen wol von \textbf{schulden} klagen.\\ 
 & \textbf{diu vrouwe hete} getragen\\ 
 & ein kint, daz in ir lîbe stiez,\\ 
 & die man âne helfe ligen liez.\\ 
5 & ahzehen wochen hete gelebt,\\ 
 & des muoter mit dem tôde strebt,\\ 
 & vrou Herzeloyde, diu künegîn.\\ 
 & die anderen heten kranken sin,\\ 
 & daz si \textbf{niht hulfen} dem wîbe,\\ 
10 & wan \textbf{diu} truoc in ir lîbe,\\ 
 & der aller ritter bluome wirt,\\ 
 & ob in sterben \textbf{hie} verbirt.\\ 
 & Dô kom ein \textbf{altwîser} man\\ 
 & durch \textbf{klage} über die vrouwen \textbf{sân},\\ 
15 & dâ \textbf{diu} mit dem tôde ranc.\\ 
 & die zene er ir von ein ander twanc.\\ 
 & \textbf{man} gôz ir wazzer in den munt.\\ 
 & \textbf{al} dâ wart ir versinnen kunt.\\ 
 & Si sprach: "owê, war kom mîn trût?"\\ 
20 & diu vrouwe \textbf{in klagete} überlût:\\ 
 & "mînes herzen vreude breit\\ 
 & was Gahmuretes werdecheit.\\ 
 & \textbf{daz} nam mir sîn vrechiu ger.\\ 
 & ich was vil jünger dann er\\ 
25 & unt bin sîn muoter und sîn wîp.\\ 
 & ich trage al hie \textbf{doch} sînen lîp\\ 
 & unt sînes verhes sâmen.\\ 
 & den gâben und nâmen\\ 
 & unser zweier minne.\\ 
30 & \textbf{hât} got getriuwe sinne,\\ 
\end{tabular}
\scriptsize
\line(1,0){75} \newline
D Fr33 \newline
\line(1,0){75} \newline
\textbf{1} \textit{Initiale} D  \textbf{13} \textit{Initiale} Fr33   $\cdot$ \textit{Majuskel} D  \textbf{19} \textit{Majuskel} D  \newline
\line(1,0){75} \newline
\textbf{1} Die] Sie Fr33  $\cdot$ von] mit Fr33 \textbf{3} ein] Sin Fr33 \textbf{8} sin] schin Fr33 \textbf{9} niht hulfen] hulfen niht Fr33 \textbf{12} sterben] ein sterben Fr33 \textbf{14} klage] clagen Fr33 \textbf{18} al] \textit{om.} Fr33 \textbf{20} in klagete] claget in Fr33 \textbf{22} Gahmuretes] Gahmvretes D Gamuretes Fr33 \textbf{23} daz] Den Fr33 \textbf{24} was] \textit{om.} Fr33 \textbf{26} trage al hie] tragen Fr33 \textbf{30} hât] habe Fr33 \newline
\end{minipage}
\hspace{0.5cm}
\begin{minipage}[t]{0.5\linewidth}
\small
\begin{center}*m
\end{center}
\begin{tabular}{rl}
 & \textbf{die} muosen wol von \textbf{sünden} klagen.\\ 
 & \textbf{dô hete diu vrouwe} getragen\\ 
 & ein kint, daz in ir lîbe stiez,\\ 
 & die man âne helfe ligen liez.\\ 
5 & ahzehen wochen hete gelebet,\\ 
 & des muoter mit dem tôde strebet,\\ 
 & vrouwe Herczeloid\textit{e}, diu künigîn.\\ 
 & die andern heten kranken sin,\\ 
 & daz si \textbf{niht hulfen} dem wîbe,\\ 
10 & wanne \textbf{diu} truoc in ir lîbe,\\ 
 & der aller ritter bluom\textit{e} \textit{w}irt,\\ 
 & ob in sterben \textbf{hie} verbirt.\\ 
 & dô kam ein \textbf{altwîse} man\\ 
 & durch \textbf{klage} über die vrouwen \textbf{sân},\\ 
15 & d\textit{â} \textbf{si} mit dem tôde \textit{r}anc.\\ 
 & die zene e\textit{r} \textit{i}r von ein ander twanc.\\ 
 & \textbf{man} gôz ir wazzer in den munt.\\ 
 & \textbf{al}dâ wart ir versinnen kunt.\\ 
 & si sprach: "owê, war kam mî\textit{n} trût?"\\ 
20 & diu vrouwe \textbf{klagete in} überlût:\\ 
 & "mînes herzen vröude breit\\ 
 & was Gahmuretes wirdicheit.\\ 
 & \textbf{den} nam mir sîn vrechi\textit{u} ger.\\ 
 & ich was vil jünger denne er\\ 
25 & und bin sîn muoter und sîn wîp.\\ 
 & ich trage alhie \textbf{doch} sînen lîp\\ 
 & und sînes ver\textit{hes} sâmen.\\ 
 & den gâben und nâmen\\ 
 & unser zweier minne.\\ 
30 & \textbf{hât} got getriu\textit{w}e sinne,\\ 
\end{tabular}
\scriptsize
\line(1,0){75} \newline
m n o \newline
\line(1,0){75} \newline
\newline
\line(1,0){75} \newline
\textbf{1} die] Do n  $\cdot$ muosen] muͯsten n o  $\cdot$ klagen] sagen n \textbf{3} lîbe] liebe o \textbf{6} des] Das o \textbf{7} Herczeloide] herczeloiden m hertzeloẏd n herczeleide o \textbf{10} diu] sú n (o) \textbf{11} bluome wirt] bluͦmmen truͦg vnd wirt m bluͦmen wirt n o \textbf{12} hie] sie o \textbf{13} altwîse] alter wiser n o \textbf{14} vrouwen] frouwe n \textbf{15} dâ] Do m n o  $\cdot$ dem] denn o  $\cdot$ ranc] trang m \textbf{16} er ir] er in ir m \textbf{18} versinnen] versinnent o \textbf{19} mîn] mit m \textbf{22} Gahmuretes] gahmurettes m gamiretes n gamúretes o \textbf{23} sîn] mẏn o  $\cdot$ vrechiu] frechi m frechen o \textbf{25} sîn wîp] sine wip n \textbf{26} doch] durch n \textbf{27} verhes sâmen] versomen m werhes somen o \textbf{30} hât] Hette n  $\cdot$ getriuwe] getruge m \newline
\end{minipage}
\end{table}
\newpage
\begin{table}[ht]
\begin{minipage}[t]{0.5\linewidth}
\small
\begin{center}*G
\end{center}
\begin{tabular}{rl}
 & \textbf{si} muosen wol von \textbf{schulden} klagen.\\ 
 & \textbf{diu vrouwe hete} getragen\\ 
 & ein kint, daz in ir lîbe stiez,\\ 
 & die man ân helfe ligen liez.\\ 
5 & ahzehen wochen hete gelebet,\\ 
 & des muoter mit dem tôde strebet,\\ 
 & vrô Herzeloide, diu künigîn.\\ 
 & die anderen heten kranken sin,\\ 
 & daz si \textbf{hulfen niht} dem wîbe,\\ 
10 & wan \textbf{si} truoc in ir lîbe,\\ 
 & der aller rîter bluome wirt,\\ 
 & obe in \textbf{ein} sterben \textbf{hie} verbirt.\\ 
 & dô kom ein \textbf{altwîse} man\\ 
 & durch \textbf{klage} über die vrouwen \textbf{gegân},\\ 
15 & \textbf{al} dâ \textbf{si} mit dem tôde ranc.\\ 
 & die zene er ir von \textit{ein} ander twanc.\\ 
 & \textbf{man} gôz ir wazzer in den munt.\\ 
 & dô wart ir versinnen kunt.\\ 
 & si sprach: "owê, war kom mîn trût?"\\ 
20 & diu vrouwe \textbf{klagte in} überlût:\\ 
 & "mînes herzen vröude breit\\ 
 & was Gahmu\textit{r}etes werdicheit.\\ 
 & \textbf{den} nam mir sîn vrechiu ger.\\ 
 & ich was vil jünger dane er\\ 
25 & \begin{large}U\end{large}nd bin sîn muoter und sîn wîp.\\ 
 & ich trage al hie \textbf{doch} sînen lîp\\ 
 & und sînes verhes sâmen.\\ 
 & den gâben und nâmen\\ 
 & unser zweier minne.\\ 
30 & \textbf{habe} got getriwe sinne,\\ 
\end{tabular}
\scriptsize
\line(1,0){75} \newline
G I O L M Q R Z \newline
\line(1,0){75} \newline
\textbf{1} \textit{Initiale} I O L Q R Z  \textbf{19} \textit{Initiale} I M  \textbf{25} \textit{Initiale} G  \textbf{28} \textit{Initiale} L  \newline
\line(1,0){75} \newline
\textbf{1} si] ÷i O  $\cdot$ muosen] muͯsten in L \textbf{4} ân] an alle O \textbf{5} daz chint ahtzehen wochen in ir het gelept I  $\cdot$ hete] hat M \textbf{6} mit] hie mit I  $\cdot$ strebet] strep I \textbf{7} vrô] \textit{om.} O Q R  $\cdot$ Herzeloide] herzenlaude I Herzelavde O Hertzelauͯde L herczeloide M Herzeloude Q Herczelaude R herzelovde Z \textbf{8} heten] hette L \textbf{9} si hulfen niht] si nih hulfen I (L) (Q) hvlfen niht O \textbf{11} bluome] ein bluͤme I blumen Q \textbf{12} hie] da O L M Q R Z \textbf{13} dô] Da M Z  $\cdot$ altwîse] alde wise M \textbf{14} durch] Doch Q  $\cdot$ klage] chlagen O (L) (M) (Q) (R) (Z)  $\cdot$ die] sin I  $\cdot$ gegân] gan G san O L M Q Z \textbf{15} dem] den R \textbf{16} von ein ander] von ander G vf I von ein andren R  $\cdot$ twanc] dranch O \textbf{17} in den munt] \textit{om.} \textit{nachträglich hinzugefügt:} in den mvnt Z \textbf{18} dô] Da M R Z  $\cdot$ versinnen] versumen I \textbf{19} owê] awe I O \textbf{20} klagte in] chlagt in I (Q) (R) (Z) in chlagte O \textbf{21} herzen] herren L  $\cdot$ vröude] frewden Q  $\cdot$ breit] bereit R \textbf{22} Gahmuretes] gahmvetes G Gamvretes O Gahmuͯretes L gamuretis M gamúretes Q gamuretes Z \textbf{23} mir sîn] [si]: mir sin O sie sein Q [s]: mir sin R  $\cdot$ vrechiu] vrecher O (R) [hercze]: freche  M frewde Q \textbf{24} vil jünger] juͯnger vil L  $\cdot$ dane] wan M \textbf{25} Und bin] vnd I Jch bin L Q  $\cdot$ und] nit R \textbf{26} ich] vnd I  $\cdot$ al hie doch] hie al R \textbf{27} verhes] herzen I \textit{om.} M werbes R [*verkes]: verhes Z \textbf{30} habe] hat I \newline
\end{minipage}
\hspace{0.5cm}
\begin{minipage}[t]{0.5\linewidth}
\small
\begin{center}*T (U)
\end{center}
\begin{tabular}{rl}
 & \textbf{\begin{large}S\end{large}i} muosen \textbf{in} wol von \textbf{schulden} \textit{klagen},\\ 
 & \textbf{wan} \textbf{diu vrouwe hete} getragen\\ 
 & ein kint, daz in ir lîbe stiez,\\ 
 & die man âne helfe ligen liez.\\ 
5 & ahzehen wochen hete\textbf{z} gelebet,\\ 
 & des muoter mit dem tôde strebet,\\ 
 & vrô Herzeloyde, diu künegîn.\\ 
 & die andern heten kranken sin,\\ 
 & daz si \textbf{niht hulfen} dem wîbe,\\ 
10 & wan \textbf{si} truoc in ir lîbe,\\ 
 & der aller rîter bluome wirt,\\ 
 & ob in \textbf{ein} sterben \textbf{dâ} verbirt.\\ 
 & dô kam ein \textbf{alter wîser} man\\ 
 & durch \textbf{klagen} über die vrouwe \textbf{sân},\\ 
15 & \textbf{al} dâ \textbf{si} mit dem tôde ranc.\\ 
 & die zene er ir von ein ander twanc\\ 
 & \textbf{und} gôz ir wazzer in den munt.\\ 
 & \textbf{al} dâ wart ir versinnen kunt.\\ 
 & si sprach: "owê, war kam mîn trût?"\\ 
20 & diu vrouwe \textbf{klageten} überlût:\\ 
 & "mînes herzen vreude breit\\ 
 & was Gahmuretes wirdecheit.\\ 
 & \textbf{den} nam mir sîn vrechiu ger.\\ 
 & ich was vil jünger dan er\\ 
25 & und bin sîn muoter und sîn wîp.\\ 
 & ich trage alhie \textbf{den} sînen lîp\\ 
 & und sînes verhes sâmen.\\ 
 & den gâben und nâmen\\ 
 & unser zweier m\textit{inn}e.\\ 
30 & \textbf{habe} got getriuwe s\textit{inn}e,\\ 
\end{tabular}
\scriptsize
\line(1,0){75} \newline
U V W T \newline
\line(1,0){75} \newline
\textbf{1} \textit{Initiale} U V W T  \textbf{2} \textit{Majuskel} T  \textbf{13} \textit{Majuskel} T  \textbf{19} \textit{Majuskel} T  \newline
\line(1,0){75} \newline
\textbf{1} muosen in] muͤsten in V muͤssen W (T)  $\cdot$ wol] \textit{om.} V  $\cdot$ klagen] \textit{om.} U \textbf{2} wan] \textit{om.} T \textbf{4} man] \textit{om.} W  $\cdot$ helfe] craft T \textbf{5} hetez] hat ez T \textbf{6} des] desn T \textbf{7} Herzeloyde] Herzeleide U Hertzelaude V hertzeloyde W \textbf{9} niht hulfen] niht gúlffen W hvlfen niht T \textbf{12} dâ verbirt] do verbirt V icht verwirt W \textbf{14} die vrouwe sân] dise frawen kan W \textbf{16} er ir] si T  $\cdot$ twanc] dranck W \textbf{17} und] man T \textbf{18} al dâ] do T \textbf{19} war] \textit{om.} T  $\cdot$ mîn] mir mein W \textbf{20} klageten] claget in V in klagete W \textbf{21} mînes] vnde mins T \textbf{22} Gahmuretes] Gahmuͦretes U Gamuretes V (W) \textbf{25} und bin] Jch bin V \textbf{26} ich] vnd V  $\cdot$ den] \textit{om.} W doch T \textbf{29} unser] vns T  $\cdot$ minne] muͦme U \textbf{30} sinne] suͦme U \newline
\end{minipage}
\end{table}
\end{document}
