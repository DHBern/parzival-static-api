\documentclass[8pt,a4paper,notitlepage]{article}
\usepackage{fullpage}
\usepackage{ulem}
\usepackage{xltxtra}
\usepackage{datetime}
\renewcommand{\dateseparator}{.}
\dmyyyydate
\usepackage{fancyhdr}
\usepackage{ifthen}
\pagestyle{fancy}
\fancyhf{}
\renewcommand{\headrulewidth}{0pt}
\fancyfoot[L]{\ifthenelse{\value{page}=1}{\today, \currenttime{} Uhr}{}}
\begin{document}
\begin{table}[ht]
\begin{minipage}[t]{0.5\linewidth}
\small
\begin{center}*D
\end{center}
\begin{tabular}{rl}
\textbf{594} & \begin{large}V\end{large}on Logroys, diu clâre.\\ 
 & wem kumt si sus ze vâre?\\ 
 & der Turkote ist mit ir komen,\\ 
 & \textbf{von dem} sô dicke ist vernomen,\\ 
5 & \textbf{daz sîn herze} ist unverzagt.\\ 
 & er hât mit \textbf{speren} prîs bejagt,\\ 
 & \textbf{es} wæren \textbf{gehêrt} driu lant.\\ 
 & Gein sîner werlîchen hant\\ 
 & \textbf{sult ir strîten} mîden nû.\\ 
10 & strîten ist iu gar ze vruo,\\ 
 & ir sît \textbf{ûf} strît \textbf{ze} sêre wunt.\\ 
 & ob ir \textbf{halt} wæret wol gesunt,\\ 
 & ir solt doch strîten gein im lân."\\ 
 & Dô sprach mîn hêr Gawan:\\ 
15 & "ir jeht, ich sul \textbf{hie} hêrre sîn.\\ 
 & swer denne ûf \textbf{al} die \textbf{êre} mîn\\ 
 & rîterschaft sô nâhe suochet,\\ 
 & \textbf{sît} \textbf{er} strîtes \textbf{geruochet},\\ 
 & \multicolumn{1}{l}{ - - - }\\ 
 & \multicolumn{1}{l}{ - - - }\\ 
 & vrouwe, ich sol mîn harnasch hân."\\ 
20 & Des wart grôz weinen \textbf{dâ} getân\\ 
 & von den vrouwen allen vieren.\\ 
 & si sprâchen: "welt ir zieren\\ 
 & iwer \textbf{sælde} und iwer\textit{n} prîs,\\ 
 & sô strîtet niht \textbf{decheinen gewîs}.\\ 
25 & læget ir \textbf{dâ vor im} tôt,\\ 
 & \textbf{alrêst wüehse} unser nôt.\\ 
 & sult aber ir vor im genesen,\\ 
 & \multicolumn{1}{l}{ - - - }\\ 
 & \multicolumn{1}{l}{ - - - }\\ 
 & welt ir in harnasche \textbf{wesen},\\ 
 & iu nement iwer \textbf{êrsten} wundenz leben;\\ 
30 & sô sîn wir an den tôt gegeben."\\ 
\end{tabular}
\scriptsize
\line(1,0){75} \newline
D Z Fr7 \newline
\line(1,0){75} \newline
\textbf{1} \textit{Initiale} D Z  \textbf{3} \textit{Initiale} Fr7  \textbf{8} \textit{Majuskel} D  \textbf{14} \textit{Majuskel} D  \textbf{20} \textit{Majuskel} D  \newline
\line(1,0){75} \newline
\textbf{1} Logroys] Logrois Z (Fr7) \textbf{3} Turkote] Tvrkoyte D Turkoit Z durkoite Fr7 \textbf{5} herze] hant Z \textbf{6} speren] sper Z \textbf{7} es] Ez Z \textbf{9} strîten] strit Z \textbf{10} gar] nach Z \textbf{11} strît] striten Z \textbf{12} halt wæret] halt [werer]: weret Z ioch were Fr7 \textbf{13} solt] soldet Z  $\cdot$ doch strîten] striten doch Fr7 \textbf{16} al] \textit{om.} Z \textbf{18} sît] Ob Z  $\cdot$ geruochet] gerne het Fr7 \textbf{21} den] den den Z \textbf{22} sprâchen] sprach Fr7 \textbf{23} sælde] leben Z  $\cdot$ iwern prîs] iwer prîs D ewern lip Z \textbf{24} strîtet] enstritet Z ::: Fr7  $\cdot$ decheinen gewîs] keinen wis Z deheine wis Fr7 \textbf{25} Leget vor im ir tot Z \textbf{27} aber ir] ir aber Z \newline
\end{minipage}
\hspace{0.5cm}
\begin{minipage}[t]{0.5\linewidth}
\small
\begin{center}*m
\end{center}
\begin{tabular}{rl}
 & von Logrois, diu clâre.\\ 
 & wem kumt si sus zuo vâre?\\ 
 & der Turk\textit{oi}te ist mit ir komen,\\ 
 & \textbf{von dem} sô dicke ist vernomen,\\ 
5 & \textbf{daz sîn herz} ist unverzaget.\\ 
 & er het mit \textbf{spern} prîs bejaget,\\ 
 & \textbf{es} wæren \textbf{geêret} driu lant.\\ 
 & gegen sîner werlîchen hant\\ 
 & \textbf{solt ir strîten} mîden nû.\\ 
10 & strîten ist iu gar ze vruo,\\ 
 & ir sît \textbf{ûf} strît \textbf{sô} sêre wunt.\\ 
 & ob ir \textbf{halt} wæret wol gesunt,\\ 
 & ir solt doch strîten gegen im lân."\\ 
 & dô sprach mîn hêr Gawan:\\ 
15 & "ir jeht, i\textit{ch} sul\textit{e} hêrre sîn.\\ 
 & wer dan ûf \textbf{alle} die \textbf{êre} mîn\\ 
 & ritterschaft sô nâhe suochet,\\ 
 & \textbf{sît} \textbf{er} strîtes \textbf{geruochet},\\ 
 & \multicolumn{1}{l}{ - - - }\\ 
 & \multicolumn{1}{l}{ - - - }\\ 
 & vrouwe, ich sol mîn harnasch hân."\\ 
20 & des wart grôz weinen \textbf{d\textit{â}} getân\\ 
 & von den vrouwen al vieren.\\ 
 & si sprâchen: "wolt ir zieren\\ 
 & iuwer \textbf{sælde} und iuwer\textit{n} prîs,\\ 
 & sô strîte\textit{t} niht \textbf{dekein wîs}.\\ 
25 & læget ir \textbf{d\textit{â} vor im} tôt,\\ 
 & \textbf{sô w\textit{üe}hs\textit{e} êrst} unser nôt.\\ 
 & solt aber ir vor im genesen,\\ 
 & \multicolumn{1}{l}{ - - - }\\ 
 & \multicolumn{1}{l}{ - - - }\\ 
 & wolt ir in harnasch \textbf{wesen},\\ 
 & iu nement iuwer wunden daz leben;\\ 
30 & sô sîn wir an den tôt gegeben."\\ 
\end{tabular}
\scriptsize
\line(1,0){75} \newline
m n o \newline
\line(1,0){75} \newline
\newline
\line(1,0){75} \newline
\textbf{2} zuo] zur o \textbf{3} Turkoite] turkiot m torkeit n turkoit o  $\cdot$ ir] mir n \textbf{8} werlîchen] werlicher n \textbf{11} sô] zuͯ n  $\cdot$ sêre] >sere< o \textbf{13} \textit{Die Verse 594.13-14 fehlen} o   $\cdot$ solt] soltent n \textbf{14} hêr] herre her n \textbf{15} ich sule] ir suͯlt m \textbf{16} wer dan] Wenne denne wer n \textbf{20} dâ] do m n o \textbf{21} al] alle n \textbf{23} sælde] solde o  $\cdot$ iuwern] uͯwer m (o) \textbf{24} sô] Do o  $\cdot$ strîtet] stritten m  $\cdot$ dekein] do keine n \textbf{25} dâ] do m n o \textbf{26} wüehse] wuhs m wuches o \textbf{27} solt aber ir] Solten ir aber n \textbf{30} sîn] sint n o  $\cdot$ tôt gegeben] do gege:en o \newline
\end{minipage}
\end{table}
\newpage
\begin{table}[ht]
\begin{minipage}[t]{0.5\linewidth}
\small
\begin{center}*G
\end{center}
\begin{tabular}{rl}
 & von Logroys, diu clâre.\\ 
 & wem kumet si sus ze vâre?\\ 
 & der Turkoite ist mit ir komen,\\ 
 & \textbf{dâ von} sô dicke ist vernomen,\\ 
5 & \textbf{daz sîn hant} ist unverzaget.\\ 
 & er hât mit \textbf{spern} prîs bejaget,\\ 
 & \textbf{ez} wæren \textbf{gehêrt} driu lant.\\ 
 & gein sîner werlîchen hant\\ 
 & \textbf{solt ir strîten} mîden nû.\\ 
10 & strîten ist iu gar ze vruo,\\ 
 & ir sît \textit{\textbf{ze}} strîte \textbf{ze} sêre wunt.\\ 
 & ob ir wært wol gesunt,\\ 
 & ir solt doch strîten gein im lân."\\ 
 & dô sprach mîn hêrre Gawan:\\ 
15 & "ir jehet, ich sule \textbf{hie} hêrre sîn.\\ 
 & swer denne ûf die \textbf{erde} mîn\\ 
 & rîtersch\textit{a}ft sô nâhen suochet,\\ 
 & \textbf{ob} \textbf{der} strîtes \textbf{ruochet}\\ 
 & oder rîterschefte gert,\\ 
 & des wirt er von mir gewert.\\ 
 & vrouwe, ich sol mîn harnasch hân."\\ 
20 & des wart grôz weinen \textbf{dâ} getân\\ 
 & von den vrouwen allen vieren.\\ 
 & si sprâchen: "welt ir zieren\\ 
 & iuwer \textbf{leben} unde iuwern prîs,\\ 
 & sô strîtet niht \textbf{deheine wîs}.\\ 
25 & \begin{large}L\end{large}\textit{æ}get ir \textbf{dâ vor im} tôt,\\ 
 & \textbf{alrêrste wüehse} unser nôt.\\ 
 & sult aber ir vor im genesen,\\ 
 & - daz muoz an grôzem glücke wesen -,\\ 
 & wande, lieber hêrre mîn,\\ 
 & welt ir in harnasche \textbf{sîn},\\ 
 & iu n\textit{e}me\textit{n}t iuwer wunden daz leben;\\ 
30 & sô sîn wir an den tôt gegeben."\\ 
\end{tabular}
\scriptsize
\line(1,0){75} \newline
G I L M Z \newline
\line(1,0){75} \newline
\textbf{1} \textit{Initiale} L M Z  \textbf{13} \textit{Initiale} I  \textbf{25} \textit{Initiale} G  \textbf{27} \textit{Initiale} I  \newline
\line(1,0){75} \newline
\textbf{1} Logroys] logrois G (Z) ligrois M \textbf{2} ze vâre] zcware M \textbf{3} Turkoite] turchoit G Turkoyt I Tvrkoýt L torkoteis M Turkoit Z  $\cdot$ ist] \textit{om.} M \textbf{4} dâ von] von dem I (Z) \textbf{6} spern] sper M Z \textbf{7} wæren] warn I (L)  $\cdot$ gehêrt driu] gehertie L geherit dy M \textbf{8} werlîchen] wertlichen M \textbf{9} ir] \textit{om.} I  $\cdot$ strîten] strit L \textbf{10} gar] nach Z \textbf{11} ze] \textit{om.} G vf L (M) Z  $\cdot$ strîte] striten M Z  $\cdot$ ze sêre] sere I gar zcu sere M \textbf{12} wært] ouch wart L halt weret M halt [werer]: weret Z \textbf{13} solt] soltet I (Z)  $\cdot$ strîten] strite I strit M  $\cdot$ im] \textit{om.} L \textbf{14} dô] Da M  $\cdot$ hêrre Gawan] ergawan M \textbf{15} jehet] sprecht M  $\cdot$ sule] sol L \textbf{16} swer] Wer L M  $\cdot$ erde] erden I ere L M Z \textbf{17} rîterschaft] Riterscheft G \textbf{18} ob der] oder I Ob er Z  $\cdot$ ruochet] geruchet M Z \textbf{18} \textit{Vers 594.18¹ fehlt} L M Z  \textbf{18} \textit{Vers 594.18² fehlt} M Z   $\cdot$ \textit{nach 594.18²:} Die wile mich der lip wert L   $\cdot$ des wirt er] der wirt er I Er wirt es L \textbf{20} wart] wart wart L \textbf{21} den] den den Z \textbf{22} sprâchen] sprach L \textbf{23} iuwern] vwer L  $\cdot$ prîs] lip Z \textbf{24} strîtet] en stritet M (Z)  $\cdot$ deheine] keinen Z \textbf{25} Læget] Liget G  $\cdot$ ir dâ vor im] ir von im da L ir vor yme da M vor im ir Z \textbf{27} aber ir] ir aber Z  $\cdot$ vor] von L \textbf{27} \textit{Die Verse 594.27¹-27² fehlen} L M Z  \textbf{28} sîn] wesen L (M) Z \textbf{29} nement] nimet G (L) (M)  $\cdot$ wunden] [wunde*]: wunden G erste L erste wunde M ersten wunden Z \textbf{30} sîn] sý L  $\cdot$ an] in L \newline
\end{minipage}
\hspace{0.5cm}
\begin{minipage}[t]{0.5\linewidth}
\small
\begin{center}*T
\end{center}
\begin{tabular}{rl}
 & von Logrois, diu klâre.\\ 
 & wem kumt si sust zuo vâre?\\ 
 & der Turkoyte ist mit i\textit{r} komen,\\ 
 & \textbf{dâ von} sô dicke ist vernomen,\\ 
5 & \textbf{sô daz er} ist unverzagt.\\ 
 & er hât mit \textbf{sper} prîs bejagt,\\ 
 & \textbf{es} wæren \textbf{gehêrt} driu lant.\\ 
 & gên sîner werlîchen hant\\ 
 & \textbf{ir solt strît} mîden nû.\\ 
10 & strîten ist iu gar zuo vruo,\\ 
 & ir sît \textbf{ûf} strît \textbf{zuo} sêre wunt.\\ 
 & ob ir \textbf{halt} wæret wol gesunt,\\ 
 & ir solt doch strîten gên im lân."\\ 
 & dô sprach mîn hêrre Gawan:\\ 
15 & "ir jehet, ich solle \textbf{hie} hêrre sîn.\\ 
 & wer danne ûf die \textbf{êre} mîn\\ 
 & ritterschaft sô nâhe suochet,\\ 
 & \textbf{ob} \textbf{der} strîtes \textbf{geruochet},\\ 
 & \multicolumn{1}{l}{ - - - }\\ 
 & \multicolumn{1}{l}{ - - - }\\ 
 & vrouwe, ich sol mîn harnasch hân."\\ 
20 & des wart grôz weinen getân\\ 
 & von den vrouwen allen vieren.\\ 
 & si sprâchen: "wolt ir zieren\\ 
 & iuwer \textbf{leben} und iuwer\textit{n} prîs,\\ 
 & sô \textbf{en}strîtet niht \textbf{keinen wîs}.\\ 
25 & læget ir \textbf{vor im dâ} tôt,\\ 
 & \textbf{alrêst wüehse} unser nôt.\\ 
 & solt aber ir vor im genesen,\\ 
 & \multicolumn{1}{l}{ - - - }\\ 
 & \multicolumn{1}{l}{ - - - }\\ 
 & wolt ir in harnasch \textbf{wesen},\\ 
 & iu nement iuwer \textbf{êrste} wunden daz leben;\\ 
30 & sô sîn wir an den tôt gegeben."\\ 
\end{tabular}
\scriptsize
\line(1,0){75} \newline
Q R W V U \newline
\line(1,0){75} \newline
\textbf{1} \textit{Initiale} Q  \newline
\line(1,0){75} \newline
\textbf{1} \textit{Die Verse 553.1-599.30 fehlen} U   $\cdot$ Logrois] logroys Q logroẏs V \textbf{3} Turkoyte] turkoite Q (V) truͦkoyt R  $\cdot$ ir] im Q \textbf{4} dâ von] [D*]: Von dem V \textbf{5} Das sin hercze (ist W V ) vnuerczagt R (W) (V) \textbf{6} mit] mir den W  $\cdot$ sper] spern R (W) V \textbf{7} wæren] geweret R  $\cdot$ gehêrt] [*eret]: geeret V \textbf{9} ir solt strît] Soͯlt ir strit R (W) Soͤllent ir striten V \textbf{11} ûf strît zuo] ze strit R \textbf{12} wæret] weren R  $\cdot$ wol] wolt W \textbf{13} solt] soͯltent R (W) (V)  $\cdot$ strîten] strit R streites W \textbf{16} wer] Swer V \textbf{19} mîn] minen V \textbf{20} weinen] winen do R (W) (V) \textbf{21} allen] [*]: allen V \textbf{23} leben] [*]: selde V  $\cdot$ iuwern] ewer Q \textbf{24} niht keinen] enkeine R nicht keine W (V) \textbf{25} læget] Liget W  $\cdot$ vor im dâ] da von Jm R do vor im W vor im V \textbf{26} unser] vns R \textbf{27} aber ir vor] ir aber R aber ir von W \textbf{28} wolt] Woͤlt W \textbf{29} nement] nymt W  $\cdot$ êrste wunden] [*]: wunden V  $\cdot$ daz] úwer R [*]: daz V \newline
\end{minipage}
\end{table}
\end{document}
