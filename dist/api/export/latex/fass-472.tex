\documentclass[8pt,a4paper,notitlepage]{article}
\usepackage{fullpage}
\usepackage{ulem}
\usepackage{xltxtra}
\usepackage{datetime}
\renewcommand{\dateseparator}{.}
\dmyyyydate
\usepackage{fancyhdr}
\usepackage{ifthen}
\pagestyle{fancy}
\fancyhf{}
\renewcommand{\headrulewidth}{0pt}
\fancyfoot[L]{\ifthenelse{\value{page}=1}{\today, \currenttime{} Uhr}{}}
\begin{document}
\begin{table}[ht]
\begin{minipage}[t]{0.5\linewidth}
\small
\begin{center}*D
\end{center}
\begin{tabular}{rl}
\textbf{472} & "\textbf{\begin{large}M\end{large}ac} rîterschaft des lîbes prîs\\ 
 & unt \textbf{doch} der sêle pardîs\\ 
 & bejagen mit schilde unt \textbf{ouch} \textbf{mit} sper,\\ 
 & sô was ie rîterschaft mîn ger.\\ 
5 & ich streit ie, swâ \textbf{ich} \textbf{strîten} vant,\\ 
 & sô daz mîn werlîchiu hant\\ 
 & sich \textbf{næhert} dem prîse.\\ 
 & ist got an strît wîse,\\ 
 & der sol mich dar benennen,\\ 
10 & daz si mich dâ bekennen.\\ 
 & mîn hant dâ \textbf{strîtes} niht verbirt."\\ 
 & Dô sprach aber sîn kiuscher wirt:\\ 
 & "ir \textbf{müeset} aldâ \textbf{vor} hôchvart\\ 
 & mit senften willen sîn bewart.\\ 
15 & iuch verleitet lîhte iwer jugent,\\ 
 & daz ir \textbf{der} kiusche \textbf{bræchet} tugent.\\ 
 & hôchvart \textbf{ie} seic und viel",\\ 
 & sprach der wirt. ieweder ouge im wiel,\\ 
 & dô er an \textbf{diz} mære \textbf{dâhte},\\ 
20 & daz er \textbf{dâ} mit rede volbrâhte.\\ 
 & dô sprach er: "hêrre, ein künec \textbf{dâ} was,\\ 
 & der hiez unt heizet noch Anfortas.\\ 
 & daz sol iuch und mich armen\\ 
 & immer mêr erbarmen\\ 
25 & umb sîne herzebære nôt,\\ 
 & die hôchvart im ze lône bôt.\\ 
 & sîn jugent unt sîn rîcheit\\ 
 & der werlde an im vuogte leit\\ 
 & unt daz er gerte minne\\ 
30 & ûzerhalp der \textbf{kiusche} sinne.\\ 
\end{tabular}
\scriptsize
\line(1,0){75} \newline
D \newline
\line(1,0){75} \newline
\textbf{1} \textit{Initiale} D  \textbf{12} \textit{Majuskel} D  \newline
\line(1,0){75} \newline
\newline
\end{minipage}
\hspace{0.5cm}
\begin{minipage}[t]{0.5\linewidth}
\small
\begin{center}*m
\end{center}
\begin{tabular}{rl}
 & "\textbf{mac} ritterschaft des lîbes prîs\\ 
 & und \textbf{doch} der sêlen paradîs\\ 
 & bejagen mit schilt und sper,\\ 
 & sô was ie ritterschaft mîn ger.\\ 
5 & ich streit ie, wâ \textbf{ich} \textbf{strîte} vant,\\ 
 & sô daz mîn werlîchiu hant\\ 
 & sich \textbf{næherte} dem prîse.\\ 
 & ist got an strîte wîse,\\ 
 & der sol mich dar benenn\textit{e}n,\\ 
10 & daz si mich d\textit{â} bekennen.\\ 
 & mîn hant d\textit{â} \textbf{dienst} niht verbirt."\\ 
 & dô sprach aber sîn kiuscher wirt:\\ 
 & "ir \dag muoste\dag  al dâ \textbf{vor} hôchvart\\ 
 & mit senftem willen sîn bewart.\\ 
15 & i\textit{u}ch verleite lîht iuwer jugent,\\ 
 & daz ir \textbf{des} kiusche \textbf{brechet} tugent.\\ 
 & hôchvart, \textbf{diu} seic und viel",\\ 
 & sprach der wirt. ietweder ouge im wiel,\\ 
 & dô er  \textbf{daz} mære \textbf{gedâht},\\ 
20 & daz er \textbf{d\textit{â}} mit rede volbrâht.\\ 
 & dô sprach er: "hêrre, ein künic was,\\ 
 & der hiez und heizet noch Anfortas.\\ 
 & daz sol iuch und mich armen\\ 
 & iemer \textit{m}ê erbarmen\\ 
25 & umb sîn herzebære nôt,\\ 
 & die hôchvart ime zuo lône bôt.\\ 
 & sîn jugent und sîn rîcheit\\ 
 & der werlte an im vuogete leit\\ 
 & und daz er gerte minne\\ 
30 & ûzerhalp der \textbf{kiuschen} \textit{s}inne.\\ 
\end{tabular}
\scriptsize
\line(1,0){75} \newline
m n o \newline
\line(1,0){75} \newline
\newline
\line(1,0){75} \newline
\textbf{3} und] vnd mit n \textbf{5} ie] e o \textbf{6} \textit{Versdoppelung (²o); Lesarten des vorausgehenden Verses mit ¹o bezeichnet} o   $\cdot$ mîn] man \textsuperscript{1}\hspace{-1.3mm} o \textbf{9} benennen] [benennet]: benennetn m \textbf{10} dâ] do m n o \textbf{11} dâ] do m n o \textbf{15} iuch] Jch m o \textbf{16} daz] Des n  $\cdot$ des] dasz o  $\cdot$ brechet] brechen n \textbf{17} diu] ie n o \textbf{18} wiel] miel o \textbf{19} mære] [nie]: mer m \textbf{20} dâ] do m n o \textbf{22} hiez] hiesse n  $\cdot$ Anfortas] an fortas n \textbf{24} mê] nie m \textbf{25} umb] Jemer [n*]: ummb o \textbf{30} der] des o  $\cdot$ sinne] mynne m \newline
\end{minipage}
\end{table}
\newpage
\begin{table}[ht]
\begin{minipage}[t]{0.5\linewidth}
\small
\begin{center}*G
\end{center}
\begin{tabular}{rl}
 & "\textbf{\begin{large}N\end{large}âch} rîterschaft, des lîbes prîs\\ 
 & unde \textbf{ouch} der sêle paradîs\\ 
 & bejagen mit schilt und\textit{e} \textbf{\textit{m}it} sper,\\ 
 & sô was ie rîterschaft mîn ger.\\ 
5 & ich streit ie, swâ \textbf{man} \textbf{strîten} vant,\\ 
 & sô daz \textit{mîn} werlîchiu hant\\ 
 & sich \textbf{nâhete} dem brîse.\\ 
 & ist got an strîte wîse,\\ 
 & der sol mich dar benennen,\\ 
10 & daz si mich dâ bekennen.\\ 
 & mîn hant dâ \textbf{dienst} niht verbirt."\\ 
 & dô sprach aber sîn kiuscher wirt:\\ 
 & "ir \textbf{müeset} aldâ \textbf{von} hôchvart\\ 
 & mit senften willen sîn bewart.\\ 
15 & iuch verleite\textit{t} lîhte iuwer jugent,\\ 
 & daz ir \textbf{der} kiusche \textbf{bræchet} \textbf{ir} tugent.\\ 
 & hôchvart \textbf{ie} seic unde viel",\\ 
 & sprach der wirt. ietweder ouge im wiel,\\ 
 & dô er an \textbf{daz} mære \textbf{dâhte},\\ 
20 & daz er \textbf{dâ} mit rede volbrâhte.\\ 
 & dô sprach er: "hêrre, ein künic \textbf{dâ} was,\\ 
 & der hiez unde heizet noch Anfortas.\\ 
 & daz sol iuch unde mich armen\\ 
 & immer mêr erbarmen\\ 
25 & umbe sîne herzebære nôt,\\ 
 & die hôchvart im ze lône bôt.\\ 
 & sîn jugent unde sîn rîcheit\\ 
 & der werlte \textit{an im vuogte} leit\\ 
 & unt daz er gerte minne\\ 
30 & ûzerhalp der \textbf{kiusche} sinne.\\ 
\end{tabular}
\scriptsize
\line(1,0){75} \newline
G I O L M Z Fr18 \newline
\line(1,0){75} \newline
\textbf{1} \textit{Initiale} G I O L Z Fr18  \textbf{15} \textit{Initiale} I  \newline
\line(1,0){75} \newline
\textbf{1} Nâch] ÷ach O Mach L (Z) Fr18 Sage M \textbf{2} ouch] doch O L (M) Z Fr18 \textbf{3} unde mit] vnde oͮch mit G \textbf{5} swâ man] swa [min]: man G da man I wan ich L wa man M wo ich Z  $\cdot$ vant] want I \textbf{6} mîn] \textit{om.} G  $\cdot$ werlîchiu] wertliche M \textbf{7} nâhete] huͤp nach I nahet O M Fr18 nehert Z \textbf{8} strîte] \textit{om.} O \textbf{10} bekennen] nennen O \textbf{11} dienst] striten Z \textbf{12} dô] Da M  $\cdot$ sîn kiuscher] der kevsche Z \textbf{13} von] vor O L M Z Fr18 \textbf{14} senften] shlehtem I senftem O L Z Fr18 senfftē M  $\cdot$ sîn] sie M \textbf{15} verleitet] uirleite G \textbf{16} kiusche] kuschen M  $\cdot$ bræchet] brechtit M  $\cdot$ ir] \textit{om.} L Z \textbf{18} ietweder ouge] sinev augen I  $\cdot$ im] \textit{om.} I Z \textbf{19} dô] Da O M Z  $\cdot$ daz] die L ditz Z  $\cdot$ dâhte] gedahte L Z \textbf{20} rede] die rede O redin M  $\cdot$ volbrâhte] wol brachte M \textbf{21} dô] Da O M  $\cdot$ er] ir M  $\cdot$ dâ] \textit{om.} I \textbf{22} heizet noch] heizet O Fr18 noch heiszit M  $\cdot$ Anfortas] Amfortas L \textbf{25} herzebære] hertzenbaren L \textbf{26} die] diu I \textbf{27} unde sîn] vnd I \textbf{28} an im vuogte] fuegte an im G an im suͯchte L \textbf{29} minne] libe M \textbf{30} kiusche] chusshen I (O) (M) (Fr18)  $\cdot$ sinne] chvniginne O minne Fr18 \newline
\end{minipage}
\hspace{0.5cm}
\begin{minipage}[t]{0.5\linewidth}
\small
\begin{center}*T
\end{center}
\begin{tabular}{rl}
 & "\textbf{mac} rîterschaft des lîbes prîs\\ 
 & unde \textbf{doch} der sêle paradîs\\ 
 & bejagen mit schilte unde \textbf{mit} sper,\\ 
 & sô was ie rîterschaft mîn ger.\\ 
5 & ich streit ie, sw\textit{â} \textbf{\textit{m}an} \textbf{strîten} vant,\\ 
 & sô daz mîn werlîchiu hant\\ 
 & sich \textbf{nâhete} dem prîse.\\ 
 & ist got an strîte wîse,\\ 
 & der sol mich dar benennen,\\ 
10 & daz si mich dâ bekennen.\\ 
 & mîn hant dâ \textbf{dienst} niht verbirt."\\ 
 & Dô sprach aber sîn kiuscher wirt:\\ 
 & "ir \textbf{müezet} aldâ \textbf{vor} hôchvart\\ 
 & mit senften willen sîn bewart.\\ 
15 & iuch ver\textit{l}e\textit{i}te lîhte iuwer jugent,\\ 
 & daz ir \textbf{der} kiusche \textbf{brechet} \textbf{ir} tugent.\\ 
 & hôchvart \textbf{ie} seic unde viel",\\ 
 & sprach der wirt. ietweder ouge im wiel,\\ 
 & dô er an \textbf{daz} mære \textbf{gedâhte},\\ 
20 & daz er mit rede volbrâhte.\\ 
 & \begin{large}D\end{large}ô sprach er: "hêrre, ein künec \textbf{dâ} was,\\ 
 & der hiez unde he\textit{i}zet noch Anfortas.\\ 
 & daz sol iuch unde mich armen\\ 
 & iemer mêr erbarmen\\ 
25 & umbe sîne herzebære nôt,\\ 
 & die hôchvart im ze lône bôt.\\ 
 & sîn jugent unde sîn rîcheit\\ 
 & der werlte an im vuocte leit\\ 
 & unde daz er gerte minne\\ 
30 & ûzerhalp der \textbf{kiuschen} sinne.\\ 
\end{tabular}
\scriptsize
\line(1,0){75} \newline
T U V W Q R Fr42 \newline
\line(1,0){75} \newline
\textbf{12} \textit{Majuskel} T  \textbf{21} \textit{Initiale} T V W Fr42  \newline
\line(1,0){75} \newline
\textbf{1} \textit{Die Verse 453.1-502.30 fehlen} U   $\cdot$ mac] Macht Q \textbf{2} sêle] sole W \textbf{3} schilte] schilten Q \textbf{4} rîterschaft] riterscat Fr42  $\cdot$ ger] beger R \textbf{5} streit] strit R  $\cdot$ swâ man] swain an T [wa*]: wa men V wo man W Q (R)  $\cdot$ strîten] [*]: striten V zuͦ streiten W \textbf{10} dâ] do V W Q  $\cdot$ bekennen] erkennen W \textbf{11} dâ] do V W Q \textbf{12} sîn kiuscher] der kunsche R \textbf{13} ir müezet] Erst must Q  $\cdot$ vor] von R \textbf{14} senften] senfftem W (R) senfftē Q  $\cdot$ bewart] [bespart]: bewart Fr42 \textbf{15} iuch] iv T  $\cdot$ verleite] verteilte T verleitet V verleit Q \textbf{16} ir] \textit{om.} Q  $\cdot$ der] [*]: der V \textbf{17} viel] wiel R [neic]: viel Fr42 \textbf{18} wiel] wil Q \textbf{19} \textit{Versfolge 472.20-19} V   $\cdot$ er an daz mære] an das mere er Q  $\cdot$ gedâhte] dachte R \textbf{20} mit] do mit V W (Q) da mit Fr42 \textbf{21} dâ was] [*]: was V do waz W (Q) R \textbf{22} heizet] heiezet T \textbf{23} iuch] îv T  $\cdot$ armen] vil armen Q [erb armen]: armen Fr42 \textbf{28} vuocte] [*]: fuͦte V \textbf{29} minne] [*]: minne V \textbf{30} [*]: Vsserhalb der kv́schen sinne V  $\cdot$ kiuschen] kewsche Q (Fr42)  $\cdot$ sinne] sinnde Q \newline
\end{minipage}
\end{table}
\end{document}
