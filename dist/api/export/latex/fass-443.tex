\documentclass[8pt,a4paper,notitlepage]{article}
\usepackage{fullpage}
\usepackage{ulem}
\usepackage{xltxtra}
\usepackage{datetime}
\renewcommand{\dateseparator}{.}
\dmyyyydate
\usepackage{fancyhdr}
\usepackage{ifthen}
\pagestyle{fancy}
\fancyhf{}
\renewcommand{\headrulewidth}{0pt}
\fancyfoot[L]{\ifthenelse{\value{page}=1}{\today, \currenttime{} Uhr}{}}
\begin{document}
\begin{table}[ht]
\begin{minipage}[t]{0.5\linewidth}
\small
\begin{center}*D
\end{center}
\begin{tabular}{rl}
\textbf{443} & \textit{\begin{large}A\end{large}}l sîner vröude er dô vergaz.\\ 
 & ich wæne, er het gevrâget baz,\\ 
 & wære er ze Munsalvæsche komen,\\ 
 & denne als ir ê hât vernomen.\\ 
5 & Nû lât in rîten. war sol er?\\ 
 & \textbf{dort} gein im kom geriten her\\ 
 & ein man, dem was \textbf{daz} houbet blôz,\\ 
 & \textbf{sîn} wâpenroc von koste grôz,\\ 
 & dâr \textbf{undenz} harnasch blanc gevar.\\ 
10 & ân daz houbet was er gewâpent gar.\\ 
 & gein Parzivale er vaste reit.\\ 
 & \textbf{dô sprach er}: "hêrre, mir ist leit,\\ 
 & daz ir mînes hêrren walt sus bant.\\ 
 & ir wert schiere \textbf{drumbe} ermant,\\ 
15 & dâ von sich iwer gemüete sent.\\ 
 & Munsalvæsche ist niht gewent,\\ 
 & daz \textbf{iemen ir} sô nâhe rite,\\ 
 & ez enwære, der angestlîche strite\\ 
 & oder \textbf{der} a\textit{l}solhen wandel bôt,\\ 
20 & als man vor dem walde heizet tôt."\\ 
 & Einen helm er in der hende\\ 
 & vuorte, des gebende\\ 
 & wâren snüere sîdîn,\\ 
 & unt \textbf{eine} scharpfe glevîn,\\ 
25 & dâr inne \textbf{al} niwe was der schaft.\\ 
 & der helt bant mit zornes kraft\\ 
 & den helm ûfez houbet ebene.\\ 
 & ez \textbf{en}stuont in niht vergebene\\ 
 & an den selben zîten\\ 
30 & sîn drôn unt \textbf{ouch} sîn strîten.\\ 
\end{tabular}
\scriptsize
\line(1,0){75} \newline
D Fr5 Fr31 \newline
\line(1,0){75} \newline
\textbf{1} \textit{Initiale} D Fr5 Fr31  \textbf{5} \textit{Majuskel} D  \textbf{21} \textit{Majuskel} D  \newline
\line(1,0){75} \newline
\textbf{1} Al sîner] ÷l siner D Di sine Fr31 \textbf{3} ze Munsalvæsche] ce Mvnsælvæsce D zi muntsaluasch Fr5 zemvntschalvasch Fr31 \textbf{6} dort] Der Fr31 \textbf{9} undenz] vndir Fr5 (Fr31) \textbf{11} Parzivale] Parcifale D parcifal Fr5 Parzifal Fr31 \textbf{13} sus] sivs Fr5 \textbf{15} gemüete] mvͦt Fr5 (Fr31) \textbf{16} Munsalvæsche] Mvntsælvæsce D Muntsaluasch Fr5 Mintschalnasch Fr31 \textbf{17} nâhe] nahir Fr5  $\cdot$ rite] giritte Fr5 (Fr31) \textbf{18} angestlîche] angistlichir Fr5 (Fr31) \textbf{19} alsolhen] ansolhen D dera selhen Fr31 \textbf{20} als] [al]: als D \textbf{23} snüere] wiz Fr5 Fr31 \textbf{24} eine scharpfe] einin scharpfin Fr5 \textbf{28} enstuont] stuͦnt Fr5 \textbf{30} ouch] \textit{om.} Fr5 \newline
\end{minipage}
\hspace{0.5cm}
\begin{minipage}[t]{0.5\linewidth}
\small
\begin{center}*m
\end{center}
\begin{tabular}{rl}
 & al sîner vröuden er dô vergaz.\\ 
 & ich wæne, er hete gevrâget baz,\\ 
 & wær er ze Mun\textit{t}salvasche komen,\\ 
 & danne als ir ê habet vernomen.\\ 
5 & \begin{large}N\end{large}û lât in rîten. war sol er?\\ 
 & \textbf{dar} gegen ime kam geriten her\\ 
 & ein man, dem was \textbf{sîn} houbet blôz,\\ 
 & \textbf{sî\textit{n}} \textit{w}âpenroc von koste grôz,\\ 
 & dâr \textbf{under} harnasch blanc gevar.\\ 
10 & âne daz houbet was er gewâpent gar.\\ 
 & gegen Parcifale er vaste reit.\\ 
 & \textbf{er sprach}: "hêrre, mir ist leit,\\ 
 & daz ir mînes hêrren walt sus bant.\\ 
 & ir werdet schiere \textbf{von im} ermant,\\ 
15 & dâ von sich iuwer gemüete sent.\\ 
 & Mun\textit{t}salvasche ist niht gewent,\\ 
 & daz \textbf{ieme\textit{n} ir} sô nâhe rite,\\ 
 & ez enwære, der angestlîchen strite\\ 
 & oder \textbf{der} alsolhe\textit{n} wandel bôt,\\ 
20 & als man vor dem walde heizet tôt."\\ 
 & einen helm er in der hende\\ 
 & vuorte, des gebende\\ 
 & wâren snüer\textit{e} sîdîn,\\ 
 & und \textbf{eine} scharfe glevîn,\\ 
25 & dâr inne \textbf{al} niu was der schaft.\\ 
 & der helt bant mit zornes kraft\\ 
 & den helm ûfe\textit{z} houbet ebene.\\ 
 & ez \textbf{en}stuont in niht vergebene\\ 
 & an den selben zîten\\ 
30 & sîn drôn und \textbf{ouch} sîn strîten.\\ 
\end{tabular}
\scriptsize
\line(1,0){75} \newline
m n o \newline
\line(1,0){75} \newline
\textbf{5} \textit{Initiale} m   $\cdot$ \textit{Capitulumzeichen} n  \newline
\line(1,0){75} \newline
\textbf{3} Muntsalvasche] munsaluasce m muntsaluasce n munt saluasce o \textbf{6} dar] Dort n o \textbf{7} sîn] das n (o) \textbf{8} sîn wâpenroc] Sin rock vnd wappen rock m  $\cdot$ koste] toste o \textbf{11} Parcifale] parcifal n o \textbf{14} von im] dar vmb n (o) \textbf{15} sich] so sich n \textbf{16} Muntsalvasche] Munsaluasce m Monsaluasce n Mansaluasce o \textbf{17} iemen] yemer m  $\cdot$ rite] [ritten]: ritte o \textbf{18} enwære] were n o  $\cdot$ angestlîchen] angstluche o \textbf{19} alsolhen] alsolhem m einen sollichen n ein soluchen o \textbf{23} snüere] snvrin m \textbf{25} al niu] alúmbe o \textbf{27} ûfez] vfer m \textbf{30} drôn] drauͯm o  $\cdot$ ouch] \textit{om.} n o \newline
\end{minipage}
\end{table}
\newpage
\begin{table}[ht]
\begin{minipage}[t]{0.5\linewidth}
\small
\begin{center}*G
\end{center}
\begin{tabular}{rl}
 & \begin{large}A\end{large}\textit{l} \textit{s}îner vröude er dô vergaz.\\ 
 & ich wæne, er hete gevrâget baz,\\ 
 & wær er ze Muntsalvatsche komen,\\ 
 & danne als ir ê habet vernomen.\\ 
5 & nû lât in rîten. war sol er?\\ 
 & \textbf{dort} gein im kom geriten her\\ 
 & ein man, dem was \textbf{daz} houbet blôz,\\ 
 & \textbf{sîn} wâpenroc von koste grôz,\\ 
 & dâr \textbf{under daz} harnasch blanc gevar.\\ 
10 & ân \textit{d}az houb\textit{et} was er gewâpent gar.\\ 
 & gegen Parzival er vaste reit.\\ 
 & \textbf{dô sprach er}: "hêrre, mir ist leit,\\ 
 & daz ir mînes hêrren walt sus \textit{b}ant.\\ 
 & ir werdet schier \textbf{drumbe} ermant,\\ 
15 & dâ von sich iuwer gemüete sent.\\ 
 & Muntsalvatsche ist niht gewent,\\ 
 & daz \textbf{iemen ir} sô nâhen rite,\\ 
 & ez enwære, de\textit{r} angestlîchen strite\\ 
 & ode \textbf{der} alsolhen wandel bôt,\\ 
20 & als man vor dem walde heizet tôt."\\ 
 & einen helm er in der hende\\ 
 & vuorte, des gebende\\ 
 & wâren snüere sîdîn,\\ 
 & unde scharfe glavîn,\\ 
25 & dâr inne \textbf{al} niuwe was der schaft.\\ 
 & der helt bant mit zornes kraft\\ 
 & de\textit{n} helm ûf daz houbet eben.\\ 
 & ez \textbf{en}stuont in niht vergeben\\ 
 & an den selben zîten\\ 
30 & sîn drôn unde sîn strîten.\\ 
\end{tabular}
\scriptsize
\line(1,0){75} \newline
G I O L M Z \newline
\line(1,0){75} \newline
\textbf{1} \textit{Initiale} G I O L Z  \textbf{11} \textit{Initiale} I  \newline
\line(1,0){75} \newline
\textbf{1} Al sîner] Al nah siner G ÷ller siner O  $\cdot$ vröude] frevden O (M) (Z)  $\cdot$ dô] \textit{om.} O da M Z \textbf{3} ze Muntsalvatsche] zemvntschaluatsche G ze muntshaluasche I Munsalvatsche M zv montsalvatsche Z \textbf{4} habet] hap I \textbf{5} in rîten] riten yn M in varn Z \textbf{6} gein im kom] gein im kome G chom Gein im I chvmt gein im O \textbf{7} ein] Eyme M  $\cdot$ dem] \textit{om.} M \textbf{8} sîn] Ein O L \textbf{9} daz] \textit{om.} L \textbf{10} ân daz houbet] Anez huͦpte G  $\cdot$ er] \textit{om.} I \textbf{11} Parzival] parziual G parzifal I M Barcifal O parzifale L parcifal Z \textbf{12} dô] Da M \textbf{13} sus] \sout{durc} I  $\cdot$ bant] hant G \textbf{14} ir] [er]: ir I \textbf{16} Muntsalvatsche] Muntsalvatsch G Munshalvasche I Munsalvatsche M Montsalvatsch Z  $\cdot$ niht] \textit{om.} I \textbf{18} enwære] were M  $\cdot$ der] des G \textit{om.} L \textbf{19} der] \textit{om.} M \textbf{21} einen] Sinen L \textbf{22} vuorte] vurt I (Z) \textbf{24} scharfe] ein scharfe O (L) (M) (Z) \textbf{25} dâr inne] \textit{om.} I \textbf{26} helt] \textit{om.} O  $\cdot$ bant] \textit{om.} I \textbf{27} den] dem G bant den I \textbf{28} enstuont] stunt I (O) \textbf{30} drôn] dro I (L)  $\cdot$ unde] vnde ovch O (L) (M) (Z)  $\cdot$ strîten] riten M \newline
\end{minipage}
\hspace{0.5cm}
\begin{minipage}[t]{0.5\linewidth}
\small
\begin{center}*T
\end{center}
\begin{tabular}{rl}
 & \begin{large}A\end{large}l sîner vröude er dô vergaz.\\ 
 & Ich wæne, er hete gevrâget baz,\\ 
 & wærer ze Munsalvasche komen,\\ 
 & danne als ir ê habet vernomen.\\ 
5 & Nû lât in rîten. war sol er?\\ 
 & \textbf{dort} gegen im kom geriten her\\ 
 & ein man, dem was \textbf{daz} houbt blôz,\\ 
 & \textbf{ei\textit{n}} wâpenroc von koste grôz,\\ 
 & dâr \textbf{under} harnasch blanc gevar.\\ 
10 & âne daz houbet was er gewâpent gar.\\ 
 & gegen Parcifal er vaste reit.\\ 
 & \textbf{dô sprach er}: "hêrre, mir ist leit,\\ 
 & daz ir mînes hêrren walt sus bant.\\ 
 & ir werdet schiere \textbf{drumbe} ermant,\\ 
15 & dâ von sich iuwer gemüete sent.\\ 
 & Munsalvasche ist niht gewent,\\ 
 & daz \textbf{ir ieman} sô nâhe rite,\\ 
 & ez \textit{en}wære, der angestlîchen strite\\ 
 & oder alsolhen wandel \textbf{drumbe} bôt,\\ 
20 & alse man vor dem walde heizet tôt."\\ 
 & Einen helm er in der hende\\ 
 & vuorte, des gebende\\ 
 & wâren s\textit{n}üere sîdîn,\\ 
 & unde \textbf{eine} scharpfe glevîn,\\ 
25 & dâr inne niuwe was der schaft.\\ 
 & der helt bant mit zornes kraft\\ 
 & den helm ûf daz houbet ebene.\\ 
 & ez stuont in niht vergebene\\ 
 & an den selben zîten\\ 
30 & sîn dröun unde \textbf{ouch} sîn strîten.\\ 
\end{tabular}
\scriptsize
\line(1,0){75} \newline
T U V W Q R \newline
\line(1,0){75} \newline
\textbf{1} \textit{Überschrift:} Awentewr wy partzifal sein rosz zu tode fiel vnd er selbe behinck an dem zweige Q   $\cdot$ \textit{Großinitiale} T U   $\cdot$ \textit{Initiale } V W Q  \textbf{2} \textit{Majuskel} T  \textbf{5} \textit{Überschrift:} Hie stichet parzifal mit eime templeis Der waz von mvntschevasche ein ritter vomme grale V   $\cdot$ \textit{Initiale} R   $\cdot$ \textit{Majuskel} T  \textbf{21} \textit{Majuskel} T  \newline
\line(1,0){75} \newline
\textbf{1} Al] ALler U Alre V ALler W  $\cdot$ vröude] vreiden U (W)  $\cdot$ dô] \textit{om.} W \textbf{2} er] \textit{om.} Q \textbf{3} ze] gen W  $\cdot$ Munsalvasche] mvnsalvasce T Muntsalvatsche U [muntschal*]: muntschalvasche V montsaluatsch W muntsalfalsche Q \textbf{4} ê habet] e hant U [*]: habent e V e het Q \textbf{5} sol er] sol [her]: er Q er sol R \textbf{6} gegen im kom] kam gen Jm R \textbf{7} man] [mam]: man R \textbf{8} ein] einen T [*in]: Sin V Sein Q (R) \textbf{11} Parcifal] Parcifale U parzifal V partzifal W Q parczifal R \textbf{13} mînes] In mins R  $\cdot$ sus] sos U ausz Q \textbf{14} schiere drumbe] sin schier R \textbf{16} Munsalvasche] mvnsalvasce T Muntsalvatsche U [Mvntschal*]: Mvntschalvache V Montsaluatsch W Muntsalvasche Q Munsalualesche R \textbf{17} ir ieman sô nâhe] ir ieman so [nahe*]: nahe T ieman ir so nahe (nahen W ) U (W) ieman [*]: ir so nahe V yman ir [sultnahe]: sult nahe Q nẏeman so nache R \textbf{18} ez enwære] ez were T Er enwere W Es werre R  $\cdot$ der] \textit{om.} R  $\cdot$ strite] [strite*]: strite T \textbf{19} oder] [O*]: Oder der V Oder der Q R  $\cdot$ alsolhen] al sulche Q also hochen R  $\cdot$ drumbe] \textit{om.} V W Q R \textbf{20} alse man] [A*]: Als man V  $\cdot$ heizet tôt] [*ch]: heisset not V heiset not R \textbf{21} er] \textit{om.} R \textbf{22} des] das R \textbf{23} snüere] svͦre T schnuͯr R \textbf{24} eine scharpfe] ein scharpffen R \textbf{25} niuwe was der] neúwe was das W al newe was der Q was ein núwer R \textbf{27} daz] sin R \textbf{28} ez stuont] [E*]: Ez enstvͦnt V Es enstuͦnd W (Q) \textbf{29} an] In W \textbf{30} sîn dröun] [din]: Sin drovn T  $\cdot$ ouch] \textit{om.} R  $\cdot$ strîten] schreiten Q \newline
\end{minipage}
\end{table}
\end{document}
