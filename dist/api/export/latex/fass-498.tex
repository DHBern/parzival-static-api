\documentclass[8pt,a4paper,notitlepage]{article}
\usepackage{fullpage}
\usepackage{ulem}
\usepackage{xltxtra}
\usepackage{datetime}
\renewcommand{\dateseparator}{.}
\dmyyyydate
\usepackage{fancyhdr}
\usepackage{ifthen}
\pagestyle{fancy}
\fancyhf{}
\renewcommand{\headrulewidth}{0pt}
\fancyfoot[L]{\ifthenelse{\value{page}=1}{\today, \currenttime{} Uhr}{}}
\begin{document}
\begin{table}[ht]
\begin{minipage}[t]{0.5\linewidth}
\small
\begin{center}*D
\end{center}
\begin{tabular}{rl}
\textbf{498} & \begin{large}I\end{large}n mîne herberge er vuor.\\ 
 & vür dise rede ich dicke swuor\\ 
 & manegen ungestabten eit.\\ 
 & dô er \textbf{mich} sô vil an gestreit,\\ 
5 & verholn ich\textbf{z} im \textbf{dô} sagte,\\ 
 & \textbf{des} er vreude vil bejagte.\\ 
 & er gap sîn kleinœde mir;\\ 
 & swaz ich im gap, daz was sîn gir.\\ 
 & \textbf{mîne} kefsen, die dû sæhe ê\\ 
10 & - diu ist noch grüener denne der klê -,\\ 
 & \textbf{hiez ich würken} ûz eime steine,\\ 
 & \textbf{den} \textbf{mir gap} der reine.\\ 
 & sînen neven er mir ze knehte liez,\\ 
 & Ithern, den sîn herze hiez,\\ 
15 & daz aller valsch an im verswant,\\ 
 & den künec von Kukumerlant.\\ 
 & wir mohten vart niht lenger sparn,\\ 
 & wir muosen von ein ander varn.\\ 
 & er kêrte, dâ der bâruc was,\\ 
20 & unt ich \textbf{vuor} vür den Rohas.\\ 
 & ûz \textbf{Zilje} ich vür \textbf{den} Rohas reit;\\ 
 & drî \textbf{mântage} ich dâ vil gestreit.\\ 
 & mich dûhte, ich het \textbf{dâ} wol gestriten.\\ 
 & dar nâch ich schierste kom geriten\\ 
25 & in die wîten Gandine,\\ 
 & dâ nâch der an dîne\\ 
 & Gandin wart genennet.\\ 
 & \textbf{dâ wart Ither} \textbf{bekennet}.\\ 
 & diu selbe stat lît al dâ,\\ 
30 & dâ diu Greian in die Tra,\\ 
\end{tabular}
\scriptsize
\line(1,0){75} \newline
D Fr11 \newline
\line(1,0){75} \newline
\textbf{1} \textit{Initiale} D Fr11  \newline
\line(1,0){75} \newline
\textbf{2} dise] disivͯ Fr11 \textbf{5} im] uͯns Fr11 \textbf{6} vreude] frauͯdn Fr11 \textbf{9} kefsen] chebsen Fr11  $\cdot$ sæhe] sacht Fr11 \textbf{14} Ithern] Jthern D ythern Fr11 \textbf{16} den] der Fr11  $\cdot$ Kukumerlant] Chvnchvmerlant D Kuchuͯmerlant Fr11 \textbf{18} muosen] muͯstn Fr11 \textbf{19} er kêrte] :::chert Fr11 \textbf{20} vuor] \textit{om.} Fr11  $\cdot$ Rohas] roas Fr11 \textbf{21} Zilje] Cylie D Zilie Fr11  $\cdot$ den] \textit{om.} Fr11  $\cdot$ Rohas] roas Fr11 \textbf{22} dâ] \textit{om.} Fr11 \textbf{25} die] divͯ Fr11 \textbf{28} Ither] Jther D \textbf{30} Greian] Grêian D  $\cdot$ Tra] Trâ D \newline
\end{minipage}
\hspace{0.5cm}
\begin{minipage}[t]{0.5\linewidth}
\small
\begin{center}*m
\end{center}
\begin{tabular}{rl}
 & \begin{large}I\end{large}n mîn herberge er vuor.\\ 
 & vür dise rede  dicke swuor\\ 
 & manigen unges\textit{t}abten eit.\\ 
 & dô er \textbf{mich} sô vil \textbf{d\textit{â}} an gestreit,\\ 
5 & verholn ich \textbf{ez} im sagete,\\ 
 & \textbf{daz} er vröuden vil bejagete.\\ 
 & er gap sîn kleinœte mir;\\ 
 & waz ich im gap, daz was sîn gir.\\ 
 & \textbf{mîn} ke\textit{f}sen, die dû sæhe ê\\ 
10 & - diu ist noch grüener dan der klê -,\\ 
 & \textbf{hiez ich wirken} ûz einem stein,\\ 
 & \textbf{den} \textbf{mir gap} der rein.\\ 
 & sînen neven er mir zuo knehte liez,\\ 
 & I\textit{t}hern, den sîn herze \textit{h}iez,\\ 
15 & daz aller valsch an im verswant,\\ 
 & den künic von Kukumerlant.\\ 
 & wir m\textit{o}hten \textbf{die} vart niht langer sparn,\\ 
 & wir muosen von ein ander varn.\\ 
 & er kêrte, d\textit{â} der bâruc was,\\ 
20 & und \textit{i}ch vür den Rohas.\\ 
 & ûz \textbf{Zilje} ich vür \textbf{den} Rohas reit;\\ 
 & drîe \textbf{mântage} ich d\textit{â} vil gestreit.\\ 
 & mich dûhte, ich het \textbf{d\textit{â}} wol gestriten.\\ 
 & dar nâch ich schierste kam geriten\\ 
25 & in die wîten Gandin,\\ 
 & dar nâch der ane dîn\\ 
 & Gandin wart genennet.\\ 
 & \textbf{d\textit{â} wart I\textit{t}her} \textbf{bekennet}.\\ 
 & diu selbe stat lît aldâ,\\ 
30 & d\textit{â} diu Grei\textit{a}n in die \textit{T}ra,\\ 
\end{tabular}
\scriptsize
\line(1,0){75} \newline
m n o \newline
\line(1,0){75} \newline
\textbf{1} \textit{Initiale} m  \newline
\line(1,0){75} \newline
\textbf{3} ungestabten] vngeschaptten m \textbf{4} sô] do so n  $\cdot$ dâ] do m o \textit{om.} n \textbf{5} im] ẏm do o \textbf{9} kefsen] [lesszen]: kesszen m leffczen o \textbf{10} klê] [sne]: cle o \textbf{13} zuo] z: o \textbf{14} Ithern] Jchern m Jthern n o  $\cdot$ hiez] lies m \textbf{16} Kukumerlant] kukumer lant m kucumerlant n kacumerlant o \textbf{17} mohten] moͯhtten m (n) \textbf{18} muosen] muͯssen m (n) o \textbf{19} dâ] do m n o \textbf{20} ich] sich m  $\cdot$ Rohas] baruͯg roas o \textbf{21} Zilje] zilie m n o  $\cdot$ Rohas] roas n \textbf{22} Jch drige mendage vil gestreit n  $\cdot$ ich dâ] jch do m (o) \textbf{23} dûhte] dúchte o  $\cdot$ dâ] do m n o \textbf{25} Gandin] gaudin n \textbf{27} Gandin] Gaudin n \textbf{28} dâ] Do m n o  $\cdot$ Ither] icher m ich ter n iter o \textbf{30} dâ] Do m n  $\cdot$ Greian] greien m greẏan n  $\cdot$ Tra] [r]: cra m cro n cra o \newline
\end{minipage}
\end{table}
\newpage
\begin{table}[ht]
\begin{minipage}[t]{0.5\linewidth}
\small
\begin{center}*G
\end{center}
\begin{tabular}{rl}
 & \begin{large}\textit{I}\end{large}n mîne herberge er vuor.\\ 
 & vür dise rede ich dicke swuor\\ 
 & manigen ungestabeten eit.\\ 
 & dô er \textbf{mich} sô vil an gestreit,\\ 
5 & verholn ich\textbf{z} im \textbf{dô} sagete,\\ 
 & \textbf{de\textit{s}} er vröude vil bejagete.\\ 
 & er gap sî\textit{n} kleinœde mir;\\ 
 & swaz ich im gap, daz was sîn gir.\\ 
 & \textbf{mîne} kefsen, die dû sæhe ê\\ 
10 & - diu ist noch grüener danne der klê -,\\ 
 & \textbf{hiez ich würken} ûz einem steine,\\ 
 & \textbf{den} \textbf{mir gap} der reine.\\ 
 & sînen neven er \textit{m}ir ze knehte liez,\\ 
 & Itheren, de\textit{n} sîn herze hiez,\\ 
15 & daz aller valsch an im verswant,\\ 
 & den künic von Kukumerlant.\\ 
 & wir mohten vart niht lenger sparn,\\ 
 & wir muosen von ein ander varn.\\ 
 & er kêrt, dâ der bâruc was,\\ 
20 & unde ich \textbf{vuor} vür den Roas.\\ 
 & ûz \textbf{Zilje} ich vür Roas reit;\\ 
 & drî \textbf{mântage} ich dâ vil gestreit.\\ 
 & mich dûhte, ich het \textbf{dâ} wol gestriten.\\ 
 & dar nâch ich schierste kom geriten\\ 
25 & in die wîten Gandine,\\ 
 & dâ nâch der ane dîne\\ 
 & Gandin wart genennet.\\ 
 & \textbf{Ither dâ wart} \textbf{erkennet}.\\ 
 & diu selbe stat lît al dâ,\\ 
30 & dâ diu Greian in die Tra,\\ 
\end{tabular}
\scriptsize
\line(1,0){75} \newline
G I L M Z Fr61 \newline
\line(1,0){75} \newline
\textbf{1} \textit{Initiale} G I L Z  \textbf{15} \textit{Initiale} I  \textbf{19} \textit{Initiale} M  \newline
\line(1,0){75} \newline
\textbf{1} In] An G  $\cdot$ mîne] min I  $\cdot$ er] hor M \textbf{2} dise] diseu Fr61  $\cdot$ rede] reise L  $\cdot$ ich] er L \textbf{3} ungestabeten] gestabten I vngeswabten Fr61 \textbf{4} dô] Da M  $\cdot$ gestreit] streit L \textbf{5} ichz] ich M Fr61  $\cdot$ im] ims Fr61  $\cdot$ dô] da M Z \textbf{6} des] der G da von I Das M  $\cdot$ vröude] vreuden I  $\cdot$ bejagete] beiaget G behabete I \textbf{7} sîn] sine G \textbf{8} swaz] Waz L (M) \textbf{9} kefsen] kefse M  $\cdot$ sæhe] sæcht Fr61  $\cdot$ ê] \textit{om.} M \textbf{10} der] ein Fr61 \textbf{12} den] dem I  $\cdot$ gap] gabt I \textbf{13} mir] ir G \textbf{14} Itheren] Jthern I (M) (Fr61) Ihtern L Jchern Z  $\cdot$ den] der G \textbf{15} daz] ÷az I  $\cdot$ an] in L \textbf{16} den] der I (L) (M)  $\cdot$ Kukumerlant] chunchvmerlant G kukumberlant I kvcumer lant L kukuͯmerlant M [Chv*]: Chvncumerlant Z Chumchvmer lant Fr61 \textbf{17} wir] Mir Z  $\cdot$ vart] die vart I  $\cdot$ lenger] lenge Z \textbf{18} muosen] muesen G \textbf{19} kêrt] het L karte M (Z) \textbf{20} vuor vür] fuͯr L (M)  $\cdot$ Roas] Rohas Z Fr61 \textbf{21} Fuͤr zilie ich fuer der walt ist brait Fr61  $\cdot$ Zilje] zilie G L lilie I cicilie M Cilie Z  $\cdot$ Roas] den roas I Rohas Z \textbf{22} dâ vil] vil I vil da Z \textbf{23} dâ] \textit{om.} I M \textbf{24} schierste] sere M schier Fr61 \textbf{25} wîten] wite M  $\cdot$ Gandine] kandein Fr61 \textbf{27} Gandin] Kandein Fr61 \textbf{28} Ither dâ wart] Jther da wart I Da wart Jehter (jther M [ Fr61 ] icher Z ) L (M) (Z) (Fr61)  $\cdot$ erkennet] bekennet L (M) Z (Fr61) \textbf{29} \textit{Vers 498.29 fehlt} M   $\cdot$ diu] Die Fr61  $\cdot$ lît] lach L \textbf{30} diu] der L M  $\cdot$ Greian] grea I gregan M Greẏa Fr61  $\cdot$ Tra] trâ G \newline
\end{minipage}
\hspace{0.5cm}
\begin{minipage}[t]{0.5\linewidth}
\small
\begin{center}*T
\end{center}
\begin{tabular}{rl}
 & In mîne herberge er vuor.\\ 
 & vür dise rede ich dicke swuor\\ 
 & manegen ungestabeten eit.\\ 
 & dô er sô vil an \textbf{mich} gestreit,\\ 
5 & verholne ich im \textbf{dô} sagete,\\ 
 & \textbf{des} er vröuden vil bejagete.\\ 
 & er gap sîne kleinœde mir;\\ 
 & swaz ich im gap, daz was sîn gir.\\ 
 & \textbf{eine} kaf\textit{s}e, die dû sæhe ê\\ 
10 & - diu ist noch grüener danne der klê -,\\ 
 & \textbf{geworht} ûz einem steine.\\ 
 & \textbf{di\textit{e}} \textbf{gap mir} der reine.\\ 
 & sînen neven er mir ze knehte liez,\\ 
 & Ithern, den sîn herze hiez,\\ 
15 & daz aller valsch an im verswant,\\ 
 & de\textit{n} künec von Kukumerlant.\\ 
 & wir mohten \textbf{die} vart niht langer sparn,\\ 
 & wir muosen von ein ander varn.\\ 
 & er kêrte, dâ der bâruc was,\\ 
20 & unde ich \textbf{vuor} vür den Roas.\\ 
 & ûz \textbf{Cecilye} ich vür \textbf{den} Roas reit;\\ 
 & drîe \textbf{morgene} ich dâ vil gestreit.\\ 
 & mich dûhte, ich hete wol gestriten.\\ 
 & dâ nâch \textit{ich} schierest kom geriten\\ 
25 & in die wîten Gandin,\\ 
 & dâ nâch der ane dîn\\ 
 & Gandin wart genennet.\\ 
 & \textbf{dâ wart Ither} \textbf{erkennet}.\\ 
 & \multicolumn{1}{l}{ - - - }\\ 
30 & \multicolumn{1}{l}{ - - - }\\ 
\end{tabular}
\scriptsize
\line(1,0){75} \newline
T U V W O Q R Fr39 \newline
\line(1,0){75} \newline
\textbf{1} \textit{Initiale} W O Q   $\cdot$ \textit{Majuskel} T  \newline
\line(1,0){75} \newline
\textbf{1} \textit{Die Verse 453.1-502.30 fehlen} U   $\cdot$ In] ÷n O  $\cdot$ er] er do Q \textbf{2} rede] reise R  $\cdot$ ich] er O \textbf{3} ungestabeten] vngestalten O \textbf{4} an mich gestreit] mich ane gestreit V (W) (O) (Q) (R) (Fr39) \textbf{5} ich im] iches im V (W) (Fr39) ich imz O ich michs Q \textbf{6} vröuden] frevde O (Q) (R) (Fr39) \textbf{8} swaz] Was W Q (R) \textbf{9} eine] Mine V  $\cdot$ kafse] kaffe T kebsen Q kaffen Fr39 \textbf{10} ist] \textit{om.} W  $\cdot$ der] ein Q \textbf{11} geworht] [*]: Geworht V \textbf{12} die] div T Fr39 [Den]: Die V \textbf{13} er] ê Fr39 \textbf{14} Ithern] Jthern T Ytern V Itheren W Jchern Q Jhtern R yhtern Fr39  $\cdot$ herze] heisse Fr39 \textbf{16} den] der T (W) (R) [Den]: Der  V  $\cdot$ Kukumerlant] kukumer lant W kvcvmerlant O kuͦmuͦerlant R kuchumerlant Fr39 \textbf{17} die] \textit{om.} W O Q R Fr39  $\cdot$ langer] lange R \textbf{18} muosen] mvesen T miessen W \textbf{19} dâ] do V W Q dar do Fr39  $\cdot$ der] er W \textbf{20} ich] auch W \textit{om.} R  $\cdot$ vuor] \textit{om.} O  $\cdot$ vür den] gen W  $\cdot$ Roas] Rohas O Fr39 \textbf{21} ûz] Ausz den Q  $\cdot$ Cecilye] Cecilie V cyle W cilie O (R) (Fr39) cilien Q  $\cdot$ Roas] Rohas O Fr39 \textbf{22} morgene] mentag W (O) (Q) (R) (Fr39)  $\cdot$ dâ] do V W Q Fr39  $\cdot$ gestreit] streit O \textbf{23} hete wol] hette [*l]: do wol V \textbf{24} ich] \textit{om.} T  $\cdot$ schierest kom] kam schier R \textbf{25} Gandin] gandine V \textbf{26} der] dar Q \textbf{27} Gandin] Gaudin W Gandein Q \textbf{28} dâ] Do V W Q (Fr39)  $\cdot$ Ither] Jther T [ytern]: yter V yther W ihther Q Jhter R  $\cdot$ erkennet] bekennet W (O) Q R \textbf{29} \textit{Die Verse 498.29-30 fehlen} T R   $\cdot$ Die selbe stat lit al da V (W) (O) (Q) (Fr39) \textbf{30} Do die [greia*]: greian in die tra V Do die greian (gran O greyan Q Graͤian Fr39 ) in den (die O Q Fr39 ) tra W (O) (Q) (Fr39) \newline
\end{minipage}
\end{table}
\end{document}
