\documentclass[8pt,a4paper,notitlepage]{article}
\usepackage{fullpage}
\usepackage{ulem}
\usepackage{xltxtra}
\usepackage{datetime}
\renewcommand{\dateseparator}{.}
\dmyyyydate
\usepackage{fancyhdr}
\usepackage{ifthen}
\pagestyle{fancy}
\fancyhf{}
\renewcommand{\headrulewidth}{0pt}
\fancyfoot[L]{\ifthenelse{\value{page}=1}{\today, \currenttime{} Uhr}{}}
\begin{document}
\begin{table}[ht]
\begin{minipage}[t]{0.5\linewidth}
\small
\begin{center}*D
\end{center}
\begin{tabular}{rl}
\textbf{108} & \textbf{versigelt} ûfez kriuze \textbf{obeme} grabe.\\ 
 & sus \textbf{sagent} die buochstabe:\\ 
 & "durch disen helm ein tjoste sluoc\\ 
 & den werden, der ellen truoc.\\ 
5 & Gahmuret \textbf{was er} genant,\\ 
 & gewaldec \textbf{künec} über \textbf{driu} lant.\\ 
 & \textbf{ieglîchez} im der krône jach,\\ 
 & dâ \textbf{giengen rîche vürsten} nâch.\\ 
 & er was von Anschouwe \textbf{erborn}\\ 
10 & und hât vor Baldac verlorn\\ 
 & den lîp durch den bâruc.\\ 
 & sîn prîs gap sô hôhen ruc,\\ 
 & niemen reichet an sîn zil,\\ 
 & swâ man \textbf{noch} ritter prüeven wil.\\ 
15 & er ist von muoter ungeborn,\\ 
 & zuo de\textit{m} sîn ellen habe gesworn,\\ 
 & ich meine, der schildes ambet hât.\\ 
 & \textbf{helfe} unt \textbf{menlîchen} rât\\ 
 & gap er mit \textbf{stæte} vriunden sîn.\\ 
20 & er leit durch wîp vil scherpfen pîn.\\ 
 & er truoc den touf unt Kristen ê.\\ 
 & sîn tôt tet Sarrazinen wê,\\ 
 & \textbf{sunder} liegen, \textbf{daz} ist wâr.\\ 
 & sîner zît versunnenlîchiu jâr\\ 
25 & sîn ellen sô nâch prîse warp,\\ 
 & mit ritterlîchem \textbf{prîse} er starp.\\ 
 & er hete \textbf{der} valscheit an gesigt.\\ 
 & nû \textbf{wünschet} im heiles, der hie ligt."\\ 
 & diz was alsô der knappe jach.\\ 
30 & Waleise man \textbf{vil} weinen sach.\\ 
\end{tabular}
\scriptsize
\line(1,0){75} \newline
D Fr33 \newline
\line(1,0){75} \newline
\textbf{29} \textit{Initiale} Fr33  \newline
\line(1,0){75} \newline
\textbf{1} ûfez] vfme Fr33 \textbf{2} sagent] sageten Fr33 \textbf{5} Der was Gamuͦret genant Fr33  $\cdot$ Gahmuret] Gahmvret D \textbf{7} ieglîchez] Jrgelicher Fr33  $\cdot$ krône] cronen Fr33 \textbf{9} Anschouwe] Anscoͮwe D Angiowe Fr33  $\cdot$ erborn] geborn Fr33 \textbf{10} Baldac] Baldach D \textbf{11} den lîp] Sinen lip Fr33 \textbf{13} reichet] reit Fr33 \textbf{16} dem] den D \textbf{18} menlîchen] menlich Fr33 \textbf{21} Kristen] christen D cristen Fr33 \textbf{22} Sarrazinen] Zarrazinen D \textbf{23} daz] [da*]: daz D \textbf{24} versunnenlîchiu] virswendlichiv Fr33 \textbf{26} ritterlîchem] ritterlichen Fr33  $\cdot$ er starp] irstarp Fr33 \textbf{30} Waleise] Waleisen Fr33 \newline
\end{minipage}
\hspace{0.5cm}
\begin{minipage}[t]{0.5\linewidth}
\small
\begin{center}*m
\end{center}
\begin{tabular}{rl}
 & \textbf{versigelt} ûf daz kriuze \textbf{ob dem} grabe.\\ 
 & sus \textbf{sagent} die buochstabe:\\ 
 & "\begin{large}D\end{large}urch disen helm ein juste sluoc\\ 
 & den werden \textbf{degen}, der ellen truoc.\\ 
5 & Gahmuret \textbf{was er} genant,\\ 
 & gewaltic \textbf{künic} über \textbf{alliu} lant.\\ 
 & \textbf{iegelîcher} ime der krône jach,\\ 
 & dâ \textbf{rîche vürsten giengen} nâch.\\ 
 & er was von Anschouwe \textbf{geborn}\\ 
10 & und hât vor Baldac verlorn\\ 
 & den lîp durch den bâruc.\\ 
 & sîn prîs gap sô hôhen ruc,\\ 
 & \textbf{daz} niemen re\textit{i}chte an sîn zil,\\ 
 & wâ man \textbf{noch} ritter brüefen wil.\\ 
15 & er ist von muoter ungeborn,\\ 
 & zuo dem sîn elle\textit{n} habe gesworn,\\ 
 & ich meine, der schiltes ambet hât.\\ 
 & \textbf{helfe} und \textbf{manlîchen} rât\\ 
 & \textit{g}a\textit{p} er mit \textbf{strîte} vriunden sîn.\\ 
20 & er l\textit{ei}t durch wîp vil scharpfe pîn.\\ 
 & er truoc den touf und Kristen ê.\\ 
 & sîn tôt tet Sarrazinen wê,\\ 
 & \textbf{sunder} liegen, \textbf{daz} ist wâr.\\ 
 & sîner zît versunnenlîchiu jâr\\ 
25 & sîn ellen sô nâch prîse warp,\\ 
 & mit ritterlîchem \textbf{ende} er starp.\\ 
 & er hete \textbf{der} valscheit an gesig\textit{e}t.\\ 
 & nû \textbf{wünsch} ime heiles, der hie lig\textit{e}t."\\ 
 & \begin{large}D\end{large}iz was alsô der knappe jach.\\ 
30 & Waleise man \textbf{vil} weinen sach.\\ 
\end{tabular}
\scriptsize
\line(1,0){75} \newline
m n o \newline
\line(1,0){75} \newline
\textbf{3} \textit{Initiale} m n  \textbf{29} \textit{Initiale} m o   $\cdot$ \textit{Capitulumzeichen} n  \newline
\line(1,0){75} \newline
\textbf{1} kriuze] rucz o \textbf{3} disen] dise o \textbf{4} der] den n \textbf{5} Gahmuret] Gamiret n Gamuͯret o \textbf{6} alliu] dis n die o \textbf{8} dâ] Do n  $\cdot$ vürsten] fúrsten jme n \textbf{9} Anschouwe] an schouwe n anschowe o \textbf{10} hât] hette n  $\cdot$ Baldac] baldack m baldag n o \textbf{13} reichte] rechte m reichet n o \textbf{14} brüefen] prisen n \textbf{15} ungeborn] hoch geborn o \textbf{16} ellen] ellende m \textbf{19} gap] Das m \textbf{20} leit] [lipt]: liet m warp leit o  $\cdot$ vil] wil o \textbf{22} tet] der n  $\cdot$ Sarrazinen] sarrasinen m sarazúnenn o \textbf{24} versunnenlîchiu] von sinneclichem n von symelichen o \textbf{27} hete der valscheit] hatt den falcheyt o  $\cdot$ gesiget] gesigent m \textbf{28} wünsch] wunschet n (o)  $\cdot$ heiles] [heln]: heliles o  $\cdot$ liget] ligent m \textbf{30} Waleise] Walleise m  $\cdot$ man vil] vil man n o \newline
\end{minipage}
\end{table}
\newpage
\begin{table}[ht]
\begin{minipage}[t]{0.5\linewidth}
\small
\begin{center}*G
\end{center}
\begin{tabular}{rl}
 & \textbf{versigelt} ûfez k\textit{riu}ze \textbf{ûf dem} grabe.\\ 
 & sus \textbf{sageten} die buochstabe:\\ 
 & "\begin{large}D\end{large}urch disen helm ein tjoste sluoc\\ 
 & den werden, der ellen truoc.\\ 
5 & Gahmuret \textbf{er was} genant,\\ 
 & gewaltic über \textbf{driu} lant.\\ 
 & \textbf{ieslîchez} im der krône jach,\\ 
 & dâ \textbf{giengen rîche vürsten} nâch.\\ 
 & er was von Anschouwe \textbf{geboren}\\ 
10 & unde hât vor Baldac verloren\\ 
 & den lîp durch den bâruc.\\ 
 & sîn brîs gap sô hôhen ruc,\\ 
 & \textit{n}iemen reichet an sîn zil,\\ 
 & swâ man \textbf{nû} rîter brüeven wil.\\ 
15 & er ist von muoter ungeboren,\\ 
 & zuo dem sîn ellen habe gesworen,\\ 
 & ich meine, der schiltes ambet hât.\\ 
 & \textbf{helfe} und \textbf{manlîch} rât\\ 
 & gap er mit \textbf{stæte} \textbf{den} vriunden sîn.\\ 
20 & er leit durch wîp vil scharfen pîn.\\ 
 & er truoc den touf und Kristen ê.\\ 
 & sîn tôt tet Sarrazinen wê,\\ 
 & \textbf{âne} liegen, \textbf{daz} ist wâr.\\ 
 & sîner zît versunniclîchiu jâr\\ 
25 & sîn ellen sô nâch prîse warp,\\ 
 & mit rîterlîchem \textbf{prîse} er starp.\\ 
 & e\textit{r} hete valscheit an gesiget.\\ 
 & \textit{nû} \textbf{wünschet} im heiles, d\textit{e}r hie liget."\\ 
 & diz was als der knappe jach.\\ 
30 & \textbf{vil} Waleise man \textbf{dâ} weinen sach.\\ 
\end{tabular}
\scriptsize
\line(1,0){75} \newline
G I O L M Q R Z \newline
\line(1,0){75} \newline
\textbf{1} \textit{Initiale} O  \textbf{3} \textit{Initiale} G  \textbf{5} \textit{Initiale} M  \textbf{9} \textit{Initiale} I  \newline
\line(1,0){75} \newline
\textbf{1} versigelt] ÷er sigelt O  $\cdot$ ûfez] vff einem Q vf dem Z  $\cdot$ kriuze] chvrze G  $\cdot$ ûf dem] [vb]: ob dem I ob dem O (L) (Q) Z obnan R  $\cdot$ grabe] graben R \textbf{2} sageten] sagent I (O) L (M) (Q) R  $\cdot$ buochstabe] buͤchstabn I (Q) (R) \textbf{3} disen] den L disem R [disem]: disen Z \textbf{4} den werden] Den werden helt L Der werden Q  $\cdot$ der] der ie I  $\cdot$ ellen] eren Q \textbf{5} Gahmuret] Gamvret O Z GAmuret M (Q) Z Gahmuͯret L  $\cdot$ er was] was er I O (M) Q (R) Z  $\cdot$ genant] genans Q \textbf{6} gewaltic] Gewaltich chvnich O (L) (M) (Q) (R) (Z)  $\cdot$ über driu] drier O M (Q) \textbf{7} ieslîchez] Jeslicher O  $\cdot$ krône] cronen M \textbf{8} dâ] Jm L Do Q  $\cdot$ rîche] dry M reichen Q \textbf{9} er] Der I Es Q  $\cdot$ Anschouwe] anschoͮwe G antschau I anschowe O (L) M R anshowe Q Z \textbf{10} hât] hatte M  $\cdot$ Baldac] baldach G (O) (L) Baldec R \textbf{11} durch] \textit{om.} O \textbf{12} brîs] lib R \textbf{13} niemen] daz niemen G  $\cdot$ reichet] richeit O reichte M Q (R) (Z) \textbf{14} swâ] Da L Wa M (Q) R  $\cdot$ nû] noch I nach Z \textbf{15} ungeboren] geboren Q \textbf{16} ellen habe] eren haben Q \textbf{17} ambet] annucht M \textbf{18} manlîch] manlichen O L (Z) maneclichin M nemliche Q manliche R  $\cdot$ rât] tat R \textbf{19} stæte den] steten I \textbf{20} vil] die L  $\cdot$ scharfen] starcke Q scharpfe R (Z) \textbf{21} truoc] \textit{om.} M  $\cdot$ den] die Q  $\cdot$ touf] toiffe M  $\cdot$ und] in L  $\cdot$ Kristen] christen G O cristen I M Q Z Cristes L cristan R \textbf{22} Sarrazinen tet sin sterben we O  $\cdot$ Den sarracenyn fet sein sterben we Q  $\cdot$ Sarrazinen] sarazinen G den sarrazinen I den saraszinen L sarraczinen M sarrasinen R \textbf{23} daz ist] ist ez I diz ist O M (Q) (Z) \textbf{24} Sie begant siuzit svmeliche iar L  $\cdot$ sîner] siniu I  $\cdot$ versunniclîchiu] verswundenlichiu I was er versvͦnlichiv O versinnliche R versvmeliche Z \textbf{25} ellen] ere Q  $\cdot$ sô] \textit{om.} R  $\cdot$ warp] wap L \textbf{26} rîterlîchem] Rittenlichem R \textbf{27} er] e G  $\cdot$ hete] hat R  $\cdot$ valscheit] der falscheit Q \textbf{28} nû] \textit{om.} G  $\cdot$ der] da er G \textbf{29} diz] daz I  $\cdot$ als] \textit{om.} R  $\cdot$ knappe] knappen R \textbf{30} vil Waleise] walæis I Waleise vil O Walayse L Zu waleise Q Walose Z  $\cdot$ dâ] da vil I \textit{om.} O vil L M Q R Z \newline
\end{minipage}
\hspace{0.5cm}
\begin{minipage}[t]{0.5\linewidth}
\small
\begin{center}*T (U)
\end{center}
\begin{tabular}{rl}
 & \textbf{b\textit{e}s\textit{i}gelt} ûf daz kriuze \textbf{ûf dem} grabe.\\ 
 & sus \textbf{sageten} die buochstab\textit{e}:\\ 
 & "durch disen helm ein jost sluoc\\ 
 & den werden, der ellen truoc.\\ 
5 & Gahmuret \textbf{was er} genant,\\ 
 & gewaltic über \textbf{driu} lant.\\ 
 & \textbf{ieclîchez} im der krône jach,\\ 
 & dâ \textbf{giengen rîche vürsten} nâch.\\ 
 & er was von Anschouwe \textbf{geborn}\\ 
10 & und het vor Baldac verlor\textit{n}\\ 
 & den lîp durch den bâruc.\\ 
 & sîn prîs gap sô hôhen ruc,\\ 
 & nieman reichet an sîn zil,\\ 
 & wâ man \textbf{noch} rîter prüeven wil.\\ 
15 & er ist von muoter ungeborn,\\ 
 & zuo dem sîn ellen habe gesworn,\\ 
 & ich meine, der schiltes ambet hât.\\ 
 & \textbf{triuwe} und \textbf{manlîchen} rât\\ 
 & gap er mit \textbf{stæte} \textbf{den} v\textit{r}i\textit{un}den sîn.\\ 
20 & er leit durch wîp vil scharpfe pîn.\\ 
 & er truoc den touf und Kristen ê.\\ 
 & sîn tôt tet Sarrazinen wê,\\ 
 & \textbf{âne} liegen, \textbf{diz} ist wâr.\\ 
 & sîner zît vers\textit{u}nne\textit{n}lîchiu jâr\\ 
25 & sîn ellen sô nâch prîse warp,\\ 
 & mit ritterlîchem \textbf{prîse} er starp.\\ 
 & er hete valscheit an gesiget.\\ 
 & nû \textbf{wünschet} im heiles, der hie liget."\\ 
 & diz was als der knappe jach.\\ 
30 & Waleisen man \textbf{vil} weinen sach.\\ 
\end{tabular}
\scriptsize
\line(1,0){75} \newline
U V W T \newline
\line(1,0){75} \newline
\textbf{3} \textit{Majuskel} T  \newline
\line(1,0){75} \newline
\textbf{1} versigelt vf crvce vnd vf grap T  $\cdot$ besigelt] Bosegelt U  $\cdot$ ûf dem] obe dem V (W) \textbf{2} svs saget vns der bvͦchstap T  $\cdot$ sageten] sprachen W  $\cdot$ buochstabe] buͦch staben U \textbf{3} helm] schluͦg W  $\cdot$ jost] helm T \textbf{4} der] den der T  $\cdot$ ellen] ie ellen V \textbf{5} Gahmuret] Gahmuͦret U Gamuret V W \textbf{6} gewaltic] gewaltig [*]: kv́nig V gewaltic kvnec T \textbf{7} ieclîchez] iesliches T  $\cdot$ im der krône] der kron im W \textbf{8} dâ] do V (W) \textbf{9} Anschouwe] Anschowe U V antschowe W Anschoͮwe T \textbf{10} het] hat W T  $\cdot$ Baldac] Baldag V haldac W  $\cdot$ verlorn] verlor U \textbf{12} ruc] zug V W \textbf{13} reichet] [rehte]: reihte T \textbf{14} wâ] swa V T  $\cdot$ noch] nv T \textbf{18} helfe vnd manliche tât T \textbf{19} vriunden] vreiden U (W) \textbf{20} wîp] vroͮwen T  $\cdot$ vil scharpfe] vil scharpfen V manegen T \textbf{21} Kristen] cristen U V T \textbf{22} Sarrazinen] den Sarrazinen V den sarazinen W \textbf{23} diz] das W \textbf{24} versunnenlîchiu] versinneliche U versumecliche V versvmelichiv T \textbf{25} sô] \textit{om.} T \textbf{26} er] \textit{om.} W \textbf{27} hete] hat T \textbf{28} heiles] heller W \textbf{30} Waleisen] [*]: walleise V Waleysen W waleise T  $\cdot$ vil] do vil W \newline
\end{minipage}
\end{table}
\end{document}
