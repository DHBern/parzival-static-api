\documentclass[8pt,a4paper,notitlepage]{article}
\usepackage{fullpage}
\usepackage{ulem}
\usepackage{xltxtra}
\usepackage{datetime}
\renewcommand{\dateseparator}{.}
\dmyyyydate
\usepackage{fancyhdr}
\usepackage{ifthen}
\pagestyle{fancy}
\fancyhf{}
\renewcommand{\headrulewidth}{0pt}
\fancyfoot[L]{\ifthenelse{\value{page}=1}{\today, \currenttime{} Uhr}{}}
\begin{document}
\begin{table}[ht]
\begin{minipage}[t]{0.5\linewidth}
\small
\begin{center}*D
\end{center}
\begin{tabular}{rl}
\textbf{120} & dar nâch sîn snelheit verre spranc.\\ 
 & er lernte \textbf{den} \textbf{gabilôtes} swanc,\\ 
 & dâ mit er manegen hirz erschôz,\\ 
 & des sîn muoter unt ir volc genôz.\\ 
5 & ez wære \textbf{æber} oder \textbf{snê},\\ 
 & dem wilde tet sîn schiezen wê.\\ 
 & nû hœret vremdiu mære:\\ 
 & swenne er \textbf{schôz} daz swære,\\ 
 & \textbf{des} \textbf{wære ein mûl} geladen genuoc,\\ 
10 & \textbf{als} \textbf{unzerworht} \textbf{hin} heim erz truoc.\\ 
 & \begin{large}E\end{large}ines tages gie er den weideganc\\ 
 & an \textbf{einer} halden, diu was lanc.\\ 
 & er brach durch \textbf{blates} stimme ein zwîc.\\ 
 & \textbf{dâ nâhen bî im gie} ein stîc.\\ 
15 & dâ \textbf{hôr\textit{t}e}r schal von huofslegen.\\ 
 & sîn gabilôt begunder wegen.\\ 
 & dô sprach er: "waz hân ich vernomen?\\ 
 & \textbf{wan} wolt \textbf{êt} nû der tiuvel komen\\ 
 & mit grimme \textbf{zorneclîche},\\ 
20 & \textbf{den} bestüende \textbf{ich} sicherlîche.\\ 
 & mîn muoter vreisen von im sagt.\\ 
 & ich wæne, \textbf{ir ellen} sî verzagt."\\ 
 & Alsus stuont er in strîtes ger.\\ 
 & nû seht, dort \textbf{kom} \textbf{geschûftet} her\\ 
25 & \textbf{ritter} nâch wunsche var,\\ 
 & von vuoz ûf gewâpent gar.\\ 
 & der k\textit{n}appe wânde sunder spot,\\ 
 & daz ieslîcher wære ein got.\\ 
 & Dô stuont \textbf{ouch er} niht langer hie:\\ 
30 & in \textbf{daz} pfat viel er ûf \textbf{sîniu} knie.\\ 
\end{tabular}
\scriptsize
\line(1,0){75} \newline
D \newline
\line(1,0){75} \newline
\textbf{11} \textit{Initiale} D  \textbf{23} \textit{Majuskel} D  \textbf{29} \textit{Majuskel} D  \newline
\line(1,0){75} \newline
\textbf{15} hôrter] horer \textit{nachträglich korrigiert zu:} horter D \textbf{27} knappe] kappe D \newline
\end{minipage}
\hspace{0.5cm}
\begin{minipage}[t]{0.5\linewidth}
\small
\begin{center}*m
\end{center}
\begin{tabular}{rl}
 & dar nâch sîn snelheit verre spranc.\\ 
 & er lernte \textbf{den} \textbf{gabilôtes} swanc,\\ 
 & dâ mite er manigen hirz erschôz,\\ 
 & des sîn muoter und ir volc genôz.\\ 
5 & ez wære \textbf{regen} oder \textbf{snê},\\ 
 & dem wilde tet sîn schiezen wê.\\ 
 & nû hœret vremdiu mære:\\ 
 & wenne er \textbf{erschôz} daz swære,\\ 
 & \textbf{des} \textbf{ein m\textit{û}l wære} geladen genuoc,\\ 
10 & \textbf{als} \textbf{u\textit{n}ervorht} \textbf{hie} heim erz truoc.\\ 
 & \begin{large}E\end{large}ines tages gienc er den weideganc\\ 
 & an \textbf{einer} halden, diu was lanc.\\ 
 & er brach durch \textbf{blates} stimme ein zwîc.\\ 
 & \textbf{d\textit{â} nâhen gien\textit{c} bî ime} ein stîc.\\ 
15 & d\textit{â} \textbf{h\textit{ô}rt} er schal von huofslegen.\\ 
 & sîn gabilôt begunde er wegen.\\ 
 & dô sprach er: "waz hân ich vernomen?\\ 
 & \textbf{wanne} wolte \textbf{eht} nû der tiuvel komen\\ 
 & mit grimme \textbf{zorneclîche},\\ 
20 & \textbf{ich} bestüende \textbf{in} sicherlîche.\\ 
 & mîn muoter vreis\textit{en} vo\textit{n} ime saget.\\ 
 & ich wæne, \textbf{ir ellen} sî verzaget."\\ 
 & alsus stuont er in strîtes ger.\\ 
 & nû seht, dort \textbf{komet} \textbf{geschi\textit{u}ftet} her\\ 
25 & \textbf{drîe ritter} nâch \textbf{dem} wunsche var,\\ 
 & von vuoze ûf gewâpent gar.\\ 
 & der knappe wânde sunder spot,\\ 
 & daz ieglîcher wære ein got.\\ 
 & dô stuond\textbf{er ouch} niht langer hie:\\ 
30 & in \textbf{daz} pfat viel er ûf \textbf{diu} knie\\ 
\end{tabular}
\scriptsize
\line(1,0){75} \newline
m n o \newline
\line(1,0){75} \newline
\textbf{11} \textit{Initiale} m   $\cdot$ \textit{Capitulumzeichen} n  \newline
\line(1,0){75} \newline
\textbf{2} lernte den] lerte den n lerte >den< o \textbf{7} hœret] erhoͯrent n \textbf{9} des] Das n o  $\cdot$ mûl] mal m  $\cdot$ geladen] lebende n \textbf{10} unervorht] [vns]: vncz erforcht m  $\cdot$ hie heim erz] ers heim n (o) \textbf{14} dâ] Do m n o  $\cdot$ gienc bî ime] ginge bi ime m by jme ging n (o) \textbf{15} dâ] Do m n o  $\cdot$ hôrt] hert m  $\cdot$ von] vor n \textbf{18} nû] jme n (o) \textbf{21} vreisen von] freisam vom m \textbf{22} ellen sî] wellen sin n \textbf{23} alsus] Also n o \textbf{24} komet] kam n o  $\cdot$ geschiuftet] geschiffttet m geschefftet n gescheffet o \textbf{29} niht] ni n \textbf{30} daz] den n o  $\cdot$ diu] sine n (o) \newline
\end{minipage}
\end{table}
\newpage
\begin{table}[ht]
\begin{minipage}[t]{0.5\linewidth}
\small
\begin{center}*G
\end{center}
\begin{tabular}{rl}
 & dar nâch sîn snelheit verre spranc.\\ 
 & er lernete \textbf{den} \textbf{gabilôtes} swanc,\\ 
 & dâ mit er manigen hirz erschôz,\\ 
 & des \textit{sîn muoter} und \textit{ir} volc genôz.\\ 
5 & ez wære \textbf{æber} oder \textbf{snê},\\ 
 & dem wilde tet sîn schiezen wê.\\ 
 & nû hœret vrömdiu mære:\\ 
 & swenner \textbf{erschôz} daz swære,\\ 
 & \textbf{es} \textbf{wære ein mûl} geladen genuoc,\\ 
10 & \textbf{als} \textbf{unzerworht} \textbf{hin} heim erz truoc.\\ 
 & eines tages gienger den weideganc\\ 
 & an \textbf{eine} halden, diu was lanc.\\ 
 & er brach durch \textbf{blate} stimme ein zwîc.\\ 
 & \textbf{bî im nâhen gienc} ein stîc.\\ 
15 & dâ \textbf{hôrt} er schal von huofslegen.\\ 
 & sîn gabilôt begunder wegen.\\ 
 & dô sprach er: "waz hân ich vernomen?\\ 
 & \textbf{wan} wolt \textbf{êt} nû der tiuvel komen\\ 
 & mit grimme, \textbf{zornes rîche},\\ 
20 & \textbf{ich} bestüende \textbf{in} sicherlîche.\\ 
 & mîn muoter vreise von im saget.\\ 
 & ich wæne, \textbf{si ellens} sî verzaget."\\ 
 & alsus stuont er in strîtes ger.\\ 
 & nû seht, dort \textbf{kom} \textbf{geschûftet} her\\ 
25 & \textbf{\textit{drî} rîter} nâch wunsche var,\\ 
 & von vuoze ûf gewâpent gar.\\ 
 & \begin{large}D\end{large}er knappe wânde sunder spot,\\ 
 & daz ieslîcher wære ein got.\\ 
 & dô stuont \textbf{ouch \textit{er}} niht lenger hie:\\ 
30 & in \textbf{dem} pfade viel er ûf \textbf{sîniu} knie.\\ 
\end{tabular}
\scriptsize
\line(1,0){75} \newline
G I O L M Q R Z Fr36 \newline
\line(1,0){75} \newline
\textbf{1} \textit{Initiale} I O  \textbf{7} \textit{Initiale} R  \textbf{11} \textit{Initiale} L Q R Z Fr36  \textbf{17} \textit{Initiale} I  \textbf{25} \textit{Überschrift mit Illustration:} wie dry Ritter gen Im Rittent R  \textbf{27} \textit{Initiale} G  \newline
\line(1,0){75} \newline
\textbf{1} dar] ÷ar O  $\cdot$ sîn] \textit{om.} Z  $\cdot$ verre] vor M  $\cdot$ spranc] erspranc I \textbf{2} er] Der M  $\cdot$ lernete] lernt I O Z  $\cdot$ den] des I (R)  $\cdot$ swanc] sang R \textbf{3} hirz] herz M hirzen Q \textbf{4} des er vnd sin volch genoz G  $\cdot$ Des er vnde sin mvͦter wol genoz O  $\cdot$ volc] gesind R \textbf{5} wære] weren R  $\cdot$ snê] re L R \textbf{6} wilde] gewilde R  $\cdot$ schiezen] hitze Q \textbf{7} vrömdiu] ein froͤmdez I wilde Q \textbf{8} swenner] Wenne er L (Q) R  $\cdot$ erschôz] Geshoz I schoz O L (M) (R) \textbf{9} des ein muͤl wer [v]: Geladen genuͤc I  $\cdot$ es] Des O (L) M Q Z (Fr36) \textbf{10} Als vnzerworcht truͦg ers heim gnuͦg R  $\cdot$ unzerworht] vns erworcht M vnzerborhtz Fr36  $\cdot$ hin heim erz] erz hin heim O er daz heim L hin heym es Q \textbf{11} tages] morgens R  $\cdot$ gienger den weideganc] ging der wigant Q gieng er der weide gang R \textbf{12} eine halden] einer halden L einer heide Z \textbf{13} durch] durch durch R  $\cdot$ blate] blattes L (Z) \textbf{14} nâhen] nahm Q \textbf{15} dâ] Do Q R  $\cdot$ hôrt er] was ein O her horte M erhort er Z \textbf{16} sîn] Jm Q \textbf{17} dô] Da O M Z \textit{om.} L  $\cdot$ sprach er] Er sprach L doch er Q \textbf{18} êt] \textit{om.} M Q Z \textbf{19} zornes rîche] zornicliche Z \textbf{20} ich bestüende in] Den bestvͦnde ich O (L) (M) (Q) (R) (Z) \textbf{21} muoter] \textit{om.} O  $\cdot$ vreise] freisen Z \textbf{22} si ellens] er ellens I ir ellent O (L) (M) (R) (Z) ir eren Q \textbf{23} alsus] als I  $\cdot$ er in] er O irs M \textbf{24} kom] chomen I O (M) (Q) (R)  $\cdot$ geschûftet] Geshufet I geschuͯpfet L gesticht Q \textit{om.} R \textbf{25} drî] zwene G  $\cdot$ wunsche] wuncsche I wensche L  $\cdot$ var] Geuar I gar R \textbf{26} von] Von dem O  $\cdot$ vuoze] vuͤzzen I (Q)  $\cdot$ gar] var R \textbf{27} sunder] svnder svnder O sundern M \textbf{28} daz] daz ir I \textit{om.} L \textbf{29} dô] Daen M Do ne Q Darumb R  $\cdot$ ouch] \textit{om.} O L M Q R  $\cdot$ er] \textit{om.} G \textbf{30} in] An M  $\cdot$ dem] den L dē M  $\cdot$ ûf sîniu] vur si an diu I [*]: er nider vf div O an die L uff dy M (Q) \newline
\end{minipage}
\hspace{0.5cm}
\begin{minipage}[t]{0.5\linewidth}
\small
\begin{center}*T (U)
\end{center}
\begin{tabular}{rl}
 & dar nâch sîn snelheit verre spranc.\\ 
 & er lernete \textbf{gabylôten} swanc,\\ 
 & dâ mit er manegen hirz erschôz,\\ 
 & des sîn muoter und ir volc genôz.\\ 
5 & ez wære \textbf{eber} oder \textbf{rê},\\ 
 & dem wilde tet sîn schiezen wê.\\ 
 & nû hœret vremediu mære:\\ 
 & wan er \textbf{erschôz} daz swære,\\ 
 & \textbf{ez} \textbf{wære eime mûl} geladen genuoc,\\ 
10 & \textbf{unzerbrochen} \textbf{hin} heim erz truoc.\\ 
 & eines tages gienc er den weideganc\\ 
 & an \textbf{eine} halde, diu was lanc.\\ 
 & er brach durch \textbf{blate} stimme ein zwîc.\\ 
 & \textbf{bî im nâhe gienc} ein stîc.\\ 
15 & dâ \textbf{erhôrt} er schal von \textit{h}uofslegen.\\ 
 & sîn gabilôt begunde er we\textit{g}en.\\ 
 & dô sprach er: "waz hân ich vernomen?\\ 
 & wolte nû der tiuvel komen\\ 
 & mit grimme, \textbf{zornes rîche},\\ 
20 & \textbf{den} bestüende \textbf{ich} sicherlîche.\\ 
 & mîn muoter vreisen von im saget.\\ 
 & ich wæne, \textbf{ir ellen} sî verzaget."\\ 
 & alsus stuont er in strîtes ger.\\ 
 & nû seht, dort \textbf{kômen} \textbf{drî rîter} her,\\ 
25 & \textbf{die wâren} nâch wunsche var,\\ 
 & von vuoze ûf gewâpent gar.\\ 
 & der knabe wânde sunder spot,\\ 
 & daz ieclîcher wære ein got.\\ 
 & dô stuont \textbf{ouch er} niht langer hie:\\ 
30 & in \textbf{den} pfat viel er ûf \textbf{sîniu} knie.\\ 
\end{tabular}
\scriptsize
\line(1,0){75} \newline
U V W T \newline
\line(1,0){75} \newline
\textbf{1} \textit{Majuskel} T  \textbf{3} \textit{Majuskel} T  \textbf{7} \textit{Initiale} T  \textbf{11} \textit{Initiale} V W   $\cdot$ \textit{Majuskel} T  \textbf{24} \textit{Majuskel} T  \textbf{29} \textit{Majuskel} T  \newline
\line(1,0){75} \newline
\textbf{1} snelheit] snelliv T  $\cdot$ verre spranc] sere rang W \textbf{2} gabylôten] den [gabelote*]: gabelotes V ein gabiloten W den gabylotes T \textbf{4} sîn muoter und ir] er vnde sin mvͦter T \textbf{6} sîn schiezen] er vil W \textbf{7} nû] Hie W \textbf{8} wan er] Swen er V alser T  $\cdot$ erschôz] schoz V icht erschos W  $\cdot$ swære] was schwere W \textbf{9} ez] des T  $\cdot$ eime] ein V T  $\cdot$ geladen] gewesen W \textbf{10} unzerbrochen hin heim erz] [a*]: alvnzerworht hin heim ers V All vneruorcht ers haim W al vn zerworht erz heim T \textbf{11} tages] mals W \textbf{12} eine] eime W  $\cdot$ halde] halden T \textbf{13} durch] zer T  $\cdot$ blate] blatens W \textbf{15} dâ] Do U V W  $\cdot$ erhôrt er] hort er V W (T)  $\cdot$ huofslegen] vfslegen U \textbf{16} sîn] Sinen V Seine W  $\cdot$ begunde] gebond V  $\cdot$ wegen] wenden U \textbf{18} wolte] Wan wolt eht V (W) wan wolte T \textbf{19} grimme] grimes V sinem T  $\cdot$ zornes] zorne T \textbf{20} den] dem T \textbf{21} vreisen] freise W \textbf{22} wæne] wone W  $\cdot$ ellen] ellend W  $\cdot$ sî] sit T \textbf{24} Nun sach er dort komen geritten her W Nv seht dort cômen geriten her T  $\cdot$ kômen] [k*ment]: koment V \textbf{25} die wâren] Drei ritter W (T) \textbf{26} von vuoze] Von fvͦzen V Vom fuͦß W \textbf{28} daz] ir T \textbf{29} ouch er] er V T auch W \textbf{30} den] dem T  $\cdot$ sîniu] die T \newline
\end{minipage}
\end{table}
\end{document}
