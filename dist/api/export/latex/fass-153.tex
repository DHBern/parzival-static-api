\documentclass[8pt,a4paper,notitlepage]{article}
\usepackage{fullpage}
\usepackage{ulem}
\usepackage{xltxtra}
\usepackage{datetime}
\renewcommand{\dateseparator}{.}
\dmyyyydate
\usepackage{fancyhdr}
\usepackage{ifthen}
\pagestyle{fancy}
\fancyhf{}
\renewcommand{\headrulewidth}{0pt}
\fancyfoot[L]{\ifthenelse{\value{page}=1}{\today, \currenttime{} Uhr}{}}
\begin{document}
\begin{table}[ht]
\begin{minipage}[t]{0.5\linewidth}
\small
\begin{center}*D
\end{center}
\begin{tabular}{rl}
\textbf{153} & "Got weiz, hêr scheneschalt,\\ 
 & daz Cunneware \textbf{de Lalant}\\ 
 & durch den knappen ist zerbert.\\ 
 & iwer vreude \textbf{es} wirt verzert\\ 
5 & noch von sîner hende,\\ 
 & er\textbf{n} \textbf{sî} nie sô ellende."\\ 
 & "Sît iwer êrste rede mir dröut,\\ 
 & ich wæne, irs wênic iuch gevröut."\\ 
 & sîn brât wart gealûnet,\\ 
10 & mit slegen \textbf{vil} gerûnet\\ 
 & dem \textbf{witzehaftem} tôren\\ 
 & mit viusten in sîn ôren.\\ 
 & daz tet Keie sunder twâl.\\ 
 & Dô muose der junge Parzival\\ 
15 & disen kumber schouwen,\\ 
 & Anthanors unt der vrouwen.\\ 
 & im was \textit{v}on herzen leit ir nôt.\\ 
 & vil dicker greif zem gabilôt.\\ 
 & \textbf{Vor der künegîn was} sölch gedranc,\\ 
20 & daz er durch daz vermeit den swanc.\\ 
 & urloup nam dô Iwanet\\ 
 & ze\textbf{m} fillu roy Gahmuret.\\ 
 & \textit{\begin{large}D\end{large}}es reise al eine wart getân\\ 
 & hin ûz gein Ither ûf den plân.\\ 
25 & dem saget er sölhiu mære,\\ 
 & daz \textbf{niemen dinne} wære.\\ 
 & \multicolumn{1}{l}{ - - - }\\ 
 & \multicolumn{1}{l}{ - - - }\\ 
 & "Ich sagte, \textbf{als} dû mir \textbf{verjæhe},\\ 
30 & wie ez âne danc geschæhe,\\ 
\end{tabular}
\scriptsize
\line(1,0){75} \newline
D \newline
\line(1,0){75} \newline
\textbf{1} \textit{Majuskel} D  \textbf{7} \textit{Majuskel} D  \textbf{14} \textit{Majuskel} D  \textbf{19} \textit{Majuskel} D  \textbf{23} \textit{Initiale} D  \textbf{29} \textit{Majuskel} D  \newline
\line(1,0){75} \newline
\textbf{13} Keie] kaye D \textbf{17} von] won D \textbf{21} Iwanet] Jwanet D \textbf{22} Gahmuret] Gahmvret D \textbf{23} Des] ÷es D \textbf{24} Ither] Jther D \textbf{27} \textit{Die Verse 153.27-28 fehlen} D  \newline
\end{minipage}
\hspace{0.5cm}
\begin{minipage}[t]{0.5\linewidth}
\small
\begin{center}*m
\end{center}
\begin{tabular}{rl}
 & "goteweiz, hêrre schinischant,\\ 
 & daz Cunneware \textbf{de Lalant}\\ 
 & durch den knappen ist zerbert.\\ 
 & iuwer vröude \textbf{des} wirt verzert\\ 
5 & noch von sîner hende,\\ 
 & er \textbf{en}\textbf{sî} nie sô ellende."\\ 
 & "sît iuwer êrste rede mir dröuwet,\\ 
 & ich wæne, irs wênic iuch gevröuwet."\\ 
 & sîn b\textit{râ}t wart gealûnet,\\ 
10 & mit slegen \textbf{vil} gerûnet\\ 
 & dem \textbf{wizzenthaften} tôren\\ 
 & mit v\textit{iu}sten in sîn ôren.\\ 
 & daz tet Keie sunder twâl.\\ 
 & dô muose der junge Parcifal\\ 
15 & disen kumber schouwen,\\ 
 & Anthan\textit{o}rs und der vrouwen.\\ 
 & im was von herzen leit ir nôt.\\ 
 & vil dicke er greif zem gabilôt.\\ 
 & \textbf{d\textit{ô} wa\textit{s v}or der künigîn} solich gedranc,\\ 
20 & daz er durch daz vermeit den swanc.\\ 
 & urloup nam dô Iwanet\\ 
 & ze\textbf{m} fili rois Gahmuret.\\ 
 & des reise aleine wart getân\\ 
 & hin ûz gegen I\textit{t}her ûf den plân.\\ 
25 & dem sagete er solichiu mære,\\ 
 & daz \textbf{niemen dinne} wære,\\ 
 & der justierens gerte.\\ 
 & "der künic mich gebe werte.\\ 
 & ich sagete, \textbf{als} dû mir \textbf{jæhe},\\ 
30 & wiez âne danc geschæhe,\\ 
\end{tabular}
\scriptsize
\line(1,0){75} \newline
m n o \newline
\line(1,0){75} \newline
\newline
\line(1,0){75} \newline
\textbf{2} Cunneware] cúnneware n Cumeiẏare o \textbf{3} zerbert] zerbúrt n (o) \textbf{6} nie] nuͯ n \textbf{9} brât] bart m  $\cdot$ gealûnet] gelúnet n \textbf{10} mit] So mit n  $\cdot$ gerûnet] gerúmet n \textbf{12} viusten] frosten m slegen n  $\cdot$ in] vmb n [vnnd]: vnnb o \textbf{13} Keie] keẏe n \textbf{14} muose] muͯste n (o) \textbf{16} Anthanors] Anthanars m Anthenors n Antenors o \textbf{18} gabilôt] babilot o \textbf{19} dô was vor] [V]: Der was der vor m  $\cdot$ gedranc] getwang o \textbf{20} den] das o \textbf{21} dô] da o  $\cdot$ Iwanet] jwanet m n ẏwanet o \textbf{22} rois] ros n  $\cdot$ Gahmuret] gamuret n gamuͯret o \textbf{24} Ither] icher m ichter n iter o  $\cdot$ den] dem n \textbf{28} gebe werte] gewerte n o \textbf{29} sagete] sage o \newline
\end{minipage}
\end{table}
\newpage
\begin{table}[ht]
\begin{minipage}[t]{0.5\linewidth}
\small
\begin{center}*G
\end{center}
\begin{tabular}{rl}
 & "goteweiz, hêr seneschalt,\\ 
 & daz \textbf{vrou} Kuneware \textbf{de Lalant}\\ 
 & durch den knappen ist zerbert.\\ 
 & iwer vröude \textbf{es} wirt verzert\\ 
5 & noch von sîner hende.\\ 
 & er \textbf{ist} nie sô ellende."\\ 
 & "sît iwer êrstiu rede mir dröut,\\ 
 & ich wæne, irs wênic iuch gevröut."\\ 
 & sîn brât wart gealûnet,\\ 
10 & mit slegen \textbf{vil} gerûnet\\ 
 & \textit{dem} \textbf{witzehaften} tôren\\ 
 & mit viusten in sîn ôren.\\ 
 & daz tet Kay sunder twâl.\\ 
 & dô muose der junge Parzival\\ 
15 & disen kumber schouwen,\\ 
 & Antanors und der vrouwen.\\ 
 & im was von herzen leit ir nôt.\\ 
 & vil dicker greif zem gabilôt.\\ 
 & \textbf{vor der künigîn was} solch gedranc,\\ 
20 & daz er durch daz vermeit den swanc.\\ 
 & urloup nam dô Ywanet\\ 
 & ze filiroys Gahmuret.\\ 
 & des reise al eine wart getân\\ 
 & hin ûz gein Ither ûf den plân.\\ 
25 & dem sagter solhiu mære,\\ 
 & daz \textbf{dâ inne niemen} wære,\\ 
 & der tjostierns gerte.\\ 
 & "der künic mich gâbe werte.\\ 
 & ich sagte, \textbf{als} dû mir \textbf{jæhe},\\ 
30 & wiez ân danc geschæhe,\\ 
\end{tabular}
\scriptsize
\line(1,0){75} \newline
G I O L M Q R Z \newline
\line(1,0){75} \newline
\textbf{7} \textit{Initiale} I  \textbf{17} \textit{Initiale} Q  \textbf{25} \textit{Initiale} I O L R Z  \newline
\line(1,0){75} \newline
\textbf{1} hêr] her key O her kay Q  $\cdot$ seneschalt] senethsal:t I senetzant O sineshalt L sinerschalt M senetschalt Q sinetschant R Z \textbf{2} Kuneware] gunwar I kvnwar O Gvnware M konware Q kunne R kvnneware Z  $\cdot$ de] der O von Q da R  $\cdot$ Lalant] lalalt Q wer zu land R \textbf{3} den] disen I (L)  $\cdot$ zerbert] gebert I O L Q \textbf{4} es] drumbe I \textit{om.} O M die Q \textbf{6} er ist] ern ist I Er si noch O Er en sý L (M) (Q) (R) (Z) \textbf{7} dröut] so drowet M \textbf{8} irs wênic iuch] irs luzzel evch I ivchs wenich O ir euchs vil wenick Q ir wenig uch R \textbf{9} brât] brait L bart M Q  $\cdot$ wart] wert M \textbf{11} dem] mit G  $\cdot$ witzehaften] wiszhafft\%-i M witze haftem Q \textbf{12} viusten] susen I fuͯrsten L sufften M funsten R  $\cdot$ in] vmb Q Z \textbf{13} Kay] kai G Gey I key O M Z cay Q keẏ R \textbf{14} dô] Da O M Z  $\cdot$ Parzival] [parzifal]: Parzifal I Parcifal O L (Z) parzeval M [partifal]: partzifal Q parzifal R \textbf{16} Antanors] Antonors R  $\cdot$ der] die I \textbf{17} herzen] hertz Q  $\cdot$ ir] sein Q \textbf{18} vil] \textit{om.} R  $\cdot$ dicker greif] diche greif er I \textbf{19} vor] Var O  $\cdot$ der künigîn] dem chvnige O (L) (Q)  $\cdot$ solch] so Groz I \textbf{20} durch daz] do durch Q  $\cdot$ vermeit den swanc] megtin swanc I vermeit verswanc Z \textbf{21} dô] er ze I da M R Z die Q  $\cdot$ Ywanet] ẏwanet G iuuanet I jwanet L (R) \textbf{22} ze] ye I  $\cdot$ Gahmuret] gahmvret G Gamvret O Ghmuͯret L Gamuͯret M Gamúret Q gamuret Z \textbf{23} des] Div O (Q) \textbf{24} Ither] itern I [*]: Jther O Jhter L Jther M Q Ihter R ycher Z \textbf{25} dem] ÷em O  $\cdot$ sagter] seit er I (Q) (Z) \textbf{26} dâ] \textit{om.} L  $\cdot$ niemen] \textit{om.} O \textbf{27} tjostierns] Tiostiern I tiostires Q (R) \textbf{28} mich] mit I mich der R  $\cdot$ gâbe] gebe M  $\cdot$ werte] gewerte O M Q \textbf{29} ich sagte] Ouch sagit ich M Jch sagt im Z \textbf{30} danc] din danc I \newline
\end{minipage}
\hspace{0.5cm}
\begin{minipage}[t]{0.5\linewidth}
\small
\begin{center}*T (U)
\end{center}
\begin{tabular}{rl}
 & "got weiz, hêrre scheneschalt,\\ 
 & daz \textbf{vrou} Kunneware \textbf{von iuwer gewalt}\\ 
 & durch den knappen ist zerbert.\\ 
 & iuwer vreude \textbf{dar umb} wirt verzert\\ 
5 & noch von sîner hende,\\ 
 & er \textbf{en}\textbf{sî} nie sô ellende."\\ 
 & "sît iuwer êrste rede mir dröut,\\ 
 & ich wæne, irs wênic iuch gevröut."\\ 
 & sîn brâ\textit{t} wart gealûnet,\\ 
10 & mit slegen \textbf{wol} gerûnet\\ 
 & dem \textbf{witzehaften} tôren\\ 
 & mit viusten in sîne ôren.\\ 
 & daz tet Key sunder \textit{t}wâl.\\ 
 & dô muose der junge Parzifal\\ 
15 & disen kumber schouwen,\\ 
 & Antenors und der vrouwen.\\ 
 & im was von herzen leit ir nôt.\\ 
 & vil dicke er greif zem gabilôt.\\ 
 & \textbf{vor der küneginne was} solich gedranc,\\ 
20 & daz er durch daz vermeit den swanc.\\ 
 & \begin{large}U\end{large}rloup nam dô Ywanet\\ 
 & zuo filliroi Gahmuret.\\ 
 & des reise aleine wart getân\\ 
 & hin ûz gein Ither ûf den plân.\\ 
25 & dem sagete er solichiu mære,\\ 
 & daz \textbf{dâ inne nieman} wære,\\ 
 & der tjostierens gerte.\\ 
 & "der künec mich gâbe werte.\\ 
 & ich sagete, \textbf{des} dû mir \textbf{jæhe},\\ 
30 & wie ez âne \textbf{dînen} danc geschæhe,\\ 
\end{tabular}
\scriptsize
\line(1,0){75} \newline
U V W T \newline
\line(1,0){75} \newline
\textbf{7} \textit{Majuskel} T  \textbf{11} \textit{Majuskel} \textsuperscript{2}\hspace{-1.3mm} T  \textbf{14} \textit{Majuskel} T  \textbf{21} \textit{Initiale} U V T  \textbf{25} \textit{Initiale} W  \textbf{29} \textit{Majuskel} T  \newline
\line(1,0){75} \newline
\textbf{1} scheneschalt] schinneschalt W Senescalt T \textbf{2} daz vrou] daz [irn]: ir vrov̂n T  $\cdot$ Kunneware] Cuͦmewar U kvnnewar V (W) kvnnewaren T  $\cdot$ von iuwer gewalt] [*]: delelant V mit gewalt T \textbf{4} dar umb wirt] es wirt W drvmbe wart T \textbf{6} sô] \textit{om.} W \textbf{7} êrste rede] ersten [*]: wort V \textbf{8} wênic iuch gevröut] eúch wenig frawet W  $\cdot$ irs] irz T \textbf{9} \textit{Die Verse 153.9-10 fehlen} W   $\cdot$ brât] [b*]: brach U \textbf{10} wol] vil T \textbf{11} \textit{Versdoppelung 153.11-12 (²T) nach 153.11; Fassungstext *T nach ¹T mit Lesarten der nachfolgenden Verse (²T) im Apparat} T   $\cdot$ Er schluͦg den wissenhafften toren W  $\cdot$ witzehaften] wîselôsen \textsuperscript{2}\hspace{-1.3mm} T \textbf{12} Mit der hand zuͦ den oren W · mit sv̂sene vmbe sin ôren \textsuperscript{2}\hspace{-1.3mm} T \textbf{13} Key] keyn V  $\cdot$ twâl] wal U \textbf{14} muose] mvese T  $\cdot$ Parzifal] partzifal W \textbf{15} schouwen] selber schawen W \textbf{16} Antenors] von anthenors V Anthenors W  $\cdot$ vrouwen] iunckfrawen W \textbf{21} Ywanet] ẏwanet V Jwanet T \textbf{22} zuo filliroi] Zvͦ dez kv́nges svn V Zuͦ dem filli roys W  $\cdot$ Gahmuret] Gahmuͦret U [gam*]: gamvret V gamuret W \textbf{23} wart] hie ward W \textbf{24} ûz] vf T  $\cdot$ Ither] ytern V yther W \textbf{25} sagete] saget V W \textbf{26} inne nieman] niemant inne W \textbf{29} des dû mir jæhe] [d*v]: daz dv mir [*]: veriehe V im als du mir veriehe W als dvmir iehe T \textbf{30} dînen] \textit{om.} T  $\cdot$ danc] willen W \newline
\end{minipage}
\end{table}
\end{document}
