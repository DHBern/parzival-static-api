\documentclass[8pt,a4paper,notitlepage]{article}
\usepackage{fullpage}
\usepackage{ulem}
\usepackage{xltxtra}
\usepackage{datetime}
\renewcommand{\dateseparator}{.}
\dmyyyydate
\usepackage{fancyhdr}
\usepackage{ifthen}
\pagestyle{fancy}
\fancyhf{}
\renewcommand{\headrulewidth}{0pt}
\fancyfoot[L]{\ifthenelse{\value{page}=1}{\today, \currenttime{} Uhr}{}}
\begin{document}
\begin{table}[ht]
\begin{minipage}[t]{0.5\linewidth}
\small
\begin{center}*D
\end{center}
\begin{tabular}{rl}
\textbf{516} & sus ahtbæren gesellen",\\ 
 & sprach si. "got müeze \textbf{iuch} vellen."\\ 
 & \begin{large}S\end{large}wer nû \textbf{des wil} volgen mir,\\ 
 & der mîde valsche rede gein ir.\\ 
5 & niemen sich verspreche,\\ 
 & er\textbf{n} wizze \textbf{ê}, waz er reche,\\ 
 & unz er gewinne künde,\\ 
 & wie ez umb ir herze stüende.\\ 
 & ich künde ouch \textbf{wol} gerechen dar\\ 
10 & gein der vrouwen \textbf{wol gevar};\\ 
 & swaz si hât gein Gawan\\ 
 & in ir zorne missetân\\ 
 & oder \textbf{daz} \textbf{si} noch \textbf{getuot} gein im,\\ 
 & die râche ich alle von ir nim.\\ 
15 & Orgeluse, diu rîche,\\ 
 & vuor ungeselleclîche.\\ 
 & zuo Gawane si kom geriten\\ 
 & mit alsô \textbf{zornlîchen} siten,\\ 
 & daz \textbf{ich michs} wênec trôste,\\ 
20 & daz si mich von sorgen lôste.\\ 
 & Si riten dannen beide\\ 
 & \textbf{ûf} eine liehte heide.\\ 
 & ein krût \textbf{dâ stênde Gawan} sach,\\ 
 & \textbf{des} \textbf{würze} er wunden helfe jach.\\ 
25 & Dô erbeizte der werde\\ 
 & nider \textbf{zuo der} erde;\\ 
 & \textbf{er} gruob si, wider ûf er saz.\\ 
 & diu vrouwe ir rede \textbf{ouch} niht vergaz;\\ 
 & si sprach: "kan der geselle mîn\\ 
30 & arzet und rîter sîn,\\ 
\end{tabular}
\scriptsize
\line(1,0){75} \newline
D \newline
\line(1,0){75} \newline
\textbf{3} \textit{Initiale} D  \textbf{21} \textit{Majuskel} D  \textbf{25} \textit{Majuskel} D  \newline
\line(1,0){75} \newline
\newline
\end{minipage}
\hspace{0.5cm}
\begin{minipage}[t]{0.5\linewidth}
\small
\begin{center}*m
\end{center}
\begin{tabular}{rl}
 & sus ahtbæren gesellen",\\ 
 & sprach si. "got müez \textbf{in} vellen."\\ 
 & wer nû \textbf{wel des} volgen mir,\\ 
 & der m\textit{î}de valsch rede gegen ir.\\ 
5 & niemen sich verspreche,\\ 
 & er wizze \textbf{ê}, waz er reche,\\ 
 & unz er gewinne künde,\\ 
 & wie ez umb ir herz stüende.\\ 
 & ich künde ouch gerechen dar\\ 
10 & gegen der vrouwen \textbf{wolgevar};\\ 
 & waz si het gegen Gawan\\ 
 & in ir zorn missetân\\ 
 & oder noch \textbf{getuot} gegen ime,\\ 
 & die râche ich alle von ir nime.\\ 
15 & Urgeluse, diu rîch,\\ 
 & vuor unge\textit{sell}eclîch.\\ 
 & zuo Gawan si \textit{kam} geriten\\ 
 & mit alsô \textbf{zorniclîchem} siten,\\ 
 & daz \textbf{ich mich} wênic trôst,\\ 
20 & daz si mich von sorgen lôst.\\ 
 & si riten dannen beide\\ 
 & \textbf{ûf} eine liehte heide.\\ 
 & ein krût \textbf{Gawan d\textit{â} stên} sach,\\ 
 & \textbf{des} \textbf{würze} er wunden helfe \textit{j}ach.\\ 
25 & dô erbeizte der werde\\ 
 & nider \textbf{zuo der} erde\\ 
 & \textbf{und} gruop si; wider ûf er saz.\\ 
 & diu vrouwe ir rede \textbf{ouch} niht vergaz;\\ 
 & si sprach: "\textit{kan} der geselle mîn\\ 
30 & arzet und ritter sîn,\\ 
\end{tabular}
\scriptsize
\line(1,0){75} \newline
m n o \newline
\line(1,0){75} \newline
\newline
\line(1,0){75} \newline
\textbf{2} müez] muͯs m (n) (o) \textbf{3} wel des] wil des n des wil o \textbf{4} mîde] muͯde m \textbf{7} gewinne künde] gewinnen kuͯnne o \textbf{11} het] hette n \textbf{15} Urgeluse] Vrgeluͯse m Vrgelúse o \textbf{16} vuor] Fuͯre n  $\cdot$ ungeselleclîch] vngefugeclich m [vngelleclich]: vngesseclich o \textbf{17} kam] \textit{om.} m \textbf{20} von] [vor]: von n \textbf{23} dâ] do m n o \textbf{24} wunden helfe] helffen wunden n wuͯnden helffen o  $\cdot$ jach] nach m \textbf{29} kan] \textit{om.} m \newline
\end{minipage}
\end{table}
\newpage
\begin{table}[ht]
\begin{minipage}[t]{0.5\linewidth}
\small
\begin{center}*G
\end{center}
\begin{tabular}{rl}
 & \textit{\begin{large}S\end{large}}us ahtbæren gesellen",\\ 
 & sprach si. "got muoze \textbf{iuch} vellen."\\ 
 & swer nû \textbf{des wil} volgen mir,\\ 
 & der mî\textit{de} valsche rede gein ir.\\ 
5 & niemen sich versprech\textit{e},\\ 
 & er\textbf{n} wizze \textbf{ê}, waz er reche,\\ 
 & unze er gewinne künde,\\ 
 & wie ez umbe ir herze stüende.\\ 
 & ich künde ouch \textbf{wol} gerechen dar\\ 
10 & gein der vrouwen \textbf{wol gevar};\\ 
 & swaz si hât gein Gawan\\ 
 & in ir zorne missetân\\ 
 & ode \textbf{swaz} \textbf{si} noch \textbf{tuot} gein im,\\ 
 & die râche ich alle von ir nim.\\ 
15 & Orgeluse, diu rîche,\\ 
 & vuor ungeselleclîche.\\ 
 & zuo Gawane si kom geriten\\ 
 & mit alsô \textbf{zornlîchen} siten,\\ 
 & daz \textbf{ich michs} wênic trôste,\\ 
20 & daz si mich von sorgen lôste.\\ 
 & si riten dannen beide\\ 
 & \textbf{ûf} eine liehte heide.\\ 
 & ein krût \textbf{Gawan dâ stênde} sach.\\ 
 & \textbf{des} \textbf{krût} er wunden helfe jach.\\ 
25 & dô erbeizet der werde\\ 
 & nider \textbf{zuo der} erde;\\ 
 & \textbf{er} gruop si, wider ûf er saz.\\ 
 & diu vrouwe ir rede \textbf{ouch} niht vergaz;\\ 
 & si sprach: "kan der geselle mîn\\ 
30 & arzet unde rîter sîn,\\ 
\end{tabular}
\scriptsize
\line(1,0){75} \newline
G I L M Z Fr23 \newline
\line(1,0){75} \newline
\textbf{1} \textit{Initiale} G I L Z  \textbf{13} \textit{Initiale} I  \newline
\line(1,0){75} \newline
\textbf{1} Sus] Vvs G  $\cdot$ ahtbæren] ehtigen I achtebere M \textbf{2} sprach si] si sprach I  $\cdot$ muoze] muͤz I (L) (Z) \textbf{3} swer] Wer L M Z  $\cdot$ nû des] [is]: des mir nu I des nu Z \textbf{4} mîde] mit G \textbf{5} verspreche] uirsprechen G \textbf{6} ê] \textit{om.} I M \textbf{7} er] er ir I \textbf{9} ouch] uͯch L (M)  $\cdot$ gerechen] geraichen I \textbf{11} swaz] Waz L (M) Z  $\cdot$ Gawan] gewan G Gawane L \textbf{12} in ir] mit I \textbf{13} swaz] daz L (M) Z Fr23  $\cdot$ tuot] getuͯt L (Z) (Fr23) \textbf{14} ich] ich noch M  $\cdot$ von ir] gein im L vor ir M \textbf{15} Orgeluse] Orguluse I Orgelýse L Orgiluse Fr23  $\cdot$ diu] die Fr23 \textbf{17} Gawane] Gawan I (Z) Fr23 \textbf{18} zornlîchen] zorntlichen L \textbf{19} ich michs wênic] ich mich wenc I ichz wenig mich L (Fr23) ichs mich wennic M \textbf{22} liehte] lýchte L (Fr23) lichtin M  $\cdot$ heide] heidin M \textbf{23} dâ stênde] dort sten L \textbf{24} des krût er] dem er I Des kraft den L Der we ir M Des wurtz er Z Des wrz zeder Fr23  $\cdot$ jach] rach M \textbf{25} dô] Da M Z  $\cdot$ erbeizet] erbaizte I (L) (M) (Z) \textbf{26} erde] erdin M \textbf{27} gruop si] grvpsz L \textbf{28} ouch] \textit{om.} I M Fr23 \textbf{30} unde] vnd ouch Z \newline
\end{minipage}
\hspace{0.5cm}
\begin{minipage}[t]{0.5\linewidth}
\small
\begin{center}*T
\end{center}
\begin{tabular}{rl}
 & sus ahtbæren gesellen",\\ 
 & sprach si. "got muoz\textbf{iuch} vellen."\\ 
 & Swer nû \textbf{des welle} volgen mir,\\ 
 & der mîde valsche rede gein ir.\\ 
5 & nieman sich verspreche,\\ 
 & er\textbf{n} wizze, waz er reche,\\ 
 & unz er gewinne künde,\\ 
 & wiez umb ir herze stüende.\\ 
 & ich künde ouch \textbf{wol} ger\textit{e}chen dar\\ 
10 & gegen der vrouwen \textbf{lieht gevar};\\ 
 & Swaz si hât gegen Gawan\\ 
 & in ir zorne missetân\\ 
 & oder \textbf{daz} \textbf{si} noch \textbf{getuot} gein im,\\ 
 & die râche ich alle von ir nim.\\ 
15 & Orgeluse, diu rîche,\\ 
 & vuor ungeselleclîche.\\ 
 & ze Gawane si kom geriten\\ 
 & mit alsô \textbf{zornlîchen} siten,\\ 
 & daz \textbf{ichs mich} wênic trôste,\\ 
20 & daz si mich von sorgen lôste.\\ 
 & si riten dannen beide\\ 
 & \textbf{über} eine liehte heide.\\ 
 & ein krût \textbf{Gawan dâ stênde} sach.\\ 
 & \textbf{der} \textbf{würze} er \textbf{zer} wunden helfe jach.\\ 
25 & dô erbeizete der werde\\ 
 & nider \textbf{ûf die} erde;\\ 
 & \textbf{er} gruop si \textbf{ûz}, wider ûf er saz.\\ 
 & Diu vrouwe ir rede niht vergaz;\\ 
 & si sprach: "kan der geselle mîn\\ 
30 & arzât unde rîter sîn,\\ 
\end{tabular}
\scriptsize
\line(1,0){75} \newline
T U V W O Q R Fr40 \newline
\line(1,0){75} \newline
\textbf{1} \textit{Initiale} O  \textbf{3} \textit{Majuskel} T  \textbf{11} \textit{Majuskel} T  \textbf{28} \textit{Majuskel} T  \newline
\line(1,0){75} \newline
\textbf{1} sus] ÷vs O Alles Q \textbf{2} muoziuch] mvͦziv T mvͤze [*]: v́ch V mvͦz ivch O (Q) (R) \textbf{3} Swer] Wer U W R Wes Q  $\cdot$ nû des] dez nv V (Q) (R) nv O  $\cdot$ welle] wille U wil O  $\cdot$ volgen] volge U velgen Q \textbf{5} sich] nit R \textbf{6} ern wizze] Ern wússe e V Er wis dann W Ern wisse e Q Er wise e R  $\cdot$ reche] gereche U \textbf{7} unz] Mit U \textbf{9} gerechen] gereichen T U gesprechen O \textbf{11} Swaz] Waz U (W) (Q) (R)  $\cdot$ Gawan] her Gawan R \textbf{13} oder] [D*]: Oder V  $\cdot$ daz si noch getuot] das ich noch [tun]: tunt Q was sy noch tuͦt R \textbf{14} ich alle] alle ich U ich rable Q  $\cdot$ ir] \textit{om.} W \textbf{15} Orgeluse] Orgilvse O Orgeliuse Q Orguluse R \textbf{17} Gawane] gawanen W Gawan O (Q) Gawain R  $\cdot$ si] \textit{om.} R \textbf{18} alsô] so R  $\cdot$ zornlîchen] zorniglichem W zorlichem Q \textbf{19} ichs mich] ich mich ez V (Q) (Fr40) ich mich W O sy mich so R \textbf{20} si mich] mich sie U mich R \textbf{23} Gawan dâ] do gawan W Gawain da R  $\cdot$ stênde] stuͦde R \textbf{24} der würze] [*er]: Der wurze V Des wurtze W (O) Die wurczen R des wurz: Fr40  $\cdot$ er] \textit{om.} W  $\cdot$ zer] \textit{om.} V W Q R Fr40 \textbf{26} ûf die] zvͦ der O \textbf{27} ûf er] er auff W vs er O  $\cdot$ saz] sach Fr40 \textbf{28} niht] auch nit U (Q) (Fr40) \textbf{30} arzât unde rîter] Ritter vnd arczat R \newline
\end{minipage}
\end{table}
\end{document}
