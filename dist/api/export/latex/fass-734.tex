\documentclass[8pt,a4paper,notitlepage]{article}
\usepackage{fullpage}
\usepackage{ulem}
\usepackage{xltxtra}
\usepackage{datetime}
\renewcommand{\dateseparator}{.}
\dmyyyydate
\usepackage{fancyhdr}
\usepackage{ifthen}
\pagestyle{fancy}
\fancyhf{}
\renewcommand{\headrulewidth}{0pt}
\fancyfoot[L]{\ifthenelse{\value{page}=1}{\today, \currenttime{} Uhr}{}}
\begin{document}
\begin{table}[ht]
\begin{minipage}[t]{0.5\linewidth}
\small
\begin{center}*D
\end{center}
\begin{tabular}{rl}
\textbf{734} & \begin{large}V\end{large}il liute \textbf{des} hât verdrozzen,\\ 
 & den diz mære was \textbf{vor beslozzen};\\ 
 & Genuoge kundenz nie ervarn.\\ 
 & Nû wil ich \textbf{daz} niht langer sparn,\\ 
5 & ich tuonz iu kunt mit rehter sage,\\ 
 & Wande ich in dem munde trage\\ 
 & Daz slôz dirre âventiure,\\ 
 & wie der süeze unt der gehiure\\ 
 & Anfortas wart wol gesunt.\\ 
10 & uns tuot diu âventiure kunt,\\ 
 & wie von Pelrapeire diu künegîn\\ 
 & ir kiuschen wîplîchen \textbf{sin}\\ 
 & behielt \textbf{unz} an ir lônes stat,\\ 
 & \textbf{dâ} si in hôhe sælde trat.\\ 
15 & Parzival daz wirbet,\\ 
 & \textbf{ob mîn} kunst niht verdirbet.\\ 
 & ich sage alrêst sîn arbeit.\\ 
 & swaz sîn hant ie gestreit,\\ 
 & daz was mit kinden her getân.\\ 
20 & m\textit{ö}hte ich \textbf{dises} mæres wandel hân,\\ 
 & \textbf{ungern wolt ich in} wâgen;\\ 
 & des kunde ouch mich betrâgen.\\ 
 & Nû bevilh ich sîn gelücke\\ 
 & \textbf{sîme herzen}, der sælden stücke,\\ 
25 & dâ \textbf{diu vrevel} bî der kiusche lac,\\ 
 & wand ez nie zagheit gepflac.\\ 
 & daz \textbf{müeze} im vestenunge geben,\\ 
 & daz \textbf{er} behalde \textbf{nû} sîn leben,\\ 
 & sît ez sich hât an den gezogt,\\ 
30 & in bestêt ob \textbf{allem strîte} ein vogt\\ 
\end{tabular}
\scriptsize
\line(1,0){75} \newline
D \newline
\line(1,0){75} \newline
\textbf{1} \textit{Großinitiale} D  \textbf{3} \textit{Majuskel} D  \textbf{4} \textit{Majuskel} D  \textbf{6} \textit{Majuskel} D  \textbf{7} \textit{Majuskel} D  \textbf{23} \textit{Majuskel} D  \newline
\line(1,0){75} \newline
\textbf{15} Parzival] Parcifal D \textbf{20} möhte] mohte D \newline
\end{minipage}
\hspace{0.5cm}
\begin{minipage}[t]{0.5\linewidth}
\small
\begin{center}*m
\end{center}
\begin{tabular}{rl}
 & \begin{large}V\end{large}il liute \textbf{dis} hât verdrozzen,\\ 
 & den diz mære was \textbf{vor beslozzen};\\ 
 & genuoge kundenz nie ervarn.\\ 
 & nû wil ich \textbf{ez} niht langer s\textit{p}arn,\\ 
5 & ich tuo ez iu kunt mit rehter sage,\\ 
 & wan ich in dem munde trage\\ 
 & daz slôz diser âventiur,\\ 
 & wie der süeze und der gehiur\\ 
 & A\textit{n}fortas wart wol gesunt.\\ 
10 & uns tuot diu âventiur kunt,\\ 
 & wie von Pelraperie diu künigîn\\ 
 & ir kiuschen wîplîchen \textbf{schîn}\\ 
 & behielt \textbf{im} an ir lônes stat,\\ 
 & \textbf{dô} si in hôhe sælde trat.\\ 
15 & Parcifal daz wirbet,\\ 
 & \textbf{ob mîn} kunst niht verdirbet.\\ 
 & ich sage allerêrst sîn arbeit,\\ 
 & \textbf{wan} waz sîn hant ie gestreit,\\ 
 & daz was mit kinden her getân.\\ 
20 & m\textit{ö}hte ich \textbf{dis} mæres wandel hân,\\ 
 & \textit{\textbf{ungerne wolte ich in} wâgen;}\\ 
 & des kunde ouch mich betrâgen.\\ 
 & nû bevilh ich sîn glücke\\ 
 & \textbf{sînem herzen}, der sæl\textit{d}en stücke,\\ 
25 & d\textit{â} \textbf{diu vrevel} bî der kiusche lac,\\ 
 & wa\textit{n}t ez nie zagheit gepflac.\\ 
 & daz \textbf{muoste} im vestenunge geben,\\ 
 & daz \textbf{er} \textit{b}ehalt \textit{\textbf{nû}} sîn leben,\\ 
 & sît ez sich het an den gezoget,\\ 
30 & i\textit{n} bestât ob \textbf{allem strît} ein voget\\ 
\end{tabular}
\scriptsize
\line(1,0){75} \newline
m n o V V' \newline
\line(1,0){75} \newline
\textbf{1} \textit{Überschrift:} Hie kvmmet parzefal zvͦ sime bruͦder vnde vindet den von geschiht fervis anschefin vnde wurt mit imme vehtende V  Hie komet parzifal zv sinem bruder ferevis vnd vichtet mit yme V'   $\cdot$ \textit{Initiale} m V V'   $\cdot$ \textit{Capitulumzeichen} n  \newline
\line(1,0){75} \newline
\textbf{1} dis] des V (V') \textbf{2} diz] dise n \textbf{4} sparn] starn m \textbf{5} ich tuo ez iu kunt] Jch tuͯ uͯch es kuͯnt o Jch entuͤge úch kvnt V Jch tu uch kvnt V' \textbf{6} dem] dan V' \textbf{7} diser] der V' \textbf{8} wie der] [*]: Wie der V  $\cdot$ gehiur] gehuse V' \textbf{9} Anfortas] Arfortas m Anfortes V' \textbf{10} diu âventiur] \sout{anfort} die auenture V' \textbf{11} Pelraperie] pelraberie m pelrapier n o Belrepere V (V')  $\cdot$ diu künigîn] der konig o \textbf{12} schîn] sin V V' \textbf{13} behielt] behalt o  $\cdot$ im] jme das n (o) vnze V (V')  $\cdot$ ir] [des]: ir o \textbf{14} sælde] wirde V' \textbf{15} Parcifal] Parzefal V Parzifal V' \textbf{17} allerêrst] alle erst n \textbf{18} waz] swaz V  $\cdot$ gestreit] gesreit o \textbf{20} \textit{Versdoppelung (mit Anteil aus Vers 734.21):} Vngerne wolte ich dis meres [wa*]: wandel han o   $\cdot$ Abir uon parzifal so vahe ich an V'  $\cdot$ möhte] Mohtte m (n) (o) (V) \textbf{21} \textit{Vers 734.21 fehlt} m   $\cdot$ \textit{Die Verse 734.21-735.4 fehlen} V'   $\cdot$ wâgen] frogen n o \textbf{22} des kunde ouch mich] Des kunde mich ouch n Des kinden auch auch mich o \textbf{24} sînem herzen] [Sin* herze*]: Sin herze V  $\cdot$ sælden] selben m selten o \textbf{25} dâ] Do m n o V  $\cdot$ vrevel] fruel n \textbf{26} want] Wart m \textbf{27} muoste] muͤsse V \textbf{28} behalt] hehalt m  $\cdot$ nû] ẏm m o \textbf{29} het] hette n  $\cdot$ gezoget] gezeiget n gezoigen o \textbf{30} in] Jnne m  $\cdot$ allem] allen n o \newline
\end{minipage}
\end{table}
\newpage
\begin{table}[ht]
\begin{minipage}[t]{0.5\linewidth}
\small
\begin{center}*G
\end{center}
\begin{tabular}{rl}
 & \begin{Large}V\end{Large}il liute \textbf{des} hât verdrozzen,\\ 
 & den ditze mære was \textbf{verslozzen};\\ 
 & genuoge kundenz nie ervarn.\\ 
 & nû\textbf{ne} wil ich \textbf{daz} niht langer sparn,\\ 
5 & ich \textbf{en}tuo ez iu kunt mit rehter sage,\\ 
 & wan ich in dem munde trage\\ 
 & daz slôz dirre âventiure,\\ 
 & wie der süeze unde der gehiure\\ 
 & Anfortas wart wol gesunt.\\ 
10 & uns tuot diu âventiure kunt,\\ 
 & wie von Peilrapeire diu künigîn\\ 
 & ir kiuschen wîplîchen \textbf{sin}\\ 
 & behielt \textbf{unze} an ir lônes stat,\\ 
 & \textbf{daz} si in hôhe sælde trat.\\ 
15 & Parcival daz wirbet,\\ 
 & \textbf{obe mîn} kunst niht verdirbet.\\ 
 & ich\textbf{n} sage alrêst sîn arbeit.\\ 
 & swaz sîn hant ie gestreit,\\ 
 & daz was mit kinden her getân.\\ 
20 & möhte ich \textbf{des} mæres wandel hân,\\ 
 & \textbf{ich wol\textit{d}e in ungerne} wâgen;\\ 
 & des kunde ouch mich betrâgen.\\ 
 & nû bevilhe ich sîn gelücke,\\ 
 & \textbf{sîn herze} der sælden stücke,\\ 
25 & dâ \textbf{diu übel} bî der kiusche lac,\\ 
 & wan ez nie zageheit gepflac.\\ 
 & daz \textbf{müeze} im vestenunge geben,\\ 
 & daz \textbf{er} behalte \textbf{nû} sîn leben,\\ 
 & sît ez sich hât an den gezoge\textit{t},\\ 
30 & in bestêt ob \textbf{allem strîte} ein voget\\ 
\end{tabular}
\scriptsize
\line(1,0){75} \newline
G I L M Z Fr18 Fr24 \newline
\line(1,0){75} \newline
\textbf{1} \textit{Überschrift:} Hie ligt der kunic gramoflantz bi siner frowen vnd wil hohzit haben vnd hat sich parcifal da von verstoln vnd ritet da hin Z   $\cdot$ \textit{Großinitiale} Z Fr18   $\cdot$ \textit{Initiale} G L M Fr24  \textbf{23} \textit{Initiale} I  \newline
\line(1,0){75} \newline
\textbf{1} des] das M  $\cdot$ verdrozzen] bedrozzen Fr24 \textbf{2} ditze] die L das M (Fr24)  $\cdot$ was] waz vor L (Z)  $\cdot$ verslozzen] bislozzen M (Z) \textbf{3} \textit{Versdoppelung (²Z); Lesarten des vorausgehenden Verses mit ¹Z bezeichnet:} Genvge kvndenz nie ervarn / Genuge enkundez nie ervarn Z   $\cdot$ kundenz] enkonden isz M (\textsuperscript{2}\hspace{-1.3mm} Z \textbf{4} nûne] Nv L Z  $\cdot$ ich daz] isz M  $\cdot$ langer] \sout{bc} lengir M \textbf{5} ez] \textit{om.} Z  $\cdot$ iu] \textit{om.} L \textbf{7} \textit{Versfolge 734.8-7} Fr18   $\cdot$ dirre] der Z \textbf{8} der] daz L \textbf{9} Anfortas] Amfortas L :nfortas Fr18  $\cdot$ wol] \textit{om.} M \textbf{11} Peilrapeire] pailrapair I pelrapeire L M Z Pelrapeẏre Fr18 Pelrapeyre Fr24 \textbf{12} sin] shin I \textbf{13} lônes] [b]: lones I \textbf{14} trat] [rat]: trat M \textbf{15} Parcival] parcifal G (Z) (Fr18) (Fr24) Parzifal I L M  $\cdot$ daz] der L \textbf{16} niht] \textit{om.} M  $\cdot$ verdirbet] [erwerbit]: ersterbit M \textbf{17} ichn sage] Jch sage M Z (Fr18) Jch sage iv Fr24 \textbf{18} swaz] Waz L (M) \textbf{20} möhte] Moht L (M) Z Fr24  $\cdot$ des] dissis M (Z) ditse Fr18 (Fr24) \textbf{21} wolde] wolge G \textbf{22} ouch] \textit{om.} I \textbf{23} ich] \textit{om.} M  $\cdot$ sîn] syme M (Z) (Fr18) \textbf{24} stücke] rvcke Fr18 \textbf{25} dâ diu übel] diu da vbel I Do die vbel L (Fr24) Das y M Da die frevel Z Daz div frevil Fr18  $\cdot$ kiusche] chushait I \textbf{26} ez] er Fr18  $\cdot$ gepflac] phlac M \textbf{27} müeze] musz M (Z) (Fr18) (Fr24)  $\cdot$ vestenunge] vestegunge I \textbf{28} behalte nû] Nu bihilt M behaltet nv Fr18  $\cdot$ sîn] daz I \textbf{29} hât an den] an den hat L (Fr24)  $\cdot$ gezoget] gezogen G \textbf{30} ob allem strîte] ob allen striten I alleine strite M \newline
\end{minipage}
\hspace{0.5cm}
\begin{minipage}[t]{0.5\linewidth}
\small
\begin{center}*T
\end{center}
\begin{tabular}{rl}
 & \begin{Large}V\end{Large}il liute \textbf{des} hâte verdrozzen,\\ 
 & den ditze mære was \textbf{beslozzen};\\ 
 & genuoge kunden ez nie ervarn.\\ 
 & nû \textbf{en}wil ich \textbf{daz} niht langer sparn,\\ 
5 & ich tuon ez iu kunt mit rehter sage,\\ 
 & wan\textit{d} ich in dem munde trage\\ 
 & daz slôz dirre âventiure,\\ 
 & wie der süeze und der gehiure\\ 
 & Anfortas wart wol gesunt.\\ 
10 & uns tuot diu âventiure kunt,\\ 
 & wie von Peilrapere diu künigîn\\ 
 & ir kiuschen wîplîchen \textbf{schîn}\\ 
 & behielt \textbf{unz} an ir lônes stat,\\ 
 & \textbf{daz} si in hôhe sælde trat.\\ 
15 & Parcifal daz wirbet,\\ 
 & \textbf{oder im} kunst niht verdirbet.\\ 
 & ich sage alrêst sîn arbeit.\\ 
 & waz sîn hant ie gestreit,\\ 
 & daz was mit kinden her getân.\\ 
20 & m\textit{ö}ht ich \textbf{di\textit{ses}} \textit{mæres} wandel hân,\\ 
 & \textbf{ich wolt in ungerne} wâgen;\\ 
 & des kunde ouch mich betrâgen.\\ 
 & nû bevilhe ich sîn gelücke,\\ 
 & \textbf{sîn herze} der sælden stücke,\\ 
25 & d\textit{â} \textbf{der vrevel} bî der kiusche lac,\\ 
 & wan ez nie zageheit gepflac.\\ 
 & daz \textbf{muoz} im vestenunge geben,\\ 
 & daz behalte \textbf{dâ} sîn leben,\\ 
 & sît ez sich het an den gezogt,\\ 
30 & \textit{in bestêt ob \textbf{allen strîten} ein vogt}\\ 
\end{tabular}
\scriptsize
\line(1,0){75} \newline
U W Q R \newline
\line(1,0){75} \newline
\textbf{1} \textit{Überschrift:} Hye streit heyden vnd parcifall durch des grals mal Q   $\cdot$ \textit{Großinitiale} U R   $\cdot$ \textit{Initiale} W Q  \newline
\line(1,0){75} \newline
\textbf{1} des] daz R  $\cdot$ hâte] hat W (Q) her R \textbf{2} den] Des Q  $\cdot$ ditze] dise U  $\cdot$ was beslozzen] was vor beslosszen Q (R) \textbf{3} kunden ez] kundes Q  $\cdot$ nie] nit R \textbf{4} nû enwil] Nun wil W R  $\cdot$ niht langer] lenger nit R \textbf{5} ez iu] úchs W (Q) \textbf{6} wand ich] Wan dich U \textbf{9} Anfortas] Anfortes R \textbf{11} Peilrapere] Peilraper U pelrapeir W pelrapeire Q pelrapire R \textbf{12} schîn] sin W Q R \textbf{14} hôhe sælde] hochen selden R \textbf{15} Parcifal] Parzifal U Herr partzifal W Partzifal Q Parczifal R \textbf{16} oder im] Ob mein W Q (R) \textbf{17} sîn] mein W \textbf{19} her] gar R \textbf{20} möht] Mocht U Q  $\cdot$ dises mæres wandel] dirre wandel U des meres wandels mit fuͯge R \textbf{22} des] Das W \textbf{24} der] des Q \textbf{25} dâ] do U W Q  $\cdot$ der vrevel] die freuel W Q (R) \textbf{26} ez] er Q  $\cdot$ gepflac] pflag R \textbf{27} vestenunge] nun vestunge W \textbf{28} behalte dâ] er behaltet nun W er behelte nun Q \textbf{29} het] \textit{om.} W \textbf{30} \textit{Vers 734.30 fehlt} U   $\cdot$ in] Ym Q  $\cdot$ allen strîten] allem streit Q allen sitten R \newline
\end{minipage}
\end{table}
\end{document}
