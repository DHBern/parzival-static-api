\documentclass[8pt,a4paper,notitlepage]{article}
\usepackage{fullpage}
\usepackage{ulem}
\usepackage{xltxtra}
\usepackage{datetime}
\renewcommand{\dateseparator}{.}
\dmyyyydate
\usepackage{fancyhdr}
\usepackage{ifthen}
\pagestyle{fancy}
\fancyhf{}
\renewcommand{\headrulewidth}{0pt}
\fancyfoot[L]{\ifthenelse{\value{page}=1}{\today, \currenttime{} Uhr}{}}
\begin{document}
\begin{table}[ht]
\begin{minipage}[t]{0.5\linewidth}
\small
\begin{center}*D
\end{center}
\begin{tabular}{rl}
\textbf{65} & "ist er gezimieret hie?\\ 
 & âvoy, sô sol man schouwen, \textbf{wie}\\ 
 & sîn lîp den poinder irret,\\ 
 & wie erz \textbf{mit hurte} wirret.\\ 
5 & Der stolze \textbf{künec} Hardiz\\ 
 & hât mit zorne sînen vlîz\\ 
 & nû lange \textbf{vaste} an mich gewant.\\ 
 & den sol \textit{h}ie Gahmuretes hant\\ 
 & mit sîner tjoste neigen.\\ 
10 & mîn sælde ist niht der veigen."\\ 
 & \textbf{sîne} boten sant er sân,\\ 
 & dâ Gaschier der Oriman\\ 
 & mit \textbf{grôzer} messenîe lac\\ 
 & unt der liehte Killirjacac.\\ 
15 & die wâren dâ durch sîne bete.\\ 
 & z\textbf{em} poulûn \textbf{si} \textbf{mit} \textbf{Kaylete}\\ 
 & \textbf{vuoren mit} geselleschaft.\\ 
 & dô enpfiengen si \textbf{durch} liebe kraft\\ 
 & den werden künec von Zazamanc.\\ 
20 & si dûhte ein beiten gar ze lanc,\\ 
 & daz si in \textbf{niht ê} gesâhen.\\ 
 & des si mit triwen jâhen.\\ 
 & Dô vrâgeter \textbf{si} der mære,\\ 
 & wer dâ ritter wære.\\ 
25 & dô sprach sîner muomen kint:\\ 
 & "ûz \textbf{verrem lande} \textbf{hie} sint\\ 
 & ritter, die diu minne jagt,\\ 
 & vil küener helde \textbf{nû} unverzagt.\\ 
 & \textit{\begin{large}H\end{large}}ie hât manegen Bertun\\ 
30 & \textbf{der künec} Utepandragun.\\ 
\end{tabular}
\scriptsize
\line(1,0){75} \newline
D Fr9 \newline
\line(1,0){75} \newline
\textbf{5} \textit{Majuskel} D  \textbf{23} \textit{Majuskel} D  \textbf{25} \textit{Initiale} Fr9  \textbf{29} \textit{Initiale} D  \newline
\line(1,0){75} \newline
\textbf{1} ist] Sus ist Fr9 \textbf{2} sô] nv Fr9 \textbf{5} Hardiz] hardîz D hardẏz Fr9 \textbf{8} hie] die D  $\cdot$ Gahmuretes] Gahmvretes D gamvretes Fr9 \textbf{9} tjoste] zẏoste Fr9 \textbf{12} Gaschier] Gascier D Gatzẏer Fr9  $\cdot$ Oriman] norman Fr9 \textbf{13} grôzer] sẏner Fr9 \textbf{14} Killirjacac] killiriakach D Fr9 \textbf{16} Kaylete] kailete D \textbf{18} dô enpfiengen si durch] vnde vntfiengen da von Fr9 \textbf{19} Zazamanc] Zazamanch D \textbf{20} dûhte ein beiten] duchtez beiden Fr9 \textbf{21} gesâhen] ensahen Fr9 \textbf{26} verrem lande] verren landen Fr9 \textbf{28} nû] \textit{om.} Fr9 \textbf{29} Hie] ÷ie D  $\cdot$ Bertun] bertvͦn D bẏrtvn Fr9 \textbf{30} Utepandragun] Vtrepandragvͦn D vte pandragvn Fr9 \newline
\end{minipage}
\hspace{0.5cm}
\begin{minipage}[t]{0.5\linewidth}
\small
\begin{center}*m
\end{center}
\begin{tabular}{rl}
 & "ist er gezimieret hie?\\ 
 & â\textit{v}oy, sô sol man schouwen \textbf{hie},\\ 
 & sîn lîp den ponder irret,\\ 
 & wie er ez \textbf{mit h\textit{u}rte} wirret.\\ 
5 & der stolze, \textbf{küene} Hardiz\\ 
 & hât mit zorne sînen vlîz\\ 
 & nû lange \textbf{vaste} an mich gewant.\\ 
 & den sol hie Gahmuretes hant\\ 
 & mit sî\textit{n}er juste neigen.\\ 
10 & mîn sælde ist niht der veigen."\\ 
 & \textbf{sîne} boten sant er sân,\\ 
 & dâ Gaschier der Oriman\\ 
 & mit \textbf{grôzer} massenîe lac\\ 
 & und der liehte Kiliriacac.\\ 
15 & die wâren dâ durch sîne bete.\\ 
 & z\textbf{em} poulûn \textbf{mit} \textbf{Kailete}\\ 
 & \textbf{vuoren si mit} geselleschaft.\\ 
 & dô enpfiengen si \textbf{durch} liebe kraft\\ 
 & den werden künic von Zazamanc.\\ 
20 & si dûhte ein beiten gar ze lanc,\\ 
 & daz si in \textbf{niht ê} gesâhen.\\ 
 & des si mit triuwen jâhen.\\ 
 & dô vrâgete  \textbf{si} der mære,\\ 
 & wer d\textit{â} rîter wære.\\ 
25 & \begin{large}D\end{large}ô sprach sîner muomen kint:\\ 
 & "ûz \textbf{verrem lande} \textbf{si} sint,\\ 
 & ritter, die diu minne jaget,\\ 
 & vil küener hel\textit{de} unverzaget.\\ 
 & \textit{h}ie hât manigen Britun\\ 
30 & \textbf{rois} \textit{Ut}r\textit{a}p\textit{a}ndragun.\\ 
\end{tabular}
\scriptsize
\line(1,0){75} \newline
m n o \newline
\line(1,0){75} \newline
\textbf{25} \textit{Initiale} m   $\cdot$ \textit{Capitulumzeichen} n  \newline
\line(1,0){75} \newline
\textbf{1} \textit{Vers 65.1 fehlt} n o  \textbf{2} âvoy] Anoi m (n) (o)  $\cdot$ sô] \textit{om.} n o \textbf{3} den] der o \textbf{4} hurte] herte m n o  $\cdot$ wirret] werret o \textbf{5} Hardiz] hardis m o hardisz n \textbf{8} Gahmuretes] gahmurettes m gamiretes n gamuͯretes o \textbf{9} sîner] simer m \textbf{10} sælde] sele o  $\cdot$ niht] mit n nit mit o \textbf{11} sîne boten] Sin botte o \textbf{12} dâ] Do n o  $\cdot$ Gaschier] gascier m n o  $\cdot$ Oriman] oriman \textit{nachträglich korrigiert zu:} noriman m arme man n ariman o \textbf{13} lac] gelag o \textbf{14} Kiliriacac] Killiria kag m killiakag n kiliakag o \textbf{15} dâ] do n o \textbf{16} zem] Zuͯ n  $\cdot$ poulûn] paueluͯm o  $\cdot$ Kailete] kailette m kaylette n kaẏlet o \textbf{17} mit] durch n o \textbf{19} Zazamanc] zazamang m n o \textbf{20} beiten] biten n (o) \textbf{22} des] Vnd n o \textbf{23} der] die o \textbf{24} dâ] do m der n o \textbf{25} muomen] muͦter n \textbf{26} si] hie n o \textbf{27} die diu] die n  $\cdot$ jaget] jagent o \textbf{28} helde] helt m \textbf{29} hie] Nie \textit{nachträglich korrigiert zu:} Hie m Nie n  $\cdot$ hât] hette n  $\cdot$ Britun] brittun m brituͦn n beituͯn o \textbf{30} rois] Ros o  $\cdot$ Utrapandragun] vetter pendragun m vetter pendraguͦn n v:tter pendragún o \newline
\end{minipage}
\end{table}
\newpage
\begin{table}[ht]
\begin{minipage}[t]{0.5\linewidth}
\small
\begin{center}*G
\end{center}
\begin{tabular}{rl}
 & "ist er gezimiert hie?\\ 
 & âvoy, sô sol man schouwen, \textbf{wie}\\ 
 & sîn lîp den ponder irret,\\ 
 & \begin{large}W\end{large}ierz \textbf{mit hurten} wirret.\\ 
5 & der stolze \textbf{künic} Hardiz\\ 
 & hât mit zorne sînen vlîz\\ 
 & nû lange \textbf{vaste} \textit{an} mich gewant.\\ 
 & den sol hie Gahmuretes hant\\ 
 & mit sîner tjoste neigen.\\ 
10 & mîn sælde ist niht der veigen."\\ 
 & \textbf{sîne} boten \textit{sande er} sân,\\ 
 & dâ Gatschier der Norman\\ 
 & mit \textbf{sîner} massenîe lac\\ 
 & unt der liehte Kiliriakac.\\ 
15 & die wâren dâ durch sîne bet.\\ 
 & z\textbf{em} pavelûne \textbf{si} \textbf{mit} \textbf{Kailet}\\ 
 & \textbf{vuoren durch} geselleschaft.\\ 
 & dô enpfiengen si \textbf{mit} liebe kraft\\ 
 & den werden künic von Zazamanc.\\ 
20 & si dûht ein beiten gar ze lanc,\\ 
 & daz sin \textbf{niht ê} gesâhen.\\ 
 & des si mit triwen jâhen.\\ 
 & dô vrâgter \textbf{si} der mære,\\ 
 & wer dâ rîter wære.\\ 
25 & dô sprach sîner muomen kint:\\ 
 & "ûz \textbf{verren landen} \textbf{hie} sint\\ 
 & rîter, die diu minne jaget,\\ 
 & vil küener helde unverzaget.\\ 
 & hie hât \textbf{vil} manigen Britun\\ 
30 & \textbf{roys} Utpandragun.\\ 
\end{tabular}
\scriptsize
\line(1,0){75} \newline
G I O L M Q R Z Fr37 Fr44 \newline
\line(1,0){75} \newline
\textbf{1} \textit{Initiale} O  \textbf{4} \textit{Initiale} G  \textbf{11} \textit{Initiale} I L  \textbf{29} \textit{Initiale} I M Q R Z Fr44  \newline
\line(1,0){75} \newline
\textbf{1} er] ez I \textit{om.} L Z Fr37 Fr44 ein Q  $\cdot$ gezimiert] gezimer Q \textbf{2} âvoy] Awe O  $\cdot$ sô] nu Fr44  $\cdot$ sol man] svlt ir L  $\cdot$ wie] hie M \textbf{3} sîn lîp] Er L  $\cdot$ den] den don R dem Fr37  $\cdot$ irret] wirret I \textbf{4} hurten] hvrte O (L) (M) (Q) (R) Z (Fr37) (Fr44)  $\cdot$ wirret] firret I \textbf{5} stolze] solcze R  $\cdot$ Hardiz] [harz]: hardiz G Bardiz O hardis L M R hardiesz Q hardiez Fr37 Hardîez Fr44 \textbf{6} hât] der hat O L (M) (Q) (R) (Z) (Fr37)  $\cdot$ mit zorne] mit hazze I stol mit zorne Q \textbf{7} an] vf G \textbf{8} hie] die O  $\cdot$ Gahmuretes] Gamvretes O Gahmuͯres L gamuretis M gamuertes Q gamuretes Z (Fr44) \textbf{9} sîner tjoste] [seiner]: seinem trost Q  $\cdot$ neigen] negen Q \textbf{11} DO sante er sinen boten san L  $\cdot$ sîne] Sinen I (Q)  $\cdot$ sande er] er do sande G \textbf{12} dâ] Do Q Fr44  $\cdot$ Gatschier] Gatshier O Gahmuͯret L gatschir M Q gaschier Fr37  $\cdot$ Norman] noreman G \textbf{13} sîner] grozer O (L) (M) (Q) (R) (Z) Fr37 Fr44 \textbf{14} liehte] lýchte L (M) (Q) (Fr44) o\textit{m. } Z  $\cdot$ Kiliriakac] kiliriakach G kiliriachac I kyliriacach O killiriakach L kiliriacac M kallicriack Q kylliriakac R killiriakac Z Fr44 kalliriach Fr37 \textbf{15} durch sîne] mit siner I \textbf{16} zem] Zuͯ der L Dem R  $\cdot$ si] \textit{om.} L  $\cdot$ Kailet] Gahilet I kaylet O M R Fr44 kaýlet L Gailet Z \textbf{17} vuoren] Fuͯren sie L  $\cdot$ durch] mit I O (M) Q R Fr37 Fr44  $\cdot$ geselleschaft] geschelschafft M \textbf{18} dô] Da Z  $\cdot$ mit] dvrch O (M) (Q) (Fr37) (Fr44)  $\cdot$ liebe] lieber I (L) \textbf{19} Zazamanc] zazamanch G O L zazamant Q zasamanc R \textbf{20} beiten] biten L  $\cdot$ gar] \textit{om.} M \textbf{21} sin] sie Z  $\cdot$ niht ê] ê nit Fr44  $\cdot$ gesâhen] sahin M (Fr44) \textbf{22} des] Das R  $\cdot$ triwen] truwe M \textbf{23} dô] Da M R Z  $\cdot$ vrâgter] vragt er I O (R) (Z) fragiten M  $\cdot$ der] \textit{om.} L \textbf{24} dâ] der I do Q  $\cdot$ wære] [mere]: were M \textbf{25} dô] Da M Z \textbf{26} ûz] vzer I  $\cdot$ verren landen] verrem lande I (M) (R) (Z) \textbf{27} diu] da M do Q \textbf{28} helde] ritter Fr44 \textbf{29} hie] Sie Fr44  $\cdot$ vil] \textit{om.} O L M Q R Z Fr44  $\cdot$ Britun] pritun I brittvn L brytun M brittún Q \textbf{30} Utpandragun] vter prandiguͦn I vterpan dragun M (Fr44) vszpandragún Q vspandagrun R \newline
\end{minipage}
\hspace{0.5cm}
\begin{minipage}[t]{0.5\linewidth}
\small
\begin{center}*T (U)
\end{center}
\begin{tabular}{rl}
 & "ist er gezimieret hie?\\ 
 & âvoy, sô sol man schouwen, \textbf{wie}\\ 
 & sîn lîp den poynder irret,\\ 
 & wie er ez \textbf{enmitten} wirret.\\ 
5 & der stolze \textbf{künec} Hardiz\\ 
 & hât mit zorne sînen vlîz\\ 
 & nû lange an mich gewant.\\ 
 & den sol hie Gahmuretes hant\\ 
 & mit sîner jost neigen.\\ 
10 & mîne sælde ist niht d\textit{er} veige\textit{n}."\\ 
 & \textbf{sînen} boten sante er sân,\\ 
 & d\textit{â} Gatschier der Norman\\ 
 & mit \textbf{sîner} massenîe lac\\ 
 & und der liehte Kylliriakac.\\ 
15 & die wâren d\textit{â} durch sîne bet.\\ 
 & zuo \textbf{eim} pavelûn \textbf{zuo} \textbf{Gahmuret}\\ 
 & \textbf{vuor \textbf{die}} geselleschaft.\\ 
 & dô entviengen si \textbf{durch} liebe kraft\\ 
 & den werden künec von Zazamanc.\\ 
20 & si dûhte ein beiten gar zuo lanc,\\ 
 & daz sin \textbf{ê niht} gesâhen.\\ 
 & des si mit triuwen jâhen.\\ 
 & \begin{large}D\end{large}ô vrâgeter der mære,\\ 
 & wer d\textit{â} rîter wære.\\ 
25 & dô sprach sîner muomen kint:\\ 
 & "ûz \textbf{verreme lande} \textbf{hie} sint\\ 
 & ritter, die diu minne jaget,\\ 
 & vil küener helde unverzaget.\\ 
 & hie hât manegen Britun\\ 
30 & \textbf{ro\textit{y}s} Utpandragun.\\ 
\end{tabular}
\scriptsize
\line(1,0){75} \newline
U V W T \newline
\line(1,0){75} \newline
\textbf{1} \textit{Majuskel} T  \textbf{2} \textit{Majuskel} T  \textbf{5} \textit{Majuskel} T  \textbf{11} \textit{Initiale} T  \textbf{23} \textit{Initiale} U V W   $\cdot$ \textit{Majuskel} T  \textbf{25} \textit{Majuskel} T  \textbf{29} \textit{Initiale} V   $\cdot$ \textit{Majuskel} T  \textbf{30} \textit{Majuskel} T  \newline
\line(1,0){75} \newline
\textbf{1} er] es W \textbf{4} wie er ez enmitten] Wie ers (wierz T ) mit hurtte V (W) (T)  $\cdot$ wirret] virret W \textbf{5} Hardiz] Hardys U [*]: hardis V hardis W \textbf{6} hât mit zorne] der hat allen T \textbf{7} an] vaste an V her an W mit hazze an T \textbf{8} Gahmuretes] Gahmuͦretes U [Gamutetes]: Gamuretes V gamuretes W \textbf{9} jost] hand W \textbf{10} mîne] Mir W  $\cdot$ der veigen] die veige U \textbf{11} Do sante er sinen botten san V  $\cdot$ sînen] Seine W (T) \textbf{12} dâ] Do U V W  $\cdot$ Gatschier] Gascier T \textbf{13} sîner] grosser V W (T) \textbf{14} Kylliriakac] Kylliacac U gilliriacag V kilriatag W kylliriacak T \textbf{15} dâ] do U V W \textbf{16} Zvͦ deme paualun [s*]: sv́ mit [*ayolette]: kayolette V · Zuͦ dem pauilun sy mit gamurette W · zem pavelvn mit gahmvrete T  $\cdot$ Gahmuret] Gahmuͦret U \textbf{17} vuor die] Rittent mit V Euͦren durch W vuͦren mit T \textbf{18} entviengen] entpfieng W  $\cdot$ durch liebe] mit grosser V durch liehe W \textbf{19} den werden] Der werde W  $\cdot$ Zazamanc] zazamang V W \textbf{20} beiten] bîten T \textbf{21} ê niht gesâhen] niht e [gesohen*]: gesohen V nicht ensahen W niht ê gesahen T \textbf{22} des] Das W  $\cdot$ triuwen] gantzen treúwen W \textbf{23} vrâgeter] fraget er sy W vrageter si T \textbf{24} dâ] do U V W \textbf{26} ûz] von T  $\cdot$ verreme lande] fremden landen W \textbf{27} ritter] \textit{om.} T  $\cdot$ jaget] hat her geiaget T \textbf{28} si sint ze strîte noch vnverzaget T \textbf{29} hie] So T  $\cdot$ manegen] vil manigen V manger W hie manegen T  $\cdot$ Britun] Brituͦn U brittvn V \textbf{30} roys] Ros U  $\cdot$ Utpandragun] vtpantragun U vtepandragun V vterpandragun W \newline
\end{minipage}
\end{table}
\end{document}
