\documentclass[8pt,a4paper,notitlepage]{article}
\usepackage{fullpage}
\usepackage{ulem}
\usepackage{xltxtra}
\usepackage{datetime}
\renewcommand{\dateseparator}{.}
\dmyyyydate
\usepackage{fancyhdr}
\usepackage{ifthen}
\pagestyle{fancy}
\fancyhf{}
\renewcommand{\headrulewidth}{0pt}
\fancyfoot[L]{\ifthenelse{\value{page}=1}{\today, \currenttime{} Uhr}{}}
\begin{document}
\begin{table}[ht]
\begin{minipage}[t]{0.5\linewidth}
\small
\begin{center}*D
\end{center}
\begin{tabular}{rl}
\textbf{23} & rîchiu kleider dar getragen.\\ 
 & diu leit er an. \textbf{alsus} hôrt ich sagen,\\ 
 & daz diu \textbf{tiwer} wæren.\\ 
 & a\textit{n}ker, die swæren,\\ 
5 & von arâbischem golde\\ 
 & \textbf{wâren} drûfe, als er wolde.\\ 
 & Dô saz der minnen geltes lôn\\ 
 & ûf \textbf{ein} ors, daz ein Babylon\\ 
 & gein im \textbf{durch} tjostieren reit;\\ 
10 & den stach er drabe. daz was \textbf{dem} leit.\\ 
 & ob sîn wirt \textbf{iht mit im} var?\\ 
 & er unt sîne rîter gar.\\ 
 & \textbf{jâ deiswâr, si sint es} vrô.\\ 
 & si riten mit ein ander dô\\ 
15 & unt erbeizten vor dem palas,\\ 
 & dâ manec \textbf{rîter} ûf was.\\ 
 & \textbf{die} \textbf{muosen wol gekleidet} sîn.\\ 
 & sîniu kinder liefen vor im în,\\ 
 & \begin{large}I\end{large}e zwei ein ander \textbf{an} der hant.\\ 
20 & ir hêrre manege vrouwen vant,\\ 
 & \textbf{gekleidet} wünneclîche.\\ 
 & der küneginne rîche\\ 
 & \textbf{ir ougen vuogten} \textbf{hôhen} pîn,\\ 
 & dô si \textbf{gesach} den Anschevin.\\ 
25 & der was minneclîche \textbf{gevar},\\ 
 & daz er entslôz ir herze gar,\\ 
 & ez wære ir liep oder leit;\\ 
 & daz \textbf{beslôz} dâ vor ir wîpheit.\\ 
 & ein wênec si \textbf{gein im dô} \textbf{trat}.\\ 
30 & \textbf{ir gast si sich küssen bat}.\\ 
\end{tabular}
\scriptsize
\line(1,0){75} \newline
D Fr9 Fr14 \newline
\line(1,0){75} \newline
\textbf{7} \textit{Versal} D  \textbf{19} \textit{Initiale} D  \newline
\line(1,0){75} \newline
\textbf{2} leit] tete Fr9  $\cdot$ alsus] sus Fr9 \textbf{3} daz diu] Die harte Fr9 \textbf{4} anker] acher D \textbf{8} Babylon] babẏlon Fr9 \textbf{9} tjostieren] ziusteren Fr9 \textbf{10} daz] iz Fr9 \textbf{11} var] [ware]: vare Fr9 \textbf{13} Die warens ẏnnichliche vro Fr9 \textbf{14} si] Vnde Fr9 \textbf{15} unt] Sie Fr9 \textbf{22} küneginne] kvninginnen Fr9 \textbf{23} vuogten] suchten Fr9 \textbf{24} Anschevin] Anscivin D anzevin Fr9 ansevin Fr14 \textbf{25} was] was so Fr9 Fr14  $\cdot$ minneclîche gevar] mẏnnichlicher vare Fr9 \textbf{27} wære] wære Fr14 \textbf{29} gein im dô] do kegen im Fr9 \textbf{30} ir] Jrn Fr9 \newline
\end{minipage}
\hspace{0.5cm}
\begin{minipage}[t]{0.5\linewidth}
\small
\begin{center}*m
\end{center}
\begin{tabular}{rl}
 & rîchiu kleider dar getragen.\\ 
 & diu l\textit{ei}t er an. \textbf{sunst} hôrt ich sagen,\\ 
 & daz diu \textbf{tiur\textit{e}} wæren.\\ 
 & anker, die swæren,\\ 
5 & von ar\textit{â}bische\textit{m} golde\\ 
 & \textbf{wâren} dâr ûf, als er wolde.\\ 
 & \begin{large}D\end{large}ô saz der minnen geltes lôn\\ 
 & ûf \textbf{einem} ros, daz ein Babilon\\ 
 & gegen ime justieren reit;\\ 
10 & den stach er dâr abe. daz was \textbf{im} leit.\\ 
 & ob sîn wirt \textbf{\textit{mi}t ime iht} var?\\ 
 & er und sîne ritter gar.\\ 
 & \textbf{jâ des zwâr, \textit{si} sint es} vrô.\\ 
 & si riten mit ein ander dô\\ 
15 & und erbeizeten vor dem palas,\\ 
 & dâ menic \textbf{ritter} ûf was,\\ 
 & \textbf{die} \textbf{wol \textit{ge}k\textit{l}eidet muosen} sîn.\\ 
 & sîniu kinder liefen vor im \textbf{\textit{h}in} în,\\ 
 & ie zwei ein ander \textbf{mit} der hant.\\ 
20 & ir hêrr\textit{e} manig\textit{e} vrowen vant,\\ 
 & \textbf{\textit{be}kleidet} wünneclîche.\\ 
 & der küniginn\textit{e} rîche\\ 
 & \textbf{ir ougen suochten} \textbf{hôhe} pîn,\\ 
 & dô si \textbf{ie sach} den A\textit{n}schevin.\\ 
25 & der was \textbf{sô} minneclîch \textbf{gevar},\\ 
 & daz er ents\textit{lô}z ir herze gar,\\ 
 & ez wære ir liep oder leit;\\ 
 & daz \textbf{verslôz} dâ vor ir wîpheit.\\ 
 & ein wênic si \textbf{gegen ime dô} \textbf{trat}.\\ 
30 & \textbf{ir gast si sich dô küssen bat}.\\ 
\end{tabular}
\scriptsize
\line(1,0){75} \newline
m n o \newline
\line(1,0){75} \newline
\textbf{7} \textit{Illustration mit Überschrift:} Also gamiret mit sinen rittern vff sas vnd reit do bekam [e*]: jme einer vnder wegen den stach er nẏder n   $\cdot$ \textit{Überschrift:} Also Gahmuret vff sas vnd mit sinen rittern reit do bekam ime einre vnder wegen den stach er uͯber abe m   $\cdot$ \textit{Initiale} m n o  \newline
\line(1,0){75} \newline
\textbf{1} \textit{Die Verse 22.19-23.6 fehlen} o   $\cdot$ getragen] tragen n \textbf{2} leit er] lietter \textit{nachträglich korrigiert zu:} leÿt er m  $\cdot$ hôrt] hoͯre n \textbf{3} tiure] toͯren m tore n \textbf{5} arâbischem] aribiscen \textit{nachträglich korrigiert zu:} arabiscen m \textbf{6} wâren] Was n  $\cdot$ er] er das n \textbf{8} einem] ein n  $\cdot$ Babilon] babẏlon o \textbf{9} justieren] duͯrch justieren o \textbf{10} dâr] \textit{om.} n o \textbf{11} mit] nuͯt m \textbf{13} des] \textit{om.} n o  $\cdot$ si] \textit{om.} m  $\cdot$ es] des n o \textbf{15} vor] [reit]: vor n \textbf{16} dâ] Do m n o  $\cdot$ ûf] \textit{om.} n \textbf{17} wol gekleidet] wolkeÿdet \textit{nachträglich korrigiert zu:} wolgekleÿdet m wol gecleiten o \textbf{18} hin] in m \textbf{19} ein ander] mit einander n \textbf{20} hêrre] herren m  $\cdot$ manige] manichen m  $\cdot$ vrowen] frouwe n (o) \textbf{21} bekleidet] Erkleidet m \textbf{22} küniginne] kuniginnen m (o) \textbf{23} suochten] fochten n o \textbf{24} Anschevin] ausceuin \textit{nachträglich korrigiert zu:} ansceuin m angest sin n o \textbf{26} entslôz] entschols \textit{nachträglich korrigiert zu:} entschlos m enslosse n \textbf{28} vor] von n o \textbf{29} ein wênic] Enwenig o \textbf{30} dô] \textit{om.} n o  $\cdot$ ir] Jren m n o \newline
\end{minipage}
\end{table}
\newpage
\begin{table}[ht]
\begin{minipage}[t]{0.5\linewidth}
\small
\begin{center}*G
\end{center}
\begin{tabular}{rl}
 & rîchiu kleider dar getragen.\\ 
 & diu leit er an. \textbf{sus} hôrt ich sagen,\\ 
 & daz diu \textbf{tiure} wæren.\\ 
 & anker, die swæren,\\ 
5 & von arâbeschem golde\\ 
 & \textbf{lâgen} drûfe, als er wolde.\\ 
 & dô saz der minnen geltes lôn\\ 
 & \begin{large}Û\end{large}f \textbf{ein} ors, daz ein Babilon\\ 
 & gein im \textbf{durch} tjostieren reit;\\ 
10 & den stach er drab. daz was \textbf{dem} leit.\\ 
 & op sîn wirt \textbf{mit im iht} var?\\ 
 & \textbf{jâ} er und sîne rîter gar,\\ 
 & \textbf{die wârens al gelîche} vrô.\\ 
 & si riten mit ein ander dô\\ 
15 & unde erbeizten vor dem palas,\\ 
 & dâ manic \textbf{rîter} ûffe was.\\ 
 & \textbf{die} \textbf{muosen wol gekleidet} sîn.\\ 
 & sîniu kinder liefen vor im în,\\ 
 & ie zwei ein ander \textbf{an} der hant.\\ 
20 & ir hêrre manige vrouwen vant,\\ 
 & \textbf{gekleidet} wünniclîche.\\ 
 & der küniginne rîche\\ 
 & \textbf{ir ougen vuogeten} \textbf{grôzen} pîn,\\ 
 & dô si \textbf{gesach} den Antschevin.\\ 
25 & der was \textbf{sô} minniclîch \textbf{gevar},\\ 
 & daz er entslôz ir herze gar,\\ 
 & ez wære ir liep oder leit;\\ 
 & daz \textbf{beslôz} dâ vor ir wîpheit.\\ 
 & ein wênic si\textbf{m engegene} \textbf{trat}.\\ 
30 & \textbf{ir gast si sich küssen bat}\\ 
\end{tabular}
\scriptsize
\line(1,0){75} \newline
G O L M Q R W Z Fr29 Fr32 Fr36 Fr71 \newline
\line(1,0){75} \newline
\textbf{1} \textit{Initiale} O M Fr29  \textbf{8} \textit{Initiale} G  \textbf{15} \textit{Initiale} Fr71  \textbf{17} \textit{Versal} Fr32  \textbf{19} \textit{Initiale} R W  \newline
\line(1,0){75} \newline
\textbf{1} rîchiu] Riche R (Fr32)  $\cdot$ kleider] wat L \textbf{2} leit er an] legter an Fr29 Fr32 leit an ir Fr36  $\cdot$ sus] sich L \textit{om.} W  $\cdot$ ich] er R \textbf{3} tiure] tvrer L  $\cdot$ wæren] waren L (M) \textbf{4} die] den Fr71 \textbf{5} arâbeschem] arabenschem G arabischem O (L) Q W Z (Fr29) (Fr32) arabischen M (Fr36) Arabyschem R (Fr71) \textbf{6} lâgen] Warn W (Z) \textbf{7} dô] Da O L R W Z  $\cdot$ minnen geltes] minne geltes O (L) (M) Q W Fr29 Fr32 Fr71 minen geltes R minnen gerndes Z \textbf{8} ein] einem Fr36  $\cdot$ Babilon] Babýlon L (Fr32) babylon Fr71 \textbf{9} tjostieren] stritte R \textbf{10} drab] abe Q  $\cdot$ daz was dem] daz was im O (M) (R) Z Fr71 das im L dem wasz daz Q ::: was dem Fr29 (Fr32) \textbf{11} mit im iht] icht mit im Q (Z) Fr71 [*cht]: icht sandim R  $\cdot$ var] gar R \textbf{12} jâ er] Er O M Q R Z Fr29 (Fr32) o\textit{m. } L  $\cdot$ sîne] siner O ouch sine L \textbf{13} Ja [*]: Das war des sint si fro O  $\cdot$ Ja dest war sie sint es fro L (M) (R) (W) (Fr29) (Fr32)  $\cdot$ Ja zwar sie sein desz fro Q  $\cdot$ Ja des waren sie sin fro Z  $\cdot$ Si sint all geliche vrô Fr71 \textbf{14} si riten] svst riten si Fr71  $\cdot$ ander] andren R \textbf{15} dem] den L dē Q \textbf{17} muosen] muͤssen W \textbf{18} kinder] kint die L kient M  $\cdot$ im în] min O in W \textbf{19} ie] ÷E R  $\cdot$ zwei ein ander] zwei vnd zwei O zweý vnd zwey ein ander L zwene ein andren R zwey fuͦrten einander W  $\cdot$ an] bey Q \textbf{20} vrouwen] frawe Q \textbf{21} gekleidet] Becleidet L \textbf{22} küniginne] kúnige Q kúnginen R  $\cdot$ rîche] gliche M \textbf{23} vuogeten] feuchten Q  $\cdot$ grôzen] grosze M Q (R) (Z) \textbf{24} dô] Da M Z  $\cdot$ gesach] gesahen Fr71  $\cdot$ den] dem O  $\cdot$ Antschevin] anschevin G O (Fr71) Anshevin L (Z) (Fr29) (Fr32) anschyn M [anschein]: anscheuin Q ansheuin R antscheuin W \textbf{25} minniclîch] myndiglich Q  $\cdot$ gevar] var M R (Fr32) \textbf{26} er entslôz] entschlos L R W her yn slosz M \textbf{27} ez] Er \textit{nachträglich korrigiert zu:} Esz Q \textbf{28} daz] Da L  $\cdot$ beslôz dâ vor] vor beslosz Q beschos da vor R  $\cdot$ wîpheit] wipplicheit R \textbf{29} sim engegene] sie do gegen im L (W) \textbf{30} ir] Ern M (Z) (Fr71) \newline
\end{minipage}
\hspace{0.5cm}
\begin{minipage}[t]{0.5\linewidth}
\small
\begin{center}*T
\end{center}
\begin{tabular}{rl}
 & rîch\textit{iu} kleider dar getragen.\\ 
 & diu leit er an. \textbf{sus} hôrt ich sagen,\\ 
 & daz diu \textbf{rîche} wæren.\\ 
 & Anker, die swæren,\\ 
5 & von arâbischem golde\\ 
 & \textbf{lâgen} drûfe, als er wolde.\\ 
 & dô saz der minnegeltes lôn\\ 
 & ûf \textbf{ein} ors, daz ein Babylon\\ 
 & gegen im \textbf{durch} tjostieren reit;\\ 
10 & den stach er drabe. daz was \textbf{dem} leit.\\ 
 & Ob sîn wirt \textbf{mit im iht} var?\\ 
 & er und sîne rîter gar.\\ 
 & \textbf{Jâ deiswâr, si sint e\textit{s}} vrô.\\ 
 & si riten mit ein ander dô\\ 
15 & und erbeizten vor dem palas,\\ 
 & dâ manec \textbf{vrouwe} ûffe was.\\ 
 & \textbf{si} \textbf{muosen wol gekleidet} sîn.\\ 
 & Sîn\textit{iu} kinder liefen vor im în,\\ 
 & ie zwei ein ander \textbf{an} der hant.\\ 
20 & ir hêrre manege vrouwen vant,\\ 
 & \textbf{gekleidet} wünneclîche.\\ 
 & Der küneginne rîche\\ 
 & \textbf{vuogten ir ougen} \textbf{grôzen} pîn,\\ 
 & dô si \textbf{gesach} den Anschevin.\\ 
25 & der was \textbf{sô} minneclîche \textbf{var},\\ 
 & daz er entslôz ir herze gar,\\ 
 & ez wære ir liep oder leit;\\ 
 & daz \textbf{beslôz} dâ vor ir wîpheit.\\ 
 & Ein wênic si \textbf{nâher gegen im} \textbf{gienc}.\\ 
30 & \textbf{vil minneclîchen sin enpfienc}\\ 
\end{tabular}
\scriptsize
\line(1,0){75} \newline
T U V \newline
\line(1,0){75} \newline
\textbf{4} \textit{Majuskel} T  \textbf{11} \textit{Majuskel} T  \textbf{13} \textit{Majuskel} T  \textbf{18} \textit{Majuskel} T  \textbf{22} \textit{Majuskel} T  \textbf{29} \textit{Majuskel} T  \newline
\line(1,0){75} \newline
\textbf{1} rîchiu] rîche T \textbf{5} arâbischem] Arabescem T arabeschem U \textbf{8} Babylon] Babylôn T babilon V \textbf{10} drabe] abe U  $\cdot$ dem] im V \textbf{11} wirt] [wip]: wirt T \textbf{13} es] ez T \textbf{16} Do vffe manige swarze vreuͦwe was U (V) \textbf{17} muosen] mvesen T (V)  $\cdot$ gekleidet] bekleidet V \textbf{18} Sîniu] Sine T  $\cdot$ kinder] kinde U  $\cdot$ liefen] liefent alle V \textbf{19} zwei] zwei fvͦrten V \textbf{20} vrouwen] vreuͦwe U (V) \textbf{23} grôzen] groze U (V) \textbf{24} Anschevin] Anscevin T Anscheuin V \textbf{25} var] gevar U (V) \textbf{26} er entslôz] [*los]: entslos V \textbf{30} enpfienc] vmbe vieng V \newline
\end{minipage}
\end{table}
\end{document}
