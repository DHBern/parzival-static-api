\documentclass[8pt,a4paper,notitlepage]{article}
\usepackage{fullpage}
\usepackage{ulem}
\usepackage{xltxtra}
\usepackage{datetime}
\renewcommand{\dateseparator}{.}
\dmyyyydate
\usepackage{fancyhdr}
\usepackage{ifthen}
\pagestyle{fancy}
\fancyhf{}
\renewcommand{\headrulewidth}{0pt}
\fancyfoot[L]{\ifthenelse{\value{page}=1}{\today, \currenttime{} Uhr}{}}
\begin{document}
\begin{table}[ht]
\begin{minipage}[t]{0.5\linewidth}
\small
\begin{center}*D
\end{center}
\begin{tabular}{rl}
\textbf{365} & \textit{\begin{large}S\end{large}}wem \textbf{wâriu} \textbf{liebe} ie erholte,\\ 
 & daz er \textbf{herzeminne} dolte,\\ 
 & herzen minne ist des \textbf{erkant},\\ 
 & daz herze ist \textbf{rehter} minne \textbf{ein} pfant,\\ 
5 & alsô versetzet unt verselt,\\ 
 & dechein munt ez nimmer \textbf{gar} volzelt,\\ 
 & waz \textbf{minne wunders} vüegen kan.\\ 
 & ez sî wîb oder man,\\ 
 & die krenket herzen minne\\ 
10 & vil dicke an hôhem sinne.\\ 
 & Obie unt Melyanz,\\ 
 & \textbf{ir} zweier minne was \textbf{sô} ganz\\ 
 & unt stuont mit solhen triwen,\\ 
 & sîn zorn iuch solde riwen,\\ 
15 & daz er \textbf{mit zorne} von ir reit.\\ 
 & des gab \textbf{in} trûren solhez leit,\\ 
 & daz ir kiusche wart gein zorne balt.\\ 
 & unschuldec Gawan des engalt\\ 
 & unt ander, die ez \textbf{mit ir dâ} liten.\\ 
20 & si \textbf{kom} dicke ûz \textbf{vrouwelîchen} siten.\\ 
 & sus vlaht ir kiusche sich in zorn.\\ 
 & \textbf{ez} was ir bêder ougen dorn,\\ 
 & \textbf{swâ} si den werden man gesach.\\ 
 & ir herze Melyanze jach,\\ 
25 & er \textbf{m\textit{üe}se} vor ûz der \textbf{hôste} sîn.\\ 
 & si dâhte: "ob er mich \textbf{lêret} pîn,\\ 
 & den \textbf{sol} ich gerne \textbf{durch in} hân.\\ 
 & den \textbf{jungen, werden, süezen} man\\ 
 & vor alder werlde ich minne,\\ 
30 & dar jagent mich herzen sinne."\\ 
\end{tabular}
\scriptsize
\line(1,0){75} \newline
D Fr3 Fr4 \newline
\line(1,0){75} \newline
\textbf{1} \textit{Initiale} D  \textbf{11} \textit{Initiale} Fr3 Fr4  \newline
\line(1,0){75} \newline
\textbf{1} Swem] ÷wem D  $\cdot$ wâriu] varwe Fr3 Fr4 \textbf{3} Her:::es erkant pfant Fr3  $\cdot$ herzen] herze Fr4 \textbf{4} pfant] \textit{om.} Fr3 \textbf{5} verselt] verselt celt Fr3 \textbf{6} volzelt] vol Fr3 \textbf{9} herzen] :::ze Fr3 herze Fr4 \textbf{10} hôhem] hohen Fr4 \textbf{11} Obie] Obŷe D Obẏe Fr4  $\cdot$ Melyanz] mel:anz Fr3 melẏanz Fr4 \textbf{15} reit] reit balt Fr3 \textbf{16} in] ir Fr3 \textbf{17} balt] \textit{om.} Fr3 \textbf{20} vrouwelîchen] vrowen Fr3 \textbf{22} dorn] zlon Fr3 \textbf{24} Melyanze] melianze Fr3 melẏanze Fr4 \textbf{25} müese] mvͦse D (Fr3) (Fr4) \textbf{30} herzen] herze Fr3 \newline
\end{minipage}
\hspace{0.5cm}
\begin{minipage}[t]{0.5\linewidth}
\small
\begin{center}*m
\end{center}
\begin{tabular}{rl}
 & \begin{large}W\end{large}em \textbf{w\textit{â}riu} \textbf{minne} \textit{ie} erholte,\\ 
 & daz er \textbf{herzeminne} dolte,\\ 
 & herzeminne ist des \textbf{erkant},\\ 
 & daz herze ist \textbf{rehter} minne pfant,\\ 
5 & alsô versetzet und verselt,\\ 
 & d\textit{e}kein munt ez niemer volzelt,\\ 
 & waz \textbf{minne wunders} vüegen kan.\\ 
 & ez sî wîp oder man,\\ 
 & die krenket herzeminne\\ 
10 & vil dicke an hôhem sinne.\\ 
 & Obie und \textit{M}elianz,\\ 
 & \textbf{ir} zwe\textit{i}er minne was \textbf{sô} ganz\\ 
 & und stuont mit solichen triuwen,\\ 
 & sîn zorn iuch solte riuwen,\\ 
15 & daz er \textbf{mit zorne} von ir reit.\\ 
 & des gap \textbf{in} trûren solichez leit,\\ 
 & daz ir kiusche war\textit{t} gegen zorn bal\textit{t}.\\ 
 & unschuldic G\textit{a}wan des engal\textit{t}\\ 
 & und ander, die ez \textbf{dâ mit im} liten.\\ 
20 & si \textbf{kam} dicke ûz \textbf{vroulîchen} siten.\\ 
 & sus vlaht ir kiusche sich in zorn.\\ 
 & \textbf{ez} was ir beider ougen dorn,\\ 
 & \textbf{wâ} si den werde\textit{n m}an gesach.\\ 
 & ir herze Melianze jach,\\ 
25 & er \textbf{müese} vor ûz der \textbf{hœheste} sîn.\\ 
 & si dâhte: "obe er mich \textbf{lêre} pîn,\\ 
 & den \textbf{sol} ich gerne \textbf{durch in} hân.\\ 
 & den \textbf{jungen, süezen, werden} man\\ 
 & vor aller der werlte ich minne,\\ 
30 & dar jagent mich herzen sinne."\\ 
\end{tabular}
\scriptsize
\line(1,0){75} \newline
m n o \newline
\line(1,0){75} \newline
\textbf{1} \textit{Initiale} m   $\cdot$ \textit{Capitulumzeichen} n  \newline
\line(1,0){75} \newline
\textbf{1} Wem wâriu] Wem were m Wen wol were n Der wore o  $\cdot$ ie] niht m e o \textbf{3} \textit{Die Verse 365.3-4 fehlen} n o  \textbf{6} dekein] Do kein m n  $\cdot$ niemer] [nẏmen]: nẏmer o \textbf{11} Obie] Obye n Obe o  $\cdot$ und] >vnd< o  $\cdot$ Melianz] [m*]: neliancz m meliantz n meliancz o \textbf{12} zweier] zweiwer m \textbf{13} mit] in o \textbf{14} iuch] mich n \textit{om.} o \textbf{15} mit] nit o \textbf{16} in trûren] dasz gap ẏn trurigen o \textbf{17} ir] er o  $\cdot$ wart] war m  $\cdot$ balt] balder m balde n o \textbf{18} Gawan] gewan m  $\cdot$ des] das o  $\cdot$ engalt] engalte m (n) \textbf{19} ez dâ] \textit{om.} n o \textbf{20} vroulîchen] froͯlichen m froͯmlichem n fruͯnlichen o \textbf{21} in] an o \textbf{23} den] \textit{om.} o  $\cdot$ werden man] werden werden man m \textbf{24} Melianze] meliantz n meliancz o \textbf{25} müese] musse m muste o \textbf{26} dâhte] gedochte n (o)  $\cdot$ lêre] leret n o \textbf{30} jagent mich herzen] jagt mich hercze o \newline
\end{minipage}
\end{table}
\newpage
\begin{table}[ht]
\begin{minipage}[t]{0.5\linewidth}
\small
\begin{center}*G
\end{center}
\begin{tabular}{rl}
 & swem \textbf{rehtiu} \textbf{liebe} ie erholte,\\ 
 & daz er \textbf{herzeliebe} dolte,\\ 
 & herzeminne ist des \textbf{bekant},\\ 
 & daz herze ist \textbf{rehter} minne pfant,\\ 
5 & alsô versetzet unde verselt,\\ 
 & dehein munt ez nimer \textbf{gar} volzelt,\\ 
 & waz \textbf{minne wunders} vüegen kan.\\ 
 & ez sî wîp oder man,\\ 
 & \begin{large}D\end{large}ie krenket herzeminne\\ 
10 & vil dicke an hôhem sinne.\\ 
 & Obie unde Melianz,\\ 
 & \textbf{der} zweier minne was \textbf{sô} ganz\\ 
 & unde stuont mit solhen triwen,\\ 
 & sîn zorn iuch solt riwen,\\ 
15 & daz er \textbf{sô \textit{zorn}ic} von ir reit.\\ 
 & des gap \textbf{ir} trûren solhez leit,\\ 
 & daz ir kiusche wart gein zorne balt.\\ 
 & unschul\textit{d}ic Gawan des engalt\\ 
 & unde ander, diez \textbf{mit ir dâ} liten.\\ 
20 & si \textbf{kom} dicke ûz \textbf{vrou\textit{lîch}en} siten.\\ 
 & sus vlaht ir kiusche sich in zorn.\\ 
 & \textbf{ez} was ir beider ougen dorn,\\ 
 & \textbf{swâ} si den werden man gesach.\\ 
 & ir herze Melianz jach,\\ 
25 & er \textbf{solt} vor ûz der \textbf{beste} sîn.\\ 
 & si dâhte: "ober mich \textbf{lêret} pîn,\\ 
 & den \textbf{wil} ich gerne \textbf{von im} hân.\\ 
 & den \textbf{werde\textit{n}, junge\textit{n}, süezen} man\\ 
 & vor alder werlt ich minne,\\ 
30 & dar jagent mich herzen sinne."\\ 
\end{tabular}
\scriptsize
\line(1,0){75} \newline
G I O L M Q R Z Fr21 \newline
\line(1,0){75} \newline
\textbf{1} \textit{Initiale} O L M Z Fr21   $\cdot$ \textit{Capitulumzeichen} R  \textbf{3} \textit{Initiale} I  \textbf{9} \textit{Initiale} G  \textbf{17} \textit{Initiale} I  \newline
\line(1,0){75} \newline
\textbf{1} swem] swen I ÷wem O DEm L Wenn Q Wem R  $\cdot$ rehtiu] rechte R  $\cdot$ erholte] erholt Z \textbf{2} er] \textit{om.} I ir M  $\cdot$ herzeliebe] heize minne O hertzen mýnne L (M) (R) (Z) (Fr21) herte minne Q  $\cdot$ dolte] dolt Z \textbf{4} rehter] reht der Fr21 \textbf{5} verselt] verseret Q verschwelt R \textbf{6} gar] \textit{om.} L R  $\cdot$ volzelt] gezelt I (M) erczelt R \textbf{7} waz] Wan O (Q) Z Wans M  $\cdot$ wunders] wunder I (Q)  $\cdot$ vüegen] fuge M \textbf{9} herzeminne] herte minne Q \textbf{10} \textit{Versdoppelung 365.10 (²R) nach 365.12; Lesarten des vorausgehenden Verses mit ¹R bezeichnet} R   $\cdot$ vil] Vnd vil \textsuperscript{1}\hspace{-1.3mm} R  $\cdot$ an] von I  $\cdot$ hôhem] hochen R \textbf{11} Obie] Obŷe O Oblye Q Obye R Z Obẏe Fr21  $\cdot$ Melianz] Melyanz O melyans Q meliancz R meliantz Z \textbf{12} zweier] weir Q  $\cdot$ ganz] [grosz]: gantz Q \textbf{14} iuch solt] euch sode Q soͯlt uch R \textbf{15} zornic] trvrch G  $\cdot$ von] vor Q \textbf{16} des] Da M \textbf{17} ir] \textit{om.} O  $\cdot$ kiusche] kusse M  $\cdot$ gein zorne] von zorn I zu Q gen Jm zorne R \textbf{18} unschuldic] vnschulch G \textbf{19} ir] im I  $\cdot$ dâ liten] do liten Q doltent R \textbf{20} kom] chomen I  $\cdot$ vroulîchen] froͮwen G frolichen I frevelichen O \textbf{21} sus] Vsz M  $\cdot$ ir kiusche sich] sich ir chuske I (O)  $\cdot$ zorn] dorn R \textbf{22} dorn] zorn R \textbf{23} swâ] Wo L (M) Q (R)  $\cdot$ den] dem I  $\cdot$ gesach] er sach O (L) (M) (Q) (R) (Z) \textbf{24} Melianz] Melianzen I Fr21 Melyanzen O Meliantze L (Z) Melianze M (Q) Meliancze R \textbf{25} vor ûz] vor O uͯsz vor L  $\cdot$ der beste] den hosten O der hohste L (M) (Q) (R) Z (Fr21) \textbf{26} lêret] lerte I Q \textbf{28} werden jungen süezen] werde ivnge soͮzen G werden suͤzen iungen I ivngen werden svͦzen O (L) (M) (Q) (R) (Z) (Fr21) \textbf{29} alder] alle der R  $\cdot$ ich] ich in I \textbf{30} dar jagent] dar iaget I Do iaget Q Der iagt Z  $\cdot$ herzen] herte Q herze O (L) (M) (R) Fr21 \newline
\end{minipage}
\hspace{0.5cm}
\begin{minipage}[t]{0.5\linewidth}
\small
\begin{center}*T
\end{center}
\begin{tabular}{rl}
 & \textit{swem} \textit{\textbf{rehtiu}} \textit{\textbf{liebe}} \textit{ie erholte,}\\ 
 & \textit{daz er \textbf{herzeminne} dolte,}\\ 
 & herzen minne ist des \textbf{bekant},\\ 
 & daz herze ist \textbf{der} minnen pfant,\\ 
5 & alsô versetzet unde verselt,\\ 
 & dehein munt ez niemer volzelt,\\ 
 & waz \textbf{wunders minne} vüegen kan.\\ 
 & ez sî wîp oder man,\\ 
 & d\textit{ie} krenket herzen minne\\ 
10 & vil dicke an hôhe\textit{m} sinne.\\ 
 & Obye unde Melyanz,\\ 
 & \textbf{der} zweier minne was ganz\\ 
 & \textbf{ê}, unde stuont mit sölhen triuwen,\\ 
 & sîn zorn iu\textit{ch} solte riuwen,\\ 
15 & daz er \textbf{sô trûric} von ir reit.\\ 
 & des gab \textbf{im} trûren solhez leit,\\ 
 & \textit{daz ir kiusche wart gegen zorne balt.}\\ 
 & unschuldic Gawan des engalt\\ 
 & unde andere, diez \textbf{mit ir dâ} liten.\\ 
20 & si \textbf{komen} dicke ûz \textbf{vriuntlîchen} siten.\\ 
 & sus vlaht ir kiusche sich in zorn.\\ 
 & \textbf{daz} was ir beider ougen dorn,\\ 
 & \textbf{sô} si den werden man gesach.\\ 
 & ir herze Melyanze jach,\\ 
25 & er \textbf{solte} vor ûz der \textbf{hôste} sîn.\\ 
 & \textit{si dâhte: "ob er mich \textbf{lêret} pîn,}\\ 
 & den \textbf{wil} ich gerne \textbf{von im} hân.\\ 
 & den \textbf{jungen, werden, süezen} man,\\ 
 & vor alder werlt ich \textbf{in} minne,\\ 
30 & dar \textbf{zuo} jagent mich \textbf{mînes} herzen sinne."\\ 
\end{tabular}
\scriptsize
\line(1,0){75} \newline
T V W \newline
\line(1,0){75} \newline
\textbf{1} \textit{Initiale} W  \newline
\line(1,0){75} \newline
\textbf{1} \textit{Die Verse 365.1-2 fehlen (Zeilen ausgespart)} T   $\cdot$ swem] WEm W  $\cdot$ erholte] erholt W \textbf{2} herzeminne dolte] hiesse minne dolt W \textbf{3} herzen] Hei zuͦ W \textbf{4} der minnen] [reht*]: rehter minnen V recht der minne W \textbf{6} volzelt] gar gezelt W \textbf{7} wunders minne] minne wunders W \textbf{9} die krenket] der krenket T Die [kr*ket]: krenket V Die krencken W  $\cdot$ herzen] herze V \textbf{10} hôhem] hohen T \textbf{11} Obye] Obẏe V Obie W  $\cdot$ Melyanz] melianz V meliantz W \textbf{12} ganz] so ganz V (W) \textbf{13} ê] \textit{om.} V \textbf{14} iuch] îu T \textbf{15} trûric von ir] [*]: zornig von ir V zornig vor eúch W \textbf{16} gab im] gap in V gewan nie W \textbf{17} \textit{Vers 365.17 fehlt (Zeile nach Vers 365.18 ausgespart)} T   $\cdot$ ir] er W  $\cdot$ wart gegen zorne] gen zoren waz so W \textbf{19} ir dâ] [*]: im do V im do W \textbf{20} komen] kam V (W)  $\cdot$ vriuntlîchen] [v*]: froͮwelichen V \textbf{21} ir] die W \textbf{24} Melyanze] melianze V meliantz W \textbf{26} \textit{Vers 365.26 fehlt (Zeile ausgespart)} T   $\cdot$ dâhte] gedachte W \textbf{28} süezen] suͤsseu W \textbf{29} in] \textit{om.} V W \textbf{30} zuo jagent] iaget W \newline
\end{minipage}
\end{table}
\end{document}
