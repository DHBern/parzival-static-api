\documentclass[8pt,a4paper,notitlepage]{article}
\usepackage{fullpage}
\usepackage{ulem}
\usepackage{xltxtra}
\usepackage{datetime}
\renewcommand{\dateseparator}{.}
\dmyyyydate
\usepackage{fancyhdr}
\usepackage{ifthen}
\pagestyle{fancy}
\fancyhf{}
\renewcommand{\headrulewidth}{0pt}
\fancyfoot[L]{\ifthenelse{\value{page}=1}{\today, \currenttime{} Uhr}{}}
\begin{document}
\begin{table}[ht]
\begin{minipage}[t]{0.5\linewidth}
\small
\begin{center}*D
\end{center}
\begin{tabular}{rl}
\textbf{551} & \begin{large}G\end{large}awan \textbf{tragen} alle drî\\ 
 & und eine salsen dar bî.\\ 
 & diu \textbf{juncvrouwe} niht vermeit,\\ 
 & mit \textbf{guoten} zühten si sneit\\ 
5 & Gawane süeziu mursel\\ 
 & ûf \textbf{einem} \textbf{blankem} \textbf{wastel}\\ 
 & mit ir \textbf{clâren} henden.\\ 
 & \textbf{dô} sprach \textbf{si}: "ir sult senden\\ 
 & dirre \textbf{gebrâten} vogel einen\\ 
10 & - wan si hât enkeinen -,\\ 
 & hêrre, mîner muoter dar."\\ 
 & er sprach zer meide wol gevar,\\ 
 & daz er gern \textbf{ir willen} tæte\\ 
 & \textbf{dâr an} oder \textbf{swes} si bæte.\\ 
15 & Ein galander wart gesant\\ 
 & der wirtinne. Gawans hant\\ 
 & wart mit zühten \textbf{vil} genigen\\ 
 & unt des wirtes danken niht verswigen.\\ 
 & \textbf{Dô} brâht ein des wirtes sun\\ 
20 & \textbf{Purzeln} und lâtûn,\\ 
 & gebrochen in \textbf{den} vînæger.\\ 
 & \textbf{ze} grôzer kraft daz unwæger\\ 
 & ist die lenge \textbf{solhiu} nar;\\ 
 & man wirt ir \textbf{ouch} niht wol gevar.\\ 
25 & solch varwe tuot die wârheit kunt,\\ 
 & die man sloufet in den munt.\\ 
 & gestrichen varwe ûfez vel\\ 
 & ist selten worden lobes \textbf{hel}.\\ 
 & swelch wîplîch herze \textbf{ist stæte} ganz,\\ 
30 & ich wæne, diu treit den besten \textbf{glanz}.\\ 
\end{tabular}
\scriptsize
\line(1,0){75} \newline
D \newline
\line(1,0){75} \newline
\textbf{1} \textit{Initiale} D  \textbf{15} \textit{Majuskel} D  \textbf{19} \textit{Majuskel} D  \textbf{20} \textit{Majuskel} D  \newline
\line(1,0){75} \newline
\newline
\end{minipage}
\hspace{0.5cm}
\begin{minipage}[t]{0.5\linewidth}
\small
\begin{center}*m
\end{center}
\begin{tabular}{rl}
 & Gawan \textbf{tragen} alle drî\\ 
 & und ein salsen dar bî.\\ 
 & diu \textbf{juncvrouwe} niht vermeit,\\ 
 & mit \textbf{guoten} zühten si sneit\\ 
5 & Gawane süeziu mursel\\ 
 & ûf \textbf{einen} \textbf{blanken} \textbf{wastel}\\ 
 & mit ir \textbf{clâren} henden.\\ 
 & \textbf{dô} sprach \textbf{si}: "ir sulle\textit{t} \textit{s}enden\\ 
 & diser vogel einen\\ 
10 & - wan si het dekeinen -,\\ 
 & hêrre, mîner muoter dar."\\ 
 & er sprach zer megde wol gevar,\\ 
 & da\textit{z} er gerne \textbf{ir willen} tæte\\ 
 & \textbf{dâr an} oder \textbf{waz} si bæte.\\ 
15 & ein galander wart gesant\\ 
 & der wirtinne. Gawanes hant\\ 
 & wart mit zühten \textbf{des} genigen\\ 
 & und des wirtes danken niht verswigen.\\ 
 & \textbf{dô} brâht ein des wirtes sun\\ 
20 & \textbf{p\textit{u}rzeln} und lâtûn,\\ 
 & gebrochen in \textbf{den} vîn\textit{æ}ge\textit{r}.\\ 
 & \textbf{zuo} grôzer \textit{k}raft daz unwæger\\ 
 & ist die lenge \textbf{solichiu} \textit{n}ar;\\ 
 & man wirt ir \textbf{ouch} niht wol gevar.\\ 
25 & solich varwe tuot die wârheit kunt,\\ 
 & die man sloufet in den munt.\\ 
 & gestrichen varwe ûf daz vel\\ 
 & is\textit{t} \textit{s}elten worden lobes \textbf{hel}.\\ 
 & welich wîplîch herz \textbf{ist stæte} ganz,\\ 
30 & ich wæne, diu treit den besten \textbf{kranz}.\\ 
\end{tabular}
\scriptsize
\line(1,0){75} \newline
m n o \newline
\line(1,0){75} \newline
\newline
\line(1,0){75} \newline
\textbf{1} Gawan] Gawanen n \textbf{5} Gawane] Gawanen n Gawan o \textbf{6} einen] einem n o \textbf{8} sullet senden] suͯllent selle senden m \textbf{9} vogel] [vol]: vogel n \textbf{10} dekeinen] do keinen n \textbf{13} daz] Dar m \textbf{16} der wirtinne] Die wurtnic o  $\cdot$ Gawanes] gawans o \textbf{20} purzeln] Parceln m (n) \textbf{21} vînæger] vingere m \textbf{22} kraft] rafft m  $\cdot$ unwæger] vngeweger o \textbf{23} nar] var m \textbf{28} ist selten] Jst der seltten m \textbf{29} welich] Wolich o \textbf{30} \textit{Vers 551.30 fehlt} o   $\cdot$ kranz] glantz n \newline
\end{minipage}
\end{table}
\newpage
\begin{table}[ht]
\begin{minipage}[t]{0.5\linewidth}
\small
\begin{center}*G
\end{center}
\begin{tabular}{rl}
 & Gawan \textbf{tragen} alle drî\\ 
 & unde eine salsen dar bî.\\ 
 & diu \textbf{juncvrouwe} niht vermeit,\\ 
 & mit \textbf{guoten} zühten si sneit\\ 
5 & Gawan süeziu mursel\\ 
 & ûf \textbf{einen} \textbf{blanken} \textbf{wastel}\\ 
 & mit ir \textbf{blanken} henden.\\ 
 & \textbf{dô} sprach \textbf{si}: "ir sult senden\\ 
 & dirre \textbf{gebrâten} vogel einen\\ 
10 & - wan s\textit{i} hât neheinen -,\\ 
 & hêrre, mîner muoter dar."\\ 
 & er sprach zer meide wol gevar,\\ 
 & daz er gerne \textbf{ir willen} tæte\\ 
 & \textbf{dâr an} ode \textbf{swes} si bæte.\\ 
15 & ein galander wart gesant\\ 
 & der wirtinne. Gawanes hant\\ 
 & wart mit zühten \textbf{vil} genigen\\ 
 & unde des wirtes danken niht verswigen.\\ 
 & \textbf{dô} brâhte ein des wirtes sun\\ 
20 & \textbf{purzeln} unde lâtûn,\\ 
 & gebrochen in \textbf{den} vînæger.\\ 
 & \textbf{ze} grôzer kraft daz unwæger\\ 
 & ist die lenge \textbf{al}\textbf{solhiu} nar;\\ 
 & man wirt ir \textbf{ouch} niht wol gevar.\\ 
25 & sol\textit{h} varwe tuot die wârheit kunt,\\ 
 & die man sloufet in den munt.\\ 
 & gestrichen varwe ûffez vel\\ 
 & ist selten worden lobes \textbf{hel}.\\ 
 & swelh wîplîch herze \textbf{stæte ist} ganz,\\ 
30 & ich wæne, diu treit den besten \textbf{glanz}.\\ 
\end{tabular}
\scriptsize
\line(1,0){75} \newline
G I L M Z \newline
\line(1,0){75} \newline
\textbf{1} \textit{Initiale} L Z  \textbf{3} \textit{Initiale} M  \textbf{15} \textit{Initiale} I  \newline
\line(1,0){75} \newline
\textbf{1} Gawan] Gawane L Gawanen Z \textbf{2} salsen] saszen L \textbf{3} juncvrouwe] juncfrouwen M \textbf{5} Gawan] Gawane L M Gawanen Z \textbf{6} einen] eine I einem L (M) Z  $\cdot$ wastel] kastel Z \textbf{7} blanken] claren L (M) (Z) \textbf{8} dô] Da M \textbf{9} gebrâten] \textit{om.} L \textbf{10} si] sin G  $\cdot$ hât neheinen] hat ir keinen Z \textbf{11} hêrre] \textit{om.} I \textbf{13} ir] zcu M \textbf{14} swes] wez L (M) \textbf{16} Gawanes] Gawans I L (Z) \textbf{18} danken] danc I (M) \textbf{19} dô] Dach M \textbf{20} purzeln] porcellen I Bvrzel L Parcziln M (Z) \textbf{21} den] em L  $\cdot$ vînæger] eber I \textbf{23} die lenge] lange L  $\cdot$ alsolhiu] solche L (M)  $\cdot$ nar] lýp nar L \textbf{24} ouch] \textit{om.} I \textbf{25} solh] Solhe G \textbf{26} man] \textit{om.} M \textbf{27} gestrichen] Gestrichnev I \textbf{29} swelh] Welch L (M)  $\cdot$ wîplîch] wipplich G I liplich M  $\cdot$ stæte ist] ist state L (M) (Z) \textbf{30} wæne] gihe I  $\cdot$ besten glanz] hohesten cranz I \newline
\end{minipage}
\hspace{0.5cm}
\begin{minipage}[t]{0.5\linewidth}
\small
\begin{center}*T
\end{center}
\begin{tabular}{rl}
 & Gawane \textbf{getragen} alle drî\\ 
 & unde eine salse dâ bî.\\ 
 & Diu \textbf{maget} \textbf{ir dienst} niht vermeit:\\ 
 & mit \textbf{grôzen} zühten si sneit\\ 
5 & Gawane süeziu mursel\\ 
 & ûf \textbf{einem} \textbf{blanken} \textbf{platel}\\ 
 & mit ir \textbf{clâren} henden.\\ 
 & \textbf{si} sprach: "ir sult senden\\ 
 & dirre \textbf{gebrâten} vogel einen\\ 
10 & - wande si hât dekeinen -,\\ 
 & hêrre, mîner muoter dar."\\ 
 & er sprach zer megede wol gevar,\\ 
 & daz er \textbf{daz} gerne tæte\\ 
 & oder \textbf{swes} si\textbf{n} \textbf{anders} bæte.\\ 
15 & Eine galander wart gesant\\ 
 & der wirtîn. Gawanes hant\\ 
 & wart mit zühten \textbf{vil} genigen\\ 
 & unde des wirtes danken niht verswigen.\\ 
 & \textbf{\textit{\begin{large}N\end{large}}û} brâhte ein \textbf{knappe}, des wirtes sun,\\ 
20 & \textbf{burzel} unde latechûn,\\ 
 & gebrochen in vînæger.\\ 
 & \textbf{gegen} grôzer kraft daz unwæger\\ 
 & ist die lenge \textbf{disiu} nar;\\ 
 & man wirt ir niht wol gevar.\\ 
25 & Sölch varwe tuot di\textit{e} wârheit kunt,\\ 
 & die man sloufet in den munt.\\ 
 & gestrichen varwe ûfez vel\\ 
 & ist selten worden lobes \textbf{snel}.\\ 
 & swelch wîplîch herze \textbf{ist stæte} ganz,\\ 
30 & ich wæne, diu treit den besten \textbf{kranz}.\\ 
\end{tabular}
\scriptsize
\line(1,0){75} \newline
T U V W O Q R Fr39 \newline
\line(1,0){75} \newline
\textbf{1} \textit{Initiale} O Q   $\cdot$ \textit{Capitulumzeichen} R  \textbf{3} \textit{Majuskel} T  \textbf{15} \textit{Majuskel} T  \textbf{19} \textit{Initiale} T U V  \textbf{25} \textit{Majuskel} T  \newline
\line(1,0){75} \newline
\textbf{1} Gawane] ÷awane O Gawan Q Gawin R \textbf{2} salse] sase V \textbf{3} ir] irn U V \textit{om.} R \textbf{4} grôzen] \textit{om.} W O  $\cdot$ si] si dem herren O \textit{om.} R \textbf{5} Gawane] Gawinen R \textbf{6} einem] einē Q einen R  $\cdot$ blanken] blanchem O  $\cdot$ platel] wastel U V W O Q R \textbf{9} dirre] Disen Q \textbf{10} hât] enhat W \textbf{12} megede] unckfren R  $\cdot$ gevar] gar R \textbf{13} daz gerne] gerne irn willen V (W) (Q) (R) ir willen gerne O \textbf{14} oder] Dar an oder V W O (Q) (R)  $\cdot$ swes] wes U W was Q (R)  $\cdot$ sin] sie U (V) (W) (O) Q (R)  $\cdot$ anders] \textit{om.} V W O Q R \textbf{15} Eine] Ein W O Q R \textbf{16} Gawanes] Gawans U (W) O von gawans Q Gawains R \textbf{17} wart] Ym wart Q  $\cdot$ genigen] gedigen W \textbf{18} danken] dank R \textbf{19} Nû] ÷v T  $\cdot$ ein knappe] dar O \textbf{20} burzel] Pozzoln U Porzoln V Purtzeln W Q Bvceln O  $\cdot$ latechûn] latun W (O) (Q) lattuͦm R \textbf{21} in] in einem ezzîch in O \textbf{22} daz] \textit{om.} R  $\cdot$ unwæger] vnwnger R \textbf{23} die] div O \textbf{24} ir] ir doch W ovch O ir auch Q (R) (Fr39)  $\cdot$ wol gevar] wol gewar U R gevar Fr39 \textbf{25} varwe] fraw Q  $\cdot$ die] div T \textbf{26} sloufet] schleússet W \textbf{27} ûfez] vffens V \textbf{28} lobes] lones Q  $\cdot$ snel] hel V O Q R Fr39 \textbf{29} swelch] Welch U W Q (R)  $\cdot$ wîplîch] wibs O \textbf{30} besten] selben O hoͯsten R  $\cdot$ kranz] glantz V W (O) Q (R) (Fr39) \newline
\end{minipage}
\end{table}
\end{document}
