\documentclass[8pt,a4paper,notitlepage]{article}
\usepackage{fullpage}
\usepackage{ulem}
\usepackage{xltxtra}
\usepackage{datetime}
\renewcommand{\dateseparator}{.}
\dmyyyydate
\usepackage{fancyhdr}
\usepackage{ifthen}
\pagestyle{fancy}
\fancyhf{}
\renewcommand{\headrulewidth}{0pt}
\fancyfoot[L]{\ifthenelse{\value{page}=1}{\today, \currenttime{} Uhr}{}}
\begin{document}
\begin{table}[ht]
\begin{minipage}[t]{0.5\linewidth}
\small
\begin{center}*D
\end{center}
\begin{tabular}{rl}
\textbf{279} & wander hetes vil getân\\ 
 & vor Clamide ze Brandigan.\\ 
 & Keie durch daz sîn dienst liez:\\ 
 & unsælde ins vürsten swester hiez\\ 
5 & ze sêre alûnen mit \textbf{eime} stabe.\\ 
 & durch zuht \textbf{entweich} er dienstes abe.\\ 
 & \textbf{ouch was diu schulde} \textbf{niht verkorn}\\ 
 & von der meide wol geborn.\\ 
 & doch schuof er spîse dar genuoc;\\ 
10 & Kingrun \textbf{ez vür Orilusen} truoc.\\ 
 & Cunneware, diu \textbf{lobes} wîse,\\ 
 & sneit ir bruoder sîne spîse\\ 
 & mit ir blanken, linden hant.\\ 
 & vrou Jeschute von Karnant\\ 
15 & mit wîplîchen zühten az.\\ 
 & Artus, der künec, niht vergaz,\\ 
 & er enkœme, dâ \textbf{diu zwei} sâzen\\ 
 & unt \textbf{vriwentlîchen} âzen.\\ 
 & \textbf{Dô sprach er}: "geezzet ir \textbf{hînte} übel hie,\\ 
20 & \textbf{ez} \textbf{en}wart iedoch mîn wille nie.\\ 
 & ir \textbf{en}gesâzet nie über wirtes brôt,\\ 
 & der\textbf{z iu} mit bezzerem willen bôt,\\ 
 & sô gar âne wankes vâre.\\ 
 & mîn vrou Cunneware,\\ 
25 & ir sult \textbf{iwers} bruoder hie wol pflegen.\\ 
 & guote naht geb iu \textbf{der gotes} segen."\\ 
 & Artus vuor slâfen dô.\\ 
 & Oriluse wart gebettet sô,\\ 
 & \textbf{dâ} sîn vrou Jeschute pflac\\ 
30 & geselleclîche unz an den tac.\\ 
\end{tabular}
\scriptsize
\line(1,0){75} \newline
D \newline
\line(1,0){75} \newline
\textbf{19} \textit{Majuskel} D  \newline
\line(1,0){75} \newline
\textbf{6} dienstes] diens D \textbf{10} Kingrun] kingrv̂n D \textbf{14} Jeschute] Jescvte D \textbf{24} Cunneware] Cvnnewáre D \textbf{29} Jeschute] Jescvte D \newline
\end{minipage}
\hspace{0.5cm}
\begin{minipage}[t]{0.5\linewidth}
\small
\begin{center}*m
\end{center}
\begin{tabular}{rl}
 & wand er hetes vil getân\\ 
 & vor Clamide ze Brandigan.\\ 
 & Keie durch daz sînen dienest liez:\\ 
 & unsælde in des vürsten swester hiez\\ 
5 & ze sêre alûnen mit \textbf{einem} stabe.\\ 
 & durch zuht \textbf{entweich} er dienstes abe,\\ 
 & \textbf{wanne diu schulde was} \textbf{unverkorn}\\ 
 & von der mege\textit{de} wol geborn.\\ 
 & doch schuof er spîse dar genuoc;\\ 
10 & Kingrun \textbf{si vür Orilusen} truoc.\\ 
 & C\textit{u}nn\textit{e}w\textit{a}re, diu \textbf{lobes} wîse,\\ 
 & sneit ir bruoder \textit{sîne sp}îse\\ 
 & mit ir blanken, linden hant.\\ 
 & vrouwe Jeschut\textit{e} von Karnant\\ 
15 & mit wîplîchen zühten az.\\ 
 & Artus, der künic, niht vergaz,\\ 
 & er enk\textit{œ}me, dâ \textbf{si} sâzen\\ 
 & und \textbf{vrœlîchen} âzen.\\ 
 & \textbf{er sp\textit{r}ach}: "geezzet ir übel hie,\\ 
20 & \textbf{ez} \textbf{en}wart iedoch mîn wille nie.\\ 
 & ir \dag engeâzet\dag  nie über wirtes brôt,\\ 
 & der \textbf{si} mit bezzerem willen bôt,\\ 
 & sô gar âne wankes vâre.\\ 
 & mîn vrouwe Cunn\textit{e}ware,\\ 
25 & ir sult \textbf{mînes} bruoder hie wol pflegen.\\ 
 & guote naht gebe iu \textbf{den} segen."\\ 
 & Artus vuor slâfen dô.\\ 
 & Oriluse wart gebettet sô,\\ 
 & \textbf{d\textit{â}} sîn vrouwe Jeschute pflac\\ 
30 & geselleclîchen unz an den tac.\\ 
\end{tabular}
\scriptsize
\line(1,0){75} \newline
m n o \newline
\line(1,0){75} \newline
\newline
\line(1,0){75} \newline
\textbf{2} Brandigan] brandian o \textbf{3} Keie] Keẏe n Keẏ o  $\cdot$ daz] \textit{om.} n o \textbf{4} des] dasz o \textbf{6} entweich] etweich o \textbf{8} megede] mege m \textbf{10} Kingrun] Kingrunt n o  $\cdot$ si] vor sie o \textbf{11} Cunneware] Connwere m Conneware n Conne ware o \textbf{12} sîne spîse] wise m \textbf{14} Jeschute] jescutten m jescute n jescuten o  $\cdot$ Karnant] kornant n (o) \textbf{16} Artus] Artuͯs o \textbf{17} enkœme] enkome m enkomen n (o)  $\cdot$ dâ] do n o \textbf{18} vrœlîchen] frintlichen n (o) \textbf{19} sprach] spach m \textbf{20} enwart iedoch] [w*]: wart doch n \textbf{21} engeâzet] engassent m (n) engussent o \textbf{24} Cunneware] Cunnware m conneware n kumme ware o \textbf{25} mînes] mins \textit{nachträglich korrigiert zu:} vwern m \textbf{26} den] der gottes n der guͯte o \textbf{27} Artus] Artuͯs o  $\cdot$ dô] da o \textbf{28} Oriluse] Orilus n Orilús o \textbf{29} dâ] Do m n o  $\cdot$ Jeschute] jescutte m jescúte n jescuͯte o \newline
\end{minipage}
\end{table}
\newpage
\begin{table}[ht]
\begin{minipage}[t]{0.5\linewidth}
\small
\begin{center}*G
\end{center}
\begin{tabular}{rl}
 & wan er hetes vil getân\\ 
 & vor Clamide ze Brandigan.\\ 
 & Kay durch daz sîn dienst liez:\\ 
 & unsælde ins vürsten \textit{swes}ter hiez\\ 
5 & ze sêre alûnen mit \textbf{dem} stabe.\\ 
 & durch zuht \textbf{tet} er \textbf{sich} dienstes abe.\\ 
 & \textbf{ouch was diu schulde} \textbf{niht verkoren}\\ 
 & von der meide wolgeboren.\\ 
 & doch schuof er spîse dar genuoc;\\ 
10 & Kingrun \textbf{ez vür Orillus} truoc.\\ 
 & \textbf{vrou} Kuneware, diu wîse,\\ 
 & sneit ir bruoder sîne spîse\\ 
 & mit ir blanken, linden hant.\\ 
 & vrou Jeschute von Karnant\\ 
15 & mit wîplîchen zühten az.\\ 
 & Artus, der künic, niht vergaz,\\ 
 & er enkœme, dâ \textbf{diu zwei} sâzen\\ 
 & unde \textbf{\textit{l}i\textit{eplîch}e\textit{n}} âzen.\\ 
 & \textbf{er sprach}: "geezzet ir übele hie,\\ 
20 & \textbf{daz} wart iedoch mîn wille nie.\\ 
 & ir gesâzet nie über wirtes brôt,\\ 
 & der\textbf{z iu} mit bezzerem willen bôt,\\ 
 & sô gar âne wankes vâre.\\ 
 & mîn vrou Kuneware,\\ 
25 & ir sult \textbf{iuwers} bruoder hie wol pflegen.\\ 
 & guote naht gebe iu \textbf{der gotes} segen."\\ 
 & Artus vuor slâfen dô.\\ 
 & Orillus wart gebet sô,\\ 
 & \textbf{daz} sîn vrou Jeschute pflac\\ 
30 & geselliclîch unze an den tac.\\ 
\end{tabular}
\scriptsize
\line(1,0){75} \newline
G I O L M Q R Z Fr30 \newline
\line(1,0){75} \newline
\textbf{3} \textit{Initiale} O L R Z Fr30  \textbf{11} \textit{Initiale} L  \textbf{19} \textit{Initiale} I  \newline
\line(1,0){75} \newline
\textbf{1} hetes] het sin I hattis M  $\cdot$ vil] dicke L \textbf{2} Clamide] Glamide O clamẏde Fr30  $\cdot$ ze Brandigan] zebrandigan G Fr30 zuͯ Brandegan L Brandegan L \textbf{3} Kay] kaẏ G kain I ÷ey O KAý L Key M R Z Keẏ Fr30  $\cdot$ sîn] sinen L  $\cdot$ dienst] dienen Fr30 \textbf{4} ins] ym des M  $\cdot$ swester] tohter G \textbf{5} ze sêre] so ser I \textit{om.} O  $\cdot$ alûnen] Galunen I bluwen Z  $\cdot$ dem] einem O L (M) (Q) (R) Z Fr30  $\cdot$ stabe] stare Q \textbf{6} tet er sich] tet er sich des I entweich er O L Q R Z (Fr30) etweich er M  $\cdot$ dienstes] dienst R \textbf{7} niht verkoren] vnverkorn L nicht erkorn Q \textbf{8} wolgeboren] wol geworn Fr30 \textbf{9} spîse dar] dar spise L (R) Fr30 \textbf{10} Kingrun] kingruͤn I Kyngrvn O (M) (Q) (Fr30) Kýngrvn L Kygrun R  $\cdot$ ez] \textit{om.} I  $\cdot$ Orillus] Orilus I (O) (Q) R (Z) Orilusen M Oryllen Fr30  $\cdot$ truoc] ez truͤc I \textbf{11} \textit{Versfolge 279.12-11} Fr30   $\cdot$ Kuneware] kunuare I kvnware O (M) Cvneware L konware Q Cuͦnware R kunneware Z kvrnęware Fr30  $\cdot$ diu] des Z  $\cdot$ wîse] lobs wise O (L) (M) (Q) (R) (Z) \textbf{12} sneit] Ez sneit Fr30  $\cdot$ sîne] die R \textbf{13} linden] linde Q \textbf{14} vrou] Frowen Z  $\cdot$ Jeschute] ieschute G ieskute I Jescv̂te O jescuͯte L iescute M Q Jscute R iescuten Z  $\cdot$ von] vor Fr30 \textbf{16} Artus der künic] Artuͯs der kvnig L der chvnic Artus Fr30 \textbf{17} enkœme] knie R  $\cdot$ dâ] do Q \textbf{18} lieplîchen] mit ein ander G fruntlichen Z \textbf{19} er sprach] Do sprach er I O L (M) Q Z Fr30 \textbf{20} wart] enwart O L (M) Q (Z) (Fr30)  $\cdot$ iedoch] doch L Fr30  $\cdot$ wille] willen Q [hie]: nie Z \textbf{21} ir] Jrn O (M) Q Z (Fr30)  $\cdot$ nie] \textit{om.} L dach ny M  $\cdot$ brôt] gebot M \textbf{22} derz iu] Der ivz O (Z) Der uch R  $\cdot$ bezzerem] besszirn M (Z) (Fr30) \textbf{24} vrou] vrouwe fro M  $\cdot$ Kuneware] kunuare I kvnwar O (M) Cvnevare L konware Q kuͯnware R kunneware Z kvrneware Fr30 \textbf{25} hie wol] wol R wol hie Z \textbf{26} gebe] die geb I  $\cdot$ der] \textit{om.} I R \textbf{27} vuor slâfen] vor slaffin M schluͦffen fuͦr R  $\cdot$ dô] dar M \textbf{28} Orillus] Orilus I (O) Q R Z Oriluse M Orẏllen Fr30  $\cdot$ sô] do so Q \textbf{29} Da er siner frowen pflac Fr30  $\cdot$ daz] Do O Da L M  $\cdot$ sîn] min R  $\cdot$ Jeschute] ieschute G ieskute I Jescv̂te O Jescuͯte L iescute M Q Z Jscute R \textbf{30} geselliclîch] Gesellechlichen I (Q)  $\cdot$ an] vf I \newline
\end{minipage}
\hspace{0.5cm}
\begin{minipage}[t]{0.5\linewidth}
\small
\begin{center}*T
\end{center}
\begin{tabular}{rl}
 & wan \textit{er} hete\textit{s} vil getân\\ 
 & vor Clamide ze Brandigan.\\ 
 & Key durch daz sîn dienst liez:\\ 
 & unsælde in des vürsten swester hiez\\ 
5 & ze sêre alûnen mit \textbf{einem} stabe.\\ 
 & durch zuht \textbf{entweich} er dienstes abe.\\ 
 & \multicolumn{1}{l}{ - - - }\\ 
 & \multicolumn{1}{l}{ - - - }\\ 
 & doch schuof er spîse dar genuoc;\\ 
10 & Kyngrun \textbf{vür Orilus ez} truoc.\\ 
 & \textbf{Vrou} Cunnewar, diu \textbf{lobes} wîse,\\ 
 & sneit ir bruoder sîne spîse\\ 
 & mit ir blanken, linden hant.\\ 
 & vrou Jeschute von Garnant\\ 
15 & mit wîplîchen zühten az.\\ 
 & Artus, der künec, niht vergaz,\\ 
 & er enkæme, dâ \textbf{di\textit{u} zwei} sâzen\\ 
 & unde \textbf{lieplîchen} âzen.\\ 
 & \textbf{er sprach}: "geezzet ir übel hie,\\ 
20 & \textbf{daz} \textbf{en}wart iedoch mîn wille nie.\\ 
 & ir gesâzet nie über wirtes brôt,\\ 
 & der\textbf{z iu} mit bezzern willen bôt,\\ 
 & sô gar âne wankes vâr.\\ 
 & mîn vrou Cunnewar,\\ 
25 & ir sult \textbf{iuwers} bruoder hie wol pflegen.\\ 
 & guote naht gebiu \textbf{der gotes} segen."\\ 
 & Artus vuor slâfen dô.\\ 
 & Oriluse wart gebettet sô,\\ 
 & \textbf{aldâ} sîn vrou Jeschute pflac\\ 
30 & geselleclîche un\textit{z} an den tac.\\ 
\end{tabular}
\scriptsize
\line(1,0){75} \newline
T U V W \newline
\line(1,0){75} \newline
\textbf{11} \textit{Majuskel} T  \textbf{27} \textit{Majuskel} T  \newline
\line(1,0){75} \newline
\textbf{1} er hetes] hetez T  $\cdot$ vil] wol W \textbf{2} Clamide] klamide W \textbf{3} Key] Keyn V  $\cdot$ sîn] sinen V (W) \textbf{5} alûnen] allen U  $\cdot$ einem] [einen]: einem T \textbf{7} \textit{Die Verse 279.7-8 sind am Rand nachgetragen und später radiert:} ::: w:: die ::: / von der :egede wol geborn V   $\cdot$ [*]: Oͮch waz die schulde niht verkorn V \textbf{8} [*]: Von der megede wol geborn V \textbf{10} Kyngrun] Kyngruͦn U Kingrun W  $\cdot$ vür Orilus ez] [*]: ez fúr orilus V fúr orilus sy W \textbf{11} Cunnewar] kvnnewar T (W) Cvnneware U kvnneware V  $\cdot$ diu] \textit{om.} W \textbf{12} sîne] die W \textbf{14} Jeschute] Jescvte T (U) iescute V iestute W  $\cdot$ Garnant] [Ganant]: Garnant U karnant V \textbf{17} enkæme] queme U (V) (W)  $\cdot$ dâ] do V W  $\cdot$ diu] die T \textbf{18} lieplîchen] lieblich mit einander W \textbf{20} enwart] wart U V (W)  $\cdot$ iedoch] doch W \textbf{22} bezzern] besserme V (W) \textbf{24} Cunnewar] [cunnewar*]: cunneware V kunnewar W \textbf{26} gebiu der] gebe dir U \textbf{28} Oriluse] Orilus W  $\cdot$ gebettet] gebettes W \textbf{29} Jeschute] Jescvte T (U) (V) iestute W \textbf{30} geselleclîche] Húbschlichen W  $\cdot$ unz] vn T mit U \newline
\end{minipage}
\end{table}
\end{document}
