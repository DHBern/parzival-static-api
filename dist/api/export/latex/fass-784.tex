\documentclass[8pt,a4paper,notitlepage]{article}
\usepackage{fullpage}
\usepackage{ulem}
\usepackage{xltxtra}
\usepackage{datetime}
\renewcommand{\dateseparator}{.}
\dmyyyydate
\usepackage{fancyhdr}
\usepackage{ifthen}
\pagestyle{fancy}
\fancyhf{}
\renewcommand{\headrulewidth}{0pt}
\fancyfoot[L]{\ifthenelse{\value{page}=1}{\today, \currenttime{} Uhr}{}}
\begin{document}
\begin{table}[ht]
\begin{minipage}[t]{0.5\linewidth}
\small
\begin{center}*D
\end{center}
\begin{tabular}{rl}
\textbf{784} & \begin{large}Ü\end{large}ber al \textbf{den rinc} wart vernomen,\\ 
 & "Cundrie \textbf{la surziere} \textbf{ist} komen",\\ 
 & unt waz ir mære meinde.\\ 
 & Orgeluse \textbf{durch} liebe weinde,\\ 
5 & daz diu vrâge von Parzivale\\ 
 & die Anfortases quâle\\ 
 & solde machen wendec.\\ 
 & Artus, der prîses genendec,\\ 
 & ze Cundrien mit zühten sprach:\\ 
10 & "vrouwe, rîtet an iwer gemach,\\ 
 & lât iwer pflegen, lêret selbe wie."\\ 
 & Si sprach: "ist Arnive hie?\\ 
 & swelch gemach mir diu gît,\\ 
 & des wil ich leben dise zît,\\ 
15 & unze daz mîn hêrre hinnen vert.\\ 
 & ist \textbf{ir} \textbf{gevencnisse} erwert,\\ 
 & sô erloubet, daz ich müeze schouwen\\ 
 & si unt andere vrouwen,\\ 
 & den Clinschor teilte sînen vâr\\ 
20 & mit \textbf{gevancnisse} \textbf{nû} manec jâr."\\ 
 & zwêne ritter huoben si ûf \textbf{ir} pfert.\\ 
 & zArniven reit diu maget wert.\\ 
 & Nû was \textbf{ez} ouch zît, daz man dâ geaz.\\ 
 & Parzival bî sînem bruoder saz.\\ 
25 & den bat er gesellecheit.\\ 
 & Feirefiz was im \textbf{al} bereit\\ 
 & gein Munsalvæsche ze rîten\\ 
 & an den selben zîten.\\ 
 & \textbf{si stuonden} ûf über al den rinc.\\ 
30 & Feirefiz warp hôhiu dinc.\\ 
\end{tabular}
\scriptsize
\line(1,0){75} \newline
D \newline
\line(1,0){75} \newline
\textbf{1} \textit{Initiale} D  \textbf{12} \textit{Majuskel} D  \textbf{23} \textit{Majuskel} D  \newline
\line(1,0){75} \newline
\textbf{2} Cundrie la surziere] Cvndrîe lasvrziere D \textbf{5} Parzivale] Parcifale D \textbf{6} Anfortases] Anfortass D \textbf{9} Cundrien] Cvndrîen D \textbf{12} Arnive] Arnîve D \textbf{19} Clinschor] Clinscor D \textbf{24} Parzival] Parcifal D \textbf{27} Munsalvæsche] Mvnsalvæsce D \newline
\end{minipage}
\hspace{0.5cm}
\begin{minipage}[t]{0.5\linewidth}
\small
\begin{center}*m
\end{center}
\begin{tabular}{rl}
 & über al \textbf{den rinc} wart vernomen,\\ 
 & Condrie \textbf{la surzier} \textbf{wære} komen\\ 
 & und waz ir mære meinde.\\ 
 & Urgeluse \textbf{durch} liebe weinde,\\ 
5 & daz diu vrâge von Parcifal\\ 
 & die Anfortases quâl\\ 
 & solte machen wendic.\\ 
 & Artus, der prîses genen\textit{d}ic,\\ 
 & zuo C\textit{o}ndrien mit zühten sprach:\\ 
10 & "vrowe, rîtet an iuwer gemach,\\ 
 & lât iuwer pflegen, lêret selbe wie."\\ 
 & si sprach: "\textit{ist} A\textit{r}n\textit{iv}e hie?\\ 
 & welich gemach mir diu gît,\\ 
 & des wil ich leben dise zît,\\ 
15 & unz daz mîn hêrre hinnen vert.\\ 
 & ist \textbf{ir} \textbf{gevencnüsse} erwert,\\ 
 & sô erlo\textit{u}bet, daz ich müeze schouwen\\ 
 & si und ander vrowen,\\ 
 & den Clinsor teilte \textit{sînen} vâr\\ 
20 & mit \textbf{gevancnüsse} \textbf{nû} manic jâr."\\ 
 & zwên ritter huobens ûf \textbf{ir} pfert.\\ 
 & zuo A\textit{r}n\textit{iv}en reit diu maget wert.\\ 
 & \begin{large}N\end{large}û was ouch zît, daz man d\textit{â} geaz.\\ 
 & Parcifal bî sînem bruoder saz.\\ 
25 & den bat er gesellecheit.\\ 
 & Ferefiz was im \textbf{al}bereit\\ 
 & gegen Muntsalvasche zuo rîten.\\ 
 & an den selben zîten\\ 
 & \textbf{stuondens} ûf über al den rinc.\\ 
30 & Ferefiz warp hôhiu dinc.\\ 
\end{tabular}
\scriptsize
\line(1,0){75} \newline
m n o V V' W Fr6 \newline
\line(1,0){75} \newline
\textbf{1} \textit{Initiale} V Fr6  \textbf{12} \textit{Majuskel} Fr6  \textbf{23} \textit{Initiale} m V V' Fr6   $\cdot$ \textit{Capitulumzeichen} n  \newline
\line(1,0){75} \newline
\textbf{1} al] alle V V' \textbf{2} Condrie] Cundrie o (Fr6) Kvndrie V V' (W)  $\cdot$ la surzier] lasúrzier o Lesurziere V (V') lasursiere W lazvrziere Fr6 \textbf{3} \textit{Die Verse 784.3-7 fehlen} V'  \textbf{4} Urgeluse] Orgeluse V W (Fr6)  $\cdot$ weinde] weine o \textbf{5} Parcifal] parzefale V herr partzifal W parcifale Fr6 \textbf{6} Anfortases] anfortas m n o W Anfortasses V (Fr6) \textbf{8} \textit{Verse 784.8-9 kontrahiert zu:} Artus zv kvndrie do sprach V'   $\cdot$ genendic] gnendeig m \textbf{9} zuo Condrien] Zuͯ cuͯndrien m Zuͯ kundrien n (o) (V) (W) zecvndrien Fr6 \textbf{11} pflegen] pflege V' pflegen vnd W  $\cdot$ selbe] \textit{om.} W \textbf{12} ist] \textit{om.} m  $\cdot$ Arnive] arune m arniwe n arnife V' arnyue W \textbf{13} welich] Swelich V (V') (Fr6) \textbf{14} dise] disen n an dirre V' disiv Fr6 \textbf{15} \textit{Die Verse 784.15-22 fehlen} V'  \textbf{17} erloubet] erlobet m \textbf{18} \textit{Versdoppelung} o  \textbf{19} den] Das n  $\cdot$ Clinsor] clinsors V klinshor W  $\cdot$ sînen] \textit{om.} m den sinen n \textbf{20} nû] \textit{om.} W Fr6 \textbf{21} huobens] hubencz o (V)  $\cdot$ ûf ir] auffs W \textbf{22} Arniven] arunen m arniwe n arnive o Arnifen V arnyue W \textbf{23} ouch] es W  $\cdot$ dâ] do m n o V V' W  $\cdot$ geaz] aß W \textbf{24} Parcifal] Parzefal V Parzifal V' Herr partzifal W  $\cdot$ bî] zu V'  $\cdot$ sînem] sinen n \textbf{25} den bat er] Vnd bat in V' \textbf{26} Ferefiz] Ferefis m Ferrefis n Ferevis V V' Ferafis W  $\cdot$ albereit] albes bereit n \textbf{27} Muntsalvasche] muntsaluasce m n (o) Muntschalfasche V mvntschalfalsche V' montsaluatsch W mvnsalvasche Fr6  $\cdot$ zuo rîten] riten o [*]: zeriten V \textbf{28} \textit{nach 784.28:} Kv́nig artus in grosze froͤude kam / Durch die mere die er do vernam / Vnde alle die geselleschaft daz geschach / Jn froͤuden gros der kv́nig sprach / Er wolte mit parzefale (parzifal V'  ) dar hin keren (dar keren V'  ) / Sine ere helfen meren (Vnd sine freude do meren V'  ) / Den kv́nig vnde die rittere uf der stat / Parzefal (Parzifal V'  ) sv́ des alles bat / Waz er herschefte do vant / Sv́ gelobetenz imme alzehant (Einschub entspr. 'Troisième Continuation', Ep. 29-30, V. 42464-42473) V (V')  \textbf{29} stuondens] Sv́ stuͦndent V (V') (W)  $\cdot$ ûf] \textit{om.} o  $\cdot$ al] alle o \textbf{30} Ferefiz] Ferefis m Ferrefis n o Fereuis V V' Ferafis W \newline
\end{minipage}
\end{table}
\newpage
\begin{table}[ht]
\begin{minipage}[t]{0.5\linewidth}
\small
\begin{center}*G
\end{center}
\begin{tabular}{rl}
 & \begin{large}Ü\end{large}ber al \textbf{daz her} wart vernomen,\\ 
 & Gundrie \textbf{wære} komen\\ 
 & unde waz ir mære meinde.\\ 
 & Orgeluse \textbf{vor} liebe weinde,\\ 
5 & daz diu vrâge von Parzivale\\ 
 & die Anfortases quâle\\ 
 & solde machen wendec.\\ 
 & Artus, der brîses genendec,\\ 
 & ze Gundrien mit zühten sprach:\\ 
10 & "vrouwe, \textbf{nû} rîtet an iwer gemach,\\ 
 & lât iwer pflegen, lêret selbe wie."\\ 
 & si sprach: "ist Arnive hie?\\ 
 & swelch gemach mir diu gît,\\ 
 & des wil ich leben dise zît,\\ 
15 & unze daz mîn hêrre hinnen vert.\\ 
 & ist \textit{\textbf{ir}} \textbf{gevancnüsse} erwert,\\ 
 & sô erloubet, daz ich müeze schouwen\\ 
 & si unde ander vrouwen,\\ 
 & den Clinsor teilte sînen vâr\\ 
20 & mit \textbf{vancnüsse} manic jâr."\\ 
 & zwêne rîter huoben si ûffe\textbf{z} pfert.\\ 
 & zuo Arniven reit diu maget wert.\\ 
 & nû was ouch zît, daz man dâ geaz.\\ 
 & Parzival bî sînem bruoder saz.\\ 
25 & den bat er gesellecheit.\\ 
 & Feirafiz was im bereit\\ 
 & \hspace*{-.7em}\big| an den selben zîten\\ 
 & \hspace*{-.7em}\big| gein Muntsalfatsche rîten.\\ 
 & \textbf{si stuonden} ûf über al den rinc.\\ 
30 & Feirafiz warp hôhiu dinc.\\ 
\end{tabular}
\scriptsize
\line(1,0){75} \newline
G I L M Z \newline
\line(1,0){75} \newline
\textbf{1} \textit{Initiale} G I L Z  \textbf{17} \textit{Initiale} I  \newline
\line(1,0){75} \newline
\textbf{1} daz her] daz mere I (Z) dis L \textbf{2} Gundrie] kvͦndrie G Kvnderie L Das kundrie M (Z)  $\cdot$ komen] da her komen Z \textbf{3} waz] daz I \textbf{4} Orgeluse] Orgillv̂se G Orguluse I Orgelyse L \textbf{5} Parzivale] parcifale G parzifale I L M parcifal Z \textbf{6} Anfortases] Anfortas G (M) (Z) Anfortasses I Amfortasses L \textbf{7} wendec] wemdic M \textbf{8} der] des M Z \textbf{9} Gundrien] kvndrien G L (Z) Gundrie M \textbf{10} rîtet] vart I \textbf{11} pflegen] phlege M  $\cdot$ lêret] spreht Z  $\cdot$ selbe] selben M \textbf{12} Arnive] arniue I \textbf{13} swelch] welch L (M) \textbf{14} ich] \textit{om.} I \textbf{15} daz] \textit{om.} L  $\cdot$ hinnen] von hinnan I  $\cdot$ vert] virt M \textbf{16} ir gevancnüsse] gevanchnvͦsse G ir diu vancnusse I \textbf{18} ander] \textit{om.} I \textbf{19} Clinsor] Clinisor L clinsors M Clingsor Z  $\cdot$ teilte] teil M teilt Z  $\cdot$ sînen] sine I \textbf{20} vancnüsse] gefengnisse M (Z) \textbf{22} Arniven] Arniuen I \textbf{23} dâ geaz] da az I asz M \textbf{24} Parzival] parcifal G (Z) Parzifal I L M \textbf{25} den] dem I \textbf{26} Feirafiz] feirefiz G (Z) Ferefiz L Feirafisz M  $\cdot$ bereit] vil bireit M (Z) \textbf{27} Muntsalfatsche] muntshaluasce I Mvntsalvatsche L Musalvatsche M montsalvatsch Z  $\cdot$ rîten] zuͯ riten L (Z) \textbf{29} al] \textit{om.} Z \textbf{30} Feirafiz] Ferefiz L Feirefisz M Feirefiz Z \newline
\end{minipage}
\hspace{0.5cm}
\begin{minipage}[t]{0.5\linewidth}
\small
\begin{center}*T
\end{center}
\begin{tabular}{rl}
 & über al \textbf{den rinc} wart vernomen,\\ 
 & "Kundrie \textbf{la surziere} \textbf{ist} komen",\\ 
 & und waz ir mære meinte.\\ 
 & Orgeluse \textbf{vor} liebe weinte,\\ 
5 & daz diu vrâge von Parcifale\\ 
 & die Anfortasses quâle\\ 
 & solte machen wende\textit{c}.\\ 
 & Artus, der prîses genende\textit{c},\\ 
 & zuo Kundrien mit zühten sprach:\\ 
10 & "vrouwe, \textbf{nû} rîtet an iuwer gemach,\\ 
 & lât iuwer p\textit{f}legen, lêrt selbe wie."\\ 
 & si sprach: "ist Arnyve hie?\\ 
 & welch gemach mir diu gît,\\ 
 & des wil ich leben dise zît,\\ 
15 & unz daz mîn hêrre hinnen vert.\\ 
 & ist \textbf{gemach uns} erwert,\\ 
 & sô erloubet, daz ich müeze schouwen\\ 
 & si und ander vrouwen,\\ 
 & den Clynsor teilte sîne vâr\\ 
20 & mit \textbf{gevancnisse} manec jâr."\\ 
 & zwêne rîter huoben si ûf \textbf{daz} pfert.\\ 
 & zuo Arnyven reit diu maget wert.\\ 
 & nû was ouch zît, da\textit{z} man dâ geaz.\\ 
 & Parcifal bî sîme bruoder saz.\\ 
25 & den bat er gesellecheit.\\ 
 & Ferefis was im \textbf{vil} bereit\\ 
 & gein Munsalvasche zuo rîten\\ 
 & an den selben zîten.\\ 
 & \textbf{\begin{large}S\end{large}i stuonden} ûf über al den rinc.\\ 
30 & Ferefis war\textit{p} hôhiu dinc.\\ 
\end{tabular}
\scriptsize
\line(1,0){75} \newline
U Q R \newline
\line(1,0){75} \newline
\textbf{1} \textit{Initiale} R  \textbf{29} \textit{Initiale} U  \newline
\line(1,0){75} \newline
\textbf{1} über] Aber Q \textbf{2} la surziere] lazuzire Q \textbf{4} Orgeluse] Orguluse R \textbf{5} Parcifale] Parzifale U partzifale Q parczifale R \textbf{6} Anfortasses] anfortas U anfortes R \textbf{7} wendec] wendet U \textbf{8} prîses] prisen R  $\cdot$ genendec] genendet U \textbf{9} \textit{Die Verse 784.9-789.19 fehlen} Q   $\cdot$ Kundrien] kuͦndrien U kundryen R \textbf{11} pflegen] pelegen U  $\cdot$ selbe] selber R \textbf{12} Arnyve] Arnyue R \textbf{13} welch] Swelich R \textbf{15} unz] Mit U \textbf{16} gemach uns] gevanknússe R \textbf{17} erloubet] erlobent R \textbf{19} Clynsor] clinshor R \textbf{21} huoben si] huͦbencz R \textbf{22} Arnyven] arniuen R \textbf{23} daz] da U \textbf{24} Parcifal] Parzifal U Parczifal R \textbf{25} gesellecheit] [geselleh*]: geselleheit R \textbf{26} Ferefis] Feirefis R \textbf{27} \textit{Versfolge 784.28-27} R   $\cdot$ Munsalvasche] Muntsalvatsche U munsaluashe R  $\cdot$ zuo rîten] ritten R \textbf{29} stuonden] stuͦnd R \textbf{30} Ferefis] Feirefis R  $\cdot$ warp] wart U  $\cdot$ hôhiu] hoche R \newline
\end{minipage}
\end{table}
\end{document}
