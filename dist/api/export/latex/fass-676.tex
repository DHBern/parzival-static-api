\documentclass[8pt,a4paper,notitlepage]{article}
\usepackage{fullpage}
\usepackage{ulem}
\usepackage{xltxtra}
\usepackage{datetime}
\renewcommand{\dateseparator}{.}
\dmyyyydate
\usepackage{fancyhdr}
\usepackage{ifthen}
\pagestyle{fancy}
\fancyhf{}
\renewcommand{\headrulewidth}{0pt}
\fancyfoot[L]{\ifthenelse{\value{page}=1}{\today, \currenttime{} Uhr}{}}
\begin{document}
\begin{table}[ht]
\begin{minipage}[t]{0.5\linewidth}
\small
\begin{center}*D
\end{center}
\begin{tabular}{rl}
\textbf{676} & kein unbilde dran \textbf{geschach},\\ 
 & swâ man in bî sælden sach.\\ 
 & wie der von Norwæge\\ 
 & sînes volkes pflæge,\\ 
5 & der rîter unt der vrouwen?\\ 
 & dâ \textbf{mohten} rîcheit schouwen\\ 
 & Artus unt sîn gesinde\\ 
 & von des werden Lotes kinde.\\ 
 & Si sulen ouch slâfen, dô man geaz;\\ 
10 & ir ruowens hân ich selten haz.\\ 
 & \textbf{s}morgens kom vor tage geriten\\ 
 & volc mit \textbf{werlîchen} siten,\\ 
 & der herzoginne rîter gar.\\ 
 & man nam ir \textbf{zimierde} war\\ 
15 & al bî \textbf{des mânen schîne},\\ 
 & dâ Artus unt die sîne\\ 
 & lâgen; durch die zogten sie\\ 
 & \textbf{unz} anderhalp, dâ Gawan hie\\ 
 & lac mit wîtem ringe.\\ 
20 & swer solhe helfe ertwinge\\ 
 & mit sîner ellenthafter hant,\\ 
 & den mac man hân vür prîs erkant.\\ 
 & Gawan sînen marschalc bat\\ 
 & \textbf{in} zeigen herberge stat.\\ 
25 & als der herzoginne marschalc riet,\\ 
 & von Logroys \textbf{diu werde} diet\\ 
 & manegen rinc wol sunder zierten.\\ 
 & ê si geloschierten,\\ 
 & ez was wol mitter morgen.\\ 
30 & \textbf{hie} næhet ez niwen sorgen.\\ 
\end{tabular}
\scriptsize
\line(1,0){75} \newline
D \newline
\line(1,0){75} \newline
\textbf{9} \textit{Majuskel} D  \newline
\line(1,0){75} \newline
\textbf{8} Lotes] Lots D \newline
\end{minipage}
\hspace{0.5cm}
\begin{minipage}[t]{0.5\linewidth}
\small
\begin{center}*m
\end{center}
\begin{tabular}{rl}
 & kein unbilde dâr an \textbf{beschach},\\ 
 & wâ man in bî sælden sach.\\ 
 & wie der von Norwæge\\ 
 & sînes volkes pflæge,\\ 
5 & der ritter und der vrouwen?\\ 
 & d\textit{â} \textbf{mohte man} rîcheit schouwen,\\ 
 & Artus und sîn gesinde,\\ 
 & von des werden Lotes kinde.\\ 
 & si soln ouch slâfen, dô man geaz;\\ 
10 & ir ruowens hân ich selten haz.\\ 
 & morgens kam vor tage geriten\\ 
 & volc mit \textbf{werlîchem} siten,\\ 
 & der herzogîn ritter gar.\\ 
 & man nam ir \textbf{zimierde} war\\ 
15 & al bî \textbf{des mânen schîn},\\ 
 & d\textit{â} Artus und die sîn\\ 
 & lâgen; durch die zogten \textit{s}ie\\ 
 & \textbf{unz} anderhalp, d\textit{â} Gawan hie\\ 
 & lac mit wîtem ringe.\\ 
20 & wer solhe helf ertwinge\\ 
 & mit sîner ellenthaften hant,\\ 
 & den mac man hân vür prîs erkant.\\ 
 & Gawan sînen marschalc bat\\ 
 & \textbf{im} zeigen herberge stat.\\ 
25 & als der herzogîn marschalc riet,\\ 
 & von Logrois \textbf{die werden} diet\\ 
 & manigen rinc wol sunder zierten.\\ 
 & ê si geloschierten,\\ 
 & ez was wol mittemorgen.\\ 
30 & \textbf{hie} nâhet ez niuwen sorgen.\\ 
\end{tabular}
\scriptsize
\line(1,0){75} \newline
m n o Fr69 \newline
\line(1,0){75} \newline
\newline
\line(1,0){75} \newline
\textbf{1} beschach] geschah Fr69 \textbf{3} Norwæge] norwege m n o \textbf{6} dâ] Do m n o  $\cdot$ mohte] moͯchte n \textbf{7} Artus] Artuͯs o \textbf{8} Lotes] lotz m n o \textbf{15} al] Alle n \textbf{16} dâ] [Do*]: Do m Do n o \textbf{17} sie] hie m \textbf{18} dâ] do m n o \textbf{20} solhe] sol ie o \textbf{24} zeigen] zoigen o \textbf{29} mittemorgen] mitten morgen n \textbf{30} nâhet] nohen o \newline
\end{minipage}
\end{table}
\newpage
\begin{table}[ht]
\begin{minipage}[t]{0.5\linewidth}
\small
\begin{center}*G
\end{center}
\begin{tabular}{rl}
 & \begin{large}N\end{large}ehein unbilde dran \textbf{geschach},\\ 
 & swâ man in bî sælden sach.\\ 
 & wie der von Norwæge\\ 
 & sînes volkes pflæge,\\ 
5 & der rîter unde der vrouwen?\\ 
 & dâ \textbf{mohte} rîcheit schouwen\\ 
 & Artus unde sîn gesinde\\ 
 & von des werden Lotes kinde.\\ 
 & si sulen ouch slâfen, dô man geaz;\\ 
10 & ir ruowens hân ich selten haz.\\ 
 & \textbf{des} morgens kom vor tage geriten\\ 
 & volc mit \textbf{werlîchen} siten,\\ 
 & der herzoginne rîter gar.\\ 
 & man nam ir \textbf{ze unwirde} war\\ 
15 & al bî \textbf{dem mânschîne},\\ 
 & dâ Artus unde die sîne\\ 
 & lâgen; durch die zogeten sie\\ 
 & anderhalb, dâ Gawan hie\\ 
 & lac mit wîtem ringe.\\ 
20 & swer solhe helfe ertwinge\\ 
 & mit sîner ellenthaften hant,\\ 
 & den \textit{mac} man hân vür brîs erkant.\\ 
 & Gawan sînen marschalc bat\\ 
 & \textbf{in} zeigen herberge stat.\\ 
25 & als der herzoginne marschalc riet,\\ 
 & von Logroys \textbf{diu werde} diet\\ 
 & manigen  wol sunder zierten,\\ 
 & ê si geloschierten,\\ 
 & ez was wol mitter morgen.\\ 
30 & \textbf{nû} nâhet ez niwen sorgen.\\ 
\end{tabular}
\scriptsize
\line(1,0){75} \newline
G I L M Z Fr24 Fr61 \newline
\line(1,0){75} \newline
\textbf{1} \textit{Initiale} G L  \textbf{3} \textit{Initiale} Z  \textbf{11} \textit{Initiale} I Fr24 Fr61  \newline
\line(1,0){75} \newline
\textbf{1} Nehein] Eyn M (Z)  $\cdot$ unbilde] vbel I \textbf{2} swâ] Wa L M  $\cdot$ bî] pei den Fr61 \textbf{3} von Norwæge] von norwâge G von Norwege I L Z vorwege M von Norwæg Fr61 \textbf{5} Retter vnf vrowen Fr61 \textbf{6} mohte] moht ir I moht man Z mochten Fr61 \textbf{7} Artus] Artuͯs L Artaus Fr61 \textbf{8} Lotes] lotis M lothes Fr61 \textbf{9} sulen] giengen Fr24  $\cdot$ ouch] \textit{om.} Fr61  $\cdot$ dô] da M Z so Fr61  $\cdot$ geaz] hat gaz Fr61 \textbf{10} ruowens] riwens I ruͤ Fr61  $\cdot$ selten] senten L \textbf{11} des] Der M  $\cdot$ tage] tages I \textbf{12} volc] Folche Fr24 Daz uolch Fr61  $\cdot$ werlîchen] wertlichen M \textbf{13} der] Der er Fr61 \textbf{14} ze unwirde] ze vnwirden I zuͯ wirde L (Fr61) zimierde Z \textbf{15} Bei dem Aschewein Fr61  $\cdot$ al] alle I  $\cdot$ mânschîne] Mas shin I manen schine Z \textbf{16} dâ] \textit{om.} I  $\cdot$ Artus] Artuͯs L Artaus Fr61  $\cdot$ die] \textit{om.} M \textbf{17} zogeten] her zogeten M  $\cdot$ sie] [fre*]: sie G seu wie Fr61 \textbf{20} swer] Wer L M \textbf{21} ellenthaften] ellenthafter I (Fr24) (Fr61) \textbf{22} mac] sol G  $\cdot$ hân vür brîs] zuͤ prise haben I \textbf{23} Gawan] Gawa: Fr24 \textbf{25} herzoginne] kvnigýne L (Fr61) \textbf{26} Logroys] lôgroẏs G Logroýs L L:groys Fr24 Logroẏs Fr61 \textbf{27} manigen wol sunder] mangen rinc si I Mannige wol sunder M manigen rinc sunder Z Den rinch do Fr61 \textbf{28} si] \textit{om.} I  $\cdot$ geloschierten] geleisierten G L geloisierten I Z geloiserten M :geloisîerten Fr24 glotschierten Fr61 \textbf{30} nû] Hie Z  $\cdot$ nâhet] nahent I \newline
\end{minipage}
\hspace{0.5cm}
\begin{minipage}[t]{0.5\linewidth}
\small
\begin{center}*T
\end{center}
\begin{tabular}{rl}
 & kein unbilde dran \textbf{geschach},\\ 
 & wâ man in bî sælden sach.\\ 
 & wie der von Norwæge\\ 
 & sînes volkes pflæge,\\ 
5 & der ritter und der vrouwen?\\ 
 & d\textit{â} \textbf{mohte} rîcheit schouwen\\ 
 & Artus und sîn gesinde\\ 
 & von des \textit{w}e\textit{r}den Lotes kinde.\\ 
 & si sollen ouch slâfen, dô man geaz;\\ 
10 & ir ruowens hân ich selten haz.\\ 
 & \textbf{des} morgens kam vor tac geriten\\ 
 & volc mit \textbf{werlîchen} siten,\\ 
 & der herzogîn ritter gar.\\ 
 & man nam ir \textbf{z\textit{i}mierde} war\\ 
15 & al \textit{bî} \textbf{dem mânschîne},\\ 
 & d\textit{â} Artus und die sîne\\ 
 & lâgen; durch die zogten sie\\ 
 & anderhalp, d\textit{â} Gawan hie\\ 
 & la\textit{c} mit wîtem ringe.\\ 
20 & wer solhe helfe ertwinge\\ 
 & mit sîne\textit{r} e\textit{ll}enthafter hant,\\ 
 & den mac man hân vür prîs erkant.\\ 
 & Gawan sînen marschalc bat\\ 
 & \textbf{in} \textit{z}ei\textit{g}en herberge stat.\\ 
25 & als der herzoginne marschalc riet,\\ 
 & von Logrois \textbf{diu werde} diet\\ 
 & manegen rinc wol sunder zierten.\\ 
 & ê si geloschierten,\\ 
 & ez was wol mitter morgen.\\ 
30 & \textbf{hie} nâhet ez \textit{niuw}en sorgen.\\ 
\end{tabular}
\scriptsize
\line(1,0){75} \newline
Q R W V \newline
\line(1,0){75} \newline
\textbf{23} \textit{Initiale} W  \newline
\line(1,0){75} \newline
\textbf{1} kein] Ein R Fin W [*in]: kein V \textbf{2} wâ] Swa V  $\cdot$ bî] nit by R (W) (V) \textbf{3} Norwæge] norwege Q (R) W V \textbf{6} dâ] Do Q W  $\cdot$ mohte] mocht man R moͤchte W (V) \textbf{7} Artus] Kúnig artus W \textbf{8} werden] frewden Q  $\cdot$ Lotes] Lottes R (W) \textbf{9} sollen] solten W giengen V \textbf{10} hân] hat W \textbf{12} werlîchen] werlichem R W \textbf{13} herzogîn] herczoginnen R \textbf{14} [*r]: Man nam ir zimierde war V  $\cdot$ man] Nam R  $\cdot$ zimierde] zúr mirde Q \textbf{15} al bî dem] Aldem Q  $\cdot$ mânschîne] [*]: manen schine V \textbf{16} \textit{Vers 676.16 ist am Rand nachgetragen} R   $\cdot$ dâ] Do Q W V  $\cdot$ Artus] kúnig artus W \textbf{17} durch] si durch R \textbf{18} dâ] do Q W V  $\cdot$ Gawan] Gawine R \textbf{19} lac] Lat Q \textbf{20} wer] Swer V \textbf{21} sîner] seine: Q  $\cdot$ ellenthafter] erenthaffter Q ellenthafftten R (W) \textbf{22} hân] \textit{om.} R \textbf{23} Gawan] Gawin R [J]: Gawan V \textbf{24} zeigen] zweyen Q \textbf{25} herzoginne] herczoginnen R kúniginne W \textbf{26} Logrois] lichroisz Q Logoris R logroys V \textbf{27} zierten] zierren R \textbf{29} mitter] mitten W \textbf{30} niuwen] meinen Q \newline
\end{minipage}
\end{table}
\end{document}
