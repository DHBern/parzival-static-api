\documentclass[8pt,a4paper,notitlepage]{article}
\usepackage{fullpage}
\usepackage{ulem}
\usepackage{xltxtra}
\usepackage{datetime}
\renewcommand{\dateseparator}{.}
\dmyyyydate
\usepackage{fancyhdr}
\usepackage{ifthen}
\pagestyle{fancy}
\fancyhf{}
\renewcommand{\headrulewidth}{0pt}
\fancyfoot[L]{\ifthenelse{\value{page}=1}{\today, \currenttime{} Uhr}{}}
\begin{document}
\begin{table}[ht]
\begin{minipage}[t]{0.5\linewidth}
\small
\begin{center}*D
\end{center}
\begin{tabular}{rl}
\textbf{69} & dâ vriesch der künec von Zazamanc,\\ 
 & daz die poynder wît unt lanc\\ 
 & \textbf{wâren ze velde} worden\\ 
 & al nâch rîters orden.\\ 
5 & \textit{\begin{large}E\end{large}}r huob \textbf{ouch sich} des endes \textbf{dar}\\ 
 & mit maneger baniere lieht gevar.\\ 
 & er\textbf{n} kêrte sich niht an gâhez \textbf{schehen}.\\ 
 & \textbf{müezeclîche er wolde} \textbf{ersehen},\\ 
 & wie ez ze \textbf{bêder sît} \textbf{dâ} wære getân.\\ 
10 & Sînen tepich \textbf{leit man} ûf \textbf{die} plân,\\ 
 & Dâ sich die ponder wurren\\ 
 & \textbf{unt} diu ors von stichen kurren.\\ 
 & von \textbf{knappen} \textbf{was} umbe in ein rinc,\\ 
 & dâ bî von swerten \textbf{klingâ} klinc.\\ 
15 & wie si nâch \textbf{prîse} rungen,\\ 
 & \textbf{der} klingen alsus klungen.\\ 
 & von spern was grôz krachen dâ.\\ 
 & \textbf{er}\textbf{n} dorfte \textbf{niemen} vrâgen wâ.\\ 
 & poynder \textbf{wâren} sîne wende,\\ 
20 & die worhten rîters hende.\\ 
 & diu rîterschaft sô nâhe was,\\ 
 & daz die vrouwen \textbf{ab} dem palas\\ 
 & \textbf{wol} sâhen der \textbf{helde} arbeit.\\ 
 & \textbf{doch} was der küneginne leit,\\ 
25 & daz sich der künec von Zazamanc\\ 
 & \textbf{dâ mit den andern} \textbf{niht en}dranc.\\ 
 & si sprach: "\textbf{wê}, war ist \textbf{er} komen,\\ 
69.28 & von dem ich wunder hân vernomen?"\\ 
70.7 & Ez wart dâ harte \textbf{guot} getân\\ 
 & von manegem \textbf{küenem armman},\\ 
\end{tabular}
\scriptsize
\line(1,0){75} \newline
D Fr33 \newline
\line(1,0){75} \newline
\textbf{5} \textit{Initiale} D Fr33  \textbf{10} \textit{Majuskel} D  \textbf{11} \textit{Majuskel} D  \textbf{21} \textit{Initiale} Fr33  \textbf{70.7} \textit{Majuskel} D  \newline
\line(1,0){75} \newline
\textbf{1} Zazamanc] Zazamanch D zazaman: Fr33 \textbf{5} Er] ÷r D \textbf{25} Zazamanc] Zazamanch D \textbf{70.8} küenem armman] kuenen armen man Fr33 \newline
\end{minipage}
\hspace{0.5cm}
\begin{minipage}[t]{0.5\linewidth}
\small
\begin{center}*m
\end{center}
\begin{tabular}{rl}
 & \begin{large}D\end{large}ô \dag verhies\dag  der künic von Zazamanc,\\ 
 & daz die p\textit{o}inder wît und lanc\\ 
 & \textbf{wâren ze velde} \textit{word}en\\ 
 & al nâch ritters \textit{ord}en.\\ 
5 & er huop \textbf{sich ouch} des endes \textbf{gar}\\ 
 & mit maniger banier lieht gevar.\\ 
 & er kêrte sich niht an g\textit{â}hez \textbf{schehen}.\\ 
 & \textbf{innerclîch er wolde} \textbf{ersehen},\\ 
 & wie ez ze \textbf{beider sîte} \textbf{dâ} wære getân.\\ 
10 & sînen teppich \textbf{leite man} ûf \textbf{den} plân,\\ 
 & d\textit{â} sich die poinder wurren\\ 
 & \textbf{und} diu ros von stichen kurren.\\ 
 & von \textbf{knussen} \textbf{was} umb in ein rinc,\\ 
 & dâ bî vo\textit{n} swerten \textbf{klingen} klinc.\\ 
15 & wie s\textit{i} nâch \textbf{prîsen} rungen,\\ 
 & \textbf{der} klingen alsus klungen.\\ 
 & von speren was grôz krachen d\textit{â}.\\ 
 & \textbf{er} \textbf{en}dorfte \textbf{niemer} vrâgen wâ.\\ 
 & poinder \textbf{wâren} sîne wende,\\ 
20 & die \dag vorhten\dag  ritters hende.\\ 
 & diu ritterschaft sô nâhe was,\\ 
 & daz die vrowen \textbf{ab} dem palas\\ 
 & \textbf{wol} sâhen der \textbf{helde} arbeit.\\ 
 & \textbf{doch} was der küniginn\textit{e} leit,\\ 
25 & daz sich der künic von Zazamanc\\ 
 & \textbf{d\textit{â} mit den andern} \textbf{niht en}dranc.\\ 
 & si sprach: "\textbf{wê}, war ist \textbf{er} komen,\\ 
69.28 & von dem ich \textit{wunder} hân vernomen?"\\ 
70.7 & ez wart d\textit{â} harte \textbf{guot} getân\\ 
 & von manigem \textbf{küenen armen man},\\ 
\end{tabular}
\scriptsize
\line(1,0){75} \newline
m n o \newline
\line(1,0){75} \newline
\textbf{1} \textit{Initiale} m   $\cdot$ \textit{Capitulumzeichen} n  \newline
\line(1,0){75} \newline
\textbf{1} Zazamanc] zazamanck m zazamang n o \textbf{2} poinder] painder m \textbf{3} velde] solde n  $\cdot$ worden] gelegen m \textbf{4} orden] gelegen m \textbf{5} gar] dar n o \textbf{7} gâhez] gehes m  $\cdot$ schehen] sehen n o \textbf{8} innerclîch] Jnneclich n o \textbf{9} dâ] do n \textit{om.} o \textbf{10} leite] leit n o \textbf{11} dâ] Do m n o \textbf{12} von] zuͦ n o  $\cdot$ stichen] strichen o \textbf{14} von] vom m mit o  $\cdot$ klingen] clinga n o \textbf{15} si] sich m  $\cdot$ prîsen] prise n o \textbf{16} klungen] klingen o \textbf{17} dâ] do m n \textbf{18} endorfte] dorfft n (o) \textbf{23} sâhen] sehent o \textbf{24} küniginne] kuͯnginnen m \textbf{25} Zazamanc] zazamang m n o \textbf{26} dâ] Do m \textit{om.} n o \textbf{27} wê] owe o \textbf{28} wunder] \textit{om.} m \textbf{70.7} dâ] do m n  $\cdot$ harte] hatt n \textbf{70.8} küenen armen] keinen n kuͯnen o \newline
\end{minipage}
\end{table}
\newpage
\begin{table}[ht]
\begin{minipage}[t]{0.5\linewidth}
\small
\begin{center}*G
\end{center}
\begin{tabular}{rl}
 & dô vriesch der künic von Zazamanc,\\ 
 & daz die ponder wît und lanc\\ 
 & \textbf{ze velde wâren} worden\\ 
 & al nâch rîters orden.\\ 
5 & er huop \textbf{ouch sich} des endes \textbf{dar}\\ 
 & mit maniger banier lieht gevar.\\ 
 & er kêrte sich niht an gâhez \textbf{schehen}.\\ 
 & \textbf{er wolte müeziclîche} \textbf{ersehen},\\ 
 & wie ez ze \textbf{beider sîte} \textbf{dâ} wær getân.\\ 
10 & sînen tepech \textbf{leit man} ûf \textbf{den} plân,\\ 
 & \textit{dâ} sich die ponder wurren,\\ 
 & diu ors von stichen kurren.\\ 
 & von \textbf{knappen} \textbf{wart} umbe in ein rinc,\\ 
 & dâ bî von swerten \textbf{klingâ} klinc.\\ 
15 & wie si nâch \textbf{prîse} rungen,\\ 
 & \textbf{der} klingen alsus klungen.\\ 
 & von spern was grôz krachen dâ.\\ 
 & \textbf{dô}\textbf{ne} dorfte \textbf{niemen} \textit{v}r\textit{âg}en wâ.\\ 
 & ponder \textbf{wâren} sîne wende,\\ 
20 & die worhten rîters hende.\\ 
 & diu rîterschaft sô nâhen was,\\ 
 & daz die vrouwen \textbf{ûf} dem palas\\ 
 & sâhen der \textbf{rîter} arbeit.\\ 
 & \textbf{dô} was der küniginne leit,\\ 
25 & daz sich der künic von Zazamanc\\ 
 & \textbf{dâ \textit{bî} den anderen} \textbf{niene} dranc.\\ 
 & si sprach: "\textbf{owê}, war ist \textbf{der} komen,\\ 
69.28 & von dem ich wunder hân vernomen?"\\ 
70.7 & \begin{large}E\end{large}z wart dâ harte \textbf{guot} getân\\ 
 & von manigem \textbf{küenen armen man},\\ 
\end{tabular}
\scriptsize
\line(1,0){75} \newline
G I O L M Q R Z Fr21 \newline
\line(1,0){75} \newline
\textbf{1} \textit{Initiale} I O  \textbf{17} \textit{Initiale} I  \textbf{70.7} \textit{Initiale} G L Q R Z Fr21  \newline
\line(1,0){75} \newline
\textbf{1} dô] Da M Z  $\cdot$ vriesch] irfrisch M  $\cdot$ von] >von< M \textit{om.} Q  $\cdot$ Zazamanc] zazamanch G O L zazamat Q zasmanc R \textbf{2} daz] Der Q  $\cdot$ die] \textit{om.} R \textbf{4} al] Als M R \textbf{5} ouch sich] sich avch O (L) sich ::: Fr21 \textbf{6} lieht] lieh I liht O (M) (Q) liechtte R  $\cdot$ gevar] var R \textbf{7} er] ern I (M) (Fr21)  $\cdot$ kêrte] kert I (O) R Z  $\cdot$ gâhez schehen] gæhes spehen I (L) gehez sehen O gahezsin M geche geschechen R (Z) \textbf{8} müeziclîche] mvͦzlichen O  $\cdot$ ersehen] besehen I er spen M sechen R \textbf{9} wie ez] Wierz Fr21  $\cdot$ beider] beiden L (Fr21)  $\cdot$ sîte] siten L M (Q) Fr21  $\cdot$ dâ wær] wer I (L) wasz Q \textbf{10} sînen] sin I (M) (Q) (R) Sine L  $\cdot$ leit] leite M (R)  $\cdot$ man] er I man im O Fr21  $\cdot$ den] di Fr21 \textbf{11} dâ] wie G Do O Q  $\cdot$ die] div O \textbf{12} diu] Vnd div O (L) (M) (Q) (R) (Z) (Fr21)  $\cdot$ stichen] streyte Q \textbf{13} in ein] eyn M (Q) in Z \textbf{14} dâ bî] \textit{om.} I  $\cdot$ klingâ] grozzer chlinga I klange M \textbf{15} si] die Q \textbf{16} klingen] swert L clangin M  $\cdot$ klungen] ercluͯngen L (Z) \textbf{17} spern] grossen spern Q  $\cdot$ was] wart Fr21  $\cdot$ grôz] grozzen Z  $\cdot$ krachen dâ] krach alda Q \textbf{18} dône] ezn I (L) Ja en O (M) (Q) (Fr21) Man R Ern Z  $\cdot$ dorfte] dorfftin M  $\cdot$ niemen] meinen R  $\cdot$ vrâgen] sprechen G  $\cdot$ wâ] [da]: wa I \textbf{19} ponder] Pondiren Q Poyndie R  $\cdot$ wâren] wern Q \textbf{20} worhten] [wochten]: worchten L wortin M forchten Q  $\cdot$ rîters] sine R  $\cdot$ hende] [ende]: hende Fr21 \textbf{22} daz die] Da dú R  $\cdot$ dem] den O \textbf{23} sâhen] Gahen L  $\cdot$ der rîter] all ir I der helde O L M Z (Fr21) die helde Q der helden R \textbf{24} dô] Da M R Z  $\cdot$ küniginne] kúnginen R \textbf{25} sich] \textit{om.} Q  $\cdot$ Zazamanc] zazamanch O L zazamant Q zasamanc R \textbf{26} dâ bî] da mit G Bi O (L) (M) (Q) (R) Z Fr21  $\cdot$ niene] nih I da niht O L (M) (R) Z Fr21 do nicht Q  $\cdot$ dranc] endranc I (O) \textbf{27} owê] awe I owi R \textit{om.} Z  $\cdot$ der] er I O Z Fr21 \textbf{70.7} dâ] \textit{om.} O do Q  $\cdot$ guot] wol Q \textbf{70.8} manigem] machein Q  $\cdot$ küenen] chvͦnem O (Fr21) [kvnem]: kvnen  Z  $\cdot$ armen] arm I Z \textit{om.} Q \newline
\end{minipage}
\hspace{0.5cm}
\begin{minipage}[t]{0.5\linewidth}
\small
\begin{center}*T (U)
\end{center}
\begin{tabular}{rl}
 & dô vriesch der künec von Zazamanc,\\ 
 & daz die poynder wît und lanc\\ 
 & \textbf{zuo velde wâren} worden\\ 
 & al \textbf{dâ} nâch ritters orden.\\ 
5 & er huop \textbf{sich} des endes \textbf{dar}\\ 
 & mit maniger baniere lieht gevar.\\ 
 & er kêrte sich niht an gâhez \textbf{sehen}.\\ 
 & \textbf{er wolte müezeclîche} \textbf{erspehen},\\ 
 & wie ez zuo \textbf{beiden sîten} wære getân.\\ 
10 & Sînen teppich \textbf{man leite} ûf \textbf{den} plân,\\ 
 & dâ sich die poynder wurren\\ 
 & \textbf{und} diu ors von stichen kur\textit{r}en.\\ 
 & von \textbf{knappen} \textbf{wart} umb in ein rinc,\\ 
 & dâ bî von swerten \textbf{klingâ} klinc.\\ 
15 & wie si nâch \textbf{prîse} rungen,\\ 
 & \textbf{die} klingen alsus klungen.\\ 
 & von spern was grôz krachen dâ.\\ 
 & \textbf{jâ dô} dorfte \textbf{nieman} vrâgen wâ.\\ 
 & poynder \textbf{wæren} sîne wende,\\ 
20 & d\textit{ie} wor\textit{ht}en ritters hende.\\ 
 & \begin{large}D\end{large}iu ritterschaft sô nâhe was,\\ 
 & daz die vrouwen \textbf{von} dem palas\\ 
 & sâhen der \textbf{helde} arbeit.\\ 
 & \textbf{dô} was der küneginne leit,\\ 
25 & daz sich der künec von Zazamanc\\ 
 & \textbf{mit den andern dâ} \textbf{niht} dranc.\\ 
 & si sprach: "\textbf{owê}, war ist \textbf{er} komen,\\ 
69.28 & von dem ich wunder hân vernomen?"\\ 
70.7 & ez wart dâ harte \textbf{wol} getân\\ 
 & von manigem \textbf{armen küenen man},\\ 
\end{tabular}
\scriptsize
\line(1,0){75} \newline
U V W T \newline
\line(1,0){75} \newline
\textbf{10} \textit{Majuskel} T  \textbf{21} \textit{Initiale} U T  \textbf{70.7} \textit{Initiale} W  \newline
\line(1,0){75} \newline
\textbf{1} vriesch] erfreisch V erfuͦr W  $\cdot$ Zazamanc] zazamang V W \textbf{4} al dâ] Al V (W) (T) \textbf{5} sich] auch sich W (T) \textbf{6} maniger] mangem W  $\cdot$ lieht] lith U \textbf{7} er] Ern V (T)  $\cdot$ gâhez] gahes T \textbf{8} müezeclîche] mvezliche T  $\cdot$ erspehen] spehen V ersehen T \textbf{10} sînen] sine V Sein W (T)  $\cdot$ man leite] leite men im V laiten sy W leite man T \textbf{11} dâ] Do V W \textbf{12} kurren] kuͦrten \textit{nachträglich korrigiert zu:} kurren U \textbf{14} klingâ klinc] kluͦge ding W \textbf{16} die] der T \textbf{17} dâ] do V W \textbf{18} jâ dô dorfte] ioch dorfte man V Ienen dorffte W ezen dorften T \textbf{19} poynder] \textit{om.} W dem poynder T \textbf{20} die worhten] Do worden U \textbf{22} von] vs V ab W vf T \textbf{25} Zazamanc] zazamang V W \textbf{26} mit] Do mit W bi T  $\cdot$ dâ] do V \textit{om.} W  $\cdot$ dranc] erdrang W \textbf{27} owê] ôuwi T  $\cdot$ er] \textit{om.} T \textbf{28} ich] ich hie W \textbf{70.7} wart] was W  $\cdot$ dâ] [*]: do V do W  $\cdot$ wol] gvͦte T \newline
\end{minipage}
\end{table}
\end{document}
