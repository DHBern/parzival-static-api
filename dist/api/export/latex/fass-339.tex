\documentclass[8pt,a4paper,notitlepage]{article}
\usepackage{fullpage}
\usepackage{ulem}
\usepackage{xltxtra}
\usepackage{datetime}
\renewcommand{\dateseparator}{.}
\dmyyyydate
\usepackage{fancyhdr}
\usepackage{ifthen}
\pagestyle{fancy}
\fancyhf{}
\renewcommand{\headrulewidth}{0pt}
\fancyfoot[L]{\ifthenelse{\value{page}=1}{\today, \currenttime{} Uhr}{}}
\begin{document}
\begin{table}[ht]
\begin{minipage}[t]{0.5\linewidth}
\small
\begin{center}*D
\end{center}
\begin{tabular}{rl}
\textbf{339} & \begin{large}G\end{large}awan, der reht gemuote,\\ 
 & sîn ellen pflac der huote,\\ 
 & sô daz diu wâre zageheit\\ 
 & an prîse \textbf{im} nie gevrumte leit.\\ 
5 & \textbf{sîn herze was ze velde} ein burc,\\ 
 & gein scharpfen \textbf{strîten} wol sô kurc\\ 
 & in strîtes gedrenge man in sach.\\ 
 & vriunt unt vîent im des jach,\\ 
 & sîn krîe wære gein prîse \textbf{hel},\\ 
10 & swie gerne in Kingrimursel\\ 
 & mit kampfe hete dâ von genomen.\\ 
 & nû was \textit{v}on Artuse komen\\ 
 & - des enweiz ich niht wie manegen tac -\\ 
 & Gawan, der manheite pflac.\\ 
15 & \textbf{sus} reit der \textbf{werde} degen balt\\ 
 & sîne rehte strâzen \textbf{ûz} \textbf{einem} walt\\ 
 & mit sîme gezog \textbf{durch} einen grunt.\\ 
 & dâ wart im ûf \textbf{dem} bühel kunt\\ 
 & ein dinc, daz angest lêrte\\ 
20 & unt sîne manheit mêrte:\\ 
 & dâ \textbf{ersach} der helt vür unbetrogen\\ 
 & nâch maneger baniere zogen\\ 
 & \textbf{mit} \textbf{grôzer} vuore, niht ze kranc.\\ 
 & \textbf{er dâhte}: "mir ist der wec ze lanc,\\ 
25 & vlühtic wider \textbf{gein} dem walde."\\ 
 & dô hiez er gürten balde\\ 
 & \textbf{einem} orse, daz im Orilus\\ 
 & gap. daz was genennet sus:\\ 
 & mit den \textbf{rôten} ôren Gringuljet.\\ 
30 & er enpfieng ez ân \textbf{aller slahte} bet.\\ 
\end{tabular}
\scriptsize
\line(1,0){75} \newline
D \newline
\line(1,0){75} \newline
\textbf{1} \textit{Initiale} D  \newline
\line(1,0){75} \newline
\textbf{10} Kingrimursel] kingrimvrzel D \textbf{12} von] won D \newline
\end{minipage}
\hspace{0.5cm}
\begin{minipage}[t]{0.5\linewidth}
\small
\begin{center}*m
\end{center}
\begin{tabular}{rl}
 & \begin{large}G\end{large}awan, der rehte gemuote,\\ 
 & sîn ellen pflac der huote,\\ 
 & sô daz diu wâre zagheit\\ 
 & an prîse \textbf{ime} nie gevrumete leit.\\ 
5 & \textbf{sîn herz was ze velde} ei\textit{n b}urc,\\ 
 & gegen scharfen \textbf{prîsen} wol sô kurc\\ 
 & in strîtes gedrenge man in sach.\\ 
 & vriunt und vîent ime des jach,\\ 
 & sîn krîe wære gegen prîse \textbf{snel},\\ 
10 & wie gerne in Kingri\textit{m}u\textit{r}sel\\ 
 & mit kampfe het dâ von genomen.\\ 
 & nû was von Artuse komen\\ 
 & - des enweiz ich niht wie manigen tac -\\ 
 & Gawan, der manheite pflac.\\ 
15 & \textbf{sus} reit der \textbf{werde} degen balt\\ 
 & sîne rehte strâze \textbf{ûz} \textbf{einem} walt\\ 
 & mit sînem gezoge \textbf{durch} einen grunt.\\ 
 & dô wart im ûf \textbf{dem} bühele kunt\\ 
 & ein dinc, daz angest lêrte\\ 
20 & und sîne manheit mêrte:\\ 
 & dô \textbf{ersach} der helt vür unbetrogen\\ 
 & nâch maniger banier zogen\\ 
 & \textbf{vil} \textbf{grôzer} vuore, niht ze kranc.\\ 
 & \textbf{er dâhte}: "mir ist der wec ze lanc,\\ 
25 & v\textit{l}ühtic wider \textbf{gegen} dem walde."\\ 
 & dô hiez \textit{er} gürten balde\\ 
 & \textbf{einem} ros, daz ime Orilus\\ 
 & gap. daz was genennet sus:\\ 
 & mit den \textbf{rôten} ôren Gringulete.\\ 
30 & e\textit{r} enpfienc ez âne \textbf{aller slahte} bete.\\ 
\end{tabular}
\scriptsize
\line(1,0){75} \newline
m n o \newline
\line(1,0){75} \newline
\textbf{1} \textit{Initiale} m  \newline
\line(1,0){75} \newline
\textbf{5} ein burc] ein hercz vnd burg m \textbf{6} sô] [eyn]: so o \textbf{8} des] dasz o \textbf{10} Kingrimursel] kingrinvsel m kingrumorsel n komgruͯmorsel o \textbf{16} einem] einen n \textbf{17} gezoge] gezage m gezuge n o \textbf{18} bühele] behel o \textbf{21} ersach] sach n o  $\cdot$ vür] vil n o \textbf{22} nâch maniger] Vil manig n o  $\cdot$ zogen] vmb zogen n vmmb zoigen o \textbf{23} vil grôzer] Vnd grosse n o \textbf{24} dâhte] gedochte n (o) \textbf{25} vlühtic] Fuhttig m \textbf{26} er] \textit{om.} m \textbf{27} einem] Sinem n o  $\cdot$ ime] sich o \textbf{29} ôren] [orten]: oren o \textbf{30} er] Es m  $\cdot$ aller] alle o \newline
\end{minipage}
\end{table}
\newpage
\begin{table}[ht]
\begin{minipage}[t]{0.5\linewidth}
\small
\begin{center}*G
\end{center}
\begin{tabular}{rl}
 & Gawan, der reht gemuote,\\ 
 & sîn ellen pflac der huote,\\ 
 & \textit{sô} daz diu wâre zageheit\\ 
 & an prîse \textbf{im} nie gevrumte leit.\\ 
5 & \textbf{sîn herze was ze velde} ein burc,\\ 
 & gein scharfen \textbf{strîten} wol sô kurc\\ 
 & in strîtes gedrenge man in sach.\\ 
 & vriunt unde vîent im des jach,\\ 
 & sîn krîe wære gein prîse \textbf{snel},\\ 
10 & swie gerne in Kingrimursel\\ 
 & mit kampfe het dâ von genomen.\\ 
 & nû was \textbf{ouch} von Artuse komen\\ 
 & - desne weiz ich niht wie manigen tac -\\ 
 & Gawan, der manheite pflac.\\ 
15 & \textbf{sus} reit der \textbf{mære} degen balt\\ 
 & sîne reht strâze \textbf{vür} \textbf{einen} walt\\ 
 & mit sînem gezoge \textbf{\textit{d}ur\textit{ch}} einen grunt.\\ 
 & dô wart im ûf \textbf{dem} bühele kunt\\ 
 & ein dinc, daz angest lêrte\\ 
20 & \begin{large}U\end{large}nde sîne manheit mêrte:\\ 
 & dâ \textbf{sach} der helt vür unbetrogen\\ 
 & nâch maniger baniere zogen\\ 
 & \textbf{vil} \textbf{grôzer} vuore, niht ze kranc.\\ 
 & \textbf{dô dâhter}: "mir \textit{ist} der wec ze lanc,\\ 
25 & vlühtic \textit{wider} \textbf{gein} dem walde."\\ 
 & dô hiez er gürten balde\\ 
 & \textbf{sînem} orse, daz im Orillus\\ 
 & gap. daz was genennet sus:\\ 
 & mit den \textbf{rôten} ôren Gringuliet.\\ 
30 & er enpfieng ez âne \textbf{alle} bet.\\ 
\end{tabular}
\scriptsize
\line(1,0){75} \newline
G I O L M Q R Z Fr39 Fr40 \newline
\line(1,0){75} \newline
\textbf{1} \textit{Überschrift:} Hie lis hern gawans auentevr Z   $\cdot$ \textit{Initiale} I O L Q Z Fr39   $\cdot$ \textit{Capitulumzeichen} R  \textbf{20} \textit{Initiale} G  \textbf{27} \textit{Initiale} I  \newline
\line(1,0){75} \newline
\textbf{1} Gawan] ÷Awan O \textbf{2} ellen] eren Q  $\cdot$ der] vil O \textbf{3} sô] \textit{om.} G  $\cdot$ wâre] warú R \textbf{4} An sinem libe nie frumte leit R  $\cdot$ gevrumte] gefvͦgte O \textbf{5} burc] bruck Q (R) \textbf{6} scharfen] sharphem I starchem L [st*]: starrcken Q starcken Fr39  $\cdot$ strîten] strite L strites Fr39  $\cdot$ kurc] chlvͦch O kruͯg R \textbf{7} gedrenge] gedinge I gedrange L Fr39 \textbf{8} des] daz R  $\cdot$ jach] sprach M \textbf{9} prîse] vigenden R  $\cdot$ snel] hel O L M Q R Z Fr39 \textbf{10} swie] Wie L (M) (Q) R  $\cdot$ in] yme M  $\cdot$ Kingrimursel] kingrimurshel I kyngrimvrsel O kingrýmvrsel L kingrymursel M kingrinműrsel Q kúngrimursel R \textbf{11} kampfe] chraft O  $\cdot$ het dâ von] da von het O (L) (Fr39) do het fúr R \textbf{12} ouch] auch nun Q \textit{om.} R  $\cdot$ Artuse] Artuͯse L Artus R (Z) \textbf{13} desne] des I (M) (R)  $\cdot$ manigen] manich O (Q) mange Fr40 \textbf{15} mære] ware I O M Z Fr40 werde L R Fr39 \textbf{16} sîne] seinen Fr40  $\cdot$ reht strâze] strase recht R  $\cdot$ vür] vz O (L) (M) (Q) (R) Z (Fr39) Fr40  $\cdot$ einen] einem O R Fr39 einē L (M) Q Fr40 \textbf{17} \textit{Versfolge 339.18-17} L Fr39   $\cdot$ durch] vur G in I \textbf{18} dô] Dv L Da M  $\cdot$ ûf] vsz L (Fr39)  $\cdot$ dem] einen R \textbf{19} lêrte] merte R \textbf{20} mêrte] [lerte]: merte Q lerte R \textbf{21} dâ] Do O Q R Fr39 (Fr40)  $\cdot$ vür] \textit{om.} R \textbf{23} grôzer] grosze L (Fr39)  $\cdot$ ze kranc] erkant Q \textbf{24} dâhter] gedacht er R  $\cdot$ ist] \textit{om.} G \textbf{25} vlühtic] Fliehens O  $\cdot$ wider] \textit{om.} G  $\cdot$ dem] \textit{om.} L Q R Fr40 \textbf{26} dô] Da Z \textbf{27} sînem] Einem O (L) (M) (Q) Z Fr39 (Fr40) Ein R  $\cdot$ im] im gab R  $\cdot$ Orillus] [orlus]: orilus G Orilus I (O) (M) (Q) (R) (Z) Fr39 (Fr40) Orilluͯs L \textbf{28} gap] \textit{om.} R  $\cdot$ genennet] genuͯwet L \textbf{29} den rôten ôren] den roten ougen L (Fr39) deme rotin Munde M den kurtzen oren Q (R) (Fr40) den kurtzen roten oren Z  $\cdot$ Gringuliet] kringuliet G I (O) (Fr40) gringuͯlget L kryngulet M kringuleit Q kringulet R Z gringulget Fr39 \textbf{30} er enpfieng] Er pfiench O ern pfiez Fr39  $\cdot$ alle] sine I aller slahte O L (M) (Q) Z Fr39 Fr40 \newline
\end{minipage}
\hspace{0.5cm}
\begin{minipage}[t]{0.5\linewidth}
\small
\begin{center}*T
\end{center}
\begin{tabular}{rl}
 & \begin{large}G\end{large}awan, der rehte gemuot,\\ 
 & sîn ellen pflac der \textbf{êren} huot,\\ 
 & sô daz diu wâre zageheit\\ 
 & an prîse nie gevrumte leit.\\ 
5 & \textbf{Ze velde was sîn herze} ein burc,\\ 
 & gegen scharpfen \textbf{strîten} wol sô kurc\\ 
 & in strîtes gedrenge man in sach.\\ 
 & vriunt unde vîent im des jach,\\ 
 & sîn krîe wære gegen prîse \textbf{hel},\\ 
10 & swie gerne in Kyngrimursel\\ 
 & mit kampfe hete dâ von genomen.\\ 
 & nû was \textbf{er} \textbf{ouch} von Artuse komen\\ 
 & - des enweiz ich niht wie manegen tac -,\\ 
 & Gawan, der manheite pflac.\\ 
15 & \textbf{Dô} reit der \textbf{wâre} degen balt\\ 
 & sîne rehte strâze \textbf{ûf} \textbf{einen} walt\\ 
 & mit sînem gezoge \textbf{in} einen grunt.\\ 
 & dâ wart im ûf \textbf{einem} bühele kunt\\ 
 & ein dinc, daz \textbf{in} angest lêrte\\ 
20 & unde sîne manheit mêrte:\\ 
 & dô \textbf{sach} der helt vür unbetrogen\\ 
 & \textbf{ie} nâch maneger banier zogen\\ 
 & \textbf{vil} \textbf{grôze} vuore, niht ze kranc.\\ 
 & \textbf{er dâhte}: "mir ist der wec ze lanc,\\ 
25 & vlühtic wider \textbf{z}em walde."\\ 
 & dô hiez er gürten balde\\ 
 & \textbf{einem} orse, daz im Orilus\\ 
 & gap. daz was genennet sus:\\ 
 & mit den \textbf{kurzen} ôren Krynguliet.\\ 
30 & er enpfieng ez âne \textbf{aller slahte} bet.\\ 
\end{tabular}
\scriptsize
\line(1,0){75} \newline
T V W \newline
\line(1,0){75} \newline
\textbf{1} \textit{Initiale} T W  \textbf{5} \textit{Majuskel} T  \textbf{15} \textit{Majuskel} T  \newline
\line(1,0){75} \newline
\textbf{1} rehte gemuot] wolgemuͦt W \textbf{4} [*vmete]: An prise im nie gefrvmete leit V \textbf{5} [*]: Sin herze waz ze uelde ein burg V  $\cdot$ Sein hertz was zuͦ velde ein burg W \textbf{6} Gegen scharpfem streit ein kurg W  $\cdot$ strîten] [*]: striten V \textbf{9} krîe] preiß W  $\cdot$ prîse hel] [pris*]: prise snel V kriege schnel W \textbf{10} swie gerne in] Wie vil im W  $\cdot$ Kyngrimursel] kẏngrimursel V kingrunursel W \textbf{11} mit kampfe hete] Kampffes hat W \textbf{12} er ouch] [*h]: er oͮch V auch W \textbf{13} des] Das W \textbf{15} Dô] [D*]: Suz V  $\cdot$ wâre] [w*]: werde V mere W \textbf{16} ûf einen] [*]: vz einen V \textbf{17} in] [*]: durch V durch W \textbf{18} dâ] Do V W  $\cdot$ einem bühele] dem bv́hel V der buckel W \textbf{19} daz in] daz >in< T daz V (W) \textbf{20} mêrte] herte W \textbf{22} ie nâch maneger] Hin nach im vil mange W \textbf{23} vil grôze] Mit grosser W \textbf{25} zem] ze V \textbf{29} den] \textit{om.} W  $\cdot$ Krynguliet] kringvliet T [kringul*]: kringulette V kringulet W \textbf{30} aller slahte] \textit{om.} V \newline
\end{minipage}
\end{table}
\end{document}
