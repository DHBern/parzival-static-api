\documentclass[8pt,a4paper,notitlepage]{article}
\usepackage{fullpage}
\usepackage{ulem}
\usepackage{xltxtra}
\usepackage{datetime}
\renewcommand{\dateseparator}{.}
\dmyyyydate
\usepackage{fancyhdr}
\usepackage{ifthen}
\pagestyle{fancy}
\fancyhf{}
\renewcommand{\headrulewidth}{0pt}
\fancyfoot[L]{\ifthenelse{\value{page}=1}{\today, \currenttime{} Uhr}{}}
\begin{document}
\begin{table}[ht]
\begin{minipage}[t]{0.5\linewidth}
\small
\begin{center}*D
\end{center}
\begin{tabular}{rl}
\textbf{59} & der was \textbf{nâch} rîterschefte gevarn,\\ 
 & dâ man niht schilde dorfte sparn.\\ 
 & Dô hiez \textbf{ouch \textit{er}} bereiten sich\\ 
 & - sus \textbf{wert diu âventiure} mich -\\ 
5 & mit spern wol gemâlen,\\ 
 & \textbf{mit} grüenen zindâlen.\\ 
 & \textbf{ieslîchez} \textbf{hete} ein banier.\\ 
 & drî hermîn enker dran sô fier,\\ 
 & daz man ir jach vür rîcheit.\\ 
10 & \textbf{si} wâren lang unt breit\\ 
 & unt reichten vaste \textbf{unz} ûf die hant,\\ 
 & \textbf{dô} mans zuo\textbf{me} \textbf{sperîser} bant\\ 
 & \textbf{dâ} \textbf{niderhalp} eine spanne.\\ 
 & der wart dem küenen manne\\ 
15 & hundert \textbf{dâ} bereitet\\ 
 & unt wol \textbf{hin} nâch geleitet\\ 
 & von sînes neven liuten.\\ 
 & êren unt \textbf{triuten}\\ 
 & kunden \textbf{si im} mit werdecheit.\\ 
20 & daz \textbf{was} ir hêrren niht ze leit.\\ 
 & er streich - ine weiz wie lange - nâch,\\ 
 & unz er geste herberge \textbf{ersach}\\ 
 & ime lande ze Wâleis.\\ 
 & dâ was geslagen \textbf{vür} Kanvoleis\\ 
25 & manec poulûn ûf \textbf{die} plâne.\\ 
 & ine sag ez iu niht \textbf{nâch} wâne,\\ 
 & \textit{\begin{large}G\end{large}}ebiet \textbf{er}, sô ist ez wâr.\\ 
 & sîn volc hiez er ûf halden gar.\\ 
 & \textbf{der hêrre} sande vor \textbf{hin} în\\ 
30 & den kluogen \textbf{meisterknappen} sîn.\\ 
\end{tabular}
\scriptsize
\line(1,0){75} \newline
D \newline
\line(1,0){75} \newline
\textbf{3} \textit{Majuskel} D  \textbf{27} \textit{Initiale} D  \newline
\line(1,0){75} \newline
\textbf{3} er] \textit{om.} D \textbf{27} Gebiet] ÷ebiet D \newline
\end{minipage}
\hspace{0.5cm}
\begin{minipage}[t]{0.5\linewidth}
\small
\begin{center}*m
\end{center}
\begin{tabular}{rl}
 & der was \textbf{nâch} ritterschaft gevarn,\\ 
 & d\textit{â} man niht schilte dorfte sparn.\\ 
 & dô hiez \textbf{ouch er} bereiten sich\\ 
 & - sus \textbf{diu âventiure wert} mich -\\ 
5 & mit speren wol gemâlen,\\ 
 & \textbf{mit} grüenen zindâlen.\\ 
 & \textbf{iegelîcher} \textbf{hete} eine banier.\\ 
 & drîe hermî\textit{n} anker dran sô fier,\\ 
 & daz man ir jach vür rîcheit.\\ 
10 & \textbf{si} wâren lanc und breit\\ 
 & und reicheten vaste ûf die hant,\\ 
 & \textbf{dô} man si zuo\textbf{m} \textbf{sperîsen} bant\\ 
 & \textbf{d\textit{â}} \textbf{niderhalp} ein spanne.\\ 
 & der wart dem küenen manne\\ 
15 & hundert \textbf{d\textit{â}} bereitet\\ 
 & und wol \textbf{hin} nâch geleitet\\ 
 & von sînes neven liuten.\\ 
 & êren und \textbf{triuten}\\ 
 & kunden  \textbf{in} mit wirdicheit.\\ 
20 & daz \textbf{wart} ir hêrren niht zuo leit.\\ 
 & er streich - in weiz wie lange - nâch,\\ 
 & unz er geste herberge \textbf{ersach}\\ 
 & im lande zuo Wâleis.\\ 
 & d\textit{â} was geslagen \textbf{vür} K\textit{a}n\textit{v}oleis\\ 
25 & manic poulûn ûf \textbf{die} plâne.\\ 
 & \textit{in}e sag\textit{e} \textit{e}z iu niht \textbf{nâch} wâne.\\ 
 & gebiete\textit{t} \textbf{ir}, sô ist ez wâr.\\ 
 & sîn volc hiez er ûf halden gar.\\ 
 & \textbf{der hêrre} sante vor \textbf{ime} în\\ 
30 & den kluogen \textbf{meisterknappen} sîn.\\ 
\end{tabular}
\scriptsize
\line(1,0){75} \newline
m n o \newline
\line(1,0){75} \newline
\newline
\line(1,0){75} \newline
\textbf{1} nâch] zú o \textbf{2} dâ] Do m n o \textbf{3} ouch er] er ouch n \textbf{4} wert] beweret n beweren o \textbf{6} grüenen] gruͯnem n \textbf{7} eine] ein n (o) \textbf{8} hermîn] her minẏ \textit{nachträglich korrigiert zu:} herminẏn m \textbf{11} und] Jch o \textbf{12} bant] fant o \textbf{13} dâ] Do m n o \textbf{15} dâ] do m n o \textbf{17} sînes] [sinen]: sines o \textbf{20} wart] was n o \textbf{21} streich] treich o \textbf{23} Wâleis] Walleis n wagleisse o \textbf{24} dâ] Do m n o  $\cdot$ Kanvoleis] konnoleis \textit{nachträglich korrigiert zu:} konuoleis m kanfoleis n kamfoleis o \textbf{26} ine sage ez] Me sage ich es m n o \textbf{27} gebietet] Gebietten m (o) \textbf{28} hiez] hiesse n \textbf{29} în] hin o \newline
\end{minipage}
\end{table}
\newpage
\begin{table}[ht]
\begin{minipage}[t]{0.5\linewidth}
\small
\begin{center}*G
\end{center}
\begin{tabular}{rl}
 & der was \textbf{durch} rîterschaft gevaren,\\ 
 & dâ man niht schilte dorfte sparen.\\ 
 & dô hiez \textbf{ouch er} bereiten sich\\ 
 & - sus \textbf{wert diu âventiure} mich -\\ 
5 & mit speren wol gemâlen\\ 
 & \textbf{von} grüenen zendâlen.\\ 
 & \textbf{an} \textbf{ieslîchez} ein banier,\\ 
 & drî härmîn anker dran sô fier,\\ 
 & daz man ir jach vür rîcheit.\\ 
10 & \textbf{si} wâren lanc und breit\\ 
 & unde reichten vaste ûf die hant,\\ 
 & \textbf{dâ} man si zuo \textbf{dem} \textbf{îsen} bant\\ 
 & \textbf{niderhalp} ei\textit{n} spanne.\\ 
 & der wart dem küenen manne\\ 
15 & \textbf{wol} hundert \textbf{dâr} bereitet\\ 
 & unde wol \textbf{hin} nâch geleitet\\ 
 & von sînes neven liuten.\\ 
 & \textit{ê}ren und \textbf{getriuten},\\ 
 & kunden \textbf{sin} mit werdecheit.\\ 
20 & daz \textbf{was} ir hêrren niht ze leit.\\ 
 & er streich - ine weiz wie lange - nâch,\\ 
 & unzer geste herberge \textbf{ersach}\\ 
 & in dem lande ze Wâleis.\\ 
 & dâ was geslagen \textbf{vor} Kanvoleis\\ 
25 & manic pavelûn ûf \textbf{den} plân.\\ 
 & ich ensagez iu niht \textbf{vür} wân.\\ 
 & gebiet \textbf{ir}, sô ist ez wâr.\\ 
 & sîn volc hiez er ûf halden gar\\ 
 & \textbf{unde} sande vor \textbf{im} \textbf{hin} în\\ 
30 & den kluogen \textbf{knappen meister} sîn.\\ 
\end{tabular}
\scriptsize
\line(1,0){75} \newline
G I O L M Q R Z Fr21 Fr37 Fr44 \newline
\line(1,0){75} \newline
\textbf{3} \textit{Überschrift:} Auentiwer wie gahmuret vrowen Herzenlavdn gewan I   $\cdot$ \textit{Initiale} I  \textbf{21} \textit{Initiale} I  \textbf{27} \textit{Initiale} M Fr21 Fr37 Fr44  \newline
\line(1,0){75} \newline
\textbf{1} \textit{Die Verse 58.9-63.24 fehlen (Blattverlust)} R   $\cdot$ der] Er O  $\cdot$ durch] nach O (M) (Q) Z Fr21 Fr37 Fr44 \textbf{2} dâ] Do O Fr44  $\cdot$ niht schilte] schilt niht I  $\cdot$ dorfte] solde M \textbf{3} dô] Du I Da O Z  $\cdot$ hiez] liez Q  $\cdot$ ouch er] er ouch L M (Q) (Z) \textbf{4} wert] wer Z \textbf{5} gemâlen] gemal Fr37 \textbf{6} von] Mit Q  $\cdot$ grüenen zendâlen] grunē czundalen M gruͦnem zendal Fr37 \textbf{7} an] \textit{om.} O L M Q Z Fr21 Fr37 Fr44  $\cdot$ ieslîchez] iegeshlichem I Jeslicher het O (L) (Fr37) (Fr44) Jglicher had M Jctzlichs hett Q (Z) (Fr21)  $\cdot$ ein] eine Fr37 \textbf{8} anker dran sô] ancher O ander Q Anker so Fr44  $\cdot$ fier] viere Q \textbf{9} ir] \textit{om.} O  $\cdot$ jach] sagit M \textbf{11} unde] Sie Z  $\cdot$ ûf] vnz vf I (O) (M) (Z) (Fr37) vncz an Q \textbf{12} dâ] So O L M Q Z Fr21 Fr37 Fr44  $\cdot$ man si] mans O (L) (M) Q Z Fr21 Fr37 Fr44  $\cdot$ dem] den I des spers O L (Q) Fr21 Fr37 deme spers M (Z) speres Fr44 \textbf{13} niderhalp] Da nider halb O (M) (Z) (Fr21) (Fr37) Do niderhalp L (Q) (Fr44)  $\cdot$ ein] einer G \textbf{14} küenen] chuͤnem I (Fr21) (Fr37) kunē M (Fr44) kúne Q \textbf{15} wol] \textit{om.} O L M Q Z Fr21 Fr37 Fr44  $\cdot$ dâr] do O Q Fr44 \textbf{16} geleitet] [bereytet]: beleytet Q \textbf{17} von] Vonses L \textbf{18} êren] geren G  $\cdot$ getriuten] travten O (L) (M) (Z) (Fr21) (Fr44) [trew:nten]: trewenten  Q \textbf{19} kunden] Begvnden O (Fr21)  $\cdot$ sin] sie L \textbf{20} ir] orin M  $\cdot$ hêrren] hertzen Z \textbf{21} er] Do I Der Fr37  $\cdot$ streich] [strich]: streich O streht Q  $\cdot$ ine weiz wie] er im so I in ich enweiz wie L (Fr44)  $\cdot$ lange] [w]: slage Fr44 \textbf{22} unzer geste] vnz er der gest I (L) Unsern gestin M  $\cdot$ herberge] herbergen O Z (Fr37)  $\cdot$ ersach] sach I O L M Z Fr37 Fr44 \textbf{23} Wâleis] walays I waleiz L waleisz M Q waleys Fr37 \textbf{24} dâ] do I (O) (Q) (Fr44)  $\cdot$ Kanvoleis] kanpholeiz G kanfolais I convaleiz O kanvoleisz M kanúoleisz Q kamfoleis Z Canvoleis Fr21 kamvoleis Fr37 kauoleis Fr44 \textbf{25} manic] vil manc I  $\cdot$ den plân] die plane O dem plane L Q \textbf{26} ensagez] sag es Q (Z) (Fr44)  $\cdot$ iu] \textit{om.} O M  $\cdot$ vür wân] nach wane O L von wan M (Q) Z (Fr21) (Fr44) \textbf{27} ist] \textit{om.} I  $\cdot$ ez] \textit{om.} M \textbf{28} er] \textit{om.} Fr44  $\cdot$ ûf] wol Fr37 \textbf{29} unde] Der herre O L (M) Q Z Fr21 (Fr37) Fr44  $\cdot$ im hin] in O hin L (M) Q Z Fr21 Fr37 Fr44 \textbf{30} knappen meister] meister knapin M (Z) (Fr44) \newline
\end{minipage}
\hspace{0.5cm}
\begin{minipage}[t]{0.5\linewidth}
\small
\begin{center}*T (U)
\end{center}
\begin{tabular}{rl}
 & der was \textbf{nâch} ritterschaft gevarn,\\ 
 & d\textit{â} man niht schilte dorfte sparn.\\ 
 & dô hiez \textbf{er ouch} bereiten sich\\ 
 & - sus \textbf{w\textit{e}rt diu âventiure} mich -\\ 
5 & mit spern wol gemâle\textit{n}\\ 
 & \textbf{von} grüene\textit{n} zindâle\textit{n}.\\ 
 & \textbf{ieglîcher} \textbf{hete} ein banier.\\ 
 & drî härmîn enker dran sô fier,\\ 
 & daz man ir jach vür rîcheit.\\ 
10 & \textbf{die} wâren lanc und breit\\ 
 & und reicheten vaste \textbf{un\textit{z}} ûf die hant,\\ 
 & \textbf{sô} man\textit{s} zuo \textbf{sperîsene} bant\\ 
 & \textbf{dâ} \textbf{nidene} eine spanne.\\ 
 & der wart dem küenen manne\\ 
15 & hundert \textbf{wol} bereit\\ 
 & und wol \textbf{hinden} nâch geleit\\ 
 & von sînes neven liuten.\\ 
 & êren und \textbf{triuten}\\ 
 & kunden \textbf{si} mit wirdecheit.\\ 
20 & daz \textbf{was} i\textit{r} hêrren niht ze leit.\\ 
 & er streich - ine weiz wie lange - nâch,\\ 
 & unz er \textbf{zuo der} geste herberge \textbf{sach}\\ 
 & in dem lande ze Wâleis.\\ 
 & dâ was geslagen \textbf{vor} Kanvoleis\\ 
25 & manec pavelûn ûf \textbf{dem} plâne.\\ 
 & ich ensagez iu niht \textbf{von} wâne.\\ 
 & gebietet \textbf{ir}, sô ist ez wâr.\\ 
 & sîn volc hiez er ûf halten gar.\\ 
 & \textbf{der hêrre} sante vor \textbf{im} \textbf{hin} în\\ 
30 & den kluogen \textbf{knappen meister} sîn.\\ 
\end{tabular}
\scriptsize
\line(1,0){75} \newline
U V W T \newline
\line(1,0){75} \newline
\textbf{3} \textit{Majuskel} T  \textbf{4} \textit{Majuskel} T  \textbf{23} \textit{Majuskel} T  \textbf{27} \textit{Initiale} W  \textbf{29} \textit{Majuskel} T  \newline
\line(1,0){75} \newline
\textbf{2} dâ] Do U (V) W  $\cdot$ dorfte] dorste V \textbf{3} dô] Da V T  $\cdot$ er] \textit{om.} T \textbf{4} wert] wirt U \textbf{5} gemâlen] [gemalen]: gemale U \textbf{6} grüenen zindâlen] gruͦnem zindale U gruͤnem zendalen W \textbf{7} ieglîcher] Jegeliches V (T) \textbf{9} man ir] mens im V \textbf{10} die] si T \textbf{11} und] Die V  $\cdot$ unz] uns U \textit{om.} T \textbf{12} So man sy an die sper geband W  $\cdot$ mans] man U  $\cdot$ zuo sperîsene] zuͦ [*]: deme sperisin V zez spers Jsene T \textbf{13} dâ nidene] Da niderthalp V Nyderhalb dem eysen W vnde vürbaz T \textbf{15} wol] da V T schoͤne W \textbf{16} hinden nâch] [hin*]: hinnach V hin nach W T \textbf{19} kunden si] [Kunden*]: Kundenz in V kvnden sin T \textbf{20} ir] irn U \textbf{21} ine weiz] im enweiß W \textbf{22} unz] Bit U  $\cdot$ zuo] \textit{om.} V W T  $\cdot$ der] \textit{om.} T \textbf{23} Wâleis] waleise U walleis V waleiß W \textbf{24} dâ] do V Das W  $\cdot$ Kanvoleis] kanuoleis V kanuoleiß W \textbf{25} dem] die W T \textbf{26} ensagez iu] sage v́ch es V (W) \textbf{27} ir] irs V \textbf{29} vor im hin] vorhin W vor im T \newline
\end{minipage}
\end{table}
\end{document}
