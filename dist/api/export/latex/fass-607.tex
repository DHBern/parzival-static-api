\documentclass[8pt,a4paper,notitlepage]{article}
\usepackage{fullpage}
\usepackage{ulem}
\usepackage{xltxtra}
\usepackage{datetime}
\renewcommand{\dateseparator}{.}
\dmyyyydate
\usepackage{fancyhdr}
\usepackage{ifthen}
\pagestyle{fancy}
\fancyhf{}
\renewcommand{\headrulewidth}{0pt}
\fancyfoot[L]{\ifthenelse{\value{page}=1}{\today, \currenttime{} Uhr}{}}
\begin{document}
\begin{table}[ht]
\begin{minipage}[t]{0.5\linewidth}
\small
\begin{center}*D
\end{center}
\begin{tabular}{rl}
\textbf{607} & \textbf{\begin{large}S\end{large}ine} \textbf{getwungen} mich sô sêre nie.\\ 
 & ich hân ir \textbf{kleinœde} \textbf{al} hie.\\ 
 & \textbf{nû} \textbf{gelobet} ouch mîn dienst dar\\ 
 & gein der meide wol gevar.\\ 
5 & ouch \textbf{getrouwe} ich \textbf{wol}, si sî mir holt,\\ 
 & wand ich hân nôt durch si gedolt.\\ 
 & sît Orgeluse, diu rîche,\\ 
 & mit worten herzenlîche\\ 
 & ir minne mir versagete,\\ 
10 & ob ich \textbf{sît} prîs bejagete,\\ 
 & mir \textbf{würde} wol ode wê,\\ 
 & daz schuof diu werde Itonje.\\ 
 & i\textbf{ne} hân ir leider niht gesehen.\\ 
 & wil iwer \textbf{trôst} mir \textbf{helfe} jehen,\\ 
15 & sô bringet diz kleine vingerlîn\\ 
 & der clâren, süezen vrouwen mîn.\\ 
 & ir sît hie strîtes ledec gar,\\ 
 & ez \textbf{en}wære \textbf{grœzer} iwer schar,\\ 
 & zwêne oder mêre.\\ 
20 & wer jæhe mir \textbf{des} vür êre,\\ 
 & ob ich iuch slüege oder sicherheit\\ 
 & \textbf{twünge}? den \textbf{strît} mîn \textbf{hant ie meit}."\\ 
 & Dô sprach mîn hêr Gawan:\\ 
 & "ich bin doch werlîch ein man.\\ 
25 & wolt ir des niht prîs bejagen,\\ 
 & würde ich von iwerer hant \textbf{erslagen},\\ 
 & sô\textbf{ne} hân \textbf{ouch ich}\textbf{s} decheinen prîs,\\ 
 & daz ich gebrochen hân \textbf{diz} rîs.\\ 
 & wer jæhe mir\textbf{s} vür êre grôz,\\ 
30 & ob ich iuch slüege alsus blôz?\\ 
\end{tabular}
\scriptsize
\line(1,0){75} \newline
D Z \newline
\line(1,0){75} \newline
\textbf{1} \textit{Initiale} D Z  \textbf{23} \textit{Majuskel} D  \newline
\line(1,0){75} \newline
\textbf{1} Sine getwungen] Die twungen Z \textbf{2} al] \textit{om.} Z \textbf{3} mîn] minen Z \textbf{5} getrouwe] trew Z \textbf{7} rîche] rise Z \textbf{8} herzenlîche] hertencliche Z \textbf{12} Itonje] Jtonie D Jconie Z \textbf{14} trôst] gvte Z \textbf{18} enwære] wer danne Z \textbf{22} twünge] Betwuͤnge Z \textbf{27} ouch ichs] ich Z \textbf{28} diz] daz Z \newline
\end{minipage}
\hspace{0.5cm}
\begin{minipage}[t]{0.5\linewidth}
\small
\begin{center}*m
\end{center}
\begin{tabular}{rl}
 & \textbf{betwungen} mich sô sêre nie.\\ 
 & ich hab ir \textbf{kleinœtes} \textbf{n\textit{û}} hie.\\ 
 & \textbf{nû} \textbf{dinget} ouch mîn dienst dar\\ 
 & gegen der megde wol gevar.\\ 
5 & ouch \textbf{trûwe} ich \textbf{wol}, si sî mir holt,\\ 
 & wan ich hab nôt durch sî gedolt.\\ 
 & sît Urgeluse, diu rîche,\\ 
 & mit worten herzeclîche\\ 
 & ir minne mir versagte,\\ 
10 & ob ich \textbf{sît} prîs bejagte,\\ 
 & mir \textbf{werde} wol oder wê,\\ 
 & daz schuof diu werde Ithonie.\\ 
 & ich hân ir \textit{l}eider niht gesehen.\\ 
 & wil iuwer \textbf{trôst} mir \textbf{helfe} jehen,\\ 
15 & sô bringet diz kleine vingerlî\textit{n}\\ 
 & der clâren, süezen vrouwen mîn.\\ 
 & ir sît hie strîtes ledi\textit{c g}ar,\\ 
 & ez wær \textbf{den grœzer} iuwer schar,\\ 
 & \textbf{doch} zwêne oder mêre.\\ 
20 & wer jæhe mir \textbf{des} vür êre,\\ 
 & ob ich iuch slüege oder sicherheit\\ 
 & \textbf{twünge}? den \textbf{strît} mîn \textbf{hant ie streit}."\\ 
 & \begin{large}D\end{large}ô sprach mîn hêr Gawan:\\ 
 & "ich bin doch werlîch ein man.\\ 
25 & wolt ir des niht prîs bejagen,\\ 
 & würde ich von iuwer \textit{hant} \textit{\textbf{erslagen}},\\ 
 & \textit{sô hân \textbf{ich ouch} deheinen prîs},\\ 
 & daz ich gebrochen hân \textbf{daz} rîs.\\ 
 & \textit{w}er jæhe mir\textbf{s} vür êre grôz,\\ 
30 & ob ich iuch slüege alsus \dag grôz\dag ?\\ 
\end{tabular}
\scriptsize
\line(1,0){75} \newline
m n o \newline
\line(1,0){75} \newline
\textbf{15} \textit{Initiale} o   $\cdot$ \textit{Capitulumzeichen} n  \textbf{23} \textit{Initiale} m  \newline
\line(1,0){75} \newline
\textbf{2} kleinœtes] cleinoͯte n (o)  $\cdot$ nû] nẏe m (o) \textbf{3} dinget] dringet n  $\cdot$ dienst] dienste m \textbf{5} trûwe ich] getruwe ich ir n \textbf{12} Ithonie] jtonie m o itonie n \textbf{13} leider] beider m \textbf{15} \textit{nach 607.15:} Mit hoffelichem sitte adellich m   $\cdot$ bringet] bring o  $\cdot$ vingerlîn] fingerlich m \textbf{16} \textit{nach 607.16:} Vnd das aller schonst jungfrowelin m  \textbf{17} ledic gar] ledig vnd gar m \textbf{20} jæhe] ie o \textbf{22} twünge] Twinge n  $\cdot$ streit] [leit]: streit o \textbf{23} hêr] herre her n \textbf{25} niht prîs] >nit< prises o \textbf{26} \textit{Verse 607.26-27 kontrahiert zu:} Wurde ich von uͯwerm pris m  \textbf{27} ich ouch] auch >ich< o  $\cdot$ deheinen] do heinen n \textbf{29} wer jæhe] Veriehe m \textbf{30} grôz] gros \textit{nachträglich korrigiert zu:} blos m \newline
\end{minipage}
\end{table}
\newpage
\begin{table}[ht]
\begin{minipage}[t]{0.5\linewidth}
\small
\begin{center}*G
\end{center}
\begin{tabular}{rl}
 & \textbf{die} \textbf{betwungen} mich sô sêre nie.\\ 
 & ich hân ir \textbf{kleinœde} hie.\\ 
 & \textit{\textbf{nû}} \textbf{ge\textit{l}o\textit{b}t} \textit{ouch} mîn dienst dar\\ 
 & gein der meide wol gevar,\\ 
 & \hspace*{-.7em}\big| wan ich hân nôt durch si gedolt.\\ 
5 & \hspace*{-.7em}\big| ouch \textbf{trouwe} ich \textbf{wol}, si sî mir holt,\\ 
 & sît Orgeluse, diu rîche,\\ 
 & mit worten herzenlîche\\ 
 & ir minne mir versagete,\\ 
10 & ob ich prîs bejagete,\\ 
 & mir \textbf{würde} wol oder wê,\\ 
 & daz schuof diu werde Itonie.\\ 
 & ich\textbf{ne} hân ir leider niht gesehen.\\ 
 & wil iuwer \textbf{güete} mir \textbf{helfe} jehen,\\ 
15 & sô bringet diz klein vingerlîn\\ 
 & der clâren, süezen vrouwen mîn.\\ 
 & ir sît hie strîtes ledic gar,\\ 
 & ez \textbf{en}wære \textbf{danne grœzer} iuwer schar,\\ 
 & zwêne oder mêre.\\ 
20 & wer jæhe mir \textbf{des} vür êre,\\ 
 & ob ich iu\textit{ch} slüege oder sicherheit\\ 
 & \textbf{betwünge}? den \textbf{stiurte} mîn \textbf{manheit}."\\ 
 & \begin{large}D\end{large}ô sprach mîn hêr Gawan:\\ 
 & "ich bin doch werlîch ein man.\\ 
25 & welt ir des niht prîs bejagen,\\ 
 & würde ich von iuwer hant \textbf{geslagen},\\ 
 & sô\textbf{ne} hân \textbf{ouch \textit{ich}} deheinen prîs,\\ 
 & daz ich gebrochen hân \textbf{daz} rîs.\\ 
 & wer jæhe mir\textbf{s} vür êre grôz,\\ 
30 & ob \textit{ich} iuch slüege alsus blôz?\\ 
\end{tabular}
\scriptsize
\line(1,0){75} \newline
G I L M Z Fr51 \newline
\line(1,0){75} \newline
\textbf{1} \textit{Initiale} L Z  \textbf{3} \textit{Initiale} I  \textbf{5} \textit{Initiale} Fr51  \textbf{23} \textit{Initiale} G I M  \newline
\line(1,0){75} \newline
\textbf{1} die] Dene Fr51  $\cdot$ betwungen] twungen M (Z) (Fr51)  $\cdot$ mich] mir Fr51 \textbf{2} ir] \textit{om.} M \textbf{3} nû gelobt ouch] geholt unt G Geholt vnde ich I Gelobt vnd L (M) Vnde gelobet Fr51 \textbf{6} \textit{Versfolge 607.5-6} L M Z Fr51   $\cdot$ durch] \textit{om.} L \textbf{5} oͮch getruͤ ich ir wol si mir holt I \textbf{7} Orgeluse] Orguluse I Orgelise L  $\cdot$ rîche] rise Z \textbf{8} herzenlîche] hertencliche Z \textbf{10} ich] ich sit Z \textbf{11} wê] [wege]: wê G \textbf{12} diu werde] \textit{om.} Fr51  $\cdot$ Itonie] Jtonie G I (L) Jthonie M Jconie Z eltonie Fr51 \textbf{13} ichne] ich I (L) (M) \textbf{14} helfe] helfen I (M) \textbf{15} bringet] bring ich I \textbf{18} enwære] wer Z  $\cdot$ danne] \textit{om.} Fr51 \textbf{19} mêre] viere I \textbf{20} wer] Was Fr51  $\cdot$ jæhe mir] ie mir M hattich Fr51  $\cdot$ vür] \textit{om.} Fr51 \textbf{21} iuch] iv G  $\cdot$ sicherheit] twunge sicherheit I \textbf{22} den strit min hant ie gerne meit I  $\cdot$ betwünge] Betwuͯngen L  $\cdot$ stiurte mîn manheit] strit mýnhant ýe meit L (M) (Z) (Fr51) \textbf{23} Dô] DA M  $\cdot$ mîn] \textit{om.} Fr51  $\cdot$ hêr Gawan] ergawan M \textbf{24} doch werlîch] ouch werlichen M \textbf{25} welt ir] Ne woltir Fr51 \textbf{26} geslagen] erslagen I L (M) Z (Fr51) \textbf{27} ouch ich] oͮch G ich ouch M (Fr51) ich Z \textbf{29} jæhe] spricht M gaes Fr51  $\cdot$ mirs] mir Fr51 \textbf{30} ich] \textit{om.} G  $\cdot$ alsus] also L (M) \newline
\end{minipage}
\hspace{0.5cm}
\begin{minipage}[t]{0.5\linewidth}
\small
\begin{center}*T
\end{center}
\begin{tabular}{rl}
 & \textbf{die} \textbf{twungen} mich sô sêre nie.\\ 
 & ich hân ir \textbf{kleinœde} hie.\\ 
 & \textbf{gelobe\textit{t}} \textit{o}uch mîn dienst dar\\ 
 & gein der megede wol gevar.\\ 
5 & ouch \textbf{trûwe} ich, si sî mir holt,\\ 
 & wan ich hân nôt durch si gedolt.\\ 
 & sît Or\textit{ge}luse, diu rîche,\\ 
 & mit worten her\textit{z}eclîche\\ 
 & ir minne mir versagete,\\ 
10 & ob ich prîs bejagete,\\ 
 & mir \textbf{würde} wol oder wê,\\ 
 & daz schuof diu werde Itonie.\\ 
 & ich hân ir leider niht gesehen.\\ 
 & wil iuwer \textbf{güete} mir \textbf{helfen} jehen,\\ 
15 & sô bringe\textit{t} \textit{d}iz kleine vingerlîn\\ 
 & der clâren, süezen vrouwen mîn.\\ 
 & ir sît hie strîtes ledic gar,\\ 
 & ez \textbf{en}wære \textbf{grœzer dan} iuwer schar,\\ 
 & zwêne oder mêre.\\ 
20 & wer jæhe mir \textbf{daz} vür êre,\\ 
 & ob ich iuch slüege oder sicherheit\\ 
 & \textbf{betwünge}? den \textbf{strît} mîn \textbf{hant ie meit}."\\ 
 & \begin{large}D\end{large}ô sprach mîn hêr Gawan:\\ 
 & "ich bin doch werlîch ein man.\\ 
25 & wolt ir des niht prîs bejagen,\\ 
 & würde ich von iuwer hant \textbf{erslagen},\\ 
 & sô hân \textbf{ouch \textit{ich}} dekeinen prîs,\\ 
 & daz ich gebrochen hân \textbf{daz} rîs.\\ 
 & wer jæhe mir \textbf{daz} vür êre grôz,\\ 
30 & ob ich iuch slüege als blôz?\\ 
\end{tabular}
\scriptsize
\line(1,0){75} \newline
U V W Q R \newline
\line(1,0){75} \newline
\textbf{1} \textit{Capitulumzeichen} R  \textbf{23} \textit{Initiale} U V W R  \newline
\line(1,0){75} \newline
\textbf{1} twungen] zwingent R  $\cdot$ nie] nicht ni: R \textbf{2} hân] [an]: han V trag W \textbf{3} gelobet ouch] Gelobet mir auch U Ich gelobet [*]: oͮch V Gelobt eúch W (Q) Glopt Jch R  $\cdot$ mîn] [*]: minen V  $\cdot$ dienst] dieste R \textbf{4} gein] Ge R \textbf{5} ich] ich wol V W Q R \textbf{6} hân nôt] not han Q  $\cdot$ gedolt] dolt Q geholt R \textbf{7} Orgeluse] Oriluse U orgelusze Q orguluse R \textbf{8} herzeclîche] hertecliche U \textbf{10} ob ich] [O*]: Ob ich sit V \textbf{12} Itonie] Jtonie U Q R [y*onie]: ytonie V ytonie W \textbf{13} hân] enhan W  $\cdot$ leider] leidor V \textbf{14} mir] \textit{om.} W  $\cdot$ helfen] helfe V (W) \textbf{15} diz] mir diz U das R \textbf{18} grœzer dan] danne grozer V (W) (Q) (R)  $\cdot$ enwære] werde W werre R  $\cdot$ iuwer] die selbe úwer R \textbf{20} daz] dez V (W) (Q) (R) \textbf{22} [*]: Twúnge den strit min hant ie meit V  $\cdot$ Betwing den streit ich ie vermeit W  $\cdot$ Betunge das stúnd min hercz vngemeit R \textbf{23} Gawan] Gawin R \textbf{25} des niht] dez [*]: niht V den der Icht R \textbf{26} iuwer] v́were V \textbf{27} hân ouch ich] han auch U ich auch Q (R) \textbf{28} daz rîs] dis reis W \textbf{29} daz] dez V (W) \textit{om.} (Q)  $\cdot$ vür] fúr ein W \textbf{30} ich] \textit{om.} R  $\cdot$ als] alsus W \newline
\end{minipage}
\end{table}
\end{document}
