\documentclass[8pt,a4paper,notitlepage]{article}
\usepackage{fullpage}
\usepackage{ulem}
\usepackage{xltxtra}
\usepackage{datetime}
\renewcommand{\dateseparator}{.}
\dmyyyydate
\usepackage{fancyhdr}
\usepackage{ifthen}
\pagestyle{fancy}
\fancyhf{}
\renewcommand{\headrulewidth}{0pt}
\fancyfoot[L]{\ifthenelse{\value{page}=1}{\today, \currenttime{} Uhr}{}}
\begin{document}
\begin{table}[ht]
\begin{minipage}[t]{0.5\linewidth}
\small
\begin{center}*D
\end{center}
\begin{tabular}{rl}
\textbf{147} & vriunt, \textbf{nû} \textbf{saget} der künegîn,\\ 
 & ich begüzzes ân den willen mîn,\\ 
 & al dâ die werden sâzen,\\ 
 & die rehter wer vergâzen.\\ 
5 & ez \textbf{sîn} künege oder vürsten,\\ 
 & wes lâzent si \textbf{ir} wirt \textbf{erdürsten}?\\ 
 & wan holent si \textbf{im} hie sîn goltvaz?\\ 
 & ir sneller prîs wirt anders laz."\\ 
 & Der knappe sprach: "ich wirbe dir,\\ 
10 & swaz dû gesprochen hâst ze mir."\\ 
 & \textbf{er} reit \textbf{von} im ze Nantes în.\\ 
 & dâ volgeten im diu kindelîn\\ 
 & ûf den hof vür den palas,\\ 
 & dâ maneger \textbf{slahte} vuore was.\\ 
15 & schiere wart umb in gedranc.\\ 
 & Iwanet dar nâher spranc,\\ 
 & \textbf{ein} knappe valsches vrîe,\\ 
 & \textbf{der} bôt im kumpânîe.\\ 
 & \textbf{\textit{\begin{large}D\end{large}}er knappe sprach}: "got halde dich,\\ 
20 & bat \textbf{reden mîn muoter} mich,\\ 
 & ê daz ich schiede von ir hûs.\\ 
 & ich sihe hie manegen Artus.\\ 
 & Wer sol mich ritter machen?"\\ 
 & Iwanet begunde lachen.\\ 
25 & er sprach: "dû\textbf{ne} sihst \textbf{des rehten} niht,\\ 
 & daz aber schiere nû geschiht."\\ 
 & \textbf{er vuorte}n \textbf{în} zem palas,\\ 
 & dâ diu werde massenîe was.\\ 
 & sus vil kunder in schalle,\\ 
30 & er sprach: "got halde iuch \textbf{hêrren} alle,\\ 
\end{tabular}
\scriptsize
\line(1,0){75} \newline
D \newline
\line(1,0){75} \newline
\textbf{9} \textit{Majuskel} D  \textbf{19} \textit{Initiale} D  \textbf{23} \textit{Majuskel} D  \newline
\line(1,0){75} \newline
\textbf{16} Iwanet] Jwanet D \textbf{19} Der] ÷eR \textit{nachträglich korrigiert zu:} DeR D \textbf{24} Iwanet] Jwanet D \newline
\end{minipage}
\hspace{0.5cm}
\begin{minipage}[t]{0.5\linewidth}
\small
\begin{center}*m
\end{center}
\begin{tabular}{rl}
 & vriunt \textbf{mîn}, \textbf{sage} der künigîn,\\ 
 & ich begüzze si âne den willen mîn,\\ 
 & aldâ die werden sâzen,\\ 
 & die rehter we\textit{r} vergâzen.\\ 
5 & ez \textbf{sî} künige oder vürsten,\\ 
 & wes lânt si \textbf{ir} wirt \textbf{erdürsten}?\\ 
 & wanne holent s\textbf{inne} hie sîn goltvaz?\\ 
 & ir sneller prîs wirt anders laz."\\ 
 & \begin{large}D\end{large}er knabe sprach: "ich wirbe dir,\\ 
10 & waz dû gesprochen hâst ze mir."\\ 
 & \textbf{er} reit \textbf{von} ime ze Nantes în.\\ 
 & dô volgeten ime diu kindelîn\\ 
 & ûf den hof vür den palas,\\ 
 & d\textit{â} maniger \textbf{hande} vuore was.\\ 
15 & schiere wart umb in gedranc.\\ 
 & Iwanet dar nâher spranc,\\ 
 & \textbf{der} knappe valsches vrîe,\\ 
 & \textbf{er} bôt im kompânîe.\\ 
 & \textbf{der knappe sprach}: "got halte dich,\\ 
20 & bat \textbf{mîn muoter reden} mich,\\ 
 & ê daz ich schiede von ir hûs.\\ 
 & ich sihe hie manigen Artus.\\ 
 & wer sol mich \textbf{hie} ritter machen?"\\ 
 & Iwanet begunde lachen.\\ 
25 & er sprach: "dû \textbf{en}sihest \textbf{des rehten} niht,\\ 
 & daz aber schiere nû geschiht."\\ 
 & \textbf{er vuorte} in zuo dem palas,\\ 
 & dâ diu werde massenîe was.\\ 
 & sus vil kunde er in schalle,\\ 
30 & er sprach: "got halte iuch alle,\\ 
\end{tabular}
\scriptsize
\line(1,0){75} \newline
m n o \newline
\line(1,0){75} \newline
\textbf{9} \textit{Initiale} m   $\cdot$ \textit{Capitulumzeichen} n  \textbf{27} \textit{Illustration mit Überschrift:} Also die herren wurdent gefuͯret zuͦ dem palast vnd in (ir o  ) grosse zucht erbotten wart n (o)   $\cdot$ \textit{Initiale} n o  \newline
\line(1,0){75} \newline
\textbf{1} sage der] sprach die n \textbf{2} âne] ond n \textbf{4} die] Die do n  $\cdot$ wer] wert m \textbf{5} ez] Er n o \textbf{6} erdürsten] erdorfften o \textbf{7} holent] hoͯlet o  $\cdot$ goltvaz] fasz n (o) \textbf{10} waz] Das n \textbf{11} von] zú o \textbf{12} volgeten] volgen n \textbf{14} dâ] Do m n o  $\cdot$ vuore] vor n \textbf{16} Iwanet] Jwanet m n o  $\cdot$ dar] der o \textbf{18} kompânîe] kampanie m kamponie o \textbf{21} schiede] schiet n \textbf{22} hie] \textit{om.} o  $\cdot$ Artus] artús o \textbf{24} Iwanet] Jwanet m n o  $\cdot$ begunde] beguͯndet o \textbf{26} geschiht] beschicht n (o) \textbf{27} vuorte] fuͯrt n (o) \textbf{28} dâ] Do n  $\cdot$ werde] selige n \textbf{29} schalle] [schande]: schalle m \newline
\end{minipage}
\end{table}
\newpage
\begin{table}[ht]
\begin{minipage}[t]{0.5\linewidth}
\small
\begin{center}*G
\end{center}
\begin{tabular}{rl}
 & vriunt, \textbf{nû} \textbf{sage} der künigîn,\\ 
 & ich begüzze si âne den willen mîn,\\ 
 & al dâ die werden sâzen,\\ 
 & die rehter wer vergâzen.\\ 
5 & ez \textbf{sîn} künige oder vürsten,\\ 
 & wes lânt si \textbf{ir} wirt \textbf{erdürsten}?\\ 
 & wan holent s\textbf{im} hie sîn goltvaz?\\ 
 & ir sneller brîs wirt anders laz."\\ 
 & der knappe sprach: "ich wirbe dir,\\ 
10 & swaz dû gesprochen hâst ze mir."\\ 
 & \textbf{er} reit \textbf{vor} im ze Nantis în.\\ 
 & dô volgeten im diu kindelîn\\ 
 & ûf den hof vür den palas,\\ 
 & dâ maniger \textbf{hande} vuore was.\\ 
15 & \textbf{vil} schiere wart umbe in gedranc.\\ 
 & Ywanet dar nâher spranc,\\ 
 & \textbf{der} knappe valsches vrîe,\\ 
 & \textbf{unde} bôt im kumpânîe.\\ 
 & \textbf{dô sprach der gast}: "got halde dich.\\ 
20 & \textbf{alsus} bat \textbf{reden mîn muoter} mich,\\ 
 & ê daz ich schiede von ir hûs.\\ 
 & ich sihe hie manigen Artus.\\ 
 & wer sol mich rîter machen?"\\ 
 & Ywanet begunde lachen.\\ 
25 & er sprach: "dû sihest \textbf{des rehten} niht,\\ 
 & daz aber schiere nû geschiht."\\ 
 & \textbf{dô vuorter}n \textbf{hin} zem palas,\\ 
 & dâ \textit{diu} \textit{wer}de \textit{massenî}e was.\\ 
 & sus vil kunder in schalle,\\ 
30 & er sprach: "got halde iuch \textbf{hêrren} alle,\\ 
\end{tabular}
\scriptsize
\line(1,0){75} \newline
G I O L M Q R Z Fr65 \newline
\line(1,0){75} \newline
\textbf{9} \textit{Initiale} I  \textbf{11} \textit{Initiale} Q  \textbf{19} \textit{Initiale} R Z Fr65  \textbf{25} \textit{Initiale} I  \newline
\line(1,0){75} \newline
\textbf{2} begüzze si] begosz sie M begrusse sie Q begvͤz Z \textbf{4} rehter] ritter Q \textbf{5} sîn] si I sint M R \textbf{6} lânt] lat I  $\cdot$ erdürsten] duͯrsten L (M) (R) \textbf{7} holent sim] holnt sie O haltens im R holent niht Z  $\cdot$ hie] \textit{om.} M R  $\cdot$ sîn] ir O \textbf{9} sprach] sprich R  $\cdot$ wirbe] werde M \textbf{10} swaz] Waz L (Q) (R)  $\cdot$ dû] de R \textbf{11} er] vnd I Der L M Q R Z (Fr65)  $\cdot$ vor] von I (M) (Q) R Z  $\cdot$ Nantis] nanes G nantes I (L) M Nantys R \textbf{12} dô] vnd I Da M Z \textbf{13} den] ein Q \textbf{14} dâ] Do Q \textbf{15} umbe in] ein L im ein Q vmb in ein R  $\cdot$ gedranc] tranck Q \textbf{16} Ywanet dar] ẏwanet dar G ywein dar I Jwanet dar O L (M) Ywane dor Q Je welher R Iwanet d::: Fr65  $\cdot$ nâher] fuͯr L \textbf{17} valsches] vashes I \textbf{18} kumpânîe] kampanîe O (R) \textbf{19} dô] Da M Z  $\cdot$ gast] \textit{om.} Z \textbf{20} alsus bat reden] also bat I  $\cdot$ mîn] die R \textbf{21} ê daz ich] ê dann ich I Er das M  $\cdot$ ir] ieren R \textbf{22} Artus] Artuͯs L \textbf{24} Ywanet] ẏwanet G ywan I Jwanet O L M Ywaneck Q Jwan R Iwanet Fr65 \textbf{25} dû sihest] dun sihes I (M) \textbf{26} schiere nû] schier I Nu schire M \textbf{27} dô] Da M Z  $\cdot$ vuortern] furt ern Q fvrt er >in< O  $\cdot$ hin] \textit{om.} O L M Q R in Z \textbf{28} dâ] Do Q  $\cdot$ diu werde massenîe] manger hande foͮre G \textbf{29} sus] So L Vsz Q  $\cdot$ vil kunder in] enchunde er nih mit I vil chvnige er vant mit O vil konde er M kunde er vil in Q \textbf{30} hêrren] \textit{om.} O L M Q R Z \newline
\end{minipage}
\hspace{0.5cm}
\begin{minipage}[t]{0.5\linewidth}
\small
\begin{center}*T (U)
\end{center}
\begin{tabular}{rl}
 & vriunt, \textbf{nû} \textbf{sage} der künegîn,\\ 
 & ich begüzze si âne den willen mîn,\\ 
 & aldâ die werden sâzen,\\ 
 & die rehter wer vergâzen.\\ 
5 & ez \textbf{sîn} künege oder vürsten,\\ 
 & wes lânt si \textbf{den} wirt \textbf{dürsten}?\\ 
 & \textit{wan holent si \textbf{im} hie \textbf{niht} sîn goltvaz?}\\ 
 & \textit{ir sneller prîs wirt anders laz."}\\ 
 & \begin{large}D\end{large}er knabe sprach: "ich wirbe dir,\\ 
10 & waz dû gesprochen hâst zuo \textit{mir}."\\ 
 & \textbf{der} reit \textbf{von} im zuo Nantes în.\\ 
 & dô volgeten im diu kindelîn\\ 
 & ûf den hof vür den palas,\\ 
 & dâ maneger \textbf{hande} vuore was.\\ 
15 & \textbf{vil} schiere wart umb in gedranc.\\ 
 & Ywanet dar nâher spranc,\\ 
 & \textbf{der} knappe valsches vrîe,\\ 
 & \textbf{und} bôt im kompânîe.\\ 
 & \textbf{dô sprach der gast}: "got halte dich.\\ 
20 & \textbf{alsus} bat \textbf{diu muoter} mich,\\ 
 & ê daz ich schiede von ir hûs.\\ 
 & ich sihe hie manegen Artus.\\ 
 & wer sol mich rîter machen?"\\ 
 & Ywanet begunde lachen.\\ 
25 & er sprach: "dû sihest \textbf{rehte} niht,\\ 
 & daz aber schiere nû geschiht."\\ 
 & \textbf{er vuort}in \textbf{hin} zuo dem palas,\\ 
 & d\textit{â} diu werde massenîe was.\\ 
 & sus vil kunder in schalle,\\ 
30 & er sprach: "got halt iuch alle,\\ 
\end{tabular}
\scriptsize
\line(1,0){75} \newline
U V W T \newline
\line(1,0){75} \newline
\textbf{9} \textit{Initiale} U V W T  \textbf{16} \textit{Majuskel} T  \textbf{19} \textit{Initiale} W T  \textbf{27} \textit{Majuskel} T  \textbf{29} \textit{Majuskel} T  \newline
\line(1,0){75} \newline
\textbf{3} werden] recken T \textbf{4} rehter wer] rechtes W helde craft T \textbf{5} sîn künege] sei kúnig W \textbf{6} wirt dürsten] kúnig erdúrsten W \textbf{7} \textit{Die Verse 147.7-8 fehlen} U   $\cdot$ si im hie niht] sy im W si hie T \textbf{10} waz] Swaz V (T)  $\cdot$ mir] \textit{om.} U \textbf{11} der] [D*r]: Er V er T  $\cdot$ Nantes] nantis W \textbf{12} dô] da T \textbf{14} dâ] Do U V W  $\cdot$ vuore] leúte W \textbf{16} Ywanet] Jwanet T  $\cdot$ dar nâher] der nahir V do nahe W san dar naher T \textbf{18} und bôt] Got W  $\cdot$ kompânîe] Campanie U \textbf{19} dô sprach der gast] Der knappe sprach T \textbf{20} alsus bat diu] [*]: Suz bat reden min V sus hiez [div]: min T \textbf{22} hie] \textit{om.} W T \textbf{25} rehte] [*]: dez rehte V des rechten W (T) \textbf{26} schiere nû] nun vil W sciere doch T \textbf{27} er vuortin] Do vuͦrtern T  $\cdot$ zuo dem] fúr den W \textbf{28} dâ] Do U (V) W  $\cdot$ was] sas V \textbf{29} vil] \textit{om.} W \textbf{30} er sprach] \textit{om.} W  $\cdot$ alle] herren alle W herren [*]: alle T \newline
\end{minipage}
\end{table}
\end{document}
