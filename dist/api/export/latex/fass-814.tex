\documentclass[8pt,a4paper,notitlepage]{article}
\usepackage{fullpage}
\usepackage{ulem}
\usepackage{xltxtra}
\usepackage{datetime}
\renewcommand{\dateseparator}{.}
\dmyyyydate
\usepackage{fancyhdr}
\usepackage{ifthen}
\pagestyle{fancy}
\fancyhf{}
\renewcommand{\headrulewidth}{0pt}
\fancyfoot[L]{\ifthenelse{\value{page}=1}{\today, \currenttime{} Uhr}{}}
\begin{document}
\begin{table}[ht]
\begin{minipage}[t]{0.5\linewidth}
\small
\begin{center}*D
\end{center}
\begin{tabular}{rl}
\textbf{814} & "\begin{large}O\end{large}b ich durch iuch \textbf{gein} touf kum,\\ 
 & ist mir der touf ze \textbf{minnen} vrum?"\\ 
 & sprach der heiden, Gahmuretes kint.\\ 
 & "ez was \textbf{ie jenen} her ein wint,\\ 
5 & \textbf{swaz} \textbf{mich} \textbf{strît} oder minne twanc.\\ 
 & \textbf{des} sî kurz oder lanc,\\ 
 & \textbf{daz mich} \textbf{êrster} \textbf{schilt} übervienc,\\ 
 & \textbf{sît ich} nie grœzer nôt enpfienc.\\ 
 & durch zuht \textbf{solt ich minne} heln.\\ 
10 & nû\textbf{ne} \textbf{mag} ir\textbf{z} herze niht versteln."\\ 
 & Wen meinstû?", sprach Parzival.\\ 
 & "\textbf{êt} \textbf{jene} magt lieht gemâl,\\ 
 & mînes gesellen swester hie.\\ 
 & Wiltû mir helfen umbe sie,\\ 
15 & ich tuon ir rîcheit bekant,\\ 
 & sô daz ir dienent wîtiu lant."\\ 
 & Wiltû dich toufes lâzen wern",\\ 
 & sprach der wirt, "sô mahtû ir minne gern.\\ 
 & ich mac nû wol duzen dich.\\ 
20 & unser \textbf{rîchtuom} \textbf{nâch} gelîchet sich,\\ 
 & mînhalp vons Grâles krefte."\\ 
 & Hilf mir geselleschefte",\\ 
 & sprach Feirefiz Anschevin,\\ 
 & "bruoder, umbe die muomen dîn.\\ 
25 & holt man den touf mit strîte,\\ 
 & dar schaffe mich bezîte\\ 
 & \textbf{unt} lâz mich dienen umb ir lôn.\\ 
 & ich hôrte ie gerne solhen dôn,\\ 
 & \textbf{dâ} von \textbf{tjoste} sprîzen sprungen\\ 
30 & \textbf{unt} dâ swert ûf helmen \textbf{klungen}."\\ 
\end{tabular}
\scriptsize
\line(1,0){75} \newline
D \newline
\line(1,0){75} \newline
\textbf{1} \textit{Initiale} D  \textbf{11} \textit{Majuskel} D  \textbf{14} \textit{Majuskel} D  \textbf{17} \textit{Majuskel} D  \textbf{22} \textit{Majuskel} D  \newline
\line(1,0){75} \newline
\textbf{3} Gahmuretes] Gahmvrets D \textbf{11} Parzival] Parcifal D \textbf{23} Anschevin] Anscevin D \newline
\end{minipage}
\hspace{0.5cm}
\begin{minipage}[t]{0.5\linewidth}
\small
\begin{center}*m
\end{center}
\begin{tabular}{rl}
 & "ob ich durch iuch \textbf{zuo} touf kume,\\ 
 & ist mir der touf zuo \textbf{mînem} vrume?"\\ 
 & sprach der heiden, Gahmuretes kint.\\ 
 & "\textit{ez} was \textbf{ie j\textit{e}n\textit{e}n} her ein wint,\\ 
5 & \textbf{daz} \textbf{mich} \textbf{strît} oder minne twanc.\\ 
 & \textbf{des} sî kurz oder lanc,\\ 
 & \textbf{daz mich} \textbf{êrste der} \textbf{schilt} übervienc,\\ 
 & \textbf{sît ich} nie grœzer nôt enpfienc.\\ 
 & durch zuht \textbf{solt ich minne} heln.\\ 
10 & nû \textbf{mac} ir\textbf{z} herze niht versteln."\\ 
 & "wen meinestû?", sprach Parcifal.\\ 
 & "\textbf{eht} \textbf{jene} maget lieht gemâl,\\ 
 & mînes gesellen swester hie.\\ 
 & wiltû mir helfen \textit{um}b sie,\\ 
15 & ich tuon ir rîcheit bekant,\\ 
 & sô daz ir diene\textit{n}t wîtiu lant."\\ 
 & "wiltû dich toufes lâzen wern",\\ 
 & sprach der wirt, "sô mahtû ir minne gern.\\ 
 & ich mac nû wol dutzen dich.\\ 
20 & unser \textbf{rîchtuom} \textbf{nû} glîchet sich,\\ 
 & mînhalp von des Grâles krefte."\\ 
 & "hilf mir geselleschefte",\\ 
 & sprach Ferefiz A\textit{n}schevin,\\ 
 & "bruoder, umb die muomen dîn.\\ 
25 & holt man den touf mit st\textit{r}ît,\\ 
 & dar schaff \textbf{dû} mich bî zît\\ 
 & \textbf{und} lâz mich dienen umb ir lôn.\\ 
 & ich hôrte i\textit{e} gerne solhen dôn,\\ 
 & \textbf{dâ} von \textbf{juste} sprîzen sprungen\\ 
30 & \textbf{und} d\textit{â} swert ûf helm \textbf{erklungen}."\\ 
\end{tabular}
\scriptsize
\line(1,0){75} \newline
m n V V' W \newline
\line(1,0){75} \newline
\textbf{11} \textit{Initiale} W  \newline
\line(1,0){75} \newline
\textbf{1} \textit{Die Verse 808.12-816.5 fehlen} V'  \textbf{2} mînem] mynnen n (V)  $\cdot$ vrume] fromen W \textbf{3} Gahmuretes] gamurettes m gamiretes n gameretes V gamuretes W \textbf{4} ez] Jch m  $\cdot$ jenen her] jnnan her m (n) vnze her V meinem hertzen W \textbf{5} daz] Swaz V \textbf{7} daz] [Da*]: Da V \textbf{8} Nie groͤsser not ich seit entpfieng W \textbf{10} mac] enmag ich V kan W  $\cdot$ herze] hertzen V \textbf{11} Parcifal] Parzefal V partzifal W \textbf{14} umb] ob m \textbf{16} dienent] dienet m \textbf{17} toufes] toͮffen V \textbf{18} ir] irre V \textbf{20} rîchtuom] reichait W  $\cdot$ nû] noch V \textbf{23} Ferefiz] ferefis m ferrevis n ferevis V ferafis W  $\cdot$ Anschevin] auscevin m n antscheuein W \textbf{24} bruoder] So bruͦder n  $\cdot$ muomen] muͦme V W \textbf{25} strît] stit m \textbf{26} dû] \textit{om.} V \textbf{27} umb ir] vmbiren n (V) (W)  $\cdot$ lôn] lan W \textbf{28} ie] ich m  $\cdot$ dôn] tan W \textbf{29} dâ] Do n V W  $\cdot$ sprîzen] spriesse W  $\cdot$ sprungen] springen n W \textbf{30} dâ] do m n V \textit{om.} W  $\cdot$ helm] helmen W  $\cdot$ erklungen] erclingen n clungen V klingen W \newline
\end{minipage}
\end{table}
\newpage
\begin{table}[ht]
\begin{minipage}[t]{0.5\linewidth}
\small
\begin{center}*G
\end{center}
\begin{tabular}{rl}
 & "\begin{large}O\end{large}be ich durch iuch \textbf{ze} touf kum,\\ 
 & ist mir der touf ze \textbf{minnen} vrum?"\\ 
 & sprach der heiden, Gahmuretes kint.\\ 
 & "ez was \textbf{ie ennen} her ein wint,\\ 
5 & \textbf{swaz} \textbf{minne} \textbf{strît} ode minne twanc.\\ 
 & \textbf{diu wîl} sî kurz ode lanc,\\ 
 & \textbf{dô der schilt von} \textbf{êrst} \textbf{mich} übervienc,\\ 
 & nie grœzer nôt \textbf{ich sît} enpfienc.\\ 
 & durch zuht \textbf{ich minne solde} helen.\\ 
10 & nû\textbf{ne} \textbf{mag} ir\textbf{z} herze niht verstelen."\\ 
 & "wen meinstû?", sprach Parzival.\\ 
 & "\textbf{êt} \textbf{eine} maget lieht gemâl,\\ 
 & mînes gesellen swester hie.\\ 
 & wil dû mir helfen umbe sie,\\ 
15 & ich tuon ir rîcheit bekant,\\ 
 & sô daz ir dienent wîtiu lant."\\ 
 & "wil dû dich toufes lâzen weren",\\ 
 & sprach der wirt, "sô mahtû ir minne gern.\\ 
 & ich mac \textbf{doch} nû wol duzen dich.\\ 
20 & unser \textbf{rîcheit} \textbf{nâch} gelîchet sich,\\ 
 & mînhalp vons Grâles krefte."\\ 
 & "hilf mir geselleschefte",\\ 
 & sprach Feirafiz Antschevin,\\ 
 & "bruoder, umbe die muomen dîn.\\ 
25 & holt man den touf mit strîte,\\ 
 & dar schaffe mich bezîte.\\ 
 & lâ mich dienen umbe ir lôn.\\ 
 & ich hôrt ie gern solhen dôn,\\ 
30 & \hspace*{-.7em}\big| dâ swert ûf helm \textbf{klungen}\\ 
 & \hspace*{-.7em}\big| \textbf{unde} von \textbf{tjosten} sprîzen sprungen."\\ 
\end{tabular}
\scriptsize
\line(1,0){75} \newline
G I L Z \newline
\line(1,0){75} \newline
\textbf{1} \textit{Initiale} G L Z  \textbf{9} \textit{Initiale} I  \newline
\line(1,0){75} \newline
\textbf{1} Obe] lob I  $\cdot$ kum] chomen I \textbf{2} minnen] mynne L  $\cdot$ vrum] vrumen I \textbf{3} \textit{Versfolge 814.4-3} L   $\cdot$ Gahmuretes] Gahmvretes G [G*]: Gahmvretes L gamuretes Z \textbf{4} ie ennen] yenem L ienne Z \textbf{5} swaz] Waz L  $\cdot$ minne] mich Z \textbf{7} dô] Da Z  $\cdot$ von êrst mich] mich von erste L (Z) \textbf{9} ich minne solde] solte ich mýnne L (Z) \textbf{10} nûne] Nv Z \textbf{11} Parzival] parzifal I parcifal L Z \textbf{12} eine] iene L Z \textbf{17} toufes] des Taufes I \textbf{18} mahtû] maht du wol I maht Z \textbf{19} doch nû wol] nu wol I wol nu Z \textbf{21} vons] von Z \textbf{23} Feirafiz] feiraviz G ferefis L feirefiz Z  $\cdot$ Antschevin] anschevin G enthseuin I Anshevin L (Z) \textbf{26} schaffe] chauf ich I \textbf{28} ie] \textit{om.} I \textbf{30} ûf helm] vnde helm I vf helmen L \textbf{29} tjosten] spern I tiost L Z \newline
\end{minipage}
\hspace{0.5cm}
\begin{minipage}[t]{0.5\linewidth}
\small
\begin{center}*T
\end{center}
\begin{tabular}{rl}
 & "\begin{large}O\end{large}b ich durch iuch \textbf{zuo} toufe kome,\\ 
 & ist mir der touf zuo \textbf{minnen} vrome?"\\ 
 & sprach der heiden, Gahmuretes kint.\\ 
 & "ez was \textbf{unz} her ein wint,\\ 
5 & \textbf{waz} \textbf{minne} \textbf{streit} oder mi\textit{nne} twanc.\\ 
 & \textbf{diu wîle} sî kurz oder lanc,\\ 
 & \textbf{dô der schilt von} \textbf{êrst} \textbf{mich} übervienc,\\ 
 & nie grœzer nôt \textbf{ich sît} enpfienc.\\ 
 & durch zuht \textbf{kan ich minne} heln.\\ 
10 & nû \textbf{en}\textbf{kan} ir herze niht versteln."\\ 
 & "wen meinestû?", sprach Parcifal.\\ 
 & "\textbf{jene} maget lieht gemâl,\\ 
 & mînes gesellen swester hie.\\ 
 & wiltû mir helfen umb sie,\\ 
15 & ich tuon \textit{i}r rîcheit bekant,\\ 
 & sô daz ir dienent wîtiu lant."\\ 
 & "wiltû dich toufes lâzen wern",\\ 
 & sprach der wirt, "\textit{sô mah}tû ir minne gern.\\ 
 & ich mac \textbf{doch} nû wol duzen dich.\\ 
20 & unser \textbf{rîcheit} \textbf{n\textit{â}ch} glîchet sich,\\ 
 & mînhalp von des Grâles krefte."\\ 
 & "hilf mir geselleschefte",\\ 
 & sprach Ferefis Anschevin,\\ 
 & "bruoder, umb die muomen dîn.\\ 
25 & holt man den touf mit strîte,\\ 
 & dar schaffe mich bezîte.\\ 
 & lâ mich dienen umb ir lôn.\\ 
 & ich hôrte ie gerne solichen dôn,\\ 
30 & \hspace*{-.7em}\big| d\textit{â} swert û\textit{f} helmen \textbf{klungen}\\ 
 & \hspace*{-.7em}\big| \textbf{und} \textit{von} \textbf{josten} sprîzen sprungen.\\ 
\end{tabular}
\scriptsize
\line(1,0){75} \newline
U Q R \newline
\line(1,0){75} \newline
\textbf{1} \textit{Initiale} U R  \newline
\line(1,0){75} \newline
\textbf{1} kome] kam Q \textbf{2} ist] Vnd ist R  $\cdot$ der] die Q  $\cdot$ minnen] minem R \textbf{3} Gahmuretes] Gahmuͦretes U gamúretes Q Gahmurtes R \textbf{4} unz] mit U \textbf{5} minne twanc] mit twanc U \textbf{7} mich] nicht Q \textbf{9} kan] sold Q (R)  $\cdot$ minne] mi͑nnen Q \textbf{10} nû enkan] Nun enmac Q Minne mag R  $\cdot$ herze] hertzen Q (R) \textbf{11} wen] Wa R  $\cdot$ Parcifal] Parzifal U partzifal Q parczifal R \textbf{12} lieht] lúchtet R \textbf{15} ir] dir U úcht R \textbf{16} wîtiu] reiche Q witte R \textbf{18} sô mahtû] wiltuͦ U  $\cdot$ ir] \textit{om.} R \textbf{19} \textit{Versfolge 814.20-19} U   $\cdot$ doch] \textit{om.} Q  $\cdot$ duzen] ducken R \textbf{20} nâch] noch U nahen Q \textbf{21} mînhalp] Einhalb R \textbf{23} Ferefis] feirefisz Q feirefis R  $\cdot$ Anschevin] anshevin R \textbf{25} den] die Q \textbf{26} mich] mir Q \textbf{27} ir] irn U \textbf{30} dâ] Do U Q  $\cdot$ ûf] vz U  $\cdot$ helmen] helme Q R \textbf{29} und von] Vnd U Von von R  $\cdot$ josten] tyost R  $\cdot$ sprîzen] spuzen Q spiczen R \newline
\end{minipage}
\end{table}
\end{document}
