\documentclass[8pt,a4paper,notitlepage]{article}
\usepackage{fullpage}
\usepackage{ulem}
\usepackage{xltxtra}
\usepackage{datetime}
\renewcommand{\dateseparator}{.}
\dmyyyydate
\usepackage{fancyhdr}
\usepackage{ifthen}
\pagestyle{fancy}
\fancyhf{}
\renewcommand{\headrulewidth}{0pt}
\fancyfoot[L]{\ifthenelse{\value{page}=1}{\today, \currenttime{} Uhr}{}}
\begin{document}
\begin{table}[ht]
\begin{minipage}[t]{0.5\linewidth}
\small
\begin{center}*D
\end{center}
\begin{tabular}{rl}
\textbf{255} & \begin{large}E\end{large}r sprach: "ich hân gevrâget niht."\\ 
 & "owê, daz iuch mîn ouge siht",\\ 
 & sprach diu \textbf{jâmerbæriu} magt,\\ 
 & "\textbf{sît} ir vrâgens \textbf{sît} verzagt!\\ 
5 & ir sâhet doch \textbf{sölch} wunder grôz -\\ 
 & daz iuch vrâgens dô verdrôz,\\ 
 & \textbf{al} dâ ir wâret dem Grâle bî -,\\ 
 & \textbf{manege} vrouwen valsches vrî,\\ 
 & die werden Garschiloyen\\ 
10 & unt Repanse de schoyen,\\ 
 & \textbf{unt} \textbf{snîdende} silber unt bluotec sper!\\ 
 & owê, waz \textbf{wolt ir} zuo mir her?\\ 
 & geunêrter lîp, vervluochet man!\\ 
 & ir \textbf{truoget} den eiterwolves zan,\\ 
15 & dâ diu galle in der triwe\\ 
 & an iu \textbf{bekleip} \textbf{sô} niwe.\\ 
 & \textbf{iuch solte iwer wirt} erbarmet hân,\\ 
 & an dem got \textbf{wunder hât} getân,\\ 
 & unt het gevrâget sîner nôt.\\ 
20 & ir lebt unt sît an sælden tôt."\\ 
 & \textbf{Dô} sprach er: "liebiu niftel mîn,\\ 
 & tuo bezzeren willen gein mir schîn.\\ 
 & ich wandel, hân ich iht getân."\\ 
 & "Ir sult wandels sîn erlân",\\ 
25 & sprach diu maget, "mir ist wol bekant,\\ 
 & ze Munsalvæsche an iu verswant\\ 
 & êre unt rîterlîcher prîs.\\ 
 & ir \textbf{en}vindet \textbf{nû} \textbf{dekeinen gewîs}\\ 
 & dekeine \textbf{geinrede} an mir."\\ 
30 & Parzival \textbf{sus} schiet von ir.\\ 
\end{tabular}
\scriptsize
\line(1,0){75} \newline
D \newline
\line(1,0){75} \newline
\textbf{1} \textit{Initiale} D  \textbf{21} \textit{Majuskel} D  \textbf{24} \textit{Majuskel} D  \newline
\line(1,0){75} \newline
\textbf{9} Garschiloyen] Garsciloyen D \textbf{10} Repanse de schoyen] Repansce de scoyen D \textbf{22} bezzeren] [be*zeren]: bezzeren D \textbf{26} Munsalvæsche] Mvnsalvæsce D \textbf{28} gewîs] [g*îs]: gwîs D \newline
\end{minipage}
\hspace{0.5cm}
\begin{minipage}[t]{0.5\linewidth}
\small
\begin{center}*m
\end{center}
\begin{tabular}{rl}
 & er sprach: "i\textbf{ne} hân gevrâget niht."\\ 
 & "ouwê, daz iuch mîn ouge siht",\\ 
 & sprach diu \textbf{jâmerbære} maget,\\ 
 & "\textbf{sît} ir vrâgens \textbf{sît} verzaget!\\ 
5 & ir sâhet doch \textbf{solichiu} wunder grôz -\\ 
 & daz \textit{iuch} vrâgens dô verdrôz,\\ 
 & \textbf{al}dâ ir wâret dem Grâle bî -,\\ 
 & \textbf{manige} vrouwen valsches vrî\\ 
 & \textbf{ir sâhet}, die werden Gars\textit{ch}i\textit{l}oyen\\ 
10 & und Repanse de schoyen,\\ 
 & \textbf{snîdende} silber und bluotic sper!\\ 
 & owê, waz \textbf{wolt\textit{e}stû} zuo mir her?\\ 
 & geunêrter lîp, vervluoche\textit{t} \textit{m}an!\\ 
 & ir \textbf{truoget} den eiterwolves zan,\\ 
15 & dô diu galle in der triuwe\\ 
 & an iu \textbf{bekl\textit{ei}p} \textbf{sô} niuwe.\\ 
 & \textbf{iuch solt iuwer wirt} erbarmet hân,\\ 
 & an dem got \textbf{hât wunder} tân,\\ 
 & und hetet gevrâget sîner nôt.\\ 
20 & ir lebt und sît an sælden tôt."\\ 
 & \textbf{dô} sprach er: "liebiu niftel mîn,\\ 
 & tuo bezzeren willen gegen mir schîn.\\ 
 & ich wande\textit{l}, hân ich iht getân."\\ 
 & "ir sullet wandeles sîn erlân",\\ 
25 & sprach diu maget, "mir ist wol bekant,\\ 
 & ze Mu\textit{nt}s\textit{alv}asche an iu verswant\\ 
 & êre und ritterlîcher prîs.\\ 
 & ir \textbf{en}\textit{v}indet \textbf{nû} \textbf{dekeine wîs}\\ 
 & dekeine \textbf{gegenrede} an mir."\\ 
30 & Parcifal \textbf{sus} schiet von ir.\\ 
\end{tabular}
\scriptsize
\line(1,0){75} \newline
m n o Fr69 \newline
\line(1,0){75} \newline
\newline
\line(1,0){75} \newline
\textbf{1} ine] ich n o \textbf{3} jâmerbære] jomerliche n (o) \textbf{4} verzaget] verczagez o \textbf{5} sâhet] [sollent]: sohent o  $\cdot$ solichiu] sollich n (o)  $\cdot$ wunder] wunders n \textbf{6} iuch] \textit{om.} m  $\cdot$ dô] doch do n \textbf{8} vrouwen] frouwe m n (o) \textbf{9} werden] werde n  $\cdot$ Garschiloyen] garsi lioien m gasilioien n [salioien]: gesalioien o \textbf{10} Repanse de schoyen] reppanse de scoien m repansen de scoẏen n repansen der scoien o \textbf{11} snîdende] Schinende n Scheinde o  $\cdot$ sper] spor n sprer o \textbf{12} woltestû] wolttustu m wollestuͯ o \textbf{13} geunêrter] Geferweter n  $\cdot$ vervluochet man] verfluhet lip vnd man m \textbf{16} bekleip] becliep m bekeip o \textbf{18} hât wunder] wnder hat Fr69  $\cdot$ tân] getan n o Fr69 \textbf{21} \textit{Die Verse 255.21-30 fehlen} o  \textbf{22} schîn] \textit{om.} n \textbf{23} wandel] wande m \textbf{25} wol bekant] bekat n \textbf{26} Muntsalvasche] mvsel it asce m múntsaluasch n \textbf{27} ritterlîcher] ritterlichen n \textbf{28} envindet] enwindent m vindent n  $\cdot$ dekeine] do keine n \textbf{29} dekeine] Do heine n \textbf{30} sus schiet] schiet sus Fr69 \newline
\end{minipage}
\end{table}
\newpage
\begin{table}[ht]
\begin{minipage}[t]{0.5\linewidth}
\small
\begin{center}*G
\end{center}
\begin{tabular}{rl}
 & er sprach: "ich hân gevrâget niht."\\ 
 & "\begin{large}O\end{large}wê, daz iuch mîn ouge siht",\\ 
 & sprach diu \textbf{jâmer bernde} maget,\\ 
 & "\textbf{daz} ir vrâgens \textbf{sît} verzaget!\\ 
5 & ir sâhet doch \textbf{solch} wunder grôz -\\ 
 & daz iuch vrâgens dâ verdrôz,\\ 
 & dô ir wâret dem Grâle bî -,\\ 
 & \textbf{manige} vrouwen valsches vrî,\\ 
 & \textit{die werden} \textit{Garschiloien}\\ 
10 & \textit{und} \textit{Urrepanse de schoien},\\ 
 & \textbf{snîden} silber unde bluotic sper!\\ 
 & owê, waz \textbf{wolt ir} zuo mir her?\\ 
 & geunêrt lîp, vervluocht man!\\ 
 & ir \textbf{traget} den eiterwolves zan,\\ 
15 & dâ diu galle \textit{in} der triwe\\ 
 & an iu \textbf{beleip} \textbf{sô} niwe.\\ 
 & \textbf{iuch solt iwer wirt} erbarmet hân,\\ 
 & an dem got \textbf{wunder hât} getân,\\ 
 & unde het gevrâget sîner nôt.\\ 
20 & ir lebet unde sît an sælden tôt."\\ 
 & \textbf{dô} sprach er: "liebiu niftel mîn,\\ 
 & tuo bezzeren willen gein mir schîn.\\ 
 & ich wandel\textbf{z}, hân ich iht getân."\\ 
 & "ir sult wandels sîn erlân",\\ 
25 & sprach diu maget, "mirst wol bekant,\\ 
 & ze Muntsalvatsche an iu verswant\\ 
 & êre unde rîterlîcher brîs.\\ 
 & ir vindet \textbf{nimer} \textbf{deheine wîs}\\ 
 & deheine \textbf{gegenrede} an mir."\\ 
30 & Parzival \textbf{dô} schiet von ir.\\ 
\end{tabular}
\scriptsize
\line(1,0){75} \newline
G I O L M Q R Z Fr21 Fr40 Fr60 \newline
\line(1,0){75} \newline
\textbf{1} \textit{Initiale} L Q R Z Fr40  \textbf{2} \textit{Initiale} G  \textbf{9} \textit{Initiale} Fr21  \textbf{13} \textit{Initiale} M  \textbf{19} \textit{Initiale} I  \textbf{27} \textit{Initiale} O  \newline
\line(1,0){75} \newline
\textbf{1} ich] ich en M (Q) (Fr40)  $\cdot$ hân gevrâget] han gefrag R fraget Z \textbf{2} iuch] [mich]: ivch O dich Q \textbf{3} diu] de R  $\cdot$ jâmer bernde] iamerlichev I iæmerliche O (Fr21) yamerbere L (M) (Q) (R) (Z) iamerbærev Fr40 \textbf{4} daz] Seint Q (Fr40)  $\cdot$ vrâgens] frages Q \textbf{5} sâhet doch] saget doch M seheet Q sahet Fr40  $\cdot$ solch] solhev I sehlich Fr21 \textbf{6} dâ] do I O R Fr40 dach M \textit{om.} Q \textbf{7} dô] Da O L R Alda Z  $\cdot$ Grâle] grabin M \textbf{8} manige vrouwen] Mannige vrouwe M Mengú frow R \textbf{9} \textit{Die Verse 255.9-10 fehlen} G   $\cdot$ werden] werde Fr21  $\cdot$ Garschiloien] Gatschilogen O Garshiloien L (Fr40) gatschiloyen M gar shiloyen Q Gartschiloyen Z Gatschiloẏen Fr21 \textbf{10} und] Von L  $\cdot$ Urrepanse de schoien] ewrepanschi de shoien I vͤrrepans adeschoyen O Vrrepansa deshoien L vrrepansa detschoie M repansze de schoye Q Ranpanse deschoien R vrrepansen de schoyen Z vrrepansa de schoẏen Fr21 repanse de schoien Fr40 \textbf{11} snîden] vnd snidic I (O) Snidende L Vnde sniten M Vnd sneiden Q (R) (Z) (Fr21) (Fr40)  $\cdot$ silber] daz silber Z  $\cdot$ unde] \textit{om.} L \textbf{12} wolt] woldet I (L) \textbf{13} geunêrt] ÷evnerter M Geúnter Q  $\cdot$ vervluocht] verfluchtit M ver fluchtick Q \textbf{14} traget] trvͦget O (M) (Q) (R) (Z)  $\cdot$ eiterwolves zan] andren wolues zan I eiter wolfes zam Q elter wolff zan R \textbf{15} dâ] do I (O) (L) (M) (Q) (R) (Fr21)  $\cdot$ galle] gelle L  $\cdot$ in der] bi der G in O [b]: inder M  $\cdot$ triwe] riwe L \textbf{16} iu] ir R  $\cdot$ beleip] bekleip Q (Z) \textbf{18} wunder hât] hott wúnder Q wunder het R \textbf{19} het gevrâget] soͯlt gefraget han R \textbf{20} sælden] freuden I [frævden]: selden O \textbf{21} dô] Da Z \textbf{22} tuo] Tvͦt O (Fr60) tun Q  $\cdot$ willen] wille Q  $\cdot$ gein] an I O R Z \textbf{23} wandelz] wandel L  $\cdot$ hân ich iht] waz ich han L icht han ich R \textbf{24} wandels sîn] wandelosz si M \textbf{25} diu maget] div O si Fr21  $\cdot$ bekant] kekant R \textbf{26} ze Muntsalvatsche] zemvntsalvatsche G zemunschauasche I Munsalvatsche M Zu múnsalvasch Q Ze Munschaluasche R Zv montsalvatsche Z Ze Mvnsalvæsche Fr21 Ze mvntschalvasche Fr60 \textbf{27} êre] ÷re O \textbf{28} ir] Jrn O (M) Z Fr21  $\cdot$ nimer deheine] ev dehain I nv dehæinen O (Q) niht mer dekeine L Nu icheine M (R) (Fr21) mer keinen Z \textbf{29} deheine] Da heine M  $\cdot$ gegenrede] sin rede O  $\cdot$ an] von I \textbf{30} Parzival] parzifal I (M) Parcifal O L Z Fr21 Partzifal Q Parczifal R  $\cdot$ dô] da M Z \newline
\end{minipage}
\hspace{0.5cm}
\begin{minipage}[t]{0.5\linewidth}
\small
\begin{center}*T
\end{center}
\begin{tabular}{rl}
 & \begin{large}E\end{large}r sprach: "i\textbf{ne} habe gevrâget niht."\\ 
 & "ouwê, daz iuch mîn ouge siht",\\ 
 & sprach diu \textbf{jâmerbæriu} maget,\\ 
 & "\textbf{sît} ir vrâgens verzaget!\\ 
5 & ir sâhet doch \textbf{solh} wunder grôz -\\ 
 & daz iuch vrâgens dô verdrôz,\\ 
 & \textbf{al}dâ ir wâret dem Grâle bî -,\\ 
 & \textbf{unde} \textbf{maneger} vrouwen valsches vrî,\\ 
 & die werden Garschiloien\\ 
10 & unde Repansen de joien,\\ 
 & \textbf{unde} \textbf{snîdende} si\textit{l}ber unde bluotic sper!\\ 
 & ouwê, waz \textbf{wolt ir} zuo mir her?\\ 
 & geunêreter lîp, vervluocheter man!\\ 
 & ir \textbf{truoget} den eiter\textit{w}olves zan,\\ 
15 & dô diu galle in der triuwe\\ 
 & an iu \textbf{bleip} \textbf{al} niuwe.\\ 
 & \textbf{iuwer wirt iuch solte} erbarmet hân,\\ 
 & an dem got \textbf{wunder hât} getân,\\ 
 & unde hetet \textbf{ir} gevrâget sîner nôt.\\ 
20 & ir lebet unde sît an sælden tôt."\\ 
 & \hspace*{-.7em}\big| "Tuo bezzern w\textit{i}llen gegen mir schîn",\\ 
 & \hspace*{-.7em}\big| sprach er, "lieb\textit{iu} niftele mîn.\\ 
 & ich wandele, hân ich iht getân."\\ 
 & "Ir sult wandels sîn erlân",\\ 
25 & sprach diu maget, "mir ist wol bekant,\\ 
 & ze Munsalvasche an iu verswant\\ 
 & êre unde rîterlîcher prîs.\\ 
 & ir \textbf{en}vindet \textbf{nû} \textbf{deheine wîs}\\ 
 & deheine \textbf{widerrede} an mir."\\ 
30 & Parcifal \textbf{sus} schiet von ir.\\ 
\end{tabular}
\scriptsize
\line(1,0){75} \newline
T U V W Fr26 \newline
\line(1,0){75} \newline
\textbf{1} \textit{Initiale} T U V W Fr26  \textbf{22} \textit{Majuskel} T  \textbf{24} \textit{Majuskel} T  \newline
\line(1,0){75} \newline
\textbf{1} ine habe] in han ich U ich han V W \textbf{2} iuch] îv T \textbf{3} jâmerbæriu] iemerliche U W iamerbere V \textbf{4} vrâgens] vragendes sint V fragens so W \textbf{5} solh] selic U \textbf{6} iuch] îv T \textbf{7} aldâ] \textit{om.} Fr26 \textbf{8} Unde] \textit{om.} Fr26  $\cdot$ maneger] manige V \textbf{9} [J*]: Jr sohent die werden Garsiloien V  $\cdot$ Garschiloien] Garscilôien T karsi loyen W \textbf{10} Repansen de joien] Repansen deioien T Repanse de ioien U repansen de ẏoẏen V vrrepans de schoyen W \textbf{11} snîdende silber] snîdende siber T sendende U \textbf{12} wolt] woltent W  $\cdot$ ir] [*]: ir V \textbf{14} truoget] truͦgen W  $\cdot$ eiterwolves] eiter volves T \textbf{16} bleip] be cleip U (V)  $\cdot$ al] so U V W \textbf{17} iuwer] Vwert U  $\cdot$ wirt] wir W  $\cdot$ iuch solte] îv solte T solt eúch W \textbf{19} ir] \textit{om.} W \textbf{22} \textit{Versfolge 255.21-22} W   $\cdot$ willen] wellen T \textbf{21} sprach] [Sprach]: Do sprach V Do sprach W  $\cdot$ liebiu] liebe T \textbf{23} wandele] wandeln U wandelez V (W) \textbf{24} wandels] wandelens W \textbf{25} sprach] Do sprach W  $\cdot$ wol] \textit{om.} W  $\cdot$ bekant] v́rkant V \textbf{26} ze] Das zuͦ W  $\cdot$ Munsalvasche] Mvntsalvasce T Muntsalvasche U (V) montsaluatz W \textbf{28} envindet] vindet U (W) \textbf{29} deheine] \textit{om.} W  $\cdot$ widerrede] [*e]: gegen rede V \textbf{30} Parcifal] parzifal T (V) Partzifal W \newline
\end{minipage}
\end{table}
\end{document}
