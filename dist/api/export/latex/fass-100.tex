\documentclass[8pt,a4paper,notitlepage]{article}
\usepackage{fullpage}
\usepackage{ulem}
\usepackage{xltxtra}
\usepackage{datetime}
\renewcommand{\dateseparator}{.}
\dmyyyydate
\usepackage{fancyhdr}
\usepackage{ifthen}
\pagestyle{fancy}
\fancyhf{}
\renewcommand{\headrulewidth}{0pt}
\fancyfoot[L]{\ifthenelse{\value{page}=1}{\today, \currenttime{} Uhr}{}}
\begin{document}
\begin{table}[ht]
\begin{minipage}[t]{0.5\linewidth}
\small
\begin{center}*D
\end{center}
\begin{tabular}{rl}
\textbf{100} & "nû \textbf{habt} iuch an mîne pflege."\\ 
 & si wîst in heinlîche wege.\\ 
 & \textbf{sîner} geste pflac man wol ze vrumen,\\ 
 & swar halt \textbf{ir} wirt wære kumen.\\ 
5 & daz \textbf{gesinde} wart gemeine.\\ 
 & doch vuor er dan al eine,\\ 
 & wan zwei junchêrrelîn.\\ 
 & juncvrouwen unt diu künegîn\\ 
 & in vuorten, dâ er vreude vant\\ 
10 & unt al sîn trûren gar verswant.\\ 
 & enschumpfiert wart sîn riwe\\ 
 & \textbf{unt} sîn \textbf{hôchgemüete} al niwe.\\ 
 & daz \textbf{muose iedoch} \textbf{bî} liebe sîn.\\ 
 & vrou Herzeloyde, diu künegîn,\\ 
15 & ir \textbf{magetuomes} dâ âne wart.\\ 
 & die munde wâren ungespart.\\ 
 & die begunden si mit küssen zern\\ 
 & unt dem jâmer von \textbf{den vreuden} wern.\\ 
 & \begin{large}D\end{large}ar nâch er eine zuht begienc:\\ 
20 & si wurden ledic, die er dâ vienc.\\ 
 & Hardizen und Kaylet,\\ 
 & \textbf{seht}, die versuonde Gahmuret.\\ 
 & dâ ergienc ein sölhiu hôchgezît,\\ 
 & swer der hât gelîchet sît,\\ 
25 & des hant iedoch gewaldes pflac.\\ 
 & Gahmuret sich bewac,\\ 
 & sîn habe \textbf{was} \textbf{vil} ungespart.\\ 
 & arabesch golt geteilet wart\\ 
 & \textbf{armen} rîtern al gemeine\\ 
30 & \textbf{unt} den küneg\textit{e}n edele gesteine\\ 
\end{tabular}
\scriptsize
\line(1,0){75} \newline
D \newline
\line(1,0){75} \newline
\textbf{19} \textit{Initiale} D  \newline
\line(1,0){75} \newline
\textbf{21} Hardizen] Hardiezen D \textbf{22} Gahmuret] Gahmvret D \textbf{26} Gahmuret] Gahmvret D \textbf{28} arabesch] Aræbsc D \textbf{30} künegen] kvnegin D \newline
\end{minipage}
\hspace{0.5cm}
\begin{minipage}[t]{0.5\linewidth}
\small
\begin{center}*m
\end{center}
\begin{tabular}{rl}
 & "nû \textbf{habt} iuch, \textbf{hêrre}, an mî\textit{n}e pflege."\\ 
 & si wîsete in heimlîche wege.\\ 
 & \textbf{\begin{large}D\end{large}er} geste pflac man wol ze vromen,\\ 
 & war halt \textbf{der} wirt wære komen.\\ 
5 & daz \textbf{gesinde} wart gemeine.\\ 
 & doch vuor er dan aleine,\\ 
 & wan zwei junchêr\textit{r}elîn.\\ 
 & juncvrouwen und diu künigîn\\ 
 & in vuorten, dâ er vröuden vant\\ 
10 & und alle\textit{z} sîn trûren gar verswant.\\ 
 & entschumpfieret wart sîn riuwe\\ 
 & \textbf{und} sîn \textbf{hôchgemüete} al niuwe.\\ 
 & daz \textbf{muose iedoch} \textbf{in} liebe sîn.\\ 
 & vrouwe Herczeloid\textit{e}, diu künigîn,\\ 
15 & ir \textbf{magetuomes} dâ âne wart.\\ 
 & die munde wâren ungespart.\\ 
 & die begunden si mit küs\textit{s}en zern\\ 
 & und dem jâmer von \textbf{der vröude} wern.\\ 
 & dar nâch er eine zuht begienc:\\ 
20 & si wurden l\textit{e}d\textit{i}c, die er d\textit{â} vienc.\\ 
 & \textit{H}ardizen unde Kailet,\\ 
 & die versuonde Gahmuret.\\ 
 & d\textit{â} ergienc ein solich hôchzît,\\ 
 & wer der hât glîchet sît,\\ 
25 & des hant iedoch gewaltes pflac.\\ 
 & Gahmuret sich \textbf{des} bewac,\\ 
 & \textbf{daz} sîn habe \textbf{wart} \textbf{vil} ungespart.\\ 
 & arabesch golt geteilet wart\\ 
 & \textbf{arman}, rittern algemeine\\ 
30 & \textbf{und} den künigen edel gesteine\\ 
\end{tabular}
\scriptsize
\line(1,0){75} \newline
m n o \newline
\line(1,0){75} \newline
\textbf{3} \textit{Illustration mit Überschrift:} Also die junge konnigin mit iren jungfrouwen die herren fuͯrten zuͯ schouwen die burg n   $\cdot$ \textit{Initiale} m n o  \newline
\line(1,0){75} \newline
\textbf{1} \textit{Die Verse 99.22-100.2 fehlen} o   $\cdot$ mîne] mime m \textbf{7} junchêrrelîn] jungher telin m \textbf{9} vuorten] fursten o  $\cdot$ dâ] do n o  $\cdot$ vröuden] freide n o \textbf{10} allez] alle m o \textbf{12} al] \textit{om.} n \textbf{13} muose] muͯsse m (n) (o) \textbf{14} Herczeloide] herczeloiden m hertzeloit n herczeleid o \textbf{15} magetuomes] magetthum n (o)  $\cdot$ dâ] do n o \textbf{16} wâren] worent do n \textbf{17} begunden] begunde n (o)  $\cdot$ küssen] cusen m kusse o \textbf{18} dem] den n o  $\cdot$ vröude] freiden n o \textbf{20} ledic] leidg m  $\cdot$ er dâ] er do m (o) er n \textbf{21} Hardizen] Nardiczen m Nardisen n o  $\cdot$ Kailet] kaliet o \textbf{22} Gahmuret] gamiret n gamuret o \textbf{23} dâ] Do m n o  $\cdot$ hôchzît] hoch gezit n (o) \textbf{24} hât] hette n \textbf{26} Gahmuret] Gamiret n Gamuͯret o \textbf{29} arman] Armen n o  $\cdot$ algemeine] alle gemeine n \textbf{30} künigen] konnigin n konig o \newline
\end{minipage}
\end{table}
\newpage
\begin{table}[ht]
\begin{minipage}[t]{0.5\linewidth}
\small
\begin{center}*G
\end{center}
\begin{tabular}{rl}
 & "nû \textbf{halt} iuch an mîne pflege."\\ 
 & si wîste in heinlîche wege.\\ 
 & \textbf{sîner} geste pflac man wol ze vromen,\\ 
 & swar halt \textbf{ir} wirt wære komen.\\ 
5 & daz \textbf{gesinde} wart gemeine.\\ 
 & doch vuor er dan aleine,\\ 
 & wan zwei junchêrrelîn.\\ 
 & juncvrouwen unt diu künigîn\\ 
 & in vuorten, dâ er vröude vant\\ 
10 & unde al sîn trûren gar verswant.\\ 
 & entschumpfiert wart sîn riwe,\\ 
 & sîn \textbf{hôchgemüete} al niwe.\\ 
 & daz \textbf{muose iedoch} \textbf{von} liebe sîn.\\ 
 & vrô Herzeloide, diu künigîn,\\ 
15 & ir \textbf{magettuom} dâ âne wart.\\ 
 & die munde wâren ungespart.\\ 
 & die begunden si mit küssen zeren\\ 
 & \textit{und} dem jâmer von \textbf{den vröuden} weren.\\ 
 & dar nâch er eine zuht begienc:\\ 
20 & si wurden ledic, die er dâ vienc.\\ 
 & Hardiz und Kailet,\\ 
 & die versuonde Gahmuret.\\ 
 & dâ ergienc ein solch hôchzît,\\ 
 & swer der hât gelîchet sît,\\ 
25 & des hant iedoch gewaltes pflac.\\ 
 & Gahmuret sich \textbf{des} bewac,\\ 
 & sîn habe, \textbf{diu} \textbf{was} ungespart.\\ 
 & arabensch golt geteilt wart\\ 
 & \textbf{armen} rîteren algemeine.\\ 
30 & den künigen edel gesteine\\ 
\end{tabular}
\scriptsize
\line(1,0){75} \newline
G I O L M Q R Z Fr48 \newline
\line(1,0){75} \newline
\textbf{1} \textit{Initiale} I O M  \textbf{5} \textit{Capitulumzeichen} L  \textbf{19} \textit{Initiale} I R Z Fr48  \newline
\line(1,0){75} \newline
\textbf{1} nû] ÷v O  $\cdot$ halt] hapt I (O) (Q) \textbf{2} wîste in] wist in I weiste O wustin M westen Z \textbf{3} sîner] der sinen I  $\cdot$ geste] gesten R  $\cdot$ wol ze vromen] zcu [fmen]: fvmen M \textbf{4} swar] Wa L (Q) Was M Wer R  $\cdot$ ir] der Z  $\cdot$ wirt wære] vrivnde weren O \textbf{6} doch] Do L Ouch M  $\cdot$ dan] dar R (Z) \textbf{7} junchêrrelîn] Jvnchfrowelin L (Q) \textbf{9} vuorten] furte Q  $\cdot$ dâ] do Q \textbf{10} unde] \textit{om.} L  $\cdot$ al] ob O als Q \textbf{11} wart] waz L  $\cdot$ riwe] ruwen L \textbf{12} sîn] Vnde sîn O (L) (M) (Q) (R) (Z) (Fr48)  $\cdot$ hôchgemüete] hochmut M  $\cdot$ niwe] nuͯwen L \textbf{13} daz] Do Q Da R  $\cdot$ muose] muͤs I  $\cdot$ von] vor O L (M) Q R Z wi Fr48 \textbf{14} Herzeloide] herzelaude I herzenlavde O Hertzelauͯde L herczeloide M herzeloude Q herczeleide R herzelovde Z herceloude Fr48 \textbf{15} ir] irn I (Fr48)  $\cdot$ magettuom] magetvmes L  $\cdot$ dâ] do Q \textbf{16} wâren] warn da I \textbf{17} si] [s*]: sich I sich R \textbf{18} und] \textit{om.} G  $\cdot$ dem] \textit{om.} L  $\cdot$ jâmer] trovren O  $\cdot$ den] dem O dē M Q \textbf{19} begienc] begingᵉ Fr48 \textbf{20} er] \textit{om.} L  $\cdot$ dâ] do Q  $\cdot$ vienc] vingᵉ Fr48 \textbf{21} Hardiz] Hardisz Q Hardis R  $\cdot$ Kailet] Gahilet I Kaylet O L (Q) (R) Fr48 Gailet Z \textbf{22} die] Seht die O L (M) (Q) (R) Z (Fr48)  $\cdot$ versuonde] versuͦnet R (Fr48)  $\cdot$ Gahmuret] Gamvret O (M) (Z) Gahmuͯret L gamúret Q Gachmuret R Fr48 \textbf{23} dâ] Do M Q  $\cdot$ solch] solheu I solches L selic M (Fr48)  $\cdot$ hôchzît] hochgezit I (O) L R \textbf{24} swer] Wer L M Q R  $\cdot$ der] dem L  $\cdot$ gelîchet] gelicher I O \textbf{25} pflac] pfagk Q \textbf{26} Gahmuret] Gamvret O (M) (Z) Gahmuͯret L Gamúret Q Gachmuret Fr48  $\cdot$ bewac] verwagk Q \textbf{27} diu] \textit{om.} R \textbf{28} arabensch] arabischez I (O) Arabisch L Q R Z Fr48 Arabische M \textbf{30} den] der I Vnd den Z Fr48  $\cdot$ künigen] kunginne I  $\cdot$ edel] edeln Q \newline
\end{minipage}
\hspace{0.5cm}
\begin{minipage}[t]{0.5\linewidth}
\small
\begin{center}*T (U)
\end{center}
\begin{tabular}{rl}
 & "nû \textbf{halt} iuch an mîne pflege."\\ 
 & si wîstin heimelîche wege.\\ 
 & \textbf{sîner} geste pflac man wol zuo vromen,\\ 
 & war halt \textbf{ir} wirt wære komen.\\ 
5 & daz \textbf{ingesinde} wart gemeine.\\ 
 & doch vuor er dan aleine,\\ 
 & wan zwei junchêrrelîn.\\ 
 & juncvrouwen und diu künegîn\\ 
 & in vuorte\textit{n}, dâ er vreude vant\\ 
10 & und al sîn trûren gar verswant.\\ 
 & ents\textit{ch}umpfieret wart sîn riuwe\\ 
 & \textbf{und} sîn \textbf{hôher muot} a\textit{l} \textit{niuw}e.\\ 
 & daz \textbf{muoz doch} \textbf{von} liebe sîn.\\ 
 & vrou Herzeloyde, diu künegîn,\\ 
15 & ir \textbf{magetuomes} dâ âne wart.\\ 
 & die munde wâren ungespart.\\ 
 & die begunden si mit küssen z\textit{ern}\\ 
 & und dem jâmer von \textbf{der vreuden} wern.\\ 
 & \begin{large}D\end{large}ar nâch er ein zuht begienc:\\ 
20 & si wurden ledic, d\textit{i}er d\textit{â} vienc.\\ 
 & Hardysen und Kaylet,\\ 
 & \textbf{seht}, die versuo\textit{n}te Gahmuret.\\ 
 & dâ ergienc ein solich hôchgezît,\\ 
 & wer d\textit{er} hât gelîchet sît,\\ 
25 & des hant iedoch gewaltes pflac.\\ 
 & Gahmuret sich \textbf{des} bewac,\\ 
 & sîn habe, \textbf{diu} \textbf{was} ungespart.\\ 
 & arabesch golt geteilet wart\\ 
 & \textbf{armen} rittern algemeine.\\ 
30 & den künegen edel gesteine\\ 
\end{tabular}
\scriptsize
\line(1,0){75} \newline
U V W T \newline
\line(1,0){75} \newline
\textbf{3} \textit{Majuskel} T  \textbf{8} \textit{Majuskel} T  \textbf{19} \textit{Initiale} U V W T  \textbf{23} \textit{Majuskel} T  \textbf{30} \textit{Majuskel} T  \newline
\line(1,0){75} \newline
\textbf{1} halt] habent W \textbf{2} wîstin] wiset in V \textbf{3} sîner] [Siner]: Der V \textbf{4} war halt] swer halt V (T) Wo yoch W \textbf{5} ingesinde] gesinde V W T \textbf{9} \textit{Versdoppelung von 100.15 statt 100.9:} ir magetvͦm da ane wart T   $\cdot$ vuorten] vuͦrte U  $\cdot$ dâ] do V W \textbf{10} vnd alsin riͮwe da verschart T  $\cdot$ und] \textit{om.} W  $\cdot$ al sîn] als in U Als sein W  $\cdot$ gar] do W \textbf{11} entschumpfieret] Entsumpfieret U \textbf{12} hôher muot] hochgemuͤt W  $\cdot$ al niuwe] alle wege U \textbf{13} muoz] muͤste V (T) muͦse W  $\cdot$ doch] iedoch T  $\cdot$ von] vor W \textbf{14} Herzeloyde] Herzeleide U Hertzelaude V hertzeloyle W \textbf{15} magetuomes dâ] magetuͦmes do U V W magetvͦm da T \textbf{16} ungespart] do nit gespart W \textbf{17} zern] zim U \textbf{18} von] mit W  $\cdot$ der] den W T \textbf{20} si wurden] Er ließ sy W  $\cdot$ dier] der U  $\cdot$ dâ] do U V \textit{om.} W T \textbf{21} Hardysen] Hardyzen U Hardẏsen V Hardis W Hardŷs T  $\cdot$ Kaylet] kalet U Kaẏlet V gaylet W \textbf{22} seht] \textit{om.} W T  $\cdot$ versuonte] versuͦte U versuͤnde auch W  $\cdot$ Gahmuret] Gahmuͦret U Gamuret V (W) \textbf{23} dâ] Do U V W  $\cdot$ ein] \textit{om.} V \textbf{24} wer] swer V (T)  $\cdot$ der] do U W \textbf{25} hant] band W \textbf{26} Gahmuret] Gahmuͦret U Gamuret V W \newline
\end{minipage}
\end{table}
\end{document}
