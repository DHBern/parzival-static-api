\documentclass[8pt,a4paper,notitlepage]{article}
\usepackage{fullpage}
\usepackage{ulem}
\usepackage{xltxtra}
\usepackage{datetime}
\renewcommand{\dateseparator}{.}
\dmyyyydate
\usepackage{fancyhdr}
\usepackage{ifthen}
\pagestyle{fancy}
\fancyhf{}
\renewcommand{\headrulewidth}{0pt}
\fancyfoot[L]{\ifthenelse{\value{page}=1}{\today, \currenttime{} Uhr}{}}
\begin{document}
\begin{table}[ht]
\begin{minipage}[t]{0.5\linewidth}
\small
\begin{center}*D
\end{center}
\begin{tabular}{rl}
\textbf{478} & \begin{large}D\end{large}ô Frimutel den lîp verlôs,\\ 
 & mîn vater, nâch im man dô kôs\\ 
 & sînen eldesten sun ze künege dar,\\ 
 & ze vogte dem Grâle unt \textbf{des Grâles} schar.\\ 
5 & daz was mîn bruoder Anfortas,\\ 
 & \textbf{des} krône und rîcheit wirdec was.\\ 
 & \multicolumn{1}{l}{ - - - }\\ 
 & \multicolumn{1}{l}{ - - - }\\ 
 & dannoch wir wênec wâren.\\ 
 & dô mîn bruoder gein den jâren\\ 
 & kom vür der \textbf{gran sprünge} zît,\\ 
10 & mit sölher jugent hât minne \textbf{ir} strît.\\ 
 & sô twinget si ir vriunt sô sêre,\\ 
 & man mac es \textbf{ir} jehen \textbf{z}unêre.\\ 
 & swelch Grâles hêrre aber minne gert\\ 
 & anders, denne diu schrift in wert,\\ 
15 & der muoz \textbf{es} komen \textbf{ze} arbeit\\ 
 & unt in \textbf{siufzebæriu} herzeleit.\\ 
 & mîn hêrre unt der bruoder mîn\\ 
 & kôs im eine \textbf{vriundîn},\\ 
 & \textbf{des} in dûhte \textbf{mit} \textbf{guotem} site.\\ 
20 & \textbf{swer diu was}, \textbf{daz sî dâ} mite.\\ 
 & in \textbf{ir} dienst er sich zôch,\\ 
 & sô daz diu zagheit in vlôch.\\ 
 & des wart von sîner clâren hant\\ 
 & verdürkelt \textbf{manec} schildes rant.\\ 
25 & dâ bejagte an âventiure\\ 
 & der süeze unt der gehiure:\\ 
 & wart ie hôher prîs \textbf{erkant}\\ 
 & über elliu rîterlîchiu lant,\\ 
 & \textbf{von} dem mære was er \textbf{der} vrîe.\\ 
30 & 'Amor' was sîn krîe.\\ 
\end{tabular}
\scriptsize
\line(1,0){75} \newline
D Fr31 \newline
\line(1,0){75} \newline
\textbf{1} \textit{Initiale} D  \newline
\line(1,0){75} \newline
\textbf{1} Frimutel] Frimvtel D \textbf{12} man mac es ir] Wan mac irs Fr31 \textbf{14} diu schrift in] in div schrift Fr31 \textbf{15} ze] in Fr31 \textbf{19} mit] in Fr31 \textbf{24} manec] manegez Fr31 \textbf{29} von] Vor Fr31 \newline
\end{minipage}
\hspace{0.5cm}
\begin{minipage}[t]{0.5\linewidth}
\small
\begin{center}*m
\end{center}
\begin{tabular}{rl}
 & dô Fr\textit{imu}tel den lîp verlôs,\\ 
 & mîn vater, nâch \textit{im} man dô kôs\\ 
 & sînen altesten sun zuo künige dar,\\ 
 & zuo v\textit{o}gte dem Grâl und \textbf{sîner} schar.\\ 
5 & daz was mîn bruoder Anfort\textit{a}s,\\ 
 & \textbf{der} krône und rîcheit wirdic was,\\ 
 & wan er sich i\textit{e} \textit{s}êre\\ 
 & verliez \textit{ûf} triuwe und êre.\\ 
 & dannoch wir wênic wâren.\\ 
 & dô mîn bruoder gegen den jâren\\ 
 & kam vür der \textbf{g\textit{r}a\textit{n}sprunge} zît,\\ 
10 & mit solicher jugent het minne strît.\\ 
 & sô twinget si ir vriunt sô sêre,\\ 
 & man mac es jehen \textbf{vür} unêre.\\ 
 & welich Grâles hêrre aber minne gert\\ 
 & anders, dan diu schrift in wert,\\ 
15 & der muoz \textbf{es} komen \textbf{zuo} arbeit\\ 
 & und \textit{in} \textbf{siu\textit{f}z\textit{e}bæriu} herzeleit.\\ 
 & mîn hêrre und der bruoder mîn\\ 
 & kôs im ein \textbf{künigîn},\\ 
 & \textbf{diu} in dûhte \textbf{solicher} site,\\ 
20 & \textbf{daz triuwe und zuht im wonte} mite.\\ 
 & in \textbf{der} dienst er sich zôch,\\ 
 & sô daz diu zagheit in vlôch.\\ 
 & des wart von sîner clâren hant\\ 
 & verdürkelt \textbf{maniges} schiltes rant.\\ 
25 & dô bejagete an âventiure\\ 
 & der süeze und der gehiure:\\ 
 & wart i\textit{e} hôher prîs \textbf{erkant}\\ 
 & über alliu ritterlîchiu \textit{l}ant,\\ 
 & \textbf{vor} dem mære was er \textbf{der} vrîe.\\ 
30 & 'Amor' was sîn krîe.\\ 
\end{tabular}
\scriptsize
\line(1,0){75} \newline
m n o \newline
\line(1,0){75} \newline
\newline
\line(1,0){75} \newline
\textbf{1} Frimutel] fronittel m frimútel n fruͯmutel o \textbf{2} im] min \textit{nachträglich korrigiert zu:} Jm m \textbf{4} vogte] fuͯgtte m \textbf{5} bruoder] brúde o  $\cdot$ Anfortas] anfortes m \textbf{6} ie sêre] ẏe mere vnd sere m \textbf{6} verliez] Verliesse n  $\cdot$ ûf] \textit{om.} m \textbf{9} der] \textit{om.} n  $\cdot$ gransprunge] garsprunge m \textbf{10} solicher] solichen o  $\cdot$ strît] ir strit n \textbf{11} sêre] [serer]: sere m \textbf{13} hêrre] herren o  $\cdot$ minne] [minne]: mẏne m \textbf{14} schrift in wert] geschrifft in gewert n \textbf{15} muoz] muͯsse n \textbf{16} in siufzebæriu] suffszerber m sin súffzeber n \textbf{18} kôs] Der kosz n  $\cdot$ künigîn] frindin n (o) \textbf{19} dûhte] dúchte o \textbf{24} maniges] manig n \textbf{25} dô] Das n \textbf{27} ie] yr m \textbf{28} lant] hant m \newline
\end{minipage}
\end{table}
\newpage
\begin{table}[ht]
\begin{minipage}[t]{0.5\linewidth}
\small
\begin{center}*G
\end{center}
\begin{tabular}{rl}
 & dô Frimutel den lîp verlôs,\\ 
 & mî\textit{n} vater, nâch im man dô kôs\\ 
 & sînen e\textit{l}testen sun ze künige dar,\\ 
 & ze vogete dem Grâl unde \textbf{des Grâles} schar.\\ 
5 & daz was mîn bruoder Anfortas,\\ 
 & \textbf{der} krône unde rîcheit wirdic was.\\ 
 & \multicolumn{1}{l}{ - - - }\\ 
 & \multicolumn{1}{l}{ - - - }\\ 
 & dannoch wir wênic wâren.\\ 
 & dô mîn bruoder gegen den jâren\\ 
 & kom vür de\textit{r} \textbf{gran sprünge} zît,\\ 
10 & mit solher jugent hât minne \textbf{ir} strît.\\ 
 & sô twinget si ir vriunt sô sêre,\\ 
 & man mac es \textbf{ir} jehen \textbf{ze} unêre.\\ 
 & swelch Grâles hêrre aber minne gert\\ 
 & anders, danne diu schrift in wert,\\ 
15 & der muoz \textbf{des} komen \textbf{in} arbeit\\ 
 & unde in \textbf{siuftebæriu} herzeleit.\\ 
 & mîn hêrre und der bruoder mîn\\ 
 & kôs im eine \textbf{vriundîn},\\ 
 & \textbf{des} in dûhte \textbf{mit} \textbf{guotem} site.\\ 
20 & \textbf{swer diu was}, \textbf{daz sî dâ} mite.\\ 
 & in \textbf{ir} dienste er sich zôch,\\ 
 & sô daz diu zageheit in vlôch.\\ 
 & des wart von sîner clâren hant\\ 
 & verdürkelt \textbf{maniges} schildes rant.\\ 
25 & dâ bejaget an âventiure\\ 
 & der süeze unde der gehiure:\\ 
 & wart \textit{ie} hôher brîs \textbf{erkant}\\ 
 & über elliu rîterlîch\textit{iu} lant,\\ 
 & \textbf{von} dem mære was er vrîe.\\ 
30 & 'Amor' was sîn krîe.\\ 
\end{tabular}
\scriptsize
\line(1,0){75} \newline
G I O L M Z Fr18 Fr49 \newline
\line(1,0){75} \newline
\textbf{1} \textit{Initiale} O L Z Fr18  \textbf{13} \textit{Initiale} I  \newline
\line(1,0){75} \newline
\textbf{1} dô] ÷o O Da M Z  $\cdot$ Frimutel] frimvtel G frumuntel I Frimvntel O Frýmvtel L frymutel M Frẏmvtel Fr18 Frimuntel Fr49  $\cdot$ den] [sin]: den G \textbf{2} mîn] Minen G L  $\cdot$ nâch im man] man nach im I Fr49 nach im O (M) Fr18  $\cdot$ dô] da I M Z Fr49 \textbf{3} eltesten] entesten G elsten O \textbf{4} ze vogete dem] vber den I (Fr49)  $\cdot$ des Grâles] der O Fr18 \textbf{5} was] \textit{om.} M  $\cdot$ Anfortas] Amfortas L \textbf{8} dô] Da M Z  $\cdot$ mîn] mir Z \textbf{9} der] den G M Z \textbf{10} ir] \textit{om.} O M Fr18 \textbf{11} sô sêre] so so sere I zuͯ sere L \textbf{12} es ir] ir ez Fr49 \textbf{13} swelch] Swelhs I Welchs L Fr49 Wilch M  $\cdot$ aber] \textit{om.} Fr49 \textbf{14} diu schrift in] diu shrift I in div schrift O der schrift Fr49 \textbf{15} des] ez O L (M) sin Z  $\cdot$ in] zv Z \textbf{16} \textit{Vers 478.16 fehlt} M  \textbf{17} hêrre] hercze M \textbf{18} vriundîn] [frau]: freundein Fr49 \textbf{19} des] Dy M  $\cdot$ mit] in O L  $\cdot$ guotem] guͤten I gutē Fr49 \textbf{20} swer] Wer L M (Fr49)  $\cdot$ sî] im L \textbf{23} des] Das M  $\cdot$ sîner] [sinen]: siner L \textbf{24} \textit{Die Verse 478.24-27 fehlen} M   $\cdot$ verdürkelt] durchel I Fr49  $\cdot$ maniges] manc I (Z) (Fr49) \textbf{25} dâ] do I  $\cdot$ bejaget] beiagte Z  $\cdot$ an] ein I \textbf{27} wart ie] Wart so G War ie O \textbf{28} rîterlîchiu] riterlichen G \textbf{29} von] Vor O L M Z  $\cdot$ vrîe] do frîe O (L) der frie Z \newline
\end{minipage}
\hspace{0.5cm}
\begin{minipage}[t]{0.5\linewidth}
\small
\begin{center}*T
\end{center}
\begin{tabular}{rl}
 & \begin{large}D\end{large}ô Frimutel den lîp verlôs,\\ 
 & mîn vater, nâch im \textit{man} dô kôs\\ 
 & sînen eltesten sun ze künege dar,\\ 
 & ze vogete dem Grâle unde \textbf{des} schar.\\ 
5 & daz was mîn bruoder Anfortas,\\ 
 & \textbf{der} krône unde rîcheit wirdic was.\\ 
 & \multicolumn{1}{l}{ - - - }\\ 
 & \multicolumn{1}{l}{ - - - }\\ 
 & dannoch wir wênic wâren.\\ 
 & Dô mîn bruoder gegen den jâren\\ 
 & kom vür de\textit{r} \textbf{gran sprünge} zît,\\ 
10 & mit sölher \textit{jugent} hât minne strît.\\ 
 & sô twinget sir vriunt sô \textit{s}êre,\\ 
 & man mac es \textbf{ir} jehen \textbf{z}unêre.\\ 
 & swelch Grâles hêrre aber minnen gert\\ 
 & anders, danne diu schrift in wert,\\ 
15 & der muoz \textbf{es} komen \textbf{in} arbeit\\ 
 & unde in \textbf{siuftebærez} herzeleit.\\ 
 & Mîn hêrre unde der bruoder mîn\\ 
 & kôs im eine \textbf{vriundîn},\\ 
 & \textbf{diu} in dûhte \textbf{in} \textbf{guotem} site.\\ 
20 & \textbf{swer diu was}, \textbf{daz s\textit{î} dâ} mite.\\ 
 & in \textbf{ir} dienste er sich zôch,\\ 
 & sô daz diu zageheit in vlôch.\\ 
 & des wart von sîner clâren hant\\ 
 & verdürkelt \textbf{manec} schiltes rant.\\ 
25 & dâ bejagete an âventiure\\ 
 & der süeze unde der gehiure:\\ 
 & wart ie hôher prîs \textbf{bekant}\\ 
 & über alliu rîterlîchiu \textit{l}ant,\\ 
 & \textbf{vor} dem mære was er \textbf{dô} \textbf{der} vrîe.\\ 
30 & 'Amor' was \textit{sîne} krîe.\\ 
\end{tabular}
\scriptsize
\line(1,0){75} \newline
T U V W Q R \newline
\line(1,0){75} \newline
\textbf{1} \textit{Initiale} T V W Q  \textbf{8} \textit{Majuskel} T  \textbf{17} \textit{Majuskel} T  \newline
\line(1,0){75} \newline
\textbf{1} \textit{Die Verse 453.1-502.30 fehlen} U   $\cdot$ Frimutel] frymvtel T frimuntel V friműtel Q \textbf{2} im] in W  $\cdot$ man] \textit{om.} T R [*]: man V  $\cdot$ dô] \textit{om.} W \textbf{3} sînen] Sin R  $\cdot$ künege] kunigar Q \textbf{4} des] [d*]: siner V des grals W Q (R) \textbf{6} \textit{Die Verse 478.6¹-6² sind am Rand nachgetragen und später radiert:} wande ::: sich ie sere / verlies vf truwe ::: ere V  \textbf{7} wir] [*]: wir V mir do R \textbf{9} der] den T W [*]: der V die R  $\cdot$ gran sprünge] garn sprungen R \textbf{10} jugent] minne T [*]: iugent V  $\cdot$ minne strît] [mir]: minne strit T [*]: minne strit V meine streit Q \textbf{11} sêre] srere T \textbf{12} es ir] [*]: ez ir V irs R \textbf{13} swelch] Welchs W Welh Q (R)  $\cdot$ minnen] [*]: minne V minne W (Q) R \textbf{14} diu schrift in wert] [*]: die schrift in wert V in die schrift wert Q die geschrifftt in lert R \textbf{15} \textit{Versdoppelung 478.15-16 nach 478.14} V   $\cdot$ es] des Q \textbf{16} siuftebærez] [sv́fzeber*]: sv́fzebere V sufzebern Q sunffzenbere R \textbf{18} eine vriundîn] ein breűndin Q \textbf{19} diu] Des W Q R  $\cdot$ in guotem] [*tte]: solher V in gutē Q ein gutte R  $\cdot$ site] siten Q \textbf{20} [S*]: Daz zvht vnde ere ir wonte mitte V  $\cdot$ swer] Wer W Q R  $\cdot$ was] \textit{om.} W  $\cdot$ daz sî dâ mite] daz siv damite T sy do mitte W da sie do miten Q \textbf{21} ir] irn W  $\cdot$ zôch] do zoch R \textbf{23} des] Do W \textbf{24} verdürkelt] Verdvnkelt V Wirdikeit R  $\cdot$ rant] rang R \textbf{25} Do beiaget [in]: an auentv́re V  $\cdot$ dâ] Do W Q R \textbf{27} bekant] [*]: erkant V erkant W Q R \textbf{28} lant] hant T [h*]: lant V \textbf{29} dem] den R  $\cdot$ dô der] [*]: der V do W Q R \textbf{30} sîne krîe] krie T sein krei W (R) sein kye Q \newline
\end{minipage}
\end{table}
\end{document}
