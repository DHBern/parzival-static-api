\documentclass[8pt,a4paper,notitlepage]{article}
\usepackage{fullpage}
\usepackage{ulem}
\usepackage{xltxtra}
\usepackage{datetime}
\renewcommand{\dateseparator}{.}
\dmyyyydate
\usepackage{fancyhdr}
\usepackage{ifthen}
\pagestyle{fancy}
\fancyhf{}
\renewcommand{\headrulewidth}{0pt}
\fancyfoot[L]{\ifthenelse{\value{page}=1}{\today, \currenttime{} Uhr}{}}
\begin{document}
\begin{table}[ht]
\begin{minipage}[t]{0.5\linewidth}
\small
\begin{center}*D
\end{center}
\begin{tabular}{rl}
\textbf{603} & \begin{large}D\end{large}az ors sô verre \textbf{hin} nider vlôz,\\ 
 & des loufens in dar nâch verdrôz,\\ 
 & \textbf{wande}\textbf{r swære harnas} truoc;\\ 
 & er hete \textbf{wunden ouch} genuoc.\\ 
5 & Nû treip ez \textbf{ein werve} her,\\ 
 & daz erz erreichte mit dem sper,\\ 
 & \textbf{al} dâ der regen unt \textbf{de\textit{r}} guz\\ 
 & \textbf{erbrochen} \textbf{hete} wîten vluz\\ 
 & an einer tiefen halden.\\ 
10 & \textbf{daz uover} was gespalden,\\ 
 & daz Gringuljeten nerte.\\ 
 & mit dem \textbf{sper} erz kêrte\\ 
 & sô nâhe \textbf{her} zuo anz lant,\\ 
 & \textbf{den zoum ergreif er} mit der hant.\\ 
15 & su\textit{s} zôch mîn hêr Gawan\\ 
 & daz ors hin \textbf{ûz} ûf den plân.\\ 
 & Ez schutte sich, dô ez genas,\\ 
 & der schilt dâ niht bestanden was.\\ 
 & er gurte dem orse unt nam den schilt.\\ 
20 & swen sînes kumbers niht bevilt,\\ 
 & daz lâze ich sîn; er het doch nôt,\\ 
 & sît ez diu minne im gebôt.\\ 
 & Orgeluse, diu glanze,\\ 
 & in jagete nâch dem kranze;\\ 
25 & daz \textbf{was} ein ellenthaftiu vart.\\ 
 & \textbf{der boum was alsô} bewart,\\ 
 & wæren Gawans zwêne, die m\textit{üe}sen ir leben\\ 
 & umbe den kranz hân gegeben.\\ 
 & des pflac der künec Gramoflanz.\\ 
30 & Gawan brach iedoch den kranz.\\ 
\end{tabular}
\scriptsize
\line(1,0){75} \newline
D Z \newline
\line(1,0){75} \newline
\textbf{1} \textit{Initiale} D  \textbf{2} \textit{Initiale} Z  \textbf{5} \textit{Majuskel} D  \textbf{17} \textit{Majuskel} D  \newline
\line(1,0){75} \newline
\textbf{2} dar] hin Z \textbf{3} swære] swern Z \textbf{4} wunden ouch] ouch wunden Z \textbf{7} al dâ] Do Z  $\cdot$ der guz] des gvzz D \textbf{8} erbrochen hete] Gebrochen heten Z \textbf{11} Gringuljeten] Gringvlieten D (Z) \textbf{13} her zuo] zv im Z \textbf{14} Daz erz er greif mit der hant Z \textbf{15} sus] svz D \textbf{17} dô] da Z \textbf{27} müesen] mvͦsen D (Z) \textbf{29} Gramoflanz] Gramoͮlanz D Gramoflantz Z \newline
\end{minipage}
\hspace{0.5cm}
\begin{minipage}[t]{0.5\linewidth}
\small
\begin{center}*m
\end{center}
\begin{tabular}{rl}
 & daz ros sô verre \textbf{hin} nider vlôz,\\ 
 & des loufens in dar nâch verdrôz,\\ 
 & \textbf{wan} \textbf{er swær harnasch} truoc;\\ 
 & er hete \textbf{wunden ouch} genuoc.\\ 
5 & nû treip ez \textbf{alsô dâ} her,\\ 
 & daz erz erreichte mit dem sper,\\ 
 & \textbf{al}dâ der regen und \textbf{der} guz\\ 
 & \textbf{erbrochen} \textbf{hete} wîten vluz\\ 
 & an einer tiefen halden.\\ 
10 & \textbf{daz uover} was gespalden,\\ 
 & daz Gringuleten nerte.\\ 
 & mit dem \textbf{sper} erz kêrte\\ 
 & sô nâhe \textbf{her} zuo an daz lant,\\ 
 & \textbf{den z\textit{o}um ergrei\textit{f e}r} mit der hant.\\ 
15 & sus zôch mîn hêr Gawan\\ 
 & daz ros hin ûf den plân.\\ 
 & ez schutte sich, dô ez genas,\\ 
 & der schilt d\textit{â} niht bestanden was.\\ 
 & er gurte dem ros und nam den schilt.\\ 
20 & wen sînes kumbers niht bevilt,\\ 
 & daz lâz i\textit{ch} sîn; er het doch nôt,\\ 
 & sît ez diu minne im gebôt.\\ 
 & Urgeluse, diu glanze,\\ 
 & i\textit{n} jagete nâch dem kranze;\\ 
25 & daz \textbf{was} ein ellenthaftiu vart.\\ 
 & \textbf{der boum was alsô} bewart,\\ 
 & wæren Gawans zwên, die müest\textit{en} ir leben\\ 
 & umb den kranz hân gegeben.\\ 
 & des pflac der künic Gramolanz.\\ 
30 & Gawan brach iedoch den kranz.\\ 
\end{tabular}
\scriptsize
\line(1,0){75} \newline
m n o \newline
\line(1,0){75} \newline
\textbf{21} \textit{Initiale} n  \newline
\line(1,0){75} \newline
\textbf{2} des] Das o \textbf{4} ouch] ouch ouch n \textbf{5} ez] >es< o \textbf{8} vluz] flos o \textbf{10} daz] Sas o  $\cdot$ was] wer n \textbf{11} Gringuleten] gringuletten m \textbf{14} Den zum ergreiff ergreiff er mit der hant m  $\cdot$ den] Dem o \textbf{15} hêr] herre her n \textbf{18} dâ] do m n o \textbf{21} ich] in m \textbf{23} Urgeluse] Vrgelusse m \textbf{24} in] Jne m \textbf{27} Gawans] gawanes o  $\cdot$ müesten] muͯst m o \textbf{29} Gramolanz] gramolancz m gramonlantz n gramonlancz o \textbf{30} brach] bracht o \newline
\end{minipage}
\end{table}
\newpage
\begin{table}[ht]
\begin{minipage}[t]{0.5\linewidth}
\small
\begin{center}*G
\end{center}
\begin{tabular}{rl}
 & daz ors sô verre nider vlôz,\\ 
 & des loufe\textit{n}s in dar nâch verdrôz,\\ 
 & \textbf{swæren harnasch er} truoc;\\ 
 & er het \textbf{ouch wunden} genuoc.\\ 
5 & nû treip ez \textbf{ein werbe} her,\\ 
 & daz erz erreichete mit dem sper,\\ 
 & dâ der regen unde guz\\ 
 & \textbf{gebrochen} \textbf{heten} wîten vluz\\ 
 & an einer tiefen halden.\\ 
10 & \textbf{dâr ûf her} was gespalden,\\ 
 & daz Gringulieten nerte.\\ 
 & mit dem \textbf{zoume} erz kêrte\\ 
 & sô nâhen zuo \textbf{im} an daz lant,\\ 
 & \textbf{daz erz ergreif} mit der hant.\\ 
15 & sus zôch mîn hêr Gawan\\ 
 & daz ors hin \textbf{ûz} ûf den plân.\\ 
 & ez schutte sich, dô ez genas,\\ 
 & der schilt dâ niht bestanden was.\\ 
 & er gurt dem orse unde nam den schilt.\\ 
20 & swen sînes kumbers niht bevilt,\\ 
 & daz lâz ich sîn; er het doch nôt,\\ 
 & sît ez diu minne im gebôt.\\ 
 & Orgeluse, diu glanze,\\ 
 & in jagete nâch deme kranze,\\ 
25 & daz ein ellenthaftiu vart\\ 
 & \textbf{in brâht ze dem boume, der was} bewart.\\ 
 & wæren Gawans zwêne, die müesen ir leben\\ 
 & umb den kranz hân gegeben.\\ 
 & des pflac der künic Gramoflanz.\\ 
30 & Gawan brach iedoch den kranz.\\ 
\end{tabular}
\scriptsize
\line(1,0){75} \newline
G I L M Z Fr51 \newline
\line(1,0){75} \newline
\textbf{1} \textit{Initiale} L  \textbf{2} \textit{Initiale} Z  \textbf{5} \textit{Initiale} Fr51  \textbf{15} \textit{Initiale} I  \newline
\line(1,0){75} \newline
\textbf{1} nider] hin nider Z \textbf{2} loufens] loͮfes G  $\cdot$ dar] hin Z \textbf{3} swæren] Wan er swern Z Zwar Fr51  $\cdot$ er] her an Fr51 \textbf{6} erreichete] raichte I erreichet L rechte M la:gt Fr51  $\cdot$ mit dem] :::den Fr51 \textbf{7} dâ] Do Z  $\cdot$ der regen] des regens I  $\cdot$ unde] \textit{om.} I  $\cdot$ guz] der gvͯz L (M) (Z) (Fr51) \textbf{8} gebrochen heten] het gebrochen I Gebrochin hette M (Fr51) \textbf{9} halden] Malden L \textbf{10} dâr ûf her] dar vf er I Daz vf her L Daz vfer Z Da ober Fr51 \textbf{11} Gringulieten] chringulieten G Gringuͯlieten L kringulieten M gringuleten Fr51 \textbf{12} zoume] sper Z \textbf{13} sô] Wol so Fr51  $\cdot$ zuo im] vz I \textit{om.} Fr51 \textbf{14} ergreif] bigreiff M (Fr51) \textbf{15} zôch mîn] zo Fr51 \textbf{16} hin] \textit{om.} Fr51  $\cdot$ ûz ûf] vf an I vf vf Fr51 \textbf{17} dô] da M Z \textbf{19} gurt dem] gordez Fr51 \textbf{20} swen] Wem L Fr51 Wen M \textbf{21} het] leit L ha::: Fr51 \textbf{22} ez diu minne im] yme dy mynne esz M iz im de minne Fr51  $\cdot$ gebôt] bot Fr51 \textbf{23} Orgeluse] orguluse I Orgelýse L Orgiluse M \textbf{24} deme kranze] den k::: Fr51 \textbf{25} daz ein] wande daz in sin I Daz waz ein L (M) (Z)  $\cdot$ ellenthaftiu vart] ellenthafter ::: Fr51 \textbf{26} braht zuͤ dem baume der was bewart I  $\cdot$ Der bovm was also bewart L (M) (Z) (Fr51) \textbf{27} wæren] Waren L Fr51 Wer M  $\cdot$ Gawans] Gawansz L gawanes Fr51  $\cdot$ müesen] musten I \textbf{29} des] Das M  $\cdot$ Gramoflanz] Gramuflanz I [gamorsflanz]: gramorsflanz M Gramoflantz Z g::: Fr51 \textbf{30} iedoch] do Fr51 \newline
\end{minipage}
\hspace{0.5cm}
\begin{minipage}[t]{0.5\linewidth}
\small
\begin{center}*T
\end{center}
\begin{tabular}{rl}
 & \begin{large}D\end{large}az ors sô verre nider vlôz,\\ 
 & des loufens in dâ nâch verdrôz,\\ 
 & \textbf{swære harnasch er} truoc;\\ 
 & er hete \textbf{ouch wunden} genuoc.\\ 
5 & nû treib ez \textbf{ein gewerbe} her,\\ 
 & daz er ez erreichete mit dem sper,\\ 
 & d\textit{â} der regen und \textbf{der} guz\\ 
 & \textbf{gebrochen} \textbf{heten} wîten vluz\\ 
 & an einer tiefen halden.\\ 
10 & \textbf{d\textit{â} ez ûf her} was gespalden,\\ 
 & daz Krynguliete nerte.\\ 
 & mit dem \textbf{zoume} er ez kêrte\\ 
 & sô nâhe zuo \textbf{i\textit{m}} an daz lant,\\ 
 & \textbf{daz er ez ergreif} mit der hant.\\ 
15 & sus zôch mîn hêr Gawan\\ 
 & daz ors hin \textbf{ûz} ûf den plân.\\ 
 & ez schutte sich, dô ez genas,\\ 
 & der schilt d\textit{â} niht bestanden was.\\ 
 & er gurte dem orse und nam den schilt.\\ 
20 & wen sînes kumbers niht bevilt,\\ 
 & daz lâz ich sîn; er hete doch nôt,\\ 
 & sît ez diu minne im gebôt.\\ 
 & Orgeluse, diu glanze,\\ 
 & in jagete nâch dem kra\textit{n}ze;\\ 
25 & daz \textbf{was} ein ellenthaftiu vart.\\ 
 & \textbf{der boum was alsô} bewart,\\ 
 & wæren Gawans zwêne, die m\textit{ües}en ir leben\\ 
 & umb den kranz hân gegeben.\\ 
 & des pflac der künec Gramoflanz.\\ 
30 & Gawan brach iedoch den kranz.\\ 
\end{tabular}
\scriptsize
\line(1,0){75} \newline
U V W Q R \newline
\line(1,0){75} \newline
\textbf{1} \textit{Initiale} U V Q   $\cdot$ \textit{Capitulumzeichen} R  \newline
\line(1,0){75} \newline
\textbf{1} nider] [*]: hin nider V \textbf{2} dâ nâch] dannoch W \textbf{3} [S*]: Wande er sweren harnesch trvͦg V  $\cdot$ Schwereniharnasch er truͦg W  $\cdot$ Sweren harnasch er truck Q (R) \textbf{5} gewerbe] werbe W Q R \textbf{6} erreichete] [erreicht]:  erreichet Q errechtte R \textbf{7} dâ] Do U V W Q R  $\cdot$ und der guz] vnd des guß W nider gos R \textbf{8} heten] hette V hette ein W \textbf{10} dâ] Do U V W  $\cdot$ ûf her] her auff W  $\cdot$ gespalden] gepflanden R \textbf{11} Krynguliete] kringulierte U kringulete V kringulieten W Q kringulten R \textbf{12} dem zoume] [*]: dem sper V  $\cdot$ er ez] es R \textbf{13} im] in U \textbf{14} ez] \textit{om.} R  $\cdot$ ergreif] begreiff W Q \textbf{15} sus] Als Q \textbf{16} hin ûz ûf] hin [*]: vz vf V hin auff W (R) \textbf{17} dô] daz R \textbf{18} dâ] do U V W Q \textbf{20} wen] swen V Wann Q  $\cdot$ sînes kumbers] sein kumber W \textbf{22} sît] Sint das Q \textbf{23} Orgeluse] Orgelusen W Orguluse R \textbf{24} kranze] craze U \textbf{25} ellenthaftiu] erenthaffte Q ellenthafftte R \textbf{26} was] der waz V \textbf{27} Gawans] Gawins R  $\cdot$ die] sy W  $\cdot$ müesen] muͦzen U \textbf{28} hân gegeben] musten geben Q \textbf{29} des] [D*]: Dez V  $\cdot$ Gramoflanz] gramaflanz V gramoflantz W Q gramoflancz R \textbf{30} Gawan] Gawin R  $\cdot$ brach] bracht W \newline
\end{minipage}
\end{table}
\end{document}
