\documentclass[8pt,a4paper,notitlepage]{article}
\usepackage{fullpage}
\usepackage{ulem}
\usepackage{xltxtra}
\usepackage{datetime}
\renewcommand{\dateseparator}{.}
\dmyyyydate
\usepackage{fancyhdr}
\usepackage{ifthen}
\pagestyle{fancy}
\fancyhf{}
\renewcommand{\headrulewidth}{0pt}
\fancyfoot[L]{\ifthenelse{\value{page}=1}{\today, \currenttime{} Uhr}{}}
\begin{document}
\begin{table}[ht]
\begin{minipage}[t]{0.5\linewidth}
\small
\begin{center}*D
\end{center}
\begin{tabular}{rl}
\textbf{21} & \textbf{unt} iesch vil grôziu botenbrôt.\\ 
 & \textbf{er sprach}: "\textbf{vrouwe}, unser nôt\\ 
 & \textbf{ist} mit \textbf{vreuden} zergangen.\\ 
 & den wir hie haben enpfangen,\\ 
5 & daz ist ein rîter \textbf{sô} getân,\\ 
 & daz wir ze \textbf{vlêhen} immer hân\\ 
 & unsern goten, \textbf{die in uns brâhten},\\ 
 & daz \textbf{si des ie gedâhten}."\\ 
 & "Nû sage mir ûf die triwe dîn,\\ 
10 & wer der ritter muge sîn."\\ 
 & "vrouwe, \textbf{er} ist ein degen fier,\\ 
 & des bâruckes soldier,\\ 
 & ein Anschevin von hôher art.\\ 
 & \textbf{âvoy}, wie \textbf{wênic} wirt gespart\\ 
15 & sîn lîp, swâ man in læzet an!\\ 
 & wie reht er dar unt dan\\ 
 & entwîchet unt kêret!\\ 
 & die vîende er schaden lêret.\\ 
 & \begin{large}I\end{large}ch sach in strîten schône,\\ 
20 & dâ \textit{d}ie Babylone\\ 
 & Alexandrie lœsen solten\\ 
 & unde dô si dannen wolten\\ 
 & den bâruc trîben mit gewalt.\\ 
 & waz \textbf{ir} dâ nider wart gevalt\\ 
25 & an der schumpfentiure!\\ 
 & \textbf{dâ} begienc der gehiure\\ 
 & mit sîme lîbe sölhe tât,\\ 
 & si heten \textbf{vliehens} decheinen rât.\\ 
 & dar zuo hôrt ich in nennen,\\ 
30 & man \textbf{solte} wol \textbf{erkennen},\\ 
\end{tabular}
\scriptsize
\line(1,0){75} \newline
D Fr9 Fr14 \newline
\line(1,0){75} \newline
\textbf{9} \textit{Majuskel} D  \textbf{19} \textit{Initiale} D  \newline
\line(1,0){75} \newline
\textbf{1} grôziu] groz ein Fr9 \textbf{2} er sprach] Do sprach her Fr9 (Fr14) \textbf{3} zergangen] ergangen Fr9 \textbf{6} vlêhen] vliehene Fr9 \textbf{11} er] iz Fr9 \textbf{13} Anschevin] Anscivin D anzevin Fr9 \textbf{15} læzet] leizet Fr9 \textbf{20} die] bi D die von Fr9  $\cdot$ Babylone] babẏlone Fr9 \textbf{21} Alexandrie] Alexandrẏe Fr9 \textbf{28} si] Sie ne Fr9 \newline
\end{minipage}
\hspace{0.5cm}
\begin{minipage}[t]{0.5\linewidth}
\small
\begin{center}*m
\end{center}
\begin{tabular}{rl}
 & \textbf{und} h\textit{ie}sch vil grôz boten brôt.\\ 
 & \textbf{er sprach}: "\textbf{vrowe}, unser nôt\\ 
 & \textbf{ist} mit \textbf{ungenâden} zergangen.\\ 
 & den wir hie haben enpfangen,\\ 
5 & daz ist \textit{e}in ritter \textbf{wol}getâ\textit{n},\\ 
 & daz wir zuo \textbf{vlêhen} iemer hân\\ 
 & unseren goten, \textbf{die in uns brâhten},\\ 
 & daz \textbf{si \textit{der} ie gedâhten}."\\ 
 & "nû sage mir ûf die triuwe dîn,\\ 
10 & wer der ritter muge sîn."\\ 
 & "vr\textit{o}we, \textbf{ez} ist \textit{ein} degen fier,\\ 
 & des b\textit{âru}ckes soldier,\\ 
 & ein A\textit{n}schevin von hôher art.\\ 
 & \textbf{â}, wie \textbf{wêni\textit{c}} wirt gespart\\ 
15 & sîn lîp, wâ man in lâze\textit{t} \textit{a}n!\\ 
 & wie rehte er dar und dan\\ 
 & entwîchet und kêret!\\ 
 & die vîende er schaden lêret.\\ 
 & ich sach in strîten schône,\\ 
20 & dâ die Babilone\\ 
 & Alexandrie lœsen solten\\ 
 & und d\textit{ô} si dannen wolten\\ 
 & den bâruc trîben mit gewalt.\\ 
 & waz \textbf{in} dâ nider wart gevalt\\ 
25 & an der s\textit{chumpfentiur}e!\\ 
 & begienc der \textit{ge}hiure\\ 
 & mit sînem lîbe soliche t\textit{â}t,\\ 
 & si hetten \textbf{slîchens} keinen rât.\\ 
 & dar zuo hôrt ich \textbf{ouch} in nennen,\\ 
30 & man \textbf{solt} \textbf{in} wol \textbf{erkennen},\\ 
\end{tabular}
\scriptsize
\line(1,0){75} \newline
m n o \newline
\line(1,0){75} \newline
\newline
\line(1,0){75} \newline
\textbf{1} und hiesch] Vnd heisch m Er hiesz o  $\cdot$ grôz] grosses n grosze o \textbf{5} ein] min \textit{nachträglich korrigiert zu:} ein m  $\cdot$ wolgetân] [wolgetag]: wolgetang m so getan n o \textbf{6} vlêhen] fliehen o \textbf{8} der] \textit{om.} m \textbf{9} dîn] min n (o) \textbf{11} ein] \textit{om.} m \textbf{12} bâruckes] branckes \textit{nachträglich korrigiert zu:} barucks m barnckes o \textbf{13} Anschevin] ausceuin \textit{nachträglich korrigiert zu:} ansceuin m auscenin n anscenin o \textbf{14} â] Ey n (o)  $\cdot$ wênic] wennige m  $\cdot$ wirt] wurt das n \textbf{15} lâzet an] lasset beben an m losset leben an n o \textbf{16} er] er die o \textbf{20} dâ] Do n o  $\cdot$ die] die von n o \textbf{21} Alexandrie] Allexandrie m n o \textbf{22} dô] da m \textbf{24} dâ] do n o  $\cdot$ gevalt] gewalt o \textbf{25} schumpfentiure] stunt kamorie m scúncken múre n scúnckamuͯre o \textbf{26} gehiure] huͯre \textit{(abweichende Reklamante:} gehuͯre\textit{)} m \textbf{27} soliche] manige n  $\cdot$ tât] tort \textit{nachträglich korrigiert zu:} taͯt m \textbf{28} slîchens] sliches o \textbf{29} ouch] \textit{om.} n o \newline
\end{minipage}
\end{table}
\newpage
\begin{table}[ht]
\begin{minipage}[t]{0.5\linewidth}
\small
\begin{center}*G
\end{center}
\begin{tabular}{rl}
 & \textbf{er} iesch vil grôziu botenbrôt.\\ 
 & "\textbf{vrouwe}, \textbf{nû ist} unser nôt\\ 
 & mit \textbf{vröuden} zergangen.\\ 
 & den wir hie haben enpfangen,\\ 
5 & daz ist ein rîter \textbf{sô} getân,\\ 
 & \begin{large}D\end{large}az wir ze \textbf{dankene} imer hân\\ 
 & unseren goten, \textbf{dies gedâhten},\\ 
 & daz \textbf{sin uns her brâhten}."\\ 
 & "nû sage mir ûf die triwe dîn,\\ 
10 & wer der rîter muge sîn."\\ 
 & "vrouwe, \textbf{ez} ist ein degen fier,\\ 
 & des bâruckes soldier,\\ 
 & ein Antschevin von hôher art.\\ 
 & \textbf{âvoy}, wie \textbf{lützel} wirt gespart\\ 
15 & sîn lîp, swâ man in lâzet an!\\ 
 & wie rehter dar und dan\\ 
 & entwîchet und kêret!\\ 
 & die vînde er schaden lêret.\\ 
 & ich sach in strîten schône,\\ 
20 & \textbf{al} dâ die Babilone\\ 
 & Alexandrie lœsen solten\\ 
 & unt dô si dannen wolten\\ 
 & den bâruc trîben mit gewalt.\\ 
 & waz \textbf{ir} dâ nider wart gevalt\\ 
25 & an der schumpfentiure!\\ 
 & \textbf{dâ} begie der gehiure\\ 
 & mit sînem lîbe solhe tât,\\ 
 & si\textbf{ne} heten \textbf{vliehens} deheinen rât.\\ 
 & dar zuo hôrte ich in nennen,\\ 
30 & man \textbf{moht} \textbf{in} wol \textbf{bekennen},\\ 
\end{tabular}
\scriptsize
\line(1,0){75} \newline
G O L M Q R W Z Fr29 Fr32 Fr36 Fr55 Fr71 \newline
\line(1,0){75} \newline
\textbf{1} \textit{Initiale} O Fr29  \textbf{5} \textit{Versal} Fr32  \textbf{6} \textit{Initiale} G  \textbf{9} \textit{Initiale} M  \textbf{19} \textit{Initiale} W Fr71  \textbf{21} \textit{Versal} Fr32  \textbf{25} \textit{Initiale} L Q Z Fr32 Fr36  \textbf{29} \textit{Initiale} Fr55  \newline
\line(1,0){75} \newline
\textbf{1} er] ÷r O Vnd Z  $\cdot$ iesch] gehiez ir Fr32  $\cdot$ vil] da Z  $\cdot$ grôziu] grosz L M (Fr32) (Fr71) hoesz Q groses R  $\cdot$ botenbrôt] boten brot O (L) (M) (Q) (R) Z (Fr32) (Fr71) bote::: Fr29 \textbf{2} vrouwe] Er sprach fraw Q (R) (Z) (Fr32)  $\cdot$ nû ist] \textit{om.} Z  $\cdot$ unser] vnsz M R  $\cdot$ nôt] tod Q \textbf{3} mit] Jst mit Z  $\cdot$ zergangen] gar zergangen L (M) W \textbf{4} hie] \textit{om.} L \textbf{5} daz] Der Z  $\cdot$ sô] also M \textbf{6} ze dankene] wol zelobenne W  $\cdot$ imer] \textit{om.} Fr71 \textbf{7} unseren] Vnser W Z  $\cdot$ goten] goͤtte W  $\cdot$ dies gedâhten] die in vns brahten O (L) (Q) Z Fr29 Fr32 Fr71 dy an vns brachten M die vns Jn brachtent R die in har brachten W \textbf{8} sin uns her brâhten] si des ie gedahten O (L) (M) (Q) (R) (W) (Z) (Fr29) Fr32 Fr71 \textbf{10} muge] moͤge W  $\cdot$ sîn] gesien M (Q) \textbf{11} ez] er Z (Fr29)  $\cdot$ fier] frier R \textbf{12} des bâruckes] Des brauches Q Der da riches Z  $\cdot$ soldier] soldener M \textbf{13} Antschevin] anschevin O R Fr71 anshevin L Z Fr32 naschfyn M ansheuin Q antscheuin W an:::evin Fr29  $\cdot$ hôher] hohart L \textbf{14} âvoy] Awi O Owe L W Avoya M A R  $\cdot$ lützel] weynnig M \textbf{15} swâ man] wo man L (M) (Q) W Z wan R  $\cdot$ an] dan M \textbf{16} dar] her Fr32  $\cdot$ dan] dar M \textbf{18} die] Dem Q  $\cdot$ schaden] schade M schanden Fr71 \textbf{19} ich] Man L  $\cdot$ in] eynen M  $\cdot$ strîten schône] stritesschone M streyde schone Q \textbf{20} al] \textit{om.} O L M Q R Z Fr29 Fr32 Fr36 Fr55  $\cdot$ dâ] Do Q  $\cdot$ Babilone] babẏlone Fr32 babilon Fr36 babylone Fr71 \textbf{21} Alexandrie] Allexandria M Allexandrie Q R Alexandrîe Fr29 Alexandrîen Fr32 Alexandri Fr36 :::lexa::: Fr55  $\cdot$ solten] solte L \textbf{22} dô] da O M Z \textit{om.} L  $\cdot$ si] die Z \textbf{23} bâruc] Roub R \textbf{24} \textit{Vers 21.24 fehlt} R   $\cdot$ ir] \textit{om.} Fr32  $\cdot$ dâ] do Q W \textbf{25} an der] Inder Fr32 \textbf{26} dâ] \textit{om.} L W Fr71 Do Q R (Fr55)  $\cdot$ gehiure] vngehuͤre W \textbf{27} solhe] soͯlichen R  $\cdot$ tât] getat W \textbf{28} sine] Si O (Q)  $\cdot$ heten] hercz R het Fr71  $\cdot$ vliehens] vlihen L (Fr36) \textbf{29} \textit{Versfolge 21.30-29} Fr71   $\cdot$ dar zuo hôrte ich] Wan ich hort Fr71  $\cdot$ in] \textit{om.} O W in wol L \textbf{30} moht] solte L (W) (Fr29) moch Q  $\cdot$ bekennen] erchennen O (L) (M) (Q) (R) (W) (Z) (Fr32) (Fr36) (Fr71) \newline
\end{minipage}
\hspace{0.5cm}
\begin{minipage}[t]{0.5\linewidth}
\small
\begin{center}*T
\end{center}
\begin{tabular}{rl}
 & \textbf{er} iesch vil grôz boten brôt.\\ 
 & \textbf{er sprach}: "\textbf{nû ist} unser nôt\\ 
 & mit \textbf{vröuden} \textbf{gar} zergangen.\\ 
 & den wir hie hân enpfangen,\\ 
5 & daz ist ein rîter \textbf{sô} getân,\\ 
 & daz wir ze \textbf{dankene} iemer hân\\ 
 & unsern goten, \textbf{dien uns brâhten},\\ 
 & daz \textbf{si des ie gedâhten}."\\ 
 & "Nû sage mir ûf die triuwe dîn,\\ 
10 & wer der rîter muge sîn."\\ 
 & "vrouwe, \textbf{er} ist ein degen fier,\\ 
 & des bâruckes soldier,\\ 
 & ein Anschevin von hôher art.\\ 
 & \textbf{Âvoy}, wie \textbf{lützel} wirt gespart\\ 
15 & sîn lîp, swâ man in lâzet an!\\ 
 & wie rehter dar und dan\\ 
 & entwîchet und kêret!\\ 
 & die vîende er schaden lêret.\\ 
 & ich sach in strîten schône,\\ 
20 & dâ die Babylone\\ 
 & Alexandrien lœsen solten\\ 
 & und dô si dannen wolten\\ 
 & den bâruc trîben mit gewalt.\\ 
 & waz \textbf{ir} dâ nider wart gevalt\\ 
25 & an der schumpfentiure!\\ 
 & \textbf{dâ} begie der gehiure\\ 
 & mit sînem lîbe solhe tât,\\ 
 & si heten \textbf{vliehens} keinen rât.\\ 
 & dar zuo hôrt ich in nennen,\\ 
30 & man \textbf{moht} \textbf{in} wol \textbf{erkennen},\\ 
\end{tabular}
\scriptsize
\line(1,0){75} \newline
T U V \newline
\line(1,0){75} \newline
\textbf{9} \textit{Majuskel} T  \textbf{14} \textit{Majuskel} T  \newline
\line(1,0){75} \newline
\textbf{1} iesch] iach U \textbf{2} er sprach] vrowe V \textbf{7} dien] daz sie in U  $\cdot$ uns] her V \textbf{8} des] daz U (V) \textbf{11} er] iz U (V) \textbf{12} soldier] soldinier V \textbf{13} Anschevin] Anscevin T anscheuin V \textbf{15} swâ] wa U \textbf{20} Babylone] babilone U V \textbf{21} Alexandrien] Alexandern U \textbf{22} dô] da V \textbf{24} waz ir] Bit er U  $\cdot$ dâ] do V \textbf{25} \textit{Versfolge 21.26-25} T   $\cdot$ schumpfentiure] schunpfertúre V \textbf{26} dâ begie] Do beiach U \textbf{28} heten] hete U \textbf{30} moht] moͤht V \newline
\end{minipage}
\end{table}
\end{document}
