\documentclass[8pt,a4paper,notitlepage]{article}
\usepackage{fullpage}
\usepackage{ulem}
\usepackage{xltxtra}
\usepackage{datetime}
\renewcommand{\dateseparator}{.}
\dmyyyydate
\usepackage{fancyhdr}
\usepackage{ifthen}
\pagestyle{fancy}
\fancyhf{}
\renewcommand{\headrulewidth}{0pt}
\fancyfoot[L]{\ifthenelse{\value{page}=1}{\today, \currenttime{} Uhr}{}}
\begin{document}
\begin{table}[ht]
\begin{minipage}[t]{0.5\linewidth}
\small
\begin{center}*D
\end{center}
\begin{tabular}{rl}
\textbf{366} & \begin{large}V\end{large}on minnen noch zornes vil geschiht,\\ 
 & nûne wîzet ez Obien niht.\\ 
 & \textbf{nû} hœret, wie ir vater sprach,\\ 
 & dô er \textbf{den werden} Gawan sach\\ 
5 & unt ern in daz lant enpfienc,\\ 
 & wie erz mit rede dô ane vienc.\\ 
 & \textbf{Dô sprach er}: "hêrre, iwer kumen,\\ 
 & daz mac \textbf{mit} sælden uns gevrumen.\\ 
 & ich hân gevaren manege vart,\\ 
10 & sô süeze \textbf{in} mînen ougen wart\\ 
 & nie von angesihte.\\ 
 & \textbf{zuo} dirre ungeschihte\\ 
 & sol iwer künfteclîcher tac\\ 
 & \textbf{uns} trœsten, wander trœsten mac."\\ 
15 & \textbf{Er bat in tuon dâ} ritters tât.\\ 
 & "ob ir \textbf{harnasches} mangel hât,\\ 
 & \textbf{des} \textbf{lât} iuch \textbf{wol} bereiten gar.\\ 
 & welt ir, sît, hêrre, in mîner schar."\\ 
 & Dô sprach der werde Gawan:\\ 
20 & "\textbf{ich} wære des ein bereiter man.\\ 
 & ich hân harnasch unt starke lide,\\ 
 & wan daz \textbf{mîn strîten} stêt mit vride\\ 
 & unz an eine benante stunde.\\ 
 & \textbf{ir læget} \textbf{ob} oder unde,\\ 
25 & daz wolt ich \textbf{durch iuch} lîden.\\ 
 & nû muoz ichz durch daz mîden,\\ 
 & hêrre, unz ein mîn kampf ergêt,\\ 
 & dâ mîn triwe sô \textbf{hôhe} pfandes stêt.\\ 
 & durch aller werden liute gruoz\\ 
30 & ich \textbf{si} mit kampfe lœsen muoz.\\ 
\end{tabular}
\scriptsize
\line(1,0){75} \newline
D Fr3 Fr4 \newline
\line(1,0){75} \newline
\textbf{1} \textit{Initiale} D  \textbf{3} \textit{Initiale} Fr3 Fr4  \textbf{7} \textit{Majuskel} D  \textbf{15} \textit{Majuskel} D  \textbf{19} \textit{Initiale} Fr4   $\cdot$ \textit{Majuskel} D  \newline
\line(1,0){75} \newline
\textbf{1} zornes] \textit{om.} Fr3 \textbf{2} wîzet ez] :::dîz Fr3  $\cdot$ Obien] Obyen D obẏen Fr4 \textbf{4} Gawan sach] gawanin gesach Fr4 \textbf{6} dô] \textit{om.} Fr3 Fr4 \textbf{14} wander] wan er Fr4 \textbf{26} ichz] ich Fr4 \textbf{27} hêrre unz] vnze daz Fr4 \textbf{29} werden liute] lute werdin Fr4 \newline
\end{minipage}
\hspace{0.5cm}
\begin{minipage}[t]{0.5\linewidth}
\small
\begin{center}*m
\end{center}
\begin{tabular}{rl}
 & von minnen noch zornes vil geschiht,\\ 
 & nû enwîzet ez Obien niht.\\ 
 & \textbf{nû} hœret, wie ir vater sprach,\\ 
 & d\textit{ô} er \textbf{den werden} Gawanen sach\\ 
5 & und er \textit{in} in daz lant enpfienc,\\ 
 & wie erz mit rede dô an vienc.\\ 
 & \textbf{er sprach}: "hêrre, iuwer komen,\\ 
 & daz mac \textbf{an} sælden uns gevromen.\\ 
 & ich hân gevarn manige vart,\\ 
10 & sô süeze \textbf{in} mînen ouge\textit{n w}art\\ 
 & nie \textbf{man} von angesihte.\\ 
 & \textbf{ze} dirre ungeschihte\\ 
 & sol iuwer kün\textit{ft}iclîcher tac\\ 
 & \textbf{uns} trœsten, wander trœsten mac,\\ 
15 & \textbf{daz ir hie üebet} ritters tât.\\ 
 & ob ir \textbf{harnasch} mangel hât,\\ 
 & \textbf{des} \textbf{lât} iuch \textbf{wol} bereite\textit{n g}ar.\\ 
 & welt ir, sît, hêrre, in mîner schar."\\ 
 & dô sprach der werde Gawan:\\ 
20 & "\textbf{\begin{large}I\end{large}ch} wære des ein bereiter man.\\ 
 & ich hân harnasch und starke lide,\\ 
 & want daz \textbf{mîn strît} stât mit vride\\ 
 & unz an eine benanten stunden.\\ 
 & \textbf{ir læget} \textbf{obenân} oder unden,\\ 
25 & daz wolt ich \textbf{durch iuch} lîden.\\ 
 & nû muoz ichz durch daz mîden,\\ 
 & hêrre, unz ein mîn kampf ergât,\\ 
 & d\textit{â} mîn triuwe sô \textbf{hôhes} pfandes stât.\\ 
 & durch aller werde\textit{n} liute gruoz\\ 
30 & ich \textbf{si} mit kampfe lœsen muoz.\\ 
\end{tabular}
\scriptsize
\line(1,0){75} \newline
m n o \newline
\line(1,0){75} \newline
\textbf{19} \textit{Illustration mit Überschrift:} Also ein herre des landes gerte ritterschaft an gawan den helt (an den helt gawan o  ) n (o)   $\cdot$ \textit{Initiale} n o  \textbf{20} \textit{Initiale} m  \newline
\line(1,0){75} \newline
\textbf{1} geschiht] beschiͯcht o \textbf{2} enwîzet] wuste n wihst o  $\cdot$ Obien] obie n o \textbf{4} dô] Da m  $\cdot$ Gawanen] gawan n \textbf{5} in in] in m o \textbf{8} daz] \textit{om.} n o  $\cdot$ an sælden] [selten]: an selden m \textbf{10} in] \textit{om.} o  $\cdot$ ougen wart] ougen iach wart m \textbf{11} nie man] Nieman o \textbf{13} künfticlîcher] kuniglicher m funffteklichen o \textbf{14} uns] Vnd o  $\cdot$ wander] wenne er n (o)  $\cdot$ trœsten] strosten o  $\cdot$ mac] [man]: mag m \textbf{17} des] Das n o  $\cdot$ bereiten gar] bereiten das vnd dar m \textbf{18} sît] sú n (o) \textbf{23} benanten stunden] benante stunde n (o) \textbf{24} obenân] oben n \textbf{25} iuch] \textit{om.} o \textbf{26} muoz] músse n \textbf{27} kampf] kouff n (o) \textbf{28} dâ] Do m n o \textbf{29} werden] werde m o  $\cdot$ liute] lúten n  $\cdot$ gruoz] grosz o \newline
\end{minipage}
\end{table}
\newpage
\begin{table}[ht]
\begin{minipage}[t]{0.5\linewidth}
\small
\begin{center}*G
\end{center}
\begin{tabular}{rl}
 & von minne noch zornes vil geschiht,\\ 
 & nûne wîzet ez Obien niht.\\ 
 & \textbf{nû} hœrt \textit{\textbf{ouch}}, wie ir vater sprach,\\ 
 & dô er Gawanen sach\\ 
5 & under in in daz lant enpfienc,\\ 
 & wie erz mit rede dô ane vienc.\\ 
 & \textbf{dô sprach er}: "hêrre, iwer kumen,\\ 
 & daz mag \textbf{an} sælden uns gevrumen.\\ 
 & ich hân gevaren manige vart,\\ 
10 & sô süeze \textbf{in} mînen ougen wart\\ 
 & nie von angesihte.\\ 
 & \textbf{gein} dirre ungeschihte\\ 
 & sol iwer künfticlîcher tac\\ 
 & \textbf{uns} trœsten, wan er trœsten mac."\\ 
15 & \textbf{er bat in tuon dâ} rîters tât.\\ 
 & "obe ir \textbf{harnasches} ma\textit{n}gel hât,\\ 
 & \textbf{des} \textbf{heize ich} iuch bereiten gar.\\ 
 & welt ir, sît, hêrre, in mîner schar."\\ 
 & dô sprach der werde Gawan:\\ 
20 & "\textbf{ich} wære des ein bereit man.\\ 
 & ich hân harnasch unde starke lide,\\ 
 & wan \textit{daz} \textbf{mîn strîten} stêt mit vride\\ 
 & unze an eine benant stunde.\\ 
 & \textbf{ir læget} \textbf{obe} oder unde,\\ 
25 & daz wolt ich \textbf{mit iu} lîden.\\ 
 & nû muoz ich ez durch daz mîden,\\ 
 & hêrre, unze ein mîn kampf ergêt,\\ 
 & dâ mîn triwe sô \textbf{hôh\textit{e}} pfandes stêt.\\ 
 & durch aller werden liute gruoz\\ 
30 & \textit{ich} \textbf{\textit{d}i\textit{e}} mit kampfe lœsen muoz\\ 
\end{tabular}
\scriptsize
\line(1,0){75} \newline
G I O L M Q R Z Fr21 \newline
\line(1,0){75} \newline
\textbf{1} \textit{Initiale} I O L R Z Fr21  \textbf{15} \textit{Initiale} I  \newline
\line(1,0){75} \newline
\textbf{1} von] ÷on O  $\cdot$ minne] minem I manne O minnen Z  $\cdot$ noch zornes] zorn noch I zornes Z zorns noch Fr21 \textbf{2} nûne] nu I (O) (R)  $\cdot$ ez] \textit{om.} O  $\cdot$ Obien] Obyen O (R) (Z) Fr21 oblien Q \textbf{3} nû] vnde O (L) (M) (Q) (R) (Z) (Fr21)  $\cdot$ ouch] \textit{om.} G  $\cdot$ ir] der R \textbf{4} dô] Da M Z  $\cdot$ er] der O  $\cdot$ Gawanen] hern Gawan I den werden Gawan O L (M) (Q) R (Z) (Fr21)  $\cdot$ sach] an gesach I \textbf{5} under in] Vnde er M \textbf{6} Vnd es mit worten ane fie R  $\cdot$ wie erz] Wiez Fr21  $\cdot$ dô] \textit{om.} I da M Z  $\cdot$ ane vienc] angeviench O (Z) (Fr21) \textbf{7} dô] Da M \textbf{8} daz] [Da]: Daz \sout{sprach er} O \textit{om.} L  $\cdot$ uns] mir O Q R Fr21 nv L  $\cdot$ gevrumen] fromen R \textbf{9} hân] bin I \textbf{10} Das ich es nye liebers merrne vant Q  $\cdot$ Jn minen ogen so súsze nie ward R  $\cdot$ mînen] minem O Z mynē M \textbf{11} nie von] Nieman von L Hy man Q Von R \textbf{12} dirre] dir Q \textbf{13} iwer] vns úwer R  $\cdot$ künfticlîcher] chunphelicher I chvnstechlicher O (Q) kvmflicher L \textbf{14} er] er vns I der Z \textbf{15} tuon dâ] tut do Q da tún R \textbf{16} \textit{nach 366.16:} An harnasch wirt gút Rat R   $\cdot$ obe] Er sprach ob R  $\cdot$ ir] er L  $\cdot$ harnasches mangel] harnasches magel G mangel harnaisch I harnasch mangil M (Z) (Fr21) mangel R \textbf{17} heize] wil I lat O L M Q R Z Fr21  $\cdot$ ich iuch] ivch vns O (Z) vns uͯch L (Q) ouch vns M uch herre R ivch Fr21  $\cdot$ bereiten] beraten O \textbf{18} sît hêrre] herre sit ir I so sind R  $\cdot$ in] an L  $\cdot$ mîner] minen Z \textbf{19} dô] Da M  $\cdot$ werde] wede R  $\cdot$ Gawan] her Gewan R \textbf{20} bereit] bereite G \textbf{21} starke] starcheu I \textbf{22} daz] \textit{om.} G  $\cdot$ mîn] mit Z  $\cdot$ strîten] streit Q  $\cdot$ mit] Jn R \textbf{23} benant] benanten Z \textbf{24} ir læget] ir liget I (Q) Jch lig R  $\cdot$ oder] olde G \textbf{25} daz] [wo]: daz G  $\cdot$ mit] durc I \textbf{26} mîden] leyden Q \textbf{27} hêrre] \textit{om.} I  $\cdot$ ein] eyme M  $\cdot$ kampf] strit R \textbf{28} dâ] Do O Q  $\cdot$ triwe sô hôhe] triwe so hohes G [truͯweho]: truͯwe ho L  $\cdot$ pfandes] phandes vmbe I vmb R \textbf{29} werden] werde Q welte R werder Z  $\cdot$ liute] lutten R \textbf{30} ich die] den ih G Die ich O  $\cdot$ mit kampfe] \textit{om.} R \newline
\end{minipage}
\hspace{0.5cm}
\begin{minipage}[t]{0.5\linewidth}
\small
\begin{center}*T
\end{center}
\begin{tabular}{rl}
 & von minnen noch zornes vil geschiht,\\ 
 & nûne wîzet ez Obyen niht\\ 
 & \textbf{unde} hœret \textbf{ouch}, wie ir vater sprach,\\ 
 & dô er \textbf{den werden} Gawan sach\\ 
5 & unde ern in daz lant enpfienc,\\ 
 & wierz mit rede dô ane vienc.\\ 
 & \textbf{Dô sprach er}: "hêrre, iuwer komen,\\ 
 & daz mac \textbf{an} sælden uns gevromen.\\ 
 & ich hân gevarn manege vart,\\ 
10 & sô süeze mînen ougen wart\\ 
 & nie von angesihte.\\ 
 & \textbf{gegen} dirre ungeschihte\\ 
 & sol iuwer kün\textit{f}teclîcher tac\\ 
 & trœsten, wander trœsten mac."\\ 
15 & \textbf{er bat in tuon dâ} rîters tât.\\ 
 & "ob ir \textbf{harnasches} mangel hât,\\ 
 & \textbf{daz} \textbf{lât} iu\textit{ch} \textbf{uns} bereiten gar.\\ 
 & welt ir, sît, hêrre, in mîner schar."\\ 
 & Dô sprach der werde Gawan,\\ 
20 & \textbf{er} wære des ein bereit man.\\ 
 & "ich hân harnasch unde starkiu lide,\\ 
 & wan daz \textbf{mî\textit{n} strît} \textit{stât} mit vride\\ 
 & unz an eine benante stunde.\\ 
 & \textbf{ich læge} \textbf{ob} oder unde,\\ 
25 & daz wolt ich \textbf{mit iu} lîden.\\ 
 & nû muoz ichz durch daz mîden,\\ 
 & hêrre, unz \textbf{an daz} ein mîn kampf ergêt,\\ 
 & dâ mîn triuwe sô \textbf{hôhe} pfandes stêt\\ 
 & durch aller werder liute gruoz,\\ 
30 & \textbf{daz} ich \textbf{die} mit kampfe lœsen muoz\\ 
\end{tabular}
\scriptsize
\line(1,0){75} \newline
T V W \newline
\line(1,0){75} \newline
\textbf{1} \textit{Initiale} V  \textbf{7} \textit{Majuskel} T  \textbf{19} \textit{Majuskel} T  \newline
\line(1,0){75} \newline
\textbf{1} minnen noch zornes] minne zorn noch W \textbf{2} Nun wissent obye nicht W  $\cdot$ Obyen] obẏen V \textbf{3} ouch] \textit{om.} W \textbf{7} Dô sprach er] Er sprach W \textbf{8} uns] mich W \textbf{10} mînen] [*]: in minen V in meinen W \textbf{11} von] [*]: man von V \textbf{12} gegen] [*]: Zvͦ V \textbf{13} künfteclîcher] kvnsteclicher T \textbf{14} trœsten wander] Vns troͤsten wan er V (W) \textbf{15} [*]: Daz ir hie vͤbet ritters tat V  $\cdot$ in] im W  $\cdot$ dâ] \textit{om.} W \textbf{16} harnasches] harnesch V (W) \textbf{17} iuch uns bereiten] îu vns bereiten T [*]: v́ch wol bereiten V vns heraiten W \textbf{18} sît] sein W \textbf{20} er] [*]: Jch V \textbf{22} mîn strît] mit strit T mein streiten W  $\cdot$ stât] \textit{om.} T  $\cdot$ mit] in W \textbf{24} Was ich ioch guͦtes gunde W  $\cdot$ ich læge] [*]: Jr legent V \textbf{26} ichz] ich W \textbf{27} an daz ein mîn] [*]: ein min V \textbf{28} dâ] Do V W \textbf{30} daz ich die mit] Ich die mich W \newline
\end{minipage}
\end{table}
\end{document}
