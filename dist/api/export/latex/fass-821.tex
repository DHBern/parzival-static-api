\documentclass[8pt,a4paper,notitlepage]{article}
\usepackage{fullpage}
\usepackage{ulem}
\usepackage{xltxtra}
\usepackage{datetime}
\renewcommand{\dateseparator}{.}
\dmyyyydate
\usepackage{fancyhdr}
\usepackage{ifthen}
\pagestyle{fancy}
\fancyhf{}
\renewcommand{\headrulewidth}{0pt}
\fancyfoot[L]{\ifthenelse{\value{page}=1}{\today, \currenttime{} Uhr}{}}
\begin{document}
\begin{table}[ht]
\begin{minipage}[t]{0.5\linewidth}
\small
\begin{center}*D
\end{center}
\begin{tabular}{rl}
\textbf{821} & \begin{large}S\end{large}i \textbf{muosen} machen niwe slâ\\ 
 & ûz gein Karchobra.\\ 
 & dar enbot der süeze Anfortas\\ 
 & dem, der \textbf{dâ} burcgrâve was,\\ 
5 & daz er wære des gemant,\\ 
 & ob er ie von sîner hant\\ 
 & enpfienge gâbe rîche,\\ 
 & daz er nû dienstlîche\\ 
 & sîne triwe an im \textbf{geprîste}\\ 
10 & unt \textbf{im} sînen swâger wîste\\ 
 & "unt \textbf{des} wîp, die swester \textbf{mîn}",\\ 
 & \textbf{durchz} fôreht Læprisin\\ 
 & in die wilden habe wît.\\ 
 & nû was \textbf{ez} \textbf{ouch} urloubes zît.\\ 
15 & Si\textbf{ne} solten \textbf{dô} niht vürbaz komen.\\ 
 & Cundrie \textbf{la surziere} wart genomen\\ 
 & zuo dirre botschefte dan.\\ 
 & urloup zuo dem rîchen man\\ 
 & nâmen al die templeise.\\ 
20 & hin reit der kurteise.\\ 
 & Der burcgrâve dô niht liez,\\ 
 & swaz \textbf{in Cundrie} leisten hiez.\\ 
 & Feirefiz, der rîche,\\ 
 & wart dô \textbf{rîterlîche}\\ 
25 & mit grôzer \textbf{vuore} enpfangen.\\ 
 & in dorfte dâ niht erlangen.\\ 
 & man vuort in vürbaz schiere\\ 
 & mit \textbf{werdem} condwiere.\\ 
 & i\textbf{ne} weiz, wie \textbf{manec lant} er reit\\ 
30 & unz ze Joflanze ûf den anger breit.\\ 
\end{tabular}
\scriptsize
\line(1,0){75} \newline
D \newline
\line(1,0){75} \newline
\textbf{1} \textit{Initiale} D  \textbf{15} \textit{Majuskel} D  \textbf{21} \textit{Majuskel} D  \newline
\line(1,0){75} \newline
\textbf{2} Karchobra] Charchobrâ D \textbf{12} Læprisin] [Læ*sîn]: Læprisîn D \textbf{16} Cvndrîe lasvrziere wart genomn D \newline
\end{minipage}
\hspace{0.5cm}
\begin{minipage}[t]{0.5\linewidth}
\small
\begin{center}*m
\end{center}
\begin{tabular}{rl}
 & si \textbf{muosten} machen niuwe slâ\\ 
 & ûz gegen Carkobra.\\ 
 & dar enbot der süeze Anfortas\\ 
 & dem, der \textbf{d\textit{â}} burcgrâve was,\\ 
5 & d\textit{a}z er wær des ge\textit{m}ant,\\ 
 & ob er ie von sîner hant\\ 
 & enpfienge gâbe rîch,\\ 
 & daz er nû dienstlîch\\ 
 & sîn triuwe an im \textbf{geprîste}\\ 
10 & und sînen swâger wîste\\ 
 & und \textbf{sîn} wîp, die swester \textbf{sîn},\\ 
 & \textbf{durch daz} fôreht Loprisin\\ 
 & in die wilden habe wît.\\ 
 & nû was urloubes zît.\\ 
15 & si solte\textit{n} \textbf{dô} niht vürbaz komen.\\ 
 & Condrie \textbf{lasurzier} wart genomen\\ 
 & zuo diser botschafte dan.\\ 
 & urloup zuo dem rîchen man\\ 
 & nâmen alle die t\textit{e}mpleis.\\ 
20 & hin reit der kurteis.\\ 
 & \begin{large}D\end{large}er burcgrâve dô niht \textbf{en}liez,\\ 
 & waz \textbf{Condrie in} leisten hiez.\\ 
 & Ferefiz, der rîch,\\ 
 & wart dô \textbf{ritterlîch}\\ 
25 & mit grôzer \textbf{vuor} enpfangen.\\ 
 & in dorfte d\textit{â} niht erlangen.\\ 
 & man vuorte in vürbaz schier\\ 
 & mit \textbf{werdem} condwier.\\ 
 & ich weiz, wie \textbf{manic lant} er \textit{r}eit\\ 
30 & unz zuo Joflanze ûf den anger breit.\\ 
\end{tabular}
\scriptsize
\line(1,0){75} \newline
m n V V' W \newline
\line(1,0){75} \newline
\textbf{1} \textit{Initiale} W  \textbf{21} \textit{Initiale} m V   $\cdot$ \textit{Capitulumzeichen} n  \newline
\line(1,0){75} \newline
\textbf{1} \textit{Die Verse 821.1-17 fehlen} V'   $\cdot$ muosten] muͤszen V \textbf{2} Carkobra] karkubra m kukubra n korkobra W \textbf{4} dâ] do m n V W \textbf{5} daz] Des m  $\cdot$ gemant] genant m \textbf{9} geprîste] [*]: prisete V \textbf{12} Loprisin] laprisin V lobrosin W \textbf{13} wilden] wilde W  $\cdot$ habe] [hahe]: habe n \textbf{14} was] waz oͮch V \textbf{15} si solten] Suͯ soltte m (n) [S*solten]: Sv́ ensolten  V \textbf{16} Condrie lasurzier] Kv́ndrie lasurziere V Kundrie lasurzsier W \textbf{18} \textit{statt 821.18-20:} Vrloup sie do namen do / Hin riten sie yeso V'  \textbf{19} templeis] tampleis m n \textbf{21} \textit{Die Verse 821.21-28 fehlen} V'   $\cdot$ enliez] [enlei]: enliesz n \textbf{22} waz] Swaz V  $\cdot$ Condrie] kvndrie V (W) \textbf{23} Ferefiz] Ferefis m Ferrefis n Ferevis V Ferafis W \textbf{26} dâ] do m n V W \textbf{27} vürbaz] do fúrbas W \textbf{28} werdem] werden n froͤlicher W \textbf{29} Jch enweis nit wie manic lant ferevis do reit V'  $\cdot$ ich weiz] Jn enweiz V  $\cdot$ reit] treit m \textbf{30} Joflanze] joflantz m n jofflansze V joflantze V' tschoflantz W \newline
\end{minipage}
\end{table}
\newpage
\begin{table}[ht]
\begin{minipage}[t]{0.5\linewidth}
\small
\begin{center}*G
\end{center}
\begin{tabular}{rl}
 & \begin{large}S\end{large}i \textbf{begunden} machen niwe slâ\\ 
 & ûz gein Charchopra.\\ 
 & dar enbot der süeze Anfortas\\ 
 & dem, der burcgrâve was,\\ 
5 & daz er wære des gemant,\\ 
 & ob er ie von sîner hant\\ 
 & enpfienge gâbe rîche,\\ 
 & daz er nû dienstlîche\\ 
 & sîne triwe an im \textbf{brîste}\\ 
10 & unde \textbf{im} sînen swâger wîste\\ 
 & unde \textbf{sîn} wîp, die swester \textbf{sîn},\\ 
 & \textbf{zem} fôreist Lohprisin\\ 
 & in die wilden habe wît.\\ 
 & nû was \textbf{ouch} urloubes zît.\\ 
15 & si solden \textbf{doch} niht vürbaz komen.\\ 
 & Gundrie wart genomen\\ 
 & ze dirre botschefte dan.\\ 
 & urloup ze dem rîchen man\\ 
 & nâmen alle die templeise.\\ 
20 & hin reit der kurteise.\\ 
 & der burcgrâve dô niht \textbf{en}liez,\\ 
 & swaz \textbf{in Gundrie} leisten hiez.\\ 
 & Feirafiz, der rîche,\\ 
 & wart dô \textbf{minniclîche}\\ 
25 & mit grôzer \textbf{vröude} enpfangen.\\ 
 & in dorfte dâ niht erlangen.\\ 
 & man vuorte in vürbaz schiere\\ 
 & mit \textbf{maniger} condwiere.\\ 
 & ich\textbf{ne} weiz, wie \textbf{manic lant} er \textbf{dô} reit\\ 
30 & unze ze Tschofflanze ûf den anger breit.\\ 
\end{tabular}
\scriptsize
\line(1,0){75} \newline
G I L Z \newline
\line(1,0){75} \newline
\textbf{1} \textit{Initiale} G I L Z  \textbf{15} \textit{Initiale} I  \newline
\line(1,0){75} \newline
\textbf{2} Charchopra] chrachobra I charcobra L charthopra Z \textbf{3} dar] daz I  $\cdot$ enbot] einbot L  $\cdot$ Anfortas] Amfortas L \textbf{4} der] der da L Z \textbf{5} daz] Der L  $\cdot$ des] daz L \textbf{7} enpfienge] enphienc I Einpfinge L \textbf{10} im] \textit{om.} L \textbf{11} sîn wîp] des wip L Z \textbf{12} zem] zuͤ den I  $\cdot$ fôreist] rechten L  $\cdot$ Lohprisin] leh prisin I (L) (Z) \textbf{14} ouch urloubes] vrlaubes auch I \textbf{15} si] Sine I  $\cdot$ doch] ouch L \textbf{16} Gundrie] kvndiz G Kvndrie L Z \textbf{21} dô] da L Z  $\cdot$ enliez] liez L \textbf{22} swaz] Waz L  $\cdot$ Gundrie] kvndrie G L (Z) \textbf{23} Feirafiz] Ferefis L Feirefiz Z  $\cdot$ rîche] vil riche I \textbf{24} dô] do Gar I da Z \textbf{25} vröude] vuge I fuͯre L (Z) \textbf{26} dâ] \textit{om.} I  $\cdot$ erlangen] belangen I Z \textbf{27} vuorte] fuͯrt L \textbf{28} maniger] groszer L grozzem Z \textbf{29} dô] durc I \textit{om.} L da Z \textbf{30} ze Tschofflanze] ze tschoflanz G zeshoffanze I Tschoflanze L zv Tschofflantz Z  $\cdot$ den] dem L \newline
\end{minipage}
\hspace{0.5cm}
\begin{minipage}[t]{0.5\linewidth}
\small
\begin{center}*T
\end{center}
\begin{tabular}{rl}
 & si \textbf{begunden} machen niuwe slâ\\ 
 & ûz gein Kachopra.\\ 
 & dar enbot der süeze Anfortas\\ 
 & dem, der \textbf{d\textit{â}} burcgrâve was,\\ 
5 & daz er wære des gemant,\\ 
 & ob er ie von sîner hant\\ 
 & enpfienge gâbe rîche,\\ 
 & daz er nû dienstlîche\\ 
 & sîne triuwe an im \textbf{nû} \textbf{prîste}\\ 
10 & und \textbf{im} sînen swâger wîste\\ 
 & und \textbf{des} wîp, die swester \textbf{sîn},\\ 
 & \textit{\textbf{zem} fôrest Lehprisin}\\ 
 & in die wilden habe wît.\\ 
 & nû was \textbf{ouch} urloubes zît.\\ 
15 & si \textbf{en}solten \textbf{doch} niht vürbaz komen.\\ 
 & Kundrie wart genomen\\ 
 & zuo dirre botschefte dan.\\ 
 & urloup zuo dem rîchen man\\ 
 & nâmen alle die templeise.\\ 
20 & hin reit der kurteise.\\ 
 & \begin{large}D\end{large}er burcgrâve dô niht \textbf{en}liez,\\ 
 & waz \textbf{in Kundrie} leisten hiez.\\ 
 & Ferefis, der rîche,\\ 
 & wart dô \textbf{minneclîche}\\ 
25 & mit grôzer \textbf{vuore} enpfangen.\\ 
 & in durfte dâ niht erlangen.\\ 
 & man vuort in vürbaz schiere\\ 
 & mit \textbf{grôzeme} cundewiere.\\ 
 & ich \textbf{en}weiz, wie \textbf{manege mîle} er reit\\ 
30 & unz zuo Tschoflanze ûf den anger breit.\\ 
\end{tabular}
\scriptsize
\line(1,0){75} \newline
U Q R \newline
\line(1,0){75} \newline
\textbf{1} \textit{Initiale} Q  \textbf{21} \textit{Initiale} U  \newline
\line(1,0){75} \newline
\textbf{2} Kachopra] karkopra Q karthoga R \textbf{4} dâ] do U die Q \textbf{5} wære des] des were Q \textbf{7} enpfienge gâbe] Empfangen gaben R \textbf{9} nû] \textit{om.} Q R \textbf{12} \textit{Vers 821.12 fehlt} U   $\cdot$ Lehprisin] lechprisin R \textbf{13} wilden] wilde Q \textbf{14} ouch] \textit{om.} R \textbf{15} ensolten] soltent R \textbf{16} Kundrie] Kuͦndrie U Kundire R  $\cdot$ wart] wart wart R \textbf{19} templeise] tempelise R \textbf{21} \textit{Die Verse 821.21-826.30 fehlen} Q   $\cdot$ enliez] lies R \textbf{22} Kundrie] kuͦndrie U \textbf{23} Ferefis] Feirefis R \textbf{26} dâ] do R  $\cdot$ erlangen] belangen R \textbf{29} manege mîle] menig land R  $\cdot$ er reit] er do reit R \textbf{30} unz] Mit U  $\cdot$ Tschoflanze] schoflancze R \newline
\end{minipage}
\end{table}
\end{document}
