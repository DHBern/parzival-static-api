\documentclass[8pt,a4paper,notitlepage]{article}
\usepackage{fullpage}
\usepackage{ulem}
\usepackage{xltxtra}
\usepackage{datetime}
\renewcommand{\dateseparator}{.}
\dmyyyydate
\usepackage{fancyhdr}
\usepackage{ifthen}
\pagestyle{fancy}
\fancyhf{}
\renewcommand{\headrulewidth}{0pt}
\fancyfoot[L]{\ifthenelse{\value{page}=1}{\today, \currenttime{} Uhr}{}}
\begin{document}
\begin{table}[ht]
\begin{minipage}[t]{0.5\linewidth}
\small
\begin{center}*D
\end{center}
\begin{tabular}{rl}
\textbf{256} & \begin{large}D\end{large}az er vrâgens \textit{w}as sô laz,\\ 
 & \textbf{daz} er bî dem trûrigem wirte saz,\\ 
 & Daz \textbf{rou} dô grœzlîche\\ 
 & den \textbf{helt} ellens rîche.\\ 
5 & durch klage unt \textbf{durch} den tac sô heiz\\ 
 & Begunde netzen in der sweiz.\\ 
 & durch den luft \textbf{von im er} bant\\ 
 & den helm unt vuorten in der hant.\\ 
 & er entstricte die vintâlen sîn.\\ 
10 & durch îsers râm was lieht sîn schîn.\\ 
 & er kom ûf eine niwe slâ,\\ 
 & wandez gienc vor im al dâ\\ 
 & ein ors, daz was wol beslagen,\\ 
 & unt ein barvuoz pfert muose tragen\\ 
15 & eine vrouwen, die er sach.\\ 
 & nâch der ze rîten im geschach.\\ 
 & \textbf{Ir} pfert \textbf{gein} kumber was verselt:\\ 
 & man het im \textbf{wol} durch hût gezelt\\ 
 & elliu sîniu rippe gar.\\ 
20 & als ein harm ez was gevar.\\ 
 & \textbf{ein bästîn halfter lac dâr} an.\\ 
 & unz ûf den huof swang im diu man,\\ 
 & \textbf{sîn} ougen tief, die gruoben wît.\\ 
 & ouch \textbf{was} \textbf{der vrouwen} runzît\\ 
25 & vertwâlet unt vertrecket,\\ 
 & durch \textbf{hunger} dicke erwecket.\\ 
 & ez was dürre als ein zunder.\\ 
 & sîn gên, daz was wunder,\\ 
 & wand ez reit ein vrouwe wert,\\ 
30 & diu selten \textbf{kunrierte} pfert.\\ 
\end{tabular}
\scriptsize
\line(1,0){75} \newline
D \newline
\line(1,0){75} \newline
\textbf{1} \textit{Großinitiale} D  \textbf{3} \textit{Majuskel} D  \textbf{6} \textit{Majuskel} D  \textbf{17} \textit{Majuskel} D  \newline
\line(1,0){75} \newline
\textbf{1} was] vas D \newline
\end{minipage}
\hspace{0.5cm}
\begin{minipage}[t]{0.5\linewidth}
\small
\begin{center}*m
\end{center}
\begin{tabular}{rl}
 & \begin{large}D\end{large}az er vrâgens was sô laz,\\ 
 & \textbf{daz} er bî dem t\textit{r}ûrigen wirte saz,\\ 
 & daz \textbf{gerouwe} dô grœzeclîche\\ 
 & den \textbf{degen} ellens rîche.\\ 
5 & durch klage und den tac sô h\textit{ei}z\\ 
 & begunde netzen in der sweiz.\\ 
 & durch den luft \textbf{an ime er} bant\\ 
 & den helm und vuorte in in der hant.\\ 
 & er entstri\textit{c}te die fantailen sîn.\\ 
10 & durch îsers râm was lieht sîn schîn.\\ 
 & er kam ûf eine niuwe slâ,\\ 
 & wand ez gien\textit{c} vor ime aldâ\\ 
 & ein ros, daz was wol beslagen,\\ 
 & und ein barvuoz pfert, \textbf{daz} muose tragen\\ 
15 & eine vrouwen, die er sach.\\ 
 & \textit{nâch der ze rîten ime geschach}.\\ 
 & \textbf{ir} pfert \textbf{gegen} kumber was verselt:\\ 
 & man hete ime durch \textbf{die} hût gezelt\\ 
 & \textbf{\textit{wol}} \textit{alliu sîniu rippe gar}.\\ 
20 & \textit{alsô ein harm ez was gevar}.\\ 
 & \textbf{ein bes\textit{t}în ha\textit{l}fter lac dâr} ane.\\ 
 & unz ûf den huof swanc ime diu mane,\\ 
 & \textbf{sîn} ougen tief, die gruoben wît.\\ 
 & ouch \textbf{was} \textbf{daz selbe} runzît\\ 
25 & vertwâlet und vertrecket,\\ 
 & durch \textbf{hunger} dicke erwecket.\\ 
 & ez was dürre als ein zunder.\\ 
 & sîn gên, daz was \textbf{ein} wunder,\\ 
 & want ez reit ein vrouwe wert,\\ 
30 & diu selten \textbf{turnierte} pfert.\\ 
\end{tabular}
\scriptsize
\line(1,0){75} \newline
m n o Fr69 \newline
\line(1,0){75} \newline
\textbf{1} \textit{Überschrift:} Wie parcifal mit stritte Jescutten die hulde erwarp m  Also parcifal mit strite froͧwe jescuten die hulde erwarp n   $\cdot$ \textit{Großinitiale} n   $\cdot$ \textit{Initiale} m o  \newline
\line(1,0){75} \newline
\textbf{2} Daz] Do n o Fr69  $\cdot$ trûrigen] turigen m  $\cdot$ saz] was n o \textbf{3} gerouwe] er gerouwe n \textbf{4} ellens rîche] ellentrich n o \textbf{5} heiz] hies m \textbf{8} und] er n o  $\cdot$ vuorte in] fuͯrt n \textbf{9} entstricte] enstritte m (o)  $\cdot$ fantailen] fantolien o \textbf{12} gienc] ginge m \textbf{14} muose] muͯsse m (n) (o) \textbf{15} vrouwen] frouwe n (o) \textbf{16} \textit{Vers 256.16 fehlt} m  \textbf{17} verselt] erselt n versolt o \textbf{18} ime] in n  $\cdot$ die] sin o \textbf{19} \textit{Die Verse 256.19-20 fehlen} m  \textbf{21} bestîn halfter] beschinhaffter m besten halffter n \textbf{22} den] die n  $\cdot$ swanc] twang n  $\cdot$ ime diu] in die o im der Fr69 \textbf{23} gruoben] gruͯp eyn o \textbf{24} runzît] ein zit n rinzit o \textbf{27} als] alsam n \textbf{28} gên] ganck o \textbf{30} turnierte] kundewierte n kunde wirtin o \newline
\end{minipage}
\end{table}
\newpage
\begin{table}[ht]
\begin{minipage}[t]{0.5\linewidth}
\small
\begin{center}*G
\end{center}
\begin{tabular}{rl}
 & daz er vrâgens was sô laz,\\ 
 & \textbf{dô}r bî dem trûregen wirte saz,\\ 
 & daz \textbf{rou} dô grœzlîche\\ 
 & den \textbf{helt} ellens rîche.\\ 
5 & durch klage unde \textbf{durch} den tac sô heiz\\ 
 & begunde netzen in der sweiz.\\ 
 & durch den luft \textbf{er von im} bant\\ 
 & den helm unde vuortin in der hant.\\ 
 & er entstricte die vinteilen sîn.\\ 
10 & durch îsers râm was lieht sîn schîn.\\ 
 & er kom ûf eine niuwe slâ,\\ 
 & wan ez \textit{gie} vor im al dâ\\ 
 & ein ors, daz was wol beslagen,\\ 
 & unde ein barvuoz pfert, \textbf{daz} muose tragen\\ 
15 & eine vrouwen, die er sach.\\ 
 & nâch der ze rîtene im geschach.\\ 
 & \textbf{\begin{large}D\end{large}az} pfert \textbf{gein} kumber was verselt:\\ 
 & man het im durch \textbf{die} hût gezelt\\ 
 & elliu sîniu rippe gar.\\ 
20 & als ein harm ez was gevar.\\ 
 & \textbf{ein bästîn halfter lac dâr} ane.\\ 
 & unze ûf den huof swanc im diu mane,\\ 
 & \textbf{sîn} ougen tief, die gruoben wît.\\ 
 & ouch \textbf{was} \textbf{der vrouwen} runzît\\ 
25 & vertwâlet unde vertrecket,\\ 
 & durch \textbf{hunger} dicke erwecket.\\ 
 & ez was dürre als ein zunder.\\ 
 & sîn gên, daz was wunder,\\ 
 & wan ez reit ein vrouwe wert,\\ 
30 & diu selten \textbf{kunrierte} pfert.\\ 
\end{tabular}
\scriptsize
\line(1,0){75} \newline
G I O L M Q R Z Fr21 Fr60 \newline
\line(1,0){75} \newline
\textbf{1} \textit{Überschrift:} beyde ritterlichen hye streit partzifal vnd der hertzoge orilus et sic Q  Hie qvam parcifal zv frowen iescvten Z   $\cdot$ \textit{Initiale} O L Q R Z Fr21   $\cdot$ \textit{Capitulumzeichen} R  \textbf{7} \textit{Initiale} I  \textbf{13} \textit{Überschrift:} Aventiwer wie Parzifal frowen Sigvͦnen in der chlosen lie I   $\cdot$ \textit{Initiale} I  \textbf{17} \textit{Initiale} G  \textbf{27} \textit{Initiale} I  \newline
\line(1,0){75} \newline
\textbf{1} daz] ÷Az O \textbf{2} dôr] Da er M Z  $\cdot$ trûregen] \textit{om.} I trvͦrigem O \textbf{3} dô] da Z \textbf{4} ellens rîche] ellentrichen R \textbf{5} durch] \textit{om.} O \textbf{7} den] die M der Q  $\cdot$ er von im bant] von im er bant O (M) er abe do bant L von im erbant Z (Fr21) \textbf{8} vuortin] vuͤrt in I  $\cdot$ in] an L Z \textbf{9} er entstricte] Ern striht O Er [enstichte]: enstrichte L  $\cdot$ die vinteilen] die sinteiln Q den fintelle R \textbf{10} îsers] isen I (O) ysens L M  $\cdot$ lieht] [liht]: lieht O hýht L licht M Q R (Fr21) \textbf{11} eine niuwe] eyne nuwen M einen newen Q (R) \textbf{12} ez] dize I  $\cdot$ gie] \textit{om.} G R  $\cdot$ al] \textit{om.} I \textbf{14} ein] \textit{om.} O  $\cdot$ barvuoz] \textit{om.} I  $\cdot$ pfert] ros R \textbf{16} der] dem O (M) \textbf{17} Daz] Jr Q R  $\cdot$ gein kumber was] nach chvmber was O was von hunger R  $\cdot$ verselt] geselt I O Q Fr21 \textbf{18} durch] wol dvrch O (M) (Q) (R) (Z) (Fr21)  $\cdot$ die] \textit{om.} O Q R Fr21  $\cdot$ hût] huͯt wol L \textbf{20} \textit{Vers 256.20 fehlt} R   $\cdot$ harm] harin L \textbf{22} huof] fvͦz O  $\cdot$ swanc im] im swanc I swang in L  $\cdot$ diu] der Q \textbf{23} tief] trvͦbe I wit M  $\cdot$ die] vnd I sin Z  $\cdot$ gruoben] ogen R [trv]: grvben Z  $\cdot$ wît] tiff M \textbf{24} vrouwen] vrwon L  $\cdot$ runzît] ir runzit I \textbf{25} vertwâlet] ir twelt I Verwalt R  $\cdot$ vertrecket] ver decket Q (R) verterket Z \textbf{26} durch] vor I  $\cdot$ dicke] dich I  $\cdot$ erwecket] erwerket Z \textbf{27} ez] Er O  $\cdot$ dürre] durre als durre I \textbf{28} wunder] ein wunder O (Fr21) \textbf{30} kunrierte pfert] rait kekunriertiu phert I kurriren gert Q \newline
\end{minipage}
\hspace{0.5cm}
\begin{minipage}[t]{0.5\linewidth}
\small
\begin{center}*T
\end{center}
\begin{tabular}{rl}
 & \begin{large}D\end{large}az er vrâgens was sô laz,\\ 
 & \textbf{dô} er bî dem trûrigen wirte saz,\\ 
 & daz \textbf{rou} dô grôzlîche\\ 
 & den \textbf{helt} ellens rîche.\\ 
5 & Durch klage unde \textbf{durch} den tac sô heiz\\ 
 & begunde netzen in der sweiz.\\ 
 & durch den luft \textbf{er von im} bant\\ 
 & den helm unde vuortin in der hant.\\ 
 & er entstricte die vinteilen sîn.\\ 
10 & durch îsers râm was lieht sîn schîn.\\ 
 & er kom ûf eine niuwe slâ,\\ 
 & wan ez gienc vor im aldâ\\ 
 & ein ors, daz was wol beslagen,\\ 
 & unde ein barvuoz pfert, \textbf{daz} muose tragen\\ 
15 & eine vrouwen, dier sach.\\ 
 & nâch der ze rîtene im geschach.\\ 
 & \textbf{daz} pfert \textbf{von} kumber was verselt:\\ 
 & man het im durch \textbf{di\textit{e}} hût gezelt\\ 
 & \textbf{wol} alliu sîniu rippe gar.\\ 
20 & als ei\textit{n} harm ez was gevar.\\ 
 & \textbf{dâ lac ein bestîn britel} an.\\ 
 & unz ûf den huof swanc im diu man,\\ 
 & \textbf{diu} ougen tief, die gruoben wît.\\ 
 & ouch \textbf{der vrouwen} runzît\\ 
25 & - vertwâlet unde vertrecket,\\ 
 & durch \textbf{kumber} dicke erwecket -,\\ 
 & ez was dürre als ein zunder.\\ 
 & sîn gân, daz was wunder,\\ 
 & wan ez reit ein vrouwe wert,\\ 
30 & di\textit{u} selten \textbf{kunrierte} pfert.\\ 
\end{tabular}
\scriptsize
\line(1,0){75} \newline
T U V W Fr26 \newline
\line(1,0){75} \newline
\textbf{1} \textit{Überschrift:} Hie streit her partzifal mit orilius vnd erwarb frawe iestuten sein hulde W   $\cdot$ \textit{Großinitiale} T  · Initiale U W Fr26  \textbf{5} \textit{Majuskel} T  \textbf{11} \textit{Überschrift:} Hie kam parzifal zvͦ orilus vnd Zvͦ sinem wibe jescuten V   $\cdot$ \textit{Initiale} V  \newline
\line(1,0){75} \newline
\textbf{2} wirte] kuͦnege U (V) \textbf{3} rou] [*]: geroͮ V raw in W \textbf{4} ellens] al zu U \textbf{5} durch den] den W \textbf{6} der] dem U \textbf{8} in] [*]: in V \textbf{9} die] von W  $\cdot$ vinteilen] fintellen T (W) fintelle U vantellen V \textbf{10} durch îsers] Duͦrch ysens U Der eisen W  $\cdot$ lieht] lecht W  $\cdot$ sîn] \textit{om.} U \textbf{12} vor] al vor W  $\cdot$ aldâ] da W \textbf{14} barvuoz] baruͦstic U  $\cdot$ muose] mvese T \textbf{15} vrouwen] vreuͦwe U (V) \textbf{16} ze rîtene] zartin U \textbf{17} daz] [*]: Jr V  $\cdot$ von] mit U [*]: gegen V \textbf{18} durch] wol durch W  $\cdot$ die] div T \textit{om.} W \textbf{19} Seine rippe allegar W  $\cdot$ alliu sîniu] alle sine T \textbf{20} ein] eim T \textit{om.} W  $\cdot$ harm] arm W  $\cdot$ ez was] waz ez V (W) \textbf{21} bestîn] beste U lang bestin V  $\cdot$ britel] halffter W \textbf{22} unz] Mit U  $\cdot$ den] die W  $\cdot$ huof] [v*]: vuͦz U fvͦz V  $\cdot$ swanc im] im swanc U (V) in swanc W  $\cdot$ diu] der V \textbf{23} diu] [*]: Sin V \textbf{24} ouch] Auch was U (V) W  $\cdot$ der vrouwen] [*]: daz selbe V \textbf{25} vertrecket] verstrecket V verdecket W \textbf{26} kumber] [*]: hvnger V \textbf{28} wunder] ein wunder V \textbf{30} diu] die T  $\cdot$ kunrierte] gunduirte W \newline
\end{minipage}
\end{table}
\end{document}
