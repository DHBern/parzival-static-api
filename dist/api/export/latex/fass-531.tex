\documentclass[8pt,a4paper,notitlepage]{article}
\usepackage{fullpage}
\usepackage{ulem}
\usepackage{xltxtra}
\usepackage{datetime}
\renewcommand{\dateseparator}{.}
\dmyyyydate
\usepackage{fancyhdr}
\usepackage{ifthen}
\pagestyle{fancy}
\fancyhf{}
\renewcommand{\headrulewidth}{0pt}
\fancyfoot[L]{\ifthenelse{\value{page}=1}{\today, \currenttime{} Uhr}{}}
\begin{document}
\begin{table}[ht]
\begin{minipage}[t]{0.5\linewidth}
\small
\begin{center}*D
\end{center}
\begin{tabular}{rl}
\textbf{531} & \begin{large}D\end{large}em pferde was der rücke \textbf{krump} \textbf{unt} \textbf{junc}.\\ 
 & wære drûf ergangen \textbf{dâ} sîn sprunc,\\ 
 & im wære \textbf{der} rücke \textbf{gar} zervarn.\\ 
 & daz muoser allez \textbf{dô} bewarn.\\ 
5 & \textbf{es} het in etswenne bevilt:\\ 
 & \textbf{er zôch} ez und truoc den schilt\\ 
 & unt \textbf{eine} glevîn.\\ 
 & sîner scharpfen pîn\\ 
 & diu vrouwe sêre lachte,\\ 
10 & diu im vil kumbers machte.\\ 
 & sînen schilt er ûfez pfert bant.\\ 
 & Si sprach: "vüert ir krâmgewant\\ 
 & in mîme lande veile?\\ 
 & wer gap mir ze teile\\ 
15 & einen arzet unt eines \textbf{krâmes} pflege?\\ 
 & hüetet \textbf{iuch vor zolle} ûfem wege.\\ 
 & etslîch mîn zolnære\\ 
 & iuch sol machen vreuden lære."\\ 
 & Ir scharpfiu salliure\\ 
20 & \textbf{in dûhte} sô gehiure,\\ 
 & daz ern ruochte, waz si sprach,\\ 
 & \textbf{wan} immer, swenner an si sach,\\ 
 & sô was sîn pfant \textbf{ze} riwe quît.\\ 
 & si was im \textbf{reht ein} meien zît,\\ 
25 & vor allem blicke ein flôrî,\\ 
 & ougen süeze unt sûwer dem herzen bî.\\ 
 & sît vlust unt \textbf{vinden} an ir was\\ 
 & unt des siechiu vreude wol genas,\\ 
 & daz vrumt in zallen stunden\\ 
30 & ledec unt sêre gebunden.\\ 
\end{tabular}
\scriptsize
\line(1,0){75} \newline
D Fr7 Fr11 Fr31 \newline
\line(1,0){75} \newline
\textbf{1} \textit{Initiale} D Fr7  \textbf{11} \textit{Initiale} Fr31  \textbf{12} \textit{Majuskel} D Fr31  \textbf{19} \textit{Majuskel} D  \newline
\line(1,0){75} \newline
\textbf{1} Och waz daz phæridelin so kranch Fr31  $\cdot$ unt junc] \textit{om.} Fr7 Fr11 \textbf{2} Daz er drvf niht sprach Fr31  $\cdot$ ergangen] gegangen Fr11  $\cdot$ dâ] \textit{om.} Fr7 Fr11 \textbf{4} muoser] mvͦse Fr31  $\cdot$ allez] alles Fr11  $\cdot$ dô] da Fr7 Fr11 \textbf{5} es] Ez Fr31  $\cdot$ etswenne] enswenne Fr7 \textbf{6} er zôch ez] ez zoch ::: Fr7 Do zoch ers Fr31 \textbf{7} eine] sine Fr31  $\cdot$ glevîn] gleivine Fr7 glaeveine Fr11 \textbf{8} scharpfen] scharpheit Fr31 \textbf{10} vil kumbers] kummers vil Fr7 kvmbers [vl]: vil Fr31 \textbf{11} bant] gebant Fr7 \textbf{12} Si sprach vüert] :::n Fr11 \textbf{13} mîme] dem Fr31 \textbf{15} eines krâmes] einen cram Fr7 einen chromes Fr11  $\cdot$ pflege] pflegen Fr7 \textbf{16} vor zolle] ver zelle Fr31  $\cdot$ ûfem wege] vf den wegen Fr7 \textbf{17} zolnære] Zollere Fr7 \textbf{18} vreuden] fraeẅde Fr11 \textbf{19} scharpfiu] scha::fe Fr7 scharphe Fr31  $\cdot$ salliure] sazluͯre Fr11 schallivre Fr31 \textbf{20} in dûhte] in dauͯch Fr11 Dvhte in Fr31 \textbf{21} ern ruochte] er di ruͦhte Fr7 er e ruͯcht Fr11 \textbf{22} wan immer] Von iamer Fr31  $\cdot$ sach] gesach Fr7 \textbf{23} sô] da Fr11  $\cdot$ ze] gein Fr11  $\cdot$ riwe] riͮwen Fr31 \textbf{25} Vor allen bliken ain floryn Fr31 \textbf{26} sûwer] svre Fr7  $\cdot$ dem herzen bî] in dem herzelin Fr31 \newline
\end{minipage}
\hspace{0.5cm}
\begin{minipage}[t]{0.5\linewidth}
\small
\begin{center}*m
\end{center}
\begin{tabular}{rl}
 & dem pferde was der rücke \textbf{junc}.\\ 
 & wær dar ûf ergangen sîn sprunc,\\ 
 & im wær \textbf{der} rück \textbf{gar} zervarn.\\ 
 & daz muost er allez \textbf{dô} bewarn.\\ 
5 & \textbf{des} het in etwen bevilt:\\ 
 & \textbf{er zôch} ez und truoc den schilt\\ 
 & und \textbf{ouch} \textbf{die} \textit{g}levîn.\\ 
 & \textit{s}îner scharpfen pîn\\ 
 & diu vrouwe sêre lachete,\\ 
10 & diu im vil kumbers machete.\\ 
 & sînen schilt er ûf daz pfert bant.\\ 
 & si sprach: "\textit{vü}eret ir k\textit{r}â\textit{m}gewant\\ 
 & in mî\textit{n}em lande veile?\\ 
 & wer gap mir zuo teile\\ 
15 & einen arzet und eines \textit{\textbf{krâmes}} pflege?\\ 
 & hüete\textit{t} \textbf{vor zol \textit{iuch}} ûf dem wege,\\ 
 & \textbf{wan} etlîch mîn zollære\\ 
 & iuch sol machen vröuden \textit{l}ære."\\ 
 & ir scharpfiu s\textit{a}lliure\\ 
20 & \textbf{in dûhte} sô gehiure,\\ 
 & daz er enruochte, waz si sprach;\\ 
 & \textbf{und} iemer, wan er an si sach,\\ 
 & sô was sîn pfant \textbf{an} riuw\textit{e} \textit{q}uît.\\ 
 & si was im \textbf{ein rehte} meien zît,\\ 
25 & vor allem blic ein flôrî,\\ 
 & ougen süeze und \textit{s}ûr dem herzen bî.\\ 
 & sît vlust und \textbf{vinde} an ir was\\ 
 & und des siechiu vröude wol genas,\\ 
 & daz vr\textit{o}mte in zuo allen stunden\\ 
30 & ledic und sêre gebunden.\\ 
\end{tabular}
\scriptsize
\line(1,0){75} \newline
m n o \newline
\line(1,0){75} \newline
\newline
\line(1,0){75} \newline
\textbf{5} des] Do n \textbf{6} zôch ez und truoc] truͦg vnd zoch o \textbf{7} glevîn] gelevin m \textbf{8} sîner] Einer m \textbf{10} machete] [macht]: machette m \textbf{11} pfert] [schilt]: pfert o \textbf{12} vüeret] vneret m n o  $\cdot$ krâmgewant] caingewant m \textbf{13} mînem] mimem m \textbf{15} krâmes] \textit{om.} m \textbf{16} hüetet] Huͯtte m  $\cdot$ iuch] \textit{om.} m \textbf{18} lære] mere m \textbf{19} salliure] silluͯre m (n) (o) \textbf{20} in] Jch n  $\cdot$ dûhte] dúchte o \textbf{21} er] es o \textbf{23} riuwe quît] [*]: ruͯwe kint vnd quit m truwen quit n \textbf{24} si] An \sout{rechten} o \textbf{26} ougen] Auͯge o  $\cdot$ sûr] fuͯr m \textbf{28} siechiu] sich o \textbf{29} vromte] fremtte m freuiten o \newline
\end{minipage}
\end{table}
\newpage
\begin{table}[ht]
\begin{minipage}[t]{0.5\linewidth}
\small
\begin{center}*G
\end{center}
\begin{tabular}{rl}
 & \begin{large}D\end{large}em pferde was der rücke \textbf{krump}.\\ 
 & wære drûf ergangen \textbf{dâ} sîn sprunc,\\ 
 & im wære \textbf{der} rücke \textbf{gar} zervaren.\\ 
 & daz muos er allez \textbf{dâ} bewaren.\\ 
5 & \textbf{es} hete in etwenne bevilt:\\ 
 & \textbf{er zôch} ez unde truoc den schilt\\ 
 & unde \textbf{ein} glavîn.\\ 
 & sîner scharfen pîn\\ 
 & diu vrouwe sêre lachte,\\ 
10 & diu im vil kumbers machte.\\ 
 & sînen schilt er ûf daz pfert bant.\\ 
 & si sprach: "vüeret ir krâmgewant\\ 
 & in mînem lande veile?\\ 
 & wer gab mir ze teile\\ 
15 & einen arzet unde eines \textbf{krâmers} pflege?\\ 
 & hüet \textbf{iuch vor zolle} ûf dem wege.\\ 
 & eteslîch mîn zolnære\\ 
 & iuch sol machen vröuden lære."\\ 
 & ir scharpfiu salliure\\ 
20 & \textbf{in dûhte} sô gehiure,\\ 
 & daz ern ruohte, waz \textit{si} sprach,\\ 
 & \textbf{wan} immer, swenne er an si sach,\\ 
 & sô was sîn pfant \textbf{ze} riuwe quît.\\ 
 & si was im \textbf{rehte ei\textit{n}} meien zît,\\ 
25 & vor allem blicke ein flôrî,\\ 
 & ougen süeze unde sûr dem herzen bî.\\ 
 & sît vlust unde \textbf{vinden} an ir was\\ 
 & unde des siechiu vröude wol genas,\\ 
 & daz vrumet in zallen stunden\\ 
30 & l\textit{e}dic unde sêre gebunden.\\ 
\end{tabular}
\scriptsize
\line(1,0){75} \newline
G I L M Z \newline
\line(1,0){75} \newline
\textbf{1} \textit{Überschrift:} Hie hat her gawan sin orss verlorn vnd ritet fvrbaz mit siner frowen vf einem bosen pferdelin Z   $\cdot$ \textit{Initiale} G I L Z  \textbf{19} \textit{Initiale} I  \newline
\line(1,0){75} \newline
\textbf{1} Dem] Eyme M \textbf{2} dâ] \textit{om.} L M \textbf{4} dâ] do L \textbf{5} es] ez I  $\cdot$ in] \textit{om.} L \textbf{8} scharfen] sharpher I starchen L hohen Z \textbf{10} diu] Das M \textbf{12} ir] \textit{om.} I \textbf{13} mînem] dem I \textbf{15} eines krâmers] einen cram L eynes kramis M (Z) \textbf{16} ûf] vnde uff M \textbf{17} eteslîch] etliche I \textbf{18} iuch sol machen] suln evch machen I Sol uͯch machen L Vch macht M \textbf{19} scharpfiu] sharphez I \textbf{20} sô] also M \textbf{21} ern] er L  $\cdot$ waz] swaz I  $\cdot$ si] \textit{om.} G \textbf{22} swenne] so I wenne L \textbf{23} ze riuwe] zuͦ der triwen I geyn ruwen M \textbf{24} im] \textit{om.} I  $\cdot$ ein] eine G sin I \textbf{25} allem blicke] allen blichen I (M) \textbf{26} unde] \textit{om.} I \textbf{27} sît] Sin L  $\cdot$ vinden] gewinnen I gewin Z \textbf{28} des] der Z  $\cdot$ siechiu vröude] iuncfroͮwe \textit{nachträglich korrigiert zu:} siechiv froͮde G siechen vreude I \textbf{29} in] im I (M) \textbf{30} ledic] Leidich G \newline
\end{minipage}
\hspace{0.5cm}
\begin{minipage}[t]{0.5\linewidth}
\small
\begin{center}*T
\end{center}
\begin{tabular}{rl}
 & Dem pferde was der rücke \textbf{krump}.\\ 
 & wære drûffe ergangen \textbf{dâ} sîn sprunc,\\ 
 & im wære \textbf{sîn} rücke zervarn.\\ 
 & daz muoser allez bewarn.\\ 
5 & \textbf{ez} hete in etswenne bevilt.\\ 
 & \textbf{dô zôch er}z unde truoc den schilt\\ 
 & unde \textbf{sîne} glevîn.\\ 
 & sîner scharpfen pîn\\ 
 & diu vrouwe sêre lachete,\\ 
10 & diu im vil kumbers machete.\\ 
 & sînen schilt er ûf daz pfert bant.\\ 
 & \begin{large}S\end{large}i sprach: "vüert ir krâmgewant\\ 
 & in mînem lande veile?\\ 
 & wer gap mir ze teile\\ 
15 & einen arzât unde eines \textbf{krâmers} pflege?\\ 
 & hüet \textbf{iuch vor zolle} ûf dem wege.\\ 
 & etslîch mîn zollære\\ 
 & iuch sol machen vröuden lære."\\ 
 & ir scharpf\textit{iu} salliure\\ 
20 & \textbf{dûhtin} sô gehiure,\\ 
 & daz ern ruochte, waz si sprach,\\ 
 & \textbf{wande} iemer, swenner an si sach,\\ 
 & sô was sîn pfant \textbf{gein} riuwen quît.\\ 
 & si was im \textbf{rehte eines} meien zît,\\ 
25 & vor alle\textit{m} blicke ein flôrî,\\ 
 & ougen süeze und sûre dem herzen bî.\\ 
 & Sît vlust unde \textbf{vinden} an ir was\\ 
 & unde des siech\textit{iu} vröude wol genas,\\ 
 & daz vrumt in zallen stunden\\ 
30 & ledic unde sêre gebunden.\\ 
\end{tabular}
\scriptsize
\line(1,0){75} \newline
T U V W O Q R Fr40 \newline
\line(1,0){75} \newline
\textbf{1} \textit{Majuskel} T  \textbf{7} \textit{Initiale} O  \textbf{12} \textit{Majuskel} T  \textbf{19} \textit{Initiale} W  \textbf{27} \textit{Majuskel} T  \newline
\line(1,0){75} \newline
\textbf{1} Do was daz pfærdelin so chranch O \textbf{2} Daz er drvf niht en spranch O  $\cdot$ dâ] do U V W so R \textbf{3} sîn] [si*]: sin \sout{herze} U der W O R  $\cdot$ zervarn] gar zeruaren W (O) (Q) (R) \textbf{4} Daz muͦz allez er do bewarn U  $\cdot$ muoser] mveser T mvͤst er V muͦsse R  $\cdot$ bewarn] do bewarn V (W) O (Q) R \textbf{5} ez] Es O \textbf{6} dô zôch erz] Er zoch es R \textbf{7} unde] ÷nde O  $\cdot$ glevîn] gline U glewe Q geleuine R \textbf{8} sîner] Zvͦ siner V Sine R \textbf{11} sînen] Den R  $\cdot$ daz] \textit{om.} W \textbf{12} vüert] vneret U \textbf{15} einen] Eins R  $\cdot$ eines] ein V  $\cdot$ krâmers] crames U V [chrams]: chrames O \textbf{16} iuch] iv T  $\cdot$ zolle] mvͦte O \textbf{18} iuch sol machen] iv sol machen T Eúch súllen machen W Jivch machet O Sol úch machen R \textbf{19} scharpfiu] scarpfen T \textbf{20} dûhtin] Jn duchte U (V) (O) (Q) R In dauchte doch W  $\cdot$ gehiure] vngehúrre R \textbf{21} si] er W \textbf{22} iemer] [*]: vmer U  $\cdot$ swenner] wan er U (W) (Q) so er R  $\cdot$ an si] an siche Q sy ane R \textbf{23} pfant] pfat Q  $\cdot$ riuwen] treúwen W (O) \textbf{24} eines] ein W O Q R \textbf{25} allem] allen T  $\cdot$ blicke] bliken R  $\cdot$ flôrî] forl Q \textbf{26} sûre] fᵫre R \textbf{27} Sît] Sein Q  $\cdot$ ir] [in]: ir O \textbf{28} des] der W  $\cdot$ siechiu] sieche T [siechie]: [siechio]: siechiv O siechtum R  $\cdot$ vröude] froͤden W \textbf{29} daz] Do W  $\cdot$ in] im W \newline
\end{minipage}
\end{table}
\end{document}
