\documentclass[8pt,a4paper,notitlepage]{article}
\usepackage{fullpage}
\usepackage{ulem}
\usepackage{xltxtra}
\usepackage{datetime}
\renewcommand{\dateseparator}{.}
\dmyyyydate
\usepackage{fancyhdr}
\usepackage{ifthen}
\pagestyle{fancy}
\fancyhf{}
\renewcommand{\headrulewidth}{0pt}
\fancyfoot[L]{\ifthenelse{\value{page}=1}{\today, \currenttime{} Uhr}{}}
\begin{document}
\begin{table}[ht]
\begin{minipage}[t]{0.5\linewidth}
\small
\begin{center}*D
\end{center}
\begin{tabular}{rl}
\textbf{567} & \begin{large}I\end{large}mmer, als dicke er trat,\\ 
 & daz bette vuor von \textbf{sîner} stat,\\ 
 & \textbf{daz} ê was gestanden.\\ 
 & Gawane wart enblanden,\\ 
5 & daz er den swæren schilt \textbf{getruoc},\\ 
 & den im sîn wirt bevalch genuoc.\\ 
 & Er \textbf{dâhte}: "wie kum ich ze dir,\\ 
 & wiltû wenken sus \textbf{vor} mir?\\ 
 & ich sol dich innen bringen,\\ 
10 & ob ich dich mege erspringen."\\ 
 & dô \textbf{gestuont} im daz bette vor.\\ 
 & er huop sich zem sprunge enbor\\ 
 & unt spranc \textbf{rehte} mitten dran.\\ 
 & die snelheit \textbf{vreischet} \textbf{nie mêr} man,\\ 
15 & wie daz bette her unt dar sich stiez.\\ 
 & der vier wende decheine ez liez:\\ 
 & mit hurte an ieslîchez swanc,\\ 
 & daz \textbf{al} diu burc dâ von erklanc.\\ 
 & Sus reit er manegen poynder grôz.\\ 
20 & swaz der doner ie gedôz\\ 
 & unt al \textbf{die} busûnære,\\ 
 & ob der êrste wære\\ 
 & bî dem \textbf{jungesten} dinne\\ 
 & und bliesen nâch gewinne,\\ 
25 & ez \textbf{en}dorfte niht mêr dâ krachen.\\ 
 & Gawan muose wachen,\\ 
 & swie er an dem bette læge.\\ 
 & wes der helt dô pflæge?\\ 
 & des galmes het in \textbf{sô} bevilt,\\ 
30 & daz er zucte über sich den schilt.\\ 
\end{tabular}
\scriptsize
\line(1,0){75} \newline
D \newline
\line(1,0){75} \newline
\textbf{1} \textit{Initiale} D  \textbf{7} \textit{Majuskel} D  \textbf{19} \textit{Majuskel} D  \newline
\line(1,0){75} \newline
\newline
\end{minipage}
\hspace{0.5cm}
\begin{minipage}[t]{0.5\linewidth}
\small
\begin{center}*m
\end{center}
\begin{tabular}{rl}
 & iemer, als dicke er trat,\\ 
 & daz \textit{b}ette vuor von \textbf{sîner} stat,\\ 
 & \textbf{daz} ê was gestanden.\\ 
 & Gawan wart enblanden,\\ 
5 & daz er den swæren schilt \textbf{getruoc},\\ 
 & den im sîn wirt bevalch genuoc.\\ 
 & er \textbf{dâhte}: "wie kom ich zuo dir,\\ 
 & wiltû wenken sus \textbf{vor} mir?\\ 
 & ich sol dich innen bringen,\\ 
10 & ob ich dich müge erspringen."\\ 
 & dô \textbf{gestuont} im daz bette vor.\\ 
 & er huop sich zuom sprung\textit{e} enbor\\ 
 & und spranc enmitten dar an.\\ 
 & die snelheit \textbf{vrei\textit{s}c\textit{he}t} \textbf{\textit{d}ekein} man,\\ 
15 & wie daz bette her und dâ sich stiez.\\ 
 & der vier wende dekeinez liez:\\ 
 & mit hurte an ieglîch ez swanc,\\ 
 & daz \textbf{al} diu burc dâ von erklanc.\\ 
 & sus reit er manigen \dag kumber\dag  grôz.\\ 
20 & waz der toner ie \textit{g}e\textit{d}ôz\\ 
 & und alle busûnære,\\ 
 & ob der êrste wære\\ 
 & bî dem \textbf{jungsten} dâr in\\ 
 & und bliesen nâch gewin,\\ 
25 & ez durft niht mê dâ krachen.\\ 
 & Gawan muost\textit{e} \textit{w}achen,\\ 
 & wie er an dem bette læge.\\ 
 & wes der helt dô pflæge?\\ 
 & des galmes het in \textbf{sô} bevilt,\\ 
30 & daz er zuhte übe\textit{r} \textit{s}ich den schilt.\\ 
\end{tabular}
\scriptsize
\line(1,0){75} \newline
m n o \newline
\line(1,0){75} \newline
\newline
\line(1,0){75} \newline
\textbf{1} iemer] Jomer o \textbf{2} bette] hett m  $\cdot$ vuor] fuͯre m n suͯre o \textbf{5} swæren] swer o \textbf{6} bevalch] befalck o \textbf{7} dâhte] gedochte n \textbf{8} vor] von n \textbf{11} vor] f:r o \textbf{12} sprunge] sprungen m \textbf{13} spranc] spang n \textbf{14} die] De o  $\cdot$ vreischet] freichsceit m frischet o  $\cdot$ dekein] die kein m do kein n \textbf{16} dekeinez] do keines n \textbf{17} ieglîch] igliches o \textbf{20} toner] torne m (n) o  $\cdot$ gedôz] degos m \textbf{21} busûnære] besuͯndere o \textbf{25} dâ] do n o \textbf{26} muoste wachen] muͯste lachen vnd wachen m \textbf{28} der] de: o \textbf{30} über sich] uͯber sin sich m vber mich o \newline
\end{minipage}
\end{table}
\newpage
\begin{table}[ht]
\begin{minipage}[t]{0.5\linewidth}
\small
\begin{center}*G
\end{center}
\begin{tabular}{rl}
 & \begin{large}I\end{large}mmer, als dicke er trat,\\ 
 & daz bette vuor von \textbf{sîner} stat,\\ 
 & \textbf{daz} ê was gestanden.\\ 
 & Gawane wart enblanden,\\ 
5 & daz er den swæren schilt \textbf{getruoc},\\ 
 & den im sîn wirt bevalch genuoc.\\ 
 & er \textbf{dâhte}: "wie kume ich ze dir,\\ 
 & wil dû wenken sus \textbf{vor} mir?\\ 
 & ich sol dich innen bringen,\\ 
10 & ob ich dich muge erspringen."\\ 
 & dô \textbf{gestuont} im daz bette vor.\\ 
 & er huop sich zem sprunge enbor\\ 
 & unde spranc \textbf{rehte} enmitten dran.\\ 
 & die snelheit \textbf{gevreischet} \textbf{niemer} man,\\ 
15 & wie daz bette her unde dar sich stiez.\\ 
 & der vier wende deheinez liez:\\ 
 & mit hurte an ieteslîche ez swanc,\\ 
 & daz \textbf{al} diu burc dâ von erklanc.\\ 
 & sus reit er manigen poynder grôz.\\ 
20 & swaz der doner ie g\textit{ed}ôz\\ 
 & unde al \textbf{die} busûnære,\\ 
 & op der êrste wære\\ 
 & bî dem \textbf{jungeste\textit{n}} dinne\\ 
 & unde bliesen nâch gewinne,\\ 
25 & ez \textbf{en}dorfte niht mê dâ krachen.\\ 
 & Gawan muose wachen,\\ 
 & swie er an dem bette læge.\\ 
 & wes der helt dô pflæge?\\ 
 & des galmes hete in \textbf{sô} bevilt,\\ 
30 & daz er zucte über sich den schilt.\\ 
\end{tabular}
\scriptsize
\line(1,0){75} \newline
G I L M Z \newline
\line(1,0){75} \newline
\textbf{1} \textit{Initiale} G I Z  \textbf{19} \textit{Initiale} I  \newline
\line(1,0){75} \newline
\textbf{1} Immer] Ie mer Z  $\cdot$ er] als er Z \textbf{3} daz] da ez I (L) (M)  $\cdot$ ê] >e< G \textbf{4} Gawane] Gawan I L Z  $\cdot$ wart] was I \textbf{5} getruoc] truͤc I (L) \textbf{7} dâhte] gedachte L  $\cdot$ kume] choͤm I (Z) \textbf{8} wenken sus] sust wenchen I  $\cdot$ vor mir] mir I von mir L (M) \textbf{11} \textit{Versfolge 567.12-11} I   $\cdot$ dô] Da M Z  $\cdot$ gestuont] stuͤnt I \textbf{13} enmitten] denmitten I mitten M \textbf{14} gevreischet] gehorte I freischet L (Z)  $\cdot$ niemer] nie dehain I \textbf{15} sich] \textit{om.} M \textbf{17} hurte] \textit{om.} Z  $\cdot$ an ieteslîche ez] angestliche sich I \textbf{20} swaz] Waz L  $\cdot$ gedôz] groz G \textbf{23} jungesten] ivngiste G \textbf{24} nâch] y noch M \textbf{25} ez endorfte] Ezuͯ dorfte L  $\cdot$ niht mê dâ] da mer I da mere nit L nicht mer M (Z) \textbf{26} Gawan] Gawain I  $\cdot$ muose] muͤste I  $\cdot$ wachen] wachchen I \textbf{27} swie er] Wer L Wie her M \textbf{28} der helt] Gawan I  $\cdot$ dô] Nu M da Z \textbf{30} sich] in I \newline
\end{minipage}
\hspace{0.5cm}
\begin{minipage}[t]{0.5\linewidth}
\small
\begin{center}*T
\end{center}
\begin{tabular}{rl}
 & iemer, alse dicke er trat,\\ 
 & daz bette vuor von \textbf{der} stat,\\ 
 & \textbf{d\textit{â} ez} ê was gestanden.\\ 
 & Gawan wart enblanden,\\ 
5 & daz er den swæren schilt \textbf{truoc},\\ 
 & den im sîn wirt b\textit{e}va\textit{l}ch genuoc.\\ 
 & Er \textbf{gedâhte}: "wie kum ich zuo dir,\\ 
 & wiltû wenken sus \textbf{von} mir?\\ 
 & ich sol dich innen bringen,\\ 
10 & ob ich dich muge erspringen."\\ 
 & dô \textbf{gesaz} im daz bette vor.\\ 
 & er huop sich zem sprunge enbor\\ 
 & unde spranc enmitten dran.\\ 
 & die snelheit \textbf{vriesch} \textbf{nie kein} man,\\ 
15 & wie daz bette her unde dar sich stiez.\\ 
 & der vier \textit{w}ende deheine ez liez:\\ 
 & mit hurte an iegeslîchez swanc,\\ 
 & daz diu burc dâ von erklanc.\\ 
 & Sus reit er manegen poynder grôz.\\ 
20 & swaz der doner ie gedôz\\ 
 & unde alle \textbf{die} busûnære,\\ 
 & ob der êrste wære\\ 
 & bî dem \textbf{andern} drinne\\ 
 & unde bliesen nâch gewi\textit{nne},\\ 
25 & ez \textbf{en}dorfte nimer dâ krachen.\\ 
 & Gawan muose wachen,\\ 
 & swie er an dem bette læge.\\ 
 & wes der helt dô pflæge?\\ 
 & des galmes het in \textbf{dô} bevilt,\\ 
30 & daz er zucte über sich den schilt.\\ 
\end{tabular}
\scriptsize
\line(1,0){75} \newline
T U V W Q R Fr25 Fr39 \newline
\line(1,0){75} \newline
\textbf{1} \textit{Initiale} Q Fr25 Fr39  \textbf{7} \textit{Majuskel} T  \textbf{19} \textit{Majuskel} T  \newline
\line(1,0){75} \newline
\textbf{1} \textit{Die Verse 553.1-599.30 fehlen} U   $\cdot$ dicke] offt R \textbf{2} der] [*]: siner V seiner W Q (R) (Fr25) (Fr39) \textbf{3} dâ] daz T Do V W \textit{om.} Q  $\cdot$ ez ê was] ez waz e V E es do was Q ez was Fr25 \textbf{4} Gawan] Gawane W Q Fr39 Gawin R \textbf{6} den] Dem R Fr25  $\cdot$ im sîn] in Fr25  $\cdot$ bevalch] b:vach T \textbf{8} wenken] wancken W  $\cdot$ sus] als Q vil Fr25  $\cdot$ von] vor Q Fr25 Fr39 \textbf{11} gesaz] [*]: gestvͦnt V stuͦnd W gestunde Q (R) (Fr25) (Fr39) \textbf{12} sich] sich auff W  $\cdot$ sprunge] sprongen Fr25 \textbf{14} vriesch] erfriesch V gefrisch Q getorst R \textbf{15} Wie sich das bett har vnd hin schos Vnd Sties R  $\cdot$ dar sich stiez] [*]: dar sich stiez V \textbf{16} wende] ende T  $\cdot$ ez] er Q \textbf{17} hurte] hurtten R  $\cdot$ iegeslîchez] [iegelich*]: iegeliche ez V  $\cdot$ swanc] spang W spranc Fr25 \textbf{18} diu] aldv́ V (W) (Q) (R) (Fr25) (Fr39) \textbf{19} Sus] Als Q  $\cdot$ reit] [reit]: leit V leit R  $\cdot$ poynder] Poyndier T [*]: pvnder V kumer R (Fr39) \textbf{20} swaz] Was W Q R \textbf{21} die] \textit{om.} W  $\cdot$ busûnære] besunder besunere R \textbf{22} êrste] ersten Q \textbf{23} bî dem] Bi den V [D]: bey dem Q [Sy]: By dem R  $\cdot$ andern] iungesten V (W) (Q) (R) \textbf{24} gewinne] gewi::: T \textbf{25} dâ] do V W R  $\cdot$ krachen] crachten R \textbf{26} Gawan] Gawin R  $\cdot$ muose] mvese T [*]: mvͤste V  $\cdot$ wachen] [lachen]: wachen Q \textbf{27} swie] Wie W Q R  $\cdot$ er] er do Q \textbf{28} pflæge] legens pflege Q \textbf{29} in] Im R  $\cdot$ dô] so V W R Fr39 \newline
\end{minipage}
\end{table}
\end{document}
