\documentclass[8pt,a4paper,notitlepage]{article}
\usepackage{fullpage}
\usepackage{ulem}
\usepackage{xltxtra}
\usepackage{datetime}
\renewcommand{\dateseparator}{.}
\dmyyyydate
\usepackage{fancyhdr}
\usepackage{ifthen}
\pagestyle{fancy}
\fancyhf{}
\renewcommand{\headrulewidth}{0pt}
\fancyfoot[L]{\ifthenelse{\value{page}=1}{\today, \currenttime{} Uhr}{}}
\begin{document}
\begin{table}[ht]
\begin{minipage}[t]{0.5\linewidth}
\small
\begin{center}*D
\end{center}
\begin{tabular}{rl}
\textbf{634} & \begin{large}D\end{large}ô sprach si: "hêrre, ich sihe nû wol,\\ 
 & ob ich sô vor iu sprechen sol,\\ 
 & daz ir von im rîtet,\\ 
 & nâch dem mîn herze strîtet.\\ 
5 & ob ir der zuht \textbf{ir reht nû} tuot,\\ 
 & hêrre, diu lêrt iuch helenden muot.\\ 
 & Disiu gâbe ist mir \textbf{ouch} \textbf{ê} gesant\\ 
 & von des werden küneges hant.\\ 
 & von im \textbf{sagt} wâr diz vingerlîn,\\ 
10 & er enpfiengez von der hende mîn.\\ 
 & swaz \textbf{er} kumbers ie gewan,\\ 
 & dâ bin ich gar unschuldec an,\\ 
 & wan sînen lîp hân ich gewert\\ 
 & mit \textbf{gedanken}, \textbf{swes} er an mich gert.\\ 
15 & er hete schiere daz vernomen,\\ 
 & m\textit{ö}ht ich iemer vürbaz komen.\\ 
 & Orgelusen ich geküsset hân,\\ 
 & diu sînen tôt sus werben kan.\\ 
 & daz was ein kus, den Judas truoc,\\ 
20 & dâ von man sprichet noch genuoc.\\ 
 & Elliu triwe an mir verswant,\\ 
 & daz der Turkote Florant\\ 
 & unt der herzoge von Gowerzin\\ 
 & von mir geküsset \textbf{solden} sîn.\\ 
25 & \textbf{mîn suone wirt in} doch nimmer ganz,\\ 
 & die gein dem künege Gramoflanz\\ 
 & mit stæte ir hazzen kunnen tragen.\\ 
 & \textbf{mîne} muoter \textbf{sult ir daz} verdagen\\ 
 & unt mîne swester Cundrie."\\ 
30 & des bat Gawan Itonje.\\ 
\end{tabular}
\scriptsize
\line(1,0){75} \newline
D Z Fr63 \newline
\line(1,0){75} \newline
\textbf{1} \textit{Initiale} D Z Fr63  \textbf{7} \textit{Majuskel} D  \textbf{21} \textit{Majuskel} D  \newline
\line(1,0){75} \newline
\textbf{1} sihe] \textit{om.} Z \textbf{4} mîn] \textit{om.} Fr63  $\cdot$ strîtet] striten Z \textbf{5} ir reht] reht Fr63 \textbf{12} gar] doch Z \textbf{13} sînen] [einen]: sinen Fr63 \textbf{16} möht] moht D (Z) (Fr63) \textbf{19} Judas] iudas Z \textbf{20} man] \textit{om.} Z \textbf{22} Turkote] Tvrkoit Z Turkoite Fr63 \textbf{26} Gramoflanz] Gramoflantz Z Gramovlan Fr63 \textbf{28} mîne] Miner Z \textbf{29} mîne] miner Z  $\cdot$ Cundrie] Cvndrîe D kundrie Z \textbf{30} Gawan] Gawanen Fr63  $\cdot$ Itonje] Jtonîe D Jconie Z Jtonie Fr63 \newline
\end{minipage}
\hspace{0.5cm}
\begin{minipage}[t]{0.5\linewidth}
\small
\begin{center}*m
\end{center}
\begin{tabular}{rl}
 & dô sprach si: "hêrre, ich sihe nû wol,\\ 
 & ob ich sô vor iu sprechen sol,\\ 
 & daz ir von im rîtet,\\ 
 & nâch dem mîn herz strîtet.\\ 
5 & ob ir der zuht \textbf{ir reht nû} tuot,\\ 
 & hêrre, diu lêret iuch helden muot.\\ 
 & disiu gâbe ist mir \textbf{ie} gesant\\ 
 & von des werden küniges hant.\\ 
 & von im \textbf{seite} wâr diz vingerlîn,\\ 
10 & er enpfienc ez von der hende mîn.\\ 
 & waz \textbf{er} kumbers ie gewan,\\ 
 & dâ bin ich gar unschuldic an,\\ 
 & wan sînen lîp hân ich gewert\\ 
 & mit \textbf{gedanken}, \textbf{waz} er \textit{an} mich gert.\\ 
15 & er het schier daz vernomen,\\ 
 & m\textit{ö}hte ich iemer vürbaz komen.\\ 
 & Urgelusen ich geküsset hân,\\ 
 & diu sînen tôt sus werben kan.\\ 
 & daz was ein kus, den Judas truoc,\\ 
20 & dâ von man \textit{s}pr\textit{i}chet noch genuoc.\\ 
 & alliu triuwe an mir verswant,\\ 
 & daz der Turkoite Florant\\ 
 & un\textit{d der h}erzog\textit{e} von Gowertzin\\ 
 & von mir geküsse\textit{t} \textbf{solte} sîn.\\ 
25 & \textbf{mîn suone wirt in} doch nimmer ganz,\\ 
 & die gegen dem künige Gramolanz\\ 
 & mit stæte ir hazzen kunnen tragen.\\ 
 & \textbf{mîn} muoter \textbf{ir daz solt} ver\textit{d}agen\\ 
 & und mîn swester Condrie."\\ 
30 & des bat Gawanen Ithonie.\\ 
\end{tabular}
\scriptsize
\line(1,0){75} \newline
m n o \newline
\line(1,0){75} \newline
\newline
\line(1,0){75} \newline
\textbf{2} iu] úsz o \textbf{5} der] ir o \textbf{14} an] \textit{om.} m \textbf{16} möhte] Mohtte m (o) \textbf{17} Urgelusen] Vrgelúse o \textbf{20} sprichet] scprchet m \textbf{22} Turkoite] turkoit m o túrkoit n \textbf{23} und der herzoge] Vndertzogin m  $\cdot$ Gowertzin] gowerczen o \textbf{24} geküsset] gekussette m \textbf{26} Gramolanz] gramolantz m gramonlantz n gramolancz o \textbf{28} verdagen] vertragen m n \textbf{30} des] Das o  $\cdot$ Gawanen] gawanene n  $\cdot$ Ithonie] jthonie m ẏthonie n \newline
\end{minipage}
\end{table}
\newpage
\begin{table}[ht]
\begin{minipage}[t]{0.5\linewidth}
\small
\begin{center}*G
\end{center}
\begin{tabular}{rl}
 & dô sprach si: "hêrre, ich sih nû wol,\\ 
 & ob ich sô vor iu sprechen sol,\\ 
 & daz ir von im rîtet,\\ 
 & nâch dem mîn herze strîtet.\\ 
5 & obe ir der zühte \textbf{nû ir rehte} tuot,\\ 
 & hêrre, diu lêrt \textit{iu}ch helende\textit{n} muot.\\ 
 & disiu gâbe ist mir \textbf{ouch} \textbf{ê} gesant\\ 
 & von des werden küniges hant.\\ 
 & von im \textbf{saget} wâr diz vingerlîn,\\ 
10 & er enpfienc \textit{ez} von der hende mîn.\\ 
 & swaz \textbf{er} kumbers ie gewan,\\ 
 & dâ bin ich gar unschuldic an,\\ 
 & wan sînen lîp hân ich gewert\\ 
 & mit \textbf{gedanken}, \textbf{\textit{sw}es} er an mich gert.\\ 
15 & er het schier daz vernomen,\\ 
 & m\textit{ö}ht ich immer vürbaz komen.\\ 
 & Orgelusen ich ge\textit{k}üsset hân,\\ 
 & diu sînen tôt sus werben kan.\\ 
 & daz was ein kus, den Judas truoc,\\ 
20 & dâ von man sprichet noch genuoc.\\ 
 & elliu triuwe an mir verswant,\\ 
 & daz der Turkoite Florant\\ 
 & unde der herzoge von Gowerzin\\ 
 & von mir geküsset \textbf{solde} sîn.\\ 
25 & \textbf{mîn suone wirt in} doch nimmer ganz,\\ 
 & die gein dem künige Gramoflanz\\ 
 & mit stæte ir hazzen kunnen tragen.\\ 
 & \textbf{mîne} muoter \textbf{sult ir daz} verdagen\\ 
 & unde mîn swester Gundrie."\\ 
30 & des bat Gawan Itonie.\\ 
\end{tabular}
\scriptsize
\line(1,0){75} \newline
G I L M Z Fr18 Fr51 \newline
\line(1,0){75} \newline
\textbf{1} \textit{Initiale} L Z Fr51  \textbf{7} \textit{Initiale} I  \newline
\line(1,0){75} \newline
\textbf{1} dô sprach si] Da sprach sy M Se sprach Fr51  $\cdot$ sih] sy M \textit{om.} Z  $\cdot$ nû] \textit{om.} L \textbf{2} sô vor iu] vor uͯch so L \textbf{3} im] dem M \textbf{4} dem] den Fr51  $\cdot$ strîtet] striten Z \textbf{5} nû ir rehte] nu rehte I ir reht nv L (Z) recht Nu M \textbf{6} iuch] mich G  $\cdot$ helenden] helende G heldes L \textbf{8} werden] edelen I \textbf{9} diz] daz L \textbf{10} ez] \textit{om.} G \textbf{11} swaz] Waz L (M) \textbf{12} gar] \textit{om.} M doch Z \textbf{14} swes] des G wez L (Fr51)  $\cdot$ an mich] \textit{om.} I \textbf{15} er] Et L \textbf{16} möht] moht G I (L) (M) wolt Fr51 \textbf{17} Orgelusen] orgulusen I Orgelýsen L  $\cdot$ geküsset] gehuset G \textbf{19} kus] kusch M  $\cdot$ Judas] iudas G I Z Fr51 \textbf{20} dâ von] von dem I  $\cdot$ man] \textit{om.} Z  $\cdot$ noch] \textit{om.} M \textbf{22} Turkoite] Turchoyde I Tuͯrkoite L Tvrkoit Z Tyrkoẏte Fr18  $\cdot$ Florant] floriant G I \textbf{23} Gowerzin] Gouerzin I gowerczin M gowercin Fr51 \textbf{24} geküsset solde] solten gekushet I gekuͯsset solten L (Z) (Fr18) (Fr51) \textbf{25} in] \textit{om.} I yme M  $\cdot$ doch] \textit{om.} Fr51  $\cdot$ nimmer] nirgen M \textbf{26} dem] den Fr51  $\cdot$ Gramoflanz] gramorflanz M Gramoflantz Z Fr18 gramoflans Fr51 \textbf{27} stæte] strit L (Fr51)  $\cdot$ ir hazzen] ir haz I haz Fr51  $\cdot$ kunnen] kunne M \textbf{28} mîne] Myner M (Z) (Fr51) \textbf{29} mîn] miner I (M) Z Fr51  $\cdot$ Gundrie] [chvnd*]: chvndre G kvndrie L (Z) Fr18 kundrien M (Fr51) \textbf{30} des] Svs Fr18  $\cdot$ Gawan] Gawanen L Fr18 \textit{om.} Fr51  $\cdot$ Itonie] ẏtonie G ytonie I jtonie L Jthonien M Jconie Z ẏtonẏe Fr18 eltonie en Fr51 \newline
\end{minipage}
\hspace{0.5cm}
\begin{minipage}[t]{0.5\linewidth}
\small
\begin{center}*T
\end{center}
\begin{tabular}{rl}
 & dô sprach si: "hêrre, ich sihe nû wol,\\ 
 & ob ich sô vor iu sprechen sol,\\ 
 & daz ir von im rîtet,\\ 
 & nâch dem mîn herze strîtet.\\ 
5 & ob ir der zuht \textbf{ir reht nû} tuot,\\ 
 & hêrre, diu lêret iuch helnde\textit{n} muot.\\ 
 & disiu gâbe ist mir \textbf{ouch} \textbf{ê} gesant\\ 
 & von des werden küneges hant.\\ 
 & von im \textbf{seit} wâr diz vingerlîn,\\ 
10 & er entviengez von der hende mîn.\\ 
 & waz \textbf{ich} kumbers ie gewan,\\ 
 & dâ bin ich gar unschuldic an,\\ 
 & wan sînen lîp hân ich gewert\\ 
 & mit \textbf{gedanke}, \textbf{wes} er an mich gert.\\ 
15 & er hete schiere daz vernomen,\\ 
 & m\textit{ö}ht \textit{ich} iemer vürbaz komen.\\ 
 & Orgelusen ich geküsset hân,\\ 
 & diu sînen tôt sus werben kan.\\ 
 & daz was ein kus, den Judas truoc,\\ 
20 & dâ von man sprichet noch genuoc.\\ 
 & alliu triuwe an mir verswant,\\ 
 & daz der Turkoyte Florant\\ 
 & und der herzoge von Gowerzin\\ 
 & von mir geküsset \textbf{solte} sîn.\\ 
25 & \textbf{minne in wirt} doch niemer ganz,\\ 
 & die gein dem künege Gramoflanz\\ 
 & mit stæte ir hazzen kunnen tragen.\\ 
 & \textbf{mîner} muoter \textbf{solt ir daz} verdagen\\ 
 & und mîne swester Kuondrie."\\ 
30 & des bat Gawan Itonie.\\ 
\end{tabular}
\scriptsize
\line(1,0){75} \newline
U V W Q R Fr40 \newline
\line(1,0){75} \newline
\textbf{1} \textit{Initiale} W R  \newline
\line(1,0){75} \newline
\textbf{1} ich sihe nû] ich syhe W nun sehe ich Q \textbf{2} sô] \textit{om.} Q \textbf{5} ir] nun R \textbf{6} helnden] helnde U (Q) heldes W \textbf{7} ê] \textit{om.} Q  $\cdot$ disiu] dise R \textbf{9} diz] das Q R \textbf{11} waz] Swaz V  $\cdot$ ich] er V W Q R \textbf{12} bin] bey Q \textbf{14} gedanke] gedanken V (Q) gedencken W (R)  $\cdot$ wes] swez V was W (R) \textbf{15} hete] [*]: hette V \textbf{16} möht ich] Mocht U [*ch]: Moͤht ich V Mocht im Q \textbf{17} Orgelusen] Orgulusen R \textbf{18} sus] als Q \textbf{19} Judas] iudas V W Q \textbf{21} alliu] Alle R  $\cdot$ mir] im Q \textbf{22} der] \textit{om.} W  $\cdot$ Turkoyte] tvrkoite V (Q) [Tur*oite]: Turkoite  R \textbf{23} Gowerzin] kawerzin Q goverzin Fr40 \textbf{24} solte] solten V (W) (R) (Fr40) \textbf{25} minne in wirt] Min [*]: svͦne wúrt in V Mein suͦne wirt W Mein sune wirt in Q (R) (Fr40) \textbf{26} die] \textit{om.} W  $\cdot$ Gramoflanz] gramaflanz V gramoflantz W Q Gramoflancz R \textbf{27} hazzen] hasse R  $\cdot$ kunnen] kúnnet W \textbf{28} mîner muoter] [M*ter]: Miner mvͦter V Meine muter Q (R)  $\cdot$ verdagen] vertragen R \textbf{29} mîne] [*]: miner V miner (R) (Fr40)  $\cdot$ Kuondrie] kundrie U (V) W Q (Fr40) kondrie R \textbf{30} Gawan] gawanen V gawann Q Gawinen R  $\cdot$ Itonie] Jtonie U [J*onie]: Jtonie V ytonie W Q \newline
\end{minipage}
\end{table}
\end{document}
