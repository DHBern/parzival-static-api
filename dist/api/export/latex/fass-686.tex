\documentclass[8pt,a4paper,notitlepage]{article}
\usepackage{fullpage}
\usepackage{ulem}
\usepackage{xltxtra}
\usepackage{datetime}
\renewcommand{\dateseparator}{.}
\dmyyyydate
\usepackage{fancyhdr}
\usepackage{ifthen}
\pagestyle{fancy}
\fancyhf{}
\renewcommand{\headrulewidth}{0pt}
\fancyfoot[L]{\ifthenelse{\value{page}=1}{\today, \currenttime{} Uhr}{}}
\begin{document}
\begin{table}[ht]
\begin{minipage}[t]{0.5\linewidth}
\small
\begin{center}*D
\end{center}
\begin{tabular}{rl}
\textbf{686} & \begin{large}B\end{large}ene unders küneges \textbf{armen} saz;\\ 
 & diu liez den kampf gar âne haz.\\ 
 & si het des küneges manheit\\ 
 & sô vil gesehen, dâ \textbf{der} streit,\\ 
5 & daz si\textbf{z} wolte ûz\textbf{en} sorgen lân.\\ 
 & wiste ab si, daz Gawan\\ 
 & ir vrouwen bruoder wære\\ 
 & unt daz disiu \textbf{strengen} mære\\ 
 & ûf ir hêrren \textbf{wæren} gezogen,\\ 
10 & si wære an vreuden dâ betrogen.\\ 
 & Si brâht dem künege ein vingerlîn,\\ 
 & daz Itonje, diu \textbf{junge} künegîn,\\ 
 & \textbf{hete durch minne im} gesant,\\ 
 & daz ir bruoder wert erkant\\ 
15 & \textbf{holte} über Sabbins.\\ 
 & Bene ûf \textbf{dem} Poynzaclins\\ 
 & kom in eime seytiez.\\ 
 & \textbf{disiu} mære si niht liez:\\ 
 & "Von Schastel Marvale gevarn\\ 
20 & ist mîn vrouwe mit vrouwen scharn."\\ 
 & si mant in triwe unt êre\\ 
 & von ir vrouwen mêre,\\ 
 & denne ie kint manne enbôt,\\ 
 & unt daz er dæhte an ir nôt,\\ 
25 & sît si vür \textbf{alle} gewinne\\ 
 & dienst büte nâch sîner minne.\\ 
 & daz machte den künec hôch gemuot.\\ 
 & unreht er \textbf{doch Gawane} tuot.\\ 
 & solt ich engelten \textbf{sus} der swester mîn,\\ 
30 & ich wolt ê âne swester sîn.\\ 
\end{tabular}
\scriptsize
\line(1,0){75} \newline
D \newline
\line(1,0){75} \newline
\textbf{1} \textit{Initiale} D  \textbf{11} \textit{Majuskel} D  \textbf{19} \textit{Majuskel} D  \newline
\line(1,0){75} \newline
\textbf{12} Itonje] Jtonie D \textbf{19} Schastel Marvale] Scastel marvale D \newline
\end{minipage}
\hspace{0.5cm}
\begin{minipage}[t]{0.5\linewidth}
\small
\begin{center}*m
\end{center}
\begin{tabular}{rl}
 & Bene under des küniges \textbf{armen} saz;\\ 
 & diu liez den kampf gar âne haz.\\ 
 & si het des küniges manheit\\ 
 & sô vil gesehen, dô \textbf{er} streit,\\ 
5 & daz si  wolt ûz sorgen lân.\\ 
 & \textit{w}uste aber si, daz Gawan\\ 
 & ir vrowen bruode\textit{r w}ære\\ 
 & und daz  \textbf{strengen} mære\\ 
 & ûf ir hêrren \textbf{wæren} gezogen,\\ 
10 & si wær an vröuden dô betrogen.\\ 
 & si brâhte dem künic ein vingerlîn,\\ 
 & d\textit{az} Ithonie, diu künigîn,\\ 
 & \textbf{het durch minne im} gesant,\\ 
 & daz ir bruoder wert erkant\\ 
15 & \textbf{holt} über \textbf{den} Sabins.\\ 
 & Bene ûf \textbf{den} Poinzacli\textit{n}s\\ 
 & kam in eine\textit{m} se\textit{it}iez.\\ 
 & \textbf{diu} \textit{m}ære si niht liez:\\ 
 & "von Schahtel Marv\textit{e}ile varn\\ 
20 & ist mîn vrowe mit vrowen scharn."\\ 
 & si mante in triuwe und êre\\ 
 & von ir vrowen mêre,\\ 
 & dan ie kint man enbôt,\\ 
 & und daz er d\textit{æ}hte an ir nôt,\\ 
25 & sît si vür \textbf{aller} gewinne\\ 
 & dienst büte nâch sîner minne.\\ 
 & daz maht den künic hôchgemuot.\\ 
 & unreht er \textbf{Gawane doch} tuot.\\ 
 & solt ich engelten \textbf{sus} der swester mîn,\\ 
30 & ich wolte ê âne swester sîn.\\ 
\end{tabular}
\scriptsize
\line(1,0){75} \newline
m n o Fr69 \newline
\line(1,0){75} \newline
\newline
\line(1,0){75} \newline
\textbf{4} er] er er o \textbf{6} wuste] Muͯste m \textbf{7} bruoder wære] bruͯder was vnd were m \textbf{12} daz] Do m  $\cdot$ Ithonie] jthonie m ithonẏe n [jtonie]: jthonie o  $\cdot$ künigîn] junge kv́nigin n (o) \textbf{13} gesant] sant o \textbf{14} wert] wart n \textbf{15} den] dan o \textbf{16} Poinzaclins] pointzaclis m [bom]: bonizaclins n poinczaclins o \textbf{17} in einem] in einen m (n) o  $\cdot$ seitiez] secies m seicies n o \textbf{18} diu] Dise n Disise o  $\cdot$ mære] were m \textbf{19} Schahtel Marveile] schahttel maruaile m schathel marueile n [schattele]: schattelamorneẏle o  $\cdot$ varn] gefarn n o \textbf{20} ist] Jch o \textbf{21} mante] nante o \textbf{24} dæhte] dahtte m (o) gedochte n \textbf{28} Gawane] gawanen o \textbf{29} solt] Sol Fr69 \textbf{30} âne] nie o \newline
\end{minipage}
\end{table}
\newpage
\begin{table}[ht]
\begin{minipage}[t]{0.5\linewidth}
\small
\begin{center}*G
\end{center}
\begin{tabular}{rl}
 & \textbf{\begin{large}V\end{large}rou} Bene unders küniges \textbf{arme} saz;\\ 
 & diu lie den kampf gar âne haz.\\ 
 & si het des küniges manheit\\ 
 & sô vil gesehen, dâ \textbf{er} streit,\\ 
5 & daz si\textbf{z} wolde ûz \textbf{den} sorgen lân.\\ 
 & wesse aber si, daz Gawan\\ 
 & ir vrouwen bruoder wære\\ 
 & unde daz disiu \textbf{strengen} mære\\ 
 & ûf ir hêrren \textbf{w\textit{æ}ren} gezogen,\\ 
10 & si wære an vröuden dâ betrogen.\\ 
 & si brâhte dem künige ein vingerlîn,\\ 
 & daz Itonie, diu \textbf{junge} künigîn,\\ 
 & \textbf{im durch minne hât} gesant,\\ 
 & daz ir bruoder wert erkant\\ 
15 & \textbf{holt} über \textbf{den} Sabins.\\ 
 & Bene ûf \textbf{den} Poinsaclins\\ 
 & kom in einem seitiez.\\ 
 & \textbf{disiu} mære si niht \textbf{en}liez:\\ 
 & "\textit{von} Tschastel Marveile gevarn\\ 
20 & ist mîn vrouwe mit vrouwen scharn."\\ 
 & si mante in triwe unde êre\\ 
 & von ir vrouwen mêre,\\ 
 & danne ie kint manne enbôt,\\ 
 & unde daz er d\textit{æ}hte an ir nôt,\\ 
25 & sît si vür \textbf{alle} gewinne\\ 
 & dienst büte nâch sîner minne.\\ 
 & daz machet den künic hôch gemuot.\\ 
 & unreht er \textbf{Gawan doch} tuot.\\ 
 & solde ich engelten der swester mîn,\\ 
30 & ich wolde ê ân swester sîn.\\ 
\end{tabular}
\scriptsize
\line(1,0){75} \newline
G I L M Z Fr18 Fr20 Fr52 \newline
\line(1,0){75} \newline
\textbf{1} \textit{Initiale} G I L M Z Fr18 Fr20  \textbf{25} \textit{Initiale} I  \newline
\line(1,0){75} \newline
\textbf{1} Vrou] ÷rowe Fr20 \textbf{4} vil] wol Fr18 \textbf{5} Si wold ::: Fr52 \textbf{9} wæren] wâren G (L) (Fr18) wer I M \textbf{10} wære] waren M  $\cdot$ dâ] Gar I (Z) \textbf{11} brâhte] brah I \textbf{12} Itonie] Jthonie M Jconie Z Jtonẏe Fr18 \textbf{13} hât] het I L (M) Z Fr18 (Fr20) \textbf{14} wert] wirt I were ir M wær Fr18 \textbf{15} holt] holte Fr20  $\cdot$ Sabins] Sabinz L Sabẏns Fr18 \textbf{16} den] dem L (M) Z Fr18  $\cdot$ Poinsaclins] poysaclins I poynsaclinsz L poyns sadins M poinzaclins Z poẏnsaclins Fr18 poinsacllins Fr20 \textbf{17} in einem] in einen I im einem L  $\cdot$ seitiez] sectiez L \textbf{18} niht] niene M Fr18  $\cdot$ enliez] verliez I liez L (M) Z (Fr18) Fr20 \textbf{19} von] \textit{om.} G  $\cdot$ Tschastel Marveile] shahtemarveile I kastel marveil L schachtil Marveile M tschachtel marveil Z tschastel marveile Fr20 Fr52 \textbf{20} scharn] schar M \textbf{21} mante] mant I L \textbf{22} ir] in Z \textbf{23} ie kint] kint I ir ie kinch L ir kint M  $\cdot$ enbôt] Gebot I \textbf{24} dæhte] dâhte G (L) (M)  $\cdot$ ir] die Z \textbf{25} gewinne] gwinne \sout{dienist} Fr20 \textbf{26} büte] biutet I bite M  $\cdot$ sîner] \textit{om.} I \textbf{27} machet] mac het I mahte Z  $\cdot$ hôch gemuot] so hochgemuͤt I \textbf{28} er Gawan] Gawan er I \textbf{29} der] sust der Z  $\cdot$ swester] swer L \newline
\end{minipage}
\hspace{0.5cm}
\begin{minipage}[t]{0.5\linewidth}
\small
\begin{center}*T
\end{center}
\begin{tabular}{rl}
 & \textbf{\begin{large}V\end{large}rou} Bene under des küneges \textbf{arme} saz;\\ 
 & diu liez den kampf gar âne haz.\\ 
 & si hete des küneges manheit\\ 
 & sô vil gesehen, dô \textbf{er} streit,\\ 
5 & daz si \textbf{ez} wolte ûz \textbf{den} sorgen lân.\\ 
 & wiste aber si, daz Gawan\\ 
 & ir vrouwen bruoder wære\\ 
 & und daz disiu \textbf{strenge} mære\\ 
 & ûf ir hêrren \textbf{wære} gezogen,\\ 
10 & si wære an vreuden dô betrogen.\\ 
 & si brâht dem künege ein vingerlîn,\\ 
 & daz Itonie, diu \textbf{junge} künegîn,\\ 
 & \textbf{im durch minne hete} gesant,\\ 
 & daz ir bruoder w\textit{e}rt erkant\\ 
15 & \textbf{brâhte} über \textbf{den} Sabins.\\ 
 & Bene ûf \textbf{dem} Poynzaclins\\ 
 & kam in eime seitiez.\\ 
 & \textbf{disiu} mære si niht liez:\\ 
 & "von Tschahtel Marvele gevarn\\ 
20 & ist mîn vrouwe mit vrouwen scharn."\\ 
 & si mante \textit{in} triuwe und êre\\ 
 & von ir vrouwen mêre,\\ 
 & dan ie kint manne enbôt,\\ 
 & und daz er dæhte an ir nôt,\\ 
25 & sît si vür \textbf{alle} gewinne\\ 
 & dienst büt\textit{e} nâch sîner minne.\\ 
 & \begin{large}D\end{large}az machte den künec hôch gemuot.\\ 
 & unrehte er \textbf{Gawane doch} tuot.\\ 
 & soltich engelten der swester mîn,\\ 
30 & ich wolte ê âne swester sîn.\\ 
\end{tabular}
\scriptsize
\line(1,0){75} \newline
U V W Q R \newline
\line(1,0){75} \newline
\textbf{1} \textit{Initiale} U  \textbf{27} \textit{Initiale} U V  \newline
\line(1,0){75} \newline
\textbf{1} arme] Rechten Arme R \textbf{2} gar] \textit{om.} R \textbf{6} Gawan] gewan Q \textbf{8} disiu strenge] diesen strengen Q dise strenge R \textbf{9} wære] weren Q \textbf{10} dô] gar W \textbf{12} Itonie] Jtonie U R [y*onie]: ytonie V ytonie W [i*]: itonie Q \textbf{13} minne] [*]: minne V ir minne W \textbf{14} wert] wirt U ward R \textbf{15} brâhte] Holte Q  $\cdot$ Sabins] Roitschesabins V \textbf{16} Poynzaclins] poinzaklins W pinzaclins Q \textbf{17} in] vff Q \textbf{18} liez] enlies V \textbf{19} Tschahtel Marvele] Tschatel marveile U Scatelmarveile V kastel marfeile W schachtel marueile Q schaltel marveile R  $\cdot$ gevarn] varen W \textbf{21} mante in] mante U [mant*]: mante in V mant Jn R \textbf{24} dæhte] gedechte W \textbf{25} alle] alle ir Q \textbf{26} büte] butet U (W) bút R \textbf{28} Gawane] her gawan W Gawin R  $\cdot$ doch] [*]: doch V \newline
\end{minipage}
\end{table}
\end{document}
