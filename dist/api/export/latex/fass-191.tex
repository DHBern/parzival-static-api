\documentclass[8pt,a4paper,notitlepage]{article}
\usepackage{fullpage}
\usepackage{ulem}
\usepackage{xltxtra}
\usepackage{datetime}
\renewcommand{\dateseparator}{.}
\dmyyyydate
\usepackage{fancyhdr}
\usepackage{ifthen}
\pagestyle{fancy}
\fancyhf{}
\renewcommand{\headrulewidth}{0pt}
\fancyfoot[L]{\ifthenelse{\value{page}=1}{\today, \currenttime{} Uhr}{}}
\begin{document}
\begin{table}[ht]
\begin{minipage}[t]{0.5\linewidth}
\small
\begin{center}*D
\end{center}
\begin{tabular}{rl}
\textbf{191} & Teilen \textbf{ez} hiez diu künegîn,\\ 
 & \textbf{dar zuo die kæse, daz vleisch}, den wîn\\ 
 & dirre kreftelôsen diet.\\ 
 & Parzival, ir gast, daz riet.\\ 
5 & \textbf{des} beleip in zwein vil kûme ein snite.\\ 
 & die teiltens âne bâgens site.\\ 
 & \begin{large}D\end{large}iu wirtschaft was ouch verzert,\\ 
 & dâ mite maneges tôt erwert,\\ 
 & den der hunger leben liez.\\ 
10 & \textbf{dem} gaste man dô \textbf{beiten} hiez\\ 
 & sanfte, des ich wænen wil.\\ 
 & wæren die burgære vederspil,\\ 
 & si\textbf{ne} wæren überkrüpfet niht,\\ 
 & des noch ir tischgerihte giht.\\ 
15 & Si truogen alle \textbf{hungers mâl},\\ 
 & \textbf{wan} der junge Parzival.\\ 
 & der nam \textbf{slâfes} urloup.\\ 
 & ob sîne kerzen wæren schoup?\\ 
 & nein, si wâren bezzer gar.\\ 
20 & dô gienc der junge, wol gevar\\ 
 & an ein bette rîche,\\ 
 & \textbf{gehêrt} küneclîche,\\ 
 & niht nâch \textbf{armüete} kür.\\ 
 & \textbf{ein teppich was geleit dar} vür.\\ 
25 & \textbf{Er bat die ritter} wider gên.\\ 
 & diene liez er dâ niht langer stên.\\ 
 & kint \textbf{im} entschuohten. sân er slief,\\ 
 & unz im der wâre jâmer rief\\ 
 & \textbf{unt} liehter ougen herzen regen.\\ 
30 & \textbf{die} \textbf{wachten} schiere den werden degen.\\ 
\end{tabular}
\scriptsize
\line(1,0){75} \newline
D \newline
\line(1,0){75} \newline
\textbf{1} \textit{Majuskel} D  \textbf{7} \textit{Initiale} D  \textbf{15} \textit{Majuskel} D  \textbf{25} \textit{Majuskel} D  \newline
\line(1,0){75} \newline
\newline
\end{minipage}
\hspace{0.5cm}
\begin{minipage}[t]{0.5\linewidth}
\small
\begin{center}*m
\end{center}
\begin{tabular}{rl}
 & \multicolumn{1}{l}{ - - - }\\ 
 & \multicolumn{1}{l}{ - - - }\\ 
 & \multicolumn{1}{l}{ - - - }\\ 
 & \multicolumn{1}{l}{ - - - }\\ 
5 & \multicolumn{1}{l}{ - - - }\\ 
 & \multicolumn{1}{l}{ - - - }\\ 
 & \begin{large}D\end{large}iu wirtschaft was ouch verzert,\\ 
 & dâ mite manige\textit{s} tôt erwert,\\ 
 & den der hunger leben liez.\\ 
10 & \textbf{dem} gaste man dô \textbf{betten} hiez\\ 
 & sanfte, des ich wænen wil.\\ 
 & wæren die burgære vederspil,\\ 
 & si \textbf{en}wæren über\textit{k}rüpfet niht,\\ 
 & des noch ir tischgerihte giht.\\ 
15 & si truogen alle \textbf{hungers mâl},\\ 
 & \textbf{wand} der junge Parcifal.\\ 
 & de\textit{r} nam \textbf{slâfens} urloup.\\ 
 & ob sîne kerzen wære\textit{n} schoup?\\ 
 & nein, si wâren bezzer gar.\\ 
20 & dô gienc der junge, wol gevar\\ 
 & an \textit{e}in bette rîche,\\ 
 & \textbf{gehêret} küneclîche,\\ 
 & niht nâch \textbf{armuot} küre.\\ 
 & \textbf{ein teppich was geleit dar} vüre.\\ 
25 & \textbf{er bat die ritter} wider gên.\\ 
 & die enliez er d\textit{â} niht langer stên.\\ 
 & kint \textbf{ime} entschuoheten. sân er slief,\\ 
 & unz ime der wâre jâmer rief\\ 
 & \textbf{und} liehter ougen herzen regen.\\ 
30 & \textbf{die} \textbf{wachten} schiere den werden degen.\\ 
\end{tabular}
\scriptsize
\line(1,0){75} \newline
m n o Fr69 \newline
\line(1,0){75} \newline
\textbf{7} \textit{Initiale} m   $\cdot$ \textit{Capitulumzeichen} n  \newline
\line(1,0){75} \newline
\textbf{1} \textit{Die Verse 191.1-6 fehlen} m n o  \textbf{8} maniges] manigen m \textbf{10} dem gaste] Den gast n Das den gast o \textbf{13} überkrüpfet] uͯber tropfet m (n) (o) \textbf{14} tischgerihte] disch gerichtet n (o) \textbf{17} der nam] Dennam m Dennen n Dannen o  $\cdot$ slâfens] sloffen n o \textbf{18} wæren] were m \textbf{20} gevar] gewar o \textbf{21} ein] in m \textbf{22} küneclîche] mẏnneclich n (o) \textbf{25} wider] nẏder n vnder o \textbf{26} enliez] liesse n lies o  $\cdot$ dâ] do m n o \textbf{27} entschuoheten] enschutten o \newline
\end{minipage}
\end{table}
\newpage
\begin{table}[ht]
\begin{minipage}[t]{0.5\linewidth}
\small
\begin{center}*G
\end{center}
\begin{tabular}{rl}
 & teilen hiez diu künigîn\\ 
 & \textbf{daz vleisch, di\textit{e} kæse unde} den wîn\\ 
 & dirre kreftelôsen diet.\\ 
 & Parzival, ir gast, daz riet.\\ 
5 & \textbf{es} beleip in zwein vil kûme ein snite.\\ 
 & die teilten si âne bâgens site.\\ 
 & diu wirtschaft was ouch verzert,\\ 
 & dâ mite maniges tôt erwert,\\ 
 & den der hunger leben liez.\\ 
10 & \textbf{ir} gaste man dô \textbf{betten} hiez\\ 
 & sanfte, des ich wænen wil.\\ 
 & w\textit{æ}ren die burgære vederspil,\\ 
 & si\textbf{ne} w\textit{æ}ren überkrüpfet niht.\\ 
 & des noch ir tischgerihte giht.\\ 
15 & si truogen alle \textbf{hungermâl},\\ 
 & \textbf{wan} der junge Parzival.\\ 
 & \begin{large}D\end{large}er \textit{n}a\textit{m} \textbf{slâfes} urloup.\\ 
 & obe sîne kerzen w\textit{æ}rn schoup?\\ 
 & nein, si wâren bezzer gar.\\ 
20 & dô gie der junge, wol gevar\\ 
 & an ein bette rîche,\\ 
 & \textbf{gehêrt} küniclîche,\\ 
 & niht nâch \textbf{âventiure} kür.\\ 
 & \textbf{ein tepech was geleit dar} vür.\\ 
25 & \textbf{er hiez die rîter} wider gên.\\ 
 & diene liez er dâ niht lenger stên.\\ 
 & kint \textbf{in} entschuohten. sân er slief,\\ 
 & unze im der wâre jâmer rief\\ 
 & \textbf{unde} liehter ougen herzen regen.\\ 
30 & \textbf{die} \textbf{erwachten} schiere den werden degen.\\ 
\end{tabular}
\scriptsize
\line(1,0){75} \newline
G I O L M Q R Z Fr47 \newline
\line(1,0){75} \newline
\textbf{7} \textit{Initiale} I  \textbf{13} \textit{Initiale} O  \textbf{17} \textit{Initiale} G  \textbf{19} \textit{Initiale} Z  \textbf{25} \textit{Initiale} R  \newline
\line(1,0){75} \newline
\textbf{1} hiez] hiez ez I Z ez hiez O (Q) (R) hiez er M \textbf{2} daz vleisch die kæse] daz fleisch div chase G den chese fleish brot I Dar zvͦ den chæse daz fleisch O Die kese daz fleýsch daz brot L Dar zcu dy kese das fleisch M (Q) (R) (Z)  $\cdot$ unde den] vnd I Z den O L (M) Q R \textbf{3} dirre] Jr O Q  $\cdot$ kreftelôsen] kreftelose Q \textbf{4} Parzival] Parzifal I L Parcifal O Z Parzeval M Partzifal Q Parczifal R  $\cdot$ ir] der I (Z)  $\cdot$ gast] gas R  $\cdot$ riet] riete O \textbf{5} \textit{Versfolge 191.6-5} O   $\cdot$ in] ye R  $\cdot$ vil kûme] \textit{om.} R kovm Z \textbf{6} bâgens] wegens Z  $\cdot$ site] mitte R \textbf{7} was ouch] wart ouch M (Q) wart R \textbf{8} Vnd mengem sin leben gemeret R \textbf{9} der hunger] den hungern M \textbf{10} ir] Dem R  $\cdot$ dô] da M Z \textit{om.} R  $\cdot$ betten] beiten L \textbf{11} des] so M \textbf{12} wæren] waren G (L) \textbf{13} sine] ÷ine O Sie Q  $\cdot$ wæren] waren G (M)  $\cdot$ überkrüpfet] vber rvpfet O aber truppfit Q \textbf{14} des] Der Q  $\cdot$ noch] nach Z  $\cdot$ tischgerihte] tish gerihtet I (Z) tische gerichttet R Tischriht Fr47 \textbf{15} hungermâl] hungers mal Z \textbf{16} wan] Won allein R  $\cdot$ junge] werde Q  $\cdot$ Parzival] Parzifal I Parcifal O (L) (Z) Fr47 parczifal M R partzifal Q \textbf{17} nam] :a:: G  $\cdot$ slâfes] slafens I M (Fr47) \textbf{18} obe sîne] Vff sinen M Sine R  $\cdot$ wærn] warn G (R) wæren ein O \textbf{20} dô] Da Z  $\cdot$ gie] \textit{om.} R \textbf{21} \textit{Versfolge 191.22-21} R  \textbf{22} gehêrt] Ge erit M Gelebt R  $\cdot$ küniclîche] wunnechlichen I \textbf{23} nâch âventiure] nach armvͦte O (L) (M) (R) (Z) (Fr47) armute Q \textbf{24} ein] ern I  $\cdot$ was] wart I O Fr47 \textbf{25} hiez] bat Q  $\cdot$ wider] wide R \textbf{26} diene] ern I Die O L R  $\cdot$ er] ir I  $\cdot$ dâ] do O Q  $\cdot$ lenger] lange R \textbf{27} in entschuohten] entshuͤctan im I im entschuͯchten L (Z) yme en schufften M Jn enchuten Fr47  $\cdot$ slief] entschlieff R \textbf{28} unze] Vnd L  $\cdot$ im] :::n Fr47  $\cdot$ der wâre jâmer] deware iamer O Jamer ware R \textbf{29} liehter] lýchter L (M) lichten Q  $\cdot$ herzen] herze O (R) \textbf{30} erwachten] wachtan I (O) (L) (M) (Q) \newline
\end{minipage}
\hspace{0.5cm}
\begin{minipage}[t]{0.5\linewidth}
\small
\begin{center}*T
\end{center}
\begin{tabular}{rl}
 & Teilen hiez diu künegîn\\ 
 & \textbf{die kæse, daz vleisch, daz brôt}, den wîn\\ 
 & dirre kreftelôsen diet.\\ 
 & Parcifal, ir gast, daz riet.\\ 
5 & \textbf{ez} bleip in zwein vil kûme ein snite.\\ 
 & die teiltens âne bâgens site.\\ 
 & \begin{large}D\end{large}iu wirtschaft, \textbf{diu} was ouch verzert,\\ 
 & dâ mite maneges tôt erwert,\\ 
 & den der hunger leben liez.\\ 
10 & \textbf{dem} gaste man dô \textbf{betten} hiez\\ 
 & sanfte, des ich wænen wil.\\ 
 & wæren die burgære vederspil,\\ 
 & si wæren überkr\textit{ü}pfet niht,\\ 
 & des noch ir ti\textit{s}chgerihte giht.\\ 
15 & si truogen alle \textbf{hungermâl},\\ 
 & \textbf{niuwan} der junge Parcifal.\\ 
 & der nam \textbf{slâfens} urloup.\\ 
 & ob sîne kerzen wæren schoup?\\ 
 & nein, si wâren bezzer gar.\\ 
20 & dô gie der junge, wol gevar\\ 
 & an ein bette rîche,\\ 
 & \textbf{gezieret} küneclîche,\\ 
 & niht nâch \textbf{armüete} kür.\\ 
 & \textbf{dâ wart geleit ein teppich} vür.\\ 
25 & \textbf{Die rîter muosen} wider gân.\\ 
 & diene liez er dâ niht lenger stân.\\ 
 & kint \textbf{im} entschuohten. sân er slief,\\ 
 & unz im der wâre jâmer rief.\\ 
 & \textbf{ein} liehter ougen herzen regen,\\ 
30 & \textbf{er} \textbf{wachte} schiere den werden degen.\\ 
\end{tabular}
\scriptsize
\line(1,0){75} \newline
T U V W \newline
\line(1,0){75} \newline
\textbf{1} \textit{Majuskel} T  \textbf{7} \textit{Initiale} T U V W  \textbf{25} \textit{Majuskel} T  \newline
\line(1,0){75} \newline
\textbf{1} hiez] hiez ez V \textbf{2} daz vleisch daz brôt] daz brot daz flaisch W \textbf{3} kreftelôsen] kreffteloser W \textbf{4} Parcifal] Parzifal V Partzifal W \textbf{5} in] ie V  $\cdot$ vil] \textit{om.} W \textbf{7} diu] \textit{om.} W \textbf{13} überkrüpfet] vber krofphet T \textbf{14} des noch] Darnach W  $\cdot$ tischgerihte] ticsch gerihte T disch gerechte U \textbf{16} niuwan] Nuͦme dan U  $\cdot$ Parcifal] Parzifal V partzifal W \textbf{17} slâfens] zuͦ schafen U schlaffes W \textbf{22} gezieret] Geheret W \textbf{24} dâ] Do U V W \textbf{25} [*i*]: Er bat die ritter wider gen V  $\cdot$ muosen] mvesen T \textbf{26} diene] Die V W  $\cdot$ dâ] do V \textit{om.} W  $\cdot$ lenger] lenger do W \textbf{27} im entschuohten] v́ntschvͦheten [im]: in V enschúchten in W  $\cdot$ sân] zuͦ hand W \textbf{28} unz] Bit daz U \textbf{29} ein] [E*]: Vnde V  $\cdot$ ougen herzen] hertzen augen W \textbf{30} er wachte schiere] [*hiere]: Die wahtent schiere V \newline
\end{minipage}
\end{table}
\end{document}
