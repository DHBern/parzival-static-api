\documentclass[8pt,a4paper,notitlepage]{article}
\usepackage{fullpage}
\usepackage{ulem}
\usepackage{xltxtra}
\usepackage{datetime}
\renewcommand{\dateseparator}{.}
\dmyyyydate
\usepackage{fancyhdr}
\usepackage{ifthen}
\pagestyle{fancy}
\fancyhf{}
\renewcommand{\headrulewidth}{0pt}
\fancyfoot[L]{\ifthenelse{\value{page}=1}{\today, \currenttime{} Uhr}{}}
\begin{document}
\begin{table}[ht]
\begin{minipage}[t]{0.5\linewidth}
\small
\begin{center}*D
\end{center}
\begin{tabular}{rl}
\textbf{795} & \begin{large}V\end{large}rœlîche unt doch mit jâmers siten.\\ 
 & er sprach: "ich hân unsanfte erbiten,\\ 
 & wirde ich immer von iu vrô.\\ 
 & ir schiedet nû jungest von mir \textbf{alsô},\\ 
5 & pfl\textit{e}gt ir \textbf{herzenlîcher} triwe,\\ 
 & man siht iuch drumbe in riwe.\\ 
 & Würde ie prîs von iu gesagt,\\ 
 & hie sî rîter oder magt,\\ 
 & \textbf{werbet} mir \textbf{dâ z}in den tôt\\ 
10 & unt lât sich enden mîne nôt.\\ 
 & sît ir genant Parzival,\\ 
 & sô \textbf{wert} mîn sehen an den Grâl\\ 
 & siben naht und aht tage.\\ 
 & dâ mite ist wendec al mîn klage.\\ 
15 & i\textbf{ne} getar iuch \textbf{anders} warnen niht.\\ 
 & wol iu, ob man iu helfe giht!\\ 
 & Iwer geselle ist hie ein vremder man.\\ 
 & sînes stêns ich im vor mir niht gan.\\ 
 & \textbf{wan} lât \textbf{ir}n varn \textbf{an} sîn gemach?"\\ 
20 & \textbf{Al weinende} Parzival dô sprach:\\ 
 & "\textbf{saget} mir, wâ der Grâl hie lige.\\ 
 & ob diu gotes güete an mir gesige,\\ 
 & des wirt wol innen disiu schar."\\ 
 & sîne venje \textbf{viel er} des endes dar\\ 
25 & drîstunt ze êren der trinitât.\\ 
 & er warp, daz m\textit{üe}se werden rât\\ 
 & des trûrigen mannes herzesêr.\\ 
 & \textbf{er rihte sich} ûf und sprach \textbf{dô} mêr:\\ 
 & "\textit{\begin{large}Œ\end{large}}heim, waz wirret dir?"\\ 
30 & der durch \textbf{sande} Silvestern einen stier\\ 
\end{tabular}
\scriptsize
\line(1,0){75} \newline
D \newline
\line(1,0){75} \newline
\textbf{1} \textit{Initiale} D  \textbf{7} \textit{Majuskel} D  \textbf{17} \textit{Majuskel} D  \textbf{20} \textit{Majuskel} D  \textbf{29} \textit{Initiale} D  \newline
\line(1,0){75} \newline
\textbf{5} pflegt] pfligt D \textbf{11} Parzival] Parcifal D \textbf{20} Parzival] Parcifal D \textbf{26} müese] mvͦse D \textbf{29} Œheim] ÷heim D \newline
\end{minipage}
\hspace{0.5cm}
\begin{minipage}[t]{0.5\linewidth}
\small
\begin{center}*m
\end{center}
\begin{tabular}{rl}
 & vrœlîch und doch mi\textit{t} jâmersiten.\\ 
 & er sprach: "ich hân unsanft erbiten,\\ 
 & wirde ich iemer von iu vrô.\\ 
 & ir sch\textit{ie}de\textit{t} nû jungst von mir \textbf{sô},\\ 
5 & pfleg\textit{t} ir \textbf{helflîcher} triuwe,\\ 
 & man siht iuch dâr umb i\textit{n} riuwe.\\ 
 & würde ie prîs von iu gesaget,\\ 
 & hie sî ritter oder maget,\\ 
 & \textbf{werbt} mir \textbf{d\textit{â} zuo} in den tôt\\ 
10 & und lât sich enden mîne nôt.\\ 
 & sît ir genant Parcifal,\\ 
 & sô \textbf{wert} mîn sehen an den Grâl\\ 
 & siben naht und aht tage.\\ 
 & dâ mit ist wendic al mîn klage.\\ 
15 & ich getar iuch \textit{\textbf{anders}} warnen niht.\\ 
 & wol iuch, ob man iu helfe giht!\\ 
 & iuwer geselle ist hie ein vremder man.\\ 
 & sîn\textit{es} stêns ich im vor mir niht gan.\\ 
 & \textbf{wan} lât \textbf{ir} in varn \textbf{in} sîn gemach?"\\ 
20 & Parcifal \textbf{zuo im} dô sprach:\\ 
 & "\textbf{sagt} mir, wâ der Grâl hie lige.\\ 
 & ob diu gotes güete an mir gesige,\\ 
 & des wirt wol innen disiu schar."\\ 
 & sîn venje \textbf{viel er} des endes dar\\ 
25 & drîstunt zuo êren der trinitât.\\ 
 & er warp, daz müeste werden rât\\ 
 & des trûrigen mannes herzen sêre.\\ 
 & \textbf{er rihte sich} ûf und sprach \textbf{dô} mêre:\\ 
 & "œheim, waz wirret dir?"\\ 
30 & d\textit{e}r durch \textbf{sant} Silvester einen stier\\ 
\end{tabular}
\scriptsize
\line(1,0){75} \newline
m n o V V' W \newline
\line(1,0){75} \newline
\textbf{21} \textit{Initiale} W  \newline
\line(1,0){75} \newline
\textbf{1} \textit{statt 794.30-795.1:} Er enphinc sie alle mit siten V'   $\cdot$ mit] mich m  $\cdot$ jâmersiten] iamers sitten W \textbf{2} er] Jch o  $\cdot$ unsanft] vngerne V' \textbf{3} \textit{Versdoppelung 795.3-4 (²o) nach 795.6; Lesarten der vorausgehenden Verse mit ¹o bezeichnet} o   $\cdot$ wirde] Wurde m n o V (W) Werd V' \textbf{4} schiedet] scheiden m \textsuperscript{2}\hspace{-1.3mm} o schieden n \textsuperscript{1}\hspace{-1.3mm} o  $\cdot$ nû jungst von mir] nu gunst von mir \textsuperscript{2}\hspace{-1.3mm} o zu iungest von mir V' von mir iúngst W \textbf{5} \textit{Die Verse 795.5-10 fehlen} V'   $\cdot$ pflegt] Pfleg m \textbf{6} in] ir m an o \textbf{9} dâ] do m n o V \textit{om.} W  $\cdot$ in] in in n \textbf{11} Parcifal] parzefal V parzifal V' partzifal W \textbf{12} wert] [*]: wendent V wendet V' \textbf{13} aht] \textit{om.} n \textbf{14} dâ mit] So V'  $\cdot$ wendic] vol endet V' \textbf{15} getar] en getar V entar V' tar W  $\cdot$ anders] \textit{om.} m  $\cdot$ warnen] wamen o sagen V' \textbf{16} iuch] uf V' \textbf{17} Vwer geselleschaft sint hie froͤmede man V Vwer geselleschaft daz sint fremde man V' \textbf{18} Sin stens vor mir ich ime nit gan n · Jr stendes ich in vor mir nút gan V · Jr sten ich in nit vor mir gar V'  $\cdot$ sînes] Sin m \textbf{19} ir in] ir o sv́ V (V') in W  $\cdot$ varn] fuͤren W  $\cdot$ in sîn] an ir V V' an sein W \textbf{20} Parcifal] Parzefal V Parzifal V' Herr partzifal W  $\cdot$ zuo im dô] do zv ime V' \textbf{21} hie] do o hin V \textbf{22} diu] \textit{om.} W  $\cdot$ güete] gnad W \textbf{23} des] Der W  $\cdot$ wol] \textit{om.} V' \textbf{24} des] dasz o \textbf{25} êren der] der heiligen V' \textbf{26} warp] [warf]: warp V'  $\cdot$ daz] des W  $\cdot$ müeste] muͦste o (V') W \textbf{27} trûrigen] trurigem n trurigez o  $\cdot$ herzen sêre] herze sere V \textbf{30} der] Dar m n >dar< o  $\cdot$ sant] \textit{om.} V'  $\cdot$ Silvester] siluester m o (V) V' W silueste: o \newline
\end{minipage}
\end{table}
\newpage
\begin{table}[ht]
\begin{minipage}[t]{0.5\linewidth}
\small
\begin{center}*G
\end{center}
\begin{tabular}{rl}
 & \begin{large}V\end{large}rœlîche unde doch mit jâmers siten.\\ 
 & er sprach: "ich hân unsanfte erbiten,\\ 
 & wirde ich imer von iu vrô.\\ 
 & ir schiedet nû jungest von mir \textbf{sô},\\ 
5 & pfleget ir \textbf{helflîcher} triuwe,\\ 
 & man siht iuch drumbe in riuwe.\\ 
 & würde ie brîs von iu gesaget,\\ 
 & hie sî rîter oder maget,\\ 
 & \textbf{sô} \textbf{werbet} mir \textbf{datze} in den tôt\\ 
10 & unde lât sich enden mîne nôt.\\ 
 & sît ir genant Parzival,\\ 
 & sô \textbf{wert} mîn sehen an den Grâl\\ 
 & siben naht unde aht tage.\\ 
 & dâ mit ist wendic al mîn klage.\\ 
15 & ich \textbf{en}getar iuch \textbf{vürbaz} warnen niht.\\ 
 & wol iuch, op man iu helfe giht!\\ 
 & iwer geselle ist hie ein vrömder man.\\ 
 & sînes stêns ich im vor mir niht gan.\\ 
 & \textbf{nû} lât \textbf{in} varn \textbf{an} sîn gemach."\\ 
20 & \textbf{al weinde} Parzival dô sprach:\\ 
 & "\textbf{nû zeiget} mir, wâ der Grâl hie lige.\\ 
 & op diu gotes güete ane mir gesige,\\ 
 & des wirt wol innen disiu schar."\\ 
 & sîn venje \textbf{er viel} des endes dar\\ 
25 & drîstunt zêren der trinitât.\\ 
 & er war\textit{p}, daz m\textit{üe}se werden rât\\ 
 & des trûrigen mannes herzesêr.\\ 
 & \textbf{dô stuont er} ûf unde sprach mêr:\\ 
 & "œheim, waz wirret dir?"\\ 
30 & der durch Silvestern einen stier\\ 
\end{tabular}
\scriptsize
\line(1,0){75} \newline
G I L M Z \newline
\line(1,0){75} \newline
\textbf{1} \textit{Initiale} G I L Z  \textbf{19} \textit{Initiale} I M  \newline
\line(1,0){75} \newline
\textbf{1} doch] \textit{om.} L \textbf{3} wirde] Wurde Z \textbf{5} pfleget] Pflagt L  $\cdot$ helflîcher] hofelicher M \textbf{6} siht] sehe I \textbf{7} \textit{Die Verse 795.7-10 fehlen} L   $\cdot$ würde] Wart M \textbf{9} werbet] irwerbit M  $\cdot$ datze] das M  $\cdot$ in] im Z \textbf{10} lât] [sich]: lat sich Z \textbf{11} Parzival] parcifal G Z parzifal I L M \textbf{12} wert] went Z \textbf{13} siben] Sobin M \textbf{14} wendic] endec I erwendet M \textbf{15} engetar] Getar I entar M (Z) \textbf{18} ich im vor mir] vor mir ich im I \textbf{19} varn] vuͤrn I \textbf{20} Parzival] parcifal G Z parzifal I L M  $\cdot$ dô] da M Z \textbf{21} nû] \textit{om.} I  $\cdot$ hie] hin L ynne M hinne Z \textbf{23} des] der I  $\cdot$ wol] \textit{om.} L \textbf{24} er viel] viel er L \textbf{26} warp] warf G erwarp L  $\cdot$ daz] der M des Z  $\cdot$ müese werden] moͮse werden G muste werde M \textbf{27} herzesêr] herzen ser I (M) (Z) ser L \textbf{28} dô] Da M Z \textbf{29} wirret] [*irret]: wirret L werret M \textbf{30} Silvestern] Siluestren I [Silve*rn]: Silvestern L Siluestern Z \newline
\end{minipage}
\hspace{0.5cm}
\begin{minipage}[t]{0.5\linewidth}
\small
\begin{center}*T
\end{center}
\begin{tabular}{rl}
 & vrœlîche und doch mit jâmers siten.\\ 
 & er sprach: "ich hân unsanfte erbiten,\\ 
 & wirde ich imer von iu vrô.\\ 
 & ir schiedet nû jungest von mir \textbf{sô},\\ 
5 & pfleget ir \textbf{helflîcher} triuw\textit{e},\\ 
 & man siht iuch dâr umb in riuw\textit{e}.\\ 
 & würde ie prîs von iu gesaget,\\ 
 & hie sî rîter oder maget,\\ 
 & \textbf{sô} \textbf{erwerbet} mir \textbf{hin z}in den tôt\\ 
10 & und lât sich enden mîne nôt.\\ 
 & sît ir genant Parcifal,\\ 
 & sô \textbf{wendet} mîn sehen an den Grâl\\ 
 & siben naht und ahte tage.\\ 
 & dâ mit ist wendic al mîn klage.\\ 
15 & ich \textbf{en}getar iuch \textbf{vürbaz} warnen niht.\\ 
 & wol iuch, ob man iu helfe giht!\\ 
 & iuwer geselle ist hie ein vremder man.\\ 
 & sînes stênes ich im vor mir niht gan.\\ 
 & \textbf{nû} lât \textbf{in} varn \textbf{an} sîn gemach."\\ 
20 & \textbf{a\textit{l} weinde} Parcifal dô sprach:\\ 
 & "\textbf{nû zeiget} mir, wâ der Grâl hie lige.\\ 
 & ob diu gotes güete an mir gesige,\\ 
 & des wirt wol innen disiu schar."\\ 
 & sîne venje \textbf{er viel} des endes dar\\ 
25 & drîstunt zêren der trinitât.\\ 
 & er warp, daz m\textit{üe}se werden rât\\ 
 & des trûrigen mannes herzesêr.\\ 
 & \textbf{dô stuont er} ûf und sprach mêr:\\ 
 & "œheim, waz wirret dir?"\\ 
30 & der durch Silvestern einen stier\\ 
\end{tabular}
\scriptsize
\line(1,0){75} \newline
U Q R \newline
\line(1,0){75} \newline
\textbf{1} \textit{Initiale} Q  \newline
\line(1,0){75} \newline
\textbf{2} erbiten] gebitten R \textbf{3} wirde] Wurde R \textbf{5} triuwe] truwen U trew_e \textit{nachträglich korrigiert zu:} trewe Q \textbf{6} siht] sich Q  $\cdot$ riuwe] ruͦwen U \textbf{7} ie prîs] pris ymer R \textbf{9} hin] do Q R  $\cdot$ zin den] zem R \textbf{10} enden] ende Q R \textbf{11} Parcifal] Parzifal U partzifal Q Parczifal R \textbf{12} an den] andem R \textbf{14} dâ] Do Q \textbf{15} ich engetar] Jchn dorff Q Jch getar R  $\cdot$ vürbaz] \textit{om.} R \textbf{18} sînes] Sin R  $\cdot$ stênes] sten Q  $\cdot$ im] hie Q \textbf{20} al] Alle U  $\cdot$ Parcifal] Parzifal U partzifal Q parczifal R  $\cdot$ dô] \textit{om.} R \textbf{21} hie] hin Q  $\cdot$ lige] liget R \textbf{22} gesige] geliget R \textbf{25} zêren] der eren Q \textbf{26} warp] erwarb R  $\cdot$ müese] muͦze U muste Q (R) \textbf{27} herzesêr] herzen ser Q \textbf{29} wirret] wirdet Q \textbf{30} durch] \textit{om.} Q  $\cdot$ Silvestern] siluestern U konigk siluestern Q Siluester R  $\cdot$ einen] eicicren Q \newline
\end{minipage}
\end{table}
\end{document}
