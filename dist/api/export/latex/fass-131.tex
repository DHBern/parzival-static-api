\documentclass[8pt,a4paper,notitlepage]{article}
\usepackage{fullpage}
\usepackage{ulem}
\usepackage{xltxtra}
\usepackage{datetime}
\renewcommand{\dateseparator}{.}
\dmyyyydate
\usepackage{fancyhdr}
\usepackage{ifthen}
\pagestyle{fancy}
\fancyhf{}
\renewcommand{\headrulewidth}{0pt}
\fancyfoot[L]{\ifthenelse{\value{page}=1}{\today, \currenttime{} Uhr}{}}
\begin{document}
\begin{table}[ht]
\begin{minipage}[t]{0.5\linewidth}
\small
\begin{center}*D
\end{center}
\begin{tabular}{rl}
\textbf{131} & \textbf{ouch} spranc der knappe wolgetân\\ 
 & von dem teppiche \textbf{an} daz bette sân.\\ 
 & \begin{large}D\end{large}iu \textbf{süeze kiusche unsamfte} erschrac,\\ 
 & dô der knappe an ir arme lac.\\ 
5 & si muost iedoch erwachen.\\ 
 & mit schame, alsunder lachen\\ 
 & diu vrouwe zuht gelêret\\ 
 & sprach: "wer hât mich entêret?\\ 
 & junchêrre, \textbf{es} ist iu gar ze vil,\\ 
10 & ir m\textit{ö}htet iu nemen ander \textbf{zil}."\\ 
 & diu vrouwe \textbf{lûte} klagete.\\ 
 & ern ruochte, waz si sagete,\\ 
 & ir munt er an den sînen twanc.\\ 
 & dâ nâch was dô niht ze lanc,\\ 
15 & \textbf{ê} er druhte an sich die herzogîn\\ 
 & \textbf{unt} nam ir ouch ein vingerlîn.\\ 
 & an ir hemde ein vürspan er \textbf{dâ} \textbf{sach}.\\ 
 & ungevuoge erz dannen brach.\\ 
 & diu vrouwe was mit wîbes wer.\\ 
20 & ir was sîn kraft ein ganze\textit{z} her,\\ 
 & doch wart \textbf{dâ ringens vil} getân.\\ 
 & Der knappe klagete \textbf{den} hunger sân.\\ 
 & diu vrouwe was ir \textbf{liebes lîht}.\\ 
 & si sprach: "ir sult mîn ezzen niht.\\ 
25 & wært ir ze vrumen wîse,\\ 
 & ir næmet iu ander spîse.\\ 
 & dort stêt brôt und wîn\\ 
 & unt \textbf{ouch} zwei \textbf{pardrîsekîn},\\ 
 & als\textbf{s} ein juncvrouwe brâhte,\\ 
30 & dius wênec iu gedâhte."\\ 
\end{tabular}
\scriptsize
\line(1,0){75} \newline
D Fr13 \newline
\line(1,0){75} \newline
\textbf{3} \textit{Initiale} D  \textbf{22} \textit{Majuskel} D  \newline
\line(1,0){75} \newline
\textbf{10} möhtet] mohtet D \textbf{15} Er her:::n sich dr:::herzogin Fr13 \textbf{16} ouch] \textit{om.} Fr13 \textbf{17} dâ] \textit{om.} Fr13 \textbf{18} ungevuoge] vnfugelichen Fr13 \textbf{20} ganzez] ganzes D \textbf{23} was] wart Fr13 \textbf{24} mîn] mich Fr13 \textbf{27} dort] Seht dort Fr13 \newline
\end{minipage}
\hspace{0.5cm}
\begin{minipage}[t]{0.5\linewidth}
\small
\begin{center}*m
\end{center}
\begin{tabular}{rl}
 & \textbf{ouch} spranc der knappe wol getân\\ 
 & vonme teppich \textbf{an} daz bette sân.\\ 
 & diu \textbf{kiusche süeze unsanfte} erschrac,\\ 
 & dô der knappe an ir arme lac.\\ 
5 & si muoste iedoch erwachen\\ 
 & mit scham, alsunder lachen.\\ 
 & \begin{large}D\end{large}iu vrouwe zuht gelêret\\ 
 & sprach: "wer hât mich entêret?\\ 
 & junchêr, \textbf{es} ist iu gar ze vil,\\ 
10 & ir m\textit{ö}htet iu nemen \textbf{ein} ander \textbf{zil}."\\ 
 & \textbf{wie vil} diu vrouwe klagete,\\ 
 & er enruochete, waz si sagete,\\ 
 & ir munt er an den sînen twanc.\\ 
 & dâ nâch was dô niht zuo lanc,\\ 
15 & er \textbf{en}druhte an sich die herzogîn\\ 
 & \textbf{und} nam ir ouch ein vingerlîn.\\ 
 & an ir hemede ein vürspange er \textbf{gesach}.\\ 
 & ungevuoge er ez dannen brach.\\ 
 & diu vrouwe was mit wîbes wer.\\ 
20 & ir was sîn kraft ein ganzez her,\\ 
 & doch wart \textbf{d\textit{â} ringe\textit{n}s vil} getân.\\ 
 & der knappe klagete hunger sân.\\ 
 & diu vrouwe was ir \textbf{lîbes lieht}.\\ 
 & si sprach: "ir sullet mîn ezzen niht.\\ 
25 & wæret ir ze vromen wîse,\\ 
 & ir næmet iu andere spîse.\\ 
 & dort stât brôt und wîn\\ 
 & und zwei \textbf{pardrîsekîn},\\ 
 & als \textbf{si} ein juncvrouwe brâhte,\\ 
30 & diu es wênic iu gedâhte."\\ 
\end{tabular}
\scriptsize
\line(1,0){75} \newline
m n o \newline
\line(1,0){75} \newline
\textbf{7} \textit{Initiale} m n  \newline
\line(1,0){75} \newline
\textbf{4} ir] iren n \textbf{5} si] So o  $\cdot$ muoste] muͯste n o \textbf{8} mich] [mir]: mich n \textbf{9} vil] [wil]: vil o \textbf{10} möhtet] mochttent m \textbf{11} vil] wil o \textbf{12} si] [ist]: sie o \textbf{14} dô] \textit{om.} n o \textbf{15} endruhte] druchte n [duͯcht]: druͯcht o  $\cdot$ herzogîn] herczogen o \textbf{17} vürspange] fúrspan n (o) \textbf{18} ez] des o  $\cdot$ dannen] dennen n o \textbf{21} dâ ringens] do ringes m o do ringens n \textbf{22} hunger] [suͯnder]: huͯnder o \textbf{23} lieht] licht n (o) \textbf{26} iu] auch o \newline
\end{minipage}
\end{table}
\newpage
\begin{table}[ht]
\begin{minipage}[t]{0.5\linewidth}
\small
\begin{center}*G
\end{center}
\begin{tabular}{rl}
 & \textbf{dô} spranc der knappe wolgetân\\ 
 & von dem tepech \textbf{ûf} daz bette sân.\\ 
 & diu \textbf{süeze kiusche unsanfte} erschrac,\\ 
 & dô der knappe an ir arme lac.\\ 
5 & si muose iedoch erwachen.\\ 
 & \textit{mit} schame\textit{n}, \textit{al}sunder lachen\\ 
 & diu vrouwe zuht gelêret\\ 
 & sprach: "wer hât mich entêret?\\ 
 & junchêrre, \textbf{es} ist iu gar ze vil,\\ 
10 & ir m\textit{ö}ht iu nemen \textbf{ein} ander \textbf{spil}."\\ 
 & diu vrouwe \textbf{lûte} klagte.\\ 
 & er enruohte, waz si sagte,\\ 
 & ir munt er an den sînen twanc.\\ 
 & dar nâch was dô niht ze lanc,\\ 
15 & er druckt an sich die herzogîn\\ 
 & \textbf{unde} nam ir ouch \textit{e}i\textit{n} vingerlîn.\\ 
 & an ir hemde ein vürspan er \textbf{dô} \textbf{sach}.\\ 
 & ungevuoge erz danen brach.\\ 
 & \begin{large}D\end{large}iu vrouwe was mit wîbes wer.\\ 
20 & ir was sîn kraft ein ganzez her,\\ 
 & doch wart \textbf{vil ringens dâ} getân.\\ 
 & der knappe klagte hunge\textit{r} sân.\\ 
 & diu vrouwe was ir \textbf{lîbes lieht}.\\ 
 & si sprach: "ir sult mîn ezzen niht.\\ 
25 & wæret ir ze vrumen wîse,\\ 
 & ir næmet iu ander spîse.\\ 
 & dort stêt brôt und wîn\\ 
 & unt zwei \textbf{rephuonlîn},\\ 
 & als ein juncvrouwe brâhte,\\ 
30 & dius wênic iu gedâhte."\\ 
\end{tabular}
\scriptsize
\line(1,0){75} \newline
G I O L M Q R Z Fr35 \newline
\line(1,0){75} \newline
\textbf{5} \textit{Initiale} Q Fr35  \textbf{11} \textit{Initiale} I  \textbf{19} \textit{Initiale} G  \newline
\line(1,0){75} \newline
\textbf{1} \textit{Vers 131.1 fehlt} Q   $\cdot$ dô] Da M \textbf{2} ûf] an L \textbf{3} süeze kiusche] kunsche suͯssen R \textbf{4} der knappe an ir] ir der knappe an dem L \textbf{6} Mit scham begvnde sie lachen O  $\cdot$ mit schamen] alschamende G Mit scham L (Q) (R) Z (Fr35)  $\cdot$ alsunder] sunder G Z oͯn sunder R \textbf{8} sprach] Vnde sprach O \textbf{9} es ist iu] iv ist sin O ez ist L \textbf{10} möht] moht G (O) (L) (M) (R) Z moch Q  $\cdot$ ein] \textit{om.} M Q R Z Fr35  $\cdot$ ander] andern M \textbf{12} enruohte] ern rucht I (Z) \textbf{13} an] an an L \textbf{14} was] en was I  $\cdot$ dô] ez I noch O doch M Z  $\cdot$ ze] \textit{om.} L \textbf{15} er] Er en M  $\cdot$ druckt] dructe I (L) M (Fr35)  $\cdot$ sich] sy R \textbf{16} ir ouch] ouch L ir M auch ir Q  $\cdot$ ein] ir G \textbf{17} ein vürspan er] er ein fvͦrspan O (M) (Q) (Fr35) ein fuͯrspang er L er ein fᵫrspang R  $\cdot$ dô] \textit{om.} I O L M Q R Fr35 da Z  $\cdot$ sach] stach Q fand R \textbf{18} ungevuoge] Mit vnfuͦg R  $\cdot$ erz] er daz L  $\cdot$ danen] dar abe M  $\cdot$ brach] schwand R \textbf{19} wîbes] wibe O L \textbf{20} sîn] sin eines L  $\cdot$ ganzez] \textit{om.} L \textbf{21} doch] Do O Durch M  $\cdot$ wart] wer Q was R  $\cdot$ ringens] ringes L M Q R  $\cdot$ dâ] do Q R \textbf{22} hunger] hungern G \textbf{23} lieht] lecht L (M) (Q) (Z) \textbf{24} sprach] en sprach M (R)  $\cdot$ ir] irn I (M) Fr35  $\cdot$ mîn] ::: \textit{nachträglich korrigiert zu:} mich O mich R  $\cdot$ niht] meht L \textbf{25} wæret] Wer Z  $\cdot$ ze vrumen] icht frúmes Q (R) (Fr35) \textbf{26} næmet] menet Q  $\cdot$ ander] ein ander I \textbf{27} brôt] bro R \textbf{28} unt] vnd auch I (O) (L) (M) (Q) (Z)  $\cdot$ rephuonlîn] parelin I legelin O M bardriesekin L (Z) pardisekin Q par túblin R \textbf{29} als] Als ez I (Q) Alz sie L \textbf{30} dius wênic iu] diu ev sin wenc I (O) Disz uch wenning M (R) Die es auch wenick Q  $\cdot$ gedâhte] dedacht R \newline
\end{minipage}
\hspace{0.5cm}
\begin{minipage}[t]{0.5\linewidth}
\small
\begin{center}*T (U)
\end{center}
\begin{tabular}{rl}
 & \textbf{dô} spranc der knappe wol getân\\ 
 & von dem teppich \textbf{ûf} daz bette sân.\\ 
 & diu \textbf{süeze, unsenfte} erschrac,\\ 
 & dô der knappe an ir arme lac.\\ 
5 & si muos iedoch erwachen.\\ 
 & mit schame, al sunder lachen\\ 
 & diu vrouw\textit{e} zuht gelêret\\ 
 & sprach: "wer hât mich entêre\textit{t}?\\ 
 & junchêrre, \textbf{des} ist iu gar zuo vil,\\ 
10 & ir m\textit{ö}htet iu nemen \textbf{ein} ander \textbf{spil}."\\ 
 & diu vrouwe \textbf{sêre} klagete.\\ 
 & er enruochte, waz si sagete,\\ 
 & ir munt er an den sînen twanc.\\ 
 & dar nâch was dô niht zuo lanc,\\ 
15 & er dructe an sich die herzogîn.\\ 
 & \textbf{er} nam ir ouch ein vingerlîn.\\ 
 & an ir hemede ein vürspan er \textbf{sach}.\\ 
 & ungevuoge er ez dannen brach.\\ 
 & diu vrouwe was mit wîbes wer.\\ 
20 & ir was sîn kraft ein ganzez her,\\ 
 & doch wart \textbf{vil ringens dâ} getân.\\ 
 & der knappe klagete \textbf{den} hunger sân.\\ 
 & diu vrouwe was ir \textbf{lîbes lieht}.\\ 
 & si sprach: "ir\textbf{n} solt mîn ezzen niht.\\ 
25 & wæret ir zuo vromen wîse,\\ 
 & ir næmet iu ander spîse.\\ 
 & dort stêt brôt und wîn\\ 
 & und \textbf{ouch} zwei \textbf{pardrîsekîn},\\ 
 & als ein juncvrouwe \textbf{ez} brâhte,\\ 
30 & dius wênic iu gedâhte."\\ 
\end{tabular}
\scriptsize
\line(1,0){75} \newline
U V W T \newline
\line(1,0){75} \newline
\textbf{1} \textit{Initiale} W   $\cdot$ \textit{Majuskel} T  \textbf{3} \textit{Majuskel} T  \textbf{7} \textit{Majuskel} T  \textbf{12} \textit{Majuskel} T  \textbf{17} \textit{Majuskel} T  \textbf{19} \textit{Majuskel} T  \textbf{22} \textit{Majuskel} T  \textbf{23} \textit{Majuskel} T  \newline
\line(1,0){75} \newline
\textbf{3} süeze unsenfte] svͤze kv́sche vnsanfte V (W) (T) \textbf{5} iedoch] doch T \textbf{6} al] \textit{om.} T \textbf{7} vrouwe] vreuͦwen U \textbf{8} entêret] enteren U suß enteret W \textbf{9} des] [*]: daz V das W ez T  $\cdot$ gar] \textit{om.} V \textbf{10} möhtet] mocht U (T) \textbf{11} sêre] laute W (T) \textbf{14} was dô] waz V wart T \textbf{15} er] Ern V \textbf{16} er] vnd T \textbf{17} ein vürspan er] [*]: ein fv́rspan er V er ein vurspan T \textbf{18} ungevuoge] mit vnvuͦge T  $\cdot$ ez] in V \textbf{20} ganzez] kranckes W \textbf{21} vil ringens dâ] vil ringens do V W da ringens vil T \textbf{22} klagete] [*er]: der klagete V  $\cdot$ den] \textit{om.} T \textbf{23} was ir] die waz ir V was des W  $\cdot$ lieht] licht W \textbf{24} irn] ir V W T  $\cdot$ mîn] mein doch W \textbf{28} pardrîsekîn] hardisgin W \textbf{29} ein juncvrouwe ez] es ein iunckfrauwe W ein ivncvroͮwe T \textbf{30} wênic iu] eúch wenig W (T) \newline
\end{minipage}
\end{table}
\end{document}
