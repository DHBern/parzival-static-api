\documentclass[8pt,a4paper,notitlepage]{article}
\usepackage{fullpage}
\usepackage{ulem}
\usepackage{xltxtra}
\usepackage{datetime}
\renewcommand{\dateseparator}{.}
\dmyyyydate
\usepackage{fancyhdr}
\usepackage{ifthen}
\pagestyle{fancy}
\fancyhf{}
\renewcommand{\headrulewidth}{0pt}
\fancyfoot[L]{\ifthenelse{\value{page}=1}{\today, \currenttime{} Uhr}{}}
\begin{document}
\begin{table}[ht]
\begin{minipage}[t]{0.5\linewidth}
\small
\begin{center}*D
\end{center}
\begin{tabular}{rl}
\textbf{171} & - daz ist \textbf{ein} unsüeze arbeit -,\\ 
 & dem sult ir helfe sîn bereit.\\ 
 & swenne ir dem tuot kumbers buoz,\\ 
 & sô nâhet iu der gotes gruoz.\\ 
5 & im ist noch wirs denne \textbf{den}, die gênt\\ 
 & nâch \textbf{porte}, \textbf{al} dâ diu venster stênt.\\ 
 & \begin{large}I\end{large}r sult bescheidenlîche\\ 
 & sîn arm unt rîche,\\ 
 & wan swâ der hêrre gar vertuot,\\ 
10 & daz ist niht hêrrenlîcher muot.\\ 
 & samnet er aber \textbf{schaz} \textbf{ze sêre},\\ 
 & daz sint \textbf{ouch} unêre.\\ 
 & gebt rehter mâze ir orden.\\ 
 & ich bin wol innen worden,\\ 
15 & daz ir râtes \textbf{dürftic} sît.\\ 
 & nû lât der \textbf{unvuoge} \textbf{ir} strît.\\ 
 & Ir \textbf{en}sult niht vil gevrâgen.\\ 
 & ouch \textbf{en}sol iuch niht betrâgen\\ 
 & \textbf{bedâhter gegenrede}, diu gê\\ 
20 & reht, als jenes \textbf{vrâgen} stê,\\ 
 & \textbf{der} \textbf{iuch} wil mit worten spehen.\\ 
 & ir kunnet hœren unt sehen,\\ 
 & entseben unt \textit{d}ræhen,\\ 
 & daz \textbf{solt} iuch \textbf{witzen} næhen.\\ 
25 & lât \textbf{die} \textbf{erbärme} bî der vrevel sîn.\\ 
 & sus tuot mir râtes volge schîn:\\ 
 & An swem ir strîtes sicherheit\\ 
 & bezalt, ern hab iu \textbf{sölhiu} leit\\ 
 & getân, diu herzekumber wesen,\\ 
30 & di\textit{e} nemt und lâzet in genesen.\\ 
\end{tabular}
\scriptsize
\line(1,0){75} \newline
D \newline
\line(1,0){75} \newline
\textbf{7} \textit{Initiale} D  \textbf{17} \textit{Majuskel} D  \textbf{27} \textit{Majuskel} D  \newline
\line(1,0){75} \newline
\textbf{23} dræhen] bræhen D \textbf{30} die] div D \newline
\end{minipage}
\hspace{0.5cm}
\begin{minipage}[t]{0.5\linewidth}
\small
\begin{center}*m
\end{center}
\begin{tabular}{rl}
 & - daz ist \textbf{ein} unsüeze arbeit -,\\ 
 & dem sullet ir helfe sîn bereit.\\ 
 & wenne \textit{i}r de\textit{m} t\textit{u}o\textit{t k}umbers buoz,\\ 
 & sô nâhet iu der gotes gruoz.\\ 
5 & ime ist noch \dag wurst\dag  danne die gênt\\ 
 & nâch \textbf{br\textit{ô}te}, \textbf{al} dâ diu venster stênt.\\ 
 & ir sullet bescheidenlîche\\ 
 & sîn arm und rîche,\\ 
 & wan wâ der hêrre gar vertuot,\\ 
10 & daz ist niht hêrrenlîcher muot.\\ 
 & samene\textit{t} er aber \textbf{schatz} \textbf{ze sêre},\\ 
 & daz sint \textbf{ouch} unêre.\\ 
 & gebet rehter mâze ir orden.\\ 
 & ich bin wol innen worden,\\ 
15 & daz ir râtes \textbf{bedürftic} sît.\\ 
 & nû lât der \textbf{ungevuoge} \textbf{ir} strît.\\ 
 & ir sullet niht vil gevrâgen.\\ 
 & ouch sol iuch niht betrâgen\\ 
 & \textbf{bedâhter gegenrede}, \textit{diu g}ê\\ 
20 & reht, als jenes \textbf{vrâgen} stê,\\ 
 & \textbf{der} \textbf{iuch} \textbf{dâ} wil mit worten spehen.\\ 
 & ir kunnet hœren und sehen,\\ 
 & ent\textit{s}eben und \textit{d}ræhen,\\ 
 & daz \textbf{solte} iuch \textbf{witzen} næhen.\\ 
25 & lât \textbf{erbermde} bî der vrevel sîn.\\ 
 & sus tuot mir râtes volge schîn:\\ 
 & an wem ir strîtes sicherheit\\ 
 & bezalt, er enhabe iu \textbf{solich} leit\\ 
 & getân, diu herze\textit{k}umber wesen,\\ 
30 & die nemt und lâzet in genesen.\\ 
\end{tabular}
\scriptsize
\line(1,0){75} \newline
m n o Fr69 \newline
\line(1,0){75} \newline
\newline
\line(1,0){75} \newline
\textbf{3} Wenne mir den tot den kumbers buͦs m \textbf{4} sô] Do o \textbf{6} brôte] bratte m brahte o \textbf{8} und] oder n \textbf{9} wâ] do n \textbf{11} samenet] Samene m  $\cdot$ schatz] stacz o \textbf{14} bin wol] wol [j]: bin n [w]: bin wol o \textbf{15} bedürftic] durfftig n (o) \textbf{16} ungevuoge] vnfuͯge n \textbf{19} diu gê] erge m \textbf{21} dâ] do n o \textbf{23} entseben] entleben \textit{(krit. Text emendiert nach V#'* ͫ)} m Git leben n (o)  $\cdot$ dræhen] entrehen m \textbf{24} iuch] auch o \textbf{26} sus] Das n o  $\cdot$ mir] mit n o  $\cdot$ râtes] \textit{om.} o \textbf{29} herzekumber] hercze tumber m herczen kommer o \textbf{30} in] úch n \newline
\end{minipage}
\end{table}
\newpage
\begin{table}[ht]
\begin{minipage}[t]{0.5\linewidth}
\small
\begin{center}*G
\end{center}
\begin{tabular}{rl}
 & - daz ist unsüeziu arbeit -,\\ 
 & dem sult ir helfe sîn bereit.\\ 
 & swenne ir dem tuot kumbers buoz,\\ 
 & sô nâhet iu der gotes gruoz.\\ 
5 & im ist noch wirs denne \textbf{den}, die gênt\\ 
 & nâch \textbf{brôte}, \textbf{al} dâ diu venster stênt.\\ 
 & ir sult bescheidenlîche\\ 
 & sîn arm unde rîche,\\ 
 & \textit{wan} swâ der hêrre gar vertuot,\\ 
10 & daz ist niht hêrlîcher muot.\\ 
 & sament er aber \textbf{schatzes} \textbf{\textit{ê}re},\\ 
 & daz sint \textbf{ouch} unêre.\\ 
 & gebet rehter mâze ir orden.\\ 
 & ich bin wol innen worden,\\ 
15 & daz ir râtes \textbf{dürftic} sît.\\ 
 & \textit{nû} lât der \textbf{ungevuoge} \textbf{ir} strît.\\ 
 & \begin{large}I\end{large}r sult niht vil gevrâgen.\\ 
 & ouch sol iuch niht betrâgen\\ 
 & \textbf{bedâhter gegenrede}, diu gê\\ 
20 & rehte, als jenes \textbf{vrâge} stê,\\ 
 & \textbf{swer} \textbf{iuch} wil mit worten spehen.\\ 
 & ir kunnet hœren unde sehen,\\ 
 & entsebe\textit{n} unde dræhen,\\ 
 & daz \textbf{sol} iuch \textbf{witzen} næhen.\\ 
25 & lât \textbf{die} \textbf{erbärme} bî der vrevele sîn.\\ 
 & sus tuot mir râtes volge schîn:\\ 
 & an swem ir strîtes sicherheit\\ 
 & bezelt, erne habe iu \textbf{solch} leit\\ 
 & getân, diu herzen kumber wesen,\\ 
30 & die nemet unde lât in genesen.\\ 
\end{tabular}
\scriptsize
\line(1,0){75} \newline
G I O L M Q R Z Fr21 \newline
\line(1,0){75} \newline
\textbf{1} \textit{Initiale} Q  \textbf{7} \textit{Initiale} I O L Z Fr21  \textbf{17} \textit{Initiale} G  \newline
\line(1,0){75} \newline
\textbf{1} daz] Do Q  $\cdot$ unsüeziu] ein vnsvͦzev O (Q) (R) eyn susze M (Fr21) \textbf{2} helfe] helffin M heffen R \textbf{3} swenne ir] wan swenn ir I (Z) Wan so ir O Fr21 Wenne ir L (Q) Wan wannir M Wen wen er R  $\cdot$ dem] den O \textbf{4} der] \textit{om.} Fr21 \textbf{5} im ist] Jn is M (Z) Ym Q  $\cdot$ noch] noch noch I  $\cdot$ denne den] danne O (Q) Fr21 noch den R  $\cdot$ die] di do O \textbf{6} al dâ] da I L M R Z Fr21 do O Q  $\cdot$ venster] venstern L \textbf{7} ir] ÷r O \textbf{8} arm unde rîche] armen vnde riehen O armen vnd Riche R \textbf{9} wan] \textit{om.} G  $\cdot$ swâ] wo L M Q (R) \textbf{10} ist] en ist M \textbf{11} sament] Samlet R  $\cdot$ er aber] ir aber O aber er Fr21  $\cdot$ schatzes êre] schatzes mære G schatz zuͯ sere L (M) schatz sere Fr21 \textbf{12} sint] ist Fr21 \textbf{13} gebet] Gelt R \textbf{15} râtes] dratis M  $\cdot$ dürftic] durffic M (Fr21) durffent R \textbf{16} nû] \textit{om.} G  $\cdot$ ungevuoge] vnfuͯge L (Q) (Z) vngefugen M \textbf{17} Ir] irn I (M) (Z) (Fr21)  $\cdot$ niht] oͮch niht Fr21 \textbf{18} sol] ensol I (O) R Z Fr21 \textbf{19} bedâhter] Gedahter I Behaiter Z  $\cdot$ gegenrede] rede gein der O  $\cdot$ gê] gerecht als ienes Q ste Z \textbf{20} jenes] ens O eines Z  $\cdot$ vrâge] vragen I (Z)  $\cdot$ stê] steen Q ge Z \textbf{21} swer] Wer L M Q R Der Z  $\cdot$ iuch wil] wil euch Q (R) \textbf{22} kunnet] kvnnet wol Z \textit{om.} Fr21 \textbf{23} entseben] entsebe G entsten I Entsehen O (Fr21)  $\cdot$ dræhen] draben L \textbf{24} sol] solde M (Q) (R) (Z)  $\cdot$ witzen] witze L (M) Q (R) witzen vnd Z \textbf{25} lât] Daz lat Z  $\cdot$ die] \textit{om.} I L ewer Z  $\cdot$ erbärme] [erbar*]: erbarmde I erbærmde O (L) (Q) (R) (Z) erbarmunge M  $\cdot$ der] dem R \textbf{26} volge] wolge Q \textbf{27} swem] swen I wem L (M) Q R  $\cdot$ ir] ist M \textbf{28} erne] er O Q  $\cdot$ iu] auch Q  $\cdot$ solch leit] hertzeleit L \textbf{29} diu] daz ez muge ân des I \textbf{30} die] Du R \newline
\end{minipage}
\hspace{0.5cm}
\begin{minipage}[t]{0.5\linewidth}
\small
\begin{center}*T
\end{center}
\begin{tabular}{rl}
 & - daz ist \textbf{ein} unsüeze arbeit -,\\ 
 & dem sult ir helfe sîn bereit.\\ 
 & \textbf{wan} swenne ir dem tuot kumbers buoz,\\ 
 & sô nâhet iu der gotes gruoz.\\ 
5 & im ist noch wirs danne \textbf{den}, die gânt\\ 
 & nâch \textbf{brôte}, dâ diu venster stânt.\\ 
 & Ir sult bescheidenlîche\\ 
 & sîn arm unde rîche,\\ 
 & wan swâ der hêrre gar vertuot,\\ 
10 & daz \textbf{en}ist niht hêrlicher muot.\\ 
 & samnet er aber \textbf{schatz} \textbf{mêre},\\ 
 & daz sint \textbf{grôz} unêre.\\ 
 & gebt rehter mâze ir orden.\\ 
 & ich bin wol innen worden,\\ 
15 & daz ir râtes \textbf{dürftic} sît.\\ 
 & nû lât der \textbf{unvuoge} strît.\\ 
 & Ir sult niht vil gevrâgen.\\ 
 & ouch sol iuch niht betrâgen\\ 
 & \textbf{beidenthalp diu rede}, diu gê\\ 
20 & rehte, als jenes \textbf{vrâgen} stê,\\ 
 & \textbf{swer} \textbf{iuwer} wil mit worten spehen.\\ 
 & ir kunnet hœren unde sehen,\\ 
 & entseben unde dræhen,\\ 
 & daz \textbf{sol} iuch \textbf{witze} næhen.\\ 
25 & lât \textbf{die} \textbf{erbermede} bî der vrevel sîn.\\ 
 & sus tuot mir râtes volge schîn:\\ 
 & an swem ir strîtes sicherheit\\ 
 & bezalt, ern habiu leit\\ 
 & getân, di\textit{u} herzen kumber wesen,\\ 
30 & die nemet unde lât in genesen.\\ 
\end{tabular}
\scriptsize
\line(1,0){75} \newline
T U V W \newline
\line(1,0){75} \newline
\textbf{7} \textit{Majuskel} T  \textbf{13} \textit{Initiale} W  \textbf{17} \textit{Majuskel} T  \newline
\line(1,0){75} \newline
\textbf{3} swenne] wen U (W)  $\cdot$ dem] den W \textbf{5} den die] [*]: danne die V die do W \textbf{6} dâ] [*]: aldo V do W  $\cdot$ diu] die T \textbf{8} arm unde rîche] armen vnde richen V \textbf{9} swâ] wo U do W \textbf{10} enist] ist W \textbf{11} er aber] aber er U V  $\cdot$ schatz] schatzes W  $\cdot$ mêre] [*]: ze sere V \textbf{12} sint] seint auch W \textbf{15} râtes] ritter W  $\cdot$ dürftic] dorfte U dúrstig W \textbf{16} der unvuoge] der vnvuͦge irn U (V) den vngefuͤge W \textbf{17} \textit{Versfolge 171.18-17} V   $\cdot$ sult] soͤllen V  $\cdot$ niht] \textit{om.} U \textbf{19} Ir gedencken wie eúwer gegen rede W  $\cdot$ beidenthalp diu] [*]: Bedahter gegenrede V  $\cdot$ diu gê] die ge T \textbf{20} Recht vnd als iens frage geste W  $\cdot$ als jenes] als eins U \textbf{21} swer iuwer] Wer vch wer U Swer v́ch V Wer eúcb W \textbf{22} Vnd eúwer sitten kúnnen vnd sehen W \textbf{23} [Ent*ben]: Entzeben vnde [*ehen]: trehen V · Dem gond mit schoͤnen zúchten nach W \textbf{24} daz [sol*]: solte v́ch witzen nehen V · Lat eúch mit rede nit zuͦ gach W \textbf{25} erbermede bî] [*]: erbermede bi V bern mit W  $\cdot$ der] dem U \textit{om.} W \textbf{27} swem] wem U W \textbf{28} habiu] habe U habe [*]: v́ch soliche V habe eúch soͤlches W \textbf{29} diu] die T  $\cdot$ herzen] herze V \newline
\end{minipage}
\end{table}
\end{document}
