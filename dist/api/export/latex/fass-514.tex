\documentclass[8pt,a4paper,notitlepage]{article}
\usepackage{fullpage}
\usepackage{ulem}
\usepackage{xltxtra}
\usepackage{datetime}
\renewcommand{\dateseparator}{.}
\dmyyyydate
\usepackage{fancyhdr}
\usepackage{ifthen}
\pagestyle{fancy}
\fancyhf{}
\renewcommand{\headrulewidth}{0pt}
\fancyfoot[L]{\ifthenelse{\value{page}=1}{\today, \currenttime{} Uhr}{}}
\begin{document}
\begin{table}[ht]
\begin{minipage}[t]{0.5\linewidth}
\small
\begin{center}*D
\end{center}
\begin{tabular}{rl}
\textbf{514} & \textbf{\begin{large}D\end{large}er} sprach: "welt ir râtes pflegen,\\ 
 & ir sult \textbf{dises pferdes iuch} bewegen.\\ 
 & ez enwert iu doch niemen hie;\\ 
 & getât \textbf{ab} ir daz wægeste ie,\\ 
5 & sô sult irz \textbf{pfert} hie lâzen.\\ 
 & mîn vrouwe sî verwâzen,\\ 
 & daz si sô manegen werden man\\ 
 & von dem lîbe scheiden kan."\\ 
 & Gawan sprach, er\textbf{n} liezes niht.\\ 
10 & "ouwê \textbf{des} \textbf{dâ nâch} geschiht!",\\ 
 & sprach der grâwe rîter wert.\\ 
 & die halfteren lôster vome pfert.\\ 
 & Er sprach: "ir sult niht langer stên;\\ 
 & lât diz pfert nâch iu gên.\\ 
15 & des hant daz mer gesalzen hât,\\ 
 & der geb iu vür kumber rât.\\ 
 & hüet, daz iuch iht gehœne\\ 
 & mîner vrouwen schœne,\\ 
 & wan \textbf{diu} ist bî der süeze \textbf{al} sûr,\\ 
20 & reht als ein sunnenblicker schûr."\\ 
 & "Nû walt\textbf{s} got", sprach Gawan.\\ 
 & urloup nam \textbf{er} zem grâwen man;\\ 
 & als tet er hie unt dort.\\ 
 & si sprâchen alle klagend\textit{iu} wort.\\ 
25 & Daz pfert gie einen smalen \textbf{wec}\\ 
 & zer porten ûz nâch im ûf den \textbf{stec}.\\ 
 & sînes herzen vogt er dâ vant;\\ 
 & diu was vrouwe über daz lant.\\ 
 & swie sîn herze gein ir vlôch,\\ 
30 & vil kumbers si im \textbf{doch} drîn \textbf{gezôch}.\\ 
\end{tabular}
\scriptsize
\line(1,0){75} \newline
D \newline
\line(1,0){75} \newline
\textbf{1} \textit{Initiale} D  \textbf{13} \textit{Majuskel} D  \textbf{21} \textit{Majuskel} D  \textbf{25} \textit{Majuskel} D  \newline
\line(1,0){75} \newline
\textbf{24} klagendiu] chlagende D \newline
\end{minipage}
\hspace{0.5cm}
\begin{minipage}[t]{0.5\linewidth}
\small
\begin{center}*m
\end{center}
\begin{tabular}{rl}
 & \textbf{er} sprach: "welt ir râtes pflegen,\\ 
 & ir solt \textbf{des pferdes iuch} bewegen.\\ 
 & ez enwert iu doch niemen hie;\\ 
 & getâtet \textbf{aber} ir daz wægest ie,\\ 
5 & sô solt irz hie lâzen.\\ 
 & mîn vrouwe sî verwâzen,\\ 
 & daz si sô manigen werden man\\ 
 & von dem lîbe scheiden kan."\\ 
 & Gawan sprach, er liez es niht.\\ 
10 & "owê \textbf{des} \textbf{dannoch} geschiht!",\\ 
 & sprach der grâwe ritter wert.\\ 
 & die halftern lôste er von dem pfert.\\ 
 & er sprach: "ir solt niht langer stân;\\ 
 & lât diz pfert nâch iu gân.\\ 
15 & des hant daz mer gesalzen hât,\\ 
 & der gebe iu vür kumber rât.\\ 
 & hüetet, daz iuc\textit{h} iht gehœne\\ 
 & mîner vrouwen schœne,\\ 
 & wan \textbf{si} ist bî der süeze sûr,\\ 
20 & reht als ein sunnenblicker schûr."\\ 
 & "nû walt \textbf{sîn} got", sprach \textbf{hêr} Gawan.\\ 
 & urloup nam \textbf{der} zuom grâwen man;\\ 
 & als tet er hie und dort.\\ 
 & si sprâchen alle klagendiu wort.\\ 
25 & daz pfert gienc einen smalen \textbf{wec}\\ 
 & zuo \textit{der} porten ûz nâch im ûf den \textbf{stec}.\\ 
 & sînes herzen vogt er d\textit{â} vant;\\ 
 & diu was vrouwe über daz lant.\\ 
 & wie sîn herze gegen ir vlôch,\\ 
30 & vil kumbers si im dar în \textbf{zôch}.\\ 
\end{tabular}
\scriptsize
\line(1,0){75} \newline
m n o \newline
\line(1,0){75} \newline
\newline
\line(1,0){75} \newline
\textbf{2} des] disz n das o \textbf{5} hie] [jie]: hie o \textbf{9} liez] liesse n \textbf{12} pfert] ::eert o \textbf{15} gesalzen] [geseltzen]: gesaltzen n \textbf{16} kumber] kemer o \textbf{17} hüetet] Huͯten n  $\cdot$ iuch] uͯcht m \textbf{19} süeze] suͯssen o \textbf{20} sunnenblicker] suͯnnen blicken o \textbf{21} hêr] \textit{om.} n o \textbf{22} der] er n o \textbf{25} wec] [steck]: weg o \textbf{26} der] \textit{om.} m \textbf{27} dâ] do m n o \newline
\end{minipage}
\end{table}
\newpage
\begin{table}[ht]
\begin{minipage}[t]{0.5\linewidth}
\small
\begin{center}*G
\end{center}
\begin{tabular}{rl}
 & \textbf{\begin{large}E\end{large}r} sprach: "welt ir râtes pflegen,\\ 
 & ir sult \textbf{disses pferdes iuch} bewegen.\\ 
 & ezn wert iu doch niemen hie;\\ 
 & getât \textbf{aber} ir daz wægest ie,\\ 
5 & sô sult irz hie lâzen.\\ 
 & mîn vrouwe sî verwâzen,\\ 
 & daz si sô manigen werden man\\ 
 & vonem lîbe scheiden kan."\\ 
 & Gawan sprach, er liezes niht.\\ 
10 & "owê \textbf{des} \textbf{danne dâ nâch} geschiht!",\\ 
 & sprach der grâwe rîter wert.\\ 
 & die halftern lôst er von dem pfert.\\ 
 & er sprach: "ir sult niht lenger stên;\\ 
 & lât ditze pfert nâch iu gên.\\ 
15 & des hant daz mer gesalzen hât,\\ 
 & der gebe iu vür kumber rât.\\ 
 & hüete\textit{t}, daz iuch iht gehœne\\ 
 & mîner vrouwen schœne,\\ 
 & wan \textbf{diu} ist bî der süeze \textbf{al} sûr,\\ 
20 & rehte als ein sunnenblicker schûr."\\ 
 & "nû walte\textbf{s} got", sprach Gawan.\\ 
 & urloup nam \textbf{er} ze dem grâwen man;\\ 
 & als tet er hie unde dort.\\ 
 & si sprâchen alle klagendiu wort.\\ 
25 & daz pfert gie einen smalen \textbf{wec}\\ 
 & ze de\textit{r} porten ûz nâch im ûf de\textit{n} \textbf{stec}.\\ 
 & sînes herzen voget er dâ vant;\\ 
 & diu was vrouwe überz lant.\\ 
 & swie sîn herze gein ir vlôch,\\ 
30 & vil \textit{kumbers} sim \textbf{doch} drîn \textbf{zôch}.\\ 
\end{tabular}
\scriptsize
\line(1,0){75} \newline
G I L M Z \newline
\line(1,0){75} \newline
\textbf{1} \textit{Initiale} G L Z  \textbf{13} \textit{Initiale} I  \newline
\line(1,0){75} \newline
\textbf{2} disses pferdes iuch] des pherides evch I uch duses pherdis M \textbf{3} iu] \textit{om.} L \textbf{4} getât] Getet I  $\cdot$ aber ir] ir aber I (Z) \textbf{5} \textit{Die Verse 514.5-8 fehlen} L   $\cdot$ irz] irsz phert M (Z) \textbf{6} mîn vrouwe] Frouwe mynne M \textbf{8} scheiden] gescheiden Z \textbf{9} er liezes] >er< liezes G ern liez sin I ich enlaszes L er liesz M \textbf{10} des] waz I das M  $\cdot$ dâ] her L \textbf{12} halftern] halfter I (L)  $\cdot$ lôst er] losteher M \textbf{13} sult] en sult M \textbf{14} ditze] das M \textbf{17} hüetet] huete G  $\cdot$ gehœne] hone I M \textbf{20} sunnenblicker] sunnen pliches I svnnen bliche L \textbf{22} ze] von I  $\cdot$ grâwen] grahen L \textbf{26} ze der] zeden G  $\cdot$ porten] porte L  $\cdot$ nâch im] \textit{om.} I  $\cdot$ den stec] dem [wech]: stech G dē stec M \textbf{28} überz] ubir allis M \textbf{29} swie] Wie L (M) \textbf{30} kumbers] [chumbers]: tiͮvelsnezze G  $\cdot$ doch] \textit{om.} M Z \newline
\end{minipage}
\hspace{0.5cm}
\begin{minipage}[t]{0.5\linewidth}
\small
\begin{center}*T
\end{center}
\begin{tabular}{rl}
 & \textbf{Er} sprach: "welt ir râtes pflegen,\\ 
 & ir sult \textbf{iuch disses pferdes} bewegen.\\ 
 & ez enwert iu doch nieman hie;\\ 
 & getât ir daz wægest ie,\\ 
5 & sô sult irz hie lâzen.\\ 
 & mîn vrouwe sî verwâzen,\\ 
 & daz si sô manegen werden man\\ 
 & von dem lîbe scheiden kan."\\ 
 & Gawan sprach, er\textbf{n} liezes niht.\\ 
10 & "Ouwê, \textbf{waz} \textbf{dan dar nâch} geschiht!",\\ 
 & sprach der grâwe rîter wert.\\ 
 & die halftern lôster vonme pfert.\\ 
 & er sprach: "ir sult niht langer stên;\\ 
 & lât diz pfert nâch iu gên.\\ 
15 & des hant daz mer gesalzen hât,\\ 
 & der gebiu vür kumber rât.\\ 
 & hüet, daz iuch iht gehœne\\ 
 & mîner vrouwen schœne,\\ 
 & wan \textbf{diu} ist bî der süeze \textbf{al}sûr,\\ 
20 & reht als ein sunnenblicker schûr."\\ 
 & "\begin{large}N\end{large}û walte\textbf{s} got", sprach Gawan.\\ 
 & urloup nam \textbf{er} zem grâwen man;\\ 
 & als tet er hie unde dort.\\ 
 & si sprâchen alle klagend\textit{iu} wort.\\ 
25 & daz pfert gienc einen smalen \textbf{stec}\\ 
 & zer porten ûz nâch im ûf den \textbf{wec}.\\ 
 & sînes herzen voget er dâ vant;\\ 
 & diu was vrouwe überz lant.\\ 
 & swie sîn herze gegen ir vlôch,\\ 
30 & vil kumbers sim \textbf{doch} drîn \textbf{gezôch}.\\ 
\end{tabular}
\scriptsize
\line(1,0){75} \newline
T U V W O Q R Fr40 \newline
\line(1,0){75} \newline
\textbf{1} \textit{Initiale} W O Q Fr40   $\cdot$ \textit{Majuskel} T  \textbf{10} \textit{Majuskel} T  \textbf{21} \textit{Initiale} T U V  \newline
\line(1,0){75} \newline
\textbf{1} Er] ÷r O \textbf{2} iuch disses pferdes] îv disses pferdes T daz pfert nicht W des pferdes ivch O dises pferdes euch Q (R) \textbf{3} ez enwert] Zu wert Q \textbf{4} ir] [i*]: aber ir V aber ir O Q R ir aber W \textbf{6} mîn vrouwe] Frvͦ minne O \textbf{7} manegen] \textit{om.} Q \textbf{9} Gawan] Gawin R  $\cdot$ ern liezes] [er*]: ern liezes V ich en laze ez O er enlies sin R \textbf{10} waz] das W Q (R) des O  $\cdot$ dan] da mit R \textbf{11} grâwe] [*]: grawe U graffe Q \textbf{13} sult niht] ensúlt nit W sultt mit R \textbf{14} diz] daz O \textbf{15} gesalzen] versalzen V \textbf{16} gebiu] gebe U \textbf{17} iuch] iv T \textbf{19} alsûr] also suͦr U svr O (R) \textbf{20} sunnenblicker] sunne blickes W svnne bliche O (Q) sunnen blick R \textbf{22} zem grâwen] vom grawe W \textbf{24} alle] alle alle Q  $\cdot$ klagendiu] clagende T (R)  $\cdot$ wort] do wort Q \textbf{25} stec] wek Q \textbf{26} nâch im] nam im U \textit{om.} R [*]: nach im Q  $\cdot$ den] dem U [dem]: den V  $\cdot$ wec] steck Q :::tek Fr40 \textbf{27} voget] [weit]: wit Q  $\cdot$ dâ] do U V W Q \textbf{28} was vrouwe] frow was R \textbf{29} swie] Wie U W (Q) R \textbf{30} doch] [*]: doch V \textit{om.} O R  $\cdot$ gezôch] [*]: zoch V zoch W O R \newline
\end{minipage}
\end{table}
\end{document}
