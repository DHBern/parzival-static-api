\documentclass[8pt,a4paper,notitlepage]{article}
\usepackage{fullpage}
\usepackage{ulem}
\usepackage{xltxtra}
\usepackage{datetime}
\renewcommand{\dateseparator}{.}
\dmyyyydate
\usepackage{fancyhdr}
\usepackage{ifthen}
\pagestyle{fancy}
\fancyhf{}
\renewcommand{\headrulewidth}{0pt}
\fancyfoot[L]{\ifthenelse{\value{page}=1}{\today, \currenttime{} Uhr}{}}
\begin{document}
\begin{table}[ht]
\begin{minipage}[t]{0.5\linewidth}
\small
\begin{center}*D
\end{center}
\begin{tabular}{rl}
\textbf{78} & het er \textbf{versûmet} sîne tât.\\ 
 & al \textbf{hie} was genuoger rât.\\ 
 & si \textbf{solden} tjustieren,\\ 
 & \textbf{dort} mit \textbf{rotten} punieren.\\ 
5 & \begin{large}S\end{large}i geloub\textit{t}en sich der sliche,\\ 
 & die man heizet vriwendes stiche.\\ 
 & heinlîch gevaterschaft\\ 
 & wart \textbf{dâ} zervuort mit \textbf{zornes} kraft.\\ 
 & dâ \textbf{wirt} diu krumbe selten sleht.\\ 
10 & man sprach dâ \textbf{lützel} rîters reht.\\ 
 & swer iht gewan, der \textbf{habt} im daz.\\ 
 & ern ruochte, hetes der ander haz.\\ 
 & si wâren \textbf{von} \textbf{manegen} landen,\\ 
 & die dâ mit ir handen\\ 
15 & schildes ambet worhten\\ 
 & unt schaden \textbf{lützel} vorhten.\\ 
 & \textbf{Al dâ} wart von Gahmurete\\ 
 & geleistet Ampflisen bete,\\ 
 & daz er ir ritter wære.\\ 
20 & ein brief sagt im daz mære.\\ 
 & Âvoy, \textbf{nû} wart er lâzen an!\\ 
 & ob minne unt ellen in des man?\\ 
 & grôz\textit{iu} liebe unt starkiu triwe\\ 
 & \textbf{sîne kraft im vrumt} al niwe.\\ 
25 & nû sach er, wâ der künic Lot\\ 
 & sînen schilt gein der herten bôt.\\ 
 & der was \textbf{umb nâch} gewant.\\ 
 & daz werte Gahmuretes hant.\\ 
 & mit \textbf{hurte} er den poinder brach.\\ 
30 & den künec von Arragun er stach\\ 
\end{tabular}
\scriptsize
\line(1,0){75} \newline
D \newline
\line(1,0){75} \newline
\textbf{5} \textit{Initiale} D  \textbf{17} \textit{Majuskel} D  \textbf{21} \textit{Majuskel} D  \newline
\line(1,0){75} \newline
\textbf{5} geloubten] geloͮben D \textbf{17} Gahmurete] Gahmvrete D \textbf{23} grôziu] groze D \textbf{25} Lot] Loth D \textbf{28} Gahmuretes] Gahmvretes D \newline
\end{minipage}
\hspace{0.5cm}
\begin{minipage}[t]{0.5\linewidth}
\small
\begin{center}*m
\end{center}
\begin{tabular}{rl}
 & hete er \textbf{versuonet} sîne tât.\\ 
 & al \textbf{hie} was genuoger rât.\\ 
 & si \textbf{solte\textit{n}} \textit{j}ustieren,\\ 
 & \textbf{dort} mit \textbf{rotten} \textit{pu}nieren.\\ 
5 & si gelo\textit{u}beten sich \textit{d}er sliche,\\ 
 & die man heiz\textit{e}t vriundes stiche.\\ 
 & heimlîch gevaterschaft\\ 
 & wart \textbf{dâ} zervüer\textit{et} mi\textit{t} \textit{k}raft.\\ 
 & d\textit{â} \textbf{wart} diu krumbe selten sleht.\\ 
10 & man sprach d\textit{â} \textbf{wênic} ritters reht.\\ 
 & wer ih\textit{t} gewan, der \textbf{hete} ime daz.\\ 
 & er enruochte, het es der ander haz.\\ 
 & si wâren \textbf{von} \textbf{manigen} landen,\\ 
 & die dâ mit ir handen\\ 
15 & schiltes ambet worhten\\ 
 & und schaden \textbf{wênic} vorhten.\\ 
 & \textbf{aldâ} wart von Gahmurete\\ 
 & geleistet A\textit{mpf}lisen bete,\\ 
 & daz er ir ritter wære.\\ 
20 & ein brief sagete ime daz mære.\\ 
 & â\textit{v}oy, \textbf{nû} wart er lâzen ane!\\ 
 & ob minne und \textit{e}llen \textit{in} des mane?\\ 
 & grôziu liebe und starkiu triuwe\\ 
 & \textbf{sîn kraft im vrumte} al niuwe.\\ 
25 & nû sach er, wâ der künic Lot\\ 
 & sînen schilt gegen der herte bôt.\\ 
 & der was \textbf{umbe nâch} gewant.\\ 
 & daz werte Gahmuretes hant.\\ 
 & mit \textbf{herte} er den poinder brach.\\ 
30 & den künic von Aragun er stach\\ 
\end{tabular}
\scriptsize
\line(1,0){75} \newline
m n o \newline
\line(1,0){75} \newline
\newline
\line(1,0){75} \newline
\textbf{1} versuonet] versumet n (o) \textbf{3} solten justieren] solten solten iustieren m \textbf{4} punieren] banieren m (n) \textbf{5} si geloubeten] Sigelon betten m n (o)  $\cdot$ der sliche] ersliche m \textbf{6} heizet] heissent m \textbf{8} dâ zervüeret] da zefuͯr m zerfúret do n do zur fuͯret o  $\cdot$ mit kraft] mit gevatterschaft craft m \textbf{9} dâ] Do m n o \textbf{10} dâ] do m n o \textbf{11} iht] ich m \textbf{12} es] sin n o  $\cdot$ ander] andern o \textbf{14} dâ] do n o \textbf{15} \textit{Verse 78.15-16 kontrahiert zu:} Schyltes ambacht forchten o  \textbf{17} Gahmurete] gamiret n gamúret o \textbf{18} Ampflisen] anfelisen m (n) (o) \textbf{20} sagete] sagt o \textbf{21} âvoy] Anoi m (n) (o)  $\cdot$ nû] \textit{om.} n o \textbf{22} ellen in] allen m \textbf{23} liebe] krafft n \textbf{24} sîn] An n o  $\cdot$ vrumte] fremte o  $\cdot$ al] alle n \textbf{25} sach] sagt n sag o \textbf{27} der] Dar n o  $\cdot$ umbe] jme n (o) \textbf{28} Gahmuretes] gahmurettes m gamires n gamutes o \textbf{30} Aragun] aragún m araguͯn o \newline
\end{minipage}
\end{table}
\newpage
\begin{table}[ht]
\begin{minipage}[t]{0.5\linewidth}
\small
\begin{center}*G
\end{center}
\begin{tabular}{rl}
 & heter \textbf{versûmet} sîne tât.\\ 
 & al \textbf{dâ} was genuoger rât.\\ 
 & si \textbf{solten} tjostieren,\\ 
 & mit \textbf{hurte} punieren.\\ 
5 & si geloubten sich der sliche,\\ 
 & die man heizet vriundes stiche.\\ 
 & heinlîch gevaterschaft\\ 
 & wart \textbf{dâ} zervüeret mit \textbf{zornes} kraft.\\ 
 & dâ \textbf{wart} diu krumbe selten sleht.\\ 
10 & man sprach dâ \textbf{wênic} rîters reht.\\ 
 & swer iht gewan, der \textbf{het} im daz.\\ 
 & er enruochte, het es der ander haz.\\ 
 & si wâren \textbf{ûz} \textbf{verren} landen,\\ 
 & die dâ mit ir handen\\ 
15 & schiltes ambet worhten\\ 
 & unde schaden \textbf{wênic} vorhten.\\ 
 & \textbf{dô} wart \textbf{ouch} von Gahmuret\\ 
 & geleistet Anphlisen bet,\\ 
 & daz er ir rîter wære.\\ 
20 & ein brief seit im daz mære.\\ 
 & âvoy, \textbf{dô} wart er lâzen an!\\ 
 & op minne und ellen in des man?\\ 
 & grôz liebe und starkiu triwe\\ 
 & \textbf{sîne kraft im vrumet} al niwe.\\ 
25 & nû sach er, wâ der künic Lot\\ 
 & sînen schilt gein der herte bôt.\\ 
 & \textit{d}er was \textbf{vil} \textbf{nâch al umbe} gewant.\\ 
 & daz werte Gahmuretes hant.\\ 
 & mit \textbf{hurte} er den ponder brach.\\ 
30 & den künic von Arragun er stach\\ 
\end{tabular}
\scriptsize
\line(1,0){75} \newline
G I O L M Q R Z Fr56 \newline
\line(1,0){75} \newline
\textbf{1} \textit{Initiale} O  \textbf{5} \textit{Initiale} I  \textbf{25} \textit{Initiale} I  \newline
\line(1,0){75} \newline
\textbf{1} heter] al het er I ÷et O  $\cdot$ sîne] sinen R  $\cdot$ tât] getat O Fr56 tac R \textbf{2} Hie was schlag wider schlac R  $\cdot$ al dâ] Al hie O (L) (M) (Q) (Z) (Fr56) \textbf{3} si solten] Si wolden O (L) (M) (R) Die wolten Q \textbf{4} \textit{Versdoppelung 309.13-14 nach 78.4:} Man sprach ir recht vf blvmen velt / Da enirte stude noch gezelt L   $\cdot$ mit] Dort mit O L (M) Q (R) Z  $\cdot$ hurte] hurten I rote O (L) M (R) rotten Q Z \textbf{5} sliche] stiche I schlichte R \textbf{6} man] man da I man do O (Q)  $\cdot$ vriundes stiche] vriundesliche I der vrivnde stich O frúndes richtte R \textbf{8} wart dâ] War O Wart do Q  $\cdot$ zervüeret] geoͯgt R \textbf{9} dâ] Do O Q  $\cdot$ sleht] slhet L \textbf{10} dâ] \textit{om.} O do Q  $\cdot$ wênic] selten I \textbf{11} swer] Wer L M Q R  $\cdot$ het] habt I O R \textbf{12} er enruochte] ern ruch I (M) Ern rvchet Z (Fr56)  $\cdot$ es] sin I Z er R  $\cdot$ ander] andren R  $\cdot$ haz] fvr haz O \textbf{13} verren] manigen O (L) (M) (Q) (R) Z Fr56 \textbf{14} dâ] do Q  $\cdot$ ir] orm M ir selbs R \textbf{16} schaden] saden I  $\cdot$ wênic] lvzel O \textbf{17} dô] Da M Z  $\cdot$ ouch] avch da O (L) (M) (R) (Z) auch do Q  $\cdot$ Gahmuret] Gamvret O [Gh]: Gahmuͯret L [gamut]: gamurete M gamuret Q (Z) \textbf{18} geleistet] Geschliget Q Gelestert Z  $\cdot$ Anphlisen] anphalisen I amphilisen O anfolisen L [anfilzen]: anfilizen M an flieszen Q amflẏsen R amflisen Z \textbf{20} Der stolze vnd der gewere L  $\cdot$ seit] saite M  $\cdot$ daz] die R \textbf{21} âvoy] Awe O Aveý L  $\cdot$ dô] da O L M nu Z  $\cdot$ lâzen] verlazen R \textbf{22} minne] nuͯn Q \textbf{23} liebe] liebú R \textbf{24} im vrumet] fugt nún Q in frumpt R \textbf{25} er] sie Z  $\cdot$ wâ] do Q \textbf{26} der herte] den herren R \textbf{27} der] er G Das M  $\cdot$ vil nâch al umbe] vmbe nach O L R Z Fr56 ome noch M (Q)  $\cdot$ gewant] gebant Z \textbf{28} daz] Des R  $\cdot$ werte] were Q wert Fr56  $\cdot$ Gahmuretes] Gachmuretis I Gamvretes O Gahmuͯretes L gamuretes M Q Z gahmvrets Fr56 \textbf{29} er] er uff M \textbf{30} Arragun] aragun G R arrogûn I arragon M arraguͯn Q \newline
\end{minipage}
\hspace{0.5cm}
\begin{minipage}[t]{0.5\linewidth}
\small
\begin{center}*T (U)
\end{center}
\begin{tabular}{rl}
 & hete er \textbf{versûmet} sîne tât.\\ 
 & al\textbf{hie} was genuoger \textbf{worden} rât.\\ 
 & si \textbf{wolten} jostieren,\\ 
 & \textbf{dort} mit \textbf{rotten} punieren.\\ 
5 & si geloubeten sich der sliche,\\ 
 & die man heizet vriundes stiche.\\ 
 & heimelîche gevaterschaft\\ 
 & wart zervüeret mit \textbf{zornes} kraft.\\ 
 & d\textit{â} \textbf{wart} diu krumbe selten sleht.\\ 
10 & man sprach d\textit{â} \textbf{wênec} ritters reht.\\ 
 & wer iht gewan, der \textbf{habe\textit{t}} im daz.\\ 
 & er\textit{n} ruochte, het e\textit{s} der ander haz.\\ 
 & si wâren \textbf{ûz} \textbf{manegen} landen,\\ 
 & die d\textit{â} mi\textit{t} ir hande\textit{n}\\ 
15 & schiltes ambet worh\textit{t}en\\ 
 & und schaden \textbf{wênic} vorhten.\\ 
 & \textbf{\begin{large}N\end{large}û} wart \textbf{ouch} \textbf{d\textit{â}} von Gahmuret\\ 
 & geleistet \textbf{vrou\textit{n}} Anflisen bet,\\ 
 & daz er ir ritter wære.\\ 
20 & ein brief sagete im die mære.\\ 
 & âvoy, \textbf{dô} wart \textit{er} gelâzen a\textit{n}!\\ 
 & ob minne und \textit{e}lle\textit{n} in de\textit{s} man?\\ 
 & grôziu liebe und starkiu triuwe\\ 
 & \textbf{vrumte sîne kraft} al niuwe.\\ 
25 & nû sach er, wâ der künec Lot\\ 
 & sînen schilt gein der herte bôt.\\ 
 & der was \textbf{al umb nâch} gewant.\\ 
 & daz werte Gahmuretes hant.\\ 
 & mit \textbf{hurte} er den poynder brach.\\ 
30 & den künec von Arragun er stach\\ 
\end{tabular}
\scriptsize
\line(1,0){75} \newline
U V W T \newline
\line(1,0){75} \newline
\textbf{9} \textit{Majuskel} T  \textbf{13} \textit{Majuskel} T  \textbf{17} \textit{Initiale} U   $\cdot$ \textit{Majuskel} T  \textbf{25} \textit{Initiale} T  \textbf{30} \textit{Majuskel} T  \newline
\line(1,0){75} \newline
\textbf{1} versûmet] versuͤnet W  $\cdot$ tât] getat V W T \textbf{2} worden] \textit{om.} W T \textbf{3} si] so T \textbf{4} \textit{Versdoppelung 309.13-14 nach 78.4:} Men sprach ir reht uf bluͦmen velt / do en irrete stude noch gezelt V   $\cdot$ rotten] rotte T \textbf{5} geloubeten] erloͮbeten V verloͮbeten T \textbf{6} heizet] da heizet T \textbf{8} wart] [w*]: wart do V Ward do W wart da T \textbf{9} dâ] Do U W Des V  $\cdot$ krumbe] minne T \textbf{10} dâ] do U V \textit{om.} W \textbf{11} wer] Swer V (T)  $\cdot$ habet] habe U het V behabt W \textbf{12} ern ruochte] Er ruͦchte U Er enruͦchte vnd W ern rvͦht T  $\cdot$ es] iz U (T)  $\cdot$ der ander] einander W \textbf{14} dâ] do U W  $\cdot$ mit] mir U  $\cdot$ handen] hande U \textbf{15} schiltes] riters T  $\cdot$ worhten] worden U \textbf{16} schaden wênic] wenig [scha*]: schaden V schadens wenig W \textbf{17} Nû] Do T  $\cdot$ ouch] \textit{om.} T  $\cdot$ dâ] do U V W  $\cdot$ Gahmuret] Gahmuͦret U Gamurete V gamuret W \textbf{18} vroun] vruͦnt U \textit{om.} W  $\cdot$ Anflisen] anflizen U anfolisen W \textbf{20} sagete im] saget im V  $\cdot$ die] [d*]: daz V das W div T \textbf{21} âvoy] \textit{om.} T  $\cdot$ er] \textit{om.} U V W  $\cdot$ gelâzen] verlâzen T  $\cdot$ an] a U \textbf{22} ellen] alle U  $\cdot$ des] dem U \textbf{23} starkiu] starke T \textbf{24} Sein krafft fruͦmete al neúwe W  $\cdot$ mahte im die creffte niͮwe T \textbf{26} herte] hvrte T \textbf{27} al umb] wider T \textbf{28} daz werte] Der werte W do wertin T  $\cdot$ Gahmuretes] Gahmuͦretes U Gamuretes V (W) \textbf{29} mit craft er die hvrte brach T \textbf{30} Arragun] Araguͦn U arragund W  $\cdot$ stach] do stach W \newline
\end{minipage}
\end{table}
\end{document}
