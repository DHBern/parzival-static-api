\documentclass[8pt,a4paper,notitlepage]{article}
\usepackage{fullpage}
\usepackage{ulem}
\usepackage{xltxtra}
\usepackage{datetime}
\renewcommand{\dateseparator}{.}
\dmyyyydate
\usepackage{fancyhdr}
\usepackage{ifthen}
\pagestyle{fancy}
\fancyhf{}
\renewcommand{\headrulewidth}{0pt}
\fancyfoot[L]{\ifthenelse{\value{page}=1}{\today, \currenttime{} Uhr}{}}
\begin{document}
\begin{table}[ht]
\begin{minipage}[t]{0.5\linewidth}
\small
\begin{center}*D
\end{center}
\begin{tabular}{rl}
\textbf{765} & \begin{large}D\end{large}er heiden jach vür \textbf{werdiu} dinc.\\ 
 & sus reit an Gawans rinc\\ 
 & Artus mit sînem wîbe\\ 
 & \textbf{unt} \textbf{mit} manegem clârem lîbe,\\ 
5 & mit rîtern unt \textbf{mit} vrouwen.\\ 
 & der heiden mohte schouwen,\\ 
 & daz ouch dâ liute wâren,\\ 
 & junc \textbf{mit} solhen jâren,\\ 
 & \textbf{daz} si pflâgen \textbf{varwe} \textbf{glanz}.\\ 
10 & Dô was der künec Gramoflanz\\ 
 & dennoch in Artuses pflege.\\ 
 & dâ reit \textbf{ouch} ûf dem selben wege\\ 
 & Itonje, sîn amîe,\\ 
 & diu süeze valsches vrîe.\\ 
15 & Dô erbeizte der tavelrunde schar\\ 
 & mit ma\textit{n}eger vrouwen \textbf{wol} gevar.\\ 
 & Ginover \textbf{liez} Itonje\\ 
 & ir neven, den heiden, küssen ê.\\ 
 & si selbe \textbf{dô} dar nâher gienc,\\ 
20 & Feirefizen si mit kusse enpfienc.\\ 
 & Artus unt \textbf{ouch} Gramoflanz\\ 
 & mit \textbf{getriulîcher} liebe ganz\\ 
 & enpfiengen \textbf{disen} heiden.\\ 
 & dâ wart im von in beiden\\ 
25 & mit dienste erboten êre,\\ 
 & unt sîner mâge mêre\\ 
 & im tâten guoten willen schîn.\\ 
 & Feirefiz Anschevin\\ 
 & was dô ze guoten vriwenden komen.\\ 
30 & daz het er schiere an in vernomen.\\ 
\end{tabular}
\scriptsize
\line(1,0){75} \newline
D Fr12 \newline
\line(1,0){75} \newline
\textbf{1} \textit{Initiale} D  \textbf{10} \textit{Majuskel} D  \textbf{15} \textit{Majuskel} D  \newline
\line(1,0){75} \newline
\textbf{1} heiden] heinden Fr12 \textbf{2} Gawans] Gawanes Fr12 \textbf{4} clârem] claren Fr12 \textbf{10} Gramoflanz] Gramanz Fr12 \textbf{11} Artuses] Artvs D \textbf{13} Itonje] Jtonîe D Jtonie Fr12 \textbf{14} süeze] selbe Fr12 \textbf{16} maneger] mager D \textbf{17} Ginover] Ginoͤver D Gynever Fr12  $\cdot$ Itonje] Jtonîe D Jtonie Fr12 \textbf{20} Feirefizen] [Feirefi*en]: Feirefizen D feyrefizen Fr12 \textbf{21} Gramoflanz] Gramanz Fr12 \textbf{28} Feirefiz] Feyrefiz Fr12  $\cdot$ Anschevin] Anscivin D (Fr12) \newline
\end{minipage}
\hspace{0.5cm}
\begin{minipage}[t]{0.5\linewidth}
\small
\begin{center}*m
\end{center}
\begin{tabular}{rl}
 & der heiden jach vür \textbf{werdiu} dinc.\\ 
 & sus reit an Gawanes rinc\\ 
 & Artus mit sînem wîbe\\ 
 & \textbf{und} manige\textit{m} clâren lîbe,\\ 
5 & mit ritter\textit{n} und vrowen.\\ 
 & der heiden mohte schouwen,\\ 
 & daz ouch d\textit{â} liute wâren,\\ 
 & junc \textbf{von} solichen jâren,\\ 
 & \textbf{dô} si pflâgen \textbf{varwe} \textbf{ganz}.\\ 
10 & dâ was der künic Gramolanz\\ 
 & dannoch in Artuses pflege.\\ 
 & dô reit \textbf{ouch} ûf dem selben wege\\ 
 & Ithonie, sîn amîe,\\ 
 & diu süeze valsches vrîe.\\ 
15 & dô erbeizte der tavelrunde schar\\ 
 & mit maniger vrowen \textbf{wol} gevar.\\ 
 & Ginover \textbf{hiez} Ithonie\\ 
 & ir neven, den heiden, küssen ê.\\ 
 & si selbe, \textbf{diu} dar nâher gienc,\\ 
20 & Ferefizen si mit kusse enpfienc.\\ 
 & Artus und Gramolanz\\ 
 & mit \textbf{getriulîcher} liebe ganz\\ 
 & enpfiengen \textbf{disen} heiden.\\ 
 & dô wart im von in beiden\\ 
25 & mit dienst erboten êre,\\ 
 & und sîner mâge mêre\\ 
 & im tâten guoten willen schîn.\\ 
 & Ferefiz A\textit{n}schevin\\ 
 & was dô zuo guoten vriunden komen.\\ 
30 & daz het er schier an in vernomen.\\ 
\end{tabular}
\scriptsize
\line(1,0){75} \newline
m n o V V' W \newline
\line(1,0){75} \newline
\newline
\line(1,0){75} \newline
\textbf{1} jach] iachs W  $\cdot$ werdiu] froliche V' \textbf{2} Gawanes] Gawans V (V') herr gawans W \textbf{3} Artus] Artuͯs o Kúnig artus W \textbf{4} und] [M*]: mit V Mit V' W  $\cdot$ manigem] mangen m o  $\cdot$ clâren] clarem V (W) \textbf{5} rittern] ritter m  $\cdot$ und] vnde mit V (V') (W) \textbf{6} mohte] moͯchte n o (V) \textbf{7} dâ] do m n o V V' die W \textbf{9} dô] Daz V V' (W)  $\cdot$ pflâgen] pflegen o  $\cdot$ ganz] glantz n (o) (V) (V') W \textbf{10} \textit{Vers 765.10 fehlt} o   $\cdot$ dâ] Do n V W  $\cdot$ Gramolanz] gramolantz m n Gramaflantz V (V') gramoflantz W \textbf{11} dannoch] [Do sie]: Donnach o  $\cdot$ Artuses] artuͯses o \textbf{12} \textit{Vers 765.12 fehlt} n   $\cdot$ dô] Der W  $\cdot$ selben] \textit{om.} V' selbem W \textbf{13} Ithonie] Jtonie m o Jthonie n V Ythonie V' Ytonie W \textbf{14} valsches] wandels V V' \textbf{15} dô erbeizte] D: erbeist do o  $\cdot$ tavelrunde] tafelrunder n (V) (V') (W) \textbf{16} maniger] mangen o  $\cdot$ gevar] gewar V' \textbf{17} Ginover] Ginofer m n o Gynover V Gynoger V' Tschinouer W  $\cdot$ hiez] lies V (V')  $\cdot$ Ithonie] Jtonie m (o) ythonye V ytonye W \textbf{19} si] Die o  $\cdot$ selbe] selber V (V') selbes W  $\cdot$ diu] do V V' \textbf{20} Ferefizen] Ferrefisen n Fereficzen o Ferevisen V Fereuisen V' Ferafißen W  $\cdot$ si] \textit{om.} W \textbf{21} Gramolanz] gramolantz m gramulantz n gramunlancz o gramaflantz V V' gramoflantz W \textbf{22} getriulîcher] getruwer o  $\cdot$ liebe] liebú V liehe W \textbf{24} im] in V V' \textbf{25} êre] [mere]: [here]: ere V' \textbf{28} Ferefiz] Ferefis m Ferrefis n o Ferevis V' Ferafis W  $\cdot$ Anschevin] auscevin m n ansce vin o anscheuin V' antscheuein W \textbf{29} guoten] \textit{om.} W  $\cdot$ vriunden] froͮden V (V') \textbf{30} daz] Des o Do V'  $\cdot$ het] hat W  $\cdot$ an] von V' \newline
\end{minipage}
\end{table}
\newpage
\begin{table}[ht]
\begin{minipage}[t]{0.5\linewidth}
\small
\begin{center}*G
\end{center}
\begin{tabular}{rl}
 & \begin{large}D\end{large}er heiden jach vür \textbf{richiu} dinc.\\ 
 & sus reit an Gawans rinc\\ 
 & Artus mit sînem wîbe,\\ 
 & \textbf{mit} manige\textit{m} clâren lîbe,\\ 
5 & mit rîtern unde \textbf{mit} vrouwen.\\ 
 & der heiden mohte schouwen,\\ 
 & daz ouch dâ liute wâren,\\ 
 & junc \textbf{mit} solhen jâren,\\ 
 & \textbf{daz} si pflâgen \textbf{varwe} \textbf{glanz}.\\ 
10 & dô was der künic Gramoflanz\\ 
 & dannoch in Artuses pflege.\\ 
 & dô reit \textbf{ouch} ûf dem selben wege\\ 
 & Itonie, sîn amîe,\\ 
 & diu süeze valsches vrîe.\\ 
15 & dô erbeizte der tavelrunder schar\\ 
 & mit maniger vrouwen \textbf{lieht} gevar.\\ 
 & Schinover \textbf{liez} Itonie\\ 
 & ir neven, den heiden, küssen ê.\\ 
 & si selbe \textbf{dô} dar nâher gienc,\\ 
20 & Feirafiz si mit kusse enpfienc.\\ 
 & Artus unde Gramoflanz\\ 
 & mit \textbf{getriulîcher} liebe ganz\\ 
 & enpfiengen \textbf{disen} heiden.\\ 
 & dô wart im von in beiden\\ 
25 & mit dienst erboten êre,\\ 
 & unde sîner mâge mêre\\ 
 & im tâten guoten willen schîn.\\ 
 & Feirafiz Antschevin\\ 
 & was dâ ze guoten vriunden komen.\\ 
30 & daz het er schiere an in vernomen.\\ 
\end{tabular}
\scriptsize
\line(1,0){75} \newline
G I L M Z Fr45 \newline
\line(1,0){75} \newline
\textbf{1} \textit{Initiale} G I L Z  \textbf{15} \textit{Initiale} I  \newline
\line(1,0){75} \newline
\textbf{1} Der heiden jach] Iach der haiden I  $\cdot$ richiu] richen Z \textbf{2} reit] rait er I  $\cdot$ Gawans] Gawansz L \textbf{3} Artus] Artuͯs L \textbf{4} mit] vnd mit I  $\cdot$ manigem] mangen G  $\cdot$ clâren] chlarem I \textbf{6} heiden] haide I (M)  $\cdot$ schouwen] schouwe M \textbf{10} dô] \textit{om.} I Da M (Fr45)  $\cdot$ Gramoflanz] gramoflantz Z \textbf{11} in] \textit{om.} I \textbf{12} dô] Da M Z  $\cdot$ selben] selbem I \textbf{13} Itonie] Jtonie G I L M Jconie Z Jthonie Fr45 \textbf{14} süeze] suͤzzev I selbe Fr45 \textbf{15} dô erbeizte] Do erbaizt I Da irbeiszte M (Z) Dor beẏzte Fr45 \textbf{16} maniger] maningen M  $\cdot$ lieht] wol L M Z Fr45 \textbf{17} Schinover] kinover G Ginofer I Gýnover L Ginover M Gynover Z Gẏnouer Fr45  $\cdot$ liez] hiez do I  $\cdot$ Itonie] Jtonie I (M) [jtanie]: jtonie  L Jconie Z Jthonie Fr45 \textbf{18} neven] neve M  $\cdot$ küssen] kuͯsse L \textbf{19} selbe dô] selbir da M \textbf{20} Feirafiz] Ferefiz L Feirafisz M Feirefiz Z feẏrafiz Fr45  $\cdot$ kusse] kvsse da Z \textbf{21} Gramoflanz] gramoflantz Z Gramoflanz Fr45 \textbf{22} getriulîcher] triwelicher I (Fr45) getruͯwer L \textbf{23} disen] si den I (M) den L \textbf{24} dô] Vnd L Da M  $\cdot$ im] \textit{om.} I \textbf{27} im] Jn L \textbf{28} Feirafiz] Feirefiz G Z Ferefiz L Feirefisz M  $\cdot$ Antschevin] Anschevin G entshevin I anshevin L Z vnd ansevin M \textbf{30} daz] Da M  $\cdot$ in] ir Z \newline
\end{minipage}
\hspace{0.5cm}
\begin{minipage}[t]{0.5\linewidth}
\small
\begin{center}*T
\end{center}
\begin{tabular}{rl}
 & der heiden jach vür \textbf{werdiu} dinc.\\ 
 & sus reit an Gawanes rinc\\ 
 & Artus mit sîme wîbe,\\ 
 & \textbf{mit} manegem clâren lîbe,\\ 
5 & mit rîtern und \textbf{mit} vrouwen.\\ 
 & der heiden mohte schouwen,\\ 
 & daz ouch dâ liute wâren,\\ 
 & junc \textbf{mit} solichen jâren,\\ 
 & \textbf{daz} si pflâgen \textbf{varwen} \textbf{glanz}.\\ 
10 & dô was der künec Gramoflanz\\ 
 & dannoch in Artuses pflege.\\ 
 & dô reit \textbf{doch} ûf dem selben wege\\ 
 & Itonie, sîniu amîe,\\ 
 & diu süeze valsches vrîe.\\ 
15 & \begin{large}D\end{large}ô erbeizte der \textit{tavelrund}er \textit{schar}\\ 
 & mit maneger vrouwen \textbf{lieht} gevar.\\ 
 & Gynover \textbf{liez} Itonie\\ 
 & ir neven, den heiden, küssen ê.\\ 
 & si selbe \textbf{dô} dar nâher gienc,\\ 
20 & Ferefisen si mit kusse enpfienc.\\ 
 & Artus \textit{und} Gramoflanz,\\ 
 & mit \textbf{triuwelîcher} liebe ganz\\ 
 & enpfiengen \textbf{si den} heiden.\\ 
 & dô wart im von in beiden\\ 
25 & mit dienste erboten êre,\\ 
 & und sîner mâge mêre\\ 
 & im tâten guoten willen schîn.\\ 
 & Ferefis Anschewin\\ 
 & was dô zuo guoten vriunden komen.\\ 
30 & daz het er schiere an in vernomen.\\ 
\end{tabular}
\scriptsize
\line(1,0){75} \newline
U Q R \newline
\line(1,0){75} \newline
\textbf{1} \textit{Initiale} Q  \textbf{15} \textit{Initiale} U  \newline
\line(1,0){75} \newline
\textbf{1} \textit{Die Verse 764.13-774.30 fehlen} R  \textbf{2} sus] Als Q  $\cdot$ Gawanes] gawans Q \textbf{7} dâ] do Q \textbf{8} mit] von Q \textbf{9} varwen] frawe Q \textbf{10} Gramoflanz] gramoflantz Q \textbf{12} doch] auch Q \textbf{13} Itonie] Jtonie U Ydonie Q \textbf{15} tavelrunder schar] er U \textbf{16} lieht] wol Q \textbf{17} Gynover] Gynofer Q  $\cdot$ Itonie] Jtonie U ytonie Q \textbf{19} selbe] selber Q \textbf{20} Ferefisen] feyrefissen Q \textbf{21} und] von U  $\cdot$ Gramoflanz] gramoflantz Q \textbf{22} triuwelîcher] getrewlicher Q \textbf{28} Ferefis] feirefisz Q  $\cdot$ Anschewin] Anschevin U anshevin Q \textbf{29} vriunden] frewden Q \textbf{30} het] hott Q \newline
\end{minipage}
\end{table}
\end{document}
