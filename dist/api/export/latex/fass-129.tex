\documentclass[8pt,a4paper,notitlepage]{article}
\usepackage{fullpage}
\usepackage{ulem}
\usepackage{xltxtra}
\usepackage{datetime}
\renewcommand{\dateseparator}{.}
\dmyyyydate
\usepackage{fancyhdr}
\usepackage{ifthen}
\pagestyle{fancy}
\fancyhf{}
\renewcommand{\headrulewidth}{0pt}
\fancyfoot[L]{\ifthenelse{\value{page}=1}{\today, \currenttime{} Uhr}{}}
\begin{document}
\begin{table}[ht]
\begin{minipage}[t]{0.5\linewidth}
\small
\begin{center}*D
\end{center}
\begin{tabular}{rl}
\textbf{129} & des wirt gevelschet manec lîp.\\ 
 & doch solten nû getriwiu wîp\\ 
 & heiles wünschen disem knaben,\\ 
 & der sich \textbf{hie von ir hât} erhaben.\\ 
5 & \begin{large}D\end{large}ô \textbf{reit} der knabe wolgetân\\ 
 & g\textit{e}in dem fôrest \textbf{in} Brizljan.\\ 
 & er kom an einen bach geriten,\\ 
 & \textbf{d\textit{e}n} hete ein han wol überschriten.\\ 
 & swie dâ stuonden bluomen unt gras,\\ 
10 & durch \textbf{daz} \textbf{sîn vluz} sô tunkel was,\\ 
 & der knappe \textbf{den vurt} dâr an vermeit.\\ 
 & den tag er \textbf{gar} dâr neben reit,\\ 
 & als \textbf{ez} sînen witzen tohte.\\ 
 & er beleip \textbf{die naht}, swie er mohte,\\ 
15 & \textbf{unz im der liehte} tag erschein.\\ 
 & \textbf{der knappe} sich dan al ein\\ 
 & \textbf{huop} zeime vurte \textbf{lûter} wolgetân.\\ 
 & \textbf{dâ} was anderhalp der plân\\ 
 & mit eime gezelt gehêret,\\ 
20 & grôz rîcheit dran gekêret\\ 
 & von drîer \textbf{varwe} samît.\\ 
 & ez was hôch und wît.\\ 
 & ûf den næten lâgen borten guot.\\ 
 & dâ hienc ein liderîn huot,\\ 
25 & den man drüber ziehen solte,\\ 
 & immer \textbf{swenne} ez regenen wolte.\\ 
 & \textbf{Der herzoge} Orilus de Lalander,\\ 
 & des wîp dort unde vander\\ 
 & \textbf{ligende} \textbf{minneclîche}.\\ 
30 & \textbf{diu} herzoginne rîche,\\ 
\end{tabular}
\scriptsize
\line(1,0){75} \newline
D Fr13 \newline
\line(1,0){75} \newline
\textbf{5} \textit{Initiale} D  \textbf{27} \textit{Majuskel} D  \newline
\line(1,0){75} \newline
\textbf{6} gein] gin D  $\cdot$ Brizljan] Prizlian D \textbf{8} den] din D \textbf{9} dâ] do Fr13 \textbf{11} vurt] vart Fr13 \textbf{13} sînen] siner Fr13 \textbf{14} swie] wi Fr13 \textbf{15} unz] Biz Fr13  $\cdot$ liehte] lichte Fr13 \textbf{16} der knappe] Du hub her Fr13 \textbf{17} huop] \textit{om.} Fr13  $\cdot$ vurte lûter] uorchte lut und Fr13 \textbf{27} Lalander] lalânder D \newline
\end{minipage}
\hspace{0.5cm}
\begin{minipage}[t]{0.5\linewidth}
\small
\begin{center}*m
\end{center}
\begin{tabular}{rl}
 & des wirt ge\textit{v}elsch\textit{e}t manic lîp!\\ 
 & doch solten \textit{nû} getr\textit{iu}wiu wîp\\ 
 & heiles wünsche\textit{n} disem knaben,\\ 
 & der sich \textbf{hie von ir hât} erhaben.\\ 
5 & \begin{large}D\end{large}ô \textbf{kêrte} der knappe wol getân\\ 
 & gegen dem fôr\textit{e}st \textbf{in} \textit{P}ricilan.\\ 
 & er kam an einen bach geriten,\\ 
 & \textbf{den} hete ein ha\textit{n} wol überschriten.\\ 
 & wie \textit{dâ} stuonden bluomen und gras,\\ 
10 & durch \textbf{daz} \textbf{sîn vliez} sô dunkel was,\\ 
 & der knappe \textbf{vürte} dâr a\textit{n v}ermeit.\\ 
 & den tac er \textbf{gar} dâr ne\textit{b}en reit,\\ 
 & als \textbf{ez} sînen witzen tohte.\\ 
 & er beleip, wie er mohte,\\ 
15 & \textbf{unz ime der liehte} tac erschein.\\ 
 & \textbf{dô huop er} sich dan alein\\ 
 & zuo \textit{einem} \textit{v}urte \textbf{lûter} wol getân.\\ 
 & \textbf{daz} was anderhalp der plân\\ 
 & mit einem gezelt gehêret,\\ 
20 & grôz rîcheit dran gekêret\\ 
 & von drîger \textbf{varwe} samît.\\ 
 & ez was hôch und wît.\\ 
 & ûf den næten lâgen borten guot.\\ 
 & d\textit{â} hienc ein liderîn huot,\\ 
25 & den man dar über ziehen solte,\\ 
 & iemer \textbf{wenne} ez regnen wolte.\\ 
 & \textbf{duc} Orilus de Lalander,\\ 
 & des wîp dort under \dag ein ander\dag \\ 
 & \textbf{ligende} \textbf{wünneclîche},\\ 
30 & \textbf{die} herzoginne rîche,\\ 
\end{tabular}
\scriptsize
\line(1,0){75} \newline
m n o \newline
\line(1,0){75} \newline
\textbf{5} \textit{Initiale} m   $\cdot$ \textit{Capitulumzeichen} n  \newline
\line(1,0){75} \newline
\textbf{1} des] Das o  $\cdot$ gevelschet] geselschaft m  $\cdot$ lîp] wip o \textbf{2} solten] selten n o  $\cdot$ nû getriuwiu] im getrwe m \textbf{3} wünschen] wunsches m wuͯnsch o \textbf{6} fôrest] forast m o  $\cdot$ Pricilan] gricilan m britalan n gritalan o \textbf{8} han] habe m \textbf{9} dâ] \textit{om.} m do n  $\cdot$ stuonden] stuͯnd o \textbf{10} vliez] flusz n (o) \textbf{11} vürte] fúr n vor o  $\cdot$ an vermeit] an ein vermeit m \textbf{12} neben] negen m nemen n \textbf{14} beleip] bleip die nacht n bleip die nach o \textbf{17} einem vurte] wurte m einer fert o \textbf{21} drîger] drien o \textbf{24} dâ] Do m n o \textbf{27} duc] Vntze n (o)  $\cdot$ de] do n o  $\cdot$ Lalander] balander n \textbf{29} ligende] Ligen n  $\cdot$ wünneclîche] winneclich o \newline
\end{minipage}
\end{table}
\newpage
\begin{table}[ht]
\begin{minipage}[t]{0.5\linewidth}
\small
\begin{center}*G
\end{center}
\begin{tabular}{rl}
 & des wirt gevelschet manic lîp.\\ 
 & doch solten nû getriwiu wîp\\ 
 & \textit{heiles wünschen} disem knaben,\\ 
 & der sich \textbf{hie hât von ir} erhaben.\\ 
5 & dô \textbf{kêrt} der knappe wolgetân\\ 
 & gein dem fôreise \textbf{ze} Brizilan.\\ 
 & er kom an einen bach geriten.\\ 
 & \textbf{in} hete ein hane wol überschriten.\\ 
 & swie dâ stuonden bluomen und gras,\\ 
10 & durch \textbf{daz} \textbf{der vliez} sô tunkel was,\\ 
 & der knappe \textbf{den vurt} dâr ane vermeit.\\ 
 & den tag er \textbf{gar} dâr neben reit,\\ 
 & alse\textbf{z} sînen witzen tohte.\\ 
 & er beleip \textbf{die naht}, swier mohte.\\ 
15 & \textbf{des morgens, dô der} tac erschein,\\ 
 & \textbf{der knapp huop} sich dan alein\\ 
 & zeinem vurte wolgetân.\\ 
 & \textbf{dâ} was anderhalp der plân\\ 
 & mit einem gezelte gehêret,\\ 
20 & grôz rîcheit dran gekêret\\ 
 & \begin{large}V\end{large}on drîger \textbf{varwe} samît.\\ 
 & ez was hôch und wît.\\ 
 & ûf den næten lâgen borten guot.\\ 
 & dâ hienc ein liderîn huot,\\ 
25 & den man drüber ziehen solte,\\ 
 & imer \textbf{sô}z regnen wolte.\\ 
 & \textbf{duc} Orillus de Lalander,\\ 
 & des wîp dort unden vander\\ 
 & \textbf{ligende} \textbf{wünniclîche},\\ 
30 & \textbf{die} herzoginne rîche,\\ 
\end{tabular}
\scriptsize
\line(1,0){75} \newline
G I O L M Q R Z Fr35 \newline
\line(1,0){75} \newline
\textbf{1} \textit{Initiale} I  \textbf{5} \textit{Überschrift:} Wie parcifal quam insz getzelt zu iescuten vnd in das furspann vnd das fingerlein name Q   $\cdot$ \textit{Illustration mit Überschrift:} Hie rittet barczifal vsz vnd gnadet siner muͦtter vnd kompt zu einer herczogin die schlieff vnder einen gezeltt allein R   $\cdot$ \textit{Initiale} I O L Q R Z  \textbf{21} \textit{Initiale} G I  \textbf{27} \textit{Initiale} O L R Z  \newline
\line(1,0){75} \newline
\textbf{1} wirt] wurd R  $\cdot$ gevelschet] getúrret R  $\cdot$ lîp] wip L \textbf{2} solten] sulde M seltten R  $\cdot$ getriwiu] getrúwe R \textbf{3} \textit{Vers 129.3 fehlt} R   $\cdot$ heiles wünschen] wunschen heiles G \textbf{4} \textit{Vers 129.4 fehlt} Q   $\cdot$ hie hât von ir] von ir nu hat I von ir hie hat O (M) R Fr35 hie von ir het L von ir hat Z \textbf{5} dô] ÷o O Da Z  $\cdot$ kêrt] kerte L (M) Z Fr35 \textbf{6} dem] den M  $\cdot$ fôreise ze] forais in I (M) (Q) (R) (Z) (Fr35) forest gein O fuͯrste L  $\cdot$ Brizilan] briziliam I Bresiliam O Brezilian L Z briziliā M breszzilian Q berczelon R Brezzilian Fr35 \textbf{8} in] den I (M) (Z)  $\cdot$ hane] hahan R \textbf{9} swie] Wie L (Q) o\textit{m. } R  $\cdot$ dâ] \textit{om.} L do Q \textbf{10} daz] \textit{om.} R  $\cdot$ der] sin Z  $\cdot$ vliez] fluͤz I (O) (L) (R) (Z)  $\cdot$ sô] gar Z  $\cdot$ tunkel] tukel R \textbf{11} knappe] chappe I  $\cdot$ den] die L  $\cdot$ ane] Gar I \textit{om.} R Z \textbf{12} gar dâr] do Q \textbf{13} witzen] witze Q \textbf{14} swier] wie er L (Q) R Z \textbf{15} dô] da M Z \textbf{17} zeinem] Zu einer R  $\cdot$ vurte] fvͦrt lvter vnde O (L) (M) (Q) (R) (Z) \textbf{18} dâ] do I (L) (Q)  $\cdot$ der] ein I \textbf{19} gezelte] czelde M  $\cdot$ gehêret] geziret vnd geheret L \textbf{21} varwe samît] hande varwe samît O varwen samýt L samet farwe Q \textbf{22} was] was och R  $\cdot$ hôch] hoht O scone M \textbf{23} Aussen borten genete gut Q \textbf{24} dâ] Do Q  $\cdot$ liderîn] ledener M \textbf{26} Jammer swenner regen wolde M  $\cdot$ imer] \textit{om.} L  $\cdot$ sôz] swanne ez O (Z) Wenne ez L (Q) (R) \textbf{27} duc] ÷vrch O DOrch L  $\cdot$ Orillus] orilus I (O) (M) (Q) (R) (Z) Orẏlus Fr35  $\cdot$ de] der I [der]: de O  $\cdot$ Lalander] lander I (R) lalalander M \textbf{28} dort unden] dar vnder I (Q) (R) \textbf{29} ligende] ligen I (O) (L) (M) (Q) (Fr35) \textbf{30} die] diu I (R) \newline
\end{minipage}
\hspace{0.5cm}
\begin{minipage}[t]{0.5\linewidth}
\small
\begin{center}*T (U)
\end{center}
\begin{tabular}{rl}
 & des wirt gevelschet manec lîp.\\ 
 & doch solten nû getriuwiu wîp\\ 
 & heiles wünschen disem knaben,\\ 
 & der sich \textbf{von ir hie hât} erhaben.\\ 
5 & \begin{large}D\end{large}ô \textbf{kêrte} der knappe wol getân\\ 
 & gein dem fôreht \textbf{in} Prezilian.\\ 
 & er kam an eine bach geriten.\\ 
 & \textbf{in} hete ein hane wol überschriten.\\ 
 & wie dâ stuonden bluomen und gras,\\ 
10 & durch \textbf{sîn vliez} sô dunkel was,\\ 
 & der knappe \textbf{den vurt} \textit{dâr an} vermeit.\\ 
 & den tac er dâr neben reit,\\ 
 & als sînen witzen tohte.\\ 
 & er bleip \textbf{die naht}, wie er mohte.\\ 
15 & \textbf{sît im der mitten} tac erschein,\\ 
 & \textbf{der knappe huop} sich dan alein\\ 
 & zuo einer vurte \textbf{lûter und} wolgetân.\\ 
 & \textbf{d\textit{â}} was anderhalp der plân\\ 
 & mit eime gezelte gehêret,\\ 
20 & grôze rîcheit dran gekêret\\ 
 & von drîer \textbf{hande} samît.\\ 
 & ez was hôch und wît.\\ 
 & ûf den næten lâgen borten guot.\\ 
 & d\textit{â} hienc ein liderîn huot,\\ 
25 & den man dar über z\textit{iehe}n solte,\\ 
 & imer \textbf{wan} ez regen wolte.\\ 
 & \textbf{duc} Orilus de Lalander,\\ 
 & des wîp dort under vander\\ 
 & \textbf{ligen} \textbf{wünneclîche},\\ 
30 & \textbf{die} herzoginne rîche,\\ 
\end{tabular}
\scriptsize
\line(1,0){75} \newline
U V W T \newline
\line(1,0){75} \newline
\textbf{5} \textit{Initiale} U V W T  \textbf{7} \textit{Majuskel} T  \textbf{9} \textit{Majuskel} T  \textbf{15} \textit{Majuskel} T  \textbf{17} \textit{Initiale} V  \textbf{27} \textit{Initiale} W T  \newline
\line(1,0){75} \newline
\textbf{1} lîp] [wip]: lip T \textbf{4} von ir hie] hie von ir W von ir T  $\cdot$ hât] het V \textbf{6} Prezilian] Bricilian U brecilian V \textbf{7} eine] einen W T \textbf{8} wol] schier W \textbf{9} wie] Swie V T  $\cdot$ dâ] do V W \textbf{10} durch] [D*h]: Doch V Durch das W (T)  $\cdot$ vliez] flus V fleiß W [vlies]: vliez T  $\cdot$ sô] do V \textbf{11} den vurt dâr an] den vuͦrt vnd U dar an die fvrt V \textbf{12} er] er gar W T \textbf{13} als] alsez T \textbf{14} wie er] swier T \textbf{15} Des morgens do der tac er scein T  $\cdot$ sît] Vnz V (W)  $\cdot$ tac] morgen V W \textbf{16} sich] \textit{om.} V \textbf{17} zuo einer] Zuͦ einem W (T)  $\cdot$ lûter] der waz lauter W \textbf{18} dâ] Do U V W daz T \textbf{21} hande] varwe T \textbf{24} dâ] Do U V W  $\cdot$ hienc] hieng auch W \textbf{25} ziehen] zuͦ in U \textbf{26} imer wan ez] Jemer swenne ez V alsez T \textbf{27} duc] Herzoge V DEs stoltzen W  $\cdot$ Orilus] Oriluͦs U \textbf{28} des] Sein W  $\cdot$ dort under] darunder W \textbf{29} ligen] ligende T \newline
\end{minipage}
\end{table}
\end{document}
