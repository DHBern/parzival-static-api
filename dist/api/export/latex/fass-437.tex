\documentclass[8pt,a4paper,notitlepage]{article}
\usepackage{fullpage}
\usepackage{ulem}
\usepackage{xltxtra}
\usepackage{datetime}
\renewcommand{\dateseparator}{.}
\dmyyyydate
\usepackage{fancyhdr}
\usepackage{ifthen}
\pagestyle{fancy}
\fancyhf{}
\renewcommand{\headrulewidth}{0pt}
\fancyfoot[L]{\ifthenelse{\value{page}=1}{\today, \currenttime{} Uhr}{}}
\begin{document}
\begin{table}[ht]
\begin{minipage}[t]{0.5\linewidth}
\small
\begin{center}*D
\end{center}
\begin{tabular}{rl}
\textbf{437} & \textit{\begin{large}E\end{large}}r \textbf{vrâgte} der gegenrede \textbf{aldâ}:\\ 
 & "ist \textbf{iemen} dinne?" si sprach: "jâ."\\ 
 & dô er \textbf{hôrte}, daz ez vrouwen stimme was,\\ 
 & her dan ûf ungetretet gras\\ 
5 & warf erz ors vil drâte.\\ 
 & ez dûht in alze spâte.\\ 
 & daz er niht \textbf{was} erbeizet ê,\\ 
 & diu selbe schame tet im wê.\\ 
 & Er bant daz ors vil vaste\\ 
10 & zeines \textbf{gevallen} ronen aste.\\ 
 & sînen dürkeln schilt hienger \textbf{ouch} dran.\\ 
 & dô der kiusche, vrävel man\\ 
 & \textbf{durch zuht} sîn swert von im gebant,\\ 
 & er gienc vürz venster zuo der want.\\ 
15 & dô wolt er vrâgen mære.\\ 
 & diu klôse was vreuden lære,\\ 
 & dar zuo \textbf{aller schimpfe} blôz.\\ 
 & er vant dâ niht wan jâmer grôz.\\ 
 & Er gert ir anz venster dar.\\ 
20 & diu juncvrouwe bleich gevar\\ 
 & mit \textbf{zuht} ûf von ir venje stuont.\\ 
 & dennoch was im hart unkunt,\\ 
 & wer si wære oder mohte sîn.\\ 
 & si truog ein hemde hærîn\\ 
25 & under grâwem rocke ze næhst \textbf{ir} hût.\\ 
 & grôz jâmer was ir \textbf{sundertrût}.\\ 
 & \textbf{der} hete ir hôhen muot gelegt,\\ 
 & vonme herzen \textbf{siufzens vil} erwegt.\\ 
 & mit \textbf{zuht} diu magt zem venster gienc.\\ 
30 & mit süezen worten si in enpfienc.\\ 
\end{tabular}
\scriptsize
\line(1,0){75} \newline
D Fr31 \newline
\line(1,0){75} \newline
\textbf{1} \textit{Initiale} D  \textbf{9} \textit{Majuskel} D  \textbf{19} \textit{Majuskel} D  \newline
\line(1,0){75} \newline
\textbf{1} Er] ÷R D \newline
\end{minipage}
\hspace{0.5cm}
\begin{minipage}[t]{0.5\linewidth}
\small
\begin{center}*m
\end{center}
\begin{tabular}{rl}
 & er \textbf{gerte} der gegenrede \textbf{dâ}:\\ 
 & "ist \textbf{niemen} dinne?" si sprach: "jâ."\\ 
 & dô er \textbf{hôrte}, daz ez vrouwen stimme wa\textit{s},\\ 
 & her dan ûf ungetret gras\\ 
5 & warf er daz ros vil drâte.\\ 
 & ez dûhte in a\textit{l}ze spâte.\\ 
 & d\textit{az} er nih\textit{t} \textbf{\textit{w}as} erbeizet ê,\\ 
 & diu selbe scham tet ime wê.\\ 
 & er bant daz ros vil vaste\\ 
10 & ze eines ronen aste.\\ 
 & \textit{sîn}en dürkelen schilt hienc er dâr an.\\ 
 & dô der kius\textit{ch}e, vrevel man\\ 
 & \textbf{durch zuht} sîn swert von im gebant,\\ 
 & er gienc vür daz venster zuo der want.\\ 
15 & dô wolte er vrâgen mære.\\ 
 & diu klôse was vröuden lære,\\ 
 & dar zuo \textbf{aller schimpf} blôz.\\ 
 & er vant d\textit{â} niht wanne jâmer grôz.\\ 
 & er gerte ir an daz venster dar.\\ 
20 & diu juncvrouwe bleich gevar\\ 
 & mit \textbf{zuht} ûf von ir venje stuont.\\ 
 & dannoch was ime harte unkunt,\\ 
 & we\textit{r} si wære oder mohte sîn.\\ 
 & si truoc ein hemede hærîn\\ 
25 & under \textbf{eime} grâwen rocke ze næhst \textbf{an} \textbf{der} hût.\\ 
 & grôz jâmer was ir \textbf{sundertrût}.\\ 
 & \textbf{der} hette ir hôhen muot geleget,\\ 
 & von dem herzen \textbf{siufzens vil} erweget.\\ 
 & mit \textbf{zuht} diu maget zem venster gienc.\\ 
30 & mit süezen worten si \textit{in} enpfienc.\\ 
\end{tabular}
\scriptsize
\line(1,0){75} \newline
m n o \newline
\line(1,0){75} \newline
\newline
\line(1,0){75} \newline
\textbf{1} gegenrede] gegen reite o  $\cdot$ dâ] do n \textbf{2} niemen] ẏemen n (o)  $\cdot$ dinne] do jnne n (o)  $\cdot$ si] [so]: sẏ n \textbf{3} hôrte] hort o  $\cdot$ ez] \textit{om.} n o  $\cdot$ vrouwen] frowe o  $\cdot$ was] war m \textbf{4} ungetret] vngetretten n \textbf{6} dûhte] ducht o  $\cdot$ alze spâte] alle zespatte m also spate o \textbf{7} daz] Do m  $\cdot$ niht] nit e m  $\cdot$ erbeizet] erbeiset o \textbf{8} selbe] selb o \textbf{9} bant] bat o \textbf{10} ronen] romes n rones o \textbf{11} sînen] Gutten m  $\cdot$ dâr an] dar in o \textbf{12} der] er n  $\cdot$ kiusche] kusse m kusc o \textbf{13} durch zuht sîn swert] Sin swert durch zucht n (o) \textbf{14} zuo der] an die n o \textbf{16} klôse] cluse n o \textbf{17} aller schimpf] alles schimppfes n (o) \textbf{18} dâ] do m n o  $\cdot$ wanne] wan o \textbf{21} ûf von ir venje stuont] sú do vff gestunt n sie uff gestunt o \textbf{23} wer] Were m n  $\cdot$ oder] [*]: oder o  $\cdot$ mohte] moͯchte n \textbf{24} truoc ein] trugen m \textbf{25} næhst] [reht]: rehst o  $\cdot$ der hût] dem hat o \textbf{26} grôz jâmer] Jomer grosz n \textbf{27} ir] iren n \textbf{28} siufzens] súfftzen n (o) \textbf{30} süezen] suͯffczen o  $\cdot$ in] \textit{om.} m \newline
\end{minipage}
\end{table}
\newpage
\begin{table}[ht]
\begin{minipage}[t]{0.5\linewidth}
\small
\begin{center}*G
\end{center}
\begin{tabular}{rl}
 & \begin{large}E\end{large}r \textbf{gerte} der gegenrede \textbf{aldâ}:\\ 
 & "ist \textbf{iemen} drinne?" si sprach: "jâ."\\ 
 & dô er \textbf{hôrte}, daz ez vrouwen stimme was,\\ 
 & her dan ûf ungetret gras\\ 
5 & warf erz ors vil drâte.\\ 
 & ez dûhte in alze spâte.\\ 
 & daz er niht \textbf{was} erbeizet ê,\\ 
 & diu selbe scham tet im \textbf{vil} wê.\\ 
 & er bant daz ors vil vaste\\ 
10 & zeins \textbf{gevallen} ronen aste.\\ 
 & sînen dürkeln schilt hie er \textbf{ouch} dran.\\ 
 & dô der kiusche, vrävel man\\ 
 & \textbf{durch zuhte} sîn swert von im gebant,\\ 
 & er gie vür daz venster zuo der want.\\ 
15 & dô wolde er vrâgen mære.\\ 
 & diu klôse was vröuden lære,\\ 
 & dar zuo \textbf{aller schimpfe} blôz.\\ 
 & er vant dâ niht wan jâmer grôz.\\ 
 & er gert ir anz venster dar.\\ 
20 & diu juncvrouwe bleich gevar\\ 
 & mit \textbf{zühten} ûf von ir venje stuont.\\ 
 & dan\textit{no}ch was im harte unkunt,\\ 
 & wer si wære ode mohte sîn.\\ 
 & si truoc ein hemede hærîn\\ 
25 & under grâwe\textit{m} rocke ze næheste \textbf{ir} hût.\\ 
 & grôz jâmer was ir \textbf{trût}.\\ 
 & \textbf{diu} het ir hôhen muot geleget,\\ 
 & von dem herzen \textbf{vil siuftens} erweget.\\ 
 & mit \textbf{zühten} diu maget ze dem venster gie.\\ 
30 & mit süezen worten sin enpfie.\\ 
\end{tabular}
\scriptsize
\line(1,0){75} \newline
G I L M Z Fr25 \newline
\line(1,0){75} \newline
\textbf{1} \textit{Initiale} G I L M Z Fr25  \textbf{15} \textit{Initiale} I  \newline
\line(1,0){75} \newline
\textbf{1} Er] ÷R Fr25  $\cdot$ der gegenrede] dehain rede I \textbf{3} dô] Da M Z  $\cdot$ hôrte] hort Z \textbf{4} ungetret] vngetreten I vngetrette M \textbf{5} erz] er Fr25 \textbf{8} vil] \textit{om.} I L M Z Fr25 \textbf{10} \textit{Vers 437.10 fehlt} M   $\cdot$ zeins] zuͤ eim I  $\cdot$ gevallen] geuallem I gevallens L \textbf{11} sînen] sinn I (M) (Z) Sine L  $\cdot$ dürkeln] durkil M \textbf{12} dô] Da M Z \textbf{13} durch zuhte sîn swert] daz swert durch zuht I  $\cdot$ gebant] gebant avch dran Fr25 \textbf{14} vür daz] fuͯrt L  $\cdot$ venster] venster sten M  $\cdot$ zuo] bi Z \textbf{15} dô] Da M Z \textbf{16} klôse] cluse M (Fr25)  $\cdot$ was vröuden] vroide wasz M was [fræden]: frevden Fr25 \textbf{17} dar] :a Fr25  $\cdot$ schimpfe] schiffe M \textbf{18} dâ] \textit{om.} M  $\cdot$ wan] nẅan L  $\cdot$ jâmer] [gamer]: iamer G kummer M \textbf{21} zühten] zcucht M (Z)  $\cdot$ von] \textit{om.} M  $\cdot$ venje] venster Z \textbf{22} dannoch] Danch G  $\cdot$ harte] vil I \textbf{23} wer] :wer Fr25  $\cdot$ wære ode mohte] mochte M moͤhte Fr25 \textbf{25} under grâwem] Vndergrawen G Vnd grawe M \textbf{26} ir] ir herzen I ir svnder L (M) (Z) (Fr25) \textbf{27} diu] Der L \textbf{28} von] :::an Fr25  $\cdot$ dem] \textit{om.} Z  $\cdot$ vil siuftens] groz sufften het I svfzens vil L (M) (Z) Fr25 \textbf{29} zühten] zuͯht L (Z) Fr25 \newline
\end{minipage}
\hspace{0.5cm}
\begin{minipage}[t]{0.5\linewidth}
\small
\begin{center}*T
\end{center}
\begin{tabular}{rl}
 & \begin{large}E\end{large}r \textbf{gerte} der gegenrede \textbf{aldâ}:\\ 
 & "Ist \textbf{ieman} dinne?" si sprach: "jâ."\\ 
 & dô er \textbf{erhôrte}, daz ez vrouwen stimme was,\\ 
 & her dan ûf ungetretet gras\\ 
5 & warf er daz ors vil drâte.\\ 
 & ez dûhtin alze spâte.\\ 
 & daz er niht erbeizet ê,\\ 
 & diu selbe schame tet im wê.\\ 
 & er bant daz ors vil vaste\\ 
10 & zuo eines \textbf{gevallen} ronen aste.\\ 
 & sînen dürkeln schilt hienc er \textbf{ouch} dran.\\ 
 & dô der kiusche, vrevel man\\ 
 & sîn swert von im gebant,\\ 
 & er gie vür daz venster zuo der want.\\ 
15 & dô wolter vrâgen mære.\\ 
 & diu klôse was vröuden lære,\\ 
 & dar zuo \textbf{alles schimpfes} blôz.\\ 
 & er vant dâ niht wan jâmer grôz.\\ 
 & er gertir an daz venster dar.\\ 
20 & Diu juncvrouwe bleich gevar\\ 
 & mit \textbf{zuht} ûf von ir v\textit{e}nje stuont.\\ 
 & dannoch was im harte unkunt,\\ 
 & wer si wære oder mohte sîn.\\ 
 & si truoc ein hemde hærîn\\ 
25 & under \textbf{einem} grâwen rocke ze næhest \textbf{an} \textbf{ir} hût.\\ 
 & grôz jâmer was ir \textbf{sundertrût}.\\ 
 & \textbf{der} het ir hôhen muot geleget,\\ 
 & von dem herzen \textbf{siuftens vil} erweget.\\ 
 & mit \textbf{zuht} diu maget zem venster gie.\\ 
30 & mit süezen worten sin enpfie.\\ 
\end{tabular}
\scriptsize
\line(1,0){75} \newline
T U V W O Q R \newline
\line(1,0){75} \newline
\textbf{1} \textit{Initiale} T U V W O   $\cdot$ \textit{Capitulumzeichen} R  \textbf{2} \textit{Majuskel} T  \textbf{20} \textit{Majuskel} T  \newline
\line(1,0){75} \newline
\textbf{1} Er] ÷r O  $\cdot$ gerte] [*]: gerte V  $\cdot$ der gegenrede] gein der rede U gen reden Q  $\cdot$ aldâ] aldo V W Q \textbf{3} er erhôrte] erhorte U (V) (O) (R) erhort W  $\cdot$ daz ez] daz U der R  $\cdot$ was] daz waz R \textbf{4} her] [Do]: Her O  $\cdot$ ungetretet] getredet U vngetrettes W (R) \textbf{5} daz ors] des orses U  $\cdot$ vil] \sout{vnge} vil Q \textbf{6} dûhtin] dunktt in R  $\cdot$ alze] zu R \textbf{7} niht] niht was O  $\cdot$ erbeizet] waz erbeisset V (W) (Q) waz erbeicze R \textbf{9} er] [E]: Er O  $\cdot$ er bant] Der batt R \textbf{10} gevallen] \textit{om.} R  $\cdot$ ronen] rones U W (R) \textbf{11} sînen] Sin V R  $\cdot$ er] \textit{om.} Q  $\cdot$ ouch] \textit{om.} W \textbf{13} sîn] Dvrch tvht sin O \textbf{16} klôse] chlvse O [s*]: claűsen Q closen R \textbf{17} dar] Do V Dor Q Das R \textbf{18} dâ] do U V W Q  $\cdot$ wan] dann Q \textbf{19} gertir] gert ir U Q R gert W O  $\cdot$ venster] [veneter]: venster O \textbf{20} juncvrouwe] frow R  $\cdot$ bleich gevar] bleches var R \textbf{21} zuht] zvhten O (R)  $\cdot$ ir venje] ir veinie T der veine W \textbf{22} harte] gar W \textbf{23} wære ode mohte] were oder moͤhte V (W) (R) mohte O \textbf{24} hemde hærîn] herrin [heml*]: hemdlin R \textbf{25} einem grâwen] grawem O irem Q  $\cdot$ ze næhest] aller nest U naͯchst R  $\cdot$ an ir] bi ir U ir V W O ander R \textbf{27} der] Div O  $\cdot$ ir] irn U (Q) (R) \textbf{28} \textit{Vers 28 nachträglich hinzugefügt} V   $\cdot$ von dem] [Vo*]: Vonme V Vnd im R  $\cdot$ suftens] súfzen V (W) (R) \textbf{29} zuht] zuͦchten U \textbf{30} worten] warten O \newline
\end{minipage}
\end{table}
\end{document}
