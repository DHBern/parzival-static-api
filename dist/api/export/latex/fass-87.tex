\documentclass[8pt,a4paper,notitlepage]{article}
\usepackage{fullpage}
\usepackage{ulem}
\usepackage{xltxtra}
\usepackage{datetime}
\renewcommand{\dateseparator}{.}
\dmyyyydate
\usepackage{fancyhdr}
\usepackage{ifthen}
\pagestyle{fancy}
\fancyhf{}
\renewcommand{\headrulewidth}{0pt}
\fancyfoot[L]{\ifthenelse{\value{page}=1}{\today, \currenttime{} Uhr}{}}
\begin{document}
\begin{table}[ht]
\begin{minipage}[t]{0.5\linewidth}
\small
\begin{center}*D
\end{center}
\begin{tabular}{rl}
\textbf{87} & "swaz mînes rehtes an iu sî,\\ 
 & dâ sult ir mich lâzen bî.\\ 
 & dar zuo mîn dienest genâden gert.\\ 
 & wird ich der \textbf{bete} hie gewert,\\ 
5 & sol \textbf{iu} \textbf{daz} prîs verkrenken,\\ 
 & sô lât mich \textbf{vürbaz} wenken."\\ 
 & Der künegîn Ampflisen,\\ 
 & der kiuschen unt der wîsen,\\ 
 & \begin{large}Û\end{large}f spranc \textbf{balde} \textbf{ir} kappelân.\\ 
10 & er sprach: "\textbf{niht}! \textit{i}n sol ze rehte hân\\ 
 & mîn vrouwe, diu mich in ditze lant\\ 
 & nâch \textbf{sîner minne} hât gesant.\\ 
 & \textbf{diu} lebt nâch \textbf{im ins} lîbes zer.\\ 
 & ir minne hât an im gewer.\\ 
15 & \textbf{diu} sol behalden sînen lîp,\\ 
 & wan si ist \textbf{im holt} vür elliu wîp.\\ 
 & hie sint ir boten, vürsten drî,\\ 
 & kint vor missewende vrî.\\ 
 & der heizet \textbf{einer} Lanzidant,\\ 
20 & von hôher art ûz Gruonlant.\\ 
 & \textbf{der} ist ze Kerlingen komen\\ 
 & unt hât die sprâche an sich genomen.\\ 
 & der ander heizet Liadarz\\ 
 & filii cuons \textbf{Schiolarz}."\\ 
25 & wer \textbf{nû} der dritte wære,\\ 
 & des \textbf{hœret} \textbf{ouch} ein mære:\\ 
 & des muoter \textbf{hiez} Beaflurs\\ 
 & unt sîn vater \textbf{Pansamurs},\\ 
 & die wâren von der \textbf{feien} art;\\ 
30 & daz kint \textbf{hiez} Liahturteltart.\\ 
\end{tabular}
\scriptsize
\line(1,0){75} \newline
D \newline
\line(1,0){75} \newline
\textbf{7} \textit{Majuskel} D  \textbf{9} \textit{Initiale} D  \newline
\line(1,0){75} \newline
\textbf{10} in sol] ensol D \textbf{20} Gruonlant] Grvͦnlant D \textbf{21} Kerlingen] kærlingen D \textbf{23} Liadarz] Leidarz D \textbf{24} Schiolarz] Sciolarz D \newline
\end{minipage}
\hspace{0.5cm}
\begin{minipage}[t]{0.5\linewidth}
\small
\begin{center}*m
\end{center}
\begin{tabular}{rl}
 & "\begin{large}W\end{large}az mînes rehtes an iu sî,\\ 
 & dâ solt ir mich lâzen bî.\\ 
 & dar zuo mîn dienst gnâden gert.\\ 
 & wird ich der \textbf{bete} hie gewert,\\ 
5 & sol \textbf{iu} \textbf{daz} prîs verkrenken,\\ 
 & sô lât mich \textbf{vürder} wenken."\\ 
 & der küniginne A\textit{mp}flisen,\\ 
 & der kiuschen und der wîsen,\\ 
 & ûf spranc \textbf{dâr} \textbf{balde} \textbf{ir} kappelân.\\ 
10 & er sprach: "\textbf{niht}! in sol ze rehte hân\\ 
 & mîn vrowe, diu mich in diz lant\\ 
 & nâch \textbf{sîner minne} hât gesant.\\ 
 & \textbf{diu} lebet nâch \textbf{ime ins} lîbes zer.\\ 
 & ir minne hât  im gewer.\\ 
15 & \textbf{diu} sol behalten sînen lîp,\\ 
 & wanne si ist vür alliu wîp.\\ 
 & hie sint ir boten, vürsten drî,\\ 
 & kin\textit{t} vor missewende vrî.\\ 
 & der heizet \textbf{einer} Lanczidant,\\ 
20 & von hôher art \textit{ûz} Grunlant.\\ 
 & \textbf{er} ist ze Kerlingen komen\\ 
 & und hât die sprâche an sich genomen.\\ 
 & der ander heizet Leidarz\\ 
 & fili con\textit{s} \textbf{Schiolarz}."\\ 
25 & wer \textit{\textbf{nû}} der dritte wære,\\ 
 & des \textbf{hœrt} \textbf{ouch} ein mære:\\ 
 & des muoter, \textbf{diu} Beaflurs,\\ 
 & und sîn vater \textbf{Pamsamurs},\\ 
 & die wâren von der \textbf{selben} art;\\ 
30 & daz kint Liahturteltart.\\ 
\end{tabular}
\scriptsize
\line(1,0){75} \newline
m n o \newline
\line(1,0){75} \newline
\textbf{1} \textit{Initiale} m  \newline
\line(1,0){75} \newline
\textbf{1} rehtes] rechten n (o) \textbf{3} gert] [gerst]: gert n \textbf{4} wird ich] Wir dich n o \textbf{7} küniginne] konig o  $\cdot$ Ampflisen] an flisen m anflisen n anflissen o \textbf{8} der wîsen] wisen n \textbf{9} dâr] \textit{om.} n o \textbf{10} in] \textit{om.} n o \textbf{12} hât] hette n \textbf{13} ime] vmmb o  $\cdot$ zer] zier n (o) \textbf{14} gewer] gewier n \textbf{16} ist] ist halt n o \textbf{18} kint] Kin m  $\cdot$ vor] fur o \textbf{19} der] Sie o  $\cdot$ Lanczidant] lantzidant n \textbf{20} ûz] irs m n o  $\cdot$ Grunlant] gruͯnlant n gruͯnbant o \textbf{21} er] Es o \textbf{22} hât] hette o \textbf{23} Leidarz] leidars m liedarsz n liedarz o \textbf{24} cons] con m concz o  $\cdot$ Schiolarz] Sciolars m sẏolars n siolars o \textbf{25} nû] im m o \textbf{27} des] Dies n Die o  $\cdot$ Beaflurs] beaflúrs n beaflurrs o \textbf{28} Pamsamurs] pansamúrs n pandamuͯrs o \textbf{29} \textit{Versdoppelung nach 87.30 (mit Anteil aus Vers 88.1):} Die lieffen an der selben art o   $\cdot$ die] Do o \textbf{30} daz] Die o  $\cdot$ Liahturteltart] liachtúrtel kart n liachtortelkart o \newline
\end{minipage}
\end{table}
\newpage
\begin{table}[ht]
\begin{minipage}[t]{0.5\linewidth}
\small
\begin{center}*G
\end{center}
\begin{tabular}{rl}
 & "swaz mînes rehtes an iu sî,\\ 
 & dâ sult ir mich lâzen bî.\\ 
 & dar zuo mîn dienst genâden geret.\\ 
 & wirde ich der \textbf{beider} hie geweret,\\ 
5 & sol \textbf{mir} \textbf{daz} prîs verk\textit{r}enken,\\ 
 & sô lât mich \textbf{sunder} wenken."\\ 
 & der künigîn Anphlisen,\\ 
 & der kiuschen unt der wîsen,\\ 
 & ûf spranc \textbf{balde} \textbf{ir} kappelân.\\ 
10 & er sprach: "in sol ze rehte hân\\ 
 & mîn vrouwe, diu mich in diz lant\\ 
 & nâch \textbf{sîner minne} hât gesant.\\ 
 & \textbf{si} lebet nâch \textbf{im in} lîbes zer.\\ 
 & ir minne hât an im gewer.\\ 
15 & \textbf{si} sol behalten sînen lîp,\\ 
 & \textit{wande} sist \textbf{im holt} vür elliu wîp.\\ 
 & hie sint ir boten, vürsten drî,\\ 
 & \textbf{driu} kint vor missewende vrî.\\ 
 & der heizet \textbf{einer} Lazidant,\\ 
20 & von hôher art ûz Gruonelant.\\ 
 & \textbf{der} ist \textbf{her} ze Charlingen komen\\ 
 & unde hât die sprâche an sich genomen.\\ 
 & der ander heizet Liadarz\\ 
 & filicunt \textbf{de} \textbf{Tschierarz}."\\ 
25 & wer der drite wære,\\ 
 & des \textbf{seiter} \textbf{ouch} ein mære:\\ 
 & \textit{des muoter} \textit{\textbf{hiez}} \textit{Beaflurs}\\ 
 & \textit{und sîn vater} \textit{\textbf{Gausamurs}},\\ 
 & \textit{die wâren von der} \textit{\textbf{feien}} \textit{art};\\ 
30 & \textit{daz kint} \textit{\textbf{hiez}} \textit{Liekurteltart}.\\ 
\end{tabular}
\scriptsize
\line(1,0){75} \newline
G I O L M Q R Z Fr21 \newline
\line(1,0){75} \newline
\textbf{1} \textit{Initiale} O M  \textbf{7} \textit{Initiale} Q R Z  \textbf{9} \textit{Initiale} I  \newline
\line(1,0){75} \newline
\textbf{1} swaz] ÷waz O Waz L (Q) (R) Z DO was M  $\cdot$ rehtes] Riches R rehten Z  $\cdot$ iu] uͦch L \textbf{3} genâden] gnad M \textbf{4} wirde] Wurde R  $\cdot$ beider] bet I \textbf{5} mir] wir Z  $\cdot$ daz] myn M  $\cdot$ verkrenken] verchenchen G \textbf{6} sunder] fvder O Z wider L furder M Q (R)  $\cdot$ wenken] denkin M schenken Z \textbf{7} Anphlisen] anphisen I amphileisen O Anfolisen L an filisen M anflizen Q amflizen R ampflisen Z \textbf{9} balde ir] balde ein I der R \textbf{10} er sprach] \textit{om.} I  $\cdot$ in] nicht in Q R \textbf{11} mîn] Dy M  $\cdot$ in diz] her in daz I in dasz Q (R) \textbf{12} hat nach siner minne gesant I  $\cdot$ minne] Minnen M minn mich Q \textbf{13} si] Hie M Die Q R  $\cdot$ lebet] leb I  $\cdot$ nâch im] an O  $\cdot$ in] in ins O in des L (M) insz Q (Z)  $\cdot$ zer] lêr I \textbf{15} si] Die Q (R)  $\cdot$ sînen] sinen pris O \textbf{16} \textit{Versfolge 87.17-16} R   $\cdot$ wande] \textit{om.} G  $\cdot$ vür] [fvn]: fvr O wor Q fvrt Z \textbf{18} driu kint] Chint O (M) (Q) (R) (Z) (Fr21) Die sint L \textbf{19} einer] [ein]: einer von I  $\cdot$ Lazidant] lanzidant I (L) Fr21 Lancidant O lanczidant M (Q) (R) lantzidant Z \textbf{20} Gruonelant] groͮnelant G granlant I grvͦne lant O Fr21 Grvnlant L grune lant M (Z) grúnlant Q Gruͦnlant R \textbf{21} der ist her] Erst Q  $\cdot$ Charlingen] kairingen I cherlingen O kerlingen L (M) Q herlingen R karlingen Z karlinge Fr21 \textbf{23} Liadarz] Lẏadarz L liedars Q lyadarcz R lyandarz Z \textbf{24} filicunt] Fili lv kvnt L (Fr21)  $\cdot$ de] der O \textit{om.} L M Q R Z Fr21  $\cdot$ Tschierarz] tschaialarz G schialarz I Tschihelarz O Tschielarz L (Z) Fr21 ischialars M schielars Q schielarcz R \textbf{25} drite] Rtter R \textbf{26} des] Der M  $\cdot$ seiter] seit er I O L (M) (Q) R Z Fr21  $\cdot$ ein] \textit{om.} Z \textbf{27} \textit{Die Verse 87.27-30 fehlen} G   $\cdot$ hiez] div hiez O (L) heizet Fr21  $\cdot$ Beaflurs] Beafluͯrs L beaflurs M Beafluͦrs R \textbf{28} \textit{Vers 87.28 fehlt} Q   $\cdot$ Gausamurs] Besamvrs O Beas Amuͯrs L beiasamuͯrs M Beasamurs R pansamvrs Z bensamvrs Fr21 \textbf{29} feien] fryen L R feren Q \textbf{30} Liekurteltart] liahtvrteltart O (Q) (Z) Fr21 liehtuͯrtelhart L hachturteltart M lyaheuͯrteltart R \newline
\end{minipage}
\hspace{0.5cm}
\begin{minipage}[t]{0.5\linewidth}
\small
\begin{center}*T (U)
\end{center}
\begin{tabular}{rl}
 & "waz mînes rehtes an iu sî,\\ 
 & dâ solt ir mich lâzen bî.\\ 
 & dar zuo mîn dienst gnâden gert.\\ 
 & w\textit{i}rdich der \textbf{beider} hie gewert,\\ 
5 & sol \textbf{mir} \textbf{der} prîs verkrenken,\\ 
 & sô lâ\textit{t} mich \textbf{vürd\textit{e}r} wenken."\\ 
 & d\textit{er} küneginne Ampflisen,\\ 
 & der kiuschen und der wîsen,\\ 
 & ûf spranc \textbf{der} kapelân.\\ 
10 & er sprach: "in sol zuo rehte hân\\ 
 & mîn vrouwe, diu mich in diz lant\\ 
 & nâch \textbf{sînen minnen} hât gesant.\\ 
 & \textbf{si} lebete \textbf{ie} nâch \textbf{sînes} lîbes zer.\\ 
 & ir minne hât an im gewer.\\ 
15 & \textbf{si} sol \textbf{in} behalten, sînen lîp,\\ 
 & wan si ist \textbf{im holt} vür alliu wîp.\\ 
 & hie sint ir boten, vürsten drî,\\ 
 & kint vor missewende vrî.\\ 
 & der heizet \textbf{eines} Lanzidant,\\ 
20 & von hôher art ûz Gruonlant.\\ 
 & \textbf{der} ist \textbf{her} zuo Kärlingen komen\\ 
 & und hât die sprâche an sich genomen.\\ 
 & der ander heizet Leidarz\\ 
 & figunt \textbf{Schiolarz}."\\ 
25 & wer der drite wære,\\ 
 & des \textbf{sagete er} \textbf{iu} ein mære:\\ 
 & des muoter, \textbf{diu} \textbf{hiez} Beaflurs\\ 
 & und sîn vater \textbf{Pansamurs},\\ 
 & die wâren von der \textbf{feien} art;\\ 
30 & daz kint \textbf{hiez} Liahturteltart.\\ 
\end{tabular}
\scriptsize
\line(1,0){75} \newline
U V W T \newline
\line(1,0){75} \newline
\textbf{7} \textit{Initiale} W   $\cdot$ \textit{Majuskel} T  \textbf{13} \textit{Majuskel} T  \textbf{23} \textit{Majuskel} T  \textbf{25} \textit{Majuskel} T  \newline
\line(1,0){75} \newline
\textbf{1} waz] swas V (T)  $\cdot$ rehtes] rechten W \textbf{4} der beider sol ich sin T  $\cdot$ wirdich der] Wer dich der U wurd ich ir V wird ich W \textbf{5} So sol ich mein preiß verkrencken W  $\cdot$ mir] mich T  $\cdot$ der] daz V \textbf{6} lât] laz U  $\cdot$ vürder] vor dir U \textbf{7} der] Die U  $\cdot$ Ampflisen] Anflizen U Anflisen V T anfolysen W \textbf{9} der] balde ir V W T \textbf{10} zuo] von W \textbf{12} sînen minnen] [sine*]: siner minnen V seinem mynnen W siner minne T \textbf{13} lebet in siner minne T  $\cdot$ lebete] lebet V  $\cdot$ zer] ser W \textbf{14} ir] div T  $\cdot$ im] ir T \textbf{15} sol in] sollen U sol V W T  $\cdot$ behalten] behaben T \textbf{17} boten] botteu W \textbf{19} eines] ainer W (T)  $\cdot$ Lanzidant] Lanzedant U T Lazedant V \textbf{20} ûz] von T  $\cdot$ Gruonlant] Gruͦenlant U [gruͦn*]: gruͦnlant V Grvenlant T \textbf{21} zuo] [ze]: von T  $\cdot$ Kärlingen] kerlingen U V W Franzen T \textbf{23} heizet] hies W  $\cdot$ Leidarz] leodarz U (V) leodars W lyadarz T \textbf{24} figunt] Figuͦnt U [f*]: filli kunt de  V Figunt de W fylly cons de T  $\cdot$ Schiolarz] Schelarz U [*]: Schelarz V schielars W tschalarz T \textbf{26} des saget er oͮch mere V · des sagt er vnß auch ein mere W · den seiter an dem mêre T \textbf{27} des] Sin V  $\cdot$ diu] \textit{om.} W  $\cdot$ Beaflurs] Beaflors U peaflurß W \textbf{28} Pansamurs] Pansamors U pansamurß W \textbf{29} feien] [*]: frien V \textbf{30} Liahturteltart] leituͦrtelart U Leoturteltart V leiturtelart W Latvrteltart T \newline
\end{minipage}
\end{table}
\end{document}
