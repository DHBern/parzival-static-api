\documentclass[8pt,a4paper,notitlepage]{article}
\usepackage{fullpage}
\usepackage{ulem}
\usepackage{xltxtra}
\usepackage{datetime}
\renewcommand{\dateseparator}{.}
\dmyyyydate
\usepackage{fancyhdr}
\usepackage{ifthen}
\pagestyle{fancy}
\fancyhf{}
\renewcommand{\headrulewidth}{0pt}
\fancyfoot[L]{\ifthenelse{\value{page}=1}{\today, \currenttime{} Uhr}{}}
\begin{document}
\begin{table}[ht]
\begin{minipage}[t]{0.5\linewidth}
\small
\begin{center}*D
\end{center}
\begin{tabular}{rl}
\textbf{447} & \begin{large}P\end{large}arzival, der werde degen,\\ 
 & het \textbf{des} lîbes sô gepflegen,\\ 
 & daz sîn zimierde rîche\\ 
 & stuont gar rîterlîche.\\ 
5 & in sölhem harnasch er reit,\\ 
 & dem ungelîch \textbf{was} jeniu kleit,\\ 
 & di\textit{u} gein im truoc der grâwe man.\\ 
 & daz ors ûzem pfade sân\\ 
 & kêrter mit dem zoume.\\ 
10 & dô nam sîn vrâgen goume\\ 
 & umbe der guoten liute vart.\\ 
 & mit süezer rede ers innen wart.\\ 
 & dô was des grâwen \textbf{rîters} klage,\\ 
 & daz im die heileclîchen tage\\ 
15 & niht h\textit{ü}lfen gein \textbf{al}sölhem site,\\ 
 & daz er sunder wâpen rite\\ 
 & ode daz er barvuoz gienge\\ 
 & unt des tages zît \textbf{begienge}.\\ 
 & Parzival \textbf{sprach zim} dô:\\ 
20 & "hêrre, ich erkenne sus noch sô,\\ 
 & wie des jâres \textbf{urhap} \textbf{gestêt}\\ 
 & ode wie der wochen zal gêt.\\ 
 & swie die tage sint genant,\\ 
 & daz ist mir allez unbekant.\\ 
25 & ich diende einem, der heizet got,\\ 
 & ê daz sô lasterlîchen spot\\ 
 & sîn gunst über mich \textbf{erhancte}.\\ 
 & mîn sin im nie gewancte,\\ 
 & von dem mir helfe was gesagt.\\ 
30 & nû ist sîn helfe an mir verzagt."\\ 
\end{tabular}
\scriptsize
\line(1,0){75} \newline
D Fr5 \newline
\line(1,0){75} \newline
\textbf{1} \textit{Initiale} D Fr5  \newline
\line(1,0){75} \newline
\textbf{1} Parzival] Parcifal D (Fr5) \textbf{6} was] warin Fr5 \textbf{7} diu] di D \textbf{9} kêrter] Kert er Fr5 \textbf{11} liute] livtin Fr5 \textbf{14} heileclîchen] heiligin Fr5 \textbf{15} hülfen] hvlfen D (Fr5)  $\cdot$ alsölhem] solchim Fr5 \textbf{18} begienge] inpfienge Fr5 \textbf{19} Parzival] Parcifal D Fr5 \textbf{20} ich erkenne] inirkenne Fr5 \textbf{21} urhap] zit Fr5 \textbf{22} ode] Alde Fr5  $\cdot$ wochen] wochun Fr5 \textbf{27} erhancte] virhancte Fr5 \newline
\end{minipage}
\hspace{0.5cm}
\begin{minipage}[t]{0.5\linewidth}
\small
\begin{center}*m
\end{center}
\begin{tabular}{rl}
 & \begin{large}P\end{large}arcifal, der werde degen,\\ 
 & hete \textbf{des} lî\textit{b}es sô gep\textit{f}legen,\\ 
 & daz sîn zimierde rîche\\ 
 & stuont gar ritterlîche.\\ 
5 & in solichem harnasch er reit,\\ 
 & dem ungelîch \textbf{was} jeniu kleit,\\ 
 & diu gegen ime truoc der grâwe man.\\ 
 & daz ros ûz dem pfade sân\\ 
 & kêret er mit dem zoume.\\ 
10 & dô nam sîn vrâgen goume\\ 
 & umb der guoten liute vart.\\ 
 & mit süeze\textit{r} rede ers innen wart.\\ 
 & dô was des grâwen \textbf{hêrren} klage,\\ 
 & daz ime die heileclîchen tage\\ 
15 & niht h\textit{ü}lfen gegen \textbf{al}solichem site,\\ 
 & daz er sunder wâpen rite\\ 
 & oder daz er barvuoz gienge\\ 
 & und des tages zît \textbf{begienge}.\\ 
 & Parcifal \textbf{sprach zuo ime} dô:\\ 
20 & "hêrre, ich erkenne sus noch sô,\\ 
 & wie des jâres \textbf{urhap} \textbf{stât}\\ 
 & oder wie der wochen zal gât.\\ 
 & wie die tage sint genant,\\ 
 & daz ist mir allez unbekant.\\ 
25 & ich diende einem, der heizet got,\\ 
 & ê daz sô lasterlîchen spot\\ 
 & sîn gunst über mich \textbf{verhan\textit{c}te}.\\ 
 & mîn \textit{sin} ime \textbf{ê} nie gewan\textit{c}te,\\ 
 & von dem mir helfe was gesaget.\\ 
30 & nû ist sîn helfe an mir verzaget."\\ 
\end{tabular}
\scriptsize
\line(1,0){75} \newline
m n o \newline
\line(1,0){75} \newline
\textbf{1} \textit{Initiale} m o   $\cdot$ \textit{Capitulumzeichen} n  \newline
\line(1,0){75} \newline
\textbf{2} lîbes] lides m  $\cdot$ gepflegen] gepfplegen m \textbf{3} zimierde] zume o \textbf{5} solichem] solich o \textbf{6} jeniu] jnne o \textbf{10} vrâgen] froge n frowen o \textbf{12} süezer] suͯssen m  $\cdot$ innen] [vmen]: jnnen o \textbf{13} des] [der]: des m  $\cdot$ klage] clagte o \textbf{14} heileclîchen] heillichen o \textbf{15} hülfen] hulffen m (n) hilffen o  $\cdot$ site] sitten n \textbf{16} rite] ritten n \textbf{17} barvuoz] furbasz o \textbf{18} begienge] begunne o \textbf{19} dô] da o \textbf{21} stât] gefar n o \textbf{22} gât] gar n o \textbf{23} \textit{Die Verse 447.23-24 fehlen} n o  \textbf{27} verhancte] verhante m \textbf{28} sin] \textit{om.} m sinne n  $\cdot$ nie] vs n  $\cdot$ gewancte] gewante m o \textbf{29} was] hett o \newline
\end{minipage}
\end{table}
\newpage
\begin{table}[ht]
\begin{minipage}[t]{0.5\linewidth}
\small
\begin{center}*G
\end{center}
\begin{tabular}{rl}
 & \begin{large}P\end{large}arzival, der werde degen,\\ 
 & het \textbf{des} \textit{lîbes} sô gepflegen,\\ 
 & daz sîn zimierde rîche\\ 
 & stuont gar rîterlîche.\\ 
5 & in solhem harnasche er reit,\\ 
 & dem \textit{u}ngelî\textit{ch} \textbf{was} jeniu kleit,\\ 
 & di\textit{u} gegen im truoc der grâwe man.\\ 
 & daz ors ûz dem pfade sân\\ 
 & kêrt er mit dem zoume.\\ 
10 & dô nam sîn vrâgen goume\\ 
 & umbe der guoten liute vart.\\ 
 & mit süezer rede ers innen wart.\\ 
 & dô was des grâwe\textit{n} \textbf{rîters} klage,\\ 
 & daz im die heiliclîchen tage\\ 
15 & niht h\textit{ü}lfen gein solhem site,\\ 
 & daz er sunder wâpen rite\\ 
 & oder daz er barvuoz gienge\\ 
 & unt des tages zît \textbf{enpfienge}.\\ 
 & Parzival \textbf{sprach ze im} dô:\\ 
20 & "hêrre, ich erkenne sus noch sô,\\ 
 & wie des jâres \textbf{zît} \textbf{gestêt}\\ 
 & oder wie der wochen zal gêt.\\ 
 & swie die tage sint genant,\\ 
 & daz ist mir allez unbekant.\\ 
25 & ich dien\textit{t}e einem, der heizet got,\\ 
 & ê daz sô lasterlîchen spot\\ 
 & sîn gunst über mich \textbf{verhancte}.\\ 
 & mîn sin im nie gewancte,\\ 
 & von dem mir helfe was gesaget.\\ 
30 & nû ist sîn helfe an mir verzaget."\\ 
\end{tabular}
\scriptsize
\line(1,0){75} \newline
G I O L M Z \newline
\line(1,0){75} \newline
\textbf{1} \textit{Initiale} G I O L Z  \textbf{13} \textit{Initiale} I  \newline
\line(1,0){75} \newline
\textbf{1} Parzival] PaRcival G Parzifal I L M ÷arcifal O Parcifal Z \textbf{2} het] Hat M  $\cdot$ lîbes] \textit{om.} G \textbf{3} sîn] sine L \textbf{4} gar] so Z \textbf{6} ungelîch] ivngelinge G  $\cdot$ was] waren O L (M) (Z)  $\cdot$ jeniu] ir I \textbf{7} diu] die G (O)  $\cdot$ im] \textit{om.} M \textbf{8} ûz] er vz O \textbf{9} er] \textit{om.} O \textbf{10} dô] Da O M Z  $\cdot$ vrâgen] vrage I L fragen die Z \textbf{11} liute] luten M \textbf{12} süezer rede] suszin redin M  $\cdot$ ers] er sin I M er des O \textbf{13} dô] Da O M Z  $\cdot$ grâwen] grawes G \textbf{14} heiliclîchen] hailigen I (M) heimlichen L \textbf{15} hülfen] hulfen G I (O) (M) Z  $\cdot$ solhem site] alsulchen siten M alsolhem site Z \textbf{16} rite] rete M \textbf{18} enpfienge] begienge I O L (M) \textbf{19} Parzival] Parzifal I L M Parcifal O Z  $\cdot$ dô] da M \textbf{20} hêrre] \textit{om.} I L  $\cdot$ sus] noch sus I \textbf{21} zît] vrhap O L M Z \textbf{23} swie] wie I (L) (M) \textbf{25} diente] diene G  $\cdot$ einem der] im der da I der L \textbf{26} lasterlîchen spot] lesterlicher [got]: spot M \textbf{27} verhancte] erhancte Z \textbf{28} sin] gunst I \newline
\end{minipage}
\hspace{0.5cm}
\begin{minipage}[t]{0.5\linewidth}
\small
\begin{center}*T
\end{center}
\begin{tabular}{rl}
 & \begin{large}P\end{large}arcifal, der werde degen,\\ 
 & hete \textbf{sînes} lîbes sô gepflegen,\\ 
 & daz sîn zimierde rîche\\ 
 & stuont gar rîterlîche.\\ 
5 & in solhem harnasche er reit,\\ 
 & dem ungelîch \textbf{wâren} jen\textit{iu} kleit,\\ 
 & di\textit{u} gegen im truoc der grâwe man.\\ 
 & Daz ors ûz dem pfade sân\\ 
 & kêrter mit dem zoume.\\ 
10 & dô nam sîn vrâgen goume\\ 
 & umb der guoten liute vart.\\ 
 & mit süezer rede ers innen wart.\\ 
 & Dô was des grâwen \textbf{rîters} klage,\\ 
 & daz im die heileclîchen tage\\ 
15 & niht h\textit{ü}lfen gegen sölhem site,\\ 
 & daz er sunder wâpen rite\\ 
 & oder daz er barvuoz gienge\\ 
 & unde des tages zît \textbf{begienge}.\\ 
 & Parcifal \textbf{zim sprach} dô:\\ 
20 & "hêrre, ich erkenne sus noch sô,\\ 
 & wie des jâres \textbf{urhap} \textbf{gestêt}\\ 
 & oder wie der wochen zal gêt.\\ 
 & swie die tage sin\textit{t} genant,\\ 
 & daz ist mir allez unbekant.\\ 
25 & ich diende einem, der heizet got,\\ 
 & ê daz sô lasterlîchen spot\\ 
 & sîn gunst über mich \textbf{verhancte}.\\ 
 & mîn sin im nie gewancte,\\ 
 & von dem mir helfe was gesaget.\\ 
30 & nû ist sîn helfe an mir verzaget."\\ 
\end{tabular}
\scriptsize
\line(1,0){75} \newline
T U V W Q R \newline
\line(1,0){75} \newline
\textbf{1} \textit{Initiale} T U W  \textbf{8} \textit{Majuskel} T  \textbf{13} \textit{Majuskel} T  \newline
\line(1,0){75} \newline
\textbf{1} Parcifal] Parzifal V PArtzifal W Parczifal R \textbf{2} sînes] dez V (W) (Q) (R)  $\cdot$ gepflegen] verwegen vnd gepflegen R \textbf{3} zimierde] [*]: zimierde V  $\cdot$ rîche] richen R \textbf{6} dem ungelîch] [Derunnglich]: Der unnglich Q Dem ye vngelich R  $\cdot$ jeniu] iene T ein U solhe V núwe R \textbf{7} diu] die T  $\cdot$ grâwe] werde W \textbf{8} ûz dem] ausserem W vff dem Q R \textbf{9} kêrter] Kerte W Kert er Q R  $\cdot$ zoume] zeme Q \textbf{11} guoten] gute Q \textbf{13} klage] clagen R \textbf{14} heileclîchen] [heili*]: heiliclichen V heimlichen Q \textbf{15} hülfen] hvlfen T (U) (V) (W) (Q) huffent R  $\cdot$ sölhem] solichen U (W) \textbf{19} Parcifal] Parzifal V Pattzifal W Partzifal Q Parczifal R  $\cdot$ zim sprach] sprach zuͦ im W (Q) (R) \textbf{21} wie] Mit Q  $\cdot$ gestêt] erget V \textbf{22} gêt] gestet V \textbf{23} swie] Wie U W Q R  $\cdot$ sint] sin T (Q) \textbf{24} unbekant] vnerkant R \textbf{25} diende] dienes W dinet Q \textbf{27} über] vmb W  $\cdot$ verhancte] erhanckte W (Q) \textbf{28} sin] sinne W  $\cdot$ im] [*]: im e V \textbf{29} was] [was*]: was V \newline
\end{minipage}
\end{table}
\end{document}
