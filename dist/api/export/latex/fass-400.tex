\documentclass[8pt,a4paper,notitlepage]{article}
\usepackage{fullpage}
\usepackage{ulem}
\usepackage{xltxtra}
\usepackage{datetime}
\renewcommand{\dateseparator}{.}
\dmyyyydate
\usepackage{fancyhdr}
\usepackage{ifthen}
\pagestyle{fancy}
\fancyhf{}
\renewcommand{\headrulewidth}{0pt}
\fancyfoot[L]{\ifthenelse{\value{page}=1}{\today, \currenttime{} Uhr}{}}
\begin{document}
\begin{table}[ht]
\begin{minipage}[t]{0.5\linewidth}
\small
\begin{center}*D
\end{center}
\begin{tabular}{rl}
\textbf{400} & \begin{large}A\end{large}ls mir diu âventiure sagete,\\ 
 & ir vederspil dâ jagete\\ 
 & den kranch oder swaz \textbf{vor in dâ} vlôch.\\ 
 & ein râvît von \textbf{Spane} hôch\\ 
5 & reit der künec Vergulaht.\\ 
 & sîn blic was tac wol bî der naht.\\ 
 & sîn geslehte sante Mazadan\\ 
 & vür den berc ze Famorgan.\\ 
 & sîn art was von \textbf{der} feien.\\ 
10 & in dûhte, er sæhe den meien\\ 
 & in rehter zît von bluomen gar,\\ 
 & swer nam des küneges varwe war.\\ 
 & Gawanen des bedûhte,\\ 
 & dô der künec sô gein im lûhte,\\ 
15 & ez wære der ander Parzival\\ 
 & unt daz er Gahmuretes mâl\\ 
 & hete, alsô diz mære weiz,\\ 
 & dô der reit în ze Kanvoleiz.\\ 
 & \textbf{Ein} reiger tet durch vluht entwîch\\ 
20 & in einen muorigen tîch.\\ 
 & den brâhten valken dar gehurt.\\ 
 & der künec suochte unrehten vurt,\\ 
 & in valken hilfe wart er naz.\\ 
 & sîn ors verlôs er umbe daz,\\ 
25 & dar zuo al di\textit{u} kleider sîn,\\ 
 & doch schiet \textbf{er} valken von ir pîn.\\ 
 & daz nâmen die valkenære.\\ 
 & ob daz ir reht iht wære?\\ 
 & ez was ir reht, si soltenz hân.\\ 
30 & man muose ouch si bî rehte lân.\\ 
\end{tabular}
\scriptsize
\line(1,0){75} \newline
D \newline
\line(1,0){75} \newline
\textbf{1} \textit{Initiale} D  \textbf{19} \textit{Majuskel} D  \newline
\line(1,0){75} \newline
\textbf{15} Parzival] Parzifal D \textbf{16} Gahmuretes] Gahmvretes D \textbf{25} diu] di D \newline
\end{minipage}
\hspace{0.5cm}
\begin{minipage}[t]{0.5\linewidth}
\small
\begin{center}*m
\end{center}
\begin{tabular}{rl}
 & als mir diu âventiure sagete,\\ 
 & ir vederspil d\textit{â} jagete\\ 
 & den kranich oder waz \textbf{vor in dâ} vlôch.\\ 
 & ein râvît von \textbf{Hispanie} hôch\\ 
5 & reit der küni\textit{c} \textit{V}ergulaht.\\ 
 & sîn blic was tac wol bî der naht.\\ 
 & sîn geslehte sante Ma\textit{za}dan\\ 
 & vür den berc ze Famorgan.\\ 
 & sîn art was von \textbf{den} feien.\\ 
10 & in dûhte, er sæhe den meien\\ 
 & in rehter zît von bluome\textit{n} \textit{g}ar,\\ 
 & wer nam des küniges varwe war.\\ 
 & Gawanen des bedûhte,\\ 
 & dô der künic sô gegen ime lûhte,\\ 
15 & ez wære der ander Parcifal\\ 
 & und daz er Gahmuretes mâl\\ 
 & hete, alsô diz mære weiz,\\ 
 & dô der reit în ze Kanvoleiz.\\ 
 & \textbf{\begin{large}E\end{large}in} reiger tet durch vluht entwîch\\ 
20 & in einen muorigen tîch.\\ 
 & den brâhte\textit{n} valken dar gehurt.\\ 
 & der künic suochte unrehten vurt,\\ 
 & in valken helfe wart er naz.\\ 
 & sîn ros verlôs er umbe daz,\\ 
25 & dar zuo alliu diu kleider sîn,\\ 
 & doch schiet \textbf{der} valken von ir pîn.\\ 
 & daz nâmen die valkenære.\\ 
 & ob daz ir reht iht wære?\\ 
 & ez was ir reht, si soltenz hân.\\ 
30 & man muose ouch si bî rehte lân.\\ 
\end{tabular}
\scriptsize
\line(1,0){75} \newline
m n o \newline
\line(1,0){75} \newline
\textbf{19} \textit{Initiale} m  \newline
\line(1,0){75} \newline
\textbf{1} sagete] saget n o \textbf{2} dâ] do m n o  $\cdot$ jagete] jaget n [sa*]: jaget o \textbf{3} den] Der n  $\cdot$ dâ] do n o \textbf{4} râvît] ranit n  $\cdot$ Hispanie] hẏspanie n rispanie o \textbf{5} künic Vergulaht] kunig von vergulaht m konig vergúlaht o \textbf{6} wol] \textit{om.} n \textbf{7} Mazadan] Mare dan m maradan n o \textbf{8} Famorgan] famorgen o \textbf{11} bluomen gar] blummen dar vnd gar m \textbf{12} varwe] frowe o \textbf{16} Gahmuretes] gamurehtes n gahamutes o \textbf{17} diz] dise n \textbf{18} în] \textit{om.} o  $\cdot$ ze Kanvoleiz] zekanvoleis m zuͯ kanfoleis n zuͦ kamfoleis o \textbf{20} Jn einem múrigem (muͯrigen o ) dicht ([tisch]: tich o ) n (o) \textbf{21} brâhten] brahte m  $\cdot$ gehurt] gehort o \textbf{22} suochte] sucht n o \textbf{23} er] ir o \textbf{26} der] er n o \textbf{30} muose] mus m muͯsz n o \newline
\end{minipage}
\end{table}
\newpage
\begin{table}[ht]
\begin{minipage}[t]{0.5\linewidth}
\small
\begin{center}*G
\end{center}
\begin{tabular}{rl}
 & als mir diu âventiure sagete,\\ 
 & ir vederspil dâ jagete\\ 
 & den kranch oder swaz \textbf{dâ vor in} vlôc\textit{h}.\\ 
 & ein râvît von \textbf{Spange} hôch\\ 
5 & reit der künic Vergulaht.\\ 
 & sîn blic was tac wol bî der naht.\\ 
 & sîn geslähte sante Mazadan\\ 
 & vür den berc ze Phimurgan.\\ 
 & sîn \textit{art} was von \textbf{der} feien.\\ 
10 & in dûhter sæhe den meien\\ 
 & in rehter zît von bluomen gar,\\ 
 & swer nam des küniges varwe war.\\ 
 & Gawanen des bedûhte,\\ 
 & dô der künic sô gein im lûhte,\\ 
15 & ez wære der ander Parzival\\ 
 & unt daz er Gahmuretes mâl\\ 
 & hete, als diz mære weiz,\\ 
 & dô \textit{d}er reit în ze Kanvoleiz.\\ 
 & \textbf{ein} reiger tet durch vluht entwîch\\ 
20 & in einen muorigen tîch.\\ 
 & den brâhten valken dar gehurt.\\ 
 & der künic suochte unrehten vurt,\\ 
 & in valken hilfe wart er naz.\\ 
 & sîn ors verlôs er umbe daz,\\ 
25 & dar zuo al diu kleider sîn,\\ 
 & doch schiet \textbf{er} valken von ir pîn.\\ 
 & \textit{da}z nâmen die valkenære.\\ 
 & op daz ir reht iht wære?\\ 
 & ez was ir reht, si soltenz hân.\\ 
30 & man muose ouch si bî rehte lân.\\ 
\end{tabular}
\scriptsize
\line(1,0){75} \newline
G I O L M Q R Z \newline
\line(1,0){75} \newline
\textbf{1} \textit{Überschrift:} Aber hern gawans auentevre eine Z   $\cdot$ \textit{Initiale} I O L M Z   $\cdot$ \textit{Capitulumzeichen} R  \textbf{17} \textit{Initiale} I  \newline
\line(1,0){75} \newline
\textbf{1} \textit{Die Verse 370.13-412.12 fehlen} Q   $\cdot$ als] ÷ls O  $\cdot$ sagete] sagt O \textbf{2} dâ] do R \textbf{3} den] einen I Der O  $\cdot$ swaz] waz L (R) swarcz M  $\cdot$ dâ vor in] vor im da I vor in do O vor in da L vor yn M vor Jn R (Z)  $\cdot$ vlôch] floc G \textbf{4} râvît] ros I gereit R  $\cdot$ Spange] ýspanie L spane O Spanie M Z spangen R \textbf{5} Vergulaht] [vergvlaht]: Vergvlaht O vergvlaht L Z vergulacht M \textbf{6} wol bî] Gein I bi O \textbf{7} geslähte] geschechte R gesihte Z  $\cdot$ Mazadan] Marzadan R \textbf{8} ze] \textit{om.} I  $\cdot$ Phimurgan] feimurgan I famorgan O Z famvͯrgan L feimorgan M frandrigan R \textbf{9} art] \textit{om.} G  $\cdot$ der] den O R \textbf{12} swer] Wer M R \textbf{13} Gawanen] Gawan I (M)  $\cdot$ bedûhte] duͯhte L \textbf{14} dô] Da M Z \textbf{15} Parzival] parzifal I L M Barceval O parczifal R parcifal Z \textbf{16} Gahmuretes] gahmvrets G Gamuretes O (Z) Gahmuͯretes L gamuͯretis M gamahurtes R \textbf{17} als] als ich I \textbf{18} dô] Da M Z  $\cdot$ der reit în] er reit in G der raitter in I der reit L  $\cdot$ ze Kanvoleiz] zekanvoleiz G zechanfoleiz I ze canvoleiz O zcu kamvoleisz M ze kanvoleis R zv kanfoleiz Z \textbf{19} tet] der R Z  $\cdot$ entwîch] entweic I entweich R Z \textbf{20} Jn eyneme morgen rich M \textbf{21} valken] [valschen]: valchen I \textbf{22} suochte] svͦht O (Z)  $\cdot$ unrehten] vnrehte I \textbf{23} naz] [nach]: naz Z \textbf{26} er] >er< O  $\cdot$ ir] \textit{om.} R \textbf{27} daz] ez G  $\cdot$ nâmen] nemen M  $\cdot$ die] \textit{om.} O L R \textbf{28} daz] diz L  $\cdot$ iht] \textit{om.} L \textbf{30} muose] muͤst I  $\cdot$ ouch si] si O (R) ouch sie ouch M  $\cdot$ bî] bi ir O \newline
\end{minipage}
\hspace{0.5cm}
\begin{minipage}[t]{0.5\linewidth}
\small
\begin{center}*T
\end{center}
\begin{tabular}{rl}
 & \begin{large}A\end{large}ls mir diu âventiure sagete,\\ 
 & ir vederspil dâ jagete\\ 
 & den kranch oder swaz \textbf{vor in dâ} vlôch.\\ 
 & Ein râvît von \textbf{Spanie} hôch\\ 
5 & reit der künec Vergulaht.\\ 
 & sîn blic was tac wol bî der naht.\\ 
 & sîn geslehte sante Mazadan\\ 
 & vür den berc ze Feimorgan.\\ 
 & sîn art was von \textbf{der} feien.\\ 
10 & in dûhte, er sæhe den meien\\ 
 & in rehter zît von bluomen gar,\\ 
 & swer nam des küneges varwe war.\\ 
 & Gawan des bedûhte,\\ 
 & dô der künec sô gegen im lûhte,\\ 
15 & ez wære der ander Parcifal\\ 
 & unde daz er Gahmuretes mâl\\ 
 & hete, alse diz mære weiz,\\ 
 & dô der reit în ze Kanvoleiz.\\ 
 & \textbf{\begin{large}S\end{large}în} rei\textit{ger} tet durch vluht entwîch\\ 
20 & in einen muorigen tîch.\\ 
 & den brâhten valken dar gehurt.\\ 
 & der künec suochte unrehten vurt,\\ 
 & in valken hilfe wart er naz.\\ 
 & sîn ors verlôs er umbe daz,\\ 
25 & dar zuo aldiu kleider sîn,\\ 
 & doch schiet \textbf{er} \textbf{die} valken von ir pîn.\\ 
 & daz nâmen die valkenære.\\ 
 & ob daz ir reht iht wære?\\ 
 & ez was ir reht, si soltenz hân.\\ 
30 & man muose ouch si bî rehte lân.\\ 
\end{tabular}
\scriptsize
\line(1,0){75} \newline
T U V W \newline
\line(1,0){75} \newline
\textbf{1} \textit{Initiale} T U V W  \textbf{4} \textit{Majuskel} T  \textbf{19} \textit{Initiale} T U  \newline
\line(1,0){75} \newline
\textbf{2} dâ] do U V W \textbf{3} kranch] kúnig W  $\cdot$ swaz] waz U (W)  $\cdot$ dâ] do U V W \textbf{4} râvît] ravt U  $\cdot$ Spanie] ẏspanie V spange W \textbf{5} Vergulaht] vergulacht U (W) virgulaht V \textbf{6} wol] \textit{om.} W \textbf{8} vür] Eúr W  $\cdot$ Feimorgan] Fêimorgan T femurgan V famorgan W \textbf{9} der] den V \textbf{12} swer] Wer U W \textbf{13} Gawan] Gawanen U V W \textbf{14} sô] \textit{om.} W \textbf{15} Parcifal] Parzifal T (V) partzifal W \textbf{16} Gahmuretes] Gahmvretes T gamurettes V W \textbf{17} diz] dise U \textbf{18} dô der] Der do W  $\cdot$ Kanvoleiz] kanuoleiß W \textbf{19} Sîn] Ein U V W  $\cdot$ reiger] reit T \textbf{21} dar] \textit{om.} U \textbf{22} unrehten] vnrehte V \textbf{26} die] \textit{om.} W \textbf{27} die] \textit{om.} W \textbf{28} iht] ich W \textbf{29} ir] \textit{om.} W \textbf{30} muose] mvese T muͤste V \newline
\end{minipage}
\end{table}
\end{document}
