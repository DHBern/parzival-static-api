\documentclass[8pt,a4paper,notitlepage]{article}
\usepackage{fullpage}
\usepackage{ulem}
\usepackage{xltxtra}
\usepackage{datetime}
\renewcommand{\dateseparator}{.}
\dmyyyydate
\usepackage{fancyhdr}
\usepackage{ifthen}
\pagestyle{fancy}
\fancyhf{}
\renewcommand{\headrulewidth}{0pt}
\fancyfoot[L]{\ifthenelse{\value{page}=1}{\today, \currenttime{} Uhr}{}}
\begin{document}
\begin{table}[ht]
\begin{minipage}[t]{0.5\linewidth}
\small
\begin{center}*D
\end{center}
\begin{tabular}{rl}
\textbf{143} & \textbf{Dô bôt im} der knappe \textbf{sân}\\ 
 & vroun Jeschuten vürspan.\\ 
 & dô \textbf{ez} der \textbf{vilân} \textbf{ersach},\\ 
 & sîn munt dô lachete unt sprach:\\ 
5 & "wiltû belîben, \textbf{liebez} kint,\\ 
 & dich êrent alle, die \textbf{hinne} sint."\\ 
 & "wiltû mich hînte wol spîsen\\ 
 & unt morgen rehte wîsen\\ 
 & gein Artuse, dem bin ich holt,\\ 
10 & sô mac belîben \textbf{dir} daz golt."\\ 
 & "\textbf{Diz} tuon ich", sprach der vilân.\\ 
 & "ine gesach nie lîp sô wolgetân.\\ 
 & ich bringe dich durch wunder\\ 
 & vür des küneges tavelrunder."\\ 
15 & \textit{\begin{large}D\end{large}}ie naht beleip der knappe dâ,\\ 
 & man sach in \textbf{s}morgens anderswâ.\\ 
 & des tages er kûme erbeite,\\ 
 & der wirt ouch sich bereite\\ 
 & unt lief im vor, der knappe nâch\\ 
20 & reit. dô was in beiden gâch.\\ 
 & Mîn hêr Hartman von Ouwe\\ 
 & \textbf{unt} vrou Ginover, iwer vrouwe,\\ 
 & unt iwer hêrre, \textbf{der} künec Artus,\\ 
 & \textbf{den} kumt ein mîn \textbf{gast} ze hûs.\\ 
25 & bit hüeten sîn vor spotte.\\ 
 & ern ist gîge noch \textbf{diu} rotte.\\ 
 & si sulen ein ander gampel nemen,\\ 
 & des lâzen sich durch zuht gezemen.\\ 
 & Anders iwer vrou Enide\\ 
30 & unt ir muoter \textbf{Karsnafide}\\ 
\end{tabular}
\scriptsize
\line(1,0){75} \newline
D \newline
\line(1,0){75} \newline
\textbf{1} \textit{Majuskel} D  \textbf{11} \textit{Majuskel} D  \textbf{15} \textit{Initiale} D  \textbf{21} \textit{Majuskel} D  \textbf{29} \textit{Majuskel} D  \newline
\line(1,0){75} \newline
\textbf{2} Jeschuten] Jescvten D \textbf{15} Die] ÷ie \textit{nachträglich korrigiert zu:} Die D \textbf{21} Ouwe] Oͮwe D \textbf{29} Enide] Enîde D \textbf{30} Karsnafide] karsnafîde D \newline
\end{minipage}
\hspace{0.5cm}
\begin{minipage}[t]{0.5\linewidth}
\small
\begin{center}*m
\end{center}
\begin{tabular}{rl}
 & \textbf{dô bôt ime} der knappe \textbf{sân}\\ 
 & vrouwen J\textit{e}sch\textit{u}ten vürspa\textit{n}.\\ 
 & dô \textbf{daz} der \textbf{vilân} \textbf{ersach},\\ 
 & sîn munt dô lachete und sprach:\\ 
5 & "wiltû blîben, \textbf{süezez} kint,\\ 
 & dich êrent alle, die \textbf{innen} sint."\\ 
 & "wilt dû mich hînt wol spîsen\\ 
 & und morgen rehte wîsen\\ 
 & gegen Artuse, dem bin ich holt, ..."\\ 
10 & "sô mac belîben \textbf{mir} daz golt.\\ 
 & \textbf{daz} \textit{tuon} ich", sprach der vilân.\\ 
 & "\textit{in}e gesach nie lîp sô wol getân.\\ 
 & ich bringe dich durch wunder\\ 
 & vür des küniges tavelrunder."\\ 
15 & die naht beleip der knappe d\textit{â},\\ 
 & man sach in morgens anderswâ.\\ 
 & des tages er kûme erbeite.\\ 
 & der wirt ouch sich bereite\\ 
 & und lief \dag in\dag  vor, der knappe nâch\\ 
20 & reit. dô was in beiden gâch.\\ 
 & mîn hêr Hartman von Ouwe,\\ 
 & vrouwe Ginover, iuwer vrouwe,\\ 
 & und iuwer hêrre, \textbf{der} künic Artus,\\ 
 & \textbf{den} kum\textit{t} ein mîn \textbf{gast} ze hûs.\\ 
25 & bittet \textbf{si} hüete\textit{n} sîn vor spotte.\\ 
 & ern ist gîge noch \textbf{diu} rotte.\\ 
 & si suln ein ander ga\textit{m}pel \textit{n}emen,\\ 
 & des lâzen sich durch zuht gezemen.\\ 
 & anders iuwer vrouwe Enite\\ 
30 & und ir muoter \textbf{Kasniafite},\\ 
\end{tabular}
\scriptsize
\line(1,0){75} \newline
m n o \newline
\line(1,0){75} \newline
\newline
\line(1,0){75} \newline
\textbf{1} bôt] bote n bate o \textbf{2} vrouwen] Frouwe m n (o)  $\cdot$ Jeschuten] iscitten m iscuten n (o)  $\cdot$ vürspan] fur spange m \textbf{4} dô lachete] erlachet n do lachet o \textbf{6} dich] Vnd dich o  $\cdot$ innen] hie n o \textbf{7} hînt] \textit{om.} n \textbf{9} Artuse] artusen n artuͯse o  $\cdot$ dem] dein o \textbf{11} tuon] \textit{om.} m \textbf{12} ine] Me m Jch n o \textbf{15} dâ] do m n o \textbf{18} der] durch o \textbf{21} Hartman] hartmann o  $\cdot$ Ouwe] auwe o \textbf{22} Ginover] ginofer n ginifuͯr o  $\cdot$ vrouwe] frage o \textbf{23} der künic] \textit{om.} n  $\cdot$ Artus] artuͯs o \textbf{24} den] Dem n  $\cdot$ kumt] kump m \textbf{25} hüeten] huttent m \textbf{27} gampel nemen] gappel niemen m \textbf{28} des] Das o  $\cdot$ lâzen] lossent n  $\cdot$ sich] sich [z]: so n \textbf{29} Enite] enitte m \textbf{30} Kasniafite] kasniafitte m kasmafite n kosniasite o \newline
\end{minipage}
\end{table}
\newpage
\begin{table}[ht]
\begin{minipage}[t]{0.5\linewidth}
\small
\begin{center}*G
\end{center}
\begin{tabular}{rl}
 & \textbf{im bôt} der knappe \textbf{wolgetân}\\ 
 & vroun Jeschuten vürspan.\\ 
 & dô \textbf{daz} der \textbf{vilân} \textbf{ersach},\\ 
 & sîn munt dô lachte und sprach:\\ 
5 & "wil dû belîben, \textbf{süezez} kint,\\ 
 & dich êrent alle, die \textbf{hinne} sint."\\ 
 & "wil dû mich hînt wol spîsen\\ 
 & unde morgen rehte wîsen\\ 
 & gein Artuse, dem bin ich holt,\\ 
10 & sô mac belîben \textbf{dir} daz golt."\\ 
 & "\textbf{daz} tuon ich", sprach der vilân.\\ 
 & "ichne gesach nie lîp sô wolgetân.\\ 
 & ich bringe dich durch wunder\\ 
 & vür des küniges tavelrunder."\\ 
15 & die naht beleip der knappe dâ,\\ 
 & man sach in\textbf{s} morgens anderswâ.\\ 
 & des tages er \textbf{vil} kûme e\textit{r}beit.\\ 
 & \textit{d}er wirt \textit{ouch} \textit{sich} bereit\\ 
 & \textit{und} lief im vor, der knappe nâch\\ 
20 & reit. dô was in beiden gâch.\\ 
 & mîn hêr Hartman von Ouwe,\\ 
 & vrô Schinover, iwer vrouwe,\\ 
 & \begin{large}U\end{large}nd iwer hêrre, \textbf{der} künic Artus,\\ 
 & \textbf{dem} kumet ein mîn \textbf{vriunt} ze hûs.\\ 
25 & bit hüeten sîn vor spote.\\ 
 & ern ist gîge noch \textbf{diu} rote.\\ 
 & si sulen ein ander gampel nemen,\\ 
 & des lâzen sich durch zuht gezemen.\\ 
 & anders iwer vrouwe Enite\\ 
30 & unde ir muoter \textbf{Kursenite}\\ 
\end{tabular}
\scriptsize
\line(1,0){75} \newline
G I O L M Q R Z \newline
\line(1,0){75} \newline
\textbf{7} \textit{Initiale} M  \textbf{11} \textit{Initiale} Q  \textbf{15} \textit{Initiale} O L R Z  \textbf{17} \textit{Überschrift:} Aventiwer wie Parzifal von vrvͦn sygnen schiet Vnde den roeten riter slvͦch I  Initiale I  \textbf{23} \textit{Initiale} G  \textbf{29} \textit{Initiale} I  \newline
\line(1,0){75} \newline
\textbf{2} vroun] Vrow L (Q) (R)  $\cdot$ Jeschuten] ieschuten G I Ieschv̂ten O Jescuͯten L Jescuten M (Q) Jesutten R iescuten Z \textbf{3} dô] Da M  $\cdot$ vilân] wirt R  $\cdot$ ersach] er giscach M der sach Z \textbf{4} dô] da M der Z  $\cdot$ lachte] lachet I L \textbf{5} süezez] liebes R \textbf{6} hinne] hie I O M \textbf{7} hînt] \textit{om.} L hút R  $\cdot$ wol spîsen] volspisen M \textbf{9} Artuse] Artus I (Q) Artuͯse L hartus M  $\cdot$ bin ich] ich bin Q \textbf{10} belîben dir] dir bliben R  $\cdot$ daz] diz I \textbf{12} ichne] ich I (O) (L) (R)  $\cdot$ gesach] ersach L sach M  $\cdot$ lîp] [wip]: lip G \textbf{13} dich] \textit{om.} Q  $\cdot$ wunder] wudir M wurder R \textbf{14} tavelrunder] tauelrunde R \textbf{15} die] ÷ie O  $\cdot$ naht] nach Q R  $\cdot$ dâ] do O L Q \textbf{16} sach] Geshac I  $\cdot$ ins morgens] in morgen I in morndes R \textbf{17} er] \textit{om.} Q  $\cdot$ vil] \textit{om.} O L M Q R Z  $\cdot$ erbeit] enbeit G erbette M \textbf{18} nv was oͮch der wirt bereit G  $\cdot$ der] Des M  $\cdot$ wirt ouch] wir R \textbf{19} vnd lief fur den chnappen der rait nach I  $\cdot$ und] der G  $\cdot$ im] yn M \textbf{20} in was beiden vil Gach I  $\cdot$ dô] da M Z doch Q \textbf{21} Ouwe] oͮwe G owe I (L) M R Z awe O ofwe Q \textbf{22} Schinover] Ginuffere I Ginover O (M) Z Gýnovire L Synorer Q gienover R \textbf{23} iwer] myn M  $\cdot$ hêrre der künic] herre her O (M) herre L Q (R)  $\cdot$ Artus] Artuͯs L \textbf{24} dem] ev I Den O M R Z  $\cdot$ mîn vriunt] frewnd Q man froͯmd R minne frvnt Z \textbf{25} bit] Beit Q Sit R  $\cdot$ hüeten] hute M \textbf{26} ern] Er O R  $\cdot$ ist] ist nicht Q  $\cdot$ diu] \textit{om.} I L \textbf{28} sich] sie sich L  $\cdot$ zuht] ir zuht I \textbf{29} Enite] enîte O enide M enrede Z \textbf{30} Kursenite] Garsinifite I karsinifîte O karsinefite L karsinafide M karsmesite Q karsmenite R kar nefrede Z \newline
\end{minipage}
\hspace{0.5cm}
\begin{minipage}[t]{0.5\linewidth}
\small
\begin{center}*T (U)
\end{center}
\begin{tabular}{rl}
 & \textbf{im bôt} der knappe \textbf{wol getân}\\ 
 & vroun Jeschuten vürspan.\\ 
 & dô \textbf{daz} der \textbf{vischer} \textbf{gesach},\\ 
 & sîn munt dô lachete und sprach:\\ 
5 & "wilt dû blîben, \textbf{süezez} kint,\\ 
 & dich êrent alle, die \textbf{hie în} sint."\\ 
 & "wilt dû mich hînt wol spîsen\\ 
 & und morgen rehte wîsen\\ 
 & gein Artuse, dem bin ich holt,\\ 
10 & sô mac blîben \textbf{dir} daz golt."\\ 
 & "\textbf{daz} tuon ich", sprach der vilân.\\ 
 & "ich engesach nie lîp sô wol getân.\\ 
 & ich bringe dich durch wunder\\ 
 & vür des küneges tavelrunder."\\ 
15 & die naht bleip der knappe dâ,\\ 
 & man sach in \textbf{des} morgens anderswâ.\\ 
 & des tages er kûme erbeite.\\ 
 & der wirt ouch sich bereite\\ 
 & und lief im vor, der knappe, nâch\\ 
20 & \textbf{er} reit. dô was in beiden gâch.\\ 
 & mîn hêrre Hartman von Ouwe\\ 
 & \textbf{und} vrou Gynover, iuwer vrouwe,\\ 
 & und iuwer hêrre, künec Artus,\\ 
 & \textbf{dô} kumet ein mîn \textbf{vriunt} zuo hûs.\\ 
25 & \textit{bitet hüeten sîn vor spotte.}\\ 
 & \textit{er enist gîge noch rotte.}\\ 
 & si soln ein ander gam\textit{p}el nemen,\\ 
 & des lâzen sich durch zuht gezemen.\\ 
 & anders iuwer vrouwe Enite\\ 
30 & und ir muoter \textbf{Karsenafite}\\ 
\end{tabular}
\scriptsize
\line(1,0){75} \newline
U V W T \newline
\line(1,0){75} \newline
\textbf{1} \textit{Majuskel} T  \textbf{7} \textit{Majuskel} T  \textbf{11} \textit{Initiale} T  \textbf{15} \textit{Initiale} V W   $\cdot$ \textit{Majuskel} T  \textbf{18} \textit{Majuskel} T  \textbf{21} \textit{Majuskel} T  \newline
\line(1,0){75} \newline
\textbf{1} getân] geboren vnd getan W \textbf{2} vroun] Vro U (W)  $\cdot$ Jeschuten] Jescuten U (T) iescuten V iestusen W \textbf{3} \textit{Versfolge 143.4-3} W   $\cdot$ vischer] vilan T  $\cdot$ gesach] ersach W \textbf{4} dô lachete und] lachende T \textbf{6} în] \textit{om.} W \textbf{7} wol] \textit{om.} T \textbf{10} blîben dir] dir blîben T  $\cdot$ daz] dis W \textbf{11} daz] Dis W (T) \textbf{12} engesach] gesach V W \textbf{14} des küneges] die T \textbf{15} dâ] do U W \textbf{16} des] \textit{om.} W \textbf{18} ouch sich] sich auch W \textbf{20} In was baiden harte gach W  $\cdot$ er] \textit{om.} T \textbf{21} Ouwe] Owe U Oͮwe V auwe W ôuwe T \textbf{22} und] \textit{om.} T  $\cdot$ Gynover] Schynover U Gynouier V tschinouer W Gynouer T \textbf{23} iuwer] \textit{om.} W  $\cdot$ künec] der kúnig W \textbf{24} dô kumet] [*]: v́ch kvmt V Den kennet W [v*]: den kvmt T \textbf{25} \textit{Die Verse 143.25-26 fehlen} U  \textbf{26} gîge] die gige W (T)  $\cdot$ rotte] die rotte W (T) \textbf{27} gampel] gamel U \textbf{28} sich] si T \textbf{29} Enite] enyte V Enîte T \textbf{30} Karsenafite] Carsenite U Carsenẏte V karsiuesite W \newline
\end{minipage}
\end{table}
\end{document}
