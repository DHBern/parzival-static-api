\documentclass[8pt,a4paper,notitlepage]{article}
\usepackage{fullpage}
\usepackage{ulem}
\usepackage{xltxtra}
\usepackage{datetime}
\renewcommand{\dateseparator}{.}
\dmyyyydate
\usepackage{fancyhdr}
\usepackage{ifthen}
\pagestyle{fancy}
\fancyhf{}
\renewcommand{\headrulewidth}{0pt}
\fancyfoot[L]{\ifthenelse{\value{page}=1}{\today, \currenttime{} Uhr}{}}
\begin{document}
\begin{table}[ht]
\begin{minipage}[t]{0.5\linewidth}
\small
\begin{center}*D
\end{center}
\begin{tabular}{rl}
\textbf{163} & dâ wil ich iu dienen nâch,\\ 
 & sît mir mîn muoter des verjach."\\ 
 & "\begin{large}S\end{large}ît ir durch râtes schulde\\ 
 & her komen, iwer hulde\\ 
5 & müezet ir mir durch râten lân,\\ 
 & \textbf{unt} wolt ir râtes volge hân."\\ 
 & Dô warf der vürste mære\\ 
 & einen mûzersperwære\\ 
 & von der hende. in die burc er swanc.\\ 
10 & ein guldîn schelle dran erklanc.\\ 
 & daz was ein bote. dô \textbf{kam in} sân\\ 
 & \textbf{vil} junchêrren wolgetân.\\ 
 & er \textbf{bat} den gast, den er dâ sach,\\ 
 & în vüeren unt schaffen sîn gemach.\\ 
15 & \textbf{der} sprach: "mîn muoter sagt alwâr:\\ 
 & \textbf{altmannes} rede stêt niht ze vâr."\\ 
 & Hin în \textbf{vuorten si in} al zehant,\\ 
 & dâ er manegen \textbf{werden} ritter vant.\\ 
 & ûf \textbf{dem hove} an einer stat\\ 
20 & ieslîcher in erbeizen bat.\\ 
 & \textbf{Dô sprach, an dem} was tumpheit schîn:\\ 
 & "mich hiez ein künec ritter sîn.\\ 
 & swaz halt \textbf{drûffe mir} geschiht,\\ 
 & ine kum von disem orse niht.\\ 
25 & gruoz gein iu riet \textbf{mîn} muoter mir."\\ 
 & si dankten beidiu im unt ir.\\ 
 & dô \textbf{daz} grüezen \textbf{wart} getân\\ 
 & - daz ors was müede unt ouch der man -,\\ 
 & maneger bete si gedâhten,\\ 
30 & \textbf{ê} si in von dem orse brâhten\\ 
\end{tabular}
\scriptsize
\line(1,0){75} \newline
D \newline
\line(1,0){75} \newline
\textbf{3} \textit{Initiale} D  \textbf{7} \textit{Majuskel} D  \textbf{17} \textit{Majuskel} D  \textbf{21} \textit{Majuskel} D  \newline
\line(1,0){75} \newline
\newline
\end{minipage}
\hspace{0.5cm}
\begin{minipage}[t]{0.5\linewidth}
\small
\begin{center}*m
\end{center}
\begin{tabular}{rl}
 & dâ wil ic\textit{h}, \textbf{\textit{h}êrre}, iu dienen nâch,\\ 
 & sît mir mîn muoter des verjach."\\ 
 & "sît ir durch râtes schulde\\ 
 & her komen, iuwer hulde\\ 
5 & müezet ir mir durch râten lân,\\ 
 & \textbf{und} wellet i\textit{r r}âtes volge hân."\\ 
 & \begin{large}D\end{large}ô warf der vürste mære\\ 
 & einen mûzersperwære\\ 
 & von der hende. in die burc er swanc.\\ 
10 & ein guldîn schelle dran erklanc.\\ 
 & daz was ein bote. d\textit{ô} \textbf{kômen} sân\\ 
 & junchêrren wol getân.\\ 
 & er \textbf{\textit{b}a\textit{t}} den gast, den er d\textit{â} sach,\\ 
 & în vüeren und schaffen sîn gemach.\\ 
15 & \textbf{der} sprach: "mîn muoter saget alwâr:\\ 
 & \textbf{altmannes} rede stât niht ze vâr."\\ 
 & hin în \textbf{sin vuorten} al zehant,\\ 
 & dâ er manigen ritter vant.\\ 
 & ûf \textbf{dem hove} an einer stat\\ 
20 & ieglîcher in erbeizen bat.\\ 
 & \textbf{dô sprach, an dem} was tumpheit schîn:\\ 
 & "mich \textit{hiez} ein künic ritter sîn.\\ 
 & waz halt \textbf{mir drûffe} geschiht,\\ 
 & \textit{in}e kume von disem rosse niht.\\ 
25 & gruoz gegen iu r\textit{ie}t \textbf{mîn} muoter mir."\\ 
 & si danketen beidiu ime und ir.\\ 
 & dô \textbf{\textit{d}az} grüezen \textbf{wart} getân\\ 
 & - daz ros was müede und ouch der man -,\\ 
 & maniger bete si gedâhten,\\ 
30 & \textbf{ê} si in von dem rosse brâhten\\ 
\end{tabular}
\scriptsize
\line(1,0){75} \newline
m n o \newline
\line(1,0){75} \newline
\textbf{7} \textit{Initiale} m   $\cdot$ \textit{Capitulumzeichen} n  \newline
\line(1,0){75} \newline
\textbf{1} ich hêrre] ich uͯch herre m \textbf{2} des] das n \textbf{3} ir] ie o \textbf{6} ir râtes] ir mir rotes m \textbf{7} mære] here n \textbf{11} dô] da m \textit{om.} n \textbf{13} bat] gap m  $\cdot$ dâ] do m n o \textbf{15} alwâr] alle war o \textbf{16} vâr] nar n o \textbf{17} sin vuorten] sú fúrtent in n sie [furs]: furten o \textbf{18} dâ] Do n o \textbf{20} erbeizen] siner basen o \textbf{22} hiez] \textit{om.} m \textbf{24} ine] Nie m Jch n o \textbf{25} riet] reit m \textbf{26} beidiu ime] ẏme beide o \textbf{27} daz] was m \textbf{28} ouch] \textit{om.} n o \newline
\end{minipage}
\end{table}
\newpage
\begin{table}[ht]
\begin{minipage}[t]{0.5\linewidth}
\small
\begin{center}*G
\end{center}
\begin{tabular}{rl}
 & dâ wil ich iu dienen nâch,\\ 
 & sît mir mîn muoter des verjach."\\ 
 & "sît ir durch râtes schulde\\ 
 & her komen, iwer hulde\\ 
5 & müezt ir mir durch râten lân,\\ 
 & \textit{\textbf{unde}} welt ir râtes volge hân."\\ 
 & dô warf der vürste mære\\ 
 & einen mûzsparwære\\ 
 & von der hende. in die burc er swanc.\\ 
10 & ein guldîn schelle dran erklanc.\\ 
 & daz was ein bote. dô \textbf{kom im} sân\\ 
 & \textbf{vil} junchêrren wolgetân.\\ 
 & er \textbf{hiez} den gast, den er dâ sach,\\ 
 & în vüeren unde schaffen sîn gemach.\\ 
15 & \textbf{der} sprach: "mîn muoter saget alwâr:\\ 
 & \textbf{altes mannes} rede stêt niht ze vâr."\\ 
 & hin în \textbf{sin vuorten} alzehant,\\ 
 & dâ er manigen \textbf{werden} rîter vant,\\ 
 & ûf \textbf{den hof}. an einer stat\\ 
20 & ieslîcher in erbeizen bat.\\ 
 & \textbf{dô sprach, an dem} was tumpheit schîn:\\ 
 & "mich hiez ein künic rîter sîn.\\ 
 & swaz halt \textbf{drûffe mir} geschiht,\\ 
 & ich enkum von disem orse niht."\\ 
25 & \multicolumn{1}{l}{ - - - }\\ 
 & \multicolumn{1}{l}{ - - - }\\ 
 & \multicolumn{1}{l}{ - - - }\\ 
 & \multicolumn{1}{l}{ - - - }\\ 
 & maniger bete si \textit{\textbf{in}} gedâhten,\\ 
30 & \textbf{unze} sin von dem orse brâhten\\ 
\end{tabular}
\scriptsize
\line(1,0){75} \newline
G I O L M Q R Z Fr17 Fr47 \newline
\line(1,0){75} \newline
\textbf{1} \textit{Initiale} Q  \textbf{3} \textit{Initiale} O L R Z Fr17  \textbf{13} \textit{Initiale} I  \newline
\line(1,0){75} \newline
\textbf{2} mir] \textit{om.} L  $\cdot$ des] das R \textbf{3} sît] ÷it O ÷it \textit{nachträglich korrigiert zu:} Sit Fr17 \textbf{4} Her kommen M \textbf{5} râten] rate O (R) rat Fr47 \textbf{6} unde] \textit{om.} G  $\cdot$ volge] volgen Q \textbf{7} dô] Da M Z \textbf{8} mûzsparwære] gemvͦzten sparwere O mvsser sperware L (M) (R) (Z) (Fr17) (Fr47) műzen sperwere Q \textbf{9} in die burc er] er in die buͯrg L (Q) (Z) indie burc M \textbf{10} erklanc] erkank R \textbf{11} daz] Dar M  $\cdot$ dô] da M Z  $\cdot$ kom im] chomen in I \textbf{13} dâ] do Q Fr47 \textbf{14} unde] vnd im R  $\cdot$ schaffen] schlaffen Q im schaffen R \textbf{15} der] er I (O) (L) Do R  $\cdot$ sprach] sagt Z  $\cdot$ saget] sagte L \textbf{17} sin] [si*]: sim Fr17 \textbf{18} dâ] Do Q  $\cdot$ werden rîter] riter werden O Ritter Fr47  $\cdot$ vant] inne vant I \textbf{19} den] dem I (O) (L) (M) (Q) (R) (Z) dē Fr47  $\cdot$ hof] hoe Q  $\cdot$ einer] der R \textbf{20} ieslîcher] Jr yettlicher R  $\cdot$ erbeizen] wilkomen R \textbf{21} dô] Da M Z  $\cdot$ was tumpheit] tvmpheit waz L \textbf{22} hiez ein künic] ein kvnic hiez Z \textbf{23} swaz] Waz L M (Q) (R)  $\cdot$ drûffe mir] mir druf I mir darum R  $\cdot$ geschiht] beschicht R \textbf{24} enkum] chum I (O) (Q) (R) (Fr47)  $\cdot$ von] abe Fr17  $\cdot$ disem] dem I (L) (Fr47) \textbf{25} \textit{Die Verse 163.25-28 fehlen} G I O L M Fr17 Fr47   $\cdot$ Grusz gen euch [reit]: riet meyn muter mir Q (Z)  $\cdot$ Gruͦs gen úch riet mir die muͦtter min R \textbf{26} Sie danckten beyde im vnd ir Q (Z)  $\cdot$ Sy tattent alle im erre schin R \textbf{27} Do dasz grussen wart (was Z ) gethon Q (R) (Z) \textbf{28} Das rosz was múde vnd (\textit{om.} R ) auch der man Q (R) (Z) \textbf{29} bete] bat M  $\cdot$ in] doch G \textit{om.} L M Q R Z Fr17 Fr47 \textbf{30} unze] Er M Wisz Q E Z \newline
\end{minipage}
\hspace{0.5cm}
\begin{minipage}[t]{0.5\linewidth}
\small
\begin{center}*T
\end{center}
\begin{tabular}{rl}
 & dâ wil ich iu dienen nâch,\\ 
 & sît mir mîn muoter des verjach."\\ 
 & "Sît ir durch râtes schulde\\ 
 & her komen, iuwer hulde\\ 
5 & müezet ir mir durch râten lân,\\ 
 & welt ir râtes volge hân."\\ 
 & Dô warf der vürste mære\\ 
 & einen mûzersperwære\\ 
 & von der hende. in die burc er swanc.\\ 
10 & ein guldîn schelle dran erklanc.\\ 
 & daz was ein bote. dô \textbf{kom im} sân\\ 
 & \textbf{vil} junchêrren wolgetân.\\ 
 & er \textbf{hiez} den gast, den er dâ sach,\\ 
 & \textbf{hin} în vüeren und\textit{e s}chaffen sîn gemach.\\ 
15 & \textbf{Er} sprach: "mîn muoter sagt alwâr:\\ 
 & \textbf{altes mannes} rede stêt niht ze vâr."\\ 
 & Hin în \textbf{sin vuorten} alzehant,\\ 
 & dâ er manegen rîter vant.\\ 
 & ûf \textbf{dem hove} an einer stat\\ 
20 & ieslîcher in erbeizen bat.\\ 
 & \textbf{An sîner sprâche} was tumpheit schîn:\\ 
 & "mich hiez ein künec rîter sîn.\\ 
 & swaz halt \textbf{drûffe mir} geschiht,\\ 
 & ine kume von disem orse niht.\\ 
25 & gruoz gegen iu riet \textbf{diu} muoter mir."\\ 
 & Si dankten beidiu im unde ir.\\ 
 & \begin{large}D\end{large}ô \textbf{diz} grüezen \textbf{was} getân\\ 
 & - daz ors was müede unde ouc\textit{h} der man -,\\ 
 & maneger bete si \textbf{dô} gedâhten,\\ 
30 & \textbf{biz} sin vom orse brâhten\\ 
\end{tabular}
\scriptsize
\line(1,0){75} \newline
T U V W \newline
\line(1,0){75} \newline
\textbf{3} \textit{Majuskel} T  \textbf{7} \textit{Initiale} W   $\cdot$ \textit{Majuskel} T  \textbf{15} \textit{Majuskel} T  \textbf{17} \textit{Majuskel} T  \textbf{21} \textit{Majuskel} T  \textbf{26} \textit{Majuskel} T  \textbf{27} \textit{Initiale} T U V  \newline
\line(1,0){75} \newline
\textbf{1} dâ] Do U (W) \textbf{2} des] das W \textbf{3} râtes] riters U \textbf{5} râten] rate W \textbf{8} mûzersperwære] [gemvzet*]: gemvzeten sperwere V \textbf{9} er] \textit{om.} U \textbf{13} dâ] do U V W \textbf{14} hin în vüeren] Hin of vuͦren U Hinfuͤren W  $\cdot$ unde schaffen] vnde vnde schaffen T \textbf{15} sagt] sagete V \textbf{16} niht ze vâr] zuͦuor W \textbf{17} în] of U  $\cdot$ sin] sy W \textbf{18} dâ] Do U (V) W  $\cdot$ manegen] vil U V mangen werden W \textbf{21} An sîner sprâche] [*]: Do sprach an dem V Sein sprach an der W \textbf{23} swaz] Waz U (W)  $\cdot$ drûffe mir] mir dar vf U (V) mir darumb W \textbf{26} Si] [si]: Ssi T  $\cdot$ dankten] dancket W \textbf{27} was] wart V (W) \textbf{28} ouch] ovc T \textit{om.} W \textbf{29} dô] da V \textbf{30} biz] Vnze V Wie W \newline
\end{minipage}
\end{table}
\end{document}
