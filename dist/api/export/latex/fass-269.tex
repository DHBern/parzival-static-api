\documentclass[8pt,a4paper,notitlepage]{article}
\usepackage{fullpage}
\usepackage{ulem}
\usepackage{xltxtra}
\usepackage{datetime}
\renewcommand{\dateseparator}{.}
\dmyyyydate
\usepackage{fancyhdr}
\usepackage{ifthen}
\pagestyle{fancy}
\fancyhf{}
\renewcommand{\headrulewidth}{0pt}
\fancyfoot[L]{\ifthenelse{\value{page}=1}{\today, \currenttime{} Uhr}{}}
\begin{document}
\begin{table}[ht]
\begin{minipage}[t]{0.5\linewidth}
\small
\begin{center}*D
\end{center}
\begin{tabular}{rl}
\textbf{269} & \begin{large}P\end{large}arzival \textbf{dô} mit triwen \textbf{vuor}:\\ 
 & er nam daz \textbf{heilictuom}, \textbf{drûf er} swuor.\\ 
 & sus stabt er selbe sînen eit.\\ 
 & er sprach: "hân ich werdecheit\\ 
5 & - ich hab si oder \textbf{en}hab \textbf{ir} niht -,\\ 
 & swer mich bîme schilde siht,\\ 
 & der prüevet \textbf{mich gein} ritterschaft.\\ 
 & des nam\textit{en} ordenlîchiu kraft,\\ 
 & als \textbf{uns} des schildes ambet sagt,\\ 
10 & hât dicke hôhen prîs bejagt.\\ 
 & \textbf{Ez} ist ouch noch ein hôher name.\\ 
 & mîn lîp gein werltlîcher schame\\ 
 & immer sî gewenket\\ 
 & unt \textbf{al} mîn prîs verkrenket.\\ 
15 & \textbf{dirre} wort sî mit werken pfant\\ 
 & mîn gelücke vor der hœhsten hant.\\ 
 & ich hânz dâ vür, die treit got.\\ 
 & nû müeze ich \textbf{vlusteclîchen} spot\\ 
 & ze bêden \textbf{lîben} immer hân\\ 
20 & \textbf{von} sîner kraft, ob missetân\\ 
 & \textbf{disiu} vrouwe habe, dô \textbf{diz} geschach,\\ 
 & daz ich ir vürspan von ir brach.\\ 
 & ouch vuort ich mêr goldes dan.\\ 
 & ich was ein tôre \textbf{unt} niht ein man,\\ 
25 & \textbf{gewahsen niht bî} witzen.\\ 
 & vil weinens, dâ bî switzen\\ 
 & mit jâmer \textbf{dolte vil} ir lîp.\\ 
 & si ist \textbf{benamen} ein unschuldic wîp.\\ 
 & dâne scheide ich ûz niht mêre;\\ 
30 & \textbf{des} sî pfant mîn sælde unt êre.\\ 
\end{tabular}
\scriptsize
\line(1,0){75} \newline
D \newline
\line(1,0){75} \newline
\textbf{1} \textit{Initiale} D  \textbf{11} \textit{Majuskel} D  \newline
\line(1,0){75} \newline
\textbf{8} namen] nam D \newline
\end{minipage}
\hspace{0.5cm}
\begin{minipage}[t]{0.5\linewidth}
\small
\begin{center}*m
\end{center}
\begin{tabular}{rl}
 & Parcifal \textbf{dô} mit triuwen \textbf{vuor}:\\ 
 & er nam daz \textbf{h\textit{ei}ltuo\textit{m}}, \textbf{dâr ûf er} swuor.\\ 
 & sus stabete er selbe sînen eit.\\ 
 & er sprach: "hân ich werdecheit\\ 
5 & - ich habe si oder habe \textbf{si} niht -,\\ 
 & wer mich bî dem schilte siht,\\ 
 & der brüefet \textbf{gegen mir} rîterschaft.\\ 
 & des namen ordenlîchiu kraft,\\ 
 & als \textbf{un\textit{s}} des schiltes ambet saget,\\ 
10 & hât dicke hôhen prîs bejaget.\\ 
 & \textbf{ez} ist ouch noch ein hôher name.\\ 
 & mîn lîp gegen werltlîcher schame\\ 
 & iemer sî gewe\textit{n}ket\\ 
 & und \textbf{alsô} mîn prîs verkrenket.\\ 
15 & \textbf{dirre} wort sî mit werken pfant\\ 
 & mîn glücke vor der hœhesten hant.\\ 
 & ich hânz dâ vür, die treit g\textit{o}t.\\ 
 & nû müeze \textit{ich} \textbf{vlusteclîchen} spot\\ 
 & ze beiden \textbf{lîben} iemer hân\\ 
20 & \textbf{von} sîner kraft, ob missetân\\ 
 & \textbf{disiu} vrouwe habe, dô \textbf{diz} geschach,\\ 
 & daz ich ir vürspan von ir brach.\\ 
 & ou\textit{ch} vuorte ich mêre goldes dan.\\ 
 & ich was ein tôre \textbf{und} niht ein man,\\ 
25 & \textbf{gewahsen niht bî} witzen.\\ 
 & vil weine\textit{n}s, dâ bî switz\textit{en}\\ 
 & mit jâmere \textbf{dolte vil} ir lîp.\\ 
 & sist ein unschuldic wîp.\\ 
 & dâne scheide ich ûz niht mêre;\\ 
30 & \textbf{des} sî pfant mîn sælde und êre.\\ 
\end{tabular}
\scriptsize
\line(1,0){75} \newline
m n o Fr69 \newline
\line(1,0){75} \newline
\newline
\line(1,0){75} \newline
\textbf{1} Parcifal dô] Do parcifal n \textbf{2} heiltuom] haltumb m \textbf{3} selbe] selbes n \textbf{5} Jch habe si [niht]: oder habe si niht m  $\cdot$ si niht] ir nicht n \textbf{8} des] Das o \textbf{9} uns] vnd m  $\cdot$ des] das o \textbf{11} noch] \textit{om.} n \textbf{12} werltlîcher] werdeklicher Fr69 \textbf{13} gewenket] gewecket m \textbf{14} alsô] aller n alle o \textbf{15} wort] worten Fr69 \textbf{17} hânz] han n o  $\cdot$ got] get m gat o \textbf{18} ich] \textit{om.} m  $\cdot$ müeze] muͯsz n (o) \textbf{19} beiden lîben] beiden lieben n beide liebe o \textbf{22} ich ir vürspan] ir fúrspan ich n \textbf{23} ouch] Ous m  $\cdot$ mêre] nie o \textbf{26} weinens] weines m  $\cdot$ switzen] swicz m \textbf{28} sist] Sú ist bynammen n (o) \textbf{29} dâne scheide] Danne scheide m Das entscheide n Das enscheide o  $\cdot$ ûz] uch o \textbf{30} des] Das n o  $\cdot$ sî] sú n myn o  $\cdot$ mîn] nym o \newline
\end{minipage}
\end{table}
\newpage
\begin{table}[ht]
\begin{minipage}[t]{0.5\linewidth}
\small
\begin{center}*G
\end{center}
\begin{tabular}{rl}
 & Parzival mit triuwen \textbf{vuor}:\\ 
 & er nam d\textit{az} \textbf{\textit{h}e\textit{ilictuom}}, \textbf{dâr ûffer} swuor.\\ 
 & sus stabter selbe sînen eit.\\ 
 & er sprach: "\textbf{hêrre}, hân ich werdicheit\\ 
5 & - ich habe si oder \textbf{ich}\textbf{ne} habe\textbf{r} niht -,\\ 
 & swer mich bî dem schilte siht,\\ 
 & der prüevet \textbf{mich gein} rîterschaft.\\ 
 & des namen ordenlîchiu kraft,\\ 
 & als \textbf{uns} des schiltes ambet saget,\\ 
10 & hât dicke hôhen brîs bejaget.\\ 
 & \textbf{ez} ist ouch noch ein hôher nam.\\ 
 & mîn lîp gein werltlîcher scham\\ 
 & immer sî gewenket\\ 
 & unde \textbf{al} mîn brîs verkrenket.\\ 
15 & \textbf{dirre} worte sî mit werken pfant\\ 
 & mîn gelücke vor der hœhesten hant.\\ 
 & ich hânz dâ vür, die treit got.\\ 
 & \textit{nû} müeze ich \textbf{vlusticlîchen} spot\\ 
 & ze beiden \textbf{leben} immer hân\\ 
20 & \textbf{von} sîner kraft, obe missetân\\ 
 & \textbf{disiu} vrouwe habe, dô \textbf{daz} geschach,\\ 
 & daz \textit{ich} ir vürspan von ir brach.\\ 
 & ouch vuorte ich \textbf{ir} mê goldes dan.\\ 
 & ich was ein tôre \textbf{unde} niht ein man,\\ 
25 & \textbf{gescheiden von den} witzen.\\ 
 & vil weinens, dâ bî switzen\\ 
 & mit jâmer \textbf{vil gedolt} ir lîp.\\ 
 & si ist \textit{\textbf{benamen}} ei\textit{n} \textit{u}nschuldic wîp.\\ 
 & dâne scheide ich ûz niht mêre;\\ 
30 & \textbf{es} sî pfant mîn sælde unde êre.\\ 
\end{tabular}
\scriptsize
\line(1,0){75} \newline
G I O L M Q R Z \newline
\line(1,0){75} \newline
\textbf{1} \textit{Initiale} I  \textbf{27} \textit{Initiale} O  \textbf{29} \textit{Initiale} Z  \newline
\line(1,0){75} \newline
\textbf{1} Parzival] Parzifal I M Parcifal O L Q Z Parczifal R  $\cdot$ mit] alda mit I do mit R da mit Z  $\cdot$ triuwen] trúrren R  $\cdot$ vuor] vor M \textbf{2} daz heilictuom] die chesse G daz heiltvͦm O (L) (R) (Z)  $\cdot$ dâr ûffer] vnd L  $\cdot$ swuor] swar M \textbf{3} stabter] stabt er I Q Z stalt er O stakt er R  $\cdot$ selbe] selbir M (R)  $\cdot$ sînen] den I \textbf{4} hêrre] \textit{om.} O L M Q R Z  $\cdot$ ich] ich selbern M \textbf{5} habe si] habs O Q hab es R (Z)  $\cdot$ ichne] ich I O Q Z ne M \textit{om.} R \textbf{6} swer] Wer L Q R  $\cdot$ dem] disem Z \textbf{7} gein] Gein der I \textit{om.} Q \textbf{8} ordenlîchiu] ordenliche I \textbf{11} ist] \textit{om.} L  $\cdot$ ouch] dach M \textbf{12} gein] in Z  $\cdot$ werltlîcher] werdechlicher I (O) (M) (Q) wertlicher L (R) \textbf{13} gewenket] gewenkent Q \textbf{15} dirre] Dise Q  $\cdot$ werken pfant] werche phant I worten pfagk Q \textbf{16} der] den Q \textbf{17} die] diu I \textbf{18} nû] so G  $\cdot$ müeze] mvͦz O (M) (Q) (Z)  $\cdot$ vlusticlîchen] fluhteclichen I fvrstlichen O verluͯstebaren L \textbf{19} leben] liben I (M) (Q) R Z liden L \textbf{20} kraft] hant O Z  $\cdot$ obe] uff M  $\cdot$ missetân] [missetat]: missetan L \textbf{21} disiu] Die L  $\cdot$ habe] \textit{om.} M  $\cdot$ dô] da M Z  $\cdot$ daz] dis L \textbf{22} ich ir] ir G ich O  $\cdot$ vürspan] vorspang L (R)  $\cdot$ von ir] abe I \textbf{23} ouch] Doch M  $\cdot$ ir mê goldes] ir goldes mer I mere ir goldes L meres goldes R mer goldes Z \textbf{24} ein tôre] ein ein tore O  $\cdot$ unde] \textit{om.} Q R \textbf{25} \textit{Vers 269.25 fehlt} I   $\cdot$ Gewahsen bi witzen O (L) gewaschin by wiczen M Gewassen bey vnwitzen Q Gewachsen mit wiczen R Gewahsen niht bi witzen Z \textbf{26} switzen] switezen G wiczen M (Q) \textbf{27} dise not an shult hat gedolt ir lip I  $\cdot$ Mit iamer dolt vil ir lip Z  $\cdot$ mit] ÷it O \textbf{28} benamen] \textit{om.} G  $\cdot$ ein unschuldic] ein hart vnschuldch G eyn vnschulde M \textbf{29} dâne scheide] Da scheide L R Do enscheide Q  $\cdot$ ûz] \textit{om.} R \textbf{30} es] des I (Z)  $\cdot$ sî] ist R  $\cdot$ pfant mîn] min pfant O  $\cdot$ unde] vnd min I (Q) (Z) min R \newline
\end{minipage}
\hspace{0.5cm}
\begin{minipage}[t]{0.5\linewidth}
\small
\begin{center}*T
\end{center}
\begin{tabular}{rl}
 & Parcifal mit triuwen \textbf{dô} \textbf{\textit{gev}uor}:\\ 
 & er nam daz \textbf{heiltuom} \textbf{unde} swuor.\\ 
 & sus stabeter selbe sînen eit.\\ 
 & er sprach: "hân ich werdecheit\\ 
5 & - ich habe si oder \textbf{ich}\textbf{n} hab\textbf{ir} niht -,\\ 
 & swer mich bî dem schilte siht,\\ 
 & der prüevet \textbf{mich gegen} rîterschaft.\\ 
 & des name\textit{n} ordenlîchiu kraft,\\ 
 & als \textbf{mir} des schiltes ambet saget,\\ 
10 & hât dicke hôhen prîs bejaget.\\ 
 & \textbf{daz} ist ouch noch ein hôher name.\\ 
 & mîn lîp gegen werltlîcher schame\\ 
 & iemer sî gewenket\\ 
 & unde \textbf{al} mîn prîs verkrenket.\\ 
15 & \textbf{der} worte sî mit werken pfant\\ 
 & mîn glücke vor der hœhesten hant.\\ 
 & ich hân\textit{z} dâ vür, die treit got.\\ 
 & nû müezich \textbf{vlühteclîchen} spot\\ 
 & ze beiden \textbf{lîben} iemer hân\\ 
20 & \textbf{vor} sîner kraft, ob missetân\\ 
 & \textbf{diu} vrouwe habe, dô \textbf{daz} geschach,\\ 
 & daz ich ir vürspan von ir brach.\\ 
 & ouch vuortich mêre goldes dan.\\ 
 & ich was ein tôre, niht ein man,\\ 
25 & \textbf{gewahsen niht bî} witzen.\\ 
 & vil weine\textit{n}s, dâ bî switzen\\ 
 & mit jâmer \textbf{dolte vil} ir lîp.\\ 
 & sist \textbf{benamen} ein unschuldic wîp.\\ 
 & dâne scheidich ûz niht mêre,\\ 
30 & \textbf{wan} \textbf{des} sî pfant mîn sælde unde êre.\\ 
\end{tabular}
\scriptsize
\line(1,0){75} \newline
T U V W \newline
\line(1,0){75} \newline
\textbf{3} \textit{Initiale} W  \newline
\line(1,0){75} \newline
\textbf{1} Parcifal] Parzifal T V Partzifal W  $\cdot$ gevuor] swuͦr T \textbf{2} unde] druf er V \textbf{3} stabeter] stabet er V \textbf{5} ichn] \textit{om.} U W ich V \textbf{6} swer] Wer U W \textbf{7} der] Daz U  $\cdot$ mich gegen] [*]: gegen mir V gen der W \textbf{8} des] Der W  $\cdot$ namen] name T U W [*]: nammen V \textbf{9} mir des] [*]: vnz dez V im das W \textbf{11} daz] Ez V  $\cdot$ ouch] \textit{om.} W \textbf{12} lîp] \textit{om.} W  $\cdot$ werltlîcher] werlicher W \textbf{14} al mîn] aller [*]: min V allen inein W  $\cdot$ verkrenket] gekrenket U \textbf{15} der] [D*]: Dirre V  $\cdot$ pfant] entpfant W \textbf{16} glücke] gelúbd W \textbf{17} hânz] hans T  $\cdot$ dâ] do U W \textbf{18} vlühteclîchen spot] [*]: verlústeclichen spot V verlustlichen spot W \textbf{20} vor] Von U V W \textbf{21} diu] Dise U V W  $\cdot$ daz] [d*]: daz V dis W \textbf{24} niht] [*h*]: vnde niht V vnd nit W \textbf{26} Ich lies sy traurig sitzen W  $\cdot$ weinens] weines T (U) \textbf{28} unschuldic] reines W \textbf{29} dâne] Do in U Danne W \textbf{30} mîn] all mein W  $\cdot$ sælde unde] \textit{om.} W  $\cdot$ êre] min ere V \newline
\end{minipage}
\end{table}
\end{document}
