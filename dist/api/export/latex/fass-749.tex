\documentclass[8pt,a4paper,notitlepage]{article}
\usepackage{fullpage}
\usepackage{ulem}
\usepackage{xltxtra}
\usepackage{datetime}
\renewcommand{\dateseparator}{.}
\dmyyyydate
\usepackage{fancyhdr}
\usepackage{ifthen}
\pagestyle{fancy}
\fancyhf{}
\renewcommand{\headrulewidth}{0pt}
\fancyfoot[L]{\ifthenelse{\value{page}=1}{\today, \currenttime{} Uhr}{}}
\begin{document}
\begin{table}[ht]
\begin{minipage}[t]{0.5\linewidth}
\small
\begin{center}*D
\end{center}
\begin{tabular}{rl}
\textbf{749} & \begin{large}\textbf{Ô}\end{large} wol diu wîp, di\textit{u} \textbf{dich} sulen sehen!\\ 
 & waz den \textbf{doch} sælden ist \textbf{geschehen}!"\\ 
 & "Ir \textbf{sprechet} wol, ich spræche baz,\\ 
 & ob ich daz kunde, ân allen haz.\\ 
5 & nû bin ich leider niht sô wîs,\\ 
 & \textbf{des} iwer werdeclîcher prîs\\ 
 & mit worten mege \textbf{gehôhet} sîn.\\ 
 & got weiz aber wol den willen mîn.\\ 
 & swaz herze unt \textbf{ougen} künste hânt\\ 
10 & an mir, diu beidiu niht \textbf{erlânt}.\\ 
 & \textbf{iwer} prîs \textbf{saget} vor, si volgent nâch.\\ 
 & daz nie von rîters hant geschach\\ 
 & mir grœzer nôt, vür wâr ich\textbf{z} weiz,\\ 
 & denne von iu", sprach der von Kanvoleiz.\\ 
15 & Dô sprach der rîche Feirefiz:\\ 
 & "\textbf{Jupiter} hât sînen vlîz,\\ 
 & werder helt, geleit an dich.\\ 
 & dû solt niht mêre irzen mich.\\ 
 & wir \textbf{heten} \textbf{bêde doch} einen vater."\\ 
20 & mit brüederlîchen triwen bat er,\\ 
 & daz er \textbf{irzens in} erlieze\\ 
 & unt in dutzenlîche hieze.\\ 
 & Diu rede was Parzivale leit.\\ 
 & \textbf{der} sprach: "bruoder, iwer rîcheit\\ 
25 & glîchet wol dem bârucke sich;\\ 
 & sô sît ir elter ouch denn ich.\\ 
 & mîn jugent unt mîn armuot\\ 
 & sol solher lôsheit sîn behuot,\\ 
 & daz ich iu duzen biete,\\ 
30 & swenn ich mich zühte niete."\\ 
\end{tabular}
\scriptsize
\line(1,0){75} \newline
D \newline
\line(1,0){75} \newline
\textbf{1} \textit{Initiale} D  \textbf{3} \textit{Majuskel} D  \textbf{15} \textit{Majuskel} D  \textbf{23} \textit{Majuskel} D  \newline
\line(1,0){75} \newline
\textbf{1} diu dich] di dich D \textbf{23} Parzivale] Parcifale D \newline
\end{minipage}
\hspace{0.5cm}
\begin{minipage}[t]{0.5\linewidth}
\small
\begin{center}*m
\end{center}
\begin{tabular}{rl}
 & \textbf{ô} wol diu wîp, diu sullen sehen,\\ 
 & waz den \textbf{doch} sælden ist \textbf{besche\textit{h}en}!"\\ 
 & "ir \textbf{sprecht} wol, ich spræche baz,\\ 
 & ob ich daz kunde, âne allen haz.\\ 
5 & nû bin ich leider niht sô wîs,\\ 
 & \textbf{des} iuwer wirdeclîcher prîs\\ 
 & mit worten muge \textbf{gehôhet} sîn.\\ 
 & got weiz aber wol den willen mîn.\\ 
 & waz herz und \textbf{ougen} kunst hânt\\ 
10 & an mir, diu beidiu ni\textit{ht} \textbf{erlânt}.\\ 
 & \textbf{iuwer} prîs \textbf{saget} vor, si volgen\textit{t} nâch.\\ 
 & daz nie von ritters hant geschach\\ 
 & mir grœzer nôt, vür wâr ich \textbf{daz} weiz,\\ 
 & dan von iu", sprach der von Kanvoleiz.\\ 
15 & dô sprach der rîch Ferefiz:\\ 
 & "\textbf{Jupiter} het sînen vlî\textit{z},\\ 
 & werder helt, geleit an dich.\\ 
 & dû solt niht mê irzen mich.\\ 
 & wir \textbf{heten} \textbf{doch beide} einen vater."\\ 
20 & mit bruoderlîchen triuwen \textit{b}at er,\\ 
 & daz er \textbf{in irzens} erl\textit{ie}ze\\ 
 & und in dutzelîch \textbf{hiute} hieze.\\ 
 & diu rede was Parcifal leit.\\ 
 & \textbf{er} sprach: "bruoder, iuwer rîcheit\\ 
25 & gelîchet wol dem bâruc sich;\\ 
 & sô sît ir elter ouch dan ich.\\ 
 & mîn jugent und mîn a\textit{r}muot\\ 
 & sol solicher lôsheit sîn behuot,\\ 
 & daz ich iu dutzen biete,\\ 
30 & wan ich mich zühte niete."\\ 
\end{tabular}
\scriptsize
\line(1,0){75} \newline
m n o V V' \newline
\line(1,0){75} \newline
\textbf{15} \textit{Initiale} V  \newline
\line(1,0){75} \newline
\textbf{1} \textit{Die Verse 748.25-750.10 fehlen} V'   $\cdot$ diu] die dich n o V \textbf{2} beschehen] [bescheht]: beschehten m geschehen V \textbf{6} des] [Der]: Des m Der o Daz V \textbf{7} gehôhet] gehoubet n \textbf{8} weiz] weil n \textbf{9} waz] Swaz V \textbf{10} niht erlânt] nider lant m \textbf{11} volgent] volgen m n o  $\cdot$ nâch] [sich]: noch o \textbf{13} daz] es V \textbf{14} der] der do n  $\cdot$ Kanvoleiz] Kanfoleis m o kanfleisz n kanvoleis V \textbf{15} Ferefiz] Ferefis m ferrefis n ferre fis o ferevis V \textbf{16} Jupiter] Juppiter n (V)  $\cdot$ het] hette n  $\cdot$ vlîz] flichs m \textbf{18} irzen] retzen n \textbf{20} bruoderlîchen] bruͦderlich o  $\cdot$ bat] hat m \textbf{21} erlieze] erleise m liessen o \textbf{22} \textit{Die Verse 749.22-28 fehlen (wohl wegen Augenabirrung von 749.22 zu 749.30)} o   $\cdot$ hiute] \textit{om.} n V \textbf{23} Parcifal] parzefale V \textbf{27} armuot] aramuͯt m \textbf{28} sol] Soͤllent V \textbf{29} biete] beiten o \textbf{30} \textit{Versdoppelung 749.30 vor 749.29 (mit Anteil aus dem in anderen Hss. vorhandenen Vers 749.22: Und; Augenabirrung?):} Vnd ich mich zuͯchte niete o   $\cdot$ wan] \sout{V́ch} Swenne V  $\cdot$ zühte] ziete o \newline
\end{minipage}
\end{table}
\newpage
\begin{table}[ht]
\begin{minipage}[t]{0.5\linewidth}
\small
\begin{center}*G
\end{center}
\begin{tabular}{rl}
 & \begin{large}W\end{large}ol diu wîp, diu \textbf{dich} sülen sehen!\\ 
 & waz den \textbf{doch} sælden ist \textbf{geschehen}!"\\ 
 & "ir \textbf{seht} wol, ich spræche baz,\\ 
 & obe ich daz kunde, ân allen haz.\\ 
5 & nû bin ich leider niht sô wîs,\\ 
 & \textbf{daz} iwer werdeclîcher brîs\\ 
 & mit worten müge \textbf{geholen} sîn.\\ 
 & got weiz aber wol den willen mîn.\\ 
 & swaz herze unde \textbf{ouge} künste hânt\\ 
10 & an mir, diu beidiu niht \textbf{enlânt}.\\ 
 & \textbf{iwern} brîs \textbf{s\textit{age ich}} vor, si volgent nâch.\\ 
 & daz nie von rîters hant geschach\\ 
 & mir grœzer nôt, vür wâr ich\textbf{z} weiz,\\ 
 & danne von iu", sprach der \textit{von} Kanvoleiz.\\ 
15 & dô sprach der rîche Feirafiz:\\ 
 & "\textbf{got} hât \textbf{rehte} sînen vlîz,\\ 
 & werder helt, geleit an dich.\\ 
 & dû\textbf{ne} solt niht mêre irzen mich.\\ 
 & wir \textbf{haben} \textbf{bêde doch} einen vater."\\ 
20 & mit bruoderlîchen triwe\textit{n} \textit{b}at er,\\ 
 & daz er \textbf{irzens in} erlieze\\ 
 & unde in duzeclîchen hieze.\\ 
 & diu rede was Parzival leit.\\ 
 & \textbf{er} sprach: "bruoder, iwer rîcheit\\ 
25 & gelîchet wol dem bâruc sich;\\ 
 & sô sît ir elter \textit{ouch} danne ich.\\ 
 & mîn jugent unde mîn armuot\\ 
 & sol solher lôsheit sîn behuot,\\ 
 & daz ich iu duzen biete,\\ 
30 & swenne ich mich zühte niete."\\ 
\end{tabular}
\scriptsize
\line(1,0){75} \newline
G I L M Z \newline
\line(1,0){75} \newline
\textbf{1} \textit{Initiale} G L Z  \textbf{9} \textit{Initiale} I  \textbf{27} \textit{Initiale} I  \newline
\line(1,0){75} \newline
\textbf{1} Wol] Owol L D wol Z  $\cdot$ diu wîp] den wiben I  $\cdot$ diu dich] dy s dich M \textbf{2} den doch] den I dach den M \textbf{3} seht] sprachit M sprechet Z  $\cdot$ spræche] spreche I M spriche Z \textbf{4} daz] ez L  $\cdot$ allen] \textit{om.} I \textbf{5} nû] nun I  $\cdot$ niht] >nih< G \textbf{7} geholen] Geuellet I gehohet L Z gehohir M \textbf{9} swaz] Waz L (M)  $\cdot$ ouge] augen I (L) (M) (Z) \textbf{10} Da mit sẏ uwer lop niht lant L  $\cdot$ niht] uch nicht M (Z)  $\cdot$ enlânt] irbant M [verlant]: erlant Z \textbf{11} sage ich vor] si vor G ich sage L  $\cdot$ si volgent] so volget L \textbf{13} mir] Mit L  $\cdot$ grœzer] nie so groze I  $\cdot$ vür wâr ichz weiz] [verwaz]: verwaiz I \textbf{14} Danne sprach von vns von kanvoleisz M  $\cdot$ von] \textit{om.} G  $\cdot$ Kanvoleiz] kanuoleuz I Kamvoleiz L kamfoleiz Z \textbf{15} dô] Da M  $\cdot$ Feirafiz] firefiz G Ferefiz L Feirafeisz M feirefiz Z \textbf{16} got] Jupiter M Z  $\cdot$ rehte] \textit{om.} L M Z \textbf{18} niht mêre] nimmer Z \textbf{19} haben bêde doch] haben doch bede I hetten doch L \textbf{20} bruoderlîchen triwen] bruͯderlicher truͯwe L  $\cdot$ triwen bat] triwen doh bat G \textbf{21} irzens in] yn irzcens M \textbf{22} in] \textit{om.} L  $\cdot$ duzeclîchen] droszlichen M \textbf{23} Parzival] parcifal G Parzifale I (L) parzifal M parcifaln Z \textbf{24} er sprach] \textit{om.} I Der sprach L Z Da sprach ir M \textbf{25} dem baruc Gelichet sich I  $\cdot$ wol] \textit{om.} Z \textbf{26} elter] edeler I  $\cdot$ ouch] \textit{om.} G \textbf{28} solher] so vor L \textbf{29} duzen] ere L daruszin M \textbf{30} swenne] Wenne L (M) \newline
\end{minipage}
\hspace{0.5cm}
\begin{minipage}[t]{0.5\linewidth}
\small
\begin{center}*T
\end{center}
\begin{tabular}{rl}
 & wol diu wîp, diu \textbf{dich} soln sehen!\\ 
 & waz den sælden ist \textbf{geschehen}!"\\ 
 & "\begin{large}I\end{large}r \textbf{sprechet} wol, ich spr\textit{æ}ch\textit{e} baz,\\ 
 & ob ich daz kunde, âne allen haz.\\ 
5 & nû \textbf{en}bin ich leider niht sô wîs,\\ 
 & \textbf{daz} iuwer wirdeclîcher prîs\\ 
 & mit worten muge \textbf{gehôhet} sîn.\\ 
 & got weiz aber wol den willen mîn.\\ 
 & waz herze und \textbf{ougen} kunst hânt\\ 
10 & an mir, diu beidiu niht \textbf{erlânt},\\ 
 & \textbf{iuwer} prîs \textbf{saget} vor, si volgent nâch.\\ 
 & daz nie von rîters hant geschach\\ 
 & mir grœzer nôt, vür \textit{wâr} ich \textbf{ez} weiz,\\ 
 & dan von iu", sprach der von Kanvoleiz.\\ 
15 & dô sprach der rîche Ferefis:\\ 
 & "\textbf{Jupiter} hât sînen vlîz,\\ 
 & werder helt, geleget an dich.\\ 
 & dû \textbf{en}solt niht mê ir\textit{z}en mich.\\ 
 & wir \textbf{heten} \textbf{beide} einen vater."\\ 
20 & mit bruoderlîchen triuwen bat er,\\ 
 & daz er \textbf{irzens in} erlieze\\ 
 & und in duzlîchen hieze.\\ 
 & diu rede was Parcifale leit.\\ 
 & \textbf{dô} sprach \textbf{er}: "bruoder, iuwer rîcheit\\ 
25 & gelîchet wol dem bâruc sich;\\ 
 & sô sît ir alter ouch dan ich.\\ 
 & mîn jugent und mîn armuot\\ 
 & sol solicher lôsheit sîn behuot,\\ 
 & daz ich iu duzen biete,\\ 
30 & wan ich mich zühte niete."\\ 
\end{tabular}
\scriptsize
\line(1,0){75} \newline
U W Q R \newline
\line(1,0){75} \newline
\textbf{1} \textit{Initiale} R  \textbf{3} \textit{Initiale} W  \textbf{23} \textit{Initiale} W  \newline
\line(1,0){75} \newline
\textbf{1} wol] O wol Q R  $\cdot$ soln] \textit{om.} Q \textbf{2} den sælden] den doch selden Q doch den [se*]: selden R \textbf{3} spræche] sprechen U spriche W (R) \textbf{4} allen] \textit{om.} R \textbf{5} enbin] bin R \textbf{8} willen] wllen W \textbf{9} herze und ougen] hertze vnd auge W (Q) ogen vnd hercze R \textbf{10} diu beidiu] die baide úch W euch beyde die Q  $\cdot$ erlânt] erkant Q \textbf{11} iuwer] Eúwern W (Q)  $\cdot$ vor] \textit{om.} W \textbf{13} vür wâr] vur U do fúr W  $\cdot$ ez] \textit{om.} Q \textbf{14} Kanvoleiz] kanuoleiß W kanvoleisz Q Kanuoleis R \textbf{15} Ferefis] ferafis W feirefisz Q feirefiz R \textbf{16} Jupiter] Iupiter W Juppiter Q \textbf{18} ensolt] solt Q R  $\cdot$ irzen] irren U  $\cdot$ mich] dich R \textbf{19} beide] doch baide W (Q) (R) \textbf{20} bruoderlîchen triuwen] Brúdenlicher trúwe R \textbf{21} erlieze] erließ W (Q) \textbf{22} duzlîchen] druczlichen R  $\cdot$ hieze] hieß W (Q) \textbf{23} Parcifale] Parzifale U partzifali W partzifalen Q parczifaln R \textbf{28} sol] So R \textbf{30} niete] mite Q \newline
\end{minipage}
\end{table}
\end{document}
