\documentclass[8pt,a4paper,notitlepage]{article}
\usepackage{fullpage}
\usepackage{ulem}
\usepackage{xltxtra}
\usepackage{datetime}
\renewcommand{\dateseparator}{.}
\dmyyyydate
\usepackage{fancyhdr}
\usepackage{ifthen}
\pagestyle{fancy}
\fancyhf{}
\renewcommand{\headrulewidth}{0pt}
\fancyfoot[L]{\ifthenelse{\value{page}=1}{\today, \currenttime{} Uhr}{}}
\begin{document}
\begin{table}[ht]
\begin{minipage}[t]{0.5\linewidth}
\small
\begin{center}*D
\end{center}
\begin{tabular}{rl}
\textbf{466} & \begin{large}V\end{large}on dem wâren minnære\\ 
 & \textbf{sagent} \textbf{disiu} \textbf{süezen} mære.\\ 
 & der ist ein durchliuhtec lieht\\ 
 & \textbf{und} \textbf{wenket sîner minne} niht.\\ 
5 & swem er minne erzeigen sol,\\ 
 & \textbf{dem} wirt mit sîner minne wol.\\ 
 & die selben sint geteilet:\\ 
 & aller werlde ist geveilet\\ 
 & bêdiu sîn minne unt \textbf{ouch} \textit{s}în haz.\\ 
10 & \textbf{nû} prüevet, \textbf{wederz} helfe baz.\\ 
 & der schuldige âne riwe\\ 
 & vliuhet die götlîchen triwe.\\ 
 & swer aber wandelt \textbf{sünden} schulde,\\ 
 & der dient nâch werder hulde.\\ 
15 & die treit, der durch gedanke vert.\\ 
 & gedanc sich sunnen blickes wert.\\ 
 & gedanc ist âne slôz \textbf{gesp\textit{ar}t},\\ 
 & vor aller crêatûre bewart.\\ 
 & gedanc ist vinster âne schîn.\\ 
20 & diu gotheit kan lûter sîn.\\ 
 & si glestet durch \textbf{der} vinster want\\ 
 & und hât den helnden sprunc gerant,\\ 
 & der endiuzet \textbf{noch} \textbf{enklinget},\\ 
 & sô er vome herzen \textbf{springet}.\\ 
25 & ez ist \textbf{dehein} gedanc sô snel,\\ 
 & ê er vome herzen vürz vel\\ 
 & kum, er\textbf{n} sî versuochet.\\ 
 & des kiuschen got geruochet.\\ 
 & sît got gedanke \textbf{speht} sô wol,\\ 
30 & owê der \textbf{brœden} werke dol!\\ 
\end{tabular}
\scriptsize
\line(1,0){75} \newline
D \newline
\line(1,0){75} \newline
\textbf{1} \textit{Initiale} D  \newline
\line(1,0){75} \newline
\textbf{9} sîn haz] min haz D \textbf{17} gespart] gesprat D \newline
\end{minipage}
\hspace{0.5cm}
\begin{minipage}[t]{0.5\linewidth}
\small
\begin{center}*m
\end{center}
\begin{tabular}{rl}
 & von dem wâren minnære\\ 
 & \textbf{sagent} \textbf{disiu} \textbf{süezen} mære.\\ 
 & der ist ein durchliuhtec lieht\\ 
 & \textbf{und} \textbf{wenk\textit{e}t sîner minne} niht.\\ 
5 & wem er minne erzöugen sol,\\ 
 & \textbf{dem} wirt mit sîner minne wol.\\ 
 & die selben sint geteilet:\\ 
 & al\textbf{der} werlt ist geveilet\\ 
 & beidiu sîn minne und sîn haz.\\ 
10 & \textbf{nû} brüefet \textbf{werder} helfe baz.\\ 
 & der schuldige âne riuwe\\ 
 & vliuhet die götlîchen \dag minne\dag .\\ 
 & wer aber wandelt \textbf{sunder} schulde,\\ 
 & der dienet nâch werder hulde.\\ 
15 & die treit, der durch gedanke vert.\\ 
 & gedanc sich s\textit{un}ne\textit{n} blickes wert.\\ 
 & gedanc ist âne slôz \textbf{bespart},\\ 
 & vor aller crêatiur bewart.\\ 
 & gedanc ist vinster âne schîn.\\ 
20 & diu gotheit kan lûter sîn.\\ 
 & si glestet durch \textbf{der} vinster want\\ 
 & und het den helden sprunc gerant,\\ 
 & der e\textit{n}diu\textit{z}et \textbf{und} \textbf{e\textit{n}springet},\\ 
 & sô er von dem herzen \textbf{dringet}.\\ 
25 & ez ist \textbf{noch ein} gedanc sô snel,\\ 
 & ê er von dem herzen vür daz vel\\ 
 & ku\textit{m}e, er sî versuochet.\\ 
 & des kiuschen got \textit{g}e\textit{r}uochet.\\ 
 & sît got gedanke \textbf{speht} sô wol,\\ 
30 & ouwê der \textbf{blœden} werke dol!\\ 
\end{tabular}
\scriptsize
\line(1,0){75} \newline
m n o \newline
\line(1,0){75} \newline
\newline
\line(1,0){75} \newline
\textbf{1} minnære] mẏnneren o \textbf{2} süezen] suͯsse n (o) \textbf{4} wenket] wenckent m n o \textbf{5} wem] Wan o  $\cdot$ erzöugen] erzeigen n \textbf{7} selben] selbe o \textbf{12} götlîchen] gotliche m n (o)  $\cdot$ minne] mynne nuwe n \textbf{13} wandelt] wandel o \textbf{14} der] Die o \textbf{15} der] er o \textbf{16} sich] sin o  $\cdot$ sunnen] sines m \textbf{17} gedanc] [Gedandang]: Gedang n \textbf{23} endiuzet] erduffttet m  $\cdot$ und] noch n o  $\cdot$ enspringet] entspringet m \textbf{24} er] er er o  $\cdot$ dringet] trinnet o \textbf{27} kume] Kuͯne m  $\cdot$ sî versuochet] sie versuchen o \textbf{28} geruochet] versuͯchet m geruchen o \textbf{29} speht] spette o \newline
\end{minipage}
\end{table}
\newpage
\begin{table}[ht]
\begin{minipage}[t]{0.5\linewidth}
\small
\begin{center}*G
\end{center}
\begin{tabular}{rl}
 & \begin{large}V\end{large}on dem wâren minnære\\ 
 & \textbf{sagent} \textbf{disiu} \textbf{süezen} mære.\\ 
 & der ist ein durchliuhtic lieht,\\ 
 & \textbf{der} \textbf{s\textit{î}ne\textit{r} minne wenket} niht.\\ 
5 & swem er minne erzeigen sol,\\ 
 & \textbf{dem} wirt mit sîne\textit{r} minn\textit{e} wol.\\ 
 & die selben sint geteilt:\\ 
 & al \textbf{der} werlde ist geveilt\\ 
 & beidiu sîn minne unde sîn haz.\\ 
10 & \textbf{nû} prüevet, \textbf{wederz} helfe baz.\\ 
 & d\textit{er} schuldige ân riuwe\\ 
 & vliuht die götelîchen triuwe.\\ 
 & swer aber wandelt \textbf{sünden} schulde,\\ 
 & der dient nâch werder hulde.\\ 
15 & die treit, der durch gedanke vert.\\ 
 & gedanc sich sunnen blickes wert.\\ 
 & gedanc ist ân slôz \textbf{\textit{versp}art},\\ 
 & \textit{vor aller crêatûr bewart}.\\ 
 & gedanc ist vinster ân schîn.\\ 
20 & diu gotheit kan lûter sîn.\\ 
 & si glestet durch \textbf{der} vinster want\\ 
 & unde hât den helden sprunc gerant,\\ 
 & der endiuzet \textbf{noch} \textbf{enklinget},\\ 
 & sô er von dem herzen \textbf{springet}.\\ 
25 & ez ist \textbf{dehein} gedanc sô snel,\\ 
 & ê er vome herzen \textit{v}ür \textit{daz} vel\\ 
 & kom, er\textbf{n} sî versuochet.\\ 
 & des kiuschen got geruochet.\\ 
 & sît got gedanke \textbf{siht} sô wol,\\ 
30 & owê der \textbf{brœden} werke dol!\\ 
\end{tabular}
\scriptsize
\line(1,0){75} \newline
G I O L M Z Fr18 Fr22 Fr61 \newline
\line(1,0){75} \newline
\textbf{1} \textit{Initiale} G I O L Z Fr18 Fr61  \textbf{15} \textit{Initiale} I  \newline
\line(1,0){75} \newline
\textbf{1} Von] ÷on O  $\cdot$ dem] den I M \textbf{2} sagent] sagtan I  $\cdot$ süezen] \textit{om.} I O L Fr18 \textbf{3} der] Er Fr61  $\cdot$ ein] \textit{om.} L \textbf{4} der] vnde O (L) (M) (Z) (Fr18) (Fr61)  $\cdot$ sîner minne wenket] sinne minne wenchet G wenchet siner minne O (L) (M) (Z) (Fr18) entwenchet seiner minne Fr61 \textbf{5} swem] Wem L (M) \textbf{6} sîner minne] sinen minnen G \textbf{8} al der] Allir M (Z) (Fr61)  $\cdot$ ist] sint I \textbf{9} beidiu] Baẏden Fr61  $\cdot$ unde] vnd ouch Z \textbf{10} wederz] welhez O (L) widir osz M \textbf{11} der] du G \textbf{12} die götelîchen] goͤtleichev Fr61 \textbf{13} swer] Wer L M Z  $\cdot$ aber] \textit{om.} Fr61  $\cdot$ sünden] \textit{om.} O svnde Z \textbf{15} der] er O Fr61  $\cdot$ vert] wert Fr61 \textbf{16} sich] sint O L Fr22  $\cdot$ sunnen] svnne O L svnden Fr61  $\cdot$ blickes] bliche L (Fr22) lasters Fr61 \textbf{17} ist] sich M  $\cdot$ ân] ein I Fr22  $\cdot$ verspart] biwart G bespart O M Z (Fr22) \textbf{18} \textit{Vers 466.18 fehlt} G  \textbf{21} glestet] schient M  $\cdot$ der] die I \textbf{22} helden] ellenden Z \textbf{23} der endiuzet] Dort duzet M \textbf{24} sô er] Der L \textbf{25} ez] Ezen O (L) (M)  $\cdot$ gedanc] danch O  $\cdot$ sô] \textit{om.} I \textbf{26} ê] \textit{om.} M  $\cdot$ vür daz] wurze G furesz M \textbf{27} kom] Kvͤme Z  $\cdot$ ern] her M \textbf{29} got] er L Fr22  $\cdot$ siht] speht O L (M) Z Fr22 \textbf{30} brœden] vroudin M \newline
\end{minipage}
\hspace{0.5cm}
\begin{minipage}[t]{0.5\linewidth}
\small
\begin{center}*T
\end{center}
\begin{tabular}{rl}
 & von dem wâren minnære\\ 
 & \textbf{saget} \textbf{uns} \textbf{diz} mære.\\ 
 & der ist ein durchliuhtic lieht\\ 
 & \textbf{unde} \textbf{wenk\textit{e}t sîner minne} nieht.\\ 
5 & swem er minne erzeigen sol,\\ 
 & \textbf{der} wirt mit sîner minne wol.\\ 
 & die selben sint ge\textit{t}ei\textit{l}et:\\ 
 & aller \textbf{der} werlte ist geveilet\\ 
 & beidiu sîn minne unde sîn haz,\\ 
10 & \textbf{unde} prüevet, \textbf{wederz} helfe baz.\\ 
 & der schuldige âne riuwe\\ 
 & vliuhet die götelîchen triuwe.\\ 
 & swer aber wandelt \textbf{sünden} schulde,\\ 
 & der dient nâch werder hulde.\\ 
15 & \dag der\dag , der durch gedenke vert.\\ 
 & gedenke \dag sint sünde\dag  blickes wert.\\ 
 & gedanc ist âne slôz \textbf{bespart},\\ 
 & vor aller crêatûre bewart.\\ 
 & gedanc ist vinster âne schîn.\\ 
20 & diu goteheit kan lûter sîn.\\ 
 & si glestet durch \textbf{die} vinster want\\ 
 & unde hât den helnden sprunc gerant,\\ 
 & der \textit{e}ndiuzet \textbf{noch} \textbf{enklinget},\\ 
 & sô er von dem herzen \textbf{\textit{spr}inget}.\\ 
25 & ez ist \textbf{dehein} gedanc sô snel,\\ 
 & ê er vome herzen vür daz vel\\ 
 & kom\textit{e}, er\textbf{n} sî versuochet.\\ 
 & des kiusche\textit{n} got geruochet.\\ 
 & sît \textit{got} gedanke \textbf{spehet} sô wol,\\ 
30 & ouwê der \textbf{brœden} werke dol!\\ 
\end{tabular}
\scriptsize
\line(1,0){75} \newline
T U V W Q R Fr42 \newline
\line(1,0){75} \newline
\textbf{1} \textit{Initiale} Q   $\cdot$ \textit{Capitulumzeichen} R  \newline
\line(1,0){75} \newline
\textbf{1} \textit{Die Verse 453.1-502.30 fehlen} U   $\cdot$ dem] den W R \textbf{2} saget] Sagent W (Q) R  $\cdot$ uns diz] vns [*]: dise V vnß die W (Q) (R) \textbf{4} wenket] wenkent T \textbf{5} swem] Wem W Q R  $\cdot$ erzeigen] erzoͤigen V zeigen R  $\cdot$ sol] \textit{om.} Q \textbf{6} der] dem V W Q R \textbf{7} geteilet] geleitet T \textbf{9} minne] mein Q \textbf{10} unde] Nv V (W) (Q) (R)  $\cdot$ prüevet] pruͯffen R \textbf{11} schuldige] vnschuldig W \textbf{12} götelîchen] gotliche V  $\cdot$ triuwe] trewen Q \textbf{13} swer] Wer W Q R \textbf{15} der der] Die treit [*]: der V Der tregt der W Der teeit der Q Die treit er R \textbf{16} gedenke sint sünde] [Gedank*]: Gedanke sine svnne V Gedenke sind sunnen R \textbf{17} ist] seint Q  $\cdot$ slôz] alle schlos W  $\cdot$ bespart] gespart Q R \textbf{18} vor] Von R \textbf{21} glestet] gleiste Q  $\cdot$ die] [*]: der V der W Q \textbf{22} hât] hant Q \textbf{23} endiuzet] ern dvzet T endruczet R  $\cdot$ enklinget] enspringet V \textbf{24} herzen] herten W hertze Q  $\cdot$ springet] klinget T [*]: dringet V \textbf{25} ist] enist V (Q) R  $\cdot$ gedanc] danck Q (R) \textbf{27} kome] comen T  $\cdot$ ern sî] er sei W \textbf{28} kiuschen] kvsce T \textbf{29} got] \textit{om.} T  $\cdot$ gedanke] gedancken Q \textbf{30} brœden werke] boͯsen werken R \newline
\end{minipage}
\end{table}
\end{document}
