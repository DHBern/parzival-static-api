\documentclass[8pt,a4paper,notitlepage]{article}
\usepackage{fullpage}
\usepackage{ulem}
\usepackage{xltxtra}
\usepackage{datetime}
\renewcommand{\dateseparator}{.}
\dmyyyydate
\usepackage{fancyhdr}
\usepackage{ifthen}
\pagestyle{fancy}
\fancyhf{}
\renewcommand{\headrulewidth}{0pt}
\fancyfoot[L]{\ifthenelse{\value{page}=1}{\today, \currenttime{} Uhr}{}}
\begin{document}
\begin{table}[ht]
\begin{minipage}[t]{0.5\linewidth}
\small
\begin{center}*D
\end{center}
\begin{tabular}{rl}
\textbf{313} & \begin{large}D\end{large}iu maget witze rîche\\ 
 & was gevar \textbf{den} ungelîche,\\ 
 & die man \textbf{dâ} heizet bêâ schent.\\ 
 & ein brûtlachen von Gent,\\ 
5 & noch blâwer denne ein lâsûr,\\ 
 & \textbf{het} an \textbf{geleit} der vreuden schûr.\\ 
 & daz was ein kappe wol gesniten\\ 
 & al nâch der Franzoyser siten.\\ 
 & \textbf{drunde} an ir lîbe was pfelle guot.\\ 
10 & von Lunders ein \textbf{pfæwîn huot},\\ 
 & gefurriert mit einem plîat\\ 
 & - der huot \textbf{was} niwe, diu snuor niht alt -,\\ 
 & der hieng ir an dem rücke.\\ 
 & ir mære was ein brücke:\\ 
15 & \textbf{über vreude ez jâmer} truoc.\\ 
 & si zuct in schimpfes dâ genuoc.\\ 
 & Über den huot ein zopf ir swanc\\ 
 & unz ûf den mûl. der was \textbf{sô} lanc,\\ 
 & swarz, herte \textbf{unt} niht ze clâr,\\ 
20 & linde als \textbf{eines} swînes \textbf{rückehâr}.\\ 
 & Si was genast als ein hunt.\\ 
 & zwêne ebers zene ir vür den munt\\ 
 & giengen wol \textbf{spannen} lanc.\\ 
 & ietweder wintbrâ sich \textbf{dranc}\\ 
25 & mit zöpfen vür die hâres snuor.\\ 
 & mîn zuht \textbf{mit} wârheit missevuor,\\ 
 & daz ich sus muoz von vrouwen sagen.\\ 
 & nehein enderiu darf ez von mir klagen.\\ 
 & Cundrie truog ôren als ein ber,\\ 
30 & niht nâch vriwendes minne ger.\\ 
\end{tabular}
\scriptsize
\line(1,0){75} \newline
D \newline
\line(1,0){75} \newline
\textbf{1} \textit{Initiale} D  \textbf{17} \textit{Majuskel} D  \textbf{21} \textit{Majuskel} D  \newline
\line(1,0){75} \newline
\textbf{5} lâsûr] laswͦr D \textbf{29} Cundrie] Cvndrîe D \newline
\end{minipage}
\hspace{0.5cm}
\begin{minipage}[t]{0.5\linewidth}
\small
\begin{center}*m
\end{center}
\begin{tabular}{rl}
 & \begin{large}D\end{large}iu maget witze rîche\\ 
 & was gevar \textbf{den} ungelîche,\\ 
 & die man heizet bêâ schent.\\ 
 & ein brûtlachen von Gent,\\ 
5 & noch blâwer \textit{denne} ein lâsûr,\\ 
 & \textbf{hete} an \textbf{geleit} der vröude\textit{n} \textit{s}chûr.\\ 
 & daz was ein \textit{k}appe wolgesniten\\ 
 & al nâch der Franzoiser siten.\\ 
 & \textbf{drunde} an ir lîbe was pfelle guot.\\ 
10 & von L\textit{un}ders ein \textbf{pfæwîn huot},\\ 
 & gefurrieret mit einem blîalt\\ 
 & - der huot niuwe, diu snuor niht alt -,\\ 
 & der hienc ir an dem rücke.\\ 
 & ir mære was ein brücke,\\ 
15 & \textbf{diu jâmer über die vröude} truoc.\\ 
 & si zuckete in schimpfes d\textit{â} genuoc.\\ 
 & über den huot ein zopf ir swanc\\ 
 & unz ûf den mûl. der was \textbf{sô} lanc,\\ 
 & swarz, herte \textbf{und} niht ze clâr,\\ 
20 & lind als \textbf{ein} swînes \textbf{rückehâr}.\\ 
 & si was gena\textit{s}et als ein hunt.\\ 
 & zwêne ebers zene ir vür den munt\\ 
 & giengen wol \textbf{vingers} lanc.\\ 
 & ietweder wint\textit{b}râ sich \textbf{dranc}\\ 
25 & mit zöpfen vür die hârs\textit{n}uor.\\ 
 & mîn zuht \textbf{durch} wârheit missevuor,\\ 
 & daz ich sus muoz von vrouwen sagen.\\ 
 & \textbf{si}, kein anderiu, darf ez von mir klagen.\\ 
 & Condrie truoc ôren als ein ber,\\ 
30 & niht nâch vriundes minne ger.\\ 
\end{tabular}
\scriptsize
\line(1,0){75} \newline
m n o \newline
\line(1,0){75} \newline
\textbf{1} \textit{Initiale} m n o  \newline
\line(1,0){75} \newline
\textbf{1} witze] witzen n (o) \textbf{2} den] dem n din o \textbf{3} heizet] heissent o  $\cdot$ bêâ schent] [bratent]: brascent o \textbf{4} brûtlachen] bruttelachen m (n) (o)  $\cdot$ von Gent] fundent o \textbf{5} blâwer] blaher o  $\cdot$ denne] \textit{om.} m  $\cdot$ lâsûr] [s]: lasuͯr o \textbf{6} vröuden schûr] froͯden lasur vnd schur m freuͯiden [sur]: [schur]: schuͯr o \textbf{7} kappe] knappe m \textbf{8} al] Alles n  $\cdot$ Franzoiser] franczosser m frantzoiser n franczoser o \textbf{9} lîbe] \textit{om.} n \textbf{10} Lunders] lauders m \textbf{11} gefurrieret] Gefruriert n \textbf{12} huot] huͯte m n o \textbf{13} rücke] rucken m (n) \textbf{14} brücke] brucken m n \textbf{15} jâmer über] jame vbe o \textbf{16} dâ] do m n o \textbf{17} ir swanc] erswang n \textbf{18} mûl der] milder o \textbf{19} und] \textit{om.} n o \textbf{20} ein] ein linde n \textbf{21} genaset] genasset m \textbf{22} ebers] eber n o \textbf{24} wintbrâ] winthera m ougbrowe n (o) \textbf{25} hârsnuor] har suͯr m \textbf{28} si] Nie n o \textbf{29} Condrie] Condri n o \newline
\end{minipage}
\end{table}
\newpage
\begin{table}[ht]
\begin{minipage}[t]{0.5\linewidth}
\small
\begin{center}*G
\end{center}
\begin{tabular}{rl}
 & diu maget witze rîche\\ 
 & was gevar \textbf{dem} ungelîche,\\ 
 & die man \textbf{dâ} heizet bêâ schent.\\ 
 & ein brûtlachen von Gent,\\ 
5 & noch blâwer danne ein lâzûr,\\ 
 & \textbf{\textit{he}t} an \textbf{i\textit{r}} der vröuden schûr.\\ 
 & daz was ein kappe wolgesniten\\ 
 & al nâch der Franzoiser siten.\\ 
 & \textbf{unden} an ir lîbe \textit{was} pfelle guot.\\ 
10 & von Lunders ein \textbf{pfâwenhuot},\\ 
 & gefurrie\textit{r}t mit einem blîalt\\ 
 & - der huot \textbf{was} niwe, diu snuor niht alt -,\\ 
 & der hienc ir an dem rücke.\\ 
 & ir mære was ein brücke:\\ 
15 & \textbf{über vröude ez jâmer} truoc.\\ 
 & si zucte in schimpfes dâ genuoc.\\ 
 & über den huot ein zopf ir swanc\\ 
 & unz ûf den mûl. der was \textbf{sô} lanc,\\ 
 & swarz, herte, niht ze clâr,\\ 
20 & linde als \textbf{eines} swînes \textbf{hâr}.\\ 
 & si was genaset als ein hunt.\\ 
 & zwêne ebers zen ir vür den munt\\ 
 & giengen wol \textbf{spanne} lanc.\\ 
 & ietweder wintbrâ sich \textbf{dranc}\\ 
25 & mit zöpfen vür die hârsnuor.\\ 
 & mîn zuht \textbf{durch} wârheit missevuor,\\ 
 & daz ich sus muoz von vrouwen sagen.\\ 
 & dehein anderiu darf ez von mir klagen.\\ 
 & \textit{Gundr}i\textit{e} truoc ôren als ein ber,\\ 
30 & niht nâch vriundes minn\textit{e} ger.\\ 
\end{tabular}
\scriptsize
\line(1,0){75} \newline
G I O L M Q R Z Fr64 \newline
\line(1,0){75} \newline
\textbf{5} \textit{Initiale} L  \textbf{13} \textit{Initiale} I  \textbf{17} \textit{Initiale} Z  \textbf{29} \textit{Initiale} I  \newline
\line(1,0){75} \newline
\textbf{1} \textit{Die Verse 312.7-313.4 fehlen} L   $\cdot$ witze] wiczen R \textbf{2} dem] den Q R Z \textbf{3} die] Div O  $\cdot$ dâ] do Q  $\cdot$ bêâ schent] beadschent G Beatschent O begament R \textbf{5} danne] wan M (R)  $\cdot$ ein] \textit{om.} R  $\cdot$ lâzûr] lazvͦr O lazuͯr L lasur Q (R) \textbf{6} het an ir] foͮrt an im G \textbf{7} daz] Da R  $\cdot$ kappe] knape M (Q)  $\cdot$ wolgesniten] von geschnitten R \textbf{8} al] Alz L (Q) o\textit{m. } Z  $\cdot$ Franzoiser] franzoyser G fronzoysen I [francoisen]: franzoisen O frantzoiser L Q franziosir M franzosser R \textbf{9} Vndan an dem libe waz der pfeller guͯot R  $\cdot$ unden] Vnd den Q  $\cdot$ was] \textit{om.} G \textbf{10} Lunders] luͯders L lundirs M (Z) luders Q  $\cdot$ pfâwenhuot] phawe huͤt I pfewin hvͦt O (Z) \textbf{11} gefurriert] gefurriet G Gevormeret M (Q)  $\cdot$ mit] mit mit R  $\cdot$ blîalt] pfalt Q \textbf{12} der] Dy M  $\cdot$ diu snuor] die snvre L vnd Z \textbf{13} der hienc] Er ginc I Der M  $\cdot$ dem] den I  $\cdot$ rücke] Rugen R \textbf{15} über] daz vber I Aber Q R  $\cdot$ ez] \textit{om.} I daz O (M) \textbf{16} Vil freuͯden sie do nýder sluͯg L  $\cdot$ in] im Q  $\cdot$ schimpfes dâ] shinphes I (Z) da schinfes M schimpes do R \textbf{17} ein] ir I  $\cdot$ zopf] zepff R  $\cdot$ ir] er L Fr64 \textbf{18} ûf] \textit{om.} O  $\cdot$ den] [dem]: den Z dem Fr64  $\cdot$ der] des Q \textit{om.} R \textbf{19} swarz] So was M  $\cdot$ niht] vnde niht O (L) (M) (Q) (R) (Z) (Fr64)  $\cdot$ ze] so L Q \textit{om.} R \textbf{20} eines] \textit{om.} O ein Q Z  $\cdot$ hâr] rvke har O (Q) (R) (Z) ruͯches har L zcucke har M \textbf{22} ebers] eber Fr64  $\cdot$ ir vür] fvr O (L) (M) (Z) gieng ir fᵫr R \textbf{23} spanne] spannen L (Q) R Fr64 \textbf{24} wintbrâ] wang R \textbf{25} Fúr zoͯpffen mit der har schnuͦr R  $\cdot$ zöpfen] koppfen Q zopfer Z  $\cdot$ hârsnuor] harsnvre Z \textbf{26} mîn] mit I (M)  $\cdot$ durch] mit Q R Z  $\cdot$ missevuor] missefvre Z \textbf{27} muoz] [furte]: muste M  $\cdot$ sagen] [chlagen]: sagen I \textbf{28} ez] \textit{om.} R  $\cdot$ klagen] nit clagen R \textbf{29} Gundrie] si G Kvndrie O (Q) Z Kondrie M R  $\cdot$ truoc] trvͦge O truͯhc L  $\cdot$ ôren] horn L ore M  $\cdot$ ber] her O \textbf{30} vriundes] frides O \textit{om.} M  $\cdot$ minne] minnen G \newline
\end{minipage}
\hspace{0.5cm}
\begin{minipage}[t]{0.5\linewidth}
\small
\begin{center}*T
\end{center}
\begin{tabular}{rl}
 & \begin{large}D\end{large}iu maget witze rîche\\ 
 & was gevar \textbf{dem} unglîche,\\ 
 & die man \textbf{dâ} heizet bêâ gent.\\ 
 & ein brûtlachen von Gent,\\ 
5 & noch blâwer danne ein lâsûr,\\ 
 & \textbf{hât} an \textbf{geleit} der vröuden schûr.\\ 
 & \multicolumn{1}{l}{ - - - }\\ 
 & \multicolumn{1}{l}{ - - - }\\ 
 & \textbf{drunder} an ir lîbe was pfelle guot.\\ 
10 & von Lunders ein \textbf{pfæwîn huot},\\ 
 & gefurrieret mit einem blîalt\\ 
 & - der huot \textbf{was} niuwe, diu snuor niht alt -,\\ 
 & der hienc ir an dem rücke.\\ 
 & ir mære was ein brücke\\ 
15 & \textbf{über vröude, diu jâmer} truoc.\\ 
 & si zuht in schimpfes dâ genuoc.\\ 
 & über den huot ein zopf ir swanc\\ 
 & unz ûf den mûl. der was lanc,\\ 
 & swarz, herte \textbf{unde} niht ze clâr,\\ 
20 & linde als \textbf{eines} swînes \textbf{rückehâr}.\\ 
 & si was genase\textit{t} als ein hunt.\\ 
 & zwêne ebers zene ir vür den munt\\ 
 & gien\textit{gen} wol \textbf{spannen} lanc.\\ 
 & ietweder wintbrâ sich \textbf{twanc}\\ 
25 & mit zöpfen vür die hârsnuo\textit{r}.\\ 
 & mîn zuht \textbf{durch} wârheit missevuor,\\ 
 & daz ich sus muoz von vrouwen sagen.\\ 
 & dehein anderiu darf ez von mir klagen,\\ 
 & \textbf{wan} Kundrie truoc ôren als ein ber,\\ 
30 & niht nâch vriundes minne ger.\\ 
\end{tabular}
\scriptsize
\line(1,0){75} \newline
T U V W \newline
\line(1,0){75} \newline
\textbf{1} \textit{Initiale} T U W  \textbf{25} \textit{Initiale} V  \newline
\line(1,0){75} \newline
\textbf{1} witze] witzen V \textbf{2} dem] den V W \textbf{3} dâ] do U V W  $\cdot$ bêâ gent] von beazent W \textbf{4} brûtlachen] stuͦlachen W \textbf{5} lâsûr] lasuͦr U \textbf{6} hât] Het U Hette V \textbf{7} \textit{Die Verse 313.7-8 fehlen} T U W   $\cdot$ Daz was ein cappe wol gesnitten V \textbf{8} Al nach der franzoẏser sitten V \textbf{9} Darunder was ein pfellel guͦt W  $\cdot$ pfelle] plege U \textbf{10} Lunders] luͦnders U  $\cdot$ ein] einen W \textbf{11} gefurrieret] Geformieret U  $\cdot$ blîalt] plyat W \textbf{12} Die schnuͦr was neúwe zuͦ der wat W \textbf{13} der] Der huͦt W  $\cdot$ ir] \textit{om.} V  $\cdot$ dem] irme V \textbf{15} diu] das W \textbf{16} zuht] zuͦch U [z*]: zvhte V  $\cdot$ dâ] do V W \textbf{17} huot] huͦp U  $\cdot$ ein] ir U  $\cdot$ ir] \textit{om.} W \textbf{18} unz] Mit U \textbf{20} eines swînes rückehâr] ein schweines har W \textbf{21} genaset] genase T W \textbf{22} ebers zene] zene giengen W \textbf{23} giengen wol] gienc wol T Wol einer grossen W \textbf{24} wintbrâ sich] bra sy W  $\cdot$ twanc] drang V \textbf{25} mit] Mit den W  $\cdot$ hârsnuor] har snvͦc T \textbf{27} sus] \textit{om.} W  $\cdot$ vrouwen] froͤden W \textbf{28} anderiu] fraw W \textbf{29} Kundrie] kuͦndrie U dundrie W \textbf{30} Ir nacht was nit nacht frúndes ger W \newline
\end{minipage}
\end{table}
\end{document}
