\documentclass[8pt,a4paper,notitlepage]{article}
\usepackage{fullpage}
\usepackage{ulem}
\usepackage{xltxtra}
\usepackage{datetime}
\renewcommand{\dateseparator}{.}
\dmyyyydate
\usepackage{fancyhdr}
\usepackage{ifthen}
\pagestyle{fancy}
\fancyhf{}
\renewcommand{\headrulewidth}{0pt}
\fancyfoot[L]{\ifthenelse{\value{page}=1}{\today, \currenttime{} Uhr}{}}
\begin{document}
\begin{table}[ht]
\begin{minipage}[t]{0.5\linewidth}
\small
\begin{center}*D
\end{center}
\begin{tabular}{rl}
\textbf{322} & \begin{large}H\end{large}êr Gawan sol sich niht \textbf{verschemen},\\ 
 & ob er \textbf{geselleschaft} wil nemen\\ 
 & ob der tavelrunder,\\ 
 & diu dort stêt besunder.\\ 
5 & \textbf{der} reht wære \textbf{gebrochen} sân,\\ 
 & sæze drob ein triwelôser man.\\ 
 & I\textbf{ne} bin her niht durch schelten komen.\\ 
 & geloubet, sît irz habt vernomen,\\ 
 & ich vorder kampf vür schelten,\\ 
10 & der niht wan tôt sol gelten\\ 
 & oder leben \textbf{mit} êren,\\ 
 & swenz \textbf{wil diu sælde} lêren."\\ 
 & Der künec swîgete unt \textbf{was} unvrô,\\ 
 & doch antwurte er der rede alsô:\\ 
15 & "hêrre, er ist mîner swester sun.\\ 
 & wære Gawan tôt, ich wolde tuon\\ 
 & den kampf, ê \textbf{sîn} gebeine\\ 
 & læge triwelôs unreine.\\ 
 & wil gelücke, iu sol Gawans hant\\ 
20 & \textbf{mit kampfe} tuon daz wol bekant,\\ 
 & daz sîn lîp mit triwen vert\\ 
 & unt sich\textbf{s} valsches hât erwert.\\ 
 & \textbf{hab} iu anders iemen leit\\ 
 & getân, sô\textbf{ne} machet niht sô breit\\ 
25 & sîn laster âne schulde;\\ 
 & wan \textbf{erwirbet} er iwer hulde,\\ 
 & sô daz sîn lîp unschuldec ist,\\ 
 & ir habt \textbf{in} dirre kurzen vrist\\ 
 & gesagt, daz iweren prîs\\ 
30 & krenket, sint die liute wîs."\\ 
\end{tabular}
\scriptsize
\line(1,0){75} \newline
D \newline
\line(1,0){75} \newline
\textbf{1} \textit{Initiale} D  \textbf{7} \textit{Majuskel} D  \textbf{13} \textit{Majuskel} D  \newline
\line(1,0){75} \newline
\newline
\end{minipage}
\hspace{0.5cm}
\begin{minipage}[t]{0.5\linewidth}
\small
\begin{center}*m
\end{center}
\begin{tabular}{rl}
 & hêr Gawan sol sich niht \textbf{verschemen},\\ 
 & ob er \textbf{geselleschaft} wil nemen\\ 
 & ob der tavelrunder,\\ 
 & diu dort stât besunder.\\ 
5 & \textbf{der} reht wære \textbf{gebr\textit{o}chen} sân,\\ 
 & sæze drobe ein triuwelôser man.\\ 
 & ich bi\textit{n h}er niht durch schelten komen.\\ 
 & geloubet, sît irz hât vernomen,\\ 
 & ich vordere kamp\textit{f v}ür schelten,\\ 
10 & der niht wan tôt sol gelten\\ 
 & oder \textbf{aber} leben \textbf{mit} êren,\\ 
 & wenne ez \textbf{wil diu sælde} lêren."\\ 
 & der künic sweic und \textbf{was} unvrô,\\ 
 & doch antwurte er \textbf{ime} der rede alsô:\\ 
15 & "hêrre, er ist mîner swester sun.\\ 
 & wære Gawan tôt, ich wolte tuon\\ 
 & den kampf, ê \textbf{mîn} gebeine\\ 
 & læge triuwelôs unreine.\\ 
 & wil glücke, iu sol Gawanes hant\\ 
20 & \textbf{mit kampfe} tuon daz wol bekant,\\ 
 & daz sîn lîp mit triuwen vert\\ 
 & und sich valsches hât erwert.\\ 
 & \textbf{habe} iu anders iemen leit\\ 
 & getân, sô machet niht sô breit\\ 
25 & sîn laster âne schulde;\\ 
 & wanne \textbf{erwirbet} er iuwer hulde,\\ 
 & sô daz sîn lîp unschuldic ist,\\ 
 & ir habt \textbf{in} dirre kurzer vrist\\ 
 & \textbf{vo\textit{n} ime} gesaget, daz iuweren prîs\\ 
30 & krenket, sint die liute wîs."\\ 
\end{tabular}
\scriptsize
\line(1,0){75} \newline
m n o \newline
\line(1,0){75} \newline
\newline
\line(1,0){75} \newline
\textbf{1} Gawan] gewan o  $\cdot$ verschemen] verschamen m (n) (o) \textbf{2} wil] sol o \textbf{5} gebrochen] gebrachen m \textbf{6} drobe ein triuwelôser] daruͯff eyn ruweloͯser o \textbf{7} bin] bin hie m \textbf{9} kampf] kampf vnd m \textbf{14} antwurte] antwuͯrt o  $\cdot$ er] \textit{om.} n o \textbf{16} Wer gewan ich dot wol thuͯn o \textbf{18} læge] Legege o \textbf{26} erwirbet er] er erwirbet n o \textbf{28} in] an n  $\cdot$ dirre kurzer] disen kurtzen n (o) \textbf{29} von] Vo m  $\cdot$ iuweren] iren m uwer n (o) \newline
\end{minipage}
\end{table}
\newpage
\begin{table}[ht]
\begin{minipage}[t]{0.5\linewidth}
\small
\begin{center}*G
\end{center}
\begin{tabular}{rl}
 & hêr Gawan sol sich niht \textbf{verschemen},\\ 
 & ober \textbf{gesellicheit} wil nemen\\ 
 & obe der tavelrunder,\\ 
 & diu dort stêt besunder.\\ 
5 & \textbf{ir} reht wære \textbf{gebrochen} sân,\\ 
 & sæze drobe ein triuwelôser man.\\ 
 & ich\textbf{ne} bin her niht durch schelten komen.\\ 
 & geloubet, sît irz habet vernomen,\\ 
 & ich vordere kampf vür schelten,\\ 
10 & der niht wan tôt sol gelten\\ 
 & oder leben \textbf{nâch} êren,\\ 
 & swen ez \textbf{wil diu sælde} lêren."\\ 
 & der künic sweic unde \textbf{wart} unvrô,\\ 
 & doch antwurte er der rede alsô:\\ 
15 & "hêrre, er ist mîner swester sun.\\ 
 & wære Gawan tôt, ich wolt tuon\\ 
 & den kampf, ê \textbf{sîn} gebeine\\ 
 & læge triuwelôs unreine.\\ 
 & wil gelücke, iu sol Gawanes hant\\ 
20 & \textbf{mit kampfe} tuon daz wol bekant,\\ 
 & daz sîn lîp mit triuwen vert\\ 
 & unde sich \textbf{des} valsches hât erwert.\\ 
 & \textbf{hâ\textit{t}} iu anders iemen leit\\ 
 & getân, sô machet niht sô breit\\ 
25 & sîn laster âne schulde;\\ 
 & wan \textbf{gewinnt} er iuwer hulde,\\ 
 & sô daz sîn lîp unschuldic ist,\\ 
 & ir habet \textbf{an} dirre kurzen vrist\\ 
 & \textbf{von im} gesaget, daz iuweren prîs\\ 
30 & krenket, sint die liute wîs."\\ 
\end{tabular}
\scriptsize
\line(1,0){75} \newline
G I O L M Q R Z Fr22 Fr39 Fr40 \newline
\line(1,0){75} \newline
\textbf{9} \textit{Initiale} O L Fr22  \textbf{13} \textit{Initiale} R  \textbf{15} \textit{Initiale} Z  \textbf{25} \textit{Initiale} Q Fr40  \newline
\line(1,0){75} \newline
\textbf{1} Gawan] [*]: gawan L gewan Fr40  $\cdot$ sich] ich L Fr39  $\cdot$ niht] \textit{om.} O Q \textbf{2} gesellicheit] geselheit R gisellischaft Fr22  $\cdot$ wil] welle R \textbf{3} tavelrunder] tauelrunde I \textbf{4} diu dort] die dor Fr39  $\cdot$ stêt] sten Q \textbf{5} reht] \textit{om.} L Fr22  $\cdot$ gebrochen] zerbrochen R \textbf{7} ichne] ich I (O) (Q) (R)  $\cdot$ her niht] niht her L Fr39 \textbf{8} geloubet] gelaup ez I Gelovpt ez O (Q) (Fr40) Gelobt es R \textbf{9} ich] ÷ch O  $\cdot$ vordere] fuͯrder L  $\cdot$ schelten] selten I \textbf{10} wan] \textit{om.} M den Q \textbf{11} nâch] mit Z \textbf{12} Vnde sinin pris gemêrin Fr22  $\cdot$ swen] swar I Wen L R Wann Q  $\cdot$ wil diu sælde] diu selde wil I (M) wil die selden Q  $\cdot$ lêren] chern I lere Z \textbf{13} sweic] swigt I swigete L  $\cdot$ wart] was O (L) M (Q) (R) Z Fr22 Fr40 \textbf{16} tôt] niht O \textbf{17} sîn] [s*]: ich sin Z \textbf{18} solde vuln indem meine I  $\cdot$ læge] Liezze ligen Z \textbf{19} wil gelücke] \textit{om.} I  $\cdot$ iu] so L M Fr22  $\cdot$ Gawanes] Gawans I O (M) R Z Fr22 Fr39 Gawanz L gewans Q \textbf{20} daz] \textit{om.} Fr22  $\cdot$ wol bekant] wol wirt erkant L (Fr39) wolkekant R \textbf{21} daz] >daz< G \textbf{23} hât] habe G  $\cdot$ anders iemen] iemen anders I anders nymant Q anderre ieman Fr39 \textbf{24} sô machet] son machet I (Q) (Fr40) Somachencz R  $\cdot$ sô] sie Q \textbf{26} wan gewinnt er] Wan er gewýnnet L (Fr39) Wender giwinnit Fr22 wand gewinnet Fr40 \textbf{28} an dirre kurzen] andirs kurcze M an diesem kurtze Q \textbf{29} iuweren] iwerm O \textbf{30} sint] sein Q \newline
\end{minipage}
\hspace{0.5cm}
\begin{minipage}[t]{0.5\linewidth}
\small
\begin{center}*T
\end{center}
\begin{tabular}{rl}
 & Hêr Gawan sol sich \textbf{des} niht \textbf{beschemen},\\ 
 & ob er \textbf{geselleschaft} wil nemen\\ 
 & ob der tavelrunder,\\ 
 & di\textit{u} dort stât besunder.\\ 
5 & \textbf{ir} reht wære \textbf{zerbrochen} sân,\\ 
 & sæze drobe ein triuwelôser man.\\ 
 & ich bin her niht durch schelten komen.\\ 
 & geloubet, sît irz habt vernomen,\\ 
 & ich vordere kampf vür schelten,\\ 
10 & der niht wan tôt sol gelten\\ 
 & oder leben \textbf{nâch} êren,\\ 
 & swenz \textbf{diu sælde wil} lêren."\\ 
 & \begin{large}D\end{large}er künec sweic unde \textbf{wart} unvrô,\\ 
 & doch antwurter der rede alsô:\\ 
15 & "hêrre, er ist mîner swester sun.\\ 
 & wære Gawan tôt, ich wolte tuon\\ 
 & den kampf, ê \textbf{sîn} gebeine\\ 
 & læge triuwelôs \textbf{unde} unreine.\\ 
 & wil glücke, iu sol Gawans hant\\ 
20 & tuon daz wol bekant,\\ 
 & daz sîn lîp mit triuwen vert\\ 
 & unde sich \textbf{des} valsches hât erwert.\\ 
 & \textbf{hât} iu anders ieman leit\\ 
 & getân, sô machet niht sô breit\\ 
25 & sîn laster âne schulde;\\ 
 & wan \textbf{gewinnet} er iuwer hulde,\\ 
 & sô daz sîn lîp unschuldic ist,\\ 
 & ir habt \textbf{an} dirre kurzen vrist\\ 
 & \textbf{von im} gesaget, daz iuwern prîs\\ 
30 & krenket, sint die liute wîs."\\ 
\end{tabular}
\scriptsize
\line(1,0){75} \newline
T U V W \newline
\line(1,0){75} \newline
\textbf{1} \textit{Majuskel} T  \textbf{13} \textit{Initiale} T U W  \newline
\line(1,0){75} \newline
\textbf{1} des] \textit{om.} U V W \textbf{4} diu] die T \textbf{7} niht durch schelten] durch schelten nit W \textbf{8} sît] [*]: sit V  $\cdot$ irz] ir W \textbf{9} vordere] vor der U \textbf{11} leben nâch] [*]: aber leben mit V \textbf{12} swenz] Wan iz U (W)  $\cdot$ lêren] [*]: leren V eren W \textbf{13} wart] waz W \textbf{14} antwurter] antwúrt er V (W)  $\cdot$ alsô] so U W \textbf{19} wil] [*]: Wil V Wil es W  $\cdot$ iu] ich W  $\cdot$ Gawans] Gawanes U \textbf{20} tuon] [*]: Mit canpfe tuͦn V Mit kampffe thuͦn W \textbf{22} valsches] valchen W \textbf{23} iu] auch W \textbf{28} habt] hahet W \newline
\end{minipage}
\end{table}
\end{document}
