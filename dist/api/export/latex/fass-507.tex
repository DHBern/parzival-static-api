\documentclass[8pt,a4paper,notitlepage]{article}
\usepackage{fullpage}
\usepackage{ulem}
\usepackage{xltxtra}
\usepackage{datetime}
\renewcommand{\dateseparator}{.}
\dmyyyydate
\usepackage{fancyhdr}
\usepackage{ifthen}
\pagestyle{fancy}
\fancyhf{}
\renewcommand{\headrulewidth}{0pt}
\fancyfoot[L]{\ifthenelse{\value{page}=1}{\today, \currenttime{} Uhr}{}}
\begin{document}
\begin{table}[ht]
\begin{minipage}[t]{0.5\linewidth}
\small
\begin{center}*D
\end{center}
\begin{tabular}{rl}
\textbf{507} & \begin{large}I\end{large}ch \textbf{en}wânde niht, daz ez kœme alsus.\\ 
 & Lischoys Gwelljus\\ 
 & hât mich sêre geletzet\\ 
 & unt hinderz ors gesetzet\\ 
5 & mit einer tjoste rîche;\\ 
 & diu ergienc sô hurteclîche\\ 
 & durch mînen schilt unt durch den lîp.\\ 
 & dô half mir \textbf{ditze} \textbf{guote} wîp\\ 
 & ûf ir \textbf{pfert} an \textbf{dise} stat."\\ 
10 & Gawanen er sêre belîben bat.\\ 
 & Gawan sprach, er wolde \textbf{sehen},\\ 
 & wâ im der schade \textbf{dâ} wære geschehen.\\ 
 & "lît Logroys sô nâhen,\\ 
 & mac ich in dâr vor ergâhen,\\ 
15 & sô muoz er antwürten mir.\\ 
 & ich \textbf{vrâge} in, waz er ræche an dir."\\ 
 & "\textbf{des} \textbf{en}tuo niht", sprach der wunde man,\\ 
 & "der wârheit ich dir jehen kan;\\ 
 & dar engêt niht kinde reise,\\ 
20 & \textbf{ez} mac wol heizen vreise."\\ 
 & Gawan die wunden verbant\\ 
 & mit der vrouwen houbtgewant.\\ 
 & er sprach zer wunden \textbf{wunden} segen;\\ 
 & \textbf{er} bat got man unt wîbes pflegen.\\ 
25 & er vant \textbf{al} bluotec ir slâ,\\ 
 & als ein hirze wære erschozzen dâ.\\ 
 & daz \textbf{en}liez niht irre \textbf{in} rîten;\\ 
 & er sach in kurzen zîten\\ 
 & Logroys, die gehêrten.\\ 
30 & vil l\textit{i}ute mit lobe si êrten.\\ 
\end{tabular}
\scriptsize
\line(1,0){75} \newline
D \newline
\line(1,0){75} \newline
\textbf{1} \textit{Initiale} D  \newline
\line(1,0){75} \newline
\textbf{2} Lischoys Gwelljus] Lishoys Gwellivs D \textbf{29} Logroys] Logroẏs D \textbf{30} liute] lvte D \newline
\end{minipage}
\hspace{0.5cm}
\begin{minipage}[t]{0.5\linewidth}
\small
\begin{center}*m
\end{center}
\begin{tabular}{rl}
 & \dag nie wan niht der\dag  kœme alsus.\\ 
 & Lischois G\textit{w}elli\textit{u}s\\ 
 & het mich sêre geletzet\\ 
 & und h\textit{i}nder daz ros gesetzet\\ 
5 & mit einer juste rîch;\\ 
 & diu ergienc sô h\textit{u}rteclîch\\ 
 & durch mîn schilt und durch den lîp.\\ 
 & dô half mir \textbf{daz} \textbf{guote} wîp\\ 
 & ûf ir \textbf{pferde} an \textbf{ir} stat."\\ 
10 & Gawan er sêre blîben bat.\\ 
 & Gawan sprach, er wolte \textbf{sehen},\\ 
 & wâ im der schade wære geschehen.\\ 
 & "lît Logr\textit{oi}s sô nâhen,\\ 
 & mac ich in dâr vor ergâhen,\\ 
15 & sô muoz er antwürten mir.\\ 
 & ich \textbf{vrâge} in, waz er ræche an dir."\\ 
 & "\textbf{daz} tuo niht", sprach der wunde man,\\ 
 & "der wârheit ich dir jehen kan;\\ 
 & dar engât niht k\textit{i}nde reise.\\ 
20 & \textbf{er} mac wol heizen vreise."\\ 
 & \begin{large}G\end{large}awan die wunden verbant\\ 
 & mit der vrouwen houbtgewant.\\ 
 & er sprach \textit{z}uor wunden se\textit{g}en\\ 
 & \textbf{und} bat got man \textit{und} wîbes pflegen.\\ 
25 & er vant \textbf{al}bluotic i\textit{r} \textit{s}lâ,\\ 
 & als ein hirz wære erschozzen dâ.\\ 
 & daz liez niht irre rîten;\\ 
 & er sach in kurzen zîten\\ 
 & Logrois, die gehêrten.\\ 
30 & vil liute mit lobe \textit{si} êrten.\\ 
\end{tabular}
\scriptsize
\line(1,0){75} \newline
m n o \newline
\line(1,0){75} \newline
\textbf{21} \textit{Initiale} m n  \newline
\line(1,0){75} \newline
\textbf{1} nie] [Ni*]: Niht o  $\cdot$ der] die o \textbf{2} Liscois girellirs m  $\cdot$ Niscois gewellius n  $\cdot$ Liscois gefeller susz o \textbf{4} hinder] hnder m \textbf{6} hurteclîch] hertteklich m (n) (o) \textbf{9} ir stat] diser stat n dise stat o \textbf{10} sêre] \textit{om.} n \textbf{13} Logrois] logrius m lagrois o \textbf{14} dâr] gar o \textbf{17} tuo] dút o \textbf{19} kinde] kunde m (o) \textbf{20} er] Es n o \textbf{23} zuor] fuͯr m  $\cdot$ segen] sehen m o \textbf{24} und wîbes] wibes m \textbf{25} ir slâ] ẏr schar vnd sla m \textbf{26} wære] wart o \textbf{27} liez] liesse n  $\cdot$ niht] in nit n in mit o \textbf{30} si] \textit{om.} m  $\cdot$ êrten] :rten o \newline
\end{minipage}
\end{table}
\newpage
\begin{table}[ht]
\begin{minipage}[t]{0.5\linewidth}
\small
\begin{center}*G
\end{center}
\begin{tabular}{rl}
 & \begin{large}I\end{large}ch\textbf{ne} wând\textit{e} \textit{n}iht, daz ez kœme alsus.\\ 
 & Lishois Gewelljus\\ 
 & hât mich sêre geletzet\\ 
 & unde hinderz ors gesetzet\\ 
5 & mit einer tjoste rîche;\\ 
 & di\textit{u} ergienc sô hurteclîche\\ 
 & durch mînen schilt unde durch den lîp.\\ 
 & dô half mir \textbf{ditze} \textbf{guote} wîp\\ 
 & ûf ir \textbf{pfert} an \textbf{dise} stat."\\ 
10 & Gawanen er sêre belîben bat.\\ 
 & Gawan sprach, er wolde \textbf{sehen},\\ 
 & wâ im der schad\textit{e} \textit{w}ære geschehen.\\ 
 & "lît Logroys sô nâhen,\\ 
 & mac \textit{ich} in dâr vor ergâhen,\\ 
15 & sô muoz er antwürten mir.\\ 
 & ich \textbf{vrâg\textit{e}} in, waz er ræche an dir."\\ 
 & "\textbf{des} \textbf{en}tuo niht", sprach der wunde man,\\ 
 & "der wârheit ich dir jehen kan;\\ 
 & dar engêt niht kinde reise,\\ 
20 & \textbf{ez} mac wol heizen vreise."\\ 
 & Gawan die wunden verbant\\ 
 & mit der vrouwen houbtgewant.\\ 
 & er sprach zer wunden \textbf{wunden} segen;\\ 
 & \textbf{er} bat got man unde wîbes pflegen.\\ 
25 & er vant \textbf{al} bluotic ir slâ,\\ 
 & als ein hirze wære ers\textit{chozz}en dâ.\\ 
 & daz \textbf{en}liez niht irre \textbf{in} rîten;\\ 
 & er sach in kurzen zîten\\ 
 & Logroys, die gehêrten.\\ 
30 & vil liute mit lobe si êrten.\\ 
\end{tabular}
\scriptsize
\line(1,0){75} \newline
G I L M Z Fr22 Fr57 \newline
\line(1,0){75} \newline
\textbf{1} \textit{Initiale} G I L Z  \textbf{17} \textit{Initiale} I  \textbf{23} \textit{Initiale} M  \newline
\line(1,0){75} \newline
\textbf{1} Ichne] Ich L  $\cdot$ wânde niht] wande oͮch niht G  $\cdot$ kœme] kom L  $\cdot$ alsus] sus I \textbf{2} Lishoẏs gwellivs G  $\cdot$ lischoys Gwellius I  $\cdot$ Lýshoýs gwelliuͯs L  $\cdot$ Luscoẏs gwellis M  $\cdot$ Lishois Gwellivs Z \textbf{5} rîche] richen I \textbf{6} diu] die G  $\cdot$ hurteclîche] hurtechlchen I \textbf{7} unde durch] in I \textbf{8} dô] Da M  $\cdot$ ditze] daz L \textbf{9} pfert] pheride I (L) \textbf{10} Gawanen] Gawan I \textbf{11} Do bat er in des verýehen L  $\cdot$ sehen] besen M \textbf{12} schade wære] schade da ware G \textbf{13} Logroys] logrois G (Z) logroýs L \textbf{14} ich] \textit{om.} G  $\cdot$ dâr vor] \textit{om.} Z \textbf{16} vrâge] fraget G \textbf{17} entuo] tuͯ L  $\cdot$ wunde] iunge I \textit{om.} M \textbf{18} jehen] geýehen L sprechen M \textbf{19} dar engêt] dan erget I  $\cdot$ kinde] kindes M \textbf{21} verbant] bant M \textbf{22} houbtgewant] hoch gewant M \textbf{23} zer wunden] dar zuͦ der I zcur M \textbf{24} man] mansz L \textbf{25} ir] \textit{om.} M \textbf{26} erschozzen] erslagin G \textbf{27} enliez] liez I (L) (M) (Fr22)  $\cdot$ niht irre in] in nih irre I (L) (M) (Fr22) \textbf{28} sach] shac I  $\cdot$ in] an Fr22 \textbf{29} Logroys] Logroýs L Logrois M Z Logroẏs Fr22 \textbf{30} liute] \textit{om.} Z \newline
\end{minipage}
\hspace{0.5cm}
\begin{minipage}[t]{0.5\linewidth}
\small
\begin{center}*T
\end{center}
\begin{tabular}{rl}
 & \begin{large}I\end{large}ch wânde niht, daz ez kœme alsus.\\ 
 & Lyschoys Gewellius\\ 
 & hât mich sêre geletzet\\ 
 & unde hinder\textit{z} ors gesetzet\\ 
5 & Mit einer tjost rîche;\\ 
 & diu ergienc sô hurteclîche\\ 
 & durch mînen schilt unde durch den lîp.\\ 
 & dô half mir \textbf{diz} \textbf{getriuwe} wîp\\ 
 & ûf ir \textbf{pferde} an \textbf{dise} stat."\\ 
10 & Gawanen er sêre belîben bat.\\ 
 & Gawan sprach, er wolde \textbf{besehen},\\ 
 & wâ im der schade \textbf{dâ} wære geschehen.\\ 
 & "Liget Logrois sô nâhen,\\ 
 & Mag ich in dâr vor ergâhen,\\ 
15 & Sô muoz er antwürten mir.\\ 
 & ich \textbf{vrâget} in, waz er ræche \textit{an} dir."\\ 
 & "\textbf{de\textit{s}} \textbf{en}tuo niht", sprach der wunde man,\\ 
 & "der wârheit ich dir jehen kan;\\ 
 & dar engêt niht kinde reise,\\ 
20 & \textbf{ez} mac wol heizen vreise."\\ 
 & Gawan die wunden verbant\\ 
 & Mi\textit{t} der vrowen houbtgewant.\\ 
 & er sprach zer wunden \textbf{wunden} segen;\\ 
 & \textbf{er} bat got man unde wîbes pflegen.\\ 
25 & er vant bluotic ir slâ,\\ 
 & als ein hirze wære erschozzen dâ.\\ 
 & daz\textbf{n} liez niht irre \textbf{in} rîten;\\ 
 & Er sach in kurzen zîten\\ 
 & Logrois, die gehêrten.\\ 
30 & vil liute mit lobe si êrten.\\ 
\end{tabular}
\scriptsize
\line(1,0){75} \newline
T U V W O Q R Fr39 Fr40 \newline
\line(1,0){75} \newline
\textbf{1} \textit{Initiale} T U W O Q Fr39 Fr40  \textbf{5} \textit{Majuskel} T  \textbf{13} \textit{Majuskel} T  \textbf{14} \textit{Majuskel} T  \textbf{15} \textit{Majuskel} T  \textbf{21} \textit{Initiale} U V Fr39  \textbf{22} \textit{Majuskel} T  \textbf{28} \textit{Majuskel} T  \textbf{29} \textit{Überschrift:} Hie kvmet Gawan zvͦ logroẏs der herzoͤginne orgolusen burg vnde oͮch zvͦ ir V  \textbf{30} \textit{Überschrift:} Hie vant orgelulen die hertzogin von logrois W  \newline
\line(1,0){75} \newline
\textbf{1} Ich] ÷ch O  $\cdot$ wânde] in wante U enwonde V (Fr39)  $\cdot$ ez] er W  $\cdot$ alsus] suͦs U \textbf{2} Lẏschois gwellivs T  $\cdot$ Lyschoys gewellus U  $\cdot$ [L*]: Lychoys gewellius V  $\cdot$ Lyshoys gewellius W  $\cdot$ Las hoys Gvvellivs O  $\cdot$ Lishoys gwellius Q  $\cdot$ Lychschois gewellius R  $\cdot$ Lishoẏs gewellivs Fr39  $\cdot$ liskois gwellius Fr40 \textbf{4} hinderz] hinders T hinder U \textbf{5} rîche] richer O \textbf{7} unde] \textit{om.} R  $\cdot$ den] minen O \textbf{9} pferde] pfært O [pferd*]: pferd Fr40  $\cdot$ dise] [*]: dise V \textbf{10} Gawanen] Gawan V Gawainen R  $\cdot$ sêre] sere an U \textit{om.} O \textbf{11} Gawan] Gawain R  $\cdot$ besehen] sehen Fr40 \textbf{12} wâ] Wie Q  $\cdot$ dâ] \textit{om.} U V Q  $\cdot$ geschehen] beschehen W (R) \textbf{13} Liget] [L*grys]: Lit V  $\cdot$ Logrois] logroys U V (O) Q lẏgroys Fr39 \textbf{14} in dâr] indert Q \textbf{15} \textit{Versfolge 507.16-15} U V   $\cdot$ Sô] Des U (V) (W) \textbf{16} Was er Reche an dir R  $\cdot$ vrâget in] vragen in U vrage in V (W) (O) (Fr39) frage Q  $\cdot$ an] \textit{om.} T \textbf{17} des] desn T  $\cdot$ entuo] thuͦn W tvͦ O (Fr40)  $\cdot$ wunde] iunge Fr39 \textbf{18} der] Die U V W R \textbf{19} dar] Do U (R) Fr39  $\cdot$ engêt] erget U W Fr39 Fr40 get O \textbf{21} Gawan] Gawain R \textbf{22} Mit] Mir T \textbf{23} wunden segen] denwund segen Q wundes segen R wunde segen Fr40 \textbf{24} Er bat ir got beider pflegen O  $\cdot$ Er bat ir baiden got pflegen R \textbf{25} bluotic] al bluͦtic U (V) (W) (O) (Q) (Fr39) (Fr40) ablutig R  $\cdot$ ir] er U \textbf{27} dazn liez] Daz liez O Das erlies R  $\cdot$ niht irre in] in nit irre U (O) (Fr39) in irre niht V nit ir Ritten R \textbf{29} Logrois] Logroys U (O) Q Fr39 Logeris R [Logrois]: Lagrois Fr40  $\cdot$ gehêrten] geherte U herten R \newline
\end{minipage}
\end{table}
\end{document}
