\documentclass[8pt,a4paper,notitlepage]{article}
\usepackage{fullpage}
\usepackage{ulem}
\usepackage{xltxtra}
\usepackage{datetime}
\renewcommand{\dateseparator}{.}
\dmyyyydate
\usepackage{fancyhdr}
\usepackage{ifthen}
\pagestyle{fancy}
\fancyhf{}
\renewcommand{\headrulewidth}{0pt}
\fancyfoot[L]{\ifthenelse{\value{page}=1}{\today, \currenttime{} Uhr}{}}
\begin{document}
\begin{table}[ht]
\begin{minipage}[t]{0.5\linewidth}
\small
\begin{center}*D
\end{center}
\begin{tabular}{rl}
\textbf{55} & dâ vor gevüeret. \textbf{er} brâht in dar.\\ 
 & er was niht als ein môr gevar.\\ 
 & Der marnære wîse\\ 
 & sprach: "\textbf{ir sult heln} lîse\\ 
5 & \textbf{vor den}, die tragent \textbf{daz swarze vel}.\\ 
 & mîne \textbf{kocken} sint sô snel,\\ 
 & \textbf{sine mugen} \textbf{uns} niht \textbf{genâhen}.\\ 
 & wir sulen von hinnen gâhen."\\ 
 & Sîn golt hiez er ze schiffe tragen.\\ 
10 & nû muoz ich iu von scheiden sagen.\\ 
 & \textbf{die} naht vuor dan der werde man.\\ 
 & daz wart verholne getân.\\ 
 & dô \textbf{er entran} dem wîbe,\\ 
 & dô hete si in ir lîbe\\ 
15 & zwelf wochen \textbf{lebendic ein} kint.\\ 
 & vaste \textbf{ment} in dan der wint.\\ 
 & Diu vrouwe in ir \textbf{biutel} vant\\ 
 & einen brief, den schreib ir mannes hant\\ 
 & \textbf{en} franzoys, daz si kunde.\\ 
20 & diu \textbf{schrift} ir sagen begunde:\\ 
 & "\textit{\begin{large}H\end{large}}ie \textbf{enbiutet} liep ein ander liep.\\ 
 & ich bin dirre verte ein diep.\\ 
 & die \textbf{muose} ich dir durch jâmer steln.\\ 
 & \textbf{vrouwe, ine mac dich} niht verheln:\\ 
25 & wære dîn orden in mîner ê,\\ 
 & sô wære mir immer nâch dir wê\\ 
 & \textbf{unt} hân \textbf{doch} immer nâch dir pîn.\\ 
 & werde unser zweier kindelîn\\ 
 & an dem \textbf{antlütze} einem man gelîch,\\ 
30 & deiswâr, \textbf{der} wirt \textbf{ellens} rîch.\\ 
\end{tabular}
\scriptsize
\line(1,0){75} \newline
D \newline
\line(1,0){75} \newline
\textbf{3} \textit{Majuskel} D  \textbf{9} \textit{Majuskel} D  \textbf{17} \textit{Majuskel} D  \textbf{21} \textit{Initiale} D  \newline
\line(1,0){75} \newline
\textbf{21} Hie] ÷ie D \newline
\end{minipage}
\hspace{0.5cm}
\begin{minipage}[t]{0.5\linewidth}
\small
\begin{center}*m
\end{center}
\begin{tabular}{rl}
 & dâ vor gevüeret. \textbf{er} brâhte in dar.\\ 
 & e\textit{r} was niht als ein môr gevar.\\ 
 & der marnære wîse\\ 
 & sprach: "\textbf{ir sullet ez heln} lîse\\ 
5 & \textbf{vor den}, die tragent \textbf{swarziu velle}.\\ 
 & mîne \textbf{kocken} sint sô snelle,\\ 
 & \textbf{si mugen} \textbf{uns} niht \textbf{gevâhen}.\\ 
 & wir sullen von hinnen gâhen."\\ 
 & sîn golt hiez er zuo schiffe tragen.\\ 
10 & nû muoz ich iu von scheiden sagen.\\ 
 & \textbf{die} naht vuor dan der werde man.\\ 
 & daz wart v\textit{e}rh\textit{o}ln getân.\\ 
 & dô \textbf{entran er} dem wîbe.\\ 
 & dô hette si in ir lîbe\\ 
15 & zwelf w\textit{o}chen \textbf{lebendic ein} kint.\\ 
 & vaste \textbf{wegete} in dan der wint.\\ 
 & \begin{large}D\end{large}iu vrowe in ir \textbf{seckel} vant\\ 
 & einen brief, den schreip ir mannes hant\\ 
 & \textbf{in} franzois, daz si kunde.\\ 
20 & diu \textbf{geschrift} ir sagen begunde:\\ 
 & "hie \textbf{enbiutet} liep ein ander liep.\\ 
 & \textit{i}ch \textit{bin} diser verte ein diep.\\ 
 & die \textbf{muose} ich dir durch jâmer steln.\\ 
 & \textbf{vrowe, \textit{in}e ma\textit{c} \textit{d}ir} niht verheln:\\ 
25 & wære dîn orde\textit{n} in mîner ê,\\ 
 & sô wære mir iemer nâch dir wê\\ 
 & \textbf{und} hân \textbf{doch} iemer nâch dir pîn.\\ 
 & wer\textit{d}e unser zweier kindelîn\\ 
 & an dem \textbf{an\textit{t}litz} einem man gelîch,\\ 
30 & d\textit{az ist} wâr, \textbf{er} wirt \textbf{ellens} rîch.\\ 
\end{tabular}
\scriptsize
\line(1,0){75} \newline
m n o \newline
\line(1,0){75} \newline
\textbf{17} \textit{Initiale} m n o  \newline
\line(1,0){75} \newline
\textbf{2} er] Ee m \textbf{5} vor] Fúr o  $\cdot$ tragent] do tragent n  $\cdot$ swarziu velle] swartz fel n \textbf{6} mîne kocken] Mẏn kocker o  $\cdot$ snelle] snell n \textbf{8} von] \textit{om.} n o \textbf{12} verholn] vor heln m \textbf{13} dô] Die o \textbf{15} wochen] wunchen m  $\cdot$ lebendic] lebende n o \textbf{16} wegete] weget n o  $\cdot$ dan] das o \textbf{19} franzois] franczos m frantzos n o \textbf{20} geschrift] schrifft o \textbf{22} ich bin] Jich m \textbf{23} muose] muͦsz n (o) \textbf{24} ine mac dir] me mag ich dir m ich mag dir n o \textbf{25} orden] orde \textit{nachträglich korrigiert zu:} orden m orde n o \textbf{27} doch] \textit{om.} n \textbf{28} werde] Were m Werdet o \textbf{29} antlitz] antzlit m anczlit o \textbf{30} daz ist] Der m  $\cdot$ wâr] vor o  $\cdot$ ellens rîch] ellentrich n o \newline
\end{minipage}
\end{table}
\newpage
\begin{table}[ht]
\begin{minipage}[t]{0.5\linewidth}
\small
\begin{center}*G
\end{center}
\begin{tabular}{rl}
 & dâ vor gevüeret \textbf{und} brâht in dar.\\ 
 & er was niht als ein môr gevar.\\ 
 & der marnære wîse\\ 
 & sprach: "\textbf{nû helt ez} lîse\\ 
5 & \textbf{vor den}, die tragent \textbf{daz swarze vel}.\\ 
 & mîn \textbf{kocken} sint sô snel,\\ 
 & \textbf{den mac} niht \textbf{genâhen}.\\ 
 & wir sulen von hinnen gâhen."\\ 
 & sîn golt hiez er ze scheffe tragen.\\ 
10 & nû muoz ich iu von scheiden sagen.\\ 
 & \textbf{die} naht vuor dan der werde man.\\ 
 & daz wart verholne getân.\\ 
 & dô \textbf{er entran} dem wîbe,\\ 
 & dô het si in ir lîbe\\ 
15 & zwelf wochen \textbf{lebendic ein} kint.\\ 
 & vaste \textbf{mente} in dan der wint.\\ 
 & \begin{large}D\end{large}iu vrouwe in ir \textbf{biutel} vant\\ 
 & einen brief, den schreip ir mannes hant\\ 
 & \textbf{en} franzois, daz si kunde.\\ 
20 & diu \textbf{schrift} ir sagen begunde:\\ 
 & "hie \textbf{enbiut} liep ein ander liep.\\ 
 & ich bin dirre verte ein diep.\\ 
 & die \textbf{muose} ich dir durch jâmer steln.\\ 
 & \textbf{ichne mac dich es, vrouwe}, niht verheln:\\ 
25 & wære dîn orden in mîner ê,\\ 
 & sô wære mir imer nâch dir wê\\ 
 & \textbf{unde} hân \textbf{sus} imer nâch di\textit{r} pîn.\\ 
 & werde unser zweier kindelîn\\ 
 & anme \textbf{lîbe} einem man gelîch,\\ 
30 & dêswâr, \textbf{der} wirt \textbf{ellens} rîch\\ 
\end{tabular}
\scriptsize
\line(1,0){75} \newline
G I O L M Q R Z Fr21 Fr37 \newline
\line(1,0){75} \newline
\textbf{1} \textit{Initiale} O  \textbf{3} \textit{Initiale} M  \textbf{13} \textit{Initiale} I  \textbf{17} \textit{Initiale} G L  \textbf{21} \textit{Capitulumzeichen} L  \textbf{27} \textit{Überschrift:} Hye ist die awentewr wie von der [morein]: moren Gamuert nûn fur vff das mer Q   $\cdot$ \textit{Initiale} I Q Z Fr21 Fr37   $\cdot$ \textit{Capitulumzeichen} L  \newline
\line(1,0){75} \newline
\textbf{1} dâ] ÷a O  $\cdot$ und] er O L M Q R Z Fr21 Fr37  $\cdot$ dar] [dasz]: dar M \textbf{2} er] Er on M (Z) (Fr21) Eren n\textit{achträglich korrigiert zu: }Er Q \textbf{4} Sprach ir svlt iz (\textit{om.} M ) heln lîse (wise och linse R ) O (L) (M) (Q) (R) (Z) (Fr21) (Fr37) \textbf{5} vor] von I  $\cdot$ die] die da O Z Fr21 die do Q  $\cdot$ daz swarze] die swarczin M swartze Q (Fr21) \textbf{6} kocken] koche L Z (Fr21)  $\cdot$ sint sô] diu sint I \textbf{7} den mac] Sine mvgen vns O (M) (Q) (R) (Z) (Fr21) (Fr37) Sie mvgent vns L  $\cdot$ niht] niht nih I  $\cdot$ genâhen] gnaden Fr21 \textbf{8} gâhen] iaghen Q \textbf{9} scheffe] schiffen Q \textbf{10} iu] \textit{om.} L  $\cdot$ scheiden] saiden I schadenne L \textbf{11} die] Bÿ M  $\cdot$ dan] \textit{om.} Q da Fr37 \textbf{12} verholn daz wart getan I \textbf{13} dô] Da M R Z  $\cdot$ entran] tran Z \textbf{14} dô] Da M Z  $\cdot$ het] hat R  $\cdot$ in ir] [on ir]: yn irin M mir Z \textbf{15} wochen] \textit{om.} Z  $\cdot$ lebendic] lemtich O lebendes L  $\cdot$ ein] \textit{om.} L \textbf{16} mente] traip I nam L ierret R  $\cdot$ dan] \textit{om.} I da R \textbf{17} Diu] Der Fr21  $\cdot$ in ir] im Fr37 \textbf{18} den] \textit{om.} Fr37  $\cdot$ ir] ores M  $\cdot$ mannes] [manne]: mannes Fr21 \textbf{19} en] \textit{om.} I  $\cdot$ Franzois] franzoẏs G fronzoys I franzoys O R Fr37 franciosch M franczosz Q frantzois Z franzeis Fr21  $\cdot$ kunde] wol kuͯnde L \textbf{20} schrift] briff M geschrifftt die R  $\cdot$ sagen] do sagen I  $\cdot$ begunde] konde R \textbf{21} enbiut] on pitet M  $\cdot$ ein] an I einen O \textbf{22} dirre] dir dirr I \textbf{23} muose] muͤs I mvͦz O (L) (M) (Q) (R) (Z) (Fr21) (Fr37)  $\cdot$ durch] vor L \textbf{24} Vrowe ich mauͦg dirs niht verheln L  $\cdot$ ichne] ich I (O) (Q) (Fr21)  $\cdot$ dich es] dich I O Z Fr21 Fr37 dirs L dir M R [dich]: dir* Q \textbf{25} wære] War Fr21  $\cdot$ in mîner] múmer Q in mine Z \textbf{26} mir imer nâch dir] noch dir immer O myr noch dir vmmer M mir [*ch]: nach dir ymmer Q nach dir immer Z (Fr21) \textbf{27} unde] Jch O L M (Q) (R) (Fr21) (Fr37)  $\cdot$ hân] \textit{om.} L  $\cdot$ sus] doch Z  $\cdot$ imer nâch dir] imer nach din G nach dir ymmer Q (Fr21) iamer nach dir R \textbf{28} werde] Wer O \textbf{29} lîbe] antlvze O (L) (M) (Q) (R) (Z) Fr21 (Fr37) \textbf{30} dêswâr] Czwar Q (Z) Des ward R  $\cdot$ ellens] ellent O R eren Q \newline
\end{minipage}
\hspace{0.5cm}
\begin{minipage}[t]{0.5\linewidth}
\small
\begin{center}*T (U)
\end{center}
\begin{tabular}{rl}
 & dâ vor gevüeret \textbf{und} brâht in dar.\\ 
 & er was niht als ein môr gevar.\\ 
 & \textit{d}er marnære wîse\\ 
 & sprac\textit{h}: "\textbf{\textit{n}û h\textit{e}lt ez} lîse,\\ 
5 & \textbf{wan} die tragent \textbf{daz swarze vel}.\\ 
 & mîne \textbf{kiele} sint sô snel,\\ 
 & \textbf{si enmugen} \textbf{uns} niht \textbf{genâhen}.\\ 
 & wir soln von hinnen gâhen."\\ 
 & sîn golt hiez er ze schiffe tragen.\\ 
10 & nû muoz ich iu von scheiden sagen.\\ 
 & \textbf{in der} naht vuor dan der werde man.\\ 
 & daz wart verholn getân.\\ 
 & dô \textbf{er entran} dem wîbe,\\ 
 & dô hatte si in ir lîbe\\ 
15 & zwelf wochen \textbf{ein} \textbf{lebendec} kint.\\ 
 & vaste \textbf{wâte} in dan der wint.\\ 
 & diu vrouwe in ir \textbf{biutel} vant\\ 
 & einen brief, den schreib ir mannes hant\\ 
 & \textbf{in} franzois, daz si kunde.\\ 
20 & diu \textbf{schrift} ir sagen begunde,\\ 
 & hie \textbf{enbüte} liebe ein ander liep.\\ 
 & "ich bin dirre verte ein diep.\\ 
 & die \textbf{muoz} ich dir durch jâmer steln.\\ 
 & \textbf{ich enmac dich es, vrouwe}, niht verheln:\\ 
25 & wære dîn orden in mîner ê,\\ 
 & sô wære mir iemer nâch dir wê.\\ 
 & \textbf{ich} hân \textbf{sus} iemer nâch dir pîn.\\ 
 & wer\textit{de} unser zweier kindelîn\\ 
 & an dem \textbf{antlitze} einem man gelîch,\\ 
30 & deiswâr, \textbf{der} wirt \textbf{alsô} rîch.\\ 
\end{tabular}
\scriptsize
\line(1,0){75} \newline
U V W T \newline
\line(1,0){75} \newline
\textbf{3} \textit{Majuskel} T  \textbf{9} \textit{Majuskel} T  \textbf{17} \textit{Initiale} W   $\cdot$ \textit{Majuskel} T  \textbf{19} \textit{Majuskel} T  \textbf{21} \textit{Majuskel} T  \newline
\line(1,0){75} \newline
\textbf{1} brâht] fuͤrt W \textbf{3} der] Zuͦ der U \textbf{4} sprach nû helt ez] Sprach er nuͦ halt iz U sprach [er]: nv helent es V Sprach ir soͤlt es helen W (T) \textbf{5} wan] [*]: vor den V Vor den W (T)  $\cdot$ tragent] do tragen W  $\cdot$ daz swarze] schwartz W swarzes T \textbf{6} kiele] kocken W kocke T  $\cdot$ snel] nahen schnel W \textbf{7} si enmugen] Sy múget W  $\cdot$ niht] \textit{om.} W  $\cdot$ genâhen] [*]: gevohen V \textbf{10} iu] \textit{om.} W  $\cdot$ scheiden] scheidende V \textbf{11} in der] [*]: Bi der V Dy W (T)  $\cdot$ werde man] wiseman T \textbf{12} wart] waz V \textbf{13} dô] Das V  $\cdot$ er entran] entran er T \textbf{14} dô hatte si] die hete T \textbf{15} zwelf] wol zwelf T  $\cdot$ ein lebendec] ein lebende V ein lebendes W lebende ein T \textbf{16} wâte in] in fuͦrte V fuͦrt in W treip in T \textbf{17} biutel] sekel V \textbf{19} franzois] franzoys U W T Franzoẏs V \textbf{21} enbüte] embútet V (W) (T) \textbf{22} dirre] der W diner T \textbf{23} durch] vor W \textbf{24} enmac] mag W  $\cdot$ dich es] dirs V (W) mich T  $\cdot$ vrouwe] \textit{om.} T  $\cdot$ verheln] geheln T \textbf{27} sus] doch W \textbf{28} werde] Weren U \textbf{29} antlitze] libe T \textbf{30} deiswâr] benamen T  $\cdot$ wirt] \textit{om.} W  $\cdot$ alsô] [ellen*]: ellenz V ellends W (T) \newline
\end{minipage}
\end{table}
\end{document}
