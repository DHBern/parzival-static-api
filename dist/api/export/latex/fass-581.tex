\documentclass[8pt,a4paper,notitlepage]{article}
\usepackage{fullpage}
\usepackage{ulem}
\usepackage{xltxtra}
\usepackage{datetime}
\renewcommand{\dateseparator}{.}
\dmyyyydate
\usepackage{fancyhdr}
\usepackage{ifthen}
\pagestyle{fancy}
\fancyhf{}
\renewcommand{\headrulewidth}{0pt}
\fancyfoot[L]{\ifthenelse{\value{page}=1}{\today, \currenttime{} Uhr}{}}
\begin{document}
\begin{table}[ht]
\begin{minipage}[t]{0.5\linewidth}
\small
\begin{center}*D
\end{center}
\begin{tabular}{rl}
\textbf{581} & der êren rîche und lasters arm\\ 
 & lag \textbf{al} sanfte unt im was warm.\\ 
 & etswenne in doch \textbf{in} slâfe vrôs,\\ 
 & daz er \textbf{hes\textit{ch}ete} unde nôs,\\ 
5 & allez von der \textbf{salben} kraft.\\ 
 & Von vrouwen grôz geselleschaft\\ 
 & giengen ûz, \textbf{die andern} în;\\ 
 & die truogen liehten, werden schîn.\\ 
 & Arnive, diu alte,\\ 
10 & gebôt mit ir gewalte,\\ 
 & daz ir enkeiniu riefe\\ 
 & die wîle der helt dâ sliefe.\\ 
 & si bat ouch den palas\\ 
 & besliezen. swaz dâ rîter was,\\ 
15 & scharjande, burgære,\\ 
 & der nekeiner disiu mære\\ 
 & vries\textit{ch} vor dem andern tage.\\ 
 & dô kom den vrouwen niwe klage.\\ 
 & Sus slief der helt unz an die naht.\\ 
20 & diu küneginne was \textbf{sô} bedâht,\\ 
 & die würze si \textbf{im} ûze\textit{m} munde nam:\\ 
 & er \textbf{erwachte}, trinkens \textbf{in} gezam.\\ 
 & Dô hiez dar tragen diu wîse\\ 
 & trinken unt guote spîse.\\ 
25 & er rihte sich ûf und saz,\\ 
 & mit \textbf{guoten vreuden} er az.\\ 
 & vil manec vrouwe vor im stuont.\\ 
 & im wart nie werder dienst kunt;\\ 
 & \textit{\begin{large}I\end{large}}r dienst mit zühten wart getân.\\ 
30 & dô prüevete mîn hêr Gawan\\ 
\end{tabular}
\scriptsize
\line(1,0){75} \newline
D Fr7 \newline
\line(1,0){75} \newline
\textbf{1} \textit{Initiale} Fr7  \textbf{6} \textit{Majuskel} D  \textbf{19} \textit{Majuskel} D  \textbf{23} \textit{Majuskel} D  \textbf{29} \textit{Initiale} D  \newline
\line(1,0){75} \newline
\textbf{2} unt] wan Fr7 \textbf{3} doch in] docht der Fr7 \textbf{4} heschete] hessete D hvͦstet Fr7 \textbf{11} riefe] rief Fr7 \textbf{15} burgære] burogere Fr7 \textbf{17} vriesch] vries D  $\cdot$ vor] von Fr7 \textbf{21} ûzem] vͦzen D \textbf{29} Ir] ÷R D \textbf{30} prüevete] pruͤuet Fr7  $\cdot$ Gawan] gauͮan Fr7 \newline
\end{minipage}
\hspace{0.5cm}
\begin{minipage}[t]{0.5\linewidth}
\small
\begin{center}*m
\end{center}
\begin{tabular}{rl}
 & der êren rîch und lasters arm,\\ 
 & \textbf{er} lac sanft und im was warm.\\ 
 & etwan in doch \textbf{in} slâfe vrôs,\\ 
 & daz er \textbf{gehset} und nôs,\\ 
5 & allez von der \textbf{salben} kraft.\\ 
 & von vrouwen grôziu geselleschaft\\ 
 & giengen \textbf{dô} ûz \textbf{und} în;\\ 
 & die truogen liehten, werden schîn.\\ 
 & Arn\textit{iv}e, diu alte,\\ 
10 & gebôt mit ir gewalte,\\ 
 & daz ir dekeiniu riefe\\ 
 & die wîle der helt d\textit{â} sliefe.\\ 
 & si bat ouch den palas\\ 
 & besliezen. waz d\textit{â} ritter was,\\ 
15 & sarjande \textbf{oder} burgære,\\ 
 & der deheine\textit{r} \textit{disiu} mære\\ 
 & vr\textit{ie}sch vor dem andern tage.\\ 
 & dô kam den vrouwen niuwiu klage.\\ 
 & sus slief der helt unz an die naht.\\ 
20 & diu künigîn was \textbf{dô} bedâht,\\ 
 & die würz si ûz dem munde nam:\\ 
 & er \textbf{erwachet}, trinkens \textbf{im} gezam.\\ 
 & dô hie dar tragen diu wîse\\ 
 & trinken und guote spîse.\\ 
25 & er riht sich ûf und saz,\\ 
 & mit \textbf{guoten vröuden} er az.\\ 
 & vil manic vrouw\textit{e} vor im stuont.\\ 
 & im wart nie werder dienest kunt;\\ 
 & ir dienst mit zühten wart getân.\\ 
30 & dô brüefete mîn hêr Gawan\\ 
\end{tabular}
\scriptsize
\line(1,0){75} \newline
m n o \newline
\line(1,0){75} \newline
\newline
\line(1,0){75} \newline
\textbf{3} slâfe] dem sloffe n \textbf{4} gehset] gehsset n \textbf{9} Arnive] Arnuͯwe m o Arnuwe n  $\cdot$ alte] alten o \textbf{10} ir] jrm m (n) \textbf{11} dekeiniu] do keine n \textbf{12} dâ] do m n o \textbf{14} dâ] do m n o \textbf{16} deheiner disiu] deheine m deheine dise o \textbf{17} vriesch] Freisch m \textbf{18} den] der o \textbf{20} dô] so n o \textbf{22} erwachet] enwachet o \textbf{23} tragen] [tagen]: tragen m \textbf{27} vrouwe] frouͯwen m \textbf{30} hêr] herre her n \newline
\end{minipage}
\end{table}
\newpage
\begin{table}[ht]
\begin{minipage}[t]{0.5\linewidth}
\small
\begin{center}*G
\end{center}
\begin{tabular}{rl}
 & \begin{large}D\end{large}er êren rîche unde lasters arm\\ 
 & lac \textbf{al}samfte unde im was warm.\\ 
 & etswenne in doch \textbf{in} slâfe vrôs,\\ 
 & daz er \textbf{heschet} unde nôs,\\ 
5 & allez von der \textbf{salben} kraft.\\ 
 & von vrouwen grôz geselleschaft\\ 
 & giengen ûz, \textbf{die anderen} în;\\ 
 & die truogen liehten, werden schîn.\\ 
 & Arnive, diu alte,\\ 
10 & gebôt mit ir gewalte,\\ 
 & daz ir deheiniu riefe\\ 
 & die wîle der helt dâ sliefe.\\ 
 & si bat ouch den palas\\ 
 & besliezen. swaz dâ rîter was,\\ 
15 & sarjande, burgære,\\ 
 & der deheiner disiu mære\\ 
 & vriesch vor dem anderen tage.\\ 
 & dô kom den vrouwen niu\textit{w}iu klage.\\ 
 & sus slief der helt unze an die naht.\\ 
20 & diu künegîn was \textbf{sô} bedâht,\\ 
 & die würze s\textbf{im} ûz dem munde nam.\\ 
 & \textbf{dô} er \textbf{wachete}, trinkens \textbf{in} gezam.\\ 
 & dô hiez dar tragen diu wîse\\ 
 & trinken unde guote spîse.\\ 
25 & er rihte sich ûf unde saz,\\ 
 & mit \textbf{guotem willen} er az.\\ 
 & vil manic vrouwe vor im stuont.\\ 
 & ime wart nie werder dienst kunt;\\ 
 & ir dienst mit zühten wart getân.\\ 
30 & \begin{large}D\end{large}ô prüevet mîn hêrre Gawan\\ 
\end{tabular}
\scriptsize
\line(1,0){75} \newline
G I L M Z Fr19 \newline
\line(1,0){75} \newline
\textbf{1} \textit{Initiale} G I L Z Fr19  \textbf{19} \textit{Initiale} I  \textbf{30} \textit{Initiale} G  \newline
\line(1,0){75} \newline
\textbf{1} unde] des L \textbf{2} Lach alz sanfte im waz och warm L  $\cdot$ alsamfte] sanft I  $\cdot$ unde im was] vnd was im Z \textbf{4} heschet] hehzt I ieschet M \textbf{5} salben] selbin M \textbf{7} giengen] Gienc I \textbf{8} die] Hie M Fr19  $\cdot$ liehten werden] werden lýchten L \textbf{9} Arnive] Arnave G Armia I Ar Nuwe M \textbf{12} dâ] \textit{om.} I \textbf{14} swaz] waz L (M) Z  $\cdot$ was] da waz L \textbf{15} sarjande] Sarianden Z \textbf{16} der] Daz Z  $\cdot$ deheiner] dehanev I \textbf{17} anderen] andrem I \textbf{18} den] der I  $\cdot$ niuwiu] niͮweiv G \textbf{20} sô] \textit{om.} Z  $\cdot$ bedâht] verdaht L \textbf{21} würze] wurzen I \textbf{22} dô er wachete] do erwacht er I Er wachte L Her erwachte M (Z) (Fr19)  $\cdot$ in] im L (M) \textbf{23} \textit{Die Verse 581.23-24 fehlen} I   $\cdot$ dô] Da M \textbf{26} guotem willen] guͦten froͮden L (M) (Z) (Fr19) \textbf{27} vrouwe] \textit{om.} L \textbf{28} wart] enwart Fr19 \textbf{29} zühten] zuͯht L  $\cdot$ wart] \textit{om.} M \textbf{30} Dô] Da M Z  $\cdot$ hêrre Gawan] ergawan M \newline
\end{minipage}
\hspace{0.5cm}
\begin{minipage}[t]{0.5\linewidth}
\small
\begin{center}*T
\end{center}
\begin{tabular}{rl}
 & Der êrenrîche und lasters arm\\ 
 & lac \textbf{als} sanfte und im was warm.\\ 
 & etwenne in doch \textbf{im} slâfe vrôs,\\ 
 & daz er \textbf{heschet} und nôs,\\ 
5 & allez von der \textbf{selben} kraft.\\ 
 & von vrouwen grôz geselleschaft\\ 
 & giengen ûz, \textbf{die andern} în;\\ 
 & die truogen liehten, werden schîn.\\ 
 & Arnyve, diu alte,\\ 
10 & \textbf{diu} gebôt mit ir gewalte,\\ 
 & daz ir deheiniu riefe\\ 
 & die wîle der helt d\textit{â} sliefe.\\ 
 & si bat ouch den palas\\ 
 & besliezen. waz dâ ritter was,\\ 
15 & sarjande, burgære,\\ 
 & der keiner disiu mære\\ 
 & vriesch vor dem andern tage.\\ 
 & dô kam den vrouwen niuwiu klage.\\ 
 & sus slief der helt unz an die naht.\\ 
20 & diu küniginne was \textbf{sô} bedâht,\\ 
 & die würz si \textbf{im} ûz dem munde nam:\\ 
 & er \textbf{erwachte}, trinke\textit{n}s \textbf{in} gezam.\\ 
 & dô hiez dar tragen diu wîse\\ 
 & trinken und guote spîse.\\ 
25 & er rihte sich ûf und saz,\\ 
 & mit \textbf{guoten vreuden} er az.\\ 
 & vil manic vrouwe vor im stuont.\\ 
 & im wart nie werder dienst kunt;\\ 
 & ir dienst mit zühten wart getân.\\ 
30 & dô prüefte mîn hêrre Gawan\\ 
\end{tabular}
\scriptsize
\line(1,0){75} \newline
Q R W V U \newline
\line(1,0){75} \newline
\textbf{1} \textit{Initiale} Q R W V  \textbf{23} \textit{Initiale} W V  \newline
\line(1,0){75} \newline
\textbf{1} \textit{Die Verse 553.1-599.30 fehlen} U   $\cdot$ lasters] der lasters R des lasters V \textbf{2} als sanfte] all sanffte W (V)  $\cdot$ und] \textit{om.} R \textbf{3} im] in V \textbf{4} heschet] hiczget R erheschet W gihzete V \textbf{5} selben] salben R V \textbf{7} giengen ûz] [*]: Gieng eine vz V  $\cdot$ andern] ander R \textbf{9} Arnyve] Arniúe Q Arnẏue R Arnyue W Arniue V \textbf{10} diu] \textit{om.} R W V  $\cdot$ ir] irem Q W \textbf{11} deheiniu] ir keine W \textbf{12} dâ] do Q V W \textit{om.} R \textbf{14} dâ] do W V \textbf{15} Sariande [*g*]: oder burgere V \textbf{16} keiner] keine W  $\cdot$ disiu] dise R \textbf{17} vriesch] Vernem R Wúste V \textbf{18} niuwiu] núwe R \textbf{19} sus] Als Q \textbf{20} sô] also W \textbf{21} würz si im] muͯs im R \textbf{22} trinkens] trinckes Q  $\cdot$ in] Jm R (W) (V)  $\cdot$ gezam] [gelzemen]: gezemen R \textbf{26} vreuden] zv́hten V \textbf{28} wart] \textit{om.} W \newline
\end{minipage}
\end{table}
\end{document}
