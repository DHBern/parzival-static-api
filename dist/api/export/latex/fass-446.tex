\documentclass[8pt,a4paper,notitlepage]{article}
\usepackage{fullpage}
\usepackage{ulem}
\usepackage{xltxtra}
\usepackage{datetime}
\renewcommand{\dateseparator}{.}
\dmyyyydate
\usepackage{fancyhdr}
\usepackage{ifthen}
\pagestyle{fancy}
\fancyhf{}
\renewcommand{\headrulewidth}{0pt}
\fancyfoot[L]{\ifthenelse{\value{page}=1}{\today, \currenttime{} Uhr}{}}
\begin{document}
\begin{table}[ht]
\begin{minipage}[t]{0.5\linewidth}
\small
\begin{center}*D
\end{center}
\begin{tabular}{rl}
\textbf{446} & \begin{large}S\end{large}wer\textbf{z} ruochet vernemen, dem tuon ich kunt,\\ 
 & wie \textbf{im sîn} dinc dâ nâch gestuont.\\ 
 & Desn prüeve ich niht der wochen zal,\\ 
 & über wie lanc \textbf{sider} Parzival\\ 
5 & reit durch âventiure als ê.\\ 
 & eines morgens was ein dünner snê,\\ 
 & iedoch sô \textbf{dicke} \textbf{wol} gesnît,\\ 
 & als der noch vrost den liuten gît.\\ 
 & \textbf{ez} was ûf einem grôzen walt.\\ 
10 & im widergienc ein rîter alt,\\ 
 & des bart al grâ was gevar,\\ 
 & dâ bî sîn vel \textbf{lieht} unde clâr.\\ 
 & die selben varwe truoc sîn wîp.\\ 
 & di\textit{u} bêdiu über blôzen lîp\\ 
15 & truogen grâwe röcke herte\\ 
 & ûf ir \textbf{bîhte} verte.\\ 
 & sîniu kint, zwô juncvrouwen,\\ 
 & die man gerne mohte schouwen,\\ 
 & \textbf{dâ} giengen in der selben wât.\\ 
20 & daz riet \textbf{in} \textbf{kiusches herzen} rât.\\ 
 & si giengen alle barvuoz.\\ 
 & Parzival bôt sînen gruoz\\ 
 & dem grâwen rîter, der dâ gienc,\\ 
 & von des râte er \textbf{sît} gelücke enpfienc.\\ 
25 & \textbf{ez} mohte wol ein hêrre sîn.\\ 
 & dâ liefen vrouwen breckelîn.\\ 
 & mit \textbf{senften siten}, niht ze hêr,\\ 
 & gienc dâ rîter unt knappen mêr\\ 
 & mit zühten ûf der gotes vart,\\ 
30 & genuoge sô \textbf{junc}, gar âne bart.\\ 
\end{tabular}
\scriptsize
\line(1,0){75} \newline
D Fr5 Fr31 \newline
\line(1,0){75} \newline
\textbf{1} \textit{Initiale} D Fr5 Fr31  \textbf{3} \textit{Majuskel} D Fr31  \textbf{5} \textit{Majuskel} Fr31  \newline
\line(1,0){75} \newline
\textbf{1} SWer nv ruͦchit horin war kumit Fr5  $\cdot$ vernemen] hoeren Fr31 \textbf{2} Den Auenture hat vz gifrumit Fr5 \textbf{4} sider] sit Fr5  $\cdot$ Parzival] Parcifal D (Fr5) \textbf{6} was] lac Fr5 \textbf{11} al grâ was] was al gra Fr5 \textbf{12} lieht] linde Fr5 \textbf{14} diu] di D \textbf{16} bîhte] gibite Fr5 \textbf{22} Parzival] Parcifal D Fr5 \textbf{30} gar] \textit{om.} Fr5 \newline
\end{minipage}
\hspace{0.5cm}
\begin{minipage}[t]{0.5\linewidth}
\small
\begin{center}*m
\end{center}
\begin{tabular}{rl}
 & swer ruochet vernemen, dem tuon ich kunt,\\ 
 & wie \textbf{des heldes} dinc dâ nâch gestuont.\\ 
 & \begin{large}D\end{large}es enbrüefe ich niht der wochen zal,\\ 
 & über wie lanc \textbf{sider} Parcifal\\ 
5 & reit durch âventiure als ê.\\ 
 & eines morgens was e\textit{i}n dün\textit{n}er snê,\\ 
 & iedoch sô \textbf{dünne} \textbf{wol} gesnît,\\ 
 & als der noch vrost den liuten gît.\\ 
 & \textbf{ez} was û\textit{f} \textit{e}in\textit{em} grôzen walt.\\ 
10 & ime widergienc ein ritter alt,\\ 
 & des bart algrâ was gevar,\\ 
 & dâ b\textit{î} sîn vel \textbf{linde} und clâr.\\ 
 & die selben varwe truoc sîn wîp.\\ 
 & diu beidiu über blôzen lîp\\ 
15 & truogen grâw\textit{e} röcke herte\\ 
 & ûf ir \textbf{bete} verte.\\ 
 & sîniu kint, zwô juncvrouwen,\\ 
 & die man gerne mohte schouwen,\\ 
 & \textbf{dâ} giengen in der selben wât.\\ 
20 & daz riet \textbf{ir} \textbf{kiusches herzen} rât.\\ 
 & si giengen alle barvuoz.\\ 
 & Parcifal bôt sînen gruoz\\ 
 & dem grâwen ritter, der d\textit{â} gienc,\\ 
 & von des râte er \textbf{sît} gelücke enpfienc.\\ 
25 & \textbf{er} m\textit{o}hte wol ein hêrre sîn.\\ 
 & dâ liefen vrouwen breckelîn.\\ 
 & mit \textbf{senftem site}, niht ze hêr,\\ 
 & gienc d\textit{â} ritter und knappen mêr\\ 
 & mit zühten ûf der gotes vart,\\ 
30 & genuoge s\textit{ô} \textit{g}ar âne bart.\\ 
\end{tabular}
\scriptsize
\line(1,0){75} \newline
m n o \newline
\line(1,0){75} \newline
\textbf{1} \textit{Illustration mit Überschrift:} Wie parcifal dem grawen riter [j*en der ke*t]: justieren m  Also parcifal mit dem grouwen ritter justierte vnd stach n (o)   $\cdot$ \textit{Initiale} n o  \textbf{3} \textit{Initiale} m  \newline
\line(1,0){75} \newline
\textbf{1} swer] WEr n o \textbf{2} dinc] dang o \textbf{3} enbrüefe] prúffe n (o) \textbf{4} über] [Vͯver]: Vͯber m  $\cdot$ sider] sẏ der m sú der n der o \textbf{5} reit] Der reit n \textbf{6} ein dünner] en dinrer m eyn doͯnner n (o) \textbf{7} iedoch] E doch o  $\cdot$ dünne] dannen o \textbf{8} liuten] lúter o \textbf{9} ez was] Er wasse o  $\cdot$ ûf einem] vff ein von ein m vff einen n \textbf{11} des] [A]: Das o  $\cdot$ algrâ was] was algrowe n \textbf{12} bî] bin m \textbf{13} selben] selbe n o \textbf{15} grâwe] growen m (n) \textbf{17} sîniu] Sint o \textbf{18} mohte] moͯchte n \textbf{19} dâ] Do n o \textbf{22} bôt] but o \textbf{23} dâ] do m n o \textbf{24} sît] sich o \textbf{25} er mohte] Er moͯhtte m Es moͯchte n \textbf{26} dâ] Do n o \textbf{27} senftem site] senfften sitten n o  $\cdot$ ze] so o \textbf{28} dâ] do m n o \textbf{30} sô gar] so mag gar m \newline
\end{minipage}
\end{table}
\newpage
\begin{table}[ht]
\begin{minipage}[t]{0.5\linewidth}
\small
\begin{center}*G
\end{center}
\begin{tabular}{rl}
 & \begin{large}S\end{large}we\textit{r} \textit{r}uochet vernemen, dem tuon ich kunt,\\ 
 & wie \textbf{im sîn} dinc dar nâch \textit{ge}stuont.\\ 
 & des enprüeve ich niht der wochen zal,\\ 
 & über wie lanc \textbf{sider} Parzival\\ 
5 & reit durch âventiure als ê.\\ 
 & eines morgens was ei\textit{n} dünner snê,\\ 
 & iedoch sô \textbf{dicke} gesnît,\\ 
 & als der noch vrost den liuten gît.\\ 
 & \textbf{ez} was ûf eine\textit{m} grôzen walt.\\ 
10 & im widergienc ein rîter alt,\\ 
 & des bart al grâ was gevar,\\ 
 & dâ bî sîn vel \textbf{linde} unde clâr.\\ 
 & die selben varwe truoc sîn wîp.\\ 
 & diu beidiu über blôzen lîp\\ 
15 & truogen grâwe röcke herte\\ 
 & ûf ir \textbf{bîhte} verte.\\ 
 & sîniu kint, zwô juncvrouwen,\\ 
 & die man gerne mohte schouwen,\\ 
 & \textbf{dâ} giengen in der selben wât.\\ 
20 & daz riet \textbf{ir} \textbf{kiusches herzen} rât.\\ 
 & si giengen alle barvuoz.\\ 
 & Parzival bôt sînen gruoz\\ 
 & dem grâwen rîter, der dâ gienc,\\ 
 & von des râte er \textbf{sîn} gelück enpfienc.\\ 
25 & \textbf{ez} mohte wol ein hêrre sîn.\\ 
 & dâ liefen vrouwen breckelîn.\\ 
 & mit \textbf{senften siten}, niht ze hêr,\\ 
 & gienc dâ rîter unde knappen mêr\\ 
 & mit zühten ûf der gotes vart,\\ 
30 & genuoge sô \textbf{junc}, gar ân bart.\\ 
\end{tabular}
\scriptsize
\line(1,0){75} \newline
G I O L M Z \newline
\line(1,0){75} \newline
\textbf{1} \textit{Überschrift:} Hie ist aber ein auentevr ergangen von parcifal vmb den gral vnd hat einen ritter er stochen vnd ritet nv aber fvrbaz suchende den gral Z   $\cdot$ \textit{Großinitiale} Z   $\cdot$ \textit{Initiale} G I O L  \textbf{17} \textit{Initiale} I  \textbf{21} \textit{Initiale} M  \newline
\line(1,0){75} \newline
\textbf{1} Swer ruochet] Swer iz roͮchet G ÷wer rvͦcht O WErz rvͯchet L So wersz ruchet M  $\cdot$ vernemen] \textit{om.} M  $\cdot$ tuon] tuͤni I \textbf{2} dinc] dienc M  $\cdot$ nâch] \textit{om.} M  $\cdot$ gestuont] stuͦnt G \textbf{3} \textit{Versfolge 446.4-3} G   $\cdot$ enprüeve] enbruͤe I prvffe Z  $\cdot$ der] die L \textbf{4} über] oder I  $\cdot$ wie] wile O M  $\cdot$ sider] si daz I seit O (M) sýt her L  $\cdot$ Parzival] parziual G parzifal I L M Barcifal O parcifal Z \textbf{6} was ein] was eine G lac ein I ein O  $\cdot$ dünner] dvnne L \textbf{7} dicke] diche wol I (L) (Z) dike was O dicke was s M \textbf{8} der] dar L \textbf{9} einem] einen G  $\cdot$ grôzen] grozem I \textbf{10} alt] balt L alde M Z \textbf{11} des] Der Z  $\cdot$ al grâ was] alz gra waz L was algra M \textbf{12} vel] vil M  $\cdot$ linde] was linde O linden M lieht Z  $\cdot$ unde] [va]: vnde O \textbf{16} ir bîhte] buchte M \textbf{19} dâ] die I (O) (L) (M) \textbf{20} daz] do I  $\cdot$ ir] in O L  $\cdot$ kiusches] chushen I (M) \textbf{22} Parzival] Parzifal I L M Barcifal O Parcifal Z  $\cdot$ bôt] bot in Z \textbf{23} grâwen] grawem I \textbf{24} sîn] sýt L (Z) \textbf{25} ez] Er L \textbf{26} dâ] do I \textbf{27} siten] sietin M \textbf{29} der] des M \textbf{30} gar] vnd I \textit{om.} O L \newline
\end{minipage}
\hspace{0.5cm}
\begin{minipage}[t]{0.5\linewidth}
\small
\begin{center}*T
\end{center}
\begin{tabular}{rl}
 & \begin{large}S\end{large}wer\textbf{z} ruochet vernemen, dem tuon ich kunt,\\ 
 & wie \textbf{im sîn} dinc dar nâch gestuont.\\ 
 & des enprüeve ich niht der wochen zal,\\ 
 & über wie lanc \textbf{daz} Parcifal\\ 
5 & reit durch âventiure als ê.\\ 
 & eines morgens was ein dünner snê,\\ 
 & iedoch sô \textbf{dicke} \textbf{wol} gesnît,\\ 
 & alse der noch \textit{vrost} den liuten gît.\\ 
 & \textbf{daz} was ûf einem grôzen walt.\\ 
10 & im widergienc ein rîter alt,\\ 
 & des bart algrâ was gevar,\\ 
 & dâ bî sîn vel \textbf{linde} unde clâr.\\ 
 & die selben va\textit{r}we truoc sîn wîp.\\ 
 & diu beidiu über blôzen lîp\\ 
15 & truogen grâwe röcke herte\\ 
 & ûf ir \textbf{bite} verte.\\ 
 & Sîniu kint, zwô juncvrouwen,\\ 
 & die man gerne mohte schouwen,\\ 
 & \textbf{die} giengen in der selben wât.\\ 
20 & daz riet \textbf{ir} \textbf{herzen kiuscher} rât,\\ 
 & si\textbf{n} giengen alle barvuoz.\\ 
 & Parcifal bôt sînen gruoz\\ 
 & dem grâwen rîter, der dâ gienc,\\ 
 & von des rât er \textbf{sît} glücke enpfienc.\\ 
25 & \textbf{er} mohte wol ein hêrre sîn.\\ 
 & dâ liefen vrouwen breckelîn.\\ 
 & mit \textbf{senften siten}, niht ze hêr,\\ 
 & gienc dâ rîter unde knappen mêr\\ 
 & mit zühten ûf der gotes vart,\\ 
30 & genuoge sô \textbf{junc}, gar âne bart.\\ 
\end{tabular}
\scriptsize
\line(1,0){75} \newline
T U V W Q R \newline
\line(1,0){75} \newline
\textbf{1} \textit{Überschrift:} Hye kam partzifal zu dem garaem ritter do er die betevart ginck amekarfreitage Q   $\cdot$ \textit{Großinitiale} T U Q R   $\cdot$ \textit{Initiale} V W  \textbf{9} \textit{Überschrift:} Hie kvmet parzifal zvͦ eime rittere der hiez kahenis vnde zvͦ sime wibe vn zvͦ zwein sinre toͤhteren amme heil gen karfritage vnde dar nach zvͦ sime oͤheime trefrezent der ein einsidelle waz vnde ein heilig man V  \textbf{17} \textit{Majuskel} T  \newline
\line(1,0){75} \newline
\textbf{1} Swerz] Wer ez V (W) Wer Q R  $\cdot$ ruochet] wel W  $\cdot$ dem] den Q  $\cdot$ kunt] >kvnt< V \textbf{2} im sîn dinc] [*]: dez heldez ding V  $\cdot$ dar nâch] dannoch Q  $\cdot$ gestuont] stunt W (Q) \textbf{3} des enprüeve] Disen pruͦve U Dez prvͤf V Disem pruͤff W Dor zu pruffe Q Des prise R \textbf{4} daz] sider U (W) (R) [*]: sider  V sey der Q  $\cdot$ Parcifal] parzifal T [*]: parzifal V partzifal W Q parczifal R \textbf{5} ê] \textit{om.} Q \textbf{7} iedoch] Doch Q Je R  $\cdot$ sô dicke] so [d*]: dicke V dicke so R \textbf{8} noch vrost] noch T vrost noch V \textbf{9} daz] Jz U (V) (Q) (R) Er W  $\cdot$ ûf] in V \textbf{10} widergienc] vider gie Q \textbf{11} algrâ was] aller was gra U do was Q \textbf{12} bî] bu Q \textbf{13} selben] selbe V  $\cdot$ varwe] varvwe T fraw Q  $\cdot$ wîp] [*]: wip T [*eip]: :eip Q \textbf{14} diu beidiu] Die beiden U Die [beibe]: beide Q Die beidú R  $\cdot$ über] vber irn U vmbe ir V \textbf{16} bite] bette V beichte W (Q) (R) \textbf{17} Sîniu] Sine R \textbf{19} der] >der< U \textbf{20} [D*]: Daz riet ir [h*]: kv́sches herzen rat V  $\cdot$ ir herzen kiuscher] ir herze kuscher U irs hertzen keúscher W irsz [hewschen]: kewschen hertzen Q Jn kúnsches herczen R \textbf{21} sin giengen] Sie [*]: gingen U Sv́ giengen V (W) (Q) (R) \textbf{22} Parcifal] parzifal T (V) Partzifal W Q Parczifal R  $\cdot$ bôt] bot in V W bat Q \textbf{23} dâ] do U V W Q \textbf{24} rât] raten W  $\cdot$ sît] [*]: sit T \textit{om.} W \textbf{25} er] Es W Q \textbf{26} dâ] Do U V W Q R  $\cdot$ breckelîn] hv́ndelin V \textbf{28} dâ] [*]: do V do W Q der R \textbf{29} der] [de*]: der V \textbf{30} junc] lang R  $\cdot$ gar] \textit{om.} U V R \newline
\end{minipage}
\end{table}
\end{document}
