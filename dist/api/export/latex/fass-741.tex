\documentclass[8pt,a4paper,notitlepage]{article}
\usepackage{fullpage}
\usepackage{ulem}
\usepackage{xltxtra}
\usepackage{datetime}
\renewcommand{\dateseparator}{.}
\dmyyyydate
\usepackage{fancyhdr}
\usepackage{ifthen}
\pagestyle{fancy}
\fancyhf{}
\renewcommand{\headrulewidth}{0pt}
\fancyfoot[L]{\ifthenelse{\value{page}=1}{\today, \currenttime{} Uhr}{}}
\begin{document}
\begin{table}[ht]
\begin{minipage}[t]{0.5\linewidth}
\small
\begin{center}*D
\end{center}
\begin{tabular}{rl}
\textbf{741} & \begin{large}D\end{large}er heiden tet dem getouften wê.\\ 
 & \textbf{des} schilt was holz, \textbf{hiez} aspindê,\\ 
 & daz vûlet noch enbrinnet.\\ 
 & er was von ir geminnet,\\ 
5 & diu \textbf{in im} gap, des sît gewis.\\ 
 & turkoyse, crisoprassis,\\ 
 & smarâde, rubîne,\\ 
 & vil steine mit sunder schîne\\ 
 & wâren \textbf{verwiert} durch kostenlîchen prîs\\ 
10 & \textbf{al umbe} ûf diu \textbf{buckelrîs}.\\ 
 & ûf dem \textbf{buckelhûse} stuont\\ 
 & ein stein, des namen tuon ich iu kunt:\\ 
 & antrax dort genennet,\\ 
 & karfunkel hie bekennet.\\ 
15 & Durch der minne condwier\\ 
 & ecidemôn, daz reine tier,\\ 
 & het im ze wâpene gegeben,\\ 
 & \textbf{in} der genâden er wolde leben,\\ 
 & diu küneginne Secundille.\\ 
20 & diz wâpen was ir wille.\\ 
 & Dâ streit der triwen lûterheit;\\ 
 & grôz triwe aldâ mit triwen streit.\\ 
 & \textbf{durch} minne heten si \textbf{gegeben}\\ 
 & \textbf{mit} kampfe ûf urteil \textbf{bêde ir leben}.\\ 
25 & ieweders hant was sicherbote.\\ 
 & der getoufte wol getrûwete gote,\\ 
 & sît er von Trevrizende schiet,\\ 
 & der im sô herzenlîche riet,\\ 
 & er solte helfe an \textbf{den} gern,\\ 
30 & der in \textbf{sorge} vreude kunde wern.\\ 
\end{tabular}
\scriptsize
\line(1,0){75} \newline
D \newline
\line(1,0){75} \newline
\textbf{1} \textit{Initiale} D  \textbf{15} \textit{Majuskel} D  \textbf{21} \textit{Majuskel} D  \newline
\line(1,0){75} \newline
\textbf{7} smarâde] Smaraide D  $\cdot$ rubîne] Rvbbîne D \textbf{27} Trevrizende] Trevriscende D \newline
\end{minipage}
\hspace{0.5cm}
\begin{minipage}[t]{0.5\linewidth}
\small
\begin{center}*m
\end{center}
\begin{tabular}{rl}
 & der heiden tet dem getouften wê.\\ 
 & \textbf{des} schilt \textit{was} holz, \textbf{\textit{heizet}} aspindê,\\ 
 & daz vûlet noch enbrinnet.\\ 
 & er was von ir geminnet,\\ 
5 & diu \textbf{in im} gap, des sît gewis.\\ 
 & \textit{t}ur\textit{k}oyse \textbf{und} \textit{c}ri\textit{s}oprassis,\\ 
 & smaragte \textbf{und} rubîne,\\ 
 & vi\textit{l} stein mit sunder schîne\\ 
 & wâren \textbf{verwieret} d\textit{ur}ch kostlîch\textit{en} prîs\\ 
10 & \textbf{al umbe} ûf diu \textbf{buckelrîs}.\\ 
 & ûf dem \textbf{buckelhûse} stuont\\ 
 & ein stein, des namen tuon ich iu kunt:\\ 
 & antrax dort genennet,\\ 
 & karfunkel hie bekennet.\\ 
15 & durch der minne condwier\\ 
 & ecidemôn, daz reine tier,\\ 
 & het im ze wâpen gegeben,\\ 
 & \textbf{an} der gnâde er wolte leben,\\ 
 & diu künigîn Secundill\textit{e}.\\ 
20 & diz wâpen was ir wille.\\ 
 & d\textit{â} streit der triuwen lûterkeit;\\ 
 & grôz triuwe aldâ mit triuwen streit.\\ 
 & \textbf{mit} minne heten si \textbf{gegeben}\\ 
 & \textbf{durch} kampf ûf urtei\textit{l} \textbf{\textit{b}eide \textit{ir} leben}.\\ 
25 & ietweders hant was sicherbot.\\ 
 & der getoufte wol getrûwet got,\\ 
 & sît er von Tr\textit{e}vri\textit{z}ende schiet,\\ 
 & der im sô herzeclîchen riet,\\ 
 & er solte helfe an \textbf{in} gern,\\ 
30 & der in \textbf{sorge} vröude kunde wern.\\ 
\end{tabular}
\scriptsize
\line(1,0){75} \newline
m n o V V' Fr69 \newline
\line(1,0){75} \newline
\textbf{1} \textit{Initiale} V  \newline
\line(1,0){75} \newline
\textbf{1} tet dem] [nam]: det dem o \textbf{2} was] hies m (n) o  $\cdot$ heizet] was m n o daz heiszet V'  $\cdot$ aspindê] aspene o \textbf{5} im] vmbe V'  $\cdot$ des] das o \textbf{6} \textit{Die Verse 741.6-18 fehlen} V'   $\cdot$ turkoyse] Gurtoisse m Turtoise n Turteise o Turcoyse V  $\cdot$ crisoprassis] pripopprassis m crisoparis V \textbf{7} smaragte] Smaragtte m Smaragde n o V  $\cdot$ rubîne] robine n ruͯbine o \textbf{8} vil] vie m \textbf{9} verwieret] verwircket n (Fr69)  $\cdot$ durch kostlîchen] doch kostlich m \textbf{10} al] Alle n  $\cdot$ buckelrîs] buckel wisz o \textbf{12} tuon] des duͯn n \textbf{14} karfunkel] Karfunckel m n Karfúnckel o Carfunkel V \textbf{16} \textit{Versdoppelung} m   $\cdot$ ecidemôn] Ezidamon V \textbf{18} an] Jn V  $\cdot$ er] in o \textbf{19} \textit{statt 741.19-20:} Die kvniginne daz edil ris (vgl. 741.10: buckelrîs) / Die secundille waz genant V'   $\cdot$ Secundille] secundillene m secuͯndille o \textbf{21} \textit{Die Verse 741.21-22 fehlen} V'   $\cdot$ dâ] Do m n o V \textbf{23} \textit{statt 741.23-24:} Die mynne hat sie beide gemant V'   $\cdot$ mit] Durch V \textbf{24} durch kampf] Vff kampff o Mit kampfe V  $\cdot$ urteil beide ir leben] vrteil sẏ beide leben m \textbf{25} \textit{Die Verse 741.25-30 fehlen} V'   $\cdot$ sicherbot] sicher gebot o \textbf{26} getrûwet] [getauffet]: getruͯwet o getruwete V \textbf{27} Trevrizende] triuriende m tremrizende n tremanzende o Trefrischente V \textbf{29} in] den n o V \textbf{30} in] imme V \newline
\end{minipage}
\end{table}
\newpage
\begin{table}[ht]
\begin{minipage}[t]{0.5\linewidth}
\small
\begin{center}*G
\end{center}
\begin{tabular}{rl}
 & \begin{large}D\end{large}er heiden tet dem getouften wê.\\ 
 & \textbf{der} schilt was holz, \textbf{hiez} aspindê,\\ 
 & daz \textbf{en}vûlet nochne brinnet.\\ 
 & er was von ir geminnet,\\ 
5 & diu \textbf{in im} gap, des sît gewis.\\ 
 & turkoyse, crisoprassis,\\ 
 & smaragde \textbf{unde} rubîne,\\ 
 & vil steine mit sunder schîne\\ 
 & wâren \textbf{verwieret} durch kostelîchen prîs\\ 
10 & \textbf{ze loben} ûf diu \textbf{buckelrîs}.\\ 
 & ûf dem \textbf{buckelhûse} stuont\\ 
 & ein stein, des namen tuon ich iu kunt:\\ 
 & antrax dort genennet,\\ 
 & karfunkel hie bekennet.\\ 
15 & \textit{durch} der minne condwier\\ 
 & ecidemôn, daz reine tier,\\ 
 & het im ze wâpen gegeben,\\ 
 & \textbf{in} der gnâde er wolde leben,\\ 
 & diu künigîn Secundille.\\ 
20 & ditze wâpen was ir wille.\\ 
 & dâ streit der triwen lûterheit;\\ 
 & grôz triwe aldâ mit triwen streit.\\ 
 & \textbf{durch} minne heten si \textbf{ir leben}\\ 
 & \textbf{mit} kampfe ûf urteil \textbf{gegeben}.\\ 
25 & ietweders hant was sicherbot.\\ 
 & der getoufte wol getrûwet got,\\ 
 & sît er von Trevrizzent schiet,\\ 
 & der im sô herzenlîchen riet,\\ 
 & er solde helfe an \textbf{in} gern,\\ 
30 & der in \textbf{sorgen} vröude kunde wern.\\ 
\end{tabular}
\scriptsize
\line(1,0){75} \newline
G I L M Z Fr24 Fr50 \newline
\line(1,0){75} \newline
\textbf{1} \textit{Initiale} G L Z Fr24  \textbf{11} \textit{Initiale} I  \newline
\line(1,0){75} \newline
\textbf{1} heiden] heide M  $\cdot$ getouften] getouftē G (M) Getauftem I \textbf{2} der] Des M Z Fr24  $\cdot$ hiez] daz I \textbf{3} nochne brinnet] noch en brunet M noch brinnet Fr24 \textbf{6} turkoyse] Tvrkoise G (I) (M) Tuͯrkoys L Tvrkois Z Tvrk::se Fr24  $\cdot$ crisoprassis] chrisoprasis G crisoprasis I chrisoprassis Z \textbf{7} smaragde] smarage G smarade I Smarande M Smareide Z Sma::de Fr24  $\cdot$ rubîne] rubrin M Rvbyne Fr24 \textbf{9} verwieret] vnverwiert L  $\cdot$ kostelîchen] [sto]: chostelichen I koste L richen M hohen Fr50 \textbf{10} ze loben] Alvmbe Z  $\cdot$ diu buckelrîs] daz puchel ris I dy pokil ris M \textbf{11} \textit{Versfolge 741.12-11} Fr50   $\cdot$ dem buckelhûse] der bucchel hohe I \textbf{12} namen] \textit{om.} M  $\cdot$ iu] \textit{om.} L Z \textbf{13} antrax] Antrox G Antrais M \textbf{14} karfunkel] Carfvnchel G karuunchel I Karfunkil M  $\cdot$ bekennet] erchennet Fr50 \textbf{15} durch] \textit{om.} G \textbf{17} ze wâpen] gewappen Fr50 \textbf{18} in] An L (Fr24) (Fr50)  $\cdot$ gnâde] Gnaden I  $\cdot$ er wolde] wold er Fr50 \textbf{19} Secundille] Secuntille I Secvͯndille L \textbf{20} ditze] daz I (L) \textbf{21} der triwen] div reine Fr50 \textbf{22} triwen] truwe M \textbf{25} ietweders] Jetweder L Jclichir M \textbf{26} getrûwet] Getruwete I (M) \textbf{27} Trevrizzent] Treuerescente I Trevrizent L trefrezcent M Trefreszent Fr24 trevrezent Fr50 \textbf{28} sô] \textit{om.} Fr50 \textbf{29} in] ir I M im Z \textbf{30} in sorgen vröude] in freude vnde sorgen I insorge vroide M in vrovde Fr50 \newline
\end{minipage}
\hspace{0.5cm}
\begin{minipage}[t]{0.5\linewidth}
\small
\begin{center}*T
\end{center}
\begin{tabular}{rl}
 & \begin{large}D\end{large}er heide\textit{n} \textit{t}et dem getouften wê.\\ 
 & \textbf{des} schilt was holz \textbf{von} aspindê,\\ 
 & daz vûlet noch \textit{e}nbrinnet.\\ 
 & er was von ir geminnet,\\ 
5 & diu \textbf{im \textit{in}} gap, des sît gewis.\\ 
 & turkoyse, crisop\textit{r}assis,\\ 
 & smarâde \textbf{und} rubîne,\\ 
 & vil steine mit sunder schîne\\ 
 & wâren \textbf{verwirket} durch kostlîchen prîs\\ 
10 & \textbf{alle umbe} ûf diu \textbf{kuppelrîs}.\\ 
 & ûf dem \textbf{kuppelhûse} stuont\\ 
 & ein stein, des namen tuon ich iu kuont:\\ 
 & antra\textit{x} dort genennet,\\ 
 & karfunkel hie bekennet.\\ 
15 & durch der minne cundewier\\ 
 & ecidemôn, daz reine tier,\\ 
 & het im zuo wâpen gegeben,\\ 
 & \textbf{an} der genâden er wolte leben,\\ 
 & diu küneginne Secundille.\\ 
20 & diz wâpen was ir wille.\\ 
 & d\textit{â} streit der triuwen lûterheit;\\ 
 & grôziu triuwe aldâ mit triuwen streit.\\ 
 & \textbf{durch} minne heten si \textbf{ir leben}\\ 
 & \textbf{mit} kampfe ûf urteil \textbf{gegeben}.\\ 
25 & ietweders ha\textit{n}t was sicherbote.\\ 
 & der getoufte wol getrûwete gote,\\ 
 & sît er von Trefrizente schiet,\\ 
 & der im sô herzeclîche riet,\\ 
 & er solte helfe an \textbf{in} gern,\\ 
30 & der in \textbf{sorgen} vreude kunde wern.\\ 
\end{tabular}
\scriptsize
\line(1,0){75} \newline
U W Q R \newline
\line(1,0){75} \newline
\textbf{1} \textit{Initiale} U Q R  \newline
\line(1,0){75} \newline
\textbf{1} heiden tet] heiden den det U \textbf{2} von] hiesz Q (R) \textbf{3} vûlet] enfaulet W (Q) (R)  $\cdot$ enbrinnet] ein brinnet U \textbf{4} er] Es Q \textbf{5} im in] im U (Q) in Im R  $\cdot$ sît] [sis]: sit U \textbf{6} turkoyse] Tuͦrkoyse R  $\cdot$ crisoprassis] chrisopassis U krisopassiß W \textbf{7} smarâde] Smarede U Schmaragde W Smaragde Q \textbf{9} verwirket] verwirt W verworcht Q  $\cdot$ durch] mit Q  $\cdot$ kostlîchen] hofflichen R \textbf{10} alle umbe] Allumbe W (Q) (R)  $\cdot$ diu kuppelrîs] die buckel reiß W des bukels reisz Q dú buckes Ris R \textbf{11} dem kuppelhûse] dem buckel hause W (R) des buckels reisz Q \textbf{12} iu] \textit{om.} W Q  $\cdot$ kuont] och kunt R \textbf{13} antrax] Antrar U Atrax R \textbf{14} karfunkel] Karvuͦnkel U Karfunckel W Q \textbf{16} ecidemôn] Essidemon W Etidemon R \textbf{17} gegeben] geben R \textbf{18} an der genâden] An der gnade W Jnder gnade Q Jn der gnaden R \textbf{19} Secundille] Secuͦndille U secúndille Q \textbf{21} dâ] Do U W Q \textbf{22} triuwen] trewe Q \textbf{23} \textit{Die Verse 741.23-26 fehlen} W  \textbf{24} urteil] vrteilt Q  $\cdot$ gegeben] geben R \textbf{25} hant] hat U \textbf{27} Trefrizente] Treifritente U trefrisente W triffrissen Q trefrizent R \textbf{28} im] nu Q \textbf{29} in] im Q \textbf{30} sorgen] sorge Q \newline
\end{minipage}
\end{table}
\end{document}
