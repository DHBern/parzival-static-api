\documentclass[8pt,a4paper,notitlepage]{article}
\usepackage{fullpage}
\usepackage{ulem}
\usepackage{xltxtra}
\usepackage{datetime}
\renewcommand{\dateseparator}{.}
\dmyyyydate
\usepackage{fancyhdr}
\usepackage{ifthen}
\pagestyle{fancy}
\fancyhf{}
\renewcommand{\headrulewidth}{0pt}
\fancyfoot[L]{\ifthenelse{\value{page}=1}{\today, \currenttime{} Uhr}{}}
\begin{document}
\begin{table}[ht]
\begin{minipage}[t]{0.5\linewidth}
\small
\begin{center}*D
\end{center}
\begin{tabular}{rl}
\textbf{556} & \begin{large}G\end{large}awan sprach: "hie ist niht \textbf{geschehen},\\ 
 & wan \textbf{des} wir \textbf{vor iu wellen} \textbf{jehen}.\\ 
 & ich vrâgte dise magt ein teil;\\ 
 & daz dûhte si \textbf{mîn} unheil\\ 
5 & und bat mich\textbf{z}, daz ichz lieze.\\ 
 & Ob iuch des niht verdrieze,\\ 
 & sô lât mîn dienst umb iuch bejagen,\\ 
 & \textbf{wirt}, daz ir mir\textbf{z} \textbf{ruochet} sagen\\ 
 & umbe \textbf{die} vrouwen \textbf{ob uns} hie.\\ 
10 & ich \textbf{en}\textbf{vriesch} in \textbf{al den} landen nie,\\ 
 & dâ man m\textit{ö}hte schouwen\\ 
 & sô manege \textbf{clâre} \textbf{vrouwen}\\ 
 & mit \textbf{sô liehtem} gebende."\\ 
 & Der wirt want sîne hende;\\ 
15 & \textbf{dô sprach er}: "\textbf{hêrre}, vrâget\textbf{s} niht durch got.\\ 
 & \textbf{hêrre}, \textbf{dâ} ist nôt \textbf{ob aller} nôt."\\ 
 & "Sô muoz ich doch ir kumber klagen",\\ 
 & sprach Gawan. "wirt, ir sult mir sagen:\\ 
 & war umbe ist iu mîn vrâgen leit?"\\ 
20 & "Hêrre, durch iwer manheit -\\ 
 & kunnet ir \textbf{vrâgen} niht \textbf{verbern},\\ 
 & sô welt ir lîhte vürbaz gern.\\ 
 & daz lêret iuch herzen swære\\ 
 & unt machet uns vreuden lære,\\ 
25 & mich unt elliu mîniu kint,\\ 
 & die iu ze dienste \textbf{erborn} sint."\\ 
 & Gawan sprach: "ir sult mir\textbf{z} sagen.\\ 
 & welt aber ir \textbf{mich}\textbf{z} gar verdagen,\\ 
 & daz iwer mære mich vergêt,\\ 
30 & ich vreische \textbf{iedoch} wol, wie ez dâ stêt."\\ 
\end{tabular}
\scriptsize
\line(1,0){75} \newline
D \newline
\line(1,0){75} \newline
\textbf{1} \textit{Initiale} D  \textbf{6} \textit{Majuskel} D  \textbf{14} \textit{Majuskel} D  \textbf{17} \textit{Majuskel} D  \textbf{20} \textit{Majuskel} D  \newline
\line(1,0){75} \newline
\textbf{11} möhte] mohte D \newline
\end{minipage}
\hspace{0.5cm}
\begin{minipage}[t]{0.5\linewidth}
\small
\begin{center}*m
\end{center}
\begin{tabular}{rl}
 & \begin{large}G\end{large}awan sprach: "hie ist niht \textbf{beschehen},\\ 
 & wan \textbf{daz} wir \textbf{vor iu wellen} \textbf{jehen}.\\ 
 & ich vrâgte dise maget ein teil;\\ 
 & daz d\textit{û}hte si \textbf{ein} unheil\\ 
5 & und bat mich, daz ich ez lieze.\\ 
 & ob iuch des niht verdrieze,\\ 
 & sô lât mîn dienst umb iuch bejagen,\\ 
 & \textbf{hêr} \textbf{wirt}, daz ir mir\textbf{z} \textbf{ruochet} sagen\\ 
 & umb \textbf{die} vrouwen \textbf{ob uns} hie.\\ 
10 & ich \textbf{ervr\textit{ie}sch} in \textbf{allen} landen nie,\\ 
 & dô man möhte schouwen\\ 
 & sô manige \textbf{clâre} \textbf{juncvrouwen}\\ 
 & mit \textbf{solichem} gebende."\\ 
 & der wir\textit{t} \textit{w}ant sîn hende;\\ 
15 & \textbf{er sprach}: "vrâgt \textit{niht} durch got.\\ 
 & \textbf{hêrre}, \textbf{d\textit{â}} ist nôt \textbf{ob aller} nôt."\\ 
 & "sô muoz ich doch ir kumber klagen",\\ 
 & sprach Gawan. "wirt, ir solt mir sagen:\\ 
 & war umb i\textit{st} iu mîn vrâgen leit?"\\ 
20 & "hêrre, durch iuwer manheit -\\ 
 & \dag dô\dag  ku\textit{nn}et ir \textbf{vrâgen} niht \textbf{enbern},\\ 
 & sô wellet ir lîht vürbaz gern.\\ 
 & daz lêret iuch herzen swære\\ 
 & und mach\textit{e}t uns vröuden lære,\\ 
25 & mich und alliu mîniu kint,\\ 
 & die iu zuo dienst \textbf{erborn} sint."\\ 
 & Gawan sprach: "ir solt mir\textbf{z} sagen.\\ 
 & welt aber ir \textbf{mir}\textbf{z} gar ver\textit{d}agen,\\ 
 & daz iuwer mære mich vergât,\\ 
30 & ich vreisch \textbf{doch} wol, wie ez d\textit{â} stât."\\ 
\end{tabular}
\scriptsize
\line(1,0){75} \newline
m n o \newline
\line(1,0){75} \newline
\textbf{1} \textit{Initiale} m   $\cdot$ \textit{Capitulumzeichen} n  \newline
\line(1,0){75} \newline
\textbf{1} beschehen] beschicht n \textbf{4} dûhte] dúhtte m  $\cdot$ ein] ein min n mẏn o \textbf{6} des] das m n o \textbf{7} bejagen] veriagen o \textbf{8} ruochet] ruͯchen m (o) \textbf{9} vrouwen] frouwe n \textbf{10} ervriesch] erfreisch m \textbf{11} möhte] mocht o \textbf{14} wirt want] wirt nam vnd wand m \textbf{15} niht] in m \textbf{16} dâ] do m n o \textbf{17} ir] uweren n \textbf{19} ist] ich m \textbf{21} kunnet] koment m o \textbf{24} machet] machent m \textbf{28} verdagen] vertragen m vertage n (o) \textbf{29} mich] úch n \textbf{30} dâ] do m n o \newline
\end{minipage}
\end{table}
\newpage
\begin{table}[ht]
\begin{minipage}[t]{0.5\linewidth}
\small
\begin{center}*G
\end{center}
\begin{tabular}{rl}
 & Gawan sprach: "hie \textit{i}s\textit{t} niht \textbf{geschehen},\\ 
 & wan \textbf{des} wir \textbf{vor iu wellen} \textbf{jehen}.\\ 
 & ich vrâgete dise maget ein teil;\\ 
 & daz dûhte si \textbf{mîn} unheil\\ 
5 & unde bat mich, daz ichz lieze.\\ 
 & ob iuch des niht verdrieze,\\ 
 & sô lât mîn dienst umbe iuch bejagen,\\ 
 & \textbf{wirt}, daz ir mir\textbf{z} \textbf{ruochet} sagen\\ 
 & umbe \textbf{die} vrouwen, \textbf{die} \textbf{obe uns} hie.\\ 
10 & ich \textbf{en}\textbf{gevriesch} in \textbf{allen} landen nie,\\ 
 & dâ man m\textit{ö}hte schouwen\\ 
 & sô manige \textbf{clâre} \textbf{vrouwen}\\ 
 & mit \textbf{sô liehtem} gebende."\\ 
 & der wirt want sîne hende;\\ 
15 & \textbf{dô sprach er}: "\textbf{hêrre}, vrâget \textbf{es} niht durch got.\\ 
 & \textbf{dâ} ist nôt \textbf{al über} nôt."\\ 
 & "sô muoz ich doch ir kumber klagen",\\ 
 & sprach Gawan. "wirt, ir sult mir sagen:\\ 
 & war umbe ist iu mîn vrâge\textit{n} leit?"\\ 
20 & "hêrre, durch iuwer manheit -\\ 
 & kunnet ir \textbf{vrâgen} niht \textbf{verberen},\\ 
 & sô welt ir lîhte vürbaz geren.\\ 
 & daz lêret iuch herzen swære\\ 
 & unde machet uns vröuden lære,\\ 
25 & mich unde elliu mîniu kint,\\ 
 & diu iu ze dienste \textbf{erboren} sint."\\ 
 & Gawan sprach: "ir sult mir\textbf{z} sagen.\\ 
 & welt aber ir \textbf{mich} \textbf{ez} gar verdagen,\\ 
 & daz iuwer mære mich vergêt,\\ 
30 & ich vreische \textbf{iedoch} wol, wie ez dâ stêt."\\ 
\end{tabular}
\scriptsize
\line(1,0){75} \newline
G I L M Z Fr23 Fr62 \newline
\line(1,0){75} \newline
\textbf{1} \textit{Initiale} L Z Fr23 Fr62  \textbf{15} \textit{Initiale} I  \newline
\line(1,0){75} \newline
\textbf{1} hie ist] hies G \textbf{2} des] daz L  $\cdot$ vor] von Fr23 \textbf{4} mîn] ein Fr23 \textbf{5} mich] michz Z \textbf{6} niht verdrieze] bedrieze Fr23 \textbf{7} mîn] mich L (Fr62)  $\cdot$ iuch] mih Fr23 \textbf{8} mirz] mir L M Fr23 \textbf{9} umbe] vmd I  $\cdot$ die obe] ob L (M) Z Fr23 Fr62 \textbf{10} ich engevriesch] me gefraishte ich I Jch friesch M Z (Fr23) (Fr62)  $\cdot$ allen] al den L aldin M allen den Z \textbf{11} möhte] mohte G I (L) (M) Z (Fr23) Fr62 \textbf{12} clâre] shone I (Fr62) \textbf{13} sô liehtem] [solich*em]: solichem L so lichteme M \textbf{15} dô] Da M vnd Fr62  $\cdot$ hêrre] \textit{om.} L M Z Fr62  $\cdot$ vrâget es] fragt sin I vragz osz M \textbf{16} dâ] Herre da L (M) Z wande da Fr62  $\cdot$ al über] vnd vber I ob aller L Z uff aller M uor alle Fr62 \textbf{18} wirt] wert M  $\cdot$ mir] \textit{om.} Fr23 \textbf{19} vrâgen] frage G \textbf{21} vrâgen] vragens I (Z) frage L  $\cdot$ niht] icht M  $\cdot$ verberen] enbern I (Z) \textbf{23} lêret iuch] uh leret Fr62  $\cdot$ herzen] herre Fr62 \textbf{24} machet uns] uns machet Fr62 \textbf{26} erboren] geborn I Z Fr23 \textbf{27} mirz] mir I \textbf{28} ir mich ez] irz mich I mirz L ir misz M ir mirz Z \textbf{30} vreische] verneme M  $\cdot$ iedoch] doch I (Fr23) (Fr62)  $\cdot$ dâ] hie L \newline
\end{minipage}
\hspace{0.5cm}
\begin{minipage}[t]{0.5\linewidth}
\small
\begin{center}*T
\end{center}
\begin{tabular}{rl}
 & Gawan sprach: "hie \textbf{n}ist niht \textbf{geschehen},\\ 
 & wan \textbf{des} wir \textbf{wellen vor iu} \textbf{verjehen}.\\ 
 & ich vrâgete dise maget ein teil;\\ 
 & daz dûhte si \textbf{mîn} unheil\\ 
5 & unde bat mich, daz ichz lieze.\\ 
 & ob iuch des niht verdrieze,\\ 
 & sô lât mîn dienst umbe iuch bejagen,\\ 
 & daz ir mir \textbf{geruochet} sagen\\ 
 & umbe \textbf{dise} vrouwen hie.\\ 
10 & ich\textbf{n} \textbf{veriesch} in \textbf{allen} landen nie,\\ 
 & dô man m\textit{ö}hte schouwen\\ 
 & sô manege \textbf{schœne} \textbf{vrouwen}\\ 
 & mit \textbf{sô liehtem} gebende."\\ 
 & Der wirt want sîne hende;\\ 
15 & \textbf{er sprach}: "\textbf{hêrre}, vrâget niht durch got.\\ 
 & \textbf{diz} ist \textbf{ein} nôt \textbf{vor aller} nôt."\\ 
 & "Sô muoz ich doch ir kumber klagen",\\ 
 & sprach Gawan. "\textbf{hêr} wirt, ir sult mir sagen:\\ 
 & war umbe ist iu mîn vrâgen leit?"\\ 
20 & "Hêrre, durch iuwer manheit -\\ 
 & kunnet ir \textbf{vrâgens} niht \textbf{enbern},\\ 
 & sô welt ir lîhte vürbaz gern.\\ 
 & daz lêret iuch herzen swære\\ 
 & unde machet uns vröuden lære,\\ 
25 & mich unde alliu mîniu kint,\\ 
 & die iu ze dienste \textbf{geborn} sint."\\ 
 & \textit{\begin{large}G\end{large}}awan sprach: "ir sult mir sagen.\\ 
 & welt aber ir \textbf{mich}\textbf{s} gar verdagen,\\ 
 & daz iuwer mære mich vergêt,\\ 
30 & ich vreische \textbf{doch} wol, wiez dâ stêt."\\ 
\end{tabular}
\scriptsize
\line(1,0){75} \newline
T U V W Q R Fr39 Fr40 \newline
\line(1,0){75} \newline
\textbf{1} \textit{Initiale} Q Fr39   $\cdot$ \textit{Capitulumzeichen} R  \textbf{14} \textit{Majuskel} T  \textbf{17} \textit{Majuskel} T  \textbf{20} \textit{Majuskel} T  \textbf{27} \textit{Überschrift:} Hie kvmet gawan zvͦ der burge die do heisset kastel marueile V   $\cdot$ \textit{Initiale} T  \newline
\line(1,0){75} \newline
\textbf{1} \textit{Die Verse 553.1-599.30 fehlen} U   $\cdot$ Gawan] Gawin R  $\cdot$ nist] ist V W Q R is: Fr39  $\cdot$ geschehen] beschehen V \textbf{2} wellen vor iu] vor eúch súllen W vor euch hye wollen Q vor uch wellen R (Fr39)  $\cdot$ verjehen] iehen V W Q (R) Fr39 \textbf{3} vrâgete] frage Q \textbf{4} \textit{Vers 556.4 fehlt} R  \textbf{5} ichz] ich W \textbf{6} iuch] iv T  $\cdot$ verdrieze] verdrusse Q \textbf{7} iuch] iv T \textbf{9} hie] [*e]: obe vnz hie V \textbf{10} ichn veriesch] Jch erfvͦr V Ich enfriesch W Jch enforsch R \textbf{11} dô] Da R  $\cdot$ möhte] mohte T (W) (Q) Fr39 \textbf{12} manege] manchen Q  $\cdot$ schœne] clare V (R) Fr39 klaren Q \textbf{13} liehtem] lichtem Q \textbf{16} vor] ob V W R Fr39 durch Q \textbf{18} sprach Gawan] Gawan sprach W Sprach Gawin R  $\cdot$ hêr] \textit{om.} V W Q R Fr39 Fr40  $\cdot$ wirt] \textit{om.} W  $\cdot$ mir] mirs W \textbf{19} vrâgen] frage R \textbf{21} vrâgens] vragen V (W) (Q) (R) Fr39 Fr40  $\cdot$ enbern] verbern V W Q Fr39 (Fr40) \textbf{23} iuch] iv T  $\cdot$ herzen] herze V \textbf{25} alliu mîniu] alle min R \textbf{27} Gawan] ÷Awan T Gawin R  $\cdot$ ir] r R  $\cdot$ mir] mirs Q R Fr39 mirz Fr40 \textbf{28} ir michs] irs mich V michs W mirs R ir mirs Fr39  $\cdot$ verdagen] vertagen W vertragen R \textbf{30} vreische] vorsche V frage R  $\cdot$ doch wol] doch V R wol doch W es doch wol Q  $\cdot$ dâ] do V Fr39 \textit{om.} W R \newline
\end{minipage}
\end{table}
\end{document}
