\documentclass[8pt,a4paper,notitlepage]{article}
\usepackage{fullpage}
\usepackage{ulem}
\usepackage{xltxtra}
\usepackage{datetime}
\renewcommand{\dateseparator}{.}
\dmyyyydate
\usepackage{fancyhdr}
\usepackage{ifthen}
\pagestyle{fancy}
\fancyhf{}
\renewcommand{\headrulewidth}{0pt}
\fancyfoot[L]{\ifthenelse{\value{page}=1}{\today, \currenttime{} Uhr}{}}
\begin{document}
\begin{table}[ht]
\begin{minipage}[t]{0.5\linewidth}
\small
\begin{center}*D
\end{center}
\begin{tabular}{rl}
\textbf{90} & swenne \textbf{ir denne} \textbf{unbetwungen} sît,\\ 
 & \textbf{mîn dienest gelebt noch} die zît,\\ 
 & daz ir mich z\textbf{einem} vriwende nemt.\\ 
 & ir m\textit{ö}ht \textbf{iuch} \textbf{nû} wol hân verschemt.\\ 
5 & swaz \textbf{halt} mir von iu geschiht,\\ 
 & mich \textbf{enslüege} doch iwer \textbf{swester} niht."\\ 
 & der rede \textbf{si lacheten} überal.\\ 
 & \textbf{dô} wart getrüebet in der schal.\\ 
 & \textit{\begin{large}D\end{large}}en wirt sîn triwe mente,\\ 
10 & daz er sich \textbf{wider} sente,\\ 
 & \textbf{wan} jâmer ist ein scherpfer gart.\\ 
 & ir ieslîcher innen wart,\\ 
 & daz sîn lîp mit kumber ranc\\ 
 & unt al sîn vrœde was ze kranc.\\ 
15 & dô zurende sîner muomen sun.\\ 
 & er sprach: "dû kanst \textbf{unvuoge} tuon."\\ 
 & "Nein, ich muoz bî riwen sîn.\\ 
 & ich sen mich nâch der künegîn.\\ 
 & ich liez ze Patelamunt\\ 
20 & - \textbf{dâ von} \textbf{mir ist} mîn herze wunt -\\ 
 & in reiner art ein süeze wîp.\\ 
 & ir werdiu kiusche \textbf{mir den} lîp\\ 
 & nâch ir minne \textbf{jâmers} mant.\\ 
 & si gap mir \textbf{liute} und lant.\\ 
25 & mich tuot vrô Belakane\\ 
 & manlîcher vreuden âne.\\ 
 & ez ist \textbf{doc\textit{h}} vil manlîch,\\ 
 & swer minnen wankes schamt sich.\\ 
 & Der vrouwen huote mich ûf bant,\\ 
30 & \textbf{daz} ich niht rîterschefte vant.\\ 
\end{tabular}
\scriptsize
\line(1,0){75} \newline
D \newline
\line(1,0){75} \newline
\textbf{9} \textit{Initiale} D  \textbf{17} \textit{Majuskel} D  \textbf{29} \textit{Majuskel} D  \newline
\line(1,0){75} \newline
\textbf{4} möht] moht D \textbf{9} Den] ÷en D \textbf{27} doch] doc D \newline
\end{minipage}
\hspace{0.5cm}
\begin{minipage}[t]{0.5\linewidth}
\small
\begin{center}*m
\end{center}
\begin{tabular}{rl}
 & wenne \textbf{\textit{i}r denne} \textbf{unbetwungen} sît,\\ 
 & \textbf{mîn dienst gelebet noch} die zît,\\ 
 & daz ir mich zuo \textbf{einem} vriunde nemet.\\ 
 & ir möhte\textit{t} \textbf{iuch} \textbf{nû} wol hân verschemet.\\ 
5 & waz \textbf{halt} mir von iu geschiht,\\ 
 & mich \textbf{enslüege} doch iuwer \textbf{swester} niht."\\ 
 & der rede \textbf{lacheten si} überal.\\ 
 & \textbf{dô} wart getrüebet in der schal.\\ 
 & den wirt sîn triuwe mente,\\ 
10 & daz er sich \textbf{wider} s\textit{e}nte,\\ 
 & \textbf{wanne} jâmer ist ein scharfer gart.\\ 
 & i\textit{r} \textit{i}eglîcher innen wart,\\ 
 & daz sîn lîp mit kumber ranc\\ 
 & und alliu sîne vröude was ze kranc.\\ 
15 & dô zurnde sîner muomen suon.\\ 
 & er sprach: "dû kanst \textbf{unvuoge} tuon."\\ 
 & "nein, ich muoz bî riuwe sîn.\\ 
 & ich sene mich nâch der künigîn.\\ 
 & ich liez ze Patelamunt\\ 
20 & - \textbf{dâ von} \textbf{mir ist} mîn herze wunt -\\ 
 & in reiner art ein süeze wîp.\\ 
 & ir werdiu kiusche \textbf{mir den} lîp\\ 
 & nâch ir minne \textbf{jâmers} mant.\\ 
 & si gap mir \textbf{liute} und lant.\\ 
25 & mich tuot vrouwe Belakane\\ 
 & manlîcher vröuden âne.\\ 
 & ez ist \textbf{doch} vil manlîch,\\ 
 & wer minne wanke\textit{s} schamet sich.\\ 
 & der vrouwen huote mich ûf bant,\\ 
30 & \textbf{dô} ich niht ritterschafte vant.\\ 
\end{tabular}
\scriptsize
\line(1,0){75} \newline
m n o \newline
\line(1,0){75} \newline
\newline
\line(1,0){75} \newline
\textbf{1} ir] mir m \textbf{2} dienst] \textit{om.} o \textbf{4} möhtet] moͯchten m morchten o  $\cdot$ nû] \textit{om.} n o  $\cdot$ verschemet] versmehet o \textbf{5} geschiht] beschiht o \textbf{6} enslüege] sluͯge n (o)  $\cdot$ swester] swester suͦn n (o) \textbf{7} rede] reden o \textbf{8} getrüebet] betruͯbet n (o) \textbf{10} sente] sante m \textbf{12} ir ieglîcher] Jr ÿer [ÿglicher]: ÿeglicher m \textbf{15} zurnde] zurne n \textbf{16} unvuoge] vngefuge o \textbf{19} Patelamunt] pathelamunt n \textbf{25} Belakane] belakanne m belakan n o \textbf{28} wankes] wancken m \newline
\end{minipage}
\end{table}
\newpage
\begin{table}[ht]
\begin{minipage}[t]{0.5\linewidth}
\small
\begin{center}*G
\end{center}
\begin{tabular}{rl}
 & swenne \textbf{aber ir} \textbf{unbetwungen} sît,\\ 
 & \textbf{sô gelebet mîn dienst wol} die zît,\\ 
 & daz ir mich ze vriunde nemet.\\ 
 & ir m\textit{ö}ht \textbf{iuch} \textit{\textbf{ê}} \textit{wol} hân verschemet.\\ 
5 & swaz \textbf{halt} mir von iu geschiht,\\ 
 & mich \textbf{slüege} doch iwer \textbf{swester} niht."\\ 
 & der rede \textbf{si lachten} überal,\\ 
 & \textbf{doch} wart getrüebet in der schal.\\ 
 & den wirt sîn triwe mente,\\ 
10 & daz er sich \textbf{sêre} sente.\\ 
 & jâmer ist ein scharfer gart.\\ 
 & ir iegelîcher innen wart,\\ 
 & daz sîn lîp mit kumber ranc\\ 
 & unde al sîn vröude was ze kranc.\\ 
15 & dô zurnde sîner muomen sun.\\ 
 & er sprach: "dû kanst \textbf{unvuoge} tuon."\\ 
 & "nein, ich muoz bî riwen sîn.\\ 
 & ich sene mich nâch der künigîn,\\ 
 & \textbf{die} ich lie ze Patelamunt.\\ 
20 & \textbf{von der} \textbf{ist} mîn herze wunt.\\ 
 & \begin{large}I\end{large}n reiner art ein süeze wîp,\\ 
 & ir werdiu kiusche \textbf{mînen} lîp\\ 
 & nâch ir minne \textbf{jâmers} mant.\\ 
 & si gap mir \textbf{liute} und lant.\\ 
25 & mich tuot vrô Belacane\\ 
 & manlîcher vröuden âne.\\ 
 & ez ist \textbf{iedoch} vil manlîch,\\ 
 & swer min\textit{n}en wankes schamet sich.\\ 
 & der vrouwen huote mich ûf bant,\\ 
30 & \textbf{daz} ich niht rîterschefte vant.\\ 
\end{tabular}
\scriptsize
\line(1,0){75} \newline
G I O L M Q R Z Fr21 Fr48 Fr56 \newline
\line(1,0){75} \newline
\textbf{1} \textit{Initiale} O  \textbf{5} \textit{Initiale} I  \textbf{7} \textit{Initiale} I R Z Fr21 Fr48  \textbf{9} \textit{Initiale} L  \textbf{21} \textit{Initiale} G  \textbf{27} \textit{Initiale} I  \newline
\line(1,0){75} \newline
\textbf{1} swenne] ÷wenne O Wenne L (M) (Q) (R)  $\cdot$ aber] ob Z ab Fr21 \textbf{2} wol] \textit{om.} Fr21 \textbf{3} mich] mich wol I \textbf{4} möht] moht G O (L) (M) (Q) Z Fr21  $\cdot$ ê wol] vorne G nv [moht]: wol O nv wol L (M) (Q) Z (Fr21) wol R  $\cdot$ verschemet] volschemit M \textbf{5} swaz] Waz L (M) (Q) (R) Z  $\cdot$ geschiht] gesiht I beschicht R \textbf{6} mich] Jch M  $\cdot$ slüege] ensluͤg I (M) (Q) (R) (Z) slvͦc Fr21  $\cdot$ doch] \textit{om.} I \textbf{7} \textit{Versfolge 90.8-7} O   $\cdot$ der] ÷er I  $\cdot$ überal] aber al Q \textbf{8} doch] Da Z Fr48  $\cdot$ getrüebet] betrubit M (R)  $\cdot$ in] im O \textbf{9} den] Der Q R  $\cdot$ mente] ment Fr56 \textbf{10} sêre] wider O L M Q R Z (Fr21) Fr48 Fr56  $\cdot$ sente] sent I Fr56 sande Q \textbf{11} jâmer] Wann iamer Q (R) (Z) (Fr48)  $\cdot$ scharfer] sherpher I  $\cdot$ gart] grat Q \textbf{14} al] als Fr21 \textbf{15} \textit{Versfolge 90.16-15} O   $\cdot$ dô] Da Z Fr48  $\cdot$ zurnde] zurnter R zvrnd Fr21 (Fr48) \textbf{16} unvuoge] vngevuͤge I (O) (L) (Fr21)  $\cdot$ tuon] rvn L \textbf{17} bî] mit O  $\cdot$ riwen] truwen L (Q) (Z) \textbf{19} ze] daz O  $\cdot$ Patelamunt] pantelamút Q Betalamvnt Fr21 patelamvͦnt Fr56 \textbf{20} mîn] mir min O Z Fr21 inner mein Q \textbf{21} In] von I Jr Q  $\cdot$ süeze] shuzze I \textbf{22} werdiu] werde I wirdie L werder R  $\cdot$ kiusche] kusse M  $\cdot$ mînen lîp] minne lip R \textbf{23} jâmers] mich Jarmers R \textbf{24} liute] lip I O L M (Q) (R) Fr21 Fr56 \textbf{25} Belacane] bellicân I Belecane L belakane Q Z Fr56 \textbf{26} manlîcher] Manger L  $\cdot$ vröuden] froͯwde R \textbf{27} iedoch] doch L Q  $\cdot$ manlîch] mennich M \textbf{28} swer] Wer L Q R  $\cdot$ minnen wankes] minen wanches G minne wanches O (Fr56) minne wantsch Q  $\cdot$ schamet] schamp Q \textbf{29} vrouwen] frawe Q  $\cdot$ huote] hand R \newline
\end{minipage}
\hspace{0.5cm}
\begin{minipage}[t]{0.5\linewidth}
\small
\begin{center}*T (U)
\end{center}
\begin{tabular}{rl}
 & swenn \textbf{aber ir} \textbf{nû} \textbf{ledic} sî\textit{t},\\ 
 & \textbf{sô \textit{ge}l\textit{e}bet mîn dienst noch} die zît,\\ 
 & daz ir mich zuo vriunde nemet.\\ 
 & ir m\textit{ö}ht \textbf{in} \textbf{nû} wol hân verschemet.\\ 
5 & waz \textbf{joch} mir von iu geschiht,\\ 
 & mich \textbf{ensluoc} doch iuwer \textbf{swestersun} niht."\\ 
 & \begin{large}D\end{large}er rede \textbf{lacheten si} überal,\\ 
 & \textbf{doch} wart getrüebet in der schal.\\ 
 & den wirt sîn triuwe mente,\\ 
10 & daz er sich \textbf{sêre} sente.\\ 
 & jâmer ist ein scharpfer gart.\\ 
 & ir ieclîcher innen wart,\\ 
 & daz sîn lîp mit kumber ranc\\ 
 & und al sîn vröude was ze kranc.\\ 
15 & dô zurnte sîner muomen suon.\\ 
 & er sprach: "dû kanst \textbf{ungevüege} tuon."\\ 
 & "nein, ich muoz bî riuwen sîn.\\ 
 & ich sene mich nâch der künegîn,\\ 
 & \textbf{die} ich liez zuo Patelamunt.\\ 
20 & \textbf{von d\textit{e}r} \textbf{ist mir} mîn herze wunt.\\ 
 & in reiner art ein süeze wîp,\\ 
 & ir werdiu kiusche \textbf{mînen} lîp\\ 
 & nâch ir minne \textbf{jâmer} mant.\\ 
 & si gap mir \textbf{l\textit{î}p} und l\textit{an}t.\\ 
25 & mich tuot vrô Belacane\\ 
 & menlîcher vröuden âne.\\ 
 & ez ist \textbf{iedoch} vil menlîch,\\ 
 & wer minnen wankes schamet sich.\\ 
 & der vrouwen huote mich ûf bant,\\ 
30 & \textbf{daz} ich niht ritterschefte vant.\\ 
\end{tabular}
\scriptsize
\line(1,0){75} \newline
U V W T \newline
\line(1,0){75} \newline
\textbf{4} \textit{Majuskel} T  \textbf{7} \textit{Initiale} U V W   $\cdot$ \textit{Majuskel} T  \textbf{11} \textit{Majuskel} T  \textbf{12} \textit{Majuskel} T  \textbf{15} \textit{Initiale} T  \textbf{17} \textit{Majuskel} T  \textbf{29} \textit{Majuskel} T  \newline
\line(1,0){75} \newline
\textbf{1} swenn aber ir] wenne aber ir V (W) swennir abr T  $\cdot$ nû ledic] vnbetzwungen W (T)  $\cdot$ sît] sin U \textbf{2} \textit{nach Vers 90.2:} Das ich on alles laster sprechen mag / Nun seit mein frúnt hie ist der tag W   $\cdot$ gelebet] lobet U lebt W  $\cdot$ noch] wol T \textbf{3} ir] [er]: *r V  $\cdot$ nemet] múget nemen W \textbf{4} Vnd eúch des nymer dúrffet schemen W  $\cdot$ möht in] mocht in U mohtet îv T \textbf{5} waz joch mir] swas mir ioch V swaz halt mir T  $\cdot$ von] vnd W \textbf{6} ensluoc] sluͦg V (W) (T)  $\cdot$ swestersun] swester V W T \textbf{7} lacheten si] si lacheten T \textbf{8} doch] do T \textbf{9} sîn] ir W \textbf{10} Also er do bescheinde W  $\cdot$ sêre] [*]: wider V vaste T \textbf{11} jâmer] Groß iamer W  $\cdot$ gart] gar T \textbf{12} innen] do inne W \textbf{13} kumber] kummers W \textbf{14} al sîn] als in U daz sin T  $\cdot$ was ze] wart so V was vil W \textbf{15} zurnte] zvrne T \textbf{16} ungevüege] vnfuͦge V W (T) \textbf{17} bî] in V mit T \textbf{19} Patelamunt] Patelamuͦnt U Panthelamunt V \textbf{20} der ist mir] dir ist mir U der ist V der mir ist T \textbf{21} in] Von W  $\cdot$ süeze] raines W \textbf{22} werdiu] wirde W  $\cdot$ kiusche] knútschet W \textbf{23} nâch] Wann mich W  $\cdot$ jâmer] iamers W T \textbf{24} lîp] liep U  $\cdot$ lant] leit U \textbf{25} Belacane] Balacane U Belecane V pelacane W \textbf{26} menlîcher] stêter T  $\cdot$ vröuden] freúde W \textbf{28} swer minne wankes [*]: schamit sich V · Wer minnen schame fleisset sich W · swer nach minnen senet sich T \newline
\end{minipage}
\end{table}
\end{document}
