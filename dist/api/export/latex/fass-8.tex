\documentclass[8pt,a4paper,notitlepage]{article}
\usepackage{fullpage}
\usepackage{ulem}
\usepackage{xltxtra}
\usepackage{datetime}
\renewcommand{\dateseparator}{.}
\dmyyyydate
\usepackage{fancyhdr}
\usepackage{ifthen}
\pagestyle{fancy}
\fancyhf{}
\renewcommand{\headrulewidth}{0pt}
\fancyfoot[L]{\ifthenelse{\value{page}=1}{\today, \currenttime{} Uhr}{}}
\begin{document}
\begin{table}[ht]
\begin{minipage}[t]{0.5\linewidth}
\small
\begin{center}*D
\end{center}
\begin{tabular}{rl}
\textbf{8} & \begin{large}G\end{large}ahmuret sprach aver sân:\\ 
 & "sehzehen knappen ich hân,\\ 
 & \textbf{der} sehse von îser sint.\\ 
 & dar zuo gebt mir vier kint\\ 
5 & \textbf{mit} guoter zuht \textbf{an} hôher art.\\ 
 & vor den wirt nimmer niht gespart,\\ 
 & des ie bejagen mac mîn hant.\\ 
 & ich wil kêren in diu lant.\\ 
 & ich hân \textbf{ouch ê ein teil} gevarn.\\ 
10 & ob mich gelücke wil bewarn,\\ 
 & sô \textbf{erwirbe} ich \textbf{guotes wîbes} gruoz.\\ 
 & ob ich \textbf{ir} dâr nâch dienen muoz\\ 
 & und ob ich des \textbf{wirdec} bin,\\ 
 & sô rætet mir mîn bester sin,\\ 
15 & daz ich\textbf{s} mit rehten triwen pflege.\\ 
 & got wîse mich der sælden wege!\\ 
 & wir vuoren geselleclîche.\\ 
 & dennoch het iwer rîche\\ 
 & unser vater Gandin.\\ 
20 & manegen kumberlîchen pîn\\ 
 & wir bêde dolten umbe liep.\\ 
 & \textbf{ir wâret} ritter unde diep,\\ 
 & ir \textbf{kundet} dienen unde heln.\\ 
 & wan kunde ouch ich nû minne steln!\\ 
25 & owê, \textbf{wan} het ich iwer kunst\\ 
 & und anderhalp \textbf{die} wâren gunst!"\\ 
 & der künec \textbf{siufzete} unde sprach:\\ 
 & "owê, daz ich dich ie gesach,\\ 
 & sît dû mit \textbf{schimpflîchen} siten\\ 
30 & mîn ganzez herze hâst versniten\\ 
\end{tabular}
\scriptsize
\line(1,0){75} \newline
D \newline
\line(1,0){75} \newline
\textbf{1} \textit{Initiale} D  \newline
\line(1,0){75} \newline
\textbf{1} Gahmuret] GAhmvͦret D \newline
\end{minipage}
\hspace{0.5cm}
\begin{minipage}[t]{0.5\linewidth}
\small
\begin{center}*m
\end{center}
\begin{tabular}{rl}
 & Gahmuret sprach aber sân:\\ 
 & "sehzehen knaben ich hân,\\ 
 & \textbf{der} sehs von îser sint.\\ 
 & dar zuo gebt mir vieriu kint\\ 
5 & \textbf{mit} guoter zuht, \textbf{von} hôher art.\\ 
 & vor den wirt niemer niht gespart,\\ 
 & des ie \textit{b}ej\textit{a}gen ma\textit{c} mîn hant.\\ 
 & ich wil kêren in di\textit{u} lant.\\ 
 & ich hân \textbf{ouch ein teil} gevarn.\\ 
10 & ob mich glücke wil bewarn,\\ 
 & sô \textbf{wirbe} ich \textbf{guotes wîbes} gruoz.\\ 
 & ob ich dâr nâch dienen muoz\\ 
 & und ob ich des \textbf{wirdi\textit{c}} bin,\\ 
 & sô rætet mir mîn bester sin,\\ 
15 & daz ich mit rehten triuwen pflege.\\ 
 & got wîse mich der sælden wege!\\ 
 & wir vuoren geselleclîche.\\ 
 & dennoch hete iuwer rîche\\ 
 & unser \textbf{beider} vater Gandin.\\ 
20 & menige kumberlîche pîn\\ 
 & wir beide dolten umb liep.\\ 
 & \textbf{wir wâren} ritter und diep,\\ 
 & ir \textbf{wolte\textit{t}} dienen unde heln.\\ 
 & wanne kunde \textit{ouch ich nû minne} steln!\\ 
25 & owê, \textbf{wenne} het ich iuwerre kunst\\ 
 & und anderhalb \textbf{des} wâren gunst!"\\ 
 & \begin{large}D\end{large}er künic \textbf{siufzte} und sprach:\\ 
 & "owê, daz ich dich ie gesach,\\ 
 & sît dû mit \textbf{semlîchem} siten\\ 
30 & mîn ganzez herz hest versniten\\ 
\end{tabular}
\scriptsize
\line(1,0){75} \newline
m n o W \newline
\line(1,0){75} \newline
\textbf{1} \textit{Initiale} W  \textbf{27} \textit{Initiale} m n o W  \newline
\line(1,0){75} \newline
\textbf{1} Gahmuret] [Ga*muͯret]: Gahmuͯret m Gamiret n Gamúret o GAmuret W \textbf{4} mir] [mit]: mir m \textbf{5} von] vnd W \textbf{7} des] Das n  $\cdot$ ie bejagen] ÿe veigen \textit{nachträglich korrigiert zu:} ÿe Sigen m [j*]: bejagen n  $\cdot$ mac] macht m \textbf{8} diu] din m n mẏnn o \textbf{11} wirbe] wurbe m erwirbe n (o) erwúrb W \textbf{13} ob ich] ouch ob ich n ob ich auch o W  $\cdot$ des] das o  $\cdot$ wirdic] wirdige m wiedig o \textbf{15} rehten] rehter o (W)  $\cdot$ triuwen] trúwe W \textbf{16} sælden] selben n o W \textbf{18} dennoch] Dar noch n (o) (W)  $\cdot$ hete] hat n \textbf{19} Gandin] gaudin W \textbf{21} dolten] dulden n [dotten]: dolten o \textbf{23} woltet] wolten m n o  $\cdot$ heln] helm o \textbf{24} ouch ich nû] ich ouch mÿnne vnd m ich eúch minne W  $\cdot$ steln] stelm o \textbf{25} wenne het] kunde n  $\cdot$ iuwerre] jre m (n) ie o ir W \textbf{26} des] der n o W \textbf{27} siufzte] súfftzet n (o) (W) \textbf{28} gesach] gesacb W \textbf{30} mîn] Arm W  $\cdot$ herz] \textit{om.} o W  $\cdot$ hest] het o \newline
\end{minipage}
\end{table}
\newpage
\begin{table}[ht]
\begin{minipage}[t]{0.5\linewidth}
\small
\begin{center}*G
\end{center}
\begin{tabular}{rl}
 & Gahmuret sprach aber sân:\\ 
 & "sehzehen knappen ich hân.\\ 
 & sehse \textbf{dâr} von îser sint.\\ 
 & dar zuo gebet mir vier kint\\ 
5 & \textbf{an} guoter zuht, \textbf{von} hôher art.\\ 
 & vor den wirt nimer niht gespart,\\ 
 & des ie bejagen mac mîn hant.\\ 
 & ich wil kêrn in diu lant.\\ 
 & ich hân \textbf{ouch ê ein teil} gevaren.\\ 
10 & obe mich gelücke wil bewaren,\\ 
 & sô \textbf{erwirbe} ich \textbf{guotes wîbes} gruoz.\\ 
 & obe ich \textbf{ir} dâr nâch dienen muoz\\ 
 & unde obe ich des \textbf{wirdic} bin,\\ 
 & sô rætet mir mîn bester sin,\\ 
15 & \begin{large}D\end{large}az ich \textbf{es} mit rehten triwen pflege.\\ 
 & got wîse mich der sælden wege!\\ 
 & wir vuoren geselliclîche.\\ 
 & dannoch het iwer rîche\\ 
 & unser vater Gandin.\\ 
20 & \textbf{vil} manigen kumberlîchen pîn\\ 
 & wir bêde dolten umbe liep.\\ 
 & \textbf{ir wâret} rîter und diep,\\ 
 & ir \textbf{kundet} dienen und helen.\\ 
 & wan kunde ouch ich nû minne stelen!\\ 
25 & owê, \textbf{wan} hete ich iwer kunst\\ 
 & unde anderhalp \textbf{die} wâren gunst!"\\ 
 & der künic \textbf{siufte} und sprach:\\ 
 & "owê, daz ich dich ie gesach,\\ 
 & sît dû mit \textbf{schimpflîchen} siten\\ 
30 & mîn ganzez herze hâst versniten\\ 
\end{tabular}
\scriptsize
\line(1,0){75} \newline
G O L M Q Z Fr29 Fr32 \newline
\line(1,0){75} \newline
\textbf{1} \textit{Initiale} O L M Z Fr29 Fr32  \textbf{15} \textit{Initiale} G   $\cdot$ \textit{Versal} Fr32  \textbf{19} \textit{Capitulumzeichen} L  \newline
\line(1,0){75} \newline
\textbf{1} Gahmuret] Gahmvret G (L) ÷Amvret O Gachmuͯret M Gamúer \textit{nachträglich korrigiert zu:} Gamúret Q Gamuret Z Gahmvͦret Fr29 Gamvret Fr32 \textbf{2} sehzehen] Sehtzig L \textbf{3} dâr] die O L M Fr32  $\cdot$ îser] syser M [eẏ*]: eẏserin Q sere Z \textbf{4} mir] \textit{om.} L ir mir Q \textbf{5} an] Mitt Q (Fr32)  $\cdot$ von] an Q Fr32 \textbf{6} gespart] verspart Q (Z) \textbf{7} des] Daz L  $\cdot$ ie] iht L sie M  $\cdot$ mac] \textit{om.} Fr32 \textbf{9} ouch ê ein teil] ein tayl auch êe O ouch ein teil L (M) (Q) ouch ê teil Fr32 \textbf{11} guotes wîbes] guͦter wibe O \textbf{12} ir] \textit{om.} O L M Z Fr29 in Q Fr32 \textbf{13} wirdic] nun wirdig Q \textbf{14} rætet mir] rate mir nun Q  $\cdot$ bester] beste Fr32 \textbf{15} es] sin Z  $\cdot$ rehten triwen] rechter Q trewen Z  $\cdot$ pflege] [phlig]: phleg O \textbf{16} sælden] selben Fr32 \textbf{18} het] e O had M  $\cdot$ iwer] vnser L (Q) \textbf{19} Gandin] kandin O Gaudin Fr29 \textbf{20} manigen] manniger M  $\cdot$ kumberlîchen] kúmerlicher Q \textbf{21} dolten umbe] woren im Q \textbf{22} wâret] vatir M \textbf{23} kundet] kvnden L \textbf{24} ouch ich] ich ouch M ich Q ovch Z  $\cdot$ minne] munne Q \textbf{25} owê] Owi O (M) Awe Q  $\cdot$ wan] vnd Z  $\cdot$ kunst] :::ûst Fr32 \textbf{27} siufte] sevft O (Q) Z \textbf{28} owê] Awe Q \textbf{29} dû] dir M \textbf{30} mîn] Nein \textit{nachträglich korrigiert zu:} Mein Q  $\cdot$ hâst versniten] had versnyten M versnide Q \newline
\end{minipage}
\hspace{0.5cm}
\begin{minipage}[t]{0.5\linewidth}
\small
\begin{center}*T
\end{center}
\begin{tabular}{rl}
 & Gahmuret sprach aber sân:\\ 
 & "sehzehen knappen ich hân,\\ 
 & sehse, \textbf{die} von îser sint.\\ 
 & dar zuo gebt mir vier kint\\ 
5 & \textbf{an} guoter zuht, \textbf{von} hôher art.\\ 
 & vor den wirt niemer niht gespart,\\ 
 & des ie bejagen mac mîn hant.\\ 
 & ich wil kêren in diu lant.\\ 
 & ich hân \textbf{ein teil ouch ê} gevarn.\\ 
10 & ob mich gelücke wil bewarn,\\ 
 & sô \textbf{erwirb} ich \textbf{guoter wîbe} gruoz.\\ 
 & ob ich \textbf{in} dâr nâch dienen muoz\\ 
 & und ob ich des \textbf{wirdic} bin,\\ 
 & sô rætet mir mîn bester sin,\\ 
15 & daz ich\textit{\textbf{s}} mit rehten triuwen pflege.\\ 
 & got wîse mich der sælden wege!\\ 
 & Wir vuoren geselleclîche.\\ 
 & dannoch hete iuwer rîche\\ 
 & unser vater Gandin.\\ 
20 & \textbf{vil} manegen kumberlîchen pîn\\ 
 & wir beide dolten umbe liep.\\ 
 & \textbf{ir wâret} rîter und diep,\\ 
 & ir \textbf{kundet} dienen und heln.\\ 
 & wan kund ouch ich nû minne steln!\\ 
25 & ouwê, \textbf{und} het ich iuwerre kunst\\ 
 & und anderhalp \textbf{die} wâren gunst!"\\ 
 & \begin{large}D\end{large}er künec \textbf{ersûfte} und sprach:\\ 
 & "ouwê, daz ich dich ie gesach,\\ 
 & sît dû mit \textbf{schimpflîchen} siten\\ 
30 & mîn ganzez herze hâst versniten\\ 
\end{tabular}
\scriptsize
\line(1,0){75} \newline
T U V \newline
\line(1,0){75} \newline
\textbf{1} \textit{Initiale} U V   $\cdot$ \textit{Majuskel} T  \textbf{17} \textit{Majuskel} T  \textbf{27} \textit{Initiale} T U V  \newline
\line(1,0){75} \newline
\textbf{1} Gahmuret] Gahmvret T Gahmuͦret U Gamuret V \textbf{2} sehzehen] Sie ziehin U \textbf{3} îser] ysern U \textbf{6} wirt] [wir]: wirt U \textbf{7} bejagen] beiahin U \textbf{8} diu] din U \textbf{11} guoter wîbe] [guͦte*]: guͦts wibes V  $\cdot$ gruoz] groz U \textbf{12} in] ir V \textbf{13} \textit{Versfolge 8.14-13} V  \textbf{15} ichs] ichz T \textbf{19} Gandin] Gaudin U \textbf{22} rîter] viter U \textbf{24} wan kunde [*ne steln]: oͮch ich nv minne steln V \textbf{26} wâren] wore V \textbf{27} ersûfte] ersúftzet V \textbf{29} schimpflîchen] [*]: kintlichen V \newline
\end{minipage}
\end{table}
\end{document}
