\documentclass[8pt,a4paper,notitlepage]{article}
\usepackage{fullpage}
\usepackage{ulem}
\usepackage{xltxtra}
\usepackage{datetime}
\renewcommand{\dateseparator}{.}
\dmyyyydate
\usepackage{fancyhdr}
\usepackage{ifthen}
\pagestyle{fancy}
\fancyhf{}
\renewcommand{\headrulewidth}{0pt}
\fancyfoot[L]{\ifthenelse{\value{page}=1}{\today, \currenttime{} Uhr}{}}
\begin{document}
\begin{table}[ht]
\begin{minipage}[t]{0.5\linewidth}
\small
\begin{center}*D
\end{center}
\begin{tabular}{rl}
\textbf{287} & ze\textbf{m fasân} inz dornach.\\ 
 & \textbf{swem} \textbf{sîn} ze suochen wære gâch,\\ 
 & der vünden \textbf{bî} den schellen;\\ 
 & die kunden lûte hellen.\\ 
5 & \begin{large}S\end{large}us \textbf{vuor} der unbescheiden helt\\ 
 & zuo dem, der minne was verselt.\\ 
 & weder ern sluoc \textbf{dô} noch enstach,\\ 
 & ê er \textbf{widersagen} hin zim sprach.\\ 
 & unversunnen hielt \textbf{dâ} Parzival.\\ 
10 & daz vuogten im diu bluotes mâl\\ 
 & unt ouch diu strenge minne,\\ 
 & diu mir dicke nimt sinne\\ 
 & unt mir daz herze unsanfte regt.\\ 
 & ach, nôt ein wîp an mich legt!\\ 
15 & wil \textbf{si mich} alsus twingen\\ 
 & unt selten hilfe bringen,\\ 
 & ich \textbf{sol} \textbf{sis} underziehen\\ 
 & unt von ir trôste vliehen.\\ 
 & \textbf{Nû} hœret \textbf{ouch} von \textbf{jenen} beiden,\\ 
20 & umb ir komen unt \textbf{umb} ir scheiden.\\ 
 & Segramors sprach alsô:\\ 
 & "ir gebâret, \textbf{hêrre}, als ir sît \textbf{vrô},\\ 
 & daz hie ein künec mit vo\textit{l}ke ligt.\\ 
 & swie unhôhe i\textit{u} daz wigt,\\ 
25 & ir müezet \textbf{im} \textbf{drumbe wandel} geben,\\ 
 & oder ich verliuse mîn leben.\\ 
 & ir sît \textbf{ûf} strît ze nâhe geriten.\\ 
 & \textbf{doch wil ich iuch durch zuht} biten:\\ 
 & \textbf{ergebt} iuch in mîne gewalt\\ 
30 & oder ir sît schiere von mir \textbf{bezalt},\\ 
\end{tabular}
\scriptsize
\line(1,0){75} \newline
D \newline
\line(1,0){75} \newline
\textbf{5} \textit{Initiale} D  \textbf{19} \textit{Majuskel} D  \newline
\line(1,0){75} \newline
\textbf{23} volke] vorche \textit{nachträglich korrigiert zu:} vorchte D \textbf{24} iu] ivch D \newline
\end{minipage}
\hspace{0.5cm}
\begin{minipage}[t]{0.5\linewidth}
\small
\begin{center}*m
\end{center}
\begin{tabular}{rl}
 & ze\textbf{m fas\textit{â}ne} inz d\textit{o}rnach.\\ 
 & \textbf{wem} \textbf{sîn} ze suochene wære gâch,\\ 
 & der vünde in \textbf{bî} den schellen;\\ 
 & die kunden lûte hellen.\\ 
5 & sus \textbf{vuor} der unbescheiden helt\\ 
 & zuo dem, der minne was verselt.\\ 
 & weder \textit{er} ensluoc \textbf{dô} noch enstach,\\ 
 & ê er \textbf{widersage} hin zuo ime sprach.\\ 
 & unversunnen hielt \textbf{dô} Parcifal.\\ 
10 & daz vuogeten ime diu bluotes mâl\\ 
 & und ouch diu strenge minne,\\ 
 & diu mir dicke nimt \textbf{die} sinne\\ 
 & und mir daz herze unsanfte reget.\\ 
 & \textit{a}ch, nôt ein wîp an mich leget!\\ 
15 & wil \textbf{mich diu} alsus twingen\\ 
 & und s\textit{e}lten helfe bringen,\\ 
 & ich \textbf{sol} \textbf{sis} underziehen\\ 
 & und von ir trôste vliehen.\\ 
 & \textbf{nû} hœret \textbf{ouch} \textit{von} \textbf{in} beiden,\\ 
20 & umb ir komen und \textbf{umb} ir scheid\textit{en}.\\ 
 & \textit{\begin{large}S\end{large}}egramors sprach alsô:\\ 
 & "ir gebâret, \textbf{hêrre}, als ir sît \textbf{vrô},\\ 
 & daz hie ein künic mit volke liget.\\ 
 & wie unhôhe iu daz wiget,\\ 
25 & ir müezet \textbf{in} \textbf{drumbe wandel} ge\textit{b}en,\\ 
 & oder ich verliuse mîn leben.\\ 
 & ir sît \textbf{ûf} strît ze nâhe geriten.\\ 
 & \textbf{doch wil ich iuch durch zuht des} biten:\\ 
 & \textbf{ergebet} iuch in mî\textit{n}e gewalt\\ 
30 & oder ir sît schiere von mir \textbf{bezalt},\\ 
\end{tabular}
\scriptsize
\line(1,0){75} \newline
m n o Fr8 \newline
\line(1,0){75} \newline
\textbf{5} \textit{Versal} Fr8  \textbf{9} \textit{Versal} Fr8  \textbf{21} \textit{Initiale} m Fr8   $\cdot$ \textit{Capitulumzeichen} n  \newline
\line(1,0){75} \newline
\textbf{1} zem] Zeinem Fr8  $\cdot$ fasâne inz dornach] fasene ins darnach m fasanis darnoch n fasanis darnach o phaẏsan in dornach Fr8 \textbf{2} wem] Sweme Fr8 \textbf{3} vünde] vinde n funden o  $\cdot$ bî] wol bi Fr8 \textbf{5} sus vuor] Sie vor o  $\cdot$ unbescheiden] vnverzaget n  $\cdot$ helt] heilt n [hielt]: helt Fr8 \textbf{6} der] denn der n  $\cdot$ minne] minnen Fr8  $\cdot$ verselt] verseilt n \textbf{7} er] \textit{om.} m n o  $\cdot$ dô] da o \textit{om.} Fr8 \textbf{8} widersage] wider sagen n (Fr8)  $\cdot$ hin] \textit{om.} n o  $\cdot$ sprach] gesprach n o \textbf{9} hielt] [czielt]: hielt o  $\cdot$ dô] da Fr8  $\cdot$ Parcifal] Partzẏfal Fr8 \textbf{10} daz] Da Fr8 \textbf{12} nimt die] mine Fr8 \textbf{13} reget] weget n o \textbf{14} ach] Jch m Jn n o  $\cdot$ mich] mir Fr8 \textbf{15} mich diu] siv mich Fr8 \textbf{16} selten] solten m o \textbf{17} sis] michs n (o) \textbf{19} ouch von] ouch m von n o Fr8 \textbf{20} und umb] vnd n o Fr8  $\cdot$ scheiden] scheid m \textbf{21} Segramors] Gegramors m Gegramurs n Gegramúrs o SEgremors Fr8 \textbf{24} wie] Swe Fr8  $\cdot$ daz] das das o \textbf{25} müezet] mussent m muͯst o  $\cdot$ in] ẏm o (Fr8)  $\cdot$ geben] gegen m \textbf{26} verliuse] verlúre n verliese hie Fr8 \textbf{27} strît] stritte m  $\cdot$ ze] so n o \textbf{28} Durch zucht wil ich vh des biten Fr8  $\cdot$ iuch] \textit{om.} n o  $\cdot$ des] das o \textbf{29} [Gebe]: Gebt vch her an mine gewalt Fr8  $\cdot$ mîne] minne m \textbf{30} schiere] \textit{om.} Fr8 \newline
\end{minipage}
\end{table}
\newpage
\begin{table}[ht]
\begin{minipage}[t]{0.5\linewidth}
\small
\begin{center}*G
\end{center}
\begin{tabular}{rl}
 & ze\textbf{n fasânen} in daz dornach.\\ 
 & \textbf{swem} ze suochene wære gâch,\\ 
 & der vünde in \textbf{von} den schellen;\\ 
 & die kunden lûte hellen.\\ 
5 & sus \textbf{vuor} der unbescheidene helt\\ 
 & ze dem, der minn\textit{e} was verselt.\\ 
 & weder ern sluoc noch enstach,\\ 
 & ê er \textbf{widersagen} hin ze im \textit{s}prach.\\ 
 & unversunnen hielt \textbf{dô} Parzival.\\ 
10 & daz vuogten im diu bluotes mâl\\ 
 & unde ouch diu strenge minne,\\ 
 & diu mir dicke nimet \textbf{die} sinne\\ 
 & unde mir daz herze unsanfte reget.\\ 
 & ach, nôt ein wîp an mich leget!\\ 
15 & wil \textbf{si mich} alsô twingen\\ 
 & unde selten helfe bringen,\\ 
 & ich \textbf{muoz} \textbf{sis} underziehen\\ 
 & unde von ir trôste vliehen.\\ 
 & hœret von \textbf{jenen} beiden,\\ 
20 & umbe ir komen unde \textbf{umbe} ir scheiden.\\ 
 & Segremors sprach alsô:\\ 
 & "ir gebâret \textbf{reht}, als ir sît \textbf{unvrô},\\ 
 & daz hie ein künic mit volke liget.\\ 
 & swie unhôhe iu daz wiget,\\ 
25 & ir müezet \textbf{im} \textbf{wandel drumbe} geben,\\ 
 & oder ich verliuse mîn leben.\\ 
 & ir sît \textbf{durch} strît ze nâhen geriten.\\ 
 & \textbf{nû wil ich iuch durch zuht des} biten,\\ 
 & \textbf{ir gebet} iuch in mîne gewalt,\\ 
30 & oder ir sît schiere von mir \textbf{bezalt},\\ 
\end{tabular}
\scriptsize
\line(1,0){75} \newline
G I O L M Q R Z Fr40 \newline
\line(1,0){75} \newline
\textbf{5} \textit{Initiale} I O Q R Z Fr40   $\cdot$ \textit{Capitulumzeichen} L  \textbf{21} \textit{Initiale} I  \newline
\line(1,0){75} \newline
\textbf{1} zen fasânen] nach fasan I Zem vashan O Zuͯ dem fasan L (Z) Zcu eynem vasan M Zem fasian Q Zem fasach R Sein fasan Fr40  $\cdot$ daz dornach] dar noch M die tornach R \textbf{2} swem] swem nach in I Swem sin O Z Wem sin L (M) R Wann sein Q  $\cdot$ wære] wart I  $\cdot$ gâch] iach Z \textbf{3} von] bi Z  $\cdot$ den] der Q \textbf{4} hellen] bellen R \textbf{5} sus] ÷vs O  $\cdot$ unbescheidene] vnbekande L \textbf{6} minne] minnen G im R  $\cdot$ verselt] verhelt R \textbf{7} enstach] stach R in stach Fr40 \textbf{8} ê] \textit{om.} M [er]: êe Fr40  $\cdot$ widersagen] wider sachen Q  $\cdot$ hin] \textit{om.} O  $\cdot$ sprach] gesprach G \textbf{9} unversunnen] Vnuersunden Q Vnuersinen R  $\cdot$ dô] \textit{om.} O L M R da Z  $\cdot$ Parzival] parzifal I M Fr40 Parcifal O (L) (Z) partzifal Q parczifal R \textbf{10} daz] do Fr40  $\cdot$ vuogten] vuͤgt I fvͦgtem O fuͯchten L (Fr40) suchten Q fuget Z  $\cdot$ diu] des I di Fr40 \textbf{11} strenge] strengen M \textbf{12} diu] di Fr40  $\cdot$ dicke nimet] dick R nimt dicke Z  $\cdot$ die] mine R  $\cdot$ sinne] sinde Q \textbf{13} mir] die mir diche L dy mir M (Q) (R) (Fr40)  $\cdot$ reget] negt R \textbf{14} ach] vil I  $\cdot$ nôt ein wîp] not an wib I ein wip noch O ein wip not L noch ein weip Q \textbf{15} alsô] alsvs O (L) (M) (Q) (R) (Z) (Fr40) \textbf{16} selten] selde Q  $\cdot$ helfe] helffen Q \textbf{17} muoz] sol Z  $\cdot$ sis] sie dez L irs R siz Fr40  $\cdot$ underziehen] wider ziehen O (M) \textbf{18} unde] \textit{om.} O Vnd ver R  $\cdot$ trôste] \textit{om.} R \textbf{19} hœret] hort auch I Horet nv O Hort tuͯch L Horit ioch M Nun hort auch Q (Z) (Fr40) Nun merkt och R  $\cdot$ jenen] en M inen R \textbf{20} umbe ir] Jr R  $\cdot$ unde umbe] vnde M (R) \textbf{21} Segremors] Saýgremors L Sigremros M Seigremors R  $\cdot$ sprach] der sprach Z \textbf{22} gebâret] bewart Q  $\cdot$ reht] herre O L (M) Q (R) Z Fr40  $\cdot$ als ir] alse irs M als ir nit R sam ir Z  $\cdot$ unvrô] vro L (M) (Q) (R) (Z) \textbf{24} swie] Wie L (M) Q R  $\cdot$ unhôhe] vnhoen Q  $\cdot$ iu] ir R  $\cdot$ wiget] vicht Q \textbf{25} im] ein Z \textbf{26} verliuse] verliese L verlise Fr40  $\cdot$ mîn] dasz Q (Fr40) \textbf{27} ze] so Q R  $\cdot$ nâhen] nahe L M Q (R) \textbf{28} nû] Doch Q R Z (Fr40)  $\cdot$ iuch] \textit{om.} Q  $\cdot$ des] \textit{om.} L R Z  $\cdot$ biten] beten M \textbf{29} ir gebet] ergept I (L) (M) (R) (Fr40) Er geit Q  $\cdot$ in] an R  $\cdot$ mîne] meinen Fr40 \textbf{30} ir] \textit{om.} Q \newline
\end{minipage}
\hspace{0.5cm}
\begin{minipage}[t]{0.5\linewidth}
\small
\begin{center}*T
\end{center}
\begin{tabular}{rl}
 & ze \textbf{dem fasân} in daz dornach.\\ 
 & \textbf{dem} \textbf{sîn} ze suochene wære gâch,\\ 
 & der vündin \textbf{von} den schellen;\\ 
 & die kunden lûte hellen.\\ 
5 & \begin{large}S\end{large}us \textbf{kom} der unbescheiden helt\\ 
 & ze dem, der \textbf{mit der} minne was verselt.\\ 
 & weder er ensluoc noch enstach,\\ 
 & ê er \textbf{widersagen} hin zim sprach.\\ 
 & Unversunnen hielt Parcifal.\\ 
10 & daz vuogten im diu bluotes mâl\\ 
 & unde ouch diu strenge minne,\\ 
 & di\textit{u} mir dicke nimt \textbf{die} sinne\\ 
 & unde mir daz herze unsanfte reget.\\ 
 & ach, nôt ein wîp an mich leget!\\ 
15 & wil \textbf{si mich} alsus twingen\\ 
 & unde selten helfe bringen,\\ 
 & ich \textbf{sol}\textbf{s ir} underziehen\\ 
 & unde von ir trôste vliehen.\\ 
 & Hœret \textbf{ouch} von \textbf{jenen} beiden,\\ 
20 & umbir komen unde ir scheiden.\\ 
 & Segremors, \textbf{der} sprach alsô:\\ 
 & "ir gebâret \textbf{rehte}, als ir sît \textbf{vrô},\\ 
 & daz hie ein künec mit volke liget.\\ 
 & swie unhôhe iu daz wiget,\\ 
25 & ir müezet \textbf{im} \textbf{wandel drumbe} geben,\\ 
 & oder ich verliuse mîn leben.\\ 
 & ir sît \textbf{durch} strît ze nâhe geriten.\\ 
 & \textbf{ich wil durch zuht iuch de\textit{s}} biten:\\ 
 & \textbf{ergebet} iuch in mîne gewalt\\ 
30 & oder ir sît schiere von mir \textbf{gevalt},\\ 
\end{tabular}
\scriptsize
\line(1,0){75} \newline
T U V W \newline
\line(1,0){75} \newline
\textbf{5} \textit{Initiale} T U W  \textbf{9} \textit{Majuskel} T  \textbf{19} \textit{Majuskel} T  \newline
\line(1,0){75} \newline
\textbf{1} ze dem] Zeime V  $\cdot$ fasân] casan U \textbf{2} wære] was W \textbf{3} der] Er W  $\cdot$ vündin] wunden U  $\cdot$ von] wol von W \textbf{6} der mit der] der mit V der W  $\cdot$ verselt] [*]: verhelt U \textbf{7} ensluoc] sluͦc U (V) (W) \textbf{8} er] [*]: er ein V  $\cdot$ hin] \textit{om.} W  $\cdot$ sprach] [*]: gesprach V \textbf{9} hielt] hielt der iunge W  $\cdot$ Parcifal] Parzifal T [*]: do parzefal V partzifal W \textbf{10} vuogten] fuͤgte W  $\cdot$ diu] des W \textbf{12} diu] die T  $\cdot$ die] meine W \textbf{13} unsanfte] sanffte W \textbf{14} ein wîp] dem weib daz sich W \textbf{15} Wil mich [*gen]: die alsvs twingen V \textbf{17} sols ir] sol sie ir U sol irs V solt mich W \textbf{18} trôste] yoste U \textbf{19} jenen] ginen V inen W \textbf{20} umbir] Vmer U \textbf{21} Segremors] [S*gremors]: Sagremors V  $\cdot$ der] \textit{om.} W \textbf{22} rehte] [*]: herre V \textbf{24} swie] Wie U W \textbf{26} verliuse] verliesen U verlv́re V  $\cdot$ mîn] hie mein W \textbf{27} strît] streite W  $\cdot$ ze] so W \textbf{28} wil] wil eúch W  $\cdot$ iuch] îv T \textit{om.} W  $\cdot$ des biten] debiten T \textbf{29} ergebet] Er git U  $\cdot$ iuch] îv T \newline
\end{minipage}
\end{table}
\end{document}
