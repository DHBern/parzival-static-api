\documentclass[8pt,a4paper,notitlepage]{article}
\usepackage{fullpage}
\usepackage{ulem}
\usepackage{xltxtra}
\usepackage{datetime}
\renewcommand{\dateseparator}{.}
\dmyyyydate
\usepackage{fancyhdr}
\usepackage{ifthen}
\pagestyle{fancy}
\fancyhf{}
\renewcommand{\headrulewidth}{0pt}
\fancyfoot[L]{\ifthenelse{\value{page}=1}{\today, \currenttime{} Uhr}{}}
\begin{document}
\begin{table}[ht]
\begin{minipage}[t]{0.5\linewidth}
\small
\begin{center}*D
\end{center}
\begin{tabular}{rl}
\textbf{623} & \begin{large}V\end{large}on der burc die vrouwen\\ 
 & \textbf{dise} wirtschaft mohten schouwen.\\ 
 & anderhalp anz urvar,\\ 
 & manec wert ritter kom al dar.\\ 
5 & ir bûhurt \textbf{mit kunst wart} getân.\\ 
 & disehalp hêr Gawan\\ 
 & dankte dem verjen unt der tohter sîn\\ 
 & - als tet ouch diu herzogîn -,\\ 
 & ir güetlîchen spîse.\\ 
10 & Diu herzoginne wîse\\ 
 & sprach: "war ist der rîter komen,\\ 
 & von dem diu tjoste wart \textbf{genomen}\\ 
 & gestern, dô ich hinnen reit?\\ 
 & ob den iemen überstreit,\\ 
15 & weder schiet daz leben oder tôt?"\\ 
 & Dô sprach Plippalinot:\\ 
 & "vrouwe, ich sach in hiute leben,\\ 
 & \textbf{er} wart mir vür \textbf{ein} ors gegeben.\\ 
 & welt ir ledigen den man,\\ 
20 & dâr umbe sol ich Swalwen hân,\\ 
 & diu \textbf{der} küneginne Secundillen was\\ 
 & unt die iu sante Anfortas.\\ 
 & mac diu harpfe wesen mîn,\\ 
 & ledec ist der \textbf{herzoge} \textbf{von} Gowerzin."\\ 
25 & "\textbf{Die harpfen untz ander krâmgewant}",\\ 
 & sprach si, "\textbf{wil} er, mit sîner hant\\ 
 & mac \textbf{geben} unt behalden,\\ 
 & der hie sitzet. lât\textbf{s in} walden.\\ 
 & ob ich im sô liep wart ie,\\ 
30 & er lœset mir \textbf{Lischoysen} hie,\\ 
\end{tabular}
\scriptsize
\line(1,0){75} \newline
D Z Fr16 \newline
\line(1,0){75} \newline
\textbf{1} \textit{Initiale} D Z  \textbf{10} \textit{Majuskel} D  \textbf{16} \textit{Majuskel} D  \textbf{25} \textit{Majuskel} D  \newline
\line(1,0){75} \newline
\textbf{2} dise] Die Z \textbf{4} wert ritter] ritter wert Z \textbf{7} dankte] Danket Z \textbf{12} genomen] vernomen Z \textbf{13} dô] da Z \textbf{15} tôt] der tot Z \textbf{16} Dô] Da Z  $\cdot$ Plippalinot] plipalinot Z \textbf{19} man] selben man Fr16 \textbf{20} Swalwen] swalben Z \textbf{21} der] \textit{om.} Fr16 \textbf{24} der herzoge von] dvc de Z \textbf{27} mac] \textit{om.} Z \textbf{30} Lischoysen] Liscoysen D Lishoisen Z \newline
\end{minipage}
\hspace{0.5cm}
\begin{minipage}[t]{0.5\linewidth}
\small
\begin{center}*m
\end{center}
\begin{tabular}{rl}
 & \begin{large}V\end{large}on der burc die vrouwen\\ 
 & \textbf{dise} wirtschaft mohte\textit{n} schouwen.\\ 
 & anderhalp an daz urvar,\\ 
 & manic wert ritter kam aldar.\\ 
5 & ir bûhurt \textbf{wart mit künste} getân.\\ 
 & disehalp hêr Gawan\\ 
 & dankte dem verjen und der tohter sîn\\ 
 & - alsô tet ouch diu herzogîn -,\\ 
 & ir güetlîchen spîse.\\ 
10 & diu herzogîn wîse\\ 
 & sprach: "war ist der ritter komen,\\ 
 & von dem diu just wart \textbf{benomen}\\ 
 & gester, dô ich hinnen reit?\\ 
 & ob den ieman überstreit,\\ 
15 & weder schiet daz leben oder tôt?"\\ 
 & dô sprach Plipp\textit{a}linot:\\ 
 & "vrowe, ich sach in hiute leben,\\ 
 & \textbf{der} wart mir vür \textbf{ein} ros gegeben.\\ 
 & wolt ir ledigen den \textbf{selben} man,\\ 
20 & dâr umb sol ich Swal\textit{w}en hân,\\ 
 & diu künigîn Secundillen was\\ 
 & und die \dag ouch\dag  sante Anfortas.\\ 
 & mac diu \textit{h}arpfe wesen mîn,\\ 
 & ledic ist \textbf{ouch} \dag die\dag  G\textit{o}w\textit{e}rt\textit{z}in."\\ 
25 & "\textbf{die harpfen und daz krâmgewant}",\\ 
 & sprach si, "\textbf{wil} er, mit sîner hant\\ 
 & mac \textbf{geben} und behalten,\\ 
 & \dag die\dag  hie sitzet. lât \textbf{es in} walten.\\ 
 & ob ich im sô liep wart ie,\\ 
30 & er lœset mir \textbf{d\textit{en} herzog\textit{e}n} hie,\\ 
\end{tabular}
\scriptsize
\line(1,0){75} \newline
m n o \newline
\line(1,0){75} \newline
\textbf{1} \textit{Initiale} m n  \newline
\line(1,0){75} \newline
\textbf{2} dise] die n  $\cdot$ mohten] mohtte m \textbf{5} bûhurt] búhart o \textbf{9} güetlîchen] guͯtliche n \textbf{12} benomen] genomen n (o) \textbf{16} Plippalinot] plippolinot m \textbf{18} der] Er n o \textbf{20} Swalwen] swalmen m salwe: o \textbf{21} Secundillen] secundille n \textbf{22} Anfortas] anforttas m \textbf{23} harpfe] scharpf m \textbf{24} die] do n d:: o  $\cdot$ Gowertzin] geworczen ein m gowortzin n gawarczin o \textbf{30} den herzogen] die herczogin m (o) \newline
\end{minipage}
\end{table}
\newpage
\begin{table}[ht]
\begin{minipage}[t]{0.5\linewidth}
\small
\begin{center}*G
\end{center}
\begin{tabular}{rl}
 & von der burc die vrouwen\\ 
 & \textbf{die} wirtschaft mohten schouwen.\\ 
 & anderhalp anz urvar,\\ 
 & manic wert rîter kom al dar.\\ 
5 & ir bûhurt \textbf{mit kunst wart} getân.\\ 
 & disehalp hêr Gawan\\ 
 & dankte dem verigen unt der tohter sîn\\ 
 & - als tet ouch diu herzogîn -,\\ 
 & ir güetlîchen spîse.\\ 
10 & diu herzoginne wîse\\ 
 & \begin{large}S\end{large}prach: "war ist der rîter komen,\\ 
 & von dem diu tjost wart \textbf{genomen}\\ 
 & gester, dô ich hinnen reit?\\ 
 & ob den iemen überstreit,\\ 
15 & weder schiet daz leben oder tôt?"\\ 
 & dô sprach Pliplalinot:\\ 
 & "vrouwe, ich sach in hiute leben,\\ 
 & \textbf{er} wart mir vür \textbf{diz} ors gegeben.\\ 
 & welt ir ledegen den man,\\ 
20 & dâr umbe sol ich Swalwen hân,\\ 
 & diu \textbf{der} künegîn Secundillen was\\ 
 & unde die iu sande Anfortas.\\ 
 & mac diu harpfe wesen mîn,\\ 
 & ledic ist der \textbf{von} Gowerzin.\\ 
25 & \textbf{in vie der helt." "wert erkant}",\\ 
 & sprach \textit{si}. "\textbf{die wîle} er mit sîner hant\\ 
 & mac \textbf{gegeben} unde behalden,\\ 
 & der hie sitzet, lât \textbf{sîn} walden.\\ 
 & ob ich im sô liep wart ie,\\ 
30 & er lœset mir \textbf{Lishoisen} hie,\\ 
\end{tabular}
\scriptsize
\line(1,0){75} \newline
G I L M Z \newline
\line(1,0){75} \newline
\textbf{1} \textit{Initiale} I L Z  \textbf{11} \textit{Initiale} G  \textbf{23} \textit{Initiale} I  \newline
\line(1,0){75} \newline
\textbf{1} der burc] den brvch L \textbf{2} die] sine I \textbf{4} wert rîter] ritter wert Z  $\cdot$ al] \textit{om.} I L \textbf{5} mit kunst wart] wart mit kvnst L \textbf{7} dankte] Dancket L (M) \textbf{9} güetlîchen] vil guͤtlicher I \textbf{12} genomen] vernomen Z \textbf{13} dô] da M Z \textbf{14} überstreit] [vberstreut]: vberstreit I \textbf{15} tôt] der tot I (M) (Z) \textbf{16} dô] Da M Z  $\cdot$ sprach] sprach der verge L  $\cdot$ Pliplalinot] plipalinot I L M Z \textbf{18} mir] \textit{om.} M  $\cdot$ diz] daz L ein Z \textbf{19} ledegen] losen L \textbf{20} dâr umbe] Warvmme M  $\cdot$ Swalwen] swaluwe I swalben Z \textbf{21} Secundillen] [secullen]: secutillen I Secuͯndillen L \textbf{22} Anfortas] Amfortas L \textbf{23} harpfe] helfe I \textbf{24} der von] dvc de Z  $\cdot$ Gowerzin] gouerzin G Goruenzin I gowerzcin M \textbf{25} Die harpfe (harpfen Z ) vntz (vnde M ) ander kram gewant L (M) (Z) \textbf{26} si] \textit{om.} G  $\cdot$ die wîle] wil Z \textbf{27} mac] \textit{om.} Z  $\cdot$ gegeben] geben L (M) (Z) \textbf{28} lât sîn] latsz in L (Z) \textbf{30} Lishoisen] liscoysen I litschoýsen L lisoigen M \newline
\end{minipage}
\hspace{0.5cm}
\begin{minipage}[t]{0.5\linewidth}
\small
\begin{center}*T
\end{center}
\begin{tabular}{rl}
 & \begin{large}V\end{large}on der burc die vrouwen\\ 
 & \textbf{die} wirtschaft mohten schouwen.\\ 
 & anderhalp an daz urvar,\\ 
 & manec werder rîter kam al dar.\\ 
5 & ir bûhurt \textbf{mit kunst wart} getân.\\ 
 & disehalp hêr Gawan\\ 
 & dankete dem vergen und der tohter sîn\\ 
 & - als tet ouch diu herzogîn -,\\ 
 & ir güetlîche\textit{n} spîse.\\ 
10 & diu herzoginne wîse\\ 
 & sprach: "war ist der rîter komen,\\ 
 & von dem diu jost wart \textbf{genomen}\\ 
 & gestern, dô ich hinnen reit?\\ 
 & ob den ieman überstreit,\\ 
15 & w\textit{e}der schiet daz leben oder \textbf{der} tôt?"\\ 
 & dô sprach Plipalinot:\\ 
 & "vrouwe, ich sach in hiute leben,\\ 
 & \textbf{er} wart mir vür \textbf{diz} ors gegeben.\\ 
 & wolt ir ledigen den man,\\ 
20 & dâr umbe sol ich Swalwen hân,\\ 
 & diu \textbf{der} küneginne Secundille was\\ 
 & und die iu sante Anfortas.\\ 
 & \textit{m}ac diu harpfe wesen mîn,\\ 
 & ledic ist der \textbf{von} Gowerzin."\\ 
25 & "\textbf{die harpfe und ander krâmgewant}",\\ 
 & sprach si, "\textbf{die wîl} er mit sîner hant\\ 
 & mag \textbf{gegeben} und behalten,\\ 
 & der hie sitzet, lât \textbf{\textit{es} in} walten.\\ 
 & ob ich im sô liep wart ie,\\ 
30 & er lœset mir \textbf{Lyschoysen} hie,\\ 
\end{tabular}
\scriptsize
\line(1,0){75} \newline
U V W Q R Fr39 \newline
\line(1,0){75} \newline
\textbf{1} \textit{Initiale} U V W Fr39   $\cdot$ \textit{Capitulumzeichen} R  \newline
\line(1,0){75} \newline
\textbf{1} die] div Fr39 \textbf{2} die] [*]: dise V  $\cdot$ mohten] [mohten]: moͤhten V mochte Q \textbf{3} an daz] des Fr39  $\cdot$ urvar] úberuar W vnfar R \textbf{4} Kam manig werder ritter dar V  $\cdot$ al] \textit{om.} W \textbf{5} bûhurt] bvhart V be hur Q  $\cdot$ mit kunst] \textit{om.} R \textbf{6} disehalp] Dishap Q  $\cdot$ hêr Gawan] Min her Gewan R \textbf{7} dankete dem] Danket den R Danket dem Fr39 \textbf{9} güetlîchen] gutliche U  $\cdot$ spîse] preise W speysen Q \textbf{12} wart] [*]: wart V ist W R  $\cdot$ genomen] [*]: genomen V benomen Q \textbf{15} weder] Werder U  $\cdot$ schiet] siet in Q \textbf{16} Plipalinot] Plypalmat U plypalinot W \textbf{18} wart mir] ward W (Q) (Fr39) ward m>ir < R  $\cdot$ diz] das R \textbf{19} den] den selben V \textbf{20} Swalwen] schwalben W walwen R \textbf{21} diu der] Der die Q  $\cdot$ küneginne] kᵫnginnen R  $\cdot$ Secundille] Secuͦndille U secundillen V W (R) Fr39 secúndillen Q \textbf{22} iu] auch Q iuch so R  $\cdot$ Anfortas] antefortes R \textbf{23} mac] Wac U \textbf{24} Gowerzin] gowerzein W kaberzin Q \textbf{25} harpfe] harpfen Fr39  $\cdot$ krâmgewant] crom genant R \textbf{26} die wîl er] \sout{die wile er} R \textbf{28} lât es] lat zuͦ U den lant ez V lat sein Q lacz R (Fr39)  $\cdot$ in] \textit{om.} V Q \textbf{30} mir] [*]: mir V mit R  $\cdot$ Lyschoysen] lyschoien U lischoen V lyshoien W lishoisen Q Fr39 Litschoisen R \newline
\end{minipage}
\end{table}
\end{document}
