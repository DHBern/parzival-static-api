\documentclass[8pt,a4paper,notitlepage]{article}
\usepackage{fullpage}
\usepackage{ulem}
\usepackage{xltxtra}
\usepackage{datetime}
\renewcommand{\dateseparator}{.}
\dmyyyydate
\usepackage{fancyhdr}
\usepackage{ifthen}
\pagestyle{fancy}
\fancyhf{}
\renewcommand{\headrulewidth}{0pt}
\fancyfoot[L]{\ifthenelse{\value{page}=1}{\today, \currenttime{} Uhr}{}}
\begin{document}
\begin{table}[ht]
\begin{minipage}[t]{0.5\linewidth}
\small
\begin{center}*D
\end{center}
\begin{tabular}{rl}
\textbf{639} & \begin{large}E\end{large}z \textbf{en}sî denne gar ein vrâz,\\ 
 & welt ir, si habent genuoc \textbf{dâ} gâz.\\ 
 & man truoc die tische gar her dan.\\ 
 & Dô vrâgete mîn hêr Gawan\\ 
5 & umbe guote videlære,\\ 
 & ob \textbf{der} dâ keiner wære.\\ 
 & dâ was werder knappen vil\\ 
 & wol gelêrt ûf seitspil.\\ 
 & ir enkeines kunst was \textbf{doch} sô ganz,\\ 
10 & \textbf{sine müesten} strîchen alten tanz.\\ 
 & \textbf{niuwer} tenze \textbf{was} dâ wênec vernomen,\\ 
 & der uns von Duringen vil ist komen.\\ 
 & Nû \textbf{dankt es} dem wirte,\\ 
 & ir vreude er \textbf{si} niht irte.\\ 
15 & manec vrouwe wol gevar\\ 
 & \textbf{giengen} vür in \textbf{tanzen} dar.\\ 
 & sus wart \textbf{ir} tanz gezieret,\\ 
 & wol underparrieret\\ 
 & die rîter underz vrouwen her;\\ 
20 & gein der riwe \textbf{kômen si} ze wer.\\ 
 & ouch \textbf{muoste} man dâ schouwen\\ 
 & \textbf{ie} zwischen zwein vrouwen\\ 
 & einen clâren rîter gên;\\ 
 & man mohte vreude an in verstên.\\ 
25 & Swelch rîter pflac der sinne,\\ 
 & daz er dienst bôt \textbf{nâch} minne,\\ 
 & diu bete was urlouplîch.\\ 
 & die sorgen arm unt \textbf{die} vreuden rîch\\ 
 & mit rede vertriben \textbf{die} stunde\\ 
30 & gein manegem süezem munde.\\ 
\end{tabular}
\scriptsize
\line(1,0){75} \newline
D Z Fr1 \newline
\line(1,0){75} \newline
\textbf{1} \textit{Initiale} D Z Fr1  \textbf{4} \textit{Majuskel} D  \textbf{13} \textit{Majuskel} D   $\cdot$ \textit{Versal} Fr1  \textbf{25} \textit{Majuskel} D  \newline
\line(1,0){75} \newline
\textbf{1} denne] \textit{om.} Z \textbf{2} habent] haben Z \textbf{8} seitspil] seiten spil Z \textbf{10} sine müesten] Sie musten Z \textbf{13} dankt es] danken Z \textbf{14} vreude] frevden Z \textbf{16} tanzen] zv tantze Z \textbf{17} ir] der Z \textbf{21} muoste] moht Z \textbf{26} Daz erbot dienst nach minne Z \textbf{29} die] ir Z \textbf{30} süezem] svzzen Z \newline
\end{minipage}
\hspace{0.5cm}
\begin{minipage}[t]{0.5\linewidth}
\small
\begin{center}*m
\end{center}
\begin{tabular}{rl}
 & \begin{large}E\end{large}z sî den gar ein vrâz,\\ 
 & wolt ir, si habent genuoc \textbf{d\textit{â}} gâz.\\ 
 & man truoc die tische gar her dan.\\ 
 & dô vrâgte mîn hêr Gawan\\ 
5 & umb guote videlære,\\ 
 & ob \textbf{ir} d\textit{â} keiner wære.\\ 
 & d\textit{â} was werder knappen vil\\ 
 & wol gelêret ûf seiten spil.\\ 
 & ir keines kunst was \textbf{doch} sô ganz,\\ 
10 & \textbf{si müesten} strî\textit{ch}en alten tanz.\\ 
 & \textbf{iuwer} tenze \textbf{was} d\textit{â} wênic vernomen,\\ 
 & der uns von Duringen \textit{vil} ist komen.\\ 
 & nû \textbf{danket es} d\textit{em} wirt\textit{e},\\ 
 & ir vröude er \textbf{si} niht irt\textit{e}.\\ 
15 & manigiu vrowe wol gevar\\ 
 & \textbf{giengen} vür in \textbf{tanzen} dar.\\ 
 & sus wart \textbf{ir} tanz gezieret,\\ 
 & wol underparrieret\\ 
 & die ritter \textit{u}nder\textit{z} vrowen her\\ 
20 & gegen der riuwe \textbf{kômen} zuo wer.\\ 
 & ouch \textbf{mohte} man d\textit{â} schouwen\\ 
 & \textbf{ie} zwischen zwein vrouwen\\ 
 & einen clâren ritter gân;\\ 
 & man mohte vröude an in verstân.\\ 
25 & welch ritter pflac der sinne,\\ 
 & daz er dienst bôt \textbf{durch} minne,\\ 
 & diu bete was erlouplîch.\\ 
 & die sorgen \dag an in die vrowen\dag  rîch\\ 
 & mit rede ver\textit{triben} \textbf{ir} stund\textit{e}\\ 
30 & gegen manigem süezen mund\textit{e}.\\ 
\end{tabular}
\scriptsize
\line(1,0){75} \newline
m n o \newline
\line(1,0){75} \newline
\textbf{1} \textit{Initiale} m   $\cdot$ \textit{Capitulumzeichen} n  \newline
\line(1,0){75} \newline
\textbf{2} si] \textit{om.} n  $\cdot$ habent] haben o  $\cdot$ dâ] do m n sie o \textbf{3} gar] do o  $\cdot$ dan] han n \textbf{4} hêr] herre her n \textbf{6} dâ] do m n o \textbf{7} dâ] Do m n o \textbf{10} strîchen] stritten m \textbf{11} tenze] ::ncz o  $\cdot$ dâ] do m n o \textbf{12} Duringen] durengen o  $\cdot$ vil] \textit{om.} m \textbf{13} dem wirte] die wirttin m \textbf{14} irte] irtin m erte o \textbf{19} underz] ander m \textbf{21} mohte] moͯchte n  $\cdot$ dâ] do m n o \textbf{24} mohte] moͯchte n \textbf{26} daz] [Durch]: Das o \textbf{28} in] \textit{om.} o \textbf{29} Mit rede verstunt ir stunden m  $\cdot$ rede] reden o \textbf{30} manigem] mangen o  $\cdot$ munde] munden m \newline
\end{minipage}
\end{table}
\newpage
\begin{table}[ht]
\begin{minipage}[t]{0.5\linewidth}
\small
\begin{center}*G
\end{center}
\begin{tabular}{rl}
 & \begin{large}E\end{large}z\textbf{n} sî danne gar ein vrâz,\\ 
 & welt ir, si habent genuoc gâz.\\ 
 & man truoc die tische gar her dan.\\ 
 & dô vrâget mîn hêr Gawan\\ 
5 & umbe guote videlære,\\ 
 & op \textbf{der} dâ deheiner w\textit{æ}r\textit{e}.\\ 
 & dâ was werder knappen vil\\ 
 & wol gelêrt ûf seitspil.\\ 
 & ir neheines kunst was sô ganz,\\ 
10 & \textbf{sin\textit{e} m\textit{üe}sen} strîchen alten tanz.\\ 
 & \textbf{niuwer} tenze \textbf{was} dâ wênic vernomen,\\ 
 & der uns von Durngen vil ist komen.\\ 
 & nû \textbf{danken} dem wirte,\\ 
 & ir vröude er \textbf{si} niht irte.\\ 
15 & manic vrouwe wol gevar\\ 
 & \textbf{gienc} vür in \textbf{ze tanze} \textit{d}ar.\\ 
 & sus wart \textbf{der} tanz gezieret,\\ 
 & wol underparrieret\\ 
 & die rîter underz vrouwen her;\\ 
20 & gein der riuw\textit{e} \textbf{wâren si} ze wer.\\ 
 & ouch \textbf{moht} man dâ schouwen\\ 
 & \textbf{ie} zwischen zwein vrouwen\\ 
 & einen clâren rîter gên;\\ 
 & man moht vröude an in verstên.\\ 
25 & swelch rîter pflac der sinne,\\ 
 & daz er dienst bôt \textbf{nâch} minne,\\ 
 & diu bet was urlouplîch.\\ 
 & die sorgen arm unde \textbf{die} vröuden rîch\\ 
 & mit rede vertriben \textbf{ir} stunde\\ 
30 & gein manigem süezem munde.\\ 
\end{tabular}
\scriptsize
\line(1,0){75} \newline
G I L M Z Fr18 Fr48 \newline
\line(1,0){75} \newline
\textbf{1} \textit{Initiale} G I L Z Fr18  \textbf{15} \textit{Initiale} I  \newline
\line(1,0){75} \newline
\textbf{1} danne gar] [*]: Gar I danne L gar Z \textbf{2} genuoc] gnuc da M (Z) \textbf{4} Dô] da M  $\cdot$ vrâget] vragte I (M) (Z) (Fr18) [v*]: fragte  L  $\cdot$ hêr Gawan] ergawan M \textbf{6} der] \textit{om.} L  $\cdot$ dâ deheiner] dehainer da I (M) do dheiner Fr48  $\cdot$ wære] war G \textbf{8} seitspil] shaitspil I seýten spil L (M) (Fr48) \textbf{9} neheines] icheyner M  $\cdot$ sô] doch so Z Fr48 \textbf{10} sine] sinen G Sie Z  $\cdot$ müesen] muͦsin G (I) (Z) (Fr48) mvͤse Fr18  $\cdot$ alten] al den I \textbf{11} was dâ wênic] wart wenc da I waz wenig L waz do wenech Fr48 \textbf{12} Durngen] doringen L (M) Dvringen Z (Fr18) duͦrgen Fr48 \textbf{13} danken] danchet I danckem L danckten Fr48 \textbf{14} vröude] frevden Z  $\cdot$ er] \textit{om.} M \textbf{16} gienc] Giengen M (Z) (Fr18) (Fr48)  $\cdot$ in ze tanze] in tantzen L Fr18 eyn tanczen M  $\cdot$ Dar] gar G \textbf{17} wart] \textit{om.} L \textbf{19} underz] vnd daz I vnd der L \textbf{20} riuwe] riͮwen G  $\cdot$ wâren si] warn I sý komen L quam sy M qvamen sie Z (Fr18) (Fr48) \textbf{21} dâ] wol I do Fr48 \textbf{22} ie] \textit{om.} L M Fr18  $\cdot$ vrouwen] jvngfrauwen L \textbf{23} rîter] [froͮ]: riter G \textbf{24} in] im Fr18 \textbf{25} swelch] Welch L Willich M \textbf{26} er dienst bôt] erbot dienst Z \textbf{27} urlouplîch] erloublichen Fr48 \textbf{28} sorgen] sorge I Fr48  $\cdot$ unde] \textit{om.} L M  $\cdot$ rîch] reichen Fr48 \textbf{29} rede] redin M  $\cdot$ vertriben] vertrib I  $\cdot$ ir] die L \textbf{30} süezem] suͯszen L (Z) (Fr48) \newline
\end{minipage}
\hspace{0.5cm}
\begin{minipage}[t]{0.5\linewidth}
\small
\begin{center}*T
\end{center}
\begin{tabular}{rl}
 & \begin{large}E\end{large}z \textbf{en}sî dan gar ein vrâz,\\ 
 & wolt ir, si hânt genuoc \textbf{d\textit{â}} gâz.\\ 
 & man truoc die tische gar her dan.\\ 
 & dô vrâgete mîn hêr Gawan\\ 
5 & umbe guote videlære,\\ 
 & ob dâ dekeiner wære.\\ 
 & dô was \textbf{dâ} werder knappen vil\\ 
 & wol gelêret ûf seiten spil.\\ 
 & ir dekeines kunst was sô ganz,\\ 
10 & \textbf{er enm\textit{üe}se} strîchen alten tanz.\\ 
 & \textbf{iuwer} tenze \textbf{wâren} d\textit{â} wênic vernomen,\\ 
 & der uns von Duringen vil ist komen.\\ 
 & nû \textbf{danken} dem wirte,\\ 
 & ir vreude er niht irte.\\ 
15 & manegiu vrouwe wol gevar\\ 
 & \textbf{gienc} vür in \textbf{zuo tanze} dar.\\ 
 & sus wart \textbf{der} tanz gezieret,\\ 
 & wol underparrieret\\ 
 & die rîter under daz vrouwen her;\\ 
20 & gein der riuwe \textbf{kâmen si} zuo wer.\\ 
 & ouch \textbf{mohte} man d\textit{â} schouwen\\ 
 & zwischen zwein vrouwen\\ 
 & einen clâren rîter gên;\\ 
 & man mohte vreude an in verstên.\\ 
25 & welch rîter pflac der sinne,\\ 
 & daz er dienst bôt \textbf{nâch} minne,\\ 
 & diu bete was erlouplîch.\\ 
 & die sorgen arm und vreuden rîch\\ 
 & mit rede vertriben \textbf{ir} stunde\\ 
30 & gein manegem süezen munde.\\ 
\end{tabular}
\scriptsize
\line(1,0){75} \newline
U V W Q R Fr40 \newline
\line(1,0){75} \newline
\textbf{1} \textit{Initiale} U V Fr40  \newline
\line(1,0){75} \newline
\textbf{1} Ez ensî] Er wer W \textbf{2} ir] er W  $\cdot$ si hânt genuoc] genuͦg er W  $\cdot$ dâ] do U V W \textbf{3} die] den V  $\cdot$ her dan] ent dar Q \textbf{4} vrâgete] vragt Fr40 \textbf{6} ob] [D*]: Ob der V Ob ir W Ob der Q R (Fr40)  $\cdot$ dâ] [*]: do V do W \textit{om.} R  $\cdot$ dekeiner] [*]: deheiner V \textbf{7} dâ] [*]: do V \textit{om.} W do Q  $\cdot$ werder] der werden Fr40 \textbf{9} ir] [*]: Der V Der R  $\cdot$ dekeines] [*]: deheinez V  $\cdot$ was] [*]: was V was doch W \textbf{10} er enmüese] Er in muͦze U [*]: Er muͤste V Er muͦßte W (R) (Fr40) \textbf{11} [*]: Nuwer tenze waz do wening vernomen V  $\cdot$ iuwer] Newer W Q (R) (Fr40)  $\cdot$ tenze] tantz Q (R)  $\cdot$ wâren] waz W (Q) R (Fr40)  $\cdot$ dâ] do U W Q \textbf{12} der] [D*]: Der V Die W  $\cdot$ Duringen] duͦringen U [*]: túrningen V dúringen W turringen R  $\cdot$ vil ist] sein W \textbf{13} danken] dankent [*]: ez V dancket es W dankent R dankem Fr40  $\cdot$ dem wirte] [*]: dem wirte V \textbf{14} vreude] frewden Q (Fr40)  $\cdot$ er] er sv́ V (W) (Q) (R) (Fr40) \textbf{15} manegiu] Mnig R \textbf{16} gienc] Giengen W  $\cdot$ in zuo tanze] ir tantzen W \textbf{17} sus] Als Q  $\cdot$ der] ir W \textbf{18} underparrieret] vnter partiret Q \textbf{19} daz] den R \textbf{20} riuwe] rewen Q \textbf{21} mohte] moͤhte V  $\cdot$ dâ] do U V W \textit{om.} Q \textbf{22} zwischen] [*]: Je zwúschent V Ie zwischen W \textbf{24} mohte] moͤhte V  $\cdot$ in] im W Q \textbf{25} welch] Swelich V (Fr40) \textbf{26} nâch] vmbe W \textbf{28} Den sorgen arm den froͤiden rich V  $\cdot$ Die ritter vnd auch die frawen rich W \textbf{29} rede] reden Q  $\cdot$ vertriben ir stunde] vertriben [*]: die stunde V vertibent die stund R \textbf{30} manegem] mengen R  $\cdot$ süezen] sussem Q (Fr40) \newline
\end{minipage}
\end{table}
\end{document}
