\documentclass[8pt,a4paper,notitlepage]{article}
\usepackage{fullpage}
\usepackage{ulem}
\usepackage{xltxtra}
\usepackage{datetime}
\renewcommand{\dateseparator}{.}
\dmyyyydate
\usepackage{fancyhdr}
\usepackage{ifthen}
\pagestyle{fancy}
\fancyhf{}
\renewcommand{\headrulewidth}{0pt}
\fancyfoot[L]{\ifthenelse{\value{page}=1}{\today, \currenttime{} Uhr}{}}
\begin{document}
\begin{table}[ht]
\begin{minipage}[t]{0.5\linewidth}
\small
\begin{center}*D
\end{center}
\begin{tabular}{rl}
\textbf{728} & \textbf{aber} anders niht \textbf{decheinen gewîs},\\ 
 & wan \textbf{ob} Gawan, ir âmîs,\\ 
 & \textbf{wolte den kampf} durch si verbern,\\ 
 & sô wol\textit{t} \textbf{ouch} si \textbf{der} suone wern.\\ 
5 & \textbf{diu suone würde} \textbf{von ir} getân,\\ 
 & ob der künec wolde lân\\ 
 & \textbf{biziht} ûf ir sweher Lot.\\ 
 & bî Artuse si daz \textbf{dan} enbôt.\\ 
 & Artus, der wîse, \textbf{höfsche} man,\\ 
10 & disiu mære brâhte dan.\\ 
 & dô muose der künec Gramoflanz\\ 
 & verkiesen umbe sînen kranz\\ 
 & und swaz er hazzes pflæge\\ 
 & gein Lote vo\textit{n} Norwæge,\\ 
15 & \textbf{der} zergienc als in der sunnen snê\\ 
 & durch die clâren Itonje\\ 
 & lûterlîche ân \textbf{allen} haz.\\ 
 & daz \textbf{ergienc}, die wîle er bî ir saz.\\ 
 & \textbf{Aller} ir bete er \textbf{volge} jach.\\ 
20 & Gawanen man \textbf{dort} komen sach\\ 
 & mit clârlîchen liuten.\\ 
 & i\textbf{ne} \textbf{m\textit{ö}hte} iu niht gar bediuten\\ 
 & ir namen \textbf{und} wannen si \textbf{wâren} \textbf{erborn}.\\ 
 & dâ wart durch \textbf{liebe} leit verkorn.\\ 
25 & Orgeluse, diu fiere,\\ 
 & und ir \textbf{werden} \textbf{soldiere}\\ 
 & und ouch diu Clinschors schar,\\ 
 & ir ein teil - si\textbf{ne} wâren\textbf{z} niht gar -\\ 
 & \textbf{sach man} mit Gawane komen.\\ 
30 & Artuses gezelde \textbf{wart} genomen\\ 
\end{tabular}
\scriptsize
\line(1,0){75} \newline
D \newline
\line(1,0){75} \newline
\textbf{19} \textit{Majuskel} D  \newline
\line(1,0){75} \newline
\textbf{4} wolt] wol D \textbf{14} von] vor D \textbf{16} Itonje] Jtonie D \textbf{20} Gawanen] Gawann D \textbf{22} möhte] mohte D \textbf{27} Clinschors] Clinscors D \textbf{30} Artuses] Artvs D \newline
\end{minipage}
\hspace{0.5cm}
\begin{minipage}[t]{0.5\linewidth}
\small
\begin{center}*m
\end{center}
\begin{tabular}{rl}
 & \textbf{aber} anders niht \textbf{dekein wîs},\\ 
 & wan \textbf{ob} Gawan, ir âmîs,\\ 
 & \textbf{wolt den kampf} durch si ver\textit{be}rn,\\ 
 & sô wolt \textbf{ouch} si \textbf{der} suone wern.\\ 
5 & \textbf{diu suone würde} \textbf{aber niht} getân,\\ 
 & \textbf{dan} ob der künic \textbf{niht} wolte lân\\ 
 & \textbf{beriht} ûf ir sweher Lot.\\ 
 & bî Artuse si daz \textbf{hin} enbôt.\\ 
 & Artus, der wîse man,\\ 
10 & disiu mær brâhte \textbf{wider} dan.\\ 
 & \begin{large}D\end{large}ô muoste der künic Gramonlantz\\ 
 & verkiesen umb sînen kranz\\ 
 & und waz er hazzes pflæge\\ 
 & gegen Lot von Norwege,\\ 
15 & \textbf{der} zergien\textit{c} als in der sunnen \textbf{der} snê\\ 
 & durch die clâren Itonie\\ 
 & lûterlîch âne \textbf{allen} haz.\\ 
 & daz \textbf{ergienc}, die wîle er bî ir saz.\\ 
 & \textbf{alle} ir bete er \textbf{volge} jach.\\ 
20 & Gawanen man \textbf{dort} komen sach\\ 
 & mit clârlîchen liuten.\\ 
 & ich \textbf{m\textit{ö}hte} iu niht gar be\textit{d}iuten\\ 
 & ir namen \textbf{oder} wannen si \textbf{sint} \textbf{erborn}.\\ 
 & d\textit{â} wart durch \textbf{liebe} leit verkorn.\\ 
25 & Urgeluse, diu fier,\\ 
 & und ir \textbf{werden} \textbf{soldier}\\ 
 & und ouch diu Clinsors schar,\\ 
 & ir ein teil - si wâren\textbf{s} niht gar -\\ 
 & \textbf{sach man} mit Gawane\textit{n} komen.\\ 
30 & Artuses gezelt \textbf{was} genomen\\ 
\end{tabular}
\scriptsize
\line(1,0){75} \newline
m n o Fr69 \newline
\line(1,0){75} \newline
\textbf{11} \textit{Initiale} m n  \newline
\line(1,0){75} \newline
\textbf{1} dekein] do keine n \textbf{2} wan] Wenne n \textbf{3} verbern] verlorn m \textbf{4} ouch si] sú ouch n  $\cdot$ suone] sone o \textbf{7} biz vntz anden abent rot Fr69  $\cdot$ sweher] speher n swester o \textbf{8} \textit{Vers 728.8 fehlt} o  \textbf{9} Artus] Bẏ artuse o \textbf{11} der] er o  $\cdot$ Gramonlantz] gramolantz n gramolancz o \textbf{15} zergienc] zurgin m \textbf{16} Itonie] ithonie n jtonie o ytonie Fr69 \textbf{18} ergienc] zerging n \textbf{20} Gawanen] Gawannen o \textbf{22} möhte] mohtte m (o) moch Fr69  $\cdot$ niht] [noh]: nich Fr69  $\cdot$ bediuten] beguͯten m \textbf{23} wannen] wanne Fr69  $\cdot$ sint] weren n worent o (Fr69) \textbf{24} dâ] Do m n o \textbf{27} Clinsors] clinsor n \textbf{29} Gawanen] gawanes m \textbf{30} Artuses] Artuͯses o \newline
\end{minipage}
\end{table}
\newpage
\begin{table}[ht]
\begin{minipage}[t]{0.5\linewidth}
\small
\begin{center}*G
\end{center}
\begin{tabular}{rl}
 & \textbf{\begin{large}U\end{large}nde aber} anders niht \textbf{deheine wîs},\\ 
 & wan \textbf{op} Gawan, ir âmîs,\\ 
 & \textbf{den kampf wolde} durch si verbern,\\ 
 & s\textit{ô} wolde si \textbf{der} suone wern.\\ 
5 & \textbf{sô würde diu suone} \textbf{von ir} getân\\ 
 & \textbf{unde} op der künic wolde lân\\ 
 & \textbf{die ziht} ûf ir sweher Lot.\\ 
 & bî Artus si daz \textbf{hin} enbôt.\\ 
 & Artus, der wîse, \textbf{höfsche} man,\\ 
10 & disiu mære brâhte dan.\\ 
 & dô muose der künec Gramoflanz\\ 
 & verkiesen umbe sînen kranz\\ 
 & unde swaz er hazzes pflæge\\ 
 & gein Lot von Norwæge,\\ 
15 & \textbf{daz} zergie als in der sunnen snê\\ 
 & durch die clâren Itonie\\ 
 & lûterlîch ân \textbf{allen} haz.\\ 
 & daz \textbf{geschach}, die wîle er bî ir saz.\\ 
 & \textbf{alle} ir bete er \textbf{volge} jach.\\ 
20 & Gawan man \textbf{dô} komen sach\\ 
 & mit clârlîchen liuten.\\ 
 & ich\textbf{ne} \textbf{mag} iu niht gar bediuten\\ 
 & \textbf{umbe} ir namen \textbf{unde} wannen si \textbf{sîn} \textbf{geborn}.\\ 
 & dâ wart durch \textbf{lieb} leit verkorn.\\ 
25 & Orgeluse, diu fiere,\\ 
 & unde ir \textbf{soldeniere}\\ 
 & unde ouch diu Clinsors schar,\\ 
 & ir ein teil - si\textbf{ne} wâren\textbf{s} niht gar -,\\ 
 & \textbf{die wâren} mit Gawane komen.\\ 
30 & Artuses gezelt \textbf{was} genomen\\ 
\end{tabular}
\scriptsize
\line(1,0){75} \newline
G I L M Z Fr20 Fr24 \newline
\line(1,0){75} \newline
\textbf{1} \textit{Initiale} G L Z Fr20  \textbf{11} \textit{Initiale} I  \newline
\line(1,0){75} \newline
\textbf{1} Unde] ÷nde Fr20  $\cdot$ aber] vber I \textit{om.} L  $\cdot$ niht] ze I  $\cdot$ deheine] icheyne M keinen Z \textbf{3} den kampf wolde] wolde den champh I (L) (M) (Z) (Fr20) (Fr24)  $\cdot$ verbern] erwern M \textbf{4} \textit{Vers 728.4 fehlt} M   $\cdot$ sô] si G  $\cdot$ der] in der I \textbf{5} sô] vnde I  $\cdot$ würde] wurden Z  $\cdot$ von] vor I  $\cdot$ ir] yn M \textbf{6} op] \textit{om.} Fr24 \textbf{7} ziht] inziht I (L)  $\cdot$ Lot] loht Fr20 \textbf{8} Artus] Artuse L (M)  $\cdot$ si] \textit{om.} Fr20  $\cdot$ hin] dane L (Z) (Fr20) o\textit{m. } M \textbf{9} höfsche] hofbsher I \textit{om.} M \textbf{11} dô] Da M Z  $\cdot$ muose] mvͦser Fr20  $\cdot$ Gramoflanz] gramoflantz Z [gl]: gramoflanz Fr20 \textbf{12} sînen] disen M \textbf{13} swaz] waz L M \textbf{14} Lot] loht Fr20  $\cdot$ von] dem I  $\cdot$ Norwæge] norwege I (M) (Z) Norwage L Norwæ̂ge Fr24 \textbf{15} zergie] zcu ginc M  $\cdot$ sunnen] svnne M (Fr24) \textbf{16} durch die] von der I :::r Fr24  $\cdot$ Itonie] Itonîe G Jconie Z \textbf{17} allen] \textit{om.} L M Fr24 \textbf{19} alle] aller I  $\cdot$ er volge] der volge er I \textbf{20} Gawan] Gawanen L  $\cdot$ man] nam Fr20  $\cdot$ dô] \textit{om.} M da Z \textbf{21} clârlîchen] clare lichten L \textbf{22} ichne] ich I (L)  $\cdot$ iu] \textit{om.} L  $\cdot$ gar] \textit{om.} M \textbf{23} umbe] \textit{om.} I L  $\cdot$ wannen] wan Z  $\cdot$ si sîn] si wern I (Z) (Fr20) (Fr24) sie waren L \textit{om.} M \textbf{24} \textit{nach 728.24:} vnd ouch allir slahte zorn Fr20   $\cdot$ verkorn] verlorn L \textbf{25} Orgeluse] Orgillvsie G Orguluse I Orgelýse L :::e Fr24  $\cdot$ fiere] here Fr20 \textbf{26} soldeniere] werde soldiere L M Z Fr24 \textbf{27} Clinsors] clinisors L Clingshors Z Clinshors Fr24 \textbf{28} sine wârens] si warnz I (L) (M) (Fr24) sie enwaren sin Z sine was Fr20 \textbf{29} Gawane] Gawan I (Z) Fr24 \textbf{30} Artuses] Artus G (I) Z (Fr20) Artuͯses L \newline
\end{minipage}
\hspace{0.5cm}
\begin{minipage}[t]{0.5\linewidth}
\small
\begin{center}*T
\end{center}
\begin{tabular}{rl}
 & \textbf{wan} anders niht \textbf{dekeine wîs},\\ 
 & wan \textbf{wolte} Gawan, ir âmîs,\\ 
 & \textbf{den kampf} durch si verbern,\\ 
 & sô wolte si \textbf{die} suone wern.\\ 
5 & \textbf{sô würde diu suone} \textbf{von ir} getân\\ 
 & \textbf{und} o\textit{b} der künec wolte lân\\ 
 & \textbf{die zi\textit{h}t} ûf ir sweher Lot.\\ 
 & b\textit{î} Artuse si daz enbôt.\\ 
 & \begin{large}A\end{large}rtus, der wîse, \textbf{hövesch} man,\\ 
10 & disiu mære brâhte dan.\\ 
 & dô muose der künec Gramoflanz\\ 
 & verkiesen umb sînen kranz\\ 
 & und waz er hazzes pflæge\\ 
 & gein Lot von Norwæge,\\ 
15 & \textbf{daz} z\textit{e}rgienc als in der sunnen snê\\ 
 & durch die clâren Itonie\\ 
 & lûterlîchen âne haz.\\ 
 & daz \textbf{geschach}, die wîle er bî ir saz.\\ 
 & \textbf{aller} ir bete er \textbf{volgen} jach.\\ 
20 & Gawanen man \textbf{dô} komen sach\\ 
 & mit clârlîchen liuten.\\ 
 & ich \textbf{en}\textbf{mag} iu niht gar betiuten\\ 
 & ir namen \textbf{und} \textit{wannen si} \textbf{wâren} \textbf{geborn}.\\ 
 & d\textit{â} wart durch \textbf{liep} leit verkorn.\\ 
25 & Orgeluse, diu fiere,\\ 
 & und ir \textbf{werden} \textbf{soldiere}\\ 
 & und ouch diu Clynsors schar,\\ 
 & ir ein teil - si \textbf{en}wâren\textbf{z} niht gar -\\ 
 & \textbf{sach man} mit Gawane komen.\\ 
30 & Artuses gezelte \textbf{was} genomen\\ 
\end{tabular}
\scriptsize
\line(1,0){75} \newline
U V W Q R \newline
\line(1,0){75} \newline
\textbf{9} \textit{Überschrift:} Hie wúrt gawan versuͤnet mit gramaflanze vnd nimet sine swester zvͦ der e V   $\cdot$ \textit{Initiale} U V Q  \textbf{25} \textit{Initiale} W  \newline
\line(1,0){75} \newline
\textbf{1} wan] Vnd aber V W (Q) Vnd R  $\cdot$ dekeine] deheine V keine W keinen Q deheinen R \textbf{2} wolte] obe V (Q) (R) das W  $\cdot$ Gawan] herr gawan W Gawin R  $\cdot$ âmîs] armis V \textbf{3} den] Wolte den V W (Q) (R)  $\cdot$ verbern] erwerben Q \textbf{4} die suone] der svͦne V (W) (Q) den suͯnnen R \textbf{5} diu] der R  $\cdot$ von ir] do V vor ir Q \textbf{6} ob] oder U  $\cdot$ lân] dann Q \textbf{7} die] Den V  $\cdot$ ziht] zihit U gezig V zig R \textbf{8} bî] Bit U  $\cdot$ Artuse] artus W Q R  $\cdot$ daz] daz dan V (R) dan W das dar Q \textbf{10} dan] [*]: wider dan V \textbf{11} muose] muͤste V  $\cdot$ Gramoflanz] gramaflantz V gramoflantz W Q Gramoflancz R \textbf{13} waz] swaz V \textbf{14} Lot] Lott R  $\cdot$ Norwæge] Norwege U (V) (W) (Q) (R) \textbf{15} daz] Der V  $\cdot$ zergienc] zuͦ er ginc U ergie W er gie Q  $\cdot$ sunnen] sunne Q  $\cdot$ snê] [*]: der sne V der schne W tuͯt schne R \textbf{16} Itonie] Jtonie U Jconie V ytone W ytonie Q R \textbf{19} aller] Alle Q R  $\cdot$ volgen] volge V W Q R \textbf{20} Gawanen] Gawan W Gawane Q Gawinen R \textbf{21} clârlîchen] klaren W \textbf{22} enmag] mag R  $\cdot$ betiuten] gedeuten Q \textbf{23} wannen si] \textit{om.} U wann sie Q  $\cdot$ wâren] weren Q sind R \textbf{24} dâ] Do U V W Q R  $\cdot$ verkorn] erkoren Q \textbf{25} Orgeluse] Orgeluͦse U  $\cdot$ fiere] [viere]: vieren V frye R \textbf{26} werden] frewde Q  $\cdot$ soldiere] [soldiere]: soldieren V \textbf{27} Clynsors] clinsors V klynshors W klynszhors Q Clinshor R \textbf{28} enwârenz] enwarn W waurencz R  $\cdot$ gar] gewar R \textbf{29} Gawane] gawanen Q Gawan R \textbf{30} Artuses] Artus R  $\cdot$ was] wart V \newline
\end{minipage}
\end{table}
\end{document}
