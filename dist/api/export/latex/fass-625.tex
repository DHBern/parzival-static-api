\documentclass[8pt,a4paper,notitlepage]{article}
\usepackage{fullpage}
\usepackage{ulem}
\usepackage{xltxtra}
\usepackage{datetime}
\renewcommand{\dateseparator}{.}
\dmyyyydate
\usepackage{fancyhdr}
\usepackage{ifthen}
\pagestyle{fancy}
\fancyhf{}
\renewcommand{\headrulewidth}{0pt}
\fancyfoot[L]{\ifthenelse{\value{page}=1}{\today, \currenttime{} Uhr}{}}
\begin{document}
\begin{table}[ht]
\begin{minipage}[t]{0.5\linewidth}
\small
\begin{center}*D
\end{center}
\begin{tabular}{rl}
\textbf{625} & \begin{large}Z\end{large}Arniven sprach Gawan:\\ 
 & "vrouwe, ich sol einen boten hân."\\ 
 & ein juncvrouwe wart gesant,\\ 
 & diu brâhte einen sarjant,\\ 
5 & manlîch, mit zühten wîse,\\ 
 & in sarjandes prîse.\\ 
 & Der knappe \textbf{swuor des} einen eit,\\ 
 & er würbe lieb oder leit,\\ 
 & daz er des niemen dâ\\ 
10 & gewüege noch anderswâ,\\ 
 & wan \textbf{dâ} erz werben solte.\\ 
 & \textbf{Gawan} bat, daz man im holte\\ 
 & tincten und permint.\\ 
 & Gawan, des \textbf{künec} Lotes kint,\\ 
15 & schreip gevuoge mit der hant.\\ 
 & er enbôt ze Lœver in daz lant\\ 
 & Artuse unt des wîbe\\ 
 & dienst von sîme lîbe\\ 
 & mit triwen unverschertet;\\ 
20 & unt het \textbf{er} prîs behertet,\\ 
 & der wære an werdecheite tôt,\\ 
 & si\textbf{ne} hülfen im ze sîner nôt,\\ 
 & daz si beide an triwe \textbf{dæhten}\\ 
 & unt ze Joflanze bræhten\\ 
25 & die messenîe \textbf{mit} vrouwen schar;\\ 
 & unt \textbf{er} \textbf{kœme} ouch selbe \textbf{gein in} dar\\ 
 & durch kampf ûf al sîn êre.\\ 
 & Ernbôt \textbf{im} dennoch mêre,\\ 
 & der kampf wære alsô genomen,\\ 
30 & daz er \textbf{werdeclîche} \textbf{müese} komen.\\ 
\end{tabular}
\scriptsize
\line(1,0){75} \newline
D Z Fr16 \newline
\line(1,0){75} \newline
\textbf{1} \textit{Initiale} D Z  \textbf{7} \textit{Majuskel} D  \textbf{28} \textit{Majuskel} D  \newline
\line(1,0){75} \newline
\textbf{4} einen] im einen Z \textbf{12} Gawan] Er Z (Fr16) \textbf{14} Lotes] Lôts D \textbf{16} Lœver] Loͤver D Lover Z \textbf{17} Artuse] Artvͦse D Artus Z \textbf{22} ze] [an*]: an Z \textbf{24} ze Joflanze] Tschofflantze Z \textbf{26} er] \textit{om.} Z \textbf{28} im] in Z \textbf{30} müese] mvzze Z \newline
\end{minipage}
\hspace{0.5cm}
\begin{minipage}[t]{0.5\linewidth}
\small
\begin{center}*m
\end{center}
\begin{tabular}{rl}
 & zuo Ar\textit{ni}ven sprach Gawan:\\ 
 & "vrowe, ich sol einen boten hân."\\ 
 & ein juncvrowe wart gesant,\\ 
 & diu brâht ei\textit{n}en sarjant,\\ 
5 & manlîch, mit zühten wîse,\\ 
 & in sa\textit{r}jandes \textit{p}rîse.\\ 
 & der knappe \textbf{swuor des} einen eit,\\ 
 & er würbe liep oder leit,\\ 
 & daz er des niemen dâ\\ 
10 & ge\textit{wüe}ge noch anderswâ,\\ 
 & wan \textbf{d\textit{â}} erz werben solte.\\ 
 & \textbf{er} bat, daz man im holte\\ 
 & tinten und permint,\\ 
 & Gawan, des \textbf{küniges} Lote\textit{s} \textit{k}int,\\ 
15 & schreip gevuoge mit der hant.\\ 
 & er enbôt zuo Lo\textit{ve}r in daz lant\\ 
 & Artuse und d\textit{e}s wîbe\\ 
 & dienst von sîn\textit{em} lîbe\\ 
 & mit triuwen unverschertet;\\ 
20 & und het \textbf{er} prîs behertet,\\ 
 & der wær a\textit{n w}erdicheit tôt,\\ 
 & si hülfen ime zuo sîner nôt,\\ 
 & daz si beide an triuwe \textbf{dæhten}\\ 
 & und zuo \textit{J}oflanze bræhten\\ 
25 & die massenîe \textbf{mit} vrowen schar;\\ 
 & und \textbf{er} \dag kam\dag  ouch selbe dar\\ 
 & durch kampf ûf al sîn êre.\\ 
 & er enbôt \textbf{in} dannoch mêre,\\ 
 & der kampf wær alsô genomen,\\ 
30 & daz er \textbf{werlîch} \textbf{müeste} komen.\\ 
\end{tabular}
\scriptsize
\line(1,0){75} \newline
m n o \newline
\line(1,0){75} \newline
\newline
\line(1,0){75} \newline
\textbf{1} Arniven] aruͯwen m arniwan n anuwe o \textbf{4} einen] eyen m \textbf{6} sarjandes] saraiandes m sariendes o  $\cdot$ prîse] psrise m \textbf{7} swuor] swuͯre n \textbf{8} würbe] warp o \textbf{9} des] das o  $\cdot$ dâ] do n \textbf{10} gewüege] Genuͯg m \textbf{11} dâ] do m n o \textbf{14} Lotes kint] lotzs suͯn vnd kint m lots kint n \textbf{16} Lover] lonor m loner n louer o \textbf{17} des] das m o \textbf{18} sînem] sin m \textbf{21} an werdicheit] an froͯden vnd werdikeit m \textbf{23} triuwe] truwen o \textbf{24} zuo Joflanze] zuͯchoflancze m zuͯ choflantze n zuͦ klofflancze o \textbf{26} er] \textit{om.} o  $\cdot$ dar] gegen in dar n (o) \textbf{30} werlîch] weltlich n \newline
\end{minipage}
\end{table}
\newpage
\begin{table}[ht]
\begin{minipage}[t]{0.5\linewidth}
\small
\begin{center}*G
\end{center}
\begin{tabular}{rl}
 & ze Arniven sprach Gawan:\\ 
 & "vrouwe, ich sol einen boten hân."\\ 
 & ein juncvrouwe wart gesant,\\ 
 & diu brâht \textbf{im} einen sarjant,\\ 
5 & manlîch, mit zühte\textit{n} wîse,\\ 
 & in sarjandes prîse.\\ 
 & der knappe \textbf{swuor des} einen eit,\\ 
 & er würbe liep oder leit,\\ 
 & daz er des niemen dâ\\ 
10 & \textbf{zuo} gewüege noch anderswâ,\\ 
 & wan \textbf{dâ} erz werben solde.\\ 
 & \textbf{er} bat, daz \textit{m}an im holde\\ 
 & tinten unde permint,\\ 
 & Gawan, des \textbf{künic} Lotes kint,\\ 
15 & schreip gevuoge mit der hant.\\ 
 & er enbôt ze Lover in daz lant\\ 
 & Artuse unde des wîbe\\ 
 & dienst von sînem lîbe\\ 
 & mit triuwen unverschert\\ 
20 & unde het \textbf{ir} brîs behert,\\ 
 & der wære an werdecheit tôt,\\ 
 & si\textbf{ne} hülfen \textit{im} ze sîner nôt,\\ 
 & daz si beide an triuwe \textbf{dæhten}\\ 
 & unde ze \dag Tschanfenzune\dag  bræhten\\ 
25 & die massenîe \textbf{mit} vrouwen schar\\ 
 & unde \textbf{k\textit{œ}men} ouch selbe \textbf{gein im} dar\\ 
 & durch kampf \textit{ûf} al\textit{le} si\textit{n} êre.\\ 
 & ernbôt \textbf{in} dannoch mêre,\\ 
 & der kampf wære alsô genomen,\\ 
30 & daz er \textbf{werdeclîche} \textbf{müese} komen.\\ 
\end{tabular}
\scriptsize
\line(1,0){75} \newline
G I L M Z Fr51 \newline
\line(1,0){75} \newline
\textbf{1} \textit{Initiale} Z  \textbf{15} \textit{Initiale} I  \textbf{17} \textit{Initiale} M  \newline
\line(1,0){75} \newline
\textbf{1} Arniven] arniuen I  $\cdot$ Gawan] her Gawan L (M) \textbf{2} sol] solt I \textbf{4} im] \textit{om.} L \textbf{6} in] An L \textbf{7} einen] ein ein I \textbf{8} würbe] vvurfe I  $\cdot$ liep] leb Fr51 \textbf{9} des] \textit{om.} Fr51 \textbf{10} zuo] \textit{om.} Z Ne Fr51 \textbf{11} wan] Mer Fr51  $\cdot$ dâ erz] daz er L \textbf{12} man] iman G \textbf{13} tinten] Jnchet Fr51 \textbf{14} des künic] \textit{om.} I kvnig L  $\cdot$ Lotes] lotis M \textbf{15} gevuoge] einen brief I \textit{om.} Fr51  $\cdot$ mit der] mit siner I mit sines selbes Fr51 \textbf{16} er enbôt] ern bot I  $\cdot$ ze Lover] zelouer G zelouers I zuͯ leove L zo lovers Fr51 \textbf{17} Artuse] Artus I Z Artuͯse L  $\cdot$ des] sinem Fr51 \textbf{19} unverschert] vnuerzert I \textbf{20} het ir] er hete I her er L (Fr51) \textbf{21} der] Das her Fr51  $\cdot$ wære] were e M \textbf{22} sine] Sy M  $\cdot$ hülfen] huͯlfen im L (M) (Fr51)  $\cdot$ ze] [an*]: an Z \textbf{23} beide] beidiu I  $\cdot$ dæhten] dachten L (M) \textbf{24} ze Tschanfenzune] zeschanfenzune G zeshanfanzune I zuͯ schoflanze L zcu schofflancze M Tschofflantze Z zo schonlance Fr51  $\cdot$ bræhten] brachten L (M) \textbf{26} kœmen] chomin G (I) (L) (M) (Fr51) koͤme Z  $\cdot$ selbe] selben M selber Fr51  $\cdot$ gein im] \textit{om.} I gein in Z \textbf{27} ûf alle sin] als si G \textbf{28} dannoch] noch Fr51 \textbf{30} müese] moste Fr51 \newline
\end{minipage}
\hspace{0.5cm}
\begin{minipage}[t]{0.5\linewidth}
\small
\begin{center}*T
\end{center}
\begin{tabular}{rl}
 & \begin{large}Z\end{large}uo Arnyven sprach Gawan:\\ 
 & "vrouwe, ich sol einen boten hân."\\ 
 & ein juncvrouwe wart gesant,\\ 
 & diu brâhte\textbf{n} einen sarjant,\\ 
5 & manlîch, mit zühten wîse,\\ 
 & in sarjandes prîse.\\ 
 & der knappe \textbf{des swuor} einen eit,\\ 
 & er würbe liep oder leit,\\ 
 & daz er des niemanne dâ\\ 
10 & \textbf{zuo} gewüege noch anderswâ,\\ 
 & wan \textbf{daz} er daz werben solte.\\ 
 & \textbf{er} bat, daz man im holte\\ 
 & tinten und permint,\\ 
 & Gawan, des \textbf{küneges} Lotes kint,\\ 
15 & schreip gevuoge mit der hant.\\ 
 & er enbôt zuo Lover in daz lant\\ 
 & Artuse und des wîbe\\ 
 & dienst von sîme lîbe\\ 
 & mit triuwen unverschertet;\\ 
20 & und hete\textbf{r} prîs behertet,\\ 
 & der wære an wirdecheit tôt,\\ 
 & si \textbf{en}hülfen im zuo sîner nôt,\\ 
 & daz si beide an triuwe \textbf{gedæhten}\\ 
 & und \textit{zuo} Tschoflanze bræhten\\ 
25 & die massenîe \textbf{von} vrouwen schar\\ 
 & und \textbf{k\textit{œ}men} ouch selbe \textbf{beidiu} \textbf{gein im} dar\\ 
 & durch kampf ûf alle sîn êre.\\ 
 & er enbôt \textbf{in} dannoch mêre,\\ 
 & der kampf wære alsô genomen,\\ 
30 & daz er \textbf{wirdeclîche} \textbf{muoze} komen.\\ 
\end{tabular}
\scriptsize
\line(1,0){75} \newline
U V W Q R Fr39 \newline
\line(1,0){75} \newline
\textbf{1} \textit{Initiale} U W Fr39   $\cdot$ \textit{Capitulumzeichen} R  \newline
\line(1,0){75} \newline
\textbf{1} Arnyven] arniuen V (Q) arnyue W [Arn*en]: Arniuen R arnyuen Fr39  $\cdot$ Gawan] herre gawan W \textbf{4} brâhten] [brahtent]: brahte V brachte im W (Q) (R) (Fr39)  $\cdot$ sarjant] stiriant Q \textbf{5} mit] vnd W \textbf{6} prîse] wise R \textbf{7} des swuor] swͦr dez V (W) (Q) (R) (Fr39) \textbf{8} würbe] [*]: wurbe V wurde R Fr39  $\cdot$ oder] vnd R \textbf{9} des] das R Fr39  $\cdot$ dâ] do W \textbf{10} gewüege] gefuͦge R \textbf{11} wan daz] wan do R  $\cdot$ er daz] er V ers W Q Fr39 er [*]: es R \textbf{12} im] in W [in]: im Q \textbf{13} tinten] Tinte V \textbf{14} Gawan] Gawin R  $\cdot$ küneges Lotes] kúnig lottes W \textbf{15} gevuoge] gefuͤge W  $\cdot$ mit] mir Fr39 \textbf{16} zuo] gen W  $\cdot$ Lover] louer W ware Q lofer R loͤver Fr39 \textbf{17} Artuse] Artusen R  $\cdot$ des] seinem W \textbf{20} heter] herter R \textbf{21} wirdecheit] wedickeit Fr39 \textbf{22} enhülfen] [haffent]: halffent R hulfen Fr39  $\cdot$ im] im dan R \textbf{23} beide] beidú R (Fr39)  $\cdot$ gedæhten] dechten W (Q) R (Fr39) \textbf{24} zuo Tschoflanze] Tschoflanze U ze schoflanze V zuͦ tschoflantze W zu tschoflanze Q ze schofflancz R zetschoflanze Fr39 \textbf{25} die] Dise W Dú R  $\cdot$ von] mit W Q R Fr39  $\cdot$ vrouwen] seiner W frewden Q \textbf{26} kœmen] comen U (Q) [*]: er  V kamen W  $\cdot$ selbe] \textit{om.} W  $\cdot$ beidiu] [*]: keme V \textit{om.} W Q R Fr39  $\cdot$ gein] \textit{om.} Q  $\cdot$ im] [*]: in V im selbe W \textit{om.} Q \textbf{27} ûf] vnd Q \textbf{28} dannoch] dannoch noch Q \textbf{29} wære] \textit{om.} W \textbf{30} muoze] mvͤste V (Fr39) sei W [moc*]: mochte Q soͯlte R \newline
\end{minipage}
\end{table}
\end{document}
