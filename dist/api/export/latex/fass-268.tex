\documentclass[8pt,a4paper,notitlepage]{article}
\usepackage{fullpage}
\usepackage{ulem}
\usepackage{xltxtra}
\usepackage{datetime}
\renewcommand{\dateseparator}{.}
\dmyyyydate
\usepackage{fancyhdr}
\usepackage{ifthen}
\pagestyle{fancy}
\fancyhf{}
\renewcommand{\headrulewidth}{0pt}
\fancyfoot[L]{\ifthenelse{\value{page}=1}{\today, \currenttime{} Uhr}{}}
\begin{document}
\begin{table}[ht]
\begin{minipage}[t]{0.5\linewidth}
\small
\begin{center}*D
\end{center}
\begin{tabular}{rl}
\textbf{268} & \begin{large}M\end{large}erke diu wort unt wis der werke \textbf{ein} wer.\\ 
 & des gib mir sicherheit al her."\\ 
 & dô sprach der \textbf{herzoge} Orilus\\ 
 & \textbf{z}em \textbf{künege} Parzival \textbf{alsus}:\\ 
5 & "mac niemen dâ vür niht gegeben,\\ 
 & sô leist ichz, wande ich wil noch leben."\\ 
 & durch \textbf{die} vorhte \textbf{von} ir man\\ 
 & vrou Jeschute, \textbf{diu} wolgetân,\\ 
 & \textbf{strît} scheidens gar verzagete,\\ 
10 & ir vîendes nôt si klagete.\\ 
 & Parzival in ûf verliez,\\ 
 & \textbf{der} vroun Jeschuten \textbf{suone} gehiez.\\ 
 & Der betwungene vürste sprach:\\ 
 & "vrouwe, sît \textbf{diz} durch iuch geschach,\\ 
15 & in strîte diu schumpfentiure mîn,\\ 
 & wol her, ir sult geküsset sîn.\\ 
 & ich hân vil prîses durch iuch verlorn.\\ 
 & waz denne? \textbf{ez} ist \textbf{doch} verkorn."\\ 
 & Diu vrouwe mit ir blôzem vel\\ 
20 & was ze\textbf{m} sprunge harte snel\\ 
 & von dem pferde ûf den wasen.\\ 
 & swie daz bluot von der nasen\\ 
 & den munt \textbf{im} hete gemachet rôt,\\ 
 & si kust in, dô er \textbf{kusses} gebôt.\\ 
25 & \textbf{Dâ} wart niht langer dô gebiten:\\ 
 & si bêde unt \textbf{ouch} diu vrouwe riten\\ 
 & \textbf{vür eine} klôsen in eines velses want.\\ 
 & eine kefsen Parzival dâ vant.\\ 
 & ein gemâlt sper dâr bî \textbf{dâ} lent.\\ 
30 & der einsidel, \textbf{der} hiez Trevrizent.\\ 
\end{tabular}
\scriptsize
\line(1,0){75} \newline
D \newline
\line(1,0){75} \newline
\textbf{1} \textit{Initiale} D  \textbf{13} \textit{Majuskel} D  \textbf{19} \textit{Majuskel} D  \textbf{25} \textit{Majuskel} D  \newline
\line(1,0){75} \newline
\textbf{8} Jeschute] Jescvte D \textbf{12} Jeschuten] Jescvten D \newline
\end{minipage}
\hspace{0.5cm}
\begin{minipage}[t]{0.5\linewidth}
\small
\begin{center}*m
\end{center}
\begin{tabular}{rl}
 & merke diu wort und wis der werke wer.\\ 
 & des g\textit{i}p mir sicherheit alher."\\ 
 & \begin{large}D\end{large}ô sprach der \textbf{vürste} Orilus\\ 
 & \textbf{z}em \textbf{künige} Parcifale \textbf{alsus}:\\ 
5 & "mac niemen dâ vür niht gegeben,\\ 
 & sô leist ichz, wand ich wil noch leben."\\ 
 & durch \textbf{die} vorhte \textbf{von} ir man\\ 
 & vrouwe Jeschute, \textbf{diu} wol getân,\\ 
 & \textbf{strît} scheidens gar verzagete,\\ 
10 & ir vîendes nôt si klagete.\\ 
 & Parcifal\textbf{s}  in ûf verliez,\\ 
 & \textbf{der} vrouwen Jeschuten \textbf{suone} gehiez.\\ 
 & der betwunge\textit{n} vürste sprach:\\ 
 & "vrouwe, sît \textbf{diz} durch iuch geschach,\\ 
15 & in strîte diu schumpfentiure mîn,\\ 
 & wol her, ir sullet geküsset sîn.\\ 
 & ich hân vil prîses durch iuch verlorn.\\ 
 & waz danne? \textbf{ez} ist \textbf{doch} verkorn."\\ 
 & diu vrouwe mit ir blôzem vel\\ 
20 & was \textit{zuo} \textbf{dem} sprunge harte snel\\ 
 & von dem pferde ûf den wasen.\\ 
 & wie daz bluot von der nasen\\ 
 & den munt \textbf{ime} hete gemachet rôt,\\ 
 & si küst in, dô er \textbf{kus} gebôt.\\ 
25 & \textbf{\begin{large}D\end{large}\textit{â}} wart niht langer dô gebiten:\\ 
 & si beide und diu vrouwe riten\\ 
 & \textbf{vür eine} klôsen in eines velses want.\\ 
 & eine kafsen Parcifal d\textit{â} vant.\\ 
 & ein gemâlet sper dâr bî \textbf{d\textit{â}} l\textit{e}nt.\\ 
30 & der einsidel hiez Trevrizent.\\ 
\end{tabular}
\scriptsize
\line(1,0){75} \newline
m n o Fr69 \newline
\line(1,0){75} \newline
\textbf{3} \textit{Initiale} m Fr69  \textbf{25} \textit{Initiale} m   $\cdot$ \textit{Capitulumzeichen} n  \newline
\line(1,0){75} \newline
\textbf{1} diu] der o  $\cdot$ wis] \textit{om.} Fr69 \textbf{2} gip] gap m n o \textbf{3} Orilus] orelus o orilu::: Fr69 \textbf{4} Parcifale] parcifal n [s]: parcifal o \textbf{5} niht] icht Fr69 \textbf{7} man] nam o \textbf{8} Jeschute] jescutte m jescute n (Fr69) jescuͯten o  $\cdot$ diu] \textit{om.} Fr69 \textbf{11} Parcifals] Parcifal n o \textbf{12} Jeschuten] Jescutten m jescuten n jescuͯten o \textbf{13} betwungen] betwunge m (o) \textbf{14} geschach] beschach n \textbf{17} prîses] pris o \textbf{19} ir blôzem] irem blossen n (o) \textbf{20} zuo] \textit{om.} m \textbf{24} küst] koste o  $\cdot$ dô er] er do n  $\cdot$ kus] kost o \textbf{25} Dâ] DO m (n) (o)  $\cdot$ dô] \textit{om.} n o \textbf{26} vrouwe] frowvͯen m \textbf{28} dâ] do m n o \textbf{29} dâ] do m n o  $\cdot$ lent] lont m \textbf{30} einsidel] eyn sẏdin o  $\cdot$ Trevrizent] Treurizent m (n) trinrecent o \newline
\end{minipage}
\end{table}
\newpage
\begin{table}[ht]
\begin{minipage}[t]{0.5\linewidth}
\small
\begin{center}*G
\end{center}
\begin{tabular}{rl}
 & \begin{large}M\end{large}erke diu wort unt wis der we\textit{r}ke \textbf{ein} wer.\\ 
 & des gip mir sicherheit al her."\\ 
 & dô sprach der \textbf{herzoge} Orillus\\ 
 & \textbf{gein} dem \textbf{künige} Parzivale \textbf{sus}:\\ 
5 & "mac niemen dâ vür niht gegeben,\\ 
 & sô leist ich ez, wan ich wil noch leben."\\ 
 & durch vorhte \textbf{von} ir man\\ 
 & vrou Jeschute wolgetân\\ 
 & \textbf{ir} \textit{schei}de\textit{n}s gar verzagte,\\ 
10 & ir vîndes nôt si klagte.\\ 
 & Parzival in ûf verliez,\\ 
 & \textbf{dôr} vroun Jeschuten \textbf{hulde} gehiez.\\ 
 & der betwungene vürste sprach:\\ 
 & "vrouwe, sît \textbf{diz} durch iuch geschach,\\ 
15 & in strîte diu schumpfentiure mîn,\\ 
 & wol her, ir sult geküsset sîn.\\ 
 & ich hân vil brîses durch iuch verloren.\\ 
 & waz danne? \textbf{ez} ist \textbf{iedoch} verkoren."\\ 
 & diu vrouwe mit ir blôzem vel\\ 
20 & was ze\textbf{m} sprunge harte snel\\ 
 & von dem pferde ûf den wasen.\\ 
 & swie daz bluo\textit{t} \textit{v}o\textit{n} der nasen\\ 
 & den munt \textit{\textbf{im}} het gemachet rôt,\\ 
 & si küst in, dô er \textbf{kus} gebôt.\\ 
25 & \textbf{ez} wart niht lenger dô gebiten:\\ 
 & si bêde unde \textbf{ouch} diu vrouwe riten\\ 
 & \textbf{zeiner} klôsen in eines velses want.\\ 
 & eine kefsen Parzival dâ vant.\\ 
 & ein gemâlt sper dâr bî \textbf{dâ} lent.\\ 
30 & der einsidel hiez Trevrizzent.\\ 
\end{tabular}
\scriptsize
\line(1,0){75} \newline
G I O L M Q R Z Fr21 \newline
\line(1,0){75} \newline
\textbf{1} \textit{Initiale} G  \textbf{3} \textit{Initiale} I L R  \textbf{11} \textit{Initiale} I  \textbf{25} \textit{Initiale} L  \textbf{27} \textit{Initiale} O Z  \newline
\line(1,0){75} \newline
\textbf{1} werke] weche G  $\cdot$ ein] \textit{om.} L Q \textbf{2} gip] gibt Q  $\cdot$ mir] mir din Fr21  $\cdot$ al her] daher L (Q) her Fr21 \textbf{3} dô] Da M Z  $\cdot$ Orillus] Orilus I (O) (M) (Q) (R) (Z) (Fr21) \textbf{4} gein dem] Zedem O (L) (M) (Q) (R) (Z) (Fr21)  $\cdot$ Parzivale] [parzifal]: Parzifal I Parcifal O (Z) (Fr21) parcifale L parzifale M partzifale Q parczifal R  $\cdot$ sus] alsvs O (M) (R) (Z) Fr21 \textbf{5} niemen] ýeman L  $\cdot$ niht] ýht L \textbf{6} sô leist ich ez] Daz leiste ich L  $\cdot$ wil noch] noch wil Q wolt nach Z \textbf{7} \textit{Vers 268.7 fehlt} L   $\cdot$ vorhte] die forcht Q (Z)  $\cdot$ von] an M vor Q \textbf{8} \textit{nach 268.8:} Die waz alz ich mich verstan L   $\cdot$ Jeschute] ieschute G ieskute I Jeschvͦte O Jescuͯte L iescute M Z Jescute Q Jscute R \textbf{9} ir scheidens] ir frivndes G Jrs scheidins M Strit scheidens Z  $\cdot$ verzagte] verzagt O (L) gedachte Q getagte R \textbf{10} vîndes] weydes Q  $\cdot$ klagte] chlagt O (L) \textbf{11} Parzival] Parzifal I L M Parcifal O Q Z Parczifal R  $\cdot$ verliez] liez Z \textbf{12} dôr] Da her M (Z)  $\cdot$ vroun] vrou L (M) (Q) (R)  $\cdot$ Jeschuten] ieschvten G iekuten I Jeschvͦten O Jescuͯten L iescuten M Z Jescuten Q Jscuten R  $\cdot$ hulde] svͦn O (L) (M) (Q) (R) (Z)  $\cdot$ gehiez] [verliez]: gehiez I verhies R \textbf{13} betwungene] betungen R \textbf{14} diz] \textit{om.} Q R \textbf{15} schumpfentiure] entschvmpfntiwer O auentúre R  $\cdot$ mîn] hie R \textbf{16} geküsset sîn] kússen ye R \textbf{17} hân] \textit{om.} Z  $\cdot$ iuch] evch han Z \textbf{18} danne] da von L  $\cdot$ iedoch] doch I \textit{om.} L  $\cdot$ verkoren] verlorn L \textbf{19} ir blôzem] ir bloszen L (M) (R) (Z)  $\cdot$ vel] velt I \textbf{21} von] vf I  $\cdot$ den] dem I de R \textbf{22} swie] Wie L (Q) R Z  $\cdot$ bluot von] bloͮt im vor G blvt vz Z  $\cdot$ der] den L \textbf{23} den] Dem Q  $\cdot$ im] \textit{om.} G  $\cdot$ het] hat R \textbf{24} Sie kust in da er ir kussen bot Z  $\cdot$ küst] kvste O (L) (M)  $\cdot$ dô] da M Z  $\cdot$ er kus] hers M \textbf{25} ez] ezn I Zu Q Da Z  $\cdot$ niht] \textit{om.} Q  $\cdot$ dô] da I O L M Z  $\cdot$ gebiten] gibeten M \textbf{26} ouch] \textit{om.} L Z \textbf{27} zeiner] ÷einer O  $\cdot$ in] an I  $\cdot$ velses want] velschin hant M \textbf{28} \textit{statt 268.28:} Eine kefsen er da fand / parczifal steken by der wand R   $\cdot$ kefsen] kasse L  $\cdot$ Parzival] parzifal I M Parcifal O (L) (Q) (Z)  $\cdot$ dâ] do O Q \textbf{29} dâr bî dâ] da bi I (M) da L do bey do Q \textbf{30} einsidel] ein sidelen Q  $\cdot$ Trevrizzent] treverezent G Treurezent I Trevezent O Treviszent L trevrezent M trefrizent Q Trefirzent R Treverzzent Z \newline
\end{minipage}
\hspace{0.5cm}
\begin{minipage}[t]{0.5\linewidth}
\small
\begin{center}*T
\end{center}
\begin{tabular}{rl}
 & merke diu wort unde wis der werke \textbf{ein} wer.\\ 
 & des gip mir sicherheit alher."\\ 
 & Dô sprach der \textbf{herzoge} Orilus\\ 
 & \textbf{z}em \textbf{küenen} Parcifale \textbf{sus}:\\ 
5 & "mac niemen dâ vür niht gegeben,\\ 
 & sô leist ichz, wan ich wil noch leben."\\ 
 & \begin{large}D\end{large}urch \textbf{die} vorhte ir man\\ 
 & vrou Jeschute, \textbf{diu} wol getân,\\ 
 & \textbf{strît} scheidens gar verzagete,\\ 
10 & ir vîendes nôt si klagete.\\ 
 & Parcifal in ûf verliez,\\ 
 & \textbf{dô er} vroun Jeschuten \textbf{suone} gehiez.\\ 
 & Der betwungene vürste sprach:\\ 
 & "vrouwe, sît durch iuch geschach\\ 
15 & in strîte diu schumpfentiure mîn,\\ 
 & wol her, ir sult geküsset sîn.\\ 
 & ich hân vil prîses durch iuch verlorn.\\ 
 & waz danne? \textbf{daz} ist \textbf{nû} verkorn."\\ 
 & Diu vrouwe mit ir blôzem vel\\ 
20 & was z\textbf{einem} sprunge harte snel\\ 
 & von dem pferde ûf den wasen.\\ 
 & swie \textbf{im} daz bluot von der nasen\\ 
 & den munt hete gemachet rôt,\\ 
 & si kustin, dô er \textbf{kus} gebôt.\\ 
25 & \textbf{Dô}\textbf{ne} wart niht lenger dâ gebiten:\\ 
 & si beide unde \textbf{ouch} di\textit{u} vrouwe riten\\ 
 & \textbf{zeiner} klôsen in eines velses want.\\ 
 & eine kafsen Parcifal dâ vant.\\ 
 & ein gemâlet sper dâ bî lente.\\ 
30 & der einsidel hiez Trefrizente.\\ 
\end{tabular}
\scriptsize
\line(1,0){75} \newline
T U V W \newline
\line(1,0){75} \newline
\textbf{3} \textit{Majuskel} T  \textbf{7} \textit{Initiale} T  \textbf{13} \textit{Initiale} U   $\cdot$ \textit{Majuskel} T  \textbf{19} \textit{Majuskel} T  \textbf{25} \textit{Majuskel} T  \newline
\line(1,0){75} \newline
\textbf{1} wis] sist U \textit{om.} W  $\cdot$ der werke ein wer] der bete ein wer U der [*]: werke wer V des wercks wer W \textbf{2} des] [Der]: Des U  $\cdot$ gip] gab W \textbf{4} küenen] [*]: kvͤnen V iungen W  $\cdot$ Parcifale] parzifale T Parcifal U parzifal V partzifal W  $\cdot$ sus] alsvs V (W) \textbf{5} dâ vür niht] icht do fúr W \textbf{6} noch] \textit{om.} W \textbf{7} ir] [*]: von ir V von irm W \textbf{8} vrou] Vrowen V  $\cdot$ Jeschute] Jescvte T (U) Jescuten V iestute W \textbf{10} ir] Jrs U (W) \textbf{11} Parcifal] Parzifal T V Partzifal W  $\cdot$ verliez] do ließ W \textbf{12} vroun] vreuͦwe U (W)  $\cdot$ Jeschuten] Jescvten T (U) iescuten V iestuten W  $\cdot$ suone gehiez] schone gehiez U suͦn verhies W \textbf{13} betwungene] betwuͦngener U \textbf{14} sît] sit diz U sit [d*]: diz V seit daz W  $\cdot$ iuch] îv T \textbf{15} schumpfentiure] entschumpfenteúre W \textbf{17} prîses] preis W  $\cdot$ iuch] iv T \textbf{18} waz] Wes W  $\cdot$ nû] \textit{om.} U doch V  $\cdot$ verkorn] verlorn U \textbf{22} swie] Wie U W \textbf{25} Dône wart niht] Danne wart nit U Do ward W  $\cdot$ dâ] do U V nit W \textbf{26} diu] die T \textbf{27} klôsen] klose W \textbf{28} kafsen] kafse U V (W)  $\cdot$ Parcifal] parzifal T V partzifal W  $\cdot$ dâ] do V W \textbf{29} dâ bî] do U [*]: dar bi da V  $\cdot$ lente] lent W \textbf{30} einsidel] [einsidel*]: einsidele V  $\cdot$ Trefrizente] Trefricente U trefrischente V treuerißent W \newline
\end{minipage}
\end{table}
\end{document}
