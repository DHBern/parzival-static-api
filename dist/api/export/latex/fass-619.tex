\documentclass[8pt,a4paper,notitlepage]{article}
\usepackage{fullpage}
\usepackage{ulem}
\usepackage{xltxtra}
\usepackage{datetime}
\renewcommand{\dateseparator}{.}
\dmyyyydate
\usepackage{fancyhdr}
\usepackage{ifthen}
\pagestyle{fancy}
\fancyhf{}
\renewcommand{\headrulewidth}{0pt}
\fancyfoot[L]{\ifthenelse{\value{page}=1}{\today, \currenttime{} Uhr}{}}
\begin{document}
\begin{table}[ht]
\begin{minipage}[t]{0.5\linewidth}
\small
\begin{center}*D
\end{center}
\begin{tabular}{rl}
\textbf{619} & \begin{large}D\end{large}ô er die mîne überstreit,\\ 
 & nâch dem helde ich selbe reit.\\ 
 & ich bôt im lant unt mînen lîp.\\ 
 & er sprach, \textbf{er hete} ein \textbf{schœner} wîp\\ 
5 & unt diu im lieber wære.\\ 
 & diu rede \textbf{was} mir swære.\\ 
 & \textbf{Ich} vrâgete, wer diu m\textit{ö}hte sîn.\\ 
 & 'von Pelrapeire diu künegîn,\\ 
 & sus ist genant diu lieht gemâl.\\ 
10 & sô heize ich selbe Parzival.\\ 
 & \textbf{ich} wil iwer minne niht,\\ 
 & der Grâl mir anders kumbers giht',\\ 
 & \textbf{sus} sprach der helt mit zorne.\\ 
 & hin reit der ûzerkorne.\\ 
15 & hân ich dâr an missetân,\\ 
 & welt ir mich daz wizzen lân,\\ 
 & ob ich durch \textbf{mîne herzenôt}\\ 
 & dem werdem rîter minne bôt,\\ 
 & \textbf{sô} krenket sich mîn minne."\\ 
20 & Gawan zer herzoginne\\ 
 & sprach: "\textbf{vrouwe}, ich erkenne in alsô wert,\\ 
 & an \textbf{dem} ir minne hât \textbf{gegert},\\ 
 & het er iuch ze minne erkorn,\\ 
 & iwer prîs wære an im \textbf{\textit{niht} verlorn}."\\ 
25 & Gawan, der kurtoys,\\ 
 & unt \textbf{diu herzoginne} \textbf{von} Logroys\\ 
 & vaste an ein ander sâhen.\\ 
 & \textbf{dô riten si sô} nâhen,\\ 
 & daz man si von der burg \textbf{ersach},\\ 
30 & dâ \textbf{im} diu âventiure geschach.\\ 
\end{tabular}
\scriptsize
\line(1,0){75} \newline
D Z Fr68 \newline
\line(1,0){75} \newline
\textbf{1} \textit{Initiale} D Z Fr68  \textbf{7} \textit{Majuskel} D  \newline
\line(1,0){75} \newline
\textbf{1} Dô] Da Z  $\cdot$ mîne] minen Z \textbf{3} ich] vnde Fr68  $\cdot$ mînen] \textit{om.} Fr68 \textbf{6} was] wart Z \textbf{7} Ich vrâgete] do fractih Fr68  $\cdot$ möhte] mohte D Z Fr68 \textbf{10} Parzival] Parcifal D (Z) partsiual Fr68 \textbf{11} wil] enwil Z \textbf{17} mîne herzenôt] mines hertzen not Z \textbf{18} werdem] werden Z Fr68  $\cdot$ minne] minnen Fr68 \textbf{19} Krenket sich dar an min minne Z \textbf{20} Gawan] er sprah Fr68 \textbf{21} sprach] \textit{om.} Fr68  $\cdot$ vrouwe] \textit{om.} Z \textbf{22} dem] den Z \textbf{23} ze] zvr Z  $\cdot$ minne] minnen Fr68 \textbf{24} niht verlorn] verlorn D v:::uerlorn Fr68 \textbf{26} diu] de D [*]: der Fr68  $\cdot$ Logroys] Logrois Z (Fr68) \textbf{27} an] \textit{om.} Fr68 \textbf{28} dô] Da Z \textbf{29} ersach] sach Z \textbf{30} dâ] daz Fr68 \newline
\end{minipage}
\hspace{0.5cm}
\begin{minipage}[t]{0.5\linewidth}
\small
\begin{center}*m
\end{center}
\begin{tabular}{rl}
 & dô er die mîne überstreit,\\ 
 & nâch dem helde ich selbe reit.\\ 
 & ich bôt im lant und mînen lîp.\\ 
 & er sprach, \textbf{er het} ein \textbf{schœnez} wîp\\ 
5 & und diu im lieber wære.\\ 
 & diu rede \textbf{was} mir swære.\\ 
 & \textbf{ich} vrâgte, wer diu m\textit{ö}hte sîn.\\ 
 & 'von Pelraperie diu künigîn,\\ 
 & sus ist genant diu lieht gemâl.\\ 
10 & sô h\textit{ei}z\textit{e} ich selbe Parcifal\\ 
 & \textbf{und} wil iuwer minne niht.\\ 
 & der Grâl mir anders kumbers giht',\\ 
 & sprach der helt mit zorne.\\ 
 & hin reit der ûzerkorne.\\ 
15 & hab ich dâr an missetân,\\ 
 & wol\textit{t} i\textit{r} mich daz wizzen lân,\\ 
 & ob ich \textit{durch} \textbf{mîn herzenôt}\\ 
 & dem werden ritter minne bôt,\\ 
 & \textbf{sô} krenket sich mîn minne."\\ 
20 & Gawa\textit{n} zuor herzog\textit{i}nne\\ 
 & sprach: "\textbf{vrowe}, ich erkenn\textit{e} in alsô wert,\\ 
 & an \textbf{den} ir minne habt \textbf{begert},\\ 
 & het er iuch zuo minne erkorn,\\ 
 & iuwer prîs wær an im \textbf{unverlorn}."\\ 
25 & \begin{large}G\end{large}awan, der kurtois,\\ 
 & und \textbf{U\textit{r}geluse} \textbf{de} Logrois\\ 
 & vast an ein ander sâhen,\\ 
 & \textbf{\textit{dô si} geriten alsô} nâhen,\\ 
 & daz man si von der burc \textbf{ersach},\\ 
30 & dâ\textbf{n} diu âventiur geschach.\\ 
\end{tabular}
\scriptsize
\line(1,0){75} \newline
m n o \newline
\line(1,0){75} \newline
\textbf{25} \textit{Initiale} m   $\cdot$ \textit{Capitulumzeichen} n  \newline
\line(1,0){75} \newline
\textbf{2} selbe] selber n \textbf{6} mir] min o \textbf{7} möhte] mohtte m \textbf{8} Pelraperie] pelrapier n o \textbf{9} genant] gemacht o  $\cdot$ gemâl] gamal o \textbf{10} heize] hies m o  $\cdot$ selbe] selbes n \textbf{12} Grâl] grole n  $\cdot$ kumbers] komber n \textbf{13} mit] mir o \textbf{16} wolt] Wol m  $\cdot$ ir] ich m o  $\cdot$ daz] des n \textbf{17} durch] \textit{om.} m \textbf{20} Gawa zur [herczogninne]: herczogiynne m \textbf{21} erkenne] erkennen m (o) \textbf{24} im] uͯch o \textbf{25} kurtois] torkois n túrkeis o \textbf{26} Urgeluse] vregeluse m vrgelúse o  $\cdot$ Logrois] logreis o \textbf{28} dô si] Sẏ do m  $\cdot$ alsô] so n [*]: so o \newline
\end{minipage}
\end{table}
\newpage
\begin{table}[ht]
\begin{minipage}[t]{0.5\linewidth}
\small
\begin{center}*G
\end{center}
\begin{tabular}{rl}
 & dô er die mîn überstreit,\\ 
 & nâch dem helde ich selbe reit.\\ 
 & ich bôt im lant unde mînen lîp.\\ 
 & er sprach: '\textbf{ich hân} ein \textbf{schœner} wîp',\\ 
5 & unde diu ime lieber wære.\\ 
 & diu rede \textbf{wart} mir swære.\\ 
 & \textbf{ich} vrâget, wer diu m\textit{ö}hte sîn.\\ 
 & 'von Pelrapeire diu künegîn,\\ 
 & sus ist genant diu lieht gemâl.\\ 
10 & sô h\textit{ei}z ich selbe Parzival.\\ 
 & \textbf{ich}\textbf{ne} wil iuwer minne niht,\\ 
 & der Grâl mir anders kumbers giht',\\ 
 & \textbf{\begin{large}S\end{large}us} sprach der helt mit zorne.\\ 
 & hin reit der ûzerkorne.\\ 
15 & hân ich dâr an missetân,\\ 
 & welt ir mich daz wizzen lân,\\ 
 & ob ich durch \textbf{mînes herzen nôt}\\ 
 & dem werden rîter minne bôt,\\ 
 & krenket sich \textbf{dar an} mîn minne."\\ 
20 & Gawan zer herzoginne\\ 
 & sprach: "ich erkenne in als wert,\\ 
 & an \textbf{dem} ir minne habet \textbf{gegert},\\ 
 & het er iuch ze minnen erkorn,\\ 
 & iuwer prîs wære an im \textbf{niht verlorn}."\\ 
25 & Gawan, der kurtoys,\\ 
 & unde \textbf{diu herzoginne} \textbf{von} Logroys\\ 
 & vaste an ein ander sâhen.\\ 
 & \textbf{dô riten si sô} nâhen,\\ 
 & daz man si von der bürge \textbf{sach},\\ 
30 & dâ \textbf{im} diu âventiure geschach.\\ 
\end{tabular}
\scriptsize
\line(1,0){75} \newline
G I L M Z \newline
\line(1,0){75} \newline
\textbf{1} \textit{Initiale} I L Z  \textbf{13} \textit{Initiale} G  \textbf{25} \textit{Initiale} M  \newline
\line(1,0){75} \newline
\textbf{1} dô] Da M Z  $\cdot$ mîn] mýnen L (M) \textbf{2} selbe] selben M \textbf{3} mînen] \textit{om.} L \textbf{4} ich hân] er het Z \textbf{6} wart] was I \textbf{7} vrâget] fragt in I  $\cdot$ möhte] mohte G I (L) (M) Z \textbf{8} Pelrapeire] pailrapeir I pelrapere M \textbf{9} lieht gemâl] lýcht gemal L lichtegemal M \textbf{10} heiz] hiez G  $\cdot$ selbe] selben M  $\cdot$ Parzival] [parzifal]: Parzifal I (M) parcifal Z \textbf{11} ichne] ich I (L)  $\cdot$ wil] wel M \textbf{12} giht] icht M \textbf{13} helt mit zorne] wol geborne L \textbf{14} Vnd reit hin mit zorne L \textbf{18} werden] werdem I \textbf{19} mîn] \textit{om.} L \textbf{20} Gawan] gawa G \textbf{22} an dem] an den I (L) (Z) andē M \textbf{23} ze minnen] zuͯ mýnne L (M) zvr minne Z \textbf{24} niht verlorn] vnvorlorn M \textbf{25} Gawan] GAuwan M  $\cdot$ kurtoys] Turtoys I \textbf{26} Logroys] logroẏs G logroýs L logis M \textbf{28} dô] Da M \textbf{30} im] nv L \newline
\end{minipage}
\hspace{0.5cm}
\begin{minipage}[t]{0.5\linewidth}
\small
\begin{center}*T
\end{center}
\begin{tabular}{rl}
 & \begin{large}D\end{large}ô er die mîne überstreit,\\ 
 & nâch dem helde ich selbe reit.\\ 
 & ich bôt im lant und mînen lîp.\\ 
 & er sprach: '\textbf{ich hân} ein \textbf{schœner} wîp',\\ 
5 & und diu im lieber wære.\\ 
 & diu rede \textbf{wart} mir swære\\ 
 & \textbf{und} vrâgete, wer diu m\textit{ö}hte sîn.\\ 
 & 'von Peilrapere diu künegîn,\\ 
 & sus ist genant diu lieht gemâl.\\ 
10 & sô heiz ich selbe Parcifal.\\ 
 & \textbf{ich} \textbf{en}wil iuwer minne niht,\\ 
 & der Grâl mir anders kumbers giht',\\ 
 & \textbf{sus} sprach der helt mit zorne.\\ 
 & hin reit der ûzerkorne.\\ 
15 & hân ich dâr an missetân,\\ 
 & wolt ir mich daz wizzen lân,\\ 
 & ob ich durch \textbf{mînes herzen nôt}\\ 
 & dem werden rîter minne bôt,\\ 
 & krenket sich \textbf{dar an} mîne minne."\\ 
20 & Gawa\textit{n} \textit{z}uo der herzoginne\\ 
 & sprach: "ich erkenne in als wert,\\ 
 & an \textbf{dem} ir minne hât \textbf{gegert},\\ 
 & heter iuch zuo minne erkorn,\\ 
 & iuwer prîs wære an im \textbf{niht verlorn}."\\ 
25 & \begin{large}G\end{large}awan, der kurtois,\\ 
 & und \textbf{\textit{diu} herzoginne} \textbf{von} Logrois\\ 
 & vaste an ein ander sâhen.\\ 
 & \textbf{dô riten si sô} nâhen,\\ 
 & daz man si von der bürge \textbf{sach},\\ 
30 & dâ \textbf{im} diu âventiure geschach.\\ 
\end{tabular}
\scriptsize
\line(1,0){75} \newline
U V W Q R Fr39 \newline
\line(1,0){75} \newline
\textbf{1} \textit{Initiale} U W Q Fr39   $\cdot$ \textit{Capitulumzeichen} R  \textbf{25} \textit{Initiale} U V  \newline
\line(1,0){75} \newline
\textbf{1} mîne] meinen W minne Fr39 \textbf{2} selbe] selber V W R \textbf{4} ich hân] [*]: er hette V \textbf{6} wart] bracht W waz R \textbf{7} diu] die Fr39  $\cdot$ möhte] mochte U Q (Fr39) \textbf{8} Peilrapere] Peilrapeir U peilraper V pelrapeir W pelrapeire Q Fr39 pelrapiere R \textbf{9} sus] als Q  $\cdot$ lieht] licht Q \textbf{10} sô] Do Q  $\cdot$ selbe] selber V W  $\cdot$ Parcifal] Parzafal U partzifal W Q parczifal R \textbf{11} enwil] wil R Fr39 \textbf{12} anders] ander W [ander]: anders R  $\cdot$ kumbers] kumber W \textbf{13} sus] Als Q  $\cdot$ helt] \textit{om.} W \textbf{19} mîne] \textit{om.} W Q \textbf{20} Gawan sprach zuͦ der herzoginne U  $\cdot$ Gawin zer herczogine R \textbf{21} als] alsus Q \textbf{22} dem] den W Q R Fr39 \textbf{24} niht verlorn] vnuerlorn W \textbf{25} Gawan] Her gawan W Gawin R \textbf{26} diu] \textit{om.} U  $\cdot$ Logrois] Logroys U (V) logroyß W logroẏs Q logris R \textbf{27} ander] andren R \textbf{28} [*]: Do ritten sv́ so nahen V \textbf{29} sach] [*]: sach V \textbf{30} dâ] Do V W Q R (Fr39)  $\cdot$ im] \textit{om.} Q In R  $\cdot$ geschach] gach W \newline
\end{minipage}
\end{table}
\end{document}
