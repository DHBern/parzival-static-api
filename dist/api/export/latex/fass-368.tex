\documentclass[8pt,a4paper,notitlepage]{article}
\usepackage{fullpage}
\usepackage{ulem}
\usepackage{xltxtra}
\usepackage{datetime}
\renewcommand{\dateseparator}{.}
\dmyyyydate
\usepackage{fancyhdr}
\usepackage{ifthen}
\pagestyle{fancy}
\fancyhf{}
\renewcommand{\headrulewidth}{0pt}
\fancyfoot[L]{\ifthenelse{\value{page}=1}{\today, \currenttime{} Uhr}{}}
\begin{document}
\begin{table}[ht]
\begin{minipage}[t]{0.5\linewidth}
\small
\begin{center}*D
\end{center}
\begin{tabular}{rl}
\textbf{368} & \begin{large}L\end{large}yppaut, \textbf{der vürste}, \textbf{al} vaste bat.\\ 
 & "hêrre, durch got die rede lât",\\ 
 & \textbf{sus} sprach des \textbf{künec} Lotes sun.\\ 
 & "durch iwer zuht sult ir daz tuon\\ 
5 & unt lât mich triwe niht enbern.\\ 
 & eines dinges \textbf{wil ich} iuch gewern:\\ 
 & ich sage iu hînte bî dirre naht,\\ 
 & wes ich mich drumbe hân bedâht."\\ 
 & Lyppaut im dankte unt vuor zehant.\\ 
10 & \textbf{ame} hove er sîne tohter vant\\ 
 & unt des burcgrâven töhterlîn;\\ 
 & diu zwei, \textbf{diu} snalten vingerlîn.\\ 
 & Dô sprach er Obilote zuo:\\ 
 & "tohter, wannen kumstû?"\\ 
15 & "vater, ich var dâ nider her.\\ 
 & ich getrûwe \textbf{im} wol, daz er mich\textbf{s} \textbf{gewer}.\\ 
 & ich wil den vremden ritter biten\\ 
 & dienstes nâch lônes siten."\\ 
 & "Tohter, sô sî dir geklagt,\\ 
20 & er\textbf{n} hât mir \textbf{an noch ab} gesagt.\\ 
 & kum mîner bete anz ende nâch."\\ 
 & der meide was zem gaste gâch.\\ 
 & dô si \textbf{in die} kemenâten gienc,\\ 
 & Gawan sprang ûf, \textbf{dô er} si enpfienc.\\ 
25 & zuo der süezen er dô saz.\\ 
 & \textbf{er} dankte ir, daz si niht vergaz\\ 
 & \textbf{sîn}, dâ man im \textbf{missebôt}.\\ 
 & er sprach: "geleit ie ritter nôt\\ 
 & durch ein \textbf{sus} \textbf{wênec} vröuwelîn,\\ 
30 & dâ solt ich durch iu\textit{ch} inne sîn."\\ 
\end{tabular}
\scriptsize
\line(1,0){75} \newline
D Fr3 Fr4 \newline
\line(1,0){75} \newline
\textbf{1} \textit{Initiale} D Fr4  \textbf{9} \textit{Initiale} Fr4  \textbf{13} \textit{Majuskel} D  \textbf{19} \textit{Majuskel} D  \textbf{25} \textit{Initiale} Fr3 Fr4  \newline
\line(1,0){75} \newline
\textbf{1} Lyppaut] Lyppaot D Lippaoth Fr4 \textbf{3} künec] kunigis Fr4  $\cdot$ Lotes] Lots D lothis Fr4 \textbf{9} Lyppaut] Lyppaot D Lippaoth Fr4 \textbf{10} er] \textit{om.} Fr4 \textbf{13} Obilote] obẏlotin Fr4 \textbf{16} michs] mich iz Fr4 \textbf{23} kemenâten] :::nate Fr3 \textbf{24} Gawan si mit zvcht entfinc Fr3  $\cdot$ sprang ûf] sprach Fr4 \textbf{29} vröuwelîn] vro Fr3 \textbf{30} iuch] iu D \newline
\end{minipage}
\hspace{0.5cm}
\begin{minipage}[t]{0.5\linewidth}
\small
\begin{center}*m
\end{center}
\begin{tabular}{rl}
 & Lippo\textit{u}t, \textbf{der vürste}, \textbf{al} vaste bat.\\ 
 & "hêrre, durch got die rede lât",\\ 
 & \textbf{sus} sprach des \textbf{küniges} Lotes sun.\\ 
 & "durch iuwer zuht sullet ir daz \textit{tuon}\\ 
5 & und lât mich triuwe niht enbern.\\ 
 & eines dinges \textbf{lât mich} iuch gewern:\\ 
 & ich sage iu hînt bî dirre naht,\\ 
 & wes ich mich dâr umb hân bedâht."\\ 
 & Lippo\textit{u}t ime dankte und vuor zehant.\\ 
10 & \textbf{a\textit{m}e} hove er sîne tohter vant\\ 
 & unt des burcgrâven töhterlîn;\\ 
 & diu zwei, \textbf{diu} snalten vingerlîn.\\ 
 & dô sprach er Obilote zuo:\\ 
 & "tohter, wannen kumest dû?"\\ 
15 & "vater, ich var dâ nider her.\\ 
 & ich getrûwe wol, daz er mich\textbf{s} \textbf{gewer}.\\ 
 & ich wil den vrömden ritter biten\\ 
 & dienstes nâch lônes siten."\\ 
 & "tohter, sô sî dir \textit{ge}klaget,\\ 
20 & er \textit{hât} mir \textbf{an noch abe} gesaget.\\ 
 & kum mîner bete \textit{a}nz ende nâch."\\ 
 & der megde was zem gaste gâch.\\ 
 & dô si \textbf{in die} kemenâten gienc,\\ 
 & Gawan spranc ûf, \textbf{dô er} si enpfienc.\\ 
25 & zuo der süezen er dô saz.\\ 
 & \textbf{er} dankete ir, daz si niht vergaz\\ 
 & \textbf{sîn}, dô ma\textit{n} ime \textbf{missetât}.\\ 
 & er sprach: "geleit ie ritter nôt\\ 
 & durch ein \textbf{sô} \textbf{wênic} vröuwelîn,\\ 
30 & dâ solt ich durch i\textit{uch in}ne sîn."\\ 
\end{tabular}
\scriptsize
\line(1,0){75} \newline
m n o \newline
\line(1,0){75} \newline
\newline
\line(1,0){75} \newline
\textbf{1} Lippout] Lippoat m Lippaot n Lipoot o \textbf{3} Lotes] lots m luchs n Loths o \textbf{4} tuon] \textit{om.} m \textbf{5} triuwe] truwen n \textbf{6} iuch] \textit{om.} o \textbf{9} Lippout] Lippoat m Lippaot n Lipaot o  $\cdot$ vuor zehant] gar zuͦ o \textbf{10} ame] Jn dem n o \textbf{11} burcgrâven] [b*]: buͯrgerefferin o \textbf{12} zwei] zwie o  $\cdot$ diu snalten] snelten n o \textbf{13} Obilote] obiloten n abeloten o \textbf{15} nider] niden o \textbf{16} er michs] ich michez o \textbf{18} dienstes] Dienest o \textbf{19} sô] [do]: so n  $\cdot$ geklaget] claget m \textbf{20} hât] \textit{om.} m \textbf{21} mîner] niemer o  $\cdot$ anz] vns m \textbf{26} dankete] dangt n (o) \textbf{27} man] mans m \textbf{28} sprach geleit] [spech]: sprech o  $\cdot$ nôt] nat n \textbf{29} durch] [Da]: Durch m  $\cdot$ vröuwelîn] jungfrowelin n (o) \textbf{30} dâ] Do n o  $\cdot$ iuch inne] yr mine m \newline
\end{minipage}
\end{table}
\newpage
\begin{table}[ht]
\begin{minipage}[t]{0.5\linewidth}
\small
\begin{center}*G
\end{center}
\begin{tabular}{rl}
 & Libaut, \textit{\textbf{der vürste}}, \textbf{al} vaste bat.\\ 
 & "hêrre, durch got die rede lât",\\ 
 & sprach des \textbf{künic} Lotes sun.\\ 
 & "durch iwer zuht sult ir daz tuon\\ 
5 & unde lât mich triwe niht enberen.\\ 
 & eines dinges \textbf{wil ich} iuch geweren:\\ 
 & ich sage iu hînt bî dirre naht,\\ 
 & wes ich mich drumbe hân bedâht."\\ 
 & \textit{Libaut} \textit{im dankte} unde vuor zehant.\\ 
10 & \textbf{ûf dem} hover sîne tohter vant\\ 
 & unt des burcgrâven töhterlîn;\\ 
 & diu zwei snalten vingerlîn.\\ 
 & dô sprach er Obilote zuo:\\ 
 & "tohter, wannen kumest dû?"\\ 
15 & "vater, ich var dâ nider her.\\ 
 & ich getrûwe \textbf{im} wol, daz er mich \textbf{wer}.\\ 
 & ich wil den vrömeden rîter biten\\ 
 & dienstes nâch lônes siten."\\ 
 & "tohter, sô sî dir geklaget,\\ 
20 & er\textbf{ne} hât mir \textbf{abe noch ane} gesaget.\\ 
 & kum mîner bete anz ende nâch."\\ 
 & der meide was zem gaste gâch.\\ 
 & \begin{large}D\end{large}ô si \textbf{in die} kemenâten gienc,\\ 
 & Gawan spranc ûf, \textbf{dô er} si enpfienc.\\ 
25 & zuo der süezen er dô saz.\\ 
 & \textbf{er} dankt ir, daz si niht vergaz\\ 
 & \textbf{sîn}, dô man im \textbf{missebôt}.\\ 
 & er sprach: "geleit ie rîter nôt\\ 
 & durch ein \textbf{\textit{al}s\textit{us}} \textbf{kleine} vröuwelîn,\\ 
30 & dâ solt ich \textit{durch} iu\textit{ch} inne sîn."\\ 
\end{tabular}
\scriptsize
\line(1,0){75} \newline
G I O L M Q R Z Fr21 Fr38 \newline
\line(1,0){75} \newline
\textbf{1} \textit{Initiale} I O L M Q Z Fr21   $\cdot$ \textit{Capitulumzeichen} R  \textbf{13} \textit{Initiale} I  \textbf{23} \textit{Initiale} G  \newline
\line(1,0){75} \newline
\textbf{1} Libaut] ÷ybavt O Libavt L Libayt M Lẏbant R Lybavt Z Libovt Fr21 :::t Fr38  $\cdot$ der vürste] \textit{om.} G  $\cdot$ al] \textit{om.} M Z als R \textbf{3} künic] chvniges O (L) (M) (Q) (R) (Fr21) (Fr38)  $\cdot$ Lotes] lots G lotis M \textbf{4} daz] dat I osz M (R) \textbf{5} triwe] trúwen R \textbf{6} eines] Ein Fr21  $\cdot$ geweren] wern R \textbf{7} iu] \textit{om.} R \textbf{8} wes] Was Q R Swes Z  $\cdot$ ich] \textit{om.} Q  $\cdot$ drumbe] \textit{om.} R  $\cdot$ bedâht] verdaht L \textbf{9} Libaut] er G Lybavt O L Z Libavt M Lybaut Q Lybant R Libovt Fr21  $\cdot$ im dankte] danchte im G im dancht O (L) (Fr38) im danck Z \textbf{10} dem] den I (R)  $\cdot$ hover] hof da er I hoe er Q  $\cdot$ tohter] ritter M \textbf{13} dô] Da M  $\cdot$ er] \textit{om.} L R  $\cdot$ Obilote] obiloten I (L) Obyloten O Fr38 obylote Q Obylot Z \textbf{15} var] wil M \textbf{16} getrûwe] troͮwe O (M) (R) Fr21  $\cdot$ im] \textit{om.} O M Fr21 nv L  $\cdot$ wol] wes Q  $\cdot$ daz] \textit{om.} R  $\cdot$ wer] gewer R Z \textbf{17} rîter] rittern M \textbf{18} dienstes] dienst I (Q) \textbf{19} sô] da R \textbf{20} erne hât] Er hat O R Her bat M  $\cdot$ abe noch ane] an noch ab I (L) (M) (Q) weder ab noch an R \textbf{21} anz] vncz R \textbf{23} Dô] Da M Z \textbf{24} Gawan] Gaban Q  $\cdot$ dô] da M  $\cdot$ er si] ersz L (Z) (Fr38) \textbf{25} dô saz] gesaz I da sasz M \textbf{26} er dankt] vnde dancht O (Q) (R) (Fr21) (Fr38) Vnd danchte L (M) Vnd danck Z \textbf{27} sîn] Sint M  $\cdot$ dô] da M  $\cdot$ man] manz I \textbf{28} geleit] leid R \textbf{29} alsus] so G svs O (M) (Q) (R) (Z) Fr21 Fr38  $\cdot$ kleine] wenich O (L) (M) (Q) (R) (Z) (Fr21) (Fr38)  $\cdot$ vröuwelîn] [vingerlin]: vreuwelin I \textbf{30} dâ] Das Q Do R  $\cdot$ solt] sol I L  $\cdot$ durch] bi G  $\cdot$ iuch] iv G ir R \newline
\end{minipage}
\hspace{0.5cm}
\begin{minipage}[t]{0.5\linewidth}
\small
\begin{center}*T
\end{center}
\begin{tabular}{rl}
 & \begin{large}L\end{large}ybaut \textbf{alse} vaste bat.\\ 
 & "hêrre, durch got die rede lât",\\ 
 & sprach des \textbf{küneges} Lotes suon.\\ 
 & "durch iuwer zuht sult ir daz tuon\\ 
5 & unde lât mich triuwen niht enbern."\\ 
 & Eines dinges \textbf{wil ich} iuch gewern:\\ 
 & ich sagiu hînt bî dirre naht,\\ 
 & wes ich mich drumbe hân bedâht."\\ 
 & Lybaut im dankete unde vuor zehant.\\ 
10 & \textbf{ûf dem} hove er sîne tohter vant\\ 
 & unde des burcgrâven töhterlîn;\\ 
 & die zwei, \textbf{die} snalten vingerlîn.\\ 
 & dô sprach er Obylote zuo:\\ 
 & "tohter, wannen kumestuo?"\\ 
15 & "vater, ich var dâ nider her.\\ 
 & ich getriuwe wol, daz er mich \textbf{wer}.\\ 
 & ich wil den vremden rîter biten\\ 
 & dienstes nâch lônes siten."\\ 
 & "Tohter, sô sî dir geklaget,\\ 
20 & er hât mir \textbf{abe noch ane} gesaget.\\ 
 & kum mîner bete anz ende nâch."\\ 
 & der megede was zem gaste gâch.\\ 
 & dô s\textbf{în zer} kemenâten gienc,\\ 
 & Gawan spranc ûf, \textbf{der} si enpfienc.\\ 
25 & zuo der süezen er dô saz\\ 
 & \textbf{unde} dankete ir, daz si niht vergaz\\ 
 & \textbf{vor des}, dô man\textbf{z} im \textbf{missebôt}.\\ 
 & er sprach: "geleit ie rîter nôt\\ 
 & durch ein \textbf{sô} \textbf{wênic} vrowelîn,\\ 
30 & dâ solt ich durch iuch inne sîn."\\ 
\end{tabular}
\scriptsize
\line(1,0){75} \newline
T V W \newline
\line(1,0){75} \newline
\textbf{1} \textit{Initiale} T W  \textbf{6} \textit{Majuskel} T  \textbf{19} \textit{Majuskel} T  \newline
\line(1,0){75} \newline
\textbf{1} Lybaut] Libaut V LYbout W  $\cdot$ alse] al V der fúrste all W \textbf{6} iuch] îv T  $\cdot$ gewern] ioch weren W \textbf{7} dirre] der W \textbf{9} Lybaut] Libaut V Lybout W  $\cdot$ im dankete unde] danckte im W \textbf{12} die snalten] die snalten mit V schnalten mit W \textbf{13} Obylote] obiloten V obylot W \textbf{14} kumestuo] kumestu so fruͦ W \textbf{15} var] gan W \textbf{16} getriuwe] [*]: getruwe im V trauwe im W \textbf{20} abe noch ane] an noch ab W \textbf{21} anz] [a*]: an ein V \textbf{23} sîn zer] [*]: sv́ in die V sy in die W \textbf{24} der] do er W \textbf{25} süezen er dô saz] guͦten er gesas W \textbf{27} vor des] [S*]: Sin V Seit W  $\cdot$ manz im] man ims W \textbf{30} dâ] Do V W  $\cdot$ iuch] îv T \newline
\end{minipage}
\end{table}
\end{document}
