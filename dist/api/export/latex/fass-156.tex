\documentclass[8pt,a4paper,notitlepage]{article}
\usepackage{fullpage}
\usepackage{ulem}
\usepackage{xltxtra}
\usepackage{datetime}
\renewcommand{\dateseparator}{.}
\dmyyyydate
\usepackage{fancyhdr}
\usepackage{ifthen}
\pagestyle{fancy}
\fancyhf{}
\renewcommand{\headrulewidth}{0pt}
\fancyfoot[L]{\ifthenelse{\value{page}=1}{\today, \currenttime{} Uhr}{}}
\begin{document}
\begin{table}[ht]
\begin{minipage}[t]{0.5\linewidth}
\small
\begin{center}*D
\end{center}
\begin{tabular}{rl}
\textbf{156} & daz ez Iwanet erhôrte\\ 
 & vor der stat ans graben orte.\\ 
 & vroun Ginovern knappe unt ir mâc,\\ 
 & \textbf{der} von dem orse erhôrte den bâc\\ 
5 & unt \textbf{dô er} niemen drûffe sach\\ 
 & - von sînen triwen daz geschach,\\ 
 & die er \textbf{nâch} Parzivale truoc -,\\ 
 & dô \textit{g}âhte dar der knappe kluoc.\\ 
 & \textbf{Er vant} Ithern tôt\\ 
10 & und Parzivalen in tumber nôt.\\ 
 & snellîch er zin beiden spranc.\\ 
 & \textbf{dô} sagete\textbf{r} Parzivale danc\\ 
 & prîses, des erwarp sîn hant\\ 
 & an dem von Kukumerlant.\\ 
15 & "Got lône dir! \textbf{nû} râte, waz ich tuo.\\ 
 & ich kan hie harte wênic zuo.\\ 
 & wie bringe ich\textbf{z} \textbf{ab} im \textbf{unt} an mich?"\\ 
 & "daz kan ich wol gelêren dich",\\ 
 & \textbf{sus} sprach der stolze Iwanet\\ 
20 & ze\textbf{m} fillu roy Gahmuret.\\ 
 & entwâpent wart der tôte man\\ 
 & al dâ vor Nantes ûf dem plân\\ 
 & unt an den lebenden geleit,\\ 
 & den dannoch grôziu tumpheit \textbf{reit}.\\ 
25 & Iwanet sprach: "\textbf{diu} ribbalîn\\ 
 & sulen niht under\textbf{em îsern} sîn.\\ 
 & dû \textbf{solt} nû tragen ritters kleit."\\ 
 & diu rede was Parzivale leit.\\ 
 & \textbf{\begin{large}D\end{large}ô} sprach der knappe guoter:\\ 
30 & "swaz mir gap mîn muoter,\\ 
\end{tabular}
\scriptsize
\line(1,0){75} \newline
D \newline
\line(1,0){75} \newline
\textbf{9} \textit{Majuskel} D  \textbf{15} \textit{Majuskel} D  \textbf{29} \textit{Initiale} D  \newline
\line(1,0){75} \newline
\textbf{1} Iwanet] Jwanet D \textbf{8} gâhte] dahte D \textbf{9} Ithern] Jthern D \textbf{14} Kukumerlant] Chvchvmerlant D \textbf{19} Iwanet] Jwanet D \textbf{20} Gahmuret] Gahmvret D \textbf{25} Iwanet] Jwanet D \newline
\end{minipage}
\hspace{0.5cm}
\begin{minipage}[t]{0.5\linewidth}
\small
\begin{center}*m
\end{center}
\begin{tabular}{rl}
 & daz ez Iwanet erhôrte\\ 
 & vor der stat ans graben orte,\\ 
 & vrouwen Ginoveren knappe und ir \dag maget\dag .\\ 
 & \textbf{dô er} von dem rosse erhôrte den bâc\\ 
5 & und \textbf{dô er} niemen drûffe sach\\ 
 & - von sînen triuwen daz geschach,\\ 
 & die er \textbf{nâch} Parcifalen truoc -,\\ 
 & dô gâhete dar der knappe kluoc.\\ 
 & \textbf{er vant} I\textit{t}hern tôt\\ 
10 & und Parcifalen in tumber nôt.\\ 
 & snellîch er zuo in beiden spranc.\\ 
 & \textbf{dô} sagete \textbf{er} Parcifalen danc\\ 
 & prîses, des erwarp sîn hant\\ 
 & an dem von Kukumerlant.\\ 
15 & "got lône dir! \textbf{nû} rât, waz ich tuo.\\ 
 & ich kan hie harte wênic zuo.\\ 
 & wie bring ich \textbf{daz} \textbf{ab} im \textbf{und} an mich?"\\ 
 & "daz kan \textit{ich} wol gelêren dich",\\ 
 & sprach der stolze Iwanet\\ 
20 & ze\textbf{m} fili rois Gahm\textit{u}ret.\\ 
 & entwâpet wart der tôte man\\ 
 & aldâ vor \textit{N}antes ûf de\textit{m} plân\\ 
 & und an den lebenden geleit,\\ 
 & de\textit{n} dennoch grôziu tumpheit \textbf{reit}.\\ 
25 & Iwanet sprach: "\textbf{diu} ribbalîn\\ 
 & sullen niht under\textbf{\textit{m} îser} sîn.\\ 
 & dû \textbf{solt} nû \textit{tr}agen ritters kleit."\\ 
 & diu rede was Parcifalen leit.\\ 
 & \textbf{\begin{large}D\end{large}ô} sprach der knappe guoter:\\ 
30 & "waz mir gap mîn muoter,\\ 
\end{tabular}
\scriptsize
\line(1,0){75} \newline
m n o \newline
\line(1,0){75} \newline
\textbf{29} \textit{Initiale} m o  \newline
\line(1,0){75} \newline
\textbf{1} Iwanet] jwanet m n ẏwanet o \textbf{3} vrouwen] Frouwe m (n) (o)  $\cdot$ Ginoveren] einewern n Einevern o \textbf{4} erhôrte] erhort n o  $\cdot$ bâc] baget n (o) \textbf{6} sînen] \textit{om.} o  $\cdot$ geschach] geschasch o \textbf{7} Parcifalen] parcifal n o \textbf{8} dar der knappe] der knappe dar vil n \textbf{9} Ithern] ichern m ichtern n Jtern o \textbf{10} Parcifalen] [parc*]: parcẏfal n parcifaln o  $\cdot$ tumber] kumbers n (o) \textbf{12} Parcifalen] parcifallen m parcifale o \textbf{13} des] das n o  $\cdot$ erwarp] erwarp do n \textbf{14} Kukumerlant] Cucumber lant m kucumerlant n cacuͯmmerlant o \textbf{15} lône] lo o \textbf{16} kan] han n \textbf{17} ab] [am]: ab m  $\cdot$ und] \textit{om.} n o \textbf{18} ich] \textit{om.} m \textbf{19} Iwanet] jwanet m o [*]: jwanet o \textbf{20} Gahmuret] gahmiret m gamiret n hamuͯret o \textbf{22} aldâ] Also n  $\cdot$ Nantes] mantes m  $\cdot$ dem] den m \textbf{24} den] Dem m \textbf{25} Iwanet] Jwanet m n o  $\cdot$ diu] de n \textbf{26} underm îser] vnder in iser m vnden ysin n (o) \textbf{27} tragen] clagen m \textbf{28} Parcifalen] parcifal n o \newline
\end{minipage}
\end{table}
\newpage
\begin{table}[ht]
\begin{minipage}[t]{0.5\linewidth}
\small
\begin{center}*G
\end{center}
\begin{tabular}{rl}
 & daz e\textit{z} Ywanet erhôrte\\ 
 & vor \textit{der} \textit{stat} ans graben orte,\\ 
 & vrôn Schinoveren knappe und ir mâc.\\ 
 & \textbf{\begin{large}D\end{large}ôr} von dem orse erhôrte den bâc\\ 
5 & und \textbf{er} niemen drûfe sach\\ 
 & - von sînen triwen daz geschach,\\ 
 & die er \textbf{nâch} Parzivale truoc -,\\ 
 & dô gâhte dar der knappe kluoc.\\ 
 & \textbf{dâ vant er} Itheren tôt\\ 
10 & unde Parzivalen in tumber nôt.\\ 
 & snellîche er zuo in beiden spranc\\ 
 & \textbf{unde} sagete Parzivale danc\\ 
 & prîses, des erwarp sîn hant\\ 
 & an dem von Kukumerlant.\\ 
15 & "got lône dir! rât, waz ich tuo.\\ 
 & ich kan hie harte wênic zuo.\\ 
 & wie bringe ich\textbf{z} \textbf{abe} im an mich?"\\ 
 & "daz kan ich wol gelêren dich",\\ 
 & sprach der stolze Ywanet\\ 
20 & ze fili rois Gahmuret.\\ 
 & entwâpent wart der tôte man\\ 
 & al dâ vor Nantis ûf de\textit{m} plân\\ 
 & unde an den lebenden geleget,\\ 
 & den dannoch grôziu tumpheit \textbf{reget}.\\ 
25 & Ywanet sprach: "\textbf{diu} ribbalîn\\ 
 & sulen niht under \textbf{dem îser} sîn.\\ 
 & dû \textbf{muost} nû tragen rîters kleit."\\ 
 & diu rede was Parzivale leit.\\ 
 & \textbf{dô} sprach der knappe guoter:\\ 
30 & "swaz mir gap mîn muoter,\\ 
\end{tabular}
\scriptsize
\line(1,0){75} \newline
G I O L M Q R Z Fr36 \newline
\line(1,0){75} \newline
\textbf{4} \textit{Initiale} G  \textbf{15} \textit{Initiale} I  \textbf{27} \textit{Initiale} O Q Z Fr36  \textbf{29} \textit{Initiale} I L  \newline
\line(1,0){75} \newline
\textbf{1} ez] er G daz O  $\cdot$ Ywanet] ẏwanet G jwanet L Jwan R Iwanet Fr36 \textbf{2} vor] von I (O) (Q) Fr36  $\cdot$ der stat] \textit{om.} G  $\cdot$ ans] vnz ans O vͤn::: Fr36 \textbf{3} vrôn] Vrow L (M) (Q) (R)  $\cdot$ Schinoveren] Gunwarn I Ginovern O (M) Genovieren L gynoúeren Q Ginoren R gynovern Z ginoue::: Fr36  $\cdot$ knappe] knappen L  $\cdot$ und] \textit{om.} Q  $\cdot$ ir] \textit{om.} I \textbf{4} Dôr] Da er L M Z  $\cdot$ von dem orse erhôrte] von den orshen hort I von dem orse hort O von den roszen erhorte L erhortten von den Rosen R  $\cdot$ den] disen R  $\cdot$ bâc] hac M \textbf{5} er] daz er I O L M (Q) Z  $\cdot$ niemen] nyemet R  $\cdot$ sach] er sach I (M) Q (Z) \textbf{7} nâch] zu I  $\cdot$ Parzivale] [parzifaln]: Parzifaln I parcifale O L parzeval M partzifal Q parczifaln R parcifalen Z \textbf{8} dô] Da M Z  $\cdot$ gâhte] gidachte M gaht O Z  $\cdot$ dar] er do O das M  $\cdot$ der knappe] der helt O er knape M \textbf{9} dâ vant er] do vant er I (O) (Q) (R) Er vant L  $\cdot$ Itheren] ytern I Jthern O M jhtern L (R) ythern Q ichern Z \textbf{10} Parzivalen] [parzifaln]: Parzifal I Parcifalen O (L) (Z) parzivaln M partzifaln Q parczifaln R \textbf{11} snellîche] Vil balde L \textbf{12} unde] Do L Da Z  $\cdot$ sagete] seit I (O) (Q) sagete e>r < L sagt er Z  $\cdot$ Parzivale] Parzifalen I parcifalen O Z parcifale L parzival M partzifaln Q parczifaln R \textbf{13} prîses] Des prises L  $\cdot$ des] den O R Z \textbf{14} an dem] Anem koinge Q Ainne kungin R  $\cdot$ Kukumerlant] chukunberlant I kvcumerlant O kvͯcvͯmer lant L kucuͯmer lant M kukumerland R kvnkvmerlant Z \textbf{15} rât] nu rat I (O) (L) (M) (Q) Z \textbf{16} ich] Jchn O  $\cdot$ kan] [z]: kan L [han]: kan M [lan]: kan R  $\cdot$ hie] \textit{om.} O  $\cdot$ harte] rechte Q  $\cdot$ zuo] hiezvͦ O \sout{mit} zu R \textbf{17} an] vnde an I (O) (L) (M) (Q) (Z) \textbf{19} Ywanet] ẏwanet G iuuanet I jwanet L Iwanet R \textbf{20} ze] \textit{om.} I  $\cdot$ fili] fillo O  $\cdot$ Gahmuret] Gamvret O (Z) Gahmuͯret L gamuͯret M Gamúret Q \textbf{22} Nantis] nantes I (L)  $\cdot$ dem] den G \textbf{23} lebenden] [le]: lebendigen O lebenigen M lebending Q lebendigen R \textbf{24} dannoch] denne R  $\cdot$ grôziu] grosze R  $\cdot$ reget] wegt R \textbf{25} Ywanet] ẏwanet G Juuanes I Jwanet O L M Ywan Q Jwan R  $\cdot$ diu] du R \textbf{26} dem] \textit{om.} L Q den M  $\cdot$ îser] ysen I O L (Q) R \textbf{27} dû] ÷v O  $\cdot$ muost] solt L  $\cdot$ kleit] kelid R \textbf{28} diu rede] daz I (L)  $\cdot$ Parzivale] [parzifalen]: Parzifalen I parcifalen O Z Parcifale L parzival M partzifaln Q parczifaln R parcifaln Fr36 \textbf{29} dô] Da M  $\cdot$ knappe] degen O L Fr36 kúng R \textbf{30} swaz] Waz L (Q) (R) so swaz Fr36 \newline
\end{minipage}
\hspace{0.5cm}
\begin{minipage}[t]{0.5\linewidth}
\small
\begin{center}*T (U)
\end{center}
\begin{tabular}{rl}
 & daz ez Ywanet erhôrte\\ 
 & vor der stat an des graben orte,\\ 
 & vroun Gynovern knabe und ir mâc.\\ 
 & \textbf{dô\textit{r}} vonme orse erhôrte den bâc\\ 
5 & und \textbf{ouch} nieman dâr ûf sach\\ 
 & - von sînen triuwen daz geschach,\\ 
 & die er \textbf{zuo} Parcifale truoc -,\\ 
 & dô gâhete dar der knappe kluoc.\\ 
 & \textbf{sus vander} Ithern tôt\\ 
10 & und Parcifaln in tumber nôt.\\ 
 & snellîche er \textbf{hin} zuo in beiden spranc\\ 
 & \textbf{und} sagete Parcifale danc\\ 
 & prîses, des erwarp sîn hant\\ 
 & an dem von Kukumerlant.\\ 
15 & "got lône dir! \textbf{nû} rât, waz ich tuo.\\ 
 & ich kan hie harte wênic zuo.\\ 
 & wie bringe ich \textbf{ez} \textbf{von} im \textbf{und} an mich?"\\ 
 & "daz kan \textit{ich} wol gelêren dich",\\ 
 & sprach der stolze Ywanet\\ 
20 & zuo filli rois Gahmuret.\\ 
 & entwâpent wart der tôte man\\ 
 & al dâ vor Nantes ûf dem plân\\ 
 & und an den lebenden geleget,\\ 
 & den dannoch grôziu tumpheit \textbf{weget}.\\ 
25 & Ywanet sprach: "\textbf{disiu} ribbalîn\\ 
 & solnt niht under \textbf{îsern hosen} sîn.\\ 
 & dû \textbf{muost} nû tragen rîters kleit."\\ 
 & diu rede was Parcifale leit.\\ 
 & \textbf{sus} sprach der knappe guoter:\\ 
30 & "waz mir gap mîn muoter,\\ 
\end{tabular}
\scriptsize
\line(1,0){75} \newline
U V W T \newline
\line(1,0){75} \newline
\textbf{8} \textit{Majuskel} T  \textbf{15} \textit{Majuskel} T  \textbf{18} \textit{Majuskel} T  \textbf{21} \textit{Majuskel} T  \textbf{28} \textit{Majuskel} T  \textbf{29} \textit{Initiale} W  \newline
\line(1,0){75} \newline
\textbf{2} vor der stat] \textit{om.} T \textbf{3} vroun] Fraw W  $\cdot$ Gynovern] [*]: Schinovern U ginovern V schinouern W  $\cdot$ ir] \textit{om.} W \textbf{4} dôr] Do U dort V  $\cdot$ vonme] vorme T  $\cdot$ erhôrte] horte W \textbf{5} ouch] er auch W daz er T \textbf{7} zuo] nach W T  $\cdot$ Parcifale] parzifale V T partzifale W \textbf{9} sus] do T  $\cdot$ Ithern] Jtern U ẏtern V ythern W Jthern T \textbf{10} Parcifaln] parzifaln V T partzifaln W \textbf{11} snellîche] Snelle W  $\cdot$ hin] \textit{om.} T \textbf{12} Parcifale] Parzifale U (V) T partzifaln W \textbf{13} des] [*]: des V den do W \textbf{14} an dem] >Amme kv́nige< V In dem W  $\cdot$ Kukumerlant] kuͦkuͦmerlant U >kvkumerlant< V kukumber land W [kv*]: kvkvmerlant T \textbf{16} hie harte wênic] harte wenig dar W \textbf{17} bringe ich ez] bringes ich V  $\cdot$ von] ab V W (T)  $\cdot$ und] \textit{om.} T \textbf{18} ich] \textit{om.} U  $\cdot$ gelêren] gelernen W \textbf{19} sprach] Sus sprach W  $\cdot$ Ywanet] ẏwanet V Jwanet T \textbf{20} Gahmuret] Gahmuͦret U gemuret V gamuret W \textbf{21} tôte] rote V ivnge T \textbf{22} Nantes] nantis W T  $\cdot$ dem] dē V \textbf{23} lebenden] lebendigen T \textbf{24} den] [d*]: den V  $\cdot$ weget] [*]: reget V reget T \textbf{25} disiu] dein W div T \textbf{26} under] [vnder]: vnderm T  $\cdot$ îsern hosen] [ys*]: yser V eisen hosen W îsene T \textbf{28} Parcifale] Parzifale U (V) partzifale W \textbf{29} sus] ALsus W do T \textbf{30} waz] swaz V (T) \newline
\end{minipage}
\end{table}
\end{document}
