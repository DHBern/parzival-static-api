\documentclass[8pt,a4paper,notitlepage]{article}
\usepackage{fullpage}
\usepackage{ulem}
\usepackage{xltxtra}
\usepackage{datetime}
\renewcommand{\dateseparator}{.}
\dmyyyydate
\usepackage{fancyhdr}
\usepackage{ifthen}
\pagestyle{fancy}
\fancyhf{}
\renewcommand{\headrulewidth}{0pt}
\fancyfoot[L]{\ifthenelse{\value{page}=1}{\today, \currenttime{} Uhr}{}}
\begin{document}
\begin{table}[ht]
\begin{minipage}[t]{0.5\linewidth}
\small
\begin{center}*D
\end{center}
\begin{tabular}{rl}
\textbf{628} & \textit{\begin{large}G\end{large}}awan nâch arbeite pflac\\ 
 & \textbf{slâfens} den mitten tac.\\ 
 & im wâren sîne wunden\\ 
 & mit kunst alsô \textbf{gebunden},\\ 
5 & ob vriwendîn wære bî im gelegen,\\ 
 & \textbf{het er minne} gepflegen,\\ 
 & \textbf{daz} wære im senfte und guot.\\ 
 & er het ouch \textbf{bezzeren} slâfes muot\\ 
 & denne des nahtes, dô diu herzogîn\\ 
10 & an ungemache im gap gewin.\\ 
 & Er \textbf{erwachte} \textbf{gein} der vesperzît;\\ 
 & doch het er in slâfe strît\\ 
 & gestriten mit der minne,\\ 
 & aber mit der herzoginne.\\ 
15 & Ein sîn kamerære\\ 
 & mit tiurem golde swære\\ 
 & brâht im kleider dar getragen\\ 
 & vo\textit{n} \textbf{liehtem} pfelle, hôrt ich sagen.\\ 
 & dô sprach mîn hêr Gawan:\\ 
20 & "wir suln \textbf{der kleider mêr noch} hân,\\ 
 & diu \textbf{al} gelîche tiure sîn,\\ 
 & dem herzogen von Gowerzin\\ 
 & unt dem \textbf{clâren} Florande,\\ 
 & der in manegem lande\\ 
25 & hât gedienet werdecheit.\\ 
 & nû \textbf{schaffet}, daz diu sîn bereit."\\ 
 & Bî eime knappen er enbôt\\ 
 & sîme wirte Plippalinot,\\ 
 & daz er \textbf{sande im} Lischoysen dar.\\ 
30 & bî sîner tohter wol gevar\\ 
\end{tabular}
\scriptsize
\line(1,0){75} \newline
D Z Fr16 \newline
\line(1,0){75} \newline
\textbf{1} \textit{Initiale} D Z  \textbf{11} \textit{Majuskel} D  \textbf{15} \textit{Majuskel} D  \textbf{27} \textit{Majuskel} D  \newline
\line(1,0){75} \newline
\textbf{1} Gawan] Dawan D \textbf{9} dô] da Z \textbf{16} mit tiurem] Von tevren Z \textbf{18} von] vol D \textbf{20} noch] \textit{om.} Z \textbf{22} Gowerzin] Gauezin Fr16 \textbf{23} clâren] vursten Fr16 \textbf{28} Plippalinot] plipalinot Z \textbf{29} sande im] im sande Z  $\cdot$ Lischoysen] Liscoysen D Lishoisen Z \newline
\end{minipage}
\hspace{0.5cm}
\begin{minipage}[t]{0.5\linewidth}
\small
\begin{center}*m
\end{center}
\begin{tabular}{rl}
 & \begin{large}G\end{large}awan nâch arbeit pflac\\ 
 & \textbf{\textit{s}lâfens} \textbf{umb} den mitten tac.\\ 
 & im wâren sîne wunden\\ 
 & mit kunst alsô \textbf{gebunden},\\ 
5 & ob vriundîn \textit{w}ær\textit{e} bî ime gelegen\\ 
 & \textbf{und minne het} ge\textit{pfl}e\textit{g}en,\\ 
 & \textbf{e\textit{z}} wær im senft und \textit{g}uot.\\ 
 & er het ouch \textbf{bezzers} slâfes \textit{m}uot\\ 
 & dan des nahtes, dô diu herzogîn\\ 
10 & an ungemach im gap gewin.\\ 
 & er \textbf{erwahte} \textbf{an} der vesper zît;\\ 
 & doch het er in slâfe strît\\ 
 & gestriten mit der minne,\\ 
 & aber mit der herzo\textit{g}inne.\\ 
15 & ein sîn kamerære\\ 
 & mit tiurem golde swære\\ 
 & brâht im kleider \textbf{al}dar getragen\\ 
 & von \textbf{rîchem} pfelle, hôrt ich sagen.\\ 
 & dô sprach mîn hêr Gawan:\\ 
20 & "wir sullen \textbf{mê kleider} hân,\\ 
 & diu \textbf{alliu} glîch tiur sî\textit{n},\\ 
 & dem herzogen \textit{von} Gowertzi\textit{n}\\ 
 & und dem \textbf{vürsten} \textit{F}lorande,\\ 
 & der i\textit{n} manigem lande\\ 
25 & het gedienet wirdicheit.\\ 
 & nû \textbf{schafft}, daz diu sîn bereit."\\ 
 & bî einem knappen er enbôt\\ 
 & sînem wirt Plippalinot,\\ 
 & daz er \textbf{im sante} Lischoisen dar.\\ 
30 & bî sîner tohter wol gevar\\ 
\end{tabular}
\scriptsize
\line(1,0){75} \newline
m n o \newline
\line(1,0){75} \newline
\textbf{1} \textit{Initiale} m   $\cdot$ \textit{Capitulumzeichen} n  \newline
\line(1,0){75} \newline
\textbf{2} slâfens] Gloffens m \textbf{5} vriundîn] frúnden n (o)  $\cdot$ wære] veren m \textbf{6} het] hettent n  $\cdot$ gepflegen] gegeben m \textbf{7} ez] Er m o  $\cdot$ guot] muͯt m \textbf{8} bezzers] besser n o  $\cdot$ muot] gut m \textbf{9} dô] do do n \textbf{10} gewin] win n \textbf{14} herzoginne] hertzognẏnne m \textbf{17} aldar] dar n o \textbf{19} hêr] herre her n \textbf{20} sullen] sullent sullent o \textbf{21} tiur] [die]: dir n  $\cdot$ sîn] sint m \textbf{22} herzogen] herczogin o  $\cdot$ von] \textit{om.} m  $\cdot$ Gowertzin] [gowertz*]: gowertzint m goweczen o \textbf{23} Florande] clorande m \textbf{24} in] ẏm m  $\cdot$ manigem] mangen o \textbf{26} sîn] sint n \textbf{28} sînem] Sinen o  $\cdot$ Plippalinot] pilpamot o \textbf{29} Lischoisen] liscoisen m n o \newline
\end{minipage}
\end{table}
\newpage
\begin{table}[ht]
\begin{minipage}[t]{0.5\linewidth}
\small
\begin{center}*G
\end{center}
\begin{tabular}{rl}
 & Gawan \textit{nâch} arbeite pflac\\ 
 & \textbf{slâfes} den mitten tac.\\ 
 & im wâren sîne wunden\\ 
 & mit kunst alsô \textbf{verbunden},\\ 
5 & ob vriundîn wære bî im gelege\textit{n},\\ 
 & \textbf{het er ir minne} gepflegen,\\ 
 & \textbf{daz} wære im senfte unde guot.\\ 
 & er het ouch \textbf{bezzern} slâfes muot\\ 
 & danne des nahtes, dô diu herzogîn\\ 
10 & an ungemache ime gap gewin.\\ 
 & er\textbf{ntwachete} \textbf{gein} der vesper zît;\\ 
 & doch het er in slâfe strît\\ 
 & gestriten mit der minne,\\ 
 & aber mit der herzoginne.\\ 
15 & ein sîn kamerære\\ 
 & mit tiurem golde swære\\ 
 & brâht im kleider dar getragen\\ 
 & von \textbf{liehtem} pfelle, hôrt ich sagen.\\ 
 & \begin{large}D\end{large}ô sprach mîn hêr Gawan:\\ 
20 & "wir suln \textbf{der kleider} hân,\\ 
 & di\textit{u} \textbf{al}gelîche tiure sîn,\\ 
 & dem herzogen von Gowerzin\\ 
 & unde dem \textbf{clâren} Florande,\\ 
 & der in manigem lande\\ 
25 & hât gedienet werdecheit.\\ 
 & nû \textbf{schaf\textit{f}et}, daz diu sîn bereit."\\ 
 & bî einem knappen ernbôt\\ 
 & sînem wirte Pliplalinot,\\ 
 & daz er \textbf{sande} Lishoisen dar.\\ 
30 & bî sîner tohter wol gevar\\ 
\end{tabular}
\scriptsize
\line(1,0){75} \newline
G I L M Z Fr51 \newline
\line(1,0){75} \newline
\textbf{1} \textit{Initiale} L Z  \textbf{11} \textit{Initiale} I  \textbf{19} \textit{Initiale} G Fr51  \textbf{27} \textit{Initiale} I  \newline
\line(1,0){75} \newline
\textbf{1} nâch] \textit{om.} G \textbf{2} slâfes] slaffens I (L) (Z)  $\cdot$ den] vmb I \textbf{4} verbunden] biwunden M gebunden Z (Fr51) \textbf{5} Ob sin vrvndin mit bette gelegen Fr51  $\cdot$ ob] ob ein I  $\cdot$ gelegen] gelege G \textbf{6} ir] \textit{om.} M Z Fr51 \textbf{7} wære] war Fr51 \textbf{8} ouch] \textit{om.} Fr51  $\cdot$ bezzern slâfes] slafes bezzern I beszers slaffes L (Fr51) \textbf{9} dô] da M Z \textbf{10} ime] \textit{om.} L M \textbf{11} erntwachete] Er erwachete I (Z) Er wachte L M vurwachte Fr51  $\cdot$ gein der] ander Fr51 \textbf{12} doch] Do Fr51 \textbf{16} Mit] Von Z  $\cdot$ tiurem] [truͯwe*]: týuͯwerem L torme M tevren Z \textbf{17} im] man yme M in Fr51 \textbf{18} liehtem] lichtem L lechten Fr51 \textbf{19} Dô] Da M  $\cdot$ mîn] \textit{om.} Fr51  $\cdot$ hêr Gawan] ergawan M \textbf{20} suln] en suln M  $\cdot$ der kleider] der cleider mer I (L) Z der cleidir noch mer M mer cleider Fr51 \textbf{21} diu] die G \textbf{22} Gowerzin] goverzin G Gouerzin I gowerczin M gowercin Fr51 \textbf{23} Florande] floriande G (I) \textbf{24} manigem] manigen Fr51 \textbf{25} gedienet] der deinet Fr51 \textbf{26} schaffet] schaftet G shaffe I  $\cdot$ bereit] gereit M \textbf{27} bî einem] Si einen Fr51 \textbf{28} Pliplalinot] pliplalinon G plipalinot I (L) M plibalinot Fr51 \textbf{29} \textit{Versfolge 628.30-29} I   $\cdot$ sande] \textit{om.} L yme sende M (Z) (Fr51)  $\cdot$ Lishoisen] Liscoisen I Lýtschoýsen L lisoien M  $\cdot$ dar] sante dar L \newline
\end{minipage}
\hspace{0.5cm}
\begin{minipage}[t]{0.5\linewidth}
\small
\begin{center}*T
\end{center}
\begin{tabular}{rl}
 & Gawan nâch arbeite pflac\\ 
 & \textbf{slâf\textit{es}} \textbf{ûf} den mitten tac.\\ 
 & im wâren sîne wunden\\ 
 & mit kunst alsô \textbf{verbunden},\\ 
5 & ob vriundîn wære bî im gelegen,\\ 
 & \textbf{het er ir minne} gepflegen,\\ 
 & \textbf{daz} wære im senfte und guot.\\ 
 & er hete ouch \textbf{bezzers} slâfes muot\\ 
 & dan des nahtes, dô diu herzogîn\\ 
10 & an ungemache im gap gewin.\\ 
 & \begin{large}E\end{large}r \textbf{erwachete} \textbf{gein} der vesperzît;\\ 
 & doch heter in slâfe strît\\ 
 & gestriten mit der minne,\\ 
 & aber mit der herzoginne.\\ 
15 & ein sîn kamerære\\ 
 & mit tiurem golde swære\\ 
 & brâht im kleider dar getragen\\ 
 & von \textbf{liehtem} pfelle, hôrtich sagen.\\ 
 & dô sprach mîn hêr Gawan:\\ 
20 & "wir soln \textbf{der kleider noch mêr} hân,\\ 
 & diu \textbf{alliu} glîche tiure sîn,\\ 
 & dem herzogen von Gowerzin\\ 
 & und dem \textbf{clâren} Florande,\\ 
 & der in manegem lande\\ 
25 & hâ\textit{t} gedienet wirdecheit.\\ 
 & nû \textbf{schaffe}, daz diu sîn bereit."\\ 
 & b\textit{î} eime knappen er enbôt\\ 
 & sîme wirte Plipalinot,\\ 
 & daz er \textbf{im sante} Lyschoysen dar.\\ 
30 & bî sîner tohter wol gevar\\ 
\end{tabular}
\scriptsize
\line(1,0){75} \newline
U V W Q R \newline
\line(1,0){75} \newline
\textbf{1} \textit{Initiale} W R  \textbf{11} \textit{Initiale} U V  \newline
\line(1,0){75} \newline
\textbf{1} Gawan] Gawin R  $\cdot$ arbeite] arbeiten V \textbf{2} slâfes] Slaf U  $\cdot$ ûf] [*]: vmbe V \textit{om.} W Q R \textbf{4} verbunden] gebunden Q \textbf{6} ir] der V \textit{om.} W Q R \textbf{7} im] \textit{om.} W \textbf{8} hete] hot Q  $\cdot$ bezzers] besser V besseren W herten Q \textbf{9} des nahtes] \textit{om.} R \textbf{11} gein] nach V \textbf{12} doch] Do V Noch R  $\cdot$ heter] herter R  $\cdot$ in] in dem W \textbf{15} kamerære] kramere W \textbf{16} tiurem] trúwen R \textbf{17} brâht im] [B*]: Braht im V Brach im Q \textbf{18} liehtem] liehten V (R) lichtem Q  $\cdot$ pfelle] pfellen V \textbf{20} soln] súlter R \textbf{21} alliu] alle R \textbf{22} Gowerzin] gowerzein W kaberzin Q Gowerczin R \textbf{25} hât] Hate U  $\cdot$ gedienet] verdienet W \textbf{26} schaffe] schaffen V schafft W (Q) (R)  $\cdot$ sîn] sei W  $\cdot$ bereit] gereit Q \textbf{27} bî] [Bi*]: Bit U  $\cdot$ enbôt] gebot Q \textbf{28} Plipalinot] Plipalmot U plypalinot W \textbf{29} Lyschoysen] lyschoien U lischoien V lyshoin W lyszhoysen Q Lyschoisen R \newline
\end{minipage}
\end{table}
\end{document}
