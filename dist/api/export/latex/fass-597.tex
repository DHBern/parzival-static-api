\documentclass[8pt,a4paper,notitlepage]{article}
\usepackage{fullpage}
\usepackage{ulem}
\usepackage{xltxtra}
\usepackage{datetime}
\renewcommand{\dateseparator}{.}
\dmyyyydate
\usepackage{fancyhdr}
\usepackage{ifthen}
\pagestyle{fancy}
\fancyhf{}
\renewcommand{\headrulewidth}{0pt}
\fancyfoot[L]{\ifthenelse{\value{page}=1}{\today, \currenttime{} Uhr}{}}
\begin{document}
\begin{table}[ht]
\begin{minipage}[t]{0.5\linewidth}
\small
\begin{center}*D
\end{center}
\begin{tabular}{rl}
\textbf{597} & \textit{\begin{large}G\end{large}}awan vriesch diu mære\\ 
 & von der tjoste pfandære.\\ 
 & Plippalinot nam alsô pfant:\\ 
 & swelch tjoste \textbf{wart al dâ bekant},\\ 
5 & daz einer \textbf{viel}, der ander \textbf{saz},\\ 
 & sô enpfieng er ân ir beider haz\\ 
 & dises vlust \textbf{unt} \textbf{eines} gewin,\\ 
 & \textbf{ich meine} daz ors, \textbf{daz zôch} er hin.\\ 
 & Ern ruochte, striten si genuoc.\\ 
10 & swer prîs oder laster truoc,\\ 
 & des liez er \textbf{jehen die vrouwen},\\ 
 & \textbf{si} mohten\textbf{z} dicke \textbf{schouwen}.\\ 
 & Gawanen er vaste sitzen bat.\\ 
 & er zôch imz ors an \textbf{den} stat,\\ 
15 & \textbf{er} bôt im \textbf{schilt unde} sper.\\ 
 & \textbf{hie} kom der Turkote her\\ 
 & kalopierende als ein man,\\ 
 & der \textbf{sîne} tjoste mezzen kan\\ 
 & weder ze hôch noch ze nider.\\ 
20 & Gawan kom gein im \textbf{hin} wider.\\ 
 & von Munsalvæsche Gringuljet\\ 
 & \textbf{tet} nâch Gawans bet,\\ 
 & als ez der zoum \textbf{gelêrte}.\\ 
 & ûf den plân er kêrte.\\ 
25 & Hurtâ, \textbf{lât} die tjoste tuon!\\ 
 & hie \textbf{kom} des \textbf{künec} Lotes sun\\ 
 & manlîch \textbf{und} âne herzen schric.\\ 
 & wâ \textbf{hât} diu helmsnuor \textbf{ir} stric?\\ 
 & des Turkoten tjost \textbf{in traf} al dâ.\\ 
30 & Gawan \textbf{ruort in} anderswâ:\\ 
\end{tabular}
\scriptsize
\line(1,0){75} \newline
D Z \newline
\line(1,0){75} \newline
\textbf{1} \textit{Initiale} D  \textbf{3} \textit{Initiale} Z  \textbf{9} \textit{Majuskel} D  \textbf{25} \textit{Majuskel} D  \newline
\line(1,0){75} \newline
\textbf{1} Gawan] ÷awan D \textbf{3} Plippalinot] Plipalmot Z \textbf{4} Swelch tiost da fvr wart erkant Z \textbf{5} saz] gesaz Z \textbf{6} er] \textit{om.} Z \textbf{7} eines] ienes Z \textbf{10} laster] lasters Z \textbf{12} si] Die Z \textbf{13} \textit{Versfolge 597.14-13} Z   $\cdot$ Gawanen] Gawann D Gawan Z \textbf{14} den] daz Z \textbf{15} schilt unde] in die hant ein Z \textbf{16} hie] Nv Z  $\cdot$ Turkote] Tvrkoite Z \textbf{20} hin] her Z \textbf{21} Munsalvæsche] Mvnsalvæsce D montsalvatsch Z  $\cdot$ Gringuljet] Gringvliet D (Z) \textbf{22} tet] Fvr Z \textbf{26} Lotes] Lots D \textbf{29} Turkoten] Tvrkoiten Z \textbf{30} ruort in] ir rvrte Z \newline
\end{minipage}
\hspace{0.5cm}
\begin{minipage}[t]{0.5\linewidth}
\small
\begin{center}*m
\end{center}
\begin{tabular}{rl}
 & Gawan vr\textit{ie}sch diu mære\\ 
 & von der juste pfandære.\\ 
 & Pl\textit{i}ppal\textit{in}ot nam alsô pfant:\\ 
 & welich juste \textbf{wart d\textit{â} bekant},\\ 
5 & daz einer \textbf{viel}, der ander \textbf{saz},\\ 
 & sô enpfienc er ân ir beider haz\\ 
 & dises verlust, \textbf{jenes} gewin,\\ 
 & \textbf{ich meine} daz ros, \textbf{daz zôch} er hin.\\ 
 & \dag er\dag  ruohte, strite\textit{n} si genuoc.\\ 
10 & wer prîs oder laster truoc,\\ 
 & des liez er \textbf{jehen die vrouwen},\\ 
 & \textbf{si} mohten dicke \textbf{schouwen}.\\ 
 & Gawanen er vaste sitzen bat.\\ 
 & er zôch imz ros an \textbf{die} stat\\ 
15 & \textbf{und} bôt im \textbf{schilt und} sper.\\ 
 & \textbf{hie} kam der Turk\textit{oi}te her\\ 
 & kalopierende als ein man,\\ 
 & der \textbf{sîn} juste \textit{m}ezzen kan\\ 
 & weder ze hôch noch zuo nider.\\ 
20 & Gawa\textit{n} \textit{k}am gegen im wider.\\ 
 & von Muntsalvasche Gringulet\\ 
 & \textbf{te\textit{t}} nâch Gawanes bet,\\ 
 & als ez der z\textit{o}um \textbf{lêrte}.\\ 
 & ûf den plân er kêrte.\\ 
25 & hurtâ, \textbf{lât} die juste tuon!\\ 
 & hie \textbf{komt} des \textbf{küniges} Lo\textit{t}es sun\\ 
 & manlîch, âne herzen schric.\\ 
 & wâ \textbf{het} diu helmsnuor \textbf{ir} stric?\\ 
 & des \textit{T}ur\textit{k}oiten just \textbf{in rafft} aldâ.\\ 
30 & Gawan \textbf{ruort\textit{e} in} anderswâ:\\ 
\end{tabular}
\scriptsize
\line(1,0){75} \newline
m n o \newline
\line(1,0){75} \newline
\newline
\line(1,0){75} \newline
\textbf{1} vriesch] freisch m \textbf{2} der] \textit{om.} o \textbf{3} Plippalinot] Plppalmot m Plippalmot n o \textbf{4} dâ] do m n o \textbf{7} jenes] vnd jennes n \textbf{9} striten] strittes m \textbf{11} des] Das o  $\cdot$ jehen] [iehen]: jehen so n ienen o \textbf{12} mohten] moͯchtent o \textbf{16} Turkoite] turciot m turcoit n tortoit o \textbf{17} kalopierende] Kalaperende o \textbf{18} mezzen] nẏessen m \textbf{20} Gawan kam] Gawan kan kam m Gawam kam n  $\cdot$ wider] hin wider n o \textbf{21} Muntsalvasche] muntsaluasce m o montsaluasche n \textbf{22} tet] Des m  $\cdot$ nâch] noch n o \textbf{23} ez] er o  $\cdot$ zoum] zuͯm m  $\cdot$ lêrte] gelerte n o \textbf{24} den] dem o \textbf{26} Lotes] loczs m lotz n o \textbf{28} diu helmsnuor ir] die helm snuͯiren n \textbf{29} Turkoiten] kurtoitten m tur kulleten o  $\cdot$ rafft] crafft n o \textbf{30} ruorte] ruͯrten m \newline
\end{minipage}
\end{table}
\newpage
\begin{table}[ht]
\begin{minipage}[t]{0.5\linewidth}
\small
\begin{center}*G
\end{center}
\begin{tabular}{rl}
 & \begin{large}G\end{large}awan vriesch diu mære\\ 
 & von der tjoste pfandære.\\ 
 & Pliplalinot nam alsô \textbf{diu} pfant:\\ 
 & swelch tjoste \textbf{dâ vür würde erkant},\\ 
5 & daz einer \textbf{geviel}, der ander \textbf{gesaz},\\ 
 & sô enpfienc er ân ir bêder haz\\ 
 & dise\textit{s} vlust \textbf{unde} \textbf{jenes} gewin,\\ 
 & \textbf{wan} daz ors \textbf{vuort} er hin.\\ 
 & ern ruohte, striten si genuoc.\\ 
10 & swer prîs oder laster truoc,\\ 
 & des liez er \textbf{die vrouwen jehen},\\ 
 & \textbf{die} mohten\textbf{z} dicke \textbf{dâ wol} \textbf{sehen}.\\ 
 & \hspace*{-.7em}\big| er zôch imz ors an \textbf{daz} stat.\\ 
 & \hspace*{-.7em}\big| Gawanen er vaste sitzen bat,\\ 
15 & \textbf{er} bôt im \textbf{in die hant ein} sper.\\ 
 & \textbf{nû} kom der Turkoite her\\ 
 & galopierende als ein man,\\ 
 & der \textbf{wo\textit{l} die} tjoste mezzen kan\\ 
 & weder ze hôch noch ze nider.\\ 
20 & Gawan kom gein im \textbf{dar} wider.\\ 
 & von Muntsalvatsche Gringuliet\\ 
 & \textbf{vuor} nâch Gawans bet,\\ 
 & als ez der zoum \textbf{lêrte}.\\ 
 & ûf den plân er kêrte\\ 
25 & \multicolumn{1}{l}{ - - - }\\ 
 & \multicolumn{1}{l}{ - - - }\\ 
 & manlîche, ân herzen schric.\\ 
 & wâ \textbf{nim\textit{et}} diu helmsnuor \textbf{den} stric?\\ 
 & des Turkoiten tjost \textbf{traf in} al dâ.\\ 
30 & Gawan \textbf{in ruorte} anderswâ:\\ 
\end{tabular}
\scriptsize
\line(1,0){75} \newline
G I L M Z \newline
\line(1,0){75} \newline
\textbf{1} \textit{Initiale} G I  \textbf{3} \textit{Initiale} L Z  \textbf{19} \textit{Initiale} I  \newline
\line(1,0){75} \newline
\textbf{1} vriesch] freisch L \textbf{3} Pliplalinot] Pliplalinon G plipalinon I Plipalinot L Plibalinot M Plipalmot Z  $\cdot$ alsô] alsus I  $\cdot$ diu] \textit{om.} L M Z \textbf{4} swelch] Welch L (M)  $\cdot$ vür] \textit{om.} I L  $\cdot$ würde] wart M Z \textbf{5} geviel] viel I Z  $\cdot$ gesaz] saz I \textbf{6} er] \textit{om.} Z \textbf{7} dises] dise G \textbf{8} wan] Jch meyine L (M) (Z)  $\cdot$ vuort] daz vurt I zoch L das zcoch M (Z)  $\cdot$ hin] yn M \textbf{10} swer] Wer L M  $\cdot$ laster] lasters Z \textbf{11} des] Das M  $\cdot$ die vrouwen jehen] ýehen die vrowen L (Z) sprechen dy vrouwen M \textbf{12} mohtenz] mochten M  $\cdot$ dâ wol sehen] da gesehen I schowen L (M) Z \textbf{14} daz] die L (M) \textbf{13} Gawanen] Gawan I (M) Z \textbf{15} Er bot im nuͯ daz sper L \textbf{16} Turkoite] turchoyte G Turkoyder I Tuͯrkoite L turkoyte M \textbf{18} wol] wolde G \textit{om.} Z  $\cdot$ die] \textit{om.} I L sine Z  $\cdot$ mezzen] gemezzen M \textbf{20} dar] her I Z \textbf{21} Muntsalvatsche] muntshaluasc I montsalvatsch Z  $\cdot$ Gringuliet] gringulieten M \textbf{23} lêrte] gelerte Z \textbf{25} \textit{Die Verse 597.25-26 fehlen} G I L M   $\cdot$ Hvrta lat die tiost tvn Z \textbf{26} Hie qvam des kvnic lotes svn Z \textbf{27} manlîche] Menlich vnd Z \textbf{28} nimet] nim G hat Z  $\cdot$ diu] der M  $\cdot$ den] ir L M Z \textbf{29} des] der I  $\cdot$ Turkoiten] turchoyten G turkoyden I  $\cdot$ traf in] in traf Z  $\cdot$ al] \textit{om.} L \textbf{30} in] ir Z  $\cdot$ anderswâ] alda M \newline
\end{minipage}
\hspace{0.5cm}
\begin{minipage}[t]{0.5\linewidth}
\small
\begin{center}*T
\end{center}
\begin{tabular}{rl}
 & Gawan vriesch diu mære\\ 
 & von der tjoste pfandære.\\ 
 & Plipalinot nam alsô pfant:\\ 
 & welch tjost \textbf{dâ vür wart erkant},\\ 
5 & daz einer \textbf{viel}, der ander \textbf{gesaz},\\ 
 & sô enpfienc er ân ir bêder haz\\ 
 & dises vlust \textbf{und} \textbf{jenes} gewin,\\ 
 & \textbf{ich mein} daz ros, \textbf{daz zôch} er hin.\\ 
 & er \textit{en}ruochte, striten si genuoc\\ 
10 & \textbf{oder} wer prîs oder laster truoc.\\ 
 & des liez er \textbf{jehen d\textit{ie} vrouwen},\\ 
 & \textbf{die} mohten\textbf{z} dicke \textbf{schouwen}.\\ 
 & \hspace*{-.7em}\big| er zôch imz ros an \textbf{daz} stat.\\ 
 & \hspace*{-.7em}\big| Gawanen er vast sitzen bat,\\ 
15 & \textbf{er} bôt im \textbf{schilt und} sper.\\ 
 & \textbf{nû} kom der Turkoyte her\\ 
 & galopiernde als ein man,\\ 
 & der \textbf{die} tjost mezzen kan\\ 
 & weder zuo hôhe noch zuo nider.\\ 
20 & Gawan kam gên im \textbf{dâ} wider.\\ 
 & von Munsalvasche Krynguliet\\ 
 & \textbf{vuor} nâch Gawans bet,\\ 
 & als ez der zoum \textbf{lêrte}.\\ 
 & ûf den plân er kêrte.\\ 
25 & hurtâ, \textbf{nû} \textbf{lâ} die tjoste tuon!\\ 
 & hie \textbf{kam} des \textbf{küniges} Lotes sun\\ 
 & menlîch, âne herzeschric.\\ 
 & wâ \textbf{hât} diu helmsnuor \textbf{ir} stric?\\ 
 & des Turkoyten tjost \textbf{traf in} aldâ.\\ 
30 & Gawan \textbf{in ruorte} anderswâ:\\ 
\end{tabular}
\scriptsize
\line(1,0){75} \newline
Q R W V U \newline
\line(1,0){75} \newline
\textbf{1} \textit{Initiale} V  \textbf{3} \textit{Initiale} W   $\cdot$ \textit{Capitulumzeichen} R  \newline
\line(1,0){75} \newline
\textbf{1} \textit{Die Verse 553.1-599.30 fehlen} U   $\cdot$ Gawin erfúr die mere R \textbf{3} Plipalinot] PLypalinot W [P*palinot]: Plypalinot V  $\cdot$ pfant] pfande W \textbf{4} Welich strit do ward fᵫr erkant R  $\cdot$ Swel tiost [d*]: wart aldo bekant V  $\cdot$ erkant] erkande W \textbf{5} viel] geuiel R (V)  $\cdot$ gesaz] sas R \textbf{7} dises] Djz V  $\cdot$ jenes] des andren R \textbf{8} ros daz] ros R \textbf{9} enruochte] ruchte Q  $\cdot$ striten] streitens W \textbf{10} oder wer] Wer R W Swer V \textbf{11} des] Das R W (V)  $\cdot$ jehen] sehen V  $\cdot$ die] der Q \textbf{12} die] Das R  $\cdot$ mohtenz] moͤchtens W mohten V \textbf{14} imz] im R  $\cdot$ daz] die W den V \textbf{13} Gawanen] Gawin R \textbf{16} Turkoyte] turkoite Q (V) Trekoite R \textbf{17} galopiernde] Gepappieret R Galapierende W \textbf{18} die] wol R W V  $\cdot$ tjost] strit R \textbf{20} Gawan] Gawin R  $\cdot$ dâ] \textit{om.} R do W V \textbf{21} Munsalvasche] muntsaluasche Q Munsaluasche R montsaluatschs W [mvnt*]: mvntsalvasche V  $\cdot$ Krynguliet] kringliet Q kringulet R kringuliet W kringvlette V \textbf{22} Gawans] Gawins R Gawanes V \textbf{25} die] \textit{om.} W  $\cdot$ tjoste] tiosten Q \textbf{26} kam] kompt R (W) (V)  $\cdot$ küniges] kúnig W  $\cdot$ Lotes] lottes R W \textbf{27} herzeschric] kertzen schrick W herzen schrik V \textbf{28} diu] des W  $\cdot$ ir] ein R \textbf{29} des] Die R  $\cdot$ Turkoyten] turkoiten Q (V)  $\cdot$ in] \textit{om.} R \textbf{30} Gawan] Gawin R \newline
\end{minipage}
\end{table}
\end{document}
