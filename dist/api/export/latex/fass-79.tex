\documentclass[8pt,a4paper,notitlepage]{article}
\usepackage{fullpage}
\usepackage{ulem}
\usepackage{xltxtra}
\usepackage{datetime}
\renewcommand{\dateseparator}{.}
\dmyyyydate
\usepackage{fancyhdr}
\usepackage{ifthen}
\pagestyle{fancy}
\fancyhf{}
\renewcommand{\headrulewidth}{0pt}
\fancyfoot[L]{\ifthenelse{\value{page}=1}{\today, \currenttime{} Uhr}{}}
\begin{document}
\begin{table}[ht]
\begin{minipage}[t]{0.5\linewidth}
\small
\begin{center}*D
\end{center}
\begin{tabular}{rl}
\textbf{79} & hinderz ors mit \textbf{eime} rôr.\\ 
 & der künec hiez Schafillor.\\ 
 & daz sper was sunder banier,\\ 
 & dâ mit er valte den degen fier.\\ 
5 & er \textbf{hetz brâht} \textbf{ûz} der heidenschaft.\\ 
 & die sîne werten in mit kraft.\\ 
 & \textbf{\begin{large}D\end{large}och} vieng er den werden man.\\ 
 & die inren tâten die ûzern sân\\ 
 & vaste \textbf{rîten} \textbf{ûfz} velt.\\ 
10 & ir vesperîe gap strîtes gelt.\\ 
 & \textbf{ez} mohte sîn ein turnei,\\ 
 & \textbf{wan} dâ lac \textbf{manec sper} enzwei.\\ 
 & Dô begunde zürnen Læhelin:\\ 
 & "sul wir \textbf{sus} entêret sîn?\\ 
15 & \textbf{daz machet}, der \textbf{den} anker treit.\\ 
 & unser eintweder den andern leit\\ 
 & noch hiute, \textbf{dâ er} unsamfte ligt.\\ 
 & si hânt uns \textbf{vil nâch} an gesigt."\\ 
 & \textbf{ir} \textbf{hurte} \textbf{gab} in rûmes vil.\\ 
20 & dâ gieng ez \textbf{ûz} der kinde spil.\\ 
 & si \textbf{worhten} mit ir henden,\\ 
 & daz den walt begunde swenden.\\ 
 & \textbf{diz} was gelîche ir beider ger.\\ 
 & sperâ, hêrre, sperâ \textbf{sper}!\\ 
25 & \textbf{doch} muose \textbf{êt} dulden Læhelin\\ 
 & einen \textbf{smæhlîchen} pîn.\\ 
 & in stach der künec von Zazamanc\\ 
 & hinderz ors wol \textbf{spers} lanc,\\ 
 & daz in \textbf{ein} rôr \textbf{geschift} was.\\ 
30 & sîne sicherheit er an sich las.\\ 
\end{tabular}
\scriptsize
\line(1,0){75} \newline
D \newline
\line(1,0){75} \newline
\textbf{7} \textit{Initiale} D  \textbf{13} \textit{Majuskel} D  \newline
\line(1,0){75} \newline
\textbf{2} Schafillor] Scafillor D \textbf{27} Zazamanc] Zazamanch D \newline
\end{minipage}
\hspace{0.5cm}
\begin{minipage}[t]{0.5\linewidth}
\small
\begin{center}*m
\end{center}
\begin{tabular}{rl}
 & hinder daz ros mit \textbf{einem} rôr.\\ 
 & der künic hiez Schaffillor.\\ 
 & daz sper was sunder banier,\\ 
 & dâ mite er valete den degen fier.\\ 
5 & er \textbf{hete ez brâht} \textbf{von} der heiden\textit{sch}aft.\\ 
 & die sîne werten in mit kraft.\\ 
 & \textbf{doch} vienc er den werden man.\\ 
 & die inren tâten die \textit{ûz}eren sân\\ 
 & vaste \textbf{rîten} \textbf{ûf daz} velt.\\ 
10 & ir ve\textit{s}perîe gap strîtes gelt.\\ 
 & \textbf{ez} mohte sîn ein tur\textit{n}ei,\\ 
 & \textbf{wanne} d\textit{â} lac \textbf{manic sper} enzwei.\\ 
 & \begin{large}D\end{large}ô begunde zürnen Lehelin:\\ 
 & "sullen wir \textbf{sus} entêret sîn\\ 
15 & \textbf{von dem}, der anker treit?\\ 
 & unser eintweder den andern leit\\ 
 & noch hiute, \textbf{dâ er} unsanfte liget.\\ 
 & si habent uns \textbf{vil nâch} an gesiget."\\ 
 & \textbf{ir} \textbf{h\textit{u}rten} \textbf{gap} in rûmes vil.\\ 
20 & dô gien\textit{c} ez \textbf{ûz} er kinde spil.\\ 
 & si \textbf{vohten} mit ir henden,\\ 
 & daz den walt begunde \textit{s}wenden.\\ 
 & \textbf{diz} was glîch ir bei\textit{d}er ger.\\ 
 & sperâ, hêrre, sperâ, \textbf{hêr}!\\ 
25 & \textbf{doch} muose \textbf{eht} dulden Lehelin\\ 
 & einen \textbf{schamelîchen} pîn.\\ 
 & in stach der künic von Zazamanc\\ 
 & hinder daz ros wol \textbf{spere} lanc,\\ 
 & daz \textit{in} \textbf{ei\textit{n}} rôr \textbf{geschiftet} was.\\ 
30 & sîne sicherheit er an sich las.\\ 
\end{tabular}
\scriptsize
\line(1,0){75} \newline
m n o \newline
\line(1,0){75} \newline
\textbf{13} \textit{Initiale} m   $\cdot$ \textit{Capitulumzeichen} n  \newline
\line(1,0){75} \newline
\textbf{1} einem] einer o \textbf{3} Das banier was sunder sper o \textbf{5} hete] hat n  $\cdot$ der] \textit{om.} n o  $\cdot$ heidenschaft] heiden craft m \textbf{6} sîne] sinen m n o \textbf{8} ûzeren] verseren m \textbf{10} vesperîe] festsperie m versperie o \textbf{11} mohte] moͯchte n  $\cdot$ turnei] turei m \textbf{12} dâ] do m n o \textbf{15} der] der do n \textbf{16} eintweder] jetweder n (o) \textbf{17} dâ] do n o  $\cdot$ unsanfte] vnuersanffte o \textbf{18} uns] \textit{om.} o  $\cdot$ nâch] wol n \textbf{19} hurten] herten m n o \textbf{20} gienc] gienge m \textbf{21} vohten] forchten o \textbf{22} swenden] wenden m \textbf{23} diz was glîch] Das waglich o  $\cdot$ beider] beiger m \textbf{24} hêr] sper n o \textbf{25} muose] muͯsse m (n)  $\cdot$ eht] \textit{om.} n o \textbf{26} einen schamelîchen] Ein schemeliche n o \textbf{27} Zazamanc] [zam*]: zazamang n zazamang o \textbf{28} spere] speres n o \textbf{29} in ein] einem m  $\cdot$ rôr] [ros]: ror n \textit{om.} o  $\cdot$ geschiftet] gescheffet n \newline
\end{minipage}
\end{table}
\newpage
\begin{table}[ht]
\begin{minipage}[t]{0.5\linewidth}
\small
\begin{center}*G
\end{center}
\begin{tabular}{rl}
 & hinderz ors mit \textbf{einem} rôre.\\ 
 & der künic hiez Tschaffilore.\\ 
 & daz sper was sunder banier,\\ 
 & dâ mit er valte den degen fier.\\ 
5 & er \textbf{brâhtez} \textbf{von} der heidenschaft.\\ 
 & die sîne werten in mit kraft.\\ 
 & \textbf{iedoch} vieng er den werden man.\\ 
 & die inneren tâten die ûzeren sân\\ 
 & vaste \textbf{rîtende} \textbf{über} velt.\\ 
10 & ir vesperîe gap strîtes gelt.\\ 
 & \textbf{ez} moht \textbf{wol} sîn ein turnei.\\ 
 & dâ lac \textbf{manic sper} enzwei.\\ 
 & dô begunde zürnen Lehelin:\\ 
 & "sulen wir \textbf{alsus} entêret sîn?\\ 
15 & \textbf{daz machet}, der \textbf{den} anker treit.\\ 
 & unser eintwedere den andern leit\\ 
 & noch hiute, \textbf{dâ er} unsanfte liget.\\ 
 & si hânt uns \textbf{vil nâch} an gesiget."\\ 
 & \textbf{grôz} \textbf{hurten} \textbf{gap} in rûmes vil.\\ 
20 & \begin{large}D\end{large}ô gieng ez \textbf{ûz} de\textit{r} kind\textit{e} spil.\\ 
 & si \textbf{worhten} mit ir henden,\\ 
 & daz den walt b\textit{eg}unde swenden.\\ 
 & \textbf{daz} was gelîch ir beider ger.\\ 
 & sperâ, hêrre, sperâ \textbf{sper}!\\ 
25 & \textbf{dô} muose dulten Lehelin\\ 
 & einen \textbf{schemelîchen} pîn.\\ 
 & in stach der künic von Zazamanc\\ 
 & hinderz ors wol \textbf{spers} lanc,\\ 
 & daz in \textbf{den} rôr \textbf{geschift} was.\\ 
30 & sîne sicherheit er an sich las.\\ 
\end{tabular}
\scriptsize
\line(1,0){75} \newline
G I O L M Q R Z Fr50 \newline
\line(1,0){75} \newline
\textbf{1} \textit{Initiale} O  \textbf{3} \textit{Initiale} L Q R Z  \textbf{13} \textit{Initiale} I M  \textbf{20} \textit{Initiale} G  \newline
\line(1,0){75} \newline
\textbf{1} hinderz] ÷inderz O  $\cdot$ einem] dem O L (M) Q R Z  $\cdot$ rôre] re M \textbf{2} hiez] \textit{om.} I  $\cdot$ Tschaffilore] tschiffilore G shiuilor I tschafillor O (L) schafillore M schaffilor Q schafillor R tsaffilor Z \textbf{3} daz] der I  $\cdot$ sunder] sundern M \textbf{4} er valte] valt er I (M)  $\cdot$ den] der M  $\cdot$ degen] helt Z \textbf{6} sîne] sinen I R  $\cdot$ in] om M  $\cdot$ kraft] ir chraft O \textbf{8} die ûzeren] die vzen O den ausern Q (Z) \textbf{9} rîtende] riten I L (Fr50)  $\cdot$ über velt] vber daz velt O vber walt Q vͦf daz velt Fr50 \textbf{10} ir] die I  $\cdot$ gelt] gezelt O \textbf{11} ez] :::n Fr50  $\cdot$ moht] [mollt]: molht I moch R  $\cdot$ wol] \textit{om.} O L M Q R Z  $\cdot$ sîn] si Fr50 \textbf{12} dâ] Wan da O L M Z (Fr50) Wan do Q Wa da R  $\cdot$ manic] manges I vil L vil der R \textbf{13} dô] Da M Z  $\cdot$ zürnen] zcorne M zuren Q  $\cdot$ Lehelin] lechelin R \textbf{14} alsus] svs O (L) (M) (Q) (R) Z \textbf{16} eintwedere] einer O Q eyn M yettwedrer R \textbf{17} dâ] do Q  $\cdot$ unsanfte] vnsaste R \textbf{18} si] sin I (Z)  $\cdot$ an gesiget] hat angesigt I \textbf{19} grôz] Jr O L M Q R Z  $\cdot$ hurten] hvrte O (M) (Q) (R) (Z)  $\cdot$ in] im Q  $\cdot$ rûmes] hurtes R \textbf{20} Dô] Da M R Z  $\cdot$ ez] er Z  $\cdot$ ûz der kinde] vz dem chindes G vber der chinde I (O) (L) (M) vzer kindes Q (R) (Z) \textbf{22} begunde] bvnde G  $\cdot$ swenden] swennen I \textbf{23} daz] Diz O L (Q) (R) (Z) \textbf{24} sperâ hêrre] Spera herra Q (Z) Sper an sper R  $\cdot$ sperâ sper] spera her I (L) (M) \textbf{25} dô] Da M Z  $\cdot$ muose] múst da Q  $\cdot$ Lehelin] lechelin R \textbf{26} einen schemelîchen] einen smehlichen I Eine schameliche L \textbf{27} in] [ein]: dein I  $\cdot$ künic] kanig L  $\cdot$ von] \textit{om.} Q  $\cdot$ Zazamanc] zazamanch G O L zazamac M \textbf{29} in den] in ein L indas M R in einem Q  $\cdot$ rôr] vor Q  $\cdot$ geschift] gesiffet I geheffttet R  $\cdot$ was] [wart]: was I \newline
\end{minipage}
\hspace{0.5cm}
\begin{minipage}[t]{0.5\linewidth}
\small
\begin{center}*T (U)
\end{center}
\begin{tabular}{rl}
 & hinder\textit{z} ors mit \textbf{dem} rôr.\\ 
 & der künec hiez Schafellor.\\ 
 & daz sper was sunder banier,\\ 
 & dâ mit er valte den degen fier.\\ 
5 & er \textbf{brâht ez} \textbf{von} der heidenschaft.\\ 
 & die sîne werten in mit kraft.\\ 
 & \textbf{iedoch} vienc er den werden man.\\ 
 & die innern tâten die ûzern sân\\ 
 & vaste \textbf{rîten} \textbf{über} velt.\\ 
10 & ir vesperîe gap strîtes gelt\\ 
 & \textbf{und} mohte sîn ein turnei,\\ 
 & \textbf{wan} d\textit{â} lac \textbf{vil spere} enzwei.\\ 
 & dô begunde zürnen Lehelin:\\ 
 & "sul wir \textbf{sus} entêret \textit{sîn}?\\ 
15 & \textbf{daz machet}, der \textbf{den} anker treit.\\ 
 & unser eintwedere den andern leit,\\ 
 & noch hiute \textbf{der} unsanfte liget.\\ 
 & si hânt uns \textbf{zuo vil} an gesiget."\\ 
 & \textbf{ir} \textbf{hurte} \textbf{gâben} in roumes vil.\\ 
20 & dô gieng ez \textbf{ûzer} der kinde spil.\\ 
 & si \textbf{worh\textit{t}en} mit ir henden,\\ 
 & daz den walt begunde swenden.\\ 
 & \textbf{diz} was glîch ir beider ger.\\ 
 & sperâ, hêrre, sperâ \textbf{sper}!\\ 
25 & \textbf{dô} muoste dulten Lehelin\\ 
 & einen \textbf{schamelîchen} pîn.\\ 
 & in stach der künec von Zazamanc\\ 
 & hinder\textit{z} ors wol \textbf{eines} \textbf{spers} lanc,\\ 
 & daz in \textbf{dem} rôr \textbf{schafte} was.\\ 
30 & sîne sicherheit er an sich las.\\ 
\end{tabular}
\scriptsize
\line(1,0){75} \newline
U V W T \newline
\line(1,0){75} \newline
\textbf{3} \textit{Initiale} W  \textbf{8} \textit{Majuskel} T  \textbf{13} \textit{Initiale} V   $\cdot$ \textit{Majuskel} T  \textbf{19} \textit{Majuskel} T  \textbf{21} \textit{Majuskel} T  \textbf{23} \textit{Majuskel} T  \textbf{25} \textit{Majuskel} T  \textbf{29} \textit{Majuskel} T  \newline
\line(1,0){75} \newline
\textbf{1} hinderz] Hinder U  $\cdot$ dem] eime V (T)  $\cdot$ rôr] roß W \textbf{2} Schafellor] Schaffillore V de schaffilloß W TschafilloR T \textbf{9} rîten über] ritende v́ber V vber ienez T \textbf{10} gelt] widergelt W \textbf{11} und] ez T \textbf{12} dâ] do U V W  $\cdot$ vil] manig W (T) \textbf{13} zürnen] reden W  $\cdot$ Lehelin] Loͤhelin V \textbf{14} sîn] \textit{om.} U \textbf{16} andern] ander T \textbf{17} der] do er V W da er T \textbf{18} zuo vil] vil nahe W vaste T \textbf{19} hurte] hvrten T  $\cdot$ gâben in] gab vnß W gabin T \textbf{20} ûzer der kinde] auß der kinde W vz ir kindes T \textbf{21} worhten] worden U \textbf{23} diz] Das W \textbf{24} Spera her spera her W sper aherre sprea sper T \textbf{25} muoste] muͤste V (T)  $\cdot$ Lehelin] Loͤhelin V lehelein W \textbf{26} einen schamelîchen] eine schemeliche V (W) \textbf{27} Zazamanc] zazamang V W \textbf{28} hinderz] Hinder U  $\cdot$ wol eines spers] eins ackers W \textbf{29} dem] [*]: ein V den W T  $\cdot$ schafte] gescheftet V [*]: gesceftet T geschiffet W \textbf{30} las] laz T \newline
\end{minipage}
\end{table}
\end{document}
