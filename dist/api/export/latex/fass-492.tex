\documentclass[8pt,a4paper,notitlepage]{article}
\usepackage{fullpage}
\usepackage{ulem}
\usepackage{xltxtra}
\usepackage{datetime}
\renewcommand{\dateseparator}{.}
\dmyyyydate
\usepackage{fancyhdr}
\usepackage{ifthen}
\pagestyle{fancy}
\fancyhf{}
\renewcommand{\headrulewidth}{0pt}
\fancyfoot[L]{\ifthenelse{\value{page}=1}{\today, \currenttime{} Uhr}{}}
\begin{document}
\begin{table}[ht]
\begin{minipage}[t]{0.5\linewidth}
\small
\begin{center}*D
\end{center}
\begin{tabular}{rl}
\textbf{492} & "\begin{large}D\end{large}û rite ein angestlîche vart",\\ 
 & sprach der wirt, "\textbf{durch warte} wol bewart.\\ 
 & ieslîchiu sô besetzet ist\\ 
 & mit \textbf{rotte}, selten iemans list\\ 
5 & in hilfet gein der reise;\\ 
 & er kêrte ie gein der vreise,\\ 
 & \textbf{swer} \textbf{jenen} her dâ zuo \textbf{z}in reit.\\ 
 & si nement niemens sicherheit,\\ 
 & si wâgent ir leben \textbf{gein} \textbf{jenes} leben;\\ 
10 & daz ist vür sünde in dâ gegeben."\\ 
 & "nû kom ich âne strîten\\ 
 & \textbf{an} den selben zîten\\ 
 & geriten, dâ der künec was",\\ 
 & sprach Parzival. "des palas\\ 
15 & sach ich des âbents jâmers vol.\\ 
 & wie tet \textbf{in} jâmer dô \textbf{sô} wol?\\ 
 & ein knappe al dâ zer tür în spranc,\\ 
 & dâ von der palas jâmers klanc.\\ 
 & der truoc in sînen henden\\ 
20 & einen schaft ze\textbf{n} vier wenden,\\ 
 & dâr inne \textbf{ein} sper bluotec rôt.\\ 
 & des kom diu diet in jâmers nôt."\\ 
 & Der wirt sprach: "\textbf{neve}, sît noch \textbf{hie}\\ 
 & wart dem künege nie\\ 
25 & \textbf{sô wê}, wan dô sîn komen zeigete sus\\ 
 & der sterne Saturnus;\\ 
 & der kan mit grôzem vroste komen.\\ 
 & \textbf{drûf} legen moht uns niht gevromen,\\ 
 & als man\textbf{z} \textbf{ê drûffe} ligen sach.\\ 
30 & daz sper man in die wunden stach.\\ 
\end{tabular}
\scriptsize
\line(1,0){75} \newline
D Fr11 Fr31 \newline
\line(1,0){75} \newline
\textbf{1} \textit{Initiale} D  \textbf{23} \textit{Majuskel} D  \newline
\line(1,0){75} \newline
\textbf{1} Dv rit ain dan angestliche not Fr31 \textbf{2} Sprach der wirt dv wær bewart Fr31 \textbf{3} ieslîchiu] Jetslich wec Fr31 \textbf{4} rotte] roter Fr31 \textbf{6} kêrte] kert Fr31 \textbf{7} jenen] einen Fr31  $\cdot$ zin] im Fr31 \textbf{14} Parzival] Parcifal D \textbf{16} dô] da Fr11 \textbf{20} zen] ze Fr31 \textbf{23} neve sît] senit Fr11  $\cdot$ hie] ee Fr11 (Fr31) \textbf{24} nie] nye so we Fr11 (Fr31) \textbf{25} sô wê] \textit{om.} Fr11 Fr31  $\cdot$ dô] da Fr11 \textit{om.} Fr31 \textbf{26} Saturnus] Saturnvͯs Fr11 satvrnivs Fr31 \textbf{27} der] \textit{om.} Fr31 \textbf{28} legen] gelegen Fr31 \textbf{29} manz] mans Fr11  $\cdot$ ê] \textit{om.} Fr31 \textbf{30} in] \textit{om.} Fr31 \newline
\end{minipage}
\hspace{0.5cm}
\begin{minipage}[t]{0.5\linewidth}
\small
\begin{center}*m
\end{center}
\begin{tabular}{rl}
 & "\dag \begin{large}N\end{large}û\dag  rite ein angestlîche vart",\\ 
 & sprach der wirt, " wol bewart.\\ 
 & ieglîch sô besetzet ist\\ 
 & mit \textbf{rotten}, selten \dag ieman\dag  list\\ 
5 & \dag nû helfet\dag  gegen der reise;\\ 
 & er kêrte ie gegen der vreise,\\ 
 & \textbf{wer} \textbf{jenen} her d\textit{â} zuo \dag im\dag  reit.\\ 
 & s\textit{i} nement niemens sicherheit,\\ 
 & si wâgent ir leben \textbf{gegen}\textbf{s} leben;\\ 
10 & daz ist vür sünde in d\textit{â} gege\textit{be}n."\\ 
 & "nû kam ich âne strîten\\ 
 & \textbf{zuo} den selben zîten\\ 
 & geriten, d\textit{â} der künic was",\\ 
 & sprach Parcifal. "des palas\\ 
15 & sach ich des âbents jâmers vol.\\ 
 & wie tet \textbf{in} jâmer dô \textbf{alsô} wol?\\ 
 & ein knappe aldâ zuor \textit{türe} în spranc,\\ 
 & dâ von der palas \dag jâmer\dag  klanc.\\ 
 & der truoc in sînen henden\\ 
20 & einen schaft zuo vier wenden,\\ 
 & dâr în \textbf{ein} sper \textbf{al}bluotic rôt.\\ 
 & des kam diu diet in jâmers nôt."\\ 
 & der wirt sprach: "\textbf{neve}, sît no\textit{ch} \textbf{ê}\\ 
 & wart dem künic nie \textbf{sô wê},\\ 
25 & wan dô sîn komen zeigte sus\\ 
 & der sterne Saturnus;\\ 
 & der \dag kam\dag  mit grôzem vroste komen.\\ 
 & \textbf{dar ûf} legen moht uns niht gevromen,\\ 
 & als man \textbf{ê dâr ûf} ligen sach\\ 
30 & daz sper man in die wunden stach.\\ 
\end{tabular}
\scriptsize
\line(1,0){75} \newline
m n o \newline
\line(1,0){75} \newline
\textbf{1} \textit{Initiale} m n  \newline
\line(1,0){75} \newline
\textbf{7} jenen] jnnen o  $\cdot$ dâ] do m n o \textbf{8} si] So m \textbf{9} gegens] gegen n \textbf{10} dâ] do m n o  $\cdot$ gegeben] gegen m \textbf{13} dâ] do m n o \textbf{16} tet] tút o  $\cdot$ dô] da o  $\cdot$ alsô] so n o \textbf{17} türe] \textit{om.} m \textbf{20} zuo] zuͯ den n \textbf{21} rôt] \textit{om.} o \textbf{23} noch] [*]: nohe m \textbf{24} künic] \sout{kein} konige o \textbf{25} zeigte] zoigte o \textbf{26} Saturnus] saturnús o \textbf{28} moht] moͯchte n (o) \newline
\end{minipage}
\end{table}
\newpage
\begin{table}[ht]
\begin{minipage}[t]{0.5\linewidth}
\small
\begin{center}*G
\end{center}
\begin{tabular}{rl}
 & "\begin{large}D\end{large}û rite ein angestlîche vart",\\ 
 & sprach der wirt, "\textbf{durch wart} wol bewart.\\ 
 & iegeslîchiu sô besetzet ist\\ 
 & mit \textbf{rote}, selten iemannes list\\ 
5 & in hilfet gein der reise;\\ 
 & er kêrte ie gein der vreise,\\ 
 & \textbf{swer} \textbf{jenen} her dâ zuo in reit.\\ 
 & si nement nieme\textit{n}s sicherheit,\\ 
 & si wâgent ir leben \textbf{gein} \textbf{jenes} leben;\\ 
10 & daz ist vür sünde in dâ gegeben."\\ 
 & "nû kom ich âne strîten\\ 
 & \textbf{an} den selben zîten\\ 
 & geriten, dâ der künic was",\\ 
 & sprach Parzival. "des palas\\ 
15 & sach ich des â\textit{b}endes jâmers vol.\\ 
 & wie tet \textbf{in} jâmer dô \textbf{sô} wol?\\ 
 & ein knappe al dâ zer tür în spranc,\\ 
 & dâ von der palas jâmers klanc.\\ 
 & der truoc in sînen henden\\ 
20 & einen schaft ze\textbf{n} vier wenden,\\ 
 & dâr inne \textbf{ein} sper bluot\textit{ic} rôt.\\ 
 & des kom diu diet in jâmers nôt."\\ 
 & der wirt sprach: "\textbf{neve}, sît noch \textbf{ê}\\ 
 & wart dem künige nie \textbf{sô wê},\\ 
25 & wan dô sî\textit{n} komen zeigte \textit{s}us\\ 
 & der sterne Saturnus;\\ 
 & der kan mit grôzem vroste komen.\\ 
 & \textbf{drûf} legen moht uns niht gevromen,\\ 
 & als man\textbf{z} \textbf{ê drûfe} ligen sach.\\ 
30 & daz sper man in die wunden stach.\\ 
\end{tabular}
\scriptsize
\line(1,0){75} \newline
G I O L M Z Fr49 \newline
\line(1,0){75} \newline
\textbf{1} \textit{Initiale} G I O L Z  \textbf{17} \textit{Initiale} I  \newline
\line(1,0){75} \newline
\textbf{1} Dû rite] ÷v rite O Da rite ich M  $\cdot$ angestlîche] ængstlichiv O \textbf{2} der wirt] er O  $\cdot$ durch wart wol] die wart sint I (Fr49) dach wart her M \textbf{3} besetzet] besezzen I Fr49 \textbf{4} rote] rotten I (Fr49)  $\cdot$ iemannes] mans M  $\cdot$ list] ist O \textbf{5} in] Jm Fr49 \textbf{6} kêrte] chert I (O) (L) (Fr49)  $\cdot$ vreise] [reise]: freise M \textbf{7} swer] swenne I Wer L M Wenn Fr49  $\cdot$ jenen her] ein hêr I (Fr49) geyn yn her M ienen het Z  $\cdot$ in] im O zin Z \textbf{8} nement] namen I (Fr49)  $\cdot$ niemens] niemes G \textbf{9} wâgent] wagtan I (M) (Fr49) \textbf{10} dâ] \textit{om.} O L \textbf{11} kom ich] chomen I \textbf{14} Parzival] parziual G parzifal I L M Parcifal O (Z) (Fr49) \textbf{15} ich] \textit{om.} Z  $\cdot$ âbendes] amendes G \textbf{16} in] im I  $\cdot$ jâmer dô] do iamer I iamer da M Z \textbf{17} al] \textit{om.} I O Z \textbf{20} schaft] schilt L  $\cdot$ zen] ze I O (Z) \textbf{21} ein] daz O  $\cdot$ bluotic] blvͦte G \textbf{22} des kom diu] Da quam der M \textbf{23} neve] \textit{om.} Z \textbf{24} künige] wirte Z \textbf{25} dô] da M Z  $\cdot$ sîn] sine G  $\cdot$ sus] alsvs G her sus M \textbf{26} Saturnus] Satvrnuͯs L \textbf{27} der mit grozem vroste was chomen I (Fr49)  $\cdot$ Der kan mit groszin vrosten komen M \textbf{28} drûf legen] Daz vf legn O  $\cdot$ moht] moͤht Z mag Fr49 \textbf{29} manz ê] manz O (Fr49) man en M \newline
\end{minipage}
\hspace{0.5cm}
\begin{minipage}[t]{0.5\linewidth}
\small
\begin{center}*T
\end{center}
\begin{tabular}{rl}
 & "\textit{\begin{large}D\end{large}}û rite ein angestlîche vart",\\ 
 & sprach der wirt, "\textbf{iedoch wære dû} wol bewart.\\ 
 & ieslîch\textit{iu} \textbf{warte} sô besetzet ist\\ 
 & mit \textbf{reite}, selten ieman\textit{es} list\\ 
5 & in hilfet gegen der reise;\\ 
 & er kêrte ie gegen der vreise,\\ 
 & \textbf{swenne} \textbf{iemen} her dar zuo in reit.\\ 
 & si nement niemannes sicherheit,\\ 
 & si wâgent ir leben \textbf{umbe} \textbf{jenes} leben;\\ 
10 & daz ist vür sünde in dâ gegeben."\\ 
 & "Nû kom ich âne strîten\\ 
 & \textbf{an} den selben zîten\\ 
 & geriten, dâ der künec was",\\ 
 & sprach Parcifal. "de\textit{s} palas\\ 
15 & sach ich des âbendes jâmers vol.\\ 
 & wie tet \textbf{mîn} jâmer dô \textbf{sô} wol?\\ 
 & Ein knappe aldâ zer tür în spranc,\\ 
 & dâ von der palas \textbf{vol} jâmers klanc.\\ 
 & der truoc in sînen henden\\ 
20 & einen schaft ze vier wenden,\\ 
 & dâr inne \textbf{daz} sper bluotic rôt.\\ 
 & des kom diu diet in jâmers nôt."\\ 
 & Der wirt sprach: "sît noch \textbf{ê}\\ 
 & wart dem künege nie \textbf{sô wê},\\ 
25 & wan d\textit{ô} sîn komen zeigete sus\\ 
 & der sterne Saturnus;\\ 
 & \textit{der} kan mit grôzem vroste komen.\\ 
 & \textbf{daz ûf} legen mohtuns niht gevromen,\\ 
 & als man\textit{\textbf{z}} \textbf{drûffe ê} ligen sach.\\ 
30 & daz sper man in die wunden stach.\\ 
\end{tabular}
\scriptsize
\line(1,0){75} \newline
T U V W Q R Fr40 \newline
\line(1,0){75} \newline
\textbf{1} \textit{Initiale} T Fr40   $\cdot$ \textit{Capitulumzeichen} R  \textbf{11} \textit{Majuskel} T  \textbf{17} \textit{Majuskel} T  \textbf{23} \textit{Initiale} W R   $\cdot$ \textit{Majuskel} T  \newline
\line(1,0){75} \newline
\textbf{1} \textit{Die Verse 453.1-502.30 fehlen} U   $\cdot$ Dû] Nv T  $\cdot$ angestlîche] angtsche R \textbf{2} iedoch wære dû] [*]: durch warte V durch wort W durch warte Q R (Fr40)  $\cdot$ wol bewart] [*]: wol bewart V bewart W \textbf{3} ieslîchiu] ieslige T (R)  $\cdot$ warte] \textit{om.} W Q R Fr40  $\cdot$ besetzet] [besetzte]: besetztet Q \textbf{4} reite] rotte V W Q (Fr40) Rate R  $\cdot$ selten] solten R  $\cdot$ iemanes] ieman T [iem*n]: iemanz V \textbf{5} in] Ym Q \textbf{6} kêrte] kert Q \textbf{7} swenne] Wer W Q R Swer Fr40  $\cdot$ iemen] iene W (Q) (R) Fr40  $\cdot$ dar zuo] zu Q da von R  $\cdot$ in] [im]: in V \textbf{8} nement] nemencz R  $\cdot$ niemannes] nẏement R \textbf{9} umbe] gegen V (W) (Q) (R) (Fr40) \textbf{10} dâ] do V W Q \textbf{11} strîten] [*]: striten V stritte R \textbf{13} dâ] do V W Q R  $\cdot$ was] sas R \textbf{14} Parcifal] parzifal V Fr40 partzifal W (Q) parczifal R  $\cdot$ des] dez T V [de*]: den Fr40 \textbf{15} sach] Da sach R  $\cdot$ vol] fil R \textbf{16} mîn] [im]: in V in W Q (R) Fr40  $\cdot$ dô sô wol] oͯne zil R \textbf{17} aldâ] do W  $\cdot$ în] hin Q \textbf{18} vol] \textit{om.} V W Q R Fr40 \textbf{20} ze] [z*]: zen V zen W \textbf{21} daz] [*]: ein V \textbf{22} diu] der Q \textbf{23} sît] neve sit V (W) (Q) (R) neve [*]: seit  Fr40 \textbf{25} dô] dv T  $\cdot$ komen] krone W \textbf{27} der kan] kan T Waz V \textbf{28} daz] [Da*]: Da V  $\cdot$ mohtuns] moͤcht vns W  $\cdot$ gevromen] befrumn Q \textbf{29} manz] mans T men V  $\cdot$ drûffe ê] e druffe V (W) (Q) (R) (Fr40) \textbf{30} daz] Do W  $\cdot$ sper] der Q Fr40 \newline
\end{minipage}
\end{table}
\end{document}
