\documentclass[8pt,a4paper,notitlepage]{article}
\usepackage{fullpage}
\usepackage{ulem}
\usepackage{xltxtra}
\usepackage{datetime}
\renewcommand{\dateseparator}{.}
\dmyyyydate
\usepackage{fancyhdr}
\usepackage{ifthen}
\pagestyle{fancy}
\fancyhf{}
\renewcommand{\headrulewidth}{0pt}
\fancyfoot[L]{\ifthenelse{\value{page}=1}{\today, \currenttime{} Uhr}{}}
\begin{document}
\begin{table}[ht]
\begin{minipage}[t]{0.5\linewidth}
\small
\begin{center}*D
\end{center}
\begin{tabular}{rl}
\textbf{548} & \begin{large}D\end{large}er schifman hôrte, daz er ranc\\ 
 & mit sorge unt daz in minne twanc.\\ 
 & dô sprach er: "hêrre, ez ist hie reht,\\ 
 & ûfem plâne unt in dem fôreht\\ 
5 & unt \textbf{al}dâ Clinschor hêrre ist.\\ 
 & zagheit noch manlîch list\\ 
 & \textbf{gevüegent}\textbf{z} anders niht wan sô:\\ 
 & hiute \textbf{riwec}, morgen vrô.\\ 
 & ez ist iu lîhte unbekant,\\ 
10 & gar âventiure ist \textbf{al} ditze lant;\\ 
 & sus \textbf{vert} ez naht unt \textbf{ouch den} tac.\\ 
 & bî manheit sælde helfen mac.\\ 
 & Diu sunne kan sô nider stên;\\ 
 & hêrre, ir sult ze schiffe gên."\\ 
15 & des bat in der schifman.\\ 
 & Lischoysen vuorte Gawan\\ 
 & \textbf{mit im dannen} ûf den wâc.\\ 
 & gedulteclîch ân allen bâc\\ 
 & man den helt des volgen sach.\\ 
20 & der verje \textbf{zôch} daz ors hin nâch.\\ 
 & Sus vuoren si über an den stat.\\ 
 & der verje Gawanen bat:\\ 
 & "sît selbe wirt in mîme hûs."\\ 
 & daz stuont alsô, daz Artus\\ 
25 & ze Nantes, dâ er dicke saz,\\ 
 & niht \textbf{m\textit{ö}hte} hân gebûwet baz.\\ 
 & dâ vuort er Lischoysen în.\\ 
 & der wirt unt daz gesinde sîn\\ 
 & sich des underwunden.\\ 
30 & an den selben stunden\\ 
\end{tabular}
\scriptsize
\line(1,0){75} \newline
D \newline
\line(1,0){75} \newline
\textbf{1} \textit{Initiale} D  \textbf{13} \textit{Majuskel} D  \textbf{21} \textit{Majuskel} D  \newline
\line(1,0){75} \newline
\textbf{5} Clinschor] Clynscor D \textbf{16} Lischoysen] Liscoysen D \textbf{26} möhte] mohte D \textbf{27} Lischoysen] Liscoysen D \newline
\end{minipage}
\hspace{0.5cm}
\begin{minipage}[t]{0.5\linewidth}
\small
\begin{center}*m
\end{center}
\begin{tabular}{rl}
 & \begin{large}D\end{large}er sch\textit{i}fman hôrt, daz er ranc\\ 
 & mit sorge und daz i\textit{n} minne twanc.\\ 
 & dô sprach \textit{er}: "hêrre, ez ist hie reht,\\ 
 & ûf dem plân un\textit{d} \textit{i}n dem fôreht\\ 
5 & und \textbf{al}dâ Clins\textit{o}r hêrre ist.\\ 
 & zagheit noch \dag minnes\dag  list\\ 
 & \textbf{vüegen}\textbf{z} ander\textit{s} nih\textit{t} \textit{w}an sô:\\ 
 & hiute \textbf{riuwic}, morgen vrô.\\ 
 & ez ist iu lîhte unbekant,\\ 
10 & gar âventiur ist \textbf{al} diz lant;\\ 
 & sus \textbf{wert} ez naht und tac.\\ 
 & bî manheit sælde helfen mac.\\ 
 & diu sunne kan sô nider stân;\\ 
 & hêrre, ir sullet zuo schiffe gân."\\ 
15 & des bat in der schifman.\\ 
 & Lischoisen vuorte Gawan\\ 
 & \textbf{mit im dannen} ûf den wâc.\\ 
 & gedulteclîchen âne allen bâc\\ 
 & man den helt des volgen sach.\\ 
20 & der verje \textbf{zôch} daz ros hin nâch.\\ 
 & sus vuoren si über an den stat.\\ 
 & der verige Gawan bat:\\ 
 & "sît selbe wirt in mîme hûs."\\ 
 & daz stuont alsô, daz Artu\textit{s}\\ 
25 & zuo \textit{N}antes, d\textit{â} er dicke saz,\\ 
 & niht \textbf{d\textit{ö}rft} hân geb\textit{ûw}en baz.\\ 
 & d\textit{â} vuorte er Lischoisen în.\\ 
 & der wirt und daz gesinde sîn\\ 
 & sich des underw\textit{u}nden.\\ 
30 & an den selben stunden\\ 
\end{tabular}
\scriptsize
\line(1,0){75} \newline
m n o \newline
\line(1,0){75} \newline
\textbf{1} \textit{Initiale} m   $\cdot$ \textit{Capitulumzeichen} n  \textbf{14} \textit{Illustration mit Überschrift:} Also liscoisen der schiffman hern gawan zuͦ schiff furte vnd uͯber mer wolten o   $\cdot$ \textit{Initiale} o  \textbf{15} \textit{Illustration mit Überschrift:} Also lẏscoẏsen der schiffman her gawan zuͯ schiff fuͯrte vnd vber mer wolte n   $\cdot$ \textit{Großinitiale} n  \newline
\line(1,0){75} \newline
\textbf{1} schifman] schimpf man m \textbf{2} daz] des n  $\cdot$ in] ẏr m \textbf{3} er] \textit{om.} m o \textbf{4} und in] vnd vnd in m \textbf{5} aldâ] also o  $\cdot$ Clinsor] clinser m \textbf{6} minnes] mẏnes o \textbf{7} anders niht wan] ander nit also wan m \textbf{11} sus] So o \textbf{16} Lischoisen] Liscoisen m n o \textbf{19} volgen] vorgen n \textbf{20} der] Die o \textbf{21} über] aber n \textbf{22} verige] vierige o  $\cdot$ Gawan] gawanen n o \textbf{23} selbe] selbes n \textbf{24} Artus] artuse m n artuͯse o \textbf{25} Nantes] mantes m  $\cdot$ dâ] do m n o \textbf{26} dörft] dorft m (n) duͯrfft o  $\cdot$ gebûwen] gebliben m \textbf{27} dâ] Do m n o  $\cdot$ Lischoisen] liscoisen m n lioscoisen o \textbf{29} underwunden] vnder winden m \newline
\end{minipage}
\end{table}
\newpage
\begin{table}[ht]
\begin{minipage}[t]{0.5\linewidth}
\small
\begin{center}*G
\end{center}
\begin{tabular}{rl}
 & \begin{large}D\end{large}er schifman hôrte, daz er ranc\\ 
 & mit sorge unde daz in minne twanc.\\ 
 & dô sprach er: "hêrre, ez ist hie reht,\\ 
 & ûffe\textit{m} plâne unde \textit{in} dem fôreht\\ 
5 & unde \textbf{al} dâ Clinsor hêrre ist.\\ 
 & zageheit noch manlîch list\\ 
 & \textbf{vüegent}\textbf{z} anders niht wan sô:\\ 
 & hiute \textbf{riuwec}, morgen vrô.\\ 
 & ez ist \textit{iu lîhte} unbekant,\\ 
10 & gar âventiure ist ditze lant;\\ 
 & sus \textbf{wert} ez naht unde \textbf{ouch den} tac.\\ 
 & bî manheit sælde helfen mac.\\ 
 & diu s\textit{u}nne kan sô nider stên;\\ 
 & hêrre, ir sult ze scheffe gên."\\ 
15 & des bat in der schifman.\\ 
 & Lishoisen \textit{vuor}t Gawan\\ 
 & \textbf{mit im dannen} ûf de\textit{n} wâc.\\ 
 & gedulticlîche ân allen bâc\\ 
 & man den helt \textit{des} volgen sach.\\ 
20 & der verje \textbf{vuorte}z ors hin nâch.\\ 
 & sus vuorens über an daz stat.\\ 
 & der verje Gawanen bat:\\ 
 & "sît selbe wirt in mînem hûs."\\ 
 & daz stuont alsô, daz Artus\\ 
25 & ze Nantis, dâ er dicke saz,\\ 
 & niht \textbf{d\textit{ö}rfte} hân gebûwet baz.\\ 
 & dâ vuort er Lyshoisen în.\\ 
 & der wirt unde daz gesinde sîn\\ 
 & sich des underwunden.\\ 
30 & an den selben stunden\\ 
\end{tabular}
\scriptsize
\line(1,0){75} \newline
G I L M Z \newline
\line(1,0){75} \newline
\textbf{1} \textit{Initiale} G L Z  \textbf{11} \textit{Initiale} I  \newline
\line(1,0){75} \newline
\textbf{2} sorge] sorgen I (M) Z \textbf{3} dô] Da M  $\cdot$ ez ist hie] hie ist I isz ist M \textbf{4} ûffem] vffen G  $\cdot$ in] \textit{om.} G \textbf{5} Clinsor] clinisor G (L) klinshor Z \textbf{6} manlîch] menlichn M \textbf{7} vüegentz] vuͦgt ez I Fuͯgtenz L (M) \textbf{9} iu lîhte] lihte iv G \textbf{10} gar] Gar vol L  $\cdot$ ditze] aldisz M \textbf{11} naht] tac M  $\cdot$ ouch den] \textit{om.} I L M  $\cdot$ tac] nacht M \textbf{12} sælde] seldin M  $\cdot$ helfen mac] helfe phlac I (Z) \textbf{13} diu] disiu I  $\cdot$ sunne] senne G \textbf{16} Lishoisen] Lishosien G liscoisen I Lýtschoýsen L Lisois M  $\cdot$ vuort] bat G  $\cdot$ Gawan] her Gawan I \textbf{17} den] dem G dē M \textbf{18} gedulticlîche] Genedechlichen I  $\cdot$ allen] \textit{om.} I \textbf{19} des] \textit{om.} G \textbf{20} verje] vergen M  $\cdot$ vuortez] zoch daz L (M) Z \textbf{21} daz] den L Z \textbf{22} Gawanen] Gawan I \textbf{23} selbe] selben M (Z) \textbf{24} Artus] Artuͯs L \textbf{25} Nantis] Nantes G I Z  $\cdot$ saz] waz L \textbf{26} dörfte] dorfte G (I) L (M) (Z)  $\cdot$ gebûwet] [erf]: erbuwen I \textbf{27} Lyshoisen] liscoisen I Lytschoysen L lisoisen M \textbf{29} sich des] des sich I \newline
\end{minipage}
\hspace{0.5cm}
\begin{minipage}[t]{0.5\linewidth}
\small
\begin{center}*T
\end{center}
\begin{tabular}{rl}
 & \textit{\begin{large}D\end{large}}er schifman hôrte, daz er ranc\\ 
 & mit sorge unde daz in minne twanc.\\ 
 & dô sprach er: "hêrre, ez ist hie reht,\\ 
 & ûf dem plâne unde in dem fô\textit{r}eht\\ 
5 & unde dâ Clynsor hêrre ist.\\ 
 & zageheit noch manlîcher list\\ 
 & \textbf{vüegent} anders niht wan sô:\\ 
 & hiute \textbf{trûren}, morne vrô.\\ 
 & ez ist iu lîhte unbekant,\\ 
10 & gar âventiure ist diz lant;\\ 
 & su\textit{s} \textbf{wert} ez naht unde tac.\\ 
 & bî manheit sælde helfen mac.\\ 
 & Diu sunne kan sô nider stên;\\ 
 & hêrre, ir sult ze schiffe gên."\\ 
15 & des bat in der schifman.\\ 
 & Lyschoysen vuorte Gawan\\ 
 & \textbf{dannen mit im} ûf den wâc.\\ 
 & gedulteclîche âne allen bâc\\ 
 & man den helt des volgen sach.\\ 
20 & der verje \textbf{zôch} daz ors hin nâch.\\ 
 & Sus vuorens über an den stat.\\ 
 & der verge Gawanen bat:\\ 
 & "sît selbe wirt in mînem hûs."\\ 
 & daz stuont alsô, daz Artus\\ 
25 & ze Nantis, dâ er dicke \textit{s}a\textit{z},\\ 
 & niht \textbf{d\textit{ö}rfte} hân gebûwen baz.\\ 
 & dâ vuorter Lyschoysen în.\\ 
 & der wirt unde daz gesinde sîn\\ 
 & sich des underwunden.\\ 
30 & an den selben stunden\\ 
\end{tabular}
\scriptsize
\line(1,0){75} \newline
T U V W O Q R \newline
\line(1,0){75} \newline
\textbf{9} \textit{Initiale} O Q  \textbf{13} \textit{Majuskel} T  \textbf{21} \textit{Majuskel} T  \textbf{28} \textit{Initiale} V  \newline
\line(1,0){75} \newline
\textbf{1} Der] ÷er T \textbf{2} sorge] sorgen U V O R  $\cdot$ unde] \textit{om.} W  $\cdot$ daz in] mit Q \textbf{4} in dem] in den R  $\cdot$ fôreht] forheht T \textbf{5} dâ] do V aldo W (O) (Q) R  $\cdot$ Clynsor] klinsor V klynshor W Clinshors O klin clinszhor Q Clingshor R \textbf{7} vüegent] Fuͤgent es W (Q) Wegntz O Fuͯgtencz R  $\cdot$ wan] \textit{om.} Q  $\cdot$ sô] also W \textbf{8} trûren] traurig W (R) riwech O (Q) \textbf{9} ez] ÷z O  $\cdot$ iu] \textit{om.} Q  $\cdot$ lîhte] lechter W \textbf{10} diz] daz O \textbf{11} sus] sv T \textbf{13} sô] sy R \textbf{14} \textit{korrigierende Versdoppelung:} Herre ir sult zu helffe gen Herre ir sult ze [helffe]: schiffe gen R  \textbf{15} in] in sere W  $\cdot$ schifman] dinst man schifman Q \textbf{16} Lyschoysen] Lyscoyen T Lyschoien U Lischoien V Lyshoien W Lyshoysen O Lyszhoisen Q Lyschoisen R  $\cdot$ vuorte] fuͦrt R \textbf{18} Gedultenklich oͯn alle frag R  $\cdot$ allen] alle W \textbf{19} des] \textit{om.} W \textbf{20} daz] im das W \textbf{21} Sus] Als Q  $\cdot$ vuorens] vuͦren iz U fuͦrencz R  $\cdot$ über an den] in die W vber an daz O (Q) (R) \textbf{22} Gawanen] Gewainen R \textbf{23} selbe] selber W \textbf{24} alsô] als V  $\cdot$ Artus] artuß W \textbf{25} Nantis] nantes V Natis R  $\cdot$ dâ] do U V W O Q  $\cdot$ saz] was T \textbf{26} dörfte] dorfte T U V O (Q) (R)  $\cdot$ gebûwen] gebawet W \textbf{27} Lyschoysen] Lyscoyen T lischoien U [*]: lẏschosen V lyshoien W Lyshoysen O lishoysen Q Lyschoisen R  $\cdot$ în] [*]: in V \textbf{28} gesinde] [*]: ingesinde V geside W \newline
\end{minipage}
\end{table}
\end{document}
