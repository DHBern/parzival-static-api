\documentclass[8pt,a4paper,notitlepage]{article}
\usepackage{fullpage}
\usepackage{ulem}
\usepackage{xltxtra}
\usepackage{datetime}
\renewcommand{\dateseparator}{.}
\dmyyyydate
\usepackage{fancyhdr}
\usepackage{ifthen}
\pagestyle{fancy}
\fancyhf{}
\renewcommand{\headrulewidth}{0pt}
\fancyfoot[L]{\ifthenelse{\value{page}=1}{\today, \currenttime{} Uhr}{}}
\begin{document}
\begin{table}[ht]
\begin{minipage}[t]{0.5\linewidth}
\small
\begin{center}*D
\end{center}
\begin{tabular}{rl}
\textbf{813} & \textbf{\begin{large}I\end{large}r} bêder vater \textbf{hiez} Frimutel.\\ 
 & glîch antlütze unt glîchez vel\\ 
 & Anfortas bî sîner swester truoc.\\ 
 & der heiden sach an si genuoc\\ 
5 & unt \textbf{aber} \textbf{wider dicke} an in.\\ 
 & swie vil man her oder hin\\ 
 & spîse \textbf{truoc}, sîn munt \textbf{ir doch} niht az.\\ 
 & ezzen er doch gelîche saz.\\ 
 & Anfortas sprach ze Parzival:\\ 
10 & "hêrre, iwer bruoder hât den Grâl,\\ 
 & des ich wæne, noch niht gesehen."\\ 
 & Feirefiz \textbf{begu\textit{n}dem wirte} jehen,\\ 
 & daz er \textbf{des Grâles} niht sæhe.\\ 
 & daz dûhte \textbf{al} die rîter spæhe.\\ 
15 & \textbf{Diz mære} ouch Titurel vernam,\\ 
 & der alte betterise lam.\\ 
 & \textbf{der} sprach: "ist ez ein heidenisch man,\\ 
 & sô darf er des niht willen hân,\\ 
 & daz sîn ougen \textbf{ân}s \textbf{toufes} kraft\\ 
20 & bejagen die geselleschaft,\\ 
 & daz si den Grâl beschouwen.\\ 
 & dâ ist hâmît vür gehouwen."\\ 
 & daz enbôt er \textbf{ûf} den palas.\\ 
 & Dô sprach \textbf{der wirt} \textbf{unt} Anfortas,\\ 
25 & \textbf{daz} Feirefiz næme war,\\ 
 & wes al daz \textbf{volc} lebte gar.\\ 
 & dâ wære ein ieslîch heiden\\ 
 & mit sehen von gescheiden.\\ 
 & \textbf{si wurben}, daz er næme \textbf{\textit{d}en} touf\\ 
30 & unt \textbf{endelôsen} gewinnes kouf.\\ 
\end{tabular}
\scriptsize
\line(1,0){75} \newline
D \newline
\line(1,0){75} \newline
\textbf{1} \textit{Initiale} D  \textbf{15} \textit{Majuskel} D  \textbf{24} \textit{Majuskel} D  \newline
\line(1,0){75} \newline
\textbf{5} wider dicke] [diche wider]: wider diche D \textbf{9} Parzival] Parcifal D \textbf{12} begundem] begvdem D \textbf{29} den touf] entoͮf D \newline
\end{minipage}
\hspace{0.5cm}
\begin{minipage}[t]{0.5\linewidth}
\small
\begin{center}*m
\end{center}
\begin{tabular}{rl}
 & \textbf{ir} beider vater \textbf{hiez} Fri\textit{mu}tel.\\ 
 & glîch an\textit{t}litz und glîch vel\\ 
 & Anfortas bî sîner swester truoc.\\ 
 & der heiden sach an si genuoc\\ 
5 & und \textbf{aber} \textbf{vil dicke} an in.\\ 
 & wie vil ma\textit{n h}er oder hin\\ 
 & spîse \textbf{truoc}, \textit{s}î\textit{n} munt \textbf{ir doch} niht az.\\ 
 & ezzen er doch glîche saz.\\ 
 & \begin{large}A\end{large}nfortas sprach zuo Parcifal:\\ 
10 & "hêrre, iuwer bruoder het den Grâl,\\ 
 & des ich wæne, noch niht ge\textit{s}ehen."\\ 
 & Ferefiz \textbf{dem wirte \textit{be}gunde} jehen,\\ 
 & daz er \textbf{des Grâles} niht sæhe.\\ 
 & daz dûhte \textbf{alle} die ritter spæhe.\\ 
15 & \textbf{diz mære} ouch Titurel vernam,\\ 
 & der alte betterise \textit{l}am.\\ 
 & \textbf{er} sprach: "ist ez ein heidenscher man,\\ 
 & sô darf er des \textit{niht} willen hân,\\ 
 & daz sîn ougen \textbf{ân}s \textbf{toufes} kraft\\ 
20 & bejage\textit{n} die geselleschaft,\\ 
 & daz si den Grâl beschouwen.\\ 
 & dâ ist hâmît vür gehouwen."\\ 
 & daz enbôt er \textbf{in} den palas.\\ 
 & dô sprach \textbf{der wirt} Anfortas,\\ 
25 & \textbf{daz} Ferefiz næme war,\\ 
 & wes allez daz \textbf{dô} lebte gar.\\ 
 & d\textit{â} wær ein ieglîch heiden\\ 
 & mit sehen von gescheiden.\\ 
 & \textbf{dô wurben si}, daz er næme touf\\ 
30 & und \textbf{endelôses} gewinnes kouf.\\ 
\end{tabular}
\scriptsize
\line(1,0){75} \newline
m n V V' W \newline
\line(1,0){75} \newline
\textbf{9} \textit{Initiale} m V W   $\cdot$ \textit{Capitulumzeichen} n  \newline
\line(1,0){75} \newline
\textbf{1} \textit{Die Verse 808.12-816.5 fehlen} V'   $\cdot$ hiez] \textit{om.} n  $\cdot$ Frimutel] frinittel m frimitel n [frimuntell]: frimuntel V \textbf{2} antlitz] antzlitz m \textbf{5} vil] wider V \textit{om.} W \textbf{6} wie] Swie V  $\cdot$ man her] man hin her m man aber her W \textbf{7} sîn] ir m  $\cdot$ ir doch] \textit{om.} W \textbf{8} ezzen] Essens n Essende V W  $\cdot$ saz] vergas n \textbf{9} Parcifal] parzefal V partzifal W \textbf{11} des] Das W  $\cdot$ gesehen] geschehen m n \textbf{12} Ferefiz] Ferefis m Ferrefis n [*]: Ferevis V Ferafis W  $\cdot$ begunde] ligunde m \textbf{13} sæhe] ensehe W \textbf{14} die] \textit{om.} W \textbf{15} diz] Dise n  $\cdot$ Titurel] tituͯrel m titturel n tyturel V W \textbf{16} lam] ilam m n \textbf{17} er] Der W  $\cdot$ ez] \textit{om.} n [ein]: ez V er W \textbf{18} niht] \textit{om.} m \textbf{19} âns] an W \textbf{20} bejagen] Beiage m (n) W Beiagent V \textbf{24} Anfortas] [*]: vnd anfortas V \textbf{25} Ferefiz] ferefis m ferrefis n ferevis V ferafis W \textbf{26} allez daz] aldaz voͮlg V  $\cdot$ dô lebte] gelebte W \textbf{27} dâ] Do m n V W \textbf{30} endelôses] endelosen n [endelosem]: endelosen V  $\cdot$ kouf] louff n \newline
\end{minipage}
\end{table}
\newpage
\begin{table}[ht]
\begin{minipage}[t]{0.5\linewidth}
\small
\begin{center}*G
\end{center}
\begin{tabular}{rl}
 & \textbf{\begin{large}I\end{large}r} beider vater \textbf{was} Frimutel.\\ 
 & glîch antlütze unde gelîchez vel\\ 
 & Anfortas bî sîner swester truoc.\\ 
 & der heiden sach an si genuoc\\ 
5 & unde \textbf{aber} \textbf{dicke wider} an in.\\ 
 & swie vil man her ode hin\\ 
 & spîse \textbf{truoc}, sîn munt niht az.\\ 
 & ezzen er doch gelîch saz.\\ 
 & Anfortas sprach ze Parcival:\\ 
10 & "hêrre, iwer bruoder hât den Grâl,\\ 
 & des ich wæne, noch niht gesehen."\\ 
 & Feirafiz \textbf{begunde dem wirte} jehen,\\ 
 & daz er \textbf{den Grâl} niht sæhe.\\ 
 & daz dûhte \textbf{al} die rîter spæhe.\\ 
15 & \textbf{die rede} ouch Titurel vernam,\\ 
 & der alt betterise lam.\\ 
 & \textbf{der} sprach: "ist ez ein heidensch man,\\ 
 & sô\textbf{n} darf er des niht willen hân,\\ 
 & daz sîn\textit{iu} ouge\textit{n} \textbf{ân} des \textbf{toufes} kraft\\ 
20 & bejagen die geselleschaft,\\ 
 & daz si den Grâl beschouwen.\\ 
 & dâ ist hâmît vür gehouwen."\\ 
 & daz enbôt er \textbf{in} den palas.\\ 
 & \textit{dô sprach} \textbf{Parcival} \textbf{unde} Anfortas\\ 
25 & \textbf{ze} Feirafi\textit{z}, \textbf{\textit{d}az er} næme war,\\ 
 & wes al daz \textbf{volc} lebet gar.\\ 
 & dâ wære ein ieslîch heiden\\ 
 & mit sehen von gescheiden.\\ 
 & \textbf{si wurben}, daz er næme \textbf{den} touf\\ 
30 & unde \textbf{\textit{e}ndelôsen} gewinnes kouf.\\ 
\end{tabular}
\scriptsize
\line(1,0){75} \newline
G I L Z \newline
\line(1,0){75} \newline
\textbf{1} \textit{Initiale} G L Z  \textbf{17} \textit{Initiale} I  \newline
\line(1,0){75} \newline
\textbf{1} Frimutel] vrimuntel I frýmvtel L \textbf{2} unde] \textit{om.} I  $\cdot$ gelîchez] geliche L \textbf{3} Anfortas] Amfortas L \textbf{5} aber] sie aber Z  $\cdot$ dicke wider] wider diche L (Z) \textbf{6} swie] Wie L  $\cdot$ ode] vnde I \textbf{9} Anfortas] Amfortas L  $\cdot$ Parcival] parzival G Parzifal I (L) parcifal Z \textbf{11} gesehen] gesehet Z \textbf{12} Feirafiz] feiraviz G Ferefis L Feirefiz Z  $\cdot$ begunde dem wirte] dem wirte begunde I begvnde wirte L \textbf{15} Titurel] Týtuͯrel L tyturel Z \textbf{17} ist ez] ez ist I Z \textbf{19} sîniu ougen] sin oͮge G (L) \textbf{22} hâmît] ein hamit I  $\cdot$ gehouwen] gehwen G \textbf{24} dô sprach] \textit{om.} G  $\cdot$ Parcival] parzival G parzifal I parcifal L Z  $\cdot$ Anfortas] Amfortas L \textbf{25} Feirafiz] feirafiz do sprach G ferefiz L feirefiz Z \textbf{26} lebet] lepte I L da lebte Z \textbf{30} endelôsen] vnendelosen G elosen L \newline
\end{minipage}
\hspace{0.5cm}
\begin{minipage}[t]{0.5\linewidth}
\small
\begin{center}*T
\end{center}
\begin{tabular}{rl}
 & beider vater \textbf{hiez} Frimutel.\\ 
 & gelîch antlitze und glîchez vel\\ 
 & Anfortas \textit{bî} sîner swester truoc.\\ 
 & der heiden sach an si genuoc\\ 
5 & und \textbf{dicke wider} an in.\\ 
 & \textit{wie vil man her oder hin}\\ 
 & spîse \textbf{getruo\textit{c}}, sîn munt niht az.\\ 
 & ezzen er doch glîche saz.\\ 
 & Anfortas sprach zuo Parcifal:\\ 
10 & "hêrre, iuwer bruoder hât den Grâl,\\ 
 & des ich wæne, noch niht gesehen."\\ 
 & Ferefis \textbf{begunde dem wirte} jehen,\\ 
 & daz er \textbf{den Grâl} niht sæhe.\\ 
 & daz dûhte die rîter spæhe.\\ 
15 & \textbf{disiu mære} ouch Tyturel vernam,\\ 
 & der alte betterise lam.\\ 
 & \textbf{der} sprach: "ist ez ein heidensch man,\\ 
 & sô \textbf{en}darf \textit{er} des niht willen hân,\\ 
 & daz sîn ougen \textbf{an} des \textbf{Grâles} kraft\\ 
20 & bejagen die geselleschaft,\\ 
 & daz si den Grâl beschouwen.\\ 
 & d\textit{â} ist hâmît v\textit{ü}r \textit{ge}houwen."\\ 
 & daz enbôt er \textbf{in} den palas.\\ 
 & dô sprach \textbf{Parcifal} \textbf{und} Anfortas\\ 
25 & \textbf{zuo} Ferefis, \textbf{daz er} næme war,\\ 
 & wes al daz \textbf{volc} lebete gar.\\ 
 & d\textit{â} wære ein ieclîch heiden\\ 
 & mit sehen von gescheiden.\\ 
 & \textbf{si wurben}, daz er næme \textbf{den} touf\\ 
30 & und \textbf{endelôsen} gewinnes kouf.\\ 
\end{tabular}
\scriptsize
\line(1,0){75} \newline
U Q R \newline
\line(1,0){75} \newline
\textbf{1} \textit{Initiale} Q  \textbf{9} \textit{Initiale} R  \newline
\line(1,0){75} \newline
\textbf{1} beider] Jr beder Q (R)  $\cdot$ Frimutel] frimuͦtel U frimútel Q frimitel R \textbf{3} Anfortas] Anfortes R  $\cdot$ bî] \textit{om.} U \textbf{4} si] mir Q \textbf{5} dicke wider] aber wider dicke Q (R) \textbf{6} \textit{Vers 813.6 fehlt} U   $\cdot$ man her oder] hin oder her Q \textbf{7} getruoc] getruͦ U truck Q (R)  $\cdot$ niht] nit spise R \textbf{8} ezzen] Czu essen Q \textbf{9} Anfortas] Anfortes R  $\cdot$ Parcifal] Parzifal U partzifal Q parczifal R \textbf{11} gesehen] sehen Q \textbf{12} Ferefis] feirefisz Q Feires R  $\cdot$ begunde] gund R \textbf{14} die] all die Q (R) \textbf{15} disiu] Disz Q (R)  $\cdot$ Tyturel] dieterel U titurel Q \textbf{16} lam] sam Q \textbf{17} ist ez] es ist R \textbf{18} endarf] bedarff R  $\cdot$ er] \textit{om.} U  $\cdot$ willen] wille Q \textbf{19} Grâles] tauffes Q (R) \textbf{22} dâ] Do U  $\cdot$ vür gehouwen] verheuͦwen U \textbf{23} den] [den]: dem Q \textbf{24} Parcifal] Parzifal U partzifal Q parczifal R \textbf{25} Ferefis] feirefisz Q feirefis R \textbf{27} dâ] Do U Q  $\cdot$ ieclîch] Jeglichen R \textbf{28} von gescheiden] vngescheiden Q \textbf{29} den] die Q \newline
\end{minipage}
\end{table}
\end{document}
