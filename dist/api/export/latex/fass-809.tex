\documentclass[8pt,a4paper,notitlepage]{article}
\usepackage{fullpage}
\usepackage{ulem}
\usepackage{xltxtra}
\usepackage{datetime}
\renewcommand{\dateseparator}{.}
\dmyyyydate
\usepackage{fancyhdr}
\usepackage{ifthen}
\pagestyle{fancy}
\fancyhf{}
\renewcommand{\headrulewidth}{0pt}
\fancyfoot[L]{\ifthenelse{\value{page}=1}{\today, \currenttime{} Uhr}{}}
\begin{document}
\begin{table}[ht]
\begin{minipage}[t]{0.5\linewidth}
\small
\begin{center}*D
\end{center}
\begin{tabular}{rl}
\textbf{809} & \begin{large}D\end{large}er êrsten blic den heiden \textbf{clâr}\\ 
 & dûhte \textbf{unt reideloht ir} hâr,\\ 
 & die anderen schœner aber dâ nâch,\\ 
 & die er dô schierest komen sach\\ 
5 & unt ir aller kleider tiure.\\ 
 & süeze, minneclîch gehiure\\ 
 & was al der meide antlütze gar.\\ 
 & nâch in allen kom diu lieht gevar\\ 
 & Repanse de schoye, ein magt.\\ 
10 & sich \textbf{liez} der Grâl, \textbf{ist} \textbf{mir} gesagt,\\ 
 & die selben tragen eine\\ 
 & unt \textbf{anders} enkeine.\\ 
 & ir herzen was vil kiusche bî,\\ 
 & ir vel \textbf{des} blickes flôrî.\\ 
15 & Sage ich des dienstes urhap,\\ 
 & wie vil kamerære dâ wazzer gap\\ 
 & \textbf{unt} waz man \textbf{taveln} vür si truoc,\\ 
 & mêr denn ich\textbf{s} iu ê gewuoc,\\ 
 & wie unvuoge den palas vlôch,\\ 
20 & waz man dâ karrâschen zôch\\ 
 & mit tiuren goltvazzen\\ 
 & unt wie die rîter sâzen,\\ 
 & daz würde ein \textbf{altez}, langez spel.\\ 
 & ich wil der kürze wesen snel.\\ 
25 & Mit zuht man vor dem Grâle nam\\ 
 & spîse wilt unde zam,\\ 
 & \textbf{disem} \textbf{den} met \textbf{unt} \textbf{dem} \textbf{den} wîn,\\ 
 & als ez ir site wolde sîn:\\ 
 & \begin{large}M\end{large}ôraz, sinôpel, klâret.\\ 
30 & fillu roy Gahmuret\\ 
\end{tabular}
\scriptsize
\line(1,0){75} \newline
D \newline
\line(1,0){75} \newline
\textbf{1} \textit{Initiale} D  \textbf{15} \textit{Majuskel} D  \textbf{25} \textit{Majuskel} D  \textbf{29} \textit{Initiale} D  \newline
\line(1,0){75} \newline
\textbf{9} Repanse de schoye] Repanse de scoye D \textbf{30} Gahmuret] Gahmvret D \newline
\end{minipage}
\hspace{0.5cm}
\begin{minipage}[t]{0.5\linewidth}
\small
\begin{center}*m
\end{center}
\begin{tabular}{rl}
 & de\textit{r} êrsten blic den heiden \textbf{clâr}\\ 
 & dûht \textbf{und reideleht ir} hâr,\\ 
 & die andern schœne\textit{r} aber dâr nâch,\\ 
 & die er dô schierest komen sach\\ 
5 & und ir aller kleider tiur.\\ 
 & süeze, minneclîch gehiur\\ 
 & was al der megde antlitz gar.\\ 
 & nâch in allen kam d\textit{iu} lieht gevar\\ 
 & Rep\textit{a}nse de schoye, ei\textit{n m}aget.\\ 
10 & sich \textit{\textbf{lieze}} der Grâl, \textbf{ist} \textbf{mir} gesaget,\\ 
 & die selben tragen eine\\ 
 & und \textbf{ander ouch} dekeine.\\ 
 & ir herze\textit{n} was vil kiusche bî,\\ 
 & ir vel \textbf{des} blickes flôrî.\\ 
15 & sage ich des dienstes urhap,\\ 
 & wie vil kamerer d\textit{â} wazzer gap\\ 
 & \textbf{und} waz man \textbf{taveln} vür si truoc,\\ 
 & mê dan ich\textbf{s} iu ê ge\textit{w}uoc,\\ 
 & wie unvuoge den palas vlôch,\\ 
20 & waz man d\textit{â} karrâ\textit{t}sche\textit{n z}ôch\\ 
 & mit tiuren goltvazzen\\ 
 & und wie die ritter sâzen,\\ 
 & daz würde ein \textbf{altez}, langez spel.\\ 
 & ich wil d\textit{e}r kürze wesen snel.\\ 
25 & mit zuht man vor dem Grâl nam\\ 
 & spîse wilde und zam,\\ 
 & \textbf{disem} \textbf{den} mete, \textbf{dem} \textbf{den} wîn,\\ 
 & als ez ir site wolte sîn:\\ 
 & môraz, sinôpel, clâret.\\ 
30 & filirois Gahmuret\\ 
\end{tabular}
\scriptsize
\line(1,0){75} \newline
m n V V' W \newline
\line(1,0){75} \newline
\textbf{1} \textit{Initiale} V  \textbf{9} \textit{Initiale} W  \newline
\line(1,0){75} \newline
\textbf{1} \textit{Die Verse 808.12-816.5 fehlen} V'   $\cdot$ der] Den m n W  $\cdot$ den] der W \textbf{2} reideleht] roideliht m \textbf{3} schœner] schonen m \textbf{4} schierest] [*]: schierste V \textbf{7} al der] aller n [*]: alle der V alle der W \textbf{8} in] den W  $\cdot$ diu] der m [*]: die V  $\cdot$ lieht] licht V \textbf{9} Repanse de schoye] Repense descoie m Repense de scoye n [*]: Repanse deschoye V VRepans de tschoye W  $\cdot$ ein maget] ein in maget m \textbf{10} lieze] \textit{om.} m lies V (W) \textbf{11} selben] selbe V  $\cdot$ eine] alleine W \textbf{12} ander ouch] anders ouch n [*]: anders oͮch V auch anders W  $\cdot$ dekeine] do keine n \textbf{13} herzen] hertz m (n) \textbf{16} dâ] do m n V W \textbf{18} ichs] ich n V  $\cdot$ ê] fúr n  $\cdot$ gewuoc] genuͯg m \textbf{19} unvuoge] vngefuͦge V (W) \textbf{20} dâ] do m n V W  $\cdot$ karrâtschen zôch] karrachschen truͯg zoh m \textbf{23} altez langez] alzelangez V (W) \textbf{24} der] dir m \textbf{25} vor] von W \textbf{28} ez] \textit{om.} W \textbf{29} môraz] Moral n  $\cdot$ sinôpel] zinolpel n siropel vnd W \textbf{30} Gahmuret] gamuret m V W gamiret n \newline
\end{minipage}
\end{table}
\newpage
\begin{table}[ht]
\begin{minipage}[t]{0.5\linewidth}
\small
\begin{center}*G
\end{center}
\begin{tabular}{rl}
 & der êrsten blic den heiden \textbf{gar}\\ 
 & dûhte \textbf{lieht} \textbf{unde reideloht ir} hâr,\\ 
 & \begin{large}D\end{large}ie anderen schœner aber dâr nâch,\\ 
 & die er dô schierst komen sach\\ 
5 & unde ir aller kleider tiwer.\\ 
 & süeze, minneclîch gehiwer\\ 
 & was al der meide antlütze gar.\\ 
 & nâch in allen kom diu lieht gevar\\ 
 & Urrepanse de schoye, ein maget.\\ 
10 & si\textit{ch} \textbf{liez} der Grâl, \textbf{wart} \textbf{mir} gesaget,\\ 
 & die se\textit{l}ben tragen eine\\ 
 & unde \textbf{anders} deheine,\\ 
 & \textbf{wan} ir herzen was vil kiusche bî,\\ 
 & ir vel \textbf{was} blickes flôrî.\\ 
15 & sage ich des dienstes urhap,\\ 
 & wie vil kamerære dâ wazzer gap,\\ 
 & waz man \textbf{dâ} \textbf{tweheln} vür si truoc,\\ 
 & mê\textit{r} \textit{d}enne ich iu ê gewuoc,\\ 
 & wie ungevüege den palas vlôch,\\ 
20 & waz man dâ karratschen zôch\\ 
 & mit tiuren goltvazzen\\ 
 & unde wie die rîter sâzen,\\ 
 & daz würde ein \textbf{al ze} langez spel.\\ 
 & ich wil der kürze wesen snel.\\ 
25 & mit zuht man vorem Grâle nam\\ 
 & spîse wilde unde zam,\\ 
 & \textbf{disem} met, \textbf{jenem} wîn,\\ 
 & als ez ir sit wolde sîn:\\ 
 & môraz, siropel, clâret.\\ 
30 & \textit{f}il li r\textit{o}ys Gahmuret\\ 
\end{tabular}
\scriptsize
\line(1,0){75} \newline
G I L Z \newline
\line(1,0){75} \newline
\textbf{1} \textit{Initiale} L Z  \textbf{3} \textit{Initiale} G  \textbf{13} \textit{Initiale} I  \textbf{23} \textit{Initiale} I  \newline
\line(1,0){75} \newline
\textbf{1} êrsten blic] etlichev I  $\cdot$ gar] klar Z \textbf{2} lieht] lýcht L \textit{om.} Z  $\cdot$ reideloht] reid L \textbf{3} schœner aber] aber shoͤner I schone aber L \textbf{4} dô] da L Z \textbf{7} \textit{Die Verse 809.7-8 fehlen} L  \textbf{9} Urrepanse de schoye] vrrepanse descoye G (I) Vrrepansa de schoie L Vrrepanse de tschoie Z  $\cdot$ ein] hiez ein I \textbf{10} sich] si G \textbf{11} selben] seben G \textbf{13} herzen] herze I \textbf{14} was] des L Z \textbf{15} des] \textit{om.} L \textbf{16} dâ] daz I \textbf{18} mêr denne] Mer mer denne G  $\cdot$ iu ê] ev I (L)  $\cdot$ gewuoc] Genuͤc I \textbf{19} ungevüege] vnfvge L \textbf{21} tiuren] tvrem L \textbf{23} daz] ÷az I  $\cdot$ al] \textit{om.} L  $\cdot$ spel] spil L \textbf{25} zuht] zuhten I \textbf{26} unde] oder I \textbf{28} ez ir] sin L \textbf{29} siropel] sirop I \textbf{30} fil li roys] silirays G si lauraus I  $\cdot$ Gahmuret] Ghamvret L gamuret Z \newline
\end{minipage}
\hspace{0.5cm}
\begin{minipage}[t]{0.5\linewidth}
\small
\begin{center}*T
\end{center}
\begin{tabular}{rl}
 & \begin{large}D\end{large}e\textit{r} êrsten blic de\textit{n} heiden \textbf{clâr}\\ 
 & dûhte \textbf{wîz} \textbf{ir reideleht} hâr,\\ 
 & die andern schœne\textit{r} aber dâr nâch,\\ 
 & d\textit{ie} er dô schierest komen sach\\ 
5 & und ir aller kleider tiure.\\ 
 & süeze, minneclîche gehiure\\ 
 & was al der megede antlitze gar.\\ 
 & nâch in allen kam diu lieht gevar\\ 
 & Repanse de joie, ein maget.\\ 
10 & sich \textbf{liez} der Grâl, \textbf{wart} gesaget,\\ 
 & die selbe tragen eine\\ 
 & und \textbf{anders} dekeine,\\ 
 & \textbf{wan} ir herzen was vil kiusche bî,\\ 
 & ir vel \textbf{des} blickes flôrî.\\ 
15 & sag ich des dienstes urhap,\\ 
 & wie vil kamer\textit{ær}e dâ wazzer gap,\\ 
 & waz man \textbf{dâ} \textbf{tweheln} vür si truoc,\\ 
 & mê dan ich iu ê gewuoc,\\ 
 & wie unvuoge den palas vlôch,\\ 
20 & waz \textit{man} dâ karrâschen zôch\\ 
 & mit tiuren goltvazzen\\ 
 & und wie die riter sâzen,\\ 
 & daz würde ein \textbf{alzuo} langez spel.\\ 
 & ich wil der kürze wesen snel.\\ 
25 & mit zuh\textit{t} man vor dem Grâle nam\\ 
 & spîse wilde und zam,\\ 
 & \textbf{disen} \textbf{den} mete, \textbf{jenen} \textbf{den} wîn,\\ 
 & als ez ir site wolte sîn:\\ 
 & môraz, sinôpel, clâret.\\ 
30 & fil li roys Gahmuret\\ 
\end{tabular}
\scriptsize
\line(1,0){75} \newline
U Q R \newline
\line(1,0){75} \newline
\textbf{1} \textit{Initiale} U R  \newline
\line(1,0){75} \newline
\textbf{1} Der] Den U  $\cdot$ êrsten] erst R  $\cdot$ den] der U \textbf{2} Dauchte vnd redenliche ir har Q  $\cdot$ Ducht [Im*]: Im vnd redlichett ir har R \textbf{3} andern] ander R  $\cdot$ schœner aber] schone aber U aber schoner R  $\cdot$ dâr nâch] der [nacht]: nach U \textbf{4} die] Do U  $\cdot$ schierest] aller schierest R \textbf{7} al] Alle R  $\cdot$ antlitze] schar vnd anttlút R \textbf{8} lieht] licht Q \textbf{9} Repanse de joie] Repanse de ioie U (R) Repanse detschoye Q \textbf{10} wart] wart mir Q (R) \textbf{11} selbe] selben Q R  $\cdot$ tragen] magen R \textbf{12} anders] ander Q \textbf{14} flôrî] frẏ R \textbf{16} kamerære] kamere U  $\cdot$ dâ] \textit{om.} R \textbf{17} dâ] do Q \textit{om.} R  $\cdot$ tweheln] tuͯchlin R \textbf{18} g me denne ich uch denne gnuͯg R \textbf{19} den] dem Q \textbf{20} man] \textit{om.} U  $\cdot$ dâ] do Q R  $\cdot$ karrâschen] karthasen Q \textbf{21} tiuren] trúwen R \textbf{23} Des wurde alle ze lang ein spil R  $\cdot$ spel] spil Q \textbf{24} snel] zil R \textbf{25} zuht] zuch U \textbf{27} disen] Dise Q [*]: Die R  $\cdot$ jenen] jene Q R \textbf{29} sinôpel] syropel Q \textbf{30} Gahmuret] Gahmuͦret U gamuͯret Q \newline
\end{minipage}
\end{table}
\end{document}
