\documentclass[8pt,a4paper,notitlepage]{article}
\usepackage{fullpage}
\usepackage{ulem}
\usepackage{xltxtra}
\usepackage{datetime}
\renewcommand{\dateseparator}{.}
\dmyyyydate
\usepackage{fancyhdr}
\usepackage{ifthen}
\pagestyle{fancy}
\fancyhf{}
\renewcommand{\headrulewidth}{0pt}
\fancyfoot[L]{\ifthenelse{\value{page}=1}{\today, \currenttime{} Uhr}{}}
\begin{document}
\begin{table}[ht]
\begin{minipage}[t]{0.5\linewidth}
\small
\begin{center}*D
\end{center}
\begin{tabular}{rl}
\textbf{650} & \textbf{\begin{large}M\end{large}în} \textbf{herze} \textbf{enbôt} sîn dienst \textbf{dâ} her\\ 
 & der küneginne. ouch ist sîn ger,\\ 
 & daz al der tavelrunde schar\\ 
 & sînes dienstes \textbf{nemen} war,\\ 
5 & daz si \textbf{an} triwe \textbf{denken}\\ 
 & unt im vreude niht verkrenken,\\ 
 & sô daz si \textbf{iu} \textbf{k\textit{o}men} râten."\\ 
 & al die werden des \textbf{dâ} bâten.\\ 
 & \textbf{Artus} sprach: "\textbf{trût} geselle mîn,\\ 
10 & trac disen brief der künegîn,\\ 
 & \textbf{daz si dran} lesen und sagen,\\ 
 & wes wir uns vreuwen unt waz wir klagen,\\ 
 & Daz der künec Gramoflanz\\ 
 & \textbf{hôchvart mit} \textbf{lôsheite} ganz\\ 
15 & gein mîme künne bieten kan.\\ 
 & er wænet, mîn neve Gawan\\ 
 & sî Cidegast, den er sluoc,\\ 
 & \textbf{dâ} von er kumbers hât genuoc.\\ 
 & ich sol im kumber mêren\\ 
20 & unt \textbf{niwen} site lêren."\\ 
 & Der knappe kom gegangen,\\ 
 & dâ er wart wol enpfangen.\\ 
 & \textbf{er gap der küneginne} den brief,\\ 
 & des manec ouge über lief,\\ 
25 & dô ir süezer munt gelas,\\ 
 & \textbf{al} daz dran geschriben was,\\ 
 & Gawans \textbf{klage} und werben.\\ 
 & dô\textbf{ne} liez ouch niht verderben\\ 
 & der knappe z\textbf{al den vrouwen} warp,\\ 
30 & dâr an sîn kunst niht verdarp.\\ 
\end{tabular}
\scriptsize
\line(1,0){75} \newline
D \newline
\line(1,0){75} \newline
\textbf{1} \textit{Initiale} D  \textbf{13} \textit{Majuskel} D  \textbf{21} \textit{Majuskel} D  \newline
\line(1,0){75} \newline
\textbf{4} dienstes] diens D \textbf{7} komen] chômn D \newline
\end{minipage}
\hspace{0.5cm}
\begin{minipage}[t]{0.5\linewidth}
\small
\begin{center}*m
\end{center}
\begin{tabular}{rl}
 & \textbf{mîn} \textbf{hêrre} \textbf{enbôt} sîn dienst \textbf{dâ} her\\ 
 & der künigîn. ouch ist sîn ger,\\ 
 & daz alder tavelrunder schar\\ 
 & sînes dienstes \textbf{neme} war,\\ 
5 & daz si \textbf{an} triuwen \textbf{denken}\\ 
 & und im vröude niht ver\textit{kr}enke\textit{n},\\ 
 & sô daz si \textbf{iu} \textbf{kunnen} râten."\\ 
 & aldie werden des \textbf{d\textit{â}} bâten.\\ 
 & \textbf{Artus} sprach: "\textbf{trût} geselle mîn,\\ 
10 & trac disen brief der künigîn,\\ 
 & \textbf{daz si dâ} lese\textit{n} und sagen,\\ 
 & wes wir uns vröuwen und waz wir klagen,\\ 
 & daz der künic Gramolanz\\ 
 & \textbf{hôchvart mit} \textbf{lôsheit} ganz\\ 
15 & gegen mîne\textit{m} künn\textit{e} bieten kan.\\ 
 & er wænet, mîn neve Gawan\\ 
 & sî Zidegast, den er sluoc,\\ 
 & \textbf{dâ} von er kumbers he\textit{t} \textit{g}enuoc.\\ 
 & ich sol im kumber mêren\\ 
20 & und \textbf{niuwe} site lêren."\\ 
 & der knappe kam gegangen,\\ 
 & d\textit{â} er wart wol enpfangen.\\ 
 & \textbf{er gap der künigîn} den brief,\\ 
 & des manic ouge \textit{ü}ber lief,\\ 
25 & dô ir süezer munt gelas\\ 
 & \textbf{alsô}, daz dâr an geschriben was,\\ 
 & Gawans \textbf{klage} und \textbf{sîn} werben.\\ 
 & dô liez ouch \textit{niht} verderben\\ 
 & der knappe zuo \textbf{alle\textit{n} den vrowe\textit{n}} \textit{w}arp,\\ 
30 & dâr an sîn kunst niht verdarp.\\ 
\end{tabular}
\scriptsize
\line(1,0){75} \newline
m n o Fr69 \newline
\line(1,0){75} \newline
\newline
\line(1,0){75} \newline
\textbf{3} tavelrunder] tafelrunde n \textbf{6} verkrenken] versencket m verkrencket o \textbf{8} dâ] do m n \textbf{9} Artus] Artuͯs o \textbf{11} dâ] do n o  $\cdot$ lesen] lese m n o \textbf{13} Gramolanz] gramolantz m gramonlantz n gramolancz o \textbf{15} mînem künne] minen kunnen m \textbf{17} Zidegast] zide gast n \textbf{18} er] ir o  $\cdot$ het genuoc] het genomen genuͯg m hette genuͦg o \textbf{22} dâ] Do m n o \textbf{24} des] Das o  $\cdot$ über] lieber m \textbf{27} klage] [clag*]: clage m \textbf{28} liez] lies er o  $\cdot$ niht] \textit{om.} m \textbf{29} Der knappe zuͯ alle den frowen sprach warp m \newline
\end{minipage}
\end{table}
\newpage
\begin{table}[ht]
\begin{minipage}[t]{0.5\linewidth}
\small
\begin{center}*G
\end{center}
\begin{tabular}{rl}
 & \textbf{\begin{large}S\end{large}în} \textbf{herze} \textbf{enbiut} sîn dienst her\\ 
 & der künegîn. ouch ist sîn ger,\\ 
 & daz al der tavelrunder schar\\ 
 & sînes dienstes \textbf{nemen} war,\\ 
5 & daz si \textbf{ir} triuwe \textbf{an im} \textbf{gedenken}\\ 
 & unde im vröude niht verkrenken,\\ 
 & sô daz si \textbf{komen} râten."\\ 
 & al die werden des bâten.\\ 
 & \textbf{der künic} sprach: "geselle mîn,\\ 
10 & trac disen brief der künegîn.\\ 
 & \textbf{bit si den} lesen unde sagen,\\ 
 & wes wir uns vrouwen unde waz wir klagen,\\ 
 & daz der künic Gramoflanz\\ 
 & \textbf{mit hôchvart} \textbf{lôsheit} ganz\\ 
15 & gein mînem künne bieten kan.\\ 
 & er wænet, mîn neve Gawan\\ 
 & s\textit{î} Zidegast, den er sluoc,\\ 
 & von \textbf{dem} er kumbers hât genuoc.\\ 
 & ich sol im kumber mêren\\ 
20 & unde \textbf{niuwen} site lêren."\\ 
 & der knappe kom gegangen,\\ 
 & \textit{d}â er wart wol enpfangen.\\ 
 & \textbf{der künegîn er gap} den brief,\\ 
 & des manic ouge über lief,\\ 
25 & dô ir süezer munt gelas,\\ 
 & \textbf{al} daz dâr an geschriben was,\\ 
 & Gawans \textbf{klagen} unde \textbf{sîn} werben.\\ 
 & d\textit{ô} \textbf{en}liez ouch niht verderben\\ 
 & der knappe z\textbf{en vrouwen allen} warp,\\ 
30 & dâr an sîn kunst niht verdarp.\\ 
\end{tabular}
\scriptsize
\line(1,0){75} \newline
G I L M Z \newline
\line(1,0){75} \newline
\textbf{1} \textit{Initiale} G L Z  \textbf{7} \textit{Initiale} I  \textbf{17} \textit{Initiale} I  \newline
\line(1,0){75} \newline
\textbf{1} Sîn herze] min herre I \textbf{4} nemen] neme Z \textbf{5} ir triuwe] truͯwen L ir trewen Z \textbf{6} verkrenken] erkrenken M \textbf{7} si komen] ir komen kvnnen L sie uch komen M (Z) \textbf{8} des] in des I [*]: des da Z \textbf{13} der] \textit{om.} Z  $\cdot$ Gramoflanz] gramorflancz M gramoflantz Z \textbf{15} künne] oͤheim Z \textbf{16} wænet] wande M \textit{om.} Z \textbf{17} sî] sit G (M)  $\cdot$ Zidegast] zitegast I Cytegast L zcitegast M Cidegast Z  $\cdot$ er] her da M \textbf{20} unde niuwen] niwe I Vnde nuwe M (Z) \textbf{22} dâ] al da G  $\cdot$ er wart] >er< wart G wart er L \textbf{23} der] Dẏ M  $\cdot$ er gap] gab er L (M)  $\cdot$ den] der Z \textbf{24} des] Das M \textbf{25} dô] Da M Z \textbf{27} Gawans] Gawanz L \textbf{28} dô enliez] doh enliez G Da enliesz M (Z) \newline
\end{minipage}
\hspace{0.5cm}
\begin{minipage}[t]{0.5\linewidth}
\small
\begin{center}*T
\end{center}
\begin{tabular}{rl}
 & \textbf{mîn} \textbf{hêrre} \textbf{enbiut} sîn dienst h\textit{er}\\ 
 & der künigîn. ouch ist sîn ger,\\ 
 & daz alder tavelrunder schar\\ 
 & sînes dienstes \textbf{neme} war,\\ 
5 & daz \textit{si} \textbf{ir} triuwen \textbf{an} \textbf{gedenken}\\ 
 & und im vreude niht verkrenken,\\ 
 & sô daz si \textbf{komen} râten."\\ 
 & alle die werden \textbf{in} des bâten.\\ 
 & \textbf{der künic} sprach: "geselle mîn,\\ 
10 & trage disen brief der künigîn,\\ 
 & \textbf{daz si dran} lese\textit{n} und sage\textit{n},\\ 
 & wes wir uns vreuwen und waz wir klagen,\\ 
 & daz der künic Gramoflanz\\ 
 & \textbf{mit hôchvart} \textbf{lôsheit} ganz\\ 
15 & gên mînem künne bieten kan.\\ 
 & er wænt, mîn neve Gawan\\ 
 & sî Cydegast, den er sluoc,\\ 
 & \textbf{dâ} von er kumbers hât genuoc.\\ 
 & ich sol im kumber mêren\\ 
20 & und \textbf{niuwen} site lêren."\\ 
 & der knabe kam gegangen,\\ 
 & d\textit{â} er wart wol enpfangen.\\ 
 & \textbf{der künigîn gap er} den brief,\\ 
 & des manec ouge über lief,\\ 
25 & dô ir süezer munt gelas,\\ 
 & \textbf{al} daz dâr an geschriben was,\\ 
 & Gawans \textbf{klage} und \textbf{sîn} werben.\\ 
 & dô \textbf{en}liez ouch niht verderben\\ 
 & der knabe zuo\textbf{n vrouwen allen} warp,\\ 
30 & dâr an sîn kunst niht verdarp.\\ 
\end{tabular}
\scriptsize
\line(1,0){75} \newline
Q R W V \newline
\line(1,0){75} \newline
\textbf{1} \textit{Initiale} Q W V  \textbf{21} \textit{Initiale} R  \newline
\line(1,0){75} \newline
\textbf{1} mîn] [Ein]: Min V  $\cdot$ her] h:: Q \textbf{2} ouch] auch daz W \textbf{3} tavelrunder] tauelrunde R \textbf{4} dienstes] dienst R  $\cdot$ neme] nemen W V \textbf{5} [*]: Daz sv́ an ir trvwe gedenken V  $\cdot$ si] \textit{om.} Q W  $\cdot$ triuwen an] trúwe an Im R (W) \textbf{7} si komen] sy úch komen R úch kúnnen W [*]: sv́ v́ch kvnnen V \textbf{9} künic] kúuig W \textbf{11} daz] Bit R  $\cdot$ lesen und sagen] lese vnd sage Q \textbf{12} Was do sei vnser froͤde vnd klagen W  $\cdot$ [*gen]: Wez wir vnz frowen vnd waz wir clagen V \textbf{13} Gramoflanz] gramoflantz Q W Gramoflancz R gramaflanz V \textbf{14} lôsheit] laßhait W [h*eit]: losheit V \textbf{15} künne] frúnde R \textbf{17} Cydegast] cidegast Q (R) zytegast W gydegast V  $\cdot$ sluoc] erschluͦg W \textbf{18} dâ von] Von dem R Do [*]: von V  $\cdot$ hât] leid R \textbf{20} niuwen] núwe R (W) (V)  $\cdot$ site] sitten W \textbf{22} dâ] Do Q W V  $\cdot$ wol] schon W \textbf{24} des] Da R \textbf{27} Gawans] Gawins R [G*]: Gawanes V \textbf{28} enliez] lies R  $\cdot$ niht] [*]: niht V \textbf{29} zuon] zu R \textbf{30} niht] gar nit R \newline
\end{minipage}
\end{table}
\end{document}
