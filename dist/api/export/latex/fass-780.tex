\documentclass[8pt,a4paper,notitlepage]{article}
\usepackage{fullpage}
\usepackage{ulem}
\usepackage{xltxtra}
\usepackage{datetime}
\renewcommand{\dateseparator}{.}
\dmyyyydate
\usepackage{fancyhdr}
\usepackage{ifthen}
\pagestyle{fancy}
\fancyhf{}
\renewcommand{\headrulewidth}{0pt}
\fancyfoot[L]{\ifthenelse{\value{page}=1}{\today, \currenttime{} Uhr}{}}
\begin{document}
\begin{table}[ht]
\begin{minipage}[t]{0.5\linewidth}
\small
\begin{center}*D
\end{center}
\begin{tabular}{rl}
\textbf{780} & \begin{large}M\end{large}it triwen âne vâre.\\ 
 & diu werde, niht diu clâre\\ 
 & snellîche wider ûf \textbf{dô} spranc.\\ 
 & si \textbf{neig} \textbf{in} unt sagete in danc,\\ 
5 & die ir nâch grôzer schulde\\ 
 & \textbf{geholfen heten} hulde.\\ 
 & Si want mit ir hende\\ 
 & \textbf{wider} \textbf{ab} ir \textbf{houbetgebende}.\\ 
 & ez wære bezel oder snürrinc,\\ 
10 & daz warf si von ir \textbf{an} den rinc.\\ 
 & Cundrie la surziere\\ 
 & \textbf{wart dô} bekennet schiere\\ 
 & unt des Grâles wâpen, daz si truoc.\\ 
 & \textbf{daz wart beschouwet dô} genuoc.\\ 
15 & Si vuorte ouch noch den selben lîp,\\ 
 & den sô manec man unt wîp\\ 
 & sach zuo dem Plimizœle komen.\\ 
 & ir \textbf{antlütze ir habt} vernomen.\\ 
 & ir ougen stuonden dennoch \textbf{sus}\\ 
20 & gel als ein topâzîus,\\ 
 & ir zene lanc. ir munt gap schîn\\ 
 & als ein vîol weitîn.\\ 
 & wan daz si truoc gein prîse muot,\\ 
 & si \textbf{vuorte} ân nôt den tiweren huot\\ 
25 & ûf dem Plimizœles plân.\\ 
 & diu sunne het ir niht getân.\\ 
 & \textbf{diu}\textbf{ne} mohte ir vel durchz hâr\\ 
 & niht verselwen mit \textbf{ir} blickes vâr.\\ 
 & \textit{\begin{large}S\end{large}}i stuont mit \textbf{zuht} und sprach,\\ 
30 & des man vür hôhiu mære jach.\\ 
\end{tabular}
\scriptsize
\line(1,0){75} \newline
D \newline
\line(1,0){75} \newline
\textbf{1} \textit{Initiale} D  \textbf{7} \textit{Majuskel} D  \textbf{15} \textit{Majuskel} D  \textbf{29} \textit{Initiale} D  \newline
\line(1,0){75} \newline
\textbf{11} Cvndrîe lasvrziere D \textbf{17} Plimizœle] Plimizoͤle D \textbf{25} Plimizœles] Plimizoͤls D \textbf{29} Si] ÷i D \newline
\end{minipage}
\hspace{0.5cm}
\begin{minipage}[t]{0.5\linewidth}
\small
\begin{center}*m
\end{center}
\begin{tabular}{rl}
 & mit triuwen âne vâre.\\ 
 & diu werde, niht diu clâre\\ 
 & snelleclîch wider ûf spranc.\\ 
 & si \textbf{neigte} und seite i\textit{n} danc,\\ 
5 & die ir nâch grôzer schulde\\ 
 & \textbf{geholfe\textit{n h}eten} hulde.\\ 
 & si want mit ir hende\\ 
 & \textbf{ab} ir \textbf{houbt daz gebende}.\\ 
 & ez wær bezel oder snürrinc,\\ 
10 & daz warf si von i\textit{r} \textbf{\textit{a}n} den rinc.\\ 
 & \hspace*{-.7em}\big| \textbf{dô wart} bekennt schier\\ 
 & \hspace*{-.7em}\big| Condrie la surzier\\ 
 & und des Grâles wâpen, daz si truoc.\\ 
 & \textbf{dô wart beschouwet} genuoc.\\ 
15 & si vuorte ouch noch den selben lîp,\\ 
 & den sô manic man und wîp\\ 
 & sach zem Plimizol komen.\\ 
 & ir \textbf{antlitz ir habt} vernomen.\\ 
 & ir ougen stuonden dannoch \textbf{sus}\\ 
20 & gel als ein topâ\textit{z}îus,\\ 
 & ir zene lanc. ir munt gap schîn\\ 
 & als ein vîol we\textit{i}tîn.\\ 
 & wan daz si truoc gegen prîse muot,\\ 
 & si \textbf{vuorte} âne nôt den tiuren huot\\ 
25 & ûf dem Plimizoles plân.\\ 
 & d\textit{iu} sunne het ir niht getân.\\ 
 & \textbf{si} mohte ir vel durch daz hâr\\ 
 & niht verselwen mit \textbf{ir} blickes vâr.\\ 
 & si stuont mit \textbf{zühten} und sprach,\\ 
30 & des man vür hôhiu mær jach.\\ 
\end{tabular}
\scriptsize
\line(1,0){75} \newline
m n o V V' W \newline
\line(1,0){75} \newline
\textbf{1} \textit{Initiale} W  \newline
\line(1,0){75} \newline
\textbf{4} neigte und seite] seite vnd neite o neig vnde sagete V neig vnd saget V' neigte vnd seit W  $\cdot$ in] ẏm m (W) \textbf{5} ir] in n \textbf{6} geholfen heten] Gehulffen hulffen hetten m  $\cdot$ hulde] zu hulden V' zuͦ hulde W \textbf{8} ab] Vber V'  $\cdot$ daz] \textit{om.} W \textbf{9} bezel] besser n vessel W \textbf{10} ir an] ir wider an m \textbf{12} \textit{Die Verse 780.12, 11 und 13 fehlen} o   $\cdot$ bekennt] erkennet V' \textbf{11} Condrie lasurzier m n  $\cdot$ Kvndrie Lesurziere V  $\cdot$ Kvndrie lasurziere V'  $\cdot$ Kundrie lasursier W \textbf{14} dô] Daz V V'  $\cdot$ genuoc] do gnuͦg V (V') \textbf{15} noch] [*och]: noch V \textit{om.} V'  $\cdot$ lîp] [gesch]: lip o \textbf{17} Plimizol] plimenzol V plimenzen V' plymizol W \textbf{19} \textit{Die Verse 780.19-28 fehlen} V'  \textbf{20} topâzîus] thoparius m thobarius n thaparius o ) Topasius V thopasius W \textbf{22} vîol] wol o vyol W  $\cdot$ weitîn] wettin m (n) (o) \textbf{23} wan] Was o  $\cdot$ muot] munt o \textbf{24} vuorte] truͦg W \textbf{25} Plimizoles] plimzols m n o plimenzol V plymizols W \textbf{26} diu] Dise m \textbf{27} mohte] moͯchte n (W) enmoͤhte V  $\cdot$ hâr] far o \textbf{28} verselwen] verfelwen W \textbf{29} si] Do o \newline
\end{minipage}
\end{table}
\newpage
\begin{table}[ht]
\begin{minipage}[t]{0.5\linewidth}
\small
\begin{center}*G
\end{center}
\begin{tabular}{rl}
 & \begin{large}M\end{large}it triuwen âne vâre.\\ 
 & diu werde, niht diu clâre\\ 
 & snellîche wider ûf spranc.\\ 
 & si \textbf{neic} \textbf{in} unde sagte in danc,\\ 
5 & die ir nâch grôzer schulde\\ 
 & \textbf{geholfen heten} hulde.\\ 
 & si want mit ir hende\\ 
 & \textbf{abe} ir \textbf{houbt daz gebende}.\\ 
 & ez wære bezel oder snürrinc,\\ 
10 & daz warf si von ir \textbf{in} den rinc.\\ 
 & Gundrie lasurziere\\ 
 & \textbf{wart d\textit{â}} bekennet schiere\\ 
 & \textit{unde} des Grâles wâpen, daz si truoc.\\ 
 & \textbf{daz wart beschouwet dô} genuoc.\\ 
15 & si vuorte ouch noch den selben lîp,\\ 
 & den sô manic man unde wîp\\ 
 & sach zem Blimzol komen.\\ 
 & ir \textbf{habet ir antlütze wol} vernomen.\\ 
 & iriu ougen stuonden dannoch \textbf{dâ}\\ 
20 & gel als ein topâziâ,\\ 
 & ir zen lanc. ir munt gap schîn\\ 
 & als ein vîol weitîn.\\ 
 & wan daz si truoc gein prîse muot,\\ 
 & si \textbf{vuorte} ân nôt den tiuren huot\\ 
25 & ûf dem Blimzoles plân.\\ 
 & diu sunne het ir niht getân.\\ 
 & \textbf{diu}\textbf{ne} moht ir vel durch daz hâr\\ 
 & niht verselwen mit \textbf{ir} blickes vâr.\\ 
 & si stuont mit \textbf{zühten} unde sprach,\\ 
30 & des man vür hôhiu mære jach.\\ 
\end{tabular}
\scriptsize
\line(1,0){75} \newline
G I L M Z \newline
\line(1,0){75} \newline
\textbf{1} \textit{Initiale} G L Z  \textbf{5} \textit{Initiale} I  \textbf{21} \textit{Initiale} I  \newline
\line(1,0){75} \newline
\textbf{4} neic in] neich L neigen M  $\cdot$ sagte] seit I (L) (Z) \textbf{5} die] Daz L \textbf{6} geholfen heten] Heten gehuͯlfen L (M) (Z) \textbf{8} abe] Wider ab L (M) Z  $\cdot$ ir houbt] dem haupte I  $\cdot$ daz] \textit{om.} M \textbf{9} bezel] beckel Z \textbf{10} von ir] nider Z  $\cdot$ in den] anden M (Z) \textbf{11} kvndrie lasv̂rziere G  $\cdot$ Kvndrie la svrziere L  $\cdot$ Kundrie lasurziere M  $\cdot$ Kvndrie lasvrziere Z \textbf{12} dâ] do G  $\cdot$ bekennet] bekant L \textbf{13} unde] von G  $\cdot$ daz si] den sy M \textbf{14} daz wart] Das wart da M Wart da Z  $\cdot$ dô genuoc] da Genuͤc I (L) gnuͯc M (Z) \textbf{15} vuorte] vurt I (L) (Z) vorchte M  $\cdot$ noch] \textit{om.} M \textbf{16} sô] \textit{om.} L \textbf{17} Blimzol] plimizol I M Z plinizol L \textbf{18} Jr antlutze ir habt vernomen L (M) (Z) \textbf{19} dâ] svs Z \textbf{20} topâziâ] Topazivs Z \textbf{22} als] al I  $\cdot$ vîol] viole M vial Z \textbf{23} daz] \textit{om.} I Z \textbf{24} vuorte] vuͤrt I (L)  $\cdot$ den] der Z \textbf{25} dem] den I (L) (M)  $\cdot$ Blimzoles] plimozols I plimizolz L plimizols M Z \textbf{26} het] en hette M \textbf{27} diune] Dy M  $\cdot$ vel] vol L \textbf{28} ir] \textit{om.} Z \newline
\end{minipage}
\hspace{0.5cm}
\begin{minipage}[t]{0.5\linewidth}
\small
\begin{center}*T
\end{center}
\begin{tabular}{rl}
 & mit triuwen âne vâre.\\ 
 & diu werde \textbf{und} niht diu clâre\\ 
 & snellîche wider ûf spranc.\\ 
 & si \textbf{neic} \textbf{im} und saget in danc,\\ 
5 & die ir nâch grôzer schulde\\ 
 & \textbf{heten geholfen} hulde.\\ 
 & si want mit ir hende\\ 
 & \textbf{wider} \textbf{von} ir \textbf{houbet daz gebende}.\\ 
 & ez wære bezel oder snürrinc,\\ 
10 & daz warf si von ir \textbf{an} den rinc.\\ 
 & Kundrie lasurziere\\ 
 & \textbf{wart dâ} bekennet schiere\\ 
 & und des Grâles wâpen, daz si truoc.\\ 
 & \textbf{daz wart dô beschouwet} genuoc.\\ 
15 & si vuorte ouch noch den selben lîp,\\ 
 & den sô manec man und wîp\\ 
 & sach zuo dem Plymizol komen.\\ 
 & ir \textbf{antlitze ir hât} vernomen.\\ 
 & ir ougen stuonden dannoch \textbf{sus}\\ 
20 & gel als ein topâzîus,\\ 
 & ir zene lanc. ir munt gap schîn\\ 
 & als ein vîol weitîn.\\ 
 & wan daz si truoc gein prîse muot,\\ 
 & si \textbf{truoc} âne nôt den tiuren huot\\ 
25 & ûf dem Plymizoles plân.\\ 
 & diu sunne het ir niht getân.\\ 
 & \textbf{d\textit{i}u} \textbf{en}moht ir v\textit{el} durch daz hâr\\ 
 & niht verselwen mit blickes vâr.\\ 
 & \begin{large}S\end{large}i stuont mit \textbf{zühten} und sprach,\\ 
30 & des man vür hôhiu mære jach.\\ 
\end{tabular}
\scriptsize
\line(1,0){75} \newline
U Q R \newline
\line(1,0){75} \newline
\textbf{1} \textit{Initiale} Q  \textbf{29} \textit{Initiale} U  \newline
\line(1,0){75} \newline
\textbf{1} triuwen] trewe Q (R) \textbf{2} und] \textit{om.} Q R  $\cdot$ clâre] vare clare R \textbf{4} si neic] Neigt Q Sẏ neigt R \textbf{8} wider] \textit{om.} Q  $\cdot$ von] ab R \textbf{9} bezel] hetel Q \textbf{11} Kundrie Lasuͦrsiere U  $\cdot$ kundrie lazurziere Q  $\cdot$ Kundire lasurziere R \textbf{12} dâ] do Q \textbf{14} Des beschawet do genuck Q  $\cdot$ dô beschouwet] beschowet da R \textbf{16} sô] sy R \textbf{17} Plymizol] Plimizol U plúmizol Q Bimizol R \textbf{18} hât] vor habt Q hapt vor R \textbf{19} dannoch] \textit{om.} R \textbf{20} topâzîus] toporius R \textbf{22} vîol] viel R \textbf{24} truoc] fuͦrte R  $\cdot$ tiuren] \textit{om.} R \textbf{25} dem] des Q  $\cdot$ Plymizoles] plymizols U plimizols Q R  $\cdot$ plân] pan Q \textbf{27} diu enmoht ir vel] Du in mochtir vri U Dú moͯchte [dur]: ir vel R \textbf{28} mit] durch ir Q mit ir R \textbf{30} hôhiu] hoche R \newline
\end{minipage}
\end{table}
\end{document}
