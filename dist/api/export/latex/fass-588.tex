\documentclass[8pt,a4paper,notitlepage]{article}
\usepackage{fullpage}
\usepackage{ulem}
\usepackage{xltxtra}
\usepackage{datetime}
\renewcommand{\dateseparator}{.}
\dmyyyydate
\usepackage{fancyhdr}
\usepackage{ifthen}
\pagestyle{fancy}
\fancyhf{}
\renewcommand{\headrulewidth}{0pt}
\fancyfoot[L]{\ifthenelse{\value{page}=1}{\today, \currenttime{} Uhr}{}}
\begin{document}
\begin{table}[ht]
\begin{minipage}[t]{0.5\linewidth}
\small
\begin{center}*D
\end{center}
\begin{tabular}{rl}
\textbf{588} & \begin{large}O\end{large}b kumber sich gelîche dem,\\ 
 & swelch minnære den an sich \textbf{genem},\\ 
 & der werde alrêst wol gesunt,\\ 
 & mit pfîlen alsus sêre wunt.\\ 
5 & daz tuot im lîhte als wê\\ 
 & als sîn minnen kumber ê.\\ 
 & Gawan truoc minne unt \textbf{ander} klage.\\ 
 & \textbf{dô} begundez \textbf{liuhten} vome tage,\\ 
 & daz sîner grôzen kerzen schîn\\ 
10 & unnâch sô virrec mohte sîn.\\ 
 & Ûf rihte sich der wîgant.\\ 
 & \textbf{dô} was sîn lînîn gewant\\ 
 & nâch wunden und harnasch var.\\ 
 & \textbf{zuo z}im \textbf{was} geleit dar\\ 
15 & hemde unde bruoch von buckeram\\ 
 & - den wehsel er dô gerne nam -\\ 
 & unt eine garnaschen merderîn,\\ 
 & des selben ein kürsenlîn,\\ 
 & ob den \textbf{bêden} schürbrant\\ 
20 & von \textbf{Arraze} \textbf{al} dar gesant.\\ 
 & zwêne stivâle ouch dâ lâgen,\\ 
 & die niht \textbf{grôzer enge} pflâgen.\\ 
 & di\textit{u} niwen kleider leit er an.\\ 
 & dô gienc mîn hêr Gawan\\ 
25 & ûz \textbf{z}er kemenâten tür.\\ 
 & sus gie er wider unde vür,\\ 
 & unz er den rîchen palas vant.\\ 
 & sînen ougen \textbf{wart nie} bekant\\ 
 & rîcheit, diu dar zuo t\textit{ö}hte,\\ 
30 & \textbf{daz si dâ} gelîchen m\textit{ö}hte.\\ 
\end{tabular}
\scriptsize
\line(1,0){75} \newline
D Z \newline
\line(1,0){75} \newline
\textbf{1} \textit{Initiale} D  \textbf{7} \textit{Initiale} Z  \textbf{11} \textit{Majuskel} D  \newline
\line(1,0){75} \newline
\textbf{5} als] alsus Z \textbf{8} dô] Da Z  $\cdot$ liuhten] liehten Z \textbf{12} dô] Da Z \textbf{14} zim] im Z \textbf{16} dô] da Z \textbf{17} garnaschen] garnetsche Z \textbf{19} bêden] selben Z \textbf{20} Arraze] arraz Z \textbf{23} diu] di D \textbf{24} dô] Da Z \textbf{25} zer] der Z \textbf{29} töhte] tohte D \textbf{30} si dâ] sich dem Z  $\cdot$ möhte] mohte D Z \newline
\end{minipage}
\hspace{0.5cm}
\begin{minipage}[t]{0.5\linewidth}
\small
\begin{center}*m
\end{center}
\begin{tabular}{rl}
 & ob kumber sich glîche de\textit{m}e,\\ 
 & welich minner den an sich \textbf{neme},\\ 
 & der werde allerêrst wol \dag gesant\dag ,\\ 
 & mit pfîlen alsus sêre wunt.\\ 
5 & daz tuot im lîhte alsô wê\\ 
 & als sîn minnen kumber ê.\\ 
 & \begin{large}G\end{large}awan truoc minne und klage.\\ 
 & \textbf{dô} begunde ez \textbf{liuhten} vom tage,\\ 
 & daz sîner grôzen kerzen schîn\\ 
10 & \dag und doch\dag  sô v\textit{ir}ric \dag mohten\dag  sîn.\\ 
 & ûf \textit{r}ihte sich der wîgant.\\ 
 & \textbf{dô} was sîn lînîn gewant\\ 
 & nâch wunden un\textit{d h}arnasch var.\\ 
 & \textbf{sus} im \textbf{wart} geleget dar\\ 
15 & he\textit{m}de und bruoch von buckeram\\ 
 & - den wehsel er dô gerne nam -\\ 
 & und ein garnasche merderîn,\\ 
 & des selben ein kürsenlîn,\\ 
 & ob den \textbf{beiden} schürb\textit{r}ant\\ 
20 & von \textbf{Arsas} \textbf{al}dâ gesant.\\ 
 & zwên stivel ouch d\textit{â} lâgen,\\ 
 & die niht \textbf{grôzer enge} pflâgen.\\ 
 & diu niuwen kleider leit er an.\\ 
 & dô gienc mîn hêr Gawan\\ 
25 & ûz \textbf{z}er kemenâten tür.\\ 
 & sus gienc er wider und vür,\\ 
 & unz er den rîchen palas vant.\\ 
 & sînen ougen \textbf{nie wart} bekant\\ 
 & rîcheit, diu dar zuo t\textit{ö}hte,\\ 
30 & \textbf{daz si d\textit{â}} glîchen möhte.\\ 
\end{tabular}
\scriptsize
\line(1,0){75} \newline
m n o \newline
\line(1,0){75} \newline
\textbf{7} \textit{Initiale} m   $\cdot$ \textit{Capitulumzeichen} n  \newline
\line(1,0){75} \newline
\textbf{1} glîche] glichen o  $\cdot$ deme] denne m \textbf{2} den] denne n (o) \textbf{7} klage] ander clage n o \textbf{8} liuhten] lichten n \textbf{9} grôzen] grosser o \textbf{10} doch] ouch n  $\cdot$ virric] forig m (o) ferrig n  $\cdot$ mohten] moͯchten n \textbf{11} rihte] sichte m \textbf{12} dô] Das o \textbf{13} und harnasch] vnd hohes harnasch m \textbf{14} wart] was n (o) \textbf{15} hemde] Hende m  $\cdot$ bruoch] bruͯche n  $\cdot$ buckeram] borkeran o \textbf{18} kürsenlîn] kurnelin o \textbf{19} schürbrant] schuͯrbant m \textbf{20} Arsas] arros n arras o \textbf{21} stivel] stifelen n  $\cdot$ dâ] do m n o \textbf{24} hêr] herre her n \textbf{29} \textit{Die Verse 588.29-589.16 fehlen} n   $\cdot$ töhte] dohtte m (o) \textbf{30} dâ] do m o  $\cdot$ möhte] mochte o \newline
\end{minipage}
\end{table}
\newpage
\begin{table}[ht]
\begin{minipage}[t]{0.5\linewidth}
\small
\begin{center}*G
\end{center}
\begin{tabular}{rl}
 & ob kumber sich gelîche deme,\\ 
 & swelch minnære den an sich \textbf{neme},\\ 
 & der werde alrêrst wol gesunt,\\ 
 & mit pfîlen alsô sêre wunt.\\ 
5 & daz tuot im lîht als wê\\ 
 & als sîn minne kumber ê.\\ 
 & \begin{large}G\end{large}awan truoc minne unde \textbf{ander} klage.\\ 
 & \textbf{nû} begunde ez \textbf{liehten} von dem tage,\\ 
 & daz sîner grôzen kerzen schîn\\ 
10 & unnâch sô \textit{v}irric mohte sîn.\\ 
 & ûf rihte sich der wîgant.\\ 
 & \textbf{nû} was sîn lînîn gewant\\ 
 & nâch wunden unt \textbf{daz} harnasch var.\\ 
 & \textbf{zuo} ime \textbf{was} geleit dar\\ 
15 & hemede unde bruoch von buggeram\\ 
 & - den wehsel er dô gerne nam -\\ 
 & unde ein garnasche merderîn,\\ 
 & des selben ein kürsenlîn,\\ 
 & ob den \textbf{zwein} schürbrant\\ 
20 & von \textbf{Arzeiz} dar gesant.\\ 
 & zwên stivâle ouch dâ lâgen,\\ 
 & die niht \textbf{grœze} pflâgen.\\ 
 & diu niuwen kleider leit er an.\\ 
 & dô gienc mîn hêrre Gawan\\ 
25 & ûz der kemenâten tür.\\ 
 & sus gienc er wider unde vür,\\ 
 & unz er den rîchen palas vant.\\ 
 & sînen ougen, \textbf{den} \textbf{wart nie} bekant\\ 
 & rîcheit, diu dar zuo t\textit{ö}hte,\\ 
30 & \textbf{daz si dem} gelîchen m\textit{ö}hte.\\ 
\end{tabular}
\scriptsize
\line(1,0){75} \newline
G I L M Fr23 \newline
\line(1,0){75} \newline
\textbf{1} \textit{Initiale} M  \textbf{7} \textit{Initiale} G L Fr23  \textbf{11} \textit{Initiale} I  \newline
\line(1,0){75} \newline
\textbf{2} swelch] Welch L Wilche M  $\cdot$ minnære] libe M  $\cdot$ den an sich] an sich den I \textbf{3} werde alrêrst] erst werde Fr23 \textbf{5} als wê] alwe L alsus we M \textbf{6} sîn] im I  $\cdot$ minne] libe M \textbf{7} minne] libe M \textbf{9} kerzen] herzen L her::: Fr23 \textbf{10} virric] wirrich G wirbic I verre Fr23  $\cdot$ mohte] mohten I \textbf{12} sîn] sy M \textbf{13} daz] \textit{om.} I nach L (M) na::: Fr23 \textbf{14} zuo] [*]: Zu L \textbf{15} buggeram] bvckam L \textbf{16} dô] da M \textbf{17} ein garnasche] den harnasch Fr23 \textbf{18} kürsenlîn] churselin G (I) (L) kusselin M \textbf{19} ob] An L Vff M  $\cdot$ schürbrant] siche::: Fr23 \textbf{20} Arzeiz] arzeis I Aleriz L alazais M alahers Fr23  $\cdot$ dar] daz L \textbf{21} stivâle] stifaln M \textbf{23} diu] Den Fr23 \textbf{24} hêrre Gawan] ergawan M Ga::: Fr23 \textbf{25} ûz] Vsze M \textbf{27} unz] Bisz M \textbf{28} den] \textit{om.} I L  $\cdot$ nie] mer M \textbf{29} töhte] tohte G I (L) tochten M \textbf{30} möhte] mohte G I (L) mochten M \newline
\end{minipage}
\hspace{0.5cm}
\begin{minipage}[t]{0.5\linewidth}
\small
\begin{center}*T
\end{center}
\begin{tabular}{rl}
 & ob kumber sich glîche dem,\\ 
 & welch minnære den an sich \textbf{nem},\\ 
 & der werde alrêst wol gesunt,\\ 
 & mit pfîlen alsô sêre wunt.\\ 
5 & daz tuot im lîhte alsô wê\\ 
 & als sîn minnen kumber ê.\\ 
 & Gawan truoc minne und \textbf{ander} klage.\\ 
 & \textbf{nû} begunde ez \textbf{liuhten} von dem tage,\\ 
 & daz sîner grôzen kerzen schîn\\ 
10 & unnâch sô virrec mohte sîn.\\ 
 & ûf rihte sich der wîgant.\\ 
 & \textbf{nû} was sîn lînîn gewant\\ 
 & nâch wunden und \textbf{nâch} harnasch var.\\ 
 & \textbf{zuo} im \textbf{was} geleit dar\\ 
15 & hemde und bruoch von \textit{b}uckeram,\\ 
 & - den wehsel er dô gerne nam -\\ 
 & und ein garnasch merderîn,\\ 
 & des selben ein kürsenlîn,\\ 
 & ob den \textbf{zwein} schürbrant\\ 
20 & von \textbf{Aroys} dar gesant.\\ 
 & zwên stivel ouch d\textit{â} lâgen,\\ 
 & die niht \textbf{grœze} pflâgen.\\ 
 & diu niuwen kleider legt er an.\\ 
 & dô gienc mîn hêrre Gawan\\ 
25 & ûz der kemenâten tür.\\ 
 & sus gienc er wider und vür,\\ 
 & unz er den rîchen palas vant.\\ 
 & sînen ougen, \textbf{den} \textbf{wart nie} bekant\\ 
 & rîcheit, diu dar zuo t\textit{ö}hte,\\ 
30 & \textbf{diu sich der} glîchen m\textit{ö}hte.\\ 
\end{tabular}
\scriptsize
\line(1,0){75} \newline
Q R W V U \newline
\line(1,0){75} \newline
\textbf{7} \textit{Initiale} W V   $\cdot$ \textit{Capitulumzeichen} R  \newline
\line(1,0){75} \newline
\textbf{1} \textit{Die Verse 553.1-599.30 fehlen} U   $\cdot$ kumber sich] soͯmlich kumer R [kvm*]: kvmber sich V \textbf{2} welch] Swelich V  $\cdot$ den] \textit{om.} R [*]: den V  $\cdot$ nem] [*]: nemme V \textbf{3} der] Dem W  $\cdot$ werde] wurde R \textbf{4} alsô] [als*]: alsvs V \textbf{5} \textit{Vers 588.5 ist am Rand nachgetragen und später radiert:} Daz : als: we: V   $\cdot$ daz] [G*]: Daz V \textbf{6} sîn] im W  $\cdot$ minnen] minne R W [*]: minne V \textbf{7} Gawan] Gawin R [*]: Gawan V \textbf{8} nû] [*]: Nv V \textbf{9} grôzen kerzen] [*]: grossen kerzen V \textbf{10} [*]: Vnnach so virrig moͤhte sin V  $\cdot$ unnâch] Vnnacht W  $\cdot$ mohte] moͯchte R (W) \textbf{11} rihte] [*]: rihte V \textbf{12} lînîn] lini R \textbf{13} harnasch var] harnesvar V \textbf{14} im] zim V \textbf{15} Hemd bruch vnd buͯkram R  $\cdot$ buckeram] bruckram Q \textbf{16} dô] [*]: do V \textbf{17} ein garnasch] harnasch W \textbf{18} des] Das W \textbf{19} zwein] [*]: beiden V \textbf{20} Aroys] areis R W arres V \textbf{21} zwên] Zwo V  $\cdot$ stivel] stinal W  $\cdot$ dâ] do Q W \textbf{22} grœze] [*]: groͤzer enge V \textbf{23} niuwen] núw R \textbf{25} ûz der] Vsser R Auß zuͦ der W Fv́r [*]: der V \textbf{26} sus] als Q \textbf{28} den] \textit{om.} V \textbf{29} töhte] tochte Q (R) (V) dachte W \textbf{30} der] gar R  $\cdot$ möhte] mochte Q R (V) (W) \newline
\end{minipage}
\end{table}
\end{document}
