\documentclass[8pt,a4paper,notitlepage]{article}
\usepackage{fullpage}
\usepackage{ulem}
\usepackage{xltxtra}
\usepackage{datetime}
\renewcommand{\dateseparator}{.}
\dmyyyydate
\usepackage{fancyhdr}
\usepackage{ifthen}
\pagestyle{fancy}
\fancyhf{}
\renewcommand{\headrulewidth}{0pt}
\fancyfoot[L]{\ifthenelse{\value{page}=1}{\today, \currenttime{} Uhr}{}}
\begin{document}
\begin{table}[ht]
\begin{minipage}[t]{0.5\linewidth}
\small
\begin{center}*D
\end{center}
\begin{tabular}{rl}
\textbf{736} & \begin{large}S\end{large}în gir stuont nâch minne\\ 
 & unt \textbf{nâch} prîses gewinne.\\ 
 & \textbf{daz} gâben ouch \textbf{allez meistec} wîp,\\ 
 & dâ mite der heiden sînen lîp\\ 
5 & kostelîche \textbf{zimierte}.\\ 
 & \textbf{diu} minne condwierte\\ 
 & in sîn manlîch herze hôhen muot,\\ 
 & als si noch dem minnegernden tuot.\\ 
 & Er truoc ouch durch prîses lôn\\ 
10 & ûf dem \textbf{helme} ein ecidemôn.\\ 
 & swelhe würme \textbf{sint} eiterhaft,\\ 
 & von des selben tierlînes kraft\\ 
 & hânt si lebens \textbf{decheine vrist},\\ 
 & swenn ez von in ersmecket ist.\\ 
15 & Thopedissimonte\\ 
 & unt Assigarzionte,\\ 
 & Thasme und Arabi\\ 
 & sint \textbf{vor} solhem pfelle vrî,\\ 
 & als sîn ors truoc covertiure.\\ 
20 & der ungetoufte gehiure\\ 
 & ranc nâch \textbf{wîbe} lône,\\ 
 & des zimiert er sich \textbf{sus} schône.\\ 
 & sîn hôhez herze in des betwanc,\\ 
 & daz er nâch werder minne ranc.\\ 
25 & Der selbe werlîche knabe\\ 
 & het in einer wilden habe\\ 
 & \textbf{zem} \textbf{fôreht} geenkert ûf dem mer.\\ 
 & er hete vünf unt zweinzec her,\\ 
 & der \textbf{neheinez} sandern rede vernam,\\ 
30 & als sîner rîcheit wol gezam.\\ 
\end{tabular}
\scriptsize
\line(1,0){75} \newline
D \newline
\line(1,0){75} \newline
\textbf{1} \textit{Initiale} D  \textbf{9} \textit{Majuskel} D  \textbf{25} \textit{Majuskel} D  \newline
\line(1,0){75} \newline
\textbf{15} Thopedissimonte] Thopedissimônte D \textbf{16} Assigarzionte] Assigarzîonte D \newline
\end{minipage}
\hspace{0.5cm}
\begin{minipage}[t]{0.5\linewidth}
\small
\begin{center}*m
\end{center}
\begin{tabular}{rl}
 & sîn gir stuont nâch minne\\ 
 & und prîses gewinne.\\ 
 & \textbf{ez} gâben ouch \textbf{meistic alliu} wîp,\\ 
 & dâ mit der heiden sînen lîp\\ 
5 & kostlîch \textbf{zimierte}.\\ 
 & \textbf{des} minne condwierte\\ 
 & in sîn manlîch herze hôhen muot,\\ 
 & alsô s\textit{i} noch dem minnen gernde\textit{n} tuot.\\ 
 & er truoc ouch durch prîses lôn\\ 
10 & ûf dem \textbf{houbt\textit{e}} \textit{ein} e\textit{c}idemôn.\\ 
 & welich würme \textbf{sîn} eiterhaft,\\ 
 & von des selben tie\textit{r}lî\textit{n}es kraft\\ 
 & hânt si lebens \textbf{dekein vrist},\\ 
 & wan ez von in ersmecket ist.\\ 
15 & Doppedisse monte\\ 
 & und Assigarzionte,\\ 
 & \textit{Th}as\textit{m}e und Arabi\\ 
 & sint \textbf{vor} solhem pfelle vrî,\\ 
 & als sîn ros truoc co\textit{v}ert\textit{iu}re.\\ 
20 & der ungetoufte gehiure\\ 
 & ranc nâch \textbf{wîbe} lôn,\\ 
 & des zimierte er sich \textbf{sô} schôn.\\ 
 & sîn hôhez herz in des betwanc,\\ 
 & daz er \textit{n}â\textit{ch} werder minne ranc.\\ 
25 & \begin{large}D\end{large}er selbe werlîch knabe\\ 
 & het in einer wilden habe\\ 
 & \textbf{zuo einem} \textbf{fôreht} geankert ûf dem mer.\\ 
 & er het vünf und zweinzic her,\\ 
 & der \textbf{keinez} \textit{des} andern rede vernam,\\ 
30 & als sîner rîcheit wol gezam.\\ 
\end{tabular}
\scriptsize
\line(1,0){75} \newline
m n o V V' \newline
\line(1,0){75} \newline
\textbf{25} \textit{Initiale} m V V'   $\cdot$ \textit{Capitulumzeichen} n  \newline
\line(1,0){75} \newline
\textbf{1} minne] [minen]: minne V \textbf{2} prîses] noch prises n o (V) (V') \textbf{3} \textit{Die Verse 736.3-8 fehlen } V'   $\cdot$ alliu] alles V \textbf{6} des] Die n V Dier o \textbf{8} si] so m  $\cdot$ dem minnen gernden] dem mẏnnen gernde m den mynnegerenden n dem mẏnne gernde o \textbf{9} durch] \textit{om.} V' \textbf{10} houbte ein] hobtter m helme ein V V'  $\cdot$ ecidemôn] ettedemon m ecidemen o \textbf{11} welich] Welsh o Swelhe V  $\cdot$ sîn] sint n o V V'  $\cdot$ eiterhaft] etehafft n eterhafft o \textbf{12} tierlînes] tielicies m tier licies n tuerlicies o tieres V V' \textbf{13} lebens] [leben*s]: lebendes V lebendes V'  $\cdot$ dekein] do keine n deheines V' \textbf{14} wan ez] Swenne es V Swenne V' \textbf{15} \textit{Die Verse 736.15-24 fehlen} V'   $\cdot$ Doppedisse monte] Toppidissemonte n Coppidisse monte o Topedissinionte V \textbf{16} Assigarzionte] assigartionte o assigarsionte V \textbf{17} Thasme] Casine m Tasine n (o)  $\cdot$ Arabi] arabeẏ m araby n arabẏ o \textbf{18} solhem] solhen V \textbf{19} ros truoc] [trug]: rosz truͯg n  $\cdot$ covertiure] confertustuͯr m \textbf{20} der] Die o \textbf{21} wîbe] wibes V \textbf{22} sô] susz n (o) (V) \textbf{24} nâch] das m \textbf{27} dem] daz V' \textbf{28} vünf] fuuf V' \textbf{29} der] Des o  $\cdot$ keinez des andern] keines andern m (n) o \newline
\end{minipage}
\end{table}
\newpage
\begin{table}[ht]
\begin{minipage}[t]{0.5\linewidth}
\small
\begin{center}*G
\end{center}
\begin{tabular}{rl}
 & \begin{large}S\end{large}în gir stuont nâch minne\\ 
 & unde \textbf{nâch} prîses gewinne.\\ 
 & \textbf{daz} gâben ouch \textbf{almeistec} wîp,\\ 
 & dâ mit der heiden sînen lîp\\ 
5 & kostlîche \textbf{zimierte}.\\ 
 & \textbf{diu} minne condewierte\\ 
 & in sîn manlîch herze hôhen muot,\\ 
 & als si noch dem minnegernden tuot.\\ 
 & er truoc ouch durch prîses lôn\\ 
10 & ûf dem \textbf{helme} ein ecidemôn.\\ 
 & swelch würme \textbf{sint} eiterhaft,\\ 
 & von des selben tierlînes kraft\\ 
 & habent si lebens \textbf{kleinen list},\\ 
 & swenne ez von in ersmecket ist.\\ 
15 & \multicolumn{1}{l}{ - - - }\\ 
 & \multicolumn{1}{l}{ - - - }\\ 
 & Tasme unde Arabi\\ 
 & sint \textbf{von} solhem pfelle vrî,\\ 
 & als sîn ors truoc covertiure.\\ 
20 & der ungetoufte gehiure\\ 
 & ranc nâch \textbf{wîbes} lône,\\ 
 & des zimierte er sich schône.\\ 
 & \multicolumn{1}{l}{ - - - }\\ 
 & \multicolumn{1}{l}{ - - - }\\ 
25 & der selbe werlîche knabe\\ 
 & het in einer wilden habe\\ 
 & \textbf{zuo dem} \textbf{fôreht} geankeret ûf dem mer.\\ 
 & er hete vünf unde zweinzic her,\\ 
 & der \textbf{deheinez} des andern rede vernam,\\ 
30 & als sîner rîcheit wol gezam.\\ 
\end{tabular}
\scriptsize
\line(1,0){75} \newline
G I L M Z Fr18 Fr24 \newline
\line(1,0){75} \newline
\textbf{1} \textit{Initiale} G I L Z Fr18  \textbf{25} \textit{Initiale} I  \newline
\line(1,0){75} \newline
\textbf{3} almeistec] almeiste L (M) Fr24 \textbf{5} kostlîche] Stolczlich M \textbf{7} \textit{nach 736.7:} Vil mange grosze smertze L   $\cdot$ hôhen muot] \textit{om.} L \textbf{8} \textit{nach 736.8:} Dem ellenthafte stet sin muͯt L   $\cdot$ als si] Also M  $\cdot$ dem] den Z  $\cdot$ minnegernden] minne Gerndem I mynnegernde L (M) minnen gernden Z \textbf{9} truoc] ::vge Fr24  $\cdot$ ouch durch] nach I \textbf{10} ein] \textit{om.} L M Z Fr24 \textbf{11} swelch] Welche L (M) Solhe Z \textbf{13} kleinen list] dehainen list I kleyne vrist M keine frist Z \textbf{14} swenne] Wenne L (M) Z :::ne Fr24 \textbf{15} \textit{Die Verse 736.15-16 fehlen} G I L M Z Fr24  \textbf{17} Tasme] Tasnie L Thasme M :::me Fr24 \textbf{20} gehiure] gebure M \textbf{21} wîbes] wibe M \textbf{22} zimierte] zimiert I (Fr24) \textbf{23} \textit{Die Verse 736.23-24 fehlen} G I L M Z Fr24  \textbf{26} het] hat M \textbf{27} fôreht] voraisk I \textit{om.} M  $\cdot$ dem] daz I \textbf{29} deheinez] icheine M \newline
\end{minipage}
\hspace{0.5cm}
\begin{minipage}[t]{0.5\linewidth}
\small
\begin{center}*T
\end{center}
\begin{tabular}{rl}
 & sîn gir stuont nâch minne\\ 
 & und \textbf{nâch} prîses gewinne.\\ 
 & \textbf{daz} gâb\textit{en} \textbf{im} ouch \textbf{almeistec} wîp,\\ 
 & dâ mit der heiden sînen lîp\\ 
5 & kostlîche \textbf{zierete}.\\ 
 & \textbf{diu} minne condewierete\\ 
 & in sîn manlîch herze hôhen muot,\\ 
 & alse si noch dem minnegernden tuot.\\ 
 & er truoc ouch durch prîses lôn\\ 
10 & ûf dem \textbf{helme} ein ecidemôn.\\ 
 & welche würme \textbf{sint} eiterhaft,\\ 
 & von des selben tierlînes kraft\\ 
 & hânt si lebens \textbf{kleinen vrist},\\ 
 & wanne ez von in ersmecket ist.\\ 
15 & \multicolumn{1}{l}{ - - - }\\ 
 & \multicolumn{1}{l}{ - - - }\\ 
 & Thasme und Araby\\ 
 & sint \textbf{von} solichem pfelle vrî,\\ 
 & als sîn ors truoc covertiure.\\ 
20 & der ungetoufte gehiure\\ 
 & ranc nâch \textbf{wîbes} lône,\\ 
 & des zimierete er sich schône.\\ 
 & \multicolumn{1}{l}{ - - - }\\ 
 & \multicolumn{1}{l}{ - - - }\\ 
25 & der selbe werlîche knabe\\ 
 & hete in einer wilden habe\\ 
 & \textbf{zuo der} \textbf{vurt} geankert ûf dem mer.\\ 
 & er hete vünf und zweinzic her,\\ 
 & der \textbf{dekeiner} des andern rede vernam,\\ 
30 & als sîner rîcheit wol gezam.\\ 
\end{tabular}
\scriptsize
\line(1,0){75} \newline
U W Q R \newline
\line(1,0){75} \newline
\textbf{1} \textit{Initiale} W   $\cdot$ \textit{Capitulumzeichen} R  \newline
\line(1,0){75} \newline
\textbf{3} gâben] gab U  $\cdot$ im] \textit{om.} W Q R \textbf{5} zierete] zymierte W (Q) (R) \textbf{7} manlîch] nanlich R \textbf{8} si] ich Q  $\cdot$ dem] den R  $\cdot$ minnegernden] minne gernde Q \textbf{9} prîses] prise R \textbf{10} dem] sinem R  $\cdot$ ein] einen Q  $\cdot$ ecidemôn] essidemon W \textbf{11} eiterhaft] eitterschafftt R \textbf{12} von] Durch W \textbf{13} Hat sein leben kleines frist Q  $\cdot$ kleinen] kleine W keine R \textbf{14} ez] er W \textbf{15} \textit{Die Verse 736.15-16 fehlen} U W Q R  \textbf{17} Thasme] Thasine U  $\cdot$ Araby] arabei W arabẏ Q aribi R \textbf{18} von] vor R  $\cdot$ solichem] sulchen Q (R) \textbf{19} covertiure] Couenture R \textbf{21} wîbes] weibe Q \textbf{22} zimierete] zimert Q zierret R  $\cdot$ sich] \textit{om.} Q \textbf{23} \textit{Die Verse 736.23-24 fehlen} U W Q R  \textbf{27} zuo der vurt] Zuͦm forecht W (R) Zu den forcht Q \textbf{29} Der keiner seiner rede vernam W · Des des andern rede vernam Q · Der deheins des andren sprache recht venam R \textbf{30} als] Als es W \newline
\end{minipage}
\end{table}
\end{document}
