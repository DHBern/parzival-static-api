\documentclass[8pt,a4paper,notitlepage]{article}
\usepackage{fullpage}
\usepackage{ulem}
\usepackage{xltxtra}
\usepackage{datetime}
\renewcommand{\dateseparator}{.}
\dmyyyydate
\usepackage{fancyhdr}
\usepackage{ifthen}
\pagestyle{fancy}
\fancyhf{}
\renewcommand{\headrulewidth}{0pt}
\fancyfoot[L]{\ifthenelse{\value{page}=1}{\today, \currenttime{} Uhr}{}}
\begin{document}
\begin{table}[ht]
\begin{minipage}[t]{0.5\linewidth}
\small
\begin{center}*D
\end{center}
\begin{tabular}{rl}
\textbf{62} & ûz verrem lande,\\ 
 & den niemen dâ \textbf{erkande}.\\ 
 & "sîn volc, \textbf{daz} ist kurtoys,\\ 
 & beidiu heidensch unt franzoys.\\ 
5 & etslîcher mag ein Anschevin\\ 
 & mit sîner sprâche \textbf{iedoch} wol sîn.\\ 
 & Ir muot ist stolz, ir wât ist clâr,\\ 
 & wol gesniten al vür wâr.\\ 
 & ich was sînen knappen bî.\\ 
10 & die sint vor \textbf{missewende} vrî.\\ 
 & \textbf{si} jehent, swer habe geruoche,\\ 
 & ob der ir hêrren suoche,\\ 
 & den scheid er von swære.\\ 
 & von im vrâget ich der mære.\\ 
15 & \textbf{nû} sageten si mir\textbf{s} sunder wanc,\\ 
 & ez wære der \textbf{künec} von Zazamanc."\\ 
 & disiu mære sagete \textbf{ir} ein garzûn:\\ 
 & "âvoy, welch ein poulûn!\\ 
 & iwer krône unt iwer lant\\ 
20 & \textbf{wæren} dâr vür niht halbez pfant."\\ 
 & "dû\textbf{ne} \textbf{darft} mir\textbf{z} \textbf{sô} \textbf{loben} niht.\\ 
 & mîn munt hin wider dir \textbf{des giht}:\\ 
 & ez mac wol sîn eines werden man,\\ 
 & der niht mit armuote kan."\\ 
25 & \textbf{alsus} sprach diu künegîn.\\ 
 & "\textbf{wê}, wanne kumt er \textbf{êt} selbe drîn?"\\ 
 & \textbf{\begin{large}D\end{large}en} garzûn si des vrâgen bat.\\ 
 & höfschlîchen durch die stat\\ 
 & der helt begunde trecken,\\ 
30 & die slâfenden wecken.\\ 
\end{tabular}
\scriptsize
\line(1,0){75} \newline
D \newline
\line(1,0){75} \newline
\textbf{7} \textit{Majuskel} D  \textbf{27} \textit{Initiale} D  \newline
\line(1,0){75} \newline
\textbf{5} Anschevin] Ascevin D \textbf{16} Zazamanc] Zazamanch D \newline
\end{minipage}
\hspace{0.5cm}
\begin{minipage}[t]{0.5\linewidth}
\small
\begin{center}*m
\end{center}
\begin{tabular}{rl}
 & ûz verrem lande,\\ 
 & den n\textit{i}emen d\textit{â} \textbf{erkande}.\\ 
 & "sîn volc, \textbf{daz} ist kurtois,\\ 
 & beidiu heidensc\textit{h} und franzois.\\ 
5 & etslîcher mac ein A\textit{n}schevin\\ 
 & mit sîner sprâche \textbf{iedoch} wol sîn.\\ 
 & ir muot ist stolz, ir wât ist klâr,\\ 
 & wol gesniten al vür wâr.\\ 
 & ich was sînen knappen bî.\\ 
10 & die sint vor \textbf{missewende} vrî.\\ 
 & \textbf{die} jehen\textit{t}, wer habe geruoche,\\ 
 & ob der ir hêrren suoche,\\ 
 & den scheid er von swære.\\ 
 & von ime \dag vrâg\dag  ich der mære.\\ 
15 & \textbf{sô} sageten si mir sunder wanc,\\ 
 & ez wære der \textbf{küene} von Zazamanc."\\ 
 & disiu mære sagete \textbf{ir} ein garzûn:\\ 
 & "â\textit{v}oy, \dag velsch\dag  ein poulûn!\\ 
 & iuwer krône und iuwer lant\\ 
20 & \textbf{wæren} dâ vür niht halbez pfant."\\ 
 & "dû \textbf{bedarf\textit{t}} mir\textbf{z} \textbf{sô} \textbf{loben} niht.\\ 
 & mîn munt hin wider dir \textbf{des giht}:\\ 
 & ez mac wol sîn eines werden man,\\ 
 & der niht mit armuote kan."\\ 
25 & \textbf{alsus} sprach diu künigîn.\\ 
 & "\textbf{wê}, wanne kume\textit{t} er selbe drîn?"\\ 
 & \textbf{den} garzûn si des vrâgen bat.\\ 
 & hovelîchen durch die stat\\ 
 & der helt begunde \dag strecken\dag ,\\ 
30 & die slâfende\textit{n} wecken.\\ 
\end{tabular}
\scriptsize
\line(1,0){75} \newline
m n o \newline
\line(1,0){75} \newline
\textbf{21} \textit{Capitulumzeichen} m  \newline
\line(1,0){75} \newline
\textbf{1} verrem] farem o \textbf{2} niemen] nemen \textit{nachträglich korrigiert zu:} nieman m  $\cdot$ dâ] do m n o \textbf{4} heidensch] [hen*]: heidensche m  $\cdot$ franzois] franczos \textit{nachträglich korrigiert zu:} franczoys m frantzois n franczois o \textbf{5} mac] [*]: Mag \textit{nachträglich korrigiert zu:} ouch m  $\cdot$ Anschevin] ausceuin \textit{nachträglich korrigiert zu:} ansceuin m auscevin n aúste vin o \textbf{7} muot] munt n  $\cdot$ wât] wart n \textbf{11} die] Sú n (o)  $\cdot$ jehent] iehen m \textbf{13} den] Dem o  $\cdot$ scheid er von] schiede ich der n schiede er von o \textbf{15} sô sageten] Sie sa:ten o \textbf{16} Zazamanc] zazamanck m zazamang n o \textbf{18} âvoy] Anoÿ m (n) (o)  $\cdot$ velsch] felch n o \textbf{21} bedarft] bedarff m darfft n o \textbf{22} mîn] Mẏ o \textbf{23} ez] Er m n o  $\cdot$ eines werden] ein swerden \textit{nachträglich korrigiert zu:} ein werden m ein werder n o \textbf{26} kumet er] kumette er m koͯmet er n er kommet o  $\cdot$ selbe] selbes n selber o \textbf{27} vrâgen] [frowen]: frewen o \textbf{28} die] due o \textbf{30} slâfenden] sloffende m o \newline
\end{minipage}
\end{table}
\newpage
\begin{table}[ht]
\begin{minipage}[t]{0.5\linewidth}
\small
\begin{center}*G
\end{center}
\begin{tabular}{rl}
 & ûz verrem lande,\\ 
 & den niemen dâ \textbf{erkande}.\\ 
 & "sîn volc, \textbf{daz} ist kurtois,\\ 
 & beidiu heidensch und franzois.\\ 
5 & etslîcher mag ein Antschevin\\ 
 & mit sîner sprâche \textbf{vil} wol sîn.\\ 
 & ir muot ist stolz, ir wât ist clâr,\\ 
 & wol gesniten al vür wâr.\\ 
 & ich was sînen knappen bî.\\ 
10 & die sint vor \textbf{missewende} vrî\\ 
 & \textbf{unde} jehent, swer habe geruoche,\\ 
 & op der ir hêrren suoche,\\ 
 & den scheid er von swære.\\ 
 & von im vrâgete ich der mære.\\ 
15 & \textbf{\begin{large}D\end{large}ô} seiten si mir sunder wanc,\\ 
 & ez wære der \textbf{künic} von Zazamanc."\\ 
 & disiu mære seit ein garzûn:\\ 
 & "âvoy, welch ein pavelûn!\\ 
 & iwer krône und iwer lant\\ 
20 & \textbf{w\textit{æ}ren} dâr vür niht halbez pfant."\\ 
 & "dû\textbf{ne} \textbf{darf\textit{t}} mir\textbf{z} \textbf{alsô} \textbf{loben} niht.\\ 
 & mîn munt hin wider dir \textbf{vergiht}:\\ 
 & ez mac wol sîn eines werden man,\\ 
 & der niht mit armuote kan."\\ 
25 & \textbf{als} sprach diu künigîn.\\ 
 & "\textbf{owê}, wenne kumet er selbe drîn?"\\ 
 & \textbf{ir} garzûn si des vrâgen bat.\\ 
 & höflîchen durch die stat\\ 
 & der helt begunde trecken,\\ 
30 & die slâfenden wecken.\\ 
\end{tabular}
\scriptsize
\line(1,0){75} \newline
G I O L M Q R Z Fr37 Fr44 \newline
\line(1,0){75} \newline
\textbf{1} \textit{Initiale} O  \textbf{15} \textit{Initiale} G  \textbf{25} \textit{Initiale} I  \textbf{29} \textit{Initiale} L M Q Z Fr37 Fr44  \newline
\line(1,0){75} \newline
\textbf{1} \textit{Die Verse 58.9-63.24 fehlen (Blattverlust)} R   $\cdot$ ûz] ÷z O  $\cdot$ verrem lande] verre dem lande I gar vrome dem lande L verren landen Q uerrene lande Fr37 gar uerrem lande Fr44 \textbf{2} dâ] \textit{om.} L Fr44 do Q \textbf{3} sîn] in I  $\cdot$ daz ist] daz wêr I ist alles L ist M \textbf{4} \textit{Versfolge 62.5-16, dann 62.4} Q   $\cdot$ heidensch] haidnis I (Q) haidens O (Fr37)  $\cdot$ franzois] fronzois I franzoys O Fr44 franzoýs L francioysch M franczoysz Q frantzois Z franzevs Fr37 \textbf{5} Antschevin] Anschevin G antscheuin I anschvin O Anshevin L (Z) aischvyn M anschoűin Q ::schauein Fr37 Anscheuin Fr44 \textbf{6} vil] \textit{om.} I L ie doch O (M) (Q) (Z) \textbf{8} gesniten] gesnite I \textbf{9} \textit{Versdoppelung 62.9-10 nach 63.20} Q  \textbf{10} sint vor] warren Q sint uon Fr44 \textbf{11} swer] wer ir L wer M Z wir Q  $\cdot$ geruoche] ruͤche I gerochen M \textbf{12} der] er O L Fr37  $\cdot$ ir] iht Z  $\cdot$ hêrren] herre geruche Q \textbf{13} scheid er] schide er von Q scheidet Z \textbf{14} von] V\%-o Q Vom Z  $\cdot$ im] in Fr44  $\cdot$ ich der] er O \textbf{15} Dô] Da M Z Dv Fr37  $\cdot$ mir sunder] om sundern M \textbf{16} von] \textit{om.} I M vn Fr37  $\cdot$ Zazamanc] zazamanch G (O) L zazamanck Q zachzamanch Fr37 \textbf{17} disiu] Div O  $\cdot$ seit] sagete L M (Fr44)  $\cdot$ ein] in Z \textbf{18} âvoy] Awi O Afoya M  $\cdot$ pavelûn] babilun I \textbf{19} krône] konick Q \textbf{20} wæren] waren G wer I (L) (M)  $\cdot$ halbez] halbe L \textbf{21} dûne] Dv O (Fr44) Da en M  $\cdot$ darft] darf G bedarft I saltis M solt Fr44  $\cdot$ mirz alsô] mir ez so I (L) (Q) (Z) (Fr37) (Fr44) mir so O so M  $\cdot$ loben] lonen O \textbf{22} munt] mut Q  $\cdot$ hin wider dir] [hinder]: hinwider dir G hin wider des I (Fr37) hin wider dich des O hin wider dir des L Z (Fr44) he wider dy des M mir des wider Q  $\cdot$ vergiht] giht O (L) (M) (Q) Z (Fr44) \textbf{24} der] Der da M  $\cdot$ mit] \textit{om.} O \textbf{25} diu künigîn] den chunegine Fr37 \textbf{26} owê] Awe I O Q We L  $\cdot$ er] der I  $\cdot$ selbe] selben M selber Q \textit{om.} Fr44 \textbf{27} si des] sis L  $\cdot$ bat] [bet]: bat I \textbf{28} höflîchen] houfchleich Fr37  $\cdot$ durch die] in der L \textbf{29} trecken] shrechen I rechen Fr37 \textbf{30} slâfenden] slofende M (Fr37) flafenden Z \newline
\end{minipage}
\hspace{0.5cm}
\begin{minipage}[t]{0.5\linewidth}
\small
\begin{center}*T (U)
\end{center}
\begin{tabular}{rl}
 & ûz verre\textit{m} lande,\\ 
 & den nieman dâ \textbf{bekande}.\\ 
 & "sîn volc ist \textbf{allez} kurtoys,\\ 
 & beidiu heidensch und franzoys.\\ 
5 & etlîcher mac ein Anschevin\\ 
 & mit sîner sprâche \textbf{iedoch} wol sîn.\\ 
 & ir muot ist stolz, ir wât ist klâr,\\ 
 & wol gesniten al vür wâr.\\ 
 & ich was sînen knappen bî.\\ 
10 & die sint vor \textbf{missetæte} vrî\\ 
 & \textbf{und} jehent, wer habe geruoche,\\ 
 & ob der ir hêrre\textit{n} suoch\textit{e},\\ 
 & den sch\textit{ei}d er von swære.\\ 
 & von im vrâget ich der mære.\\ 
15 & \textbf{dô} sageten si mir sunder wanc,\\ 
 & ez wære der \textbf{künec} von Zazamanc."\\ 
 & disiu mære saget ein garzûn:\\ 
 & "âvoy, welch ein pavelûn!\\ 
 & iuwer krône und iuwer lant\\ 
20 & \textbf{wære} dâr vür niht halbez pfant."\\ 
 & "dû \textbf{en}\textbf{darft} mir \textbf{des} \textbf{sagen} n\textit{i}ht.\\ 
 & mîn munt hin wider di\textit{r} \textbf{des giht}:\\ 
 & ez mac wol sîn eines werden man,\\ 
 & der niht mit armuote kan."\\ 
25 & \textbf{alsô} sprach diu künegîn.\\ 
 & "\textbf{ouwê}, wan kumt er selbe drîn?"\\ 
 & \textbf{ir} garzûn si des vrâgen bat.\\ 
 & höveschlîche durch die stat\\ 
 & der helt begunde trecken,\\ 
30 & die slâfenden wecken.\\ 
\end{tabular}
\scriptsize
\line(1,0){75} \newline
U V W T \newline
\line(1,0){75} \newline
\textbf{17} \textit{Majuskel} T  \textbf{21} \textit{Initiale} T  \textbf{27} \textit{Initiale} W  \textbf{28} \textit{Majuskel} T  \textbf{29} \textit{Initiale} V  \newline
\line(1,0){75} \newline
\textbf{1} ûz] Dar vs V  $\cdot$ verrem] verren U vremedem T \textbf{2} dâ] do V W  $\cdot$ bekande] erkande V \textbf{3} ist] daz ist V (T)  $\cdot$ allez] \textit{om.} T \textbf{4} heidensch] heidens U  $\cdot$ franzoys] frantzós V frantzoys W \textbf{5} Anschevin] anschefin V antscheuin W anscevin T \textbf{6} iedoch] \textit{om.} W T \textbf{7} wât] muͦt W \textbf{8} wol gesniten al] Wol beschnitten all W gefranzieret wol T \textbf{10} missetæte] missewende V W T \textbf{11} wer] swer V T \textbf{12} ob der] daz er T  $\cdot$ hêrren suoche] herre suͦchen U \textbf{13} den scheid er] Den schiet er U Er scheide in V Den schaidet er W \textbf{14} von im vrâget ich] Von imme fragete ich V ich vragete von im T \textbf{15} si] \textit{om.} W \textbf{16} ez] Auch W  $\cdot$ Zazamanc] zazamang V W \textbf{17} disiu] Diz T  $\cdot$ saget] seite V sagt im W \textbf{18} âvoy] Owẏ V  $\cdot$ welch ein] welchen W \textbf{20} wære] Werent V  $\cdot$ halbez] halber V \textbf{21} endarft] darfft W (T)  $\cdot$ mir des sagen] mirs so (sere W ) loben V (W) sin so loben T  $\cdot$ niht] nht U \textbf{22} hin] das W  $\cdot$ dir des giht] dich des giht U dich vergicht W \textbf{23} ez] Auch W  $\cdot$ eines werden] ein werder V (W) \textbf{24} niht mit armuote] nicht mit auentúre W mit armvͦt wenic T \textbf{26} ouwê] Wey V  $\cdot$ wan] wenue W \textbf{28} durch die] in der V \textbf{29} trecken] strecken V \newline
\end{minipage}
\end{table}
\end{document}
