\documentclass[8pt,a4paper,notitlepage]{article}
\usepackage{fullpage}
\usepackage{ulem}
\usepackage{xltxtra}
\usepackage{datetime}
\renewcommand{\dateseparator}{.}
\dmyyyydate
\usepackage{fancyhdr}
\usepackage{ifthen}
\pagestyle{fancy}
\fancyhf{}
\renewcommand{\headrulewidth}{0pt}
\fancyfoot[L]{\ifthenelse{\value{page}=1}{\today, \currenttime{} Uhr}{}}
\begin{document}
\begin{table}[ht]
\begin{minipage}[t]{0.5\linewidth}
\small
\begin{center}*D
\end{center}
\begin{tabular}{rl}
\textbf{777} & \begin{large}A\end{large}rtus \textbf{werte} si des sân.\\ 
 & vrâge iuch wîb oder man,\\ 
 & wer trüege die rîchsten hant,\\ 
 & der ie von deheime lant\\ 
5 & über tavelrunt \textbf{gesaz},\\ 
 & ir \textbf{en}mugt \textbf{sis} niht bescheiden baz:\\ 
 & ez \textbf{was} Feirefiz Anschevin.\\ 
 & dâ mite lât \textbf{die} rede sîn.\\ 
 & Si zogten gein dem ringe\\ 
10 & mit \textbf{werdeclîchem} dinge.\\ 
 & etslîch vrouwe wart gehurt,\\ 
 & wære ir pfert niht wol gegurt,\\ 
 & si wære gevallen schiere.\\ 
 & Manege rîche baniere\\ 
15 & sach man zallen \textbf{sîten} komen.\\ 
 & dâ wart der buhurt wît genomen\\ 
 & al umbe der tavelrunde rinc.\\ 
 & ez wâren \textbf{höfschlîchiu} dinc,\\ 
 & daz ir deheiner \textbf{in} den rinc \textbf{gereit}.\\ 
20 & daz velt was ûzerhalp sô breit,\\ 
 & si mohten dors ersprengen\\ 
 & unt sich mit \textbf{hurten} mengen\\ 
 & und \textbf{ouch} mit künste \textbf{rîten} \textbf{sô},\\ 
 & \textbf{des} diu wîp ze sehen wâren vrô.\\ 
25 & Si kômen \textbf{ouch}, dâ si sâzen,\\ 
 & al dâ die werden âzen.\\ 
 & kamerære, truhsæzen, schenken\\ 
 & muosen \textbf{daz} bedenken,\\ 
 & wie manz mit \textbf{zuht} \textbf{dar} vür \textbf{getruoc}.\\ 
30 & ich wæne, man gab in dâ genuoc.\\ 
\end{tabular}
\scriptsize
\line(1,0){75} \newline
D \newline
\line(1,0){75} \newline
\textbf{1} \textit{Initiale} D  \textbf{9} \textit{Majuskel} D  \textbf{14} \textit{Majuskel} D  \textbf{25} \textit{Majuskel} D  \newline
\line(1,0){75} \newline
\textbf{7} Anschevin] Anscivin D \newline
\end{minipage}
\hspace{0.5cm}
\begin{minipage}[t]{0.5\linewidth}
\small
\begin{center}*m
\end{center}
\begin{tabular}{rl}
 & Artus \textbf{werte} \textit{si} des sân.\\ 
 & vrâge iuch \textbf{nû} wîp oder man,\\ 
 & wer trüege die rîchesten hant,\\ 
 & der ie von dekeinem lant\\ 
5 & über tavelrunde \textbf{saz},\\ 
 & ir mogt \textbf{sis} niht bescheiden baz:\\ 
 & ez \textbf{was} Fere\textit{fiz} A\textit{n}schevin.\\ 
 & dâ mit lât \textbf{die} red\textit{e s}în.\\ 
 & si zogten gegen dem ringe\\ 
10 & mit \textbf{werdeclîchem} dinge.\\ 
 & etlîch \textit{vrow}e wart \textbf{dô} gehurt,\\ 
 & wær ir pfert niht wol gegurt,\\ 
 & si \textit{wære} gevallen schier.\\ 
 & manige rîch banier\\ 
15 & sac\textit{h} man zuo allen \textbf{sîten} komen.\\ 
 & d\textit{â} wart der buhurt wît genomen\\ 
 & a\textit{l} um\textit{b} der tavelrunder rinc.\\ 
 & ez wâren \textbf{hovelîchiu} \textit{d}inc,\\ 
 & daz ir dekeiner \textbf{in} den rinc \textbf{ger\textit{ei}t}.\\ 
20 & daz velt was ûzerhalp sô breit,\\ 
 & si mohten diu ros ersprengen\\ 
 & und sich mit \textbf{hurte} mengen\\ 
 & und \textbf{ouch} mit kunst \textbf{rîten} \textbf{sô},\\ 
 & \textbf{daz} diu wîp zuo sehen wâren vrô.\\ 
25 & si kômen \textbf{ouch}, d\textit{â} si sâzen,\\ 
 & aldâ die werden âzen.\\ 
 & k\textit{am}er\textit{æ}r\textit{e}, truhsæzen, schenken\\ 
 & muosten \textbf{daz} bedenken,\\ 
 & wie man ez mit \textbf{zuht} \textbf{d\textit{â}} vür \textbf{truoc}.\\ 
30 & ich wæne, man gap in d\textit{â} genuoc.\\ 
\end{tabular}
\scriptsize
\line(1,0){75} \newline
m n o V V' W \newline
\line(1,0){75} \newline
\textbf{3} \textit{Capitulumzeichen} n  \textbf{9} \textit{Initiale} W  \newline
\line(1,0){75} \newline
\textbf{1} si] in m n o  $\cdot$ des] dasz o (V') \textbf{2} iuch nû] uch noch o ich W \textbf{3} die] dú V \textbf{4} ie] ye do W  $\cdot$ dekeinem] do keinem n \textbf{5} tavelrunde] tauelrunder V' W  $\cdot$ saz] gesas V V' \textbf{6} sis] sich o sy des W  $\cdot$ niht] \textit{om.} n \textbf{7} Ferefiz] fere m ferrefis n ferefis o ferevis V fereuis V' ferafis W  $\cdot$ Anschevin] auscevin m n anesce vin o anschefin V antscheuein W \textbf{8} rede sîn] rede din vnd sin m \textbf{9} zogten] zeigeten n zoigten o \textbf{11} vrowe] rede m  $\cdot$ gehurt] [gefar]: gehúrt o \textbf{12} ir pfert] irm [pfer*]: pferde V \textbf{13} wære] \textit{om.} m n o \textbf{15} sach] Sacha m  $\cdot$ sîten] ziten n \textbf{16} dâ] Do m n o V V' W  $\cdot$ wît] [wit]: wil o \textbf{17} al umb] Alle vmd m Alle vmmb o  $\cdot$ tavelrunder] taffelrunde n (o) (V) \textbf{18} hovelîchiu] hoͤueschliche V (V')  $\cdot$ dinc] ling m \textbf{19} ir dekeiner] ir do keiner n erdekeinen o ir nie keiner V (V')  $\cdot$ gereit] geriet m rait W \textbf{20} velt] velt velt V' \textbf{21} mohten] brochten o moͤhtent V \textbf{22} hurte] húrten V (V') \textbf{23} \textit{Die Verse 777.23-24 fehlen} V'   $\cdot$ kunst rîten] kúnsten rittent V hurte ringen W \textbf{24} diu] \textit{om.} W  $\cdot$ sehen] sehende W  $\cdot$ wâren] werden W \textbf{25} dâ] do m n V V' W \textbf{26} aldâ] Alda da o Vnd do V' \textbf{27} kamerære] Conuͯertuͯr m (n) (o) Camer V' \textbf{28} muosten] Muͯsten n o Muͤstent V \textbf{29} zuht] zuchten V' (W)  $\cdot$ dâ] do m n o V V' W  $\cdot$ truoc] [*]: getruͦg V getrug V' \textbf{30} gap] gebe W  $\cdot$ dâ] do m n o V V' W \newline
\end{minipage}
\end{table}
\newpage
\begin{table}[ht]
\begin{minipage}[t]{0.5\linewidth}
\small
\begin{center}*G
\end{center}
\begin{tabular}{rl}
 & \begin{large}A\end{large}rtus \textbf{werte} si des sân.\\ 
 & vrâge iuch wîp oder man,\\ 
 & wer trüege die rîchesten hant,\\ 
 & der ie von deheinem lant\\ 
5 & über tavelrunder \textbf{gesaz},\\ 
 & ir\textbf{ne} müget\textbf{s in} niht bescheiden baz:\\ 
 & ez \textbf{was} Feirafiz Antschevin.\\ 
 & dâ mit lât \textbf{die} rede sîn.\\ 
 & si zogten gein dem ringe\\ 
10 & mit \textbf{werdeclîchem} dinge.\\ 
 & etslîch vrouwe wart gehurt,\\ 
 & wære ir pfer\textit{t} niht wol gegurt,\\ 
 & si wære gevallen schiere.\\ 
 & manec rîche baniere\\ 
15 & sach man zallen \textbf{sîten} komen.\\ 
 & dâ wart der buhurt wît genomen\\ 
 & alumbe der tavelrunder rinc.\\ 
 & ez wâren \textbf{höfschlîchiu} dinc,\\ 
 & daz ir deheiner \textbf{an} den rinc \textbf{reit}.\\ 
20 & daz velt was ûzerhalp sô breit,\\ 
 & si mohten diu ors ersprengen\\ 
 & unde sich mit \textbf{hurte} mengen\\ 
 & unde \textbf{iedoch} mit kunst \textbf{alsô},\\ 
 & \textbf{des} diu wîp ze sehene wâren vrô.\\ 
25 & si kômen \textbf{ouch}, dâ si sâzen,\\ 
 & aldâ die werden âzen.\\ 
 & kamerære, truhsæzen, schenken\\ 
 & muosen \textbf{daz} bedenken,\\ 
 & wie manz mit \textbf{zühten} vür \textbf{truoc}.\\ 
30 & ich wæne, man gap in dâ genuoc.\\ 
\end{tabular}
\scriptsize
\line(1,0){75} \newline
G I L M Z \newline
\line(1,0){75} \newline
\textbf{1} \textit{Initiale} G L M Z  \newline
\line(1,0){75} \newline
\textbf{1} werte] wert I \textbf{2} iuch] ich I  $\cdot$ oder] olde G \textbf{3} rîchesten] richen L \textbf{4} von] vz L  $\cdot$ deheinem lant] [der haiden lant]: dehainem lant I \textbf{5} gesaz] saz L (M) Z \textbf{6} irne] ir I (Z)  $\cdot$ mügets in] mugt ez I mvgt siz L (Z) mogz sisz M \textbf{7} Feirafiz] feirefiz G Z ferefiz L feirefisz M  $\cdot$ Antschevin] anschevin G antscheuin I Anshevin L (Z) ansevin M \textbf{10} werdeclîchem] wertlichem I \textbf{12} Het man sý iender gervrt L  $\cdot$ wære ir] Weresz M  $\cdot$ pfert] pharde G \textbf{15} sîten] ziten I \textbf{16} dâ] do I  $\cdot$ genomen] [bikome]: genommen M \textbf{17} tavelrunder] Tavelrvnde L \textbf{18} wâren] weren Z  $\cdot$ höfschlîchiu] hofeliche M \textbf{19} ir] \textit{om.} I L  $\cdot$ an] in L (M) Z \textbf{20} sô] als Z \textbf{23} iedoch] ouch M doch Z  $\cdot$ alsô] so L M riten so Z \textbf{24} des] daz I Daz ez L  $\cdot$ wîp] \textit{om.} L  $\cdot$ ze sehene wâren] waren zuͯ sehen L zcu sehene weren M (Z) \textbf{25} kômen] quam Z \textbf{26} aldâ] al I \textbf{27} schenken] senchen I \textbf{28} bedenken] wol bedenchen I \textbf{29} vür] da vor L (M) (Z)  $\cdot$ truoc] getrvc Z \textbf{30} gap] Geb I (Z)  $\cdot$ dâ] \textit{om.} M \newline
\end{minipage}
\hspace{0.5cm}
\begin{minipage}[t]{0.5\linewidth}
\small
\begin{center}*T
\end{center}
\begin{tabular}{rl}
 & Artus \textbf{gewerte} si des sân.\\ 
 & vrâge iuch wîp oder man,\\ 
 & wer trüege die rîchesten hant,\\ 
 & der ie von de\textit{h}eine\textit{m} lant\\ 
5 & über tavelrunder \textbf{saz},\\ 
 & ir \textbf{en}muget \textbf{si es} niht bescheiden baz:\\ 
 & \textit{ez} \textbf{ist} Ferefis Anschevin.\\ 
 & dâ mite lât \textbf{dise} rede sîn.\\ 
 & si z\textit{o}geten gein dem ringe\\ 
10 & mit \textbf{werlîchem} dinge.\\ 
 & etslîchiu vrouwe wart gehurt,\\ 
 & \textbf{und} wære ir pfert niht wol gegurt,\\ 
 & si wære gevallen schiere.\\ 
 & manege rîche baniere\\ 
15 & sach man zuo allen \textbf{zîten} komen.\\ 
 & d\textit{â} wart der buhurt wît genomen\\ 
 & al umb der tavelrunder rinc.\\ 
 & ez wâren \textbf{höveschiu} dinc,\\ 
 & daz ir dekeiner \textbf{in} den rinc \textbf{reit}.\\ 
20 & daz velt wa\textit{s} \textit{û}zerhalp sô breit,\\ 
 & si mohten diu ors ersprengen\\ 
 & und sich mit \textbf{hurte} mengen\\ 
 & und \textbf{doch} mit kunst \textbf{rîten} \textbf{sô},\\ 
 & \textbf{daz} diu wîp zuo sehene wâren vrô.\\ 
25 & si kâmen \textbf{ûz}, dâ si sâzen,\\ 
 & al dâ die werden âzen.\\ 
 & kamerære, truhsæzen, schenken\\ 
 & muosen \textbf{dâ} bedenken,\\ 
 & wie man ez mit \textbf{zühten} vür \textbf{truoc}.\\ 
30 & ich wæne, man \textit{gap in} dâ genuoc.\\ 
\end{tabular}
\scriptsize
\line(1,0){75} \newline
U Q R \newline
\line(1,0){75} \newline
\newline
\line(1,0){75} \newline
\textbf{1} gewerte] werte Q R \textbf{3} trüege] truͦg R  $\cdot$ rîchesten] rechten Q \textbf{4} der] Die R  $\cdot$ deheinem] dem heinen U \textbf{6} ir enmuget] Jren mocht Q Jr mugent R  $\cdot$ es] \textit{om.} Q \textbf{7} ez] \textit{om.} U  $\cdot$ ist] was Q R  $\cdot$ Ferefis] feirefisz Q feriefis R  $\cdot$ Anschevin] ansheúin Q \textbf{8} lât] lag R  $\cdot$ dise] die Q R \textbf{9} zogeten] zageten U \textbf{10} werlîchem] werdiglichem Q werdenklichen R \textbf{11} vrouwe wart] frowen waurent R  $\cdot$ gehurt] [gehűrt]: gegűrt Q \textbf{12} und] \textit{om.} Q R \textbf{13} wære] weren R \textbf{15} zîten] seiten Q sitten R \textbf{16} dâ] Do U Q R  $\cdot$ buhurt] buchiert R  $\cdot$ wît] mit Q \textbf{17} Al vmb vnd vmb der tauelrund R \textbf{18} höveschiu] hobsliche Q hofflichú R \textbf{20} was ûzerhalp] wazzerhalp U \textbf{23} Vnd doch mit hurtte vnd ::: R \textbf{24} daz] Des Q  $\cdot$ wîp] frowen R  $\cdot$ sehene] sechent R \textbf{25} ûz] auch Q (R)  $\cdot$ dâ] do Q \textbf{26} al dâ] Vnd R \textbf{28} dâ] do Q das R \textbf{29} truoc] getruͦg R \textbf{30} gap in] \textit{om.} U  $\cdot$ dâ] do Q R \newline
\end{minipage}
\end{table}
\end{document}
