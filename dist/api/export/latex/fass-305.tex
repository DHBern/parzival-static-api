\documentclass[8pt,a4paper,notitlepage]{article}
\usepackage{fullpage}
\usepackage{ulem}
\usepackage{xltxtra}
\usepackage{datetime}
\renewcommand{\dateseparator}{.}
\dmyyyydate
\usepackage{fancyhdr}
\usepackage{ifthen}
\pagestyle{fancy}
\fancyhf{}
\renewcommand{\headrulewidth}{0pt}
\fancyfoot[L]{\ifthenelse{\value{page}=1}{\today, \currenttime{} Uhr}{}}
\begin{document}
\begin{table}[ht]
\begin{minipage}[t]{0.5\linewidth}
\small
\begin{center}*D
\end{center}
\begin{tabular}{rl}
\textbf{305} & "\begin{large}I\end{large}\textbf{ne} \textbf{wil} gein \textbf{dir} niht liegens pflegen",\\ 
 & \textbf{sprach Gawan}. "hie ist von tjost gelegen\\ 
 & Segramors, ein strîtes helt,\\ 
 & des tât gein prîse ie was \textbf{erwelt}.\\ 
5 & \textbf{dû tætez}, ê Keie \textbf{wart} gevalt.\\ 
 & an in bêden \textbf{hâstû} prîs bezalt."\\ 
 & Si riten mit ein ander dan,\\ 
 & der Waleis unt Gawan.\\ 
 & vil volkes zorse unt ze vuoz\\ 
10 & dort inne bôt \textbf{in} werden gruoz,\\ 
 & Gawane unt dem rîter rôt,\\ 
 & wande in ir zuht daz gebôt.\\ 
 & Gawan kêrte, dâ er sîn poulûn vant.\\ 
 & vroun Cunnewaren de Lalant,\\ 
15 & \textbf{ir} snüere \textbf{unz} an die sîne gienc.\\ 
 & \textbf{diu} wart \textbf{vil} vrô: mit vreude enpfienc\\ 
 & diu magt \textbf{ir} rîter, der si rach,\\ 
 & daz ir von Keien ê geschach.\\ 
 & Si nam ir bruoder \textbf{an die} hant\\ 
20 & unt \textbf{vroun} Jeschuten von Karnant.\\ 
 & sus sach si komen Parzival.\\ 
 & der was gevar durch îsers mâl,\\ 
 & als touwige rôsen \textbf{wæren dar} gevlogen.\\ 
 & im was sîn harnasch ab gezogen.\\ 
25 & er spranc ûf, dô er \textbf{die vrouwen} sach.\\ 
 & nû hœret, wie Cunneware sprach:\\ 
 & "\begin{large}G\end{large}ot alrêst, dar nâch mir\\ 
 & \textbf{west willekomen}, \textbf{sît} daz ir\\ 
 & belibt bî manlîchen siten.\\ 
30 & ich hete lachen gar vermiten,\\ 
\end{tabular}
\scriptsize
\line(1,0){75} \newline
D \newline
\line(1,0){75} \newline
\textbf{1} \textit{Initiale} D  \textbf{7} \textit{Majuskel} D  \textbf{19} \textit{Majuskel} D  \textbf{27} \textit{Initiale} D  \newline
\line(1,0){75} \newline
\textbf{19} an die] ande D \textbf{20} Jeschuten] Jescv̂ten D \newline
\end{minipage}
\hspace{0.5cm}
\begin{minipage}[t]{0.5\linewidth}
\small
\begin{center}*m
\end{center}
\begin{tabular}{rl}
 & "ich \textbf{en}\textbf{wil} gegen \textbf{dir} niht liegens pflegen",\\ 
 & \textbf{sprach Gawan}. "hie ist von \textit{j}ust gelegen\\ 
 & Segramors, ein strîtes helt,\\ 
 & d\textit{es} tât gegen prîse ie was \textbf{erw\textit{e}lt}.\\ 
5 & \textbf{dû tæt ez}, ê Keie \textbf{was} gevalt.\\ 
 & an in beiden \textbf{hâst dû} prîs bezalt."\\ 
 & \begin{large}S\end{large}i riten mit ein ander dan,\\ 
 & der Waleis und Gawan.\\ 
 & vil volkes ze ros und ze vuoz\\ 
10 & dort i\textit{nn}e bôt \textbf{in} werden gruoz,\\ 
 & Gawane und dem ritter rôt,\\ 
 & wand in ir zuht daz gebôt.\\ 
 & Gawan kêrte, d\textit{â} er sîn p\textit{a}v\textit{e}lûne vant.\\ 
 & vrouwen Cunnew\textit{a}ren de Lalant\\ 
15 & snüere an die sîne gienc.\\ 
 & \textbf{diu} wart vrô: mit vröude enpfienc\\ 
 & diu mag\textit{et} \textbf{ir} ritter, der si rach,\\ 
 & daz ir von Keien ê geschach.\\ 
 & si nam ir bruoder \textbf{an ir} hant\\ 
20 & und Jeschuten von Karnant.\\ 
 & sus sach si komen Parcifal.\\ 
 & der was gevar durch îsers mâl,\\ 
 & als touwige rô\textit{s}e\textit{n} \textbf{wæren dar} gevlogen.\\ 
 & im was sîn ha\textit{r}na\textit{s}ch abe gezogen.\\ 
25 & er spranc ûf, dô er \textbf{die vrouwen} sach.\\ 
 & nû hœret, wie Cunn\textit{e}w\textit{a}re sprach:\\ 
 & "gote aller êrst \textbf{und} dar nâch mir\\ 
 & \textbf{west willekomen}, \textbf{sît} daz ir\\ 
 & belib\textit{e}t bî manlîchen siten.\\ 
30 & ich hete lachen gar vermiten,\\ 
\end{tabular}
\scriptsize
\line(1,0){75} \newline
m n o Fr69 \newline
\line(1,0){75} \newline
\textbf{7} \textit{Initiale} m   $\cdot$ \textit{Capitulumzeichen} n  \newline
\line(1,0){75} \newline
\textbf{1} enwil] wil n o \textbf{2} Gawan] gewan o  $\cdot$ just] lust m \textbf{3} Segramors] Sagromors n S:::mors Fr69 \textbf{4} des] Dat m  $\cdot$ tât] dot n o  $\cdot$ erwelt] er wolt m \textbf{5} Keie] [kẏe*]: kẏe n keẏ o \textbf{8} Der waleisse vnd gewan o \textbf{10} inne] yme m  $\cdot$ in] ẏnne o \textbf{11} Gawane] Gawan n Gewan o \textbf{13} Gawan] Gewan o  $\cdot$ dâ] do m n o  $\cdot$ pavelûne] peualune m pantelin o \textbf{14} vrouwen] Frouwe m (n) (o)  $\cdot$ Cunnewaren] Cunneweren m conwaren n Conne waren o \textbf{15} snüere] Suffer n  $\cdot$ sîne] sunne n sẏnne o \textbf{16} vröude] freiden n (o) \textbf{17} maget] mag m  $\cdot$ ir] den n o \textbf{18} Keien ê] keẏen n keẏene o \textbf{19} nam] mam o  $\cdot$ ir hant] die hant n o \textbf{20} Jeschuten] jescutten m froͧwe jescuten n (o)  $\cdot$ Karnant] kornant n (o) \textbf{22} der] Des o \textbf{23} touwige] toubigen o  $\cdot$ rôsen] rosse m roisen o  $\cdot$ wæren] wer n o \textbf{24} harnasch] harsnach m harnersch o \textbf{25} ûf] \textit{om.} n o \textbf{26} Cunneware] Cunuwere m conneware n Conne waren o \textbf{28} west] Wosent o \textbf{29} belibet] Belibot m  $\cdot$ manlîchen] manlichem n o \newline
\end{minipage}
\end{table}
\newpage
\begin{table}[ht]
\begin{minipage}[t]{0.5\linewidth}
\small
\begin{center}*G
\end{center}
\begin{tabular}{rl}
 & "ich\textbf{ne} \textbf{mac} gein \textbf{dir} niht liegens pflegen",\\ 
 & \textbf{sprach Gawan}. "hie ist von tjost gelegen\\ 
 & Segremors, ein strîtes helt,\\ 
 & des tât gein prîse ie was \textbf{gezelt}.\\ 
5 & \textbf{daz was}, ê Kay \textbf{würde} gevalt.\\ 
 & an in beiden \textbf{hâstû} brîs bezalt."\\ 
 & si riten mit ein ander dan,\\ 
 & der Waleis unde Gawan.\\ 
 & vil volkes ze orse unde ze vuoz\\ 
10 & dort inne bôt \textbf{in} werden gruoz,\\ 
 & \hspace*{-.7em}\big| wan in ir zuht daz gebôt,\\ 
 & \hspace*{-.7em}\big| Gawane unde dem rîter rôt.\\ 
 & Gawan kêrte, dâ er sîn pavelûn vant.\\ 
 & vroun Kunewaren de Lalant,\\ 
15 & \textbf{der} snüere \textbf{unze} an die sîne gienc.\\ 
 & \textbf{si} wart vrô: mit vröuden \textbf{si} enpfienc,\\ 
 & diu maget, \textbf{ir} rîter, der si rach,\\ 
 & daz ir von Kay ê geschach.\\ 
 & si nam ir bruoder \textbf{an die} hant\\ 
20 & unde \textbf{vroun} Jeschuten von Karnant.\\ 
 & sus sach si komen Parzival.\\ 
 & der was gevar durch îseres mâl,\\ 
 & \begin{large}A\end{large}ls touwege rôsen \textbf{wæren dar} gevlogen.\\ 
 & im was sîn harnasch abe gezogen.\\ 
25 & er spranc ûf, dô er \textbf{si komen} sach.\\ 
 & nû hœret, wie Kuneware sprach:\\ 
 & "got alrêst, dar nâch mir\\ 
 & \textbf{willekomen sît}, daz ir\\ 
 & belibet bî manlîchen siten.\\ 
30 & ich hete lachen gar vermiten,\\ 
\end{tabular}
\scriptsize
\line(1,0){75} \newline
G I O L M Q R Z \newline
\line(1,0){75} \newline
\textbf{3} \textit{Illustration mit Überschrift:} Hie vart parczifal mit here Gawan vnd wirt empfangen von Cuͦnwaren R  \textbf{7} \textit{Initiale} L Z  \textbf{13} \textit{Initiale} O R  \textbf{16} \textit{Initiale} I  \textbf{23} \textit{Initiale} G  \textbf{27} \textit{Initiale} I M  \newline
\line(1,0){75} \newline
\textbf{1} ichne mac] ich mach I (O) (L) (M) (Q) (R) Jch enwil Z \textbf{3} Segremors] Saýgremors L Seigremors R  $\cdot$ ein] an I \textbf{4} gein prîse] geprise R  $\cdot$ ie was] was ie M \textbf{5} Kay] kai G kain I key O Q R Z keie M  $\cdot$ würde] vnder R \textbf{6} in] \textit{om.} I  $\cdot$ brîs] \textit{om.} I Z \textbf{7} si] pris si I  $\cdot$ ander] andren R \textbf{8} parzifal vnde [*]: Gawan I \textbf{9} volkes] volke Q  $\cdot$ unde] vnd auch I  $\cdot$ ze vuoz] fuͯsze L \textbf{10} dort inne bôt in] bot in dort inne I Dor inne bot in Q Dort inne boten Z  $\cdot$ gruoz] gruͯsze L \textbf{12} \textit{Versfolge 305.11-12} O L M Q R Z  \textbf{11} Gawane] Gawan I O L R Z Gawannen Q  $\cdot$ dem] der L \textbf{13} Gawan] ÷Awan O Gawann Q  $\cdot$ kêrte] kert Q Z  $\cdot$ dâ] do Q  $\cdot$ sîn pavelûn] sine pavilvne L sine gezellte R pavelun Z \textbf{14} vroun] Vrow L (M) (R)  $\cdot$ Kunewaren] Gunwarn I Gvnwaren O Cvnewaren L Z kunawaren M konwaren Q Cuͦnwarten R  $\cdot$ de] der O (M) \textbf{15} der snüere] Das swert M Die snűre Q Der schnuͦr R  $\cdot$ unze] vuͯz L  $\cdot$ die] daz L (M)  $\cdot$ sîne] sinen Z \textbf{16} mit vröuden si] mit frevden O (L) (M) (R) Z sie mit frewden Q  $\cdot$ enpfienc] in enphienc I pfiench O \textbf{17} ir] den O R irn L Z  $\cdot$ rach] sach M \textbf{18} Kay] kei G kain I key O Q R kaýen L keien M keyn Z  $\cdot$ ê] y M (R) \textbf{19} die] ir I L M Z \textbf{20} vroun] vrow L (M) (R)  $\cdot$ Jeschuten] ieschuten G ieskuten I Jescvten O Jescuͯten L letschuten M iescuten Q Z Jscuten R  $\cdot$ Karnant] carnant O \textbf{21} si komen] komen sý L  $\cdot$ Parzival] parzifal I M Barcifal O parcifal L Z partzifal Q parczifal R \textbf{22} der] Her M  $\cdot$ was] \textit{om.} Z  $\cdot$ îseres] isen I (O) ýsens L (R) \textbf{23} als er im twuͦc rosen I  $\cdot$ touwege] touge M togne R  $\cdot$ wæren dar] >warn< dar G dar waren L weren Z \textbf{24} was] wart Q \textbf{25} er] Da M  $\cdot$ dô] du M da Z \textbf{26} \textit{Versdoppelung 305.26 (\textasciicircum2M) nach 585.17; Lesarten der vorausgehenden Verse mit \textasciicircum1M bezeichnet} M   $\cdot$ hœret] hoͯren R  $\cdot$ Kuneware] Gunware do I kvnware O Cvneware L kunware \textsuperscript{1}\hspace{-1.3mm} M kundewar \textsuperscript{2}\hspace{-1.3mm} M konware Q Cuͦnware R \textbf{27} got] ÷ot I  $\cdot$ alrêst] alrest alrest I  $\cdot$ dar] vnd dar R \textbf{28} willekomen sît] Sit willichomen sit O (L) (M) (Q) (R) West willikomen sit Z \textbf{29} belibet] Bleibet Q \textbf{30} hete] hatte M  $\cdot$ lachen gar] gar lachen R \newline
\end{minipage}
\hspace{0.5cm}
\begin{minipage}[t]{0.5\linewidth}
\small
\begin{center}*T
\end{center}
\begin{tabular}{rl}
 & "Ich \textbf{mac} gegen \textbf{iu} niht liegens pflegen,\\ 
 & hie ist von \textbf{iuwerre} tjost gelegen\\ 
 & Segremors, ein strîtes helt,\\ 
 & des tât gegen prîse ie was \textbf{gezelt}.\\ 
5 & \textbf{daz was}, ê Key \textbf{würde} gevalt.\\ 
 & an in beiden \textbf{hât ir} prîs bezalt."\\ 
 & \begin{large}S\end{large}i riten mit ein ander dan,\\ 
 & der Waleis unde Gawan.\\ 
 & vil volkes ze orse unde ze vuoz,\\ 
10 & dort inne bôt \textbf{man} werden gruoz\\ 
 & Gawane unde dem rîter rôt,\\ 
 & wan in ir zuht daz gebôt.\\ 
 & Gawan kêrte, dâ er sîn pavelûn vant.\\ 
 & vroun Cunnewaren de Lalant,\\ 
15 & \textbf{der} snüere an die sîne gienc.\\ 
 & \textbf{si} wart vrô: mit vröuden \textbf{si}\textbf{n} enpfienc,\\ 
 & diu maget \textbf{den} rîter, der si rach,\\ 
 & daz ir von Key ê geschach.\\ 
 & si nam ir brouder \textbf{bî der} hant\\ 
20 & unde \textbf{vroun} Jeschuten von Garnant.\\ 
 & sus sach si komen Parcifal.\\ 
 & der was gevar durch îsers mâl,\\ 
 & als touwige rôsen \textbf{dar wæren} gevlogen.\\ 
 & im was sîn harnasch abe gezogen.\\ 
25 & Er spranc ûf, dô er \textbf{si komen} sach.\\ 
 & nû hœret, wie \textbf{vrou} Cunnewar sprach:\\ 
 & "Got aller êrst, dar nâch mir\\ 
 & \textbf{sît willekome}, \textbf{sît} daz ir\\ 
 & belibet bî manlîchen siten.\\ 
30 & ich hete lachen gar vermiten,\\ 
\end{tabular}
\scriptsize
\line(1,0){75} \newline
T U V W \newline
\line(1,0){75} \newline
\textbf{1} \textit{Majuskel} T  \textbf{7} \textit{Überschrift:} Hie reit her partzifal mit her gawan an kúnig artus hoffe vnd ward erlich enpfangen W   $\cdot$ \textit{Initiale} T U V W  \textbf{25} \textit{Majuskel} T  \textbf{27} \textit{Majuskel} T  \newline
\line(1,0){75} \newline
\textbf{1} Ich] [*]: Jch V  $\cdot$ iu] dir V W  $\cdot$ niht] \textit{om.} U \textbf{2} hie] Sprach Gawan hie U (V) (W)  $\cdot$ iuwerre tjost] [iu*]: iust V tyost W \textbf{3} Segremors] [S*gremors]: Sagremors V \textbf{5} daz was ê] [D*]: Du tet es e V  $\cdot$ Key] keẏn V \textbf{6} hât ir] hant irn U [*]: hast du V hastu W  $\cdot$ bezalt] gezalt U \textbf{8} Waleis] walleis V \textbf{10} man] man in U V man im W  $\cdot$ werden] \textit{om.} W \textbf{11} Gawane] Gawan U W \textbf{13} dâ] do U V W  $\cdot$ sîn] in sin U [si*]: sine V \textbf{14} vroun] Vrov U (W)  $\cdot$ Cunnewaren] kvnnewaren U V kunnewarn W \textbf{15} sîne] seinen W \textbf{16} mit] in W \textbf{17} den] [I]: den V \textbf{18} Key] [*]: keẏn V \textbf{20} vroun] vrov U (W) [*]: vron  V  $\cdot$ Jeschuten] Jescvten T Jescuͦte U iescuten V iestuten W \textbf{21} Parcifal] parzifâl T parzifal V partzifalen W \textbf{22} durch îsers mâl] nach eisen malen W \textbf{23} Als ein dauͦwige rose dar were gevlogen U \textbf{26} vrou] \textit{om.} W  $\cdot$ Cunnewar] kuͦmewar U kvnneware V kunnewar W \textbf{27} Got] [Gtte]: Gotte V  $\cdot$ aller êrst] zuͦm ersten W \textbf{29} belibet] Bleibent W  $\cdot$ siten] striten U \newline
\end{minipage}
\end{table}
\end{document}
