\documentclass[8pt,a4paper,notitlepage]{article}
\usepackage{fullpage}
\usepackage{ulem}
\usepackage{xltxtra}
\usepackage{datetime}
\renewcommand{\dateseparator}{.}
\dmyyyydate
\usepackage{fancyhdr}
\usepackage{ifthen}
\pagestyle{fancy}
\fancyhf{}
\renewcommand{\headrulewidth}{0pt}
\fancyfoot[L]{\ifthenelse{\value{page}=1}{\today, \currenttime{} Uhr}{}}
\begin{document}
\begin{table}[ht]
\begin{minipage}[t]{0.5\linewidth}
\small
\begin{center}*D
\end{center}
\begin{tabular}{rl}
\textbf{338} & \begin{large}D\end{large}er nie gewarb nâch schanden,\\ 
 & eine wîle zuo sînen handen\\ 
 & sol nû \textbf{dise} âventiure hân\\ 
 & der werde erkande Gawan.\\ 
5 & \textbf{diu prüevet} \textbf{manegen} âne haz\\ 
 & dâr neben oder \textbf{vür in baz}\\ 
 & den \textbf{des mæres} hêrren Parzival.\\ 
 & swer \textbf{sînen} vriunt alle mâl\\ 
 & mit worten an daz hœhste jagt,\\ 
10 & der ist prîses anderhalp verzagt.\\ 
 & \textbf{nû wære} der liute volge guot,\\ 
 & \textbf{swer} dicke lop mit wârheit tuot,\\ 
 & \textbf{wan} swaz er sprichet oder sprach,\\ 
 & diu rede belîbet âne dach.\\ 
15 & Wer sol sinnes wort behalten,\\ 
 & es enwellen die wîsen walten?\\ 
 & valsch, lügelîch ein mære,\\ 
 & daz, wæn ich, \textbf{baz noch} wære\\ 
 & âne wirt ûf \textbf{eime} snê,\\ 
20 & sô daz dem munde würde wê,\\ 
 & derz ûz vür wârheit breitet.\\ 
 & sô het in got bereitet,\\ 
 & als guoter liute wünschen stêt,\\ 
 & den ir triwe zarbeite ergêt.\\ 
25 & Swem ist ze sölhen werken gâch,\\ 
 & dâ missewende hœret nâch,\\ 
 & pflihtet werder lîp an den gewin,\\ 
 & daz muoz in lêren \textbf{kranker} sin.\\ 
 & er mîdet ez \textbf{ê}, kan er sich schemen;\\ 
30 & den site sol er ze vogte nemen.\\ 
\end{tabular}
\scriptsize
\line(1,0){75} \newline
D \newline
\line(1,0){75} \newline
\textbf{1} \textit{Großinitiale} D  \textbf{15} \textit{Majuskel} D  \textbf{25} \textit{Majuskel} D  \newline
\line(1,0){75} \newline
\newline
\end{minipage}
\hspace{0.5cm}
\begin{minipage}[t]{0.5\linewidth}
\small
\begin{center}*m
\end{center}
\begin{tabular}{rl}
 & \begin{large}D\end{large}er nie gewarp nâch schanden,\\ 
 & eine wîle ze sînen handen\\ 
 & sol nû \textbf{die} âventiure hân\\ 
 & der werde erkante Gawan.\\ 
5 & \textbf{die brüefent} \textbf{manigen} âne haz\\ 
 & dâr neben oder \textbf{vür \textit{i}n baz}\\ 
 & danne \textbf{des mæres} hêrren Parcifal.\\ 
 & wer \textbf{sînen} vriunt alle mâl\\ 
 & mit worten anz hœhest\textit{e} jaget,\\ 
10 & der ist prîses anderhalp verzaget.\\ 
 & \textbf{ime wære} der liute volge guot,\\ 
 & \textbf{wer} dicke lop mit wârheit tuot,\\ 
 & \textbf{wanne} waz er spr\textit{i}ch\textit{et} oder spr\textit{a}c\textit{h},\\ 
 & diu rede belîbet âne \textit{da}c\textit{h}.\\ 
15 & wer sol sinnes wort behalten,\\ 
 & es enwellen die wîsen walten?\\ 
 & valsch, lügelîch ein mære,\\ 
 & daz, wæ\textit{n} ich, \textbf{noch baz} wære\\ 
 & âne wirt ûf \textbf{einem} snê,\\ 
20 & sô daz dem munde würde wê,\\ 
 & \dag daz\dag  ez ûz \textit{vür} wâr\textit{h}e\textit{it} \textit{b}reitet.\\ 
 & sô het in got bereitet,\\ 
 & als guoter liute wünschen stât,\\ 
 & den ir triuwe ze\textbf{r} arbeit ergât.\\ 
25 & wem ist ze solhen wer\textit{k}en gâch,\\ 
 & dâ missewende hœret nâch,\\ 
 & pflihtet werde\textit{r l}îp an den gewin,\\ 
 & daz muoz \textit{in} lêren \textbf{kranker} sin.\\ 
 & er mîdet ez \textbf{ê}, kan er sich schemen;\\ 
30 & den site sol er ze vogete nemen.\\ 
\end{tabular}
\scriptsize
\line(1,0){75} \newline
m n o \newline
\line(1,0){75} \newline
\textbf{1} \textit{Illustration mit Überschrift:} Hie gant Gawanes offenturen (gawans offentúre n  ) an wie der ze sinem kampfe fuͯr vnd gedaget man parcifals ein lange wile m (n)  Hie gat gawans afentúre an wie zuͦ sẏnem kampff fuͯr vnd getaget man parcifals eyn lange wile o   $\cdot$ \textit{Initiale} m n o  \newline
\line(1,0){75} \newline
\textbf{1} Der] Wer n \textbf{6} in baz] hinbas m basz o \textbf{7} danne] [Das]: Dan o  $\cdot$ des mæres] des [meren]: meres m das meren o  $\cdot$ hêrren] her n \textbf{9} hœheste] hoͯhesten m \textbf{13} sprichet oder sprach] sprach oder sprichet m \textbf{14} dach] tichet m \textbf{15} sinnes] sin n o \textbf{16} enwellen] wellent n o \textbf{17} lügelîch] lúgerich n \textbf{18} wæn] wer m \textbf{19} wirt] wort n  $\cdot$ einem] eẏnen o \textbf{21} vür wârheit] bewaren m  $\cdot$ breitet] bereittet m \textbf{22} sô het] [Dasz muͯsse]: So hett o \textbf{23} wünschen] wunsch n (o) \textbf{24} zer] zuͯ n (o)  $\cdot$ ergât] [gat]: ergat o \textbf{25} werken] werden m \textbf{26} dâ] Do n o \textbf{27} werder lîp] werder man vnd lip m  $\cdot$ den] dem o \textbf{28} in] \textit{om.} m \textbf{29} kan] kande n \newline
\end{minipage}
\end{table}
\newpage
\begin{table}[ht]
\begin{minipage}[t]{0.5\linewidth}
\small
\begin{center}*G
\end{center}
\begin{tabular}{rl}
 & \begin{large}D\end{large}er nie gewarp nâch schanden,\\ 
 & eine wîle ze sînen handen\\ 
 & sol nû \textbf{dise} âventiure hân\\ 
 & der werde erkande Gawan.\\ 
5 & \textbf{diu prüevet} \textbf{manigen} ân haz\\ 
 & dâr neben oder \textbf{vür in baz}\\ 
 & dane \textbf{des mæres} hêrren Parzival.\\ 
 & swer \textbf{sînen} vriunt alle mâl\\ 
 & mit worten an daz hœheste jaget,\\ 
10 & der ist brîses anderhalp verzaget.\\ 
 & \textbf{im ist} der liute volge guot,\\ 
 & \textbf{der} dicke lop mit wârheit tuot,\\ 
 & \textbf{wan} swaz er spricht oder sprach,\\ 
 & diu rede belîbet âne dach.\\ 
15 & wer sol sinnes wort behalten,\\ 
 & e\textit{s} enwellen die wîsen walten?\\ 
 & valsch, lügelîch ein mære,\\ 
 & daz, wæne ich, \textbf{baz noch} wære\\ 
 & âne wirt ûf \textbf{einem} snê,\\ 
20 & sô daz dem munde würde wê,\\ 
 & derz ûz vür wârheit breitet.\\ 
 & sô het in got bereitet,\\ 
 & als guoter liute wünschen stêt,\\ 
 & den ir triwe ze arbeit ergêt.\\ 
25 & swem ist ze solhen werken gâch,\\ 
 & dâ missewende hœrt nâch,\\ 
 & pfliht werder lîp an den gewin,\\ 
 & daz muoz in lêren \textbf{karger} sin.\\ 
 & er mîdet ez \textbf{ê}, kan er sich schemen;\\ 
30 & den site sol er ze vogte nemen.\\ 
\end{tabular}
\scriptsize
\line(1,0){75} \newline
G I O L M Q R Z Fr39 \newline
\line(1,0){75} \newline
\textbf{1} \textit{Überschrift:} Gawan und kingrimursel begunden hie des streites snel Q   $\cdot$ \textit{Initiale} G I O L Q R Z Fr39  \textbf{25} \textit{Initiale} M  \newline
\line(1,0){75} \newline
\textbf{1} Der] ÷Er O Er M \textbf{2} sînen] sine L \textbf{3} sol] Der sol O  $\cdot$ dise âventiure] dise aventivre nv O disiv auentivre Fr39 \textbf{4} werde] wirt R  $\cdot$ Gawan] her gawan Z \textbf{5} diu prüevet] Der pruͯfete L (Fr39) Dy prufeten M [Die]: Der prufet Q Die pruͯffent R \textbf{7} hêrren] \textit{om.} R  $\cdot$ Parzival] Parzifal I (L) Barcifal O parzifals M partzifal Q parczifal R parcifal Z (Fr39) \textbf{8} swer] Wer L M Q R  $\cdot$ sînen vriunt] sine frivnde O (L) (M) (R) (Z) (Fr39)  $\cdot$ alle] ellev I \textbf{9} daz hœheste] des hochsten Q  $\cdot$ jaget] iagin M \textbf{10} brîses anderhalp] anderthalb gar R  $\cdot$ verzaget] verzcagin M \textbf{12} der] Swer O Z (Fr39) Wer L M Q R \textbf{13} wan] Was R  $\cdot$ swaz] waz L (M) (Q) (R) Z \textbf{15} sol] solt Q  $\cdot$ sinnes wort] sins wortes I \textbf{16} es enwellen] e enwellen G ezn wellen I (L) E zn wolle Q E sin woͤllen Z \textbf{18} wæne] wente M  $\cdot$ baz] das M \textbf{19} einem] einen L R Fr39 eynē M (Q) \textbf{20} würde] were I werde L Fr39 \textbf{21} Ze frevdere erz vz fvr warheit braitet O  $\cdot$ Daz were ein guͯt geleite L (Fr39)  $\cdot$ derz ûz] Der M Der ez vf Z  $\cdot$ wârheit] die warheit I \textbf{22} bereitet] bereite L Fr39 geleitet M R \textbf{23} wünschen] wnsche O (R)  $\cdot$ stêt] gestet O \textbf{24} triwe ze arbeit] ruwe zcu arbeiten M trewe zvr arbeit Z  $\cdot$ ergêt] get R \textbf{25} swem] Wem L Q R So wenic M  $\cdot$ ze solhen werken] zesolhem werche I zuͯ solchen dingen L (Fr39) \textbf{26} dâ] Do Q (Fr39)  $\cdot$ nâch] hernach R \textbf{27} pfliht] phlihte I Phliget L (R) (Fr39)  $\cdot$ lîp] wil R \textbf{28} daz] der I  $\cdot$ in lêren] im raten Q (R)  $\cdot$ karger] k:::ger I chrancher O (L) (M) (Q) (R) (Z) (Fr39) \textbf{29} mîdet ez] meidet sich Q mindrecz R  $\cdot$ ê kan] chan O e kam Q \textbf{30} site] sitten R  $\cdot$ er] ez Q  $\cdot$ ze vogte] zeuuͤge I  $\cdot$ nemen] namen Q \newline
\end{minipage}
\hspace{0.5cm}
\begin{minipage}[t]{0.5\linewidth}
\small
\begin{center}*T
\end{center}
\begin{tabular}{rl}
 & \begin{large}D\end{large}er nie gewarp nâch schanden,\\ 
 & Eine wîle ze sînen handen\\ 
 & sul nû \textbf{dis} âventiure hân\\ 
 & der werde erkante Gawan.\\ 
5 & \textbf{Daz prüevet} \textbf{maneger} âne haz\\ 
 & dâr neben oder \textbf{vürbaz}\\ 
 & danne \textbf{daz mære} hêrn Parcifal.\\ 
 & Swer \textbf{sîne} vriunt alle mâl\\ 
 & mit worten an daz hœheste jaget,\\ 
10 & der ist prîses anderhalp verzaget.\\ 
 & \textbf{im ist} der liute volge guot,\\ 
 & \textbf{der} dicke lop mit wârheit tuot,\\ 
 & \textbf{oder} swaz er sprichet oder sprach,\\ 
 & diu rede blîbet âne dach.\\ 
15 & wer sol sinnes wort behalten,\\ 
 & es enwellen die wîsen walten?\\ 
 & valsch, lügelîch ein mære,\\ 
 & daz, wænich, \textbf{\textit{b}az noch} wære\\ 
 & âne wirt ûf \textbf{einen} snê,\\ 
20 & sô daz dem munde würde wê,\\ 
 & derz ûz vür wârheit breitet.\\ 
 & sô hetin got bereitet,\\ 
 & als guoter liute wünschen stêt,\\ 
 & den ir triuwe ze arbeit ergêt.\\ 
25 & Swem ist ze solhen werken gâch,\\ 
 & dâ missewende hœret nâch,\\ 
 & pflihtet werder lîp an den gewin,\\ 
 & daz muoz in lêren \textbf{kranker} sin.\\ 
 & er mîdet ez, kan er sich schemen;\\ 
30 & den site sol er ze vogete nemen.\\ 
\end{tabular}
\scriptsize
\line(1,0){75} \newline
T V W \newline
\line(1,0){75} \newline
\textbf{1} \textit{Überschrift:} Hie vert her Gawan in daz lant zvͦ ascalvn Do er kemphen solte mit kyngrimursel V   $\cdot$ \textit{Großinitiale} T V   $\cdot$ \textit{Initiale} W  \textbf{2} \textit{Majuskel} T  \textbf{5} \textit{Majuskel} T  \textbf{8} \textit{Majuskel} T  \textbf{25} \textit{Majuskel} T  \newline
\line(1,0){75} \newline
\textbf{3} nû] ich W \textbf{5} Daz] [D*]: Die V Der W  $\cdot$ maneger] manigen V (W) \textbf{6} dâr] der T (W) [D*r]: Dar  V  $\cdot$ vürbaz] fúr in bas V fúr in was W \textbf{7} daz mære] dez [*]: merez V das merern W  $\cdot$ hêrn] herre W  $\cdot$ Parcifal] parzifal T [parzif*]: parzifal V partzifal W \textbf{8} Swer] Wer W  $\cdot$ sîne] sinen V (W) \textbf{10} verzaget] vertaget W \textbf{11} im] In W \textbf{12} der] Swer V Wer W \textbf{13} swaz] was W \textbf{16} es] Fs W \textbf{18} Daz wene [*ere]: ich noch baz were V  $\cdot$ daz wænich] Wenne ich W  $\cdot$ baz] daz T \textbf{19} einen] [einen*]: einem V \textbf{20} sô] \textit{om.} W \textbf{21} derz ûz] Der vns W \textbf{22} hetin] hat in W \textbf{24} arbeit ergêt] arbaiten get W \textbf{25} Swem] Wem W \textbf{26} dâ] Do V W \textbf{27} pflihtet] Gepflichtet W \textbf{29} ez] es ee W \textbf{30} Den sitten kan er zuͦ suͦge nemen W \newline
\end{minipage}
\end{table}
\end{document}
