\documentclass[8pt,a4paper,notitlepage]{article}
\usepackage{fullpage}
\usepackage{ulem}
\usepackage{xltxtra}
\usepackage{datetime}
\renewcommand{\dateseparator}{.}
\dmyyyydate
\usepackage{fancyhdr}
\usepackage{ifthen}
\pagestyle{fancy}
\fancyhf{}
\renewcommand{\headrulewidth}{0pt}
\fancyfoot[L]{\ifthenelse{\value{page}=1}{\today, \currenttime{} Uhr}{}}
\begin{document}
\begin{table}[ht]
\begin{minipage}[t]{0.5\linewidth}
\small
\begin{center}*D
\end{center}
\begin{tabular}{rl}
\textbf{692} & mit dem künege in den rinc geriten,\\ 
 & al dâ der kampf was \textbf{erliten}.\\ 
 & diu sach Gawanen kreftelôs,\\ 
 & den si \textbf{vür aldie werlt} erkôs\\ 
5 & z\textbf{ir} \textbf{besten} vreude krône.\\ 
 & \textbf{nâch} herzen jâmers dône\\ 
 & si \textbf{schrîende} von dem pferde spranc,\\ 
 & mit armen si in vaste umbeswanc.\\ 
 & Si sprach: "vervluochet sî diu hant,\\ 
10 & diu disen kumber hât erkant\\ 
 & gemachet an iwerem lîbe clâr,\\ 
 & \textbf{bî} allen mannen, daz ist wâr,\\ 
 & iwer varwe ein manlîch spiegel was."\\ 
 & si sazten nider \textbf{an}z gras;\\ 
15 & ir \textbf{weinens} wênec wart verdagt.\\ 
 & dô streich im diu süeze magt\\ 
 & \textbf{ab}en ougen bluot und sweiz;\\ 
 & in harnasche was im heiz.\\ 
 & Der künec Gramoflanz \textbf{dô} sprach:\\ 
20 & "Gawan, mir ist leit dîn ungemach,\\ 
 & ez \textbf{en}wære \textbf{von} mîner hant getân.\\ 
 & wiltû morgen \textbf{wider} ûf den plân\\ 
 & \textbf{gein mir} komen durch strîten,\\ 
 & \textbf{des wil ich} gerne bîten.\\ 
25 & ich bestüende gerner nû ein wîp\\ 
 & dan dînen kreftelôsen lîp.\\ 
 & waz prîses m\textit{ö}ht ich an dir bejagen,\\ 
 & i\textbf{ne} hôrte dich baz \textbf{gein} kreften sagen?\\ 
 & nû ruowe hînte, des \textbf{wirt} dir nôt,\\ 
30 & wiltû \textbf{vürstên} den künec Lot."\\ 
\end{tabular}
\scriptsize
\line(1,0){75} \newline
D \newline
\line(1,0){75} \newline
\textbf{9} \textit{Majuskel} D  \textbf{19} \textit{Majuskel} D  \newline
\line(1,0){75} \newline
\textbf{3} Gawanen] Gawann D \textbf{27} möht] moht D \newline
\end{minipage}
\hspace{0.5cm}
\begin{minipage}[t]{0.5\linewidth}
\small
\begin{center}*m
\end{center}
\begin{tabular}{rl}
 & mit dem künige in den rinc geriten,\\ 
 & aldâ der kampf was \textbf{gestriten}.\\ 
 & diu sach Gawanen kreftelôs,\\ 
 & den si \textbf{vür alle die werlt} erkôs\\ 
5 & zuo \textbf{ir} \textbf{höfschen} vröuden krône.\\ 
 & \textbf{nâch} herzen jâmers dône\\ 
 & si \textbf{schrîende} von dem pferde sp\textit{r}anc,\\ 
 & mit armen si in vaste umbeswanc.\\ 
 & si sprach: "vervluochet sî diu hant,\\ 
10 & diu disen kumber het erkant\\ 
 & gemachet an iuwerm lîbe clâr,\\ 
 & \textbf{bî} allen mannen, daz ist wâr,\\ 
 & iuwer varwe ein manlîch spiegel was."\\ 
 & si sazte in nider \textbf{ûf} daz gras;\\ 
15 & ir \textbf{weinen} wênic wart verdaget.\\ 
 & dô streich im diu süeze maget\\ 
 & \textbf{ab} den ougen bluot und sweiz;\\ 
 & in \textit{h}arnasch was im heiz.\\ 
 & der künic Gramolanz \textbf{dô} sprach:\\ 
20 & "Gawan, mir ist leit dîn ungemach,\\ 
 & ez wær \textbf{von} mîner hant getân.\\ 
 & wiltû morgen ûf den plân\\ 
 & \textbf{geg\textit{e}n mi\textit{r}} \textit{k}omen durch strîten,\\ 
 & \textbf{des wil ich} gerne bîten.\\ 
25 & ich bestüende gerner n\textit{û ein} wîp\\ 
 & dan dînen kreftelôsen lîp.\\ 
 & waz prîses m\textit{ö}ht ich an dir bejagen,\\ 
 & ich hôrte dich baz \textbf{gegen} kreften sagen?\\ 
 & nû ruowe hîn\textit{t}, d\textit{e}s \textbf{ist} dir nôt,\\ 
30 & wiltû \textbf{verstên} den künic Lot."\\ 
\end{tabular}
\scriptsize
\line(1,0){75} \newline
m n o \newline
\line(1,0){75} \newline
\newline
\line(1,0){75} \newline
\textbf{3} Gawanen] gawanes o \textbf{7} spranc] spang m \textbf{8} umbeswanc] ẏm swang o \textbf{9} vervluochet] verslúchet o \textbf{11} gemachet] Gemach o \textbf{13} manlîch] einmanlich n \textbf{16} dô] Doch o \textbf{17} den] der o \textbf{18} in] Jn dem n  $\cdot$ harnasch] narnasch m \textbf{19} Gramolanz] gramolantz m o gramolancz o \textbf{21} ez] Er o \textbf{22} morgen] morgen wider n o \textbf{23} gegen mir komen] Gegein mir stritten komen m \textbf{24} bîten] bitten m \textbf{25} gerner] gerne o  $\cdot$ nû ein] nẏe rose vnd m ẏm eyn o \textbf{26} lîp] [strit]: lip o \textbf{27} möht] moht m (o)  $\cdot$ dir] mich o \textbf{29} hînt des] hin das m  $\cdot$ ist] wurt n (o) \textbf{30} den künic] die konigin o \newline
\end{minipage}
\end{table}
\newpage
\begin{table}[ht]
\begin{minipage}[t]{0.5\linewidth}
\small
\begin{center}*G
\end{center}
\begin{tabular}{rl}
 & \begin{large}M\end{large}it dem künige in den rinc geriten,\\ 
 & al dâ der kampf was \textbf{erliten}.\\ 
 & diu sach Gawanen kreftelôs,\\ 
 & den si \textbf{vor alder werlde} erkôs\\ 
5 & z\textbf{ir} \textbf{hœhesten} vröuden krône.\\ 
 & \textbf{mit} herzen jâmers dône\\ 
 & si \textbf{schrîende} von dem pferde spranc,\\ 
 & mit armen sin vaste umbeswanc.\\ 
 & si sprach: "vervluochet sî diu hant,\\ 
10 & diu disen kumber hât erkant\\ 
 & gemachet an iurem lîbe clâr,\\ 
 & \textbf{vor} allen mannen, daz ist wâr,\\ 
 & iwer varwe ein manlîch spiegel was."\\ 
 & si sazt in nider \textbf{ûffe}z gras;\\ 
15 & ir \textbf{weinens} wênic wart verdaget.\\ 
 & dô streich im diu süeze maget\\ 
 & \textbf{von} den ougen bluot unde sweiz;\\ 
 & in \textbf{dem} harnasche was im heiz.\\ 
 & der künic Gramoflanz \textbf{dô} sprach:\\ 
20 & "Gawan, mirst leit dîn ungemach,\\ 
 & ez\textbf{ne} wære \textbf{mit} mîner hant getân.\\ 
 & wil dû morgen \textbf{gein mir} ûf den plân\\ 
 & \textbf{her wider} komen durch strîten,\\ 
 & \textbf{ich wil dîn} gerne bîten.\\ 
25 & ich bestüende gerner nû ein wîp\\ 
 & danne dînen kreftelôsen lîp.\\ 
 & waz brîses m\textit{ö}hte ich an dir bejagen,\\ 
 & ich\textbf{ne} hôrte dich baz \textbf{bî} kreften sagen?\\ 
 & nû ruowe hînt, des \textbf{wirt} dir nôt,\\ 
30 & wil dû \textbf{rechen} den künic Lot."\\ 
\end{tabular}
\scriptsize
\line(1,0){75} \newline
G I L M Z Fr20 \newline
\line(1,0){75} \newline
\textbf{1} \textit{Initiale} G Z Fr20  \textbf{3} \textit{Initiale} L  \textbf{9} \textit{Initiale} I  \newline
\line(1,0){75} \newline
\textbf{1} Mit] ÷it Fr20  $\cdot$ dem künige] einer kungin I \textbf{3} Gawanen] Gawan I \textbf{4} vor alder] vor aller M fvr al die Z  $\cdot$ erkôs] chos I \textbf{5} zir] zi der Fr20  $\cdot$ hœhesten vröuden krône] freuden hohisten crone I hochsten vroide crone M \textbf{7} friende von dem pheride si spranc I \textbf{15} weinens] weinen da L weynes M \textbf{16} dô] Da M Z \textbf{18} dem] \textit{om.} Z  $\cdot$ heiz] so [hei*]: heisz L \textbf{19} Gramoflanz] Gramorflanz M gramoflantz Z :::moflanz Fr20  $\cdot$ dô] da M \textbf{21} mit] von I dan mit Z  $\cdot$ getân] [gel]: getan G \textbf{22} morgen gein mir] gein mir morgen L \textbf{23} durch] \textit{om.} L \textbf{24} gerne] gernir Fr20 \textbf{27} möhte] mohte G (I) (L) (M) (Z) (Fr20)  $\cdot$ an] niht an Fr20 \textbf{28} ichne] Jch Z  $\cdot$ bî] gein L (M) Z \textbf{30} rechen] entschuͯlden L versten Z  $\cdot$ Lot] loth Fr20 \newline
\end{minipage}
\hspace{0.5cm}
\begin{minipage}[t]{0.5\linewidth}
\small
\begin{center}*T
\end{center}
\begin{tabular}{rl}
 & mit dem künege in den rinc geriten,\\ 
 & al dâ der kampf was \textbf{erliten}.\\ 
 & diu sach Gawanen kreftelôs,\\ 
 & den si \textbf{vor aller der werlde} erkôs\\ 
5 & zuo \textbf{der} \textbf{hœhesten} vreuden krône.\\ 
 & \textbf{mit} herzen jâmers dône\\ 
 & si \textbf{schrei}, von dem pferde \textbf{si} spranc,\\ 
 & mit armen si in vaste umbeswanc.\\ 
 & si sprach: "vervluochet sî diu hant,\\ 
10 & diu disen kumber hât erkant\\ 
 & gemachet an iuwerme lîbe clâr,\\ 
 & \textbf{bî} allen mannen, daz ist wâr,\\ 
 & iuwer varwe ein manlîch spiegel was."\\ 
 & si sazt in nider \textbf{ûf} daz gras;\\ 
15 & ir \textbf{weinen} wênic wart verdaget.\\ 
 & dô streich im diu süeze maget\\ 
 & \textbf{von} den ougen bluot und sweiz;\\ 
 & i\textit{n} \textbf{dem} harnasch was im heiz.\\ 
 & \begin{large}D\end{large}er künec Gramoflanz sprach:\\ 
20 & "Gawan, mir ist leit dîn ungemach,\\ 
 & ez \textbf{en}wære \textbf{mit} mîner hant getân.\\ 
 & wiltû morne \textbf{gein mir} ûf \textit{den pl}ân\\ 
 & \textbf{her wider} komen durch strîten,\\ 
 & \textbf{ich wil dîn} gerne bîten.\\ 
25 & ich bestüende gerne\textit{r} nû ein wîp\\ 
 & dan dînen kreftelôsen lîp.\\ 
 & waz prîses m\textit{ö}ht ich an dir bejagen,\\ 
 & ich \textbf{en}hôrte dich baz \textbf{\textit{gei}n} kreften sagen?\\ 
 & nû ruowe hînt, des \textbf{ist} dir nôt,\\ 
30 & wilt dû \textbf{rechen} den künec Lot."\\ 
\end{tabular}
\scriptsize
\line(1,0){75} \newline
U V W Q R \newline
\line(1,0){75} \newline
\textbf{19} \textit{Initiale} U V W  \newline
\line(1,0){75} \newline
\textbf{1} rinc] rit Q \textbf{3} Gawanen] herr gawan W Gawin R \textbf{4} aller der] aller W der Q \textbf{5} der] ir W (Q)  $\cdot$ hœhesten] hochten Q  $\cdot$ vreuden] vroͤide V \textbf{6} mit] [*]: Mit V Nach Q \textbf{7} schrei] schriende V (W) Q schrienden R  $\cdot$ si spranc] sprang V (W) (Q) R \textbf{11} an] vnd an W oͯn R  $\cdot$ iuwerme] úwern R \textbf{13} varwe ein manlîch] frawe ein menlich Q manlich fawe ein R \textbf{14} sazt in] sassen W \textbf{15} weinen] weinens V W Q (R) \textbf{16} streich] streiche W \textbf{18} in] Jm U \textbf{19} Gramoflanz] gramaflanz V gramoflantz W (Q) gramoflancz R  $\cdot$ sprach] do sprach V (W) Q (R) \textbf{20} Gawan] Gawin R \textbf{21} ez enwære] Czu were Q  $\cdot$ mit] [*]: von V \textbf{22} gein] [mir]: mit R  $\cdot$ den plân] of gan U \textbf{23} durch] vf V \textbf{24} Jch wil dir gern bitten R \textbf{25} gerner] gerne U [gern*]: gerner V \textbf{27} möht] [moht]: moͤht V  $\cdot$ dir bejagen] die began R \textbf{28} enhôrte] horte W Q  $\cdot$ gein kreften] von creften U gekrefften Q \textbf{29} \textit{Versdoppelung nach 692.30:} Das tuͯt dir not R   $\cdot$ hînt] \textit{om.} R  $\cdot$ des ist] das ist W daz tuͯt R  $\cdot$ nôt] [gach]: not Q \textbf{30} rechen] [*]: versten V \newline
\end{minipage}
\end{table}
\end{document}
