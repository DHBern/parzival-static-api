\documentclass[8pt,a4paper,notitlepage]{article}
\usepackage{fullpage}
\usepackage{ulem}
\usepackage{xltxtra}
\usepackage{datetime}
\renewcommand{\dateseparator}{.}
\dmyyyydate
\usepackage{fancyhdr}
\usepackage{ifthen}
\pagestyle{fancy}
\fancyhf{}
\renewcommand{\headrulewidth}{0pt}
\fancyfoot[L]{\ifthenelse{\value{page}=1}{\today, \currenttime{} Uhr}{}}
\begin{document}
\begin{table}[ht]
\begin{minipage}[t]{0.5\linewidth}
\small
\begin{center}*D
\end{center}
\begin{tabular}{rl}
\textbf{678} & sô geheilet wæren,\\ 
 & daz die mâsen in niht swæren.\\ 
 & \begin{large}E\end{large}r wolde baneken den lîp,\\ 
 & sît sô manec \textbf{man} unt wîp\\ 
5 & sînen kampf \textbf{solden} sehen,\\ 
 & \textbf{dâ} die wîsen rîter mohten spehen,\\ 
 & ob sîn unverzagtiu hant\\ 
 & des tages gein prîse würde erkant.\\ 
 & Einen knappen het er des gebeten,\\ 
10 & daz er im br\textit{æ}hte Gringuljeten.\\ 
 & \textbf{daz} begunder leischieren.\\ 
 & er wolde sich môvieren,\\ 
 & daz er untz ors wæren bereit.\\ 
 & mir \textbf{wart} sîn reise nie sô leit:\\ 
15 & \textbf{Als} eine reit mîn hêr Gawan\\ 
 & von dem her verre ûf \textbf{den} plân.\\ 
 & gelücke müezes walden!\\ 
 & \textbf{er sach einen rîter} halden\\ 
 & bî dem wazzer Sabbins,\\ 
20 & den wir \textbf{wol} m\textit{ö}hten heizen vlins\\ 
 & der manlîchen krefte.\\ 
 & \textbf{er}, schûr \textbf{der} rîterschefte,\\ 
 & sîn herze valsch nie \textbf{underswanc};\\ 
 & er was des lîbes wol sô kranc,\\ 
25 & swaz man heizet unprîs,\\ 
 & daz \textbf{en}\textbf{truog} er \textbf{nie} \textbf{decheinen gewîs}\\ 
 & halbes vingers lanc \textbf{noch} spanne.\\ 
 & von dem selbem werden manne\\ 
 & muget ir \textbf{wol ê} hân vernomen:\\ 
30 & an den rehten stam \textbf{diz mære ist} komen.\\ 
\end{tabular}
\scriptsize
\line(1,0){75} \newline
D Fr10 \newline
\line(1,0){75} \newline
\textbf{3} \textit{Initiale} D  \textbf{9} \textit{Majuskel} D  \textbf{15} \textit{Majuskel} D  \newline
\line(1,0){75} \newline
\textbf{8} gein prîse würde] wurder genn breis Fr10 \textbf{10} bræhte] brahte D  $\cdot$ Gringuljeten] Gringvlietn D Gringuletn Fr10 \textbf{15} Als] Al Fr10  $\cdot$ mîn] \textit{om.} Fr10 \textbf{17} müezes] muͦz ez Fr10 \textbf{19} Sabbins] Sabins D \textbf{20} möhten] mohten D (Fr10) \textbf{26} decheinen gewîs] chain weis Fr10 \textbf{27} halbes vingers] Halbz vinger Fr10  $\cdot$ spanne] spannen Fr10 \textbf{28} selbem] selbn Fr10 \newline
\end{minipage}
\hspace{0.5cm}
\begin{minipage}[t]{0.5\linewidth}
\small
\begin{center}*m
\end{center}
\begin{tabular}{rl}
 & sô geheilet wæren,\\ 
 & daz die mâsen i\textit{n} niht swæren.\\ 
 & er wolte baneken den lîp,\\ 
 & sît sô manic \textit{\textbf{man}} un\textit{d w}îp\\ 
5 & sînen kampf \textbf{d\textit{â}} \textbf{solten} sehen,\\ 
 & \textbf{daz} die wîsen ritter mohten spehen,\\ 
 & ob sîn unverzagtiu hant\\ 
 & des tages gegen prîs würde erkant.\\ 
 & einen knappen het er des gebeten,\\ 
10 & daz er im bræhte Gringuleten.\\ 
 & \textbf{den} begund\textit{e} er l\textit{ei}schieren.\\ 
 & er wolte sich môvieren,\\ 
 & daz er und daz ros wæren bereit.\\ 
 & mir \textbf{wart} sîn reise nie sô leit:\\ 
15 & \textbf{\begin{large}A\end{large}l} ein reit mîn hêr Gawan\\ 
 & von dem her verre ûf \textbf{dem} plân.\\ 
 & glücke müeze e\textit{s} walten!\\ 
 & \textbf{einen ritter sach er} halten\\ 
 & bî dem wazzer Sabbins,\\ 
20 & den wir \textbf{wol} möhten heizen vlins\\ 
 & der manlîchen krefte,\\ 
 & \textbf{ein} schûr \textbf{der} ritterschefte.\\ 
 & sîn herze valsch nie \textbf{underswanc};\\ 
 & er was des lîbes wol sô kranc,\\ 
25 & waz man heizet unprîs,\\ 
 & daz \textbf{truoc} er \textbf{niht} \textbf{dekein wîs}\\ 
 & halbes vinger\textit{s} lanc \textbf{und} spanne.\\ 
 & von dem selben werden manne\\ 
 & moget ir \textbf{wol ê} hân vernomen:\\ 
30 & an den rehten stam \textbf{diz mær ist} \textit{kom}en.\\ 
\end{tabular}
\scriptsize
\line(1,0){75} \newline
m n o Fr69 \newline
\line(1,0){75} \newline
\textbf{15} \textit{Illustration mit Überschrift:} Also parcifal vnd gawan mit einander fohtten m (o)  Also [pr]: parcifal vnd gawan mit ein ander stritten vnd foͯchten n   $\cdot$ \textit{Initiale} m n o  \newline
\line(1,0){75} \newline
\textbf{2} mâsen] mo:sen o  $\cdot$ in] ẏm m  $\cdot$ swæren] swerent o \textbf{4} man und wîp] wip vnd manlich lip m \textbf{5} dâ] do m n o  $\cdot$ solten] solde Fr69 \textbf{6} mohten] moͯchten n \textbf{10} Gringuleten] gringuletten m n \textbf{11} begunde] begunden m  $\cdot$ er] \textit{om.} n  $\cdot$ leischieren] laschieren m n o Fr69 \textbf{12} môvieren] moueren o ::: Fr69 \textbf{13} wæren] worent o \textbf{15} Al ein] A::ein o  $\cdot$ hêr] herr her n \textbf{16} von] V:n o  $\cdot$ dem plân] dē plan n den plan o \textbf{17} es] er m n o \textbf{20} möhten] mochten o \textbf{26} dekein] do keine n \textbf{27} vingers] finger m (n) o  $\cdot$ und] noch n [s]: nach o \textbf{30} diz] dise n  $\cdot$ komen] funden m \newline
\end{minipage}
\end{table}
\newpage
\begin{table}[ht]
\begin{minipage}[t]{0.5\linewidth}
\small
\begin{center}*G
\end{center}
\begin{tabular}{rl}
 & sô geheilt wæren,\\ 
 & daz die mâsen in niht swæren.\\ 
 & \begin{large}E\end{large}r wolde baneken den lîp,\\ 
 & sît sô manic \textbf{maget} unde wîp\\ 
5 & sînen kampf \textbf{solde\textit{n}} sehen,\\ 
 & \textbf{daz} die wîsen rîter mohten spehen,\\ 
 & op sîn unverzagetiu hant\\ 
 & des tages gein brîse würde erkant.\\ 
 & einen knappen het er des gebeten,\\ 
10 & daz er im bræhte Gringulieten.\\ 
 & \textbf{dô} begunde er leisieren.\\ 
 & er wolde \textit{sich} môvieren,\\ 
 & daz er unde daz ors w\textit{æ}ren bereit.\\ 
 & mir\textbf{ne} \textbf{was} sîn reise nie sô leit:\\ 
15 & \textbf{al} eine reit mîn hêr Gawan\\ 
 & von dem her verre ûf \textbf{den} plân.\\ 
 & gelücke muozes walden!\\ 
 & \textbf{er sach einen rîter} halden\\ 
 & bî dem wazzer Sabins,\\ 
20 & den wir m\textit{ö}hten heizen vlins\\ 
 & der manlîchen krefte.\\ 
 & \textbf{er}, schûr \textbf{an} rîterschefte,\\ 
 & sîn herze valsch nie \textbf{verswanc};\\ 
 & er was des lîbes wol sô kranc,\\ 
25 & swaz man heizet unbrîs,\\ 
 & daz \textbf{getruoc} er \textbf{nie} \textbf{neheine wîs}\\ 
 & halbes vingers lanc \textbf{noch} spanne.\\ 
 & von dem selben werden manne\\ 
 & muget ir \textbf{ê wol} hân vernomen:\\ 
30 & an den rehten stam \textbf{ist ditze mære} komen.\\ 
\end{tabular}
\scriptsize
\line(1,0){75} \newline
G I L M Z Fr18 Fr22 Fr24 Fr52 Fr61 \newline
\line(1,0){75} \newline
\textbf{1} \textit{Initiale} L Z  \textbf{3} \textit{Initiale} G I Fr24  \textbf{15} \textit{Initiale} Fr61  \textbf{23} \textit{Initiale} I  \newline
\line(1,0){75} \newline
\textbf{1} wæren] waren L \textbf{2} Daz in die masen swaren L  $\cdot$ in niht] sein icht Fr61 \textbf{5} solden] solde G \textbf{6} wîsen] werden wisen I \textit{om.} Fr61  $\cdot$ spehen] yehen L (Fr61) \textbf{8} würde] werd Fr61 \textbf{9} einen] Sinen L (Fr61) \textbf{10} Gringulieten] kringvlieten G Fr18 [Gringuliete*]: Gringulieten I Gringvleten L kringulieten Fr24 Gringoleten Fr61 \textbf{11} dô] Da M Daz Z  $\cdot$ er] \textit{om.} L \textbf{12} sich] \textit{om.} G  $\cdot$ môvieren] norviern Fr61 \textbf{13} unde daz] vnd sin Fr24  $\cdot$ wæren] waren G L \textbf{14} mirne was] mirn en wart I Mir enwart L (M) (Z) (Fr18) (Fr24) mir ward Fr61 \textbf{15} mîn] \textit{om.} L Fr52 Fr61 \textbf{16} verre] verte M \textit{om.} Fr61 \textbf{17} muozes] muͤs sin I (Z) (Fr52) \textbf{18} sach] saz Fr61  $\cdot$ einen rîter] an einer L Fr61 \textbf{19} Sabins] Sabyns Fr24 sab::: Fr52 \textbf{20} Der wol mochte heuszen flinz L (Fr61)  $\cdot$ wir möhten heizen] wir mohten heizen G wir wol mohten haizen I (Z) (Fr24) (Fr52) wir wol heiszin mochten M  $\cdot$ vlins] slins I \textbf{22} er] ein I (L) (Fr52) \textbf{23} sîn] Myn M  $\cdot$ valsch nie] allen valsch Fr52 preis nie Fr61  $\cdot$ verswanc] vnder swanch L (Z) vorswant M \textbf{24} sô] also Fr61 \textbf{25} swaz] Waz L (M) \textbf{26} des truc er in keinen wis Fr52  $\cdot$ neheine] keinen Z (Fr18) (Fr22) (Fr24) Fr52 \textbf{27} halbes] \textit{om.} Fr52 \textbf{28} selben] \textit{om.} Fr61  $\cdot$ werden] werdem I (Fr18) o\textit{m. } Fr52 \textbf{29} ê wol hân] wol han M Fr52 wol han e Z wol ê han Fr18 Fr24 \textbf{30} an] Mit Fr24  $\cdot$ ist ditze mære] ditz mer ist I (L) (M) (Z) (Fr22) (Fr24) daz mær ist Fr18 Fr61 \newline
\end{minipage}
\hspace{0.5cm}
\begin{minipage}[t]{0.5\linewidth}
\small
\begin{center}*T
\end{center}
\begin{tabular}{rl}
 & sô geheilt wæren,\\ 
 & daz die mâsen in niht swæren.\\ 
 & er wolt ba\textit{ne}k\textit{e}n den lîp,\\ 
 & sît sô manic \textbf{man} und wîp\\ 
5 & sînen k\textit{a}mp\textit{f} \textbf{solde} s\textit{eh}en,\\ 
 & \textbf{d\textit{â}} die wîsen rîter mohten spehen,\\ 
 & ob sîn unverzagtiu hant\\ 
 & des tages gên prîse würde erkant.\\ 
 & einen knaben het er des gebeten,\\ 
10 & daz er im bræhte Kryngulieten.\\ 
 & \textbf{daz} begund er leisieren.\\ 
 & er wolt sich môvieren,\\ 
 & daz er und daz ros wæren bereit.\\ 
 & mir \textbf{wart} sîn reise nie sô leit:\\ 
15 & \textbf{al}ein reit mîn hêr Gawan\\ 
 & von dem her verre ûf \textbf{den} plân.\\ 
 & gelück\textit{e} müez es walten!\\ 
 & \textbf{er sach einen ritter} halten\\ 
 & bî dem wazzer Sabins,\\ 
20 & den wir \textbf{wol} möhten heizen vlins\\ 
 & der menlîchen kreft\textit{e}.\\ 
 & \textbf{er}, schûr \textbf{der} ritterschefte,\\ 
 & sîn herze valsch nie \textbf{underswanc};\\ 
 & er was des lîbes wol sô kranc,\\ 
25 & waz man heizet unprîs,\\ 
 & daz \textbf{getruoc} er \textbf{nie} \textbf{keinen wîs}\\ 
 & halbes vingers lanc \textbf{noch} spanne.\\ 
 & von dem selben werden manne\\ 
 & mugt ir \textbf{wol ê} hân vernomen:\\ 
30 & an den rehten stam \textit{\textbf{ist diz mære}} \textit{komen}.\\ 
\end{tabular}
\scriptsize
\line(1,0){75} \newline
Q R W V \newline
\line(1,0){75} \newline
\textbf{9} \textit{Initiale} W V   $\cdot$ \textit{Überschrift:} Hie vihtet Parzifal mit Gawane do gawan mit gramaflanzen solte gekempfet han V  \newline
\line(1,0){75} \newline
\textbf{1} sô] Da R \textbf{2} in] Im R (W) [im]: in  V \textbf{3} baneken] bachern Q bewern R  $\cdot$ den] sinen R V \textbf{5} sînen] Seinē Q Sinem R  $\cdot$ kampf] kawmpt Q  $\cdot$ sehen] schawen Q [*]: sehen V \textbf{6} dâ] Do Q Das R W (V)  $\cdot$ die wîsen] dise W  $\cdot$ mohten] moͤchten W (V) \textbf{7} unverzagtiu] vnuerczagte R \textbf{8} würde] wer W \textbf{9} einen] SEinen W (V) \textbf{10} im] in R  $\cdot$ Kryngulieten] kringulieten Q [kinguletten]: kringuletten R kringulietten W kringuleten V \textbf{11} leisieren] [*ieren]: lesieren V \textbf{13} daz ros] des ros W \textbf{14} wart] enwart V  $\cdot$ sô] sol R \textbf{15} alein] [A*eine]: Alleine V \textbf{17} gelücke] Geluckes Q  $\cdot$ müez] muͦst W  $\cdot$ es] \textit{om.} R \textbf{19} Sabins] Roitschesabins V \textbf{20} möhten] [mohten]: moͤhten V \textbf{21} menlîchen] manliche R  $\cdot$ krefte] krefften Q \textbf{22} er] Ein V \textbf{26} keinen] deheine R (W) V \textbf{27} halbes] Halbe W \textbf{29} wol ê] ee wol W e V \textbf{30} stam] stamen R  $\cdot$ ist diz mære komen] \textit{om.} Q diß mer ist komen W [*]: diz mer ist komen V \newline
\end{minipage}
\end{table}
\end{document}
