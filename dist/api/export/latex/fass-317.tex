\documentclass[8pt,a4paper,notitlepage]{article}
\usepackage{fullpage}
\usepackage{ulem}
\usepackage{xltxtra}
\usepackage{datetime}
\renewcommand{\dateseparator}{.}
\dmyyyydate
\usepackage{fancyhdr}
\usepackage{ifthen}
\pagestyle{fancy}
\fancyhf{}
\renewcommand{\headrulewidth}{0pt}
\fancyfoot[L]{\ifthenelse{\value{page}=1}{\today, \currenttime{} Uhr}{}}
\begin{document}
\begin{table}[ht]
\begin{minipage}[t]{0.5\linewidth}
\small
\begin{center}*D
\end{center}
\begin{tabular}{rl}
\textbf{317} & \begin{large}D\end{large}iu stat hât erden wunsches solt,\\ 
 & hie het iu \textbf{vrâgen} mêr erholt.\\ 
 & \textbf{jenes} landes künegîn\\ 
 & Feirefiz Anschevin\\ 
5 & mit herter riterschefte erwarp,\\ 
 & an dem diu manheit niht verdarp,\\ 
 & die iwer bêder vater truoc.\\ 
 & iwer bruoder wunders pfligt genuoc;\\ 
 & \textbf{jâ} ist beidiu swarz und blanc\\ 
10 & der küneginne sun von Zazamanc.\\ 
 & Nû \textbf{denke} \textit{ich} aber an Gahmureten,\\ 
 & des herze \textbf{ie} valsches was erjeten.\\ 
 & von Anschouwe iwer vater hiez,\\ 
 & der iu ander erbe liez,\\ 
15 & denn als ir habt geworben.\\ 
 & \textbf{an} prîse ir sît verdorben.\\ 
 & het iwer muoter ie missetân,\\ 
 & sô \textbf{solt} ichz dâ vür \textbf{gerne} hân,\\ 
 & ir mäht \textbf{ir} sun niht gesîn.\\ 
20 & nein, \textbf{si} lêrte ir triwe pîn.\\ 
 & geloubet von ir \textbf{guoter} mære\\ 
 & unt daz iwer vater wære\\ 
 & manlîcher triwe wîse\\ 
 & unt \textbf{wîtvengec} \textbf{hôher} prîse.\\ 
25 & er kunde wol mit \textbf{schallen}.\\ 
 & grôz herze unt \textbf{kleine gallen},\\ 
 & dar \textbf{ob} was sîn brust \textbf{ein} dach.\\ 
 & er was riuse \textbf{unt} \textbf{vengec} vach.\\ 
 & sîn manlîchez ellen\\ 
30 & kunde \textbf{den prîs} wol \textbf{gestellen}.\\ 
\end{tabular}
\scriptsize
\line(1,0){75} \newline
D \newline
\line(1,0){75} \newline
\textbf{1} \textit{Initiale} D  \textbf{11} \textit{Majuskel} D  \newline
\line(1,0){75} \newline
\textbf{4} Anschevin] Anscevin D \textbf{10} Zazamanc] Zazamanch D \textbf{11} ich] \textit{om.} D  $\cdot$ Gahmureten] Gahmvreten D \textbf{13} Anschouwe] Anscoͮwe D \newline
\end{minipage}
\hspace{0.5cm}
\begin{minipage}[t]{0.5\linewidth}
\small
\begin{center}*m
\end{center}
\begin{tabular}{rl}
 & \dag ow\dag  stat hât erden wunsches solt,\\ 
 & hie hete iu \textbf{vrâgen} mêr erholt.\\ 
 & \hspace*{-.7em}\big| Ferefi\textit{z} A\textit{n}schevin\\ 
 & \hspace*{-.7em}\big| \textbf{jenes} landes künigîn\\ 
5 & mit herter ritterschaft erwarp,\\ 
 & an dem diu manheit niht verdarp,\\ 
 & die iuwer beider vater truoc.\\ 
 & iuwer bruoder wunders pfliget genuoc;\\ 
 & \textbf{jâ} ist beidiu swarz und blanc\\ 
10 & der künigîn sun von Zazaman\textit{c}.\\ 
 & nû \textbf{denke} ich aber an Gahmureten,\\ 
 & des herze \textbf{ie} valsches was erjeten.\\ 
 & von Anschouwe iuwer vater hiez,\\ 
 & der iu ander erbe liez,\\ 
15 & danne als ir habt geworben.\\ 
 & \textit{\textbf{in}} \textit{prîse ir sît verdorben.}\\ 
 & hete iuwer muoter ie missetân,\\ 
 & sô \textbf{solt} ichz dar vür \textbf{gerne} hân,\\ 
 & ir \textbf{en}m\textit{ö}htet \textbf{sîn} sun niht gesîn.\\ 
20 & nein, \textbf{sô} lêrte ir triuwe pîn.\\ 
 & gloubet von ir \textbf{guoter} mære\\ 
 & und daz iuwer vater wære\\ 
 & manlîcher triuwen wîse\\ 
 & und \textbf{wîtvengic} \textbf{hôher} prîse.\\ 
25 & er kunde wol mit \textbf{schalle}.\\ 
 & grôz herze und \textbf{kleiniu galle},\\ 
 & dar \textbf{obe} was sîn brust \textbf{sîn} dach.\\ 
 & er was rius\textit{e} \textbf{und} \textbf{vengic} vach.\\ 
 & sîn manlîchez ellen\\ 
30 & kunde \textbf{dem prîse} wol \textbf{gestellen}.\\ 
\end{tabular}
\scriptsize
\line(1,0){75} \newline
m n o \newline
\line(1,0){75} \newline
\newline
\line(1,0){75} \newline
\textbf{1} ow] Vwer n Wer o  $\cdot$ erden wunsches] wuͯnsches erden o \textbf{2} hete] hat n \textbf{4} Fere vir auscevin m Ferre fúr anscevin n Ferre vir ansce win o \textbf{9} jâ] Jo n \textbf{10} sun] sin o  $\cdot$ Zazamanc] zazamant m zazamang n o \textbf{11} denke] gedencke n o  $\cdot$ Gahmureten] gahmuretten m gamureten n gamuͯreten o \textbf{12} des] Das o \textbf{13} Anschouwe] anschowe o \textbf{16} \textit{Vers 317.16 fehlt} m  \textbf{17} missetân] gemissetan o \textbf{18} dar vür gerne] gerne do fúr n (o) \textbf{19} enmöhtet] enmochtent m moͯchtent n mochtent o \textbf{21} guoter] guͦten n \textbf{27} dar] Das n o  $\cdot$ sîn dach] >sin< dach o \textbf{28} riuse] rusig m rurig o  $\cdot$ vengic vach] wennig o \textbf{29} manlîchez] manlichens o \newline
\end{minipage}
\end{table}
\newpage
\begin{table}[ht]
\begin{minipage}[t]{0.5\linewidth}
\small
\begin{center}*G
\end{center}
\begin{tabular}{rl}
 & \multicolumn{1}{l}{ - - - }\\ 
 & \multicolumn{1}{l}{ - - - }\\ 
 & \textbf{eines} landes künigîn\\ 
 & Feirafiz Anschevin\\ 
5 & mit herter rîterschaft erwarp,\\ 
 & an dem diu manheit niht verdarp,\\ 
 & die iuwer beider vater truoc.\\ 
 & iuwer bruoder wunders pfliget genuoc,\\ 
 & \textbf{der}st beidiu swarz unde blanc,\\ 
10 & der künigîn sun von Zazamanc.\\ 
 & nû \textbf{denke} ich aber an Gahmureten,\\ 
 & des herze valsches was erjeten.\\ 
 & von Anschouwe iuwer vater hiez,\\ 
 & der iu ander erbe liez,\\ 
15 & danne als ir habet geworben.\\ 
 & \textbf{an} prîse ir sît verdorben.\\ 
 & het iuwer muoter ie missetân,\\ 
 & sô \textbf{wolt} ich ez dâ vür hân,\\ 
 & ir m\textit{ö}ht \textbf{sîn} sun niht gesîn.\\ 
20 & nein, \textbf{si} lêrte ir triuwe pîn.\\ 
 & \begin{large}G\end{large}eloubt von ir \textbf{guot} mære\\ 
 & unde daz iuwer vater wære\\ 
 & manlîcher triuwen wîse\\ 
 & unde \textbf{wîtvenge} \textbf{hôher} prîse.\\ 
25 & er kunde wol mit \textbf{schalle}.\\ 
 & grôz herze unde \textbf{kleiniu galle},\\ 
 & dar \textbf{über} was sîn brust \textbf{ein} dach.\\ 
 & er was r\textit{i}u\textit{s}e \textbf{unde} \textbf{vengec} vach.\\ 
 & sî\textit{n} \textit{m}anlîch ellen\\ 
30 & kunde \textbf{den prîs} wol \textbf{stellen}.\\ 
\end{tabular}
\scriptsize
\line(1,0){75} \newline
G I O L M Q R Z Fr39 Fr40 \newline
\line(1,0){75} \newline
\textbf{5} \textit{Initiale} O  \textbf{17} \textit{Initiale} I Z Fr39 Fr40   $\cdot$ \textit{Capitulumzeichen} R  \textbf{21} \textit{Initiale} G  \textbf{29} \textit{Initiale} I  \newline
\line(1,0){75} \newline
\textbf{1} \textit{Die Verse 316.7-318.8 fehlen} L   $\cdot$ \textit{Die Verse 317.1-2 fehlen} G I   $\cdot$ Div stat hat (\textit{om.} M ) erden wnsches solt O (M) (Q) (R) (Z) (Fr39) (Fr40) \textbf{2} Hie het iv fragen mer verholt (er holt M [ Q R Z Fr39 ] Fr40 ) O (M) (Q) (R) (Z) (Fr39) (Fr40) \textbf{3} eines] Gein des O Zcu eines M Jenes Q (R) Z (Fr39) (Fr40) \textbf{4} Feirafiz] veirefiz G Feireviz O Ferrefisz M Firrefisz Q Feirefiz R Z Feirefis Fr39 ferefiz Fr40  $\cdot$ Anschevin] antsheuin I Ansehvin O anshevisz Q Anshevin R (Z) (Fr39) anschevein Fr40 \textbf{5} mit] ÷it O  $\cdot$ herter] rechtir M (Q) (R) (Fr39) (Fr40) \textbf{6} niht] \textit{om.} O \textbf{8} pfliget] phlac I \textbf{9} derst] Ja ist Z  $\cdot$ beidiu] \textit{om.} O beide R \textbf{10} künigîn] kúnginen R  $\cdot$ sun] \textit{om.} O  $\cdot$ Zazamanc] zazamanch G O Zasmaanc R zazamank Fr40 \textbf{11} denke ich] gedenchet ich O gedencke ich Q denct ir Z  $\cdot$ an] \textit{om.} Q  $\cdot$ Gahmureten] Gamvreten O gamureten M gamúreten Q gemureten Z \textbf{12} \textit{Vers 317.12 fehlt} Q   $\cdot$ herze valsches] herczin valscheit M  $\cdot$ erjeten] irreten M \textbf{13} Anschouwe] antschev I anschowe O M (Q) (R) Fr40 anshowe Z Fr39  $\cdot$ iuwer] ir M \textbf{14} erbe] ere M \textbf{15} danne] Das dann Q  $\cdot$ habet] hab O had M (Z)  $\cdot$ geworben] erworben Q (R) (Fr40) \textbf{16} ir sit an bris verdorben I  $\cdot$ prîse] ir pris Z  $\cdot$ ir sît] sit ir M \textbf{18} wolt] solt O M Q (R) (Z) Fr39 (Fr40)  $\cdot$ ich ez] ichz gerne O (Q) Z (Fr40) irz M ich gern R (Fr39)  $\cdot$ vür] ur I vur gerne M \textbf{19} ir] Jren Q (Z) (Fr39) (Fr40)  $\cdot$ möht] moht G O (M) (Q) Z Fr39 Fr40  $\cdot$ sîn] [v]: sin G ir I M Q R Fr39 Fr40  $\cdot$ gesîn] sin I \textbf{20} triuwe] triwen O (M) (Z) (Fr40) \textbf{21} Geloubt] Geloibit ir M Gelavp Q  $\cdot$ guot] gvͦtiv O guter Q (R) (Z) (Fr39) (Fr40) \textbf{23} manlîcher triuwen] manlich triwe I Minnechlicher triwen O \textbf{24} wîtvenge] witvengech O (M) (Z) wintvench Q (Fr39) wittwenk R  $\cdot$ hôher] houbit M hohen Q \textbf{25} wol] \textit{om.} Z  $\cdot$ schalle] schallen Z \textbf{26} grôz] Groze O  $\cdot$ unde] vnd vnd R  $\cdot$ kleiniu] chlaine O  $\cdot$ galle] gallen Z \textbf{28} riuse] russe G M ruͦshce I kivsche Fr39 ro:sch Fr40  $\cdot$ vengec] \textit{om.} I vene Q \textbf{29} sîn manlîch] sin wert manlich G Sin mænlichez O (M) (Q) (R) (Z) (Fr39) (Fr40) \textbf{30} den prîs] dem preise Q  $\cdot$ stellen] gestellen O (M) (Q) Z Fr39 Fr40 gestellend R \newline
\end{minipage}
\hspace{0.5cm}
\begin{minipage}[t]{0.5\linewidth}
\small
\begin{center}*T
\end{center}
\begin{tabular}{rl}
 & di\textit{u} stat hât erden wunsches solt,\\ 
 & hie het iu \textbf{vrâge} mêr erholt.\\ 
 & \textbf{jenes} landes künegîn\\ 
 & Ferefis Anschevin\\ 
5 & mit herter rîterschaft erwarp,\\ 
 & an dem di\textit{u} manheit niht verdarp,\\ 
 & die iuwer beider vater truoc.\\ 
 & iuwer bruoder wunder\textit{s} pfliget genuoc,\\ 
 & \textbf{der} ist beidiu swarz unde blanc,\\ 
10 & der küneginne sun von Zazamanc.\\ 
 & nû \textbf{gedenke} ich aber an Gahmureten,\\ 
 & des herze valsches was erjeten.\\ 
 & von Anschouwe iuwer vater hiez,\\ 
 & der iu ander erbe \textit{l}iez,\\ 
15 & denn als ir habt geworben.\\ 
 & \textbf{an} prîse ir sît verdorben.\\ 
 & het iuwer muoter ie missetân,\\ 
 & sô \textbf{wolt}ichz dâ vür hân,\\ 
 & ir m\textit{ö}htet \textbf{sîn} sun niht gesîn.\\ 
20 & nein, \textbf{si} lêrte ir triuwe pîn.\\ 
 & geloubet von ir \textbf{guoter} mære\\ 
 & unde daz iuwer vater wære\\ 
 & manlîcher triuwen wîse\\ 
 & unde \textbf{wîtvenge} \textbf{an hôhen} prîse.\\ 
25 & er kunde wol mit \textbf{schalle}.\\ 
 & grôz herze unde \textbf{klein\textit{iu} galle},\\ 
 & dar \textbf{über} was sîn brust \textbf{ein} dach.\\ 
 & er was riuse, \textbf{venge} \textbf{ein} vach.\\ 
 & sîn manlîchez ellen\\ 
30 & kunde \textbf{dem prîse} wol \textbf{gestellen}.\\ 
\end{tabular}
\scriptsize
\line(1,0){75} \newline
T U V W \newline
\line(1,0){75} \newline
\textbf{1} \textit{Initiale} V  \newline
\line(1,0){75} \newline
\textbf{1} diu] die T \textbf{2} \textit{Versdoppelung 317.2 nach 317.8 (gestrichen)} T   $\cdot$ het] hat W  $\cdot$ vrâge] vragen V \textit{om.} W  $\cdot$ erholt] geholt W \textbf{3} jenes] Jn des U \textbf{4} Ferefis] Fereifiz T Fereifeiz U Ferevis V Ferafis W  $\cdot$ Anschevin] Anschovin U antscheuin W \textbf{6} diu] die T  $\cdot$ niht] nie W \textbf{8} wunders] wunderz T \textbf{10} Zazamanc] zasamang V zazamanck W \textbf{11} Gahmureten] Gahmuͦreten U gamuretten V gamureten W \textbf{12} valsches was] ie valsches waz V was valsches gar W \textbf{13} Anschouwe] Anschôuwe T anschowe U V antschowe W \textbf{14} liez] hiez T \textbf{15} denn als ir] Dan ir U (W) [*r]: Wan als Jr  V  $\cdot$ geworben] erworben W \textbf{19} möhtet] mohtet T (U)  $\cdot$ sîn] ir U V W \textbf{21} guoter] guͦte W \textbf{23} triuwen] treúwe W \textbf{24} wîtvenge an hôhen] wint venke hoher U witvengig hoher V wint wencke hohe W \textbf{26} unde] \textit{om.} W  $\cdot$ kleiniu] cleine T \textbf{27} brust] hertz W  $\cdot$ ein] sin V \textbf{28} riuse venge ein] rúse vnde vengic V der tugende ein W \textbf{30} dem prîse] den preis W \newline
\end{minipage}
\end{table}
\end{document}
