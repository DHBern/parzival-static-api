\documentclass[8pt,a4paper,notitlepage]{article}
\usepackage{fullpage}
\usepackage{ulem}
\usepackage{xltxtra}
\usepackage{datetime}
\renewcommand{\dateseparator}{.}
\dmyyyydate
\usepackage{fancyhdr}
\usepackage{ifthen}
\pagestyle{fancy}
\fancyhf{}
\renewcommand{\headrulewidth}{0pt}
\fancyfoot[L]{\ifthenelse{\value{page}=1}{\today, \currenttime{} Uhr}{}}
\begin{document}
\begin{table}[ht]
\begin{minipage}[t]{0.5\linewidth}
\small
\begin{center}*D
\end{center}
\begin{tabular}{rl}
\textbf{118} & zer \textbf{waste} \textbf{in} Soltane erzogen,\\ 
 & an küneclîcher vuore betrogen,\\ 
 & ez \textbf{en}m\textit{ö}hte an eime site sîn:\\ 
 & \textbf{bogen} unt \textit{b}ölzelîn,\\ 
5 & \textbf{die} sneit er mit sîn selbes hant\\ 
 & unt schôz vil vogele, die er vant.\\ 
 & \begin{large}S\end{large}wenne \textbf{aber} er den vogel erschôz,\\ 
 & \textbf{des} schal von sange ê was \textbf{sô} grôz,\\ 
 & sô weinde er und roufte sich;\\ 
10 & an sîn hâr kêrt er gerich.\\ 
 & sîn lîp was clâr unt fier.\\ 
 & \textbf{ûf} dem plân \textbf{ame} rivier\\ 
 & \textbf{twuog}er sich alle morgen.\\ 
 & er\textbf{ne} kunde \textbf{niht} \textbf{gesorgen},\\ 
15 & ez enwære ob im der vogel \textbf{sanc}.\\ 
 & diu süeze in sîn herze dranc;\\ 
 & daz erstracte im \textbf{sîniu} brüstelîn.\\ 
 & al weinende \textbf{er lief} zer künegîn.\\ 
 & \textbf{Sô} sprach si: "wer hât dir getân?\\ 
20 & dû wære hin ûz ûf den plân."\\ 
 & er\textbf{n} kunde ir gesagen niht,\\ 
 & als \textbf{kinden lîhte} noch geschiht.\\ 
 & Dem mære gienc si lange nâch.\\ 
 & eines tages si in kapfen sach\\ 
25 & ûf die boume nâch der vogele schal.\\ 
 & si wart wol innen, daz \textbf{zerswal}\\ 
 & von der stimme ir kindes brust;\\ 
 & des twanc i\textit{n} art unt sîn gelust.\\ 
 & vrou Herzeloyde kêrt ir haz\\ 
30 & an die vogele, sine wesse umb waz.\\ 
\end{tabular}
\scriptsize
\line(1,0){75} \newline
D \newline
\line(1,0){75} \newline
\textbf{7} \textit{Initiale} D  \textbf{19} \textit{Majuskel} D  \textbf{23} \textit{Majuskel} D  \newline
\line(1,0){75} \newline
\textbf{1} in Soltane] insoltane D \textbf{3} enmöhte] enmohte D \textbf{4} bölzelîn] loͤlzelin D \textbf{28} in] ir D \newline
\end{minipage}
\hspace{0.5cm}
\begin{minipage}[t]{0.5\linewidth}
\small
\begin{center}*m
\end{center}
\begin{tabular}{rl}
 & zer \textbf{wuoste} \textbf{in} Solitane erzogen,\\ 
 & an küniclîcher vuore betrogen,\\ 
 & ez \textbf{en}m\textit{ö}hte an einem site sîn:\\ 
 & \textbf{bogelîn} und bölzelîn,\\ 
5 & \textbf{die} sneit er mit sîn selbes hant\\ 
 & und schôz vil vogele, die er vant.\\ 
 & wenne er den vogel erschôz,\\ 
 & \textbf{des} schal von sange ê was \textbf{sô} grôz,\\ 
 & sô weinete er und roufete \textit{s}ich;\\ 
10 & an sîn hâr kêret er gerich.\\ 
 & sîn lîp was klâr und fier.\\ 
 & \textbf{ûf} dem plâne \textbf{anme} r\textit{ivi}e\textit{r}\\ 
 & \textbf{twanc} er sich alle morgen.\\ 
 & er kunde \textbf{niht} \textbf{gesorgen},\\ 
15 & ez enwære ob ime der vogel \textbf{gesanc}.\\ 
 & diu süeze in sîn herze dranc;\\ 
 & daz erstracte ime \textbf{sîn} brüstelîn.\\ 
 & al weinende \textbf{lief er} zuo der künigîn.\\ 
 & \textbf{dô} sprach si: "wer hât dir getân?\\ 
20 & dû wære hin ûz ûf den plân."\\ 
 & er \textbf{en}kunde \textbf{es} ir gesagen niht,\\ 
 & als \textbf{kinden lîhte} noch geschiht.\\ 
 & dem mære gienc si lange nâch.\\ 
 & eines tages si in kapfen sach\\ 
25 & ûf die boume nâch der vogele schal.\\ 
 & si wart wol innen, daz \textbf{zerswal}\\ 
 & von der stimme ir kindes brust;\\ 
 & des twanc in art und sî\textit{n} gelust.\\ 
 & vrouwe Herczeloide kêrte ir haz\\ 
30 & an die vogele, sine wesse umb waz.\\ 
\end{tabular}
\scriptsize
\line(1,0){75} \newline
m n o \newline
\line(1,0){75} \newline
\newline
\line(1,0){75} \newline
\textbf{1} zer] Zuͦ o  $\cdot$ Solitane] solitanẏe n solitanie o \textbf{2} küniclîcher] koͯniglich n \textbf{3} enmöhte] en mochte m moͯchte n mohte o \textbf{6} und] Er n o \textbf{7} er] er aber n o \textbf{8} sô] \textit{om.} n \textbf{9} weinete] weinet n o  $\cdot$ roufete] ruͯffet o  $\cdot$ sich] rich m \textbf{10} kêret] kerte n o \textbf{12} anme rivier] an me ruwe m in der refier n jn der rievier o \textbf{13} twanc] Toug n (o) \textbf{15} enwære] were n (o) \textbf{17} erstracte ime] erstracket jme n erstarket >im< o \textbf{19} dô] So n Sie o  $\cdot$ si] so o \textbf{21} \textit{Versdoppelung (mit Anteil aus Vers 118.22):} Er kunde es ir gesagen niht / Als kinden es ir gesagen niht o   $\cdot$ enkunde] kunde n \textbf{22} kinden] kunde o  $\cdot$ lîhte] vil lichte n es vilithe o \textbf{28} sîn] sẏ m \textbf{29} Herczeloide] herczloide m hertzeloide n herczeleide o \newline
\end{minipage}
\end{table}
\newpage
\begin{table}[ht]
\begin{minipage}[t]{0.5\linewidth}
\small
\begin{center}*G
\end{center}
\begin{tabular}{rl}
 & zer \textbf{wuosten} Soltanie erzogen,\\ 
 & an kü\textit{n}iclîcher vuore betrogen,\\ 
 & ez \textbf{en}m\textit{ö}ht an einem site sîn:\\ 
 & \textbf{bogen} und bölzelîn\\ 
5 & sneit er mit sîn selbes hant\\ 
 & unde schôz vil vogele, die er vant.\\ 
 & swenne \textbf{aber} er den vogel erschôz,\\ 
 & \textbf{des} schal von sange ê was \textbf{sô} grôz,\\ 
 & sô weinder und roufte sich;\\ 
10 & an sîn hâr kêrt er gerich.\\ 
 & sîn lîp was clâr und fier.\\ 
 & \textbf{\begin{large}Û\end{large}f} dem plân \textbf{\textit{an} einem} rivier\\ 
 & \textbf{twuog}er sich alle morgen.\\ 
 & er kunde \textbf{wênic} \textbf{sorgen},\\ 
15 & ez enwære obe im der vogele \textbf{klanc}.\\ 
 & di\textit{u} süeze in sîn herze dranc;\\ 
 & daz erstracte im \textbf{sîniu} brüstelîn.\\ 
 & al weinend\textit{e} \textbf{\textit{l}ief \textit{er}} zer künigîn.\\ 
 & \textbf{sô} sprach si: "wer hât dir getân?\\ 
20 & dû wære hin ûz ûf den plân."\\ 
 & er kunde \textbf{es} ir gesagen niht,\\ 
 & als \textbf{lîhte kinden} noch geschiht.\\ 
 & dem mære gie si lange nâch.\\ 
 & eines tages si in kapfen sach\\ 
25 & ûf die boume nâch der vogele schal.\\ 
 & si wart wol innen, daz \textbf{e\textit{r}swal}\\ 
 & von der stimme ir kindes brust;\\ 
 & des twang in art und sîn gelust.\\ 
 & \textit{vrô} \textit{Herzeloid}e kêrte ir haz\\ 
30 & a\textit{n} die vogele, sine wesse umbe waz.\\ 
\end{tabular}
\scriptsize
\line(1,0){75} \newline
G I O L M Q R Z \newline
\line(1,0){75} \newline
\textbf{3} \textit{Initiale} I  \textbf{7} \textit{Initiale} L R Z  \textbf{12} \textit{Initiale} G  \textbf{19} \textit{Initiale} I  \newline
\line(1,0){75} \newline
\textbf{1} wuosten] wasten O Q R  $\cdot$ Soltanie] saltunge I soltanîe O soldanie M soltange Q R insolitanie Z  $\cdot$ erzogen] wart erzogen L \textbf{2} küniclîcher] chundchlicher G kummerlicher M  $\cdot$ betrogen] gar betrogen I L \textbf{3} ez] Zu Q  $\cdot$ enmöht] en moht G (I) (O) (L) (R) (Q) (Z)  $\cdot$ an einem] indem I niht an einem O (M) \textbf{5} sneit] daz sneit I  $\cdot$ sîn selbes] sines selbis M sein selbers Q siner R \textbf{6} unde] er I  $\cdot$ vogele] vogelin I \textbf{7} swenne] Wenne L (M) (Q) (R)  $\cdot$ aber er] aber Q er aber R  $\cdot$ den] ein I \textbf{8} des] der I [Sasz]: Dasz Q  $\cdot$ von] mit R  $\cdot$ ê] \textit{om.} R  $\cdot$ sô] vil O \textit{om.} L M Q R \textbf{9} sô] [Do]: So L Sie M  $\cdot$ weinder] weinet er O Z wende M weinende er R  $\cdot$ und] vnd vnd Q  $\cdot$ roufte] Rofft R \textbf{10} an] vber I  $\cdot$ kêrt] kerte L (M) kerst R  $\cdot$ er] er den L  $\cdot$ gerich] gericht R \textbf{12} Ûf] bi I (O) (M)  $\cdot$ an] bi G  $\cdot$ einem] dem O L (M) (Q) (R) Z \textbf{13} twuoger] Tuͦt er R \textbf{14} kunde] enkvnde L (M) (R) (Z)  $\cdot$ wênic sorgen] niht gesorgen O L (M) (Q) (R) (Z) \textbf{15} ez] Zu Q  $\cdot$ enwære] wer O (R) enwese L  $\cdot$ obe im] so L  $\cdot$ vogele] voglein O  $\cdot$ klanc] sanc I (R) Z gesang L \textbf{16} diu] die G I \textbf{17} erstracte] ershract I  $\cdot$ sîniu] diu I \textbf{18} al] Alle R  $\cdot$ lief er] er lief G \textbf{19} sô] \textit{om.} I M O Q  $\cdot$ sprach si] Si sprach I (M) \textbf{20} dû wære] Da rynne M  $\cdot$ hin] hie R  $\cdot$ ûz ûf] vnz vf I [vf]: vz L uff M wo uff Q  $\cdot$ den] dem Q R \textbf{21} er] Ern I (L) (M) (Q) (R) Fro ern Z  $\cdot$ es ir] irz I (Q) ir O R \textbf{22} kinden] kinde M \textbf{23} si] so R \textbf{24} tages] mals R \textbf{25} die] dem R \textbf{26} erswal] et swal G swal L zv swal Z \textbf{28} des] Den Q  $\cdot$ art] sin art R  $\cdot$ und sîn gelust] vnd Guͤt Gelust I brust M \textbf{29} vrô Herzeloide] div chunginne G froͤw herzelaude I Vrow Hertzelauͯde L Frow herczeloide M Fraw herzeloúde Q Frow herczelaude R Frowe herzelovde Z  $\cdot$ kêrte] chert I O (Q) (R) (Z) \textbf{30} an] a G  $\cdot$ sine] si O (Q) (R)  $\cdot$ waz] das R \newline
\end{minipage}
\hspace{0.5cm}
\begin{minipage}[t]{0.5\linewidth}
\small
\begin{center}*T (U)
\end{center}
\begin{tabular}{rl}
 & zuo der \textbf{wuosten} Soltanie erzogen,\\ 
 & an küneclîcher vuore betrogen,\\ 
 & ez m\textit{ö}hte an eine\textit{m} sit\textit{e} sîn:\\ 
 & \textbf{b\textit{o}gen} und bölzelîn\\ 
5 & sneit er mit sîn selbes hant\\ 
 & und schôz vil vogele, die er vant.\\ 
 & wan \textbf{aber} er den vogel erschôz\\ 
 & - \textbf{der} schal von sange ê was grôz -,\\ 
 & sô weint er und roufet sich;\\ 
10 & an sîn hâr kêrter gerich.\\ 
 & sîn lîp was clâr und fier.\\ 
 & \textbf{bî} dem plâne \textbf{an der} rivier\\ 
 & \textbf{twuog} er sich alle morgen.\\ 
 & er kunde \textbf{\textit{n}i\textit{ht}} \textbf{\textit{g}e\textit{s}orgen},\\ 
15 & ez enwære ob im der vogele \textbf{klanc}.\\ 
 & diu süeze in sîn herze dranc;\\ 
 & daz erstracte im \textbf{sîn} brüstelîn.\\ 
 & al we\textit{inend}e \textbf{lief er} zuo der künegîn.\\ 
 & \textit{\textbf{sô}} \textit{sprach si: "wer hât dir getân?}\\ 
20 & \textit{dû wære hin ûz ûf den plân."}\\ 
 & er \textbf{en}kunde \textbf{ez} ir gesagen niht,\\ 
 & als \textbf{lîhte kinden} noch geschiht.\\ 
 & der mære gienc si lange nâch.\\ 
 & eines tages si \textit{i}n kapfen sach\\ 
25 & ûf die boume nâch der vogele schal.\\ 
 & si wart wol innen, daz \textbf{swal}\\ 
 & von der stimme ir kindes brust;\\ 
 & des twanc in art und sîn gelust.\\ 
 & vrou Herzeloyde kêrte ir haz\\ 
30 & an die vogele, si enwiste umb waz.\\ 
\end{tabular}
\scriptsize
\line(1,0){75} \newline
U V W T \newline
\line(1,0){75} \newline
\textbf{3} \textit{Majuskel} T  \textbf{7} \textit{Initiale} W   $\cdot$ \textit{Majuskel} T  \textbf{11} \textit{Initiale} V  \textbf{19} \textit{Majuskel} T  \textbf{23} \textit{Majuskel} T  \newline
\line(1,0){75} \newline
\textbf{1} Soltanie] saltanie V \textbf{3} ez] Er V  $\cdot$ möhte] mochte U (V) (T)  $\cdot$ einem site] einer siten U eine site V im sitte W \textbf{4} bogen] Begen U \textbf{5} mit sîn selbes] selber mit seiner W \textbf{7} wan] Swen V (T)  $\cdot$ aber er] er aber W \textbf{8} der] [De*]: Dez V Des W (T)  $\cdot$ ê was] waz vil W was ê T \textbf{9} weint er und roufet] weinet er vnde roͮfte V (W) rvͦfter vnde slvͦc T \textbf{10} kêrter] kert er W \textbf{11} clâr] starc T \textbf{12} bî] [*]: vffe V  $\cdot$ dem] der T \textbf{13} twuog] Trvͦg V \textbf{14} er] Vnd W ern T  $\cdot$ niht gesorgen] sich alle morgen U \textbf{15} klanc] sang W (T) \textbf{17} Vnd erstraicht im all sein brústelin W  $\cdot$ sîn] sine V \textbf{18} al] \textit{om.} T  $\cdot$ weinende] wege U \textbf{19} \textit{Die Verse 118.19-20 fehlen} U   $\cdot$ sô sprach si] Sun W \textbf{21} enkunde ez ir] [enkond*]: enkonde es ir V kunde es ir W enkvnde sir T \textbf{22} als] sam T  $\cdot$ geschiht] beschiht V \textbf{23} der] Dem V W T  $\cdot$ gienc si] sv́ gieng V \textbf{24} in kapfen sach] ein kappen sach U gesach T \textbf{25} ûf die boume] daz er kapfete T \textbf{26} swal] er schwal W \textbf{27} ir kindes] seine W \textbf{28} twanc] tzang W  $\cdot$ art und] harte W \textbf{29} Herzeloyde] Herzeleide U herzelaude V hertzeloyde W  $\cdot$ kêrte] kert W \textbf{30} enwiste] wúste V \newline
\end{minipage}
\end{table}
\end{document}
