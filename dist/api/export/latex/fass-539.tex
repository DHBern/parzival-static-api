\documentclass[8pt,a4paper,notitlepage]{article}
\usepackage{fullpage}
\usepackage{ulem}
\usepackage{xltxtra}
\usepackage{datetime}
\renewcommand{\dateseparator}{.}
\dmyyyydate
\usepackage{fancyhdr}
\usepackage{ifthen}
\pagestyle{fancy}
\fancyhf{}
\renewcommand{\headrulewidth}{0pt}
\fancyfoot[L]{\ifthenelse{\value{page}=1}{\today, \currenttime{} Uhr}{}}
\begin{document}
\begin{table}[ht]
\begin{minipage}[t]{0.5\linewidth}
\small
\begin{center}*D
\end{center}
\begin{tabular}{rl}
\textbf{539} & \begin{large}S\end{large}wie ez dâ \textbf{was} ergangen,\\ 
 & er hete vil enpfangen,\\ 
 & \textbf{des} er niht vürbaz wolde geben.\\ 
 & vür sicherheit bôt er sîn leben\\ 
5 & unt jach, swaz im geschæhe,\\ 
 & daz er nimer verjæhe\\ 
 & sicherheit durch twingen.\\ 
 & mit dem tôde wolder \textbf{dingen}.\\ 
 & \multicolumn{1}{l}{ - - - }\\ 
 & \multicolumn{1}{l}{ - - - }\\ 
 & \textbf{Dô} sprach \textbf{der} unde ligende:\\ 
10 & "bistû nû der \textbf{gesigende}?\\ 
 & des pflac ich, dô got wolte\\ 
 & \textbf{unt} ich \textbf{prîs} haben solte.\\ 
 & nû hât mîn prîs ein ende\\ 
 & von dîner werden hende.\\ 
15 & Swâ vreischet man oder wîp,\\ 
 & daz überkomen ist mîn lîp,\\ 
 & des prîs sô hôhe \textbf{ê} swebt enbor,\\ 
 & sô stêt mir baz ein sterben vor,\\ 
 & ê mîne vriwent \textbf{diz} mære\\ 
20 & so\textit{l} machen vreuden lære."\\ 
 & Gawan warp sicherheit an in.\\ 
 & dô stuont sîn \textbf{gir} unt \textbf{al} sîn sin\\ 
 & niwan ûfes lîbes verderben\\ 
 & oder ûf ein gæhez sterben.\\ 
25 & Dô dâhte mîn hêr Gawan:\\ 
 & "durch waz tœte ich \textbf{disen} man?\\ 
 & wolt er \textbf{sus} ze mînem gebot stên,\\ 
 & gesunt lieze ich in \textbf{hinnen} gên."\\ 
 & mit rede warb erz an in sô -\\ 
30 & daz \textbf{en}wart niht \textbf{gar} geleistet dô.\\ 
\end{tabular}
\scriptsize
\line(1,0){75} \newline
D Fr31 \newline
\line(1,0){75} \newline
\textbf{1} \textit{Initiale} D  \textbf{9} \textit{Majuskel} D  \textbf{15} \textit{Majuskel} D  \textbf{25} \textit{Majuskel} D  \newline
\line(1,0){75} \newline
\textbf{20} sol] so D \textbf{28} hinnen] von mir Fr31 \newline
\end{minipage}
\hspace{0.5cm}
\begin{minipage}[t]{0.5\linewidth}
\small
\begin{center}*m
\end{center}
\begin{tabular}{rl}
 & wie ez d\textit{â} \textbf{wære} ergangen,\\ 
 & er het vil enpfangen,\\ 
 & \textbf{daz} er niht vürbaz wolte g\textit{e}ben.\\ 
 & vür sicherheit bôt er sîn leben\\ 
5 & und jach, waz im geschæhe,\\ 
 & daz er niemer verjæhe\\ 
 & sicherheit durch twingen.\\ 
 & mit dem tôde wolt er \textbf{ê} \textbf{dingen}\\ 
 & - wær er noch zwirunt strenge\textit{r},\\ 
 & er wolt niht leben lenge\textit{r} -\\ 
 & \textbf{und} sprach \textbf{alsô} under ligende:\\ 
10 & "bistû nû der \textbf{sigende}?\\ 
 & des pflac ich, dô got wolt,\\ 
 & \textbf{daz} ich \textbf{prîs} haben solt.\\ 
 & nû het mîn prîs ein ende\\ 
 & von dîne\textit{r} \textit{w}erden hende.\\ 
15 & wâ \dag vreiset\dag  man oder wîp,\\ 
 & daz überkomen ist mîn lîp,\\ 
 & des prîs sô hôhe swebet enbor,\\ 
 & sô \dag strît\dag  mir baz ein sterben vor,\\ 
 & ê mîne vriunt \textbf{daz} mære\\ 
20 & sol machen vröuden lære."\\ 
 & Gawan warp sicherheit an in.\\ 
 & dô stuont sîn \textbf{gir} und \textbf{alle} sîn sin\\ 
 & niht wan ûf des lîbes verderben\\ 
 & oder ûf ein gæhez sterben.\\ 
25 & \begin{large}D\end{large}ô dâht mîn hêr Gawan:\\ 
 & "durch waz tœte ich \textbf{disen} man?\\ 
 & wolt er \textbf{sus} zuo mîne\textit{m} gebot stên,\\ 
 & gesunt liez ich in \textbf{hinnen} gên."\\ 
 & mit rede warp erz an \textit{in} sô -\\ 
30 & daz wart niht \textbf{gar} geleistet dô.\\ 
\end{tabular}
\scriptsize
\line(1,0){75} \newline
m n o \newline
\line(1,0){75} \newline
\textbf{25} \textit{Initiale} m n  \newline
\line(1,0){75} \newline
\textbf{1} dâ] do m n o \textbf{3} vürbaz wolte] wolte furbasz o  $\cdot$ geben] gebeben m \textbf{6} verjæhe] [vergeh*]: vergehe m \textbf{8} ê] ie o \textbf{8} strenger] strengen m \textbf{8} lenger] lengen m \textbf{10} nû] mir n  $\cdot$ sigende] gesigende n (o) \textbf{14} dîner werden] diner hende werden m \textbf{18} sterben] sternen o \textbf{19} ê mîne] [Emẏn]: E mẏn m E min n (o)  $\cdot$ daz] dise n des o \textbf{20} lære] leren o \textbf{25} hêr] herre her n \textbf{27} mînem] miner m \textbf{29} an in sô] anso m >an< sin so o \textbf{30} daz] Des n \newline
\end{minipage}
\end{table}
\newpage
\begin{table}[ht]
\begin{minipage}[t]{0.5\linewidth}
\small
\begin{center}*G
\end{center}
\begin{tabular}{rl}
 & \textit{\begin{large}S\end{large}}wie ez d\textit{â} \textbf{was} ergangen,\\ 
 & er het\textit{e v}il enpfangen,\\ 
 & \textbf{des} er niht vürbaz wolde geben.\\ 
 & vür sicherheit bôt er sî\textit{n} leben\\ 
5 & unde jach, swaz im geschæhe,\\ 
 & daz er nimmer verjæhe\\ 
 & sicherheit durch twingen.\\ 
 & mit dem tôde wold er \textbf{dingen}.\\ 
 & \multicolumn{1}{l}{ - - - }\\ 
 & \multicolumn{1}{l}{ - - - }\\ 
 & \textbf{dô} sprach \textbf{der} under ligende:\\ 
10 & "bistû nû der \textbf{gesigende}?\\ 
 & des pflag ich, dô got wolde\\ 
 & \textbf{unde} ich \textbf{brîs} haben solde.\\ 
 & nû hât mîn brîs ein ende\\ 
 & von dîner werden hende.\\ 
15 & swâ vreischet man ode wîp,\\ 
 & daz überkomen ist mîn lîp,\\ 
 & des brîs sô hôhe \textbf{ie} swebete enbor,\\ 
 & sô stêt mir baz ein sterben vor,\\ 
 & ê mîn\textit{e} vriunt \textbf{diz} mære\\ 
20 & sul machen vröuden lære."\\ 
 & Gawan \textit{w}arp sicherheit an in.\\ 
 & dô stuont sîn \textbf{gir} unde \textbf{a\textit{l}} sîn sin\\ 
 & niuwan ûffes lîbes verderben\\ 
 & oder ûf ein gæhez sterben.\\ 
25 & dô dâhte mîn hêr Gawan:\\ 
 & "durch waz tœte ich \textbf{den} man?\\ 
 & wolt er \textbf{sus} ze mîne\textit{m} gebote stên,\\ 
 & gesunt lieze ich in \textbf{hin} gên."\\ 
 & mit rede warb erz an in sô -\\ 
30 & daz \textbf{en}wart niht \textbf{gar} geleistet dô.\\ 
\end{tabular}
\scriptsize
\line(1,0){75} \newline
G I L M Z Fr19 \newline
\line(1,0){75} \newline
\textbf{1} \textit{Initiale} G I L Z Fr19  \textbf{21} \textit{Initiale} I  \newline
\line(1,0){75} \newline
\textbf{1} Swie] Wwie G Wie L (M)  $\cdot$ dâ] d: G  $\cdot$ was] wer I \textbf{2} hete vil] het: :il G \textbf{3} vürbaz wolde] vurbazzer wolte I wolde fvrbaz Z \textbf{4} sîn] sine G \textbf{5} jach] sprach M  $\cdot$ swaz] waz L (M) Z \textbf{6} er] der M \textbf{8} wold er] her wolde M (Fr19) \textbf{9} dô] \textit{om.} M Fr19 \textbf{10} bistû nû] bist duz I Bistuͯ L  $\cdot$ gesigende] Sigende I gesegende M \textbf{11} dô] da M \textbf{12} ich] \textit{om.} I \textbf{13} ein] \textit{om.} I \textbf{15} swâ] Wa L M  $\cdot$ vreischet man] man freýschet L (M) (Fr19) \textbf{17} sô hôhe ie] so hohe I E so hohe L so hoch e M Z (Fr19) \textbf{18} sô] Da M \textbf{19} ê] Er M  $\cdot$ mîne] min G \textbf{20} sul] Suͯsz L (M) (Fr19) Svͤlle Z  $\cdot$ machen] mache L Fr19 \textbf{21} Gawan] G:wan Fr19  $\cdot$ warp] erwarp G \textbf{22} dô] Da M Z  $\cdot$ al] alle G \textbf{23} niuwan] Nevr Z  $\cdot$ ûffes] vf I \textbf{25} dô] Da M Z  $\cdot$ dâhte] gedahte L  $\cdot$ hêr Gawan] ergawan M \textbf{26} den] disen I L (M) Z \textbf{27} wolt er] Woldir M  $\cdot$ sus] \textit{om.} M  $\cdot$ mînem] minen G \textbf{29} sô] also Z \textbf{30} enwart] wart I Z :ant Fr19  $\cdot$ dô] da M \newline
\end{minipage}
\hspace{0.5cm}
\begin{minipage}[t]{0.5\linewidth}
\small
\begin{center}*T
\end{center}
\begin{tabular}{rl}
 & \textit{\begin{large}S\end{large}}wie ez dâ \textbf{was} ergangen,\\ 
 & er hete vil enpfangen,\\ 
 & \textbf{des} er niht vürbaz wolte \textit{g}e\textit{b}en.\\ 
 & vür sicherheit bôt er sîn leben\\ 
5 & unde jach, swaz im geschæhe,\\ 
 & daz er niemer verjæhe\\ 
 & sicherheit durch twingen.\\ 
 & mit dem tôde wolter \textbf{ê} \textbf{ringen}.\\ 
 & \multicolumn{1}{l}{ - - - }\\ 
 & \multicolumn{1}{l}{ - - - }\\ 
 & \textbf{Dô} sprach \textbf{der} under ligende:\\ 
10 & "bistû nû der \textbf{gesigende}?\\ 
 & des pflac ich, dô got wolte\\ 
 & \textbf{unde} ich \textbf{êre} haben solte.\\ 
 & nû hât mîn prîs ein ende\\ 
 & von dîner werden hende.\\ 
15 & swâ vreischet man oder wîp,\\ 
 & daz überkomen ist mîn lîp,\\ 
 & des prîs sô hôhe swebt enbor,\\ 
 & sô stêt mir baz ein sterben vor,\\ 
 & ê mîne vriunt \textbf{diz} mære\\ 
20 & s\textit{ol} machen vröuden lære."\\ 
 & Gawan warp sicherheit an in.\\ 
 & dô stuont sîn \textbf{muot} unde sîn sin\\ 
 & niuwan ûffes lîbes verderben\\ 
 & oder ûf ein gæhez sterben.\\ 
25 & \textit{\begin{large}D\end{large}}ô dâhte mîn hêr Gawan:\\ 
 & "durch waz tœtich \textbf{disen} man?\\ 
 & wolter zuo mînem gebote stên,\\ 
 & gesunt lieze ich in \textbf{hinnen} gên."\\ 
 & mit rede warp erz an in sô -\\ 
30 & daz \textbf{en}wart niht geleistet dô.\\ 
\end{tabular}
\scriptsize
\line(1,0){75} \newline
T U V W O Q R Fr40 \newline
\line(1,0){75} \newline
\textbf{1} \textit{Initiale} T U V O  \textbf{9} \textit{Initiale} W   $\cdot$ \textit{Majuskel} T  \textbf{25} \textit{Initiale} T U  \newline
\line(1,0){75} \newline
\textbf{1} Swie] ÷wie T O Wie U W Q R  $\cdot$ dâ] do U V W Q R \textbf{3} des] Daz V (R)  $\cdot$ geben] iehen T \textbf{4} bôt er] er bot O \textbf{5} swaz] waz U (W) (Q) (R) \textbf{6} niemer] nicht Q \textbf{7} durch] mit R \textbf{8} dem] \textit{om.} O  $\cdot$ wolter ê] \textit{om.} Q  $\cdot$ ringen] dingen W O Q \textbf{8} \textit{Die Verse 539.8\textasciicircum1-8\textasciicircum2 sind am Rand nachgetragen und später radiert:} were er noch z: strenger / :aber lenger V  \textbf{9} Dô] [*]: Vnde V [Wol*]: Wold ich Q  $\cdot$ der] [*]: als V \textbf{10} nû] \textit{om.} Q  $\cdot$ gesigende] sigende O \textbf{12} êre] preysz Q \textbf{13} prîs] [*]: pris V lop Q \textbf{15} swâ] Wa U R (W) (Q)  $\cdot$ vreischet] vrieschet V vernimpt R  $\cdot$ oder] vnd R \textbf{19} ê] \textit{om.} Q  $\cdot$ diz] das Q (R) \textbf{20} sol] svln T Suß W \textit{om.} R \textbf{21} Gawan] Gawin R \textbf{22} sîn sin] sin W \textbf{23} niuwan] Nit dan U Nur Q  $\cdot$ ûffes] vf O (Q) vmb des R \textbf{25} Dô] ÷o T \textbf{26} tœtich] docht ich U \textbf{27} zuo] sus zuͦ U (V) W (R) (Fr40) \textbf{28} lieze] [leze]: lieze T  $\cdot$ hinnen] hin U O Q R \textbf{29} an in] im R \textbf{30} enwart] ward W (Q) (Fr40) was O  $\cdot$ niht] nie U niht gar V (W) O (Q) (Fr40) \newline
\end{minipage}
\end{table}
\end{document}
