\documentclass[8pt,a4paper,notitlepage]{article}
\usepackage{fullpage}
\usepackage{ulem}
\usepackage{xltxtra}
\usepackage{datetime}
\renewcommand{\dateseparator}{.}
\dmyyyydate
\usepackage{fancyhdr}
\usepackage{ifthen}
\pagestyle{fancy}
\fancyhf{}
\renewcommand{\headrulewidth}{0pt}
\fancyfoot[L]{\ifthenelse{\value{page}=1}{\today, \currenttime{} Uhr}{}}
\begin{document}
\begin{table}[ht]
\begin{minipage}[t]{0.5\linewidth}
\small
\begin{center}*D
\end{center}
\begin{tabular}{rl}
\textbf{477} & \begin{large}M\end{large}îner geswisterde \textbf{zwei noch} sint.\\ 
 & mîn swester Schoysiane ein kint\\ 
 & gebar; der vrühte lac si tôt.\\ 
 & der herzoge Kyot\\ 
5 & von Katelangen was ir man;\\ 
 & der enwolde ouch sît vreude hân.\\ 
 & Sigunen, \textbf{des selben} töhterlîn,\\ 
 & bevalch man der muoter dîn.\\ 
 & Schoysianen tôt mich smerzen\\ 
10 & \textbf{muoz} enmitten \textbf{ime} herzen.\\ 
 & ir wîplîch herze was sô guot,\\ 
 & ein arche \textbf{vür unkiusche} vluot.\\ 
 & ein magt, mîn swester, pfligt noch site,\\ 
 & sô daz ir \textbf{volget kiusche} mite.\\ 
15 & Repanse de schoye pfligt\\ 
 & des Grâles, der sô swære wigt,\\ 
 & daz in diu valschlîch menscheit\\ 
 & nimmer von der stat \textbf{getreit}.\\ 
 & ir bruoder und mîn ist Anfortas,\\ 
20 & der bêdiu ist unde was\\ 
 & von art des Grâles hêrre.\\ 
 & dem ist leider vreude verre,\\ 
 & wan daz er hât gedingen,\\ 
 & in sul sîn kumber bringen\\ 
25 & \textbf{zem} endelôsem gemache.\\ 
 & mit wunderlîcher sache\\ 
 & \textbf{ist ez} \textbf{im} komen an riwen zil,\\ 
 & als ich dir, neve, \textbf{künden} wil.\\ 
 & pfligstû denne triwe,\\ 
30 & sô erbarmet dich sîn riwe.\\ 
\end{tabular}
\scriptsize
\line(1,0){75} \newline
D Fr31 \newline
\line(1,0){75} \newline
\textbf{1} \textit{Initiale} D  \newline
\line(1,0){75} \newline
\textbf{2} Schoysiane] Tschoysiane D \textbf{5} Katelangen] Chatelange D \textbf{7} Sigunen] Sygvnen D \textbf{9} Schoysianen] Tschoysianen D \textbf{13} ::: noch >::te< Fr31 \textbf{15} Repanse de schoye] Repansse de Shôie D \textbf{19} Anfortas] :nphortas Fr31 \newline
\end{minipage}
\hspace{0.5cm}
\begin{minipage}[t]{0.5\linewidth}
\small
\begin{center}*m
\end{center}
\begin{tabular}{rl}
 & mîner geswi\textit{s}terde \textbf{noch zwei} sint.\\ 
 & mîn swester Schois\textit{i}a\textit{ne} ein kint\\ 
 & ge\textit{ba}r; der vruht lac si tôt.\\ 
 & der herzoge Kyot\\ 
5 & von Kath\textit{e}lan\textit{g}e\textit{n} was ir man;\\ 
 & der enwolte ouch sît \textbf{niht} vröude hân.\\ 
 & Sigunen, \textbf{daz selbe} töhterlîn,\\ 
 & bevalch man der muoter dîn.\\ 
 & S\textit{ch}oisia\textit{n}en tôt mich smerzen\\ 
10 & \textbf{muoz} enmitten \textbf{in mînem} herzen.\\ 
 & ir wîplîch herze was sô guot,\\ 
 & ein arc \textbf{vor unkiuscher} vluot.\\ 
 & ein maget, mîn swester, pfliget noch sit,\\ 
 & sô daz ir \textbf{volget kiusche} mit.\\ 
15 & Repanse de Schoie pfliget\\ 
 & des Grâles, der sô swær wiget,\\ 
 & daz in diu valschlîche menscheit\\ 
 & niemer von der stat \dag streit\dag .\\ 
 & ir bruoder und mîn ist Anfortas,\\ 
20 & der beidiu ist und was\\ 
 & von arte des Grâles hêrre.\\ 
 & dem ist leider vröide verre,\\ 
 & wan daz er hât gedinge\textit{n},\\ 
 & in solle sîn kumber bringe\textit{n}\\ 
25 & \textbf{zuom} \textit{en}delôsen gemach.\\ 
 & mit wunderlîcher sach\\ 
 & \textbf{ez ist} \textbf{nû} komen an riuwen zil,\\ 
 & als ich dir, neve, \textbf{künden} wil.\\ 
 & pfligestû dan triuwe,\\ 
30 & sô erbarmet \textit{d}ich sîn riuwe.\\ 
\end{tabular}
\scriptsize
\line(1,0){75} \newline
m n o \newline
\line(1,0){75} \newline
\newline
\line(1,0){75} \newline
\textbf{1} geswisterde] geswitterde \textit{nachträglich korrigiert zu:} geswisterde m \textbf{2} mîn] Miner n  $\cdot$ Schoisiane] scoisiga m scosẏan n scosian o \textbf{3} gebar] Geg ir m \textbf{4} herzoge] herczege o  $\cdot$ Kyot] kẏot m n o \textbf{5} Kathelangen] kathalanie m kathelange n kathelinge o \textbf{7} daz] des m n o \textbf{9} Schoisianen] Stoisianien m Stosianie n Stosianien o \textbf{12} vluot] flúht o \textbf{15} Repanse de Schoie] Repanse de Scoie m o Repanse de scoẏe n \textbf{16} der] des n \textbf{18} streit] gestreit n o \textbf{19} Anfortas] anfortes o \textbf{23} hât] hette n  $\cdot$ gedingen] gedinge m o \textbf{24} bringen] bringene m bringe o \textbf{25} endelôsen] delosen m \textbf{27} ez] Est o \textbf{30} dich] mich m sich n \newline
\end{minipage}
\end{table}
\newpage
\begin{table}[ht]
\begin{minipage}[t]{0.5\linewidth}
\small
\begin{center}*G
\end{center}
\begin{tabular}{rl}
 & \begin{large}M\end{large}îner geswisterde \textbf{noch zwei noch} sint.\\ 
 & mîn swester Schoysiane ein kint\\ 
 & geb\textit{a}r; der vrühte lac si tôt.\\ 
 & der herzoge Kiot\\ 
5 & von Katelange was ir man;\\ 
 & derne wolde ouch sît \textbf{niht} vröude hân.\\ 
 & Sigunen, \textbf{des selben} töhterlîn,\\ 
 & bevalch man der muoter dîn.\\ 
 & Schoysianen tôt mich smerzen\\ 
10 & \textbf{muoz} enmitten \textbf{ime} herzen.\\ 
 & ir wîplîch herze was sô guot,\\ 
 & ein arche \textbf{vür unkiusche} vluot.\\ 
 & ein maget, \textit{mîn swester}, \textit{pfliget} noch site,\\ 
 & sô daz ir \textbf{kiusche volget} mite.\\ 
15 & Urrepanse de schoye pfliget\\ 
 & des Grâles, der sô swære wiget,\\ 
 & daz in diu valschlîch menscheit\\ 
 & nimmer von der st\textit{at} \textbf{treit}.\\ 
 & ir bruoder unde mîn ist Anfortas,\\ 
20 & der beidiu ist unde was\\ 
 & von arte des Grâles hêrre.\\ 
 & dem ist leider vröude verre,\\ 
 & wan daz er hât gedingen,\\ 
 & in sul sîn kumber bringen\\ 
25 & \textbf{ze dem} endelôsen gemache.\\ 
 & mit wunderlîcher sache\\ 
 & \textbf{ist ez} \textbf{i\textit{m}} komen an riuwen zil,\\ 
 & als ich dir, neve, \textbf{künden} wil.\\ 
 & pfligestû danne triuwe,\\ 
30 & sô erbarmet dich sîn riuwe.\\ 
\end{tabular}
\scriptsize
\line(1,0){75} \newline
G I O L M Z Fr18 Fr49 \newline
\line(1,0){75} \newline
\textbf{1} \textit{Initiale} G O L Z Fr18  \textbf{5} \textit{Initiale} I  \newline
\line(1,0){75} \newline
\textbf{1} Miner geswister vir noch synt M  $\cdot$ Mîner] ÷iner O  $\cdot$ noch zwei noch] zwai noch I (Z) (Fr49) der noch zwei O noch zweý L (Fr18) \textbf{2} Miner Scos swestir ane eyn kint M  $\cdot$ Schoysiane] scoẏsiane G scosiane I iosyane O [Shosiane]: Schosiane L Tschosiane Z Josyane Fr18 Scosian Fr49 \textbf{3} gebar] Gebær G  $\cdot$ si] \textit{om.} O \textbf{4} Kiot] kyot G O M Z Kýot L kẏot Fr18 \textbf{5} von Katelange] Von katalange G Von katelangen I (Fr49) Von katlange L Volate lange M Von kathelange Z \textbf{6} derne] der I (O) (Fr18) (Fr49)  $\cdot$ wolde] wochte M  $\cdot$ sît] sie M Z \textbf{7} Sigunen] Sygvne O (Z) Sigvne M Sẏgvn Fr18 Sigun Fr49 \textbf{8} man] er L \textbf{9} Schoysianen] scoysianen G Scosianen I Sosyanen O Schosiane L Scosanen M Tschoisianen Z Tshoẏsianen Fr18 ::frau Fr49  $\cdot$ mich] mit Z \textbf{10} muoz] Muszen M  $\cdot$ enmitten ime] in minem I (Fr49) mitten in dem L (M) \textbf{11} \textit{Versfolge 477.12-11} O   $\cdot$ wîplîch] wipliche M \textbf{12} arche] art I Fr49 acker L \textbf{13} mîn swester pfliget] phliget min swester G  $\cdot$ site] \textit{om.} Fr49 \textbf{14} kiusche volget] volget chevsche O (L) (M) (Z) (Fr18) \textbf{15} Urrepanse de schoye] Repanse deschoye G kanpasse de schoy I :epanse der schoye O Vrrepansadeshoie L Repanse de schoie M Vrrepanse de tschoie Z Vrredepanse de tschoͮwe Fr18 \textbf{16} des] [Der]: Des O  $\cdot$ sô] da M \textbf{17} valschlîch] valschev I (O) (M) (Fr18) \textbf{18} nimmer] Jemmer L  $\cdot$ stat] stete G  $\cdot$ treit] geweit I getreit O L M Z Fr18 \textbf{19} Vnser brvͦder (ist Fr18 ) Anfortas O (Fr18)  $\cdot$ unde] vnd der I (Z)  $\cdot$ Anfortas] Amfortas L \textbf{25} endelôsen] endelosem I (L) (Fr18) \textbf{26} wunderlîcher] [wn*echlicher]: wnnechlicher O wnnechlicher Fr18 \textbf{27} im] in G \textit{om.} L  $\cdot$ riuwen] rewes Fr49  $\cdot$ zil] \textit{om.} Fr18 \textbf{29} pfligestû] Pfligst O \textbf{30} sîn] \textit{om.} Fr18 \newline
\end{minipage}
\hspace{0.5cm}
\begin{minipage}[t]{0.5\linewidth}
\small
\begin{center}*T
\end{center}
\begin{tabular}{rl}
 & Mîner geswisterde \textbf{zwei noch} sint.\\ 
 & Mîn swester Schosiane ein kint\\ 
 & gebar; der vrühte lac si tôt.\\ 
 & der herzoge Kyot\\ 
5 & von Katelange was ir man;\\ 
 & der enwolte ouch sît \textbf{niht} vröude hân.\\ 
 & Sygune, \textbf{des selben} töhterlîn,\\ 
 & beva\textit{l}ch man der muoter dîn.\\ 
 & Schosianen tôt \textbf{muoz} mich smerzen\\ 
10 & enmitten \textbf{in mînem} herzen.\\ 
 & ir wîplîch herze was sô guot,\\ 
 & ein arche \textbf{vür unkiusche} vluot.\\ 
 & ein maget, mîn swester, pfliget noch site,\\ 
 & sô daz ir \textbf{volget kiusche} mite.\\ 
15 & Repanse de joie pfliget\\ 
 & des Grâles, der sô swære wiget,\\ 
 & daz in diu valschlîch menscheit\\ 
 & niemer von der stat \textbf{getreit}.\\ 
 & ir bruoder unde \textbf{der} mîn ist Anfortas,\\ 
20 & der beidiu ist unde was\\ 
 & von art des Grâles hêrre.\\ 
 & dem ist leider \textit{vröude} verre,\\ 
 & wan daz er hât gedingen,\\ 
 & in sul sîn kumber bringen\\ 
25 & \textbf{zeinem} endelôsen gemache.\\ 
 & mit wunderlîcher sache\\ 
 & \textbf{ist ez} \textbf{im} komen an riuwen zil,\\ 
 & alsich dir, neve, \textbf{sagen} wil.\\ 
 & pfligest dû danne triuwe,\\ 
30 & sô erbarmet dich sîn riuwe.\\ 
\end{tabular}
\scriptsize
\line(1,0){75} \newline
T U V W Q R \newline
\line(1,0){75} \newline
\textbf{1} \textit{Majuskel} T  \textbf{2} \textit{Majuskel} T  \newline
\line(1,0){75} \newline
\textbf{1} \textit{Die Verse 453.1-502.30 fehlen} U   $\cdot$ geswisterde] geswister V (W)  $\cdot$ zwei noch] noch zwey R \textbf{2} Schosiane] Schosyane T scosiane V tschosiane W sofiane Q Shoisiane R \textbf{3} der] die R  $\cdot$ si] sid R \textbf{4} Kyot] kŷot T koyt Q Kẏott R \textbf{5} Katelange] kathelange W (Q) Kattelange R \textbf{6} enwolte] wolt W R \textbf{7} Sygune] Sygunen W R Sygűnen Q  $\cdot$ selben] sellen R \textbf{8} bevalch] bevach T [beua*ht]: beuacht Q  $\cdot$ dîn] [sein]: dein Q \textbf{9} Schosianen] Tschosianen W Sofianen Q Shoisianen R  $\cdot$ muoz] \textit{om.} W Q R \textbf{10} enmitten] Muͦß W Mus in mitten Q (R)  $\cdot$ mînem] dem Q (R) \textbf{12} vluot] guͦt R \textbf{13} noch] \textit{om.} W \textbf{14} kiusche] noch keúsche W \textbf{15} Repanse de joie] Repans de ioie T Repanse de ioge V Vrepans de tschoie W Repanze detschoye Q \textbf{17} valschlîch] valscheit Q falsch R \textbf{18} getreit] gereit R \textbf{19} unde] \textit{om.} R  $\cdot$ der] \textit{om.} W  $\cdot$ Anfortas] antefortes R \textbf{20} der] Die R \textbf{22} leider vröude] leider T vroͤide leidor V leider froͯden R \textbf{23} er hât] ir habt Q \textbf{24} in] Jm R  $\cdot$ sîn] kein Q \textbf{25} zeinem endelôsen] Zuͦ endelosem W Zu dem endelosem Q Zem Endenlose R \textbf{26} wunderlîcher] [w*licher]: wunderlicher V \textbf{27} ez im] in R \textbf{28} neve] [*]: neve V mere W  $\cdot$ sagen] kunden Q \textbf{30} dich] \textit{om.} W \newline
\end{minipage}
\end{table}
\end{document}
