\documentclass[8pt,a4paper,notitlepage]{article}
\usepackage{fullpage}
\usepackage{ulem}
\usepackage{xltxtra}
\usepackage{datetime}
\renewcommand{\dateseparator}{.}
\dmyyyydate
\usepackage{fancyhdr}
\usepackage{ifthen}
\pagestyle{fancy}
\fancyhf{}
\renewcommand{\headrulewidth}{0pt}
\fancyfoot[L]{\ifthenelse{\value{page}=1}{\today, \currenttime{} Uhr}{}}
\begin{document}
\begin{table}[ht]
\begin{minipage}[t]{0.5\linewidth}
\small
\begin{center}*D
\end{center}
\begin{tabular}{rl}
\textbf{693} & \begin{large}D\end{large}ô truoc der \textbf{starke} Parzival\\ 
 & ninder \textbf{müede lit} \textbf{noch erblichen} mâl.\\ 
 & er het an den stunden\\ 
 & sînen helm \textbf{ab} gebunden,\\ 
5 & d\textit{ô} in der werde künec sach,\\ 
 & zuo dem er zühteclîchen sprach:\\ 
 & "Hêrre, swaz mîn neve Gawan\\ 
 & gein iweren hulden hât \textbf{getân},\\ 
 & \textbf{des} lât mich \textbf{vür in} wesen pfant.\\ 
10 & ich trage noch werlîche hant.\\ 
 & \textbf{welt ir zürnen gein im kêren},\\ 
 & \textbf{daz sol ich iu mit swerten wêren}."\\ 
 & Der wirt ûz Rosche Sabbins\\ 
 & sprach: "hêrre, er gît mir morgen zins.\\ 
15 & der stêt ze gelte vür mînen kranz,\\ 
 & \textbf{des} sîn prîs wirt hôch und ganz\\ 
 & oder \textbf{daz er jaget mich} an die stat,\\ 
 & \textbf{al} dâ ich trit \textbf{ûf} lasters pfat.\\ 
 & ir muget wol anders sîn ein helt,\\ 
20 & dirre kampf ist iu doch niht erwelt."\\ 
 & Dô sprach Benen \textbf{süezer} munt\\ 
 & zem künege: "ir ungetriwer hunt!\\ 
 & iwer herze in \textbf{sîner hende} ligt,\\ 
 & \textbf{daz} iwer herze hazzes pfligt.\\ 
25 & war habt ir iuch durch minne ergeben?\\ 
 & diu muoz doch sîner gnâden leben.\\ 
 & ir sagt iuch \textbf{selben} sigelôs.\\ 
 & diu minne ir reht an iu verlôs:\\ 
 & \textbf{getruoget} ir ie minne,\\ 
30 & diu was mit valschem sinne."\\ 
\end{tabular}
\scriptsize
\line(1,0){75} \newline
D \newline
\line(1,0){75} \newline
\textbf{1} \textit{Initiale} D  \textbf{7} \textit{Majuskel} D  \textbf{13} \textit{Majuskel} D  \textbf{21} \textit{Majuskel} D  \newline
\line(1,0){75} \newline
\textbf{1} Parzival] Parcival D \textbf{5} dô] da D \textbf{13} Rosche Sabbins] Roscê Sabbins D \newline
\end{minipage}
\hspace{0.5cm}
\begin{minipage}[t]{0.5\linewidth}
\small
\begin{center}*m
\end{center}
\begin{tabular}{rl}
 & dô truoc der \textbf{starke} Parcifal\\ 
 & nidert \textbf{lide müede} \textbf{noch erblichen} mâl.\\ 
 & er hete an den stunden\\ 
 & sînen helm \textbf{ab} gebunden,\\ 
5 & dô in der werde künic sach,\\ 
 & zuo dem er zühteclîchen sprach:\\ 
 & "hêrre, waz mîn neve Gawan\\ 
 & gegen iuwern hulden het \textbf{getân},\\ 
 & \textbf{daz} lât mich \textbf{vür in} wesen pfant.\\ 
10 & ich trage noch werlîche hant.\\ 
 & \textbf{wolt ir zorn gegen im hân},\\ 
 & \textbf{daz sol mîn hant understân}."\\ 
 & der wir\textit{t û}z Rosche Sabins\\ 
 & sprach: "hêrre, er gît mir morgen zins.\\ 
15 & der stât zuo gelte vür mîne\textit{n} kranz,\\ 
 & \textbf{daz} sîn prîs wirt hôch und ganz\\ 
 & oder \textbf{daz er mich jaget} an die stat,\\ 
 & \textbf{al}dâ ich trit \textbf{ûf} lasters pfat.\\ 
 & ir moget wol anders sîn ein helt,\\ 
20 & diser kampf ist iu doch niht erwelt."\\ 
 & dô sprach Benen \textbf{süezer} munt\\ 
 & zem künige: "ir ungetriuwer hunt!\\ 
 & iuwer herze in \textbf{sîner hend\textit{e}} liget,\\ 
 & \textbf{daz} iuwer herze hazzes pfliget.\\ 
25 & war habt ir iuch durch minne ergeben?\\ 
 & diu muoz doch sîner gnâden leben.\\ 
 & ir sagt iuch \textbf{selben} sigelôs.\\ 
 & diu minne ir reht an iu verlôs:\\ 
 & \textbf{getruoget} ir ie minne,\\ 
30 & diu was mit valschem sinne."\\ 
\end{tabular}
\scriptsize
\line(1,0){75} \newline
m n o Fr69 \newline
\line(1,0){75} \newline
\newline
\line(1,0){75} \newline
\textbf{1} der] der der o \textbf{2} nidert] Niergent n  $\cdot$ lide müede] muͯde gelide n mude >lide< o  $\cdot$ mâl] [mag]: mal o \textbf{3} an] in o \textbf{13} wirt ûz] wirt vs us m  $\cdot$ Rosche] rosce m n rosse o \textbf{14} gît] [gie]: git n \textbf{15} vür mînen] vor minem m \textbf{18} trit] truͯt o \textbf{22} zem] Zauͯm o \textbf{23} hende] henden m  $\cdot$ liget] lege o \textbf{25} habt] bent n \textbf{27} iuch] v́ Fr69  $\cdot$ selben] selber n selbes o \textbf{28} an] on m \textbf{30} mit] \textit{om.} o \newline
\end{minipage}
\end{table}
\newpage
\begin{table}[ht]
\begin{minipage}[t]{0.5\linewidth}
\small
\begin{center}*G
\end{center}
\begin{tabular}{rl}
 & \begin{large}D\end{large}ô truoc der \textbf{junge} Parcival\\ 
 & niener \textbf{müede lide} \textbf{noch erblichen} mâl.\\ 
 & er het an den stunden\\ 
 & sînen helm \textbf{von im} gebunden,\\ 
5 & dô in der werde künic sach,\\ 
 & ze dem er zühticlîchen sprach:\\ 
 & "hêrre, swaz mîn neve Gawan\\ 
 & gein iwern hulden hât \textbf{missetân},\\ 
 & \textbf{dâ vür} lât mich wesen pfant.\\ 
10 & ich trage noch werlîche hant.\\ 
 & \textbf{sol er gein iu ze kampfe stên},\\ 
 & \textbf{hêrre, den lât an mir ergên}."\\ 
 & der wirt ûz Roisabins\\ 
 & sprach: "hêrre, er gît mir morgen zins.\\ 
15 & der stêt ze gelte vür mînen kranz,\\ 
 & \textbf{des} sîn brîs wirt hôch unde ganz\\ 
 & oder \textbf{der mîn geneiget} an die stat,\\ 
 & dâ ich trit \textbf{ûf} lasters pfat.\\ 
 & ir muget wol anders sîn ein helt,\\ 
20 & dirre kampf ist iu doch niht erwelt."\\ 
 & dô sprach \textbf{vroun} Benen munt\\ 
 & zem künige: "ir ungetriwer hunt!\\ 
 & iwer herze in \textbf{sînen handen} liget,\\ 
 & \textbf{dâr} iwer herze hazzes pfliget.\\ 
25 & war habet ir iuch durch minne ergeben?\\ 
 & diu muoz doch sîner gnâden leben.\\ 
 & i\textit{r} saget iuch \textbf{selben} sigelôs.\\ 
 & diu minne ir reht an iu verlôs:\\ 
 & \textbf{truoget} ir ie minne,\\ 
30 & diu was mit valschem sinne."\\ 
\end{tabular}
\scriptsize
\line(1,0){75} \newline
G I L M Z Fr20 \newline
\line(1,0){75} \newline
\textbf{1} \textit{Initiale} G I L Z  \textbf{13} \textit{Initiale} I  \newline
\line(1,0){75} \newline
\textbf{1} Dô] Da M Z  $\cdot$ Parcival] parcifal G Z Fr20 parzifal I L M \textbf{2} niener] Nirgen M  $\cdot$ müede lide] muͦdiu lit I  $\cdot$ erblichen] erblichnev I \textbf{4} von im] vf I abe L M \textbf{5} dô] Da M Z  $\cdot$ der] [den]: der Z  $\cdot$ werde] werder L \textbf{7} swaz] waz L M Z \textbf{8} gein iwern hulden] wider ewer hulde I  $\cdot$ missetân] getan I Z \textbf{9} mich] uch Fr20 \textbf{10} werlîche] wertliche M \textbf{11} Welt ir zvrnen gein im keren Z  $\cdot$ iu] \textit{om.} M  $\cdot$ ze kampfe] zephande I \textbf{12} Daz sol ich ev mit swerten weren Z  $\cdot$ ergên] irsten M \textbf{13} Roisabins] Roys sabins I Roysabinsz L roisz sabins M Rotschesabins Z \textbf{14} hêrre] \textit{om.} L M  $\cdot$ er gît] erget M  $\cdot$ mir] \textit{om.} Z \textbf{16} des] Kranz des M \textbf{17} der mîn geneiget] der mýne geiagt L myne geiagt M daz geiagt mich Z \textbf{18} dâ] Daz L  $\cdot$ trit] strit I trat M getrit Z  $\cdot$ ûf] vz L (M) an Z \textbf{19} ein] \textit{om.} M Z \textbf{21} dô] Da M Z  $\cdot$ vroun] vorw L  $\cdot$ Benen] bene I benem Z \textbf{24} dâr] Daz L  $\cdot$ pfliget] plhigt I \textbf{25} durch] \textit{om.} Z \textbf{27} ir] Jch G  $\cdot$ selben] selber Z \textbf{28} verlôs] verchos I (L) \textbf{29} truoget] Getruͯget L (Z) \newline
\end{minipage}
\hspace{0.5cm}
\begin{minipage}[t]{0.5\linewidth}
\small
\begin{center}*T
\end{center}
\begin{tabular}{rl}
 & dô truoc der \textbf{junge} Parcifal\\ 
 & ni\textit{en}der \textbf{müede lide} \textbf{und bluotigiu} mâl.\\ 
 & er hete an den stunden\\ 
 & sînen helm \textbf{abe} gebunden,\\ 
5 & dô in der werde künec sach,\\ 
 & zuo dem \textit{er} zühteclîche sprach:\\ 
 & "hêrre, waz mîn neve Gawan\\ 
 & gein iuwern hulden hât \textbf{getân},\\ 
 & \textbf{dâ vür} lât mich wesen pfant.\\ 
10 & ich trage noch werlîche hant.\\ 
 & \textbf{sol er gein iu zuo kampfe stân},\\ 
 & \textbf{hêrre, den lât an mir ergân}."\\ 
 & \begin{large}D\end{large}er wirt ûz Roitschesabins\\ 
 & sprach: "hêrre, \textit{e}r gît mir morgen zins.\\ 
15 & der stêt zuo gelte vür mînen kranz,\\ 
 & \textbf{daz} sîn prîs wirt hôch und ganz\\ 
 & oder \textbf{der mîne gejaget} an die stat,\\ 
 & d\textit{â} ich tr\textit{i}t \textbf{in} lasters pfat.\\ 
 & ir moget wol anders sîn ein helt,\\ 
20 & dirre kampf ist \textit{iu} doch niht erwelt."\\ 
 & dô sprach \textbf{vrou} Benen munt\\ 
 & zuo dem künege: "ir ungetriuwer hunt!\\ 
 & iuwer herze in \textbf{s\textit{în}en handen} liget,\\ 
 & \textbf{dâr} iuwer herze hazzes pfliget.\\ 
25 & war hât ir iuch durch minne \textit{er}geben?\\ 
 & diu muoz doch sîner gnâden leben.\\ 
 & ir saget iuch \textbf{selber} sigelôs.\\ 
 & diu minne ir reht an iu verlôs:\\ 
 & \textbf{getruoget} ir ie minne,\\ 
30 & diu was mit valscheme sinne."\\ 
\end{tabular}
\scriptsize
\line(1,0){75} \newline
U V W Q R \newline
\line(1,0){75} \newline
\textbf{1} \textit{Initiale} Q R  \textbf{13} \textit{Initiale} U V  \textbf{21} \textit{Initiale} W  \newline
\line(1,0){75} \newline
\textbf{1} Parcifal] parzefal V partzifal W Q parczifal R \textbf{2} niender] Nider U  $\cdot$ und bluotigiu] noch strites V noch bluͦte W erblichen Q noch erbliche R \textbf{6} er] \textit{om.} U \textbf{7} waz] swaz V  $\cdot$ Gawan] [ge*]: gawan Q [gewan]: gawan R \textbf{8} getân] missetan Q R \textbf{9} vür lât mich] lat mich fur in Q \textbf{11} [*]: Sol er gegen v́ch ze kampfe sten V \textbf{12} [D*]: Herre den lont an mir ergen V \textbf{13} Roitschesabins] roitschezabins W Roitsche [labins]: sabins R \textbf{14} er] abir U \textbf{16} daz] Des V W Q \textbf{18} dâ] Do U W Q [*]: Aldo V  $\cdot$ trit in] drat in U trit [*]: vf V tritte auß W (Q) (R) \textbf{20} iu doch] doch U úch W R \textbf{21} Benen] bene Q \textbf{22} ungetriuwer] vntreúwer W \textbf{23} sînen handen] smehen handen U seiner Q \textbf{24} dâr] [*]: Dar V Das Q Des R  $\cdot$ hazzes] hassen Q \textbf{25} minne] meyn Q  $\cdot$ ergeben] geben U [*]: ergeben V \textbf{26} sîner] ewr Q  $\cdot$ gnâden] gnade W \textbf{27} selber] selbe V (R) selben Q \textbf{28} ir reht an iu] an euch ir recht Q (R) \textbf{29} getruoget] Getrúwett R  $\cdot$ ie minne] minne W minne ie Q \newline
\end{minipage}
\end{table}
\end{document}
