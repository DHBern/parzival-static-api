\documentclass[8pt,a4paper,notitlepage]{article}
\usepackage{fullpage}
\usepackage{ulem}
\usepackage{xltxtra}
\usepackage{datetime}
\renewcommand{\dateseparator}{.}
\dmyyyydate
\usepackage{fancyhdr}
\usepackage{ifthen}
\pagestyle{fancy}
\fancyhf{}
\renewcommand{\headrulewidth}{0pt}
\fancyfoot[L]{\ifthenelse{\value{page}=1}{\today, \currenttime{} Uhr}{}}
\begin{document}
\begin{table}[ht]
\begin{minipage}[t]{0.5\linewidth}
\small
\begin{center}*D
\end{center}
\begin{tabular}{rl}
\textbf{640} & \begin{large}G\end{large}awan unt \textbf{Sangive}\\ 
 & unt diu künegîn Arnive\\ 
 & sâzen stille bî des tanzes schar.\\ 
 & diu herzoginne wol gevar\\ 
5 & her umbe zuo Gawane \textbf{sitzen} gienc;\\ 
 & ir hant er in die sîne \textbf{enpfienc},\\ 
 & si sprâchen sus und sô.\\ 
 & ir komens was er \textbf{zuo zim} vrô.\\ 
 & sîn riuwe \textbf{smal unt} sîn vreude breit\\ 
10 & wart dô, sus \textbf{swant} im \textbf{al} sîn leit.\\ 
 & was ir vreude \textbf{ame} tanze grôz,\\ 
 & Gawan noch minre hie verdrôz.\\ 
 & Diu künegîn Arnive sprach:\\ 
 & "hêrre, \textbf{nû} prüevet iwer gemach.\\ 
15 & ir \textbf{solt} an disen stunden\\ 
 & ruowen. ziwern wunden\\ 
 & hât sich diu herzogîn \textbf{bewegen},\\ 
 & daz si iwer \textbf{wil mit decke} pflegen\\ 
 & noch hînte geselleclîche;\\ 
20 & \textbf{diu} ist helfe unt râtes rîche."\\ 
 & Gawan sprach: "des vrâget sie."\\ 
 & "\textbf{in} iwer bêder gebot \textbf{ich} hie\\ 
 & \textbf{gar} bin", \textbf{sus} sprach diu herzogîn.\\ 
 & "er sol in mîner pflege sîn.\\ 
25 & Lât diz volc slâfen varn.\\ 
 & ich \textbf{sol} in \textbf{hînte} \textbf{sô} bewarn,\\ 
 & daz sîn \textbf{nie vriwendinne} baz gepflac.\\ 
 & Floranden von Itolac\\ 
 & unt den herzogen von Gowerzin\\ 
30 & lât in der rîter pflege sîn."\\ 
\end{tabular}
\scriptsize
\line(1,0){75} \newline
D Z Fr1 \newline
\line(1,0){75} \newline
\textbf{1} \textit{Initiale} D Z Fr1  \textbf{13} \textit{Majuskel} D   $\cdot$ \textit{Versal} Fr1  \textbf{25} \textit{Majuskel} D  \newline
\line(1,0){75} \newline
\textbf{1} Sangive] Sangîve D Fr1 Seyve Z \textbf{2} Arnive] Arnîve D Fr1 \textbf{5} Gawane] gawanen Z \textbf{8} zim] im Z \textbf{9} unt] \textit{om.} Z \textbf{10} dô] da Z  $\cdot$ swant] verswant Z \textbf{12} Gawan] Gawanen Fr1  $\cdot$ noch minre hie] hie nach minner Z \textbf{13} Arnive] Arnîve D Fr1 \textbf{14} iwer] ewern Z \textbf{15} solt] soldet Z (Fr1) \textbf{16} ziwern] ewern Z \textbf{22} in] Zv Z \textbf{23} sus] so Z \textbf{26} Frowe ich wil in so bewarn Z  $\cdot$ sô] also Fr1 \textbf{27} nie vriwendinne] frevndin nie Z \textbf{28} Itolac] Jtolach D Fr1 Jcolat Z \textbf{29} Gowerzin] Gowerzîn Fr1 \newline
\end{minipage}
\hspace{0.5cm}
\begin{minipage}[t]{0.5\linewidth}
\small
\begin{center}*m
\end{center}
\begin{tabular}{rl}
 & Gawan und \textbf{Sa\textit{n}g\textit{iv}e}\\ 
 & und diu künigîn Ar\textit{niv}e\\ 
 & sâzen stille bî des tanzes schar.\\ 
 & diu herzogîn wol gevar\\ 
5 & her umb zuo Gawan \textbf{sitzen} g\textit{ie}nc;\\ 
 & ir hant er in die sîn \textbf{enpfienc},\\ 
 & si spr\textit{â}chen sus und sô.\\ 
 & ir komens was er vrô.\\ 
 & sîn riuwe \textbf{smal}, sîn vröude breit\\ 
10 & war\textit{t} dô, sus \textbf{swant} im sîn leit.\\ 
 & was ir vröude \textbf{an dem} tanze grôz,\\ 
 & Gawan noch minner hie verdrôz.\\ 
 & diu künigîn Ar\textit{niv}e sprach:\\ 
 & "hêrre, \textbf{nû} brüefet iuwer \textit{g}emach.\\ 
15 & ir \textbf{solt} an disen stunden\\ 
 & ruowen. zuo iuwern wunden\\ 
 & het sich diu herzogîn \textbf{erwegen},\\ 
 & daz si iuwer \textbf{wil mit decke} pflegen\\ 
 & noch hînt geselleclîch;\\ 
20 & \textbf{daz} ist helfe und râtes rîch."\\ 
 & Gawan sprach: "des vrâget sie."\\ 
 & "\textbf{in} iuwer beider gebot hie\\ 
 & bin \textbf{ich}", sprach diu herzogîn.\\ 
 & "er sol in \dag iuwer\dag  pflege sîn.\\ 
25 & lât diz volc slâfen varn.\\ 
 & ich \textbf{sol} in \textbf{hînt} \textbf{wol} bewarn,\\ 
 & daz sîn \textbf{nie vriundîn} baz gepfla\textit{c}.\\ 
 & Floranden von Itola\textit{c}\\ 
 & und de\textit{n} herzog\textit{e}n von Gowertzin\\ 
30 & lât in der ritter \dag pfleger\dag  sîn."\\ 
\end{tabular}
\scriptsize
\line(1,0){75} \newline
m n o \newline
\line(1,0){75} \newline
\newline
\line(1,0){75} \newline
\textbf{1} Sangive] sagwe m sangwine n sangwe o \textbf{2} künigîn] konige o  $\cdot$ Arnive] arune m arniwe n o \textbf{5} gienc] gang m o \textbf{6} er in die sîn] sie sinne die er o \textbf{7} sprâchen] sprechen m \textbf{10} wart] war m \textbf{13} künigîn] konig: o  $\cdot$ Arnive] arune m arniwe n o \textbf{14} gemach] vngemach m \textbf{17} herzogîn] konigine o \textbf{19} noch] Doch n \textbf{21} des] das o \textbf{25} diz] das n \textbf{26} wol] so n o \textbf{27} nie vriundîn] frúndin nie n nie fruͯnde o  $\cdot$ gepflac] gepflage m \textbf{28} Itolac] jtholage m itholag n von Jtolag o \textbf{29} den herzogen] der hertzogin m  $\cdot$ Gowertzin] gowerczin o \textbf{30} pfleger] pflegen o \newline
\end{minipage}
\end{table}
\newpage
\begin{table}[ht]
\begin{minipage}[t]{0.5\linewidth}
\small
\begin{center}*G
\end{center}
\begin{tabular}{rl}
 & \begin{large}G\end{large}awan unde \textbf{Sagive}\\ 
 & unde diu künegîn Arnive\\ 
 & sâzen stille bî des tanzes schar.\\ 
 & diu herzoginne wolgevar\\ 
5 & her umbe ze Gawane gienc;\\ 
 & ir hant er in die sîne \textbf{vienc},\\ 
 & si sprâchen sus unde sô.\\ 
 & ir komens was er \textbf{zuo im} vrô.\\ 
 & sîn riuwe \textbf{smal}, sîn vröude breit\\ 
10 & wart dô, sus \textbf{verswant} im \textbf{al} sîn leit.\\ 
 & was ir vröude \textbf{an} tanze grôz,\\ 
 & Gawan noch minner hie verdrôz.\\ 
 & diu künegîn Arnive sprach:\\ 
 & "hêrre \textbf{mîn}, prüevet iuwer ge\textit{mach}.\\ 
15 & ir \textbf{soldet} an disen stunden\\ 
 & ruowen.  iuwern wunden\\ 
 & hât sich diu herzogîn \textbf{bewegen},\\ 
 & daz si iwer \textbf{mit decke wil} pflegen\\ 
 & noch hînt geselleclîch;\\ 
20 & \textbf{diu} ist helfe unde râtes rîch."\\ 
 & Gawan sprach: "des vrâget sie,\\ 
 & \textbf{ze} iuwer beider gebot \textbf{ich} hie\\ 
 & bin." \textbf{dô} sprach diu herzogîn:\\ 
 & "er sol in mîner pflege sîn.\\ 
25 & lât ditze volc slâfen varen.\\ 
 & \textbf{vrouwe}, ich \textbf{wil} in \textbf{sô} bewaren,\\ 
 & daz sîn \textbf{vriundîn nie} baz gepflac.\\ 
 & Floranden von Ytolac\\ 
 & unde den herzogen von Gowerzin\\ 
30 & lât in der rîter pflege sîn."\\ 
\end{tabular}
\scriptsize
\line(1,0){75} \newline
G I L M Z Fr18 Fr48 \newline
\line(1,0){75} \newline
\textbf{1} \textit{Initiale} G L Z Fr18 Fr48  \textbf{7} \textit{Initiale} I  \textbf{17} \textit{Initiale} I  \newline
\line(1,0){75} \newline
\textbf{1} Sagive] saiue I Segive L sarve M Seyve Z Saẏue Fr18 Seyue Fr48 \textbf{2} Arnive] arniue I (Fr48) ARnẏue Fr18 \textbf{3} stille] \textit{om.} Fr18 \textbf{5} Gawane] Gawan I Fr48 gawanen Z  $\cdot$ gienc] sitzen gienc Z (Fr48) \textbf{6} sîne] sinen I (L)  $\cdot$ vienc] enphie L (M) (Z) Fr18 (Fr48) \textbf{7} sprâchen] sprach L \textbf{8} was er zuo im] zuͤ im was er I (L) zu im ward er Fr48 \textbf{10} dô] da M Z  $\cdot$ sus] \textit{om.} L  $\cdot$ verswant] swant I  $\cdot$ im] \textit{om.} I L Fr18  $\cdot$ al] \textit{om.} Fr18 \textbf{11} sus was ir tanzens also groz I  $\cdot$ an] an dem Z (Fr48) \textbf{12} Gawan] Gawanen L Gawan:n Fr18  $\cdot$ noch minner hie] hie nach minner Z  $\cdot$ minner] nýnder L \textbf{13} Arnive] arniue I \textbf{14} mîn] nu Z (Fr48)  $\cdot$ iuwer] ewern I (Z) (Fr48)  $\cdot$ gemach] gen* \textit{nachträglich korrigiert zu:} gemach G \textbf{15} soldet] svlt L (M) (Fr18) (Fr48) \textbf{17} bewegen] des bewegen I \textbf{18} mit decke wil] wil mit rechte M wil mit decke Z (Fr18) (Fr48) \textbf{19} hînt] hute M \textbf{20} ist] \textit{om.} M  $\cdot$ rîch] geleich reiche Fr48 \textbf{22} ich] bin ich I \textbf{23} bin dô] do I Byn da M Gar bin so Z Bin gar suͦz Fr48 \textbf{25} lât] laz I (M) \textbf{27} sîn] sin sin I  $\cdot$ vriundîn nie] nie frvndin L \textbf{28} Floranden] florianden G (I) Florande L Florand:: Fr18  $\cdot$ Ytolac] Jdolac I jtolach L Jtolac M Fr48 Jcolat Z Jtelach Fr18 \textbf{29} Gowerzin] couerzin I gowerczin M (Fr48) \newline
\end{minipage}
\hspace{0.5cm}
\begin{minipage}[t]{0.5\linewidth}
\small
\begin{center}*T
\end{center}
\begin{tabular}{rl}
 & Gawan und \textbf{Seyve}\\ 
 & und diu küneginne Arnyve\\ 
 & sâzen stille bî des tanzes schar.\\ 
 & diu herzoginne wol gevar\\ 
5 & her umb zuo Gawane \textbf{sitzen} gienc;\\ 
 & ir hant er in die sîne \textbf{vienc},\\ 
 & \begin{large}S\end{large}i sprâchen sus und sô.\\ 
 & ir komens was er \textbf{zuo in} vrô.\\ 
 & sîn riuwe \textbf{val}, sîn vreude breit\\ 
10 & wart dô, sus \textbf{verswant} im sîn leit.\\ 
 & was ir vreude \textbf{an} tanze grôz,\\ 
 & Gawan noch minre hie verdrôz.\\ 
 & diu küneginne Arnyve sprach:\\ 
 & "hêrre, \textbf{nû} prüevet iuwer gemach.\\ 
15 & ir \textbf{solt} an disen stunden\\ 
 & ruowen. zuo iuwern wunden\\ 
 & he\textit{t} sich diu herzogîn \textbf{bewegen},\\ 
 & daz si iuwer \textbf{dicke wolle} pflegen\\ 
 & noch hînaht geselleclîche;\\ 
20 & \textbf{diu} ist helfe und râtes rîche."\\ 
 & Gawan sprach: "des vrâget sie."\\ 
 & "\textbf{zuo} iuwer beider gebote \textbf{ich} hie\\ 
 & bin", sprach diu herzogîn.\\ 
 & "er sol in mîner pflege sîn.\\ 
25 & lât diz volc slâfen varn.\\ 
 & \textbf{vrouwe}, ich \textbf{wil} in \textbf{sô} bewarn,\\ 
 & daz sîn \textbf{vriundîn nie} baz gepflac.\\ 
 & Florant von Itolac\\ 
 & und den herzogen von Gowerzin\\ 
30 & lât in der rîter pflege sîn."\\ 
\end{tabular}
\scriptsize
\line(1,0){75} \newline
U V W Q R Fr40 \newline
\line(1,0){75} \newline
\textbf{1} \textit{Initiale} Q R Fr40  \textbf{7} \textit{Initiale} U V  \textbf{13} \textit{Initiale} W  \newline
\line(1,0){75} \newline
\textbf{1} Gawan] ÷awann Q Gawin R  $\cdot$ Seyve] seiue V seyue W (R) seyre Q \textbf{2} Arnyve] Arnive U (Fr40) arniue V Q arneyue W Arnẏue R \textbf{5} Gawane] gawan W gawann Q Gawinen R \textbf{6} er] \textit{om.} R  $\cdot$ sîne] sinen R  $\cdot$ vienc] enpfieng W (Q) (R) (Fr40) \textbf{7} sprâchen] sprach V \textbf{8} zuo in] zvͦ zim V (Q) (Fr40) o\textit{m. } W zu Im R \textbf{9} riuwe] trúwe R  $\cdot$ val] smal V (W) Q (R) Fr40 \textbf{10} dô sus] sus do W do als Q  $\cdot$ verswant] schwand W  $\cdot$ im] [*]: im V \textit{om.} R \textbf{11} an] [*]: amme V am W R \textbf{12} Gawan] Gawanen Q Gawin R  $\cdot$ hie] \textit{om.} W Q \textbf{13} Arnyve] arniue V W Q Arnyue R \textbf{14} nû] \textit{om.} W  $\cdot$ prüevet] prvͤven V  $\cdot$ iuwer] ewren Q \textbf{15} ir] Nun Q  $\cdot$ solt] soldet Q (R)  $\cdot$ disen] disten R \textbf{17} het] Hete U  $\cdot$ bewegen] verwegen W \textbf{18} si] \textit{om.} Q  $\cdot$ dicke] mit decce V (Q) (R) o\textit{m. } W \textbf{19} [*]: Noch hint gesellecliche V \textbf{20} diu] Das W  $\cdot$ helfe und râtes] rat vnd helffes R \textbf{21} Gawan] Gawin R  $\cdot$ sprach] \textit{om.} R \textbf{22} zuo] In W  $\cdot$ beider] beyde Q  $\cdot$ gebote] bet Q  $\cdot$ ich] \textit{om.} W bin Jch R \textbf{23} bin] Bin do V Q Bin ich W Do R \textbf{24} mîner] eúwer W  $\cdot$ pflege] helfe Q \textbf{26} vrouwe] \textit{om.} W  $\cdot$ wil] sol W  $\cdot$ sô] selber [*]: noch hint V heinnacht also W \textbf{27} vriundîn] frewden Q \textbf{28} Florant] Florande U V Floranden W Q R  $\cdot$ Itolac] Jtolac U ẏtolac V ytelag W ytolat Q Itolag R \textbf{29} Gowerzin] kawerzin Q \textbf{30} pflege] [pflege*]: pflege V pfleger W pflegen Q \newline
\end{minipage}
\end{table}
\end{document}
