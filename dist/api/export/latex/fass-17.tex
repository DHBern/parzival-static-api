\documentclass[8pt,a4paper,notitlepage]{article}
\usepackage{fullpage}
\usepackage{ulem}
\usepackage{xltxtra}
\usepackage{datetime}
\renewcommand{\dateseparator}{.}
\dmyyyydate
\usepackage{fancyhdr}
\usepackage{ifthen}
\pagestyle{fancy}
\fancyhf{}
\renewcommand{\headrulewidth}{0pt}
\fancyfoot[L]{\ifthenelse{\value{page}=1}{\today, \currenttime{} Uhr}{}}
\begin{document}
\begin{table}[ht]
\begin{minipage}[t]{0.5\linewidth}
\small
\begin{center}*D
\end{center}
\begin{tabular}{rl}
\textbf{17} & \multicolumn{1}{l}{ - - - }\\ 
 & \multicolumn{1}{l}{ - - - }\\ 
 & si \textbf{tæten} \textbf{sînen} boten kunt,\\ 
 & \textbf{ez} \textbf{wære} Patelamunt.\\ 
5 & daz wart \textbf{im} \textbf{inneclîchen} enboten.\\ 
 & \textbf{si} \textbf{manten} in bî ir goten,\\ 
 & daz er in hülfe. \textbf{des} wære in nôt.\\ 
 & si rungen niht wan umben tôt.\\ 
 & \textbf{Dô} der junge Anschevin\\ 
10 & vernam ir kumberlîchen pîn,\\ 
 & er bôt sîn dienest umbe guot,\\ 
 & als \textbf{noch vil dicke} \textbf{ein rîter} tuot,\\ 
 & oder daz si \textbf{im} sageten, umbe waz\\ 
 & er solte \textbf{doln} der vîende haz.\\ 
15 & \begin{large}D\end{large}ô sprach ûz einem munde\\ 
 & der sieche unde der gesunde,\\ 
 & daz im wære al gemeine\\ 
 & ir golt unde ir gesteine;\\ 
 & des solt er alles hêrre wesen\\ 
20 & \textbf{unde} \textbf{er} m\textit{ö}hte wol \textbf{bî} in genesen.\\ 
 & \textbf{doch} bedorft er \textbf{wênec} soldes.\\ 
 & von Arabie des goldes\\ 
 & het er manegen knollen brâht.\\ 
 & liute vinster sô diu naht\\ 
25 & \textbf{wâren} alle die von Zazamanc.\\ 
 & bî den dûht in diu wîle lanc.\\ 
 & \textbf{doch} hiez er herberge nemen.\\ 
 & des moht ouch si vil wol gezemen,\\ 
 & daz \textbf{si im} die besten \textbf{gâben}.\\ 
30 & die vrouwen dennoch lâgen\\ 
\end{tabular}
\scriptsize
\line(1,0){75} \newline
D Fr9 \newline
\line(1,0){75} \newline
\textbf{9} \textit{Initiale} Fr9   $\cdot$ \textit{Majuskel} D  \textbf{15} \textit{Initiale} D  \newline
\line(1,0){75} \newline
\textbf{1} \textit{Die Verse 17.1-2 fehlen} D Fr9  \textbf{3} tæten] taten Fr9 \textbf{4} ez] [*ez]: ez D \textbf{5} im] \textit{om.} Fr9 \textbf{8} si] Sie ne Fr9 \textbf{9} Anschevin] Anscivin D anzevẏn Fr9 \textbf{20} möhte] mohte D (Fr9) \textbf{21} bedorft] dorfte Fr9 \textbf{22} Arabie] arabi Fr9 \textbf{24} vinster] dinster Fr9 \textbf{25} Zazamanc] Zazamanch D \textbf{27} doch] Do Fr9 \textbf{30} Die vrouwen hetten sich erhaben Fr9 \newline
\end{minipage}
\hspace{0.5cm}
\begin{minipage}[t]{0.5\linewidth}
\small
\begin{center}*m
\end{center}
\begin{tabular}{rl}
 & \multicolumn{1}{l}{ - - - }\\ 
 & \multicolumn{1}{l}{ - - - }\\ 
 & si \textbf{tâten} \textbf{sînem} boten kunt,\\ 
 & \textbf{ez} \textbf{wære} Patelamu\textit{n}t.\\ 
5 & daz \textit{wart} \textbf{in} \textbf{minneclîchen} enboten.\\ 
 & \textbf{si} \textbf{nâmen} in bî ir goten,\\ 
 & daz er in hülfe. \textbf{es} wære in nôt.\\ 
 & si \textit{r}ungen niht wenne umb den tôt.\\ 
 & \textbf{\begin{large}D\end{large}ô} der junge A\textit{n}schevin\\ 
10 & vernam ir kumberlîche pîn,\\ 
 & er bôt sînen dienst umb guot,\\ 
 & als \textbf{noch vil dicke} \textbf{ein ritter} tuot,\\ 
 & oder daz si sageten, umb waz\\ 
 & er solte \textbf{dulden} der vîende haz.\\ 
15 & dô sprach ûz einem munde\\ 
 & der sieche und der gesunde,\\ 
 & daz ime wære algemeine\\ 
 & ir golt und ir gesteine;\\ 
 & des solte er alles hêrre wesen\\ 
20 & \textbf{und} mehte wol \textbf{vor} in genesen.\\ 
 & \textbf{doch} bedorfte er \textbf{wênic} soldes.\\ 
 & von Arabie des goldes\\ 
 & hette er menigen knollen brâht.\\ 
 & \dag liuhte\dag  vinster sô diu naht\\ 
25 & \textbf{varn} alle die von Zazamanc.\\ 
 & bî den dûhte in diu wîle lanc.\\ 
 & \textbf{dô} hiez er herberge nemen.\\ 
 & des mohte ouch si vil wol gezemen,\\ 
 & daz \textbf{sîn} die besten \textbf{pflâgen}.\\ 
30 & die vrowen \textbf{lange} dennoch lâgen\\ 
\end{tabular}
\scriptsize
\line(1,0){75} \newline
m n o \newline
\line(1,0){75} \newline
\textbf{9} \textit{Initiale} m   $\cdot$ \textit{Capitulumzeichen} n  \newline
\line(1,0){75} \newline
\textbf{1} \textit{Die Verse 17.1-2 fehlen} m n o  \textbf{4} Patelamunt] patelamuuͦt m patalemúnt o \textbf{5} wart] \textit{om.} m \textbf{7} daz er] Was ir o  $\cdot$ hülfe] hilffe n \textbf{8} rungen] nunngent m  $\cdot$ niht] mit o \textbf{9} Anschevin] ausceuin \textit{nachträglich korrigiert zu:} ansceuin m [auscenẏe]: auscenẏne n anscenyen o \textbf{11} guot] g:t o \textbf{12} dicke] \textit{om.} o  $\cdot$ tuot] d:t o \textbf{14} dulden] \textit{om.} n  $\cdot$ vîende] winde o \textbf{17} algemeine] alle gemeine n (o) \textbf{20} und] Vnd er n o  $\cdot$ mehte] mohte o  $\cdot$ vor] von o  $\cdot$ in] jme n \textbf{21} bedorfte] bedarff o \textbf{23} hette] Ahett o \textbf{25} varn] Faren \textit{nachträglich korrigiert zu:} Waren m  $\cdot$ Zazamanc] zazamanck m zaczamang n zazamang o \textbf{26} dûhte] duch n \textbf{27} hiez] hiesse n \textbf{28} des] Das o  $\cdot$ ouch si] sú n es auch sie o  $\cdot$ gezemen] schemen n \textbf{30} lange] \textit{om.} n o \newline
\end{minipage}
\end{table}
\newpage
\begin{table}[ht]
\begin{minipage}[t]{0.5\linewidth}
\small
\begin{center}*G
\end{center}
\begin{tabular}{rl}
 & wan ir kunde nie gewan\\ 
 & \textbf{weder} er noch dehein sîn schifman.\\ 
 & si \textbf{tæten} \textbf{sînem} boten kunt,\\ 
 & \textbf{si} \textbf{hieze} Patelamunt\\ 
5 & - daz wart \textbf{im} \textbf{minneclîche} enboten -\\ 
 & \textbf{unde} \textbf{manten} in bî ir goten,\\ 
 & d\textit{a}z er in hülfe. \textbf{es} wære in nôt.\\ 
 & si rungen niht wan umben tôt.\\ 
 & \textbf{dô} der junge Antschevin\\ 
10 & vernam ir kumberlîchen bîn,\\ 
 & er bôt sîn dienst umbe guot,\\ 
 & als \textbf{ouch noch} \textbf{ein rîter} tuot,\\ 
 & oder daz si\textbf{m} seiten, umbe waz\\ 
 & er solte \textbf{dulten} der vînde haz.\\ 
15 & dô sprach ûz einem munde\\ 
 & der sieche unt der gesunde,\\ 
 & daz im wære algemeine\\ 
 & ir golt und ir gesteine;\\ 
 & des solter alles hêrre wesen,\\ 
20 & \textbf{\begin{large}E\end{large}r} m\textit{ö}hte wol \textbf{bî} in genesen.\\ 
 & \textbf{doch} bedorfter \textbf{lützel} soldes.\\ 
 & von Arabie des goldes\\ 
 & heter manigen knollen brâht.\\ 
 & liute vinster sô diu naht\\ 
25 & \textbf{wâren} alle die von Zazamanc.\\ 
 & bî den dûhte in diu wîle lanc.\\ 
 & \textbf{doch} hiez er herberge nemen.\\ 
 & des moht ouch si vil wol gezemen,\\ 
 & daz \textbf{si im} die besten \textbf{gâben}.\\ 
30 & die vrouwen dannoch lâgen\\ 
\end{tabular}
\scriptsize
\line(1,0){75} \newline
G O L M Q R W Z Fr29 Fr32 Fr36 Fr71 \newline
\line(1,0){75} \newline
\textbf{1} \textit{Initiale} O M  \textbf{3} \textit{Initiale} Fr29  \textbf{7} \textit{Versal} Fr32  \textbf{15} \textit{Initiale} L Q R W Z Fr32  \textbf{20} \textit{Initiale} G  \textbf{27} \textit{Initiale} Fr71  \newline
\line(1,0){75} \newline
\textbf{1} Wann er nie kunde gewan W  $\cdot$ wan] ÷an O  $\cdot$ ir] er ir O L (R) her M (Fr32)  $\cdot$ kunde] kunigk \textit{nachträglich korrigiert zu:} kuͯnt Q kundig R \textbf{2} Noch keiner seiner schiff man W  $\cdot$ weder er] Er O (M) (Q) (Z) (Fr32) o\textit{m. } L R  $\cdot$ dehein] \textit{om.} M  $\cdot$ sîn] \textit{om.} Q \textbf{3} sînem] sinen O (M) R Z Fr29 Fr32 seinē Q \textbf{4} hieze] hiez O hiesz in M heize Fr32  $\cdot$ Patelamunt] Pentalamvnt L patalamut \textit{nachträglich korrigiert zu:} patalamuͯt Q \textbf{5} daz] Da L  $\cdot$ im] in Q  $\cdot$ minneclîche] ynniclichen M (Fr32) menlichen Q \textbf{6} unde] Si O (L) (M) (Q) (R) (Z) (Fr32) \textbf{7} daz] d:z G  $\cdot$ hülfe] hilffe Q (R)  $\cdot$ es wære in] des were in O (L) (Z) (Fr29) das wer on M (Q) (R) ausser W daz tet in Fr32 \textbf{8} rungen] en rungen O (Q) (R) (Z) (Fr29) (Fr32) ringen W  $\cdot$ niht] nie W \textbf{9} dô] Da Z  $\cdot$ Antschevin] anschevin O [Anshev*]: Anshevin L aschvyn M ansheúin Q aschevin R antscheuin W anshevin Z Fr32 (Fr32) \textbf{10} kumberlîchen] kummerliche M (Q) (Z) \textbf{11} Sinen dienst bot er vmbe guͯt L (W)  $\cdot$ umbe] in \textit{nachträglich korrigiert zu:} umb Q \textbf{12} ouch noch] noch vil diche O (M) (Q) (R) (Z) (Fr29) (Fr32) noch dicke L \textbf{13} daz sim seiten] sy sagen ime W \textbf{14} solte dulten] dulden solde Q solte doln L (Z)  $\cdot$ vînde] vigenden R \textbf{15} dô] Da M Z  $\cdot$ einem] seinen W \textbf{16} Dy sichen vnde die gesunde M \textbf{17} daz im] Das in R Dar inne W  $\cdot$ algemeine] alle gemeyne M gemeine Q \textbf{18} gesteine] edil gesteyne M \textbf{19} solter] sal her M soltt R  $\cdot$ wesen] sein [we*]: weszen Q \textbf{20} Er] Vnd er Z  $\cdot$ möhte] mohte G O (L) (M) (Q) (Z) Fr29 Fr32  $\cdot$ wol bî in] bie mir wol M \textbf{21} doch] Avch O (M) (R) (Fr29) Dorch L  $\cdot$ bedorfter] bedarffte M  $\cdot$ soldes] [goldisz]: soldis M \textbf{22} Arabie] Arabia L (W) arabi Q Z Araby R \textbf{23} heter] Hat er W  $\cdot$ brâht] gebracht M \textbf{24} vinster] winster R  $\cdot$ sô] sam O Q also M (R) (W)  $\cdot$ diu] eyn M \textbf{25} alle] \textit{om.} O Q  $\cdot$ Zazamanc] zazamanch G O L Fr36 Fr71 sasamangk M zazamat Q zasmanac R zazamang W [zazamant]: zazamanc Z \textbf{26} lanc] niht lanch L \textbf{27} doch] Do O L W Fr36 Da M \textbf{28} des] Das Q Do R Sy W  $\cdot$ moht] mochte h M moch Q R moͤcht W torft Fr36  $\cdot$ ouch] \textit{om.} O  $\cdot$ si] des W \textit{om.} Fr32 Fr36  $\cdot$ vil] \textit{om.} L \textbf{29} Daz sie sin schone phlagen L (W)  $\cdot$ daz si im] Das im R (Z) Alz in Fr36 \textbf{30} die] Die \textit{(in neuer Zeile:)} Die O  $\cdot$ dannoch] alle W \newline
\end{minipage}
\hspace{0.5cm}
\begin{minipage}[t]{0.5\linewidth}
\small
\begin{center}*T
\end{center}
\begin{tabular}{rl}
 & wand \textbf{er} ir kunde nie gewan -\\ 
 & er, noch dehein sîn schifman.\\ 
 & si \textbf{tâten} \textbf{sînen} boten kunt,\\ 
 & \textbf{si} \textbf{heize} Patelamunt.\\ 
5 & daz wart \textbf{im} \textbf{minneclîche} enboten.\\ 
 & \textbf{si} \textbf{manten} in bî ir goten,\\ 
 & daz er in hülfe. \textbf{des} wære in nôt.\\ 
 & si rungen niht wan umbe den tôt.\\ 
 & \textbf{Als} der junge Anschevin\\ 
10 & vernam ir kumberlîchen pîn,\\ 
 & er bôt sînen dienst umbe guot,\\ 
 & als \textbf{noch dicke} \textbf{maneger} tuot,\\ 
 & oder daz si\textbf{m} sageten, umbe waz\\ 
 & er solte \textbf{doln} der vîende haz.\\ 
15 & \begin{large}D\end{large}ô sprach ûz einem munde\\ 
 & der sieche und der gesunde,\\ 
 & daz im wære algemeine\\ 
 & ir golt und ir gesteine;\\ 
 & des solter alles hêrre wesen,\\ 
20 & \textbf{er} m\textit{ö}hte wol \textbf{bî} in genesen.\\ 
 & \textbf{ouch} bedorft er \textbf{lützel} soldes.\\ 
 & von Arabie des goldes\\ 
 & het er manegen knollen brâht.\\ 
 & liute vinster sô di\textit{u} naht\\ 
25 & \textbf{wâren} alle die von Zazamanc.\\ 
 & bî den dûht in di\textit{u} wîle lanc.\\ 
 & \textbf{doch} h\textit{ie}z er herberge nemen.\\ 
 & des mohte ouch si vil wol gezemen,\\ 
 & daz \textbf{sim} die besten \textbf{gâben}.\\ 
30 & die vrouwen dannoch lâgen\\ 
\end{tabular}
\scriptsize
\line(1,0){75} \newline
T U V \newline
\line(1,0){75} \newline
\textbf{9} \textit{Majuskel} T  \textbf{15} \textit{Initiale} T  \newline
\line(1,0){75} \newline
\textbf{1} wand er] Wan der U \textbf{3} sînen] sime U \textbf{4} heize] hiezen U hiesse V  $\cdot$ Patelamunt] patelamuͦnt U \textbf{9} Anschevin] Anscevin T Anscheuin V \textbf{10} kumberlîchen] kuͦmerlich U (V) \textbf{11} guot] got U \textbf{12} maneger] ein ritter U V \textbf{14} doln] dulten U (V) \textbf{17} daz im] Dar ime U \textbf{19} alles] allez U \textbf{20} möhte] mohte T U \textbf{24} liute] lith U  $\cdot$ sô] als U sam V  $\cdot$ diu] die T \textbf{25} Zazamanc] zazamang V \textbf{26} diu] die T \textbf{27} hiez] heiz T U  $\cdot$ er] sie U \textbf{28} des mohte ouch si] [*]: Sv́ moͤhte oͮch des V \textbf{29} sim] im U \newline
\end{minipage}
\end{table}
\end{document}
