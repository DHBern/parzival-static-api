\documentclass[8pt,a4paper,notitlepage]{article}
\usepackage{fullpage}
\usepackage{ulem}
\usepackage{xltxtra}
\usepackage{datetime}
\renewcommand{\dateseparator}{.}
\dmyyyydate
\usepackage{fancyhdr}
\usepackage{ifthen}
\pagestyle{fancy}
\fancyhf{}
\renewcommand{\headrulewidth}{0pt}
\fancyfoot[L]{\ifthenelse{\value{page}=1}{\today, \currenttime{} Uhr}{}}
\begin{document}
\begin{table}[ht]
\begin{minipage}[t]{0.5\linewidth}
\small
\begin{center}*D
\end{center}
\begin{tabular}{rl}
\textbf{151} & Iwanet in an der hende zôch\\ 
 & vür eine louben niht ze \textit{h}ô\textit{ch}.\\ 
 & dô sach er vür und wider.\\ 
 & \textbf{ouch was diu loube} sô nider,\\ 
5 & daz er drûffe \textbf{hôrte} unt \textbf{ouch} \textbf{ersach},\\ 
 & dâ von ein trûren im geschach.\\ 
 & \textbf{Dâ} wolt ouch diu künegîn\\ 
 & selbe \textbf{an dem venster} sîn\\ 
 & mit rittern unt mit vrouwen.\\ 
10 & die begunden \textbf{in} alle schouwen.\\ 
 & dâ saz vrou Cunneware,\\ 
 & diu fiere unt diu clâre.\\ 
 & diu \textbf{en}lachete \textbf{decheinen gewîs},\\ 
 & si\textbf{ne} sæhe \textbf{in}, d\textit{er} den hœhsten prîs\\ 
15 & hete oder solte erwerben.\\ 
 & si wolt \textbf{ê sus} ersterben.\\ 
 & allez lachen si vermeit,\\ 
 & \textbf{unz} daz der knappe vür si reit.\\ 
 & dô erlachete ir minneclîcher munt,\\ 
20 & des wart ir rücke ungesunt.\\ 
 & \textbf{\begin{large}D\end{large}ô nam} Keie \textbf{scheneschalt}\\ 
 & vroun Cunnewaren \textbf{de Lalant}\\ 
 & \textbf{mit} ir \textbf{reiden} hâre.\\ 
 & ir lange zöpfe clâre,\\ 
25 & \textbf{die} want er umbe sîne hant,\\ 
 & \textbf{er} \textbf{spancte} si âne türbant.\\ 
 & ir rücke wart \textbf{dechein} eit gestabt,\\ 
 & \textbf{doch} wart ein stab sô dran gehabt,\\ 
 & \textbf{unze} daz sîn siusen gar verswanc.\\ 
30 & \textbf{durch die wât unt} durch \textbf{ir} vel \textbf{ez} dranc.\\ 
\end{tabular}
\scriptsize
\line(1,0){75} \newline
D \newline
\line(1,0){75} \newline
\textbf{7} \textit{Majuskel} D  \textbf{21} \textit{Initiale} D  \newline
\line(1,0){75} \newline
\textbf{1} Iwanet] Jwanet D \textbf{2} ze hôch] zegroz D \textbf{11} Cunneware] Cvnnewâre D \textbf{14} der] [d*]: div D \textbf{21} Keie] keye D \newline
\end{minipage}
\hspace{0.5cm}
\begin{minipage}[t]{0.5\linewidth}
\small
\begin{center}*m
\end{center}
\begin{tabular}{rl}
 & Iwanet in an der hende zôch\\ 
 & vür eine louben niht ze hôch.\\ 
 & dô sach er vür und wider.\\ 
 & \textbf{ouch was diu loube wol} sô nider,\\ 
5 & daz er drûf \textbf{hôrte} und \textbf{sach},\\ 
 & dâ von ein trûren im geschach.\\ 
 & \textbf{dô} wolte ouch diu künigîn\\ 
 & selbe \textbf{an de\textit{n} venstern} sîn\\ 
 & mit rittern und mit vrouwen.\\ 
10 & die begunden \textbf{in} alle schouwen.\\ 
 & d\textit{â} saz vrouwe Cunneware,\\ 
 & diu fiere und diu clâre.\\ 
 & diu \textbf{en}lachete \textbf{niht} \textbf{dekeine wîs},\\ 
 & si sæhe \textbf{in}, der den hôsten prîs\\ 
15 & he\textit{t}e oder solte erwerben.\\ 
 & si wolte ersterben.\\ 
 & allez lachen si vermeit,\\ 
 & \textbf{unz} daz der knappe vür si reit.\\ 
 & dô erlachete ir minneclîcher munt,\\ 
20 & des wart ir rücke ungesunt.\\ 
 & \hspace*{-.7em}\big| vrouwen C\textit{u}n\textit{n}ewaren \textbf{de Lalant},\\ 
 & \hspace*{-.7em}\big| \textbf{si k\textit{o}ste} Keie \textbf{sân zehant}\\ 
 & \textbf{mit} ir \textbf{reiden} hâre.\\ 
 & ir langen zöpfe clâre,\\ 
25 & \textbf{die} want er umb sîne hant,\\ 
 & \textbf{er} \textbf{spante} si âne türbant.\\ 
 & ir rücke wart \textbf{ein} eit gestabet;\\ 
 & \textbf{ez} wart ei\textit{n s}tap sô dran gehabet,\\ 
 & \textbf{unz} daz sîn sûsen gar verswan\textit{c}.\\ 
30 & \textbf{durch die wât und} durch \textbf{ir} vel \textbf{ez} dranc.\\ 
\end{tabular}
\scriptsize
\line(1,0){75} \newline
m n o \newline
\line(1,0){75} \newline
\newline
\line(1,0){75} \newline
\textbf{1} Iwanet] Jwanet m n o \textbf{2} niht] vnd nit n \textbf{4} wol sô] nit zuͦ n o \textbf{6} ein trûren] entruwen n (o)  $\cdot$ geschach] beschach o \textbf{8} selbe] selbes n  $\cdot$ den] dem m \textbf{10} die] \textit{om.} n o \textbf{11} dâ] Do m n o  $\cdot$ Cunneware] cúnnenare n (o) \textbf{13} enlachete] erlacht n erlachte o  $\cdot$ dekeine] do keine n \textbf{14} si sæhe] Sú ensehe n Ensehe o \textbf{15} hete] Herre m n o  $\cdot$ solte] \textit{om.} n \textbf{16} ersterben] E sus ersterben n (o) \textbf{19} erlachete] erlachet n (o) \textbf{22} vrouwen] Frouwe m n (o)  $\cdot$ Cunnewaren] Canmiewaren m commewaren n Comme waren o  $\cdot$ de] die o \textbf{21} koste] kuste m n  $\cdot$ Keie] keye n keige o  $\cdot$ sân] gar o \textbf{24} ir] Jren n  $\cdot$ zöpfe] zopf m (n) (o) \textbf{26} spante] spant o \textbf{28} ein stap] ein eid vnd stab m  $\cdot$ sô] \textit{om.} n (o) \textbf{29} verswanc] verswand m (n) (o) \textbf{30} durch ir] ir n o \newline
\end{minipage}
\end{table}
\newpage
\begin{table}[ht]
\begin{minipage}[t]{0.5\linewidth}
\small
\begin{center}*G
\end{center}
\begin{tabular}{rl}
 & Ywanet in an der hende zôch\\ 
 & vür eine louben niht ze hôch.\\ 
 & dô sach er vür und wider.\\ 
 & \textbf{diu loube, diu was wol} sô nider,\\ 
5 & daz er drûfe \textbf{erhôrte} und \textbf{sach},\\ 
 & dâ von ein trûren im geschach.\\ 
 & \textbf{dâ} wolt ouch diu künigîn\\ 
 & selbe \textbf{in den venstern} sîn\\ 
 & mit rîteren und mit vrouwen.\\ 
10 & die begunden alle schouwen.\\ 
 & dâ saz vrô Kuneware,\\ 
 & diu fier und diu klâre.\\ 
 & diu lachte \textbf{niht} \textbf{neheine wîs},\\ 
 & si\textbf{ne} sæhe \textbf{den}, der den hœhesten brîs\\ 
15 & hete oder solt erwerben.\\ 
 & si wolt \textbf{ê sus} ersterben.\\ 
 & \begin{large}A\end{large}llez lachen si vermeit,\\ 
 & \textbf{biz} \textit{daz} der knappe vür si reit.\\ 
 & dô erlachte ir minniclîcher munt,\\ 
20 & des wart ir rücke ungesunt.\\ 
 & \textbf{dô nam} Kay \textbf{seneschalt}\\ 
 & vrôn Kunewaren \textbf{de Lalant}\\ 
 & \textbf{mit} ir \textbf{reidem} hâre.\\ 
 & ir lange zöpfe clâre\\ 
25 & want er umbe sîne hant\\ 
 & \textbf{unde} \textbf{spante} si âne türbant.\\ 
 & ir rücke wart \textbf{dehein} eit gestabet,\\ 
 & \textbf{doch} wart ein stap sô dran gehabt,\\ 
 & \textbf{biz} \textit{daz} sîn sûsen gar verswanc.\\ 
30 & \textbf{durch die wât und} durch \textbf{ir} vel \textbf{ez} dranc.\\ 
\end{tabular}
\scriptsize
\line(1,0){75} \newline
G I O L M Q R Z Fr65 \newline
\line(1,0){75} \newline
\textbf{11} \textit{Initiale} Q  \textbf{17} \textit{Initiale} G M  \textbf{21} \textit{Initiale} I O L R Z  \newline
\line(1,0){75} \newline
\textbf{1} Ywanet] ẏwanet G Jwanet O L M Nyman Q Jwan R \textbf{2} vür eine] Vor einer L  $\cdot$ niht] div was niht O \textbf{3} dô] da I (O) (M) (R) (Z) \textbf{4} diu] \textit{om.} O L M Q R \textbf{5} erhôrte] hort O (L) Q (R)  $\cdot$ sach] ersach I Z \textbf{6} ein] zuͯ L (Q) (R)  $\cdot$ geschach] gesach Z \textbf{7} dâ] do I (L) (Q) \textbf{8} selbe] Selbin M Selbs Q  $\cdot$ den venstern] dem venster I (L) (M) dem venstern O \textbf{10} alle] in alle O L Q (R) Z alle yn M \textbf{11} dâ] Do Q R  $\cdot$ Kuneware] gunwar I kvnaware O Cvneware L kuͯnware M konware Q Cunneware R \textbf{12} fier] frie R \textbf{13} diu lachte] diu lacht I Die enlachete L (R) (Z) Dine irlachte M Dine lachten Q  $\cdot$ niht] \textit{om.} O L M Q  $\cdot$ neheine wîs] keinen wis Z \textbf{14} sine sæhe] Sy seche denne R  $\cdot$ den der] der I O M in der Z  $\cdot$ hœhesten] besten Q \textbf{16} si] Div O  $\cdot$ ê sus] auch ê I \textbf{18} daz] \textit{om.} G \textbf{19} dô] Da M Z  $\cdot$ erlachte] erlacht I (Q) R (Z)  $\cdot$ minniclîcher] munndiglicher Q \textbf{21} dô] ÷o O Da M  $\cdot$ Kay] kai G Q Gay I keye M key R Z  $\cdot$ seneschalt] thsenethsant I zehant O sinetschant L (R) senectschalt Q sine tschalant Z \textbf{22} vrôn] Vruͯw L (M) (Q) (R)  $\cdot$ Kunewaren] kunwaren G (O) Gunwarn I kvnwaren O kunne warin M kvnnewaren Z  $\cdot$ de] der O  $\cdot$ Lalant] lalalt Q :::t Fr65 \textbf{23} reidem] ræiden O (L) (R) (Z) reitē M (Q) \textbf{24} lange] reiden Z  $\cdot$ zöpfe] zophen I \textbf{25} want er] Wan der L  $\cdot$ hant] hende R \textbf{26} unde] Er O L M Q R Z  $\cdot$ spante si] spantes I (Q) (Z) spengtes O spranct es L spancte si M spant irs R  $\cdot$ türbant] ein tuͤrbant I (O) (Z) ein ende R \textbf{27} ir] vf ir I \textbf{28} ein] eins I  $\cdot$ sô] \textit{om.} M \textbf{29} biz] Eins Q Sus R  $\cdot$ daz] \textit{om.} G  $\cdot$ sûsen] suze M  $\cdot$ verswanc] verswant Q \textbf{30} die] \textit{om.} I O M Q R ir L  $\cdot$ und] \textit{om.} R  $\cdot$ ir vel] fel I (O) (L) (Q) R \textit{om.} M  $\cdot$ ez] er I  $\cdot$ dranc] drant Q \newline
\end{minipage}
\hspace{0.5cm}
\begin{minipage}[t]{0.5\linewidth}
\small
\begin{center}*T (U)
\end{center}
\begin{tabular}{rl}
 & Ywanet in an der hende zôch\\ 
 & vür ein lo\textit{u}ben niht zuo hôch.\\ 
 & dô sach er vür und wider.\\ 
 & \textbf{diu loube was wol} sô nider,\\ 
5 & daz er d\textit{rû}f\textit{e} \textbf{erhôrte} und \textbf{sach},\\ 
 & dâ von ein trûren im geschach.\\ 
 & \textbf{nû} wolt ouch diu künegîn\\ 
 & \textbf{dô} selbe \textbf{in den venstern} sîn\\ 
 & mit rîtern und mit vrouwen.\\ 
10 & die begunden \textbf{in} alle schouwen.\\ 
 & dâ saz \textbf{ouch} vrou Cunneware,\\ 
 & diu fiere und diu clâre.\\ 
 & diu lachete \textbf{dekeine wîs},\\ 
 & si\textbf{n} sæh\textbf{in}, der den hœhesten prîs\\ 
15 & hete oder solte erwerben.\\ 
 & si wolte \textbf{ê sus} ersterben.\\ 
 & allez lachen si vermeit,\\ 
 & \textbf{biz} daz der knappe vür si reit.\\ 
 & dô erlachete ir minneclîcher munt,\\ 
20 & des wart ir rücke ungesunt.\\ 
 & \textbf{dô nam} Key \textbf{scheneschalt}\\ 
 & vroun Cunnewaren \textbf{mit gewalt}\\ 
 & \textbf{bî} ir \textbf{wolgetânen} hâre.\\ 
 & ir la\textit{n}gen zöpfe clâre,\\ 
25 & \textbf{die} want er umb sîn hant,\\ 
 & \textbf{er} \textbf{spante} si âne türbant.\\ 
 & \textbf{ûf} ir rücke wart \textbf{dekein} eit gestabt,\\ 
 & \textbf{doch} wart ein stap sô dran gehabt,\\ 
 & \textbf{biz} daz sîn sûsen gar verswanc,\\ 
30 & \textbf{daz daz bluot} durch \textbf{w\textit{â}t und} \textit{vel} \textbf{ûz} dranc.\\ 
\end{tabular}
\scriptsize
\line(1,0){75} \newline
U V W T \newline
\line(1,0){75} \newline
\textbf{7} \textit{Majuskel} T  \textbf{11} \textit{Majuskel} T  \textbf{21} \textit{Initiale} W   $\cdot$ \textit{Majuskel} T  \newline
\line(1,0){75} \newline
\textbf{1} Ywanet] [Y*]: Ywonet V Jwanet T \textbf{2} louben] [l*]: loben U \textbf{3} dô] da T \textbf{4} sô] \textit{om.} T \textbf{5} drûfe] dorfte U  $\cdot$ erhôrte] horte V (T)  $\cdot$ sach] ersach W \textbf{6} dâ] Do W  $\cdot$ ein] ir T \textbf{7} nû] Da T \textbf{8} dô] da V \textit{om.} T \textbf{11} dâ] do U V W  $\cdot$ ouch] die W \textit{om.} T  $\cdot$ Cunneware] Cuͦmware U kvnneware V kunnewar W kvndeware T \textbf{12} fiere] vrie V \textbf{13} lachete] enlachete W (T) \textbf{14} sin sæhin] Sy ersehe in W siv en sêhe den T \textbf{18} biz] Vnz V (W)  $\cdot$ daz] \textit{om.} W \textbf{19} erlachete] lachete T \textbf{21} Key scheneschalt] keẏn [*]: so zehant V kaytschinet schalt W kêy Senescalt T \textbf{22} vroun] vrowe V  $\cdot$ Cunnewaren] Kuͦmewaren U [*]: kvnneware V kunnebarn W kvnnewaren T  $\cdot$ mit gewalt] [*]: de lalant V \textbf{23} bî] mit T  $\cdot$ wolgetânen] reidem V T raydeletem W \textbf{24} langen] lagen U \textbf{25} die] \textit{om.} T \textbf{26} er spante si âne] er spante sv́ an eine V Er spien sy an ein W vnde spîen si ane T \textbf{27} ûf] \textit{om.} W T  $\cdot$ rücke] rucken W \textbf{28} sô] \textit{om.} T \textbf{30} Durch ir wat vnd fel es dranck W · dvrch wât vnde dvrch vel ez dranc T  $\cdot$ daz daz] daz V  $\cdot$ wât] wuͦt U  $\cdot$ vel] \textit{om.} U \newline
\end{minipage}
\end{table}
\end{document}
