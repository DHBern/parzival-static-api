\documentclass[8pt,a4paper,notitlepage]{article}
\usepackage{fullpage}
\usepackage{ulem}
\usepackage{xltxtra}
\usepackage{datetime}
\renewcommand{\dateseparator}{.}
\dmyyyydate
\usepackage{fancyhdr}
\usepackage{ifthen}
\pagestyle{fancy}
\fancyhf{}
\renewcommand{\headrulewidth}{0pt}
\fancyfoot[L]{\ifthenelse{\value{page}=1}{\today, \currenttime{} Uhr}{}}
\begin{document}
\begin{table}[ht]
\begin{minipage}[t]{0.5\linewidth}
\small
\begin{center}*D
\end{center}
\begin{tabular}{rl}
\textbf{174} & \multicolumn{1}{l}{ - - - }\\ 
 & \textbf{mit} \textbf{schenkelen} \textbf{vliegen} schîne\\ 
 & \textbf{ûf den} poinder \textbf{solde} wenken\\ 
 & unt den schaft ze rehte senken\\ 
5 & \textbf{unt} \textbf{den schilt} gein tjoste vür sich nemen.\\ 
 & er sprach: "des lâzet iuch gezemen."\\ 
 & \begin{large}U\end{large}nvuoge er im sus werte\\ 
 & baz denne ein swankel gerte,\\ 
 & diu argen kinden brichet vel.\\ 
10 & \textbf{dô} hiez er komen ritter snel\\ 
 & gein im durch tjustieren.\\ 
 & er begunde in condwieren\\ 
 & \textbf{einem} ze gegen an den rinc.\\ 
 & dô \textbf{brâhte} der jungelinc\\ 
15 & sîn êrsten tjost durch \textbf{einen} schilt,\\ 
 & \textbf{deis} von in allen wart bevilt\\ 
 & unt daz er hinder\textit{z} ors \textbf{verswanc}\\ 
 & \textbf{einen starken} rîter, niht ze kranc.\\ 
 & \textbf{Ein tjostiure} was komen.\\ 
20 & \textbf{dô} het ouch Parzival genomen\\ 
 & einen starken \textbf{niwen} schaft.\\ 
 & sîn jugent het ellen unt kraft.\\ 
 & der junge süeze âne bart,\\ 
 & \textbf{den} twanc \textbf{diu} Gahmuretes art\\ 
25 & unt angeborniu manheit.\\ 
 & daz ors von rabbîne er reit\\ 
 & mit volleclîcher hurte dar;\\ 
 & er nam der vier nagele war.\\ 
 & des wirtes ritter niht \textbf{gesaz},\\ 
30 & al vallende er den acker maz.\\ 
\end{tabular}
\scriptsize
\line(1,0){75} \newline
D \newline
\line(1,0){75} \newline
\textbf{7} \textit{Initiale} D  \textbf{19} \textit{Majuskel} D  \newline
\line(1,0){75} \newline
\textbf{17} hinderz] hinders D \textbf{24} Gahmuretes] Gahmvrets D \newline
\end{minipage}
\hspace{0.5cm}
\begin{minipage}[t]{0.5\linewidth}
\small
\begin{center}*m
\end{center}
\begin{tabular}{rl}
 & mit sporn gruozes pîne\\ 
 & \textbf{mit} \textbf{schenkele} \textbf{vliegen} schîne\\ 
 & \textbf{ûf dem} poinder \textbf{solte} wenken\\ 
 & und den schaft ze rehte senken\\ 
5 & \textbf{und} \textbf{den schilt} gegen juste vür sich nemen.\\ 
 & er sprach: "des lâzet iuch gezemen."\\ 
 & \begin{large}U\end{large}nvuoge er ime sus werte\\ 
 & baz danne ein swankel g\textit{er}te,\\ 
 & diu argen kinden brichet vel.\\ 
10 & \textbf{dô} hiez er komen ritter snel\\ 
 & gegen ime durch justieren.\\ 
 & er begunde in condewieren\\ 
 & \textbf{einem} ze gegen an den rinc.\\ 
 & dô \textbf{brâhte} der jungelinc\\ 
15 & sîn êrste juste durch \textbf{eine\textit{n}} schilt,\\ 
 & \textbf{daz} von in allen wart bevilt\\ 
 & und daz er hinder \textit{daz} ros \textbf{verswanc}\\ 
 & \textbf{einen stolzen} ritter, niht ze kranc.\\ 
 & \textbf{ein ander justiere} was \textbf{dô} komen.\\ 
20 & \textbf{dô} hete ouch Parcifal genomen\\ 
 & einen starken \textbf{niuwen} schaft.\\ 
 & sîn jugent hete ellen und kraft.\\ 
 & der junge süeze âne bart,\\ 
 & \textbf{den} twanc \textbf{diu} Gahmuretes art\\ 
25 & und angeborniu manheit.\\ 
 & daz ros von rabbîne er reit\\ 
 & mit volleclîcher h\textit{u}rte dar;\\ 
 & er nam der vier nagel war.\\ 
 & des wirtes ritter niht \textbf{gesaz},\\ 
30 & al vallende er den acker maz.\\ 
\end{tabular}
\scriptsize
\line(1,0){75} \newline
m n o Fr69 \newline
\line(1,0){75} \newline
\textbf{5} \textit{Illustration mit Überschrift:} Also parcifal einen ritter vnder das rosz stach vor den frouwen n (o)   $\cdot$ \textit{Initiale} n o  \textbf{7} \textit{Initiale} m  \newline
\line(1,0){75} \newline
\textbf{2} vliegen] fliegende n flegende o \textbf{4} \textit{Versdoppelung (mit Anteil aus Vers 174.3):} Vnd den schaft ze schaftte wencken / Vnd den schaft ze rechte sencken m  \textbf{6} des] das n o \textbf{7} Unvuoge] Vngefúge o \textbf{8} gerte] guͯte m \textbf{10} dô] Das o \textbf{11} durch] \textit{om.} n \textbf{12} begunde] kunde n \textbf{13} den] dem o \textbf{15} einen] einem m \textbf{16} wart] \textit{om.} n \textbf{17} und] \textit{om.} n  $\cdot$ daz ros] ros m \textbf{24} diu] \textit{om.} n o  $\cdot$ Gahmuretes] gahmurettes m gamuretes n gamuͯretes o \textbf{27} volleclîcher] falleklicher o  $\cdot$ hurte] herte m n o \textbf{30} al] Alle n \newline
\end{minipage}
\end{table}
\newpage
\begin{table}[ht]
\begin{minipage}[t]{0.5\linewidth}
\small
\begin{center}*G
\end{center}
\begin{tabular}{rl}
 & mit sporen gruozes pîne\\ 
 & \textbf{nâch} \textbf{schenkelen} \textbf{vliegens} schîne\\ 
 & \textbf{ûz dem} ponder \textbf{solte} wenken\\ 
 & \textit{unde} den schaft ze rehte senken,\\ 
5 & \textbf{den schilt} gein tjoste vür sich nemen.\\ 
 & er sprach: "des lât iuch gezemen."\\ 
 & unvuoger im sus werte\\ 
 & baz danne ein swankel gerte,\\ 
 & diu argen kinden brichet vel.\\ 
10 & \textbf{dô} hiez er komen rîtære snel\\ 
 & gein im durch tjostieren.\\ 
 & er begunde in condewieren\\ 
 & \textbf{einen} ze gegen an den rinc.\\ 
 & dô \textbf{brâhte} der jungelinc\\ 
15 & sîn êrste tjost durch \textbf{einen} schilt,\\ 
 & \textbf{dês} von in allen wart bevilt\\ 
 & unt daz er hinderz ors \textbf{verswanc}\\ 
 & \textbf{einen starken} rîter, niht ze kranc.\\ 
 & \textbf{ein ander tjostiure} was komen.\\ 
20 & \textbf{nû} het ouch Parzival genomen\\ 
 & einen starken \textbf{niwen} schaft.\\ 
 & \begin{large}S\end{large}în jugent het ellen unde kraft.\\ 
 & der junge süeze âne bart,\\ 
 & \textbf{des} twang \textbf{in} Gahmuretes art\\ 
25 & unde angeborniu manheit.\\ 
 & daz ors von rabîne er reit\\ 
 & mit volliclîcher hurte dar;\\ 
 & er nam der vier nagele war.\\ 
 & des wirtes rîter niht \textbf{gesaz},\\ 
30 & al vallende er den acker maz.\\ 
\end{tabular}
\scriptsize
\line(1,0){75} \newline
G I O L M Q R Z Fr21 Fr47 \newline
\line(1,0){75} \newline
\textbf{1} \textit{Initiale} Q  \textbf{7} \textit{Initiale} O L M R Z Fr21  \textbf{11} \textit{Initiale} I  \textbf{15} \textit{Initiale} Fr47  \textbf{22} \textit{Initiale} G  \textbf{29} \textit{Initiale} I  \newline
\line(1,0){75} \newline
\textbf{1} sporen] sporne Q  $\cdot$ gruozes pîne] grosze pine M graffe pine R grozzes bein Fr47 \textbf{2} schenkelen] schenchelns O (Fr21) schenkil Q (R)  $\cdot$ vliegens] vliegen I (O) L (Fr21) fligein Fr47 \textbf{3} ûz] Vf O L (M) (Q) (R) Z Fr21 Fr47  $\cdot$ dem] den M Fr21  $\cdot$ ponder] Punir Fr47  $\cdot$ solte] soltu I \textbf{4} unde] \textit{om.} G \textbf{5} den] Vnd den L (Z) Der Fr21  $\cdot$ gein] gein der I (M) er zer O zer Q Fr21 zv Z  $\cdot$ nemen] solde nemen O (Fr21) \textbf{6} lât] sol O  $\cdot$ gezemen] so Gezemen I avch gezemen O (Fr21) \textbf{7} unvuoger] Vngefuͤge er I (Fr21) ÷ngefvͦge er O ungefuge yr M  $\cdot$ sus] \textit{om.} Q R  $\cdot$ werte] wert Fr47 \textbf{8} swankel] swanke M swanchelt Fr47  $\cdot$ gerte] gert Fr47 \textbf{9} brichet] brichtet M so brichet R  $\cdot$ vel] vil M \textbf{10} dô] Da Z  $\cdot$ er] \textit{om.} Fr21  $\cdot$ komen] chome Fr47  $\cdot$ rîtære] einen ritter L \textbf{12} begunde] begeinde O  $\cdot$ in] ym M (Fr47) o\textit{m. } Q \textbf{13} einen] Einē L M Q Einem Z  $\cdot$ ze gegen] gegen im O zaigen Fr47  $\cdot$ den] dem R \textbf{14} dô] Da O L M Z Fr21 Fr47  $\cdot$ brâhte] brach Q Fr47 \textbf{15} sîn] ÷Ein Fr47  $\cdot$ êrste] ersten L (M) est Fr21 \textbf{16} dês] Der es Fr21  $\cdot$ wart] was Q  $\cdot$ bevilt] gevielt Q \textbf{17} er hinderz] ern hindern Fr47  $\cdot$ verswanc] erswanch O (Fr21) verschwand R swanch Fr47 \textbf{18} starken] \textit{om.} I \textbf{19} ein] E in O  $\cdot$ tjostiure] tiostiern I (L) tyostier O (M) (Q) (R) tiost Fr47 \textbf{20} Parzival] [parziual]: Parziual I parcifal O Z Fr21 parcifale L partzifal Q parczifal R parhcifal Fr47 \textbf{22} Sin ellen hat ivgende vnde kraft Fr21  $\cdot$ Sîn] Eyn M  $\cdot$ jugent] tugent Q  $\cdot$ ellen] eren Q \textbf{24} des twang in] Den twanc Z  $\cdot$ Gahmuretes] Gahmurets G Gamvretes O Gahmuͯretes L gamuͯretis M gahemútes Q Gahmuͦrtes R gamuretes Z Gahmoretes Fr21 Gahmuretez Fr47 \textbf{26} rabîne] rubin I \textbf{27} hurte] horte M \textbf{29} gesaz] ensaz Z \textbf{30} al] Abe M \textit{om.} Fr47  $\cdot$ er den acker] er den ker I er den arker L den angir her M \newline
\end{minipage}
\hspace{0.5cm}
\begin{minipage}[t]{0.5\linewidth}
\small
\begin{center}*T
\end{center}
\begin{tabular}{rl}
 & mit sporn gruozes pîne,\\ 
 & \textbf{nâch} \textbf{schenkeln} \textbf{vliegens} schîne,\\ 
 & "\textbf{in dem} pondier \textbf{sult ir} wenken\\ 
 & unde den schaft ze rehte senken\\ 
5 & \textbf{unde} gegen \textbf{der} tjost vür sich nemen."\\ 
 & er sprach: "des lât iuch gezemen."\\ 
 & unvuoge er im sus werte\\ 
 & baz danne ein swankel gerte,\\ 
 & diu argen kinden brichet vel.\\ 
10 & \textbf{Die} hiez er komen, rîter snel,\\ 
 & gegen im durch tjostieren.\\ 
 & er begundin condewieren\\ 
 & \textbf{einen} ze gegene an den rinc.\\ 
 & dô \textbf{brach} der jungelinc\\ 
15 & sîn êrste tjost durch \textbf{de\textit{n}} schilt,\\ 
 & \textbf{dês} von in allen wart bevilt\\ 
 & unde daz er hinderz ors \textbf{swanc}.\\ 
 & \textbf{Ein starker} rîter, niht ze kranc,\\ 
 & \textbf{zer andern tjost} was komen.\\ 
20 & \textbf{Nû} hete ouch Parcifal genomen\\ 
 & einen starken \textbf{îwînen} schaft.\\ 
 & sîn jugent hete ellen unde kraft.\\ 
 & der junge süeze âne bart,\\ 
 & \textbf{des} twanc \textbf{in} Gahmuretes art\\ 
25 & unde angeborn\textit{iu} manheit.\\ 
 & daz ors von rabîne er reit\\ 
 & mit volleclîcher hurte dar;\\ 
 & er nam der vier nagele war.\\ 
 & Des wirtes rîter niht \textbf{vergaz},\\ 
30 & al vallende er den acker maz.\\ 
\end{tabular}
\scriptsize
\line(1,0){75} \newline
T U V W \newline
\line(1,0){75} \newline
\textbf{7} \textit{Initiale} W  \textbf{10} \textit{Majuskel} T  \textbf{18} \textit{Majuskel} T  \textbf{20} \textit{Majuskel} T  \textbf{29} \textit{Majuskel} T  \newline
\line(1,0){75} \newline
\textbf{1} gruozes] gurte auff den W \textbf{2} nâch] [*]: Mit V  $\cdot$ schenkeln vliegens] schenckels fliegen W \textbf{3} in dem] [J*]: Vf den V  $\cdot$ sult ir] [soͤl*]: solte V solt er W \textbf{4} den] \textit{om.} W \textbf{5} unde gegen der] Vnd den gein der U Vnd den [*]: schilt gegen V Den schilt gen der W \textbf{6} des] daz W  $\cdot$ iuch] îv T  $\cdot$ gezemen] al wol gezemen W \textbf{7} unvuoge] Vngevuͦge U \textbf{8} danne ein] dannen U \textbf{10} Die] [d*]: do V Do W \textbf{11} durch] \textit{om.} W \textbf{12} begundin] begunde im W \textbf{13} einen ze gegene] Eime zuͦ gingen U  $\cdot$ an] auff W \textbf{14} brach] [bra*]: brahte V  $\cdot$ jungelinc] schoͤne iúngeling W \textbf{15} êrste] ersten W  $\cdot$ den] dem T [*]: einen V \textbf{17} hinderz] hinder U [hinder]: hinderz V  $\cdot$ swanc] verswang W \textbf{18} Ein starker] Einen starcken W \textbf{19} Ein ander tyostierer was komen W \textbf{20} Parcifal] parzifal V partziual W \textbf{21} îwînen] nv́wen V (W) \textbf{22} hete ellen unde] hat ellends W \textbf{24} des] Das W  $\cdot$ Gahmuretes] Gahmuͦretes U gamuretes V gamurettes W \textbf{25} unde] Vil W  $\cdot$ angeborniu] angeborne T \textbf{29} vergaz] [*z]: gesas V gesaß W \newline
\end{minipage}
\end{table}
\end{document}
