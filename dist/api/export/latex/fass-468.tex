\documentclass[8pt,a4paper,notitlepage]{article}
\usepackage{fullpage}
\usepackage{ulem}
\usepackage{xltxtra}
\usepackage{datetime}
\renewcommand{\dateseparator}{.}
\dmyyyydate
\usepackage{fancyhdr}
\usepackage{ifthen}
\pagestyle{fancy}
\fancyhf{}
\renewcommand{\headrulewidth}{0pt}
\fancyfoot[L]{\ifthenelse{\value{page}=1}{\today, \currenttime{} Uhr}{}}
\begin{document}
\begin{table}[ht]
\begin{minipage}[t]{0.5\linewidth}
\small
\begin{center}*D
\end{center}
\begin{tabular}{rl}
\textbf{468} & \begin{large}D\end{large}er wirt sprach: "hêrre, ir sprechet wol.\\ 
 & ir sît in \textbf{rehter} kumbers dol,\\ 
 & sît ir nâch iwer selbes wîbe\\ 
 & sorgen pflihte gebt dem lîbe.\\ 
5 & werdet ir \textbf{ervunden} \textbf{an} \textbf{rehter} ê,\\ 
 & iu mac \textbf{zer helle} werden wê,\\ 
 & diu nôt sol schiere ein ende hân\\ 
 & \textbf{unt} werdet von banden \textbf{al dâ} verlân\\ 
 & mit der gotes helfe \textbf{al} sunder twâl.\\ 
10 & ir jeht, ir sent iuch umben Grâl.\\ 
 & ir tumber man, daz muoz ich klagen.\\ 
 & \textbf{jâ}\textbf{ne} mac \textbf{den Grâl niemen} bejagen,\\ 
 & wan der ze himel ist \textbf{sô} bekant,\\ 
 & daz er zem Grâle sî benant.\\ 
15 & \textbf{des} muoz ich vome Grâle jehen.\\ 
 & \textbf{ich weiz ez} und hânz \textbf{vür wâr} gesehen."\\ 
 & Parzival sprach: "wâret ir dâ?"\\ 
 & der wirt sprach gein im: "\textbf{hêrre}, jâ."\\ 
 & Parzival versweic \textbf{in} gar,\\ 
20 & daz \textbf{ouch er} was komen dar.\\ 
 & er vrâgte in von der künde,\\ 
 & wiez umben Grâl \textbf{dâ} stüende.\\ 
 & der wirt sprach: "mir ist \textbf{wol} bekant,\\ 
 & ez wont manec werlîchiu hant\\ 
25 & ze Munsalvæsche bîme Grâl.\\ 
 & durch âventiure die alle mâl\\ 
 & \textbf{riten} manege reise.\\ 
 & \textbf{die} selben templeise,\\ 
 & swâ si kumber ode prîs bejagent,\\ 
30 & vür ir sünde si daz tragent.\\ 
\end{tabular}
\scriptsize
\line(1,0){75} \newline
D \newline
\line(1,0){75} \newline
\textbf{1} \textit{Initiale} D  \newline
\line(1,0){75} \newline
\textbf{17} Parzival] Parcifal D \textbf{19} Parzival] Parcifal D \textbf{25} Munsalvæsche] Mvnsælvæsche D \newline
\end{minipage}
\hspace{0.5cm}
\begin{minipage}[t]{0.5\linewidth}
\small
\begin{center}*m
\end{center}
\begin{tabular}{rl}
 & der wirt sprach: "hêrre, ir sprechet wol.\\ 
 & ir sît in \textbf{rehter} kumbers dol,\\ 
 & sît ir \textit{nâch} iuwer selbes wîbe\\ 
 & sorgen pfliht gebt dem lîbe.\\ 
5 & werdet ir \textbf{vunden} \textbf{in} \textbf{der} ê,\\ 
 & iu mac \textbf{in w\textit{î}ze} werden wê,\\ 
 & diu nôt sol \textit{sch}ier ein ende hân.\\ 
 & \textbf{ir} werdet von \textit{band}e\textit{n} \textbf{gar} verlân\\ 
 & mit der gotes helfe sunder twâl.\\ 
10 & ir jeht, ir sent iuch umb den Grâl.\\ 
 & ir tumb\textit{e}r man, daz muoz ich klagen.\\ 
 & \textbf{j\textit{ô}} mac \textbf{den Grâl niemen} bejagen,\\ 
 & wan der zuo himel ist \textbf{sô} bekant,\\ 
 & daz er zuom Grâl sî benant.\\ 
15 & \textbf{daz} muoz ich von dem Grâl jehen.\\ 
 & \textbf{daz weiz ich} und hab ez \textbf{vür wâr} gesehen."\\ 
 & Parcifal sprach: "wâret ir dâ?"\\ 
 & der wirt sprach gegen im: "jâ."\\ 
 & Parcifal versw\textit{e}ic \textbf{im} gar,\\ 
20 & daz \textbf{ouch er} was komen dar.\\ 
 & er vrâgete in von der künde,\\ 
 & wie ez umb den Grâl \textbf{sô} stüende.\\ 
 & \begin{large}D\end{large}er wirt sprach: "mir ist \textbf{wol} bekant,\\ 
 & ez wont manic werlîchiu hant\\ 
25 & zuo Muntsalvasche bî dem Grâl.\\ 
 & durch âventiure die alle mâl\\ 
 & \textbf{rîtent} manige reise.\\ 
 & \textbf{die} selben templeise,\\ 
 & wâ si kumber oder prîs bejagent,\\ 
30 & vür ir sünde \textit{si daz} tragent.\\ 
\end{tabular}
\scriptsize
\line(1,0){75} \newline
m n o \newline
\line(1,0){75} \newline
\textbf{23} \textit{Initiale} m n o  \newline
\line(1,0){75} \newline
\textbf{1} sprechet] sprechen n \textbf{3} nâch] \textit{om.} m  $\cdot$ wîbe] libe n \textbf{4} gebt dem lîbe] geben uwerm wibe n \textbf{5} der] rechter n o \textbf{6} wîze] wicz m o witze n \textbf{7} nôt] mot o  $\cdot$ sol schier] soldier m \textbf{8} werdet] werden m n o  $\cdot$ banden] libe m  $\cdot$ gar] aldo n [also]: aldo o \textbf{11} tumber] thomar m [*]: thomer o \textbf{12} jô] Jch m Ja o  $\cdot$ den] der o  $\cdot$ Grâl] grole n  $\cdot$ bejagen] [iagen]: beiagen o \textbf{14} Grâl sî] glar wer o \textbf{16} wâr] [vor]: wor o \textbf{17} Parcifal] [Spr]: Sparcifal o  $\cdot$ dâ] do n \textbf{18} im] \textit{om.} o \textbf{19} versweic] verswig m \textbf{20} er] es o \textbf{22} Grâl] grole n  $\cdot$ sô] do n o \textbf{23} Der wirt] EEr o \textbf{25} Muntsalvasche] muͯntsaluasce m muntsaluasce n (o) \textbf{28} templeise] tompleiszen o \textbf{29} kumber] kament o \textbf{30} si daz] das sẏ m \newline
\end{minipage}
\end{table}
\newpage
\begin{table}[ht]
\begin{minipage}[t]{0.5\linewidth}
\small
\begin{center}*G
\end{center}
\begin{tabular}{rl}
 & \begin{large}D\end{large}er wirt sprach: "hêrre, ir sprechet wol.\\ 
 & ir sît in \textbf{rehter} kumbers dol,\\ 
 & sît ir nâch iuwer selbes wîbe\\ 
 & sorgen pflihte gebet dem lîbe.\\ 
5 & werdet ir \textbf{\textit{v}unden} \textbf{an} \textbf{rehter} ê,\\ 
 & iu mac \textbf{ze helle} werden wê,\\ 
 & diu nôt \textit{sol schier} ein ende hân\\ 
 & \textbf{unde} werdet von banden \textbf{al dâ} verlân\\ 
 & mit der gotes helfe \textbf{al} s\textit{un}der twâl.\\ 
10 & ir jehet, ir senet iuch umbe den Grâl.\\ 
 & ir tumber man, daz muoz ich klagen.\\ 
 & \textbf{jâ}\textbf{ne} mac \textbf{den Grâl niemen} bejagen,\\ 
 & wan der ze himel ist \textbf{sô} bekant,\\ 
 & daz er ze dem Grâle sî benant.\\ 
15 & \textbf{des} muo\textit{z} ich von dem Grâle jehen.\\ 
 & \textbf{ich weiz ez} unde hânz \textbf{vür wâr} gesehen."\\ 
 & Parzival sprac\textit{h}: "\textit{w}ârt ir dâ?"\\ 
 & der wirt sprach gein im: "\textbf{hêrre}, jâ."\\ 
 & Parzival versweic \textbf{in} gar,\\ 
20 & daz \textbf{ouch er} was komen dar.\\ 
 & er vrâget in von der künde,\\ 
 & wiez umbe den Grâl stüende.\\ 
 & der wirt sprach: "mir ist \textbf{wol} bekant,\\ 
 & ez wonet manic we\textit{r}lîchiu hant\\ 
25 & ze Muntsalvatsche bîme Grâle.\\ 
 & durch âventiure die alle mâle\\ 
 & \textbf{rîtent} manige reise.\\ 
 & \textbf{die} selben te\textit{m}pleise,\\ 
 & swâ si kumber ode brîs bejagent,\\ 
30 & vür ir sünde si daz tragent.\\ 
\end{tabular}
\scriptsize
\line(1,0){75} \newline
G I O L M Z Fr18 \newline
\line(1,0){75} \newline
\textbf{1} \textit{Initiale} G I O L Z Fr18  \textbf{15} \textit{Initiale} I  \newline
\line(1,0){75} \newline
\textbf{1} Der wirt] Er I ÷er wirt O Der Fr18  $\cdot$ sprach] spral Z  $\cdot$ sprechet] redet I \textbf{2} rehter] rehtes I (M) \textbf{3} iuwer] uwirs M  $\cdot$ selbes] selbe L  $\cdot$ wîbe] [lib]: wibe I [libe]: wibe Z \textbf{4} gebet] gebte O \textbf{5} vunden] sunden G  $\cdot$ rehter] vnrehter Z \textbf{6} mac] muͯsz M (Fr18)  $\cdot$ ze] zuͤ der I (Z)  $\cdot$ werden] [welden]: werden G nit werden L \textbf{7} sol schier] schier sol G  $\cdot$ ein] \textit{om.} I L \textbf{8} von] von den L  $\cdot$ al dâ] da O Fr18 \textit{om.} L  $\cdot$ verlân] lan L \textbf{9} der] \textit{om.} L  $\cdot$ al sunder] als wider G svnder L \textbf{10} jehet] sprecht M  $\cdot$ umbe den] nach dem L \textbf{12} jâne] ion I (M)  $\cdot$ den Grâl niemen] niemen den Gral L \textbf{13} sô] \textit{om.} I wol O L Fr18 \textbf{14} benant] genant O \textbf{15} muoz] muͦze G  $\cdot$ dem] \textit{om.} M  $\cdot$ jehen] sprechin M \textbf{16} ez] \textit{om.} M  $\cdot$ unde hânz vür wâr] vnd hanz I wan ich hanz O Fr18 fvr war vnd hanz Z \textbf{17} Parzival] Parziual G [parzifal]: Parzifal I Parcifal O Z Fr18 Parzifal L M  $\cdot$ sprach wârt] sprach zedem wirt wart G sparch wart I \textbf{18} der wirt] Er O  $\cdot$ gein im] \textit{om.} I  $\cdot$ hêrre] \textit{om.} M \textbf{19} Parzival] Parziual G parzifal I (L) (M) Parcifal O Z Fr18  $\cdot$ versweic] vor Swiget M  $\cdot$ in] im L \textit{om.} M \textbf{20} ouch er] er auch I (O) (L) (M) (Fr18) \textbf{21} vrâget] vragite M \textbf{22} stüende] da stunde I (M) (Z) \textbf{24} werlîchiu] welichiv G [wellichiv]: werlichiv O wertlichir M \textbf{25} ze Muntsalvatsche] zemuntsalfatsch G zemuntsalvasce I Zcu Munsalvatsche M Zv montsalvatsche Z Ze Mvntsalvatske Fr18  $\cdot$ Grâle] gras I \textbf{26} die] [duͯrch]: dyͯer L  $\cdot$ alle] allev I \textbf{28} templeise] tepleise G \textbf{29} swâ] Wa L M  $\cdot$ kumber ode brîs] kunber oder kunber I pris oder kvmber L \textbf{30} daz] dar Fr18 \newline
\end{minipage}
\hspace{0.5cm}
\begin{minipage}[t]{0.5\linewidth}
\small
\begin{center}*T
\end{center}
\begin{tabular}{rl}
 & Der wirt sprach: "hêrre, ir sprechet wol.\\ 
 & ir sît in \textbf{grôzer} kumbers \textit{d}o\textit{l},\\ 
 & sît ir nâch iuwer selbes wîbe\\ 
 & sorgen pflihte gebet dem lîbe.\\ 
5 & werdet ir \textbf{vunden} \textbf{an} \textbf{rehter} ê,\\ 
 & iu mac \textbf{ze helle} werden wê,\\ 
 & diu nôt sol schiere ein ende hân\\ 
 & \textbf{unde} werdet von banden \textbf{dâ} verlân\\ 
 & mit der gotes helfe \textbf{al}sunder twâl.\\ 
10 & ir jeht, ir sent iuch umben Grâl.\\ 
 & ir tumber man, daz muoz ich klagen.\\ 
 & \textbf{jô}\textbf{n} mac \textbf{nieman den Grâl} bejagen,\\ 
 & wander ze himele ist \textbf{wol} bekant,\\ 
 & daz er zem Grâle sî benant.\\ 
15 & \textbf{des} muoz ich vonme Grâle jehen.\\ 
 & \textbf{ich weiz ez} unde hânz gesehen."\\ 
 & Parcifal sprach: "wârt ir dâ?"\\ 
 & Der wirt sprach gegen im: "\textbf{hêrr\textit{e}}, \textit{j}â."\\ 
 & Parcifal versweic \textbf{in} gar,\\ 
20 & Daz \textbf{er ouch} was komen dar.\\ 
 & Er vrâgetin von der künde,\\ 
 & Wiez umbe den Grâl stüende.\\ 
 & \begin{large}D\end{large}er wirt sprach: "\textbf{hêrre}, mir ist bekant,\\ 
 & ez wont manec werlîchiu hant\\ 
25 & ze Munsalvasche bîme Grâl.\\ 
 & durch âventiure di\textit{e} alliu mâl\\ 
 & \textbf{rîtent} manege reise.\\ 
 & \textbf{dise} selben templeise,\\ 
 & swâ si kumber oder prîs bejagent,\\ 
30 & vür ir sünde si daz tragent.\\ 
\end{tabular}
\scriptsize
\line(1,0){75} \newline
T U V W Q R Fr42 \newline
\line(1,0){75} \newline
\textbf{1} \textit{Initiale} W Q Fr42   $\cdot$ \textit{Capitulumzeichen} R   $\cdot$ \textit{Majuskel} T  \textbf{17} \textit{Initiale} W Fr42   $\cdot$ \textit{Capitulumzeichen} R  \textbf{18} \textit{Majuskel} T  \textbf{20} \textit{Majuskel} T  \textbf{21} \textit{Majuskel} T  \textbf{22} \textit{Majuskel} T  \textbf{23} \textit{Initiale} T   $\cdot$ \textit{Capitulumzeichen} R  \newline
\line(1,0){75} \newline
\textbf{1} \textit{Die Verse 453.1-502.30 fehlen} U   $\cdot$ sprach] sagt W \textbf{2} grôzer] rehter V (W) (Q) (R)  $\cdot$ dol] not T \textbf{4} sorgen] Sorge R \textbf{5} vunden] niht fvnden V freűnde Q  $\cdot$ rehter] rechtere Q \textbf{6} werden] nit werden R \textbf{8} dâ] alda V (Q) do W  $\cdot$ verlân] erlon R \textbf{9} der] \textit{om.} W  $\cdot$ alsunder] an sunder Q (R) \textbf{10} iuch] iv T  $\cdot$ umben] nach dem Q \textbf{11} daz] des Q \textbf{12} jôn] Ia W Jane Q Es R  $\cdot$ mac] enmag V R \textbf{13} wol] [*]: so V \textbf{15} des] Das R \textbf{16} hânz gesehen] ha͑ns vor war geseh::: Q habs gesechen R \textbf{17} Parcifal] Parzifal V PArtzifal W Parczifal R P::: Fr42  $\cdot$ dâ] do V W Q \textbf{18} gegen] [*]: gegn T  $\cdot$ hêrre jâ] herre im îa T \textbf{19} Parcifal] Parzifal V Partzifal W Parczifal R  $\cdot$ in] im V W (Q) (R) \textbf{21} Er vrâgetin] [E*]: Er vragete in V Er fragt in W (Q) R  $\cdot$ von] nach Q \textbf{22} stüende] do stv́nde V (Q) \textbf{23} hêrre mir ist bekant] mirst wol bekant V (W) (Q) (R) \textbf{25} Munsalvasche] mvnsalvasce T [munt*]: muntsalvasche V montsaluatschs W múntsaluasche Q Munschaluasche R \textbf{26} die alliu] div alliv T div alle Fr42 \textbf{27} manege] mache Q \textbf{28} dise] Die W Q R (Fr42)  $\cdot$ selben] sellem R \textbf{29} swâ] Wo W (Q) (R)  $\cdot$ oder] vnd Q \newline
\end{minipage}
\end{table}
\end{document}
