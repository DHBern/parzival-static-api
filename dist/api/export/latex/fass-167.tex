\documentclass[8pt,a4paper,notitlepage]{article}
\usepackage{fullpage}
\usepackage{ulem}
\usepackage{xltxtra}
\usepackage{datetime}
\renewcommand{\dateseparator}{.}
\dmyyyydate
\usepackage{fancyhdr}
\usepackage{ifthen}
\pagestyle{fancy}
\fancyhf{}
\renewcommand{\headrulewidth}{0pt}
\fancyfoot[L]{\ifthenelse{\value{page}=1}{\today, \currenttime{} Uhr}{}}
\begin{document}
\begin{table}[ht]
\begin{minipage}[t]{0.5\linewidth}
\small
\begin{center}*D
\end{center}
\begin{tabular}{rl}
\textbf{167} & ine weiz, wer si des bæte,\\ 
 & juncvrouwen \textbf{mit} rîcher wæte\\ 
 & \textbf{unt} an lîbes varwe minneclîch,\\ 
 & die kômen zühte site gelîch.\\ 
5 & si \textbf{truogen} unt strichen schiere\\ 
 & von im sîn amesiere\\ 
 & \begin{large}M\end{large}it blanken, linden henden.\\ 
 & \textbf{jâ}ne dorft in niht ellenden,\\ 
 & der dô \textbf{was witze} ein weise.\\ 
10 & sus dolt er vreude unt eise.\\ 
 & tumpheit er wênic \textbf{gein in} engalt.\\ 
 & \textbf{juncvrouwen} kiusche unde balt\\ 
 & in alsus \textbf{kunrierten}.\\ 
 & \textbf{swâ} von \textbf{si parlierten},\\ 
15 & dâ kunder wol geswîgen zuo.\\ 
 & \textbf{ez} dorft in dunken niht ze vruo,\\ 
 & \textbf{wan} von in schein der ander tac.\\ 
 & der glast alsus in strîte lac,\\ 
 & sîn varwe \textbf{laschte} beidiu lieht.\\ 
20 & des was sîn lîp \textbf{versûmet} nieht.\\ 
 & Man bôt ein badelachen dar,\\ 
 & des nam er \textbf{vil} kleine war.\\ 
 & sus kunder sich \textbf{bî} vrouwen schemen;\\ 
 & vor in \textit{w}olt er\textbf{s} niht umbe nemen.\\ 
25 & juncvrouwen muosen gên,\\ 
 & \textbf{sine torsten} \textbf{dâ niht langer} stên.\\ 
 & \textbf{ich wæne}, \textbf{si} gerne heten gesehen,\\ 
 & ob im dort unde iht wære geschehen.\\ 
 & wîpheit vert mit triwen,\\ 
30 & si kan vriwendes kumber riwen.\\ 
\end{tabular}
\scriptsize
\line(1,0){75} \newline
D \newline
\line(1,0){75} \newline
\textbf{7} \textit{Initiale} D  \textbf{21} \textit{Majuskel} D  \newline
\line(1,0){75} \newline
\textbf{24} wolt] volt D \newline
\end{minipage}
\hspace{0.5cm}
\begin{minipage}[t]{0.5\linewidth}
\small
\begin{center}*m
\end{center}
\begin{tabular}{rl}
 & \textit{in}e weiz, wer si des bæte,\\ 
 & juncvrouwen \textbf{in} rîcher wæte\\ 
 & \textbf{und} an lîbes varwe minneclîch,\\ 
 & die kômen zühte site gelîch.\\ 
5 & si \textbf{twuogen} und strichen schiere\\ 
 & von ime sîn amesiere\\ 
 & mit blanken, linden henden.\\ 
 & \textbf{jô} endorfte in niht ellenden,\\ 
 & der dô \textbf{was witze} ein w\textit{e}ise.\\ 
10 & sus dolt er vröude und eise.\\ 
 & tumpheit er wênic \textbf{gegen in} engalt.\\ 
 & \textbf{juncvrowen} kiusch und balt\\ 
 & in alsus \textbf{kunrierten}.\\ 
 & \textbf{wâ} von \textbf{si parlierten},\\ 
15 & dâ kunde er wol geswîgen zuo.\\ 
 & \textbf{ez} \textbf{en}dorft in dunken niht ze vruo,\\ 
 & \textbf{wanne} von in schein der ander tac.\\ 
 & der glast alsus in strîte lac,\\ 
 & sîn varw\textit{e} \textbf{\textit{l}aste} beidiu lieht.\\ 
20 & des was sîn lîp \textbf{versw\textit{e}inet} niht.\\ 
 & man bôt ein badelachen dar,\\ 
 & des nam er \textbf{h\textit{a}rte} kleine war.\\ 
 & sus kunde er sich \textbf{bî} vrouwen schemen;\\ 
 & vor in wolt er\textbf{s} niht umbe nemen.\\ 
25 & \textbf{die} juncvrouwen muosten gân,\\ 
 & \textbf{si g\textit{e}tor\textit{s}ten} \textbf{d\textit{â} niht lange\textit{r}} \textit{s}tân.\\ 
 & \textbf{ich wæne}, \textbf{si} gerne heten gesehen,\\ 
 & ob im dort unde iht wære geschehen.\\ 
 & wîpheit vert mit triuwen,\\ 
30 & si kan \textbf{wol} vriundes kumber riuwen.\\ 
\end{tabular}
\scriptsize
\line(1,0){75} \newline
m n o Fr69 \newline
\line(1,0){75} \newline
\newline
\line(1,0){75} \newline
\textbf{1} ine] Me m \textbf{2} wæte] wetuͯ o \textbf{4} kômen] kúnnen n  $\cdot$ site gelîch] sitteclich n o \textbf{6} amesiere] amasor o \textbf{8} jô endorfte] Ja endarff o \textbf{9} weise] wise m \textbf{10} dolt] dulte n o \textbf{11} gegen in] \textit{om.} n o \textbf{13} kunrierten] kunrieten n kunreten o \textbf{14} parlierten] pareliertten m paralierten n parielierten o \textbf{16} endorft] durfft n (o) \textbf{19} varwe laste] varwe was vnd laste m farwe leschet n o \textbf{20} des] Das o  $\cdot$ versweinet] verswinet m versmemet o \textbf{22} des] Das o  $\cdot$ nam] [man]: nam o  $\cdot$ er] \textit{om.} Fr69  $\cdot$ harte] herte m  $\cdot$ kleine] kleinv Fr69 \textbf{23} bî] vor n Fr69 \textbf{25} die] Der o  $\cdot$ muosten] muͯsten n o \textbf{26} getorsten] gestortten m  $\cdot$ dâ] do m n o  $\cdot$ langer stân] langer beitten ston m \textbf{27} gerne heten] hetten gern Fr69 \textbf{28} dort unde iht] icht vnd dort n o dort vnder [ich]: icht Fr69  $\cdot$ geschehen] beschehen n o (Fr69) \textbf{29} wîpheit] Jwipheit o \textbf{30} kan] kande n  $\cdot$ vriundes] frúdes n \newline
\end{minipage}
\end{table}
\newpage
\begin{table}[ht]
\begin{minipage}[t]{0.5\linewidth}
\small
\begin{center}*G
\end{center}
\begin{tabular}{rl}
 & \begin{large}I\end{large}ne weiz, wer si des bæte,\\ 
 & juncvrouwen \textbf{in} rîcher wæte,\\ 
 & an lîbes varwe minniclîch,\\ 
 & die kômen zühte site gelîch.\\ 
5 & si \textbf{twuogen} unde strichen schier\\ 
 & von im sîn amesier\\ 
 & mit \textbf{ir} blanken, linden henden.\\ 
 & \textbf{jâ}ne dorfte in niht ellenden,\\ 
 & der dâ \textbf{was witze} ein weise.\\ 
10 & sus dolt er vröude unde eise.\\ 
 & tu\textit{m}pheit er wênic \textbf{gein in} engalt.\\ 
 & \textbf{juncvrouwen} kiusche unde balt\\ 
 & in alsus \textbf{kunrierten}.\\ 
 & \textbf{swâ} von \textbf{si parlierten},\\ 
15 & dâ kunder wol geswîgen zuo.\\ 
 & \textbf{jâ}\textbf{n} dorfte in dunken niht ze vruo,\\ 
 & \textbf{wan} von in schein \textit{der} ander tac.\\ 
 & der glast alsus in strîte lac,\\ 
 & sîn varwe \textbf{laschte} beidiu lieht.\\ 
20 & des was sîn lîp \textbf{versûmet} niht.\\ 
 & man bôt ein badelachen dar,\\ 
 & des nam er \textbf{vil} kleine war.\\ 
 & sus kunder si\textit{ch} \textbf{bî} vrouwen schemen;\\ 
 & vor in wolt er\textbf{z} niht umbe nemen.\\ 
25 & \textbf{die} juncvrouwen muosen gên,\\ 
 & \textbf{si getorsten} \textbf{dâ niht lenger} stên,\\ 
 & \textbf{die}, \textbf{wæne ich}, gerne heten gesehen,\\ 
 & obe im dort unden iht wære geschehen.\\ 
 & wîpheit vert mit triwen,\\ 
30 & si kan vriundes kumber riwen.\\ 
\end{tabular}
\scriptsize
\line(1,0){75} \newline
G I O L M Q R Z Fr17 \newline
\line(1,0){75} \newline
\textbf{1} \textit{Initiale} G Q  \textbf{5} \textit{Initiale} O R Z Fr17  \textbf{11} \textit{Initiale} I L  \textbf{15} \textit{Initiale} M  \textbf{29} \textit{Initiale} I  \newline
\line(1,0){75} \newline
\textbf{1} Ine] Jch O R  $\cdot$ wer] wær O \textbf{2} in] mit R  $\cdot$ rîcher] rechter Q \textbf{3} an] andes I Vnde O Vnde an M (Q) (R) (Z) (Fr17)  $\cdot$ varwe] frawe Q  $\cdot$ minniclîch] wunenklich R \textbf{4} kômen] konigin Q  $\cdot$ zühte] mit zvhten O  $\cdot$ site gelîch] sitechlich O (L) (M) \textbf{5} si] ÷i O Die L ÷i \textit{nachträglich korrigiert zu:} Si Fr17  $\cdot$ twuogen] twungen O (M) (Q) \textbf{7} ir] \textit{om.} O L M Q R Z \textbf{8} jâne dorfte in] in endorfte I Ja dorf in O Jo dorfte in L Jone durfften yn M Jnne bedorfftt R \textbf{9} dâ] do Q  $\cdot$ witze ein] witze vnd L eyn M \textbf{10} unde] \textit{om.} L \textbf{11} Tumpheit ein [weni*]: wenic er engalt Z  $\cdot$ tumpheit] tupheit G  $\cdot$ er wênic] er O Q er niht L (M)  $\cdot$ in] in niht O (Q) im R \textbf{13} in] Jr Z  $\cdot$ kunrierten] kúntirten Q rierten Z \textbf{14} swâ] Wa L Q R  $\cdot$ parlierten] parlieten Z \textbf{15} dâ] Do Q R  $\cdot$ geswîgen] schwigen R \textbf{16} jân dorfte] Ez endorfte L Jn dorfftte R  $\cdot$ in] \textit{om.} M R Z \textbf{17} wan von] von I Wa von O Vor L  $\cdot$ in] im I L Q (R) (Fr17)  $\cdot$ der] ein G \textbf{18} glast] [glast]: gast O  $\cdot$ alsus] sust R \textbf{19} varwe laschte] frawe erlaschte Q  $\cdot$ lieht] lýcht L (M) (Q) \textbf{20} des] Des en M (Q)  $\cdot$ versûmet] versvnnen Z \textbf{21} ein] im I L (R) im ein Fr17 \textbf{23} sich] si G  $\cdot$ bî] vor I L \textbf{24} wolt erz niht] erz niht wolde I \textbf{26} si] sin I (M) (Fr17) Dy Q  $\cdot$ dâ] do O Q \textbf{27} die wæne ich] Jch wene si O (L) (Z) \textbf{28} unden] vnder O (M) vnd hy Q \textbf{29} vert] ver Z \newline
\end{minipage}
\hspace{0.5cm}
\begin{minipage}[t]{0.5\linewidth}
\small
\begin{center}*T
\end{center}
\begin{tabular}{rl}
 & Ine weiz, wer si des bæte,\\ 
 & juncvrouwen \textbf{in} rîcher wæte\\ 
 & \textbf{unde} an lîbes varwe minneclîch,\\ 
 & die kâmen zühte site glîch.\\ 
5 & si \textbf{twuogen} unde strichen schiere\\ 
 & von im sîn amesiere\\ 
 & mit blanken, linden henden.\\ 
 & \textbf{jô}ne dorftin niht ellenden,\\ 
 & der dâ \textbf{witze was} ein weise.\\ 
10 & sus dolter vröude unde eise.\\ 
 & tumpheit er wênic \textbf{dâ} engalt.\\ 
 & \textbf{ein juncvrouwe, beidiu} kiusche unde balt,\\ 
 & in alsus \textbf{condewierte},\\ 
 & \textbf{dâ} von \textbf{er sich parelierte}.\\ 
15 & dâ kunder wol geswîgen zuo.\\ 
 & \textbf{jâ}\textbf{ne} dorfte\textbf{z} in dunken niht ze vruo,\\ 
 & \textbf{daz} von in schein der ander tac.\\ 
 & der glast alsus in strîte lac,\\ 
 & sîn varwe \textbf{gelaste} beidiu lieht.\\ 
20 & des was sîn lîp \textbf{versûmet} nieht.\\ 
 & \begin{large}M\end{large}an bôt ein badelachen dar,\\ 
 & des nam er \textbf{vil} kleine war.\\ 
 & Sus kunder sich \textbf{vor} vrouwen schemen;\\ 
 & vor in wolter\textbf{z} niht umbe nemen.\\ 
25 & \textbf{Die} juncvrouwen muosen gân,\\ 
 & \textbf{die liez er} \textbf{lenger dâ niht} stân.\\ 
 & \textbf{Ich wæne}, \textbf{si} gerne heten gesehen,\\ 
 & ob im dort unden iht wære geschehen.\\ 
 & wîpheit vert mit triuwen,\\ 
30 & si kan vriundes kumber riuwen.\\ 
\end{tabular}
\scriptsize
\line(1,0){75} \newline
T U V W \newline
\line(1,0){75} \newline
\textbf{1} \textit{Majuskel} T  \textbf{21} \textit{Initiale} T U V W  \textbf{23} \textit{Majuskel} T  \textbf{25} \textit{Majuskel} T  \textbf{27} \textit{Majuskel} T  \newline
\line(1,0){75} \newline
\textbf{1} Ine weiz] Jnenweiz V \textbf{4} zühte site glîch] zv́hte sitten glich V mit zucht sittigleich W \textbf{5} twuogen] truͦgen W \textbf{8} jône dorftin] Jagen dorftin U ia endorft V Yoch endorffte W \textbf{9} dâ] do U W  $\cdot$ witze] witzen V \textbf{11} dâ] gein im U (W) gegen in V \textbf{12} ein juncvrouwe] Iunckfrawen W  $\cdot$ beidiu] \textit{om.} U V W \textbf{13} condewierte] kuͦnrierte U [kvnrierten]: kvnrierte V kunduierten W \textbf{14} [d* *n]: Swa von sv́ parlierte V · Do von sy sich parierten W  $\cdot$ parelierte] Parlierte T (U) \textbf{15} dâ] Do U W  $\cdot$ geswîgen] geswingen U \textbf{16} jâne dorftez in] Jagen dorfte sin U Ia es endorffte sy W \textbf{17} daz] [*]: Wan V  $\cdot$ in] im U V W \textbf{18} glast] gelust W \textbf{19} gelaste] laschte U (V) (W) \textbf{20} versûmet] versinnet W \textbf{24} wolterz] wolt er W \textbf{26} lenger dâ niht] langer do niht V nit lenger do W \textbf{27} gerne heten] hetten gerne V \textbf{28} unden] niden V \newline
\end{minipage}
\end{table}
\end{document}
