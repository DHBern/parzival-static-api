\documentclass[8pt,a4paper,notitlepage]{article}
\usepackage{fullpage}
\usepackage{ulem}
\usepackage{xltxtra}
\usepackage{datetime}
\renewcommand{\dateseparator}{.}
\dmyyyydate
\usepackage{fancyhdr}
\usepackage{ifthen}
\pagestyle{fancy}
\fancyhf{}
\renewcommand{\headrulewidth}{0pt}
\fancyfoot[L]{\ifthenelse{\value{page}=1}{\today, \currenttime{} Uhr}{}}
\begin{document}
\begin{table}[ht]
\begin{minipage}[t]{0.5\linewidth}
\small
\begin{center}*D
\end{center}
\begin{tabular}{rl}
\textbf{133} & \textbf{der snüere ein teil was ûz getret}.\\ 
 & dâ \textbf{hete ein knappe} daz gras gewet.\\ 
 & \textit{\begin{large}D\end{large}}er vürste wert unt erkant\\ 
 & sîn wîp dort \textbf{unt} \textbf{al trûric} vant.\\ 
5 & dô sprach der \textbf{stolze} Orilus:\\ 
 & "owê, vrouwe, wie hân ich sus\\ 
 & mîn dienest gein iu gewendet!\\ 
 & mir ist nâch laster geendet\\ 
 & manec \textbf{rîterlîcher} prîs.\\ 
10 & ir habt ein ander âmîs."\\ 
 & \hspace*{-.7em}\big| Mit wazzerrîchen ougen\\ 
 & \hspace*{-.7em}\big| diu vrouwe \textbf{bôt} ir lougen,\\ 
 & sô daz si unschuldic wære.\\ 
 & er\textbf{n} geloubte niht \textbf{ir} mære,\\ 
15 & iedoch sprach si mit vorhte siten:\\ 
 & "dâ kom ein tôre her \textbf{zuo} geriten.\\ 
 & swaz ich liute erkennet hân,\\ 
 & i\textbf{ne} gesach nie lîp sô wolgetân.\\ 
 & mîn vürspan unt \textbf{ein} vingerlîn,\\ 
20 & \textbf{daz} nam er âne den willen mîn."\\ 
 & "\textbf{Hey}, sîn lîp iu wol gevellet!\\ 
 & ir habt iu\textit{ch} zim gesellet."\\ 
 & \textbf{dô sprach si}: "nû\textbf{ne} welle got!\\ 
 & sîniu ribbalîn, sîn gabilôt\\ 
25 & \textbf{wâren mir} doch ze nâhen.\\ 
 & diu rede \textbf{iu solte} \textbf{smâhen}.\\ 
 & vürstinne ez übele \textbf{zæme},\\ 
 & ob si dâ minne næme."\\ 
 & \textbf{Aber sprach} der vürste sân:\\ 
30 & "vrouwe, i\textbf{ne} hân iu niht getân,\\ 
\end{tabular}
\scriptsize
\line(1,0){75} \newline
D \newline
\line(1,0){75} \newline
\textbf{3} \textit{Initiale} D  \textbf{12} \textit{Majuskel} D  \textbf{21} \textit{Majuskel} D  \textbf{29} \textit{Majuskel} D  \newline
\line(1,0){75} \newline
\textbf{3} Der] IeR D \textbf{5} Orilus] Ôrilvs D \textbf{22} iuch] iv D \newline
\end{minipage}
\hspace{0.5cm}
\begin{minipage}[t]{0.5\linewidth}
\small
\begin{center}*m
\end{center}
\begin{tabular}{rl}
 & \textbf{der snüere ein teil was ûz getret},\\ 
 & d\textit{â} \textbf{der knappe hete} daz gras gewet.\\ 
 & der vürste wert und erkant\\ 
 & sîn wîp dort \textbf{inne} \textbf{al trûric} vant.\\ 
5 & dô sprach der \textbf{stolze} Orilus:\\ 
 & "owê, vrouwe, wie hân ich sus\\ 
 & mînen dienst gegen iu gewendet!\\ 
 & mir ist nâch laster geendet\\ 
 & manic \textbf{ritterlîcher} prîs.\\ 
10 & ir habet ein ander âmîs."\\ 
 & diu vrouwe \textbf{leit} ir lougen\\ 
 & mit wazzerrîchen ougen,\\ 
 & sô daz si unschuldic wære.\\ 
 & er \textbf{en}geloubete niht \textbf{ir} mære,\\ 
15 & iedoch sprach si mit vorhten siten:\\ 
 & "d\textit{â} kam ein tôre her geriten.\\ 
 & waz ich liute erkennet hân,\\ 
 & \textit{i}\textbf{\textit{n}e} gesach nie lîp sô wol getân.\\ 
 & mîn vürspange und \textbf{ein} vingerlîn,\\ 
20 & \textbf{daz} nam er ân den willen mîn."\\ 
 & "\textbf{nein}, sîn lîp iu wol gevellet!\\ 
 & ir habet iuch zuo im gesellet."\\ 
 & \textbf{dô sprach si}: "nû \textbf{en}welle got!\\ 
 & sîniu ribbalîn, sîn gabilôt\\ 
25 & \textbf{wâren mir} doch ze nâhen.\\ 
 & diu rede \textbf{iu solte} \textbf{smâhen}."\\ 
 & "vürstinne ez übele \textbf{zæme},\\ 
 & ob si dâ minne næme",\\ 
 & \textbf{\textit{sprach} aber} der vürste sân.\\ 
30 & "vrouwe, i\textbf{ne} hân iu niht getân,\\ 
\end{tabular}
\scriptsize
\line(1,0){75} \newline
m n o \newline
\line(1,0){75} \newline
\newline
\line(1,0){75} \newline
\textbf{2} dâ] Do m n o \textbf{3} erkant] ouch erkant n \textbf{10} habet] hap o  $\cdot$ ander] anderen n (o) \textbf{12} wazzerrîchen] wasserigen n \textbf{13} unschuldic] vnuerschuldig o \textbf{14} engeloubete] engelobette m gloubete n (o)  $\cdot$ ir] der n o \textbf{15} iedoch] Jie doch o  $\cdot$ vorhten] forchtem o \textbf{16} dâ] Do m n o \textbf{18} ine] Me m Jch n o \textbf{19} vürspange] fúrspan n (o) \textbf{20} daz nam er] Nam er mir n Nam er mie o \textbf{21} nein] Wem n o  $\cdot$ iu] ouch n (o) \textbf{23} nû] ẏm o  $\cdot$ enwelle] woͯlle n \textbf{24} ribbalîn] ralin vnd n ribulin o  $\cdot$ gabilôt] babilot n (o) \textbf{25} ze] so n o \textbf{28} dâ] do n o \textbf{29} sprach] \textit{om.} m \textbf{30} ine] ich n o \newline
\end{minipage}
\end{table}
\newpage
\begin{table}[ht]
\begin{minipage}[t]{0.5\linewidth}
\small
\begin{center}*G
\end{center}
\begin{tabular}{rl}
 & \textbf{der snüere ein teil was ûz getreten}.\\ 
 & dâ \textbf{hete ein knappe} daz gras geweten.\\ 
 & der vürste wert und erkant\\ 
 & sîn wîp dort \textbf{al trûric} vant.\\ 
5 & dô sprach der \textbf{herzoge} Orillus:\\ 
 & "owê, vrouwe, wie hân ich sus\\ 
 & mîn dienst gein iu gewendet!\\ 
 & mir ist nâch laster geendet\\ 
 & \textbf{vil} manic \textbf{lobelîcher} brîs.\\ 
10 & ir habt ein ander âmîs."\\ 
 & diu vrouwe \textbf{bôt} ir lougen\\ 
 & mit wazzerrîchen ougen,\\ 
 & sô daz si unschul\textit{di}c wære.\\ 
 & er geloubte niht \textbf{ir} mære,\\ 
15 & iedoch sprach si mit vorhten siten:\\ 
 & "dâ kom ein tôre her \textbf{zuo} geriten.\\ 
 & swaz ic\textit{h} \textit{l}iute erkennet hân,\\ 
 & ich gesach nie lîp sô wolgetân.\\ 
 & mîn vürspan und \textbf{mîn} vingerlîn\\ 
20 & nam er âne den willen mîn."\\ 
 & "\textbf{owê}, sîn lîp iu wol gevellet!\\ 
 & ir habt iuch zim gesellet."\\ 
 & \textbf{\textit{si} sprach}: "nû\textbf{ne} welle got!\\ 
 & sîniu ribbalîn, sîn gabilôt\\ 
25 & \textbf{wâren mir} \textit{d}och ze nâhen.\\ 
 & diu rede \textbf{iu solt} \textbf{versmâhen}.\\ 
 & vürstinne ez übel \textbf{zæme},\\ 
 & op si dâ minne næme."\\ 
 & \textbf{aber sprach} der vürste sân:\\ 
30 & "vrouwe, ich hân iu niht getân,\\ 
\end{tabular}
\scriptsize
\line(1,0){75} \newline
G I O L M Q R Z Fr35 \newline
\line(1,0){75} \newline
\textbf{3} \textit{Initiale} O L R Z  \textbf{5} \textit{Initiale} M Q  \textbf{11} \textit{Initiale} I  \textbf{27} \textit{Initiale} I  \newline
\line(1,0){75} \newline
\textbf{1} ein teil was] ein tail auch was I was eyn teil M  $\cdot$ ûz] \textit{om.} I vͯff L  $\cdot$ getreten] getretet L (Z) gittin M \textbf{2} dâ] Do O Q R  $\cdot$ gras] towe R  $\cdot$ geweten] gewetet L (Z) \textbf{3} der] ÷er O \textbf{4} sîn] Ein L  $\cdot$ dort] aldort M  $\cdot$ al trûric] in altruren I vnder al truͯrich L vnden al trúrig R \textbf{5} dô] Da M Z  $\cdot$ Orillus] [o*]: orilus I orilvs O (M) (Q) (R) (Z) \textbf{6} owê] Awe O Owý L (M) (R) \textbf{7} mîn] Minen L Z \textbf{8} nâch] nahe Q \textbf{9} vil] \textit{om.} O L Q  $\cdot$ lobelîcher] riterlicher O (L) (M) (Q) (R) (Z) \textbf{13} si] \textit{om.} Q  $\cdot$ unschuldic] vnschulch G \textbf{14} er] Er en M (Z)  $\cdot$ geloubte] gelavbt O (Q) (Z)  $\cdot$ niht ir] niht L ir nit R \textbf{15} vorhten] vorchte L (R) (Z) vorchticlichen M \textbf{16} ein tore chom her zuͤ geriten I  $\cdot$ dâ] Do O Q R Ez L  $\cdot$ zuo] \textit{om.} L R \textbf{17} swaz] Waz L (Q) (R)  $\cdot$ ich liute] ich noch lute G \textbf{18} gesach] geshach I en sach M engesach Z  $\cdot$ nie lîp] nyman M \textbf{19} vürspan] fᵫrspang R \textbf{21} owê] hat I Hei O (L) (R) (Z) Ez M Ey Q \textbf{23} si sprach] do sprach G Do sprach si O (Q) (Z) Die vrowe sprach L Da sprach sie M  $\cdot$ nûne welle got] [m]: nein enwelle got L Nune wol gitan M \textbf{24} ribbalîn] Rabalin vnde O ribalin vnd L (R) \textbf{25} doch] iedoch G gar O L Q \textbf{26} iu solt] sol I O solte uͯch L (Q) sollin M \textbf{27} ez übel] daz nih wol I obil vbile M  $\cdot$ zæme] gezême O \textbf{28} dâ] do Q \textbf{30} hân] en han M \newline
\end{minipage}
\hspace{0.5cm}
\begin{minipage}[t]{0.5\linewidth}
\small
\begin{center}*T (U)
\end{center}
\begin{tabular}{rl}
 & \textbf{die z\textit{e}lt\textit{s}n\textit{ü}ere wâren zertret}.\\ 
 & dâ \textbf{hete ein knappe} daz gras gewet.\\ 
 & der vürste wert und erkant,\\ 
 & sîn wîp \textbf{er} dort \textbf{unden} vant.\\ 
5 & dô sprach der \textbf{herzoge} Orilus:\\ 
 & "owê, vrouwe, wie hân ich sus\\ 
 & mîn dienst gein iu gewendet!\\ 
 & mir ist nâch laster geendet\\ 
 & \textbf{vil} manec \textbf{ritterlîcher} prîs.\\ 
10 & ir habt ein ander âmîs."\\ 
 & diu vrouwe \textbf{bôt} ir lougen\\ 
 & mit wazzerrîchen ougen,\\ 
 & sô daz si \textit{u}nschuldic wære.\\ 
 & er geloubete \textbf{ir} niht \textbf{der} mære,\\ 
15 & iedoch sprach \textit{si mit} vorhte siten:\\ 
 & "dâ kam ein tôre her geriten.\\ 
 & waz ich liute erkennet hân,\\ 
 & i\textbf{ne} gesach nie lîp sô wol getân.\\ 
 & mîn vürspan und \textbf{mîn} vingerlîn\\ 
20 & nam er âne den willen mîn."\\ 
 & "\textbf{wie}, sîn lîp iu wol gevellet!\\ 
 & ir habet iuch zuo im gesellet."\\ 
 & \textbf{dô sprach si}: "nû welle got!\\ 
 & sîniu ribbalîn, sî\textit{n} gabilôt\\ 
25 & \textbf{mir wâren} doch zuo nâhen.\\ 
 & diu rede \textbf{solt i\textit{u}} \textbf{versmâhen}.\\ 
 & vürstinne ez übel \textbf{gezæme},\\ 
 & ob si dâ minne næme."\\ 
 & \textbf{aber sprach} der vürste sân:\\ 
30 & "vrouwe, ich hân iu niht getân,\\ 
\end{tabular}
\scriptsize
\line(1,0){75} \newline
U V W T \newline
\line(1,0){75} \newline
\textbf{3} \textit{Initiale} W T  \textbf{5} \textit{Majuskel} T  \textbf{11} \textit{Majuskel} T  \textbf{21} \textit{Majuskel} T  \textbf{23} \textit{Majuskel} T  \textbf{29} \textit{Initiale} T  \newline
\line(1,0){75} \newline
\textbf{1} der snvere ein teil was vertren T  $\cdot$ zeltsnüere] zilt sin ere U gezelt snvͤre V \textbf{2} dâ] Do W  $\cdot$ gewet] gewehten T \textbf{3} wert] riche T \textbf{4} dort unden] dort altrurig V do vnden traurig W dort vnder trvrec T \textbf{5} Orilus] Oriluͦs U \textbf{6} owê] wafen T \textbf{7} mîn] Minen V \textbf{8} geendet] gesendet V \textbf{9} [*il *g*]: Jr hant verkrenket ritters pris V \textbf{10} ein ander] einen andern V \textbf{13} si unschuldic] sin schuldic U \textbf{14} er] er en T  $\cdot$ geloubete] geloͮbet V (W) (T) \textbf{15} si mit vorhte] vochte U si mit vorhten V (W) \textbf{16} dâ] Dort V Do W ez T  $\cdot$ her] her ein W her zvͦ T \textbf{17} waz] Swaz V (T) \textbf{18} ine] Jch V \textbf{19} mîn vürspan] Minen fv́rspan V \textbf{20} âne] v́ber V (T) mir úber W \textbf{21} wie] Ei V Ahi W Hei T \textbf{23} dô sprach si] Da sprach sy W Div vrôuwe sprach T  $\cdot$ nû welle] nun enwell W (T) \textbf{24} sîn gabilôt] si nuͦ [ge*]: gebelot U sine gabelot V \textbf{25} mir wâren doch zuo] Mir [wor*]: worent doch so V Warn mir doch zuͦ W wâren mir ze T \textbf{26} iu] iz U ir W \textbf{28} dâ] do W sine T \textbf{29} aber sprach] Do sprach aber W \newline
\end{minipage}
\end{table}
\end{document}
