\documentclass[8pt,a4paper,notitlepage]{article}
\usepackage{fullpage}
\usepackage{ulem}
\usepackage{xltxtra}
\usepackage{datetime}
\renewcommand{\dateseparator}{.}
\dmyyyydate
\usepackage{fancyhdr}
\usepackage{ifthen}
\pagestyle{fancy}
\fancyhf{}
\renewcommand{\headrulewidth}{0pt}
\fancyfoot[L]{\ifthenelse{\value{page}=1}{\today, \currenttime{} Uhr}{}}
\begin{document}
\begin{table}[ht]
\begin{minipage}[t]{0.5\linewidth}
\small
\begin{center}*D
\end{center}
\begin{tabular}{rl}
\textbf{353} & \begin{large}W\end{large}az welt ir, daz si \textbf{mêr nû} tuon?\\ 
 & \textbf{wan} dô erbeizte \textbf{der} \textbf{künec} Lotes sun,\\ 
 & al dâ er den besten schaten vant.\\ 
 & sîn kamerære truoc dar zehant\\ 
5 & einen kulter unt ein matraz,\\ 
 & dâr ûf der stolze werde saz.\\ 
 & ob im \textbf{saz} \textbf{wîbe} hers ein vluot.\\ 
 & sîn kamergewant man nider luot\\ 
 & untz harnasch von den soumen.\\ 
10 & hin dan under \textbf{den} andern boumen\\ 
 & herberge nâmen sie,\\ 
 & knappen, die \textbf{dâ} kômen hie.\\ 
 & Diu alte herzogîn sprach sân:\\ 
 & "tohter, welch koufman\\ 
15 & kunde alsus gebâren?\\ 
 & dû\textbf{ne} solt sîn \textbf{sus} niht vâren."\\ 
 & dô sprach diu junge Obilot:\\ 
 & "unvuoge ir \textbf{dennoch} mêr gebôt:\\ 
 & gein dem künege Meljanze von Liz\\ 
20 & si kêrte \textbf{ir hôchverte} vlîz,\\ 
 & dô er si bat ir minne.\\ 
 & geunêrt sîn sölhe sinne!"\\ 
 & dô sprach Obie,\\ 
 & vor zorne niht diu vrîe:\\ 
25 & "\textbf{sîn vuore ist mir} unmære.\\ 
 & dort sitzet ein wechselære,\\ 
 & \textbf{des} market \textbf{muoz} \textbf{hie} werden guot.\\ 
 & sîne soumschrîn sint sô behuot,\\ 
 & dînes ritters, \textbf{tœrschiu} swester mîn,\\ 
30 & er wil ir \textbf{selbe goumel} sîn."\\ 
\end{tabular}
\scriptsize
\line(1,0){75} \newline
D \newline
\line(1,0){75} \newline
\textbf{1} \textit{Initiale} D  \textbf{13} \textit{Majuskel} D  \newline
\line(1,0){75} \newline
\textbf{2} Lotes] Lôts D \textbf{17} Obilot] Obylot D \textbf{19} Meljanze] Melianze D  $\cdot$ Liz] Lŷz D \textbf{23} Obie] Obîe D \newline
\end{minipage}
\hspace{0.5cm}
\begin{minipage}[t]{0.5\linewidth}
\small
\begin{center}*m
\end{center}
\begin{tabular}{rl}
 & waz welt ir, \textit{d}az si \textbf{mêr nû} tuon?\\ 
 & \textbf{wanne} dô erbeizete \textbf{künic} Lotes sun,\\ 
 & aldâ er den beste\textit{n} schate vant.\\ 
 & sîn kamere\textit{r} truoc dar zehant\\ 
5 & einen kulter und ein matraz,\\ 
 & dâr ûf der stolze werde saz.\\ 
 & ob ime \textbf{saz} \textbf{wîbes} hers ein vluot.\\ 
 & sîn kamergewant man nider luot\\ 
 & untz harnasch von den \dag sunen\dag .\\ 
10 & hin dan under \textbf{den} andern \dag bumen\dag \\ 
 & herberge nâmen sie,\\ 
 & knappen, die kômen hie.\\ 
 & diu alte herzogîn sprach sân:\\ 
 & "tohter, welich koufman\\ 
15 & kunde alsus gebâren?\\ 
 & dû \textbf{en}solt sîn \textbf{sust} niht vâren."\\ 
 & dô sprach diu junge Obilot:\\ 
 & "unvuoge ir \textbf{noch} mêr gebôt:\\ 
 & gegen dem künig\textit{e} \textit{M}el\textit{i}anze von Liz\\ 
20 & si kêrte \textbf{ûf hôchvart ir} vlîz,\\ 
 & dô er si bat ir minne.\\ 
 & geunêret sîn solhe sinne!"\\ 
 & dô sprach \textbf{diu maget} Obie,\\ 
 & vor zorne niht diu vrîe:\\ 
25 & "\textbf{sîn vuore ist mir} unmære.\\ 
 & dort sitzet ein wehselære,\\ 
 & \textbf{des} market \textbf{muoz} \textbf{hie} werden guot.\\ 
 & sîne soumschrîn sint sô behuot,\\ 
 & dînes ritters, \textbf{tœrschiu} swester mîn,\\ 
30 & er wil ir \textbf{selbe goumel} sîn."\\ 
\end{tabular}
\scriptsize
\line(1,0){75} \newline
m n o \newline
\line(1,0){75} \newline
\newline
\line(1,0){75} \newline
\textbf{1} daz] was m  $\cdot$ mêr nû] nuͯ me n \textbf{2} Lotes] lots m [luche]: luchs n luckes o \textbf{3} aldâ] Do n o  $\cdot$ besten] beste m  $\cdot$ schate] schatten n o \textbf{4} kamerer] kamere m  $\cdot$ truoc] truͯg nuͯ n \textbf{5} einen] Ein n (o)  $\cdot$ kulter] kalter o \textbf{8} luot] sluͯg n (o) \textbf{9} sunen] sunnen n (o) \textbf{10} bumen] bronnen n (o) \textbf{12} die] \textit{om.} n o  $\cdot$ hie] sie o \textbf{15} alsus] alsos o \textbf{16} dû] Do n  $\cdot$ ensolt] solt n o  $\cdot$ sust] \textit{om.} n so o  $\cdot$ niht] in o \textbf{17} Obilot] abilot o \textbf{18} unvuoge] Vngefuͯge o \textbf{19} künige Melianze] kunige von meleanze m kv́nige meliantz n konige meliancz o  $\cdot$ von] [vol]: von m  $\cdot$ Liz] lisz n lis o \textbf{24} vor] Von n o \textbf{28} soumschrîn] suͯn schrin n  $\cdot$ sint] sin o \textbf{29} tœrschiu] túrste n (o) \textbf{30} goumel] gemmel n \newline
\end{minipage}
\end{table}
\newpage
\begin{table}[ht]
\begin{minipage}[t]{0.5\linewidth}
\small
\begin{center}*G
\end{center}
\begin{tabular}{rl}
 & waz welt ir, daz si \textbf{mêr} tuon?\\ 
 & dô erbeizte \textbf{des} \textbf{künic} Lotes sun,\\ 
 & al dâ er den besten schate vant.\\ 
 & sîn kamerære truoc dar zehant\\ 
5 & einen gulter unde ein matraz,\\ 
 & dâr ûf der stolze werde saz.\\ 
 & obe im \textbf{was} \textbf{wîbe} hers ein vluot.\\ 
 & sîn kamergewant man nider luot\\ 
 & unt daz harnasch von den soumen.\\ 
10 & hin dan under anderen boumen\\ 
 & \begin{large}H\end{large}erberge nâmen sie,\\ 
 & knappen, die \textbf{dâ} kômen hie.\\ 
 & diu alt herzogîn sprach sân:\\ 
 & "tohter, welch koufman\\ 
15 & kunde alsus gebâren?\\ 
 & dû\textbf{ne} solt sîn \textbf{sus} niht vâren."\\ 
 & dô sprach diu junge Obilot:\\ 
 & "unvuoge ir \textbf{dannoch} mê gebôt:\\ 
 & gein dem künige Melianze von Liz\\ 
20 & si kêrte \textbf{ir hôchverte} vlîz,\\ 
 & dô er si bat ir minne.\\ 
 & geu\textit{n}êrt sîn solhe sinne!"\\ 
 & dô sprach Obie,\\ 
 & vor zorne niht diu vrîe:\\ 
25 & "\textbf{mir ist sîn vuore} unmære.\\ 
 & dort sitzet ein wehselære;\\ 
 & \textbf{sîn} market \textbf{mac} \textbf{hie} werden guot.\\ 
 & sîniu soumschrîn sint sô behuot,\\ 
 & dînes rîters, \textbf{tœrschiu} swester mîn,\\ 
30 & er wil ir \textbf{selbe goumel} sîn."\\ 
\end{tabular}
\scriptsize
\line(1,0){75} \newline
G I O L M Q R Z Fr39 \newline
\line(1,0){75} \newline
\textbf{1} \textit{Initiale} I O Z Fr39   $\cdot$ \textit{Capitulumzeichen} R  \textbf{11} \textit{Initiale} G  \textbf{13} \textit{Initiale} I M  \newline
\line(1,0){75} \newline
\textbf{1} waz] ÷Az O  $\cdot$ daz] mer das Q  $\cdot$ si] \textit{om.} L  $\cdot$ mêr] nv mere O mir nv L mer Nu M (R) (Z) (Fr39) nu Q \textbf{2} dô] Da R Z  $\cdot$ erbeizte] erbeiszet L (Z) (Fr39)  $\cdot$ des künic Lotes sun] [gahmurests sun]: des chunc lotes sun G des chvniges lotes svͦn O (Q) (R) des konniges lotis son M \textbf{3} al] \textit{om.} I L Fr39  $\cdot$ dâ] da da Q Do Fr39  $\cdot$ den besten] \textit{om.} Q  $\cdot$ schate] schaten L Q (R) Z Fr39 schacz M \textbf{4} dar] im dar I das Q \textbf{5} einen] ein I (R) (Z) Seinen Q Eine Fr39  $\cdot$ ein] einen O (Q) eine L Fr39 \textbf{6} stolze werde] werde stolze O \textbf{7} wîbe] wibes O (Q)  $\cdot$ hers] her O L M Q Fr39 \textbf{8} kamergewant] kamerer gewant I (R)  $\cdot$ nider] abe O \textbf{9} von] ab I  $\cdot$ den] dem Q  $\cdot$ soumen] somern R \textbf{10} under] vndern Z  $\cdot$ anderen] ander O Q \textbf{11} \textit{statt 353.11-16 (Versfolge 353.13-14-11-12; Versdoppelung 353.13-14 [mit Anteil aus 353.15] nach 353.12):} Du alte herczogin sprach san / Dochter welich koffman / Herberg sy nemen / Knappen die da komen / Die herczogin sprach do / Welher koffman kan gebaren so R   $\cdot$ nâmen sie] sie namen Q \textbf{12} knappen] manc cnappe I  $\cdot$ dâ] do Q Fr39  $\cdot$ hie] \textit{om.} Q \textbf{14} tohter] Ttochir M  $\cdot$ welch] welche Fr39 \textbf{15} alsus] also M \textbf{16} dûne] Dv O  $\cdot$ sus niht] niht O niht so L Fr39 so nicht M Q (Z)  $\cdot$ vâren] gefaren L \textbf{17} dô] \textit{om.} L M Die Q [D*]: Da Z ::: Fr39 \textbf{18} unvuoge ir] vngefuͦge I Vnfuͯge er L  $\cdot$ gebôt] bot O \textbf{19} dem] den R  $\cdot$ Melianze] Melianz O (L) (Z) (Fr39)  $\cdot$ von] de I  $\cdot$ Liz] lisz M Q lis R \textbf{20} kêrte] chert I  $\cdot$ ir hôchverte] inhohverte O irn hoffertigen R \textbf{21} dô] Da M Z \textbf{22} geunêrt] gevngert G  $\cdot$ sîn solhe] sein selbe Q sig solich R \textbf{23} dô] Da M Z  $\cdot$ Obie] obŷe O oblye Q obye R \textbf{25} mir] Min M  $\cdot$ vuore] \textit{om.} L ::: Fr39 \textbf{26} sitzet] sicz R  $\cdot$ wehselære] wechsebere M weschelere Q \textbf{27} sîn] Des O L M Q R Fr39 Der Z  $\cdot$ market] markt der Z \textbf{28} sîniu] Sine R  $\cdot$ sint] sin L (Q)  $\cdot$ sô] so wol I  $\cdot$ behuot] wol behuͤt I guͦt R \textbf{29} tœrschiu] troste R \textbf{30} er] Jr Z  $\cdot$ selbe] selber I \newline
\end{minipage}
\hspace{0.5cm}
\begin{minipage}[t]{0.5\linewidth}
\small
\begin{center}*T
\end{center}
\begin{tabular}{rl}
 & Waz welt ir, daz si \textbf{nû mêre} tuon?\\ 
 & dô erbeizete \textbf{des} \textbf{küneges} Lotes suon,\\ 
 & aldâ er den beste\textit{n} schate vant.\\ 
 & Sîn kamerære truoc dar zehant\\ 
5 & einen kulter unde ein matraz,\\ 
 & dâr ûfe der stolze werde saz.\\ 
 & ob im \textbf{was} \textbf{wîbes} hers ein vluot.\\ 
 & sîn kamergewant man nider luot\\ 
 & unde daz harnasch von den soumen.\\ 
10 & hin dan under andern boumen\\ 
 & herberge nâmen sie,\\ 
 & knappen, die \textbf{dâ} kômen hie.\\ 
 & Diu alte herzogîn sprach sân:\\ 
 & "tohter, welch koufman\\ 
15 & kunde alsus gebâren?\\ 
 & dû solt sîn \textbf{sô} niht vâren."\\ 
 & \begin{large}D\end{large}ô sprach diu junge Obylot:\\ 
 & "unvuoge ir \textbf{dannoch} mêr gebôt:\\ 
 & gegen dem künege Melyanze von Liz\\ 
20 & si kêrte \textbf{ir hôchverte} vlîz,\\ 
 & dô er si bat ir minne.\\ 
 & geunêrt sîn solhe sinne!"\\ 
 & Dô sprach Obie,\\ 
 & vor zorne niht diu vrîe:\\ 
25 & "\textbf{mir ist sîn vuore} unmære.\\ 
 & dort sitzet ein wehselæ\textit{r}e,\\ 
 & \textbf{des} market \textbf{mac} werden guot.\\ 
 & sîne soumschrîne sint sô behuot,\\ 
 & dînes rîters, \textbf{liebiu} swester mîn,\\ 
30 & er wil ir \textbf{goumel selbe} sîn."\\ 
\end{tabular}
\scriptsize
\line(1,0){75} \newline
T V W \newline
\line(1,0){75} \newline
\textbf{1} \textit{Majuskel} T  \textbf{4} \textit{Majuskel} T  \textbf{13} \textit{Majuskel} T  \textbf{17} \textit{Initiale} T W  \textbf{23} \textit{Majuskel} T  \newline
\line(1,0){75} \newline
\textbf{2} Lotes] lottes W \textbf{3} besten] beste T \textbf{5} einen] Ein W \textbf{7} was wîbes hers] sas wibe hers V was weibes her W \textbf{8} sîn kamergewant] Seins kameres gewant W \textbf{9} Vnd seinen harnaß von seinem soume W \textbf{10} andern] dem W \textbf{12} dâ] do V W \textbf{13} sprach] \textit{om.} W \textbf{14} tohter] Sprach tochter W \textbf{15} gebâren] wol gebaren W \textbf{17} diu] der W  $\cdot$ Obylot] Obylôt T obilot V abilot W \textbf{19} dem] einem W  $\cdot$ Melyanze] melianze V melians W  $\cdot$ Liz] lŷz T lys V liß W \textbf{20} ir hôchverte] vf hochvart [*]: iren V \textbf{22} sîn] sint V \textbf{23} sprach] sprach die maget V sprach aber W  $\cdot$ Obie] Obye T \textbf{24} vor] von W \textbf{26} wehselære] wehselele T \textbf{27} des market mac] Dez merkert muͦz hie V Der mercket mag hie W \textbf{28} sint] sein W \textbf{29} liebiu] toͤrsche V (W) \textbf{30} ir goumel selbe] selber ir goumer W \newline
\end{minipage}
\end{table}
\end{document}
