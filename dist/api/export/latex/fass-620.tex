\documentclass[8pt,a4paper,notitlepage]{article}
\usepackage{fullpage}
\usepackage{ulem}
\usepackage{xltxtra}
\usepackage{datetime}
\renewcommand{\dateseparator}{.}
\dmyyyydate
\usepackage{fancyhdr}
\usepackage{ifthen}
\pagestyle{fancy}
\fancyhf{}
\renewcommand{\headrulewidth}{0pt}
\fancyfoot[L]{\ifthenelse{\value{page}=1}{\today, \currenttime{} Uhr}{}}
\begin{document}
\begin{table}[ht]
\begin{minipage}[t]{0.5\linewidth}
\small
\begin{center}*D
\end{center}
\begin{tabular}{rl}
\textbf{620} & \begin{large}D\end{large}ô sprach er: "vrouwe, tuot sô wol,\\ 
 & ob ich iuch \textbf{des} bitten sol:\\ 
 & lât mînen namen unerkant,\\ 
 & als mich der rîter hât genant,\\ 
5 & der mir \textbf{entreit} Gringuljeten.\\ 
 & leistet, des ich iuch hân gebeten.\\ 
 & Swer iuch des vrâgen welle,\\ 
 & sô sprechet ir: 'mîn geselle\\ 
 & ist mir des unerkennet,\\ 
10 & er wart mir nie genennet.'"\\ 
 & si sprach: "vil gern ich \textbf{si} \textbf{ez} verdage,\\ 
 & sît ir niht welt, daz ichz \textbf{in} sage."\\ 
 & Er unt diu vrouwe wol gevar\\ 
 & kêrten gein der bürge dar.\\ 
15 & \textbf{die rîter heten} dâ vernomen,\\ 
 & daz \textbf{dar} ein rîter wære komen,\\ 
 & der hete die âventiur erliten\\ 
 & unt den lewen überstriten\\ 
 & unt den Turkoten sider\\ 
20 & \textbf{ze} rehter tjost gevellet nider.\\ 
 & Innen des reit Gawan\\ 
 & gein dem urvar ûf den plân,\\ 
 & daz si in von zinnen sâhen.\\ 
 & si begunden vaste gâhen\\ 
25 & ûz der burc mit schalle.\\ 
 & dô vuorten si alle\\ 
 & rîche baniere;\\ 
 & sus kômen si schiere\\ 
 & ûf snellen ravîten.\\ 
30 & er wânde, si wolden strîten.\\ 
\end{tabular}
\scriptsize
\line(1,0){75} \newline
D Z Fr68 \newline
\line(1,0){75} \newline
\textbf{1} \textit{Initiale} D Z Fr68  \textbf{7} \textit{Majuskel} D  \textbf{13} \textit{Majuskel} D  \textbf{21} \textit{Majuskel} D  \newline
\line(1,0){75} \newline
\textbf{1} Dô] Da Z \textbf{5} Gringuljeten] Gringvlieten D kringulieten Z gwingulieten Fr68 \textbf{8} ir] \textit{om.} Fr68 \textbf{10} wart] enwart Z \textbf{11} vil gern ich si] ich vil gerne Z vil gerne ih Fr68 \textbf{12} in] \textit{om.} Z Fr68 \textbf{13} gevar] gevare Fr68 \textbf{15} nv hant di ritter alda vernumen Fr68 \textbf{16} ein rîter] \textit{om.} Fr68 \textbf{19} Turkoten] Tvrkoiten Z (Fr68) \textbf{23} daz] da Fr68  $\cdot$ von] \textit{om.} Z vo: Fr68  $\cdot$ zinnen] d:::nen Fr68 \textbf{24} do begund:::ste gahen Fr68 \textbf{26} dô] Da Z vnd Fr68  $\cdot$ si] mit in Fr68 \newline
\end{minipage}
\hspace{0.5cm}
\begin{minipage}[t]{0.5\linewidth}
\small
\begin{center}*m
\end{center}
\begin{tabular}{rl}
 & dô sprach er: "vrowe, tuot sô wol,\\ 
 & ob ich iuch bitten sol:\\ 
 & lât mî\textit{n}en name\textit{n} unerkant,\\ 
 & als mich der ritter hâ\textit{t} \textit{g}enant,\\ 
5 & der mir \textbf{entret} Gringuleten.\\ 
 & leistet, des ich iuch hân gebeten.\\ 
 & wer iuch des vrâgen welle,\\ 
 & sô sprecht ir: 'mîn geselle\\ 
 & ist mir des unerkennet,\\ 
10 & er wart mir nie genennet.'"\\ 
 & si sprach: "vil gerne ich verdage,\\ 
 & sît ir niht wellet, daz ich\textit{z} \textbf{in} sage."\\ 
 & er und diu vrowe wol gevar\\ 
 & kêrten gegen de\textit{r b}ürge dar.\\ 
15 & \textbf{jâ heten die ritter} d\textit{â} vernomen,\\ 
 & daz ein ritter wær komen,\\ 
 & der het die âventiure erliten\\ 
 & und den lewen überstriten\\ 
 & und \textbf{ouch} den Turkoiten sider\\ 
20 & \textbf{in} rehter juste gevellet nider.\\ 
 & innen des reit \textit{G}awan\\ 
 & geg\textit{e}n dem urvar ûf den plân,\\ 
 & daz si in von \textbf{den} z\textit{i}nnen sâhen.\\ 
 & si begunden vaste gâhen\\ 
25 & ûz der burc mit schalle.\\ 
 & dô vuorten si alle\\ 
 & rîche banier;\\ 
 & sus kômen si schier\\ 
 & ûf snellen ravîten.\\ 
30 & er wânde, si wolten strîten.\\ 
\end{tabular}
\scriptsize
\line(1,0){75} \newline
m n o \newline
\line(1,0){75} \newline
\newline
\line(1,0){75} \newline
\textbf{2} bitten] des bitten n o \textbf{3} mînen namen] mẏnnen namer m \textbf{4} hât genant] hat erkant vnd genant m \textbf{5} Gringuleten] gringuletten m \textbf{7} des] daz o \textbf{8} sprecht ir mîn] sprach ir mẏnne o \textbf{9} des] das o \textbf{11} gerne] gernen n  $\cdot$ ich] iches o \textbf{12} niht] \textit{om.} n  $\cdot$ ichz] ich m o  $\cdot$ in] uͯch o \textbf{14} kêrten] Die kertent n  $\cdot$ der bürge] der kuͯmien burge m \textbf{15} jâ] Jo n :: o  $\cdot$ dâ] do m n o \textbf{16} wær] dar were n (o) \textbf{19} Turkoiten] turkoitten m korkoiten n cuͯrtoiten o \textbf{21} Gawan] hegawan m \textbf{22} gegen] Gegegen m \textbf{23} zinnen] zunnen m \textbf{29} ravîten] rauitier o \newline
\end{minipage}
\end{table}
\newpage
\begin{table}[ht]
\begin{minipage}[t]{0.5\linewidth}
\small
\begin{center}*G
\end{center}
\begin{tabular}{rl}
 & dô sprach er: "vrouwe, tuot sô wol,\\ 
 & ob ich iuch \textbf{des} biten sol:\\ 
 & lât mînen namen unerkant,\\ 
 & als mich der rîter hât genant,\\ 
5 & der mir \textbf{entreit} Gringulieten.\\ 
 & leistet, des ich iuch hân gebeten.\\ 
 & swer iuch des vrâgen welle,\\ 
 & sô sprechet ir: 'mîn geselle\\ 
 & ist mir de\textit{s} unerkennet,\\ 
10 & er\textbf{ne} wart mir nie genennet.'"\\ 
 & si sprach: "vil gerne ich \textbf{si}\textbf{z} verdage,\\ 
 & sît ir niht wellet, daz ich ez sage."\\ 
 & er unt diu vrouwe wol gevar\\ 
 & kêrten gein der bürge dar.\\ 
15 & \textbf{die rîter heten} dâ vernomen,\\ 
 & daz ein rîter wære komen,\\ 
 & der het die âventiure erliten\\ 
 & unt den lewen überstriten\\ 
 & unde den Turkoiten sider\\ 
20 & \textbf{ze} rehter tjost gevellet nider.\\ 
 & innen des reit Gawan\\ 
 & gein dem urvar ûf den plân,\\ 
 & daz sin von zinnen sâhen.\\ 
 & si begunden vaste gâhen\\ 
25 & ûz der burc mit schalle.\\ 
 & dâ vuorten si alle\\ 
 & rîche baniere;\\ 
 & sus kômen si schiere\\ 
 & ûf snellen ravîten.\\ 
30 & er wânde, si wolden strîten.\\ 
\end{tabular}
\scriptsize
\line(1,0){75} \newline
G I L M Z \newline
\line(1,0){75} \newline
\textbf{1} \textit{Initiale} L Z  \textbf{11} \textit{Initiale} I  \textbf{21} \textit{Initiale} I  \newline
\line(1,0){75} \newline
\textbf{1} dô] Da M Z \textbf{5} entreit] hin rait I  $\cdot$ Gringulieten] grigulieten G (M) Gringvlieten L kringulieten Z \textbf{6} gebeten] geben L \textbf{7} swer] Wer L M \textbf{9} des] der G \textit{om.} I \textbf{10} erne] er I  $\cdot$ nie genennet] [g]: nie genennet G nih benennet I \textbf{11} vil gerne ich siz] vil gerne ichsz L (M) ich vil gerne iz Z  $\cdot$ verdage] veriage L \textbf{16} daz] Daz dar Z \textbf{19} Turkoiten] [t*]: turchoyten G Turkoyden I Tuͯrkoyten L \textbf{21} Gawan] ergawan M \textbf{23} sin] in L  $\cdot$ von] von den I \textit{om.} Z \textbf{26} dâ] do I L \newline
\end{minipage}
\hspace{0.5cm}
\begin{minipage}[t]{0.5\linewidth}
\small
\begin{center}*T
\end{center}
\begin{tabular}{rl}
 & dô sprach er: "vrouwe, tuot sô wol,\\ 
 & ob ich iuch \textbf{des} b\textit{it}en sol:\\ 
 & lât mînen namen unerkant,\\ 
 & als mich der rîter hât genant,\\ 
5 & der mir \textbf{entreit} Kryngulieten.\\ 
 & leistet, des ich iuch hân gebeten.\\ 
 & wer iuch des vrâgen welle,\\ 
 & sô sprechet ir: 'mîn geselle\\ 
 & ist mir des unerkennet,\\ 
10 & er \textbf{en}wart mir nie genennet.'"\\ 
 & si sprach: "vil gerne ich \textbf{ez} verdage,\\ 
 & sît ir niht wolt, daz ich ez sage."\\ 
 & er und diu vrouwe wol gevar\\ 
 & kêrten gein der bürge dar.\\ 
15 & \textbf{die rîtære heten} d\textit{â} vernomen,\\ 
 & daz ein rîter wære komen,\\ 
 & der hete die âventiure erliten\\ 
 & und den lewen überstriten\\ 
 & und den Turkoyten sider\\ 
20 & \textbf{zuo} rehter jost gevellet nider.\\ 
 & \begin{large}I\end{large}nne des reit Gawan\\ 
 & gein dem urvar ûf den plân,\\ 
 & daz si in von \textbf{der} zinnen sâhen.\\ 
 & si begunden vaste gâhen\\ 
25 & ûz der burc mit schalle.\\ 
 & dô vuorten si alle\\ 
 & rîche baniere;\\ 
 & sus kâmen si schiere\\ 
 & ûf snellen ravîten.\\ 
30 & er wânte, si wolten strîten.\\ 
\end{tabular}
\scriptsize
\line(1,0){75} \newline
U V W Q R Fr39 \newline
\line(1,0){75} \newline
\textbf{1} \textit{Initiale} W R Fr39  \textbf{21} \textit{Initiale} U  \newline
\line(1,0){75} \newline
\textbf{1} dô] [*]: Do V \textbf{2} biten] bieten U \textbf{5} entreit] entwert R  $\cdot$ Kryngulieten] kyngulieten U kringuletten V kringulieten W (Q) [kringulten]: kringuleten  R kringuleten Fr39 \textbf{6} leistet] Leisten Q (R) \textbf{7} wer] Swer V Fr39 \textbf{10} enwart] wart Q (R) Fr39 \textbf{11} ez] \textit{om.} V daz R  $\cdot$ verdage] [vertrage]: verdage Q \textbf{15} [D*]: Do hetten die ritter do vernomen V  $\cdot$ dâ] daz U \textbf{16} komen] [*]: dar komen V \textbf{19} Turkoyten] [kurkoiten]: Turkoiten Q Turkoten R turkoiten Fr39 \textbf{21} Inne] Jnrent V  $\cdot$ Gawan] gewan W Gawa: Fr39 \textbf{22} dem] \textit{om.} Q  $\cdot$ urvar] úberuar W \textbf{23} si] \textit{om.} W  $\cdot$ der] \textit{om.} V W Q R Fr39 \textbf{25} ûz] Zu Q Vsser R \textbf{28} sus] [*]: Als Q \textbf{30} wolten] wolte W \newline
\end{minipage}
\end{table}
\end{document}
