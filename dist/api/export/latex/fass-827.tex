\documentclass[8pt,a4paper,notitlepage]{article}
\usepackage{fullpage}
\usepackage{ulem}
\usepackage{xltxtra}
\usepackage{datetime}
\renewcommand{\dateseparator}{.}
\dmyyyydate
\usepackage{fancyhdr}
\usepackage{ifthen}
\pagestyle{fancy}
\fancyhf{}
\renewcommand{\headrulewidth}{0pt}
\fancyfoot[L]{\ifthenelse{\value{page}=1}{\today, \currenttime{} Uhr}{}}
\begin{document}
\begin{table}[ht]
\begin{minipage}[t]{0.5\linewidth}
\small
\begin{center}*D
\end{center}
\begin{tabular}{rl}
\textbf{827} & \begin{large}O\end{large}b von Troys meister Christjan\\ 
 & \textbf{disem mære} \textbf{hât unreht} getân,\\ 
 & \textbf{daz} mac wol zürnen Kyot,\\ 
 & der uns \textbf{diu rehten mære} enbôt.\\ 
5 & endehaft giht der Provenzal,\\ 
 & wie Herzeloyden kint den Grâl\\ 
 & \textbf{erwarp}, als im \textbf{daz} geordent was,\\ 
 & dô in verworhte Anfortas.\\ 
 & Von Provenz in tiuschiu lant\\ 
10 & diu rehten mære uns sint gesant\\ 
 & unt dirre âventiure endes zil.\\ 
 & \textbf{niht mêr dâ von nû} sprechen wil\\ 
 & ich, Wolfram von Esschenbach,\\ 
 & wan als dort der meister sprach.\\ 
15 & Sîniu kint, sîn hôch geslehte,\\ 
 & hân ich iu \textbf{benennet} rehte,\\ 
 & Parzivals, den ich hân brâht,\\ 
 & dar sîn doch sælde het erdâht.\\ 
 & Swes leben sich sô verendet,\\ 
20 & daz got niht wirt gepfendet\\ 
 & \textbf{der} sêle durch\textbf{s} lîbes schulde,\\ 
 & unt \textbf{der} \textbf{doch} der werlde hulde\\ 
 & \textbf{behalten} kan mit werdecheit,\\ 
 & daz ist ein nütziu arbeit.\\ 
25 & Guotiu wîp, hânt die sin,\\ 
 & deste werder ich in bin,\\ 
 & ob mir \textbf{decheiniu} guotes gan,\\ 
 & sît ich \textbf{diz mære} \textbf{volsprochen} hân.\\ 
 & ist daz durch ein wîp geschehen,\\ 
30 & diu muoz mir süezer \textbf{worte} jehen.\\ 
\end{tabular}
\scriptsize
\line(1,0){75} \newline
D \newline
\line(1,0){75} \newline
\textbf{1} \textit{Initiale} D  \textbf{9} \textit{Majuskel} D  \textbf{15} \textit{Majuskel} D  \textbf{19} \textit{Majuskel} D  \textbf{25} \textit{Majuskel} D  \newline
\line(1,0){75} \newline
\textbf{1} Christjan] Christian D \textbf{9} tiuschiu] tîvsciv D \textbf{13} Esschenbach] Esscenbach D \textbf{17} Parzivals] Parcifals D \newline
\end{minipage}
\hspace{0.5cm}
\begin{minipage}[t]{0.5\linewidth}
\small
\begin{center}*m
\end{center}
\begin{tabular}{rl}
 & \begin{large}O\end{large}b von T\textit{ro}is meister Cristian\\ 
 & \textbf{disem mære} \textbf{unrehte het} getân,\\ 
 & \textbf{des} mac wol zürnen Kiot,\\ 
 & der uns \textbf{die rehten mær} enbôt.\\ 
5 & endehaft giht der Provenzal,\\ 
 & wie Herczeloiden kint den Grâl\\ 
 & \textbf{erwarp}, als im geordent was,\\ 
 & dô in verworhte Anfortas.\\ 
 & von Profantz in tiutsch\textit{iu} lant\\ 
10 & di\textit{u} rehte\textit{n} mære uns si\textit{nt} gesant\\ 
 & und diser âventiur endes zil.\\ 
 & \textbf{niht mê dâ von nû} sprechen wil\\ 
 & ich, Wolfram von E\textit{s}chenbach,\\ 
 & wan als dort der meister sprach.\\ 
15 & sîniu kint, sîn hôch geslehte,\\ 
 & hab ich iu \textbf{benennet} rehte,\\ 
 & Parcifals, den ich hân brâht,\\ 
 & dar sîn \textit{d}och sælde het erdâht.\\ 
 & wes leben sich sô verendet,\\ 
20 & daz got niht wirt gepfendet\\ 
 & \textbf{der} sêle durch \textbf{des} lîbes schulde,\\ 
 & und \textbf{der} \textbf{doch} der werlte hulde\\ 
 & \textbf{behalten} kan mit werdicheit,\\ 
 & daz ist ein nütziu arbeit.\\ 
25 & guotiu wîp, hânt die sin,\\ 
 & deste werder ich in bin,\\ 
 & ob mir \textbf{dekeiniu} guotes gan,\\ 
 & sît ich \textbf{diz mære} \textbf{volsprochen} hân.\\ 
 & ist daz durch ein wîp geschehen,\\ 
30 & d\textit{iu} muoz mir süezer \textbf{worte} jehen.\\ 
\end{tabular}
\scriptsize
\line(1,0){75} \newline
m n V V' W \newline
\line(1,0){75} \newline
\textbf{1} \textit{Initiale} m V   $\cdot$ \textit{Capitulumzeichen} n  \textbf{11} \textit{Initiale} W  \newline
\line(1,0){75} \newline
\textbf{1} von] \textit{om.} n  $\cdot$ Trois] toris m troya V' troys W  $\cdot$ Cristian] kristian W \textbf{2} disem mære] Diesen meren V'  $\cdot$ unrehte] vnrecht n (V) V' \textbf{3} des] Daz V V'  $\cdot$ Kiot] kẏot n kyot V V' W \textbf{4} rehten] rehte V \textbf{5} \textit{Die Verse 827.5-10 fehlen} V'   $\cdot$ Provenzal] provinzal m provintzal n profenzal V prouenzal W \textbf{6} Herczeloiden] hertzeloiden m n herzelauden V hertzeloyden W \textbf{7} im] imme daz V er W \textbf{8} verworhte] verwúrcket n \textbf{9} Profantz] profantz V  $\cdot$ tiutschiu] tutschen m dútze n Túsche V \textbf{10} diu rehten] Die rehtte m (n)  $\cdot$ sint] sẏ m (n) \textbf{11} und] Von V' \textbf{12} Nit mer ich do von sagen wil V'  $\cdot$ nû] \textit{om.} W \textbf{13} Wolfram] wolffram n W wolferam V'  $\cdot$ Eschenbach] echenbach m [æ]: eschenbach V' \textbf{15} \textit{Die Verse 827.15-16 fehlen} V'   $\cdot$ sîn] sint V \textbf{16} iu] \textit{om.} W \textbf{17} Parcifals] Parzefal V Parzifal V' Herr partzifal W  $\cdot$ den ich hân] han ich V' \textbf{18} dar] Do V'  $\cdot$ doch] hoh m  $\cdot$ sælde] selten V'  $\cdot$ het] hat n W  $\cdot$ erdâht] gedocht V' (W) \textbf{19} wes] Swez V \textbf{21} sêle] selen W  $\cdot$ des] [der]: des m \textit{om.} V V' \textbf{25} hânt] haben V'  $\cdot$ die] den V' W \textbf{27} dekeiniu] do kein n  $\cdot$ gan] [h]: gan n \textbf{28} diz] dise n W das V'  $\cdot$ volsprochen] volsprechen V' \textbf{29} geschehen] beschehen W \textbf{30} \textit{nach 827.30:} Hie het dis buch ein ende / Got vns von suͯnden wende Amen m  Amen n  Einschub 827.30\textasciicircum1-30\textasciicircum5\textasciicircum8\textasciicircum0 (Epilog, Kolophon, Blattberechnung, Minneliedstrophen, dt. und afr.) V  Einschub 827.30\textasciicircum1-30\textasciicircum5\textasciicircum5\textasciicircum8 (Epilog und Blattberechnung) V'  MCCCCLXXVII W   $\cdot$ diu] Des m  $\cdot$ muoz] muͯsse n \newline
\end{minipage}
\end{table}
\newpage
\begin{table}[ht]
\begin{minipage}[t]{0.5\linewidth}
\small
\begin{center}*G
\end{center}
\begin{tabular}{rl}
 & \begin{large}O\end{large}be von Troys meister Christan\\ 
 & \textbf{disem mære} \textbf{hât unreht} getân,\\ 
 & \textbf{daz} mac wol zürnen Kiot,\\ 
 & der uns \textbf{diu mære rehte} enbôt.\\ 
5 & endehaft giht der Provenzal,\\ 
 & wie Herzeloide kint den Grâl\\ 
 & \textbf{geêrbet}, alse im geordent was,\\ 
 & dô in verworhte Anfortas.\\ 
 & von Provenze in tiutschiu lant\\ 
10 & diu rehten mære uns sint gesant\\ 
 & unde dirre âventiure endezil.\\ 
 & \textbf{dâ von ich ni\textit{ht} mêre} sprechen wil,\\ 
 & ich, Wolfram von Eschenbach,\\ 
 & wan als dort der meister sprach.\\ 
15 & sîniu kint, sîn hôch geslehte,\\ 
 & hân ich iu \textbf{genennet} rehte,\\ 
 & Parzivals, den ich hân brâht,\\ 
 & dar sîn doch sælde het erdâht.\\ 
 & swes leben sich sô verendet,\\ 
20 & daz got niht wirt gepfendet\\ 
 & \textbf{diu} sêle durch \textbf{des} lîbes schulde,\\ 
 & unde \textbf{er} der werlde hulde\\ 
 & \textbf{gedienen} kan mit werdecheit,\\ 
 & daz ist ein nütziu arbeit.\\ 
25 & guotiu wîp \textbf{unde} hânt die sin,\\ 
 & deste werder ich in bin,\\ 
 & op mir \textbf{deheiniu} guotes gan,\\ 
 & sît ich \textbf{ditze mære} \textbf{volsprochen} hân.\\ 
 & \textbf{unde} ist daz durch ein wîp geschehen,\\ 
30 & diu muoz mir süezer \textbf{mære} jehen.\\ 
\end{tabular}
\scriptsize
\line(1,0){75} \newline
G I L Z \newline
\line(1,0){75} \newline
\textbf{1} \textit{Initiale} G L Z  \textbf{5} \textit{Initiale} I  \textbf{17} \textit{Initiale} I  \newline
\line(1,0){75} \newline
\textbf{1} Troys] Troẏs G Trois Z  $\cdot$ meister] \textit{om.} I  $\cdot$ Christan] cristian I Cristan L Z \textbf{2} disem] Der L  $\cdot$ unreht] vnrehte I (L)  $\cdot$ hât] \textit{om.} I \textbf{3} zürnen] zvrnet Z  $\cdot$ Kiot] kyot L Z \textbf{5} der] auch der I  $\cdot$ Provenzal] prouinzal I \textbf{6} Herzeloide] herzeloẏde G herzenlauden I hertzelovden L herzelovden Z  $\cdot$ den] der I \textbf{7} \textit{Versfolge 827.8-7} L   $\cdot$ im] im daz L Z \textbf{8} dô] Da Z  $\cdot$ Anfortas] Amfortas L \textbf{9} Provenze] prouenz I proventz Z  $\cdot$ tiutschiu] tv̂tschiv G Tutshev I tvͯtsche L tevsche Z \textbf{10} rehten] rehtiu I  $\cdot$ uns] \textit{om.} L \textbf{11} endezil] zil L \textbf{12} ich niht mêre] ih nimere G ich nu nih me I niht me nv L ich niht mer nv Z \textbf{13} Eschenbach] [eshelbahe]: Eshelbach I Eshenbach L Esschenbach Z \textbf{16} iu] \textit{om.} L \textbf{17} Parzivals] parcifals G (Z) Parzifaln I Parzifals L \textbf{18} erdâht] gedacht L \textbf{19} swes] Wez L  $\cdot$ verendet] endet L \textbf{20} Daz in got niht enpfendet L \textbf{21} diu] Der Z  $\cdot$ des] \textit{om.} L \textbf{23} mit] duͯrch L \textbf{24} daz] Der L  $\cdot$ ein nütziu] an nvͯtzer L \textbf{25} unde] \textit{om.} L  $\cdot$ die] den L \textbf{28} ditze mære] daz L  $\cdot$ volsprochen hân] vol [sprechen]: sprochen [chan]: han I \textbf{30} \textit{nach 827.30, in roter Tinte:} Explicit parzifal Anno dominj M° Cccc° lj iar off purificacio marie virginis [war]: wart dis buͦch geschrieben von Jordan vnd ist auch sin L   $\cdot$ süezer mære] gvter sprvͤche Z \newline
\end{minipage}
\hspace{0.5cm}
\begin{minipage}[t]{0.5\linewidth}
\small
\begin{center}*T
\end{center}
\begin{tabular}{rl}
 & ob von Troys meister Cristian\\ 
 & \textbf{disen mæren} \textbf{hât unreht} getân,\\ 
 & \textbf{daz} mac wol zürnen Kyot,\\ 
 & der uns \textbf{die mære rehte} enbôt.\\ 
5 & \textit{endehaft giht der Provenzal, }\\ 
 & \textit{wie Herzeloyden kint den Grâl}\\ 
 & \textbf{erbete}, als im geordent was,\\ 
 & dô in verworhte Anfortas.\\ 
 & von Provenze in tiuschiu lant\\ 
10 & die rehten mære uns sint gesant\\ 
 & und dirre âventiure endes zil.\\ 
 & \textbf{d\textit{â} von ich nû niht mêr} sprechen wil,\\ 
 & ich, Wolfram von Eschebach,\\ 
 & wan als dort der meister sprach.\\ 
15 & sîniu kint, sîn hôch gesle\textit{h}te,\\ 
 & hân ich iu \textbf{genennet} rehte,\\ 
 & \textit{Parcifals, den ich hân brâht,}\\ 
 & \textit{dar sîn doch sælde het erdâht.}\\ 
 & wes leben sich sô verendet,\\ 
20 & daz got niht wirt gepfendet\\ 
 & \textbf{der} sêle durch lîbes schulde,\\ 
 & und \textbf{doch} der werlde hulde\\ 
 & \textit{\textbf{gedienen} kan mit werdecheit,}\\ 
 & \textit{daz ist ein nütziu arbeit.}\\ 
25 & guotiu wîp, hânt die sin,\\ 
 & deste werder ich in bin,\\ 
 & ob mir \textbf{dekeinez} guotes gan,\\ 
 & sît ich \textbf{dise mære} \textbf{von in} \textbf{gesprochen} hân.\\ 
 & \textbf{und} \textit{ist} daz durch ein wîp geschehen,\\ 
30 & diu muoz mir süezer \textbf{mære} jehen.\\ 
\end{tabular}
\scriptsize
\line(1,0){75} \newline
U Q R \newline
\line(1,0){75} \newline
\textbf{1} \textit{Initiale} Q  \textbf{13} \textit{Initiale} R  \newline
\line(1,0){75} \newline
\textbf{1} Troys] Trois U  $\cdot$ meister] meiste R  $\cdot$ Cristian] [Criatian]: Cristian U kristan Q cristan R \textbf{2} disen mæren] Diesem mere Q (R) \textbf{4} die] du R \textbf{5} \textit{Die Verse 827.5-6 fehlen} U   $\cdot$ endehaft giht] Enhafftt git R  $\cdot$ Provenzal] prouenzal Q profenzal R \textbf{6} Herzeloyden] herzeloūde Q herczelauden R \textbf{7} erbete] Geerbet Q R  $\cdot$ im] im das Q in R \textbf{8} dô] Den R \textbf{9} Provenze] prouenze Q provencze R  $\cdot$ tiuschiu] duͦsche U dutze Q tᵫtsche R \textbf{10} die rehten] Du Rechtú R  $\cdot$ sint] hat R \textbf{11} âventiure endes] ende auentᵫre R \textbf{12} nû] \textit{om.} Q R \textbf{13} Wolfram] wolffram R  $\cdot$ Eschebach] eschenbach Q R \textbf{14} als] als vns R \textbf{15} Sine geschlecht sine kint sint hoch geschechte R  $\cdot$ geslehte] geslechechte U \textbf{17} \textit{Die Verse 827.17-18 fehlen} U   $\cdot$ Parcifals] Partzifals Q Parczifals R  $\cdot$ brâht] \sout{rechte} brocht Q \textbf{19} sich] sy R  $\cdot$ sô] doch Q \textbf{20} niht wirt] wirt nicht Q \textbf{21} der] Dú R  $\cdot$ lîbes] des lebens Q des libes R \textbf{22} Vnder der werden hulde R  $\cdot$ doch] er Q \textbf{23} \textit{Die Verse 827.23-24 fehlen} U  \textbf{24} nütziu] nucze R \textbf{25} hânt die] vnd habent den Q vnde hant dú R \textbf{27} dekeinez] keine Q (R) \textbf{28} Sint ich diesz mere volsprochen hon Q  $\cdot$ Sid ich dis auentúrre gar gesprochen han R \textbf{29} ist] \textit{om.} U  $\cdot$ daz] dis R \textbf{30} \textit{nach 827.30, in schwarzer Tinte:} [he]: hie haît das buͦch ein ende / got alle falsche hertzen schende (von anderer Hand:) Ammen U  In roter Tinte: Et finitus est iste liber 2a feria ante palmarum Q  In brauner Tinte: Dis Auentúr hett ein end / Got vns sine gnade send / Vnd helff vns vser aller not / Der durch vns leid den tod / An dem heilgen Crucz fronen / Nun sprechent alle Amen (in roter Tinte:) Explicit barczifal 3a ante purificacionis marie Anno etc lxvij° per me Joh. Stemhein de Constancia (in brauner Tinte:) Dis buͦch ist Joͯrg Friburgers Von bern [1468]: 1467 R   $\cdot$ mære] mynne Q \newline
\end{minipage}
\end{table}
\end{document}
