\documentclass[8pt,a4paper,notitlepage]{article}
\usepackage{fullpage}
\usepackage{ulem}
\usepackage{xltxtra}
\usepackage{datetime}
\renewcommand{\dateseparator}{.}
\dmyyyydate
\usepackage{fancyhdr}
\usepackage{ifthen}
\pagestyle{fancy}
\fancyhf{}
\renewcommand{\headrulewidth}{0pt}
\fancyfoot[L]{\ifthenelse{\value{page}=1}{\today, \currenttime{} Uhr}{}}
\begin{document}
\begin{table}[ht]
\begin{minipage}[t]{0.5\linewidth}
\small
\begin{center}*D
\end{center}
\begin{tabular}{rl}
\textbf{383} & \textbf{\begin{large}O\end{large}uch hete} ieslîch Bertun\\ 
 & durch \textbf{bekantnisse} ein gampilûn\\ 
 & eintweder \textbf{ûf} helm oder \textbf{ûf den} schilt\\ 
 & nâch Ilynotes wâpen gezilt.\\ 
5 & \textbf{daz} was Artuses werder sun.\\ 
 & waz \textbf{mohte} Gawan \textbf{dô} tuon,\\ 
 & er\textbf{n} siufzete, dô er \textbf{diu} wâpen sach,\\ 
 & wande im sîn herze jâmers jach.\\ 
 & sînes œheims sunes tôt\\ 
10 & brâhte \textbf{Gawanen} in jâmers nôt.\\ 
 & er \textbf{bekande} wol \textbf{der} wâpen schîn.\\ 
 & dô \textbf{liefen über} diu ougen sîn.\\ 
 & Er liez die von Bertane\\ 
 & \textbf{sus} tûren ûf \textbf{dem plâne}.\\ 
15 & er wolde mit in strîten niht,\\ 
 & als man \textbf{nâch} vriwentschefte giht.\\ 
 & Er reit gein Meljanzes her.\\ 
 & \textbf{dâ wâren die burgære} ze wer,\\ 
 & daz mans in danken mohte,\\ 
20 & wan \textbf{daz} in doch niht \textbf{tohte},\\ 
 & daz velt gegen überkraft ze \textbf{behaben}.\\ 
 & si wâren entwichen gein dem graben.\\ 
 & Den burgæren manege tjost dâ bôt\\ 
 & ein ritter allenthalben rôt.\\ 
25 & der hiez der ungenante,\\ 
 & wande in niemen dâ bekante.\\ 
 & ich sagz \textbf{iu}, als ich\textbf{z} hân vernomen:\\ 
 & er was zuo Meljanze komen\\ 
 & dâ vor ame dritten tage.\\ 
30 & des kômen die burgære in klage.\\ 
\end{tabular}
\scriptsize
\line(1,0){75} \newline
D \newline
\line(1,0){75} \newline
\textbf{1} \textit{Initiale} D  \textbf{13} \textit{Majuskel} D  \textbf{17} \textit{Majuskel} D  \textbf{23} \textit{Majuskel} D  \newline
\line(1,0){75} \newline
\textbf{4} Ilynotes] Jlynots D \textbf{5} Artuses] Artvs D \textbf{10} Gawanen] Gawann D \textbf{17} Meljanzes] Melianzes D \textbf{28} Meljanze] Melianze D \newline
\end{minipage}
\hspace{0.5cm}
\begin{minipage}[t]{0.5\linewidth}
\small
\begin{center}*m
\end{center}
\begin{tabular}{rl}
 & \textbf{ouch hete} etslîcher Britu\textit{n}\\ 
 & durch \textbf{bekantnisse} ein gampilû\textit{n}\\ 
 & eintweder \textbf{ûf} helm oder \textbf{ûf den} schilt\\ 
 & nâch \textit{I}li\textit{no}t\textit{e}s  wâpen gezilt.\\ 
5 & \textbf{daz} was Artuses werder sun.\\ 
 & waz \textbf{mohte} Ga\textit{w}an \textbf{dô} tuon?\\ 
 & er siufze\textit{t}, dô er \textbf{diu} wâpen sach,\\ 
 & want ime sîn herze jâmers jach.\\ 
 & sînes œheimes sunes tôt\\ 
10 & brâhte \textbf{Gawanen} in jâmers nôt.\\ 
 & er \textbf{erkante} wol \textbf{der} wâpen schîn.\\ 
 & dô \textbf{liefen über} diu ougen sîn.\\ 
 & er liez die von Britane\\ 
 & \textbf{sus} tûren ûf \textbf{dem plâne}.\\ 
15 & er wolte mit in strîten niht,\\ 
 & als man \textbf{noch} v\textit{r}iun\textit{t}schaft giht.\\ 
 & er reit gegen Melianze\textit{s} her.\\ 
 & \textbf{d\textit{â} wâren die burgære} ze wer,\\ 
 & daz mans in danken mohte,\\ 
20 & wan \textbf{daz} in doch niht \textbf{tohte},\\ 
 & daz velt gegen überkraft ze \textbf{behaben}.\\ 
 & si wâren entwichen gegen dem graben.\\ 
 & \begin{large}D\end{large}en burgæren manige just d\textit{â} b\textit{ô}t\\ 
 & ein ritter \textit{a}llenthalben rôt.\\ 
25 & der hiez der ungenante,\\ 
 & wand in niemen d\textit{â} bekante.\\ 
 & ich \textit{sag} ez \textbf{iu}, als ich\textbf{z} hân vernomen:\\ 
 & er was ze Melianze komen\\ 
 & dâ vor an dem dritten tage.\\ 
30 & des kômen die burgære in klage.\\ 
\end{tabular}
\scriptsize
\line(1,0){75} \newline
m n o \newline
\line(1,0){75} \newline
\textbf{23} \textit{Initiale} m n  \newline
\line(1,0){75} \newline
\textbf{1} Britun] brittum m britym o \textbf{2} ein] ym o  $\cdot$ gampilûn] gampelum m gampolúm o \textbf{3} eintweder] Eitwider o \textbf{4} Ilinotes] littens m ilinors n linors o \textbf{5} Artuses] artus m  $\cdot$ werder] werden o \textbf{6} Gawan] gavan m  $\cdot$ dô] da o  $\cdot$ tuon] getuͦn n (o) \textbf{7} er] E: o  $\cdot$ siufzet] suffcze m \textbf{9} œheimes] oͯheim n (o)  $\cdot$ sunes] sẏnes o \textbf{10} Gawanen] gawonen n  $\cdot$ jâmers] james o \textbf{11} wâpen] wappe o \textbf{12} dô] Die n \textbf{13} er liez] Elies o  $\cdot$ Britane] brittane m \textbf{14} tûren] fúren o  $\cdot$ dem] den n o \textbf{15} in strîten] ẏme scẏtten o \textbf{16} vriuntschaft] fuͯnschafft m \textbf{17} Melianzes] melianczen m meliantzes n melianczez o \textbf{18} dâ] Die m Do n o \textbf{21} ze] \textit{om.} n o \textbf{22} gegen] von n o \textbf{23} manige] gegen o  $\cdot$ dâ bôt] do bat m do bot n do gap o \textbf{24} allenthalben] ellenthalben m [ellen*]: ellenthalben o  $\cdot$ rôt] [not]: rot o \textbf{26} dâ] do m n o \textbf{27} sag] \textit{om.} m  $\cdot$ ichz] ich o  $\cdot$ vernomen] vernonoen o \textbf{28} ze Melianze] zemeliancze m zuͯ meliantze n zuͦ meliancz o \textbf{30} burgære] burgen o \newline
\end{minipage}
\end{table}
\newpage
\begin{table}[ht]
\begin{minipage}[t]{0.5\linewidth}
\small
\begin{center}*G
\end{center}
\begin{tabular}{rl}
 & \textbf{ez vüert ouch} etslîch Britun\\ 
 & durch \textbf{bekantnisse} ein capelûn\\ 
 & eintweder \textbf{ûfem} helme oder \textbf{ûfeme} schilt\\ 
 & nâch Ilinotes wâ\textit{p}en gezilt.\\ 
5 & \textbf{der} was Artuses werder sun.\\ 
 & waz \textbf{mac} Gawan \textbf{nû} tuon?\\ 
 & er sûfte, dô er \textbf{diu} wâpen sach,\\ 
 & wan im sîn herze jâmers jach.\\ 
 & sîne\textit{s} œheimes sunes tôt\\ 
10 & brâht \textbf{Gawa\textit{n}} in \textit{jâm}e\textit{rs} nôt.\\ 
 & er \textbf{e\textit{r}kande} wol \textbf{der} wâpen schîn.\\ 
 & dô \textbf{über liefen} diu ougen sîn.\\ 
 & er lie die von Britanie\\ 
 & tûren ûf \textbf{der plânîe}.\\ 
15 & er\textbf{ne} wolte mit in strîten niht,\\ 
 & als man \textbf{noch} vriuntschefte giht.\\ 
 & er reit gein Melianzes her.\\ 
 & \textbf{d\textit{ie} burgære wâren} \textbf{sô} ze wer,\\ 
 & daz mans in danken mohte,\\ 
20 & wan in doch niht \textbf{getohte}\\ 
 & daz velt gein überkraft ze \textbf{haben}.\\ 
 & si wâren entwichen geime graben.\\ 
 & den burgæren manige tjost dâ bôt\\ 
 & ein rîter allenthalben rôt.\\ 
25 & der hiez der ungenande,\\ 
 & wan in niemen dâ bekande.\\ 
 & ich sagez \textbf{iu}, als ich hân vernomen:\\ 
 & er was zuo Melianze komen\\ 
 & dâ vor ame driten tage.\\ 
30 & des kômen die burgære in klage.\\ 
\end{tabular}
\scriptsize
\line(1,0){75} \newline
G I O L M Q R Z Fr21 Fr41 \newline
\line(1,0){75} \newline
\textbf{1} \textit{Initiale} I O L M Fr21   $\cdot$ \textit{Capitulumzeichen} R  \textbf{15} \textit{Initiale} I  \textbf{23} \textit{Majuskel} Fr41  \textbf{27} \textit{Initiale} O   $\cdot$ \textit{Capitulumzeichen} R  \newline
\line(1,0){75} \newline
\textbf{1} \textit{Die Verse 370.13-412.12 fehlen} Q   $\cdot$ ez vüert ouch] ÷vch het O Avch het L (M) (Z) (Fr21) Doch hett R  $\cdot$ Britun] [brtun]: britun G pritun I Biton R \textbf{2} bekantnisse] kantnvsse Fr21  $\cdot$ ein] einen L  $\cdot$ capelûn] Gabilun I gampilum R (Z) (Fr41) \textbf{3} eintweder] ietdweder I Ein >veder< O \textit{om.} R  $\cdot$ ûfem] vf I O L  $\cdot$ ûfeme] vf I O L \textit{om.} M \textbf{4} nâch Ilinotes] nahilinotes G vf cleinodes I Nach ylinotes O Nach jlinotes L (R) Nach ylinotis M Nach ylmotes Z Nach elemotes Fr21  $\cdot$ wâpen] waben G \textbf{5} was] \textit{om.} I  $\cdot$ Artuses] Artuͯses L artus R  $\cdot$ werder] werder swester I swester O werden Fr21 \textbf{6} mac] mohte Z  $\cdot$ tuon] getuͦn I (Z) \textbf{7} er sûfte] ern suͤfte I Er sevft O Er ersúnffczete R Ern schvfte Fr21  $\cdot$ dô] da O M Z  $\cdot$ diu] disiv O (L) (M) (Z) siniv Fr21 \textbf{8} jâmers] Jamer R (Z) \textbf{9} sînes] sin G Sin sines L  $\cdot$ œheimes] elems Fr21  $\cdot$ tôt] rot Z \textbf{10} Gawan] Gawanen G  $\cdot$ jâmers] groze G \textbf{11} erkande] echande G bechande I (Fr21) kantte R  $\cdot$ wol] \textit{om.} M  $\cdot$ der] diu I \textbf{12} dô] Da O M Z  $\cdot$ diu] im diu I (O) (L) (M) (R) (Z) \textbf{13} er] Der R  $\cdot$ Britanie] pritangen I [pritange]: pritanige O Brittanie L britange Z (Fr21) \textbf{14} tûren] Tuͯrnieren L (R) Sus truren Z  $\cdot$ der] dem I (L) (R) (Z) \textbf{15} erne] ER I (O) (R) (Fr21) \textbf{16} noch] \textit{om.} I Z \textbf{17} reit] treit M  $\cdot$ Melianzes] meliananzes I Melyanzes O Malianzes R meliantzes Z \textbf{18} die] do G  $\cdot$ sô] do R \textbf{19} mans in] man in I man ins Fr21 \textbf{20} wan] Wan das R (Fr41)  $\cdot$ doch] do I \textit{om.} R  $\cdot$ getohte] tohte Fr21 \textbf{21} gein] gen inen begund R  $\cdot$ ze haben] zebe haben O zuͯ hagen L zcu bihaben M (Z) (Fr21) \textbf{22} entwichen] getriben R  $\cdot$ geime] Gein den I vff den R \textbf{23} den] Der L  $\cdot$ manige] manigen M  $\cdot$ dâ] do R \textit{om.} Z \textbf{24} allenthalben] ellenthalbir M \textbf{26} niemen dâ] da nieman I (O) nyeman do R  $\cdot$ bekande] erchande I (O) (R) (Fr21) \textbf{27} ich sagez iu] ÷ch sagz iv O sage uch M sagen úchs R sagt ivz Fr21  $\cdot$ ich hân] ichz han I O (L) (M) (R) (Z) \textbf{28} zuo] Gein I  $\cdot$ Melianze] Melyanze O Meliantze L (Z) Meliencze R \textbf{29} ame] anden M \textbf{30} des] Dem L  $\cdot$ in] zuͯ L \newline
\end{minipage}
\hspace{0.5cm}
\begin{minipage}[t]{0.5\linewidth}
\small
\begin{center}*T
\end{center}
\begin{tabular}{rl}
 & \textbf{\begin{large}N\end{large}û hete ouch} iegeslîch Britun\\ 
 & durch \textbf{erkantnisse} einen gampilûn\\ 
 & einweder \textbf{ûf} helm oder \textbf{ûf} schilt\\ 
 & nâch Ylinotes wâpen gezilt.\\ 
5 & \textbf{der} was Artuses werder suon.\\ 
 & Waz \textbf{mac} Gawan \textbf{nû} tuon?\\ 
 & er sûfte, dô er \textbf{dis\textit{iu}} wâpen sach,\\ 
 & wandim sîn herze jâmers jach.\\ 
 & \hspace*{-.7em}\big| er \textbf{erkande} wol \textbf{den} wâpen schîn.\\ 
 & \hspace*{-.7em}\big| dô \textbf{über liefen} \textbf{im} diu ougen sîn.\\ 
 & \hspace*{-.7em}\big| sînes œheimes sunes tôt\\ 
10 & \hspace*{-.7em}\big| brâht\textbf{in} in jâmers nôt.\\ 
 & er lie \textbf{sich} die von Britanie\\ 
 & tûren ûf \textbf{der plânîe}.\\ 
15 & er wolte mit in strîten niht,\\ 
 & alse \textbf{dâ} man \textbf{noch} vriuntschefte giht.\\ 
 & er reit gegen Melyanzes her.\\ 
 & \textbf{die burgære wâren} \textbf{sô} ze wer,\\ 
 & daz man\textit{s} in danken mohte,\\ 
20 & wandi\textit{n} doch niht \textbf{tohte},\\ 
 & daz velt gegen überkraft ze \textbf{haben}.\\ 
 & si wâren entwichen gegen dem graben.\\ 
 & Den burgæren manege tjost dâ bôt\\ 
 & ein rîter allenthalben rôt.\\ 
25 & der hiez der ungenante,\\ 
 & wandin nieman dâ bekante.\\ 
 & ich sagez, alsich\textbf{z} hân vernomen:\\ 
 & er was ze Melyanze komen\\ 
 & dâr vor anme driten tage.\\ 
30 & des kômen die burgære in klage.\\ 
\end{tabular}
\scriptsize
\line(1,0){75} \newline
T V W \newline
\line(1,0){75} \newline
\textbf{1} \textit{Initiale} T W  \textbf{6} \textit{Majuskel} T  \textbf{23} \textit{Majuskel} T  \newline
\line(1,0){75} \newline
\textbf{1} Nû hete ouch] AVch het W  $\cdot$ Britun] brittvn V \textbf{2} erkantnisse] bekantnus W  $\cdot$ einen gampilûn] ein kampelvn V ein garapilun W \textbf{3} einweder ûf] Auff dem W  $\cdot$ schilt] [*]: den schilt V dem schilt W \textbf{4} Ylinotes] ẏlinotes V ilinotes W  $\cdot$ gezilt] gehilt W \textbf{6} mac] [*]: moͤhte V  $\cdot$ tuon] getvn V \textbf{7} disiu] dise T [d*]: die V \textbf{11} \textit{Versfolge 383.9-10-11-12} W   $\cdot$ den] der V W \textbf{12} über liefen im] [*]: liefen v́ber V \textbf{10} brâhtin] Brahte [*]: Gawanen in V Brachte gawan W  $\cdot$ jâmers] \textit{om.} W \textbf{13} sich] \textit{om.} V W  $\cdot$ Britanie] Britanîe T britange W \textbf{14} tûren] [*]: Svz dvren V  $\cdot$ der] dem W \textbf{16} dâ] \textit{om.} V W \textbf{17} Melyanzes] meliachganzes V melianzes W \textbf{19} mans in] manz in T man ins W \textbf{20} wandin] wandim T  $\cdot$ tohte] endohte V getochte W \textbf{21} überkraft] eúwer krafft W  $\cdot$ ze haben] [*]: zebehaben V \textbf{23} dâ] do V W \textbf{26} nieman dâ bekante] do niemant erkande W \textbf{27} sagez] sag úch W \textbf{28} Melyanze] melŷanze T melẏanze V melianz W \newline
\end{minipage}
\end{table}
\end{document}
