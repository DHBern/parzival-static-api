\documentclass[8pt,a4paper,notitlepage]{article}
\usepackage{fullpage}
\usepackage{ulem}
\usepackage{xltxtra}
\usepackage{datetime}
\renewcommand{\dateseparator}{.}
\dmyyyydate
\usepackage{fancyhdr}
\usepackage{ifthen}
\pagestyle{fancy}
\fancyhf{}
\renewcommand{\headrulewidth}{0pt}
\fancyfoot[L]{\ifthenelse{\value{page}=1}{\today, \currenttime{} Uhr}{}}
\begin{document}
\begin{table}[ht]
\begin{minipage}[t]{0.5\linewidth}
\small
\begin{center}*D
\end{center}
\begin{tabular}{rl}
\textbf{149} & im kunde niemen vîent sîn.\\ 
 & \textbf{dô besach} in ouch diu künegîn,\\ 
 & ê si schiede von dem palas,\\ 
 & \textbf{dâ si dâ vor} begozzen was.\\ 
5 & \textbf{Artus} an de\textit{n} knappen sach,\\ 
 & zuo dem tumben er dô sprach:\\ 
 & "junchêrre, got vergelde \textbf{iu} gruoz,\\ 
 & den ich gerne \textbf{dienen} muoz\\ 
 & mit \textbf{dem} lîbe unt mit \textbf{dem} guote.\\ 
10 & des ist mir wol ze muote."\\ 
 & "\textbf{wolt êt} got, wan wære \textbf{daz} wâr!\\ 
 & \textbf{der} wîle dunket mich ein jâr,\\ 
 & daz ich niht ritter wesen sol.\\ 
 & daz tuot mir wirs denne wol.\\ 
15 & nû\textbf{ne} sûmet mich niht mêre,\\ 
 & pflegt mîn nâch ritters êre."\\ 
 & "daz tuon ich gerne", sprach der wirt,\\ 
 & "ob werdecheit mich niht verbirt.\\ 
 & \begin{large}D\end{large}û bist wol sô gehiure,\\ 
20 & rîch an koste \textbf{stiure}\\ 
 & wirt dir mîn gâbe undertân.\\ 
 & dêswâr, ich \textbf{sol}z ungerne lân.\\ 
 & dû solt \textbf{unze} morgen beiten,\\ 
 & ich wil dich wol bereiten."\\ 
25 & der wol geborne knappe\\ 
 & hielt \textbf{gagernde} als ein trappe.\\ 
 & er sprach: "i\textbf{ne} wil hie nihtes bîten.\\ 
 & mir kom ein ritter widerriten.\\ 
 & mac mir des harnasch werden niht,\\ 
30 & ine ruoche, wer küneges gâbe giht.\\ 
\end{tabular}
\scriptsize
\line(1,0){75} \newline
D \newline
\line(1,0){75} \newline
\textbf{19} \textit{Initiale} D  \newline
\line(1,0){75} \newline
\textbf{5} den] dem D \newline
\end{minipage}
\hspace{0.5cm}
\begin{minipage}[t]{0.5\linewidth}
\small
\begin{center}*m
\end{center}
\begin{tabular}{rl}
 & ime kunde niemen vîent sîn.\\ 
 & \textbf{dô besach} in ouch diu künigîn,\\ 
 & ê si schiede von dem palas,\\ 
 & \textbf{d\textit{â} si vor} begozzen was.\\ 
5 & \textbf{Artus} an den knappen sach,\\ 
 & zuo dem tumben er dô sprach:\\ 
 & "junchêrre, got vergelte \textbf{iuwer} gruoz,\\ 
 & den ich \textbf{vil} gerne \textbf{dienen} muoz\\ 
 & mit lîbe und \textbf{ouch} mit guote.\\ 
10 & des ist mir wol ze muote."\\ 
 & "\textbf{wolte} got, wanne wær \textbf{ez} wâr!\\ 
 & \textbf{der} wîle dunket mich ein jâr,\\ 
 & daz ich niht ritter wesen sol.\\ 
 & daz tuot mir wirs danne wol.\\ 
15 & nû \textbf{en}s\textit{û}met mich niht mêre,\\ 
 & pflegt mîn nâch ritters êre."\\ 
 & "daz \textit{tuon} ich gerne", sprach der wirt,\\ 
 & "ob wirdicheit mich niht verbirt.\\ 
 & dû bist wol sô gehiure,\\ 
20 & rîch an koste \textbf{stiure}\\ 
 & wir\textit{t} dir mîn gâbe undertân.\\ 
 & daz ist wâr, ich \textbf{solte} ez ungerne lân.\\ 
 & dû solt \textbf{unz} morne beiten,\\ 
 & ich wil dich wol bereiten."\\ 
25 & der wolgeborne knappe\\ 
 & hielt \textbf{dâ gernde} als ein trappe.\\ 
 & er sprach: "i\textbf{ne} wil hie nihtes bîten.\\ 
 & mir kam ein ritter widerriten.\\ 
 & mac mir des harna\textit{s}ch werden niht,\\ 
30 & \textit{i}ch \textit{en}ruoche, wer küniges gâbe giht.\\ 
\end{tabular}
\scriptsize
\line(1,0){75} \newline
m n o \newline
\line(1,0){75} \newline
\newline
\line(1,0){75} \newline
\textbf{4} dâ] Do m n o \textbf{5} Artus] Artuͯs o \textbf{6} tumben] tamber o \textbf{7} iuwer] vwers o \textbf{8} ich] ich ich o \textbf{10} des] Das n o \textbf{11} wanne wær ez] vnd wer das n o \textbf{12} dunket] dunckel o \textbf{14} mir wirs] [mirs]: mir [*]: wirs m mir wurst n \textbf{15} ensûmet] ensamet m sument n (o) \textbf{17} tuon] \textit{om.} m \textbf{21} wirt] Wir m \textbf{22} solte] solte solte o \textbf{26} dâ] do n o  $\cdot$ gernde] gerne o \textbf{27} ine] ich n o  $\cdot$ bîten] beiten o \textbf{29} des] das o  $\cdot$ harnasch] harnach m harnersch o \textbf{30} ich enruoche] Mich ruche m \newline
\end{minipage}
\end{table}
\newpage
\begin{table}[ht]
\begin{minipage}[t]{0.5\linewidth}
\small
\begin{center}*G
\end{center}
\begin{tabular}{rl}
 & im kunde niemen vîent sîn.\\ 
 & \textbf{nû sach} in ouch diu künigîn,\\ 
 & ê si schiede von dem palas,\\ 
 & \textbf{dâr ûffe si} begozzen was.\\ 
5 & \textbf{der künic} an den knappen sach,\\ 
 & zuo dem tumben er dô sprach:\\ 
 & "junchêrre, got vergelt \textbf{iu} gruoz,\\ 
 & den ich \textbf{vil} gerne \textbf{dienen} muoz\\ 
 & mit lîbe und mit guote.\\ 
10 & des ist mir wol ze muote."\\ 
 & "\textbf{daz wolt} got, wan wære \textbf{daz} wâr!\\ 
 & \textbf{der} wîle dunket mich ein jâr,\\ 
 & daz ich niht rîter wesen sol.\\ 
 & daz tuot mir wirs danne wol.\\ 
15 & nû sûmet mich niht mêre,\\ 
 & pfleget mîn nâch rîters êre."\\ 
 & "daz tuon ich gerne", sprach der wirt,\\ 
 & "obe werdicheit mich niht verbirt.\\ 
 & dû bist wol sô gehiure,\\ 
20 & rîch an koste \textbf{tiure}\\ 
 & wirt dir mîn gâbe undertân.\\ 
 & dêswâr, ich \textbf{sol}z ungerne lân.\\ 
 & dû solt \textbf{biz} morgen beiten,\\ 
 & ich wil dich wol bereiten."\\ 
25 & der wolgeborne knappe\\ 
 & hielt \textbf{gagerende} als ein trappe.\\ 
 & er sprach: "i\textbf{ne} wil hie nihts bîten.\\ 
 & mir kom ein rîter widerriten.\\ 
 & mac mir des harnasch werden niht,\\ 
30 & ine ruoche, wer küniges gâbe giht.\\ 
\end{tabular}
\scriptsize
\line(1,0){75} \newline
G I O L M Q R Z \newline
\line(1,0){75} \newline
\textbf{11} \textit{Initiale} I Q  \textbf{17} \textit{Initiale} M  \textbf{19} \textit{Initiale} O L R Z  \textbf{25} \textit{Initiale} I  \newline
\line(1,0){75} \newline
\textbf{1} im] Jn R  $\cdot$ sîn] gesin O (M) \textbf{2} nû sach] in Geshach I Do gesach O Do besach L R Da bisach M (Z) Do geschag Q  $\cdot$ in ouch] selb I ouch L \textbf{3} ê si schiede] Er si schiden M \textbf{4} begozzen] gesessen R \textbf{5} den] dem O (M) Q R \textbf{6} tumben] tvmber L  $\cdot$ dô] da M Z \textbf{7} vergelt] vergel R  $\cdot$ iu] îv >ewern< O euch ewer Q uͯwer L ev den Z  $\cdot$ gruoz] gruͦsses R \textbf{8} ich] in Q  $\cdot$ dienen] vordine M \textbf{9} Mit dem libe vnd mit dem gvte Z  $\cdot$ und] vnd auch I \textbf{11} wolt] wol Q  $\cdot$ wan] \textit{om.} I M vnde O (R) \textbf{12} wîle] wille M R \textbf{13} wesen] werdin M \textbf{14} mir] mirs O \textbf{15} nû] vnd I  $\cdot$ sûmet] ensuͯmit M (Q) (R) \textbf{16} pfleget] Pfleg Q \textbf{17} \textit{Versfolge 149.18-17} O  \textbf{18} obe] Vff M O Q  $\cdot$ mich] ivch O \textbf{19} dû] ÷v O  $\cdot$ wol sô] also M \textbf{20} rîch an] in richer I  $\cdot$ koste tiure] kostiwere O \textbf{22} dêswâr] Entzwar Q Zwar Z  $\cdot$ solz] soldez M (Q) (Z)  $\cdot$ ungerne] gerne Q \textbf{23} biz] vntze L \textbf{25} wolgeborne] wolgebrne R \textbf{26} hielt] Gie I  $\cdot$ gagerende] gagerne M \textbf{27} ine] ich I O L Q R Z  $\cdot$ nihts] nimmer I niht L (Q) (R)  $\cdot$ bîten] enbiten L lenger beyden Q beitten R \textbf{30} küniges] kvnich L  $\cdot$ giht] siht I git R \newline
\end{minipage}
\hspace{0.5cm}
\begin{minipage}[t]{0.5\linewidth}
\small
\begin{center}*T (U)
\end{center}
\begin{tabular}{rl}
 & im \textbf{en}kunde nieman vîent sîn.\\ 
 & \textbf{dô sach} in ouch diu künegîn,\\ 
 & ê si schied\textit{e} von dem palas,\\ 
 & \textbf{d\textit{â} si dâ vor} begozzen was.\\ 
5 & \textbf{der künec} an de\textit{n} knappen sach,\\ 
 & zuo dem tumben er dô sprach:\\ 
 & "junchêrre, got vergelt \textbf{iu} gruoz,\\ 
 & den ich \textbf{vil} gerne \textbf{gedienen} muoz\\ 
 & mit lîbe und mit guote.\\ 
10 & des ist mir wol zuo muote."\\ 
 & "\textbf{daz wolte} got, wen wære \textbf{daz} wâr!\\ 
 & \textbf{die} wîle dunket mich ein jâr,\\ 
 & daz ich niht rîter wesen sol.\\ 
 & daz tuot mir wirs dan wol.\\ 
15 & nû sûmet mich niht mêre,\\ 
 & pfleget mîn nâch rîters êre."\\ 
 & "daz tuon ich gerne", sprach der wirt,\\ 
 & "ob wirdecheit mich niht verbirt.\\ 
 & dû bist wol sô gehiure,\\ 
20 & rîche an koste \textbf{tiure}\\ 
 & wirt dir mîn gâbe undertân.\\ 
 & dêswâr, ich \textbf{sol} ez \textit{u}ngerne lân.\\ 
 & dû solt \textbf{biz} morne beiten,\\ 
 & ich wil dich wol bereiten."\\ 
25 & \begin{large}D\end{large}er wolgeborn knappe\\ 
 & hielt \textbf{gagernde} als ein trappe.\\ 
 & er sprach: "ich wil hie nihtes bîten.\\ 
 & mir kam ein rîter widerriten.\\ 
 & mag mir des harnasch werden niht,\\ 
30 & ich enruoche, wer küneges gâbe giht.\\ 
\end{tabular}
\scriptsize
\line(1,0){75} \newline
U V W T \newline
\line(1,0){75} \newline
\textbf{2} \textit{Majuskel} T  \textbf{5} \textit{Majuskel} T  \textbf{7} \textit{Majuskel} T  \textbf{11} \textit{Majuskel} T  \textbf{23} \textit{Majuskel} T  \textbf{25} \textit{Initiale} U V T  \newline
\line(1,0){75} \newline
\textbf{1} enkunde] enknde V kunde W (T) \textbf{2} sach] besah T \textbf{3} schiede] schieden U \textbf{4} dâ si dâ vor] Do sie da vor U (W) dar vffe si T \textbf{5} den] dem U [dem]: den V \textbf{6} mit svezen worten er sprach T  $\cdot$ tumben] knappen W \textbf{7} gruoz] den gruͦß W \textbf{8} gedienen] dienen V \textbf{9} Mit dem leib vnd mit dem guͦt W \textbf{11} wære daz] wer es W (T) \textbf{12} die wîle] Der wille W der wile T \textbf{14} wirs] wúrster W \textbf{19} wol] \textit{om.} V \textbf{20} koste tiure] kosteúre W \textbf{22} sol ez] soltz T  $\cdot$ ungerne] invngerne U \textbf{24} wol] \textit{om.} W \textbf{26} Gagieret alß ein rappe W \textbf{27} ich] ine T  $\cdot$ nihtes bîten] nihtes beiten U nihtes bitten V nicht beiten W niht [bîten]: biten T \textbf{30} ich enruoche] sone ich T  $\cdot$ giht] git U V W \newline
\end{minipage}
\end{table}
\end{document}
