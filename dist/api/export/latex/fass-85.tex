\documentclass[8pt,a4paper,notitlepage]{article}
\usepackage{fullpage}
\usepackage{ulem}
\usepackage{xltxtra}
\usepackage{datetime}
\renewcommand{\dateseparator}{.}
\dmyyyydate
\usepackage{fancyhdr}
\usepackage{ifthen}
\pagestyle{fancy}
\fancyhf{}
\renewcommand{\headrulewidth}{0pt}
\fancyfoot[L]{\ifthenelse{\value{page}=1}{\today, \currenttime{} Uhr}{}}
\begin{document}
\begin{table}[ht]
\begin{minipage}[t]{0.5\linewidth}
\small
\begin{center}*D
\end{center}
\begin{tabular}{rl}
\textbf{85} & \textbf{dô bôt man} in \textbf{ir} trinken dar\\ 
 & in \textbf{manegem steine} \textbf{lieht} gevar,\\ 
 & smareide unt sardîn.\\ 
 & \textbf{etslîcher was ein rubîn}.\\ 
5 & vür\textbf{z poulûn} dô reit\\ 
 & zwêne ritter ûf ir sicherheit.\\ 
 & die wâren hin \textbf{ûz} gevangen\\ 
 & unt kômen \textbf{her} în gegangen.\\ 
 & \textbf{daz eine, daz was} Kaylet.\\ 
10 & \textbf{der} sach den künec Gahmuret\\ 
 & sitzen, als er wære unvrô.\\ 
 & er sprach: "wie gebârest dû sô?\\ 
 & \textit{\begin{large}D\end{large}}în prîs ist doch dâ vür erkant,\\ 
 & vrôn Herzeloyden unt ir lant\\ 
15 & \textbf{hât dîn lîp} errungen.\\ 
 & des jehent \textbf{hie} \textbf{gar} die zungen.\\ 
 & \textbf{er} sî Bertun oder Yrschman\\ 
 & \textbf{oder} swer hie welhisch sprâche kan,\\ 
 & Franzois oder Brabant,\\ 
20 & die jehent unt volgen\textit{t} dîner hant,\\ 
 & dir \textbf{enkünne} an sô \textbf{bewantem} \textbf{spiln}\\ 
 & glîche niemen \textbf{hie} geziln.\\ 
 & des lis ich \textbf{hie} den wâren brief.\\ 
 & dîn kraft mit ellen dô niht slief,\\ 
25 & dô dise \textbf{hêrren} kômen in nôt,\\ 
 & \textbf{der} hant \textbf{nie} sicherheit \textbf{gebôt}:\\ 
 & \textbf{mîn hêr} Brandelidelin\\ 
 & unt der küene Læhelin,\\ 
 & Hardiz unt Schafillor.\\ 
30 & \textbf{owê}, Razalik, der môr,\\ 
\end{tabular}
\scriptsize
\line(1,0){75} \newline
D \newline
\line(1,0){75} \newline
\textbf{13} \textit{Initiale} D  \newline
\line(1,0){75} \newline
\textbf{3} sardîn] Sardyn D \textbf{10} Gahmuret] Gahmvret D \textbf{13} Dîn] ÷in D \textbf{17} Yrschman] ẏrschman D \textbf{18} welhisch] wælhisch D \textbf{20} volgent] volgen D \textbf{29} Hardiez vnt Scaffillor D \textbf{30} Razalik] Razalich D \newline
\end{minipage}
\hspace{0.5cm}
\begin{minipage}[t]{0.5\linewidth}
\small
\begin{center}*m
\end{center}
\begin{tabular}{rl}
 & \textbf{dô bôt man} in \textbf{daz} trinken dar\\ 
 & in \textbf{manigen steinen} \textbf{wol} gevar,\\ 
 & smaragte und sardîn.\\ 
 & \textbf{etslîcher was ein rubîn}.\\ 
5 & \begin{large}V\end{large}ür \textbf{die pavelûne} dô reit\\ 
 & zwêne ritter ûf ir sicherheit.\\ 
 & die wâren hin \textbf{ûz} gevangen\\ 
 & und k\textit{ô}men \textbf{her} în gegangen.\\ 
 & \textbf{daz eine, daz was} Kailet.\\ 
10 & \textbf{der} sach den künic Gahmuret\\ 
 & sitzen, als er wære unvrô.\\ 
 & er sprach: "wie gebârest dû sô?\\ 
 & dîn brîs ist doch dâr vür erkant,\\ 
 & vrowen He\textit{r}czeloiden und ir lant\\ 
15 & \textbf{hât dîn lîp} errungen.\\ 
 & des jehent \textbf{hie} die zungen.\\ 
 & \textbf{ez} \textit{sî} Br\textit{i}tu\textit{n} oder irisch man\\ 
 & \textbf{oder} wer hie welsche sprâche kan,\\ 
 & Franzois oder Brabant,\\ 
20 & die jehent und volgent dîner hant,\\ 
 & dir \dag enkome\dag  an sô \textbf{bewanten} \textbf{ziln}\\ 
 & glîch niemen \textbf{hie} geziln.\\ 
 & des lise ich \textbf{\textit{h}i\textit{e}} den wâren brief.\\ 
 & dîn kraft mit ellen dô niht slief,\\ 
25 & dô dis\textit{e} \textbf{hêrren} kômen in nôt,\\ 
 & \textbf{der} hant \textbf{nie} sicherheit \textbf{gebôt}:\\ 
 & \textbf{mîn hêrre} Brandelidelin\\ 
 & und der küene Lehelin,\\ 
 & Hardiz und Schaffillor.\\ 
30 & \textbf{owî}, Razalic, der môr,\\ 
\end{tabular}
\scriptsize
\line(1,0){75} \newline
m n o \newline
\line(1,0){75} \newline
\textbf{5} \textit{Initiale} m   $\cdot$ \textit{Capitulumzeichen} n  \newline
\line(1,0){75} \newline
\textbf{1} man in] mann o \textbf{2} steinen] steine o \textbf{3} smaragte] Smaragde n o \textbf{4} rubîn] rúbin o \textbf{5} pavelûne] paneluͦm o \textbf{7} die] Do o \textbf{8} kômen] kumen m \textbf{9} Kailet] kaylet n kaẏlet o \textbf{10} Gahmuret] gamiret n gamuͯret o \textbf{11} er wære] wer er o \textbf{14} vrowen] Froͯwe m (n) (o)  $\cdot$ Herczeloiden] herecze loÿden m hertzoleiden n hercze leide o \textbf{15} errungen] er ruͯnigen o \textbf{16} des] Das o  $\cdot$ hie] hie gar n o \textbf{17} sî] \textit{om.} m  $\cdot$ Britun] bruttuͦm m britum n botten o \textbf{19} Franzois] Franczos m Frantzos n Franczois o  $\cdot$ Brabant] braband m brobrant n broͯbrant o \textbf{20} volgent] folget o \textbf{21} dir enkome] Du enkumest n (o)  $\cdot$ an sô] so an o  $\cdot$ bewanten] berantem n beranten o  $\cdot$ ziln] zil n \textbf{23} des lise] Dis liesse o  $\cdot$ hie] chi m  $\cdot$ wâren] more o \textbf{24} mit ellen dô] do mit ellen n o  $\cdot$ slief] sliefft o \textbf{25} dise] dissen m \textbf{27} Brandelidelin] brandelindelin o \textbf{29} Hardiz] Hardis m o Hardisz n  $\cdot$ Schaffillor] scaffillor m n o \textbf{30} owî] Owir n  $\cdot$ Razalic] razalit n o \newline
\end{minipage}
\end{table}
\newpage
\begin{table}[ht]
\begin{minipage}[t]{0.5\linewidth}
\small
\begin{center}*G
\end{center}
\begin{tabular}{rl}
 & \textbf{man bôt} in \textbf{daz} tri\textit{n}ken dar\\ 
 & in \textbf{manigem steine} \textbf{wol} gevar,\\ 
 & smaragde und sardîn.\\ 
 & \textbf{etslîcher was ein rubîn}.\\ 
5 & vür \textbf{daz pavelûn} dô reit\\ 
 & zwêne rîter ûf ir sicherheit.\\ 
 & die wâren hin \textbf{ûz} gevangen\\ 
 & unde kômen \textbf{hin} în gegangen.\\ 
 & \textbf{da\textit{z w}as \textit{einer}} \textit{K}ailet.\\ 
10 & \textit{\textbf{er}} sach den künic Gahmuret\\ 
 & sitzen, alser wære unvrô.\\ 
 & er sprach: "wie gebârstû sô?\\ 
 & dîn prîs ist doch dâ vür erkant,\\ 
 & vrôn Herzeloiden und ir lant\\ 
15 & \textbf{dîn lîp hât} errungen.\\ 
 & des jehent \textbf{dir} \textbf{gar} die zungen.\\ 
 & \textbf{ez} sî Britun oder yrisch man\\ 
 & \textbf{oder} swer hie wa\textit{l}sche sprâche kan,\\ 
 & Franzois oder Brabant,\\ 
20 & die jehent und volgent dîner hant,\\ 
 & \textbf{daz} dir an sô \textbf{gewanten} \textbf{spilen}\\ 
 & gelîch niemen \textbf{müge} gezilen.\\ 
 & des lis ich \textbf{hie} den wâren brief.\\ 
 & dîn kraft mit ellen dô niht slief,\\ 
25 & dô dise \textbf{helde} kômen in nôt,\\ 
 & \textbf{der} hant \textbf{nie} sicherheit \textbf{gebôt}:\\ 
 & \textbf{der stolze} Brandelidelin\\ 
 & unt der küene Lehelin,\\ 
 & Hardiz und Tschaffilor.\\ 
30 & \textbf{\textbf{owê}}, Razalic, der môr,\\ 
\end{tabular}
\scriptsize
\line(1,0){75} \newline
G I O L M Q R Z \newline
\line(1,0){75} \newline
\textbf{1} \textit{Initiale} O M  \textbf{5} \textit{Initiale} L Q R Z  \newline
\line(1,0){75} \newline
\textbf{1} man] ÷an O  $\cdot$ bôt] het O  $\cdot$ in] ir I  $\cdot$ trinken] trichen G \textbf{2} in] \textit{om.} O L  $\cdot$ manigem] Mangen O L manige M  $\cdot$ wol] wol bekant Q \textbf{3} smaragde] smaragede G Smaraide I Smaragkte M Smaragden R \textbf{4} rubîn] rubein Q \textbf{5} daz] die L den M  $\cdot$ dô] da Z \textbf{6} ir sicherheit] sicherheit I irricherheit O \textbf{8} kômen] kam Q  $\cdot$ hin] dar O L M R Z \textbf{9} daz eine daz was kailet G  $\cdot$ Kailet] Gahilet I kaylet O R kaẏlet L kalheit Q Gailet Z \textbf{10} er] vnde G Der L Q Z  $\cdot$ Gahmuret] Gamvret O Gahmuͯret L gamuret M Z gamúret Q \textbf{11} sitzen] Siczende R  $\cdot$ alser] als ob er R sam er Z  $\cdot$ wære] \textit{om.} I \textbf{12} er sprach] Da sprach her M Do sprach er Z  $\cdot$ gebârstû] geberstu M \textbf{13} doch] avch O (R) \textbf{14} vrôn] [frow]: frowe I Frav O (L) (Q) (R)  $\cdot$ Herzeloiden] herzenlavde I herzenlavden O Hertzelauden L herczeloiden M herzeloúden Q herczelaude R herzelovden Z \textbf{15} dîn lîp hât] Hat din lip O (L) (M) (Q) (R) Z  $\cdot$ errungen] betwungen I \textbf{16} dir] \textit{om.} Q  $\cdot$ gar] all I \textbf{17} Britun] Brittvn L brituͯn M britún Q britisch R  $\cdot$ yrisch man] ẏrsch man G yrisk man I irschman O Jresch man L irsch man M Z erschman Q irisch man R \textbf{18} oder] Vnd L  $\cdot$ swer] wer L M Q R Z  $\cdot$ hie] \textit{om.} Q  $\cdot$ walsche] wahsche G welisch I welsche O L M [werlische]: wellische Q welsch R welisch Z  $\cdot$ sprâche] sprechen L R \textit{om.} Z \textbf{19} Franzois] franzoẏs G fronzoys I Frantzois L Z Francios M Franczoisz Q Franzoys R  $\cdot$ oder] vnd I  $\cdot$ Brabant] brofant Q pravant Z \textbf{21} dir an] dy M dir Q  $\cdot$ sô] also M Q  $\cdot$ gewanten] gewonten M gewanttem R  $\cdot$ spilen] ziln I spil M \textbf{22} niemen] als nyman M  $\cdot$ müge] mogen Q (R)  $\cdot$ gezilen] [geziln]: gespiln I gezeln R \textbf{23} lis] liesz Q besee Z  $\cdot$ hie] hynne M (R) (Z) \textbf{24} dîn] Die Q  $\cdot$ ellen] alle Q  $\cdot$ dô] da M R Z \textbf{25} dô] Da M R Z  $\cdot$ helde] helden R \textbf{27} Brandelidelin] Brandlidelin O (Q) Brantlidelin L Brandeldelin R brandeliedelin Z \textbf{28} \textit{Vers 85.28 fehlt} R   $\cdot$ der küene] avch der chvne O (M) (Q) (Z) der kvnig L \textbf{29} Hardiz] hartiz I Hardiez O Hardis L R Hardisz M  $\cdot$ und] vnd der Q  $\cdot$ Tschaffilor] tschafillor G O scaphilor I Tschaffillor L scafillor M schaffillor R \textbf{30} owê] owe vnd I Awe O Owi wie R  $\cdot$ Razalic] [razalch]: razalich G Razalich O L razzalic Z \newline
\end{minipage}
\hspace{0.5cm}
\begin{minipage}[t]{0.5\linewidth}
\small
\begin{center}*T (U)
\end{center}
\begin{tabular}{rl}
 & \textbf{man bôt} in \textbf{daz} trinken dar\\ 
 & in \textbf{manegem steine} \textbf{wol} gevar,\\ 
 & smaragde und sardîne.\\ 
 & \textbf{etslîche wâren rubîne}.\\ 
5 & \begin{large}V\end{large}ür \textbf{daz pavelûn} dô reit\\ 
 & zwêne ritter ûf ir sicherheit.\\ 
 & die wâren hin gevangen\\ 
 & und kâmen \textbf{dar} în gegangen.\\ 
 & \textbf{der eine, daz was} Kaylet.\\ 
10 & \textbf{der} sach den künec Gahmuret\\ 
 & sitzen, als er wære unvrô.\\ 
 & er sprach: "wie gebâres dû sô?\\ 
 & dîn prîs ist doch dâ vür erkant,\\ 
 & vrô\textit{n} Herzeloyden und ir lant\\ 
15 & \textbf{hât dîn lîp} errungen.\\ 
 & des jehent \textbf{dir} die zungen.\\ 
 & \textbf{ez} sî Britun oder irisch man,\\ 
 & wer hie welsche sprâche kan,\\ 
 & Franzoys oder Brabant,\\ 
20 & die jehent und volgent dîner hant,\\ 
 & \textbf{daz} dir an sô \textbf{gewanten} \textbf{spiln}\\ 
 & glîche nieman \textbf{muge} gez\textit{i}ln.\\ 
 & des lise ich den wâren brief.\\ 
 & dîn kraft mit \textit{e}llen dô niht slief,\\ 
25 & dô dise \textbf{helde} kâmen in nôt,\\ 
 & \textbf{diu} hant \textbf{dir} sicherheit \textbf{dâ} \textbf{bôt}:\\ 
 & \textbf{der stolze} Brandelidelin\\ 
 & und \textbf{ouch} der küene Lehelin,\\ 
 & Hardiz und Schafillor\\ 
30 & \textbf{und} Razalic, der môr,\\ 
\end{tabular}
\scriptsize
\line(1,0){75} \newline
U V W T \newline
\line(1,0){75} \newline
\textbf{5} \textit{Initiale} U W T  \textbf{17} \textit{Majuskel} T  \textbf{27} \textit{Majuskel} T  \textbf{30} \textit{Majuskel} T  \newline
\line(1,0){75} \newline
\textbf{2} in manegem steine] Mangen teúren stain W \textbf{3} smaragde] Schmaragde W  $\cdot$ sardîne] sardin W (T) \textbf{4} Etlicher was ein rubin W (T)  $\cdot$ rubîne] Ruͦbine U \textbf{5} Vür daz] DVrch das W Dur die T \textbf{6} ir] des W \textbf{7} hin] hin vs V (T) \textbf{9} der eine daz was] Der was einer W daz eine was T  $\cdot$ Kaylet] Kaẏlet V gaylet W \textbf{10} der] vnde V er T  $\cdot$ Gahmuret] Gahmuͦret U Gamuret V (W) \textbf{11} als] sam T \textbf{12} er sprach] Er sprach nefe W do sprach er T \textbf{13} dâ] \textit{om.} T \textbf{14} vrôn] Vroin U Frauw W  $\cdot$ Herzeloyden] Herzeleide U Hertzelauden V hertzeloyden W \textbf{15} errungen] do errungen W errvnden T \textbf{16} des jehent] Das iehen W  $\cdot$ dir] dir gar V W T \textbf{17} ez] Er W  $\cdot$ Britun] Brituͦn U Brittun V Britv̂n T  $\cdot$ irisch] Jrisch U V hieescher W Ĵrisch T \textbf{18} wer] swer V oder swer T  $\cdot$ welsche] welsce T \textbf{19} Franzoys] Frantzoẏs V Eranzoyß W  $\cdot$ Brabant] braband W \textbf{20} jehent] iehen W \textbf{21} an sô gewanten] so gewantem W \textbf{22} geziln] gezeln U \textbf{23} lise] ließe W  $\cdot$ ich] ich hie V W T  $\cdot$ den] die W \textbf{24} ellen] allen U \textbf{25} kâmen] koment V \textbf{26} diu hant dir] der hant [*]: nie V der hant nie T Der hand nie me W  $\cdot$ dâ bôt] gebot V W (T) \textbf{28} der küene] herre T \textbf{29} Hardiz] Hardyz U Hardis V W  $\cdot$ und] de W  $\cdot$ Schafillor] schaffellor U Schaffilor V schartfillor W ScafilloR T \textbf{30} und] Vnd owe W Oͮwe T  $\cdot$ Razalic] Ratzalig V razzalig W \newline
\end{minipage}
\end{table}
\end{document}
