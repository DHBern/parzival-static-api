\documentclass[8pt,a4paper,notitlepage]{article}
\usepackage{fullpage}
\usepackage{ulem}
\usepackage{xltxtra}
\usepackage{datetime}
\renewcommand{\dateseparator}{.}
\dmyyyydate
\usepackage{fancyhdr}
\usepackage{ifthen}
\pagestyle{fancy}
\fancyhf{}
\renewcommand{\headrulewidth}{0pt}
\fancyfoot[L]{\ifthenelse{\value{page}=1}{\today, \currenttime{} Uhr}{}}
\begin{document}
\begin{table}[ht]
\begin{minipage}[t]{0.5\linewidth}
\small
\begin{center}*D
\end{center}
\begin{tabular}{rl}
\textbf{19} & aht ors \textbf{mit} zindâle\\ 
 & verdecket \textbf{al} ze mâle.\\ 
 & daz niunde sînen satel truoc.\\ 
 & ein schilt, des \textit{ich} ê gewuoc,\\ 
5 & \textbf{den} vuorte ein knappe vil gemeit\\ 
 & \textbf{dar} \textbf{bî}. \textbf{nâch} \textbf{den} selben reit\\ 
 & pusûner, der man \textbf{noch} bedarf.\\ 
 & ein tambûrer sluog unt warf\\ 
 & hôhe \textbf{sîne} tambûr.\\ 
10 & den hêrren nam vil untûr,\\ 
 & \textbf{dâ}\textbf{ne} riten floitierre bî\\ 
 & unt \textbf{guoter} videlære drî.\\ 
 & den was allen niht ze gâch.\\ 
 & selbe reit er hinden nâch\\ 
15 & unt sîn marnære,\\ 
 & der wîse unt der \textbf{mære}.\\ 
 & \begin{large}S\end{large}waz dâ was volkes inne,\\ 
 & môre unt mœrinne\\ 
 & was beidiu wîp unt man.\\ 
20 & der hêrre schouwen began\\ 
 & manegen schilt zerbrochen,\\ 
 & mit spern gar \textbf{durchstochen}.\\ 
 & der was dâ vil gehangen \textbf{vür}\\ 
 & an die \textbf{wende} unt an \textbf{die tür}.\\ 
25 & si heten jâmer unt guft.\\ 
 & in diu venster \textbf{gein dem} luft\\ 
 & was gebettet manege\textit{m} wunden man,\\ 
 & \textbf{swenn} er den arzât gewan,\\ 
 & daz er doch \textbf{mohte niht} genesen.\\ 
30 & \textbf{der} was bî vîenden gewesen.\\ 
\end{tabular}
\scriptsize
\line(1,0){75} \newline
D Fr9 \newline
\line(1,0){75} \newline
\textbf{17} \textit{Initiale} D  \newline
\line(1,0){75} \newline
\textbf{4} ich] \textit{om.} D  $\cdot$ gewuoc] gevuͦch Fr9 \textbf{5} vil] \textit{om.} Fr9 \textbf{6} den] dem Fr9 \textbf{7} pusûner] Eẏn busvnre Fr9 \textbf{9} sîne] sẏnen Fr9 \textbf{22} mit spern gar] Vnde mit speren Fr9 \textbf{26} dem] der Fr9 \textbf{27} manegem] manegen D \textbf{28} den] dan Fr9 \textbf{29} mohte niht] nicht mvͦchte Fr9 \newline
\end{minipage}
\hspace{0.5cm}
\begin{minipage}[t]{0.5\linewidth}
\small
\begin{center}*m
\end{center}
\begin{tabular}{rl}
 & aht ros \textbf{mit} zindâle\\ 
 & verdecket \textbf{alliu} zuo mâle.\\ 
 & daz \textit{ni}unde sînen satel truoc.\\ 
 & einen schilt, d\textit{e}s ic\textit{h} \textit{ê} gewuoc,\\ 
5 & \textbf{den} \textit{v}uorte ein knappe vil \textit{ge}meit.\\ 
 & \textbf{dô} \textbf{noch} \textbf{bî} \textbf{dem} selben reit\\ 
 & pusûn\textit{er}, der man \textbf{ouch} bedarf.\\ 
 & ein tambû\textit{r} \textit{s}luoc und warf\\ 
 & \textbf{vil} hôhe \textbf{sîne} tambûre.\\ 
10 & den hêrren nam vil untûre,\\ 
 & \textbf{denne} riten fl\textit{oit}ierr\textit{e} bî\\ 
 & und \textbf{guoter} videlære drî.\\ 
 & den was alle\textit{n} niht zuo gâch.\\ 
 & selbe reit er hinden nâch\\ 
15 & und sîn marnære,\\ 
 & der wîse und der \textbf{hêre}.\\ 
 & \begin{large}W\end{large}az d\textit{â} was volkes inne,\\ 
 & môre und mœrinne\\ 
 & was \textit{b}eid\textit{iu} wîp und man.\\ 
20 & der hêrre schouwen began\\ 
 & menigen schilt zerbrochen,\\ 
 & mit speren gar \textbf{zerstochen}.\\ 
 & der was d\textit{â} vi\textit{l} gehangen \textbf{vor}\\ 
 & \textit{an} die \textbf{wende} und an \textbf{diu tor}.\\ 
25 & si heten jâmer und guft.\\ 
 & in diu venster \textbf{durch den} luft\\ 
 & was gebettet menigem wunden man.\\ 
 & \textbf{swin\textit{d}e} er den arzet gewan,\\ 
 & daz er doch \textbf{niht moht} genesen.\\ 
30 & \textbf{er} was bî vîenden gewesen.\\ 
\end{tabular}
\scriptsize
\line(1,0){75} \newline
m n o \newline
\line(1,0){75} \newline
\textbf{17} \textit{Initiale} m  \newline
\line(1,0){75} \newline
\textbf{1} mit] mit mit o \textbf{2} alliu zuo] alzuͦ n o \textbf{3} niunde] in vnd m n o \textbf{4} des ich ê] das ich das ich e m den ich E n  $\cdot$ ê] ie o \textbf{5} vuorte] fruͦrte \textit{nachträglich korrigiert zu:} fuͦrte m  $\cdot$ vil gemeit] viel meit \textit{nachträglich korrigiert zu:} vielgmeit m \textbf{6} dô] Dar n o  $\cdot$ dem] den n o \textbf{7} pusûner] Pusunort \textit{nachträglich korrigiert zu:} pfruant m \textbf{8} tambûr] tambure m tamber o  $\cdot$ sluoc] fluͦg \textit{nachträglich korrigiert zu:} sluͦg m \textbf{10} nam] \textit{om.} o \textbf{11} denne] Den n o  $\cdot$ floitierre] flancieren m flantiere n o \textbf{13} allen] alle m  $\cdot$ gâch] joch o \textbf{14} selbe] Selbes n o \textbf{15} marnære] mannere n (o) \textbf{17} Waz] Das n o  $\cdot$ dâ] do m n \textbf{18} môre] Moͯren n \textbf{19} was] [Mas]: Was m  $\cdot$ beidiu] leide m \textbf{21} zerbrochen] zerhouwen o \textbf{22} gar zerstochen] durch stochen n gar durch strouwen o \textbf{23} dâ] do m n  $\cdot$ vil] viele m \textbf{24} an] Zuͦ m \textbf{26} den] die o \textbf{27} gebettet] gebotten n o \textbf{28} swinde] Swinne \textit{nachträglich korrigiert zu:} Kvme m \textbf{30} bî] bẏ den n \newline
\end{minipage}
\end{table}
\newpage
\begin{table}[ht]
\begin{minipage}[t]{0.5\linewidth}
\small
\begin{center}*G
\end{center}
\begin{tabular}{rl}
 & aht ors \textbf{von} zendâle\\ 
 & verdecket \textbf{al} ze mâle.\\ 
 & daz niunde sînen satel truoc.\\ 
 & einen schilt, des ich ê gewuoc,\\ 
5 & vuorte ein knappe vil gemeit.\\ 
 & \textbf{dâ} \textbf{hinden} \textbf{nâch} \textbf{dem} selben reit\\ 
 & busûnære, der man \textbf{ouch} bedarf.\\ 
 & ein tambûr sluoc und warf\\ 
 & \textbf{vil} hôhe \textbf{sînen} tambûr.\\ 
10 & den hêrren nam vil untûr,\\ 
 & \textbf{dâ} riten floitierære bî\\ 
 & unde \textbf{walscher} videlære drî.\\ 
 & den was allen niht ze gâch.\\ 
 & selbe reit er hinden nâch\\ 
15 & unde sîn marnære,\\ 
 & der wîse und der \textbf{mære}.\\ 
 & swaz dâ was volkes inne,\\ 
 & môr und mœrinne\\ 
 & was beidiu wîb und man.\\ 
20 & der hêrre schouwen began\\ 
 & manigen schilt zerbrochen,\\ 
 & \begin{large}M\end{large}it spern gar \textbf{durchstochen}.\\ 
 & der was dâ vil gehangen \textbf{vür}\\ 
 & an die \textbf{wende} und an \textbf{die tür}.\\ 
25 & si heten jâmer und guft.\\ 
 & in diu venster \textbf{gein dem} luft\\ 
 & was gebettet manigem wunden man,\\ 
 & \textbf{swenn}er den arzât gewan,\\ 
 & daz er doch \textbf{mohte niht} genesen.\\ 
30 & \textbf{der} was bî vînden gewesen.\\ 
\end{tabular}
\scriptsize
\line(1,0){75} \newline
G O L M Q R W Z Fr29 Fr32 Fr36 Fr71 \newline
\line(1,0){75} \newline
\textbf{1} \textit{Initiale} O M Fr29  \textbf{11} \textit{Initiale} Fr71  \textbf{13} \textit{Versal} Fr32  \textbf{17} \textit{Initiale} L R W Z Fr32  \textbf{22} \textit{Initiale} G  \textbf{29} \textit{Versal} Fr32  \newline
\line(1,0){75} \newline
\textbf{1} aht] ÷ht O Fr29  $\cdot$ von] mit O L M Q R Z Fr29 Fr32 \textbf{2} verdecket] werdicheit Fr71  $\cdot$ al] als R alle W \textit{om.} Fr71 \textbf{4} einen] [S]: Eynen M  $\cdot$ ê] uch E L (W) y M  $\cdot$ gewuoc] gewung M gewick \textit{nachträglich korrigiert zu} gewuͯk Q genuͦg R \textbf{5} vuorte] Den [rvͦrt]: fvͦrt O Den fuͯrte L (M) (Q) (Z) (Fr29) (Fr32) (Fr36)  $\cdot$ knappe] \textit{om.} O knecht R  $\cdot$ vil] \textit{om.} L W wol R \textbf{6} hinden nâch] bî nach O (L) (M) (Q) (R) (Fr29) (Fr32) (Fr36) nach bi Z (Fr71)  $\cdot$ selben] selbigen M selbe Fr71 \textbf{7} busûnære] [Bvsvnavnere]: Busvnaver O Busawmere Q  $\cdot$ der] des Q  $\cdot$ bedarf] da bedarf Fr71 \textbf{8} Der ieglicher seine hoͤhe warff W  $\cdot$ ein] Syn M  $\cdot$ tambûr] tamburare L kamburen Q taumbaurer Fr36 twerch Fr71 \textbf{9} Ieglicher sein tanbúre W  $\cdot$ sînen] sine R Z \textbf{10} do nam in des vil vntowr Fr71  $\cdot$ vil] dez L (W)  $\cdot$ untûr] wundir M \textbf{11} dâ riten] Do riten Q Da enritten R (Fr29) (Fr32) (Fr36) Do entritten W  $\cdot$ floitierære] flontirie Q floitiern R \textbf{12} walscher] guter Q (R) (Z) (Fr32) welscher W Fr29  $\cdot$ videlære] fidelerer Q videnler R \textbf{13} \textit{Versfolge 19.14-13} Q   $\cdot$ den] Jn Q \textbf{14} selbe] Selben M (Q) \textbf{15} marnære] mannere M (Fr32) \textbf{16} der mære] der gewêre Fr32 \textbf{17} \sout{Was da was volkes Jnne mere} Was da was volkes mere Jnne R  $\cdot$ swaz] Waz L (M) (W) o\textit{m. } Q  $\cdot$ dâ] do O (Q) (W)  $\cdot$ was volkes inne] folkes ynne was M wasz vockes inne Q was leute inne W \textbf{18} môr] Moren Q Z Moͯrin R  $\cdot$ und] vnd auch W \textbf{19} was] \textit{om.} W \textbf{21} manigen] Vil manigen O (L) (M) (Q) (R) (Fr32) \textbf{22} spern] sper O  $\cdot$ gar] dar O \textit{om.} R  $\cdot$ durchstochen] zu stochen Q \textbf{23} dâ] do Q W  $\cdot$ vil] vile M \textbf{24} an] and O  $\cdot$ tür] thor M \textbf{25} si] So M \textbf{26} in diu venster] An div fenster O Jnden fenstren M in dem venster Fr71  $\cdot$ gein dem] an den L gein der M (Q) vnd an die W \textbf{27} gebettet] gettent Q  $\cdot$ manigem] manchen Q \textbf{28} Wer den artzat nit do gewan W  $\cdot$ swenner] Wem er L Wan her M (R) Sint er Q \textbf{29} Der mocht nicht wol genesen W  $\cdot$ mohte niht] niht mohte L (M) (Q) (R) (Fr32) (Fr71) \textbf{30} \textit{Vers 19.30 fehlt} R   $\cdot$ der] Er W  $\cdot$ bî] bi den O (L) (M) pem Q  $\cdot$ gewesen] ovch geweisen Fr71 \newline
\end{minipage}
\hspace{0.5cm}
\begin{minipage}[t]{0.5\linewidth}
\small
\begin{center}*T
\end{center}
\begin{tabular}{rl}
 & ahte ors \textbf{mit} zindâle\\ 
 & verdecket \textbf{al} ze mâle.\\ 
 & daz niunde sînen satel truoc.\\ 
 & einen schilt, des ich ê gewuoc,\\ 
5 & \textbf{den} vuorte ein knappe vil gemeit\\ 
 & \textbf{dâ} \textbf{bî}. \textbf{nâch} \textbf{dem} selben reit\\ 
 & pusînære, der man \textbf{ouch} bedarf.\\ 
 & ein tambûrer sluoc und warf\\ 
 & \textbf{vil} hôhe \textbf{sînen} tambûr.\\ 
10 & den hêrren nam vil untûr,\\ 
 & \textbf{dâ} \textbf{en}riten floitære bî\\ 
 & und \textbf{welscher} videlære drî.\\ 
 & den was allen niht ze gâch.\\ 
 & selbe reit er hinden nâch\\ 
15 & und sîn marnære,\\ 
 & der wîse und der \textbf{mære}.\\ 
 & \begin{large}S\end{large}waz dâ was volkes inne,\\ 
 & môre und mœrinne\\ 
 & was beid\textit{iu} wîp und man.\\ 
20 & der hêrre schouwen began\\ 
 & manegen schilt zerbrochen,\\ 
 & mit spern gar \textbf{durchstochen}.\\ 
 & der was dâ vil gehangen \textbf{vür}\\ 
 & an die \textbf{vensteren} und an \textbf{die tür}.\\ 
25 & Si heten jâmer und guft.\\ 
 & in di\textit{u} venster \textbf{gegen dem} luft\\ 
 & was gebettet manegem wunden man,\\ 
 & \textbf{swenn}er den arzât gewan,\\ 
 & daz er doch \textbf{mohte niht} genesen.\\ 
30 & \textbf{der} was bî vîenden gewesen.\\ 
\end{tabular}
\scriptsize
\line(1,0){75} \newline
T U V \newline
\line(1,0){75} \newline
\textbf{17} \textit{Initiale} T U V  \textbf{25} \textit{Majuskel} T  \newline
\line(1,0){75} \newline
\textbf{1} mit] von U \textbf{3} niunde] in vnd U \textbf{4} ich] [*]: ich v́ch V \textbf{8} [*i*rf]: Der iegelicher sine hoͤhe warf V \textbf{9} [*]: Vnde slvͦg meisterlich sine tambur V \textbf{11} dâ] [D*]: Do V \textbf{13} niht] zit U \textbf{16} mære] gewere U V \textbf{17} Swaz] Waz U  $\cdot$ dâ] do V \textbf{18} môre] Moren V \textbf{19} beidiu] beide T \textbf{23} dâ] do V \textbf{24} vensteren und an] wende vor U (V) \textbf{26} diu] die T \textbf{28} [swen*d*]: swer den arzat do gewan V  $\cdot$ swenner] Wen er U \textbf{29} [D*h*sen]: Der enmoͤhte niht wol genesen V  $\cdot$ doch mohte niht] nit mothe doch U \newline
\end{minipage}
\end{table}
\end{document}
