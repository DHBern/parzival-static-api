\documentclass[8pt,a4paper,notitlepage]{article}
\usepackage{fullpage}
\usepackage{ulem}
\usepackage{xltxtra}
\usepackage{datetime}
\renewcommand{\dateseparator}{.}
\dmyyyydate
\usepackage{fancyhdr}
\usepackage{ifthen}
\pagestyle{fancy}
\fancyhf{}
\renewcommand{\headrulewidth}{0pt}
\fancyfoot[L]{\ifthenelse{\value{page}=1}{\today, \currenttime{} Uhr}{}}
\begin{document}
\begin{table}[ht]
\begin{minipage}[t]{0.5\linewidth}
\small
\begin{center}*D
\end{center}
\begin{tabular}{rl}
\textbf{371} & \begin{large}S\end{large}i sprach: "\textbf{vil} wênec mich \textbf{des} bevilt.\\ 
 & ich bin iwer scherm unt iwer schilt\\ 
 & \textbf{unt} iwer herze \textbf{unt} iwer trôst,\\ 
 & sît ir mich zwîvels hât erlôst.\\ 
5 & ich bin \textbf{vür} ungevelle\\ 
 & iwer geleite unt iwer geselle,\\ 
 & vür ungelückes schûr ein dach\\ 
 & bin ich iu senfteclîch gemach.\\ 
 & mîn minne sol iu vride bern,\\ 
10 & \textbf{gelückes} \textbf{vor der} angest wern,\\ 
 & daz iwer ellen niht verbirt,\\ 
 & ir \textbf{en}wert iuch vaste unz an den wirt.\\ 
 & ich bin wirt unt wirtîn\\ 
 & unt wil \textbf{in strîte bî iu} sîn.\\ 
15 & swenne ir des gedingen hât,\\ 
 & sælde unt ellen iuch niht \textbf{en}lât."\\ 
 & Dô sprach der werde Gawan:\\ 
 & "vrouwe, ich wil beidiu hân,\\ 
 & sît ich in iwerem gebote lebe,\\ 
20 & \textbf{iwerer} minne unt \textbf{iwerre} trôstes gebe."\\ 
 & die wîle \textbf{wâren} ir hendelîn\\ 
 & zwischen den handen sîn.\\ 
 & dô sprach si: "hêrre, \textbf{nû} lât mich varn.\\ 
 & ich \textbf{muoz} ouch mich dâr an bewarn.\\ 
25 & \textbf{wie} vüeret ir âne mînen solt?\\ 
 & dar zuo \textbf{wære} ich iu \textbf{al} ze holt.\\ 
 & ich sol mich arbeiten,\\ 
 & mîn kleinœte iu bereiten.\\ 
 & swenne ir daz traget, \textbf{decheinen gewîs}\\ 
30 & überhœhet iuch \textbf{nimmer ander} prîs."\\ 
\end{tabular}
\scriptsize
\line(1,0){75} \newline
D \newline
\line(1,0){75} \newline
\textbf{1} \textit{Initiale} D  \textbf{17} \textit{Majuskel} D  \newline
\line(1,0){75} \newline
\newline
\end{minipage}
\hspace{0.5cm}
\begin{minipage}[t]{0.5\linewidth}
\small
\begin{center}*m
\end{center}
\begin{tabular}{rl}
 & si sprach: "wênic mich \textbf{daz} bevilt.\\ 
 & ich bin iuwer schi\textit{rm} und iuwer sc\textit{h}i\textit{l}t,\\ 
 & iuwer herze \textbf{und} iuwer trôst,\\ 
 & sît ir mich zwîvels habt erlôst.\\ 
5 & ich bin \textbf{vür} ungevelle\\ 
 & iuwer geleite und iuwer geselle,\\ 
 & vür ungelückes schûr ein dach\\ 
 & bin ich iu senfteclîch gemach.\\ 
 & mîn minne sol iu vride bern,\\ 
10 & \textbf{gelückes} \textbf{vor der} angest wern,\\ 
 & daz iuwer ellen niht verbirt,\\ 
 & ir \textbf{en}wert iuch vaste unz an den wirt.\\ 
 & ich bin wirt und wirtîn\\ 
 & und wil \textbf{in strîte bî iu} sîn.\\ 
15 & wenne ir des gedingen hât,\\ 
 & sælde und ellen iuch niht lât."\\ 
 & \begin{large}D\end{large}ô sprach der werde Gawan:\\ 
 & "vrouwe, ich wil beidiu hân,\\ 
 & sît ich in iuwerem gebote lebe,\\ 
20 & \textbf{iuwer} minne und \textbf{iuweres} trôstes gebe."\\ 
 & die wîle \textbf{was} ir hendelîn\\ 
 & zwischen den henden sîn.\\ 
 & dô spr\textit{ach} si: "hêrre, \textbf{nû} lât mich varn.\\ 
 & ich \textbf{muoz} ouch mich dâr an bewarn.\\ 
25 & \textbf{wie} vüeret ir âne mînen solt?\\ 
 & dar zuo \textbf{wær} ich iu \textbf{a\textit{l}} ze holt.\\ 
 & ich sol mich arbeiten,\\ 
 & mîn kleinœte iu bereiten.\\ 
 & wenne ir daz traget, \textbf{dekeine wîs}\\ 
30 & überhœhet iuch \textbf{niemen an dem} prîs."\\ 
\end{tabular}
\scriptsize
\line(1,0){75} \newline
m n o \newline
\line(1,0){75} \newline
\textbf{17} \textit{Initiale} m o   $\cdot$ \textit{Capitulumzeichen} n  \newline
\line(1,0){75} \newline
\textbf{2} schirm] schilt m  $\cdot$ schilt] schrrint m \textbf{4} erlôst] erlast o \textbf{15} gedingen] gedinge n o \textbf{19} iuwerem] irem m \textbf{20} iuwer] Jr m  $\cdot$ iuweres] ires m \textbf{23} sprach] spr m  $\cdot$ nû] \textit{om.} n  $\cdot$ lât] [fart]: lat o \textbf{24} an] in n o \textbf{26} al] alle m \textit{om.} n solt o \textbf{29} dekeine] do heine n \textbf{30} niemen] \textit{om.} o \newline
\end{minipage}
\end{table}
\newpage
\begin{table}[ht]
\begin{minipage}[t]{0.5\linewidth}
\small
\begin{center}*G
\end{center}
\begin{tabular}{rl}
 & si sprach: "\textbf{vil} wênic mich \textbf{des} bevilt.\\ 
 & ich bin iwer scherm unde iwer schilt\\ 
 & \textit{\textbf{und}} iwer herze \textbf{unde} iwer trôst,\\ 
 & sît ir mich zwîvels habet erlôst.\\ 
5 & ich bin \textbf{vür} ungevelle\\ 
 & iwer geleite unde iwer geselle,\\ 
 & vür ungelückes schûr ein dach\\ 
 & bin ich iu senfticlîch gemach.\\ 
 & mîn minne sol iu vride bern,\\ 
10 & \textbf{gelückes} \textbf{vür die} angest wern,\\ 
 & daz iwer ellen niht verbirt,\\ 
 & ir\textbf{ne} wert iuch vaste unze an den wirt.\\ 
 & ich bin wirt unde wirtîn\\ 
 & unde wil \textbf{bî iu in strîte} sîn.\\ 
15 & swenne ir des gedingen hât,\\ 
 & sælde unde ellen iuch niht lât."\\ 
 & dô sprach der werde Gawan:\\ 
 & "vrouwe, ich wil bêdiu hân,\\ 
 & sît ich in iwerm gebote lebe,\\ 
20 & \textbf{iwere} minne unde \textbf{iwers} trôstes gebe."\\ 
 & die wîle \textbf{was} ir hendelîn\\ 
 & zwischen den handen sîn.\\ 
 & dô sprach si: "hêrre, \textbf{nû} lât mich varen.\\ 
 & ich \textbf{sol} ouch mich dâr ane bewaren.\\ 
25 & \textbf{wie} vüert ir âne mînen solt?\\ 
 & dar zuo \textbf{bin} ich iu ze holt.\\ 
 & ich sol mich arbeiten,\\ 
 & mîn kleinœde iu bereiten.\\ 
 & swenne ir daz traget, \textbf{deheine wîs}\\ 
30 & überhœhet iuch \textbf{dehein ander} brîs."\\ 
\end{tabular}
\scriptsize
\line(1,0){75} \newline
G I O L M Q R Z Fr24 Fr38 \newline
\line(1,0){75} \newline
\textbf{1} \textit{Initiale} I O L M Z Fr24 Fr38   $\cdot$ \textit{Capitulumzeichen} R  \textbf{17} \textit{Initiale} I  \newline
\line(1,0){75} \newline
\textbf{1} \textit{Die Verse 370.13-412.12 fehlen} Q   $\cdot$ si] ÷i O Dy M  $\cdot$ vil] \textit{om.} I  $\cdot$ des] daz R  $\cdot$ bevilt] vilt M \textbf{2} unde iwer] vnde M  $\cdot$ scherm] schrim Fr38 \textbf{3} und iwer herze] iwer herze G  $\cdot$ trôst] Swert M \textbf{4} zwîvels] zwifelns L  $\cdot$ habet] hap I \textbf{5} \textit{Die Verse 371.5-6 fehlen} L  \textbf{8} gemach] [ein dach]: gemach I \textbf{9} vride] frevde Z \textbf{10} die] \textit{om.} I der O L M R Z Fr24 (Fr38) \textbf{12} irne] Jr O Z  $\cdot$ wert] werte Fr38 \textbf{15} swenne] Wenne L (R) \textbf{16} sælde] Selden Fr38  $\cdot$ lât] verlat I (O) \textbf{17} dô] Da O M \textbf{18} wil] wil si I (R) \textbf{19} iwerm] iwer Fr38 \textbf{20} iwers] ewer I (L) \textbf{22} zwischen] Schoͮne zwischen Fr38 \textbf{23} dô] Da M  $\cdot$ hêrre] \textit{om.} I  $\cdot$ nû] \textit{om.} O L \textbf{26} ze holt] al zeholt O (Z) \textbf{28} mîn] mit I  $\cdot$ iu] \textit{om.} I \textbf{29} swenne] Wenne L R  $\cdot$ deheine] deheinen O (Z) Fr24 (Fr38) \textbf{30} überhœhet] vber [haubet]: hohet I Vberhuͯget L Vbir houbit M (Fr24)  $\cdot$ iuch] \textit{om.} L  $\cdot$ dehein] [immer]: nimmer O niemer L (R) (Z) (Fr24) (Fr38) mynner M  $\cdot$ ander] \textit{om.} I \newline
\end{minipage}
\hspace{0.5cm}
\begin{minipage}[t]{0.5\linewidth}
\small
\begin{center}*T
\end{center}
\begin{tabular}{rl}
 & \begin{large}S\end{large}i sprach: "\textbf{vil} wênic mich \textbf{des} bevilt.\\ 
 & ich bin iuwer schirm unde iuwer schilt,\\ 
 & iuwer herze, iuwer trôst,\\ 
 & sît ir mich zwîvels habt erlôst.\\ 
5 & ich bin \textbf{iuwer} ungevelle,\\ 
 & iuwer geleite unde iuwer geselle,\\ 
 & vür ungelückes schûr ein dach\\ 
 & bin ich iu \textbf{ein} senfteclîchez gemach.\\ 
 & mîn minne sol iu vride bern,\\ 
10 & \textbf{glücke} \textbf{vor der} angest wern,\\ 
 & daz iuwer ellen niht verbirt,\\ 
 & ir wert iuch vaste unz an den wirt.\\ 
 & ich bin wirt unde wirtîn\\ 
 & unde wil \textbf{bî iu in strîte} sîn.\\ 
15 & swennir des gedinge hât,\\ 
 & sælde unde ellen iuch niht lât."\\ 
 & \begin{large}D\end{large}ô sprach der werde Gawan:\\ 
 & "vrouwe, ich wil beidiu hân,\\ 
 & sît ich in iuwerme gebote lebe,\\ 
20 & \textbf{iuwer} minne unde \textbf{iuwer} trôstes gebe."\\ 
 & die wîle \textbf{was} ir hendelîn\\ 
 & zwischen den handen sîn.\\ 
 & Dô sprach si: "hêrre, lât mich varn.\\ 
 & ich \textbf{sol} ouch mich dâr an bewarn.\\ 
25 & vüeret ir âne mînen solt,\\ 
 & dâ zuo \textbf{bin} ich iu ze holt.\\ 
 & ich sol mich arbeiten,\\ 
 & mîn kleinœte iu bereiten.\\ 
 & swennir daz traget, \textbf{deheine wîs}\\ 
30 & überhœhet iuch \textbf{niemer ander} prîs."\\ 
\end{tabular}
\scriptsize
\line(1,0){75} \newline
T V W \newline
\line(1,0){75} \newline
\textbf{1} \textit{Initiale} T W  \textbf{17} \textit{Initiale} T V  \textbf{23} \textit{Majuskel} T  \newline
\line(1,0){75} \newline
\textbf{1} vil] \textit{om.} W \textbf{2} schirm unde iuwer] schrein vnde W \textbf{3} herze] herze vnde V (W) \textbf{5} iuwer] [*]: fúr V fúr W \textbf{8} bin ich] Jch bin V  $\cdot$ ein] \textit{om.} W  $\cdot$ gemach] vngemach W \textbf{10} glücke] Glúckes W \textbf{12} wert] [*]: enwert V  $\cdot$ iuch] îv T \textbf{13} bin] \textit{om.} W \textbf{15} swennir] Wann ir W \textbf{16} iuch] iv T \textbf{20} iuwer trôstes] v́wers trostes V ewern trost W \textbf{22} zwischen] In zwischen W \textbf{23} lât] nv lant V \textbf{25} vüeret] [*]: Wie fuͤrent V \textbf{26} ze holt] [*]: alzeholt V gar zuͦ holt W \textbf{28} mîn] Mit W  $\cdot$ kleinœte] cleinoͤter V \textbf{29} swennir] Wenn ir W \textbf{30} iuch] îv T  $\cdot$ ander] [*]: an dem V \newline
\end{minipage}
\end{table}
\end{document}
