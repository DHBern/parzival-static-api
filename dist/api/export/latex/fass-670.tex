\documentclass[8pt,a4paper,notitlepage]{article}
\usepackage{fullpage}
\usepackage{ulem}
\usepackage{xltxtra}
\usepackage{datetime}
\renewcommand{\dateseparator}{.}
\dmyyyydate
\usepackage{fancyhdr}
\usepackage{ifthen}
\pagestyle{fancy}
\fancyhf{}
\renewcommand{\headrulewidth}{0pt}
\fancyfoot[L]{\ifthenelse{\value{page}=1}{\today, \currenttime{} Uhr}{}}
\begin{document}
\begin{table}[ht]
\begin{minipage}[t]{0.5\linewidth}
\small
\begin{center}*D
\end{center}
\begin{tabular}{rl}
\textbf{670} & \textit{\begin{large}N\end{large}}û diz \textbf{was êt alsus} komen:\\ 
 & Gawans rinc \textbf{was} genomen\\ 
 & durch Artuses her, al dâ \textbf{der} lac.\\ 
 & waz man schouwens dâ \textbf{gepflac}!\\ 
5 & ê diz volc durch si gerite,\\ 
 & Gawan durch \textbf{hovelîchen} site\\ 
 & unt ouch \textbf{durch} \textbf{werdeclîchiu} dinc\\ 
 & hiez \textbf{an} Artuses rinc\\ 
 & die êrsten vrouwen halden.\\ 
10 & sîn marschalc muose walden,\\ 
 & daz einiu \textbf{nâhe} zuo der reit.\\ 
 & der andern deheiniu \textbf{dô} vermeit,\\ 
 & sine \textbf{habten} sus al umbe,\\ 
 & hie \textbf{diu wîse} \textbf{unt} dort \textbf{diu tumbe},\\ 
15 & bî ieslîcher ein ritter, der ir pflac\\ 
 & \textbf{unt} der sich dienstes dâr bewac.\\ 
 & Artuses rinc, den wîten,\\ 
 & \textbf{man sach} \textbf{an} \textbf{allen} sîten\\ 
 & mit vrouwen umbevangen.\\ 
20 & dô wart alrêst enpfangen\\ 
 & Gawan, der sælden rîche,\\ 
 & ich wæne des, minneclîche.\\ 
 & Arnive, ir tohter unt ir kint\\ 
 & mit Gawane erbeizet sint,\\ 
25 & von Logroys diu herzogîn\\ 
 & unt der herzoge von Gowerzin\\ 
 & unt der Turkote Florant.\\ 
 & gein disen liuten wert erkant\\ 
 & Artus ûz dem \textbf{gezelte} gienc,\\ 
30 & der si dâ \textbf{vriuntlîche} enpfienc;\\ 
\end{tabular}
\scriptsize
\line(1,0){75} \newline
D Fr8 Fr10 \newline
\line(1,0){75} \newline
\textbf{1} \textit{Initiale} D  \textbf{17} \textit{Initiale} Fr8  \newline
\line(1,0){75} \newline
\textbf{1} Nû] ÷v D  $\cdot$ êt alsus] aber alsus Fr8 et also Fr10 \textbf{3} Artuses] Artvs D (Fr10) Arthuses Fr8 \textbf{4} schouwens] do schowens Fr8 schonhait Fr10 \textbf{6} hovelîchen] hobisliche Fr8 \textbf{8} an] in Fr8  $\cdot$ Artuses] Artvs D Arthuses Fr8 \textbf{10} walden] walde Fr10 \textbf{11} einiu nâhe] ainen naher Fr10 \textbf{12} deheiniu] dehaine Fr10 \textbf{14} unt] \textit{om.} Fr8 Fr10 \textbf{15} der] die Fr8 \textbf{16} dienstes] diens D  $\cdot$ dâr] \textit{om.} Fr10 \textbf{17} Artuses] Artvs D (Fr10) ARthuses Fr8 \textbf{18} an] in Fr8 \textbf{23} Arnive] Arnîve D Arnẏue Fr8 Arniue Fr10 \textbf{24} erbeizet] ge erbt Fr10 \textbf{25} Vnd von Logroẏs div herzogin Fr8  $\cdot$ Vnd uon Logrois diu hertzoginn Fr10 \textbf{26} Gowerzin] Gomerzin Fr10 \textbf{27} Turkote] turkoẏte Fr8 turkoite Fr10 \textbf{28} wert] wart Fr10 \textbf{29} Artus] Arthus Fr8  $\cdot$ gezelte] paulune Fr8 \newline
\end{minipage}
\hspace{0.5cm}
\begin{minipage}[t]{0.5\linewidth}
\small
\begin{center}*m
\end{center}
\begin{tabular}{rl}
 & \begin{large}N\end{large}û diz \textbf{was alsô} komen:\\ 
 & Gawans rinc \textbf{was} genomen\\ 
 & durch Artuses her, aldâ \textbf{der} lac.\\ 
 & waz man schouwens d\textit{â} \textbf{gepflac}!\\ 
5 & ê diz volc durch si gerite,\\ 
 & Gawan durch \textbf{hovelîchen} site\\ 
 & und ouch \textbf{werdiclîchez} dinc\\ 
 & hiez \textbf{an} Artuses rinc\\ 
 & die êrsten vrowen halten.\\ 
10 & sîn marschalc muoste walten,\\ 
 & daz einiu \textbf{nâhe} zuo der reit.\\ 
 & der ander\textit{n} dekeiniu \textbf{daz} vermeit,\\ 
 & \textit{sin}e \textbf{habeten} sus alumbe,\\ 
 & hie \textbf{diu wîse}, dort \textbf{d\textit{iu} tumbe},\\ 
15 & bî ieglîcher ein ritter, der ir pflac\\ 
 & \textbf{und} der sich dienstes dâr bewac.\\ 
 & Artuses rinc, \textit{den} wîten,\\ 
 & \textbf{sach man} \textbf{an} \textbf{allen} sîten\\ 
 & mit vro\textit{w}en umbevangen.\\ 
20 & dâ wart allerêrst enpfangen\\ 
 & Gawan, der sælden rîche,\\ 
 & ich wæne d\textit{e}s, minneclîche.\\ 
 & Ar\textit{niv}e, ir tohter und ir kint\\ 
 & mit Gawan erbeizet sint\\ 
25 & \textbf{und} von Logrois diu herzogîn\\ 
 & und der herzoge von Gowertzin\\ 
 & und der Turkoite Florant.\\ 
 & gegen disen liuten \textit{wert} erkant\\ 
 & Artus û\textit{z dem} \textbf{pavelûne} gienc,\\ 
30 & der si d\textit{â} \textbf{vriuntlîch} enpfienc;\\ 
\end{tabular}
\scriptsize
\line(1,0){75} \newline
m n o Fr69 \newline
\line(1,0){75} \newline
\textbf{1} \textit{Initiale} m   $\cdot$ \textit{Capitulumzeichen} n  \newline
\line(1,0){75} \newline
\textbf{3} Artuses] artus m n o  $\cdot$ der] er n \textbf{4} dâ] do m n o \textbf{5} durch si] von ir n \textbf{6} hovelîchen] houesiche n \textbf{7} ouch] ouch durch n (o)  $\cdot$ werdiclîchez] werdekliche o \textbf{8} an] \textit{om.} n  $\cdot$ Artuses] artuse n \textbf{11} der] dir o \textbf{12} andern] ander m o  $\cdot$ dekeiniu] do kein n \textbf{13} sine] Me m n Nie o  $\cdot$ habeten] behabeten n \textbf{14} diu tumbe] der [tu*]: tumb m \textbf{17} Artuses] Artuͯses o  $\cdot$ den] vnd m \textbf{19} vrowen] froͯden m \textbf{20} dâ] Do n o daz Fr69  $\cdot$ allerêrst] alrers Fr69 \textbf{21} Gawan den selden richen Fr69  $\cdot$ Gawan] [Ga*]: Gawan m \textbf{22} des] das m n o ::: Fr69 \textbf{23} Arnive] Arune m Arniwe n  $\cdot$ tohter] dockter Fr69 \textbf{26} Gowertzin] gowerczin o \textbf{27} Turkoite] turkoitte m durkoite n \textbf{28} wert] \textit{om.} m wart n o \textbf{29} Artus] Artuͯs o  $\cdot$ ûz dem] vff m vs n (o) usser Fr69  $\cdot$ pavelûne] paneleme o \textbf{30} dâ] do m n o Fr69 \newline
\end{minipage}
\end{table}
\newpage
\begin{table}[ht]
\begin{minipage}[t]{0.5\linewidth}
\small
\begin{center}*G
\end{center}
\begin{tabular}{rl}
 & \textit{\begin{large}N\end{large}}û ditze \textbf{alsô was} komen:\\ 
 & Gawans rinc \textbf{wart} genomen\\ 
 & durch Artuses her, al dâ \textbf{er} lac.\\ 
 & waz man schouwens dâ \textbf{\textit{ge}pflac}!\\ 
5 & ê diz volc durch si gerite,\\ 
 & Gawan durch \textbf{ho\textit{v}elîche} site\\ 
 & unde ouch \textbf{durch} \textbf{werdeclîchiu} dinc\\ 
 & hiez \textit{\textbf{bî}} Artuses rinc\\ 
 & die êrsten vrouwen halden.\\ 
10 & sîn marschalc muose walden,\\ 
 & daz einiu \textbf{nâhen} zer \textbf{andern} reit.\\ 
 & der andern deheiniu vermeit,\\ 
 & sine \textbf{hielde} sus alumbe,\\ 
 & hie \textbf{diu wîse}, dort \textbf{diu tumbe},\\ 
15 & bî ieslîcher ein rîter, der ir pflac,\\ 
 & der sich dienstes dâr bewac.\\ 
 & Artuses rinc, den wîten,\\ 
 & \textbf{man sach} \textbf{in} \textbf{allen} sîten\\ 
 & mit vrouwen umbevangen.\\ 
20 & dâ wart alrêrst enpfangen\\ 
 & Gawan, der sælden rîche,\\ 
 & ich wæne des, minniclîche.\\ 
 & Arnive, ir tohter unde ir kint\\ 
 & mit Gawane erbeizt sint\\ 
25 & \textbf{unde} von Logroys diu herzogîn\\ 
 & unde der herzoge von Gowerzin\\ 
 & unde der Turkoite Florant.\\ 
 & gein disen liuten wert erkant\\ 
 & Artus ûz dem \textbf{pavelûn} gienc,\\ 
30 & der si dâ \textbf{vrœlîche} enpfienc;\\ 
\end{tabular}
\scriptsize
\line(1,0){75} \newline
G I L M Z Fr61 \newline
\line(1,0){75} \newline
\textbf{1} \textit{Initiale} G I L Z Fr61  \textbf{19} \textit{Initiale} I  \newline
\line(1,0){75} \newline
\textbf{1} Nû] Dv G Da M \textbf{2} Gawans] Gawanc Z Gawanes Fr61  $\cdot$ wart] waz L (Fr61) \textbf{3} Artuses] Artvs G (M) (Z) Artuͯses L Artauses Fr61  $\cdot$ her] [herz]: her M  $\cdot$ al] \textit{om.} L M Z Fr61 \textbf{4} gepflac] pflach G \textbf{5} si] \textit{om.} Fr61  $\cdot$ gerite] rite I gerte L \textbf{6} hovelîche] hôssliche G hovesliche L helffliche M hofleichen Fr61 \textbf{7} ouch] \textit{om.} Fr61  $\cdot$ werdeclîchiu] wertlich I \textbf{8} bî] \textit{om.} G er vmbe L (Fr61) an Z  $\cdot$ Artuses] Artvs G (Z) [*]: Artuses  L Artauses Fr61 \textbf{10} sîn] Der Z \textbf{11} einiu nâhen] eyn nahe M \textit{om.} Fr61 \textbf{12} andern] \textit{om.} Fr61  $\cdot$ deheiniu] diu hainev I keine da Z dehainen Fr61 \textbf{13} sine hielde] sin hielden I (Fr61) Sy hilde M \textbf{14} hie die wisen dort die tumbe I \textbf{16} der] vnd Fr61  $\cdot$ dâr] \textit{om.} M Fr61 \textbf{17} Artuses] Artvs G (I) (L) (Z) Artauses Fr61 \textbf{18} in allen] an eyner M an allen Z \textbf{20} alrêrst] schon Fr61 \textbf{22} des] \textit{om.} L Fr61 \textbf{23} Arnive] Arniue I  $\cdot$ unde ir] ieren Fr61 \textbf{24} Gawane] [einander]: Gawan I gawan M (Z) \textbf{25} unde] \textit{om.} Fr61  $\cdot$ Logroys] logroẏs G Fr61 Logroýs L logrois M (Z) \textbf{26} unde] \textit{om.} Z  $\cdot$ von] \textit{om.} I ouch von Z  $\cdot$ Gowerzin] Gouerzin I gowerzcin M Gowertzein Fr61 \textbf{27} Turkoite] tv̂rkoite G Turchoyde I Tuͯrkoyte L Tvrkoit Z Turkoyt Fr61  $\cdot$ Florant] floriant I \textbf{28} wert] wart L \textbf{29} Artus] Artaus Fr61  $\cdot$ pavelûn] [pa*]: pauilune I \textbf{30} der] \textit{om.} M  $\cdot$ dâ] \textit{om.} Fr61 \newline
\end{minipage}
\hspace{0.5cm}
\begin{minipage}[t]{0.5\linewidth}
\small
\begin{center}*T
\end{center}
\begin{tabular}{rl}
 & nû diz \textbf{alsô was} komen:\\ 
 & Gawans rinc \textbf{was} genomen\\ 
 & durch Artuses her, al dâ \textbf{er} lac.\\ 
 & waz man schouwe\textit{n}s d\textit{â} \textbf{pflac}!\\ 
5 & ê diz volc durch si gerite,\\ 
 & Gawan durch \textbf{höveschlîchen} site\\ 
 & und ouch \textbf{durch} \textbf{werd\textit{ec}lîchiu} dinc\\ 
 & hiez \textbf{bî} Artuses rinc\\ 
 & die êrsten vrouwen halden.\\ 
10 & sîn marschalc muoste walden,\\ 
 & daz einiu \textbf{nâher} zuo der \textbf{andern} reit.\\ 
 & der anderen keiniu \textbf{dô} vermeit,\\ 
 & sine \textbf{hielten} sus alumben,\\ 
 & hie \textbf{die wîsen}, dort \textbf{die tumben},\\ 
15 & bî ieslîcher ein ritter, der ir pflac,\\ 
 & der sich dienstes dâr \textit{b}ewac.\\ 
 & Artuses rinc, den wîten,\\ 
 & \textbf{man sach} \textbf{an} \textbf{einer} sîten\\ 
 & mit vrouwen umbevangen.\\ 
20 & dô wart alrêst enpfangen\\ 
 & Gawan, der sælde rîche,\\ 
 & ich wæn des, minneclîche.\\ 
 & Arnyve, ir tohter und ir kint\\ 
 & mit Gawane erbeizet sint\\ 
25 & \textbf{und} von Logrois diu herzogîn\\ 
 & und der herzoge von Gowerzin\\ 
 & und der Turkoyte Florant.\\ 
 & gên disen liuten wert erkant\\ 
 & Artus ûz dem \textbf{pavelûn} gienc,\\ 
30 & d\textit{er} si d\textit{â} \textbf{vrœlîch} enpfienc;\\ 
\end{tabular}
\scriptsize
\line(1,0){75} \newline
Q R W V \newline
\line(1,0){75} \newline
\textbf{1} \textit{Initiale} R V  \textbf{29} \textit{Initiale} W  \newline
\line(1,0){75} \newline
\textbf{1} diz alsô was] dis was also R (V) was dis also W \textbf{2} Gawans] Gawins R  $\cdot$ was] ward R \textbf{3} Artuses] artus Q (R) W \textbf{4} schouwens] schawes Q  $\cdot$ dâ] do Q R W V  $\cdot$ pflac] gepflag R W gespflag V \textbf{6} Gawan] Gawin R  $\cdot$ höveschlîchen] hoffliche R huͤffelichen W [hoveliche*]: hoveliche V \textbf{7} werdeclîchiu] werdliche Q werdenkliche R \textbf{8} Artuses] artus Q (R) W \textbf{10} muoste] muͯs R mvͤste V \textbf{11} \textit{Versfolge 670.12-11} W   $\cdot$ einiu] eine R  $\cdot$ nâher zuo der] nach zer R (V) nach der W  $\cdot$ andern] ander R [ander*]: anderen V \textbf{12} anderen] andrú R  $\cdot$ keiniu] deheine R  $\cdot$ dô] \textit{om.} R \textbf{13} sine] Sy R W (V)  $\cdot$ hielten] enhielten W  $\cdot$ sus] als Q [*]: sus V \textbf{14} [*ort]: hie die wise dort die tvmbe V  $\cdot$ hie] \textit{om.} W  $\cdot$ wîsen] weise W  $\cdot$ tumben] tumbe R W \textbf{16} der] [*]: vnde der V  $\cdot$ dâr] [*ar]: dar V  $\cdot$ bewac] gewack Q \textbf{17} Artuses] Artus Q R W \textbf{21} Gawan] Gawin R Herr gawan W  $\cdot$ sælde rîche] selden Riche R (W) (V) \textbf{22} wæn] meine R \textbf{23} Arnyve] Arniue Q V Arnyue R W \textbf{24} Gawane] Gawin R \textbf{25} Logrois] logroys Q Logoris R \textbf{26} von] \textit{om.} R  $\cdot$ Gowerzin] kawerzin Q Gowerczin R \textbf{27} Turkoyte] turkoite Q R tvrkoit V \textbf{29} ûz] vsser R  $\cdot$ pavelûn] gezelte R \textbf{30} der] Do Q  $\cdot$ dâ] do Q R W V \newline
\end{minipage}
\end{table}
\end{document}
