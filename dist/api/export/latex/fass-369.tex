\documentclass[8pt,a4paper,notitlepage]{article}
\usepackage{fullpage}
\usepackage{ulem}
\usepackage{xltxtra}
\usepackage{datetime}
\renewcommand{\dateseparator}{.}
\dmyyyydate
\usepackage{fancyhdr}
\usepackage{ifthen}
\pagestyle{fancy}
\fancyhf{}
\renewcommand{\headrulewidth}{0pt}
\fancyfoot[L]{\ifthenelse{\value{page}=1}{\today, \currenttime{} Uhr}{}}
\begin{document}
\begin{table}[ht]
\begin{minipage}[t]{0.5\linewidth}
\small
\begin{center}*D
\end{center}
\begin{tabular}{rl}
\textbf{369} & \begin{large}D\end{large}iu junge, süeze, klâre\\ 
 & sprach ân alle vâre:\\ 
 & "got sich \textbf{des} wol versinnen kan,\\ 
 & hêrre, ir sît der êrste man,\\ 
5 & der ie mîn redegeselle wart.\\ 
 & ist mîn zuht dâr an bewart\\ 
 & unt ouch mîn schamelîcher sin,\\ 
 & daz gît an vreuden mir gewin,\\ 
 & wande mir mîn meisterinne \textbf{verjach},\\ 
10 & diu rede wære des sinnes dach.\\ 
 & hêrre, ich bitte iwer unt mîn,\\ 
 & daz lêrt mich \textbf{endehafter} pîn.\\ 
 & den nenne ich iu, geruochet irs.\\ 
 & habt ir \textbf{mich} \textbf{ihtes} deste wirs,\\ 
15 & ich var doch ûf der mâze pfat,\\ 
 & wande ich dâ z\textbf{iu} mîn \textbf{selber} bat.\\ 
 & ir sît mit der wârheit ich,\\ 
 & swie die namen teilen sich.\\ 
 & mînes lîbes namen sult ir hân,\\ 
20 & \textbf{nû} sît magt unde man.\\ 
 & ich hân iwer unt mîn gegert.\\ 
 & lât ir mich, hêrre, ungewert\\ 
 & nû \textbf{schamelîche} von iu gên,\\ 
 & dâr umbe muoz ze rehte stên\\ 
25 & iwer prîs \textbf{vor} \textbf{iwer} selbes zuht,\\ 
 & sît mîn magettuomlîchiu \textbf{vluht}\\ 
 & \textbf{iwer} gnâde suochet.\\ 
 & ob ir des, hêrre, \textbf{ruochet},\\ 
 & ich wil iu geben minne\\ 
30 & mit herzelîchem sinne.\\ 
\end{tabular}
\scriptsize
\line(1,0){75} \newline
D Fr3 \newline
\line(1,0){75} \newline
\textbf{1} \textit{Initiale} D  \newline
\line(1,0){75} \newline
\textbf{5} mîn redegeselle] miner rede geselle Fr3 \textbf{9} Si wande mir meisterin viach Fr3 \textbf{10} des sinnes] sinne Fr3 \textbf{11} Here ich bîn vw:::nd::: Fr3 \textbf{12} lêrt] lere Fr3 \textbf{16} wande ich] :::an dich Fr3 \textbf{17} der] \textit{om.} Fr3 \textbf{22} ir] mîr Fr3 \textbf{23} nû schamelîche] Vnd senemliche Fr3 \textbf{25} vor iwer] von vwers Fr3 \textbf{28} Obir here des geruchet Fr3 \newline
\end{minipage}
\hspace{0.5cm}
\begin{minipage}[t]{0.5\linewidth}
\small
\begin{center}*m
\end{center}
\begin{tabular}{rl}
 & diu junge, süeze, clâre\\ 
 & \begin{large}S\end{large}prach âne alle vâre:\\ 
 & "got sich \textbf{des} wol versinnen kan,\\ 
 & hêrre, ir sît der êrste man,\\ 
5 & der ie mîn redegeselle wart.\\ 
 & ist mîn zuht dâr an bewart\\ 
 & und ouch mîn schamelîcher sin,\\ 
 & daz gît an vröuden mir gewin,\\ 
 & wand \dag ir\dag  mîn meisterîn \textbf{verjach},\\ 
10 & diu rede wær des sinnes dach.\\ 
 & hêrre, ich bitte iuwer und mîn,\\ 
 & daz lêret mich \textbf{endehafter} pîn.\\ 
 & den nem ich iu, geruochet irs.\\ 
 & habt ir\textbf{z} \textbf{ihtes} deste wirs,\\ 
15 & ich var doch ûf der mâze pfat,\\ 
 & wand ich dar zuo mîn \textbf{selber} bat.\\ 
 & ir sît mit der wârheit ich,\\ 
 & wie die namen teilen sich.\\ 
 & mînes lîbes namen sullet ir hân,\\ 
20 & \textbf{nû} sît \textbf{ir} maget und man.\\ 
 & ich hân iuwer \textit{und} mîn gegert.\\ 
 & lât ir mich, hêrre, ungewert\\ 
 & nû \textbf{schemelîch} von iu gên,\\ 
 & dâr umb muoz ze rehte stên\\ 
25 & iuwer prîs \textbf{vor} \textbf{iuwer} selbes zuht,\\ 
 & sît mîn magettuomlîchiu \textbf{vluht}\\ 
 & \textbf{iuwer} gnâde suochet.\\ 
 & ob ir des, hêrre, \textbf{ruochet},\\ 
 & ich wil iu geben minne\\ 
30 & mit herzelîchem sinne.\\ 
\end{tabular}
\scriptsize
\line(1,0){75} \newline
m n o \newline
\line(1,0){75} \newline
\textbf{1} \textit{Initiale} n  \textbf{2} \textit{Initiale} m  \textbf{16} \textit{Initiale} n  \newline
\line(1,0){75} \newline
\textbf{1} clâre] crore o \textbf{2} alle] aller o \textbf{5} der] Die o \textbf{9} verjach] veiach o \textbf{10} sinnes] suͯnes o \textbf{12} lêret] lerte n  $\cdot$ endehafter] endehaffte n o \textbf{13} Dem nen mann ich uch geruchen irs o \textbf{14} irz] ir mich n o \textbf{15} mâze] mossen n (o) \textbf{16} dar zuo mîn] mich dar zuͯ n darczú mich o  $\cdot$ selber] selben o \textbf{19} lîbes] libens o \textbf{21} und] \textit{om.} m  $\cdot$ gegert] begert o \textbf{25} vor] fúr n \textbf{26} magettuomlîchiu] magetliche n o  $\cdot$ vluht] fruht o \textbf{28} ruochet] geruͯchet n o \textbf{29} geben] \textit{om.} o \newline
\end{minipage}
\end{table}
\newpage
\begin{table}[ht]
\begin{minipage}[t]{0.5\linewidth}
\small
\begin{center}*G
\end{center}
\begin{tabular}{rl}
 & diu junge, süeze, clâre\\ 
 & sprach âne alle vâre:\\ 
 & "got sich \textbf{des} wol versinnen kan,\\ 
 & hêrre, ir sît der êrste man,\\ 
5 & der ie mîn redegeselle wart.\\ 
 & ist mîn zuht dâr ane bewart\\ 
 & unde ouch mîn schemelîcher sin,\\ 
 & daz gît an vröuden mir gewin,\\ 
 & wan mir mîn meisterinne \textbf{jach},\\ 
10 & diu rede wære des sinnes dach.\\ 
 & hêrre, ich bit iwer unde mîn,\\ 
 & daz lêrt mich \textbf{endehafter} pîn.\\ 
 & den nenne ich iu, geruochet irs.\\ 
 & habt ir \textbf{mich} \textbf{iht} deste wirs,\\ 
15 & ich var doch ûf der mâze pfat,\\ 
 & wan ich dâ ze \textbf{iu} mîn \textbf{selber} bat.\\ 
 & ir sît mit der wârheit ich,\\ 
 & swie die namen teilen sich.\\ 
 & mînes lîbes namen sult ir hân,\\ 
20 & \textbf{nû} sît maget unde man.\\ 
 & ich hân iwer unde mîn gegert.\\ 
 & lât ir mich, hêrre, ungewert\\ 
 & nû \textbf{schemelîchen} von iu gên,\\ 
 & dâr umbe muoz ze rehte stên\\ 
25 & iwer brîs \textbf{vür} \textbf{iwer} selbes zuht,\\ 
 & sît mîn magettuomlîchiu \textbf{vluht}\\ 
 & genâde \textbf{an iuch} suochet.\\ 
 & obe ir des, hêrre, \textbf{ruochet},\\ 
 & ich wil iu geben minne\\ 
30 & mit herzelîchem sinne.\\ 
\end{tabular}
\scriptsize
\line(1,0){75} \newline
G I O L M Q R Z Fr21 Fr24 Fr38 \newline
\line(1,0){75} \newline
\textbf{1} \textit{Initiale} I O L Q R Z Fr21  \newline
\line(1,0){75} \newline
\textbf{1} süeze] suͤzev I (Fr21) \textbf{2} sprach] Die sprach R  $\cdot$ alle] allen O Z \textbf{3} des] \textit{om.} O \textbf{4} sît] sit osz M (Q) (Z) (Fr21) (Fr38) \textbf{5} ie] \textit{om.} L \textbf{6} bewart] gespart Z \textbf{7} ouch] \textit{om.} M Fr24 \textbf{9} Wan mýn meister mir veriach L  $\cdot$ mir] \textit{om.} O  $\cdot$ jach] ver iach O (M) (Q) (R) (Z) (Fr21) (Fr24) \textbf{10} des sinnes] der sinnen R \textbf{11} ich] \textit{om.} Z  $\cdot$ bit] bin L pete M beite Z \textbf{12} lêrt] lerte Fr24  $\cdot$ endehafter] endehafften O ellenthafter L endenhafftte R edehafter Z  $\cdot$ pîn] [sin]: pin Z \textbf{14} mich] \textit{om.} Q  $\cdot$ iht] nih I (Q) ivt Fr38 \textbf{15} ich var] Jr fart Q \textbf{16} wan] vnde O  $\cdot$ dâ ze] \textit{om.} I daz O do zu Q doch zu R  $\cdot$ iu] \textit{om.} L R Z Fr38  $\cdot$ selber] selbes I (M) selbe O silber Fr21 \textbf{17} ir] Jch L  $\cdot$ der wârheit] warheit der L \textbf{18} swie] Wie L Q R \textbf{19} lîbes] libens Fr38 \textbf{21} hân] \textit{om.} Fr21  $\cdot$ gegert] begert Q \textbf{23} nû] Jn Q  $\cdot$ schemelîchen] semeliche Fr38 \textbf{24} muoz] mvͦz ich O (L) \textbf{25} selbes] selber I \textbf{26} mîn] \textit{om.} L  $\cdot$ magettuomlîchiu] magetlichev I (M) magtumliche R  $\cdot$ vluht] frvht O Fr21 \textbf{27} suochet] suchte Q \textbf{28} ir des hêrre] irs selbe R  $\cdot$ ruochet] geruchet M ruchte Q \textbf{29} iu] ich R  $\cdot$ geben minne] [*]: minne Q \textbf{30} sinne] sinde Q \newline
\end{minipage}
\hspace{0.5cm}
\begin{minipage}[t]{0.5\linewidth}
\small
\begin{center}*T
\end{center}
\begin{tabular}{rl}
 & \begin{large}D\end{large}iu junge, süeze, clâre\\ 
 & sprach âne alle vâre:\\ 
 & "got sich wol versinnen kan,\\ 
 & hêrre, ir sît der êrste man,\\ 
5 & der ie mîn redegeselle wart.\\ 
 & ist mîn zuht dâr an bewart\\ 
 & \textit{unde ouch mîn schemelîcher sin,}\\ 
 & daz gît an vröuden mir gewin,\\ 
 & wand mir mîn meisterîn \textbf{verjach},\\ 
10 & diu rede wære des sinnes dach.\\ 
 & hêrre, ich bitte iuwer unde mîn,\\ 
 & daz lêret mich \textbf{endehaften} pîn.\\ 
 & den nennich iu, geruochet irs.\\ 
 & habt ir \textbf{mich} \textbf{iht} deste wirs,\\ 
15 & ich var doch ûf der mâze pfat,\\ 
 & wandich dâ zuo mîn \textbf{selbes} bat.\\ 
 & ir sît mit der wârheit ich,\\ 
 & swie die namen teilen sich.\\ 
 & mînes lîbes namen sult ir hân,\\ 
20 & sît maget unde man.\\ 
 & ich hân iuwer unde mîn gegert.\\ 
 & lât ir mich, hêrre, ungewert\\ 
 & nû \textbf{semelîche} von iu gân,\\ 
 & dâr umbe muoz ze rehte stân\\ 
25 & iuwer prîs \textbf{vür} \textbf{mîn} selbes zuht,\\ 
 & sît mîn magettuomlîchiu \textbf{vruht}\\ 
 & gnâde \textbf{an iu} suochet.\\ 
 & ob ir des, hêrre, \textbf{geruochet},\\ 
 & ich wil iu geben minne\\ 
30 & mit herzeclîchem sinne.\\ 
\end{tabular}
\scriptsize
\line(1,0){75} \newline
T V W \newline
\line(1,0){75} \newline
\textbf{1} \textit{Initiale} T V W  \newline
\line(1,0){75} \newline
\textbf{3} wol] wol dez V \textbf{4} sît] sint ez V (W) \textbf{7} \textit{Vers 369.7 fehlt (Zeile ausgespart)} T  \textbf{12} endehaften] endehafter V (W) \textbf{14} iht] doch W \textbf{15} doch ûf der] auch auff W \textbf{16} dâ zuo] da zv́ch V do zuͦ eúch W  $\cdot$ selbes] selber V W \textbf{18} swie] Swie so V Wie W \textbf{20} sît] [*]: Nv sint ir V  $\cdot$ unde] vnd ich ein W \textbf{23} semelîche] schemeliche V (W) \textbf{24} muoz] muͦß ich W \textbf{25} vür mîn selbes] [*]: fur úwer selbes V vnd eúwer W \textbf{26} vruht] [f*]: flvht V \textbf{27} [*]: V́wer genade suͦchet V \textbf{28} geruochet] ruͦchet W \textbf{30} mit] Von W \newline
\end{minipage}
\end{table}
\end{document}
