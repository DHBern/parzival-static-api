\documentclass[8pt,a4paper,notitlepage]{article}
\usepackage{fullpage}
\usepackage{ulem}
\usepackage{xltxtra}
\usepackage{datetime}
\renewcommand{\dateseparator}{.}
\dmyyyydate
\usepackage{fancyhdr}
\usepackage{ifthen}
\pagestyle{fancy}
\fancyhf{}
\renewcommand{\headrulewidth}{0pt}
\fancyfoot[L]{\ifthenelse{\value{page}=1}{\today, \currenttime{} Uhr}{}}
\begin{document}
\begin{table}[ht]
\begin{minipage}[t]{0.5\linewidth}
\small
\begin{center}*D
\end{center}
\begin{tabular}{rl}
\textbf{127} & Diu vrouwe nam ein sactuoch.\\ 
 & si sneit im hemde unt bruoch,\\ 
 & daz doch an eime stücke \textbf{erschein},\\ 
 & \textbf{unz} enmitten an sîn blanke\textit{z} bein.\\ 
5 & daz wart vür tôren kleit erkant.\\ 
 & eine gugelen man \textbf{obene} drûfe vant.\\ 
 & al \textbf{vrisch rûch} kelberîn\\ 
 & \textbf{von} einer hûte zwei ribbalîn\\ 
 & nâch \textbf{sînen beinen} wart gesniten.\\ 
10 & dâ wart grôz jâmer niht vermiten.\\ 
 & Diu künegîn was alsô bedâht,\\ 
 & si bat \textbf{belîben in} die naht:\\ 
 & "dû\textbf{ne} solt niht hinnen kêren,\\ 
 & ich wil dich \textbf{liste} lêren:\\ 
15 & an ungebanten strâzen\\ 
 & \textbf{soltû tunkel vürte} lâzen.\\ 
 & die \textbf{sîhte} und lûter sîn,\\ 
 & dâ solt dû \textbf{al} balde rîten în.\\ 
 & dû solt dich site nieten,\\ 
20 & der \textbf{werlde} grüezen bieten.\\ 
 & \begin{large}O\end{large}b dich ein \textbf{grâ} wîse man\\ 
 & zuht wil lêren, als er wol kan,\\ 
 & dem soltû gerne volgen\\ 
 & unt wis im niht \textbf{erbolgen}.\\ 
25 & Sun, lâ dir bevolhen sîn:\\ 
 & swâ dû guotes wîbes vingerlîn\\ 
 & mugest erwerben unt ir gruoz,\\ 
 & daz nim; ez tuot dir kumbers buoz.\\ 
 & dû solt z\textbf{ir} kusse gâhen\\ 
30 & \textbf{unt} ir lîp vaste umbevâhen.\\ 
\end{tabular}
\scriptsize
\line(1,0){75} \newline
D \newline
\line(1,0){75} \newline
\textbf{1} \textit{Majuskel} D  \textbf{11} \textit{Majuskel} D  \textbf{21} \textit{Initiale} D  \textbf{25} \textit{Majuskel} D  \newline
\line(1,0){75} \newline
\textbf{4} blankez] blanches D \newline
\end{minipage}
\hspace{0.5cm}
\begin{minipage}[t]{0.5\linewidth}
\small
\begin{center}*m
\end{center}
\begin{tabular}{rl}
 & \begin{large}D\end{large}iu vrouwe nam ein sactuoch.\\ 
 & si sneit im hemede und bruoch,\\ 
 & daz doch an ein\textit{em} stücke \textbf{wol} \textbf{erschein},\\ 
 & \textbf{und} \textit{e}nmitten an sîn blanke\textit{z} bein.\\ 
5 & daz war\textit{t} vür tôren kleider erkant.\\ 
 & eine kugelen man \textbf{obenân} drûfe vant.\\ 
 & al \textbf{vrisch \textit{rû}ch} kelberîn\\ 
 & \textbf{von} einer h\textit{û}t zwei r\textit{i}bbalîn\\ 
 & nâch \textbf{sînen \textit{be}i\textit{n}en} wart gesniten.\\ 
10 & d\textit{â} wart grôz jâmer niht vermiten.\\ 
 & diu künigîn was alsô bedâht,\\ 
 & si bat \textbf{in be\textit{lîbe}n} die naht:\\ 
 & "dû \textbf{en}solt niht hinnen kêren,\\ 
 & ich wil dich \textbf{liste} lêren:\\ 
15 & an ungebaneten strâzen\\ 
 & \textbf{solt dû dunkele vürte} lâzen.\\ 
 & die \dag schilte\dag  und lûter sî\textit{n},\\ 
 & d\textit{â} solt dû \textbf{al}balde rîten î\textit{n}.\\ 
 & dû solt dich siten nieten,\\ 
20 & der \textbf{werde} grüezen bieten.\\ 
 & o\textit{b} dich ein \textbf{gr\textit{â}} wîser man\\ 
 & zuht wil lêren, als \textit{er} wol kan,\\ 
 & dem soltû gerne volgen\\ 
 & und wis im niht \textbf{verbolgen}.\\ 
25 & sun, lâ dir bevolhen sîn:\\ 
 & wâ dû guotes wîbes vingerlîn\\ 
 & mugest erwerben und ir gruoz,\\ 
 & daz nim; ez tuot dir kumber\textit{s} buoz.\\ 
 & dû solt zuo \textbf{ir} \dag zusse\dag  gâhen\\ 
30 & \textbf{und} ir lîp vast\textit{e} \textit{umbe}vâhen.\\ 
\end{tabular}
\scriptsize
\line(1,0){75} \newline
m n o \newline
\line(1,0){75} \newline
\textbf{1} \textit{Initiale} m   $\cdot$ \textit{Capitulumzeichen} n  \newline
\line(1,0){75} \newline
\textbf{2} si] Vnd n o \textbf{3} einem] ein m \textbf{4} und] Vntze n Vns o  $\cdot$ enmitten] anmitten m  $\cdot$ blankez] blancken m \textbf{5} wart] war m \textbf{6} kugelen] kogel n o  $\cdot$ obenân] oben n >oben< o \textbf{7} al] Elle n  $\cdot$ rûch] durch m  $\cdot$ kelberîn] kuͯlberin o \textbf{8} hût] hant m  $\cdot$ ribbalîn] rubalin m rẏbbulin o \textbf{9} beinen] snitten m \textbf{10} dâ] Do m n o \textbf{12} belîben] bebelin m \textbf{13} ensolt] solt n o \textbf{17} schilte] fúchte n (o)  $\cdot$ sîn] sint m \textbf{18} dâ] Du m Do n o  $\cdot$ în] int m \textbf{19} solt] soltuͯ o  $\cdot$ siten] sitte n o \textbf{20} Der werden grusz enbieten n (o) \textbf{21} ob] O m  $\cdot$ grâ] gros m \textbf{22} wil] vil o  $\cdot$ er] \textit{om.} m \textbf{28} kumbers] kumber m \textbf{29} zusse] kúsch n (o) \textbf{30} vaste umbevâhen] vaste gahen vnd vohen m \newline
\end{minipage}
\end{table}
\newpage
\begin{table}[ht]
\begin{minipage}[t]{0.5\linewidth}
\small
\begin{center}*G
\end{center}
\begin{tabular}{rl}
 & diu vrouwe nam ein sactuoch.\\ 
 & si sneit im hemde und bruoch,\\ 
 & daz doch an einem stücke \textbf{schein},\\ 
 & \textbf{unze} enmitten an sîn blankez bein.\\ 
5 & daz wart vür tôren kleit erkant.\\ 
 & eine kugelen man drûffe vant.\\ 
 & al \textbf{rûch vrisch} kelberîn\\ 
 & \textbf{ûz} einer hût zwei ribbalîn\\ 
 & nâch \textbf{sînem beine} wart gesniten.\\ 
10 & dâ\textbf{ne} wart grôz jâmer niht vermiten.\\ 
 & diu künigîn was alsô bedâht,\\ 
 & si bat \textbf{belîben in} die naht:\\ 
 & "dû\textbf{ne} solt niht hinnen kêren,\\ 
 & ich wil dich \textbf{site} lêren:\\ 
15 & an un\textit{g}e\textit{b}anten strâzen\\ 
 & \textbf{soltû tunkele vürte} lâzen.\\ 
 & die \textbf{sîhte} und lûter sîn,\\ 
 & dâ soltû \textbf{al} balde rîten în.\\ 
 & dû solt dich site nieten,\\ 
20 & der \textbf{werlde} grüezen bieten.\\ 
 & op dich ein \textbf{alt} wîse man\\ 
 & zuht wil lêren, alser wol kan,\\ 
 & dem soltû gerne volgen\\ 
 & unde wis im niht \textbf{erbolgen}.\\ 
25 & sun, lâ dir bevolhen sîn:\\ 
 & swâ dû guotes wîbes vingerlîn\\ 
 & mugest erwerben und ir gruoz,\\ 
 & daz nim; ez tuot dir kumbers buoz.\\ 
 & dû solt z\textbf{ir} kusse gâhen,\\ 
30 & ir lîp vaste umbevâhen.\\ 
\end{tabular}
\scriptsize
\line(1,0){75} \newline
G I O L M Q R Z \newline
\line(1,0){75} \newline
\textbf{1} \textit{Initiale} I  \textbf{19} \textit{Initiale} I L Z  \textbf{29} \textit{Initiale} O  \newline
\line(1,0){75} \newline
\textbf{2} si] Vnd L Q \textbf{3} doch] tuͤch I  $\cdot$ an] von L  $\cdot$ stücke] sturm Q  $\cdot$ schein] er schein O (M) (Q) (Z) \textbf{4} unze] vnde I  $\cdot$ blankez] blozez O \textbf{6} drûffe] oben drvfe O (L) (M) (Q) (R) ober drvf Z  $\cdot$ vant] bant Q \textbf{8} ûz] Vser R  $\cdot$ hût] hevte Z \textbf{9} sînem beine] sinen bain I seinē bein Q sinen beinen R  $\cdot$ wart] was Q \textbf{10} dâne] da I (O) Do Q R  $\cdot$ grôz] \textit{om.} O M Q R \textbf{12} bat] \textit{om.} Z  $\cdot$ belîben] das her blebe M  $\cdot$ naht] tracht Q \textbf{13} dûne] Dv O (L) \textbf{14} site] liste O M Q R Z witze L \textbf{15} ungebanten] vnbechanten G vngebanen R \textbf{17} sîhte] schleht L sichtigk Q (R)  $\cdot$ sîn] sind R \textbf{18} dâ] Do Q  $\cdot$ soltû] maht duͯ L  $\cdot$ în] hin R \textbf{19} nieten] meren Q \textbf{20} der] Die Q  $\cdot$ grüezen] gruͤz I (O) (M) (Q) (R) \textbf{21} alt wîse] gra wiser L (Q) (R) \textbf{22} zuht wil] Wil zuͯcht L  $\cdot$ wol] \textit{om.} I \textbf{24} erbolgen] verbolgen R \textbf{25} bevolhen] enpholhen I \textbf{26} swâ] Wo L Q (R)  $\cdot$ wîbes] \textit{om.} L \textbf{28} nim ez] meyn ist Q  $\cdot$ dir] dich O L dy M  $\cdot$ kumbers] sorgen L kumber R \textbf{29} dû] ÷v O Vnd Z  $\cdot$ zir] zem R zv Z  $\cdot$ kusse] chvssen O (Z)  $\cdot$ gâhen] diche nahen I \textbf{30} ir] Vnde ir O (L) (M) (Q) (R) (Z) \newline
\end{minipage}
\hspace{0.5cm}
\begin{minipage}[t]{0.5\linewidth}
\small
\begin{center}*T (U)
\end{center}
\begin{tabular}{rl}
 & diu vrouwe nam ein sactuoch.\\ 
 & si sneit im hemede und bruoch,\\ 
 & daz doch an eime stücke \textbf{schein},\\ 
 & \textbf{unz} mitten an sîn blankez bein.\\ 
5 & daz wart vür tôren kleit erkant.\\ 
 & eine kogel man \textbf{oben} drûf vant.\\ 
 & al \textbf{rûch vrisch} kelberîn\\ 
 & \textbf{ûz} einer hût zwei r\textit{i}bbalîn\\ 
 & nâch \textbf{sînen beinen} wart gesniten.\\ 
10 & d\textit{â} wart grôz jâmer niht vermiten.\\ 
 & diu künegîn was alsô bedâht,\\ 
 & si bat \textbf{blîben in} die naht:\\ 
 & \hspace*{-.7em}\big| "ich wil dich \textbf{liste} lêren:\\ 
 & \hspace*{-.7em}\big| dû solt niht hine kêren\\ 
15 & an ungebanten strâzen.\\ 
 & \textbf{dunkele \textit{vü}rte soltû} lâzen.\\ 
 & die \textbf{lieht} und lûter sîn,\\ 
 & dâ soltû balde rîten în.\\ 
 & dû solt dich site nieten,\\ 
20 & der \textbf{werlde} grüezen bieten.\\ 
 & ob dich ein \textbf{grâwe\textit{r}} wîser man\\ 
 & zuht wil lêren, als er wol kan,\\ 
 & dem soltû gerne volgen\\ 
 & und wi\textit{s} im niht \textbf{erbolgen}.\\ 
25 & sun, lâ dir bevolhen sîn:\\ 
 & wâ dû guotes wîbes vingerlîn\\ 
 & mugest erwerben und ir gruoz,\\ 
 & daz nim; ez tuot dir kumbers buoz.\\ 
 & dû solt zuo \textbf{dem} kusse gâhen\\ 
30 & \textbf{und} ir lîp vaste umbevâhen.\\ 
\end{tabular}
\scriptsize
\line(1,0){75} \newline
U V W T \newline
\line(1,0){75} \newline
\textbf{1} \textit{Majuskel} T  \textbf{7} \textit{Majuskel} T  \textbf{11} \textit{Majuskel} T  \textbf{15} \textit{Majuskel} T  \textbf{19} \textit{Initiale} W T  \textbf{21} \textit{Majuskel} T  \textbf{25} \textit{Majuskel} T  \newline
\line(1,0){75} \newline
\textbf{3} schein] erschain W \textbf{4} unz] bit U  $\cdot$ mitten] enmitte V (W) (T) \textbf{6} eine kogel] einen kvgelhvͦt T  $\cdot$ oben] \textit{om.} V T \textbf{7} al] \textit{om.} T  $\cdot$ vrisch] von eim velle W \textbf{8} ribbalîn] Rabelin U riemelin W \textbf{9} sînen beinen wart] seinem baine warn W sinem beine wart T \textbf{10} dâ] Do U V W \textbf{11} alsô] so T \textbf{12} blîben in] in beleiben do W \textbf{14} \textit{Versfolge 127.13-14} T  \textbf{13} dû] Svn dv V dvne T  $\cdot$ hine] hie hinnen W \textbf{15} ungebanten] vngebante V \textbf{16} dunkele vürte soltû] Duͦnkele worte soltu U Du solt dunckel fúrte W soltu tvnkele vurte T \textbf{17} lieht] gesiht T \textbf{18} dâ] Do W \textbf{19} site] sitten W \textbf{20} grüezen] grvͤze V (W) (T) \textbf{21} grâwer] grawe U alt T \textbf{24} wis] wil U \textbf{26} wâ] Swa V (T) \textbf{28} daz] die T \textbf{29} zuo dem kusse] zirme kvsse V (T) zuͦ ir kússen W \textbf{30} ir lîp] \textit{om.} W \newline
\end{minipage}
\end{table}
\end{document}
