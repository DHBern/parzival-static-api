\documentclass[8pt,a4paper,notitlepage]{article}
\usepackage{fullpage}
\usepackage{ulem}
\usepackage{xltxtra}
\usepackage{datetime}
\renewcommand{\dateseparator}{.}
\dmyyyydate
\usepackage{fancyhdr}
\usepackage{ifthen}
\pagestyle{fancy}
\fancyhf{}
\renewcommand{\headrulewidth}{0pt}
\fancyfoot[L]{\ifthenelse{\value{page}=1}{\today, \currenttime{} Uhr}{}}
\begin{document}
\begin{table}[ht]
\begin{minipage}[t]{0.5\linewidth}
\small
\begin{center}*D
\end{center}
\begin{tabular}{rl}
\textbf{36} & \textbf{sô} die senewen tuot daz armbrust.\\ 
 & dâ was ze dræt sîn gelust.\\ 
 & der hêrre \textbf{ân allez} slâfen lac,\\ 
 & \textbf{unz er} erkôs den grâwen tac.\\ 
5 & der gap dennoch niht liehten schîn.\\ 
 & \textbf{dô solt} ouch dâ bereite sîn\\ 
 & zer messe eîn sîn kappelân.\\ 
 & der sanc si got unt im sân.\\ 
 & \textbf{sîn harnasch truoc man} \textbf{sâ} zehant.\\ 
10 & er reit, dâ \textbf{man} tjustieren vant.\\ 
 & dô saz er an der stunde\\ 
 & ûf ein ors, daz beidiu kunde,\\ 
 & hurtlîchen dringen\\ 
 & unt snelleclîchen springen,\\ 
15 & \textbf{bekêric}, \textbf{swâ} man\textit{z} wider zôch.\\ 
 & sînen anker ûf dem helme hôch\\ 
 & man gein der porte vüeren sach.\\ 
 & aldâ \textbf{wîp unt man} verjach,\\ 
 & si\textbf{ne} gesæhen nie helt sô wünneclîch.\\ 
20 & ir gote \textbf{im solten} sîn gelîch.\\ 
 & man vuort ouch \textbf{starkiu} sper dâ bî.\\ 
 & wie er \textbf{gezimieret} sî?\\ 
 & sîn ors von \textbf{îser} truoc ein dach.\\ 
 & \textbf{daz was vür slege des gemach.}\\ 
25 & \textbf{dâr ûf ein ander decke} lac,\\ 
 & \textbf{ringe, diu} niht swære wac.\\ 
 & daz was ein grüener samît.\\ 
 & sîn wâpenroc, sîn kursît\\ 
 & was \textbf{ouch} ein grüenez achmardî.\\ 
30 & daz \textbf{wart} geworht \textbf{dâ} zArabi.\\ 
\end{tabular}
\scriptsize
\line(1,0){75} \newline
D \newline
\line(1,0){75} \newline
\newline
\line(1,0){75} \newline
\textbf{15} manz] mans D \newline
\end{minipage}
\hspace{0.5cm}
\begin{minipage}[t]{0.5\linewidth}
\small
\begin{center}*m
\end{center}
\begin{tabular}{rl}
 & \textbf{sô} die senewe tuot \textit{daz} armbrust.\\ 
 & dô was ze dræte sîn gelust.\\ 
 & \begin{large}D\end{large}er hêrre \textbf{âne allez} slâfen lac,\\ 
 & \textbf{im} erkôs den grâwen tac.\\ 
5 & der gap dannoch niht lieh\textit{t}en schîn.\\ 
 & \textbf{dô solte} ouch dâ bereit sîn\\ 
 & zer messe einer sîn kappelân.\\ 
 & der sant si gote und im sân.\\ 
 & \textbf{sîn harnasch truoc man} \textbf{dar} zehant.\\ 
10 & er reit, d\textit{â} \textbf{er} justieren vant.\\ 
 & dô saz er an der stunde\\ 
 & ûf ein ros, daz beidiu kunde,\\ 
 & h\textit{u}rteclîchen d\textit{r}ingen,\\ 
 & und snelleclîchen springen,\\ 
15 & \textbf{bekêret}, \textbf{wâ} manz wider zôch.\\ 
 & sînen anker ûf dem helme hôch\\ 
 & man gegen der porten v\textit{üe}ren sach.\\ 
 & aldâ \textbf{wîp \textit{und} man} verjach,\\ 
 & si ges\textit{æ}hen nie helt sô wünneclîch.\\ 
20 & ir gote \textbf{im solte\textit{n}} sîn gelîch.\\ 
 & man vuorte ouch \textbf{starken} sper dâ bî.\\ 
 & wie er \textbf{geformieret} sî?\\ 
 & sîn ros von \textbf{îsen} truoc ein dach.\\ 
 & \textbf{daz \textit{was} vür slege d\textit{e}s gemach.}\\ 
25 & \textbf{dâr ûf ein ander decke} lac,\\ 
 & \textbf{ringe, diu} niht swære wa\textit{c}.\\ 
 & daz was ein grüener samît.\\ 
 & sîn wâpenroc, sî\textit{n} kursît\\ 
 & was \textbf{ouch} ein grüene\textit{z} achmardî.\\ 
30 & daz \textbf{was} gewor\textit{h}t zArabi.\\ 
\end{tabular}
\scriptsize
\line(1,0){75} \newline
m n o W \newline
\line(1,0){75} \newline
\textbf{3} \textit{Initiale} m n o  \textbf{9} \textit{Initiale} W  \newline
\line(1,0){75} \newline
\textbf{1} daz armbrust] armbrost \textit{nachträglich korrigiert zu:} daz armbrost m \textbf{2} dræte] tretten W \textbf{4} im erkôs] Bitz an W \textbf{5} niht] \textit{om.} n  $\cdot$ liehten] liechen m \textbf{6} dâ] \textit{om.} n o W \textbf{7} einer] \textit{om.} W \textbf{9} harnasch] harnersch o \textbf{10} dâ] do m n o W \textbf{11} dô] Das n \textbf{12} daz] das er n  $\cdot$ kunde] kúnde o \textbf{13} hurteclîchen] Herteklichen m (n) o  $\cdot$ dringen] dingen m \textbf{16} dem helme] den helm n \textbf{17} vüeren] [furen]: faren m \textbf{18} aldâ] Alle die n  $\cdot$ und] \textit{om.} m \textbf{19} gesæhen] gesahent m gesehent n  $\cdot$ wünneclîch] mynneclich n \textbf{20} solten] solte m  $\cdot$ sîn] \textit{om.} W \textbf{21} starken] starcke n o W \textbf{24} daz] [De*]: Das n  $\cdot$ was] \textit{om.} m  $\cdot$ des gemach] das gemach \textit{nachträglich korrigiert zu:} was gemacht m \textbf{26} ringe] K:inge o  $\cdot$ wac] wa: \textit{nachträglich korrigiert zu:} wag m \textbf{28} wâpenroc] wappen ros o  $\cdot$ sîn kursît] sint kursit \textit{nachträglich korrigiert zu:} sin kursit m \textbf{29} grüenez] groͯner m (n) (o) (W)  $\cdot$ achmardî] [samit]: achmardi o acbmardin W \textbf{30} was] \textit{om.} o  $\cdot$ geworht] geworheit m  $\cdot$ zArabi] zuͦ araby n zuͦ arabẏ o zuͦ arabin W \newline
\end{minipage}
\end{table}
\newpage
\begin{table}[ht]
\begin{minipage}[t]{0.5\linewidth}
\small
\begin{center}*G
\end{center}
\begin{tabular}{rl}
 & \multicolumn{1}{l}{ - - - }\\ 
 & \multicolumn{1}{l}{ - - - }\\ 
 & der hêrre \textbf{sunder} slâfen lac,\\ 
 & \textbf{unzer} erkôs den grâwen tac.\\ 
5 & der gap dannoch niht liehten schîn.\\ 
 & \textbf{nû wolt} ouch dâ bereite sîn\\ 
 & zer messe ein sîn kappelân.\\ 
 & der sanc si gote und im sân.\\ 
 & \textbf{man truoc sîn harnasch} \textbf{dar} zehant.\\ 
10 & er reit, dâ \textbf{er} tjostieren vant.\\ 
 & dô saz er an der stunde\\ 
 & ûf ein ors, daz beidiu kunde,\\ 
 & hurticlîche dringen\\ 
 & unde snellîche springen,\\ 
15 & \textbf{kêric}, \textbf{sô} manz wider zôch.\\ 
 & sînen anker ûf dem helme hôch\\ 
 & \begin{large}M\end{large}an gein der borte vüeren sach.\\ 
 & al dâ \textbf{man und wîp} verjach,\\ 
 & si\textbf{ne} ges\textit{æ}hen nie helt sô wünniclîch.\\ 
20 & ir gote \textbf{solten im} sîn gelîch.\\ 
 & man vuort ouch \textbf{starkiu} sper dâ bî.\\ 
 & wier \textbf{gezimieret} sî?\\ 
 & sîn ors von \textbf{îser} truoc ein dach.\\ 
 & \textbf{daz was vür slege des gemach.}\\ 
25 & \textbf{ein ander decke drûffe} lac,\\ 
 & \textbf{ringe, diu} niht \textit{swær}e wac.\\ 
 & daz was ein grüener samît.\\ 
 & sîn wâpenroc, sîn kursît\\ 
 & was ein grüenez achmardî.\\ 
30 & daz \textbf{was} geworht ze Arabi.\\ 
\end{tabular}
\scriptsize
\line(1,0){75} \newline
G O L M Q R Z Fr21 Fr32 \newline
\line(1,0){75} \newline
\textbf{1} \textit{Initiale} O Fr21  \textbf{3} \textit{Initiale} M  \textbf{9} \textit{Initiale} L Q R Z   $\cdot$ \textit{Versal} Fr32  \textbf{17} \textit{Initiale} G  \newline
\line(1,0){75} \newline
\textbf{1} \textit{Die Verse 35.23-36.2 fehlen} G   $\cdot$ \textit{Vers 36.1 fehlt} Q   $\cdot$ ÷am (Sam L So M Z Fr21 Fr32 ) div senwe tvͦt daz armbrvst O (L) (M) (Z) (Fr21) (Fr32)  $\cdot$ Als die senne tuͦt dem armbrust R \textbf{2} Da (Do Q R [ Fr32 ]) was ze dræte sin gelust O (L) (M) (Q) (R) (Z) (Fr21) (Fr32) \textbf{4} unzer erkôs] Vntz erkos L Vns ir kosz M (Fr32) Vntz er Q Vncz das er erkos R Biz er kos Z  $\cdot$ grâwen] graͦben O graffen Q liehten Z  $\cdot$ tac] sack Q \textbf{5} der] Dern O (R) Fr21  $\cdot$ liehten] lychten L (M) (Q) (Fr21) \textbf{6} nû] Da O Z Do L Q R Fr21 (Fr32) Her M  $\cdot$ wolt] must Q (Fr32) solde Z  $\cdot$ ouch dâ] avch do O (Q) da ouch M  $\cdot$ bereite] bereytet Q \textbf{7} zer] Der L Fr21  $\cdot$ sîn] \textit{om.} Q Fr21 \textbf{8} sanc] sante M (R) o\textit{m. } Fr32  $\cdot$ si] sinem O (Q) sin R  $\cdot$ gote] bot R  $\cdot$ und im] zu im R vmminne Fr32 \textbf{9} sîn] im sin R \textbf{10} Do reyt do er zu streiten fant Q  $\cdot$ er reit] Ein reit Z  $\cdot$ dâ] do O L  $\cdot$ tjostieren] tiostierren vnd strit R \textbf{11} dô] Da R Z  $\cdot$ an der stunde] in der stvnde L an den stunden Q zv der stunde Z \textbf{12} daz] do Q  $\cdot$ kunde] kunden M \textbf{13} hurticlîche] Hertigliche Q (Z) \textbf{14} snellîche] behentiglich Q \textbf{15} kêric] Bekerich L (Z) Kerecht M [Erigk]: Krigk Q Ze recht R  $\cdot$ sô] als Q \textbf{16} sînen] Siner M  $\cdot$ helme] helmen M \textbf{17} Man] Man in Q (R)  $\cdot$ borte] porten O (R) Z Fr21 phortin M (Q) \textbf{18} man und wîp] wip vnd man O L (Q) (R) Z (Fr21)  $\cdot$ verjach] sprach Q \textbf{19} sine] Sinen O Sie L Q (R) Z  $\cdot$ gesæhen] gesahen G L gesehen O (M) Q Z gesechent R  $\cdot$ nie helt] helt nie O man helt M nye herren R  $\cdot$ wünniclîch] loblich Q minnenclich Fr32 \textbf{20} solten im] solt den im Q im solden Z \textbf{21} ouch] ým L  $\cdot$ starkiu] groziv O starke R \textbf{23} îser] ysen O (L) (Q) R (Z) (Fr21) (Fr32) \textbf{24} daz] Die R  $\cdot$ des gemach] dar gemacht R \textbf{25} Dar vfe ein ander deche (dekin R ) lach O (L) (M) (Q) (R) (Z) (Fr21) (Fr32) \textbf{26} ringe] Ringer Q  $\cdot$ swære] ringe G \textbf{27} \textit{Versfolge 36.28-27} R  \textbf{28} sîn] Ein L \textbf{29} was] Was avch O (L) (M) (Q) (Z) (Fr21)  $\cdot$ grüenez] grvͤner Z \textbf{30} daz] Der Z  $\cdot$ was] wart O L M R Fr21 (Fr32)  $\cdot$ geworht] geworht da O (M) (Z) Fr32 gewúrk R  $\cdot$ Arabi] araby M arrabi R Arabẏ Fr21 \newline
\end{minipage}
\hspace{0.5cm}
\begin{minipage}[t]{0.5\linewidth}
\small
\begin{center}*T
\end{center}
\begin{tabular}{rl}
 & \textbf{alse} die senewen tuot daz armbrust.\\ 
 & dô was ze dræte sîn gelust.\\ 
 & der hêrre \textbf{sunder} slâfen lac,\\ 
 & \textbf{unz er} erkôs den grâwen tac.\\ 
5 & der gap dannoch niht liehten schîn.\\ 
 & \textbf{nû wolte} ouch dâ bereit sîn\\ 
 & zer messe ein sîn kappelân.\\ 
 & der sanc si gote und im sân.\\ 
 & \textbf{man truoc sînen harnasch} \textbf{dar} zehant.\\ 
10 & er reit, dâ \textbf{er} tjostieren vant.\\ 
 & dô saz er an der stunde\\ 
 & ûf ein ors, daz beidiu kunde,\\ 
 & hurteclîchen dringen\\ 
 & unde snellîchen springen,\\ 
15 & \textbf{gerech}, \textbf{sô} man\textit{z} wider zôch.\\ 
 & sînen anker ûf dem helme hôch\\ 
 & man gegen der porte vüeren sach.\\ 
 & aldâ \textbf{wîp und man} verjach,\\ 
 & si\textbf{ne} gesæhen nie helt sô wünneclîch.\\ 
20 & ir gote \textbf{solten im} sîn glîch.\\ 
 & Man vuorte ouch \textbf{starkiu} sper dâ bî.\\ 
 & wie er \textbf{gezimieret} sî?\\ 
 & \textbf{\hspace*{-.7em}\big| dâ man den helt rîten sach,}\\ 
 & \hspace*{-.7em}\big| sîn ors von \textbf{îsene} truoc ein dach.\\ 
25 & \textbf{dâr ûfe ein ander decke} lac,\\ 
 & \textbf{diu ringe}, niht swære wac.\\ 
 & daz was ein grüener samît.\\ 
 & sîn wâpenroc, sîn kursît\\ 
 & was \textbf{ouch} ein grüenez achmardî.\\ 
30 & daz \textbf{wart} geworht ze Arabi.\\ 
\end{tabular}
\scriptsize
\line(1,0){75} \newline
T U V \newline
\line(1,0){75} \newline
\textbf{3} \textit{Initiale} U V  \textbf{21} \textit{Majuskel} T  \newline
\line(1,0){75} \newline
\textbf{1} senewen] senewe V  $\cdot$ daz] am V \textbf{2} ze dræte] zestrete U (V) \textbf{4} erkôs] kos U V \textbf{6} dâ] do U \textbf{7} messe] messen U V  $\cdot$ sîn] \textit{om.} U V \textbf{9} sînen] sin U V \textbf{10} dâ] do V \textbf{13} hurteclîchen] Hertecliche U \textbf{14} snellîchen] snellekliche V \textbf{15} gerech] [G*h]: Keren V  $\cdot$ manz] mans T \textbf{17} porte] porten U V \textbf{18} wîp und man] man vnd wip U (V) \textbf{19} sine gesæhen] Sie gesahen U Si engesehent V \textbf{21} ouch] im V \textbf{24} \textit{Versfolge 36.23-24} U V   $\cdot$ [sin ors von isene trvͦc ein tach / da man den helt rîten sach]: da man den helt rîten sach / sin ors von isene trvͦc ein tach T  $\cdot$ sîn] Sper U  $\cdot$ îsene] [yser]: ysern U [is*]: iser V \textbf{23} Daz was vur [sleg]: slege des ein dach U das waz [fur]: fúr slege des [*ach]: gemach V \textbf{26} diu ringe] Ringe die U (V) \textbf{29} ouch] \textit{om.} U V \textbf{30} Arabi] Arabŷ T [arab*]: arabi U \newline
\end{minipage}
\end{table}
\end{document}
