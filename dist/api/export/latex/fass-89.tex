\documentclass[8pt,a4paper,notitlepage]{article}
\usepackage{fullpage}
\usepackage{ulem}
\usepackage{xltxtra}
\usepackage{datetime}
\renewcommand{\dateseparator}{.}
\dmyyyydate
\usepackage{fancyhdr}
\usepackage{ifthen}
\pagestyle{fancy}
\fancyhf{}
\renewcommand{\headrulewidth}{0pt}
\fancyfoot[L]{\ifthenelse{\value{page}=1}{\today, \currenttime{} Uhr}{}}
\begin{document}
\begin{table}[ht]
\begin{minipage}[t]{0.5\linewidth}
\small
\begin{center}*D
\end{center}
\begin{tabular}{rl}
\textbf{89} & Daz lobte ir der werde man.\\ 
 & si nam urloup, \textbf{dô} vuor \textbf{si} \textbf{sân}.\\ 
 & si huop \textbf{Kaylet}, der degen wert,\\ 
 & sunder \textbf{schamel} ûf ir pfert\\ 
5 & und \textbf{gienc} von ir \textbf{hin} wider în,\\ 
 & \textbf{al} dâ er \textbf{sach} die \textbf{vrouwen} sîn.\\ 
 & \begin{large}E\end{large}r sprach ze Hardize:\\ 
 & "iwer swester Alize\\ 
 & mir minne bôt. die nam ich dâ.\\ 
10 & \textbf{diu ist} bestatet anderswâ\\ 
 & \textbf{unt} werdeclîcher denne ze mir.\\ 
 & durch iwer zuht lât zornes gir.\\ 
 & si hât der vürste Lambekin.\\ 
 & \textbf{al} sul si niht gekrœnet sîn,\\ 
15 & si hât doch werdecheit bekant.\\ 
 & Hanouwe unde Brabant\\ 
 & ir dienet unt manec ritter guot.\\ 
 & kêrt mir ze \textbf{grüezen} iweren muot.\\ 
 & lât mich in iwern hulden sîn\\ 
20 & \textbf{unt} nemt \textbf{hin} wider dienest mîn."\\ 
 & Der künec von Gascon \textbf{dô} sprach,\\ 
 & als im sîn manlîch ellen jach:\\ 
 & "iwer \textbf{rede} was ie süeze.\\ 
 & \textbf{swer} iuch dâr umbe grüeze,\\ 
25 & dem ir \textbf{vil lasters} hât getân,\\ 
 & der woltz \textbf{doch} durch vorhte \textbf{lân}.\\ 
 & mich vienc iwer muomen sun.\\ 
 & der kan an niemen missetuon."\\ 
 & "Ir werdet wol ledic \textbf{von} Gahmurete.\\ 
30 & daz sol sîn mîn êrstiu bete.\\ 
\end{tabular}
\scriptsize
\line(1,0){75} \newline
D \newline
\line(1,0){75} \newline
\textbf{1} \textit{Majuskel} D  \textbf{7} \textit{Initiale} D  \textbf{21} \textit{Majuskel} D  \textbf{29} \textit{Majuskel} D  \newline
\line(1,0){75} \newline
\textbf{2} sân] [*an]: san D \textbf{3} Kaylet] kaẏlet D \textbf{7} Hardize] Hardyse D \textbf{8} Alize] Alise D \textbf{13} Lambekin] Læmbekin D \textbf{16} Hanouwe] Hanoͮwe D \textbf{29} Gahmurete] Gahmvrete D \newline
\end{minipage}
\hspace{0.5cm}
\begin{minipage}[t]{0.5\linewidth}
\small
\begin{center}*m
\end{center}
\begin{tabular}{rl}
 & daz lobete ir der werde man.\\ 
 & si nam urloup, \textbf{dô} vuor \textbf{si} \textbf{dan}.\\ 
 & si huop \textbf{Kailet}, der degen wert,\\ 
 & sunder \textbf{schemel} ûf ir pfert\\ 
5 & und \textbf{gienc} von ir \textbf{hin} wider în,\\ 
 & \textbf{al}dâ er \textbf{sach} die \textbf{vröude} sîn.\\ 
 & er sprach ze Hardize:\\ 
 & "iuwer swester Alize\\ 
 & mir minne bôt. die nam ich dâ.\\ 
10 & \textbf{d\textit{iu}st} \textbf{nû} bestatet anderswâ\\ 
 & \textbf{und} wer\textit{d}e\textit{c}lîcher danne ze mir.\\ 
 & durch i\textit{uwe}r zuht lât zornes gir.\\ 
 & si hât der vürste Lambekin.\\ 
 & \textbf{al} \dag sullent\dag  si niht gekrœnet sîn,\\ 
15 & si hât doch werdicheit bekant.\\ 
 & Hanouwe und Brab\textit{a}nt\\ 
 & ir dienet und manic ritter guot.\\ 
 & kêret mir ze \textbf{gruoze} iuwern muot.\\ 
 & lât mich in iuwern hulden sîn\\ 
20 & \textbf{und} nem\textit{e}t \textbf{hin} wider \textbf{den} dienst mîn."\\ 
 & \begin{large}D\end{large}er künic von Gascone \textbf{dô} sprach,\\ 
 & als im sîn manlîch ellen jach:\\ 
 & "iuwer \textbf{rede} was ie süeze.\\ 
 & \textbf{wer} iuch dâr umbe grüeze,\\ 
25 & dem ir \textbf{vil lasters} habet getân,\\ 
 & der wolte ez \textbf{doch} durch vorhte \textbf{gelân}.\\ 
 & mich vienc i\textit{uwe}r muomen suon.\\ 
 & der kan an niemen missetuon."\\ 
 & "ir werdet wol ledic \textbf{von} Gahmurete.\\ 
30 & daz sol sîn mîn êrstiu bete.\\ 
\end{tabular}
\scriptsize
\line(1,0){75} \newline
m n o \newline
\line(1,0){75} \newline
\textbf{21} \textit{Initiale} m   $\cdot$ \textit{Capitulumzeichen} n  \newline
\line(1,0){75} \newline
\textbf{2} nam] \textit{om.} o  $\cdot$ dô vuor si] vnd fuͦr n (o) \textbf{3} \textit{Vers 89.3 fehlt} n   $\cdot$ Kailet] kalet o \textbf{5} hin wider] hin wider hin n wider o \textbf{7} Hardize] hardicze m o harditze n \textbf{8} Alize] alitze n alcze o \textbf{9} dâ] do n \textbf{10} diust] Duͦst m \textbf{11} werdeclîcher] werckelicher m \textbf{12} iuwer] ire m ir n o \textbf{14} gekrœnet] bekronet o \textbf{16} Hanouwe] Hanowe n Honowe o  $\cdot$ Brabant] brabent m brobrant n brabrant o \textbf{17} dienet] dienent n \textbf{20} nemet] nemenet m nemen o  $\cdot$ hin] \textit{om.} n o \textbf{21} Gascone] kascone n  $\cdot$ dô] \textit{om.} n o  $\cdot$ sprach] spach n \textbf{23} was] were n \textbf{24} grüeze] gruͦsz o \textbf{26} doch] \textit{om.} o  $\cdot$ gelân] lan n o \textbf{27} iuwer] ire m ir n o \textbf{28} niemen] nẏemans n \textbf{29} von] \textit{om.} o  $\cdot$ Gahmurete] gahmurette m gamiret n gamuͯret o \newline
\end{minipage}
\end{table}
\newpage
\begin{table}[ht]
\begin{minipage}[t]{0.5\linewidth}
\small
\begin{center}*G
\end{center}
\begin{tabular}{rl}
 & d\textit{az} lobt ir d\textit{er} \textit{we}r\textit{d}e man.\\ 
 & si nam urloup \textbf{und} vuor \textbf{von} \textbf{dan}.\\ 
 & si huop \textbf{Kailet}, der degen wert,\\ 
 & sunder \textbf{schamel} ûf ir pfert\\ 
5 & unde \textbf{kêrte} von ir wider în,\\ 
 & dâ er \textbf{vant} die \textbf{vr\textit{öu}de} sîn.\\ 
 & er sprach ze Hardize:\\ 
 & "iwer swester Alize\\ 
 & mir minne bôt. die nam ich dâ.\\ 
10 & \textbf{diu ist} bestatet anderswâ,\\ 
 & werdiclîcher dane ze mir.\\ 
 & durch iwer zuht lât zornes gir.\\ 
 & \begin{large}S\end{large}i hât der vürste Lambikin.\\ 
 & \textbf{en}sul si niht gekrœnet sîn,\\ 
15 & si hât doch werdicheit bekant.\\ 
 & Hengouwe und Brabant\\ 
 & ir dienet und manic rîter guot.\\ 
 & kêret mir ze \textbf{gruoze} iweren muot.\\ 
 & lât mich in iweren hulden sîn,\\ 
20 & nemet \textbf{hin} wider \textbf{den} dienst mîn."\\ 
 & der künic von Gascone sprach,\\ 
 & als im sîn manlîch ellen jach:\\ 
 & "iwer \textbf{rede} was ie süeze.\\ 
 & \textbf{der} iuch dâr umbe grüeze,\\ 
25 & dem ir \textbf{grôz laster} habet getân,\\ 
 & der woltez \textbf{iedoch} durch vorhte \textbf{lân}.\\ 
 & mich vienc iwer muomen sun.\\ 
 & der kan an niemen missetuon."\\ 
 & "ir werdet wol ledic \textbf{von} Gahmuret.\\ 
30 & daz sol sîn mîn êrstiu bet.\\ 
\end{tabular}
\scriptsize
\line(1,0){75} \newline
G I O L M Q R Z Fr21 \newline
\line(1,0){75} \newline
\textbf{1} \textit{Initiale} O M  \textbf{7} \textit{Initiale} L Q R Z Fr21  \textbf{13} \textit{Initiale} G  \newline
\line(1,0){75} \newline
\textbf{1} daz] ditze G ÷az O  $\cdot$ lobt] gelobit M (R)  $\cdot$ ir] ir do O Q Fr21  $\cdot$ der werde] dirre G \textbf{2} und] do O Q Fr21 da R Z  $\cdot$ von] si O (M) (Q) (Z) Fr21 \textit{om.} L sin R \textbf{3} Kailet] Gahilet I kaylet O L R Fr21 kayleten Q Gailet Z \textbf{4} sunder schamel] Sundern scam M Sunder schnel R  $\cdot$ ir] ein R \textbf{5} unde] Er L  $\cdot$ kêrte] chert I (O) (R)  $\cdot$ von ir wider] von en wider M wider von ir Z \textbf{6} dâ] Al da O (M) (Q) (R) (Z) (Fr21)  $\cdot$ er vant] [vant]: want Q  $\cdot$ die vröude] die frivnde G die vrowe L den frund R \textbf{7} er] Der Fr21  $\cdot$ Hardize] hardieze G (L) (Fr21) \textbf{8} Alize] alieze G (L) Fr21 Alixe R alise Z \textbf{9} mir] Mit Q Min Z  $\cdot$ bôt] enbot Z  $\cdot$ dâ] do Q \textbf{10} bestatet] nv bestetet O (L) (M) (R) (Z) (Fr21) \textbf{11} werdiclîcher] Vnde werdechlicher O (L) (M) (Q) (R) (Z) (Fr21) \textbf{12} durch] Doch Q  $\cdot$ lât] laz M \textbf{13} hât] bat Q  $\cdot$ vürste] kvnic Fr21  $\cdot$ Lambikin] lambechin G lambekin I lamechin O Lammekin L (M) (Q) (R) Lemmekein Z [lamecbin]: lamechin Fr21 \textbf{14} ensul] sol I Vnde sol O (Fr21) Svln L Adir sal M Als Q R Also svͤl Z  $\cdot$ si] \textit{om.} O L  $\cdot$ gekrœnet] [gekrot]: gekronet G \textbf{15} si] So O  $\cdot$ hât] hant L  $\cdot$ doch] \textit{om.} O \textbf{16} Hengouwe] hengoͮwe G henoͮwe I (Fr21) Heͮnawe O Henegowe L Henouwe M Henofire Q henoͯwe R Henegev Z  $\cdot$ Brabant] [bra*]: brabant Q Prabant Z \textbf{17} ir dienet] Jn dienent L  $\cdot$ und] \textit{om.} Z \textbf{18} ze gruoze] zegrvͦzen O (M) (Q) (Z) (Fr21) \textbf{19} iweren hulden] v́wer hulde R \textbf{20} nemet] Vnde nempt O (L) (M) (Q) (R) (Z) Vnde nem Fr21 \textbf{21} Gascone] ascone G kaschonie I Gaschvn O Gascon L (M) (Q) Z Gazo R Gascoͮn Fr21  $\cdot$ sprach] da sprach Z \textbf{22} ellen] elle M eren Q  $\cdot$ jach] veriach I \textbf{23} ie] ê O (Fr21) \textbf{24} der] Swer O Z (Fr21) Wer L M Q R \textbf{25} grôz laster] laide I \textbf{26} iedoch] doch I L (M) Q ye R  $\cdot$ vorhte] forchten Q \textbf{27} mich] Jch O Fr21  $\cdot$ iwer] iwern O Fr21 dach uwer M \textbf{28} kan] on kan M (Z) \textbf{29} werdet] werden L  $\cdot$ Gahmuret] Gamvret O (M) (Z) Gahmuͦret L gamurret Q Gahmoret Fr21 \textbf{30} sol sîn] ist nv L so sin R  $\cdot$ êrstiu] erste R \newline
\end{minipage}
\hspace{0.5cm}
\begin{minipage}[t]{0.5\linewidth}
\small
\begin{center}*T (U)
\end{center}
\begin{tabular}{rl}
 & daz lobet ir der werde man.\\ 
 & si nam urloup, \textbf{dô} vuor \textbf{si} \textbf{dan}.\\ 
 & si huop der degen wert\\ 
 & sunder \textbf{schame} ûf ir pfert\\ 
5 & und \textbf{kêrte} von ir wider în,\\ 
 & \textbf{al} dâ er \textbf{vant} die \textbf{vreude} sîn.\\ 
 & er sprach zuo Hardize:\\ 
 & "iuwer swester Alize\\ 
 & mi\textit{r} minne bôt. die nam ich dâ.\\ 
10 & \textbf{dû bist} \textbf{nû} bestatet anderswâ.\\ 
 & \multicolumn{1}{l}{ - - - }\\ 
 & \multicolumn{1}{l}{ - - - }\\ 
 & si hât der vürste Lambekin.\\ 
 & sol si niht gekrœnet sîn,\\ 
15 & si hât doch wirdecheit bekant.\\ 
 & Hanouwe und Brabant\\ 
 & ir dient und manec ritter guot.\\ 
 & kêrt mir zuo \textbf{gruoze} iuwern muot.\\ 
 & lât mich in iuwern hulden sîn\\ 
20 & \textbf{und} nemet \textbf{in} wider, \textbf{den} dienst mîn."\\ 
 & der künec von Gascone sprach,\\ 
 & als im sîn manlîch ellen jach:\\ 
 & "iuwer \textbf{ellen} was ie süeze.\\ 
 & \textbf{swer} iuch dâr umb grüeze,\\ 
25 & dem ir \textbf{laster} hât getân,\\ 
 & der wolte ez \textbf{iedoch} durch vorhte \textbf{lân}.\\ 
 & mich vienc iuwer muomen suon.\\ 
 & der \textbf{en}kan an niemanne missetuon."\\ 
 & "ir werdet wol ledic \textbf{an} Gahmurete.\\ 
30 & daz sol sîn mîn êrstiu bete.\\ 
\end{tabular}
\scriptsize
\line(1,0){75} \newline
U V W T \newline
\line(1,0){75} \newline
\textbf{1} \textit{Majuskel} T  \textbf{3} \textit{Majuskel} T  \textbf{7} \textit{Initiale} T W  \textbf{8} \textit{Majuskel} T  \textbf{19} \textit{Majuskel} T  \textbf{21} \textit{Majuskel} T  \newline
\line(1,0){75} \newline
\textbf{1} daz lobet] Dis gelobte W \textbf{2} dô vuor si] vnde fuͦr V vnde sciet T \textbf{3} der] Gaiolet der V gaylet den W kaylet der T \textbf{4} schame] [sch*]: schemel V schone W schamen T  $\cdot$ ir] sein W \textbf{6} vreude] frauwen W \textbf{7} Hardize] Hardyse U T Hardise V (W) \textbf{8} Alize] alyse U Alẏse V alise W Alŷse T \textbf{9} mir] Mit U Mir ir W  $\cdot$ dâ] do W \textbf{10} dû bist nû] die ist nv V (W) div ist T  $\cdot$ bestatet] bestat bestatet T \textbf{11} \textit{Die Verse 89.11-12 fehlen} U W   $\cdot$ Vnde werdeklicher danne zuͦ mir V (T) \textbf{12} durch uwer zuht lant zornes gir V (T) \textbf{13} der vürste] [den fúrsten]: der fúrste V  $\cdot$ Lambekin] Lamekin U Lamechin V (W) [lamekeit]: lamekin  T \textbf{14} sol si niht] [*]: Alse sv́ nv́t V alde sol si svs T \textbf{15} doch] oͮch V (W)  $\cdot$ bekant] erkant W \textbf{16} Hanouwe] Hanegowe U (V) Hennowe W Hagenôuwe T \textbf{17} ir dient] dient ir T  $\cdot$ und] \textit{om.} V \textbf{18} mir] mir es W  $\cdot$ gruoze] gvͦte T  $\cdot$ iuwern] in yren W \textbf{20} in] hin V (W) da T \textbf{21} Gascone] Gasgone U (W) T Gaschonie V \textbf{23} ellen] [*]: rede V rede W T \textbf{24} swer] Wer W \textbf{26} iedoch] ye W \textbf{28} enkan] kan V T \textbf{29} \textit{Die Verse 89.29-30 fehlen} T   $\cdot$ an] von W  $\cdot$ Gahmurete] Gamuͦret U Gamurette V gamuret W \newline
\end{minipage}
\end{table}
\end{document}
