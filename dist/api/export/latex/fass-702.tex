\documentclass[8pt,a4paper,notitlepage]{article}
\usepackage{fullpage}
\usepackage{ulem}
\usepackage{xltxtra}
\usepackage{datetime}
\renewcommand{\dateseparator}{.}
\dmyyyydate
\usepackage{fancyhdr}
\usepackage{ifthen}
\pagestyle{fancy}
\fancyhf{}
\renewcommand{\headrulewidth}{0pt}
\fancyfoot[L]{\ifthenelse{\value{page}=1}{\today, \currenttime{} Uhr}{}}
\begin{document}
\begin{table}[ht]
\begin{minipage}[t]{0.5\linewidth}
\small
\begin{center}*D
\end{center}
\begin{tabular}{rl}
\textbf{702} & \begin{large}A\end{large}rtus die bete \textbf{hôrte};\\ 
 & daz gespræche er zerstôrte,\\ 
 & mit \textbf{in} wider \textbf{an} den rinc \textbf{er} saz.\\ 
 & Gawans schenke niht vergaz,\\ 
5 & dar \textbf{en}\textbf{trüegen} junchêrrelîn\\ 
 & manegen tiwern kopf guldîn\\ 
 & mit edelem gesteine.\\ 
 & der schenke gienc niht eine.\\ 
 & Dô daz schenken \textbf{geschach},\\ 
10 & daz volc vuor \textbf{gar} an sîn gemach.\\ 
 & \textbf{dô} begunde\textbf{z} \textbf{ouch nâhen} \textbf{der} naht.\\ 
 & Parzival \textbf{was} sô bedâht,\\ 
 & \textbf{al sîn harnasch er} besach,\\ 
 & ob dem \textbf{iht} riemen gebrach.\\ 
15 & daz hiez \textit{er} wol bereiten\\ 
 & unt wünneclîche feiten\\ 
 & unt einen niwen schilt gewinnen.\\ 
 & der sîne was \textbf{ûzen} unt innen\\ 
 & zerhurtiert unt \textbf{ouch} \textbf{zerslagen};\\ 
20 & man muose im einen \textbf{starken} tragen.\\ 
 & daz tâten sarjande,\\ 
 & die \textbf{vil} \textbf{wênec er} bekande;\\ 
 & etslîcher was ein Franzeys.\\ 
 & sîn ors, daz der templeys\\ 
25 & gein im zer tjoste brâhte,\\ 
 & ein knappe des gedâhte,\\ 
 & ez wart nie baz \textbf{erstrichen} sît.\\ 
 & \textbf{dô} was ez naht unt \textbf{slâfes} zît.\\ 
 & Parzival \textbf{ouch} \textbf{slâfes} pflac;\\ 
30 & sîn harnasch gar vor im \textbf{dâ} \textbf{lac}.\\ 
\end{tabular}
\scriptsize
\line(1,0){75} \newline
D Fr66 \newline
\line(1,0){75} \newline
\textbf{1} \textit{Initiale} D Fr66  \textbf{9} \textit{Majuskel} D  \newline
\line(1,0){75} \newline
\textbf{1} Artus] AArtus Fr66 \textbf{12} Parzival] Parcival D \textbf{15} er] \textit{om.} D \textbf{29} Parzival] Parcifal D \newline
\end{minipage}
\hspace{0.5cm}
\begin{minipage}[t]{0.5\linewidth}
\small
\begin{center}*m
\end{center}
\begin{tabular}{rl}
 & \begin{large}A\end{large}rtus die bete \textbf{erhôrte};\\ 
 & daz gespræch er zerstôrte,\\ 
 & mit \textbf{in} wider \textbf{an} den rinc \textbf{er} saz.\\ 
 & Gawans schenke niht vergaz:\\ 
5 & dar \textbf{truogen} junchêrrelîn\\ 
 & manigen tiuren kopf guldîn\\ 
 & mit edelem gesteine.\\ 
 & der schenke gienc niht eine.\\ 
 & dô da\textit{z} schenken \textbf{beschach},\\ 
10 & daz volc vuor an sîn gemach.\\ 
 & \textbf{dô} begunde \textbf{ez} \textbf{nâhen ouch} \textbf{der} naht.\\ 
 & Parcifal \textbf{was} sô bedâht,\\ 
 & \textbf{al sîn harnasch er} besach,\\ 
 & ob dem \textbf{iht} riemen gebrach.\\ 
15 & daz hie er wol bereiten\\ 
 & und wünneclîchen feiten\\ 
 & und einen niuwen schilt gew\textit{i}nnen.\\ 
 & der sîn was \textbf{ûz} und innen\\ 
 & zerhurtieret und \textbf{erslagen};\\ 
20 & man muost im einen \textbf{starken} tragen.\\ 
 & daz tâten sarjande,\\ 
 & die \textbf{vil} \textbf{wênic er} bekante;\\ 
 & etlîcher was ein Franzois.\\ 
 & sîn ros, daz der \textit{t}emplois\\ 
25 & gegen im zer juste brâhte,\\ 
 & ein knappe des gedâhte,\\ 
 & \dag ich\dag  wart nie baz \textbf{durchstrichen} sît.\\ 
 & \textbf{dô} was ez naht und \textbf{slâfens} zît.\\ 
 & Parcifal \textbf{ouch} \textbf{slâfens} pflac;\\ 
30 & sîn harnasch gar vor im \textbf{gelac}.\\ 
\end{tabular}
\scriptsize
\line(1,0){75} \newline
m n o \newline
\line(1,0){75} \newline
\textbf{1} \textit{Initiale} m n  \newline
\line(1,0){75} \newline
\textbf{2} gespræch] gesprach o \textbf{9} daz] dach m \textbf{10} \textit{Die Verse 702.10-30 fehlen} o  \textbf{13} al sîn] Allen sinen n \textbf{17} gewinnen] gewunnen m \textbf{18} ûz] vssen n \textbf{19} erslagen] zerslagen n \textbf{23} Franzois] frantzois m n \textbf{24} templois] pemplois m \textbf{26} gedâhte] bedacht n \textbf{27} durchstrichen] erstrichen n \newline
\end{minipage}
\end{table}
\newpage
\begin{table}[ht]
\begin{minipage}[t]{0.5\linewidth}
\small
\begin{center}*G
\end{center}
\begin{tabular}{rl}
 & \begin{large}A\end{large}rtus die bete \textbf{hôrte};\\ 
 & daz gespræche er \textbf{gar} zerstôrte,\\ 
 & mit \textbf{in} \textbf{er} wider \textbf{in} den rinc saz.\\ 
 & Gawans schenke niht vergaz:\\ 
5 & dar \textbf{truogen} junchêrrelîn\\ 
 & manigen tiuren kopf guldîn\\ 
 & mit edelem gesteine.\\ 
 & der schenke gie niht eine.\\ 
 & dô daz schenken \textbf{geschach},\\ 
10 & daz volc vuor \textbf{gar} an sîn gemach.\\ 
 & \textbf{nû} begunde \textbf{ouch nâhen} \textbf{diu} naht.\\ 
 & Parcival sô bedâht,\\ 
 & \textbf{daz er sîn harnasch} besach,\\ 
 & op dem \textbf{deheines} riemen gebrach.\\ 
15 & daz hiez er wol bereiten\\ 
 & unde wünniclîchen feiten\\ 
 & unde einen niwen schilt gewinnen,\\ 
 & \textbf{wan} der sîn was \textbf{ûzen} unde innen\\ 
 & zerhurtiert unde \textbf{zerslagen};\\ 
20 & man muose im einen \textbf{niwen} tragen.\\ 
 & daz tâten sarjande,\\ 
 & die \textbf{wênic er} bekande;\\ 
 & etslîcher was ein Franzoys.\\ 
 & sîn ors, daz der templ\textit{oy}s\\ 
25 & gein im zer tjoste brâhte,\\ 
 & ein knappe des gedâhte,\\ 
 & ez\textbf{ne} wart nie baz \textbf{erstrichen} sît.\\ 
 & \textbf{nû} was ez naht unde \textbf{slâfe\textit{n}s} zît.\\ 
 & Parcival \textbf{dô} \textbf{slâfes} pflac;\\ 
30 & sîn harnasch gar vor im \textbf{dô} \textbf{lac}.\\ 
\end{tabular}
\scriptsize
\line(1,0){75} \newline
G I L M Z \newline
\line(1,0){75} \newline
\textbf{1} \textit{Initiale} G I L  \textbf{23} \textit{Initiale} I  \newline
\line(1,0){75} \newline
\textbf{1} hôrte] gehorte M \textbf{3} in den] an den L (M) Z \textbf{4} Gawans] Gawan I Gawanz L  $\cdot$ schenke] senchens I \textbf{5} dar truogen] Darn truͯgen L (M) (Z) \textbf{6} tiuren] \textit{om.} I \textbf{8} schenke] geschenke M  $\cdot$ gie niht] nicht gienc M  $\cdot$ eine] al eine Z \textbf{9} dô] Da M  $\cdot$ daz] daz volch L  $\cdot$ schenken] [geschenkin]: schenkin M  $\cdot$ geschach] sach L \textbf{10} daz volc] Ez L  $\cdot$ vuor gar] gar vur M  $\cdot$ sîn] sinen Z \textbf{11} ouch] \textit{om.} I \textbf{12} Parcival] parcifal G (Z) Parzifal I L M  $\cdot$ sô] sich do I waz so L (M) was ouch so Z \textbf{14} op] Vff M  $\cdot$ dem] im Z  $\cdot$ deheines riemen] dehain rieme I (L) \textbf{16} feiten] seiten M \textbf{18} ûzen] vz I Z \textbf{20} niwen] starchen L (M) (Z)  $\cdot$ tragen] dar tragen I \textbf{22} er bekande] erchande I \textbf{23} Franzoys] franzois G I franzoýs L franzois M frantzeis Z \textbf{24} temploys] templeis G (M) (Z) \textbf{25} zer] zuͯ L \textbf{26} des] dy des M \textbf{27} ezne] Ez M  $\cdot$ erstrichen] gestrichen L erstriten M \textbf{28} ez naht unde] \textit{om.} M  $\cdot$ slâfens] slaffes G \textbf{29} Parcival] Parcifal G Z Parzifal I L M  $\cdot$ dô] \textit{om.} I da M Z \textbf{30} gar] \textit{om.} L  $\cdot$ vor] bi I  $\cdot$ dô] \textit{om.} I da L M Z \newline
\end{minipage}
\hspace{0.5cm}
\begin{minipage}[t]{0.5\linewidth}
\small
\begin{center}*T
\end{center}
\begin{tabular}{rl}
 & Artus die bete \textbf{hôrte};\\ 
 & daz gespræcher \textbf{gar} \textit{ze}rstôrte,\\ 
 & mit \textbf{im} \textbf{er} wider \textbf{an} den rinc saz.\\ 
 & Gawans schenke niht vergaz,\\ 
5 & dar \textbf{en}\textbf{trüegen} junchêrrelîn\\ 
 & manegen tiuren kopf guldîn\\ 
 & mit edelem gesteine.\\ 
 & der schenke gienc niht eine.\\ 
 & dô daz schenken \textbf{geschach},\\ 
10 & daz volc vuor an sîn gemach.\\ 
 & \textbf{\begin{large}N\end{large}û} begunde \textbf{ouch nâhen} \textbf{diu} naht.\\ 
 & Parcifal \textbf{was} sô bedâht,\\ 
 & \textbf{daz er sîn harnasch} besach,\\ 
 & o\textit{b} dem \textbf{dekeines} riemen gebrach.\\ 
15 & daz hiez er wol bereiten\\ 
 & und wünneclîche feiten\\ 
 & und einen niuwen schilt gewinnen,\\ 
 & \textbf{wan} der \textit{sîne} was \textbf{ûz} und innen\\ 
 & zerhurtieret und \textbf{zerslagen};\\ 
20 & man muos im \textbf{dar} einen \textbf{starken} tragen.\\ 
 & daz tâten sarjande,\\ 
 & die \textbf{er wênic} bekande;\\ 
 & etslîcher was ein Franzeis.\\ 
 & sîn ors, daz der templeis\\ 
25 & gein im zuo der jost brâhte,\\ 
 & ein knappe des gedâhte,\\ 
 & ez \textbf{en}wart nie baz \textbf{erstrichen} sît.\\ 
 & \textbf{nû} was ez naht und \textbf{slâfens} zît.\\ 
 & Parcifal \textbf{dô} \textbf{slâfes} pflac;\\ 
30 & sîn harnasch gar vor im \textbf{d\textit{â}} \textbf{lac}.\\ 
\end{tabular}
\scriptsize
\line(1,0){75} \newline
U V W Q R \newline
\line(1,0){75} \newline
\textbf{1} \textit{Initiale} Q R  \textbf{11} \textit{Initiale} U W  \newline
\line(1,0){75} \newline
\textbf{1} Artus] Kúnig artus W  $\cdot$ hôrte] erhorte W \textbf{2} zerstôrte] zu der storte U \textbf{3} im] [e*]: in V in W Q (R) \textbf{4} Gawans] Herr gawans W Gawins R \textbf{5} entrüegen] truͦgen W \textbf{8} eine] [alleine]: eine Q alleine R \textbf{9} schenken] schenken do V \textbf{10} vuor] das fuͦr W \textbf{11} ouch] \textit{om.} W \textbf{12} Parcifal] Parzifal U Parzefal V Er partzifal W Partzifal Q Parczifal R  $\cdot$ sô] also W \textbf{14} ob] Oder U  $\cdot$ dekeines] [*]: dekeines U  $\cdot$ riemen] ringes V rimensz Q  $\cdot$ gebrach] gebrech R \textbf{18} sîne] \textit{om.} U  $\cdot$ ûz] aussen W (Q) (R) \textbf{19} zerhurtieret] Zerhurttet R \textbf{20} muos] muͤst V  $\cdot$ dar] \textit{om.} W Q R \textbf{22} er wênic] wening er V (W) (Q) o\textit{m. } R  $\cdot$ bekande] erkande R \textbf{23} Franzeis] franzois U [Franz*]: Franzeẏs V frantzeis Q franczeis R \textbf{24} templeis] [temp*]: templeẏs V templis R \textbf{25} der jost] strit R \textbf{27} enwart] ward R  $\cdot$ erstrichen] erschrigen Q \textbf{28} slâfens] schlaffes W (Q) schlaffes Rechtte R \textbf{29} Parcifal] Parzifal U Parzefal V Partzifal W Q Parczifal R  $\cdot$ slâfes] schlaffens R \textbf{30} dâ lac] do lac U (W) Q (R) [*]: do lag  V \newline
\end{minipage}
\end{table}
\end{document}
