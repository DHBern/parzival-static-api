\documentclass[8pt,a4paper,notitlepage]{article}
\usepackage{fullpage}
\usepackage{ulem}
\usepackage{xltxtra}
\usepackage{datetime}
\renewcommand{\dateseparator}{.}
\dmyyyydate
\usepackage{fancyhdr}
\usepackage{ifthen}
\pagestyle{fancy}
\fancyhf{}
\renewcommand{\headrulewidth}{0pt}
\fancyfoot[L]{\ifthenelse{\value{page}=1}{\today, \currenttime{} Uhr}{}}
\begin{document}
\begin{table}[ht]
\begin{minipage}[t]{0.5\linewidth}
\small
\begin{center}*D
\end{center}
\begin{tabular}{rl}
\textbf{417} & \begin{large}D\end{large}ô sprach der vürste Liddamus:\\ 
 & "waz \textbf{solte} \textbf{der} in mînes hêrren hûs,\\ 
 & der im sînen vater sluoc\\ 
 & unt \textbf{daz laster im} sô nâhe truoc?\\ 
5 & ist \textbf{mîn hêrre} \textbf{wert} bekant,\\ 
 & \textbf{daz} richet \textbf{alhie} sîn \textbf{selbes} hant.\\ 
 & sô gelt ein tôt den andern tôt.\\ 
 & ich wæne gelîche \textbf{sîn} \textbf{die} nôt."\\ 
 & Nû seht \textbf{ir}, wie Gawan dô stuont.\\ 
10 & alrêst was im grôz angest kunt.\\ 
 & Dô sprach Kingrimursel:\\ 
 & "swer mit der \textbf{drô} wære sô snel,\\ 
 & der solte ouch gâhen in den strît.\\ 
 & ir habt gedrenge oder wît,\\ 
15 & man mac sich iwer lîhte erwern.\\ 
 & hêr Liddamus, vil wol ernern\\ 
 & \textbf{trûwe ich vor iu disen} man.\\ 
 & swaz \textbf{iu der hete} getân,\\ 
 & ir liezet\textbf{z} \textbf{ungerochen}.\\ 
20 & ir habt iuch gar versprochen.\\ 
 & man \textbf{sol iu wol gelouben},\\ 
 & daz iuch nie mannes ougen\\ 
 & gesâhen \textbf{ze vorderst}, dâ man streit.\\ 
 & iu was ie strîten wol sô leit,\\ 
25 & daz ir der vluht begundet.\\ 
 & dennoch ir mêr \textbf{wol} kundet:\\ 
 & swâ \textbf{man} ie gein strîte dranc,\\ 
 & dâ \textbf{tætet} ir wîbes widerwanc.\\ 
 & swelich künec sich lât an iwern rât,\\ 
30 & vil twerhes \textbf{dem} \textbf{diu} krône stât.\\ 
\end{tabular}
\scriptsize
\line(1,0){75} \newline
D Fr5 \newline
\line(1,0){75} \newline
\textbf{1} \textit{Initiale} D  \textbf{9} \textit{Majuskel} D  \textbf{11} \textit{Majuskel} D  \newline
\line(1,0){75} \newline
\textbf{28} tætet] tatint Fr5 \textbf{30} dem] im Fr5 \newline
\end{minipage}
\hspace{0.5cm}
\begin{minipage}[t]{0.5\linewidth}
\small
\begin{center}*m
\end{center}
\begin{tabular}{rl}
 & dô sprach der vürste Liddam\textit{u}s:\\ 
 & "waz \textbf{solt} \textbf{er} in mînes hêrren hûs,\\ 
 & der ime sîne\textit{n} vater sluoc\\ 
 & und \textbf{daz laster ime} sô nâhe truoc?\\ 
5 & ist \textbf{mîn hêr} \textbf{an} \textbf{wer} bekant,\\ 
 & \textbf{daz} richet \textbf{an wer} sîn \textbf{selbes} hant.\\ 
 & sô gelte ein tôt den andern tôt.\\ 
 & ich wæne glîch \textbf{sîn} \textbf{die} nôt."\\ 
 & nû sehet \textbf{ir}, wie Gawan dô stuont.\\ 
10 & allerêrst was ime grôz angest kunt.\\ 
 & \begin{large}D\end{large}ô sprach Kingri\textit{m}ursel:\\ 
 & "wer mit der \textbf{drô} wære sô snel,\\ 
 & der solte ouch gâhen in den strît.\\ 
 & ir habt gedrenge oder wît,\\ 
15 & man mac sich iuwer lîhte erwern.\\ 
 & hêr Liddamus, vil wol ernern\\ 
 & \textbf{trûwe ich vor iu disen} man.\\ 
 & waz \textbf{iu der hete} getân,\\ 
 & ir liezet \textbf{daz} \textbf{ungerochen}.\\ 
20 & ir habet iuch gar versprochen.\\ 
 & man \textbf{giht daz âne lougen},\\ 
 & daz iuch nie mannes ougen\\ 
 & gesâhen \textbf{ze vorderst}, dâ man streit.\\ 
 & iu was ie strîten wol sô leit,\\ 
25 & daz ir der vluht begundet.\\ 
 & dannoch ir mêr \textbf{wol} kundet:\\ 
 & wâ \textbf{man} ie gegen strîte dranc,\\ 
 & d\textit{â} \textbf{tâte\textit{t}} ir wîbes widerwanc.\\ 
 & welich künic sich lât an iuweren rât,\\ 
30 & vil twerhes \textbf{ime} \textbf{diu} \textit{k}rô\textit{n}e stât.\\ 
\end{tabular}
\scriptsize
\line(1,0){75} \newline
m n o \newline
\line(1,0){75} \newline
\textbf{11} \textit{Illustration mit Überschrift:} Also der kv́nnig kingrumursel zuͯ dem hertzogen liddinynus sprach das er in nit an keime strit zuͯ vorderst sehe n  Also der konig [Rẏmn]: Rẏmorsel zuͦ dem herczogen liddẏnnynus sprach das er >in< niemer an keinem strit zuͦ vordest >nie< sehe o   $\cdot$ \textit{Initiale} m n o  \newline
\line(1,0){75} \newline
\textbf{1} Liddamus] liddamas m lidamis n liddamuͯs o \textbf{2} hûs] his n \textbf{3} ime] in n  $\cdot$ sînen] sine m \textbf{5} bekant] erkant n o \textbf{6} an wer] alhie n o  $\cdot$ sîn] sins o \textbf{9} sehet] scheut o  $\cdot$ dô] da o \textbf{10} kunt] [kust]: kunt o \textbf{11} Kingrimursel] kingrinvrsel m kingrunnursel n kingrumursel o \textbf{13} der] Den n o \textbf{16} hêr Liddamus] Herliddamus m Her liddenynus n Her liddmuus o \textbf{18} hete] hat n \textbf{19} liezet] liessen o \textbf{23} gesâhen] Gesehen n  $\cdot$ vorderst] furdast o  $\cdot$ dâ man streit] an dem strit n o \textbf{27} man] namm o  $\cdot$ dranc] twang o \textbf{28} dâ] Do m n o  $\cdot$ tâtet] totten m (o)  $\cdot$ widerwanc] [wider twang]: wider wang o \textbf{29} künic] [wang]: konig o  $\cdot$ an] [an]: de o  $\cdot$ iuweren] iren m \textbf{30} krône] troste m \newline
\end{minipage}
\end{table}
\newpage
\begin{table}[ht]
\begin{minipage}[t]{0.5\linewidth}
\small
\begin{center}*G
\end{center}
\begin{tabular}{rl}
 & \begin{large}D\end{large}ô sprach der vürste Lidamus:\\ 
 & "waz \textbf{solt} \textbf{der} in mînes hêrren hûs,\\ 
 & der im sînen vater sluoc\\ 
 & unde \textbf{daz laster im} sô nâhen truoc?\\ 
5 & ist \textbf{mî\textit{m} hêrre\textit{n}} \textbf{wer} bekant,\\ 
 & \textbf{ez} richet \textbf{al hie} sîn \textbf{selbes} hant.\\ 
 & sô gelte ein tôt den andern tôt.\\ 
 & ich wæne gelîch \textbf{sîn} \textbf{dise} nôt."\\ 
 & nû seht, wie Gawan dô stuont.\\ 
10 & alrêst was im grôz angest kunt.\\ 
 & dô sprach Kingrimursel:\\ 
 & "swer mit der \textbf{rede} wære sô snel,\\ 
 & der solt ouch gâhen in den strît.\\ 
 & ir habet gedrenge oder wît,\\ 
15 & man mac sich iwer lîhte erweren.\\ 
 & hêr Lidamus, vil wol erneren\\ 
 & \textbf{trûwe ich vor iu disen} man.\\ 
 & swaz \textbf{er iu hete} getân,\\ 
 & ir liezet \textbf{ez} \textbf{unerrochen}.\\ 
20 & ir habet iuch gar versprochen.\\ 
 & man \textbf{sol iu wol gelouben},\\ 
 & daz iuch nie mannes ougen\\ 
 & gesâhen \textbf{ze vorderste}, dâ man streit.\\ 
 & iu was ie strîten wol sô leit,\\ 
25 & daz ir der vluht begundet.\\ 
 & dannoch ir mêr \textbf{wol} kundet:\\ 
 & swâ \textbf{man} ie gein strîte dranc,\\ 
 & dâ \textbf{tâtet} ir wîbes widerwanc.\\ 
 & swelch künic sich læt an iweren rât,\\ 
30 & vil twerhes \textbf{dem} \textbf{sîn} krône stât.\\ 
\end{tabular}
\scriptsize
\line(1,0){75} \newline
G I O L M Q R Z \newline
\line(1,0){75} \newline
\textbf{1} \textit{Initiale} G I O L Q R Z  \textbf{15} \textit{Initiale} I  \newline
\line(1,0){75} \newline
\textbf{1} Dô] ÷o O Da M  $\cdot$ vürste] kúng R  $\cdot$ Lidamus] Liddamus O (Q) (Z) Lýddamvͯs L litdamuͯsz M lidanus R \textbf{2} solt] sol O L Q \textbf{3} sînen] [seinem]: seinen Q \textbf{4} daz laster im] im daz laster O L (R) im sein laster Q \textbf{5} mîm hêrren] min herre G (L) (M) Z mein herren Q  $\cdot$ wer] wert L M  $\cdot$ bekant] erkant Z \textbf{6} ez] Daz O L (M) (Q) (R) Z  $\cdot$ richet] reche O richtet M  $\cdot$ al] \textit{om.} M R  $\cdot$ sîn selbes] sin O L (Q) billich sin R \textbf{7} ein] den I  $\cdot$ den] ein I \textbf{8} sîn] si I O Z  $\cdot$ dise] disiu I div O (L) (M) (Q) (R) (Z) \textbf{9} wie] ir wie O (Q) Z ir M  $\cdot$ Gawan] Gawin R  $\cdot$ dô] da M Z \textbf{10} grôz] \textit{om.} I \textbf{11} dô] Da M  $\cdot$ Kingrimursel] Kyngrimvrsel O (M) kingrýmvͯrsel L kingrim vrsel Q kúngrumursel R \textbf{12} swer] Wer L Q R  $\cdot$ mit der rede wære] mit rede wer I mit der dro wer O M (Z) mit drov were L wer mit der trew Q (R) \textbf{14} oder] olde G \textbf{15} sich] \textit{om.} R  $\cdot$ iwer lîhte] ev vil lihte I iwer wol O euch lichte Q \textbf{16} Lidamus] Liddamvs O (Z) lýddamvs L litdamus M liddanus Q  $\cdot$ vil] wil Q  $\cdot$ wol] lihte O  $\cdot$ erneren] der neren Q [erwern]: ernern R \textbf{17} truͤ ich hin I  $\cdot$ trûwe] Trewet Q  $\cdot$ vor iu] wol L \textbf{18} swaz] Waz L (M) (Q) (R)  $\cdot$ er iu] iv der O (L) (M) (Q) (R) (Z)  $\cdot$ hete] had M \textbf{19} ir] Er Q  $\cdot$ liezet ez] liezez I lieszet L liesset des Q  $\cdot$ unerrochen] vngerochen O L (M) Q R Z \textbf{21} Wan esz ist gar vngelogin M \textbf{22} iuch] \textit{om.} L  $\cdot$ ougen] oͮge I \textbf{23} gesâhen ze vorderste] sach ze vordrest I Gesahen O Q (R) Vch gesahen L  $\cdot$ dâ] do O Q R wa L  $\cdot$ streit] so streit O \textbf{24} strîten] strit I \textbf{25} begundet] [begvnnet]: begvndet O beginnet Q \textbf{26} mêr wol] da mer I mir wol M  $\cdot$ kundet] kúnnet Q \textbf{27} swâ] Wo L (M) Q (R) \textbf{29} swelch] Welch L (R) Wech Q  $\cdot$ künic] \textit{om.} L  $\cdot$ læt] \textit{om.} Z \textbf{30} twerhes] weihes Q  $\cdot$ dem sîn] im diu I (O) (R) dem die L (M) Q Z \newline
\end{minipage}
\hspace{0.5cm}
\begin{minipage}[t]{0.5\linewidth}
\small
\begin{center}*T
\end{center}
\begin{tabular}{rl}
 & Dô sprach der vürste Lyddamus:\\ 
 & "waz \textbf{wolte} \textbf{der} in mînes hêrren hûs,\\ 
 & der im sînen vater sluoc\\ 
 & unde \textbf{im daz laster} sô nâhe truoc?\\ 
5 & ist \textbf{mînem hêrren} \textbf{wer} bekant,\\ 
 & \textbf{daz} richet \textbf{alhie} sîn hant.\\ 
 & sô gelte ein tôt den andern tôt.\\ 
 & ich wæne, glîch \textbf{sî} \textbf{diu} nôt."\\ 
 & Nû seht \textbf{ir}, wie Gawan dô stuont.\\ 
10 & alrêst was im grôz angest kunt.\\ 
 & Dô sprach Kyngrimursel:\\ 
 & "swer mit der \textbf{rede} wære sô snel,\\ 
 & der solt ouch gâhen in den strît.\\ 
 & ir habt gedrenge oder wît,\\ 
15 & man mac sich iuwer lîhte erwern.\\ 
 & hêr Lyddamus, vil wol ernern,\\ 
 & \textbf{ich getriuwe disem} man.\\ 
 & swaz \textbf{iu der hât} getân,\\ 
 & ir liezet\textbf{z} \textbf{ungerochen}.\\ 
20 & ir habt iuch gar versprochen.\\ 
 & man \textbf{sol iu wol gelouben},\\ 
 & daz iuch niemannes ougen\\ 
 & gesâhen, dâ man streit.\\ 
 & iu was ie strîten wol sô leit,\\ 
25 & daz ir der vluht begundet.\\ 
 & dannoch ir mêr kundet:\\ 
 & swâ ie gegen strîte dranc,\\ 
 & dâ \textbf{tâtet} ir wîbes widerwanc.\\ 
 & swelch künec sich lât an iuwern rât,\\ 
30 & vil twerhes \textbf{im} \textbf{diu} krône stât.\\ 
\end{tabular}
\scriptsize
\line(1,0){75} \newline
T U V W \newline
\line(1,0){75} \newline
\textbf{1} \textit{Majuskel} T  \textbf{9} \textit{Majuskel} T  \textbf{11} \textit{Majuskel} T   $\cdot$ \textit{Initiale} W  \newline
\line(1,0){75} \newline
\textbf{1} Lyddamus] Lydamuͦs U littamus V lidamus W \textbf{2} wolte] solte U V W \textbf{5} mînem hêrren] min herre an V \textbf{6} sîn] sin selbes V mein W \textbf{7} gelte] gilt U \textbf{9} ir] \textit{om.} U V  $\cdot$ dô] nuͦ do U \textbf{11} sprach] sprach der U  $\cdot$ Kyngrimursel] kingrimorsel U kingrimursel W \textbf{12} swer] Wer U W  $\cdot$ rede] dro W \textbf{13} in] an W \textbf{15} iuwer lîhte] licht vwer U \textbf{16} Lyddamus] littamus V lidamus W  $\cdot$ vil wol] gar schon W \textbf{17} ich getriuwe disem] Truͦwe ich disen U Truwe ich vor úch disen V (W) \textbf{18} swaz] Waz U (W)  $\cdot$ hât] hette V \textbf{20} iuch] îv T \textbf{21} Man giht dez ane loͮgen V \textbf{22} iuch] îv T \textbf{23} gesâhen] Gesohent zevorderst V  $\cdot$ dâ] Do U (V) (W) \textbf{26} kundet] wol kuͦndet U (V) (W) \textbf{27} swâ] Wo man U Swa man V Mo man W \textbf{28} dâ] Do U W \textbf{29} swelch] Welch U W \textbf{30} im diu krône] im [*]: die crone V dem sein krone W \newline
\end{minipage}
\end{table}
\end{document}
