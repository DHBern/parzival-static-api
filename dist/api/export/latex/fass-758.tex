\documentclass[8pt,a4paper,notitlepage]{article}
\usepackage{fullpage}
\usepackage{ulem}
\usepackage{xltxtra}
\usepackage{datetime}
\renewcommand{\dateseparator}{.}
\dmyyyydate
\usepackage{fancyhdr}
\usepackage{ifthen}
\pagestyle{fancy}
\fancyhf{}
\renewcommand{\headrulewidth}{0pt}
\fancyfoot[L]{\ifthenelse{\value{page}=1}{\today, \currenttime{} Uhr}{}}
\begin{document}
\begin{table}[ht]
\begin{minipage}[t]{0.5\linewidth}
\small
\begin{center}*D
\end{center}
\begin{tabular}{rl}
\textbf{758} & \begin{large}D\end{large}az harnasch \textbf{was} von \textbf{in} getân.\\ 
 & dô schouweten disen \textbf{bunten} man\\ 
 & al, die \textbf{wunders} kunden jehen,\\ 
 & \textbf{die} mohten\textbf{z} dâ mit wârheite \textbf{spehen}:\\ 
5 & Feirefiz truoc vremdiu mâl.\\ 
 & Gawan sprach ze Parzival:\\ 
 & "neve, tuo den gesellen dîn\\ 
 & mir kunt. er treit sô wæhen schîn,\\ 
 & dem ich \textbf{gelîchez} nie gesach."\\ 
10 & Parzival zuo \textbf{sîme wirte} sprach:\\ 
 & "bin ich dîn mâc, daz ist ouch er;\\ 
 & des sî Gahmuret dîn \textbf{wer}.\\ 
 & \textbf{diz} ist der künec von Zazamanch.\\ 
 & mîn vater dort mit prîse erranc\\ 
15 & Belakanen, diu disen rîter truoc."\\ 
 & Gawan den heiden \textbf{dô} genuoc\\ 
 & kuste. \textbf{der rîche} Feirefiz\\ 
 & was beidiu swarz und wîz\\ 
 & über al sîn vel, wan \textbf{daz} der munt\\ 
20 & gein \textbf{halbem} \textbf{zil} tet rœte kunt.\\ 
 & Man brâht in \textbf{beidesamt} gewant,\\ 
 & daz was vür tiwer kost erkant.\\ 
 & ûz Gawans kamern truoc man\textbf{z} dar.\\ 
 & \textbf{dô} kômen vrouwen lieht gevar.\\ 
25 & diu herzogîn liez Cundrie\\ 
 & unt Sangiven küssen ê.\\ 
 & si selbe unt Arnive \textbf{in} dô\\ 
 & kusten. Feirefiz was vrô,\\ 
 & \textbf{daz} er sô clâre vrouwen sach.\\ 
30 & ich wæne im liebe dran geschach.\\ 
\end{tabular}
\scriptsize
\line(1,0){75} \newline
D Fr12 \newline
\line(1,0){75} \newline
\textbf{1} \textit{Initiale} D  \textbf{21} \textit{Majuskel} D  \newline
\line(1,0){75} \newline
\textbf{2} dô] di Fr12 \textbf{5} Feirefiz] Feyrefiz Fr12 \textbf{6} Parzival] Parcifal D (Fr12) \textbf{10} Parcifal im sagte vnde sprach::: Fr12  $\cdot$ Parzival] Parcival D \textbf{12} Gahmuret] Gahmvret D Gamvret Fr12 \textbf{16} \textit{nach 758.16:} daz des Parcifal gewc Fr12  \textbf{17} Feirefiz] feyrefiz Fr12 \textbf{21} beidesamt] beiden Fr12 \textbf{22} vür tiwer] von tvre Fr12 \textbf{23} kamern] kamer Fr12  $\cdot$ truoc] brahte Fr12 \textbf{27} in] \textit{om.} Fr12 \textbf{28} Feirefiz] Feyrefiz Fr12 \newline
\end{minipage}
\hspace{0.5cm}
\begin{minipage}[t]{0.5\linewidth}
\small
\begin{center}*m
\end{center}
\begin{tabular}{rl}
 & \textbf{\begin{large}D\end{large}ô} daz harnasch \textbf{wart} von \textbf{in} getân,\\ 
 & dô schouweten disen \textbf{b\textit{unt}en} man\\ 
 & alle, die \textbf{wunders} k\textit{u}n\textit{d}en jehen,\\ 
 & \textbf{die} mohten\textbf{z} d\textit{â} mit wârheit \textbf{spehen}:\\ 
5 & Ferefiz truoc vremdiu mâl.\\ 
 & Gawan sprach zuo Parcifal:\\ 
 & "ne\textit{ve}, tuo den gesellen dîn\\ 
 & mir kunt. er treit sô wæhen schîn,\\ 
 & dem ich \textbf{glîches} nie \textit{gesa}ch."\\ 
10 & Parc\textit{i}fal ze \textbf{Gawane} sprach:\\ 
 & "bin ich dîn mâc, daz ist ouch er;\\ 
 & des sî Gahmuret dîn \textbf{wer}.\\ 
 & \textbf{daz} ist der künic von Zazamanc.\\ 
 & mîn vater dort mit prîse \textit{er}ranc\\ 
15 & Belakanen, diu disen ritter truoc."\\ 
 & Gawan den heiden \textbf{dô} genuoc\\ 
 & kuste, \textbf{den rîchen} Ferefiz\\ 
 & was beidiu swarz und wîz\\ 
 & über al sîn vel, wan \textbf{daz} der munt\\ 
20 & gegen \textbf{halbem} \textbf{zil} tet r\textit{œ}te kunt.\\ 
 & \textit{man brâht in} \textbf{\textit{beiden sam}en\textit{t}} \textit{gewant,}\\ 
 & \textit{daz was vür tiure koste erkant.}\\ 
 & ûz Gawanes kamer truoc man\textbf{z} dar.\\ 
 & \textbf{dô} kômen vrowen lieht gevar.\\ 
25 & diu herzogîn liez Condrie\\ 
 & und Sangiven küssen ê.\\ 
 & si selbe und Ar\textit{niv}e \textbf{in} d\textit{ô}\\ 
 & kusten. Ferefiz was vrô,\\ 
 & \textbf{dô} er sô clâre vrowen sach.\\ 
30 & ich wæne im lie\textit{b}e dâr an geschach.\\ 
\end{tabular}
\scriptsize
\line(1,0){75} \newline
m n o V V' \newline
\line(1,0){75} \newline
\textbf{1} \textit{Initiale} m V V'   $\cdot$ \textit{Capitulumzeichen} n  \newline
\line(1,0){75} \newline
\textbf{1} in] im V (V') \textbf{2} bunten] pincken m [*]: gepunten V gebunten V' \textbf{3} kunden] kuͯnnen m \textbf{4} mohtenz] moͯchtens n moͤhtent V mochten V'  $\cdot$ dâ] do m n o V V' \textbf{5} Ferefiz] Ferefis m n Ferrefis o Ferevis V V'  $\cdot$ truoc] truͦg do V (V') \textbf{6} Gawan] Gawin V'  $\cdot$ Parcifal] parzefal V parzifal V' \textbf{7} neve tuo] Nem du m n o  $\cdot$ den] \textit{om.} V' \textbf{9} ich] iches o  $\cdot$ gesach] verich m \textbf{10} Parcifal] Parcefal m Parzefal V Parzifal V'  $\cdot$ Gawane] gawanen o \textbf{12} Gahmuret] gamiret n gahmueret o Gameret V (V') \textbf{13} daz] Diz V V'  $\cdot$ Zazamanc] zazamang m n V V' [zamang]: zazamang o \textbf{14} prîse erranc] prise rang m prisz er rang n \textbf{15} Belakanen] Bellakanen m Belekanen n [Bel*]: Belakanen o \textbf{17} den rîchen] der riche V [*]: der riche V'  $\cdot$ Ferefiz] Ferefis m n o ferevis V [*vis]: ferevis V' \textbf{19} al] alle o \textbf{20} halbem] halbe o halbene V'  $\cdot$ rœte] ritte m hurte o \textbf{21} \textit{Die Verse 758.21-22 fehlen} m   $\cdot$ \textit{Versfolge 758.22-21} n   $\cdot$ beiden sament] beiden samit n beide sament o \textbf{23} ûz] Von V'  $\cdot$ Gawanes] gawans n o (V) V'  $\cdot$ kamer] kamern V'  $\cdot$ manz] man V' \textbf{24} kômen] koment V V' \textbf{25} Condrie] Cundrie o kvndrie V V' \textbf{26} Sangiven] sangwen n o sagiven V V' \textbf{27} Arnive] arune m arniwe n o  $\cdot$ dô] da m n o \textbf{28} Ferefiz] ferefis m o ferrefis n Ferevis V fereuis V' \textbf{30} liebe] lielde m liep n \newline
\end{minipage}
\end{table}
\newpage
\begin{table}[ht]
\begin{minipage}[t]{0.5\linewidth}
\small
\begin{center}*G
\end{center}
\begin{tabular}{rl}
 & daz harnasch \textbf{was} von \textbf{im} getân.\\ 
 & dô schouweten disen man\\ 
 & alle, die \textbf{werdes} kunden jehen,\\ 
 & \textbf{daz} mohten\textbf{s} dâ mit wârheit \textbf{sehen}:\\ 
5 & Feirafiz truoc vremdiu mâl.\\ 
 & Gawan sprach ze Parzival:\\ 
 & "neve, tuo den gesellen dîn\\ 
 & mir kunt. er treit sô wæhen schîn,\\ 
 & dem ich \textbf{glîches} nie gesach."\\ 
10 & Parzival ze \textbf{sînem wirte} sprach:\\ 
 & "bin ich dîn mâc, daz ist ouch er;\\ 
 & des sî Gahmuret dîn \textbf{gewer}.\\ 
 & \textbf{ditze} ist der künic von Zazamanc.\\ 
 & mîn vater dort mit prîse erranc\\ 
15 & Belacanen, diu disen rîter truoc."\\ 
 & Gawan den heiden \textbf{dô} genuoc\\ 
 & kuste, \textbf{den rîchen} Feirafiz.\\ 
 & \textbf{er} was beidiu swarz unde wîz\\ 
 & über al sî\textit{n} vel, wan \textbf{dâ} der munt\\ 
20 & gein \textbf{blanken} \textbf{teil} tet rœte kunt.\\ 
 & man brâhte in \textbf{\textit{beiden} sament} gewant,\\ 
 & daz was vür tiure kost erkant,\\ 
 & ûz Gawans kamer truoc man dar.\\ 
 & \textbf{dô} kômen vrouwen lieht gevar.\\ 
25 & diu herzogîn liez Gundrie\\ 
 & unde Sagiven küssen ê.\\ 
 & si selbe unde Arnive dô\\ 
 & kusten. Feirafiz was vrô,\\ 
 & \textbf{daz} er sô clâre vrouwen sach.\\ 
30 & ich wæne im liebe dran geschach.\\ 
\end{tabular}
\scriptsize
\line(1,0){75} \newline
G I L M Z Fr43 \newline
\line(1,0){75} \newline
\textbf{1} \textit{Initiale} L Z Fr43  \textbf{13} \textit{Initiale} I  \newline
\line(1,0){75} \newline
\textbf{1} was] wart I \textbf{2} dô] Da M Z Fr43  $\cdot$ schouweten disen] schovwete man dise L  $\cdot$ man] bvnten man Z \textbf{3} werdes] werden L wunder Z  $\cdot$ jehen] spehen Z \textbf{4} daz mohtens] Daz mohten L Die mohtens Z  $\cdot$ sehen] iehen Z \textbf{5} Feirafiz] Ferefiz L Fr43 Feirefisz M Feirefiz Z  $\cdot$ vremdiu] frevde Z \textbf{6} Parzival] parcifal G Z Parzifal I (L) (M) \textbf{8} sô] \textit{om.} M  $\cdot$ wæhen schîn] [hohen pin]: wehen shin I werden schin L \textbf{9} dem] Des Z  $\cdot$ glîches] geliche L  $\cdot$ nie] nie niht L  $\cdot$ gesach] gischach M \textbf{10} Parzival] parcifal G (Z) Parzifal I L M Parzifa: Fr43  $\cdot$ sînem wirte] Gawan I \textbf{12} des] Das M  $\cdot$ Gahmuret] Gahmv̂ret G Gamuͯret M gamuret Z G::: Fr43  $\cdot$ gewer] wer L \textbf{13} Zazamanc] Zazamanch L \textbf{14} mîn] Mit Z \textbf{15} Belacanen] belcanen G belicanen I M Z Belaca::: Fr43 \textbf{16} dô] da M Z \textbf{17} kuste] Kuͯst L  $\cdot$ den rîchen] der riche L (M)  $\cdot$ Feirafiz] firafiz G ferefiz L feirefisz M feirefiz Z \textbf{18} er was] Waz L (Fr43) Wan er was Z  $\cdot$ beidiu] [bei*de]: beide L \textbf{19} sîn] si G  $\cdot$ wan] \textit{om.} L  $\cdot$ dâ] do L das M (Z) \textbf{20} blanken teil] blaiche I blanchen teile L [*]: halbe teil M halbem zil Z h::: Fr43 \textbf{21} beiden sament] sament G \textbf{23} Gawans] Gawanz L Gaw::: Fr43  $\cdot$ man] manz I L Z  $\cdot$ dar] daz Z \textbf{24} dô] Da M Z Fr43  $\cdot$ lieht] lýcht L (M) \textbf{25} liez Gundrie] lýe kvͯndrie L ligundrie M liez kvndrie Z \textbf{26} Sagiven] sagíven G Saifen I Segiven L saiven M Seyven Z Sen::: Fr43  $\cdot$ küssen] [*]: kusten M \textbf{27} selbe] selbin M  $\cdot$ Arnive] arniue I  $\cdot$ dô] da M \textbf{28} kusten] Jn kusten Z  $\cdot$ Feirafiz] Ferefiz L ferefisz M feirefiz Z \newline
\end{minipage}
\hspace{0.5cm}
\begin{minipage}[t]{0.5\linewidth}
\small
\begin{center}*T
\end{center}
\begin{tabular}{rl}
 & der harnasch \textbf{was} von \textbf{im} getân.\\ 
 & dô schouweten disen \textbf{bunten} man\\ 
 & alle, die \textbf{wunders} kunden jehen,\\ 
 & \textbf{die} mohten d\textit{â} mit wârheit \textbf{sehen}:\\ 
5 & Ferefis truoc vremdiu mâl.\\ 
 & Gawan sprach zuo Parcifal:\\ 
 & "neve, tuo den gesellen dîn\\ 
 & mir kunt. er treit sô wæhen schîn,\\ 
 & dem ich \textbf{glîches} nie gesach."\\ 
10 & Parcifal zuo \textbf{sîme wirte} sprach:\\ 
 & "bin ich dîn mâc, daz ist ouch er;\\ 
 & des sî Gahmuret dîn \textbf{wer}.\\ 
 & \textbf{diz} ist der künec von Zazamanc.\\ 
 & mîn vater dort mit prîse erranc\\ 
15 & Belacanen, diu disen rîter truoc."\\ 
 & Gawan den heiden genuoc\\ 
 & kuste. \textbf{der rîche} Ferefis\\ 
 & was beidiu swarz und wîz\\ 
 & über al sîn vel, wan \textbf{daz} der munt\\ 
20 & gein \textbf{halbem} \textbf{teile} tet rœte kunt.\\ 
 & man brâht in \textbf{beiden samt} gewant,\\ 
 & daz was vür tiure kost erkant,\\ 
 & ûz Gawans kamer truoc man dar.\\ 
 & \textbf{dar} kômen vrouwen lieht gevar.\\ 
25 & diu herzogîn liez Kundrie\\ 
 & und Seyven küssen ê.\\ 
 & si selbe und Arnyve dô\\ 
 & kusten. Ferefis was vrô,\\ 
 & \textbf{daz} er sô clâre vrouwen sach.\\ 
30 & ich wæne im liebe dran geschach.\\ 
\end{tabular}
\scriptsize
\line(1,0){75} \newline
U W Q R \newline
\line(1,0){75} \newline
\textbf{1} \textit{Capitulumzeichen} R  \textbf{25} \textit{Initiale} W  \newline
\line(1,0){75} \newline
\textbf{1} der] Das W Q R  $\cdot$ was] ward R \textbf{2} schouweten] schawtens Q  $\cdot$ bunten] puncten W \textbf{4} mohten] mochtens W Q (R)  $\cdot$ dâ] do U Q  $\cdot$ sehen] spehen Q (R) \textbf{5} Ferefis] Ferafis W Feirefisz Q Feriefis R \textbf{6} Gawan] Gawin R  $\cdot$ Parcifal] Parzifal U partzifal W Q parczifal R \textbf{7} tuo] tun Q tuͯ mir R \textbf{8} mir kunt] Mir kun Q Kunt R  $\cdot$ wæhen] fechen R \textbf{9} glîches] gelicher R \textbf{10} Parcifal] Partzifal W Q Parczifal R \textbf{11} mâc] neue R  $\cdot$ er] ere Q \textbf{12} Gahmuret] Gahmuͦret U gamuret W gamúret Q  $\cdot$ wer] gewere Q (R) \textbf{13} Zazamanc] zazamang W zasmanc R \textbf{15} Belacanen] Belakanen W \textbf{16} Gawan] Gan W Gawin R  $\cdot$ genuoc] do gewuͦg W do genuck Q (R) \textbf{17} Ferefis] ferafis W feirefisz Q feirefis R \textbf{19} wan] was R \textbf{20} halbem teile] halbe teit R \textbf{22} erkant] genant Q \textbf{23} Gawans] Gawins R  $\cdot$ man] man es Q \textbf{24} dar] Do Q R  $\cdot$ vrouwen lieht] frawenliche Q \textbf{25} Kundrie] kuͦndrie U kúndrie Q kundrye R \textbf{26} Seyven] seynen U seiuen W seyren Q Seyuen R \textbf{27} selbe] selber W Q  $\cdot$ Arnyve] arnyue W arniue Q [arniue]: arnyue R \textbf{28} kusten] kusten in Q  $\cdot$ Ferefis] ferafis W feirefisz Q feiresen R \newline
\end{minipage}
\end{table}
\end{document}
