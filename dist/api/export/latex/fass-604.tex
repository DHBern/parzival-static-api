\documentclass[8pt,a4paper,notitlepage]{article}
\usepackage{fullpage}
\usepackage{ulem}
\usepackage{xltxtra}
\usepackage{datetime}
\renewcommand{\dateseparator}{.}
\dmyyyydate
\usepackage{fancyhdr}
\usepackage{ifthen}
\pagestyle{fancy}
\fancyhf{}
\renewcommand{\headrulewidth}{0pt}
\fancyfoot[L]{\ifthenelse{\value{page}=1}{\today, \currenttime{} Uhr}{}}
\begin{document}
\begin{table}[ht]
\begin{minipage}[t]{0.5\linewidth}
\small
\begin{center}*D
\end{center}
\begin{tabular}{rl}
\textbf{604} & \textit{\begin{large}D\end{large}}az wazzer hiez Sabbins.\\ 
 & Gawan holt \textbf{unsenften} zins,\\ 
 & \textbf{dô} er unz ors drîn bleste.\\ 
 & swie Orgeluse gleste,\\ 
5 & ich wolt ir minne alsô niht nemen.\\ 
 & ich weiz wol, wes mich sol gezemen.\\ 
 & Dô Gawan daz rîs gebrach\\ 
 & unt der kranz \textbf{wart} \textbf{des} \textbf{helmes} dach,\\ 
 & ez reit zuo \textbf{z}im ein rîter clâr.\\ 
10 & dem wâren sîner zîte jâr\\ 
 & weder ze kurz noch ze lanc.\\ 
 & sîn muot \textbf{durch hôchvart in} \textbf{twanc},\\ 
 & swie vil \textbf{im ein man} tet leit,\\ 
 & daz er doch \textbf{mit dem niht} streit,\\ 
15 & ir \textbf{en}wæren zwêne oder mêr.\\ 
 & sîn hôhez herze was sô hêr,\\ 
 & swaz im \textbf{tet} ein man,\\ 
 & \textbf{den} wolt er \textbf{âne strît} \textbf{doch} lân.\\ 
 & Fillu roy \textbf{Irot}\\ 
20 & Gawan guoten morgen bôt.\\ 
 & daz was der künec Gramoflanz.\\ 
 & \textbf{Dô} sprach \textbf{er}: "hêrre, umbe disen kranz\\ 
 & hân ich \textbf{doch} niht gar verzigen.\\ 
 & mîn \textbf{grüezen} wære \textbf{noch gar} \textbf{verswigen},\\ 
25 & ob iwer zwêne wæren.\\ 
 & \textbf{die daz} niht verbæren,\\ 
 & si\textbf{ne} holten hie durch hôhen prîs\\ 
 & \textbf{ab} mîme boume \textbf{alsus} ein rîs,\\ 
 & die müesen strît enpfâhen;\\ 
30 & daz sol mir sus versmâhen."\\ 
\end{tabular}
\scriptsize
\line(1,0){75} \newline
D Z \newline
\line(1,0){75} \newline
\textbf{1} \textit{Initiale} D Z  \textbf{7} \textit{Majuskel} D  \textbf{19} \textit{Majuskel} D  \textbf{22} \textit{Majuskel} D  \newline
\line(1,0){75} \newline
\textbf{1} Daz] ÷az D  $\cdot$ Sabbins] Sabins D Z \textbf{3} dô] Da Z \textbf{5} wolt] en wolde Z \textbf{7} Dô] Da Z \textbf{8} des] sin Z \textbf{9} zim] im Z \textbf{18} doch lân] lat Z \textbf{19} Irot] Jrot D Gyrot Z \textbf{20} guoten] gute Z \textbf{21} Gramoflanz] Gramoͮlanz D Gramoflantz Z \textbf{22} hêrre umbe] \textit{om.} Z \textbf{24} grüezen wære noch] gruz wer ev Z \textbf{26} daz] des Z \textbf{28} mîme] minen Z \newline
\end{minipage}
\hspace{0.5cm}
\begin{minipage}[t]{0.5\linewidth}
\small
\begin{center}*m
\end{center}
\begin{tabular}{rl}
 & daz wazzer, \textbf{daz} hiez Sabbins.\\ 
 & Gawan holte \textbf{unsanften} zins,\\ 
 & \textbf{dô} er und daz ros d\textit{ar î}n bleste.\\ 
 & wie Urgeluse \textbf{den} gleste,\\ 
5 & ich wolt ir minne alsô niht nemen.\\ 
 & ich weiz wol, wes mich sol gezemen.\\ 
 & \begin{large}D\end{large}ô Gawan daz rîs gebrach\\ 
 & und \textit{d}er kranz \textbf{was} \textbf{sînes} \textbf{helmes} dach,\\ 
 & ez reit zuo im ein ritter clâr.\\ 
10 & dem wâren sîner zîte jâr\\ 
 & weder zuo kurz noch zuo lanc.\\ 
 & sîn muot \textbf{durch hôchvart in} \textbf{betwanc},\\ 
 & wie vil \textbf{ein man im} tete leit,\\ 
 & daz er doch \textbf{mit dem niht} streit,\\ 
15 & ir wæren zwêne oder mêr.\\ 
 & sîn hôhez herz was sô hêr,\\ 
 & waz ime \textbf{getet} ein man,\\ 
 & \textbf{den} wolt er \textbf{âne strît} \textbf{doch} lân.\\ 
 & fily rois \textbf{Irot}\\ 
20 & Gawane guoten morgen bôt.\\ 
 & daz was der künic Gr\textit{a}m\textit{o}lanz.\\ 
 & \textbf{der} sprach: "hêrre, umb disen kranz\\ 
 & hân ich \textbf{iu} \textbf{doch} niht gar verzigen.\\ 
 & mîn \textbf{gruoz} wær \textbf{noch gar} \textbf{un\textit{v}erswigen},\\ 
25 & ob iuwer zwêne wæren.\\ 
 & \textbf{die daz} niht verbæren,\\ 
 & si holten hie durch hôhen prîs\\ 
 & \textbf{ab} mînem boum \textbf{alsus} ein rîs,\\ 
 & die müesten strît enpfâhen;\\ 
30 & daz sol mir sus versmâhen."\\ 
\end{tabular}
\scriptsize
\line(1,0){75} \newline
m n o \newline
\line(1,0){75} \newline
\textbf{7} \textit{Initiale} m   $\cdot$ \textit{Capitulumzeichen} n  \newline
\line(1,0){75} \newline
\textbf{1} Sabbins] sabins n \textbf{2} unsanften] vnsanffte n \textbf{3} dar în] den m \textbf{4} Urgeluse] vrgelúse o  $\cdot$ den] \textit{om.} n o \textbf{5} niht] [nie]: nit m \textbf{6} wes] wasz o \textbf{7} daz] des o  $\cdot$ rîs] [rich]: risz n \textbf{8} der] er m  $\cdot$ was] wart n o \textbf{19} rois] roit o  $\cdot$ Irot] jrot m (o) \textbf{20} Gawane] Gawanen o \textbf{21} Gramolanz] gromonlancz m gramonlantz n gramalancz o \textbf{24} gar] \textit{om.} n  $\cdot$ unverswigen] vnerswigen m \textbf{26} die] Dis o \textbf{28} alsus] [sus]: alsus m \newline
\end{minipage}
\end{table}
\newpage
\begin{table}[ht]
\begin{minipage}[t]{0.5\linewidth}
\small
\begin{center}*G
\end{center}
\begin{tabular}{rl}
 & \begin{large}D\end{large}az wazzer hiez Sabins.\\ 
 & Gawan holt \textbf{iedoch den} zins,\\ 
 & \textbf{dô} er unt daz ors d\textit{rîn} bleste.\\ 
 & swie Orgeluse gleste,\\ 
5 & ich\textbf{ne} wolt ir minne alsô niht nemen.\\ 
 & ich weiz wol, wes mich sol gezemen.\\ 
 & dô Gawan daz rîs gebrach\\ 
 & und der kranz \textbf{wart} \textbf{sînes} \textbf{helmes} dach,\\ 
 & ez reit zuo im ein rîter clâr.\\ 
10 & deme wâren sîner zîte jâr\\ 
 & weder ze kurz noch ze lanc.\\ 
 & sîn muot \textbf{in durch hôchvart} \textbf{betwanc},\\ 
 & swie vil \textbf{im ein man} tet leit,\\ 
 & daz er doch \textbf{mit dem niht} \textit{streit},\\ 
15 & ir\textbf{n} wæren zwêne oder mêr.\\ 
 & sîn hôhez herze was sô hêr,\\ 
 & swaz im \textbf{tet} ein man,\\ 
 & \textbf{daz} wolde er \textbf{âne strîten} lân.\\ 
 & fil roys \textbf{Gyrot}\\ 
20 & Gawa\textit{n} guote\textit{n} morgen bôt.\\ 
 & daz was der künic Gramoflanz.\\ 
 & \textbf{dô} sprach \textbf{er}: "hêrre, umb dise\textit{n} kranz\\ 
 & hân ich \textbf{iu} niht gar ver\textit{z}igen.\\ 
 & mîn \textbf{gruoz} wære \textbf{iuch} \textbf{doch} \textbf{verswigen},\\ 
25 & ob iuwer zw\textit{ê}ne wæren.\\ 
 & \multicolumn{1}{l}{ - - - }\\ 
 & si\textbf{ne} holte\textit{n} hie durch hôhen prîs\\ 
 & \textbf{abe} mînem boume \textbf{alsus} ein rîs,\\ 
 & die müesen strît enpfâhen;\\ 
30 & daz sol mir sus versmâhen."\\ 
\end{tabular}
\scriptsize
\line(1,0){75} \newline
G I L M Z Fr51 \newline
\line(1,0){75} \newline
\textbf{1} \textit{Initiale} G I L Z Fr51  \textbf{19} \textit{Initiale} I  \newline
\line(1,0){75} \newline
\textbf{1} hiez] heiz Fr51  $\cdot$ Sabins] Sabinsz L sa::: Fr51 \textbf{2} holt iedoch den] vnsanfte holt den L holte vnsanfte den M holt vnsanften Z halte vns::: Fr51 \textbf{3} dô] \textit{om.} M Da Z  $\cdot$ drîn] dem G  $\cdot$ bleste] platste L \textbf{4} swie] swie wol I Wie L (M) (Fr51)  $\cdot$ Orgeluse] Orgelise L orgelusen Fr51  $\cdot$ gleste] nu Gleste I \textbf{5} ichne] ich I (L)  $\cdot$ alsô niht] nicht so M so niht Fr51  $\cdot$ nemen] hol I \textbf{6} wes] waz I (Fr51)  $\cdot$ sol gezemen] gezemen sol I \textbf{7} dô] Da M Z  $\cdot$ gebrach] brach Fr51 \textbf{8} wart] was M  $\cdot$ sînes] sin Z \textbf{9} ez] do I (Fr51) \textbf{11} ze lanc] [zespæte]: zelanch G \textbf{12} in durch] durc I (Z)  $\cdot$ betwanc] twanch L (M) intwanc Z \textbf{13} swie] Wie L (M) (Fr51)  $\cdot$ vil] \textit{om.} L  $\cdot$ tet] getete I tate L \textbf{14} mit dem] mit im I \textit{om.} Fr51  $\cdot$ streit] enstreit G \textbf{15} irn] iener I  $\cdot$ wæren] warn L \textbf{17} swaz] Waz L (M) \textbf{18} daz] Dem L Den M Z (Fr51)  $\cdot$ wolde] volget L  $\cdot$ strîten] strit I Z Fr51  $\cdot$ lân] lat Z \textbf{19} Gyrot] chẏrot G kirot I girot M Fr51 \textbf{20} Gawan] gawanen G  $\cdot$ guoten] guͦtem G gute Z \textbf{21} Gramoflanz] grimoflanz G gramorflanz M Gramoflantz Z gramoflans Fr51 \textbf{22} dô] Da M  $\cdot$ er] \textit{om.} Fr51  $\cdot$ hêrre umb] \textit{om.} Z  $\cdot$ disen] dise G \textbf{23} iu] doch Z an iv Fr51  $\cdot$ verzigen] verligen G \textbf{24} doch] gar L M Z Fr51 \textbf{25} ob] daz I  $\cdot$ zwêne] zw:ne G  $\cdot$ wæren] waren L Fr51 \textbf{26} \textit{Vers 604.26 fehlt} G   $\cdot$ alle in den selben gebern I  $\cdot$ Die daz (des Z ) niht verbaren (verberen M ) L (M) (Z)  $\cdot$ ::: woltich uwer varen Fr51 \textbf{27} daz sie wolten holn durc ir pris I  $\cdot$ holten] holte G  $\cdot$ hie] nicht Fr51 \textbf{28} abe] :::n Fr51  $\cdot$ mînem] minen Z Fr51  $\cdot$ alsus] nicht Fr51 \textbf{29} müesen] musten I \textbf{30} daz sol] :::as Fr51 \newline
\end{minipage}
\hspace{0.5cm}
\begin{minipage}[t]{0.5\linewidth}
\small
\begin{center}*T
\end{center}
\begin{tabular}{rl}
 & \begin{large}D\end{large}az wazzer hiez Sabins.\\ 
 & Gawan holte \textbf{unsenften} zins,\\ 
 & \textbf{daz} er und daz ors dar în bleste.\\ 
 & wie Orgeluse gleste,\\ 
5 & ich wolte ir minne alsô niht nemen.\\ 
 & ich weiz wol, wes mich sol gezemen.\\ 
 & dô Gawan daz rîs gebrach\\ 
 & und der kranz \textbf{wart} \textbf{sînes} \textbf{houbetes} dach,\\ 
 & ez reit zuo im ein rîter clâr.\\ 
10 & dem wâren sîner zîte jâr\\ 
 & weder zuo kurz noch zuo lanc.\\ 
 & sîn muot \textbf{durch hôchvart in} \textbf{twanc},\\ 
 & wie vil \textbf{im ein man} te\textit{t} leit,\\ 
 & daz er doch \textbf{niht mit dem} streit,\\ 
15 & ir \textbf{en}wæren zwêne oder mêr.\\ 
 & sîn hôhez herze was sô hêr,\\ 
 & waz im \textbf{tet} ein man,\\ 
 & \textbf{daz} wolt er \textbf{ungerochen} lân.\\ 
 & fillyrois \textbf{Irot}\\ 
20 & Gawane guoten morgen bôt.\\ 
 & daz was der künec Gramoflanz.\\ 
 & \textbf{dô} sprach \textbf{er}: "hêrre, umb disen kranz\\ 
 & hân ich \textbf{iu} niht gar verzigen.\\ 
 & mîn \textbf{gruoz} wære \textbf{iuch} \textbf{gar} \textbf{verswigen},\\ 
25 & o\textit{b} iuwer zwêne wæren,\\ 
 & \textbf{daz die} niht verbæren,\\ 
 & si holten hie durch hôhen prîs\\ 
 & \textbf{von} mîme boume \textbf{sus} ein rîs,\\ 
 & die m\textit{ües}en strît enpfâhen;\\ 
30 & daz sol mir sus versmâhen."\\ 
\end{tabular}
\scriptsize
\line(1,0){75} \newline
U V W Q R \newline
\line(1,0){75} \newline
\textbf{1} \textit{Initiale} U V W   $\cdot$ \textit{Capitulumzeichen} R  \textbf{22} \textit{Überschrift:} Hie komp der kung Gramoflancz zu Gawin gritten do er den crancz gebrochen hett vnd hie die herczogin ennent wasser vnd luͯg zuͯ R  \newline
\line(1,0){75} \newline
\textbf{1} hiez] hiez der V  $\cdot$ Sabins] sabincz R \textbf{2} Gawan holte] Gawin holt R  $\cdot$ unsenften] vnsanfte Q (R) \textbf{3} daz] [Da*]: Da V Do W Q R  $\cdot$ bleste] bleczste R \textbf{4} wie] Swie V  $\cdot$ Orgeluse] orguluse R  $\cdot$ gleste] [*]: sere glaste V \textbf{5} wolte] wel W \textbf{6} wes] wie ez V  $\cdot$ sol] so R \textbf{7} dô Gawan] Das Gawin R \textbf{8} kranz] helm W  $\cdot$ houbetes] [*]: helmez V helms Q (R) \textbf{9} Er rait zuͦ einem ritter klar W \textbf{10} zîte] geburte R \textbf{12} Sin mvͦt durch hochvart [*]: in dez twang V  $\cdot$ Sin muͦt in durch hoffart zwang R \textbf{13} wie] Swie V  $\cdot$ im ein man tet] im ein man der U ein man im thet W (Q) \textbf{14} niht mit dem] mit dem niht V (R) mit dem W \textbf{15} enwæren] weren W R \textbf{17} waz] Swaz V  $\cdot$ ein] ein einzig V \textbf{19} fillyrois] Dez kv́niges svn V  $\cdot$ Irot] yrot U V W ytot Q \textbf{20} Gawane] Gawanen Q Gwainen R \textbf{21} Gramoflanz] gramaflanz V gramoflantz W Q Gramoflancz R \textbf{25} ob] Oder U \textbf{26} daz die] Die das W Q  $\cdot$ niht] das R \textbf{27} si] Seine Q  $\cdot$ holten] enholten V W (R)  $\cdot$ hôhen] \textit{om.} Q \textbf{28} von] Abe V (W) (Q) R  $\cdot$ sus] als Q \textbf{29} müesen] muͦzen U \textbf{30} mir] mich W  $\cdot$ sus] als Q \newline
\end{minipage}
\end{table}
\end{document}
