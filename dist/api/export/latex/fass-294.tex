\documentclass[8pt,a4paper,notitlepage]{article}
\usepackage{fullpage}
\usepackage{ulem}
\usepackage{xltxtra}
\usepackage{datetime}
\renewcommand{\dateseparator}{.}
\dmyyyydate
\usepackage{fancyhdr}
\usepackage{ifthen}
\pagestyle{fancy}
\fancyhf{}
\renewcommand{\headrulewidth}{0pt}
\fancyfoot[L]{\ifthenelse{\value{page}=1}{\today, \currenttime{} Uhr}{}}
\begin{document}
\begin{table}[ht]
\begin{minipage}[t]{0.5\linewidth}
\small
\begin{center}*D
\end{center}
\begin{tabular}{rl}
\textbf{294} & \begin{large}D\end{large}az ir den künec gelastert hât,\\ 
 & welt ir \textbf{mir volgen}, sô ist mîn rât,\\ 
 & unt \textbf{dunket} mich iwer bestez heil:\\ 
 & nemt iuch \textbf{selben} an ein bracken seil\\ 
5 & unt lât iuch vür in ziehen.\\ 
 & ir \textbf{en}megt mir niht enpfliehen,\\ 
 & ich bringe iuch doch \textbf{betwungen} dar,\\ 
 & sô nimt man iwer \textbf{unsanfte} war."\\ 
 & Den Waleis twanc der minnen kraft\\ 
10 & \textbf{swîgens}. Keie sînen schaft\\ 
 & ûf zôch unt vrumt im einen swanc\\ 
 & \textbf{anz houbt}, daz der helm erklanc.\\ 
 & \textbf{dô sprach er}: "dû muost wachen.\\ 
 & âne lînlachen\\ 
15 & wirt dir dîn \textbf{slâfen} hie benant.\\ 
 & ez zilt al anders \textbf{hie} mîn hant:\\ 
 & ûf den snê \textbf{dû wirst} geleit.\\ 
 & der den sac \textbf{von} der mül treit,\\ 
 & wolte man in \textbf{sô} bliwen,\\ 
20 & in m\textit{ö}hte lazheit riwen."\\ 
 & Vrou minne, hie seht ir zuo.\\ 
 & ich wæne, man\textbf{z iu} ze laster tuo,\\ 
 & wan ein gebûr spræche sân:\\ 
 & "mîme hêrren sî \textbf{diz} getân."\\ 
25 & \textbf{er klaget} \textbf{ouch}, m\textit{ö}ht \textbf{er} sprechen.\\ 
 & vrou minne, lât sich rechen\\ 
 & den werden Waleise,\\ 
 & wan liez in iwer \textbf{vreise}\\ 
 & unt iwer strenge, unsüezer last,\\ 
30 & ich wæne, sich werte dirre gast.\\ 
\end{tabular}
\scriptsize
\line(1,0){75} \newline
D \newline
\line(1,0){75} \newline
\textbf{1} \textit{Initiale} D  \textbf{9} \textit{Majuskel} D  \textbf{21} \textit{Majuskel} D  \newline
\line(1,0){75} \newline
\textbf{20} möhte] mohte D \textbf{25} möht] moht D \newline
\end{minipage}
\hspace{0.5cm}
\begin{minipage}[t]{0.5\linewidth}
\small
\begin{center}*m
\end{center}
\begin{tabular}{rl}
 & da\textit{z} ir den küni\textit{c g}elastert hât,\\ 
 & wellet ir \textbf{mir volgen}, sô ist mîn rât,\\ 
 & und \textbf{dunket} mich iuwer bestez heil:\\ 
 & nemt iuch \textbf{selben} an ein bracken seil\\ 
5 & und lât iuch vür in ziehen.\\ 
 & ir \textbf{en}muget mir niht enpfliehen,\\ 
 & ich \textbf{en}bringe iuch doch \textbf{betwungen} dar,\\ 
 & sô nimt man iuwer \textbf{sanfte} war."\\ 
 & \begin{large}D\end{large}en Waleis twanc der minnen kraft\\ 
10 & \textbf{swîgens}. Keie sînen schaft\\ 
 & ûf zôch und vrumti\textit{m} einen swanc\\ 
 & \textbf{a\textit{n}z houbet}, daz der helm erklanc.\\ 
 & \textbf{er sprach}: "dû muost wachen.\\ 
 & âne lînlachen\\ 
15 & wirt dir dîn \textbf{slâfen} hie benant.\\ 
 & ez zilt al anders \textit{mîn} hant:\\ 
 & ûf den snê \textbf{dû wirst} geleit.\\ 
 & der den sac \textbf{zuo} der mülen tr\textit{eit},\\ 
 & wolte man i\textit{n} \textbf{\textit{a}lsô} bl\textit{i}u\textit{w}en,\\ 
20 & in m\textit{ö}hte lazheit riuwen."\\ 
 & vrouwe minne, hi\textit{e s}eht ir zuo.\\ 
 & ich wæne, man \textbf{\textit{ez} iu} ze laster tuo,\\ 
 & wanne ein gebûre, \textbf{der} spræche sân:\\ 
 & "mînem hêrren sî \textbf{diz} getân."\\ 
25 & \textbf{er klagete}, m\textit{ö}hte\textbf{r} sprechen.\\ 
 & vrouwe minne, lât sich rechen\\ 
 & den werden Waleise,\\ 
 & wan lieze in iuwer \textbf{reise}\\ 
 & und iuwer strenge, unsüezer la\textit{s}t,\\ 
30 & ich wæne, sich werte dirre gast.\\ 
\end{tabular}
\scriptsize
\line(1,0){75} \newline
m n o \newline
\line(1,0){75} \newline
\textbf{9} \textit{Initiale} m  \newline
\line(1,0){75} \newline
\textbf{1} Da er den kunig von gelastert hat m \textbf{4} selben] selber n \textbf{6} enmuget] múgent n (o) \textbf{7} enbringe] bringe n o \textbf{8} sanfte] vnsanffte n (o) \textbf{9} Waleis] waleisz o  $\cdot$ minnen] mẏnne n \textbf{10} swîgens] Swigen n  $\cdot$ Keie] keẏe n o \textbf{11} vrumtim] frumtin m frumpt jme n (o)  $\cdot$ swanc] [slag]: swang n \textbf{12} anz] Aus m \textbf{16} al] \textit{om.} n o  $\cdot$ mîn] \textit{om.} m \textbf{18} treit] truͯg m \textbf{19} in alsô bliuwen] in sollich in also bluͯmen m \textbf{20} möhte] mohte m (o) \textbf{21} hie seht] hie ist vnd seht m \textbf{22} ez] \textit{om.} m \textbf{23} der] \textit{om.} n \textbf{25} möhter] mohter m (o) \textbf{27} Waleise] waleẏse n \textbf{28} in] ẏne o \textbf{29} strenge] stenger o  $\cdot$ last] lalt m \newline
\end{minipage}
\end{table}
\newpage
\begin{table}[ht]
\begin{minipage}[t]{0.5\linewidth}
\small
\begin{center}*G
\end{center}
\begin{tabular}{rl}
 & daz ir den künic gelastert hât,\\ 
 & welt ir \textbf{im wandelen}, sô ist mîn rât,\\ 
 & unt \textbf{dunkt} mich iuwer beste\textit{z} heil,\\ 
 & \textbf{ir} nemet iuch an ein bracken seil\\ 
5 & unde lât iuch vür in ziehen.\\ 
 & ir \textbf{en}muget mir niht enpfliehen,\\ 
 & ich bringe iuch doch \textbf{gevangen} dar,\\ 
 & sô nimet man iuwer \textbf{unsanfte} war."\\ 
 & den Waleis twanc der minnen kraft\\ 
10 & \textbf{swîgens}. Kay sînen schaft\\ 
 & ûf zôch unde vrumt im einen swanc\\ 
 & \textbf{ûf daz houbet}, daz der helm erklanc.\\ 
 & \textbf{dô sprach er}: "dû muost wachen.\\ 
 & âne lîlachen\\ 
15 & wirt dir dîn \textbf{slâfen} hie benant.\\ 
 & ez zilt al anders mîn hant:\\ 
 & ûf den snê \textbf{dû wirst} geleit.\\ 
 & der den sac \textbf{von} der müle treit,\\ 
 & wolte man in \textbf{alsus} bliuwen,\\ 
20 & in m\textit{ö}hte lazheit riuwen."\\ 
 & vrou minne, hie seht ir zuo.\\ 
 & ich wæne, man \textbf{iuz} ze laster tuo,\\ 
 & wan ein gebûre spræche sân:\\ 
 & "mînem hêrren sî \textbf{diz} getân."\\ 
25 & \textbf{er klagt} \textbf{ouch}, m\textit{ö}hte\textbf{r} sprechen.\\ 
 & vrou minne, lât sich rechen\\ 
 & den werden Waleise,\\ 
 & wan lieze in iuwer \textbf{vreise}\\ 
 & unt iuwer strenge, unsüezer last,\\ 
30 & ich wæne, sich werte dirre gast.\\ 
\end{tabular}
\scriptsize
\line(1,0){75} \newline
G I O L M Q R Z Fr40 \newline
\line(1,0){75} \newline
\textbf{3} \textit{Initiale} I  \textbf{9} \textit{Initiale} O R Z Fr40  \textbf{13} \textit{Initiale} M  \textbf{21} \textit{Initiale} I L R  \newline
\line(1,0){75} \newline
\textbf{1} gelastert] gelaster I \textbf{2} im wandelen] im buͤzen I im wandel R mir volgen Z  $\cdot$ sô] tuͦn R \textbf{3} bestez] beste G \textbf{4} ir nemet iuch] lat evch nemen I So nempt ivch O (L) (M) (Q) (R) (Fr40) Nemt evch Z \textbf{6} ir enmuget] ir mugt I (O) (L) (R) Jrn mochet Q  $\cdot$ mir] \textit{om.} R \textbf{8} man] \textit{om.} Q Z  $\cdot$ iuwer] euch Q \textbf{9} den] ÷en O Der Fr40  $\cdot$ minnen] minne I O (L) (M) (Q) R \textbf{10} kai swigens sinen shaft I  $\cdot$ swîgens] Swigende L (M) (R) (Fr40) Wigende Q  $\cdot$ Kay] kaẏ G Key O (M) (Q) (R) (Z) (Fr40) kaý L  $\cdot$ schaft] krafft mit dem schafftt R \textbf{11} ûf] Vz O (L) (M)  $\cdot$ vrumt] vrunt I frvmte L (M)  $\cdot$ im] in O \textbf{12} ûf daz] Vf Z  $\cdot$ daz der] dez der Fr40 \textbf{13} dô] Da M Z \textbf{14} ân bette vnd ân lilachen I \textbf{15} dir] die M  $\cdot$ dîn] min I  $\cdot$ slâfen] shaft I slaf Z \textbf{16} al] \textit{om.} I \textbf{18} müle] muͯlen L (Q) (R)  $\cdot$ treit] tret R \textbf{19} alsus] sus L M \textbf{20} möhte] mohte G I (O) L (M) (Q) Z  $\cdot$ lazheit] wol M lacheit Q \textbf{21} \textit{Die Verse 294.21-22 fehlen} M   $\cdot$ minne] meine Q  $\cdot$ hie] da I \textbf{22} man iuz] manz ev I (L) (Q) \textbf{23} spræche] spreche êê O sprach esz M \textbf{24} diz ist mim herren getan I  $\cdot$ hêrren] herczin M (Z)  $\cdot$ diz] daz O L (M) (Q) (R)  $\cdot$ getân] gan Z \textbf{25} klagt] chlagte O (L) (M)  $\cdot$ möhter] mohter G (O) (L) (M) (Q) (Z) \textbf{26} minne] meine Q  $\cdot$ sich] sich uch O es R \textbf{27} den] Dem Q  $\cdot$ Waleise] waleiýsen L waleisse R \textbf{28} lieze] liezen I  $\cdot$ vreise] prise O reiszen L \textbf{29} strenge] strenger I (Z) strengen R  $\cdot$ unsüezer] vnshuzzer I vnsusze M \newline
\end{minipage}
\hspace{0.5cm}
\begin{minipage}[t]{0.5\linewidth}
\small
\begin{center}*T
\end{center}
\begin{tabular}{rl}
 & daz ir den künec gelestert hât,\\ 
 & welt ir\textbf{z im wandeln}, sô ist mîn rât,\\ 
 & und \textbf{dûhte} mich iuwer beste he\textit{i}l:\\ 
 & nemt iuch an ein bracken seil\\ 
5 & unde lât iuch vür in ziehen.\\ 
 & ir muget mir niht enpfliehen,\\ 
 & i\textbf{ne} bringiuch doch \textbf{gevangen} dar,\\ 
 & sô nimt man iuwer \textbf{unsanfte} war."\\ 
 & \begin{large}D\end{large}en Waleis twanc der minnen kraft.\\ 
10 & \textbf{swîgende} Key sînen schaft\\ 
 & ûf zôch unde vrumt im einen swanc,\\ 
 & daz \textbf{im} der helm \textbf{ûf dem houbete} erklanc.\\ 
 & \textbf{dô sprach er}: "dû muost wachen.\\ 
 & âne lîlachen\\ 
15 & wirt dir dîn \textbf{slâf} \textbf{al}hie benant.\\ 
 & ez zilt al anders mîn hant:\\ 
 & ûf den snê \textbf{wirst dû} geleit.\\ 
 & Der den sac \textbf{ze} der müln treit,\\ 
 & wolte man in \textbf{alsus} bliuwen,\\ 
20 & in m\textit{ö}hte lazheit riuwen."\\ 
 & Vrou minne, hie seht ir \textbf{selbe} zuo.\\ 
 & ich wæne, man\textbf{\textit{z} iu} ze laster tuo,\\ 
 & wan ein gebûr spræche sân:\\ 
 & "mînem hêrren sî \textbf{daz} getân."\\ 
25 & \textbf{Ir klagetet} \textbf{ouch}, m\textit{ö}ht \textbf{ir} sprechen.\\ 
 & Vrou minne, lât sich rechen\\ 
 & den werden Waleise,\\ 
 & wan liez in iuwer \textbf{vreise}\\ 
 & unde iuwer strenger, unsüezer last,\\ 
30 & ich wæne, sich werte dirre gast.\\ 
\end{tabular}
\scriptsize
\line(1,0){75} \newline
T U V W \newline
\line(1,0){75} \newline
\textbf{9} \textit{Initiale} T U V  \textbf{18} \textit{Majuskel} T  \textbf{21} \textit{Majuskel} T  \textbf{25} \textit{Majuskel} T  \textbf{26} \textit{Majuskel} T  \newline
\line(1,0){75} \newline
\textbf{2} irz im] ir [*]: im V \textbf{3} dûhte] dunket U (V) (W)  $\cdot$ beste] bestes V W  $\cdot$ heil] hel T \textbf{4} iuch] îv T v́ch selben V \textbf{5} iuch] iv T \textbf{7} ine] Ich W  $\cdot$ bringiuch] bringîv T \textbf{8} man] man dan U  $\cdot$ unsanfte] kleine W \textbf{9} Waleis] walleis V \textbf{10} Key] [k]: keie V \textbf{12} im] \textit{om.} W  $\cdot$ ûf] von W  $\cdot$ erklanc] klang W \textbf{16} zilt] zelt U W \textbf{18} ze] von W \textbf{20} in möhte] in mohte T (U) [Jm*]: Jn moͤhte  V In muͤste W \textbf{21} minne] mine U  $\cdot$ selbe] \textit{om.} W \textbf{22} manz iu] mans îv T mens v́ch V man eúchs W \textbf{24} mînem] Meinen W  $\cdot$ daz] [d*]: diz V dis W \textbf{25} Ir klagetet] Jr claget U (W) Er [clage*]: clagete  V  $\cdot$ ouch] eúch W  $\cdot$ möht] moht T (U) moͤchten W  $\cdot$ ir] [*]: er V \textbf{26} minne] mine U  $\cdot$ sich] eúch W \textbf{27} Waleise] walleise V waleisen W \textbf{28} wan liez] [W*]: Wan liez V Wann liessen W  $\cdot$ vreise] freisen W \textbf{29} unsüezer] suͤsser W \newline
\end{minipage}
\end{table}
\end{document}
