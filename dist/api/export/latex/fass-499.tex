\documentclass[8pt,a4paper,notitlepage]{article}
\usepackage{fullpage}
\usepackage{ulem}
\usepackage{xltxtra}
\usepackage{datetime}
\renewcommand{\dateseparator}{.}
\dmyyyydate
\usepackage{fancyhdr}
\usepackage{ifthen}
\pagestyle{fancy}
\fancyhf{}
\renewcommand{\headrulewidth}{0pt}
\fancyfoot[L]{\ifthenelse{\value{page}=1}{\today, \currenttime{} Uhr}{}}
\begin{document}
\begin{table}[ht]
\begin{minipage}[t]{0.5\linewidth}
\small
\begin{center}*D
\end{center}
\begin{tabular}{rl}
\textbf{499} & \begin{large}M\end{large}it golde ein wazzer, rinnet.\\ 
 & dâ wart Ither geminnet.\\ 
 & \textbf{dise} basen er dâ vant;\\ 
 & diu was vrouwe überz lant.\\ 
5 & Gandin von Anschouwe\\ 
 & hiez si dâ wesen vrouwe.\\ 
 & \textbf{si} \textbf{heizet} Lammire;\\ 
 & sô istz lant genennet Stire.\\ 
 & swer schildes ambet üeben wil,\\ 
10 & der muoz durchstrîchen \textbf{lande} vil.\\ 
 & Nû riwet mich mîn \textbf{knappe} rôt,\\ 
 & durch den si mir \textbf{grôz} êre bôt.\\ 
 & von Ithern dû bist \textbf{erborn};\\ 
 & dîn hant die sippe hât \textbf{verkorn}.\\ 
15 & got hât \textbf{ir} \textit{niht} vergezzen doch,\\ 
 & er kan si wol geprüeven noch.\\ 
 & wil dû gein got mit triwen leben,\\ 
 & \textbf{sô soltû} im \textbf{wandel drumbe} geben.\\ 
 & mit \textbf{riwe} ich dir daz künde:\\ 
20 & dû treist zwô grôze sünde.\\ 
 & Ithern dû hâst erslagen;\\ 
 & dû solt ouch dîne muoter klagen.\\ 
 & ir grôziu triwe daz geriet,\\ 
 & dîn vart si vome \textbf{leben} schiet,\\ 
25 & die dû jungest von ir tæte.\\ 
 & nû volge mîner ræte:\\ 
 & nim buoze vür missewende\\ 
 & und sorge êt umb dîn ende,\\ 
 & daz \textbf{dir} dîn arbeit hie erhol,\\ 
30 & daz dort diu sêle ruowe dol."\\ 
\end{tabular}
\scriptsize
\line(1,0){75} \newline
D Fr11 \newline
\line(1,0){75} \newline
\textbf{1} \textit{Initiale} D  \textbf{11} \textit{Majuskel} D  \newline
\line(1,0){75} \newline
\textbf{2} Ither] Jther D \textbf{5} Anschouwe] Anschoͮwe D \textbf{7} Lammire] Lammîre D \textbf{8} Stire] Stŷre D stiͯre Fr11 \textbf{10} muoz] muͯs Fr11  $\cdot$ lande] landes Fr11 \textbf{11} mîn] der Fr11 \textbf{13} Ithern] Jthern D ythern Fr11 \textbf{15} niht] \textit{om.} D \textbf{21} Ithern] Jthern D Ythern Fr11  $\cdot$ dû] \textit{om.} Fr11 \textbf{25} die] Divͯ Fr11 \textbf{28} êt] \textit{om.} Fr11 \textbf{30} ruowe] reẅe Fr11 \newline
\end{minipage}
\hspace{0.5cm}
\begin{minipage}[t]{0.5\linewidth}
\small
\begin{center}*m
\end{center}
\begin{tabular}{rl}
 & mit golde ein wazzer, rinnet.\\ 
 & d\textit{â} wart I\textit{t}her geminnet.\\ 
 & \textbf{dîn} basen er d\textit{â} vant;\\ 
 & diu was vrouwe \textit{über} daz lant.\\ 
5 & Gandin von A\textit{nsch}o\textit{uw}e \\ 
 & hiez si d\textit{â} wesen vrouwe.\\ 
 & \textbf{si} \textbf{heizet} Lamire;\\ 
 & sô ist daz lant genennet St\textit{i}re.\\ 
 & wer schiltes ambet üebe\textit{n} wil,\\ 
10 & der muoz durchstrîchen \textbf{lande} vil.\\ 
 & nû riuwet mich mîn \textbf{knappe} rôt,\\ 
 & durch den si mir \textbf{grôz} êre bôt.\\ 
 & von I\textit{t}her dû bist \textbf{erborn};\\ 
 & dîn hant die sippe het \textbf{verlorn}.\\ 
15 & got het \textbf{ir} niht vergezzen doch,\\ 
 & er ka\textit{n} si wol gebrüefen noch.\\ 
 & wiltû gegen got mit triuwen leben,\\ 
 & \textbf{sô soltû} im \textbf{wandel} geben.\\ 
 & mit \textbf{riuwe} ich dir daz künde:\\ 
20 & d\textit{û} treist zwô grôze sünde.\\ 
 & I\textit{t}hern dû hâst erslagen;\\ 
 & dû solt ouch dîn muoter klagen.\\ 
 & ir grôziu triuwe daz geriet,\\ 
 & dîn vart si von dem \textbf{leben} schiet,\\ 
25 & die dû jungest von i\textit{r} tæte.\\ 
 & nû volge mîner ræte:\\ 
 & \textit{n}i\textit{m} buoze vür missewende\\ 
 & und sorge eht umb dîn ende,\\ 
 & daz \textbf{dir} dîn arbeit hie erhol,\\ 
30 & daz dort diu sêle ruowe dol."\\ 
\end{tabular}
\scriptsize
\line(1,0){75} \newline
m n o \newline
\line(1,0){75} \newline
\newline
\line(1,0){75} \newline
\textbf{2} dâ] Do m n o  $\cdot$ Ither] icher m iethern o \textbf{3} dâ] do m n o \textbf{4} diu] Dis o  $\cdot$ über] in m \textbf{5} Gandin] Gaudin n  $\cdot$ Anschouwe] astorie m auscouwe n anscowe o \textbf{6} dâ] do m n o \textbf{7} Lamire] lannire n \textbf{8} Stire] stiere m stier o \textbf{9} üeben] uͯber m (o) aber n \textbf{13} Ither] icher m ithern o \textbf{14} hant] [lant]: hant o  $\cdot$ verlorn] verkorn n (o) \textbf{16} kan] kam m  $\cdot$ si wol] sú wol sú wol n \textbf{18} geben] dar vmb geben n (o) \textbf{19} riuwe] truwe n \textbf{20} dû] Do m  $\cdot$ sünde] suͯnnede o \textbf{21} Ithern] Jchern m Jthern n o \textbf{22} ouch] do ouch n \textbf{25} jungest] junstest o  $\cdot$ ir] ẏm m \textbf{27} nim] Min m n (o)  $\cdot$ buoze] [misse]: buͯsse o \textbf{28} eht umb] echt vmmb echt vmmb o \textbf{29} erhol] erholt o \textbf{30} dol] dolt o \newline
\end{minipage}
\end{table}
\newpage
\begin{table}[ht]
\begin{minipage}[t]{0.5\linewidth}
\small
\begin{center}*G
\end{center}
\begin{tabular}{rl}
 & \begin{large}M\end{large}it golde ein wazzer, rinnet.\\ 
 & dâ wart Ither geminnet.\\ 
 & \textbf{dîne} basen er dâ vant;\\ 
 & diu was vrouwe überz lant.\\ 
5 & Gandin von Antschouwe\\ 
 & hiez si dâ wesen vrouwe.\\ 
 & \textbf{si} \textbf{heizet} Lammire;\\ 
 & sô ist daz lant genennet Stire.\\ 
 & swer schiltes ambet üeben wil,\\ 
10 & der muoz durchstrîchen \textbf{lande} vil.\\ 
 & nû riuwet mich mîn \textbf{knappe} rôt,\\ 
 & durch den si mir \textbf{vil} êre bôt.\\ 
 & von Ither dû bist \textbf{erborn};\\ 
 & dîn hant die sippe hât \textbf{erkorn}.\\ 
15 & got hât \textbf{ir} niht vergezzen doch,\\ 
 & er kan si wol geprüeven noch.\\ 
 & wil dû gein got mit triuwen leben,\\ 
 & \textbf{sô soltû} im \textbf{wandel drumbe} geben.\\ 
 & mit \textbf{triuwen} ich dir daz künde:\\ 
20 & dû treist z\textit{w}ô grôze sünde.\\ 
 & Ithern dû hâst erslagen;\\ 
 & dû solt ouch dîne muoter klagen.\\ 
 & ir grôziu triuwe daz geriet,\\ 
 & dîn vart si voneme \textbf{lebenne} schiet,\\ 
25 & die dû jungest von ir tæte.\\ 
 & nû volge mîner ræte:\\ 
 & nim buoze vür missewende\\ 
 & unt sorge êt umbe dîn ende,\\ 
 & daz \textbf{dir} dîn arbeit hie erhol,\\ 
30 & daz dort diu sêle ruowe dol."\\ 
\end{tabular}
\scriptsize
\line(1,0){75} \newline
G I L M Z Fr61 \newline
\line(1,0){75} \newline
\textbf{1} \textit{Initiale} G I L Z  \textbf{15} \textit{Initiale} I  \newline
\line(1,0){75} \newline
\textbf{2} Ither] Jther I (M) (Fr61) Jehter L Jcher Z  $\cdot$ geminnet] genennit M \textbf{5} Gandin] Kandein Fr61  $\cdot$ Antschouwe] Anschoͮwe G [Antshowe]: Antshawe I Anschouwe L Anschowe M Fr61 anshowe Z \textbf{6} si] \textit{om.} Fr61 \textbf{7} si] Deu Fr61  $\cdot$ Lammire] lammîre G lamiture I laminire Z Lamire Fr61 \textbf{8} sô ist] Vnd L  $\cdot$ daz] ez Z  $\cdot$ genennet] \textit{om.} L  $\cdot$ Stire] stîre G stiere M styre Z \textbf{9} swer] Wer L M  $\cdot$ üeben] werben L \textbf{10} lande] landes L (M) \textbf{11} mîn] der Fr61 \textbf{12} vil] so grosze L (M) groz Z (Fr61)  $\cdot$ êre] ern I  $\cdot$ bôt] erbot L \textbf{13} Ither] Jthern I (M) Jehter L Jchern Z Jther Fr61  $\cdot$ erborn] geborn L M \textbf{14} die sippe hât] hat die sippe I  $\cdot$ erkorn] verchorn I (L) (M) (Z) \textbf{15} ir] dein Fr61 \textbf{16} si] dich Fr61  $\cdot$ wol geprüeven] bruͦven I \textbf{19} triuwen] ruwe M (Z) \textbf{20} zwô] zuͦ G \textbf{21} Ithern] Jthern G (I) M Fr61 Jehtern L Jchern Z \textbf{23} grôziu] grozze Fr61  $\cdot$ triuwe] iamer L \textbf{24} vart] vater M  $\cdot$ lebenne] leibe Fr61 \textbf{26} nû volge] volge der Fr61  $\cdot$ mîner ræte] minen rate I (Fr61) mýnem rate L \textbf{28} êt] etwa L \textit{om.} Fr61 \textbf{30} diu] dein Fr61  $\cdot$ ruowe] icht chvmber Fr61 \newline
\end{minipage}
\hspace{0.5cm}
\begin{minipage}[t]{0.5\linewidth}
\small
\begin{center}*T
\end{center}
\begin{tabular}{rl}
 & \multicolumn{1}{l}{ - - - }\\ 
 & \multicolumn{1}{l}{ - - - }\\ 
 & \textbf{sîne} basen er dâ vant;\\ 
 & diu was \textbf{ouch} vrouwe überz lant.\\ 
5 & \multicolumn{1}{l}{ - - - }\\ 
 & \multicolumn{1}{l}{ - - - }\\ 
 & \textbf{di\textit{u}} \textbf{hiez} Lamyre;\\ 
 & sô ist daz lant genennet Styre.\\ 
 & swer schiltes ambet üeben wil,\\ 
10 & der muoz durchstrîchen \textbf{landes} vil.\\ 
 & \begin{large}N\end{large}û riuwet mich mîn \textbf{rîter} rôt,\\ 
 & durch den si mir \textbf{grôz} êre bôt.\\ 
 & von Ithere dû bist \textbf{geborn};\\ 
 & dîn hant die sippe hât \textbf{verkorn}.\\ 
15 & got hât \textbf{dîn} niht vergezzen doch,\\ 
 & er kan si wol geprüeven noch.\\ 
 & wiltû gegen gote mit triuwen leben,\\ 
 & \textbf{dû solt} im \textbf{drumbe wandel} geben.\\ 
 & mit \textbf{riuwe} ich dir daz künde:\\ 
20 & dû treist zwô grôze sünde.\\ 
 & Ithern dû hâst erslagen;\\ 
 & dû solt ouch dîne muoter klagen.\\ 
 & ir grôz\textit{iu} triuwe daz geriet,\\ 
 & dîn vart si von dem \textbf{lîbe} schiet,\\ 
25 & die dû jungest von ir tæte.\\ 
 & Nû volge mîner ræte:\\ 
 & nim buoze vür missewende\\ 
 & unde sorge eht umbe dîn ende,\\ 
 & daz dîn arbeit hie erhol,\\ 
30 & daz dort diu sêle ruowe dol."\\ 
\end{tabular}
\scriptsize
\line(1,0){75} \newline
T U V W O Q R Fr39 \newline
\line(1,0){75} \newline
\textbf{1} \textit{Initiale} Q  \textbf{11} \textit{Initiale} T V  \textbf{15} \textit{Initiale} W  \textbf{26} \textit{Majuskel} T  \newline
\line(1,0){75} \newline
\textbf{1} \textit{Die Verse 453.1-502.30 fehlen} U   $\cdot$ \textit{Die Verse 499.1-2 fehlen} T   $\cdot$ Mit golde ein wasser rinnet V W (O) (Q) (R) Fr39 \textbf{2} Do wart [ytern]: yter geminnet V  $\cdot$ Do ward yther (Jther O Fr39 yhther Q Jhter R ) geminnet W (O) (Q) (R) (Fr39) \textbf{3} sîne] [Dine]: Sine V Dine O  $\cdot$ dâ] do V W Q Fr39 \textbf{4} diu] Die selbe R  $\cdot$ was ouch] vͦch waz V waz R \textbf{5} \textit{Die Verse 499.5-6 fehlen} T O Q R W Fr39   $\cdot$ \textit{Die Verse 499.5-6 sind am Rand nachgetragen und später radiert:} :ndin von anschowe / :sv́ do wesen vrowe V  \textbf{7} diu] die T  $\cdot$ Lamyre] [*mire]: lamire V lammire W (Q) Lammyre O (Fr39) lamire R \textbf{8} Styre] [S*]: Stŷre T stire V W Q R stẏre O \textbf{9} swer] Wer W Q R  $\cdot$ üeben] vbel Q \textbf{10} landes] [*]: lande V \textbf{11} mîn rîter rôt] [d* rot]: min ritter rot V \textbf{12} bôt] entbot Q \textbf{13} Ithere] Jthere T [ytern]: yter V yther W Jther O ithern Q Jhter R \textbf{14} die] dein W  $\cdot$ verkorn] verlorn W O \textbf{16} si] dir V R [*]: dir Fr39 dich W sich O gnoden Q  $\cdot$ wol] \textit{om.} Q [*]: wol Fr39  $\cdot$ geprüeven] gehelfen V (R) [*]: gehelfen  Fr39 ergetzen W erzeigen Q \textbf{17} wiltû] Wilt Q \textbf{18} drumbe wandel] wandel darumbe R \textbf{19} riuwe] triwe O (Q)  $\cdot$ dir] \textit{om.} Fr39  $\cdot$ daz] dar W  $\cdot$ künde] verkúnde W \textbf{20} treist] werst R  $\cdot$ zwô] zw Q zu R  $\cdot$ grôze] grosser W R \textbf{21} Ithern] Jthern T Ytern V Ythern W Jhtern R  $\cdot$ dû hâst] hastu R \textbf{22} dîne] deiner W \textbf{23} grôziu] groze T  $\cdot$ daz] ir daz V O \textbf{25} die] Do R  $\cdot$ jungest] zeiungest V  $\cdot$ tæte] Ritte vnd tette R \textbf{26} mîner ræte] [mirer]: miner ræte O minem Rautte R \textbf{27} nim] [*in]: Nim V Mein Q \textbf{28} eht] auch Q \textbf{29} daz] Daz [*]: dir V Das dir W (O) R Fr39 Das die Q \textbf{30} Das dir dein sele rewe dol Q  $\cdot$ diu] dein W  $\cdot$ dol] wol W \newline
\end{minipage}
\end{table}
\end{document}
