\documentclass[8pt,a4paper,notitlepage]{article}
\usepackage{fullpage}
\usepackage{ulem}
\usepackage{xltxtra}
\usepackage{datetime}
\renewcommand{\dateseparator}{.}
\dmyyyydate
\usepackage{fancyhdr}
\usepackage{ifthen}
\pagestyle{fancy}
\fancyhf{}
\renewcommand{\headrulewidth}{0pt}
\fancyfoot[L]{\ifthenelse{\value{page}=1}{\today, \currenttime{} Uhr}{}}
\begin{document}
\begin{table}[ht]
\begin{minipage}[t]{0.5\linewidth}
\small
\begin{center}*D
\end{center}
\begin{tabular}{rl}
\textbf{647} & \begin{large}Û\end{large}f den hof dû balde drabe.\\ 
 & enruoche, ob dîn runzît \textbf{iemen} habe;\\ 
 & dâ von soltû balde gên,\\ 
 & \textbf{al} dâ die werden rîter stên.\\ 
5 & die vrâgent \textbf{dich} âventiure;\\ 
 & als dû gâhest ûz eime viure,\\ 
 & gebâre mit \textbf{rede} unt \textbf{ouch mit} siten.\\ 
 & von in vil kûme wirt \textbf{gebiten},\\ 
 & waz dû mære bringest.\\ 
10 & waz \textbf{wirret}, ob dû \textbf{dich} dringest\\ 
 & durchz volc unz an den \textbf{rehten} wirt,\\ 
 & der gein dir grüezen niht verbirt?\\ 
 & \textbf{Disen} brief gib im in die hant,\\ 
 & dâr an er schiere hât erkant\\ 
15 & dîniu mære unt dînes \textbf{hêrren} ger;\\ 
 & des ist er mit der volge wer.\\ 
 & \textbf{Noch mêr wil ich} lêren dich:\\ 
 & offenlîche soltû \textbf{sprechen} mich,\\ 
 & dâ ich unt ander vrouwen\\ 
20 & dich hœren und schouwen.\\ 
 & dâ wirbe umb uns, als dû wol kanst,\\ 
 & ob dû dîme hêrren guotes ganst,\\ 
 & \textbf{Unt} sage mir: wâ ist Gawan?"\\ 
 & \textbf{der knappe sprach}: "\textbf{daz} wirt verlân;\\ 
25 & ich sage \textbf{iu} niht, wâ \textbf{mîn hêrre} sî.\\ 
 & welt ir, er belîbet vreuden bî."\\ 
 & Der knappe \textbf{was} \textbf{ir} râtes vrô.\\ 
 & von der küneginne \textbf{er} dô\\ 
 & schiet, als ir wol habt vernomen,\\ 
30 & unt kom \textbf{ouch}, als er solde komen.\\ 
\end{tabular}
\scriptsize
\line(1,0){75} \newline
D \newline
\line(1,0){75} \newline
\textbf{1} \textit{Initiale} D  \textbf{13} \textit{Majuskel} D  \textbf{17} \textit{Majuskel} D  \textbf{23} \textit{Majuskel} D  \textbf{27} \textit{Majuskel} D  \newline
\line(1,0){75} \newline
\newline
\end{minipage}
\hspace{0.5cm}
\begin{minipage}[t]{0.5\linewidth}
\small
\begin{center}*m
\end{center}
\begin{tabular}{rl}
 & ûf den hof dû balde drabe.\\ 
 & enruoch, ob dîn runzît \textbf{niemen} habe;\\ 
 & dâ von soltû balde gân,\\ 
 & d\textit{â} die werden ritter stân.\\ 
5 & die vrâgent \textbf{dich} âve\textit{n}tiure;\\ 
 & als dû gâhest ûz einem viure,\\ 
 & gebâr mit \textbf{rede} und \textbf{mit} siten.\\ 
 & von in vil kûme w\textit{i}rt \textbf{erbiten},\\ 
 & waz dû mære bringest.\\ 
10 & waz \textbf{irret}, ob dû dringest\\ 
 & durch daz volc unz an d\textit{en} wirt,\\ 
 & der gegen dir grüezen niht verbirt?\\ 
 & \textbf{den} \textit{b}ri\textit{ef}, \textbf{den} gip im in die hant,\\ 
 & dâr an er schier het erkant\\ 
15 & dîniu mære und dînes \textbf{herzen} ger;\\ 
 & des ist er mit der volge wer.\\ 
 & \textbf{noch \textit{m}ê wil ich} lêren dich:\\ 
 & offenlîch soltû \textbf{sprechen} mich,\\ 
 & d\textit{â} ich und ander vrouwen\\ 
20 & dich hœren und schouwen.\\ 
 & d\textit{â} wirp umb uns, als dû wol kanst,\\ 
 & ob dû dîne\textit{m} hêrren guotes ganst,\\ 
 & \textbf{und} sage mir: wâ ist Gawan?"\\ 
 & \textbf{der knappe sprach}: "\textbf{daz} wirt verlân;\\ 
25 & ich sage \textbf{iu} niht, wâ \textit{\textbf{er}} sî.\\ 
 & wolt ir, \dag erwerben\dag  blîbet vröuden bî."\\ 
 & \begin{large}D\end{large}er knappe \textbf{was} \textbf{des} râtes vrô.\\ 
 & von der künigîn \dag aldâ\dag \\ 
 & schiet, als ir wol habt vernomen,\\ 
30 & und kam \textbf{ouch}, als er solte komen.\\ 
\end{tabular}
\scriptsize
\line(1,0){75} \newline
m n o \newline
\line(1,0){75} \newline
\textbf{27} \textit{Initiale} m n  \newline
\line(1,0){75} \newline
\textbf{1} dû] do o \textbf{2} niemen] ẏeman n (o) \textbf{3} soltû] [solle]: solte o \textbf{4} dâ] Do m n o \textbf{5} dich] doch n  $\cdot$ âventiure] auenfentuͯre m \textbf{6} gâhest] gast o \textbf{7} gebâr] Gabar o \textbf{8} in] ẏm o  $\cdot$ wirt] wart m \textbf{10} irret] wirret n o  $\cdot$ dringest] twingest o \textbf{11} den] das m \textbf{13} brief] pris m \textbf{15} dîniu] Din m o \textbf{16} des] Das o \textbf{17} mê] nie m o  $\cdot$ lêren dich] dich leren o \textbf{19} dâ] Do m n o \textbf{21} dâ] Do m n o \textbf{22} dînem] dinen m o \textbf{25} er] ich m \textbf{26} erwerben] ir belibent n blibet o  $\cdot$ vröuden] froiúde o \newline
\end{minipage}
\end{table}
\newpage
\begin{table}[ht]
\begin{minipage}[t]{0.5\linewidth}
\small
\begin{center}*G
\end{center}
\begin{tabular}{rl}
 & \begin{large}Û\end{large}f den hof dû balde drabe.\\ 
 & enruoche, ob dîn runzît \textbf{niemen} habe;\\ 
 & dâ von soltû balde gên,\\ 
 & dâ die werden rîter stên.\\ 
5 & die vrâgent \textbf{dich} âventiure;\\ 
 & alse dû gâhest ûz einem viure,\\ 
 & gebâre mit \textbf{rede} unt \textbf{mit} siten.\\ 
 & von in vil kûme wirt \textbf{erbiten},\\ 
 & waz dû mære bringest.\\ 
10 & waz \textbf{wirret}, ob dû \textbf{dich} dringest\\ 
 & durch daz volc unze an den wirt,\\ 
 & der gein dir grüezen niht verbirt?\\ 
 & \textbf{den} brief gib im in die hant,\\ 
 & dâr \textit{an} er schier hât erkant\\ 
15 & dîniu mære unde dînes \textbf{herzen} ger;\\ 
 & des ist er mit der volge wer.\\ 
 & \textbf{mêre wil ich noch} lêren dich:\\ 
 & offenlîchen soltû \textbf{sprechen} mich,\\ 
 & dâ ich unde ander vrouwen\\ 
20 & dich hœren unde schouwen.\\ 
 & dâ wirb umbe uns, als dû wol kanst,\\ 
 & ob dû dînem hêrren \textbf{wol} guotes ganst,\\ 
 & \textbf{sô} sag mir: wâ ist Gawan?"\\ 
 & "\textbf{vrouwe}, \textbf{diz mære} wirt verlân;\\ 
25 & ich \textbf{en}sag \textbf{iu} niht, wâ \textbf{mîn hêrre} sî.\\ 
 & welt ir, er belîbet vröuden bî."\\ 
 & der \textit{k}nappe \textbf{wart} \textbf{ir} râtes vrô.\\ 
 & von der künegî\textit{n d}ô\\ 
 & \textit{schiet \textbf{er}, als ir wol habt vernomen},\\ 
30 & unde k\textit{om}, als er solde komen.\\ 
\end{tabular}
\scriptsize
\line(1,0){75} \newline
G I L M Z \newline
\line(1,0){75} \newline
\textbf{1} \textit{Initiale} G L M Z  \textbf{13} \textit{Initiale} I  \newline
\line(1,0){75} \newline
\textbf{2} [er]: en ruͤch wer din runzin habe I  $\cdot$ dîn] daz L \textbf{3} gên] [haben]: gen M \textbf{4} werden] werde M \textbf{7} gebâre] So gibare M  $\cdot$ rede] reden I  $\cdot$ unt] vnd ouch Z \textbf{11} den] den rehten Z \textbf{12} grüezen] gruͤzzes I \textbf{13} den brief] Disen brief den Z \textbf{14} an] \textit{om.} G \textbf{15} herzen ger] hern gern M \textbf{16} wer] gewer Z \textbf{17} mêre] Wer M  $\cdot$ noch] \textit{om.} I M \textbf{18} sprechen] gesprechen I \textbf{19} unde ander] vndir M \textbf{21} dâ] So L \textbf{22} wol] \textit{om.} L Z  $\cdot$ ganst] kanst M \textbf{23} sô] \textit{om.} I Vnd Z  $\cdot$ mir] \textit{om.} L M \textbf{24} mære] \textit{om.} Z  $\cdot$ wirt] wir I  $\cdot$ verlân] verlorn M \textbf{25} ensag iu] sage M  $\cdot$ mîn hêrre] er L \textbf{26} vröuden] frovwen L \textbf{27} knappe] chanappe G  $\cdot$ ir] \textit{om.} L \textbf{28} künegîn dô] kvnegin er do G (L) (Z) konnigin er da M \textbf{29} \textit{Vers 647.29 fehlt} G   $\cdot$ er] \textit{om.} L M Z  $\cdot$ wol] \textit{om.} L \textbf{30} kom] chome G kom och L (M) (Z) \newline
\end{minipage}
\hspace{0.5cm}
\begin{minipage}[t]{0.5\linewidth}
\small
\begin{center}*T
\end{center}
\begin{tabular}{rl}
 & ûf de\textit{n} ho\textit{f d}û balde drabe.\\ 
 & enruoch, ob dîn runzît \textbf{nieman} habe;\\ 
 & dâ von saltû \textit{b}a\textit{ld}e gên,\\ 
 & \textbf{al}dâ die werden ritter stên.\\ 
5 & die vrâgent \textit{\textbf{nâch}} âventiure;\\ 
 & als dû gâhest ûz einem viure,\\ 
 & geb\textit{âr}e mit \textbf{reden} und siten.\\ 
 & von in vil kûme wirt \textbf{erbiten},\\ 
 & waz dû mære bringest.\\ 
10 & waz \textbf{wirret}, ob dû \textbf{\textit{d}ich} dringest\\ 
 & durch daz volc unz an den \textbf{rehten} wirt,\\ 
 & der gein dir grüezen niht verbirt?\\ 
 & \textbf{disen} brief gip im in die hant,\\ 
 & dâr an er schiere hât erkant\\ 
15 & dîniu mære und dînes \textbf{hêrren} ger;\\ 
 & des ist er mit der volge wer.\\ 
 & \textbf{noch mêr wil ich} lêren dich:\\ 
 & offenlîch soltû \textbf{besprechen} mich,\\ 
 & d\textit{â} ich und \textit{ander} vrouwen\\ 
20 & dich hœren und schouwen.\\ 
 & dâ wirb umb uns, als dû wol kanst,\\ 
 & ob dû dînem hêrren guotes ganst,\\ 
 & \textbf{und} sag mir: wâ ist Gawan?"\\ 
 & \textbf{der knabe sprach}: "\textbf{daz} wirt verlân;\\ 
25 & ich sag niht, wâ \textbf{mîn hêrre} sî.\\ 
 & wolt ir, er blîbet vreuden bî."\\ 
 & der knabe \textbf{wart} \textbf{ir} râtes vrô.\\ 
 & von der künigîn \textbf{er} dô\\ 
 & schiet, als ir wol habt vernomen,\\ 
30 & und kom \textbf{ouch}, als er solte komen.\\ 
\end{tabular}
\scriptsize
\line(1,0){75} \newline
Q R W V \newline
\line(1,0){75} \newline
\textbf{1} \textit{Capitulumzeichen} R  \textbf{27} \textit{Initiale} W V  \newline
\line(1,0){75} \newline
\textbf{1} den hof dû] dem hofe zu Q \textbf{2} nieman] ieman V \textbf{3} balde] habe Q selber R \textbf{5} nâch] die Q dich W V \textbf{6} viure] freúre W \textbf{7} gebâre] Geberde Q  $\cdot$ reden] Rede R (W) (V)  $\cdot$ und] vnd mit R (W) vnde oͮch mit V \textbf{8} erbiten] erbieten Q erbeitten R \textbf{10} dich] mich Q \textit{om.} V \textbf{11} rehten] \textit{om.} R W \textbf{13} disen] Disem R  $\cdot$ gip] gab R [d*]: den gip V \textbf{14} er] er sich R er gar W \textbf{15} dîniu] Dine R  $\cdot$ hêrren] herczen R (W) \textbf{16} des] Der R \textbf{17} lêren] lernen W \textbf{18} besprechen] sprechen R W gesprechen V \textbf{19} dâ] Do Q W V  $\cdot$ ander] \textit{om.} Q auch andet W \textbf{20} und] vnd auch W \textbf{21} dâ] Do W V  $\cdot$ umb] vns R \textbf{23} und] So R \textbf{24} verlân] erlan W [*]: verlan V \textbf{25} [*]: Jch ensage úch niht wa er si V  $\cdot$ sî] sein Q \textbf{27} ir] dez V \textbf{28} künigîn] kunginen R  $\cdot$ er] \textit{om.} R \textbf{29} ir wol habt] er hett R ir hant wol V \newline
\end{minipage}
\end{table}
\end{document}
