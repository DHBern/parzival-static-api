\documentclass[8pt,a4paper,notitlepage]{article}
\usepackage{fullpage}
\usepackage{ulem}
\usepackage{xltxtra}
\usepackage{datetime}
\renewcommand{\dateseparator}{.}
\dmyyyydate
\usepackage{fancyhdr}
\usepackage{ifthen}
\pagestyle{fancy}
\fancyhf{}
\renewcommand{\headrulewidth}{0pt}
\fancyfoot[L]{\ifthenelse{\value{page}=1}{\today, \currenttime{} Uhr}{}}
\begin{document}
\begin{table}[ht]
\begin{minipage}[t]{0.5\linewidth}
\small
\begin{center}*D
\end{center}
\begin{tabular}{rl}
\textbf{408} & \begin{large}H\end{large}ie \textbf{der} ritter, dort \textbf{der} koufman,\\ 
 & diu juncvrouwe erhôrte sân\\ 
 & \textbf{den} bovel komen ûz der stat.\\ 
 & mit Gawane si gegen dem turne trat.\\ 
5 & ir vriunt muose kumber lîden.\\ 
 & si bat siz dicke mîden.\\ 
 & \textbf{ir kradem unt ir dôz was} \textbf{sô},\\ 
 & daz \textbf{ez} ir decheiner marcte dô.\\ 
 & Durch \textbf{strît} si \textbf{drungen} gein der tür.\\ 
10 & Gawan stuont ze wer dâr vür.\\ 
 & ir îngên er bewarte.\\ 
 & einen rigel, der den turn besparte,\\ 
 & den zucter ûz der mûre.\\ 
 & \textbf{sîn} arge nâchgebûre\\ 
15 & entwichen im dicke mit \textbf{ir} schar.\\ 
 & diu künegîn lief her unt dar,\\ 
 & ob ûf dem turn iht wære ze wer\\ 
 & gein \textbf{disem} ungetriwem her.\\ 
 & Dô vant diu \textbf{magt} reine\\ 
20 & ein schâchzabel gesteine\\ 
 & unt ein bret wol erleit, wît.\\ 
 & daz brâhte si Gawane \textbf{in} den strît.\\ 
 & an eime îsenînem ringe ez hienc,\\ 
 & dâ \textbf{mit} ez Gawan enpfienc.\\ 
25 & ûf \textbf{disen} \textbf{vierecken} schilt\\ 
 & \textbf{was} schâchzabels vil gespilt.\\ 
 & \textbf{der} wart im sêre \textbf{zerhouwen}.\\ 
 & nû hœret ouch von der vrouwen.\\ 
 & ez wære künec oder roch,\\ 
30 & daz warf si gein den vîenden doch.\\ 
\end{tabular}
\scriptsize
\line(1,0){75} \newline
D \newline
\line(1,0){75} \newline
\textbf{1} \textit{Initiale} D  \textbf{9} \textit{Majuskel} D  \textbf{19} \textit{Majuskel} D  \newline
\line(1,0){75} \newline
\newline
\end{minipage}
\hspace{0.5cm}
\begin{minipage}[t]{0.5\linewidth}
\small
\begin{center}*m
\end{center}
\begin{tabular}{rl}
 & \hspace*{-.7em}\big| die juncvrouwe erhôrte sân\\ 
 & \hspace*{-.7em}\big| hie \textbf{den} ritter, dort \textbf{den} koufman\\ 
 & \textbf{und} \textbf{den} pavel komen ûz der stat.\\ 
 & mit Gawane si gegen dem turne trat.\\ 
5 & ir vriunt muose kumber lîden.\\ 
 & si bat siz dicke mîden.\\ 
 & \textbf{dô was ir kradem und ir dôz} \textbf{alsô},\\ 
 & daz \textbf{eht} ir dekeiner marhte dô.\\ 
 & durch \textbf{strît} si \textbf{drungen} gegen der tür.\\ 
10 & Gawan stuont ze wer dâr vür.\\ 
 & ir îngân \textit{e}r bewarte.\\ 
 & einen rigel, der den turn besparte,\\ 
 & den zuckete er ûz der mûre.\\ 
 & \textbf{sîne} argen nâchgebûre\\ 
15 & entwichen ime dicke mit \textbf{der} schar.\\ 
 & diu künigîn lief her und dar,\\ 
 & ob ûf dem turne iht wære ze wer\\ 
 & gegen \textbf{disem} ungetriuwen her.\\ 
 & dô vant diu \textbf{mage\textit{t}} \textit{r}eine\\ 
20 & ein schâfzabel gesteine\\ 
 & und ein br\textit{e}t wol erleit, wît.\\ 
 & daz brâhte si Gawane \textbf{in} den strît.\\ 
 & an einem îsenînen ringe ez hienc,\\ 
 & dâ \textbf{mite} ez Gawan enpfienc.\\ 
25 & ûf \textbf{disen} \textbf{vierecken} schilt\\ 
 & \textbf{wart} schâfzabels vil gespilt.\\ 
 & \textbf{er} wart ime sêre \textbf{verhouwen}.\\ 
 & nû hœret ouch von der vrouwen.\\ 
 & ez wære künic, \textbf{alt} oder roch,\\ 
30 & daz warf si gegen den vîenden doch.\\ 
\end{tabular}
\scriptsize
\line(1,0){75} \newline
m n o \newline
\line(1,0){75} \newline
\newline
\line(1,0){75} \newline
\textbf{2} juncvrouwe] jungherren n \textbf{3} pavel] ponenl o \textbf{4} Gawane] gawan n o  $\cdot$ trat] [tat]: trat m \textbf{5} muose] musse m muͯste n o \textbf{7} ir kradem] tradem n tradam o  $\cdot$ dôz] \textit{om.} o \textbf{8} marhte] marcke o \textbf{9} strît] [stite]: strite o  $\cdot$ der tür] dem tor n \textbf{10} ze] dar zuͦ o  $\cdot$ vür] vor n o \textbf{11} er] dar m \textbf{12} besparte] [bewarte]: besparte n \textbf{13} zuckete] zuͯcke o \textbf{17} turne iht] torneẏ ich o \textbf{19} maget reine] maget rein reine m \textbf{21} bret] breit m \textbf{22} Gawane] gawanen o \textbf{23} îsenînen] isen n (o) \textbf{27} er] Es o \textbf{30} vîenden] venden n \newline
\end{minipage}
\end{table}
\newpage
\begin{table}[ht]
\begin{minipage}[t]{0.5\linewidth}
\small
\begin{center}*G
\end{center}
\begin{tabular}{rl}
 & hie \textbf{der} rîter, dort \textbf{der} koufman,\\ 
 & diu juncvrouwe erhôrte sân\\ 
 & \textbf{einen} povel komen ûz der stat.\\ 
 & mit Gawane si gein dem turne trat.\\ 
5 & ir vriunt muose kumber lîden.\\ 
 & si bat siz dicke mîden.\\ 
 & \textbf{ir kradem unde ir dôz was} \textbf{sô},\\ 
 & daz ir deheiner marhte dô.\\ 
 & durch \textbf{strît} si \textbf{giengen} gein der tür.\\ 
10 & Gawan s\textit{tuont ze wer} dâr vür.\\ 
 & ir îngên er bewarte.\\ 
 & einen rigel, der den turen besparte,\\ 
 & den zucter ûz der mûre.\\ 
 & \textbf{sîne} arge nâchgebûre\\ 
15 & entwichen im dicke mit \textbf{ir} scha\textit{r}.\\ 
 & diu künigîn lief her unde dar,\\ 
 & obe ûf dem turne iht wære ze wer\\ 
 & gein \textbf{disem} ungetriwen her.\\ 
 & dô vant diu \textbf{junge} reine\\ 
20 & ein schâchzabel gesteine\\ 
 & unt ein bret wol erleit, wît.\\ 
 & daz brâhte si Gawane \textbf{in} den strît.\\ 
 & an einem îsenînen ringez hienc,\\ 
 & dâ \textbf{bî} ez Gawan enpfienc.\\ 
25 & ûf \textbf{disen} \textbf{vierecken} schilt\\ 
 & \textbf{wa\textit{s}} schâchzabels vil gespilt.\\ 
 & \textbf{der} wart im sêre \textbf{zerhouwen}.\\ 
 & nû hœrt ouch von der vrouwen.\\ 
 & ez wære künic oder roch,\\ 
30 & daz warf si gein den vîenden doch.\\ 
\end{tabular}
\scriptsize
\line(1,0){75} \newline
G I O L M Q R Z \newline
\line(1,0){75} \newline
\textbf{3} \textit{Initiale} I O L Z   $\cdot$ \textit{Capitulumzeichen} R  \textbf{19} \textit{Initiale} I  \newline
\line(1,0){75} \newline
\textbf{1} \textit{Die Verse 370.13-412.12 fehlen} Q   $\cdot$ hie der] hie die O \textbf{3} einen] Min I ÷er O DEn L (M) (R) (Z)  $\cdot$ komen] chumt I chom O \textbf{4} Gawane] Gawan I O (M) R (Z)  $\cdot$ si] [er]: si G  $\cdot$ gein] zcu M den R \textbf{5} muose] muͤste I mvͦsen O (Z) \textbf{6} bat] bat is M \textbf{7} kradem] [cradem]: tradem L krachin M wuͯffen R  $\cdot$ dôz] dor M \textbf{8} ir] ez ir I (L) (M) Z irz O es R  $\cdot$ marhte] maht gehorn I  $\cdot$ dô] da M \textbf{9} giengen] drvngen O (L) (M) (R) (Z)  $\cdot$ gein] zu R \textbf{10} stuont ze wer] spranch hin vz G  $\cdot$ dâr] her I \textbf{11} ir îngên] Jrn Ingang R \textbf{12} den turen] da R  $\cdot$ besparte] sparte I sprtte R \textbf{13} zucter] zoch ez L zcoich her M zuckt er R  $\cdot$ mûre] Muren M \textbf{14} sîne arge] Sin arger O \textbf{15} ir] der I (Z)  $\cdot$ schar] scharen G \textbf{16} her] hin O \textbf{17} iht wære] wer iht Z \textbf{18} disem] disen I  $\cdot$ ungetriwen] vngetriwem I O \textbf{19} dô] Da O L M Z  $\cdot$ vant] wart R  $\cdot$ junge] magt O (L) (M) (R) Z \textbf{20} schâchzabel gesteine] Schaffzalbel steine R \textbf{21} wol erleit] wol I was er leit M (R) \textbf{22} Gawane] Gawan I O L (R) Z  $\cdot$ in den strît] enzit O an den strit L R \textbf{23} an] Jn O  $\cdot$ îsenînen] isenim I (O) Jsern M \textbf{24} bî] mit O L (M) R Z  $\cdot$ enpfienc] geuienc I \textbf{25} disen] disem I duse M  $\cdot$ vierecken] vieregolten R \textbf{26} was] wart G  $\cdot$ schâchzabels] shahzabel I \textbf{27} wart] was I  $\cdot$ im] in L  $\cdot$ sêre] \textit{om.} R  $\cdot$ zerhouwen] [*]: verhouwen M \textbf{29} wære] werent R \newline
\end{minipage}
\hspace{0.5cm}
\begin{minipage}[t]{0.5\linewidth}
\small
\begin{center}*T
\end{center}
\begin{tabular}{rl}
 & Hie \textbf{der} rîter, dort \textbf{der} koufman,\\ 
 & diu juncvrouwe erhôrte sân\\ 
 & \textbf{den} povel komen ûz der stat.\\ 
 & mit Gawane si gegen dem turne trat.\\ 
5 & ir vriunt muose kumber lîden.\\ 
 & si bat siz dicke mîden.\\ 
 & \textbf{ir krademen unde ir dôz was} \textbf{sô},\\ 
 & daz \textbf{s}ir deheiner marhte dô.\\ 
 & durch \textbf{si} si \textbf{drungen} gegen der tür.\\ 
10 & Gawan stuont ze wer dâr vür.\\ 
 & ir îngân er bewarte.\\ 
 & einen rigel, der den turn besparte,\\ 
 & den zuhter ûz der mûre.\\ 
 & \textbf{si}, argen nâchgebûre,\\ 
15 & entwichen im dicke mit \textbf{ir} schar.\\ 
 & Diu künegîn lief her unde dar,\\ 
 & ob ûf dem turne iht wære ze wer\\ 
 & gegen \textbf{dem} ungetriuwen her.\\ 
 & Dô vant diu \textbf{maget} reine\\ 
20 & ein schâchzabel gesteine\\ 
 & unde ein bret wol erleget, wît.\\ 
 & daz brâhte si Gawane \textbf{an} den strît.\\ 
 & an einem îserînen ringe ez hienc,\\ 
 & dâ \textbf{mit} ez Gawan enpfienc.\\ 
25 & ûf \textbf{disem} \textbf{viereckehtem} schilt\\ 
 & \textbf{was} schâchzabels vil gespilt.\\ 
 & \textbf{der} wart im sêre \textbf{verhouwen}.\\ 
 & Nû hœret ouch von der vrouwen.\\ 
 & ez wære künec oder roch,\\ 
30 & daz warf si gegen den vîenden doch.\\ 
\end{tabular}
\scriptsize
\line(1,0){75} \newline
T U V W \newline
\line(1,0){75} \newline
\textbf{1} \textit{Majuskel} T  \textbf{16} \textit{Majuskel} T  \textbf{19} \textit{Majuskel} T  \textbf{28} \textit{Majuskel} T  \newline
\line(1,0){75} \newline
\textbf{3} den] Vnde den V \textbf{4} gegen] zuͦ W \textbf{5} muose] mvese T \textbf{7} krademen] cradem U (W) komen V \textbf{9} si si] strit sv́ V streit W \textbf{10} vür] vor U \textbf{11} îngân er] in gar U \textbf{13} mûre] mvren V \textbf{14} si] Sin U (W) Sine V  $\cdot$ argen] arge W  $\cdot$ nâchgebûre] nachgebvren V \textbf{15} ir] der W \textbf{17} iht] ir U \textbf{18} dem] disem U V W  $\cdot$ ungetriuwen] vngetreúwem W \textbf{23} îserînen] Jsen U \textbf{25} disem] disen U V W  $\cdot$ viereckehtem] vier ecken U viereckehten V (W) \textbf{26} schâchzabels] schafzovel V \textbf{29} künec] kv́nig alte V \newline
\end{minipage}
\end{table}
\end{document}
