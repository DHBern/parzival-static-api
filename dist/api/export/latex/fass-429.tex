\documentclass[8pt,a4paper,notitlepage]{article}
\usepackage{fullpage}
\usepackage{ulem}
\usepackage{xltxtra}
\usepackage{datetime}
\renewcommand{\dateseparator}{.}
\dmyyyydate
\usepackage{fancyhdr}
\usepackage{ifthen}
\pagestyle{fancy}
\fancyhf{}
\renewcommand{\headrulewidth}{0pt}
\fancyfoot[L]{\ifthenelse{\value{page}=1}{\today, \currenttime{} Uhr}{}}
\begin{document}
\begin{table}[ht]
\begin{minipage}[t]{0.5\linewidth}
\small
\begin{center}*D
\end{center}
\begin{tabular}{rl}
\textbf{429} & \begin{large}D\end{large}â ir swert wâren gehangen,\\ 
 & diu wâren \textbf{in} undergangen,\\ 
 & Gawans knappen, an \textbf{des} \textbf{strîtes} stunt,\\ 
 & daz ir decheiner was worden wunt.\\ 
5 & ein gewaltec man von der stat\\ 
 & in vrides vor den andern bat.\\ 
 & der vienc si unt leite si in prîsûn.\\ 
 & ez wære Franzeys oder Bertun,\\ 
 & starke knappen \textbf{unt} kleiniu kint,\\ 
10 & von \textbf{swelhen landen} si komen sint,\\ 
 & die brâht \textbf{man} \textbf{dô} ledeclîchen\\ 
 & \textbf{Gawane}, \textbf{dem} ellens \textbf{rîchen}.\\ 
 & Dô in diu kint ersâhen,\\ 
 & dâ wart grôz umbevâhen.\\ 
15 & ieslîchez \textbf{sich} weinende an \textbf{in} hienc.\\ 
 & daz weinen \textbf{iedoch} \textbf{von} liebe \textbf{ergienc}.\\ 
 & von Curnewals mit im dâ was\\ 
 & cons \textbf{Layz} fiz Tynas.\\ 
 & ein edel kint wont im \textbf{ouch} bî:\\ 
20 & duc \textbf{Gandiluz} fiz Gurzgri,\\ 
 & der durch Schoydelakurt den lîp verlôs,\\ 
 & dâ manec vrouwe ir \textbf{jâmer} \textbf{kôs}.\\ 
 & Liaze was des kindes base.\\ 
 & sîn munt, sîn ougen unt sîn nase\\ 
25 & was reht der minne kern.\\ 
 & al diu werlt sach in gern,\\ 
 & dar zuo sehs ander kindelîn.\\ 
 & \textbf{dise} aht junchêrren sîn\\ 
 & wâren \textbf{\textit{g}e\textit{b}ürte} des bewart,\\ 
30 & elliu von \textbf{edeler}, hôhen art.\\ 
\end{tabular}
\scriptsize
\line(1,0){75} \newline
D Fr1 Fr5 Fr68 \newline
\line(1,0){75} \newline
\textbf{1} \textit{Initiale} D Fr1 Fr5  \textbf{13} \textit{Initiale} Fr68   $\cdot$ \textit{Majuskel} D  \newline
\line(1,0){75} \newline
\textbf{1} Dâ] DA in Fr5 \textbf{2} diu] di Fr68  $\cdot$ in] \textit{om.} Fr68 \textbf{3} Gawans] Gauwans Fr5  $\cdot$ an] \textit{om.} Fr5 \textbf{4} ir] in Fr5 \textbf{6} in] Der in Fr5 (Fr68)  $\cdot$ vor] von Fr5 \textbf{7} leite] leit Fr5 \textbf{8} ez] Er Fr5  $\cdot$ Franzeys] Franzeis D franzois Fr68  $\cdot$ Bertun] prittvn Fr5 birtun Fr68 \textbf{9} \textit{Verse 429.9-13 kontrahiert zu:} Starche knappin vnd cleiniv kint irsahin Fr5   $\cdot$ unt kleiniu] oder cleine Fr68 \textbf{10} swelhen landen] swelhem lande Fr68 \textbf{11} dô] \textit{om.} Fr68 \textbf{12} dem] den Fr68 \textbf{13} diu] di Fr68 \textbf{14} dâ] do Fr68 \textbf{16} iedoch] doh Fr68  $\cdot$ von] vor Fr5 \textbf{17} Curnewals] kurniwal Fr5 kurnewals Fr68 \textbf{18} Cons Laŷz fiz Tynâs D Conslaiz fiztinas Fr5 conslaiz fiz týnas Fr68 \textbf{19} wont im] [vont]: wont im Fr5 wonetim Fr68 \textbf{20} duc] \textit{om.} Fr68  $\cdot$ Gandiluz] Gandilvz D gandeluz Fr68  $\cdot$ Gurzgri] Gvrzgrî D gorzgri Fr68 \textbf{21} Schoydelakurt] scoydelakvrt D schordelachur Fr5 dioý del kurt Fr68 \textbf{23} Liaze] Lŷaze D Lias Fr5 \textbf{24} unt] \textit{om.} Fr5 \textbf{25} minne] minnen Fr68 \textbf{26} diu] di Fr68 \textbf{27} zuo] \textit{om.} Fr68 \textbf{29} gebürte] begvrte D \textbf{30} hôhen] hoer Fr68 \newline
\end{minipage}
\hspace{0.5cm}
\begin{minipage}[t]{0.5\linewidth}
\small
\begin{center}*m
\end{center}
\begin{tabular}{rl}
 & \begin{large}D\end{large}\textit{â} ir swert wâren gehangen,\\ 
 & diu wâren undergangen\\ 
 & Gawanes knappen an \textbf{der} \textbf{selben} stunt,\\ 
 & daz ir dekeiner was worden wunt.\\ 
5 & ein gewaltec man von der stat,\\ 
 & \textbf{der} in vrides vor den andern bat,\\ 
 & der vienc si und leit si in prîsûn.\\ 
 & ez wære Franzois oder Br\textit{i}tu\textit{n},\\ 
 & starke knappen \textbf{oder} kleiniu kint,\\ 
10 & von \textbf{welh\textit{e}m lande} si komen sint,\\ 
 & die brâhte \textbf{man} \textbf{dô} lediclîchen\\ 
 & \textbf{Gawane}, \textbf{dem} ellens \textbf{rîchen}.\\ 
 & dô in diu kint ersâhen,\\ 
 & dô wart grôz umbevâhen.\\ 
15 & ieglîchez weinende an \dag in\dag  hienc.\\ 
 & daz weinen \textbf{doch} \textbf{von} liebe \textbf{gienc}.\\ 
 & von Kurnewal mit ime d\textit{â} was\\ 
 & cons \textbf{Lan} fi\textit{z} Tinas.\\ 
 & ein edelkint wont ime bî:\\ 
20 & du\textit{c} \textbf{Gandilirz} fi\textit{z} Gur\textit{z}gr\textit{i},\\ 
 & der durch Schoid\textit{e}la\textit{cur}t den lîp verlôs,\\ 
 & dâ manic vrouwe ir \textbf{jâmer} \textbf{kôs}.\\ 
 & Liaze was des kindes base.\\ 
 & sîn munt, sîn ougen und sîn nase\\ 
25 & was rehte der minnen kerne.\\ 
 & aldiu werlt sach in gerne,\\ 
 & dar zuo sehs ander kindelîn.\\ 
 & \textbf{dise} ahte junchêrren sîn\\ 
 & wâren \textbf{gebürte} des bewart,\\ 
30 & elliu von \textbf{edeler}, hôhen art.\\ 
\end{tabular}
\scriptsize
\line(1,0){75} \newline
m n o \newline
\line(1,0){75} \newline
\textbf{1} \textit{Initiale} m   $\cdot$ \textit{Capitulumzeichen} n  \newline
\line(1,0){75} \newline
\textbf{1} dâ] DO m (n) (o) \textbf{3} Gawanes] Gawans n o  $\cdot$ selben] \textit{om.} n o \textbf{4} dekeiner] do keiner n  $\cdot$ worden] wuͯrden o \textbf{5} von] vor n \textbf{7} leit si] leite o \textbf{8} Franzois] frantzois n franczois o  $\cdot$ Britun] brutuͯm m brituͦn n britymm o \textbf{10} welhem] welhelim m \textbf{12} Gawane] Gawan n Gawann o \textbf{14} dô] Da o \textbf{17} Kurnewal] kornewal n karnewal o  $\cdot$ dâ] do m n o \textbf{18} cons Lan] Coms lan m Comslaur n Comslair o  $\cdot$ fiz] fir m o fúr n \textbf{19} ime] jme ouch n (o) \textbf{20} duc] Duo m Due n Guwe o  $\cdot$ Gandilirz] gaudulurz n ganduliersz o  $\cdot$ fiz] fir m o fúr n  $\cdot$ Gurzgri] gurigrẏ m gurgri n guͯrczgri o \textbf{21} Schoidelacurt] scoidilatrut m scoidelacurt n scoidelatuͯrt o \textbf{22} dâ] Do n o \textbf{23} Liaze] Lẏas n Lias o  $\cdot$ des] das o \textbf{25} kerne] kere o \textbf{28} dise] Die o \textbf{30} elliu] Aller o  $\cdot$ hôhen] hoher n o \newline
\end{minipage}
\end{table}
\newpage
\begin{table}[ht]
\begin{minipage}[t]{0.5\linewidth}
\small
\begin{center}*G
\end{center}
\begin{tabular}{rl}
 & dâ ir swert wâren gehangen,\\ 
 & diu wâren \textbf{in} undergangen,\\ 
 & Gawanes knappen, an \textbf{der} \textbf{strîtes} stunt,\\ 
 & daz ir deheiner was worden wunt.\\ 
5 & ein gewaltic man von der stat,\\ 
 & \textbf{der} in vrides vor den anderen bat,\\ 
 & der vie si unde leit si in prîsûn.\\ 
 & ez wære Franzoys ode Britun,\\ 
 & starke knappen \textbf{unde} kleiniu kint,\\ 
10 & von \textbf{swelhem lande} si komen sint,\\ 
 & die brâhte \textbf{man} lediclîchen\\ 
 & \textbf{Gawane}, \textbf{dem} ellens \textbf{rîchen}.\\ 
 & dô in diu kint ersâhen,\\ 
 & dô wart grôz umbevâhen.\\ 
15 & etslîchez \textbf{sich} weinende an \textbf{in} hienc.\\ 
 & daz weinen \textbf{doch} \textbf{von} liebe \textbf{ergienc}.\\ 
 & von Kurnewals mit im dâ was\\ 
 & cons \textbf{Liaz} fiz Tinas.\\ 
 & ein edel kint wont im \textbf{ouch} bî:\\ 
20 & duc \textbf{Gandiluz} fiz Gurzgri,\\ 
 & der durch Tschoidelakurt den lîp verlôs,\\ 
 & dâ manic vrouwe ir \textbf{jâmer} \textbf{kôs}.\\ 
 & Liaze was des kindes base.\\ 
 & sîn munt, sîn ougen unde sîn nase\\ 
25 & wa\textit{s} \textit{r}eht der minn\textit{e} kern.\\ 
 & aldiu werlt sach in gern,\\ 
 & dar zuo sehs anderiu kindelîn.\\ 
 & \textbf{die} ahte junchêrren sîn\\ 
 & wâren \textbf{gebürte} des bewart,\\ 
30 & elliu von \textbf{edel} hôher art.\\ 
\end{tabular}
\scriptsize
\line(1,0){75} \newline
G I O L M Q R Z Fr21 \newline
\line(1,0){75} \newline
\textbf{1} \textit{Initiale} I O L Z Fr21   $\cdot$ \textit{Capitulumzeichen} R  \textbf{17} \textit{Initiale} I  \newline
\line(1,0){75} \newline
\textbf{1} dâ] ÷a O Das M Do Q  $\cdot$ ir] \textit{om.} L \textbf{2} wâren] wurden I wairent R \textbf{3} Gawanes] [gwanes]: gawanes G Gawans I O (M) Z Fr21 Gawanz L Dawans Q  $\cdot$ an der] an des I (O) (M) (Z) (Fr21) an L  $\cdot$ strîtes] \textit{om.} Q R \textbf{4} ir] \textit{om.} Fr21  $\cdot$ deheiner] dehaim I  $\cdot$ wunt] chunt I \textbf{5} von] vor L \textbf{6} in] \textit{om.} Fr21  $\cdot$ den] dem R \textbf{7} leit si] legtes Q (R) (Fr21)  $\cdot$ in] in eyne M \textbf{8} Franzoys] franzieis G fronzoys I frantzoýs L franzois M Z frantzoys Q fracyois R  $\cdot$ Britun] pritun G I Brittvn L (Q) Bertvn Fr21 \textbf{9} starke] [Starcher]: Starche Fr21  $\cdot$ kleiniu] cleine R \textbf{10} swelhem lande] welhem lande I swelhen landen O (M) welchen landen L (Q) (R) (Z) \textbf{11} man] man do O L (Q) R Fr21 da M man da Z \textbf{12} Gawane] Gawan I O L (M) Z Gawainen R  $\cdot$ dem] den L Z  $\cdot$ ellens rîchen] ellens richens M eren reiche Q [enl]: ellentrichen R \textbf{13} dô] Da M Z  $\cdot$ ersâhen] sahen L \textbf{14} dô] Da M Z  $\cdot$ wart] wart da I was R \textbf{15} sich] \textit{om.} R  $\cdot$ an in] sich an in O an sich Fr21 \textbf{16} doch] idoch O (Q) (Z) (Fr21)  $\cdot$ von liebe] vor leide O Fr21 vor liebe L (M) (Q) von trúwen R  $\cdot$ ergienc] gieng R \textbf{17} Kurnewals] chunewalisch I Kvrnwals O (Fr21) kvrnawals L curriwals Q Cornuvals R kvrnuwals Z  $\cdot$ mit im dâ] mit im I mit ym do Q do mit Jm R da mit im Z \textbf{18} cons Liaz] conslias I Cons Laiz O Z (Fr21) Kons Lars L Conslarz M Rons larz Q Kons larz R  $\cdot$ fiz Tinas] freztinas M fiz tynas Q R \textbf{19} im ouch] im I uch M \textbf{20} duc] Dvrch O Da M Diyc Q Herczog R  $\cdot$ Gandiluz] kandilus I Gandilaz L kandiuͯsz M gandilus Q Candilvz Fr21  $\cdot$ Gurzgri] Gurz gri I Gvrzgrý L guszgry M gurczgri Q \textbf{21} Tschoidelakurt] tschoidelahgurt G shoy de largut I scheidenlacvrt O schoýe delakvrt L sy delaent M soidelakurt Q Shoidelakuͯrt R Schoidelakvrt Z Scheidelacvrt Fr21 \textbf{22} dâ] Do Q  $\cdot$ jâmer] leit O L Q (R) Fr21  $\cdot$ kôs] an kos O erkos L (Q) R Fr21 \textbf{23} Liaze] lise I Lẏaz O Liasz M Lyaze Q R  $\cdot$ des] de Fr21 \textbf{24} sîn] Sint Z  $\cdot$ unde] \textit{om.} L R  $\cdot$ nase] nasen Q base Fr21 \textbf{25} was zereht der minnen chern G  $\cdot$ kern] kere Q \textbf{26} aldiu] Alle div O Alle R \textbf{27} dar] [Daz]: Dar L  $\cdot$ sehs anderiu] ander R \textbf{28} die] Dis O Fr21 (L) (M) (Q) (Z)  $\cdot$ ahte] sechs R  $\cdot$ junchêrren sîn] ivnchfrowen sin O ivncherrelin Fr21 \textbf{29} wâren] warn von I (R) Wan L  $\cdot$ gebürte] hart M \textbf{30} elliu] Alle O Fr21  $\cdot$ edel hôher] edeler hoher I (O) (L) (Fr21) ediln habin M hoer edeler Q Edler R (Z) \newline
\end{minipage}
\hspace{0.5cm}
\begin{minipage}[t]{0.5\linewidth}
\small
\begin{center}*T
\end{center}
\begin{tabular}{rl}
 & Dâ ir swert wâren gehangen,\\ 
 & di\textit{u} wâre\textit{n} \textbf{in} undergangen,\\ 
 & Gawans knappen, an \textbf{der} stunt,\\ 
 & daz ir deheiner was worden wunt.\\ 
5 & Ein gewaltic man von der stat,\\ 
 & \textbf{der} in vrides vor den anderen bat,\\ 
 & der vienc si unde legete si in prîsûn.\\ 
 & ez wære Franzoys oder Britun,\\ 
 & starke knappen \textbf{unde} kleiniu kint,\\ 
10 & von \textbf{swelhen landen} si komen sint,\\ 
 & die brâhte \textbf{dô} ledeclîche\\ 
 & \textbf{Gawan}, \textbf{der} ellens \textbf{rîche}.\\ 
 & dô in diu kint ersâhen,\\ 
 & dâ wart grôz umbevâhen.\\ 
15 & iegeslîchez weinde an \textbf{im} hienc.\\ 
 & daz weinen \textbf{iedoch} \textbf{vor} liebe \textbf{ergienc}.\\ 
 & von Curnewale mit im dâ was\\ 
 & cons \textbf{Larz} fiz Tynas.\\ 
 & ein edel kint wont im \textbf{ouch} bî:\\ 
20 & duc \textbf{Gandlus} fiz Gurzgri,\\ 
 & der durch Schoydelakurt den lîp verlôs,\\ 
 & dâ manec vrouwe ir \textbf{leit} \textbf{erkôs}.\\ 
 & Lyaze was des kindes base.\\ 
 & sîn munt, sîn ougen unde sîn nase\\ 
25 & was rehte der minnen kerne.\\ 
 & aldiu werlt sach \textit{in} gerne,\\ 
 & dar zuo sehs ander kindelîn.\\ 
 & \textbf{dise} ahte junchêrren sîn\\ 
 & wâren \textbf{von geburt} des bewart,\\ 
30 & alle von hôher art.\\ 
\end{tabular}
\scriptsize
\line(1,0){75} \newline
T U V W \newline
\line(1,0){75} \newline
\textbf{1} \textit{Initiale} V W   $\cdot$ \textit{Majuskel} T  \textbf{5} \textit{Majuskel} T  \newline
\line(1,0){75} \newline
\textbf{1} Dâ] Do U V (W) \textbf{2} diu wâren] die waret T \textbf{3} an der stunt] an der [*]: selben stunt V \textbf{5} von] vor W \textbf{7} legete si in] legetez in U leit es in die W \textbf{8} Franzoys] franzôys T franzoẏs V frantzoys W  $\cdot$ oder] [*]: oder V  $\cdot$ Britun] Britv̂n T Brituͦn U brittvn V \textbf{10} swelhen] [welchem]: welchen U swelme V welchen W  $\cdot$ landen] lande V \textbf{11} dô] men do V (W) \textbf{12} Gawan] Gawane W  $\cdot$ der ellens rîche] dem ellens richen V dem reichen W \textbf{14} dâ] Do U V W \textbf{15} \textit{Die Verse 429.15-16 fehlen} U   $\cdot$ weinde] sich weinende V  $\cdot$ im] in W \textbf{16} vor] von V  $\cdot$ liebe] leid W \textbf{17} Curnewale] Cuͦrnewale U [kornuwal*]: kornuwal V korniwals W  $\cdot$ dâ] do U V W \textbf{18} Cons] Graue V  $\cdot$ Larz] lars V  $\cdot$ fiz] svn von V  $\cdot$ Tynas] tẏnas V \textbf{19} wont im] im wont V \textbf{20} duc] Duͦt U Herzoge V  $\cdot$ Gandlus] Gandilus U (V) (W)  $\cdot$ fiz] svn V  $\cdot$ Gurzgri] Gvrzgrî T gursgri W \textbf{21} Schoydelakurt] Schoidelakvrt T Scheidelakuͦrt U schodelakvrt V soidelakurt W \textbf{22} dâ] Do U V W  $\cdot$ leit erkôs] [*]: leit kos V \textbf{23} Lyaze] Lyase V Lyaße W \textbf{26} in] si T \textbf{28} dise] [die]: dise T [D*]: Dise V \textbf{29} von] \textit{om.} U V W \textbf{30} alle] Alles V  $\cdot$ hôher] edeler hoher U W ediler [hoh*]: hoher V \newline
\end{minipage}
\end{table}
\end{document}
