\documentclass[8pt,a4paper,notitlepage]{article}
\usepackage{fullpage}
\usepackage{ulem}
\usepackage{xltxtra}
\usepackage{datetime}
\renewcommand{\dateseparator}{.}
\dmyyyydate
\usepackage{fancyhdr}
\usepackage{ifthen}
\pagestyle{fancy}
\fancyhf{}
\renewcommand{\headrulewidth}{0pt}
\fancyfoot[L]{\ifthenelse{\value{page}=1}{\today, \currenttime{} Uhr}{}}
\begin{document}
\begin{table}[ht]
\begin{minipage}[t]{0.5\linewidth}
\small
\begin{center}*D
\end{center}
\begin{tabular}{rl}
\textbf{207} & hilf mir durch dîne werdecheit\\ 
 & Cunnewaren hulde umbe krapfen breit."\\ 
 & er \textbf{bôt} ir anders wandels niht.\\ 
 & die rede lât sîn! \textbf{hœret}, waz geschiht,\\ 
5 & dâ wir \textbf{diz} mære liezen ê.\\ 
 & vür Pelrapeire kom Clamide.\\ 
 & dâ\textbf{ne} \textbf{wart grôz stürmen niht} vermiten.\\ 
 & die inren mit den ûzern striten.\\ 
 & \textit{\begin{large}S\end{large}}i heten trôst unt kraft.\\ 
10 & man vant die helde werhaft.\\ 
 & dâ \textbf{von} behabten si daz wal.\\ 
 & ir landes hêrre, Parzival,\\ 
 & streit den sînen verre vor.\\ 
 & dâ stuonden offen \textbf{gar} diu tor.\\ 
15 & mit slegen er die arme erswanc.\\ 
 & \textbf{sîn} swert durch herte helme erklanc.\\ 
 & Swaz er \textbf{dâ} ritter nider sluoc,\\ 
 & die \textbf{vunden} arbeit genuoc.\\ 
 & die kunde man si lêren\\ 
20 & \textbf{zer} halsberge gêren.\\ 
 & die burgære tâten râche schîn.\\ 
 & si \textbf{erstâchen} si zen slitzen în.\\ 
 & Parzival in werte daz.\\ 
 & dô si drumbe \textbf{erhôrten} sînen haz,\\ 
25 & zweinzec si ir \textbf{lebendic} \textbf{geviengen},\\ 
 & ê si \textbf{von} \textbf{strîte} giengen.\\ 
 & Parzival wart wol gewar,\\ 
 & daz Clamide mit sîner schar\\ 
 & rîterschaft ze\textbf{n} porten meit\\ 
30 & unt daz er \textbf{anderthalben} streit.\\ 
\end{tabular}
\scriptsize
\line(1,0){75} \newline
D \newline
\line(1,0){75} \newline
\textbf{9} \textit{Initiale} D  \textbf{17} \textit{Majuskel} D  \newline
\line(1,0){75} \newline
\textbf{6} Clamide] Chlamidê D \textbf{9} Si] ÷i \textit{nachträglich korrigiert zu:} Si D \textbf{28} Clamide] Chlamide D \newline
\end{minipage}
\hspace{0.5cm}
\begin{minipage}[t]{0.5\linewidth}
\small
\begin{center}*m
\end{center}
\begin{tabular}{rl}
 & hilf mir durch dîne wirdicheit\\ 
 & C\textit{u}nn\textit{e}w\textit{a}r\textit{e}n hulde umbe krapfen breit."\\ 
 & er \textbf{bôt} ir anders wandels niht.\\ 
 & die rede lât sî\textit{n}! \textbf{hœret}, waz geschiht,\\ 
5 & dâ wir \textbf{daz} mære liezen ê.\\ 
 & vür Pelraperie kam Clamide.\\ 
 & d\textit{â} \textbf{wart grôz stürmen niht} vermiten.\\ 
 & \textit{d}ie \textit{inn}eren mit den ûzern striten.\\ 
 & si heten trôst und kraft.\\ 
10 & man vant die helde werhaft.\\ 
 & dâ \textbf{von} behabeten si daz \textit{w}al.\\ 
 & ir landes hêrre, Parcifal,\\ 
 & streit den sînen verre vor.\\ 
 & d\textit{â} stuonden offen \textbf{gar} diu tor.\\ 
15 & mit slegen er die arme erswanc.\\ 
 & \textbf{sîn} swert durch herte helme erklanc.\\ 
 & waz er \textbf{dâ} ritter nider sluoc,\\ 
 & die \textbf{vunden} arbeite genuoc.\\ 
 & die kunde man s\textit{i} lêren\\ 
20 & \textbf{zer} halsberge gêren.\\ 
 & die burgære tâten râche schîn.\\ 
 & si \textbf{erstâchen} si zuo de\textit{n} slitzen în.\\ 
 & Parcifal in werte daz.\\ 
 & dô si dar umbe \textbf{erh\textit{ô}rten} sînen haz,\\ 
25 & zweinzic si ir \textbf{lebende} \textbf{viengen},\\ 
 & ê si \textbf{vonme} \textbf{strîte} giengen.\\ 
 & Parcifal wart wol gewar,\\ 
 & daz Clamide mit sîner schar\\ 
 & ritterschaft ze\textbf{n} porten meit\\ 
30 & und daz er \textbf{an der halden} streit.\\ 
\end{tabular}
\scriptsize
\line(1,0){75} \newline
m n o Fr69 \newline
\line(1,0){75} \newline
\newline
\line(1,0){75} \newline
\textbf{1} durch] vmb n \textbf{2} Cunnewaren] Connnweran m Connewaren n Konde waren o  $\cdot$ breit] brot o \textbf{4} rede] rode o  $\cdot$ sîn] si m \textit{om.} Fr69  $\cdot$ geschiht] beschẏht o \textbf{5} dâ] Do n o  $\cdot$ liezen] laͦssent o \textbf{6} Pelraperie] pelrapeir n pelrapier o  $\cdot$ Clamide] klamide m \textbf{7} dâ] Do m n o \textbf{8} die inneren] Si meren m \textbf{10} helde werhaft] helle verhafft o \textbf{11} wal] mol m \textbf{13} vor] fur o \textbf{14} dâ] Do m n o \textbf{16} erklanc] dranc Fr69 \textbf{17} waz] Swaz Fr69  $\cdot$ dâ] do n o \textbf{18} vunden] fuͯnden o  $\cdot$ arbeite] do arbeit n arbeiten o \textbf{19} kunde] kuͯnde o  $\cdot$ si] sich m \textbf{20} halsberge] halspergen m (o) halbergen n  $\cdot$ gêren] sú keren n \textbf{22} den] dem m \textbf{23} in] er n der o \textbf{24} erhôrten] erherten m \textbf{29} zen] zuͯm o \textbf{30} an] in o \newline
\end{minipage}
\end{table}
\newpage
\begin{table}[ht]
\begin{minipage}[t]{0.5\linewidth}
\small
\begin{center}*G
\end{center}
\begin{tabular}{rl}
 & hilf mir durch dîne werdicheit\\ 
 & Kunewaren hulde umbe krapfen breit."\\ 
 & er \textbf{bôt} ir anders wandels niht.\\ 
 & die rede lât sîn! \textbf{hœret}, waz geschiht,\\ 
5 & dâ wir \textbf{daz} mære liezen ê.\\ 
 & vür Pelrapeire kom Clamide.\\ 
 & dâ \textbf{grôz sturm niht wart} vermiten.\\ 
 & die inneren mit den ûzeren striten.\\ 
 & si heten trôst unde kraft.\\ 
10 & man vant die helde werhaft.\\ 
 & \begin{large}D\end{large}â \textbf{vor} behabten si daz wal.\\ 
 & ir landes hêrre, Parzival,\\ 
 & streit den sînen verre vor.\\ 
 & dâ stuonden offen \textbf{in} diu tor.\\ 
15 & mit slegen er die arme erswanc.\\ 
 & \textbf{daz} swert durch \textit{h}e\textit{r}te helme erklanc.\\ 
 & swaz er \textbf{der} rîter nider sluoc,\\ 
 & die \textbf{gewunnen} arbeit genuoc.\\ 
 & die kunde man si lêren\\ 
20 & \textbf{zer} halsberge gêren.\\ 
 & die burgære tâten râche schîn.\\ 
 & si \textbf{erstâchen} si zen slitzen în.\\ 
 & Parzival in werte daz.\\ 
 & dô si drumbe \textbf{erhôrten} sînen haz,\\ 
25 & zweinzic sir \textbf{lebend} \textbf{viengen},\\ 
 & ê si \textbf{vome} \textbf{sturme} giengen.\\ 
 & Parzival wart wol gewar,\\ 
 & daz Clamide mit sîner schar\\ 
 & rîterschaft ze\textbf{r} porten meit\\ 
30 & unt daz er \textbf{anderhalben} streit.\\ 
\end{tabular}
\scriptsize
\line(1,0){75} \newline
G I O L M Q R Z Fr21 \newline
\line(1,0){75} \newline
\textbf{5} \textit{Initiale} R  \textbf{11} \textit{Initiale} G  \textbf{13} \textit{Initiale} I  \textbf{17} \textit{Initiale} Z  \textbf{19} \textit{Initiale} M  \textbf{27} \textit{Initiale} L  \textbf{29} \textit{Initiale} R  \newline
\line(1,0){75} \newline
\textbf{2} Kunewaren] chunwarn I Kvnwaren O Fr21 Cvnewaren L Kunuaren M Czun waren Q Cuͦnwaren R Cvnnewaren Z \textbf{3} er] Ern O Q (R) (Z) Fr21 Her yn M \textbf{4} die rede] hort I Die M  $\cdot$ sîn hœret] sin I O Fr21 horent L sin so M  $\cdot$ waz] was noch M \textbf{5} dâ] Do Q  $\cdot$ daz] diz L (M) (Q) (R) (Z) \textbf{6} vür] fu R  $\cdot$ Pelrapeire] pailrapier I pelraperie M welarapere R pelrapeir Z peilrapeir Fr21  $\cdot$ Clamide] Glamide O \textbf{7} dâ] Do O Fr21 Do nicht Q  $\cdot$ grôz] grozzez O Z  $\cdot$ sturm] stvͦrmen O (M) (Q) (R) (Z)  $\cdot$ niht] \textit{om.} Q \textbf{10} die] da R  $\cdot$ werhaft] warhafft M werschafftt R \textbf{11} vor] von L Z  $\cdot$ behabten] hielten I hatten M behalttent R \textbf{12} landes] \textit{om.} R  $\cdot$ Parzival] Parzifal I (M) Parcifal O L (Z) (Fr21) partzifal Q barczifal R \textbf{14} dâ] do I (O) (Q) (R) (Fr21)  $\cdot$ in] gar O L M Q R Z Fr21 \textbf{15} arme] armen Q  $\cdot$ erswanc] swanc Q Z \textbf{16} daz] Sin Z  $\cdot$ herte] liehte G di herte O  $\cdot$ erklanc] dranck Q (R) (Z) \textbf{17} swaz] Waz L (Q) (R)  $\cdot$ der] da O L (M) Z Fr21 do Q R  $\cdot$ rîter nider] nider ritter O M (Fr21) \textbf{18} gewunnen] fvnden O L (M) (Q) (R) (Z) Fr21 \textbf{19} kunde] begvnde L \textbf{20} zer halsberge] zuͤ den halspergen I \textbf{21} râche] rachin M (Q) \textbf{22} erstâchen] stachen I O L (Q) Fr21  $\cdot$ zen] zv Z \textbf{23} Parzival] Parzifal I L M Parcifal O Z Fr21 Partzifal Q Parczifal R  $\cdot$ in werte] wert in I in wert O der werte in Z \textbf{24} dô] Da M Z Das R  $\cdot$ erhôrten] horten I heten O \textbf{25} sir] si O M (Z) Fr21 sie do L sie irer Q  $\cdot$ lebend] lemtich O (M) (Q) (R) (Z) (Fr21)  $\cdot$ viengen] Geviengen I \textbf{26} sturme] strite O L M (Q) (R) Z Fr21 \textbf{27} Parzival] Parzifal I (L) M Parcifal O Z Fr21 Partzifal Q Parczifal R \textbf{28} Clamide] Glamide O \textbf{29} zer] zden O (L) (M) (R) (Z) (Fr21)  $\cdot$ porten] phorten M (Q) \textbf{30} unt] [Wan]: Vnde Fr21  $\cdot$ anderhalben] ander halden O (L) (Q) Fr21 ander helden R  $\cdot$ streit] [steit]: streit G meit Fr21 \newline
\end{minipage}
\hspace{0.5cm}
\begin{minipage}[t]{0.5\linewidth}
\small
\begin{center}*T
\end{center}
\begin{tabular}{rl}
 & hilf mir durch dîne werdecheit\\ 
 & Kunnewaren hulde umbe krapfen breit."\\ 
 & er \textbf{enbôt} ir anders wandels niht.\\ 
 & Die rede lât sîn! \textbf{hœre}, waz geschiht,\\ 
5 & d\textit{â} wir \textbf{daz} mære liezen ê.\\ 
 & vür Peilrapere kom Clamide.\\ 
 & dâ\textbf{ne} \textbf{wart grôz stürmen niht} vermiten.\\ 
 & die innern mit den ûzern striten.\\ 
 & si heten trôst unde kraft.\\ 
10 & man vant die helde werhaft.\\ 
 & dâ \textbf{von} behabeten si daz wal.\\ 
 & ir landes hêrre, Parcifal,\\ 
 & streit den sînen verre vor.\\ 
 & dâ stuonden offen \textbf{gar} diu tor.\\ 
15 & mit slegen er die arme erswanc.\\ 
 & \textbf{sîn} swert durch herte helme erklanc.\\ 
 & swaz er \textbf{der} rîter nider sluoc,\\ 
 & die \textbf{gewunnen} arbeite genuoc.\\ 
 & die kunde man si lêren\\ 
20 & \textbf{zir} halsberge gêren.\\ 
 & die burgære tâten râche schîn.\\ 
 & si \textbf{stâchen} si zen slitzen în.\\ 
 & Parcifal in werte daz.\\ 
 & dô si drumbe \textbf{hôrten} sînen haz,\\ 
25 & zweinzic sir \textbf{lebendic} \textbf{viengen},\\ 
 & ê si \textbf{vomme} \textbf{strîte} gie\textit{n}gen.\\ 
 & \begin{large}P\end{large}arcifal wart wol gewar,\\ 
 & daz Clamide mit sîner schar\\ 
 & \textbf{die} rîterschaft ze\textbf{n} porten meit\\ 
30 & unde daz er \textbf{anderhalben} streit.\\ 
\end{tabular}
\scriptsize
\line(1,0){75} \newline
T U V W \newline
\line(1,0){75} \newline
\textbf{4} \textit{Majuskel} T  \textbf{27} \textit{Initiale} T U V W  \newline
\line(1,0){75} \newline
\textbf{2} Kunnewaren] Kunnewarn W \textbf{3} enbôt] bot V  $\cdot$ ir] vns W \textbf{4} sîn] \textit{om.} W  $\cdot$ hœre] horet U (V) hoͤren W \textbf{5} \textit{nach 207.5:} Daz wil ich v́ch vntsliezzen V   $\cdot$ dâ] daz T Do U W [Daz]: Da V  $\cdot$ liezen ê] e liezen V \textbf{6} \textit{nach 207.6:} Den lv́ten wolt er schaffen we V   $\cdot$ Peilrapere] [*]: belrepere V pelrapier W  $\cdot$ Clamide] klamide W \textbf{7} dâne] Do W  $\cdot$ niht] nie W \textbf{8} mit den] vnd die W \textbf{11} daz] die W \textbf{12} Parcifal] parzifal V partzifal W \textbf{14} dâ] Do U V W  $\cdot$ offen gar] gar offen W \textbf{15} erswanc] swang V verswang W \textbf{16} herte] die harte U  $\cdot$ erklanc] clang V (W) \textbf{17} swaz] Waz U (W)  $\cdot$ der] do W \textbf{18} die] Sy W \textbf{19} die] [D*]: Do U \textbf{20} zir] Zuͦ dem U Zuͦ der W \textbf{22} zen slitzen] zuͦ slitzeten U \textbf{23} Parcifal] Parzifal V Partzifal W \textbf{25} lebendic] \textit{om.} W \textbf{26} giengen] giegen T \textbf{27} Parcifal] Parzifal V PArtzifal W \textbf{28} Clamide] klamide W \textbf{30} anderhalben] [*]: ander halden V \newline
\end{minipage}
\end{table}
\end{document}
