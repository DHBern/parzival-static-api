\documentclass[8pt,a4paper,notitlepage]{article}
\usepackage{fullpage}
\usepackage{ulem}
\usepackage{xltxtra}
\usepackage{datetime}
\renewcommand{\dateseparator}{.}
\dmyyyydate
\usepackage{fancyhdr}
\usepackage{ifthen}
\pagestyle{fancy}
\fancyhf{}
\renewcommand{\headrulewidth}{0pt}
\fancyfoot[L]{\ifthenelse{\value{page}=1}{\today, \currenttime{} Uhr}{}}
\begin{document}
\begin{table}[ht]
\begin{minipage}[t]{0.5\linewidth}
\small
\begin{center}*D
\end{center}
\begin{tabular}{rl}
\textbf{699} & \begin{large}D\end{large}er Waleis zArtuse sprach:\\ 
 & "hêrre, dô ich iuch jungest sach,\\ 
 & dô wart ûf \textbf{die} êre mir gerant.\\ 
 & von prîse ich gap sô hôhiu pfant,\\ 
5 & daz ich \textbf{von prîse nâch was} komen.\\ 
 & nû hân ich, \textbf{hêrre}, \textbf{von} iu vernomen,\\ 
 & ob ir mirz saget âne vâr,\\ 
 & daz prîs \textbf{an mir ein teil} hât wâr.\\ 
 & swie unsanfte ich daz \textbf{lerne},\\ 
10 & ich \textbf{geloube}z \textbf{iu doch} gerne,\\ 
 & woldez gelouben ander diet,\\ 
 & von den ich mich \textbf{dô} schamende schiet."\\ 
 & Die dâ sâzen, jâhen sîner hant,\\ 
 & si hete den prîs über \textbf{menegiu} lant\\ 
15 & mit sô hôhem prîse erworben,\\ 
 & daz \textbf{sîn prîs} wære unverdorben.\\ 
 & Der herzoginne rîter gar\\ 
 & \textbf{ouch} \textbf{kom}, dâ der wol gevar\\ 
 & Parzival bî Artuse saz.\\ 
20 & der werde künec \textbf{des} niht vergaz,\\ 
 & er \textbf{enpfienge} \textbf{si} in des wirtes hûs,\\ 
 & der \textbf{höfsche}, \textbf{wîse} Artus,\\ 
 & Swie wît wære Gawans gezelt,\\ 
 & er saz dar vür ûfez velt;\\ 
25 & si sâzen \textbf{umb in} an den rinc.\\ 
 & sich samneten unkundiu dinc.\\ 
 & wer \textbf{dirre unt jener} wære,\\ 
 & daz würden wîtiu mære,\\ 
 & solt der kristen unt der Sarrazin\\ 
30 & kuntlîche dâ genennet sîn.\\ 
\end{tabular}
\scriptsize
\line(1,0){75} \newline
D \newline
\line(1,0){75} \newline
\textbf{1} \textit{Initiale} D  \textbf{13} \textit{Majuskel} D  \textbf{17} \textit{Majuskel} D  \textbf{23} \textit{Majuskel} D  \newline
\line(1,0){75} \newline
\textbf{1} Waleis] Waleys D \textbf{19} Parzival] Parcival D \newline
\end{minipage}
\hspace{0.5cm}
\begin{minipage}[t]{0.5\linewidth}
\small
\begin{center}*m
\end{center}
\begin{tabular}{rl}
 & der Waleise zuo Artuse sprach:\\ 
 & "hêrre, dô ich iuch \textbf{zuo} jungste sach,\\ 
 & dô wart ûf \textbf{die} êre mir gerant.\\ 
 & von prîse ich gap sô hôhiu pfant,\\ 
5 & da\textit{z} ich  \textbf{nâch was} komen.\\ 
 & nû hab ich \textbf{von} iu vernomen,\\ 
 & ob ir mirz sagt âne vâr,\\ 
 & daz prîs \textbf{ein teil an mir} het wâr.\\ 
 & wie unsanft ich daz \textbf{lerne},\\ 
10 & ich \textbf{gloubte}z \textbf{ouch noch} gerne,\\ 
 & wolt e\textit{z} glouben \textbf{al} ander diet,\\ 
 & von den ich mich schamende schiet."\\ 
 & \begin{large}D\end{large}ie dâ sâzen, jâhen sîner hant,\\ 
 & si het den prîs über \textbf{manigiu} lant\\ 
15 & mit sô hôhem prîs erworben,\\ 
 & daz \textbf{sîn prîs} wær unverdorben.\\ 
 & der herzogîn ritter gar\\ 
 & \textbf{ouch} \textbf{kômen}, d\textit{â} der wol gevar\\ 
 & Parcifal bî Artuse saz.\\ 
20 & der werde künic niht vergaz:\\ 
 & er \textbf{enpfienc} \textbf{si} in des wirtes hûs,\\ 
 & der \textbf{hœheste}, \textbf{wîse} Artus,\\ 
 & wie wî\textit{t} wære Gawans gezelt,\\ 
 & er saz dar vür ûf daz velt;\\ 
25 & si sâzen \textbf{umb in} an den rinc.\\ 
 & sich samten unkundiu dinc.\\ 
 & wer \textbf{diser und jener} wære,\\ 
 & daz würden wîtiu mære,\\ 
 & solte der kristen und der Sarrazin\\ 
30 & kuntlîche d\textit{â} genennet sîn.\\ 
\end{tabular}
\scriptsize
\line(1,0){75} \newline
m n o Fr69 \newline
\line(1,0){75} \newline
\textbf{13} \textit{Initiale} m   $\cdot$ \textit{Capitulumzeichen} n  \newline
\line(1,0){75} \newline
\textbf{1} Waleise] waleisse o ::: Fr69  $\cdot$ Artuse] artus o ::: Fr69 \textbf{2} iuch] \textit{om.} n v́ Fr69  $\cdot$ zuo] \textit{om.} Fr69 \textbf{4} ich gap] gap ich n \textbf{5} daz] Dach m  $\cdot$ komen] komenen n \textbf{7} âne] \textit{om.} n \textbf{10} gloubtez] globttes m (n)  $\cdot$ ouch noch] úch doch n (o) \textbf{11} ez glouben] erglouben m  $\cdot$ diet] det o \textbf{13} dâ] do n  $\cdot$ sîner] miner o \textbf{14} het] hat n  $\cdot$ manigiu] manig m n o \textbf{16} daz] Denne n \textbf{17} gar] [dar]: gar o \textbf{18} kômen] kome n  $\cdot$ dâ] do m n o \textbf{22} Artus] artuͯs o \textbf{23} wît] witte m (n) (o) \textbf{24} dar] \textit{om.} o \textbf{25} in] im o \textbf{26} unkundiu] vnkuͯnden o \textbf{29} Sarrazin] sarazin m sariczan o \textbf{30} dâ] do m n o \newline
\end{minipage}
\end{table}
\newpage
\begin{table}[ht]
\begin{minipage}[t]{0.5\linewidth}
\small
\begin{center}*G
\end{center}
\begin{tabular}{rl}
 & \begin{large}D\end{large}er Waleis ze Artuse sprach:\\ 
 & "hêrre, dô ich iuch jungest sach,\\ 
 & dô wart ûf êre mir gerant.\\ 
 & von prîse ich gab sô hôhiu pfant,\\ 
5 & daz ich \textbf{was nâch von brîse} komen.\\ 
 & nû hân ich, \textbf{hêrre}, \textbf{an} iu vernomen,\\ 
 & obe ir mirz saget âne vâr,\\ 
 & daz prîs \textbf{ein teil an mir} hât wâr."\\ 
 & \multicolumn{1}{l}{ - - - }\\ 
10 & \multicolumn{1}{l}{ - - - }\\ 
 & \multicolumn{1}{l}{ - - - }\\ 
 & \multicolumn{1}{l}{ - - - }\\ 
 & die dâ sâzen, \textbf{die} jâhen sîner hant,\\ 
 & si het den brîs über \textbf{manigiu} lant\\ 
15 & mit sô hôhem brîse erworben,\\ 
 & daz \textbf{si brîses} wære unverdorben.\\ 
 & der herzoginne rîter gar\\ 
 & \textbf{kômen}, dâ der wol gevar\\ 
 & Parcival bî Artuse saz.\\ 
20 & der werde künic \textbf{dô} niht vergaz:\\ 
 & er \textbf{enpfienc} \textbf{in} in des wirtes hûs,\\ 
 & der \textbf{stolze} \textbf{künic} Artus,\\ 
 & swie wît wære Gawans gezelt,\\ 
 & er saz dar vür ûf daz velt;\\ 
25 & si sâzen \textbf{nider} an den rinc.\\ 
 & sich samenten unkundiu dinc.\\ 
 & wer \textbf{dirre unde jener} wære,\\ 
 & daz würden wîtiu \textit{m}ære,\\ 
 & solde der christen unde der Sarrazin\\ 
30 & kuntlîche dâ genennet sîn.\\ 
\end{tabular}
\scriptsize
\line(1,0){75} \newline
G I L M Z \newline
\line(1,0){75} \newline
\textbf{1} \textit{Initiale} G I L Z  \textbf{23} \textit{Initiale} I  \newline
\line(1,0){75} \newline
\textbf{1} Artuse] artus Z \textbf{2} dô] \textit{om.} L da M Z  $\cdot$ iuch] evch nu I \textbf{3} dô] Da M Z  $\cdot$ êre] die ere L (M) (Z)  $\cdot$ mir] myn M \textbf{4} hôhiu] hohez I hohen M \textbf{5} was nâch] nach was Z \textbf{8} daz] Der L  $\cdot$ an mir hât] hat an mir I \textbf{9} \textit{Die Verse 699.9-12 fehlen} G I L M   $\cdot$ Swie vnsanfte ich daz lerne Z \textbf{10} Jch gelovbtez ev doch gerne Z \textbf{11} Woͤlt ez gelovben ander diet Z \textbf{12} Von den ich mich doch schamende schiet Z \textbf{13} dâ] [s]: da M  $\cdot$ die jâhen] vnd iahen L dy sprachin M \textbf{14} den] \textit{om.} L  $\cdot$ manigiu] ellev I \textbf{16} brîses wære] were prises Z \textbf{19} Parcival] Parcifal G Z Parzifal I L M  $\cdot$ Artuse] artus I Z \textbf{20} dô] da M Z \textbf{21} enpfienc in] enpfinge sý L enpfienges Z \textbf{22} stolze] solze I  $\cdot$ Artus] Artuͯs L \textbf{23} swie] Wie L (M)  $\cdot$ Gawans] Gawanz L \textbf{24} ûf] an I \textbf{25} nider an den] witen an den I vmb den L alvmme anden M vmb in an den Z \textbf{26} unkundiu dinc] vnchudiu chint I vnkvndie ding L \textbf{27} unde] [*]: oder L  $\cdot$ jener] iene M \textbf{28} daz] Da M  $\cdot$ mære] wâre G \textbf{29} Sarrazin] Sarazin L \textbf{30} kuntlîche] Chunclich I \newline
\end{minipage}
\hspace{0.5cm}
\begin{minipage}[t]{0.5\linewidth}
\small
\begin{center}*T
\end{center}
\begin{tabular}{rl}
 & der Waleis zuo Artuse sprach:\\ 
 & "hêrre, dô ich iuch jungest sach,\\ 
 & dô wart ûf \textbf{die} êre mir gerant.\\ 
 & von prîse ich gap sô hôhiu pfant,\\ 
5 & daz ich \textbf{was nâch von prîse} komen.\\ 
 & nû hân ich, \textbf{hêrre}, \textbf{an} iu vernomen,\\ 
 & ob ir mir ez saget âne vâr,\\ 
 & daz prîs \textbf{ein teil an mir} hât wâr.\\ 
 & wie unsanfte ich daz \textbf{gelerne},\\ 
10 & ich \textbf{ge\textit{lou}bete} ez \textbf{iu doch} gerne,\\ 
 & wolt e\textit{z} gelouben ander diet,\\ 
 & von den ich mich scha\textit{me}nde schiet."\\ 
 & die dâ sâzen, \textbf{die} jâhen sîner hant,\\ 
 & si hete den prîs über \textbf{alliu} lant\\ 
15 & mit sô hôhem prîse erworben,\\ 
 & daz \textbf{si prîses} wær\textit{e} unverdorben.\\ 
 & \begin{large}D\end{large}er herzoginne rîter gar\\ 
 & \textbf{kâmen}, d\textit{â} der wol gevar\\ 
 & Parcifal bî Artuse saz.\\ 
20 & der werde künec niht vergaz:\\ 
 & er \textbf{entvienc} \textbf{si} in des wirtes hûs,\\ 
 & der \textbf{stolze} \textbf{künec} Artus,\\ 
 & wie wît wære Gawans gezelt,\\ 
 & er saz dar vür ûf daz velt;\\ 
25 & si sâzen \textbf{umb} an den rinc.\\ 
 & sich samente\textit{n} unkundiu dinc.\\ 
 & wer \textbf{jener und dirre} wære,\\ 
 & daz würden wîtiu mære,\\ 
 & solte der kristen und der Sarrazin\\ 
30 & kuntlîche dâ genennet sîn.\\ 
\end{tabular}
\scriptsize
\line(1,0){75} \newline
U V W Q R \newline
\line(1,0){75} \newline
\textbf{1} \textit{Initiale} Q R  \textbf{13} \textit{Initiale} W  \textbf{17} \textit{Initiale} U V  \newline
\line(1,0){75} \newline
\textbf{1} Waleis] walleis V waleiße W  $\cdot$ Artuse] artus Q (R) \textbf{2} jungest] zeiúngest V (W) \textbf{3} Do wart mir uf die ere gerant V  $\cdot$ mir] mein W (R) \textbf{5} von prîse] [*]: von prise V preise W \textbf{6} an iu] [*]: an v́ch V an uch das R \textbf{7} ir mir ez] irs mirs R \textbf{8} an mir hât] hat an mir Q \textbf{9} wie] Swie V Wie ich so R  $\cdot$ gelerne] lerne V W Q R \textbf{10} geloubete] gebete U gelopt R  $\cdot$ iu doch] [*]: v́ch oͮch V doch Q \textbf{11} ez] ir U  $\cdot$ ander] andere W \textbf{12} mich schamende] mich schande U [*]: mich so schamende V do schamende Q (R) \textbf{13} dâ] do W \textbf{14} hete] [*]: hette V  $\cdot$ alliu] manige V (W) (Q) menge guͯttes R \textbf{16} si] sin V (W)  $\cdot$ prîses] [pri*]: pris V preiß W  $\cdot$ wære] weren U R \textbf{17} herzoginne] herczoginnen R \textbf{18} dâ] do U V W Q R \textbf{19} Parcifal] Parzifal U Parzefal V Partzifal W Q Parczifal R  $\cdot$ Artuse] artus Q R \textbf{20} niht] [*]: do niht V do nicht W Q (R) \textbf{23} wie] Swie V  $\cdot$ Gawans] Gawins R  $\cdot$ gezelt] zelt R \textbf{24} ûf daz] vffens V \textbf{25} an] [*]: in an V vff Q \textbf{26} samenten] samente U  $\cdot$ unkundiu] vnkúndicge R \textbf{27} jener und dirre] dieser vnd iener Q (R) \textbf{28} wîtiu] vast weite W witte R \textbf{29} solte] Solten W  $\cdot$ Sarrazin] Sarraszin V sarazin W sarrasin R \textbf{30} dâ] do V W Q R \newline
\end{minipage}
\end{table}
\end{document}
