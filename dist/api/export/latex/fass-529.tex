\documentclass[8pt,a4paper,notitlepage]{article}
\usepackage{fullpage}
\usepackage{ulem}
\usepackage{xltxtra}
\usepackage{datetime}
\renewcommand{\dateseparator}{.}
\dmyyyydate
\usepackage{fancyhdr}
\usepackage{ifthen}
\pagestyle{fancy}
\fancyhf{}
\renewcommand{\headrulewidth}{0pt}
\fancyfoot[L]{\ifthenelse{\value{page}=1}{\today, \currenttime{} Uhr}{}}
\begin{document}
\begin{table}[ht]
\begin{minipage}[t]{0.5\linewidth}
\small
\begin{center}*D
\end{center}
\begin{tabular}{rl}
\textbf{529} & \begin{large}V\end{large}rouwe, daz ist sîn râche ûf mich."\\ 
 & si sprach: "sich twirhet sîn gerich.\\ 
 & ich \textbf{en}wirde iu lîhte nimmer holt,\\ 
 & doch enpfæhet er drumbe \textbf{al}solhen solt,\\ 
5 & ê er scheide von mîme lande,\\ 
 & des er jehen mac vür schande.\\ 
 & Sît ez der künec dort niht rach\\ 
 & \textbf{al daz} der \textbf{vrouwen} \textbf{dâ} geschach,\\ 
 & \textbf{unt} ez sich hât an mich gezogt,\\ 
10 & ich bin nû iwer bêder vogt\\ 
 & unt enweiz doch, wer ir beidiu sît.\\ 
 & \textbf{er muoz} dar umbe enpfâhen strît,\\ 
 & durch die \textbf{vrouwen} eine\\ 
 & unt durch iuch harte kleine.\\ 
15 & man sol \textbf{unvuoge} rechen\\ 
 & mit slahen und mit stechen."\\ 
 & Gawan zuo dem pferde gienc,\\ 
 & mit \textbf{lîhtem sprunge} erz \textbf{doch} gevienc.\\ 
 & dâ was der knappe komen nâch,\\ 
20 & ze dem diu \textbf{vrouwe} heidensch sprach\\ 
 & al daz si wider ûf enbôt.\\ 
 & \textbf{nû} \textbf{næhet} ouch \textbf{Gawans} nôt.\\ 
 & Malcreatiure ze vuoz \textbf{vuor} dan.\\ 
 & dô \textbf{gesach} \textbf{ouch} mîn hêr Gawan\\ 
25 & des \textbf{hêrren} runzît;\\ 
 & da\textit{z} was ze kranc ûf einen strît.\\ 
 & ez hete der knappe dort genomen,\\ 
 & ê er von \textbf{der halden} wære komen,\\ 
 & einem vilâne.\\ 
30 & dô geschach ez Gawane\\ 
\end{tabular}
\scriptsize
\line(1,0){75} \newline
D Fr7 Fr11 Fr31 \newline
\line(1,0){75} \newline
\textbf{1} \textit{Initiale} D Fr31  \textbf{7} \textit{Majuskel} D  \textbf{17} \textit{Initiale} Fr11  \newline
\line(1,0){75} \newline
\textbf{3} lîhte] \textit{om.} Fr31 \textbf{4} alsolhen] solhen Fr31 \textbf{5} scheide] schaid Fr11 \textbf{6} mac] mvͤz Fr31 \textbf{8} al daz] alda Fr7 daz Fr11 Daz es Fr31  $\cdot$ vrouwen] frowe Fr7  $\cdot$ dâ] dan Fr7 \textit{om.} Fr31 \textbf{9} hât an mich] an mich hat Fr7 (Fr31) \textbf{10} nû] iv Fr7 \textbf{11} wer] von Fr7  $\cdot$ beidiu] beide Fr7 (Fr31) \textbf{12} muoz] muͤz Fr7 muͯs Fr11 \textbf{15} unvuoge] ungefuͤge Fr7 \textbf{17} Gawan] Gawin Fr11 \textbf{18} lîhtem sprunge] lihten sprungen Fr7 lihte sprvge Fr31  $\cdot$ erz doch] er daz Fr31 \textbf{19} dâ] Daz Fr11 \textbf{21} al daz si] Daz si hin Fr11  $\cdot$ enbôt] enbat Fr7 \textbf{22} næhet] nahet Fr7 Fr31 nachent Fr11  $\cdot$ ouch] \textit{om.} Fr11 Fr31  $\cdot$ Gawans] Gaw::: Fr11 \textbf{23} :::iͮr fvͦr zefvͦz dan Fr31  $\cdot$ Malcreatiure] Malcreatvre D (Fr7) Mal creatúre Fr11 \textbf{24} da gesach auch he::: Fr11 \textbf{25} hêrren] ivncherren Fr7 (Fr11) \textbf{26} daz] das D  $\cdot$ kranc] grane Fr7 \textbf{29} Einen vilanen Fr11 \textbf{30} Gawane] Ga::: Fr11 \newline
\end{minipage}
\hspace{0.5cm}
\begin{minipage}[t]{0.5\linewidth}
\small
\begin{center}*m
\end{center}
\begin{tabular}{rl}
 & vrouwe, daz ist sîn râche ûf mich."\\ 
 & si sprach: "sich twirhet sîn gerich.\\ 
 & ich wirde iu lîht nimmer holt,\\ 
 & doch enpfâhet er dar umb solichen solt,\\ 
5 & ê er sch\textit{ei}de von mî\textit{n}e\textit{m} lande,\\ 
 & des er jehen mac vür schande.\\ 
 & sît ez der künic dort niht rach,\\ 
 & \textbf{aldâ ez} der \textbf{vrouwen} geschach,\\ 
 & \textbf{und} ez sich het an mich gez\textit{o}get,\\ 
10 & ich bin nû \textit{i}u\textit{w}er beider voget\\ 
 & und enweiz doch, wer ir beidiu sît.\\ 
 & \textbf{er muoz} dar umb enpfâhen strît,\\ 
 & durch die \textbf{vrouwen} eine\\ 
 & und durch iuch harte kleine.\\ 
15 & man sol \textbf{sîn} \textbf{unvuoge} rechen\\ 
 & mit slahen und mit stechen."\\ 
 & Gawan zuo dem pferde gienc,\\ 
 & mit \textbf{lîhtem sprunge} erz \textbf{doch} gevienc.\\ 
 & d\textit{â} was der knappe komen nâch,\\ 
20 & zuo dem diu \textbf{vrouwe} heidensch sprach\\ 
 & allez daz si wider ûf enbôt.\\ 
 & \textbf{nû} \textbf{nâhet} ouch \textbf{Gawanen} nôt.\\ 
 & Mala creatiur zuo \textit{v}uoz \textbf{schiet} dan.\\ 
 & dô \textbf{be\textit{s}ach} mîn hêrre Gawan\\ 
25 & des \textbf{junchêrren} runzît;\\ 
 & daz was zuo kranc ûf eine\textit{n} strît.\\ 
 & ez het der knappe dort genomen,\\ 
 & ê er vo\textit{n} \textbf{der h\textit{a}lde} wær komen,\\ 
 & eine\textit{m} vi\textit{l}ân.\\ 
30 & dô geschach ez Gawan\\ 
\end{tabular}
\scriptsize
\line(1,0){75} \newline
m n o \newline
\line(1,0){75} \newline
\newline
\line(1,0){75} \newline
\textbf{2} gerich] [rich]: gerich o \textbf{3} wirde] wurde n (o)  $\cdot$ lîht] vil lichte n \textbf{5} E er schiede von mẏnnen lande m \textbf{6} des] Das o \textbf{7} dort] sit o \textbf{9} het] hette n  $\cdot$ gezoget] gezouget m gezeiget n \textbf{10} iuwer] vnder m \textit{om.} n \textbf{11} doch] \textit{om.} o \textbf{12} muoz] muͯs m o \textbf{13} vrouwen] frouͯwe m (n) (o) \textbf{14} iuch] \textit{om.} o \textbf{15} unvuoge] vngefuge o \textbf{19} dâ] Do m n o \textbf{21} allez daz] als das m n Als o \textbf{22} Gawanen] gawaner o \textbf{23} Mala creatiur] Mala creatur m Mala creatúr n o  $\cdot$ vuoz] suͯs m \textbf{24} besach] beschah m (o)  $\cdot$ hêrre] herre her n \textbf{26} einen] einem m einē o \textbf{28} von der halde] vor der hulde m \textbf{29} einem] Eine m Eẏnen o  $\cdot$ vilân] filian m n o \newline
\end{minipage}
\end{table}
\newpage
\begin{table}[ht]
\begin{minipage}[t]{0.5\linewidth}
\small
\begin{center}*G
\end{center}
\begin{tabular}{rl}
 & \begin{large}V\end{large}rouwe, daz ist sîn râche ûf mich."\\ 
 & si sprach: "sich twir\textit{h}et sîn gerich.\\ 
 & ich\textbf{ne} wirde iu lîhte nimer holt,\\ 
 & doc\textit{h} enpfæhet er drumbe solchen solt,\\ 
5 & ê er scheide von mînem lande,\\ 
 & des er jehen mac vür schande.\\ 
 & sît ez der künic dort niht rach\\ 
 & \textbf{al daz} der \textbf{vrouwen} \textbf{dâ} geschach,\\ 
 & \textbf{sît} ez sich hât an mich gezoget,\\ 
10 & ich bin nû iuwer beider voget\\ 
 & unde enweiz doch, wer ir beidiu sît.\\ 
 & \textbf{er muoz} dar umbe enpfâhen strît,\\ 
 & durch die \textbf{vrouwen} eine\\ 
 & unde durch iuch harte kleine.\\ 
15 & man sol \textbf{ungevuoge} rechen\\ 
 & mit slahen unde mit stechen."\\ 
 & Gawan zuo dem pferde gienc,\\ 
 & mit \textbf{lîhten sprünge\textit{n}} erz \textbf{doch} gevienc.\\ 
 & dâ was der knappe komen nâch,\\ 
20 & ze dem diu \textbf{vrouwe} heidensch sprach\\ 
 & al daz si wider ûf enbôt.\\ 
 & \textbf{nû} \textbf{nâhent} ouch \textbf{Gawans} nôt.\\ 
 & Mal creature ze vuoz \textbf{vu\textit{or}} dan.\\ 
 & dô \textbf{gesach} \textbf{ouch} mîn hêrre Gawan\\ 
25 & des \textbf{junchêrren} runzît;\\ 
 & daz was ze kranc ûf eine\textit{n} strît.\\ 
 & ez hete der knappe dort genomen,\\ 
 & ê er von \textbf{der halden} wære komen,\\ 
 & einem vilâne.\\ 
30 & dô geschach ez Gawane\\ 
\end{tabular}
\scriptsize
\line(1,0){75} \newline
G I L M Z \newline
\line(1,0){75} \newline
\textbf{1} \textit{Initiale} G I L Z  \textbf{17} \textit{Initiale} M  \textbf{23} \textit{Initiale} I  \newline
\line(1,0){75} \newline
\textbf{1} Vrouwe] [Prowe]: Frowe I  $\cdot$ ûf] ubir M \textbf{2} twirhet] twirbet G \textbf{3} ichne wirde] Jch wuͯrde L \textbf{4} doch] dohne G  $\cdot$ er] ir I M  $\cdot$ solchen] alsulchen M (Z) \textbf{5} ê] \textit{om.} I M  $\cdot$ lande] [libe]: lande G \textbf{6} des] Das M  $\cdot$ jehen] Sprechin M \textbf{7} niht] nirgen M \textbf{8} al daz] swaz I \textbf{9} sît] Vnd L (M) (Z)  $\cdot$ gezoget] gezcugt M \textbf{11} beidiu] beide I \textbf{12} enpfâhen] einpfahen L \textbf{14} unde] \textit{om.} I \textbf{15} ungevuoge] vnfuͯge L (M) (Z) \textbf{17} pferde] orse I \textbf{18} lîhten sprüngen] lihten sprvnge G lichteme Sprunge M (Z)  $\cdot$ doch] \textit{om.} I  $\cdot$ gevienc] vinc M \textbf{19} dâ] Do L Z \textbf{20} diu] \textit{om.} L \textbf{21} al daz] Als M  $\cdot$ enbôt] einbot L \textbf{22} nâhent] nahet L M  $\cdot$ Gawans] Gauwans I Gawansz L gawanes Z \textbf{23} Mal creature] Malcreatur I Malcreatuͯre L Male creature M Mal createvr Z  $\cdot$ vuor] fuere G \textbf{24} dô] Da L M  $\cdot$ gesach] sach Z  $\cdot$ ouch] \textit{om.} L  $\cdot$ hêrre Gawan] ergawan M \textbf{26} einen] einem G eineē L (M) \textbf{27} dort] al dort Z \textbf{28} ê] \textit{om.} M \textbf{29} einem] Einen L \textbf{30} dô] Da M Z \newline
\end{minipage}
\hspace{0.5cm}
\begin{minipage}[t]{0.5\linewidth}
\small
\begin{center}*T
\end{center}
\begin{tabular}{rl}
 & Vrouwe, daz ist sîn râche ûf mich."\\ 
 & Si sprach: "sich twerhet sîn gerich.\\ 
 & ich wirdiu lîhte niemer holt,\\ 
 & doch enpfæhet er drumbe solhen solt,\\ 
5 & ê er scheide von mînem lande,\\ 
 & des er jehen mac vür schande.\\ 
 & sît ez der künec dort niht rach,\\ 
 & \textbf{daz ez} der \textbf{juncvrouwen} geschach,\\ 
 & \textbf{unde} ez sich hât an mich gezoget,\\ 
10 & ich bin nû iuwer beider voget\\ 
 & unde enweiz doch, wer ir beide sît.\\ 
 & \textbf{doch muoz er} drumbe enpfâhen strît,\\ 
 & durch die \textbf{junc\textit{v}rouwe} eine\\ 
 & unde durch iuch harte kleine.\\ 
15 & man sol \textbf{unvuoge} rechen\\ 
 & mit slahen unde mit stechen."\\ 
 & Gawan zuo dem pferde gienc,\\ 
 & mit \textbf{lîhten sprüngen} erz gevienc.\\ 
 & dâ was der knappe komen nâch,\\ 
20 & zuo dem diu \textbf{juncvrouwe} heidensch sprach\\ 
 & aldaz si wider ûf enbôt.\\ 
 & \textbf{dô} \textbf{nâhet} ouch \textbf{Gawans} nôt.\\ 
 & \textit{\begin{large}M\end{large}}alacreature ze vuoz \textbf{vuor} dan.\\ 
 & dô \textbf{besach} mîn hêr Gawan\\ 
25 & des \textbf{junchêrre\textit{n}} runzît;\\ 
 & daz was ze kranc ûf einen strît.\\ 
 & ez hete der knappe dort genomen,\\ 
 & ê er von\textbf{me lande} wære komen,\\ 
 & einem vilâne.\\ 
30 & \begin{large}D\end{large}ô geschach ez Gawane\\ 
\end{tabular}
\scriptsize
\line(1,0){75} \newline
T U V W O Q R Fr40 \newline
\line(1,0){75} \newline
\textbf{1} \textit{Initiale} W O R Fr40   $\cdot$ \textit{Majuskel} T  \textbf{2} \textit{Majuskel} T  \textbf{17} \textit{Majuskel} T  \textbf{23} \textit{Initiale} T U V  \textbf{30} \textit{Initiale} T  \newline
\line(1,0){75} \newline
\textbf{1} Vrouwe] ÷rowe O  $\cdot$ daz ist] ist daz W  $\cdot$ ûf mich] [an mir]: vf mich O \textbf{2} sprach] spach V  $\cdot$ twerhet] zweilet sust R  $\cdot$ sîn] min O \textbf{3} wirdiu lîhte] worde lichte vch U wúrd úch villeichte W \textbf{4} doch] Do hin U Ovch O \textbf{5} ê] \textit{om.} Q  $\cdot$ scheide] schied R \textbf{6} schande] schanden Q \textbf{7} ez] er Q \textbf{8} daz] [A*]: Aldo V Da O (Fr40) Do Q R  $\cdot$ ez] \textit{om.} Fr40 \textbf{9} hât an mich] an mich hat Q R \textbf{11} beide] beider R \textbf{12} muoz] muͦs R  $\cdot$ strît] leit O \textbf{13} juncvrouwe] ivncrouwe T iuͦncvreuͦwen U (V) (W) (O) (Q) (R) (Fr40) \textbf{14} iuch] iv T \textbf{15} man] Jch O  $\cdot$ unvuoge] vngevuͦge U si vil wol O  $\cdot$ rechen] gerechen O \textbf{17} Gawan] Gawin R \textbf{18} lîhten sprüngen] leichtem sprung W (Q) (Fr40) \textbf{19} dâ] Do U V O Q R (Fr40) Dar ab W \textbf{20} juncvrouwe] vreuͦwe U (V) (W) (O) (Q) (R) (Fr40) \textbf{21} aldaz] Alz daz U  $\cdot$ enbôt] in bot U \textbf{22} Gawans] gawins R \textbf{23} Malacreature] ÷Alacreatvre T Malcreature U (Fr40) Mal creatúre V (R) Mala creatur W Malcreatvr O Mang creatur Q  $\cdot$ ze vuoz vuor] fvͦr zefvͦzen O zu fuͦs fuͦr hin R \textbf{24} dô] Doch V  $\cdot$ besach] sach W O Q R Fr40  $\cdot$ Gawan] gawin R \textbf{25} junchêrren] ivncherre T \textbf{26} ze kranc] zekranz Fr40  $\cdot$ ûf einen] in einem Q \textbf{28} vonme lande] von [*]: der halde V von der halde Q von der halden R Fr40 \textbf{29} einem] Einen Q (Fr40) \textbf{30} geschach] gesach R  $\cdot$ Gawane] hern Gawane O Gawaine R \newline
\end{minipage}
\end{table}
\end{document}
