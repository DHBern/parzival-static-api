\documentclass[8pt,a4paper,notitlepage]{article}
\usepackage{fullpage}
\usepackage{ulem}
\usepackage{xltxtra}
\usepackage{datetime}
\renewcommand{\dateseparator}{.}
\dmyyyydate
\usepackage{fancyhdr}
\usepackage{ifthen}
\pagestyle{fancy}
\fancyhf{}
\renewcommand{\headrulewidth}{0pt}
\fancyfoot[L]{\ifthenelse{\value{page}=1}{\today, \currenttime{} Uhr}{}}
\begin{document}
\begin{table}[ht]
\begin{minipage}[t]{0.5\linewidth}
\small
\begin{center}*D
\end{center}
\begin{tabular}{rl}
\textbf{33} & man diende \textbf{in} rîterlîche.\\ 
 & diu küneginne rîche\\ 
 & kom stolzlîche vür sînen tisch.\\ 
 & hie stuont der reiger, dort der visch.\\ 
5 & si was durch daz \textbf{hin} \textbf{zim} gevarn,\\ 
 & si wolde selbe \textbf{daz} bewarn,\\ 
 & daz man sîn pfl\textit{æ}ge \textbf{wol} ze vromen.\\ 
 & si was mit juncvrouwen komen.\\ 
 & \textbf{si} kniete nider. daz was im leit.\\ 
10 & mit ir \textbf{selber} hant si sneit\\ 
 & dem rîter sîner spîse ein teil.\\ 
 & diu vrouwe was ir gastes geil.\\ 
 & \textbf{dô} \textbf{huop si} im \textbf{sîn} trinken dar\\ 
 & unt pflac sîn wol. ouch nam er war,\\ 
15 & wie \textbf{was gebærde} unt ir wort.\\ 
 & zende an \textbf{sînes} tisches ort\\ 
 & sâzen sîne \textbf{spilman}\\ 
 & \textbf{unt} anderhalp sîn \textbf{kappelân}.\\ 
 & al schemende er an die vrouwen sach.\\ 
20 & harte \textbf{blûclîche} er sprach:\\ 
 & "Ich\textbf{ne} hân mi\textit{ch}\textbf{s} niht genietet,\\ 
 & als ir mir, vrouwe, bietet,\\ 
 & mînes \textbf{lîbes} mit sölhen êren.\\ 
 & ob ich iuch solde lêren,\\ 
25 & sô wære \textbf{hînte} an iuch gegert\\ 
 & \textbf{eines} pflegens, des ich wære wert.\\ 
 & sô\textbf{ne} wæret ir niht her ab geriten.\\ 
 & getar ich \textbf{iuch des, vrouwe}, \textbf{biten},\\ 
 & \textit{\begin{large}S\end{large}}ô lât mich in der mâze leben.\\ 
30 & ir habt mir \textbf{êre} ze vil gegeben."\\ 
\end{tabular}
\scriptsize
\line(1,0){75} \newline
D \newline
\line(1,0){75} \newline
\textbf{21} \textit{Majuskel} D  \textbf{29} \textit{Initiale} D  \newline
\line(1,0){75} \newline
\textbf{7} pflæge] phlege D \textbf{21} michs] mis D \textbf{29} Sô] ÷o D \newline
\end{minipage}
\hspace{0.5cm}
\begin{minipage}[t]{0.5\linewidth}
\small
\begin{center}*m
\end{center}
\begin{tabular}{rl}
 & man diente \textbf{in} ritterlîche.\\ 
 & diu küniginne rîche\\ 
 & kam stolzlîc\textit{h} vür sînen tisch.\\ 
 & hie \textit{stuont} der reiger, dort der visch.\\ 
5 & si was durch daz \textbf{zuo im} ge\textit{var}n,\\ 
 & si wolte selber \textbf{daz} \textit{b}e\textit{war}n,\\ 
 & daz man sîn pflæge \textbf{wol} ze vromen.\\ 
 & si was \textit{mit} juncvrouwen komen.\\ 
 & \textbf{si} kniewete nider. daz was im leit.\\ 
10 & mit ir \textbf{selbes} hant si sneit\\ 
 & dem ritter sîner spîse ein teil.\\ 
 & diu vrowe was ir gastes geil.\\ 
 & \textbf{dô} \textbf{bôt si} im \textbf{sîn} trinken dar\\ 
 & und p\textit{f}lac sîn wol. ouch nam er war,\\ 
15 & wie \textbf{was gebærde} und ir wort.\\ 
 & zende an \textbf{sînes} tisches ort\\ 
 & sâ\textit{z}en sîne \textbf{spilman}\\ 
 & \textbf{und} anderhalb sîn \textbf{kappelân}.\\ 
 & Alschemende er an die vrouwen sach.\\ 
20 & harte \textbf{blœdeclîchen} er sprach:\\ 
 & "i\textbf{ne} hâ\textit{n m}ich\textbf{s} niht genietet,\\ 
 & als ir mir\textbf{s}, vrowe, bietet,\\ 
 & mînes \textbf{lebendes} mit solichen êren.\\ 
 & ob ich iuch solte lêren,\\ 
25 & sô wære \textbf{hînt} \textbf{sân} an iuch gegert\\ 
 & \textbf{eines} pflegenes, des ich wære wert.\\ 
 & s\textit{ô}\textbf{ne} wære\textit{t} ir niht her abe geriten.\\ 
 & getar ich \textbf{iuch des, vrouwe}, \textbf{erbiten},\\ 
 & sô lât mich in der mâze leben.\\ 
30 & ir habt mir \textbf{êre} ze vil gegeben."\\ 
\end{tabular}
\scriptsize
\line(1,0){75} \newline
m n o W \newline
\line(1,0){75} \newline
\textbf{5} \textit{Initiale} W  \textbf{19} \textit{Initiale} W   $\cdot$ \textit{Capitulumzeichen} m  \newline
\line(1,0){75} \newline
\textbf{1} diente in] dient im W \textbf{2} küniginne] koniginnen o \textbf{3} stolzlîch] stolczlichem m \textbf{4} stuont] \textit{om.} m  $\cdot$ dort] do W \textbf{5} gevarn] geritten m \textbf{6} si] Vnd n o W  $\cdot$ bewarn] verbitten m \textbf{7} sîn] \textit{om.} W  $\cdot$ wol] \textit{om.} W  $\cdot$ ze vromen] zuͦ frowen o \textbf{8} mit] \textit{om.} m n o \textbf{9} kniewete] koniete o knúten W \textbf{11} dem] Den o \textbf{14} pflac] plag m  $\cdot$ ouch] doch W \textbf{15} wie] \textit{om.} n o W  $\cdot$ ir] ire m \textbf{17} sâzen] Saffen m \textbf{18} sîn] seine W \textbf{19} Alschemende] ALs senende W  $\cdot$ vrouwen] frauwe W \textbf{21} ine hân michs] Me han ich mihs m Nuͯ habe ich mich sin n Nu habe ich miches o Nun hab mich es W  $\cdot$ genietet] genitten o gemiet W \textbf{23} lebendes] lebens n o W \textbf{25} wære hînt sân] het mein sinne W \textbf{26} pflegenes] pflegen W \textbf{27} sône wæret] Seine weren m So weren W \textbf{28} getar] So getar n Gotar o  $\cdot$ ich iuch des vrouwe erbiten] das ich uch frowe bitten o ich eúch fraw des bitten W \textbf{29} sô] [Sÿ]: So m Sie (o)  $\cdot$ mâze] mossen n (o) (W) \textbf{30} habt] hap o haben W  $\cdot$ ze] so o  $\cdot$ gegeben] geben n W \newline
\end{minipage}
\end{table}
\newpage
\begin{table}[ht]
\begin{minipage}[t]{0.5\linewidth}
\small
\begin{center}*G
\end{center}
\begin{tabular}{rl}
 & man diente \textbf{im} rîterlîche.\\ 
 & diu küniginne rîche\\ 
 & kom stolzlîche vür sînen tisch.\\ 
 & hie stuont der reiger, dort der visch.\\ 
5 & si was durch daz \textbf{hin} \textbf{abe} gevaren,\\ 
 & si wolt \textbf{ouch} selbe bewaren,\\ 
 & daz man sîn pflæge ze vrumen.\\ 
 & si was mit juncvrouwen kumen\\ 
 & \textbf{unde} kniete nider. daz was im leit.\\ 
10 & mit ir \textbf{selber} hant si sneit\\ 
 & dem rîter sîner spîse ein teil.\\ 
 & diu vrouwe was ir gastes geil.\\ 
 & \textbf{si bôt} im \textbf{ouch} \textbf{sîn} trinken dar\\ 
 & unde pflac sîn wol. ouch nam er war,\\ 
15 & wie \textbf{was ir gebærde} und ir wort.\\ 
 & zende an \textbf{des} tisches ort\\ 
 & sâzen sîne \textbf{kappelân},\\ 
 & anderhalp sîne \textbf{spilman}.\\ 
 & al schemende er an die vrouwen sach.\\ 
20 & harte \textbf{blûclîch}er sprach:\\ 
 & "\begin{large}I\end{large}ch hân mich niht genietet,\\ 
 & als ir mir\textbf{z}, vrouwe, bietet,\\ 
 & mînes \textbf{lebens} mit solhen êren.\\ 
 & obe ich iuch solte lêren,\\ 
25 & sô wære \textbf{hiut} \textbf{sân} an iuch gegert\\ 
 & \textbf{des} pflegens, des ich wære wert.\\ 
 & sô\textbf{ne} wæret ir niht her abe geriten.\\ 
 & getar ich, \textbf{vrouwe, iuch des} \textbf{gebiten},\\ 
 & sô lât mich in der mâze leben.\\ 
30 & ir habet mir \textbf{êren} ze vil gegeben."\\ 
\end{tabular}
\scriptsize
\line(1,0){75} \newline
G O L M Q R Z Fr32 \newline
\line(1,0){75} \newline
\textbf{1} \textit{Initiale} M  \textbf{21} \textit{Initiale} G  \textbf{27} \textit{Versal} Fr32  \newline
\line(1,0){75} \newline
\textbf{1} diente] dint Q (R) (Z) (Fr32)  $\cdot$ im] in L (M) (Q) (Z) ir Fr32  $\cdot$ rîterlîche] Richlich Ritterliche R \textbf{2} \textit{Vers 32.2 fehlt} Q  \textbf{3} kom] \textit{om.} Q  $\cdot$ sînen] sinē M (Q) sinem Z \textbf{4} der reiger] der \sout{der} reiger Z  $\cdot$ dort] hy Q \textbf{5} hin abe] hinz im O (L) (M) Z (Fr32) hintz eyn n\textit{achträglich korrigiert zu: }hauͯsz eyn Q zu Jm R  $\cdot$ gevaren] hingefarn R \textbf{6} si] Vnd O L (M) Q R (Fr32)  $\cdot$ ouch] \textit{om.} L Q R Z Fr32  $\cdot$ selbe] selten Q selber R Z  $\cdot$ bewaren] daz bewarn O L (M) (Q) (R) Z Fr32 \textbf{7} pflæge] phlege O (M) (Q) (R) (Z) (Fr32)  $\cdot$ ze] wol ze O (L) (M) (Q) (R) (Z) Fr32  $\cdot$ vrumen] vrouwen Fr32 \textbf{8} [*]: si mag komen mit iuncvrowen Fr32 \textbf{9} kniete] chniet O (L) (Z)  $\cdot$ nider] \textit{om.} Q  $\cdot$ im] in Q (R) \textbf{10} selber] selbern M selben Q selbes R Z  $\cdot$ si] si on M \textbf{11} Dem ritter spise vmb sin heil Fr32  $\cdot$ ein teil] enteyle Q \textbf{13} si] [D*]: Sy Q  $\cdot$ bôt] boten Z  $\cdot$ ouch] \textit{om.} O L M Q R Z \textbf{14} unde] Sie Z \textbf{15} was ir gebærde] was gebærde O (R) ir geberde waz L (M) (Q) \textbf{16} des] sines O L M (Q) (R) Z \textbf{17} sâzen] Assen Q  $\cdot$ kappelân] spilman O (L) (M) Q R Z \textbf{18} Vnd ander thalb sin chapelan O (L) (M) (Q) (R) (Z) \textbf{19} al schemende er] Semmelichen er M Al schamender R  $\cdot$ vrouwen] frawe Q \textbf{20} blûclîcher] blodecliche er M (Z) blodlich er do Q blúklich er do R \textbf{21} Ich] Jchn O (M) (R)  $\cdot$ genietet] genieret M \textbf{22} ir] irs Q [er]: ir R \textbf{23} mînes lebens] Min leben Z \textbf{24} Solte ich iuch vrowe eren L  $\cdot$ iuch] mich Fr32 \textbf{25} wære] wir M [wert]: wer Z  $\cdot$ hiut] hynt M (Q) (Z) hin Fr32  $\cdot$ sâ] an M sam R \textbf{26} des] Ein Z  $\cdot$ pflegens] pfleges Q  $\cdot$ des ich wære] das ich were L des wer ich R \textbf{27} sône] So L Z Schone R  $\cdot$ her] er M (Q) \textbf{28} getar] Thar M Mag Q  $\cdot$ vrouwe iuch des] ivch des frawe O (L) (M) (Z) (Fr32) úch frowe des R  $\cdot$ gebiten] biten O (L) M (R) Z Fr32 \textbf{29} in der] :::ader Fr32 \textbf{30} mir] \textit{om.} L Z mich R  $\cdot$ gegeben] geben Q \newline
\end{minipage}
\hspace{0.5cm}
\begin{minipage}[t]{0.5\linewidth}
\small
\begin{center}*T
\end{center}
\begin{tabular}{rl}
 & man diend\textbf{im} rîterlîche.\\ 
 & Diu küneginne rîche\\ 
 & kom stolzlîche vür sînen tisch.\\ 
 & hie stuont der reiger, dort der visch.\\ 
5 & si was durch daz \textbf{hin} \textbf{zim} gevarn,\\ 
 & si wolte \textbf{ouch} selbe \textbf{daz} bewarn,\\ 
 & daz man sîn pflæge \textbf{wol} ze vromen.\\ 
 & si was mit \textbf{ir} juncvrouwen komen\\ 
 & \textbf{und} kniete nider. daz was im leit.\\ 
10 & mit ir \textbf{selbes} hant si sneit\\ 
 & dem rîter sîner spîse ein teil.\\ 
 & diu vrouwe was ir gastes geil.\\ 
 & \textbf{si bôt} im \textbf{daz} trinken dar\\ 
 & und pflac sîn wol. ouch nam er war,\\ 
15 & wie \textbf{ir gebærde was} und ir wort.\\ 
 & zende an \textbf{sînes} tisches ort,\\ 
 & \textbf{dâ} sâzen sîne \textbf{spilman}\\ 
 & \textbf{und} anderhalp sîn \textbf{kappelân}.\\ 
 & Al schamender an die vrouwen sach.\\ 
20 & harte \textbf{blûclîche} er sprach:\\ 
 & "ich\textbf{n} hân mich\textbf{s} niht genietet,\\ 
 & als ir mir\textbf{s}, vrouwe, bietet,\\ 
 & mînes \textbf{lebens} mit solhen êren.\\ 
 & ob ich\textbf{s} iuch solte lêren,\\ 
25 & sô wære \textbf{hin} \textbf{sân} an iu gegert\\ 
 & \textbf{des} pflegens, des ich wære wert.\\ 
 & sô wæret ir niht her abe geriten.\\ 
 & getar ich, \textbf{vrouwe, iu des} \textbf{gebiten},\\ 
 & sô lât mich in der mâze leben.\\ 
30 & ir habt mir \textbf{êren} ze vil gegeben."\\ 
\end{tabular}
\scriptsize
\line(1,0){75} \newline
T U V \newline
\line(1,0){75} \newline
\textbf{2} \textit{Majuskel} T  \textbf{19} \textit{Majuskel} T  \newline
\line(1,0){75} \newline
\textbf{1} diendim] dinte in U \textbf{8} ir] \textit{om.} U V \textbf{13} daz] dar U \textbf{15} ir wort] wort U V \textbf{16} sînes] des V \textbf{17} dâ] \textit{om.} U V \textbf{19} schamender] schemede er U  $\cdot$ die vrouwen] der U \textbf{20} blûclîche] blodecliche U [*eklich]: blúdeklich V \textbf{22} mirs] iz mir U  $\cdot$ bietet] hie bietet V \textbf{24} iuch] îv T \textbf{25} hin] hint U V  $\cdot$ sân] so V \textbf{26} ich wære wert] were gewert U \newline
\end{minipage}
\end{table}
\end{document}
