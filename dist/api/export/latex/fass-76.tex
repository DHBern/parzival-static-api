\documentclass[8pt,a4paper,notitlepage]{article}
\usepackage{fullpage}
\usepackage{ulem}
\usepackage{xltxtra}
\usepackage{datetime}
\renewcommand{\dateseparator}{.}
\dmyyyydate
\usepackage{fancyhdr}
\usepackage{ifthen}
\pagestyle{fancy}
\fancyhf{}
\renewcommand{\headrulewidth}{0pt}
\fancyfoot[L]{\ifthenelse{\value{page}=1}{\today, \currenttime{} Uhr}{}}
\begin{document}
\begin{table}[ht]
\begin{minipage}[t]{0.5\linewidth}
\small
\begin{center}*D
\end{center}
\begin{tabular}{rl}
\textbf{76} & ein wîp, die ich ê genennet hân,\\ 
 & \textbf{hie} kom \textbf{ein} ir kappelân\\ 
 & unt kleiner junchêrren drî.\\ 
 & den riten starke knappen bî.\\ 
5 & \begin{large}Z\end{large}wêne soumære \textbf{giengen} an \textbf{ir} hant.\\ 
 & die boten hete dar gesant\\ 
 & diu küneginne Ampflise.\\ 
 & ir kappelân was wîse.\\ 
 & vil schiere \textbf{bekant} er disen man.\\ 
10 & \textbf{en} franzois \textbf{er in gruozte} sân:\\ 
 & "bien seivenuz, \textbf{beassir},\\ 
 & mîner vrouwen unt mir.\\ 
 & daz ist \textbf{rêgîn} d\textit{e} Franze.\\ 
 & \textbf{die rüeret dîner minnen lanze}."\\ 
15 & \textbf{einen brief gab er im} in die hant,\\ 
 & dâr an \textbf{der hêrre} grüezen vant,\\ 
 & unt ein kleine vingerlîn.\\ 
 & daz solt ein wârer geleite sîn,\\ 
 & wan \textbf{daz} enpfienc \textbf{sîn} vrouwe\\ 
20 & von dem von Anschouwe.\\ 
 & er \textbf{neic}, dô er die \textbf{schrift} \textbf{ersach}.\\ 
 & \textbf{welt ir} nû \textbf{hœren}, wie \textbf{diu} sprach?\\ 
 & "Dir enbiutet minne unt gruoz\\ 
 & mîn lîp, dem \textbf{nie} wart kumbers buoz,\\ 
25 & sît \textbf{ich} dîner minne enpfant.\\ 
 & dîn minne \textbf{ist slôz unt bant}\\ 
 & mînes herzen unt des vröude.\\ 
 & dîn minne tuot mich töude.\\ 
 & sol \textbf{mir} dîn minne verren,\\ 
30 & sô muoz mir minne werren.\\ 
\end{tabular}
\scriptsize
\line(1,0){75} \newline
D \newline
\line(1,0){75} \newline
\textbf{5} \textit{Initiale} D  \textbf{23} \textit{Majuskel} D  \newline
\line(1,0){75} \newline
\textbf{7} Ampflise] Ampflîse D \textbf{13} de] der D \textbf{20} Anschouwe] Anscoͮwe D \newline
\end{minipage}
\hspace{0.5cm}
\begin{minipage}[t]{0.5\linewidth}
\small
\begin{center}*m
\end{center}
\begin{tabular}{rl}
 & ein wîp, die ich ê genennet h\textit{â}n,\\ 
 & \textbf{hie} kam \textbf{einer} ir kappelân\\ 
 & und kleiner junchêrren drî.\\ 
 & den riten starke knappen bî.\\ 
5 & zwêne s\textit{ou}mær\textit{e} \textbf{giengen} an \textbf{ir} hant.\\ 
 & \textit{d}ie boten hete dar gesant\\ 
 & diu küniginne \textit{Amp}flise.\\ 
 & ir kappelân was wîse.\\ 
 & vil schiere \textbf{bekante} er disen ma\textit{n}.\\ 
10 & \textbf{in} franzois \textbf{er in gruozte} sân:\\ 
 & "biensevennus, \textbf{beas sir},\\ 
 & mîner vrouwen und mir.\\ 
 & daz ist \textbf{rêgîne} de Franze.\\ 
 & \textbf{die rüeret dîner minnen lanze}."\\ 
15 & \textbf{einen brief gap er im} in die hant,\\ 
 & dâr an \textbf{der hêrre} grüezen vant,\\ 
 & und ein kleine vingerlîn.\\ 
 & daz solte ein wâr geleit sîn,\\ 
 & wanne \textbf{daz} enpfienc \textbf{diu} vrouwe\\ 
20 & von dem von Anschouwe.\\ 
 & er \textbf{neiget}, dô er die \textbf{geschrift} \textbf{an sach}.\\ 
 & \textbf{wellet ir} nû \textbf{hœren}, wie \textbf{er} sprach?\\ 
 & "\dag \begin{large}M\end{large}ir\dag  enbiutet minne und gruoz\\ 
 & mîn lîp, dem wart kumbers buoz,\\ 
25 & sît \textbf{ich} dîner minne enpfant.\\ 
 & dîn minne \textbf{ist slôz und bant}\\ 
 & mînes herzen und des vröude.\\ 
 & dîn minne tuot mich töude.\\ 
 & sol \textbf{mich} dîn minne verren,\\ 
30 & sô muoz mir minne werren.\\ 
\end{tabular}
\scriptsize
\line(1,0){75} \newline
m n o \newline
\line(1,0){75} \newline
\textbf{23} \textit{Initiale} m n  \newline
\line(1,0){75} \newline
\textbf{1} ê genennet hân] egenennethen m ie genennet han o \textbf{2} hie] Die n \textbf{4} starke] starcken n \textbf{5} soumære] sameren m  $\cdot$ an] in o \textbf{6} die] Sie m \textbf{7} Ampflise] zuͦ flisse m an flise n an flisse o \textbf{8} kappelân] Cappellen o \textbf{9} man] man Jn m \textbf{10} franzois] franczose m frantzos n franc:os o \textbf{13} rêgîne] regine vnd mir o  $\cdot$ de Franze] de francze m de frantz n defrancze o \textbf{18} wâr] gewores n fuͯr o \textbf{20} Anschouwe] anschowe o \textbf{21} neiget] neigte o  $\cdot$ geschrift] schrifft o  $\cdot$ an sach] gesach n sach o \textbf{22} nû] in o  $\cdot$ hœren] herren n \textbf{24} dem] \textit{om.} n o \textbf{25} sît] Sitte n Sit das o \textbf{27} vröude] freiden o \textbf{30} mir] mich n \newline
\end{minipage}
\end{table}
\newpage
\begin{table}[ht]
\begin{minipage}[t]{0.5\linewidth}
\small
\begin{center}*G
\end{center}
\begin{tabular}{rl}
 & ein wîp, diech ê genennet hân,\\ 
 & \textbf{dort her} kom \textbf{ein} ir kappelân\\ 
 & unde kleiner junchêrren drî.\\ 
 & den riten starke knappen bî.\\ 
5 & zwêne soumære \textbf{zugens} an \textbf{der} hant.\\ 
 & die boten hete dar gesant\\ 
 & diu künigîn Anphlise.\\ 
 & ir kapellân was wîse.\\ 
 & vil schiere \textbf{erkande}r disen man.\\ 
10 & \textbf{en} franzois \textbf{gruozter in} sân:\\ 
 & "ben sefenu, \textbf{misir},\\ 
 & mîner vrouwen und mir.\\ 
 & daz ist \textbf{ro\textit{in}} de Franze.\\ 
 & \textbf{die rüeret dîn minne lanze}."\\ 
15 & \textbf{dem helde gap er} in die hant\\ 
 & \textbf{einen brief}, dâr an \textbf{er} grüezen vant,\\ 
 & unde ein kleine vingerlîn.\\ 
 & daz solt ein wâr geleite sîn,\\ 
 & wan \textbf{ez} enpfie \textbf{sîn} vrouwe\\ 
20 & von dem von Anschouwe.\\ 
 & \begin{large}E\end{large}r \textbf{neic}, dô er die \textbf{schrift} \textbf{ersac\textit{h}}.\\ 
 & nû \textbf{hœre\textit{t}}, \textit{w}ie \textbf{diu} sprach:\\ 
 & "dir enbiut minne und gruoz\\ 
 & mîn lîp, dem \textbf{nie} wart kumbers buoz,\\ 
25 & sît \textbf{ich} dîner minne enpfant.\\ 
 & dîn minne \textbf{ist slôz und bant}\\ 
 & mînes herzen unt des vröude.\\ 
 & dîn minne tuot mich töude.\\ 
 & sol \textbf{mir} dîn minne verren,\\ 
30 & sô muoz mir minne werren.\\ 
\end{tabular}
\scriptsize
\line(1,0){75} \newline
G I O L M Q R Z Fr56 \newline
\line(1,0){75} \newline
\textbf{1} \textit{Überschrift:} Hie ruwet gamuret vnd sitzet vf ein geruwet ross Z   $\cdot$ \textit{Initiale} O L R Z  \textbf{15} \textit{Initiale} I  \textbf{21} \textit{Initiale} G  \textbf{23} \textit{Initiale} L  \newline
\line(1,0){75} \newline
\textbf{1} ein] ÷in O Sin R  $\cdot$ diech] die R  $\cdot$ ê] \textit{om.} M \textbf{2} dort her] Alhie O L (M) (Q) R (Fr56) Hie Z  $\cdot$ ein ir] eyner or M ir ein Q \textbf{3} \textit{Versfolge 76.4-3} Q   $\cdot$ junchêrren] Jungen hern M \textbf{4} den] Nv L  $\cdot$ knappen] knechtte R \textbf{5} soumære] sovmeren L  $\cdot$ zugens] giengen O (L) (M) (Q) (R) Z (Fr56)  $\cdot$ der] ir O L (M) Q R Z Fr56 \textbf{6} hete] hetten L (Q) hat R \textbf{7} Anphlise] anphelise I amphleise O anfoleise L an fliesze M an fleysz Q amfilise R amflise Z \textbf{9} vil] Als O  $\cdot$ erkander] er chande O kant er R \textbf{10} en] Ein O in I (L) (M) (Q) (R) (Z) Fr56  $\cdot$ franzois] [fronzoys]: franzoys I franzeis O frantzois L Z fracioses M franzoysz Q franczoys R frantzoys Fr56  $\cdot$ gruozter in] gruzt er >in< I er in grvͦzte O (L) (Q) (Z) er on grusze M er Jn gruͦssen R er in grvtz Fr56  $\cdot$ sân] began M gan R \textbf{11} sefenu] soe vns venvz O  $\cdot$ misir] bea sir O (L) (M) (R) (Z) (Fr56) beafir Q \textbf{13} daz] Da M  $\cdot$ roin] roẏ G la Raoine O dy royn M Ryon R  $\cdot$ de] diu I der O zu Q  $\cdot$ Franze] franz I Fr56 franzoe Q franncze R frantze Z \textbf{14} rüeret] Rúrtte R  $\cdot$ dîn] dy M diener Q (Fr56) der R  $\cdot$ minne lanze] minnen shanz I múmme lanze Q minnen lantze Z \textbf{15} helde] kvnige L \textbf{16} er] \textit{om.} O  $\cdot$ grüezen] grusse Q (R) \textbf{18} solt] sol R  $\cdot$ wâr geleite] [warzacchen]: warzaichen O gewares geleyde Q woczeiche R \textbf{19} sîn] diu I \textbf{20} von] \textit{om.} I L R  $\cdot$ Anschouwe] anschoͮwe G antschawe I anschawe O Anschowe L (Q) (R) anschouw M antschowe Z \textbf{21} neic] neigt R  $\cdot$ dô] da M Z  $\cdot$ schrift] geschriffte R  $\cdot$ ersach] ersac G sach R \textbf{22} hœret wie] horet rehte wie G mvgt ir horen wie O (L) (M) (Q) (R) (Z)  $\cdot$ diu] er O \textbf{23} minne und] mein Q  $\cdot$ gruoz] ir grvz Z \textbf{24} mîn lîp] \textit{om.} I  $\cdot$ dem nie] demme R \textbf{25} enpfant] pfant Q \textbf{27} \textit{Die Verse 76.27-28 fehlen} O   $\cdot$ des] der L Z \textbf{28} töude] reide L toͯden R \textbf{29} sol] So R  $\cdot$ mir] mich L  $\cdot$ verren] werden Q \textbf{30} mir] ich Q \newline
\end{minipage}
\hspace{0.5cm}
\begin{minipage}[t]{0.5\linewidth}
\small
\begin{center}*T (U)
\end{center}
\begin{tabular}{rl}
 & ein wîp, die ich ê genennet hân,\\ 
 & \textbf{alhie} kam ir kappelân\\ 
 & und kleiner junchêrren drî.\\ 
 & den riten starke knappen bî.\\ 
5 & zwêne soumer \textbf{giengen} an \textbf{ir} hant.\\ 
 & die boten het\textit{e} dar gesant\\ 
 & diu künigîn Anflise.\\ 
 & ir kappellân was wîse.\\ 
 & vil schiere \textbf{erkante}r disen man.\\ 
10 & \textbf{in} franzoys \textbf{er in gruozte} sân:\\ 
 & "bensevinus, \textbf{beas sir},\\ 
 & mîner vrouwen und mir.\\ 
 & daz ist \textbf{ro\textit{y}s} de Franze,\\ 
 & \textbf{der werden minnen schanze}."\\ 
15 & \textbf{dem helde gab er} in die hant\\ 
 & \textbf{einen brief}, dâr an \textit{\textbf{er}} grüezen vant,\\ 
 & und ein kleine vingerlîn.\\ 
 & daz solte ein wâ\textit{r} geleite sîn,\\ 
 & wan \textbf{ez} entvienc \textbf{sîn} vrouwe\\ 
20 & von dem von Anschouwe.\\ 
 & er \textbf{neic}, dô er die \textbf{schrift} \textbf{sach}.\\ 
 & \textbf{welt ir} nû \textbf{hœren}, wie \textbf{diu} sprach?\\ 
 & "\begin{large}D\end{large}ir enbiutet minne und gruoz\\ 
 & mîn lîp, dem \textbf{nie} wart kumbers buoz,\\ 
25 & sît \textbf{er} dîner minnen \textit{en}pfant.\\ 
 & dîn minne \textbf{tuot ein slozbant}\\ 
 & mînes herzen und des vröude.\\ 
 & dîn minne tuot mich töude.\\ 
 & sol \textbf{mir} dîn minne verren,\\ 
30 & sô muoz mir minne werren.\\ 
\end{tabular}
\scriptsize
\line(1,0){75} \newline
U V W T \newline
\line(1,0){75} \newline
\textbf{1} \textit{Initiale} W T  \textbf{9} \textit{Majuskel} T  \textbf{21} \textit{Majuskel} T  \textbf{22} \textit{Majuskel} T  \textbf{23} \textit{Überschrift:} Der kúnigin anfolysen brieff W   $\cdot$ \textit{Platz für Illustration ausgespart} W   $\cdot$ \textit{Initiale} U V W T  \textbf{26} \textit{Majuskel} T  \newline
\line(1,0){75} \newline
\textbf{1} ein] SEin W (T)  $\cdot$ die ich ê] daz ich hie W \textbf{2} ir] ein ir T \textbf{3} junchêrren] iunkerlein W \textbf{6} hete] heten U [hetten]: hette V \textbf{7} Anflise] anfolyse W \textbf{8} wîse] so weise W \textbf{10} franzoys] franzois V frantzoys W  $\cdot$ er in gruozte] grvͦste er in T \textbf{11} [*]: Beschevenús beaschir V \textbf{13} roys de] Ros de U [Ro*]: Rogẏn de V roys in W Roin de T \textbf{14} [der*s*h*]: die ruͦrte din minne lanze V  $\cdot$ div rveret der minnen lanze T  $\cdot$ minnen] mynne W \textbf{16} er] \textit{om.} U  $\cdot$ grüezen] geschriben W wunder T \textbf{17} kleine] vil klaines W \textbf{18} wâr geleite] ware geleite U wortzeichen V wares gelait W \textbf{19} ez] er W  $\cdot$ sîn] div T \textbf{20} Anschouwe] Anschowe U V antschowe W \textbf{21} neic] naig ir W  $\cdot$ die schrift sach] die [geschr*]: geschrift ersach V die schrifft ersach W den brief gesach T \textbf{22} welt ir nû] Nv horent rehte T  $\cdot$ diu] er W der T \textbf{23} Dir] Iv T \textbf{24} mîn] Ein W  $\cdot$ kumbers] kummer W \textbf{25} er] [*]: ich V ich W T  $\cdot$ minnen] minne V W T  $\cdot$ enpfant] pant U \textbf{26} tuot ein slozbant] die ist [*]: slos vnd bant V ist mir ein schlosbant W ist slôz vnde bant T \textbf{27} \textit{Die Verse 76.27-28 fehlen} W   $\cdot$ des] der T \textbf{29} mir] mich T \textbf{30} minne] froͤde W \newline
\end{minipage}
\end{table}
\end{document}
