\documentclass[8pt,a4paper,notitlepage]{article}
\usepackage{fullpage}
\usepackage{ulem}
\usepackage{xltxtra}
\usepackage{datetime}
\renewcommand{\dateseparator}{.}
\dmyyyydate
\usepackage{fancyhdr}
\usepackage{ifthen}
\pagestyle{fancy}
\fancyhf{}
\renewcommand{\headrulewidth}{0pt}
\fancyfoot[L]{\ifthenelse{\value{page}=1}{\today, \currenttime{} Uhr}{}}
\begin{document}
\begin{table}[ht]
\begin{minipage}[t]{0.5\linewidth}
\small
\begin{center}*D
\end{center}
\begin{tabular}{rl}
\textbf{158} & \begin{large}I\end{large}waneten niht bevilte,\\ 
 & er\textbf{n} lêrten underm schilte\\ 
 & künsteclîch gebâren\\ 
 & \textbf{unt} der vîende schaden vâren.\\ 
5 & er bôt im in die hant ein sper,\\ 
 & daz was gar âne \textbf{sîne} ger.\\ 
 & \textbf{doch} \textbf{vrâgt} er\textbf{n}: "war zuo ist diz vrum?"\\ 
 & "swer gein dir ze\textbf{r} tjost kum,\\ 
 & dâ soltûz balde brechen,\\ 
10 & durch sînen schilt \textbf{verstechen}.\\ 
 & wiltû des vil getrîben,\\ 
 & man lobt dich vor den wîben."\\ 
 & Als uns diu âventiure giht,\\ 
 & von Kölne noch von Mastricht\\ 
15 & dechein schiltære entwürfe\textbf{n} baz,\\ 
 & denn \textbf{alser} \textbf{ûfem} orse saz.\\ 
 & Dô sprach er zIwaneten sân:\\ 
 & "lieber vriunt, mîn kumpân,\\ 
 & ich hân \textbf{hie} erworben, des ich bat.\\ 
20 & dû solt mîn dienst in die stat\\ 
 & dem künege Artuse sagen\\ 
 & unt ouch mîn hôhez laster klagen.\\ 
 & bring im wider sîn goltvaz.\\ 
 & ein ritter \textbf{sich an mir} vergaz,\\ 
25 & daz er die juncvouwen sluoc,\\ 
 & durch daz si lachens mîn gewuoc.\\ 
 & mich müent ir jæmerlîchen wort,\\ 
 & di\textit{u}\textbf{ne} \textbf{rüerent mir} \textbf{dechein} herzen ort;\\ 
 & \textbf{jâ muoz} enmitten drinne sîn\\ 
30 & der vrouwen ungedienter pîn.\\ 
\end{tabular}
\scriptsize
\line(1,0){75} \newline
D \newline
\line(1,0){75} \newline
\textbf{1} \textit{Initiale} D  \textbf{13} \textit{Majuskel} D  \textbf{17} \textit{Majuskel} D  \newline
\line(1,0){75} \newline
\textbf{14} Kölne] Choͤlne D \textbf{17} zIwaneten] zywaneten D \textbf{28} diune] dine D \newline
\end{minipage}
\hspace{0.5cm}
\begin{minipage}[t]{0.5\linewidth}
\small
\begin{center}*m
\end{center}
\begin{tabular}{rl}
 & Iwaneten niht bevilt,\\ 
 & er lêrte in under dem schilt\\ 
 & künsteclîch gebâren\\ 
 & \textbf{und} der vîende schaden vâren.\\ 
5 & er bôt ime in die hant ein sper,\\ 
 & daz was g\textit{a}r âne \textbf{sîne} ger.\\ 
 & \textbf{doch} \textbf{vrâgete} er: "war zuo ist diz vrume?"\\ 
 & "wer gegen dir ze\textbf{r} juste kume,\\ 
 & d\textit{â} solt dûz balde brechen,\\ 
10 & durch sînen schilt \textbf{verstechen}.\\ 
 & wiltû d\textit{e}s vil getrîben,\\ 
 & man lobet dich vor de\textit{n} wîben."\\ 
 & \begin{large}A\end{large}ls uns diu âventiure giht,\\ 
 & von Köln noch von Mastri\textit{ch}t\\ 
15 & dekein schilter entwürfe\textit{\textbf{n}} \textit{b}az,\\ 
 & danne \textbf{Parcifal} \textbf{ze} rosse saz.\\ 
 & dô sprach er ze Iwanete sân:\\ 
 & "lieber vriunt, mîn kumpân,\\ 
 & ich hân erworben, des ich bat.\\ 
20 & dû solt mîn dienest in die stat\\ 
 & dem künige Artuse sagen\\ 
 & und ouch mîn hôhez laster klagen.\\ 
 & bring ime wider sîn goltvaz.\\ 
 & ein ritter \textbf{sîner zuht} vergaz,\\ 
25 & daz er die juncvrouwen sluoc,\\ 
 & durch daz si lachens mîn gewuoc.\\ 
 & mich müegent ir jâmerlîchen wort,\\ 
 & diu \textbf{verrüerent mich in} \textbf{dekein} herzen ort;\\ 
 & \textbf{jâ muoz} enmitten drinne sîn\\ 
30 & der vrouwen ungedienter pîn.\\ 
\end{tabular}
\scriptsize
\line(1,0){75} \newline
m n o \newline
\line(1,0){75} \newline
\textbf{13} \textit{Initiale} m n  \newline
\line(1,0){75} \newline
\textbf{1} Iwaneten] Jwaneten m n Jwanten o \textbf{2} lêrte] lert n o \textbf{4} schaden] schade o \textbf{5} bôt] bat o \textbf{6} gar] ger m \textbf{7} diz] das n  $\cdot$ vrume] fromen n \textbf{8} kume] komen n \textbf{9} dâ] Du m Do n o  $\cdot$ solt dûz] solte es o \textbf{10} verstechen] zerstechen n durch stechen o \textbf{11} des] das m  $\cdot$ getrîben] triben n o \textbf{12} vor] von n  $\cdot$ den] dem m \textbf{14} Köln] koͯln m coͯllen n Collen o  $\cdot$ Mastricht] mastrit m o \textbf{15} dekein] Do kein n  $\cdot$ entwürfen baz] entwurffen in bas m \textbf{17} Iwanete] jwanete m n o \textbf{18} \textit{Die Verse 158.18-160.3 fehlen (Blattverlust)} o  \textbf{19} des] den n \textbf{21} Artuse] artusen n \textbf{26} mîn] jme n \textbf{27} müegent] muͯget n \textbf{28} in dekein] nit in n \newline
\end{minipage}
\end{table}
\newpage
\begin{table}[ht]
\begin{minipage}[t]{0.5\linewidth}
\small
\begin{center}*G
\end{center}
\begin{tabular}{rl}
 & Ywaneten niht bevilte,\\ 
 & er lêrte in under dem schilte\\ 
 & künsticlîch gebâren,\\ 
 & der vînde schaden vâren.\\ 
5 & er bôt im in die hant ein sper,\\ 
 & daz was gar âne \textbf{sîne} ger.\\ 
 & \textbf{doch} \textbf{vrâgte}r: "war zuo ist diz vrum?"\\ 
 & "swer gein dir ze tjost kum,\\ 
 & dâ soltûz balde brechen,\\ 
10 & durch sînen schilt \textbf{verstechen}.\\ 
 & wil dû des vil getrîben,\\ 
 & man lobet dich vor den wîben."\\ 
 & als uns diu âventiure giht,\\ 
 & von Kölne noch von Mastrieht\\ 
15 & dehein schiltære entwürfe \textbf{in} baz,\\ 
 & danne \textbf{alser} \textbf{ûf dem} orse saz.\\ 
 & dô sprach er ze Ywanete sân:\\ 
 & "lieber vriunt, mîn kumpân,\\ 
 & ich hân \textbf{hie} erworben, des ich bat.\\ 
20 & dû solt mîn dienst in die stat\\ 
 & dem künige Artuse sagen\\ 
 & unde ouch mîn hôhez laster klagen.\\ 
 & bringe im wider sîn goltvaz.\\ 
 & ein rîter \textbf{sich an mir} vergaz,\\ 
25 & daz er \textit{die} juncvrouwen sluoc,\\ 
 & durch daz si lachens mîn gewuoc.\\ 
 & mich müent ir jæmerlîchen wort,\\ 
 & diu \textbf{rüeren mir} \textbf{dehein} herzen ort;\\ 
 & \textbf{jâ muoz} enmitten drinne sîn\\ 
30 & der vrouwen ungedienter pîn.\\ 
\end{tabular}
\scriptsize
\line(1,0){75} \newline
G I O L M Q R Z Fr36 \newline
\line(1,0){75} \newline
\textbf{1} \textit{Illustration mit Überschrift:} Hie wappnet Jwan den parczifal mit des Rotten Ritteres geczug vnd zoch im das pfaͯrt dar daruff er prang one stegreiff vnd reit en weg R   $\cdot$ \textit{Initiale} O L Q R  \textbf{7} \textit{Initiale} I  \textbf{13} \textit{Initiale} M  \newline
\line(1,0){75} \newline
\textbf{1} Ywaneten] Juuaneten I ÷waneten O Jwaneten L Jwannen Q Iwatten R  $\cdot$ bevilte] beuilt I beuilhte R \textbf{2} er lêrte] Er enlerte L Er lernte M Er lert Q Er leit R Ern lert Z  $\cdot$ in] [im]: in Z  $\cdot$ dem] den M \textbf{3} gebâren] yn gibarin M geboren Q \textbf{4} der vînde] Vnde der viende O (Q) (R) (Z) Vnde den vienden M \textbf{5} die] sin I \textbf{6} sîne] \textit{om.} M Fr36  $\cdot$ ger] wer M \textbf{7} doch vrâgter] Do vragt er I Doch fragt er O (M) (Q) R Z Er sprach L :::t er Fr36  $\cdot$ war zuo] zwiu I waz zcu M waz Fr36  $\cdot$ ist] ist mir I  $\cdot$ diz] dize I O es R \textbf{8} swer] Wer L Q R  $\cdot$ ze tjost] zeTiostiern I (O) (L) zer tioste Q mit strit R \textbf{9} dâ] Do Q  $\cdot$ brechen] berchen G \textbf{10} verstechen] solt duͤz stechen I stechen L (Fr36) zerstechen R \textbf{11} getrîben] triben O L (Q) (Fr36) \textbf{12} vor den] werden I von den M \textbf{14} von] Zv Z  $\cdot$ Kölne] cholne G choln I koͤln O Colne L kolne M R koln Q koͤlne Z  $\cdot$ noch von] noch zv Z  $\cdot$ Mastrieht] mastriht I O (L) Z mastrich M mastriech R mahstriht Fr36 \textbf{15} entwürfe in] entwarf O in entwuͯrfe L in entwurffen Q entwafnet in R in worhte Fr36 \textbf{16} alser] so er O  $\cdot$ ûf dem] ze O (L) (Q) R Fr36  $\cdot$ saz] waz L \textbf{17} dô] Da O M Dv Z  $\cdot$ Ywanete] ẏwanete G iuuanete I Jwaneten O jwanete L Jwanten M ywanen Q Jwanetten R ywaneten Z Fr36  $\cdot$ sân] sone M \textbf{18} lieber] vil lieber I \textbf{19} hie] \textit{om.} I \textbf{21} Artuse] \textit{om.} I artus R artusen Z \textbf{22} ouch] \textit{om.} I O L M Q  $\cdot$ hôhez laster] laster hohez Q \textbf{23} goltvaz] golúas Q (R) \textbf{24} sich] sin I \textit{om.} O sin selbes L  $\cdot$ mir] Jm R \textbf{25} die] eine G \textbf{26} daz] \textit{om.} I  $\cdot$ lachens] lachin M  $\cdot$ mîn] Gegen mir I mir L  $\cdot$ gewuoc] bewuch Q gnuͦg R \textbf{27} müent] múet Q (R)  $\cdot$ jæmerlîchen] iamerlichev I (O) (Q) (R) (Z) \textbf{28} diu] daz I  $\cdot$ rüeren] enruͤrt I enrvͦrent O (Z) o\textit{m. } L giengent R  $\cdot$ mir dehein] mir nih des I (L) nicht mynes M mir des Q mir andaz R mich kein Z  $\cdot$ herzen] herczes R \textbf{29} jâ] Ez L Da M Do Q  $\cdot$ muoz enmitten] muͤz ez enmitten I musszin mitten M (Q)  $\cdot$ drinne] ynne M \textbf{30} ungedienter] vnuordieneter M  $\cdot$ pîn] sin R \newline
\end{minipage}
\hspace{0.5cm}
\begin{minipage}[t]{0.5\linewidth}
\small
\begin{center}*T (U)
\end{center}
\begin{tabular}{rl}
 & Ywaneten niht bevilte,\\ 
 & er\textbf{n} lêrtin underme schilte\\ 
 & künsteclîche gebâren\\ 
 & \textbf{und} der vîende schaden \textit{v}âren.\\ 
5 & er bôt im in die hant ein sper,\\ 
 & daz was gar ân \textbf{sîner} ger.\\ 
 & \textbf{Dô} \textbf{sprach} er: "war zuo ist diz vrome?"\\ 
 & "Swer gegen dir ze\textbf{r} tjoste kome,\\ 
 & dâ soltûz balde brechen\\ 
10 & \textbf{und} durch sînen schilt \textbf{stechen}.\\ 
 & wiltû des vil getrîben,\\ 
 & man lobtich vor den wîben."\\ 
 & \begin{large}A\end{large}ls uns diu âventiure giht,\\ 
 & v\textit{on} Colne noch von Mastriht\\ 
15 & dehein schiltære entwürfe baz,\\ 
 & danne\textbf{r} \textbf{ûf dem} orse saz.\\ 
 & dô sprach er ze Ywanete sân:\\ 
 & "Lieber vriunt, mîn kumpân,\\ 
 & ich hân erworben, des ich \textbf{hie} bat.\\ 
20 & dû solt mîn dienst in die stat\\ 
 & dem künege Artuse sagen\\ 
 & und ouch mîn hôhez laster klagen.\\ 
 & bringim wider sîn goltvaz.\\ 
 & Ein rîter \textbf{sîn an mir} vergaz,\\ 
25 & daz er die juncvrouwen sluoc,\\ 
 & durch daz si lachens mîn gewuoc.\\ 
 & mich müent ir jæmerlîchen wort,\\ 
 & di\textit{u} \textbf{rüerent mir} \textbf{mînes} herzen ort\\ 
 & \textbf{und müezen} mitten drinne sîn:\\ 
30 & der vrouwen ungedienter pîn.\\ 
\end{tabular}
\scriptsize
\line(1,0){75} \newline
T U V W \newline
\line(1,0){75} \newline
\textbf{7} \textit{Majuskel} T  \textbf{8} \textit{Majuskel} T  \textbf{13} \textit{Initiale} T U V W  \textbf{18} \textit{Majuskel} T  \textbf{24} \textit{Majuskel} T  \newline
\line(1,0){75} \newline
\textbf{1} \textit{Versdoppelung 157.25-158.10 (²T) nach 158.10; Fassungstext *T nach ²T mit Lesarten der vorausgehenden Verse (¹T) im Apparat} T   $\cdot$ Ywaneten] den kappen \textsuperscript{1}\hspace{-1.3mm} T [Y*]: Ywanette V Ywanet W \textbf{2} ern] Er \sout{in} U er V (W)  $\cdot$ underme] mit dem W \textbf{3} künsteclîche] Kuͦntliche U (V) \textbf{4} vâren] waren \textsuperscript{2}\hspace{-1.3mm} T \textbf{6} ân sîner] ane sine \textsuperscript{1}\hspace{-1.3mm} T [a*]: ane sine V \textbf{7} Dô sprach er] doch vrageter \textsuperscript{1}\hspace{-1.3mm} T  $\cdot$ diz] das W \textbf{8} Swer] Wer U W  $\cdot$ zer] ze V zuͦ einer W \textbf{9} dâ] Do U V W \textbf{10} und] \textit{om.} \textsuperscript{1}\hspace{-1.3mm} T W  $\cdot$ stechen] [*]: verstechen \textsuperscript{1}\hspace{-1.3mm} T verstechen W \textbf{11} getrîben] triben U V (W) \textbf{13} Als uns] Alsus U ALs vnd W \textbf{14} von] v T  $\cdot$ Colne] koͤlle V koͤlne W  $\cdot$ Mastriht] mastri U mastricht W \textbf{15} entwürfe] [*ntwúrfe]: entwúrfe in V \textbf{17} er] \textit{om.} U  $\cdot$ Ywanete] [ẏwanet*]: ẏwanete V ywanet W \textbf{19} hie bat] [*]: bat V \textbf{21} Artuse] Artuͦse U \textbf{23} bringim] bringen U \textbf{24} sîn] sich U V  $\cdot$ an mir vergaz] [*]: an im vergas V \textbf{25} die] \textit{om.} U  $\cdot$ juncvrouwen] iunckfrawe W \textbf{26} mîn] mir V \textit{om.} W \textbf{27} müent] muͤgen W  $\cdot$ jæmerlîchen] iemerliche U V W \textbf{28} diu] die T  $\cdot$ rüerent] ruͦrten W  $\cdot$ mir] \textit{om.} V \textbf{29} und] Die W \textbf{30} ungedienter] iemerleiche W \newline
\end{minipage}
\end{table}
\end{document}
