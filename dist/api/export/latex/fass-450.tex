\documentclass[8pt,a4paper,notitlepage]{article}
\usepackage{fullpage}
\usepackage{ulem}
\usepackage{xltxtra}
\usepackage{datetime}
\renewcommand{\dateseparator}{.}
\dmyyyydate
\usepackage{fancyhdr}
\usepackage{ifthen}
\pagestyle{fancy}
\fancyhf{}
\renewcommand{\headrulewidth}{0pt}
\fancyfoot[L]{\ifthenelse{\value{page}=1}{\today, \currenttime{} Uhr}{}}
\begin{document}
\begin{table}[ht]
\begin{minipage}[t]{0.5\linewidth}
\small
\begin{center}*D
\end{center}
\begin{tabular}{rl}
\textbf{450} & \begin{large}O\end{large}b ich kleinez dinc dâr ræche,\\ 
 & ungerne ich daz verspræche,\\ 
 & ich\textbf{n} holt einen kus durch suone dâ,\\ 
 & ob si der suone spræchen "jâ".\\ 
5 & wîp sint êt immer wîp.\\ 
 & \textbf{werlîches} mannes lîp\\ 
 & hânt si schiere betwungen.\\ 
 & in ist dicke \textbf{alsus} gelungen.\\ 
 & Parzival hie unde dort\\ 
10 & mit bet hôrte ir süezen wort,\\ 
 & des vater, muoter unt der kinde.\\ 
 & er dâhte: "ob ich erwinde,\\ 
 & ich \textbf{gên} ungerne in dirre schar.\\ 
 & dise meide sint sô wol gevar,\\ 
15 & daz mîn rîten bî in übel stêt,\\ 
 & sît man und wîp ze vuoz hie gêt.\\ 
 & sich vüeget mîn scheiden von in baz,\\ 
 & sît ich gein dem trage haz,\\ 
 & den si von herzen minnent\\ 
20 & und sich helfe dâ versinnent.\\ 
 & der hât sîne helfe mir verspart\\ 
 & und mich \textbf{von} sorgen niht bewart."\\ 
 & Parzival sprach zin dô sân:\\ 
 & "\textbf{hêrre und vrouwe}, lât mich hân\\ 
25 & iwern urloup. gelücke \textbf{iu} heil\\ 
 & \textbf{gebe} unt \textbf{vreuden} vollen teil.\\ 
 & \textbf{ir} juncvrouwen süeze,\\ 
 & iwer zuht iu danken müeze,\\ 
 & \textbf{sît} ir \textbf{gundet mir gemaches} wol.\\ 
30 & iwern urloup ich haben sol."\\ 
\end{tabular}
\scriptsize
\line(1,0){75} \newline
D Fr5 \newline
\line(1,0){75} \newline
\textbf{1} \textit{Initiale} D Fr5  \newline
\line(1,0){75} \newline
\textbf{3} einen] ein Fr5 \textbf{4} suone spræchen] volge sprêche Fr5 \textbf{6} mannes] manins Fr5 \textbf{7} hânt si] Hat Fr5 \textbf{9} Parzival] Parcifal D Fr5 \textbf{10} hôrte ir] hort er Fr5  $\cdot$ süezen] sivͦziv Fr5 \textbf{12} dâhte] gidahte Fr5 \textbf{16} ze vuoz hie] zefvͦezin Fr5 \textbf{21} der] er Fr5 \textbf{22} von] vor Fr5 \textbf{23} Parzival] Parcifal D Fr5 \textbf{24} hêrre und vrouwe] Froͮwe vnd herre Fr5 \textbf{25} iwern urloup] Vrhap Fr5  $\cdot$ iu] vnd Fr5 \textbf{26} unt vreuden] iv got Fr5 \textbf{27} ir] Mine Fr5 \textbf{29} ir gundet mir] ir mir gundin Fr5 \textbf{30} iwern] Jvwir Fr5 \newline
\end{minipage}
\hspace{0.5cm}
\begin{minipage}[t]{0.5\linewidth}
\small
\begin{center}*m
\end{center}
\begin{tabular}{rl}
 & ob ich kleinez dinc dâ ræche,\\ 
 & ungerne ich daz verspræche,\\ 
 & ich holte einen kus durch suone d\textit{â},\\ 
 & ob si der suone spræchen "jâ".\\ 
5 & wîp sint eht iemer wîp.\\ 
 & \textbf{werlîches} mannes lîp\\ 
 & hâ\textit{n}t si schiere betwungen.\\ 
 & in ist dicke \textbf{alsus} gelungen.\\ 
 & Parcifal hie und dort\\ 
10 & mit bete hôrte ir süezen wort,\\ 
 & des vater, muoter und der kinde.\\ 
 & er dâhte: "ob ich erwinde,\\ 
 & ich \textbf{gên} ungerne in dirre schar.\\ 
 & dise megde sint sô wol gevar,\\ 
15 & daz mîn rîten bî in übel stât,\\ 
 & sît man und wîp ze vuoze hie gât.\\ 
 & si\textit{ch} vüeget mîn scheiden von in baz,\\ 
 & sît ich gegen dem trage haz,\\ 
 & den si von herzen minnent\\ 
20 & und sich helfe d\textit{â} versinne\textit{n}t.\\ 
 & d\textit{e}r hât sîne helfe mir vers\textit{p}art\\ 
 & und mich \textbf{vor} sorgen niht bewart."\\ 
 & Parcifal sprach zuo i\textit{n} dô sân:\\ 
 & "\textbf{hêrre und vrouwe}, lât mich \textit{h}ân\\ 
25 & iuwern urloup, glücke \textbf{und} heil,\\ 
 & \textbf{gebe} und \textbf{vröuden} vollen teil.\\ 
 & \textbf{ir} juncvrouwen süeze,\\ 
 & iuwer zuht iu danken müeze,\\ 
 & \textbf{daz} ir \textbf{mir gemaches günnet} wol.\\ 
30 & iuwer\textit{n} urloup ich haben sol."\\ 
\end{tabular}
\scriptsize
\line(1,0){75} \newline
m n o \newline
\line(1,0){75} \newline
\newline
\line(1,0){75} \newline
\textbf{3} dâ] do m n \textbf{4} suone spræchen] sine spreche o \textbf{7} hânt] Hat m \textbf{10} hôrte] hort n o \textbf{11} der] ir n \textbf{12} dâhte] gedochte n  $\cdot$ erwinde] erwuͯnde o \textbf{13} gên] gange n (o) \textbf{17} sich] Sit m \textbf{19} minnent] nymmet o \textbf{20} dâ versinnent] do versinnet m entsinnent n ensimmet o \textbf{21} der] Dar m  $\cdot$ verspart] versprart m \textbf{23} in dô] yme do m jme n in o \textbf{24} hân] san m \textbf{28} müeze] [hiesse]: muͯsse n musz o \textbf{29} günnet] gúnnen n \textbf{30} iuwern] Vͯwer m \newline
\end{minipage}
\end{table}
\newpage
\begin{table}[ht]
\begin{minipage}[t]{0.5\linewidth}
\small
\begin{center}*G
\end{center}
\begin{tabular}{rl}
 & \begin{large}O\end{large}b ich kleinez dinc dâ ræche,\\ 
 & ungern ich daz verspræche,\\ 
 & ich\textbf{ne} holte einen \textit{k}us durch suone dâ,\\ 
 & op si der suone spræchen "jâ".\\ 
5 & wîp sint êt immer wîp.\\ 
 & \textbf{wertlîches} mannes lîp\\ 
 & hânt si schier betwungen.\\ 
 & in ist dicke \textbf{alsus} gelungen.\\ 
 & Parzival hie unde dort\\ 
10 & mit bet hôrte ir süeziu wort,\\ 
 & des vater, muoter unde der kinde.\\ 
 & er dâhte: "ob ich erwinde,\\ 
 & ich \textbf{gên} ungerne in dirre schar.\\ 
 & dise meide sint sô wolgevar,\\ 
15 & daz mîn rîten bî in übel stêt,\\ 
 & sît man unde wîp ze vuoze hie gêt.\\ 
 & sich vüeget mîn scheiden von in baz,\\ 
 & sît ich gein dem trage haz,\\ 
 & den s\textit{i} von herzen minnent\\ 
20 & unt sich helfe dâ versinnent.\\ 
 & der hât sîn helfe mir verspart\\ 
 & unde mich \textbf{von} sorgen niht bewart."\\ 
 & Parzival sprach zin dô sân:\\ 
 & "\textbf{vrouwe unde hêrre}, lât mich hân\\ 
25 & iwern urloup. gelücke \textbf{iu} heil\\ 
 & \textbf{gebe} unde \textbf{vröuden} vollen teil.\\ 
 & \textbf{mîne} juncvrouwen süeze,\\ 
 & iuwer zuht iu danken müeze,\\ 
 & \textbf{sît} ir \textbf{gundet mir gemaches} wol.\\ 
30 & iuwern urloup ich haben sol."\\ 
\end{tabular}
\scriptsize
\line(1,0){75} \newline
G I O L M Z \newline
\line(1,0){75} \newline
\textbf{1} \textit{Initiale} G O L M Z  \textbf{15} \textit{Initiale} I  \newline
\line(1,0){75} \newline
\textbf{1} Ob] ÷b O  $\cdot$ dinc] [dich]: dinch G  $\cdot$ dâ] tar M \textbf{3} ichne holte] Jch halde en M  $\cdot$ kus] cius G  $\cdot$ suone] svͦnen O \textbf{4} der suone spræchen] sprechen mit suͦne I \textbf{5} êt] \textit{om.} I ouch M  $\cdot$ immer] myner M \textbf{6} wertlîches] werliches I (M) (Z) \textbf{7} hânt si] hat si I het L Han sie M \textbf{8} in] nu I  $\cdot$ alsus] wol I \textbf{9} Parzival] Parziual G Parzifale I Parcifal O Z Parzifal L M \textbf{10} süeziu] suszen M (Z) \textbf{11} muoter] der muͯter L (Z)  $\cdot$ kinde] [kint]: kinde I \textbf{12} dâhte] gedachte L \textbf{13} gên ungerne in] wil gerne Gein I \textbf{16} unde] oder I  $\cdot$ ze vuoze hie gêt] bi in zevuszen gent I zefvͦzen hie get O hie zcu fusze [stet]: get M zv fvzze get Z \textbf{18} dem] ym M \textbf{19} si] sin G \textbf{20} helfe dâ] helf an in I da helfe L \textbf{21} mir] vor mir I an mir L  $\cdot$ verspart] [verspurt]: verspart I \textbf{22} von sorgen] vor sorgen O (M) Z  $\cdot$ bewart] er art L \textbf{23} Parzival] Parziual G Parzifal I L M Parcifal O Z  $\cdot$ zin] zcu yme M (Z)  $\cdot$ dô] \textit{om.} M \textbf{24} vrouwe unde hêrre] frowen vnd herren I Herre vnde frowe O (L) (M) (Z) \textbf{25} iwern] ewer I (M) (Z) o\textit{m. } O  $\cdot$ iu] vnd I Z \textbf{26} gebe] geb ev I (Z) Gebe iv got O  $\cdot$ unde] \textit{om.} Z  $\cdot$ vollen] allen O \textbf{27} mîne] \textit{om.} I Jr O L M Z  $\cdot$ süeze] suszen M \textbf{29} gundet mir gemaches] gemaches gunnet mir so I mir gvndet gemaches O (L) mir gunnet gimachis M \textbf{30} iuwern] ewer I \newline
\end{minipage}
\hspace{0.5cm}
\begin{minipage}[t]{0.5\linewidth}
\small
\begin{center}*T
\end{center}
\begin{tabular}{rl}
 & \textit{\begin{large}O\end{large}b ich} kleine\textit{z} dinc dâ ræche,\\ 
 & ungerne ich daz verspræche,\\ 
 & i\textbf{ne} holte einen kus durch suone dâ,\\ 
 & ob si der suone spræchen "jâ".\\ 
5 & wîp sint eht iemer wîp.\\ 
 & \textbf{werlîches} mannes lîp\\ 
 & hânt si schiere betwungen.\\ 
 & in ist dicke \textbf{alsô} gelungen.\\ 
 & Parcifal hie unde dort\\ 
10 & mit bete hôrt\textit{i}r süezen wort,\\ 
 & des vater, \textbf{der} muoter unde der kinde.\\ 
 & er dâhte: "ob ich erwinde,\\ 
 & ich \textbf{gâ} ungerne in dirre schar.\\ 
 & dise megde sint sô wol gevar,\\ 
15 & daz mîn rîten bî in übel stêt,\\ 
 & sît man unde wîp ze vuoz hie gêt.\\ 
 & sich vüeget mîn scheiden von in baz,\\ 
 & sît ich gegen dem trage haz,\\ 
 & den si von herzen minnent\\ 
20 & unde sich helfe dâ versinnent.\\ 
 & der hât sîne helfe \textbf{gegen} mir verspart\\ 
 & unde mich \textbf{vor} sorgen niht bewart."\\ 
 & Parcifal sprach zin dô sân:\\ 
 & "\textbf{hêrre unde vrouwen}, lât mich hân\\ 
25 & iuwern urloup. glücke \textbf{unde} heil\\ 
 & \textbf{geb}\textbf{iu got} unde vollen teil\\ 
 & \textbf{vröuden}. juncvrouwen süeze,\\ 
 & iuwer zuht iu danken müeze,\\ 
 & \textbf{sît} ir \textbf{gundet mir gemaches} wol.\\ 
30 & iuwern urloup ich haben sol."\\ 
\end{tabular}
\scriptsize
\line(1,0){75} \newline
T U V W Q R \newline
\line(1,0){75} \newline
\textbf{1} \textit{Initiale} T U W Q   $\cdot$ \textit{Capitulumzeichen} R  \textbf{23} \textit{Initiale} W  \newline
\line(1,0){75} \newline
\textbf{1} Ob ich kleinez] K cleines T Ob ich ein cleines R  $\cdot$ dâ] dar ane V do Q \textbf{2} verspræche] spreche Q \textbf{3} ine] Jch U R  $\cdot$ einen] ein W  $\cdot$ durch suone] [du*]: durch svͦnne V  $\cdot$ dâ] do U V W \textbf{4} spræchen] spreche R \textbf{5} eht] \textit{om.} U \textbf{7} hânt] Hat R \textbf{8} dicke] \textit{om.} R  $\cdot$ alsô] alsvz V \textbf{9} Parcifal] Parzifal V Partzifal W Q Parczifal R \textbf{10} bete] \textit{om.} U  $\cdot$ hôrtir] horter T hort ir W Q  $\cdot$ süezen] suͦze U (W) (Q) (R) \textbf{11} unde] \textit{om.} U V W R \textbf{12} erwinde] erwúnde Q \textbf{14} sô] \textit{om.} U R \textbf{16} ze vuoz hie] hie zu fusze Q \textbf{17} sich vüeget mîn scheiden] [*]: Sich scheiden fuͯget mich R \textbf{18} gegen] gerne Q \textbf{19} herzen] herten Q \textbf{20} sich] \textit{om.} W  $\cdot$ dâ] do V Q an ir W \textbf{21} hât] [hate]: hat U  $\cdot$ sîne] \textit{om.} Q mir sine R  $\cdot$ gegen] \textit{om.} W Q R  $\cdot$ mir] \textit{om.} R \textbf{22} vor] von W (Q) \textbf{23} Parcifal] Parzifal V PArtzifal W (Q) Parczifal R  $\cdot$ zin] zim V (Q)  $\cdot$ dô] da U \textit{om.} W R \textbf{24} vrouwen] vrowe V (Q) \textbf{25} iuwern] V́wer R  $\cdot$ glücke] [g*e]: gelv́ke V  $\cdot$ unde] eúch W \textbf{26} gebiu got] Erbe W Got euch gebe Q gebe R  $\cdot$ unde] \textit{om.} Q  $\cdot$ vollen teil] vollez heil U [*]: froͤiden vollen teil V froͤden vollen teil W (Q) (R) \textbf{27} vröuden] [*]: Vroͤiden vil V Ir W (Q) (R) \textbf{29} gundet mir] guͦnnet mir U mir gúnt W (R) \textbf{30} iuwern] Vwer R \newline
\end{minipage}
\end{table}
\end{document}
