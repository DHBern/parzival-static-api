\documentclass[8pt,a4paper,notitlepage]{article}
\usepackage{fullpage}
\usepackage{ulem}
\usepackage{xltxtra}
\usepackage{datetime}
\renewcommand{\dateseparator}{.}
\dmyyyydate
\usepackage{fancyhdr}
\usepackage{ifthen}
\pagestyle{fancy}
\fancyhf{}
\renewcommand{\headrulewidth}{0pt}
\fancyfoot[L]{\ifthenelse{\value{page}=1}{\today, \currenttime{} Uhr}{}}
\begin{document}
\begin{table}[ht]
\begin{minipage}[t]{0.5\linewidth}
\small
\begin{center}*D
\end{center}
\begin{tabular}{rl}
\textbf{771} & \textit{\begin{large}I\end{large}}ch hete ein dinc vür schande:\\ 
 & man jach in \textbf{mîme} lande,\\ 
 & nehein \textbf{rîter bezzer} m\textit{ö}hte sîn\\ 
 & denne Gahmuret Anschevin,\\ 
5 & der ie ors \textbf{überschrite}.\\ 
 & \textbf{ez} was mîn wille unt \textbf{ouch} mîn \textbf{site},\\ 
 & daz ich \textbf{vüere}, unz ich in vünde.\\ 
 & sît gewan ich strîtes künde.\\ 
 & \textbf{Von mînen zwein landen} her\\ 
10 & \textbf{vuort ich kreftec} ûfez mer.\\ 
 & gein rîterschefte het ich muot.\\ 
 & swelech lant was werlîch unt guot,\\ 
 & daz twang ich mîner hende\\ 
 & unze verre inz ellende.\\ 
15 & Dâ werten mich ir minne\\ 
 & zwô rîche küneginne,\\ 
 & Olimpia unt Clauditte.\\ 
 & Secundille ist nû diu dritte.\\ 
 & ich hân durch wîp vil getân.\\ 
20 & hiute alrêst ich künde hân,\\ 
 & daz mîn vater \textbf{Gahmuret} ist tôt.\\ 
 & mîn bruoder \textbf{sage} ouch sîne nôt."\\ 
 & Dô sprach der werde Parzival:\\ 
 & "sît ich schiet vonme Grâl,\\ 
25 & sô hât mîn hant mit strîte\\ 
 & in \textbf{der enge} unt an der wîte\\ 
 & vil rîterschefte erzeiget,\\ 
 & etslîches prîses geneiget,\\ 
 & Der des \textbf{was ungewent} ie.\\ 
30 & \textbf{die wil ich iu nennen} hie:\\ 
\end{tabular}
\scriptsize
\line(1,0){75} \newline
D Fr2 \newline
\line(1,0){75} \newline
\textbf{1} \textit{Initiale} D Fr2  \textbf{9} \textit{Majuskel} D  \textbf{15} \textit{Majuskel} D  \textbf{23} \textit{Initiale} Fr2   $\cdot$ \textit{Majuskel} D  \textbf{29} \textit{Majuskel} D  \newline
\line(1,0){75} \newline
\textbf{1} Ich] ÷ch \textit{nachträglich korrigiert zu:} Jch D ÷ch Fr2 \textbf{3} möhte] mohte D Fr2 \textbf{4} Gahmuret] Gahmvret D Gamvret Fr2  $\cdot$ Anschevin] Anscivin D Angivin Fr2 \textbf{7} Daz ich in fvnde Fr2 \textbf{10} ûfez] vber Fr2 \textbf{18} Secundille] Secvndilla Fr2 \textbf{21} Gahmuret] Gahmvret D Gamvret Fr2 \textbf{23} Dô] ÷o Fr2  $\cdot$ Parzival] Parcifal D partzefal Fr2 \textbf{26} an] in Fr2 \newline
\end{minipage}
\hspace{0.5cm}
\begin{minipage}[t]{0.5\linewidth}
\small
\begin{center}*m
\end{center}
\begin{tabular}{rl}
 & \begin{large}I\end{large}ch het ei\textit{n d}inc vür schande:\\ 
 & man jach in \textbf{mînem} lande,\\ 
 & dekein \textbf{ritter \textit{bezzer}} m\textit{ö}hte sîn\\ 
 & dan Gahmuret A\textit{n}schevin,\\ 
5 & der ie ros \textbf{überschrite}.\\ 
 & \textbf{dô} was mîn wille und mîn \textbf{site},\\ 
 & daz ich \textbf{vüer}, unz ich in vünde.\\ 
 & sît g\textit{e}wan \textit{i}ch strîtes künde.\\ 
 & \textbf{ich vuorte harte kreftic} her\\ 
10 & \textbf{von mînen landen} ûf daz mer.\\ 
 & gegen ritterschaft het ich muot.\\ 
 & welich lant was werlîch und guot,\\ 
 & daz twanc ich mîner hende\\ 
 & unz verre \textit{in} daz ellende.\\ 
15 & d\textit{â} werten mich ir minne\\ 
 & zwô rîch küniginne,\\ 
 & Olimpia und Clauditte.\\ 
 & Secundille ist nû diu dritte.\\ 
 & ich hân durch wîp vil getân.\\ 
20 & hiute allerêrst ich künde hân,\\ 
 & daz mîn vater \textbf{Gahmuret} ist tôt.\\ 
 & mîn bruoder \textbf{sage} ouch sîn nôt."\\ 
 & dô sprach der werde Parcifal:\\ 
 & "sît ich schiet von dem Grâl,\\ 
25 & sô het mîn hant mit strît\\ 
 & in \textbf{der enge} und an der wît\\ 
 & vil ritterschaft erz\textit{ei}get,\\ 
 & etlîches prîses geneiget,\\ 
 & der des \textbf{was ungewenet} ie.\\ 
30 & \textbf{der wil ich ein teil nennen} hie:\\ 
\end{tabular}
\scriptsize
\line(1,0){75} \newline
m n o V V' W Fr6 \newline
\line(1,0){75} \newline
\textbf{1} \textit{Initiale} m V W Fr6   $\cdot$ \textit{Capitulumzeichen} n  \textbf{23} \textit{Initiale} V V' W Fr6  \newline
\line(1,0){75} \newline
\textbf{1} \textit{Die Verse 771.1-8 fehlen} V'   $\cdot$ ein dinc] ein konig ding m \textbf{2} mînem] minen o \textbf{3} dekein] Do kein n Das kein W  $\cdot$ ritter bezzer] ritter m n o besser ritter W  $\cdot$ möhte] mohte m (o) [m*ohte]: moͤhte  V \textbf{4} Gahmuret] gahmuͯret m gamiret n gahmuͯeret o Gamuret V (W) Gahmvret Fr6  $\cdot$ Anschevin] auscevin m n ansce vin o (Fr6) Anscefin V antscheuein W \textbf{6} dô] Das W (Fr6)  $\cdot$ und] vnde oͮch V (Fr6) \textbf{7} vünde] frunde o \textbf{8} gewan] gawan m n o  $\cdot$ ich] uͯch m \textbf{9} harte] gar ein V' \textbf{10} mînen] [min*]: minen V mime V'  $\cdot$ landen] lande V'  $\cdot$ ûf daz] vffens V \textbf{12} welich] Swelich V (Fr6)  $\cdot$ werlîch] werhaft V' \textbf{13} mîner] mit myner V' (W) (Fr6) \textbf{14} in] \textit{om.} m \textbf{15} \textit{Die Verse 771.15-19 fehlen} V'   $\cdot$ dâ] Do m n o V W \textbf{17} Olimpia] Ollimpia m Olinpia n Olẏmpia o  $\cdot$ Clauditte] claudite m n o klauditte W clavditte Fr6 \textbf{18} Secundille] Secund::: o  $\cdot$ nû] \textit{om.} n ẏm o \textbf{19} wîp] sy W \textbf{20} \textit{statt 771.20-22:} Nu alrest han ich ervarn die not / Daz min vater mir ist tot V'  \textbf{21} Gahmuret] gahmuͯret m gamiret n gahmnutet o [Gam*]: Gamnuret V gamuret W Gahmvret Fr6 \textbf{22} mîn] Meinem W  $\cdot$ ouch] uch o ich auch W  $\cdot$ sîn] myn o \textbf{23} Parcifal] parzefal V parzifal V' partzifal W \textbf{24} sît] seit das W  $\cdot$ schiet] \textit{om.} n \textbf{25} het] hat min hat V'  $\cdot$ mit] mir o \textbf{26} an] in W \textbf{27} \textit{statt 771.27-30:} Vil ritterschaft erzeiget [ydo]: yso / Der wil ich ein teil teil nennen do V'   $\cdot$ erzeiget] erzoͯget m erzoiget o [er*]: erzeiget V \textbf{29} ungewenet] vngewonet W \newline
\end{minipage}
\end{table}
\newpage
\begin{table}[ht]
\begin{minipage}[t]{0.5\linewidth}
\small
\begin{center}*G
\end{center}
\begin{tabular}{rl}
 & ich het ein dinc vür schande:\\ 
 & man jach in \textbf{mînem} lande,\\ 
 & \textbf{daz} dehein \textbf{bezzer rîter} möhte sîn\\ 
 & danne Gahmuret Antschevin,\\ 
5 & der ie ors \textbf{überschrite}.\\ 
 & \textbf{ez} was mîn wille unde \textbf{ouch} mîn \textbf{site},\\ 
 & daz ich \textbf{in suochte}, unze ich in vünde.\\ 
 & sît gewan ich strîtes künde.\\ 
 & \textbf{von mînen zwein landen} her\\ 
10 & \textbf{vuort ich kreftec} ûf daz mer.\\ 
 & gein rîterschefte het ich muot.\\ 
 & swelch lant was werlîch unde guot,\\ 
 & daz twanc ich mîner hende\\ 
 & unze verre inz ellende.\\ 
15 & dâ werten mich ir minne\\ 
 & z\textit{w}uo rîche küniginne,\\ 
 & Olimpia unde Claudite.\\ 
 & Secundille ist nû diu drite.\\ 
 & ich hân durch wîp vil getân.\\ 
20 & hiute alrêst ich künde hân,\\ 
 & daz mîn vater ist tôt.\\ 
 & mîn bruoder \textbf{saget} ouch sîne nôt."\\ 
 & dô sprach der werde Parzival:\\ 
 & "sît ich schiet vonme Grâl,\\ 
25 & sô hât mîn hant mit strîte\\ 
 & in \textbf{gedrenge} unde an der wîte\\ 
 & vil rîterschefte erzeiget,\\ 
 & etslîches prîses geneiget,\\ 
 & der des \textbf{vil} \textbf{ungewent was} ie.\\ 
30 & \textbf{ein teil ich der benenne} hie:\\ 
\end{tabular}
\scriptsize
\line(1,0){75} \newline
G I L M Z Fr18 Fr46 Fr72 \newline
\line(1,0){75} \newline
\textbf{15} \textit{Initiale} I  \newline
\line(1,0){75} \newline
\textbf{2} man jach] iehe man I Man sprach M  $\cdot$ in] \textit{om.} L  $\cdot$ mînem] minen I mangem L (M) (Z) (Fr18) (Fr72) \textbf{3} daz] \textit{om.} L  $\cdot$ bezzer rîter] riter bezzer I  $\cdot$ möhte] mohte I L (M) Z Fr18 (Fr72)  $\cdot$ sîn] gesin Z \textbf{4} Gahmuret] Gahmv̂ret G Gamuret M (Z)  $\cdot$ Antschevin] anschevin G Antsheuin I Anshevin L (Z) Fr18 Ansevin M :nschow: Fr72 \textbf{5} der ie] Dy ir M  $\cdot$ überschrite] ubir schriten M \textbf{6} ouch] \textit{om.} L \textbf{7} unze] vnz daz I bisz M  $\cdot$ ich in vünde] ich ich fvnde L ihn fvnde Fr18 \textbf{9} landen] lander I \textbf{12} swelch] Welch L (M) \textbf{13} twanc] bitwanc M  $\cdot$ mîner] mit myner L \textbf{15} dâ] Do I  $\cdot$ werten] werte M \textbf{16} zwuo] zoͮ G \textbf{17} Olimpia] Polimpie G (I) M Olympie L Olimpie Z Polẏmpie Fr18  $\cdot$ Claudite] Clauditte I Z (Fr18) \textbf{18} Secundille] Secuntille I \textbf{19} wîp] diu wip I \textbf{20} alrêst ich] ich alreste L \textbf{21} vater] vatter Gahmvret L (Fr18) vater gamuret M Z  $\cdot$ ist] were I \textbf{22} saget] chlagt I sage M Z Fr18 \textbf{23} dô] Da M  $\cdot$ Parzival] parcifal G Z (Fr18) Parzifal I (L) (M) \textbf{25} mîn hant mit] min hant min hant mt I \textbf{26} in] An L  $\cdot$ an der] inder I \textbf{28} etslîches] vnd etlichs I Ettislichen M Etslich Fr18 \textbf{29} des] is M  $\cdot$ vil ungewent] gewent L \textbf{30} ich der] ich ir I ich dir L der Z  $\cdot$ benenne] bekenne Z benne Fr18 \newline
\end{minipage}
\hspace{0.5cm}
\begin{minipage}[t]{0.5\linewidth}
\small
\begin{center}*T
\end{center}
\begin{tabular}{rl}
 & \textit{\begin{large}I\end{large}}ch het ein dinc vür schande:\\ 
 & man jach in \textbf{manege\textit{m}} lande,\\ 
 & \textbf{daz} dekein \textbf{rîter bezzer} m\textit{ö}hte sîn\\ 
 & \textit{danne} Gahmuret Anschevin,\\ 
5 & der ie ors \textbf{überschreit}.\\ 
 & \textbf{ez} was mîn wille und \textbf{ouch} mîn \textbf{leit},\\ 
 & daz \textit{ich} \textbf{in suochte}, unz \textbf{daz} ich in vünde.\\ 
 & sît gewan ich strîtes künde.\\ 
 & \textbf{von mînen zwein landen} her\\ 
10 & \textbf{vuor ich kreftic} ûf daz mer.\\ 
 & gein rîterschefte hât \textit{ich} muot.\\ 
 & welch lant was werlîch und guot,\\ 
 & daz twanc ich mîner hende\\ 
 & u\textit{n}z verre in daz ellende.\\ 
15 & dâ werte\textit{n} mich ir minne\\ 
 & zwô rîche küneginne,\\ 
 & Olimpia und Claudite.\\ 
 & Secundille ist nû diu drite.\\ 
 & ich \textit{hân} durch wîp vil getân.\\ 
20 & hiute alrêrst ich künde hân,\\ 
 & daz mî\textit{n v}ater \textbf{Gahmuret} ist tôt.\\ 
 & mîn bruoder \textbf{sage} ouch sîne nôt."\\ 
 & \begin{large}D\end{large}ô sprach der werde Parcifal:\\ 
 & "sît ich schiet von dem Grâl,\\ 
25 & sô hât mîn hant mit strîte\\ 
 & in \textbf{dem gedrenge} und an der wîte\\ 
 & vil rîterschefte erzeiget,\\ 
 & etslîches prîses geneiget,\\ 
 & der des \textbf{ungewent was} ie.\\ 
30 & \textbf{ein tei\textit{l i}ch der benenne} hie:\\ 
\end{tabular}
\scriptsize
\line(1,0){75} \newline
U Q R Fr53 \newline
\line(1,0){75} \newline
\textbf{1} \textit{Initiale} U Q Fr53  \textbf{23} \textit{Initiale} U  \newline
\line(1,0){75} \newline
\textbf{1} \textit{Die Verse 764.13-774.30 fehlen} R   $\cdot$ Jch hett ein schande Q  $\cdot$ Ich] Ach U \textbf{2} manegem] manegen U meim Fr53 \textbf{3} daz] \textit{om.} Fr53  $\cdot$ rîter bezzer] bezzer ritter Fr53  $\cdot$ möhte] mochte U Q (Fr53) \textbf{4} danne] \textit{om.} U  $\cdot$ Gahmuret] Gahmuͦret U gamúret Q  $\cdot$ Anschevin] antschevein Fr53 \textbf{5} überschreit] [vber strite]: vber schrite Q vber schrite Fr53 \textbf{6} ouch] \textit{om.} Fr53  $\cdot$ leit] site Q Fr53 \textbf{7} ich in suochte] in suͦchte U ich fvͦre Fr53  $\cdot$ unz daz] mit daz U bisz Q vntz Fr53 \textbf{9} mînen] meinem Q \textbf{10} vuor] furt Q (Fr53) \textbf{11} ich] \textit{om.} U \textbf{12} welch] swelch Fr53 \textbf{14} unz] Vz U \textbf{15} dâ] Do Q  $\cdot$ werten] werte U Q \textbf{17} Olimpia] Olimpe Fr53  $\cdot$ Claudite] Cladite U claditte Q chlavditte Fr53 \textbf{18} Secundille] Secuͦndille U Secűndille Q  $\cdot$ ist] \textit{om.} Fr53 \textbf{19} hân] \textit{om.} U \textbf{21} mîn] min G U  $\cdot$ Gahmuret] Gahmuͦret U gamúret Q \textit{om.} Fr53 \textbf{22} sage] sach Q sagt Fr53 \textbf{23} Parcifal] partzifal Q \textbf{25} hât] hant Q  $\cdot$ mit] an Fr53 \textbf{26} dem gedrenge] gedrenge Q der enge Fr53  $\cdot$ an] avf Fr53  $\cdot$ wîte] weide Q \textbf{28} prîses geneiget] geneiget preisz Q \textbf{29} ungewent] vil vngewent Q Fr53 \textbf{30} teil] deil deil U  $\cdot$ hie] alhie Q \newline
\end{minipage}
\end{table}
\end{document}
