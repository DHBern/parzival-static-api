\documentclass[8pt,a4paper,notitlepage]{article}
\usepackage{fullpage}
\usepackage{ulem}
\usepackage{xltxtra}
\usepackage{datetime}
\renewcommand{\dateseparator}{.}
\dmyyyydate
\usepackage{fancyhdr}
\usepackage{ifthen}
\pagestyle{fancy}
\fancyhf{}
\renewcommand{\headrulewidth}{0pt}
\fancyfoot[L]{\ifthenelse{\value{page}=1}{\today, \currenttime{} Uhr}{}}
\begin{document}
\begin{table}[ht]
\begin{minipage}[t]{0.5\linewidth}
\small
\begin{center}*D
\end{center}
\begin{tabular}{rl}
\textbf{746} & \begin{large}D\end{large}ô disiu rede von im geschach,\\ 
 & Parzival zem heiden sprach:\\ 
 & "wâ von sît ir ein Anschevin?\\ 
 & Anschouwe ist von erbe mîn,\\ 
5 & bürge, lant unt stete.\\ 
 & hêrre, ir sult durch mîne bete\\ 
 & einen anderen namen kiesen.\\ 
 & \textbf{solt} ich mîn lant verliesen\\ 
 & unt die werden stat Bealzenan,\\ 
10 & sô het ir mir gewalt getân.\\ 
 & \textbf{ist unser deweder} ein Anschevin,\\ 
 & daz sol ich von arde sîn.\\ 
 & Doch ist mir vür wâr gesagt,\\ 
 & daz ein helt unverzagt\\ 
15 & won in der heidenschaft,\\ 
 & \textbf{der} habe mit rîterlîcher kraft\\ 
 & minne unt prîs behalten,\\ 
 & daz er \textbf{muoz} beider walten.\\ 
 & der ist ze bruoder mir \textbf{benant};\\ 
20 & si hânt in dâ vür prîs erkant."\\ 
 & Aber sprach \textbf{dô} Parzival:\\ 
 & "hêrre, iwers antlützes mâl,\\ 
 & het ich diu kuntlîche ersehen,\\ 
 & sô würde iu schiere von mir verjehen,\\ 
25 & als \textbf{er} mir kunt ist getân.\\ 
 & hêrre, welt irz an mich lân,\\ 
 & sô enblœzet iwer houbet.\\ 
 & ob ir mirz geloubet,\\ 
 & mîn hant iuch strîtes gar verbirt,\\ 
30 & unz ez ander stunt gewâpent wirt."\\ 
\end{tabular}
\scriptsize
\line(1,0){75} \newline
D \newline
\line(1,0){75} \newline
\textbf{1} \textit{Initiale} D  \textbf{13} \textit{Majuskel} D  \textbf{21} \textit{Majuskel} D  \newline
\line(1,0){75} \newline
\textbf{2} Parzival] Parcifal D \textbf{3} Anschevin] Anscivin D \textbf{4} Anschouwe] Anscoͮwe D \textbf{11} Anschevin] Anscevin D \textbf{21} Parzival] Parcifal D \newline
\end{minipage}
\hspace{0.5cm}
\begin{minipage}[t]{0.5\linewidth}
\small
\begin{center}*m
\end{center}
\begin{tabular}{rl}
 & \begin{large}D\end{large}ô disiu rede von im geschach,\\ 
 & Parcifal \textit{zuo} dem heiden sprach:\\ 
 & "wâ von sît ir ein A\textit{n}schevin?\\ 
 & Anschouwe ist von erbe mîn,\\ 
5 & bürge, lant und stete.\\ 
 & hêrre, ir sulet durch mîn bete\\ 
 & einen andern namen kiesen.\\ 
 & \textbf{solt} ich mî\textit{n l}ant verliesen\\ 
 & und die werden stat Bealtzen\textit{a}n,\\ 
10 & sô het ir mir gewalt getân.\\ 
 & \textbf{ist unser d\textit{e}w\textit{e}der} ein A\textit{n}schevin,\\ 
 & daz sol ich von arde sîn.\\ 
 & doch ist mir vür wâr gesaget,\\ 
 & daz ein helt unverzaget\\ 
15 & won in der heidenschaft,\\ 
 & \textbf{der} hab mit ritterlîcher kraft\\ 
 & minne und prîs behalten,\\ 
 & daz er \textbf{muoz} beide\textit{r} walten.\\ 
 & der ist zuo bruoder mir \textbf{genant};\\ 
20 & si hânt in d\textit{â} vür prîs erkant."\\ 
 & aber sprach Parcifal:\\ 
 & "hêrre, iuwer an\textit{t}litz\textit{es} \textit{m}âl,\\ 
 & het ich diu kuntlîch ersehen,\\ 
 & sô würde i\textit{u} schier von mir \textit{ver}jehen,\\ 
25 & als \textbf{er} mir kunt ist getân.\\ 
 & hêrre, welt irz an mich lân,\\ 
 & sô enblœzet iuwer houbet.\\ 
 & ob ir mirz gloubet,\\ 
 & mîn hant iuch strîtes gar verbirt,\\ 
30 & unz ez ander stunt gewâpent wirt."\\ 
\end{tabular}
\scriptsize
\line(1,0){75} \newline
m n o V V' \newline
\line(1,0){75} \newline
\textbf{1} \textit{Initiale} m V V'   $\cdot$ \textit{Capitulumzeichen} n  \newline
\line(1,0){75} \newline
\textbf{2} Parcifal] Parzefal V Parzifal V'  $\cdot$ zuo] \textit{om.} m \textbf{3} Anschevin] auscevin m n ansce vin o \textbf{4} \textit{Die Verse 746.4-11 fehlen} n   $\cdot$ Anschouwe] Anschowe m o V' Anschoͮwe V \textbf{6} ir sulet] die sult V' \textbf{7} andern] ander o \textbf{8} mîn lant] min hant vnd lant m myn lip o \textbf{9} werden] werde m o V  $\cdot$ Bealtzenan] bealtzenon m bealczenan o bealszenan V bealzeman V' \textbf{11} \textit{Die Verse 746.11-12 fehlen} V'   $\cdot$ deweder] do wider m  $\cdot$ Anschevin] auscevin m ansce vin o \textbf{12} arde] erde o \textbf{13} ist] ich n \textbf{16} hab] habt o \textbf{18} beider] beiden m \textbf{19} genant] benant o \textbf{20} in dâ vür prîs] in do vor pris m (n) ir pris do o in do fúr [pris*]: prisig V in do prisig V' \textbf{21} sprach] sprach do n o V V'  $\cdot$ Parcifal] parzefal V parzifal V' \textbf{22} iuwer] uwers V (V')  $\cdot$ antlitzes] antzlitz m antlitz n anczlit o  $\cdot$ mâl] gemal m \textbf{23} kuntlîch] kintlich n \textbf{24} So wurde ich schier von mir iehen m (n) (o) \textbf{26} welt] woltent n  $\cdot$ irz] ir o \textbf{28} \textit{Versdoppelung} o  \textbf{30} ander stunt gewâpent] an der stunt gewoppen n [*]: ander stvnt gewopent V anderweit verwappent V' \newline
\end{minipage}
\end{table}
\newpage
\begin{table}[ht]
\begin{minipage}[t]{0.5\linewidth}
\small
\begin{center}*G
\end{center}
\begin{tabular}{rl}
 & \begin{large}D\end{large}ô disiu rede von im geschach,\\ 
 & Parzival zem heiden sprach:\\ 
 & "wâ von sît ir ein Antschevin?\\ 
 & Anschouwe ist von erbe mîn,\\ 
5 & bürge, lant unde stete.\\ 
 & hêrre, ir sult durch mîne bete\\ 
 & \textbf{iu} einen andern namen kiesen.\\ 
 & \textbf{sol} ich mîn lant verliesen\\ 
 & unde \textit{die} werden stat Belzanan,\\ 
10 & sô het ir mir gewalt getân.\\ 
 & \textbf{unde} \textbf{ist unser deweder} ein Antschevin,\\ 
 & daz sol ich von arde sîn.\\ 
 & doch ist mir vür wâr gesaget,\\ 
 & daz ein helt unverzaget\\ 
15 & wone in der heidenschaft\\ 
 & \textbf{unde} habe mit rîterlîcher kraft\\ 
 & minne unde brîs behalten,\\ 
 & daz er \textbf{müeze} bêder walten.\\ 
 & der ist ze bruoder mir \textbf{genant};\\ 
20 & si hânt in dâ vür brîs erkant."\\ 
 & aber sprach \textbf{dô} Parzival:\\ 
 & "hêrre, iwers antlützes mâl,\\ 
 & het ich diu kuntlîche ersehen,\\ 
 & sô würde iu schiere von mir verjehen,\\ 
25 & als mir kunt ist getân.\\ 
 & hêrre, welt irz ane mich lân,\\ 
 & sô enblœzet iwer houbet.\\ 
 & obe ir mirz gloubet,\\ 
 & mîn hant iuch strîtes gar verbirt,\\ 
30 & unze ez ander stunt gewâpent wirt."\\ 
\end{tabular}
\scriptsize
\line(1,0){75} \newline
G I L M Z Fr48 Fr50 \newline
\line(1,0){75} \newline
\textbf{1} \textit{Initiale} G L Z Fr48 Fr50  \textbf{5} \textit{Initiale} I  \textbf{25} \textit{Initiale} I  \newline
\line(1,0){75} \newline
\textbf{1} Dô] Da M Z \textbf{2} Parzival] Parcifal G Z (Fr50) Parzifal I L M Partzifal Fr48 \textbf{3} wâ] wan Fr50  $\cdot$ Antschevin] anschevin G Fr50 antscheuin I anshevin L Z ansevin M Ansheuin Fr48 \textbf{4} Anschouwe] Anschoͮwe G antschowe I Anschowe L Fr50 Anscowe M Anshowe Z Anshoue Fr48 \textbf{5} bürge] Burge vnd I \textbf{7} einen] an I \textbf{8} sol] solde M (Z) (Fr48) \textbf{9} die] \textit{om.} G  $\cdot$ Belzanan] ze belzenan I Fr50 Bealzanan L Z Fr48 zcu balzanan M \textbf{10} getân] tan Fr50 \textbf{11} unde] \textit{om.} I M  $\cdot$ deweder] wedir M ietweder Fr48  $\cdot$ Antschevin] Anschevin G Fr50 Antsheuin I Anshevin L Z ansevin M Ansheuin Fr48 \textbf{16} unde] Der Z Fr48 \textbf{18} er] ez L  $\cdot$ müeze bêder] ir beder muze I muͯsz beider L (M) (Z) (Fr48) \textbf{19} genant] benant I \textbf{20} si] sin Fr50  $\cdot$ dâ] \textit{om.} I \textbf{21} aber sprach dô] aber do sprach I Do sprach aber Fr50  $\cdot$ dô] da M  $\cdot$ Parzival] parcifal G Z Fr50 Parzifal I (L) (M) partzifal Fr48 \textbf{22} iwers] ewer Z \textbf{23} diu] \textit{om.} M Fr50 \textbf{24} schiere] schie Z \textit{om.} Fr50 \textbf{25} mir] er mir Z Fr48 \textbf{30} ez] bisz M  $\cdot$ gewâpent] gewappen Fr48 \newline
\end{minipage}
\hspace{0.5cm}
\begin{minipage}[t]{0.5\linewidth}
\small
\begin{center}*T
\end{center}
\begin{tabular}{rl}
 & dô disiu rede von im geschach,\\ 
 & Parcifal zuo dem heiden sprach:\\ 
 & "wâ von sît ir ein Anschevin?\\ 
 & Anschouwe ist von erbe mîn,\\ 
5 & bürge, lant und stete.\\ 
 & hêrre, ir sult durch mîne bete\\ 
 & \textbf{iu} einen andern namen kiesen.\\ 
 & \textbf{solt} ich mîn lant verliesen\\ 
 & und die werden stat \textbf{zuo} Bealzenan,\\ 
10 & sô hetet ir mir gewalt getân.\\ 
 & \textbf{sîn wir beide} ein Anschevin,\\ 
 & daz sol ich von arde sîn.\\ 
 & doch ist mir vür wâr gesaget,\\ 
 & daz ein helt unverzaget\\ 
15 & w\textit{o}n in der heidenschaft\\ 
 & \textbf{und} habe mit rîterlîcher kraft\\ 
 & minne und prîs behalten,\\ 
 & daz er \textbf{müeze} beider walten.\\ 
 & der ist zuo bruoder mir \textbf{benant};\\ 
20 & si hânt in dâ vür prîs erkant."\\ 
 & \begin{large}A\end{large}ber sprach \textbf{dô} Parcifal:\\ 
 & "hêrre, iuwers antlitzes mâl,\\ 
 & hete ich diu kuntlîche ersehen,\\ 
 & sô würde iu schiere von \textit{mir} verjehen,\\ 
25 & als mir kunt ist getân.\\ 
 & hêrre, wolt ir ez an mich lân,\\ 
 & sô enblœzet iuwer houbet.\\ 
 & ob ir mir ez geloubet,\\ 
 & mîn hant iuch strîtes gar verbirt,\\ 
30 & unz \textbf{daz} ez ander stunt gewâpent wirt."\\ 
\end{tabular}
\scriptsize
\line(1,0){75} \newline
U W Q R \newline
\line(1,0){75} \newline
\textbf{1} \textit{Initiale} W Q  \textbf{12} \textit{Illustration mit Überschrift:} Hie zerbrach Parczifal sin schwert vnd wolt der heiden nit me mit Jm fechtten vnd erkunnetent beid ein andren von Jr art vnd namen vnd sprungent vff vnd hielfent ein andren vnd waurent fro R  \textbf{21} \textit{Initiale} U W  \newline
\line(1,0){75} \newline
\textbf{1} von im] also W von In R \textbf{2} Parcifal] Herr partzifal W Partzifal Q Parczifal R \textbf{3} ein] von Q  $\cdot$ Anschevin] antscheuein W anshevin Q Aschevin R \textbf{4} Anschouwe] Anschowe U Antschowe W Anshowe Q R \textbf{7} iu] \textit{om.} R \textbf{9} werden] werde W  $\cdot$ zuo] von W  $\cdot$ Bealzenan] Balzanan U belsanan W [B*]: Bealzanan R \textbf{11} sîn wir] Ist vnser W (Q) (R)  $\cdot$ beide] deweder W (R) itweder Q  $\cdot$ Anschevin] antscheuein W anshevin Q aschevin R \textbf{13} mir vür wâr] ist fúrwar mir W \textbf{15} won] Wan U Wond Q \textbf{16} rîterlîcher] ritterschafftt R \textbf{17} behalten] behaltte R \textbf{18} müeze] musz Q \textbf{19} benant] genant Q R \textbf{20} Sin hand sige da ze pris erkant R  $\cdot$ si] Sein W  $\cdot$ dâ] do W Q \textbf{21} Parcifal] partzifal W Q parczifal R \textbf{24} iu] ich W  $\cdot$ mir] \textit{om.} U \textbf{28} ir mir ez] irs mir Q \textbf{30} unz] mit U  $\cdot$ daz] \textit{om.} W Q R \newline
\end{minipage}
\end{table}
\end{document}
