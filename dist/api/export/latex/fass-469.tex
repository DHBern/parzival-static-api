\documentclass[8pt,a4paper,notitlepage]{article}
\usepackage{fullpage}
\usepackage{ulem}
\usepackage{xltxtra}
\usepackage{datetime}
\renewcommand{\dateseparator}{.}
\dmyyyydate
\usepackage{fancyhdr}
\usepackage{ifthen}
\pagestyle{fancy}
\fancyhf{}
\renewcommand{\headrulewidth}{0pt}
\fancyfoot[L]{\ifthenelse{\value{page}=1}{\today, \currenttime{} Uhr}{}}
\begin{document}
\begin{table}[ht]
\begin{minipage}[t]{0.5\linewidth}
\small
\begin{center}*D
\end{center}
\begin{tabular}{rl}
\textbf{469} & \begin{large}D\end{large}â wont ein werlîchiu schar.\\ 
 & ich wil iu künden umb ir nar:\\ 
 & si lebent von einem steine,\\ 
 & des geslähte ist \textbf{vil} reine.\\ 
5 & hât ir des niht erkennet,\\ 
 & \textbf{der} wirt iu hie genennet:\\ 
 & er heizet \textbf{lapsit exillis}.\\ 
 & von \textbf{des steines} kraft \textbf{vil gewis}\\ 
 & \textbf{der fênix} verbrinnet, daz er zaschen wirt.\\ 
10 & diu asche im aber leben birt.\\ 
 & sus rêrt \textbf{der} fênix mûze sîn\\ 
 & unt gît dar nâch vil liehten schîn,\\ 
 & daz er schœne wirt als ê.\\ 
 & ouch wart nie menschen sô wê,\\ 
15 & swelhes tages \textbf{ez} den stein \textbf{gesiht},\\ 
 & die wochen mac \textbf{ez} \textbf{sterben} niht,\\ 
 & diu aller schierst dar nâch gestêt.\\ 
 & sîn varwe im \textbf{nimmer ouch} zergêt.\\ 
 & man muoz im sölher \textbf{varwe} jehen,\\ 
20 & dâ mit ez hât den stein gesehen,\\ 
 & ez sî magt ode man,\\ 
 & als \textbf{dô} sîn \textbf{bestiu zît} huop an.\\ 
 & \textbf{sæh}ez den stein zwei hundert jâr,\\ 
 & im \textbf{en}\textbf{würde} denne grâ sîn hâr.\\ 
25 & sölhe kraft dem \textbf{menschen} gît der stein,\\ 
 & daz im vleisch und bein\\ 
 & jugent \textbf{enpfæhet} \textbf{al} sunder twâl.\\ 
 & der stein \textbf{ist ouch} genant der Grâl.\\ 
 & dâr ûf kumt hiute ein botschaft,\\ 
30 & dâr an doch lît sîn hœhste kraft.\\ 
\end{tabular}
\scriptsize
\line(1,0){75} \newline
D \newline
\line(1,0){75} \newline
\textbf{1} \textit{Initiale} D  \newline
\line(1,0){75} \newline
\newline
\end{minipage}
\hspace{0.5cm}
\begin{minipage}[t]{0.5\linewidth}
\small
\begin{center}*m
\end{center}
\begin{tabular}{rl}
 & d\textit{â} wont ein werlîchiu schar.\\ 
 & ich wil iu künden umb ir nar:\\ 
 & si lebent von einem steine,\\ 
 & des gesleht ist reine.\\ 
5 & hât ir des niht erkennet,\\ 
 & \textbf{er} wirt iu hie genennet:\\ 
 & er heizet \textbf{lapis exilis}.\\ 
 & von \textbf{des steines} kraft \textbf{der f\textit{ê}n\textit{i}x}\\ 
 & verbrinnet, daz er ze eschen wirt.\\ 
10 & diu esche im aber leben birt.\\ 
 & sus rêret fênix \textbf{die} mûze sîn\\ 
 & und gît dar nâch vil liehten schîn,\\ 
 & daz er schœne wirt als ê.\\ 
 & ouch wart nie menschen sô wê,\\ 
15 & welches tages \textbf{er} den stein \textbf{gesiht},\\ 
 & die wochen mac \textbf{er} \textbf{sterben} niht,\\ 
 & diu aller schierste dâ nâch ges\textit{t}êt.\\ 
 & sîn varwe im \textbf{nimmer ouch} zergêt.\\ 
 & man muoz im solher \textbf{vröude} jehen,\\ 
20 & dâ mit ez het den stein gesehen,\\ 
 & ez sî maget oder man,\\ 
 & alsô \textbf{dô} sîn \textbf{bestiu zît} huop an.\\ 
 & \textbf{sehe} ez den stein zwei hundert jâr,\\ 
 & im \textbf{werde} de\textit{n} \textit{g}r\textit{â} sîn hâr.\\ 
25 & solhe kraft dem \textbf{menschen} gît der stein,\\ 
 & daz im vleisch \dag noch\dag  bein\\ 
 & jugent \textbf{enpfâhent} \textbf{al}sunder twâl.\\ 
 & der stein \textbf{ist ouch} genant der Grâl.\\ 
 & dâr ûf kumt hiute ein botschaft,\\ 
30 & dâr an doch lît sîn hœhstiu kraft.\\ 
\end{tabular}
\scriptsize
\line(1,0){75} \newline
m n o \newline
\line(1,0){75} \newline
\newline
\line(1,0){75} \newline
\textbf{1} dâ] Do m n o \textbf{3} einem] eẏnen o \textbf{4} reine] vil reine n o \textbf{6} iu] dir n \textbf{7} exilis] exilix n \textbf{8} fênix] vinx m [nenix]: venix o \textbf{9} daz er] vnd n  $\cdot$ eschen] esche m \textbf{16} wochen] wuche n  $\cdot$ mac] so mag n \textbf{17} gestêt] geset m geseit o \textbf{18} nimmer ouch] ouch nẏemer n \textbf{19} vröude] farwe n o \textbf{20} het] hette o \textbf{24} werde] [j*]: wúrde n worde o  $\cdot$ den grâ] [s]: den sin gros m \textbf{25} solhe] Solle o \textbf{27} enpfâhent] enpfohet n  $\cdot$ alsunder] sunder n \textbf{30} dâr] [Es]: Dar m  $\cdot$ an doch] doch so n \newline
\end{minipage}
\end{table}
\newpage
\begin{table}[ht]
\begin{minipage}[t]{0.5\linewidth}
\small
\begin{center}*G
\end{center}
\begin{tabular}{rl}
 & \begin{large}D\end{large}â wont ein werlîchiu schar.\\ 
 & ich wil iu künden umb ir nar:\\ 
 & si lebent von einem steine,\\ 
 & des geslähte ist \textbf{vil} reine.\\ 
5 & habet ir des niht erkennet,\\ 
 & \textbf{der} wirt iu hie genennet:\\ 
 & er heizet \textbf{lapsit exillis}.\\ 
 & von \textbf{des steines} kraft \textbf{der fênix}\\ 
 & verbrinnet, daz er ze aschen wirt.\\ 
10 & diu asche im aber leben birt.\\ 
 & sus rêrt \textbf{der} fênix mûze sîn\\ 
 & unt gît dar nâch vil liehten schîn,\\ 
 & daz er schœne wirt als ê.\\ 
 & ouch\textbf{ne} wart nie menschen sô wê,\\ 
15 & swelhes tages \textbf{ez} den stein \textbf{siht},\\ 
 & die wochen mag \textbf{ez} \textbf{sterben} niht,\\ 
 & diu aller schierste dar nâch gestêt.\\ 
 & sîn varwe im \textbf{nimer ouch} zergêt.\\ 
 & man muoz im solher \textbf{varwe} \textit{j}ehen,\\ 
20 & dâ mit ez hât den stein gesehen,\\ 
 & ez sî maget ode man,\\ 
 & als \textbf{dô} sîn \textbf{bestiu zît} huop an.\\ 
 & \textbf{sæhe} ez den stein zwei hundert jâr,\\ 
 & im \textbf{würde} danne grâ sîn hâr.\\ 
25 & solhe kraft dem \textbf{menschen} gît der stein,\\ 
 & daz im vleisch unde bein\\ 
 & jugent \textbf{enpfæhet} \textbf{al} sunder twâl.\\ 
 & der stein \textbf{ist ouch} genant der Grâl.\\ 
 & dâr ûf kumet hiute ein botschaft,\\ 
30 & dâr an doch \textit{l}ît sîn hœhestiu kraft.\\ 
\end{tabular}
\scriptsize
\line(1,0){75} \newline
G I O L M Z Fr18 \newline
\line(1,0){75} \newline
\textbf{1} \textit{Initiale} G I O L Z Fr18  \newline
\line(1,0){75} \newline
\textbf{1} Dâ] ÷a O  $\cdot$ werlîchiu] wertlichir M \textbf{3} lebent] bent I \textbf{6} genennet] in nennet I \textbf{7} er] Der Z  $\cdot$ lapsit exillis] lapis exilis I iaspis exillix O Jaspis exilis L lapis [exilli*]: exillix M Jaspis exillix Fr18 \textbf{8} steines] \textit{om.} O L Fr18 \textbf{9} verbrinnet] Verbrinn Z  $\cdot$ er] \textit{om.} I \textbf{10} diu] Der O Fr18 \textbf{11} mûze] muͤze I die mvze Z \textbf{12} gît] gebit M  $\cdot$ liehten] lýchten L (M) \textbf{13} daz] vnd daz I Da Z  $\cdot$ als] sam I \textbf{14} ouchne] auch I (L) (Z)  $\cdot$ menschen] mensch O (M)  $\cdot$ sô] alz L \textbf{15} swelhes] Wels L (M)  $\cdot$ ez] [er]: ez O  $\cdot$ stein] \textit{om.} I  $\cdot$ siht] gesiht I (M) \textbf{16} sterben] ensterben L \textbf{17} gestêt] er gestet I [geschiht]: gestet Z \textbf{18} im nimer ouch] auch nimmer I im ovch nimmer O (L) (Fr18) yme Nummer M \textbf{19} solher] sulche M  $\cdot$ jehen] sehen G \textbf{20} ez] swer I \textbf{22} als dô] Also O Also da M Al da Z  $\cdot$ sîn bestiu zît] si in besten ziten I \textbf{23} sæhe] seh I (L) Z Soge M  $\cdot$ ez] her M \textbf{24} würde] en wrde O (Fr18)  $\cdot$ danne grâ] gra nih I \textbf{25} dem] den I  $\cdot$ menschen] mensche M \textbf{26} im] man M \textbf{27} enpfæhet] enphahen L  $\cdot$ al] \textit{om.} O L Z Fr18 \textbf{28} stein] \textit{om.} M \textbf{29} ein] an I \textbf{30} lît] leit G O  $\cdot$ hœhestiu] [hostiv]: hohstiv O grosze M \newline
\end{minipage}
\hspace{0.5cm}
\begin{minipage}[t]{0.5\linewidth}
\small
\begin{center}*T
\end{center}
\begin{tabular}{rl}
 & dâ wont ein werlîchiu schar.\\ 
 & ich wil iu künden umbe ir nar:\\ 
 & Si lebent von einem steine,\\ 
 & des geslehte ist \textbf{vil} reine.\\ 
5 & habt ir des niht erkennet,\\ 
 & \textbf{der} wirt iu \textbf{al}hie genennet:\\ 
 & er heizet \textbf{jaspis ex illix}.\\ 
 & von \textbf{der} kraft \textbf{der fênix}\\ 
 & verbrinnet, daz er ze aschen wirt.\\ 
10 & diu asche im aber \textbf{daz} leben birt.\\ 
 & sus rêrt \textbf{der} fênix \textbf{die} mûze sîn\\ 
 & unde gît dar nâch vil liehten schîn,\\ 
 & daz er schœne wirt als ê.\\ 
 & ouch wart nie menschen sô wê,\\ 
15 & swelhes tages \textbf{ez} den stein \textbf{siht},\\ 
 & die wochen mag \textbf{ez} \textbf{ersterben} niht,\\ 
 & diu aller schierst dâ nâch gestêt.\\ 
 & sîn varwe im \textbf{ouch niemer} zergêt.\\ 
 & man muoz im sölher \textbf{varwe} jehen,\\ 
20 & dâ mit ez hât den stein gesehen,\\ 
 & ez sî maget oder man,\\ 
 & alsô sîn \textbf{beste zil} huop an.\\ 
 & \textbf{sæh} ez den stein zwei hundert jâr,\\ 
 & im \textbf{würde} danne grâ sîn hâr.\\ 
25 & sölhe kraft dem gît der stein,\\ 
 & daz im vleisch unde bein\\ 
 & jugent \textbf{enpfâhent} sunder twâl.\\ 
 & der stein, \textbf{der ist} genant der Grâl.\\ 
 & dâr ûf kumt hiute ein botschaft,\\ 
30 & dâr an doch liget sîn hœhest\textit{iu} kraft.\\ 
\end{tabular}
\scriptsize
\line(1,0){75} \newline
T U V W Q R Fr42 \newline
\line(1,0){75} \newline
\textbf{1} \textit{Initiale} Q  \textbf{3} \textit{Majuskel} T  \newline
\line(1,0){75} \newline
\textbf{1} \textit{Die Verse 453.1-502.30 fehlen} U   $\cdot$ dâ wont] Do wont V Do von W (Fr42)  $\cdot$ werlîchiu] werlichir Fr42 \textbf{4} des] Das Q \textbf{6} der] Er Q  $\cdot$ alhie] hie W (Q) (R) (Fr42)  $\cdot$ genennet] benennet W \textbf{7} er] Es Q  $\cdot$ jaspis ex illix] [*]: lapis exillix V iaspis exilix W iaspis exsilix Q Jaspiserillis R jaspis exillis Fr42 \textbf{8} der kraft] [de*]: dez craft V des krafft W (R) dem craft Q  $\cdot$ der fênix] [die]: der fenis Fr42 \textbf{9} verbrinnet] verbrennet T (W) (R) \textbf{10} diu asche] Die aͯschen R  $\cdot$ im] in V R  $\cdot$ daz leben] lebende V leben W (Q) R (Fr42) \textbf{11} Sus rett der fenich die fedren sin R  $\cdot$ sus] Es Q  $\cdot$ die] \textit{om.} W \textbf{12} vil] \textit{om.} Q R  $\cdot$ liehten] lichten Q \textbf{14} menschen] mensch R  $\cdot$ sô] \textit{om.} Q \textbf{15} swelhes] Welches W (Q) (R)  $\cdot$ ez] er W  $\cdot$ stein] [schein]: sctein Q \textbf{16} ez ersterben] er sterben W \textbf{18} ouch niemer] niemer oͮch V (Q) (R) \textbf{20} hât den stein] den stein hot Q \textbf{21} ez] Ezn Q \textbf{22} alsô] Also do V (Fr42) Aldo W Als das R  $\cdot$ beste zil] beste zit V (Q) (R) bestes zil W \textbf{23} sæh ez] sehez T Sicht es Q \textbf{24} im würde danne] Vmb en wurde nymmer Q Jm enwurde nit R im wurde Fr42 \textbf{25} dem] dem moͤnschen V (W) (Q) (R) den menschen Fr42 \textbf{26} vleisch] vel fleisch V fleysz Q \textbf{27} enpfâhent] enpfahet V \textbf{28} der ist] ist auch W (Q) (R) (Fr42) \textbf{30} hœhestiu] hoheste T \newline
\end{minipage}
\end{table}
\end{document}
