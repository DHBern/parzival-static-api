\documentclass[8pt,a4paper,notitlepage]{article}
\usepackage{fullpage}
\usepackage{ulem}
\usepackage{xltxtra}
\usepackage{datetime}
\renewcommand{\dateseparator}{.}
\dmyyyydate
\usepackage{fancyhdr}
\usepackage{ifthen}
\pagestyle{fancy}
\fancyhf{}
\renewcommand{\headrulewidth}{0pt}
\fancyfoot[L]{\ifthenelse{\value{page}=1}{\today, \currenttime{} Uhr}{}}
\begin{document}
\begin{table}[ht]
\begin{minipage}[t]{0.5\linewidth}
\small
\begin{center}*D
\end{center}
\begin{tabular}{rl}
\textbf{799} & \begin{large}P\end{large}arzival zuo sînem œheim sprach:\\ 
 & "ich wil si sehen, die ich nie gesach\\ 
 & \textbf{inre} vünf jâren.\\ 
 & dô wir bî ein ander wâren,\\ 
5 & si was mir liep, \textbf{als} ist si \textbf{ouch} noch.\\ 
 & dînen rât wil ich haben doch,\\ 
 & die wîle uns scheidet niht der tôt.\\ 
 & dû riete mir \textbf{ê} in grôzer nôt.\\ 
 & \textbf{Ich wil} gein mîme wîbe komen,\\ 
10 & der kunft ich gein mir hân vernomen\\ 
 & bî dem Plimizœle an einer stat."\\ 
 & urloup er im geben bat.\\ 
 & dô bevalch in gote der guote man.\\ 
 & Parzival die naht streich dan.\\ 
15 & \textbf{sînen} gesellen was der walt \textbf{wol} kunt.\\ 
 & \textbf{dô ez tagete}, dô vant er lieben vunt,\\ 
 & manec gezelt ûf geslagen.\\ 
 & ûzem lande ze Brobarz, hôrt ich sagen,\\ 
 & was vil baniere \textbf{dâ} gestecket,\\ 
20 & manec schilt dâr \textbf{nâch} \textbf{getrecket}.\\ 
 & Sînes landes vürsten lâgen dâ.\\ 
 & Parzival, \textbf{der} vrâgete, wâ\\ 
 & diu küneginne selbe læge,\\ 
 & \textbf{unt} ob si \textbf{sunderringes} pflæge\\ 
25 & \multicolumn{1}{l}{ - - - }\\ 
 & \multicolumn{1}{l}{ - - - }\\ 
 & mit \textbf{zelten} umbevangen.\\ 
 & Nû \textit{w}as von Katelangen\\ 
 & der herzoge Kyot des morgens vruo\\ 
30 & ûf gestanden. dise riten zuo.\\ 
\end{tabular}
\scriptsize
\line(1,0){75} \newline
D \newline
\line(1,0){75} \newline
\textbf{1} \textit{Initiale} D  \textbf{9} \textit{Majuskel} D  \textbf{21} \textit{Majuskel} D  \textbf{28} \textit{Majuskel} D  \newline
\line(1,0){75} \newline
\textbf{1} Parzival] Parcifal D \textbf{11} Plimizœle] Primizoͤle D \textbf{14} Parzival] Parcifal D \textbf{22} Parzival] Parcifal D \textbf{25} \textit{Die Verse 799.25-26 fehlen} D  \textbf{28} was] vas D \newline
\end{minipage}
\hspace{0.5cm}
\begin{minipage}[t]{0.5\linewidth}
\small
\begin{center}*m
\end{center}
\begin{tabular}{rl}
 & \begin{large}P\end{large}arcifal zuo sînem \textit{œheim} sprach:\\ 
 & "ich wil si sehen, die ich nie gesach\\ 
 & \textbf{in den} vünf jâren.\\ 
 & dô wir bî ein ander wâren,\\ 
5 & si was mir liep, \textbf{als} ist si noch.\\ 
 & dînen rât wil ich haben doch,\\ 
 & die wîle uns scheidet niht der tôt.\\ 
 & dû riet mir \textbf{ê} in grôzer nôt.\\ 
 & \textbf{ich wil} gegen mînem wîbe komen,\\ 
10 & der kun\textit{f}t ich gegen mir hân vernomen\\ 
 & bî dem Plimizol an einer stat."\\ 
 & urloup er im \textbf{dô} \textit{g}eben bat.\\ 
 & dô bevalch in got der guote man.\\ 
 & Parcifal die naht streich dan.\\ 
15 & \textbf{sînem} gesellen was der walt kunt.\\ 
 & \textbf{des tages} dô vant er lieben vunt,\\ 
 & manic gezelt ûf geslagen.\\ 
 & ûz dem lande zuo Brobarz, hôrt ich sagen,\\ 
 & was vil banier \textbf{d\textit{â}} gestecket,\\ 
20 & manic schilt dâr \textbf{nâch} \textbf{gestrecket}.\\ 
 & sînes landes vürsten lâgen dâ.\\ 
 & Parcifal, \textbf{der} vrâgte, wâ\\ 
 & diu künigîn \textbf{d\textit{â}} selbe læge,\\ 
 & ob si \textbf{sunders ringes} pflæge.\\ 
25 & man zougte im, aldâ si lac\\ 
 & und wol gehêrtes ringes pflac\\ 
 & mit \textbf{gezelten} umbevangen.\\ 
 & nû was von Kathelangen\\ 
 & der herzoge Kyot des morgens vruo\\ 
30 & ûf gestanden. dise riten zuo.\\ 
\end{tabular}
\scriptsize
\line(1,0){75} \newline
m n o V V' W \newline
\line(1,0){75} \newline
\textbf{1} \textit{Initiale} m V W   $\cdot$ \textit{Capitulumzeichen} n  \newline
\line(1,0){75} \newline
\textbf{1} Parcifal] Parzefal V Parzifal V' (W)  $\cdot$ œheim] \textit{om.} m oheim do V' \textbf{2} nie] [nu]: ny V' \textbf{3} in den] Jnre V Zware wol in V' \textbf{5} als] vnd V'  $\cdot$ si noch] sú ouch noch n (V) mir nach V' sy mir noch W \textbf{6} dînen rât] Dines rates V'  $\cdot$ wil ich] ich wil V \textbf{7} scheidet] scheide n scheident o  $\cdot$ niht] \textit{om.} V' \textbf{8} dû] Die V' Das W  $\cdot$ ê] ie V' \textbf{9} \textit{Verse 799.9-10 kontrahiert zu:} Jch wil gegen mir hant vernonen o  \textbf{10} kunft] kunst m  $\cdot$ gegen mir] \textit{om.} V' \textbf{11} Plimizol] plúmzol n plimczol o plimtzol V' plymizol W  $\cdot$ an einer] ein ander o auff einer W \textbf{12} geben] gegeben m \textbf{13} bevalch] befalsch o \textbf{14} Parcifal] Parzefal V Parzifal V' Herr partzifal W  $\cdot$ die naht streich] zv hant reit V'  $\cdot$ dan] von dan V (V') \textbf{15} Sînem] Sinen V V' (W)  $\cdot$ kunt] wol kvnt V (V') \textbf{16} des tages] Do ez tagete V Do ez taget V'  $\cdot$ dô] \textit{om.} W  $\cdot$ lieben] liebes o \textbf{18} ûz] Vf V'  $\cdot$ Brobarz] brobartz m n brobarcz o prabartz V' brebars W \textbf{19} dâ] do m n V V' W  $\cdot$ gestecket] gerecket W \textbf{20} nâch] \textit{om.} V' W  $\cdot$ gestrecket] angestecket W \textbf{21} dâ] do n V V' W \textbf{22} Parcifal] Parzifal V Pfarzifal V' Herr partzifal W  $\cdot$ der] \textit{om.} V' W  $\cdot$ vrâgte] fraget V' \textbf{23} künigîn] konige o  $\cdot$ dâ] do m \textit{om.} n o V V' W  $\cdot$ selbe] selbes W  $\cdot$ læge] lige V' \textbf{24} si] \textit{om.} n  $\cdot$ sunders] svnder V (V')  $\cdot$ pflæge] pflige V' \textbf{25} zougte] zeigete n V (W) zeiget V'  $\cdot$ aldâ] da o \textbf{26} gehêrtes] geheretez V herliges V' \textbf{28} nû] Mun W  $\cdot$ was] waz ouch V'  $\cdot$ Kathelangen] cathelangen n kathelangenn o cachelangen V kattelangen V' \textbf{29} der] Dez V'  $\cdot$ Kyot] kẏot m n o  $\cdot$ des] \textit{om.} n \textbf{30} ûf gestanden] Was aufgestanden W \newline
\end{minipage}
\end{table}
\newpage
\begin{table}[ht]
\begin{minipage}[t]{0.5\linewidth}
\small
\begin{center}*G
\end{center}
\begin{tabular}{rl}
 & \begin{large}P\end{large}arzival ze sînem œheim sprach:\\ 
 & "ich wil si sehen, die ich nie gesach\\ 
 & \textbf{inner} vünf jâren.\\ 
 & dô wir bî ein ander wâren,\\ 
5 & si was mir lieb, \textbf{als} ist si \textbf{ouch} noch.\\ 
 & dînen rât wil ich haben doch,\\ 
 & die wîle uns scheidet niht der tôt.\\ 
 & dû riete mi\textit{r} \textit{i}n grôzer nôt.\\ 
 & \textbf{ich wil} gein mînem wîbe komen,\\ 
10 & der kunft ich gein mir hân vernomen\\ 
 & bî dem Blimzol an einer stat."\\ 
 & urloup er im \textbf{dô} geben bat.\\ 
 & dô bevalch in got der guote man.\\ 
 & Parzival die naht streich dan.\\ 
15 & \textbf{sînen} gesellen was der walt \textbf{wol} kunt.\\ 
 & \textbf{dô ez tagte}, dô vant er lieben vunt,\\ 
 & manic gezelt ûf geslagen.\\ 
 & ûz dem lande ze Briubarz, hôrt ich sagen,\\ 
 & was vil banier gestecket,\\ 
20 & manic schilt dâ \textbf{bî} \textbf{getrecket}.\\ 
 & sînes landes vürsten lâgen dâ.\\ 
 & Parzival, \textbf{der} vrâgte, wâ\\ 
 & diu künigîn selbe læge,\\ 
 & op si \textbf{sunderringes} pflæge.\\ 
25 & man zeigte im, al dâ si lac\\ 
 & unde wol gehêrtes ringes pflac\\ 
 & mit \textbf{gezelten} umbevangen.\\ 
 & nû was  Katelangen\\ 
 & der herzoge Kiot des morgens vruo\\ 
30 & ûf gestanden. dise riten zuo.\\ 
\end{tabular}
\scriptsize
\line(1,0){75} \newline
G I L M Z Fr48 \newline
\line(1,0){75} \newline
\textbf{1} \textit{Initiale} G I L Z Fr48  \textbf{17} \textit{Initiale} I  \newline
\line(1,0){75} \newline
\textbf{1} Parzival] Parcival G Parzifal I L M Parcifal Z (Fr48) \textbf{2} die ich] die L \textbf{3} inner] Bynnen M Jn Fr48 \textbf{4} dô] da I (M) (Z)  $\cdot$ bî ein ander] en sament L \textbf{5} lieb] lie L  $\cdot$ als ist si ouch] daz ist si I sý ist och L so ist sie M (Z) \textbf{7} uns scheidet niht] niht vns shaidet I vnz niht scheidet L \textbf{8} riete mir] riete mir ie in G reiten wir M \textbf{9} ich wil] nu wil ich I \textbf{11} Blimzol] blimizol I plimizol L M Z  $\cdot$ einer] eine L (M) Z \textbf{12} dô] da M Z  $\cdot$ bat] gap M \textbf{13} dô] Da M Z \textbf{14} Parzival] parcival G [parzifal]: Parzifal I Parzifal L M Parcifal Z \textbf{15} sînen] sim I Sinē M \textbf{16} dô ez] Da esz M (Z)  $\cdot$ dô vant] da vant M (Z) \textbf{18} ûz] vf I (M)  $\cdot$ Briubarz] brubraz I Brvbarz L (M) Z \textbf{19} was] Waren M \textbf{20} bî] \textit{om.} L M  $\cdot$ getrecket] Gerechet I gestrechet L [gisteckit]: gistreckit M \textbf{21} sînes] Sinen M \textbf{22} Parzival] Parcival G Parzifal I L M Parcifal Z  $\cdot$ der] \textit{om.} L \textbf{23} selbe] selben M \textbf{24} sunderringes] svnders ringes L \textbf{25} zeigte] zaigt I (L)  $\cdot$ al] \textit{om.} L \textbf{28} was] waz von L (M) (Z)  $\cdot$ Katelangen] katlangen L \textbf{29} Kiot] kýot L kyot Z \newline
\end{minipage}
\hspace{0.5cm}
\begin{minipage}[t]{0.5\linewidth}
\small
\begin{center}*T
\end{center}
\begin{tabular}{rl}
 & Parcifal zuo sîme œheim sprach:\\ 
 & "ich wil si sehen, die ich nie gesach\\ 
 & \textbf{in} vünf jâren.\\ 
 & dô wir bî ein ander wâren,\\ 
5 & si was mir liep, \textbf{daz} ist si noch.\\ 
 & dînen rât wil ich haben doch,\\ 
 & die wîle uns scheidet niht der tôt.\\ 
 & dû riete mir \textbf{ê} in grôzer nôt.\\ 
 & \textbf{nû wil ich} gein mîme wîbe komen,\\ 
10 & der kunft ich gein mir hân vernomen\\ 
 & bî dem Plymizol an einer stat."\\ 
 & urloup er im \textbf{dô} geben bat.\\ 
 & \begin{large}D\end{large}ô bevalch in got der guote man.\\ 
 & Parcifal die naht streich dan.\\ 
15 & \textbf{sînen} gesellen was der walt \textbf{wol} kunt.\\ 
 & \textbf{dô ez tagete}, dô vant er lieben vunt,\\ 
 & manec gezelt ûf geslagen.\\ 
 & ûz dem lande zuo Brebarz, hôrt ich sagen,\\ 
 & was vil baniere \textbf{dâ} gestecket,\\ 
20 & manec schilt dâr \textbf{an} \textbf{getrecket}.\\ 
 & sînes landes vürsten lâgen dâ.\\ 
 & Parcifal \textbf{dâ} vrâget, wâ\\ 
 & diu künegîn selber læge,\\ 
 & ob si \textbf{sunderringes} pflæge.\\ 
25 & man zeiget im, aldâ si lac\\ 
 & und wol gehêrtes ringes pflac\\ 
 & mit \textbf{gezelten} umbevangen.\\ 
 & nû was von Katelangen\\ 
 & der herzoge Kyot des morgens vruo\\ 
30 & ûf gestanden. dise riten zuo.\\ 
\end{tabular}
\scriptsize
\line(1,0){75} \newline
U Q R \newline
\line(1,0){75} \newline
\textbf{1} \textit{Initiale} Q R  \textbf{13} \textit{Initiale} U  \newline
\line(1,0){75} \newline
\textbf{1} Parcifal] Parzifal U Partzifal Q Parczifal R \textbf{3} \textit{Versfolge 799.4-3} U   $\cdot$ in] Jnner Q (R) \textbf{4} ander] andren R \textbf{5} daz ist si] si ist si auch Q so ist sy och R \textbf{7} scheidet niht] nit scheidet R \textbf{9} nû wil ich] Jch wil Q R \textbf{10} ich gein mir hân] gen mir han ich R \textbf{11} Plymizol] plimizol Q plimiczol R \textbf{14} Parcifal] Parzifal U Partzifal Q Parczifal R \textbf{15} wol kunt] do gar vnkunt R \textbf{18} ûz dem] Vsserm R  $\cdot$ Brebarz] bruͦbarz U burbasz Q Burbarcz R  $\cdot$ ich] er do R \textbf{19} dâ] do Q  $\cdot$ gestecket] gestreket R \textbf{20} an] nacht Q (R) \textbf{21} dâ] do Q \textbf{22} Parcifal] Parzifal U Partzifal Q Parczifal R  $\cdot$ dâ vrâget wâ] da vraget [da]: wa U der fragte wa Q R \textbf{23} künegîn selber] konige selbe Q \textbf{25} zeiget] zeigte Q (R) \textbf{27} gezelten] geczeltte R \textbf{28} Katelangen] kachelangen Q kattelangen R \textbf{29} Kyot] koyt Q  $\cdot$ des morgens] enmorgen R \textbf{30} dise] disen R \newline
\end{minipage}
\end{table}
\end{document}
