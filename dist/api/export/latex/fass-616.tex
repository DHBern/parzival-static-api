\documentclass[8pt,a4paper,notitlepage]{article}
\usepackage{fullpage}
\usepackage{ulem}
\usepackage{xltxtra}
\usepackage{datetime}
\renewcommand{\dateseparator}{.}
\dmyyyydate
\usepackage{fancyhdr}
\usepackage{ifthen}
\pagestyle{fancy}
\fancyhf{}
\renewcommand{\headrulewidth}{0pt}
\fancyfoot[L]{\ifthenelse{\value{page}=1}{\today, \currenttime{} Uhr}{}}
\begin{document}
\begin{table}[ht]
\begin{minipage}[t]{0.5\linewidth}
\small
\begin{center}*D
\end{center}
\begin{tabular}{rl}
\textbf{616} & in\textbf{z} herze, dâ diu vreude lac,\\ 
 & dô ich Cidegastes \textbf{minne} pflac.\\ 
 & I\textbf{ne} bin sô niht verdorben,\\ 
 & i\textbf{ne} habe doch sît \textbf{geworben}\\ 
5 & des küneges schaden mit koste\\ 
 & unt manege scherpfe tjoste\\ 
 & gein sîme verhe gevrumt.\\ 
 & waz, ob mir \textbf{an iu} helfe kumt,\\ 
 & diu mich richet unt ergetzet,\\ 
10 & daz mir jâmer \textbf{ze} herze wetzet?\\ 
 & Ûf Gramoflanzes tôt\\ 
 & enpfieng ich dienst, \textbf{daz} \textbf{mir} bôt\\ 
 & ein künec, der\textbf{s} wunsches hêrre was.\\ 
 & hêrre, der \textbf{heizet} Anfortas.\\ 
15 & durch minne \textbf{ich nam} von sîner hant\\ 
 & von Thabronit daz krâmgewant,\\ 
 & daz noch vor iwerer porten stêt,\\ 
 & dâ \textbf{tiwerz gelten} gegen gêt.\\ 
 & Der künec in \textbf{mîme} dienste erwarp,\\ 
20 & dâ von \textbf{mîn vreude} gar verdarp.\\ 
 & dô ich in minne solte wern,\\ 
 & dô muos ich niwes jâmers gern.\\ 
 & In mîme dienste erwarb er sêr.\\ 
 & glîchen jâmer oder mêr,\\ 
25 & als Cidegast geben kunde,\\ 
 & gab mir Anfortases wunde.\\ 
 & \begin{large}N\end{large}û \textbf{jeht}, wie solt ich armez wîp,\\ 
 & sît ich hân getriwen lîp,\\ 
 & \textbf{al} solher nôt bî \textbf{sinne} sîn?\\ 
30 & etswenne sich krenket ouch der mîn,\\ 
\end{tabular}
\scriptsize
\line(1,0){75} \newline
D Z Fr68 \newline
\line(1,0){75} \newline
\textbf{1} \textit{Initiale} Z  \textbf{3} \textit{Initiale} Fr68   $\cdot$ \textit{Majuskel} D  \textbf{11} \textit{Majuskel} D  \textbf{19} \textit{Majuskel} D  \textbf{23} \textit{Majuskel} D  \textbf{25} \textit{Initiale} Z  \textbf{27} \textit{Initiale} D  \newline
\line(1,0){75} \newline
\textbf{1} inz] JN min Z  $\cdot$ diu] e Fr68 \textbf{2} dô] Da Z  $\cdot$ Cidegastes] Citegastes Z ci:egastes Fr68  $\cdot$ minne] minnen Fr68 \textbf{4} geworben] erworben Z \textbf{8} an iu helfe] helfe von ev Z \textbf{10} jâmer ze] iamers Z \textbf{11} Gramoflanzes] Gramoflanzs D gramoflantzes Z \textbf{13} ders] der Z \textbf{16} Thabronit] Tabrunit Z \textbf{18} gelten gegen] gelt engegen Z \textbf{19} mîme] :::en Fr68 \textbf{21} dô] Da Z \textbf{22} dô] Da Z \textbf{23} mîme] :::inen Fr68 \textbf{24} glîchen] Gelich Z \textbf{25} Cidegast] cithegast Z citegast Fr68 \textbf{26} Anfortases] Anfortas D (Z) anphortases Fr68 \textbf{29} bî sinne] bin sinnen Z \textbf{30} di erliden hat daz herce min Fr68  $\cdot$ krenket ouch] ouch krenket Z \newline
\end{minipage}
\hspace{0.5cm}
\begin{minipage}[t]{0.5\linewidth}
\small
\begin{center}*m
\end{center}
\begin{tabular}{rl}
 & in \textbf{daz} herz, d\textit{â} diu vröude lac,\\ 
 & dô ich Zidegastes \textbf{minne} pflac.\\ 
 & ich bin sô niht ver\textit{d}or\textit{b}en,\\ 
 & ich hab doch sît \textbf{erworben}\\ 
5 & des küniges schaden mit \textit{k}o\textit{st}e\\ 
 & und manige scharpfe joste\\ 
 & gegen sînem verhe gevrom\textit{t}.\\ 
 & waz, ob mir \textbf{an iu} helfe kom\textit{t},\\ 
 & diu mich richet und ergetzet,\\ 
10 & daz mir jâmer\textbf{z} herze wetzet?\\ 
 & ûf Gram\textit{o}lanzes tôt\\ 
 & enpfienc ich dienst, \textbf{daz} \textbf{mir} bôt\\ 
 & ein künic, der \textbf{des} wunsches hêrre was.\\ 
 & hêrre, der \textbf{heizet} Anfortas.\\ 
15 & durch minne \textbf{ich nam} von sîner hant\\ 
 & von T\textit{a}bronit daz k\textit{r}âmgewant,\\ 
 & daz noch vor iuwer porte stât,\\ 
 & d\textit{â} \textbf{dir daz golt} engegen gât.\\ 
 & der künic in \textbf{minnen} dienst erwarp,\\ 
20 & dâ von \textbf{mîn vröude} gar verdarp.\\ 
 & d\textit{ô} ich in minne solte wern,\\ 
 & dô muost ich niuwes jâmers gern.\\ 
 & in mînem dienst erwarp er sêr.\\ 
 & glîchen jâmer oder mêr,\\ 
25 & alsô Zi\textit{d}e\textit{g}ast geben kunde,\\ 
 & gap mir Anfortases wunde.\\ 
 & nû \textbf{jeht}, wie solt ich armez wîp,\\ 
 & sît ich hân getriuwe\textit{n} lîp,\\ 
 & \textbf{al}solicher nôt bî \textbf{sinne} sîn?\\ 
30 & etwen sich krenket ouch der mîn,\\ 
\end{tabular}
\scriptsize
\line(1,0){75} \newline
m n o \newline
\line(1,0){75} \newline
\newline
\line(1,0){75} \newline
\textbf{1} dâ] do m n o  $\cdot$ vröude] frowe o \textbf{2} Zidegastes] zidegasten o  $\cdot$ pflac] [gast]: pflag o \textbf{3} verdorben] verborgen m \textbf{4} sît] nit n \textbf{5} schaden] schade m n o  $\cdot$ koste] stoke m \textbf{7} gevromt] gefromtte m (n) (o) \textbf{8} komt] komptte m (n) kampt o \textbf{11} Gramolanzes] gramonlanczes m o gramonlantzes n \textbf{12} bôt] gebot o \textbf{16} Tabronit] tampbronit m tambre n tambroit o  $\cdot$ krâmgewant] kam [gevant]: gewant m \textbf{17} porte] porten n (o) \textbf{18} dâ] Do m n o \textbf{19} minnen] mynem n (o) \textbf{20} \textit{korrigierende Versdoppelung:} >do< Von myn freuͯide gar erwarp / Do von myn freuͯide gar verdarp o  \textbf{21} dô] Da m  $\cdot$ wern] warn o \textbf{22} gern] gert o \textbf{24} glîchen] Glichem n \textbf{25} Zidegast] zigedast m \textbf{26} Anfortases] anforttas m anfortas n o \textbf{27} jeht] \textit{om.} n \textbf{28} \textit{Versdoppelung nach 616.30} o   $\cdot$ getriuwen] getruͯwes m \textbf{29} alsolicher] Also sollicher n \newline
\end{minipage}
\end{table}
\newpage
\begin{table}[ht]
\begin{minipage}[t]{0.5\linewidth}
\small
\begin{center}*G
\end{center}
\begin{tabular}{rl}
 & in\textbf{z} herze, dâ diu vröude lac,\\ 
 & dô ich Zidegastes pflac.\\ 
 & ich \textbf{en}bin sô niht verdorben,\\ 
 & ich \textbf{en}habe doch sît \textbf{geworben}\\ 
5 & des küniges schaden mit koste\\ 
 & unde manige scharpfe tjoste\\ 
 & gein sînem verhe gevrumet.\\ 
 & waz, ob mir helfe \textbf{von iu} kumet,\\ 
 & diu mich richet unde ergetzet,\\ 
10 & daz mir \dag jâmers\dag  herze wetzet?\\ 
 & ûf Gramoflanzes tôt\\ 
 & enpfien\textit{g} ich dienest. \textbf{daz} bôt\\ 
 & ein künic, der wunsches hêrre was.\\ 
 & hêrre, der \textbf{hiez} Anfortas.\\ 
15 & durch minne \textbf{nam ich} von sîner hant\\ 
 & von Tabrunit daz krâmgewant,\\ 
 & daz noch vor iuwerre porten stêt,\\ 
 & dâ \textbf{tiefez gelt} engegen gêt.\\ 
 & \begin{large}D\end{large}er künic in \textbf{mînem} dienst erwarp,\\ 
20 & dâ von \textbf{mîn vröude} gar verdarp.\\ 
 & dô ich \textit{in} minne solde wern,\\ 
 & dô muos ich niuwes jâmers gern.\\ 
 & in mînem dienste erwarp er sêr.\\ 
 & gelîchen jâmer oder mêr,\\ 
25 & als Zidegast geben kunde,\\ 
 & gab mir Anfortases wunde.\\ 
 & nû \textbf{jehet}, wie solde ich armez wîp,\\ 
 & sît ich hân getriuwen lîp,\\ 
 & \textbf{al} solher nôt bî \textbf{sinne\textit{n}} sîn?\\ 
30 & etswenne sich krenket ouch der mîn,\\ 
\end{tabular}
\scriptsize
\line(1,0){75} \newline
G I L M Z \newline
\line(1,0){75} \newline
\textbf{1} \textit{Initiale} L Z  \textbf{3} \textit{Initiale} I  \textbf{19} \textit{Initiale} G I  \textbf{25} \textit{Initiale} Z  \newline
\line(1,0){75} \newline
\textbf{1} inz] JN min Z  $\cdot$ vröude] vrowe L \textbf{2} dô] Da M Z  $\cdot$ Zidegastes] cydegastes G zitegastes I Citegastes L Z zcitegastin M  $\cdot$ pflac] mýnne pflach L (M) (Z) \textbf{3} enbin] bin I \textbf{4} enhabe] habe I  $\cdot$ sît] \textit{om.} I  $\cdot$ geworben] erworben M \textbf{11} Gramoflanzes] Gramoflanz L Gramorflanszis M gramoflantzes Z \textbf{12} enpfieng] Enphienge G  $\cdot$ bôt] mir bot Z \textbf{14} hêrre der] Der herre M  $\cdot$ hiez] [liez]: hiez L heizzet Z  $\cdot$ Anfortas] Amfortas L \textbf{15} minne] in L  $\cdot$ nam ich] ich nam M Z \textbf{16} Tabrunit] Tanbrun I tabrvͯnit M \textbf{17} porten] borten I phorten M \textbf{18} tiefez] tuͯres L (Z) iclichsers M  $\cdot$ engegen] Gein I \textbf{20} vröude] wirde L M \textbf{21} dô] Da M Z  $\cdot$ in] \textit{om.} G \textbf{22} dô] Da M Z  $\cdot$ niuwes] niwan I \textbf{23} er] \textit{om.} I \textbf{24} gelîchen] Gelich I Z \textbf{25} Zidegast] ziteGastes I Citegast L zcitegast M cithegast Z  $\cdot$ kunde] kuͯnden L \textbf{26} gab] Gaben I  $\cdot$ Anfortases] anfortas G (I) Z Amfortassez L anfortassis M  $\cdot$ wunde] wunden I (L) \textbf{27} jehet] sehet I (M) \textbf{28} sît] \textit{om.} M \textbf{29} al] bi I An M  $\cdot$ bî] bin Z  $\cdot$ sinnen] sinne G \textbf{30} sich krenket ouch] krenkit sich ouch M sich ouch krenket Z \newline
\end{minipage}
\hspace{0.5cm}
\begin{minipage}[t]{0.5\linewidth}
\small
\begin{center}*T
\end{center}
\begin{tabular}{rl}
 & in \textbf{sîn} herze, dâ diu vreude lac,\\ 
 & dô ich Cydegastes \textbf{minne} pflac.\\ 
 & ich \textbf{en}bin sô niht verdorben,\\ 
 & ich hân doch sît \textbf{geworben}\\ 
5 & des küneges schaden mit koste\\ 
 & und manege scharpfe joste\\ 
 & gein sîme verhe gevrumt.\\ 
 & waz, ob mir helfe \textbf{von iu} kumt,\\ 
 & diu mich richet und ergetzet,\\ 
10 & daz mir jâmer \textbf{daz} herze wetzet?\\ 
 & ûf Gramoflanzes tôt\\ 
 & entvienc ich dienst. \textbf{den} bôt\\ 
 & ein künec, der wunsches hêrre was.\\ 
 & hêrre, der \textbf{hiez} Anfortas.\\ 
15 & durch minne \textbf{ich nam} von sîner hant\\ 
 & von Tabrunit daz krâmgewant,\\ 
 & daz noch vor iuwer porten stêt,\\ 
 & d\textit{â} \textbf{tiurez gelt} engein gêt.\\ 
 & der künec in \textbf{mîme} dienste erwarp,\\ 
20 & dâ von \textbf{sîn wirde} gar verdarp.\\ 
 & dô ich in minne solte wern,\\ 
 & dô muost ich niuwes jâmers gern.\\ 
 & in mîme dienste erwarb er sêre.\\ 
 & glîchen jâmer oder mêre,\\ 
25 & als Cydegaste geben kunde,\\ 
 & gap mir Anfortasses wunde.\\ 
 & nû \textbf{sehet}, wie solt ich armez wîp,\\ 
 & sît ich hân getriuwen lîp,\\ 
 & \textbf{von} solicher nôt bî \textbf{sinnen} sîn?\\ 
30 & etswanne sich krenket ouch der mîn,\\ 
\end{tabular}
\scriptsize
\line(1,0){75} \newline
U V W Q R Fr39 \newline
\line(1,0){75} \newline
\textbf{1} \textit{Initiale} Q Fr39  \newline
\line(1,0){75} \newline
\textbf{1} in sîn] Jns V (Fr39) In des W (Q)  $\cdot$ herze] hertzen W (R)  $\cdot$ dâ] do V W Q  $\cdot$ lac] [lit]: lag R \textbf{2} Cydegastes] gydegastes V zytegasteß W cidegastes Q (R) (Fr39) \textbf{3} enbin] bin W Q R Fr39 \textbf{4} ich] Jchn Q \textbf{5} schaden] schade W \textbf{6} und] Vf V  $\cdot$ scharpfe] scharpffen R \textbf{8} iu] im Q \textbf{9} diu] Das R  $\cdot$ ergetzet] seczet R \textbf{10} jâmer daz] iamers V W Q (R) (Fr39)  $\cdot$ wetzet] \textit{om.} Fr39 \textbf{11} Gramoflanzes] gramaflanzes V gramoflantzes W Q Gramoflanczes R Gramoflanz Fr39 \textbf{12} den] [de*]: den mir V das W Q (R) Fr39  $\cdot$ bôt] gebot W \textbf{14} hiez] heisset V \textbf{16} Tabrunit] Tabruͦnit U (Fr39) tamburnit V tabronit W tabrúnit Q tabernit R  $\cdot$ krâmgewant] [kamgewant]: kramgewant U [cram]: kraͮm gewand R \textbf{17} porten] burge R \textbf{18} dâ] Do U W \textbf{19} in] \textit{om.} W an R  $\cdot$ mîme] meinen W (R) minē Q \textbf{20} sîn wirde] [*]: min froͤide V \textbf{21} dô] Das W Da Fr39 \textbf{22} muost] mvͤst V (Fr39) muͦß W \textbf{24} glîchen] Gli͑ch Q \textbf{25} Cydegaste] gidegast V zytegast W Citegast Q Cidegast R Fr39 \textbf{26} Anfortasses] anfortas W \textbf{27} sehet] iecht W (Q) (R) (Fr39) \textbf{28} getriuwen] gerúwen R \textbf{29} von solicher] Alsolcher W (Fr39) An sulcher Q (R) \textbf{30} sich krenket ouch] auch W sich [och]: oͮch [kenket]: krenket R  $\cdot$ der] oder Q \sout{oder} der R \newline
\end{minipage}
\end{table}
\end{document}
