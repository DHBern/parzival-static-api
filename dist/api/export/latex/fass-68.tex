\documentclass[8pt,a4paper,notitlepage]{article}
\usepackage{fullpage}
\usepackage{ulem}
\usepackage{xltxtra}
\usepackage{datetime}
\renewcommand{\dateseparator}{.}
\dmyyyydate
\usepackage{fancyhdr}
\usepackage{ifthen}
\pagestyle{fancy}
\fancyhf{}
\renewcommand{\headrulewidth}{0pt}
\fancyfoot[L]{\ifthenelse{\value{page}=1}{\today, \currenttime{} Uhr}{}}
\begin{document}
\begin{table}[ht]
\begin{minipage}[t]{0.5\linewidth}
\small
\begin{center}*D
\end{center}
\begin{tabular}{rl}
\textbf{68} & gedenke an die sippe dîn.\\ 
 & durch re\textit{h}te liebe warte mîn."\\ 
 & Dô sprach der künec von Zazamanc:\\ 
 & "dû\textbf{ne} \textbf{darft} mir \textbf{wizzen} \textbf{keinen} danc,\\ 
5 & swaz dir mîn dienest \textbf{hie} ze êren tuot.\\ 
 & wir sulen haben einen muot.\\ 
 & stêt dîn strûz noch sunder nest,\\ 
 & dû solt dîn sarapandratest\\ 
 & gein sînem halben grîfen tragen.\\ 
10 & mîn anker vaste wirt geslagen\\ 
 & durch lenden in sînes poinders hurt.\\ 
 & er \textbf{muose selbe suochen vurt}\\ 
 & hinderm orse ûf\textbf{me} griez.\\ 
 & der uns \textbf{zein ander} liez,\\ 
15 & ich valte in oder er valte mich.\\ 
 & des wer ich an den triwen dich."\\ 
 & Kaylet ze herbergen reit\\ 
 & mit grôzen vreuden sunder leit.\\ 
 & Sich huob ein krîieren\\ 
20 & \textbf{vor} zwein helden fieren.\\ 
 & von Poytouwe \textbf{Schiolarz}\\ 
 & unt Gurnemanz \textbf{der} Graharz,\\ 
 & die tjustierten ûf dem plân.\\ 
 & sich huop diu vesperîe sân.\\ 
25 & hie riten sehse, dort wol drî.\\ 
 & den \textbf{vuor} vil l\textit{î}hte ein tropel bî.\\ 
 & si \textbf{begunden} rehte rîters tât.\\ 
 & des \textbf{en}was \textbf{êt} dechein rât.\\ 
 & \textbf{Ez} was dennoch wol mitter tac.\\ 
30 & der hêrre \textbf{under} sîme \textbf{gezelte} lac.\\ 
\end{tabular}
\scriptsize
\line(1,0){75} \newline
D Fr9 Fr33 \newline
\line(1,0){75} \newline
\textbf{3} \textit{Majuskel} D   $\cdot$ \textit{Initiale} Fr33  \textbf{19} \textit{Majuskel} D  \textbf{29} \textit{Majuskel} D  \newline
\line(1,0){75} \newline
\textbf{1} gedenke] Daz tu gedenkes Fr9 \textbf{2} rehte] rete D  $\cdot$ warte] vnde warte Fr9 \textbf{3} Zazamanc] Zazamanch D za::manc Fr33 \textbf{4} wizzen keinen] niht wizzen Fr33 \textbf{20} zwein helden] :egenen Fr33 \textbf{21} Poytouwe] Poytoͮwe D P::: Fr33  $\cdot$ Schiolarz] Scyolarz D Scelarz Fr33 \textbf{22} der] von Fr33 \textbf{25} Dort ::: hie ::: Fr33 \textbf{26} lîhte] liehte D \textbf{28} êt dechein] dekein ander Fr33 \textbf{30} under sîme gezelte] in ::: zelte Fr33 \newline
\end{minipage}
\hspace{0.5cm}
\begin{minipage}[t]{0.5\linewidth}
\small
\begin{center}*m
\end{center}
\begin{tabular}{rl}
 & gedenke an die sippe dîn.\\ 
 & durch rehte liebe warte mîn."\\ 
 & \begin{large}D\end{large}ô sprach der künic von Zazamanc:\\ 
 & "dû \textbf{solt} mir\textbf{s} \textbf{wizzen} \textbf{keinen} danc,\\ 
5 & waz dir mîn dienst \textbf{hie} zuo êren tuot.\\ 
 & wir sullen haben einen muot.\\ 
 & stât dîn strûz noch sunder \dag nehst\dag ,\\ 
 & dû solt dîn sarapandratest\\ 
 & gegen sînem halben grîfen tragen.\\ 
10 & mîn anker vaste wirt geslagen\\ 
 & durch lenden in sînes ponders hürt.\\ 
 & er \textbf{in s\textit{el}ber suochen würt}\\ 
 & hinder dem ros ûf \textbf{dem} g\textit{r}ieze.\\ 
 & der uns \textbf{zuo ein ander} \textit{l}ieze,\\ 
15 & ich valte in oder er valte mich.\\ 
 & de\textit{s} wer ich an den triuwen dich."\\ 
 & Kailet ze herberge reit\\ 
 & mit grôzen vröuden sunder leit.\\ 
 & sich huop ein krîgieren\\ 
20 & \textbf{von} zwein helden fieren.\\ 
 & von Poitouw \textbf{Schiolarz}\\ 
 & und Gurnemanz \textbf{de} Graharz,\\ 
 & die justierten ûf dem plâ\textit{n}.\\ 
 & sich huop diu vesperîe sâ\textit{n}.\\ 
25 & hie riten sehs, dort wol drî.\\ 
 & den \textbf{vuor} vil lîhte ein \textit{t}r\textit{o}pel bî.\\ 
 & si \textbf{begunden} reht ritters tât.\\ 
 & des was \textbf{dô} kein \textbf{ander} rât.\\ 
 & \textbf{ez} was dannoch wol mitter tac.\\ 
30 & der hêrre \textbf{in} sîne\textit{m} \textbf{zelte} lac.\\ 
\end{tabular}
\scriptsize
\line(1,0){75} \newline
m n o \newline
\line(1,0){75} \newline
\textbf{3} \textit{Überschrift:} Also der koͯnnig zazamang gar froͯlichen zuͯ sinen dienern rette n   $\cdot$ \textit{Initiale} m n  \newline
\line(1,0){75} \newline
\textbf{2} rehte liebe] [liebe rehte]: rehte liebe o \textbf{3} \textit{Die Verse 68.3-24 fehlen} o   $\cdot$ Zazamanc] zazamanck m zazamang n \textbf{5} hie] \textit{om.} o \textbf{7} noch] nach m \textbf{8} sarapandratest] sarap andrest n \textbf{11} hürt] hort n \textbf{12} selber] sleber m \textbf{13} grieze] [gie]: giesse m \textbf{14} lieze] verliesse m \textbf{16} des] Der \textit{nachträglich korrigiert zu:} Des m \textbf{17} Kailet] Kaẏlet n \textbf{20} von] Mit n \textbf{21} Poitouw] pontomi n  $\cdot$ Schiolarz] sciolars n \textbf{22} Gurnemanz] Gurnemans m gúrnemantz n  $\cdot$ de Graharz] degraharz m degrahars n \textbf{23} plân] plant m \textbf{24} sân] sant m \textbf{26} lîhte] luͯchte o  $\cdot$ tropel] kriupel m krippel n kruͯppel o \textbf{28} was dô] [was utt]: was do m was nuͦ do n was im do o \textbf{29} mitter] mynner n (o) \textbf{30} sînem] sinen m  $\cdot$ zelte] gezelte n o \newline
\end{minipage}
\end{table}
\newpage
\begin{table}[ht]
\begin{minipage}[t]{0.5\linewidth}
\small
\begin{center}*G
\end{center}
\begin{tabular}{rl}
 & gedenke an die \textit{s}i\textit{pp}e dîn.\\ 
 & durch rehte liebe warte mîn."\\ 
 & dô sprach der künic von Zazamanc:\\ 
 & "dû \textbf{solt} mir \textbf{wizzen} \textbf{deheinen} danc,\\ 
5 & swaz dir mîn dienst zêren tuot.\\ 
 & wir sulen haben einen muot.\\ 
 & stêt dîn strûz noch sunder nest,\\ 
 & dû solt dîn serapandratest\\ 
 & gein sînem halben grîfen tragen.\\ 
10 & mîn anker vaste wirt \textit{g}eslagen\\ 
 & durch lenden in sînes ponders hurt.\\ 
 & er \textbf{muose selbe suochen vurt}\\ 
 & hinderm orse ûf \textbf{dem} grieze.\\ 
 & der uns \textbf{zesamene} lieze,\\ 
15 & ich valt in oder er valte mich.\\ 
 & des wer ich an den triwen dich."\\ 
 & Kailet ze herbergen reit\\ 
 & mit grôzen vröuden sunder leit.\\ 
 & sich huop ein kroyieren\\ 
20 & \textbf{von} zwein \textit{h}e\textit{ld}en fieren.\\ 
 & von Potytouwe \textbf{Tschierarz}\\ 
 & unde Gurnomanz \textbf{von} Graharz,\\ 
 & die tjostierten ûf dem plân.\\ 
 & sich huop diu vesperîe sân.\\ 
25 & \begin{large}H\end{large}ie riten sehse, dort wol drî.\\ 
 & den \textbf{vuor} vil lîhte ein tropel bî.\\ 
 & si \textbf{begunden} rehte rîters tât.\\ 
 & des was \textbf{êt} \textbf{dô} dehein \textbf{ander} rât.\\ 
 & \textbf{ez} was dannoch wol mitter tac.\\ 
30 & der hêrre \textbf{in} sînem \textbf{gezelte} lac.\\ 
\end{tabular}
\scriptsize
\line(1,0){75} \newline
G I O L M Q R Z Fr21 \newline
\line(1,0){75} \newline
\textbf{1} \textit{Initiale} O  \textbf{3} \textit{Initiale} L M R Z Fr21  \textbf{17} \textit{Initiale} I  \textbf{25} \textit{Initiale} G  \textbf{29} \textit{Initiale} L Q R Z Fr21  \newline
\line(1,0){75} \newline
\textbf{1} gedenke] Gedenkin M  $\cdot$ sippe] triwe G \textbf{2} durch] Durcht M  $\cdot$ liebe] triwe I  $\cdot$ warte] wart her M gedenck Q \textbf{3} künic] \textit{om.} R  $\cdot$ Zazamanc] zazamanch G O L zazamant Q \textbf{4} dû solt] dun darft I (O) (Q) (Fr21) Duͯ darft L (R) (Z)  $\cdot$ mir wizzen] mirs wizzen I mir sagen L \textbf{5} swaz] Waz L (M) (Q) (R) Z  $\cdot$ mîn] mir Z  $\cdot$ zêren] hie eren L hie zcu eren M (Q) (R) (Z) (Fr21) \textbf{6} haben] niht haben L \textbf{7} dîn] dem Q  $\cdot$ noch] \textit{om.} Z  $\cdot$ nest] vest Z \textbf{9} sînem] sinē M Q sinen R Z Fr21  $\cdot$ halben] \textit{om.} L valben Q \textbf{10} mîn] Mit M  $\cdot$ wirt] wint R  $\cdot$ geslagen] beslagen G \textbf{11} lenden] lendens I leiden M  $\cdot$ ponders] poyndir O \textbf{12} muose] muͤz I muz Z (Fr21)  $\cdot$ selbe] selben M selber Q R  $\cdot$ suochen] schuchen I \textbf{13} hinderm orse] hinder daz ors I \textbf{14} zesamene] [zesumene]: zesamene O  $\cdot$ lieze] liesz Q \textbf{16} des] Da L Der R \textbf{17} Kailet] Gailet I Z Kaylet O Q R Kaýlet L Kaẏlet Fr21  $\cdot$ herbergen] der [heberge*]: herberge L herbin M herberge R \textbf{18} sunder] sundern M \textbf{19} kroyieren] traveren zieren L kregern M \textbf{20} von] Vnd Q  $\cdot$ helden] degenen G \textbf{21} Potytouwe] potẏtoͮwe G poitau I Poytowe O (Q) pýtowe L poytouw M poitoͯwe R portowe Z poitowe Fr21  $\cdot$ Tschierarz] scrinarz I Tschielarz O Tschilarz L ischielarz M schirlars Q Schielarcz R tschierlarz Z tschilarern Fr21 \textbf{22} Gurnomanz] gurnemanz I Gvrnamanz O [Gvrnamaz]: Gvrnamanz L gurnomans Q gvrnomantz Z Gvrnowanz Fr21  $\cdot$ von] der O Z de L M R Fr21  $\cdot$ Graharz] zgraharz I grahars Q [G*]: Graharcz R \textbf{23} tjostierten] tyostieren L (M)  $\cdot$ ûf dem] vf den I L auf der O (M) (Fr21) den Q \textbf{24} vesperîe] vespertie Q \textbf{25} dort wol] do riten Q \textbf{26} den] Da M  $\cdot$ lîhte] dicke L  $\cdot$ tropel] tripel M \textbf{27} si] die I  $\cdot$ begunden] begen Q  $\cdot$ rehte] rehter I (O) (L) (M) (Q) Fr21 rehters Z  $\cdot$ rîters] [Ritten*]: Ritters R \textit{om.} Z  $\cdot$ tât] art M \textbf{28} des] Do Fr21  $\cdot$ was] enwas O (M) Z (Fr21)  $\cdot$ êt] \textit{om.} M Q Z  $\cdot$ dô] dar M (Z)  $\cdot$ dehein] hein Fr21  $\cdot$ ander] \textit{om.} L Z \textbf{29} dannoch] da M \textbf{30} gezelte] [zelte]: gezelte I (M) \newline
\end{minipage}
\hspace{0.5cm}
\begin{minipage}[t]{0.5\linewidth}
\small
\begin{center}*T (U)
\end{center}
\begin{tabular}{rl}
 & gedenke an die sippe dîn.\\ 
 & durch rehte liebe warte mîn."\\ 
 & \begin{large}D\end{large}ô sprach der künec von Zazamanc:\\ 
 & "dû \textbf{darft} mir\textbf{s} \textbf{sagen} \textbf{kleinen} danc,\\ 
5 & waz dir mîn dienst z\textit{ê}ren tuot.\\ 
 & wir soln haben einen muot.\\ 
 & stêt dîn strûz noch sunder nest,\\ 
 & dû solt dînen sarapandertest\\ 
 & gein sîme halben grîfe tragen.\\ 
10 & mîn anker vaste wirt geslagen\\ 
 & durch lenden in sînes poynders hurt.\\ 
 & er \textbf{muoze selbe suochen vurt}\\ 
 & hinderm orse, ûf \textbf{ein} grieze.\\ 
 & der uns \textbf{zuosamene} lieze,\\ 
15 & ich valte in \textit{o}d\textit{er e}r valte mich.\\ 
 & des wer ich an den triuwen dich."\\ 
 & Kaylet zuo \textbf{der} herbergen reit\\ 
 & mit grôzen vreuden sunder leit.\\ 
 & sich huop ein crieren\\ 
20 & \textbf{von} zwein helden fieren.\\ 
 & von Poitowe \textbf{Deschelarz}\\ 
 & und Gurnemanz \textbf{de} Graharz,\\ 
 & die jostierten ûf dem plân.\\ 
 & sich huop diu vesper\textit{î}e sân.\\ 
25 & hie riten sehse, dort wol drî.\\ 
 & de\textit{n} \textbf{reit} vil lîhte ein troppel bî.\\ 
 & si \textbf{begiengen} rehte ritters tât.\\ 
 & des was \textbf{dô} kein \textbf{ander} rât.\\ 
 & \textbf{\begin{large}D\end{large}iz} was dannoch wol mitter tac.\\ 
30 & der hêrre \textbf{in} sîme \textbf{gezelte} lac.\\ 
\end{tabular}
\scriptsize
\line(1,0){75} \newline
U V W T \newline
\line(1,0){75} \newline
\textbf{3} \textit{Initiale} U V   $\cdot$ \textit{Majuskel} T  \textbf{17} \textit{Initiale} T  \textbf{19} \textit{Majuskel} T  \textbf{29} \textit{Initiale} U V W   $\cdot$ \textit{Majuskel} T  \newline
\line(1,0){75} \newline
\textbf{2} warte] gedencke W \textbf{3} der] \textit{om.} T  $\cdot$ Zazamanc] zazamang V W \textbf{4} mirs sagen] mirs wissen V (T) mir weisen W  $\cdot$ kleinen] kainen W \textbf{5} waz] swaz V (T)  $\cdot$ zêren] zorren U \textbf{8} sarapandertest] [Sarap*]: Sarapandrest V \textbf{9} sîme] sinen V (W)  $\cdot$ grîfe] griffen V (W) (T) \textbf{10} geslagen] beschlagen W \textbf{12} muoze] muͤste V muͦß W mvese T  $\cdot$ selbe] selben V W \textbf{13} \textit{Versfolge 68.14-13} T   $\cdot$ ein] dem V T den W \textbf{14} lieze] ließ W \textbf{15} oder er] al dar U \textbf{17} Kaylet] Kylet U Kaẏlet V Gaylet W  $\cdot$ zuo der] zuͦ W (T)  $\cdot$ herbergen] herberg W \textbf{20} von] vor V \textbf{21} Poitowe] Poytuͦwe U [poẏ*]: poẏtowiere V poyteweide W  $\cdot$ Deschelarz] de scelars U Schelars V yschilars W descelarz T \textbf{22} Gurnemanz] Guͦrnemans U Gurnemans V (W)  $\cdot$ Graharz] Grehars U Gregars V grahars W \textbf{24} vesperîe] vespere U \textbf{26} den] Dem U  $\cdot$ reit vil lîhte] lieff auch W  $\cdot$ troppel] [r*]: rotte V \textbf{27} \textit{Versfolge 68.28-27} T   $\cdot$ begiengen rehte] begunden rehter V (T) begunden rechte W \textbf{28} des en was dechein rât T  $\cdot$ was] waz eht V (W) \textbf{29} Diz] Es V (W) Ez T \newline
\end{minipage}
\end{table}
\end{document}
