\documentclass[8pt,a4paper,notitlepage]{article}
\usepackage{fullpage}
\usepackage{ulem}
\usepackage{xltxtra}
\usepackage{datetime}
\renewcommand{\dateseparator}{.}
\dmyyyydate
\usepackage{fancyhdr}
\usepackage{ifthen}
\pagestyle{fancy}
\fancyhf{}
\renewcommand{\headrulewidth}{0pt}
\fancyfoot[L]{\ifthenelse{\value{page}=1}{\today, \currenttime{} Uhr}{}}
\begin{document}
\begin{table}[ht]
\begin{minipage}[t]{0.5\linewidth}
\small
\begin{center}*D
\end{center}
\begin{tabular}{rl}
\textbf{696} & \begin{large}G\end{large}ot müeze ir wîplîch êre sehen.\\ 
 & ich wil \textbf{immer} vrouwen \textbf{sælden} jehen;\\ 
 & ich scham mich noch \textbf{sô} sêre,\\ 
 & ungern ich gein in kêre."\\ 
5 & "Ez muoz doch sîn", sprach Gawan.\\ 
 & \textbf{der} vuorte Parzivaln dan,\\ 
 & dâ in kusten vier künegîn.\\ 
 & die herzogîn \textbf{ez} lêrte pîn,\\ 
 & daz si den küssen solde,\\ 
10 & der ir gruozes \textbf{dô niht} wolde,\\ 
 & dô si \textbf{minne unt ir lant} im bôt\\ 
 & - des \textbf{kom} si hie \textbf{von} scham \textbf{in nôt} -,\\ 
 & dô er vor Logroys gestreit\\ 
 & unt si sô verre nâch im \textbf{reit}.\\ 
15 & Parzival, der clâre,\\ 
 & wart des âne vâre\\ 
 & überparlieret,\\ 
 & \textbf{daz} wart gecondwieret\\ 
 & elliu scham ûz sîme herzen dô.\\ 
20 & âne \textbf{bliuwecheit} wart er vrô.\\ 
 & Gawan \textbf{von} rehten schulden\\ 
 & gebôt bî sînen hulden\\ 
 & vroun Benen, daz ir \textbf{süezer} munt\\ 
 & Itonjen \textbf{des} niht tæte kunt,\\ 
25 & "daz \textbf{mich} der künec Gramoflanz\\ 
 & \textbf{sus} \textbf{hazzet} umbe sînen kranz\\ 
 & unt daz \textbf{wir} morgen ein ander strît\\ 
 & \textbf{sulen} geben ze rehter kampfes zît.\\ 
 & \textbf{mîner swester soltû des niht} sagen\\ 
30 & \textbf{unt solt dîn weinen gar} verdagen."\\ 
\end{tabular}
\scriptsize
\line(1,0){75} \newline
D \newline
\line(1,0){75} \newline
\textbf{1} \textit{Initiale} D  \textbf{5} \textit{Majuskel} D  \newline
\line(1,0){75} \newline
\textbf{6} Parzivaln] Parcifaln D \textbf{15} Parzival] Parcival D \textbf{24} Itonjen] Jtonîen D \textbf{25} Gramoflanz] [Gramofranz]: Gramoflanz D \newline
\end{minipage}
\hspace{0.5cm}
\begin{minipage}[t]{0.5\linewidth}
\small
\begin{center}*m
\end{center}
\begin{tabular}{rl}
 & got müez ir wîplîche êre sehen.\\ 
 & ich wil \textbf{iemer} vrowen \textbf{sælde} jehen;\\ 
 & ich scham mich noch \textbf{sô} sêre,\\ 
 & ungern ich gegen in kêre."\\ 
5 & "ez muoz doch sîn", sprach Gawan.\\ 
 & \textbf{er} vuorte Parcifalen dan,\\ 
 & d\textit{â} in kusten vier künigîn.\\ 
 & die herzogîn \textbf{er} lêrte pîn,\\ 
 & daz si den küssen solte,\\ 
10 & der ir gruozes \textbf{dô niht} wolte,\\ 
 & dô si \textbf{minne und ir lant} im bôt\\ 
 & - des \textbf{kam} si hie \textbf{von} scham \textbf{in nôt} -,\\ 
 & dô er vor Logrois gestreit\\ 
 & und si sô verre nâch im \textbf{reit}.\\ 
15 & Parcifal, der clâre,\\ 
 & wart des âne vâre\\ 
 & überparlieret,\\ 
 & \textbf{daz} wart gecondwieret\\ 
 & alliu scham ûz sînem herzen dô.\\ 
20 & âne \textbf{blœdeheit} wart er vrô.\\ 
 & Gawan \textbf{von} rehten schulden\\ 
 & gebôt bî sîne\textit{n} hulden\\ 
 & vrowe Benen, daz ir \textbf{süezer} munt\\ 
 & Ithonien \textbf{des} niht tæte kunt,\\ 
25 & daz \textbf{in} der künic Gramolanz\\ 
 & \textbf{sô} \textbf{hazzete} umb sînen kranz\\ 
 & und daz \textbf{si} morne ein ander strît\\ 
 & \textbf{solten} geben zuo rehter kampfzît.\\ 
30 & \hspace*{-.7em}\big| \textbf{er sprach: "dî\textit{n} weinen soltû} verdagen\\ 
 & \hspace*{-.7em}\big| \textbf{und solt niht mîner swester} sagen."\\ 
\end{tabular}
\scriptsize
\line(1,0){75} \newline
m n o Fr69 \newline
\line(1,0){75} \newline
\newline
\line(1,0){75} \newline
\textbf{1} müez] muͯs m \textbf{2} iemer] ie o \textbf{6} Parcifalen] parcifaln Fr69 \textbf{7} dâ] Do m n o  $\cdot$ kusten] kussen o  $\cdot$ künigîn] konigen o \textbf{8} er] es n o \textbf{10} dô] da o ::: Fr69 \textbf{13} vor Logrois] fúr grois o \textbf{18} Das wart gekunduwieret n \textbf{21} Gawan] Gawann o \textbf{22} sînen] sinem m \textbf{23} Benen] ::: Fr69 \textbf{24} Ithonien] Jtonien m o Itonien n \textbf{25} Gramolanz] gramolantz m n gramolancz o \textbf{26} hazzete umb] hassesete vnd o \textbf{30} dîn] dine m  $\cdot$ verdagen] vertagen m n vertragen o \newline
\end{minipage}
\end{table}
\newpage
\begin{table}[ht]
\begin{minipage}[t]{0.5\linewidth}
\small
\begin{center}*G
\end{center}
\begin{tabular}{rl}
 & \begin{large}G\end{large}ot müeze ir wîplîch êre sehen.\\ 
 & ich wil \textbf{mîner} vrouwen \textbf{sælden} jehen;\\ 
 & ich scham mich noch sêre,\\ 
 & ungerne ich gein in kêre."\\ 
5 & "ez muoz doch sîn", sprach Gawan.\\ 
 & \textbf{er} vuorte Parcivaln dan,\\ 
 & dâ in kusten vier künigîn.\\ 
 & die herzogîn \textbf{ez} lêrte pîn,\\ 
 & daz si den küssen solde,\\ 
10 & der \textbf{doch} ir gruozes \textbf{niene} wolde,\\ 
 & dô si\textbf{r} \textbf{lant unde ir minne} im bôt\\ 
 & - des \textbf{wart} si hie \textbf{vor} schame \textbf{rôt} -,\\ 
 & dô er vor Logroys gestreit\\ 
 & unde si sô verre nâch im \textbf{reit}.\\ 
15 & Parcival, der clâre,\\ 
 & wart des âne vâre\\ 
 & überparlieret.\\ 
 & \textbf{ez} wart condewieret\\ 
 & elliu scham ûz sînem herzen dô.\\ 
20 & âne \textbf{blûkeit} wart er vrô.\\ 
 & Gawan \textbf{mit} rehten schulden\\ 
 & gebôt bî sînen hulden\\ 
 & vroun Benen, daz ir munt\\ 
 & Itonien \textbf{daz} niht tæte kunt,\\ 
25 & "daz \textbf{mich} der künic Gramoflanz\\ 
 & \textbf{sus} \textbf{hazzet} umbe sînen kranz\\ 
 & unde daz \textbf{wir} morgen ein ander strît\\ 
 & \textbf{sülen} geben ze rehter kampfzît.\\ 
 & \textbf{mîner \textit{swester} soltû des niht} sagen\\ 
30 & \textbf{unde solt dîn weinen gar} verdagen."\\ 
\end{tabular}
\scriptsize
\line(1,0){75} \newline
G I L M Z \newline
\line(1,0){75} \newline
\textbf{1} \textit{Initiale} G I L M Z  \textbf{21} \textit{Initiale} I  \newline
\line(1,0){75} \newline
\textbf{1} êre sehen] \textit{om.} I \textbf{2} mîner] immer Z  $\cdot$ sælden] salde L (M) (Z) \textbf{3} sêre] so sere I (L) M \textbf{4} in] ev I yme M \textbf{6} Parcivaln] parcifaln G Z parzifaln I L parzifal M \textbf{7} in kusten] kusten yn M \textbf{8} ez lêrte] er lete L \textbf{10} Der ir kvssen da niht wolde Z  $\cdot$ doch ir] irs I  $\cdot$ gruozes] gruͯsz L  $\cdot$ wolde] enwolde I \textbf{11} Do sẏ im lant vnd mynne bot L  $\cdot$ dô] Da M Z \textbf{12} vor] von I Z \textbf{13} dô] Da M Z  $\cdot$ Logroys] logroẏs G Logrois I M Z Logroýs L \textbf{14} sô] \textit{om.} I L \textbf{15} Parcival] Parcifal G L Z Parzifal I M \textbf{16} wart] warte I \textbf{17} überparlieret] vberparlipieret I Vber parrieret L \textbf{18} condewieret] Gonduwieret I gecvndvwieret L (M) (Z) \textbf{19} elliu scham] Ellen schein Z  $\cdot$ dô] da M \textbf{20} blûkeit] vlucheit M \textbf{23} Benen] bene M \textbf{24} Itonien] Jtonien G I L (M) Jconie Z  $\cdot$ daz niht] iht I niht L \textbf{25} Gramoflanz] gramorflanz M gramoflantz Z \textbf{27} unde] \textit{om.} I  $\cdot$ ander] andern M \textbf{29} swester] \textit{om.} G  $\cdot$ soltû des] solt duz I \newline
\end{minipage}
\hspace{0.5cm}
\begin{minipage}[t]{0.5\linewidth}
\small
\begin{center}*T
\end{center}
\begin{tabular}{rl}
 & got müeze ir wîplîche êre sehen.\\ 
 & ich wil \textbf{mîner} vrouwen \textbf{s\textit{æ}lde} jehen;\\ 
 & ich schame mich noch \textbf{sô} sêre,\\ 
 & ungerne ich gein in kêre."\\ 
5 & "ez muoz doch sîn", sprach Gawan.\\ 
 & \textbf{er} vuorte Parcifaln dan,\\ 
 & d\textit{â} in kusten viere künegîn.\\ 
 & die herzoginne \textbf{ez} lêrte pîn,\\ 
 & daz si den küssen solte,\\ 
10 & der \textbf{doch} ir gruozes \textbf{niht} \textbf{en}wolte,\\ 
 & dô si \textbf{ir} \textbf{lant und minne} im bôt\\ 
 & - des \textbf{wart} si hie \textbf{von} schame \textbf{rôt} -,\\ 
 & dô er vor Logrois gestreit\\ 
 & und si sô verre nâch im \textbf{gereit}.\\ 
15 & \begin{large}P\end{large}arcifal, der clâre,\\ 
 & wart des âne vâre\\ 
 & überparlieret.\\ 
 & \textbf{ez} wart gecund\textit{w}ieret\\ 
 & alliu schame ûz sîme herzen dô.\\ 
20 & âne \textbf{bl\textit{œ}dekeit} wart er vrô.\\ 
 & Gawan \textbf{mit} rehten schulden\\ 
 & gebôt bî sînen hulden\\ 
 & vrou Bene, daz ir \textbf{süezer} munt\\ 
 & Itonien \textbf{daz} niht tæte kunt,\\ 
25 & "daz \textbf{mich} der künec Gramoflanz\\ 
 & \textbf{sus} \textbf{hazzet} umb sînen kranz\\ 
 & und daz \textbf{w\textit{ir}} morne ein ander strît\\ 
 & \textbf{soln} geben zuo rehter kampfzît.\\ 
 & \textbf{mîner swester solt dû des niht} sagen\\ 
30 & \textbf{und solt dîn weinen gar} verdagen".\\ 
\end{tabular}
\scriptsize
\line(1,0){75} \newline
U V W Q R \newline
\line(1,0){75} \newline
\textbf{15} \textit{Initiale} U V W  \newline
\line(1,0){75} \newline
\textbf{1} êre sehen] [*]: ere sehen V \textbf{2} wil] \textit{om.} Q  $\cdot$ mîner] iemer V (R) nimmer Q  $\cdot$ vrouwen] frowe R  $\cdot$ sælde] solde U \textbf{3} sô sêre] [*]: so sere V \textbf{4} in] Jm R \textbf{5} Gawan] gawan͑ Q \textbf{6} Parcifaln] Parzifaln U Parzefalen V herr partzifalen W partzifalen Q parczifaln R \textbf{7} dâ] Do U V W Q  $\cdot$ viere] vier V W Q R \textbf{8} ez] [er]: ez V er Q \textbf{10} gruozes] gruͤszen V gruͤssens W  $\cdot$ niht enwolte] nienan wolte R \textbf{11} ir lant] [*]: ir lant V  $\cdot$ minne] ir minne W (Q) (R)  $\cdot$ im] \textit{om.} R \textbf{12} des] Do Q  $\cdot$ hie von] hie vor Q des R  $\cdot$ schame] schanden Q \textbf{14} gereit] reit V (W) Q R \textbf{15} Parcifal] PArzival U Parzefal V HErr partzifal W Partzifal Q Parczifal R \textbf{18} gecundwieret] gekuͦndieret U [gecund*]: gecundwieret V \textbf{19} alliu] alle R  $\cdot$ herzen] hertze Q \textbf{20} blœdekeit] bludekeit U blvkeit V (R) \textbf{21} Gawan] Gewan W Gawin R \textbf{22} bî] mit R \textbf{23} Bene] Benen V (W) (R) ben Q  $\cdot$ süezer] \textit{om.} R \textbf{24} Itonien] Jtonien U Q [Ytonien]: Yconien V Ytonyen W Jtenien R  $\cdot$ daz niht] das icht W nicht das Q \textbf{25} mich] [m*]: in V  $\cdot$ Gramoflanz] gramaflanz V gramoflantz W Q Gramoflancz R \textbf{26} sus] Als Q  $\cdot$ umb] vmb vmb Q  $\cdot$ sînen] den R \textbf{27} wir] were U [*]: sv́ V  $\cdot$ ander] [and*]: ander V \textbf{28} soln] Solten V  $\cdot$ kampfzît] [kanpf*]: kanpfes zit V kampffes zeit W zitt R \textbf{29} swester] schweste R \textbf{30} verdagen] vertagen W vertragen Q \newline
\end{minipage}
\end{table}
\end{document}
