\documentclass[8pt,a4paper,notitlepage]{article}
\usepackage{fullpage}
\usepackage{ulem}
\usepackage{xltxtra}
\usepackage{datetime}
\renewcommand{\dateseparator}{.}
\dmyyyydate
\usepackage{fancyhdr}
\usepackage{ifthen}
\pagestyle{fancy}
\fancyhf{}
\renewcommand{\headrulewidth}{0pt}
\fancyfoot[L]{\ifthenelse{\value{page}=1}{\today, \currenttime{} Uhr}{}}
\begin{document}
\begin{table}[ht]
\begin{minipage}[t]{0.5\linewidth}
\small
\begin{center}*D
\end{center}
\begin{tabular}{rl}
\textbf{372} & \begin{large}D\end{large}annen vuor diu magt unt ir gespil.\\ 
 & si buten beide ir dienstes vil\\ 
 & Gawane, dem gaste.\\ 
 & der neig ir hulden vaste.\\ 
5 & \textbf{dô sprach er}: "sult ir werden alt,\\ 
 & trüege denne niht wan sper der walt,\\ 
 & als er\textbf{z} \textbf{am andern holze} hât,\\ 
 & \textbf{daz} \textbf{würde} iu zwein ein ringiu sât.\\ 
 & kan iwer jugent sus twingen,\\ 
10 & welt irz \textbf{in}z alter bringen,\\ 
 & iwer minne lêret noch ritters hant,\\ 
 & dâ von ie schilt gein sper verswant."\\ 
 & Dannen vuoren die meide beide\\ 
 & mit vröuden \textbf{sunder} leide.\\ 
15 & des burcgrâven töhterlîn,\\ 
 & \textbf{diu} sprach: "nû sagt mir, vrouwe mîn,\\ 
 & \textbf{wes} habt ir im ze gebene wân,\\ 
 & sît daz wir niht wan tocken hân.\\ 
 & \textbf{sîn} die mîne iht schœner baz,\\ 
20 & die gebt im âne mînen haz.\\ 
 & dâ wirt vil wênec nâch gestriten."\\ 
 & der vürste Lyppaut kom geriten\\ 
 & an dem berge enmitten.\\ 
 & Obiloten unt Clauditten\\ 
25 & sach er vor im ûf hin gên.\\ 
 & \textbf{er bat si} bêde stille stên.\\ 
 & Dô sprach diu junge Obilot:\\ 
 & "vater, mir wart nie sô nôt\\ 
 & dîner helfe, dar zuo gip mir rât:\\ 
30 & der ritter mich gewert hât."\\ 
\end{tabular}
\scriptsize
\line(1,0){75} \newline
D \newline
\line(1,0){75} \newline
\textbf{1} \textit{Initiale} D  \textbf{13} \textit{Majuskel} D  \textbf{27} \textit{Majuskel} D  \newline
\line(1,0){75} \newline
\textbf{22} Lyppaut] Lẏppaot D \textbf{24} Obiloten] Obyloten D  $\cdot$ Clauditten] Clavditten D \newline
\end{minipage}
\hspace{0.5cm}
\begin{minipage}[t]{0.5\linewidth}
\small
\begin{center}*m
\end{center}
\begin{tabular}{rl}
 & danne vuor diu maget und ir gespil.\\ 
 & si buten beide ir dienstes vil\\ 
 & Gawane, dem gaste.\\ 
 & der neic ir hulden vaste\\ 
5 & \textbf{und sprach}: "sullet ir werden alt,\\ 
 & trüege danne niuwan sper der walt,\\ 
 & alsô er \textbf{ez} \textbf{an anderem holze} hât,\\ 
 & \textbf{daz} \textbf{würde} iu zwein \textit{e}i\textit{n} ringiu sât.\\ 
 & kan iuwer jugent sus twinge\textit{n},\\ 
10 & wellet irz \textbf{in} daz alter bringe\textit{n},\\ 
 & iuwer minne lêret noch ritters hant,\\ 
 & dâ von ie schilt gegen sper verswant."\\ 
 & dannen vuoren die megde beide\\ 
 & mit vröuden \textbf{sunder} leide.\\ 
15 & des burcgrâven töhterlîn,\\ 
 & \textbf{diu} sprach: "nû sagt \textit{mir, vrouwe} mîn,\\ 
 & \textbf{wes} habt ir ime ze gebene wân,\\ 
 & sît daz wir niht wenne t\textit{o}cken hân.\\ 
 & \textbf{sint} die mîne iht schœner baz,\\ 
20 & die gebet ime âne mînen haz.\\ 
 & dâ wirt vil wênic nâch gestriten."\\ 
 & der vürste Lipp\textit{ou}t kam geriten\\ 
 & an dem berge enmitten.\\ 
 & Obiloten und Clauditten\\ 
25 & sach er vor im ûf hin gên.\\ 
 & \textbf{er bat si} beide stille stên.\\ 
 & dô sprach d\textit{iu} junge \textit{O}bilot:\\ 
 & "vater, mir wart nie sô nôt\\ 
 & dîner helfe, dar zuo gip mir rât:\\ 
30 & der ritter mich gewert hât."\\ 
\end{tabular}
\scriptsize
\line(1,0){75} \newline
m n o \newline
\line(1,0){75} \newline
\newline
\line(1,0){75} \newline
\textbf{1} diu] [in]: ir o \textbf{3} Gawane] Gawan n o  $\cdot$ dem gaste] der gast n (o) \textbf{4} neic] neigte n \textbf{6} niuwan] nit dann o  $\cdot$ der] \textit{om.} n  $\cdot$ walt] [wat]: walt o \textbf{7} anderem] dem anderen n (o) \textbf{8} ein] ẏme m \textbf{9} twingen] twingent m \textbf{10} bringen] bringent m \textbf{12} verswant] geswant o \textbf{15} des] Der o \textbf{16} sprach] sprochen n  $\cdot$ sagt] [sprache]: sagt o  $\cdot$ mir vrouwe mîn] [mir]: min m \textbf{17} wes] Was o  $\cdot$ ir] [er]: ir m ir mir n \textbf{18} tocken] decken m \textbf{20} mînen] mẏnne o \textbf{21} wirt vil] wil o \textbf{22} Lippout] lippaot m n o \textbf{24} Obiloten] [Obilo*]: Obilotten m Obeloten o  $\cdot$ Clauditten] clauditen o \textbf{25} ûf hin] hin vff n (o) \textbf{27} diu] der m  $\cdot$ Obilot] abilot m abilat o \textbf{30} gewert] gewet o \newline
\end{minipage}
\end{table}
\newpage
\begin{table}[ht]
\begin{minipage}[t]{0.5\linewidth}
\small
\begin{center}*G
\end{center}
\begin{tabular}{rl}
 & \begin{large}D\end{large}ane vuor diu maget unde ir gespil.\\ 
 & si buten bêde ir dienstes vil\\ 
 & Gawane, dem gaste.\\ 
 & der neic ir hulden vaste.\\ 
5 & \textbf{er sprach}: "\textit{\textbf{und}} sult ir werden alt,\\ 
 & trüege dane niht wan sper der walt,\\ 
 & als er\textbf{z} \textbf{an anderem holze} hât,\\ 
 & \textbf{daz} \textbf{wære} i\textit{u} zwein ein ringiu sât.\\ 
 & kan iwer jugent sus twingen,\\ 
10 & welt irz \textbf{in}z alter bringen,\\ 
 & iwer minne lêret noch rîters hant,\\ 
 & dâ von ie schilt gein sper verswant."\\ 
 & dane vuoren die magede beide\\ 
 & mit vröuden \textbf{âne} leide.\\ 
15 & des burcgrâven töhterlîn\\ 
 & sprach: "nû saget mir, vrouwe mîn,\\ 
 & \textbf{waz} habet ir im ze gebene wân,\\ 
 & sît daz wir niht \textit{wan} tocken hân.\\ 
 & \textbf{sîn} die mîne iht schœner baz,\\ 
20 & die gebet im âne mînen haz.\\ 
 & dâ wirt vil wênic nâch gestriten."\\ 
 & der vürste Libaut kom geriten\\ 
 & an dem berge enmiten.\\ 
 & Obilot unde Clauditen\\ 
25 & sach er vor im ûf hin gên,\\ 
 & \textbf{die bat er} bêde stille stên.\\ 
 & dô sprach diu junge Obilot:\\ 
 & "vater, mir wart nie sô nôt\\ 
 & dîner helfe, dar zuo gip mir rât:\\ 
30 & der rîter mich gewert hât."\\ 
\end{tabular}
\scriptsize
\line(1,0){75} \newline
G I O L M Q R Z Fr24 Fr38 \newline
\line(1,0){75} \newline
\textbf{1} \textit{Initiale} G  \textbf{3} \textit{Initiale} Fr24  \textbf{5} \textit{Initiale} I O L Z Fr38  \textbf{21} \textit{Initiale} I  \textbf{27} \textit{Initiale} Fr24  \newline
\line(1,0){75} \newline
\textbf{1} \textit{Die Verse 370.13-412.12 fehlen} Q   $\cdot$ Dane] da I (R) \textbf{2} bêde] beidu R  $\cdot$ ir] \textit{om.} O  $\cdot$ dienstes] dienst I (M) \textbf{3} Gawane] Gawan I O (M) Z Fr24  $\cdot$ dem] der Fr24 \textbf{4} der] \textit{om.} Fr24 \textbf{5} er] ÷r O  $\cdot$ und] \textit{om.} G \textbf{7} als erz an] Als ers O (Z) Alz erz in L Alse er M Diser Fr24  $\cdot$ holze] holze holze O hohze R \textbf{8} wære] wuͦrde O (L) (M) (R) (Z) (Fr24) (Fr38)  $\cdot$ iu] in G  $\cdot$ ringiu] Ringe R  $\cdot$ sât] sæt G \textbf{10} inz] mit M \textbf{11} lêret] lere L \textbf{12} ie] ir M Fr24  $\cdot$ gein] von I \textbf{13} magede] mege R \textbf{14} vröuden] vreude I \textbf{16} saget] sage Fr24 \textbf{17} waz] Wes O L M Z (Fr24) Fr38  $\cdot$ ze gebene] gegeben O  $\cdot$ wân] nv ane Z \textbf{18} niht wan] niht G \textit{om.} I \textbf{19} sîn] Si Z  $\cdot$ mîne] mynen L mynne M (Z)  $\cdot$ iht] ich R  $\cdot$ schœner] schoͤn I (Z) schoner icht R \textbf{21} dâ] Das R  $\cdot$ wênic] luczil M (Fr24) \textbf{22} Libaut] Lybavt O (Z) (Fr38) Libavt L lybant R Lybo::: Fr24 \textbf{24} Obilot] obilo I Obylot O Z Fr24 Obylote Fr38  $\cdot$ Clauditen] Glauditten I Clavditten O clauditten Z (Fr38) Clavd:::iten Fr24 \textbf{25} ûf] vsz L  $\cdot$ hin] \textit{om.} I \textbf{27} dô] Da O M Z  $\cdot$ Obilot] Obylot O (Z) Fr24 Fr38 oblet R \textbf{28} mir] mirn Fr24 \textbf{29} mir] \textit{om.} R \textbf{30} gewert] [ge*]: geerret R \newline
\end{minipage}
\hspace{0.5cm}
\begin{minipage}[t]{0.5\linewidth}
\small
\begin{center}*T
\end{center}
\begin{tabular}{rl}
 & Dannen vuor diu maget unde ir gespil.\\ 
 & si buten beide ir dienstes vil\\ 
 & Gawane, dem gaste.\\ 
 & der neic ir hulden vaste.\\ 
5 & \textbf{er sprach}: "\textbf{unde} sult ir werden alt,\\ 
 & trüege danne niht wan sper der walt,\\ 
 & alser \textbf{ander holz} hât,\\ 
 & \textbf{er} \textbf{würde} iu zwein ein ring\textit{iu} sât.\\ 
 & kan iuwer jugent sus twingen,\\ 
10 & welt irz \textbf{an}z alter bringen,\\ 
 & iuwer minne lêret noch rîters hant,\\ 
 & dâ von ie schilt gegen sper verswant."\\ 
 & \begin{large}D\end{large}annen vuoren die megede beide\\ 
 & mit vröuden \textbf{âne} leide.\\ 
15 & des burcgrâven töhterlîn\\ 
 & sprach: "nû saget mir, vrouwe mîn,\\ 
 & \textbf{wes} habt ir im ze gebenne wân,\\ 
 & sît daz wir niht wan tocken hân.\\ 
 & \textbf{sîn} die mîne iht schœner baz,\\ 
20 & die gebt im âne mînen haz.\\ 
 & dâ wirt vil wênec nâch gestriten."\\ 
 & Der vürste Lybaut kom geriten\\ 
 & \textbf{gegen i\textit{n}} an dem berge enmiten.\\ 
 & Obyloten unde Clauditen\\ 
25 & sach er vor im ûf hin gân,\\ 
 & \textbf{die bat er} beide stille stân.\\ 
 & Dô sprach diu junge Obylot:\\ 
 & "vater, mir\textbf{n} wart nie sô nôt\\ 
 & dîner helfe, dar zuo gip mir rât:\\ 
30 & der rîter mich gewert hât."\\ 
\end{tabular}
\scriptsize
\line(1,0){75} \newline
T V W \newline
\line(1,0){75} \newline
\textbf{1} \textit{Initiale} W   $\cdot$ \textit{Majuskel} T  \textbf{13} \textit{Initiale} T V  \textbf{22} \textit{Majuskel} T  \textbf{27} \textit{Majuskel} T  \newline
\line(1,0){75} \newline
\textbf{2} ir dienstes] irn dienst W \textbf{3} Gawane] Gawan W  $\cdot$ dem] den W \textbf{8} er] [*]: Daz V  $\cdot$ iu] in W  $\cdot$ ein ringiu] ein ringe T ringen W \textbf{10} anz] ins V W \textbf{17} wes] Waz V (W) \textbf{18} daz] \textit{om.} V \textbf{19} sîn] Sint V (W)  $\cdot$ mîne] minne W  $\cdot$ baz] was W \textbf{20} die gebt] So gebent sy W \textbf{21} wirt] ward W \textbf{22} Lybaut] libaut V lybout W \textbf{23} gegen in] gegn im T Hin W  $\cdot$ dem] den W \textbf{24} Obyloten] Obeloten T Obiloten V Obylot W  $\cdot$ Clauditen] clauditten V klauditen W \textbf{27} Obylot] obilot V abilot W \textbf{28} mirn] mir V W \newline
\end{minipage}
\end{table}
\end{document}
