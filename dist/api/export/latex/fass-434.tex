\documentclass[8pt,a4paper,notitlepage]{article}
\usepackage{fullpage}
\usepackage{ulem}
\usepackage{xltxtra}
\usepackage{datetime}
\renewcommand{\dateseparator}{.}
\dmyyyydate
\usepackage{fancyhdr}
\usepackage{ifthen}
\pagestyle{fancy}
\fancyhf{}
\renewcommand{\headrulewidth}{0pt}
\fancyfoot[L]{\ifthenelse{\value{page}=1}{\today, \currenttime{} Uhr}{}}
\begin{document}
\begin{table}[ht]
\begin{minipage}[t]{0.5\linewidth}
\small
\begin{center}*D
\end{center}
\begin{tabular}{rl}
\textbf{434} & \begin{large}B\end{large}eidiu iwer hêrre unt \textbf{ouch} der mîn.\\ 
 & \textbf{nû} erliuhtet mir die vuore sîn.\\ 
 & der süezen Herzeloyden barn,\\ 
 & wie hât Gahmuretes sun gevarn,\\ 
5 & sît er von Artuse reit?\\ 
 & \textbf{ob} er liep oder \textbf{herzeleit}\\ 
 & sît \textbf{habe} bezalt an strîte?\\ 
 & habt er sich an \textbf{die} wîte\\ 
 & oder hât er \textbf{sider sich} verlegen?\\ 
10 & sagt mir sîne site unt al sîn pflegen."\\ 
 & Nû tuot uns diu âventiure bekant,\\ 
 & er habe \textbf{erstrichen} manec lant\\ 
 & \textbf{zorse} unt \textbf{in schiffen} ûf de\textit{m} wâc.\\ 
 & \textbf{ez} wære lantman oder mâc,\\ 
15 & der tjoste poinder gein im maz,\\ 
 & daz der dekeiner \textbf{nie} gesaz.\\ 
 & sus kan sîn wâge seigen,\\ 
 & sîn selbes prîs ûf steigen\\ 
 & unt die andern lêren sîgen.\\ 
20 & in manegen herten wîgen\\ 
 & hât er sich schumpfentiure erwert,\\ 
 & den lîp gein \textbf{strîte} alsô \textbf{gezert},\\ 
 & swer prîs zim wolte borgen,\\ 
 & der muos ez tuon mit sorgen.\\ 
25 & sîn swert, daz im Anfortas\\ 
 & gap, dô er bîme Grâle was,\\ 
 & brast, sît dô er bestanden wart.\\ 
 & dô machet ez ganz des brunnen art\\ 
 & bî Karnant, der \textbf{dâ} heizet Lac.\\ 
30 & daz swert \textbf{gehalf} im prîses bejac.\\ 
\end{tabular}
\scriptsize
\line(1,0){75} \newline
D Fr31 \newline
\line(1,0){75} \newline
\textbf{1} \textit{Initiale} D  \textbf{11} \textit{Majuskel} D  \newline
\line(1,0){75} \newline
\textbf{2} nû] [ny]: nv D \textbf{3} Herzeloyden] Hercelôyden D herzelaudin Fr31 \textbf{4} Gahmuretes] gahmuretis Fr31 \textbf{5} Artuse] Artv̂se D \textbf{8} habt] Habe Fr31 \textbf{12} erstrichen] durch strichen Fr31 \textbf{13} unt in schiffen] ze schif vnde Fr31  $\cdot$ dem] den D Fr31 \textbf{14} ez] Er Fr31 \textbf{15} poinder] \textit{om.} Fr31 \textbf{21} sich] \textit{om.} Fr31 \textbf{29} Lac] Lach D \newline
\end{minipage}
\hspace{0.5cm}
\begin{minipage}[t]{0.5\linewidth}
\small
\begin{center}*m
\end{center}
\begin{tabular}{rl}
 & beidiu iuwer hêrre und \textbf{ouch} der mîn.\\ 
 & \textbf{nû} erliuht \textit{mir} die vuore \textit{s}în.\\ 
 & der süezen Herczeloiden barn,\\ 
 & wie hât Gahmuretes sun gevarn,\\ 
5 & sît er von Artuse reit?\\ 
 & \textbf{ob} er liep oder \textbf{herzeleit}\\ 
 & sît \textbf{habe} bezalt an strîte?\\ 
 & habt er sich ane wîte\\ 
 & oder hât er \textbf{sider sich} verlegen?\\ 
10 & sagt mir sîne site und al sîn pflegen."\\ 
 & \dag~\dag\ tuot un\textit{s} diu âventiure bekant,\\ 
 & er habe \textbf{erstrichen} manic lant\\ 
 & \textbf{ze rosse} und \textbf{ze schiffe} ûf dem wâc.\\ 
 & \textbf{er} wære lantman oder mâc,\\ 
15 & der joste poinder gegen ime maz,\\ 
 & daz der enkeiner \textbf{nie} gesaz.\\ 
 & sus ka\textit{n} sîn \textit{w}âge seigen,\\ 
 & sîn selbes brîs ûf steigen\\ 
 & und die anderen lêren sîgen.\\ 
20 & in manigen her\textit{t}en wîgen\\ 
 & hât er sich schimpfentiure erwert,\\ 
 & den lîp gegen \textbf{strîte} alsô \textbf{gezert},\\ 
 & wer prîs zuo ime wolte borgen,\\ 
 & der muos ez tuon mit sorgen.\\ 
25 & sîn swert, daz ime Anfortas\\ 
 & gap, dô er bî dem Grâle was,\\ 
 & brast, sît dô er bestanden wart.\\ 
 & dô machet ez ganz des brunne\textit{n} art\\ 
 & bî Karnant, der \textbf{dâ} heizet Lac.\\ 
30 & daz swert \textbf{half} ime prîses bejac.\\ 
\end{tabular}
\scriptsize
\line(1,0){75} \newline
m n o \newline
\line(1,0){75} \newline
\newline
\line(1,0){75} \newline
\textbf{1} ouch] \textit{om.} n o  $\cdot$ mîn] herre min n [man]: mÿn o \textbf{2} mir] uͯch m  $\cdot$ sîn] min m \textbf{3} Herczeloiden] hertzeleiden n herczoleide o \textbf{4} Gahmuretes] gahmurettes m gamuretes n o \textbf{8} sich] \textit{om.} n  $\cdot$ ane] an die n o \textbf{10} sîne] sin n o  $\cdot$ al] alle n o \textbf{11} uns] vnd m \textbf{13} und] \textit{om.} n o  $\cdot$ ûf dem] zuͯ n \textbf{14} er] Es n o  $\cdot$ oder] do oder n \textbf{16} enkeiner] enkeine o \textbf{17} kan] kam m o  $\cdot$ wâge] mage m (n) (o) \textbf{18} \textit{Versdoppelung 434.18-19} m  \textbf{20} herten] herren m \textbf{21} hât] Hette n \textbf{25} Anfortas] an fortas n \textbf{26} \textit{Versdoppelung (²m); Lesarten des vorausgehenden Verses mit ¹m bezeichnet} m   $\cdot$ gap] Brast \textsuperscript{2}\hspace{-1.3mm} m \textbf{28} machet] machte n  $\cdot$ brunnen] brunne m \textbf{29} Karnant] karnang n  $\cdot$ dâ] do n o  $\cdot$ Lac] lag m n o \newline
\end{minipage}
\end{table}
\newpage
\begin{table}[ht]
\begin{minipage}[t]{0.5\linewidth}
\small
\begin{center}*G
\end{center}
\begin{tabular}{rl}
 & beidiu iwer hêrre unde \textbf{ouch} der mîn.\\ 
 & erliuht mir die vuore sîn.\\ 
 & der süezen Herzeloide barn,\\ 
 & wie hât Gahmuretes sun gevarn,\\ 
5 & sît er von Artuse reit?\\ 
 & \textbf{ob} er liep oder \textbf{herzeleit}\\ 
 & sît \textbf{habe} bezalt an strîte?\\ 
 & habet er sich an \textbf{die} wîte\\ 
 & oder hât er \textbf{sider sich} verlegen?\\ 
10 & saget mir sîn site unde al sîn pflegen."\\ 
 & nû tuot uns diu âventiure bekant,\\ 
 & er habe \textbf{erstrichen} manic lant\\ 
 & \textbf{ze orse} unde \textbf{in scheffen} ûf dem wâc.\\ 
 & \textbf{ez} wære lantman oder mâc,\\ 
15 & der tjoste ponder gein im maz,\\ 
 & daz der deheiner \textbf{nie} gesaz.\\ 
 & sus kan sîn wâge seigen,\\ 
 & sîn selbes prîs ûf steigen\\ 
 & \textit{und} die anderen lêren sîgen.\\ 
20 & in manigen herten wîgen\\ 
 & hât er sich schumpfe\textit{n}tiure erwert,\\ 
 & den lîp gein \textbf{strît} alsô \textbf{gezert},\\ 
 & swer prîs ze im wolte borg\textit{en},\\ 
 & der muos ez tuon mit sorgen.\\ 
25 & sîn swert, daz im Anfortas\\ 
 & gap, dô er bî dem Grâle was,\\ 
 & brast, sît dô er bestanden wart.\\ 
 & dô machtez ganz de\textit{s} brunnen art\\ 
 & bî Karnant, der heizet Lac.\\ 
30 & daz swert \textbf{gehalf} im brîse\textit{s} bejac.\\ 
\end{tabular}
\scriptsize
\line(1,0){75} \newline
G I O L M Z \newline
\line(1,0){75} \newline
\textbf{1} \textit{Überschrift:} Hie ist die auentevr wider an parcifaln komen wie der gelebt habe die wile hat gestriten her gawan zv tschanfanzvn Z   $\cdot$ \textit{Initiale} I O L Z  \textbf{15} \textit{Initiale} I  \newline
\line(1,0){75} \newline
\textbf{1} beidiu] ÷ediv O  $\cdot$ hêrre unde ouch der] herre vnd der I (O) (L) vnde der herre M \textbf{2} erliuht] nu betutet I Nv erlvhtet O (L) (M) (Z)  $\cdot$ vuore] sit L \textbf{3} süezen] herzen suͤzen I  $\cdot$ Herzeloide] herzelauden I herzen lavden O Hertzeleuden L herczeloude M herzzenlouden Z \textbf{4} Gahmuretes] gahmvrets G Gamvretes O Gamvretsz L gamuretes M gamuretet Z \textbf{5} Artuse] Artuͯse L \textbf{6} ob er] Hat er O Ab ir M  $\cdot$ oder] olde G vnd L  $\cdot$ herzeleit] leit O \textbf{7} sît] Sie M \textbf{9} oder] olde G  $\cdot$ sider sich] sich sît I sit sich O sich sider L sich M \textbf{10} sîn site] sine sit I (M)  $\cdot$ al] \textit{om.} M  $\cdot$ sîn pflegen] sine phlegen I sine phlege M \textbf{11} diu] de G  $\cdot$ bekant] erchant I \textbf{13} orse] roszen L  $\cdot$ unde in scheffen] oder zesheffe I  $\cdot$ dem] den I dē M \textbf{15} der] Der die I  $\cdot$ im] \textit{om.} I yn M \textbf{17} seigen] sigen I Z \textbf{18} steigen] stigen I Z \textbf{19} und] \textit{om.} G \textbf{20} mangem hertem wige I \textbf{21} sich schumpfentiure] sich schunpheture G Thsvnfentvre L \textbf{22} alsô] so L  $\cdot$ gezert] verzert I (M) \textbf{23} swer] Wer L M  $\cdot$ prîs] \textit{om.} Z  $\cdot$ borgen] borge: G \textbf{24} muos] muͤst I  $\cdot$ ez tuon mit] dar zuͤ I \textbf{25} Anfortas] Amfortas L \textbf{26} dô] da M  $\cdot$ dem] \textit{om.} M \textbf{27} dô] da M Z \textbf{28} dô] Da M Z  $\cdot$ machtez] [machenzt]: machtenzt G macht imz I macht ez O (L)  $\cdot$ des] der G \textbf{29} Karnant] granat M  $\cdot$ der] der da O L M Z  $\cdot$ heizet] heisze M  $\cdot$ Lac] lach G O \textbf{30} daz] Des M  $\cdot$ gehalf] half I behalf Z  $\cdot$ im] \textit{om.} M  $\cdot$ brîses] brise G  $\cdot$ bejac] wach L \newline
\end{minipage}
\hspace{0.5cm}
\begin{minipage}[t]{0.5\linewidth}
\small
\begin{center}*T
\end{center}
\begin{tabular}{rl}
 & beidiu iuwer hêrre un\textit{de} der mîn.\\ 
 & \textbf{nû} erliuhtet mir die vuore sîn.\\ 
 & der süezen Herzeloyden barn,\\ 
 & wie hât Gahmuretes sun gevarn,\\ 
5 & sît er von Artuse reit?\\ 
 & \textbf{hât} er liep oder \textbf{leit}\\ 
 & sît bezalt an strîte?\\ 
 & habt er sich an \textbf{die} wîte\\ 
 & oder hât er \textbf{sich sider} verlegen?\\ 
10 & saget mir sîn site unde alsîn pflegen."\\ 
 & \textit{\begin{large}N\end{large}û} tuot uns diu âventiure bekant,\\ 
 & er habe \textbf{durchstrichen} manec lant\\ 
 & \textbf{ûf orsen} unde \textbf{in den schiffen} ûf de\textit{m} wâc.\\ 
 & \textbf{ez} wære lantman oder mâc,\\ 
15 & der tjoste poynder gegen im maz,\\ 
 & daz der deheiner \textbf{vor im} gesaz.\\ 
 & Sus kan sîn wâge seigen,\\ 
 & sîn selbes prîs ûf steigen\\ 
 & unde die andern lêren sîgen.\\ 
20 & in manegen herten wîgen\\ 
 & hât er sich schumpfentiure erwert,\\ 
 & den lîp gegen \textbf{prîse} alsô \textbf{verzert},\\ 
 & swer prîs zim wolte borgen,\\ 
 & der muos ez tuon mit sorgen.\\ 
25 & Sîn swert, daz im Anfortas\\ 
 & gap, dô er bî dem Grâle was,\\ 
 & brast, sît dô er bestanden wart.\\ 
 & dô macht ez ganz des brunnen art\\ 
 & bî Garnant, der \textbf{dâ} heizet Lac.\\ 
30 & daz swert \textbf{half} im prîses bejac.\\ 
\end{tabular}
\scriptsize
\line(1,0){75} \newline
T U V W Q R \newline
\line(1,0){75} \newline
\textbf{1} \textit{Initiale} V W   $\cdot$ \textit{Capitulumzeichen} R  \textbf{11} \textit{Initiale} T U  \textbf{17} \textit{Majuskel} T  \textbf{25} \textit{Majuskel} T  \newline
\line(1,0){75} \newline
\textbf{1} beidiu] Beider R  $\cdot$ iuwer] \textit{om.} R  $\cdot$ unde] vn T \textbf{2} erliuhtet] erluchte U (W) [*lúhte]: erlúhte  V  $\cdot$ vuore] seúre Q sinne R  $\cdot$ sîn] min R \textbf{3} der] Den U  $\cdot$ Herzeloyden] herzeleiden U herzelauden V hertzeloyde W hertzeloúden Q herczelaude R \textbf{4} Gahmuretes] Gahmvretes T Gahmuͦretes U gamurettez V gamuretes W gamúretes Q \textbf{5} Artuse] Atus R \textbf{6} leit] herzeleit V (W) (Q) (R) \textbf{7} bezalt] behaltt R \textbf{8} habt er] Hant U Oder so habt er R \textbf{9} sich sider] sider sich V W Q sich R \textbf{10} sîn] sine U V R  $\cdot$ alsîn] sin R  $\cdot$ pflegen] plege U \textbf{11} Nû] \textit{om.} T  $\cdot$ tuot] tuͦnt V \textbf{12} durchstrichen] erstrichen U V R erstritten W (Q) \textbf{13} ûf orsen] Ze orsen V (Q) Zuͦ ros W (R)  $\cdot$ unde in] zu R  $\cdot$ den schiffen] schiffen U V Q schiffe W R  $\cdot$ ûf dem wâc] vf denwâc T vnd vff dem wag R \textbf{15} poynder] poyndier T \textbf{16} der] de R  $\cdot$ vor im] nie U V W Q R \textbf{17} kan] [kam]: kan Q  $\cdot$ wâge] wagen Q  $\cdot$ seigen] [segen]: seigen Q \textbf{18} prîs ûf steigen] lob vff Stigen R \textbf{19} lêren] lere W R  $\cdot$ sîgen] singen Q \textbf{20} Jn mancher herren swingen Q  $\cdot$ Jn mengen suͯszen nygen R \textbf{21} sich] \textit{om.} Q \textbf{22} verzert] gezert V W Q ernert R \textbf{23} swer] Wer U W Q R  $\cdot$ zim] gein im U \textbf{24} der] Des R  $\cdot$ muos ez] mvesez T muͤstez V \textbf{25} Anfortas] Anfortâs T Anfortes R \textbf{26} bî dem Grâle] by Ime Im gral R \textbf{27} brast] Brach V R  $\cdot$ wart] waz R \textbf{28} dô] Da Q  $\cdot$ des] der Q \textbf{29} Sy koment da der herczog lag R  $\cdot$ Garnant] karnas U karnant V W (Q)  $\cdot$ dâ] do U V W  $\cdot$ Lac] lag V \textbf{30} half] behalff Q gab R \newline
\end{minipage}
\end{table}
\end{document}
