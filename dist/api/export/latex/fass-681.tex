\documentclass[8pt,a4paper,notitlepage]{article}
\usepackage{fullpage}
\usepackage{ulem}
\usepackage{xltxtra}
\usepackage{datetime}
\renewcommand{\dateseparator}{.}
\dmyyyydate
\usepackage{fancyhdr}
\usepackage{ifthen}
\pagestyle{fancy}
\fancyhf{}
\renewcommand{\headrulewidth}{0pt}
\fancyfoot[L]{\ifthenelse{\value{page}=1}{\today, \currenttime{} Uhr}{}}
\begin{document}
\begin{table}[ht]
\begin{minipage}[t]{0.5\linewidth}
\small
\begin{center}*D
\end{center}
\begin{tabular}{rl}
\textbf{681} & \begin{large}D\end{large}â\textbf{ne} was \textbf{dennoch niemen wan} sie.\\ 
 & welt ir nû \textbf{hœren vürbaz}, wie\\ 
 & an den selben stunden\\ 
 & Artuses boten vunden\\ 
5 & den künec Gramoflanz mit her?\\ 
 & ûf \textbf{einem} plân bî dem mer:\\ 
 & einhalb vlôz der Sabbins\\ 
 & \textbf{unt} anderhalp der Poynzaclins;\\ 
 & diu zwei wazzer \textbf{seuten} dâ,\\ 
10 & der plân \textbf{was} vester anderswâ.\\ 
 & Rosche Sabbins dort,\\ 
 & diu houbtstat, den \textbf{vierden} ort\\ 
 & begreif mit mûren unt \textbf{ouch} \textbf{mit} graben\\ 
 & unt \textbf{mit manegem turne hôhe} erhaben.\\ 
15 & Des hers loschieren \textbf{was} getân\\ 
 & wol mîlen lanc ûf \textbf{den} plân\\ 
 & und \textbf{ouch} wol halber mîle breit.\\ 
 & Artuses boten widerreit\\ 
 & manec ritter \textbf{in gar} \textbf{unbekant},\\ 
20 & Turkopele, manec sarjant\\ 
 & z\textbf{îser} unt mit lanzen.\\ 
 & \textbf{dar nâch} \textbf{begunde} swanzen\\ 
 & under \textbf{maneger} baniere\\ 
 & manec grôziu rotte schiere.\\ 
25 & von \textbf{busînen} was dâ krach.\\ 
 & daz her man \textbf{gar sich} regen sach;\\ 
 & \textbf{si} wolden an den zîten\\ 
 & gein Joflanze rîten.\\ 
 & \begin{large}V\end{large}on vrouwen \textbf{zoumen} klingâ klinc,\\ 
30 & des \textbf{künec} Gramoflanzes rinc\\ 
\end{tabular}
\scriptsize
\line(1,0){75} \newline
D Fr10 \newline
\line(1,0){75} \newline
\textbf{1} \textit{Initiale} D  \textbf{2} \textit{Initiale} Fr10  \textbf{15} \textit{Majuskel} D  \textbf{20} \textit{Majuskel} D  \textbf{29} \textit{Initiale} D  \newline
\line(1,0){75} \newline
\textbf{1} dennoch] \textit{om.} Fr10  $\cdot$ wan] dann Fr10 \textbf{4} Artuses] Artvss D Artus Fr10 \textbf{5} den] Dem Fr10  $\cdot$ Gramoflanz] Gramoflantz Fr10 \textbf{8} Poynzaclins] Pôynzaclins D Poẏnzaclins Fr10 \textbf{9} seuten] ran Fr10 \textbf{11} Rosce Sabbins dort D (Fr10) \textbf{13} mûren] [muͦwer]: muͦer Fr10  $\cdot$ ouch] \textit{om.} Fr10 \textbf{14} mit] \textit{om.} Fr10  $\cdot$ hôhe] \textit{om.} Fr10 \textbf{16} mîlen] meil Fr10 \textbf{18} Artuses] Artvs D (Fr10)  $\cdot$ widerreit] da wider rait Fr10 \textbf{19} in] \textit{om.} Fr10 \textbf{20} manec] vnd Fr10 \textbf{21} zîser] Ze Eisen Fr10 \textbf{30} Gramoflanzes] Gramoflanzs D \newline
\end{minipage}
\hspace{0.5cm}
\begin{minipage}[t]{0.5\linewidth}
\small
\begin{center}*m
\end{center}
\begin{tabular}{rl}
 & \textbf{wan} d\textit{â} was \textbf{niht wan} sie.\\ 
 & wolt ir nû \textbf{hœren vürbaz}, wie\\ 
 & an den selben stunden\\ 
 & Artuses boten vunden\\ 
5 & den künic Gramola\textit{n}zen mit her?\\ 
 & ûf \textbf{einem} plân bî dem mer:\\ 
 & einhalp \textit{v}lôz der Sabins,\\ 
 & anderhalp de\textit{r} Po\textit{i}nzaclins;\\ 
 & diu zwei wazzer \textbf{swebten} dâ,\\ 
10 & der plân \textbf{was} vester anderswâ.\\ 
 & Rosche Sabins dort,\\ 
 & diu houbtstat, den \textbf{vierden} ort\\ 
 & begrei\textit{f} mit mûren und graben\\ 
 & \textit{und} \textit{\textbf{manigen hôhen turn}} \textit{erhaben}.\\ 
15 & des hers loschieren \textbf{wart} getân\\ 
 & wol mîlen lanc ûf \textbf{dem} plân\\ 
 & und wol halber mîle breit.\\ 
 & Artuses bote\textit{n} widerreit\\ 
 & manic ritter \textbf{in gar} \textbf{unerkant},\\ 
20 & turkopel, manic sarjant\\ 
 & z\textbf{îser} und mit lanzen.\\ 
 & \textbf{dar nâch} \textbf{begunde} swanzen\\ 
 & \textit{u}nder \textbf{manic} banier\\ 
 & manic grôziu rotte schier.\\ 
25 & von \textbf{b\textit{u}sûnen} was d\textit{â k}rach.\\ 
 & daz her man \textbf{sich} regen sach;\\ 
 & \textbf{si} wolten an den zîten\\ 
 & gegen Joflanze rîten.\\ 
 & von vrowen \textbf{zoume} klingâ klin\textit{c},\\ 
30 & des \textbf{künic} Gramolanzes rin\textit{c}\\ 
\end{tabular}
\scriptsize
\line(1,0){75} \newline
m n o \newline
\line(1,0){75} \newline
\newline
\line(1,0){75} \newline
\textbf{1} dâ] do m n o \textbf{4} Artuses] Artus m \textbf{5} Gramolanzen] gramolaczen m gramolantzen n gramalanczen o \textbf{7} einhalp] En [*]: halp o  $\cdot$ vlôz] slos m (n) o \textbf{8} der] den m n o  $\cdot$ Poinzaclins] pontzaclins m ponzaclins n ponczaclins o \textbf{11} Rosche] Rosce m n o \textbf{13} begreif] Begreis m  $\cdot$ mit] \textit{om.} o \textbf{14} \textit{Vers 681.14 fehlt} m   $\cdot$ hôhen turn] tuͯrne hoch o \textbf{15} wart] was n o \textbf{16} mîlen] múlen o \textbf{17} und] Vnd ouch n  $\cdot$ mîle] milen n o \textbf{18} Artuses] Artuͯses o  $\cdot$ boten] botte m n o \textbf{19} unerkant] vnerkante m vnuerkant o \textbf{20} sarjant] sariantte m \textbf{23} under] Ander m \textbf{25} busûnen] basunnen m lasunen n b:suͯnen o  $\cdot$ dâ krach] do craff crach m do crach n o \textbf{28} Joflanze] joflantz m n joflancz o \textbf{29} zoume] zémen n (o)  $\cdot$ klinc] clinge m n o \textbf{30} des] Das o  $\cdot$ Gramolanzes] gramolantzes m n gramolancz o  $\cdot$ rinc] ringe m o [k*n]: ringe n \newline
\end{minipage}
\end{table}
\newpage
\begin{table}[ht]
\begin{minipage}[t]{0.5\linewidth}
\small
\begin{center}*G
\end{center}
\begin{tabular}{rl}
 & \begin{large}D\end{large}â\textbf{ne} was \textbf{niemen, der schiede} sie.\\ 
 & welt ir nû \textit{\textbf{hœren vürbaz}}, wie\\ 
 & an den selben stunden\\ 
 & Artuses boten vunden\\ 
5 & den künic Gramoflanz mit her?\\ 
 & ûf \textbf{dem} plâne bî dem mer:\\ 
 & einhalp vlôz der Sabins\\ 
 & \textbf{unde} anderhalp der Poinsaclins;\\ 
 & diu zwei wazzer \textbf{vluzzen} dâ,\\ 
10 & der plân \textbf{ist} vester anderswâ.\\ 
 & Roisabins dort,\\ 
 & diu houbetstat, den \textbf{fieren} ort\\ 
 & begreif mit mûren unde \textbf{mit} graben\\ 
 & unde \textbf{manigen turn hôch} erhaben.\\ 
15 & des hers loschieren \textbf{was} getân\\ 
 & wol mîle lanc ûf \textbf{den} plân\\ 
 & unde \textbf{ouch} wol halber mîle breit.\\ 
 & Artuses boten widerreit\\ 
 & manic rîter \textbf{unbekant},\\ 
20 & turkopel \textbf{unde} manic sarjant\\ 
 & zuo \textbf{îser} unde mit lanzen.\\ 
 & \textbf{dâr} \textbf{begunden} swanzen\\ 
 & under \textbf{maniger} baniere\\ 
 & manic grôziu rote schiere.\\ 
25 & von \textbf{busûne} was dâ krach.\\ 
 & daz her man \textbf{sich gar} regen sach;\\ 
 & \textbf{die} wolden an den zîten\\ 
 & gein Tschofflanze rîten.\\ 
 & von vrouwen \textbf{zoumen} klin\textit{g}â klinc,\\ 
30 & des \textbf{künic} Gramoflanzes rinc\\ 
\end{tabular}
\scriptsize
\line(1,0){75} \newline
G I L M Z Fr18 Fr24 Fr52 \newline
\line(1,0){75} \newline
\textbf{1} \textit{Initiale} G I Z Fr18  \textbf{3} \textit{Initiale} L  \textbf{19} \textit{Initiale} I  \newline
\line(1,0){75} \newline
\textbf{1} Dâne] Da Fr24  $\cdot$ niemen der schiede] dannoch nieman dan Z \textbf{2} hœren vürbaz] vurbaz hôren G \textbf{4} Artuses] Artvs G (Z) (Fr24) (Fr52) \textbf{5} Gramoflanz] Gramoflanzen I gramorflanz M Gramoflantz Z Fr18 (Fr52) \textbf{7} einhalp] ein halp ein I  $\cdot$ Sabins] sabinsz L Sabẏns Fr18 \textbf{8} Poinsaclins] poẏsaclinsz L poynsadins M poinzaclins Z poẏnsaclẏns Fr18 poyn::: Fr52 \textit{om.} Fr24 \textbf{10} ist] waz L (M) (Z) (Fr52) \textbf{11} Roisabins] Roẏs sabins G Roysabins I Roẏ sabinsz L Rois sabins M Rotsche sabins Z :::ois :: sabins Fr52 \textbf{12} fieren] virden L (M) (Z) (Fr52) \textbf{14} unde] \textit{om.} M  $\cdot$ erhaben] gehaben Fr52 \textbf{15} loschieren] loẏsiern G (L) loischiern I lesieren M (Z) (Fr18) lotschiren Fr52  $\cdot$ was] [wart]: was Z \textbf{16} ûf den] vf dem L (M) (Fr18) \textbf{17} ouch wol halber] wol I wol halber Fr18 \textbf{18} Artuses] Artus G M Z (Fr52)  $\cdot$ boten] bot L \textbf{19} unbekant] vnerchant I in gar vnbekant Z \textbf{20} turkopel] Durch kopel M  $\cdot$ manic] \textit{om.} L M \textbf{21} zuo îser] mit isen I \textbf{23} \textit{Versfolge 681.24-23} I   $\cdot$ under maniger] Vnd manig L (Fr52) \textbf{25} Da was bvsvnen krach Fr52  $\cdot$ busûne] bvsvnen L \textbf{28} Tschofflanze] tschoflanze G L shofanze I schofflanze M Tschoflantze Fr18 \textbf{29} zoumen] zwaum I zovme Fr18  $\cdot$ klingâ] [clina]: chlina G \textbf{30} künic] chunges I  $\cdot$ Gramoflanzes] gramorflanzes M Gramoflantzes Fr18 \newline
\end{minipage}
\hspace{0.5cm}
\begin{minipage}[t]{0.5\linewidth}
\small
\begin{center}*T
\end{center}
\begin{tabular}{rl}
 & \begin{large}D\end{large}â\textbf{n} was \textbf{nieman dan} sie.\\ 
 & wolt ir nû \textbf{vürbaz hœren}, wie\\ 
 & an den selben stunden\\ 
 & Artuses boten vunden\\ 
5 & den künec Gramoflanz mit her?\\ 
 & ûf \textbf{dem} plân bî dem mer:\\ 
 & einhalp vlôz der Sabins\\ 
 & \textbf{und} anderhalp der Poynzaclins;\\ 
 & diu zwei wazzer \textbf{vluzzen} dâ,\\ 
10 & der plân \textbf{was} vester anderswâ.\\ 
 & Roitschesabins dort,\\ 
 & diu houbetstat, den \textbf{werden} ort\\ 
 & begreif mit mûren und \textbf{mit} graben\\ 
 & und \textbf{manegen turn hôch} erhaben.\\ 
15 & des heres loschieren \textbf{was} getân\\ 
 & wol mîlen lanc ûf \textbf{dem} plân\\ 
 & und \textbf{ouch} wol halber mîlen breit.\\ 
 & Artuses boten widerreit\\ 
 & manec rîter \textbf{unbekant},\\ 
20 & turkopel \textbf{und} manec sarjant\\ 
 & zuo \textbf{îsene} und mit lanzen.\\ 
 & \textbf{die} \textbf{begunden} swanzen\\ 
 & under \textbf{maneger} baniere,\\ 
 & manegiu grôze rote schiere.\\ 
25 & von \textbf{busînen} was d\textit{â} krach.\\ 
 & daz her man \textbf{sich gar} regen sach;\\ 
 & \textbf{die} wolten an den zîten\\ 
 & gein Tschoflanze rîten.\\ 
 & von vrouwen \textbf{zoume} klingâ klinc,\\ 
30 & des \textbf{küneges} Gramoflanzes rinc\\ 
\end{tabular}
\scriptsize
\line(1,0){75} \newline
U V W Q R \newline
\line(1,0){75} \newline
\textbf{1} \textit{Initiale} U V W Q  \newline
\line(1,0){75} \newline
\textbf{1} DO enwas (enwas dannoch Q was denoch R ) nieman wande sie V (W) (Q) (R) \textbf{2} ir] \textit{om.} R  $\cdot$ vürbaz hœren] hoͤren fúrbas W \textbf{4} Artuses] Kúnig artus W Artus Q Arttus R \textbf{5} den] [D*]: Den V  $\cdot$ Gramoflanz] gramaflanz V gramoflantzen W gramoflantz Q Gramoflancz R  $\cdot$ mit] mir Q \textbf{7} Sabins] Roitschesabins V \textbf{8} Poynzaclins] poyzachyns U poysaclins V poynzaklins W poẏncaclins R \textbf{9} vluzzen] [*]: seweten V  $\cdot$ dâ] [*]: da V do W Q \textbf{11} Roitschesabins] Roitschabins V Rottschesabins Q Roitsche sabins R \textbf{12} werden] vierden V vieren Q (R) \textbf{13} begreif] Die brieff Q  $\cdot$ und mit] vnd R \textbf{14} und] Vnd mit W \textbf{15} loschieren] loitschierens W \textbf{16} mîlen] [*]: milen V meyle W (Q) (R)  $\cdot$ dem] dē Q den R \textbf{17} halber] halbe Q  $\cdot$ mîlen] mile V (Q) R ineile W \textbf{18} Artuses] Kúnig artus W Artus Q R  $\cdot$ widerreit] nindert reit Q \textbf{19} Manig [*]: ritter in gar vnerkant V \textbf{20} turkopel] [D*]: Durkopele V \textbf{21} zuo îsene] [Z*]: Ziser V Zuͦ eyser W (Q) (R) \textbf{22} [*]: Darnoch begvnden swanzen V  $\cdot$ die] Dar W Q R  $\cdot$ begunden] begunde R \textbf{25} was] ward W  $\cdot$ dâ] do U W Q \textbf{26} sich gar] sich V gar sich W \textbf{27} die] Sv́ V \textbf{28} gein] Von W  $\cdot$ Tschoflanze] Schoflanze V tschoflantze W schoflantze Q schofflancze R \textbf{29} zoume] zoͤmen V (W) (R) \textbf{30} Gramoflanzes] gramaflanzes V gramoflantzes W Gramoflancz R \newline
\end{minipage}
\end{table}
\end{document}
