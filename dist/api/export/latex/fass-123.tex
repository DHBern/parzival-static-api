\documentclass[8pt,a4paper,notitlepage]{article}
\usepackage{fullpage}
\usepackage{ulem}
\usepackage{xltxtra}
\usepackage{datetime}
\renewcommand{\dateseparator}{.}
\dmyyyydate
\usepackage{fancyhdr}
\usepackage{ifthen}
\pagestyle{fancy}
\fancyhf{}
\renewcommand{\headrulewidth}{0pt}
\fancyfoot[L]{\ifthenelse{\value{page}=1}{\today, \currenttime{} Uhr}{}}
\begin{document}
\begin{table}[ht]
\begin{minipage}[t]{0.5\linewidth}
\small
\begin{center}*D
\end{center}
\begin{tabular}{rl}
\textbf{123} & dû maht hie vier ritter sehen,\\ 
 & ob dû \textbf{ze rehte kundest} spehen."\\ 
 & Der knappe vrâgete vürbaz:\\ 
 & "dû nennest ritter, waz ist daz?\\ 
5 & hâstû niht götlîcher kraft,\\ 
 & sô sage mir, wer \textbf{gît} ritterschaft?"\\ 
 & "daz tuot der künec Artus.\\ 
 & junchêrre, komt ir in \textbf{des} hûs,\\ 
 & \textbf{der} bringet iuch an ritters namen,\\ 
10 & daz irs iuch \textbf{nimmer} durfet \textbf{schamen}.\\ 
 & ir mugt wol sîn von ritters art."\\ 
 & von den helden er \textbf{geschouwet} wart.\\ 
 & \begin{large}D\end{large}ô lac diu gotes \textbf{gunst} an im.\\ 
 & von der âventiure ich daz nim,\\ 
15 & diu mich \textbf{mit} wârheit des beschiet:\\ 
 & nie \textbf{mannes} varwe baz geriet\\ 
 & vor im sît Adames zît.\\ 
 & des wart sîn lop von wîben wît.\\ 
 & Aber sprach der knappe sân,\\ 
20 & dâ von ein lachen wart getân:\\ 
 & "\textbf{ay}, ritter \textbf{got}, \textbf{waz} mahtû sîn?\\ 
 & dû hâst sus manec vingerlîn\\ 
 & an \textbf{dînen} lîp gebunden,\\ 
 & dort oben unt hie unden."\\ 
25 & \textbf{al}dâ begreif des knappen hant,\\ 
 & swaz er \textbf{îsers} ame vürsten vant.\\ 
 & daz harnasch begunde er schouwen.\\ 
 & "mîner muoter juncvrouwen\\ 
 & ir vingerlîn an snüeren tragent,\\ 
30 & di\textit{u} niht sus an ein ander ragent."\\ 
\end{tabular}
\scriptsize
\line(1,0){75} \newline
D \newline
\line(1,0){75} \newline
\textbf{3} \textit{Majuskel} D  \textbf{13} \textit{Initiale} D  \textbf{19} \textit{Majuskel} D  \newline
\line(1,0){75} \newline
\textbf{30} diu] die D \newline
\end{minipage}
\hspace{0.5cm}
\begin{minipage}[t]{0.5\linewidth}
\small
\begin{center}*m
\end{center}
\begin{tabular}{rl}
 & dû maht hie vier ritter sehen,\\ 
 & ob dû \textbf{zuo rehte kundest} s\textit{p}ehen."\\ 
 & der knappe vrâgete \textbf{in} vürbaz:\\ 
 & "dû nennest ritter, waz ist daz?\\ 
5 & habest dû niht götlîcher kraft,\\ 
 & sô sage mir, wer \textbf{gap} ritterschaft?"\\ 
 & "daz tuot der künic Artus.\\ 
 & junchêr, kumet ir in \textbf{daz} hûs,\\ 
 & \textbf{er} bringet iuch an ritters namen,\\ 
10 & daz irs \textit{iuch} \textbf{nieme\textit{r}} \textit{durft} \textbf{\textit{b}esch\textit{am}en}.\\ 
 & ir muget wol sîn von ritters art."\\ 
 & von den helden er \textbf{beschouwet} wart.\\ 
 & dô lac diu gotes \textbf{gunst} an ime.\\ 
 & von der âventiure ich daz nime,\\ 
15 & diu mich \textbf{der} wârheit des besch\textit{ie}t,\\ 
 & \textbf{daz} nie \textbf{manne} varwe baz geriet\\ 
 & vor ime sît Adam\textit{e}s zît.\\ 
 & des wart sîn lop von wîben wît.\\ 
 & \begin{large}A\end{large}ber sprach der knappe sân,\\ 
20 & dâ von ein lachen wart getân:\\ 
 & "\textbf{ei}, ritter \textbf{got}, \textbf{waz} maht dû sîn?\\ 
 & dû hâst sust manic vingerlîn\\ 
 & an \textbf{dînem} lîp gebunden,\\ 
 & dort \textit{ob}en und hie unden."\\ 
25 & \textbf{al}dâ begreif des knappen hant,\\ 
 & waz er \textbf{îsers} an dem vürsten vant.\\ 
 & daz harna\textit{s}ch begunde er schouwen.\\ 
 & \textbf{er sprach}: "mîner muoter juncvrouwen\\ 
 & ir vingerlîn an snüeren tragent,\\ 
30 & diu niht sus an ein ander ragent."\\ 
\end{tabular}
\scriptsize
\line(1,0){75} \newline
m n o \newline
\line(1,0){75} \newline
\textbf{19} \textit{Initiale} m   $\cdot$ \textit{Capitulumzeichen} n  \newline
\line(1,0){75} \newline
\textbf{2} zuo] \textit{om.} o  $\cdot$ spehen] sehen m \textbf{5} niht] mit o \textbf{6} gap] git n o \textbf{7} Artus] artuͯs o \textbf{9} bringet] ringet o \textbf{10} Das irs [niemer]: niemen sullen geschehen m \textbf{12} helden] drigen helden n koͯnen holden o \textbf{14} daz] des o \textbf{15} des] das o  $\cdot$ beschiet] bescheidet m \textbf{16} manne] mannes n (o) \textbf{17} Adames] adamas m o adams n  $\cdot$ zît] [zil]: zit o \textbf{21} ei] E m o Er n \textbf{23} dînem] dinen n o \textbf{24} oben] allen m \textbf{25} begreif] begreifft o \textbf{26} îsers] ẏsens n (o) \textbf{27} harnasch] harnach m harnersch o \textbf{29} snüeren] snúre o \textbf{30} ragent] ragen n \newline
\end{minipage}
\end{table}
\newpage
\begin{table}[ht]
\begin{minipage}[t]{0.5\linewidth}
\small
\begin{center}*G
\end{center}
\begin{tabular}{rl}
 & dû maht hie vier rîter sehen,\\ 
 & op dû \textbf{si} \textbf{ze rehte kundest} spehen."\\ 
 & der knappe vrâgte vürbaz:\\ 
 & "dû nennest rîter, waz ist daz?\\ 
5 & habestû niht götelîcher kraft,\\ 
 & sô sage mir, wer \textbf{gît} rîterschaft?"\\ 
 & "daz tuot der künic Artus.\\ 
 & junchêrre, komet ir in \textbf{des} hûs,\\ 
 & \textbf{er} bringet iuch an rîters namen,\\ 
10 & daz irs iuch \textbf{niender} durft \textbf{schamen}.\\ 
 & ir muget wol sîn von rîters art."\\ 
 & von den helden er \textbf{beschouwet} wart.\\ 
 & dô lac diu gotes \textbf{kunst} an ime.\\ 
 & von der âventiure ich daz nime,\\ 
15 & diu mich \textbf{der} wârheit des beschiet:\\ 
 & nie \textbf{mannes} varwe baz geriet\\ 
 & vor im sît Adames zît.\\ 
 & des wart sîn lop von wîben wît.\\ 
 & aber sprach der knappe sân,\\ 
20 & dâ von ein lachen wart getân:\\ 
 & "\textbf{\begin{large}A\end{large}y}, ritter \textbf{guot}, \textbf{waz} maht\textit{û} sîn?\\ 
 & dû hâst sus manic vingerlîn\\ 
 & an \textbf{dînen} lîp gebunden,\\ 
 & dort oben und hie unden."\\ 
25 & \textit{dô} begrei\textit{f} \textit{d}es knappen hant,\\ 
 & swaz er \textbf{îsers} an dem vürsten vant.\\ 
 & daz harnasch begunder schouwen.\\ 
 & "mîner muoter juncvrouwen\\ 
 & ir vingerlîn an snüeren tragent,\\ 
30 & diu niht sus an ein ander ragent."\\ 
\end{tabular}
\scriptsize
\line(1,0){75} \newline
G I O L M Q R Z Fr36 \newline
\line(1,0){75} \newline
\textbf{3} \textit{Initiale} O  \textbf{13} \textit{Initiale} L Q R  \textbf{21} \textit{Initiale} G I  \newline
\line(1,0){75} \newline
\textbf{1} maht] mach R  $\cdot$ vier] me I  $\cdot$ sehen] schawen Q \textbf{2} dû si] dvz O (L) (M) (Q) (R) dv Z  $\cdot$ kundest] chanst I kunnest M (Q)  $\cdot$ spehen] sehen L \textbf{3} der] ÷er O  $\cdot$ vrâgte] fraget in L Sprach M fragt Q \textbf{5} habestû] Hast O  $\cdot$ niht] icht Q \textbf{7} der] \textit{om.} M  $\cdot$ Artus] Artvs O \textbf{8} komet] vnd kvmt Z  $\cdot$ des] dasz Q \textbf{9} er] Der Z  $\cdot$ iuch] auch Q  $\cdot$ an] in L  $\cdot$ namen] man M \textbf{10} Des ir uͯch ez duͯrfent niemer schemen L  $\cdot$ daz irs] Des ir R Daz ir sin Z  $\cdot$ niender] nimmer O (M) (Q) Z \textbf{11} \textit{Versfolge 123.12-11} R   $\cdot$ von] \textit{om.} Z  $\cdot$ rîters] hoher R \textbf{12} den] dem L \textbf{13} dô] SO L Da M Z  $\cdot$ kunst] crafft M \textbf{15} der] mit O L M Q R Z \textbf{16} \textit{Versfolge 123.17-16} R   $\cdot$ varwe] fro R  $\cdot$ baz geriet] was so lieht O \textbf{17} im] ir Q  $\cdot$ Adames] Adams L (Q) (R) adamas M  $\cdot$ zît] gezeit Q \textbf{18} wart] vatter R  $\cdot$ lop] [lip]: lop Z  $\cdot$ von] vnde M \textbf{19} \textit{Versfolge 123.20-19} R  \textbf{21} Ay] Hei I O Eyn M (Q) Ey R Z  $\cdot$ guot] \textit{om.} R  $\cdot$ mahtû] mahte G \textbf{22} sus] so I L  $\cdot$ manic] manc Guͤt I \textbf{23} dînen] dinē M (Q) dinem R \textbf{24} dort] Da Z  $\cdot$ hie] da R \textbf{25} dô] \textit{om.} G Da O M R Z Alda L  $\cdot$ begreif des] begreif gar des G begrieff des R \textbf{26} swaz] Waz L (M) (Q) (R) Z  $\cdot$ er îsers] er ysens I O L (Q) ẏsens er R  $\cdot$ an dem] anden M \textbf{27} daz] den I (R) \textbf{29} snüeren] im Q \textbf{30} niht sus] niht alsus I so niht O L (M) (Q) Fr36 sy nit R  $\cdot$ ander] andren R  $\cdot$ ragent] iagin M tragent R \newline
\end{minipage}
\hspace{0.5cm}
\begin{minipage}[t]{0.5\linewidth}
\small
\begin{center}*T (U)
\end{center}
\begin{tabular}{rl}
 & dû maht hie vier rîter sehen,\\ 
 & ob dû \textbf{ez} \textbf{kanst zuo rehte} spehen."\\ 
 & der knappe vrâgete vürbaz:\\ 
 & "dû nennest rîter, waz ist daz?\\ 
5 & hâstû niht götlîcher kraft,\\ 
 & sô sage mir, wer \textbf{gît} rîterschaft?"\\ 
 & "daz tuot der künec Artus.\\ 
 & junchêrre, komet ir in \textbf{daz} hûs,\\ 
 & \textbf{der} bringet iuch an rîters namen,\\ 
10 & daz irs iuch \textbf{niht} durfet \textbf{schamen}.\\ 
 & ir muget wol sîn von rîters art."\\ 
 & von den helden er \textbf{beschouwet} wart.\\ 
 & dô lac diu gotes \textbf{kunst} an im.\\ 
 & von der âventiure ich daz nim,\\ 
15 & diu \textbf{mit} wârheit des beschiet:\\ 
 & nie \textbf{mannes} varwe baz geriet\\ 
 & vor im sît Adames zît.\\ 
 & des wart sîn lop von wîben wît.\\ 
 & aber sprach der knappe sân,\\ 
20 & dâ von ein lachen wart getân:\\ 
 & "\textbf{ein} rîter, \textbf{got} \textbf{weiz}, mahtû \textbf{wol} sîn.\\ 
 & dû hâst sus manec vingerlîn\\ 
 & an \textbf{dînen} lîp gebunden,\\ 
 & dort obene und hie unden."\\ 
25 & \textbf{al} dâ begreif des knappen hant,\\ 
 & waz er \textbf{îserns} an dem vürsten vant.\\ 
 & den harnasch begunder schouwen.\\ 
 & "\textit{mîner muoter juncvrouwen}\\ 
 & ir vingerlîn an snüeren tragent,\\ 
30 & diu niht sus an ein ander ragent."\\ 
\end{tabular}
\scriptsize
\line(1,0){75} \newline
U V W T \newline
\line(1,0){75} \newline
\textbf{3} \textit{Majuskel} T  \textbf{7} \textit{Majuskel} T  \textbf{13} \textit{Majuskel} T  \textbf{19} \textit{Initiale} W  \textbf{21} \textit{Majuskel} T  \textbf{27} \textit{Majuskel} T  \newline
\line(1,0){75} \newline
\textbf{1} sehen] spehn T \textbf{2} Ob dv kanst rehte spehen V · ob dvse ze rehte kvndest sehn T \textbf{8} daz] dez V (W) (T) \textbf{9} der] er T  $\cdot$ an] in W \textbf{10} irs iuch] ir eúchs W  $\cdot$ niht] niemer V (W) [n*]: nîemer  T \textbf{13} diu] dez V \textbf{14} ich daz] das ich W \textbf{15} mit] mich mit V W T  $\cdot$ des] daz V \textbf{17} Adames] adamas U (T) adams W \textbf{18} wîben] frauwen W \textbf{21} [E*]: Ei ritter waz maht dv sin V · Eî riter gvot waz mahtu sin T \textbf{22} sus] so V \textbf{25} al dâ] da T \textbf{26} waz] Swaz V (T)  $\cdot$ îserns] ysins V (W) isers T \textbf{27} den] Daz T \textbf{28} \textit{Vers 123.28 fehlt} U  \textbf{29} ir vingerlîn] in vingerliv T  $\cdot$ tragent] tragen W \textbf{30} sus] alsvs V  $\cdot$ ragent] ragen W \newline
\end{minipage}
\end{table}
\end{document}
