\documentclass[8pt,a4paper,notitlepage]{article}
\usepackage{fullpage}
\usepackage{ulem}
\usepackage{xltxtra}
\usepackage{datetime}
\renewcommand{\dateseparator}{.}
\dmyyyydate
\usepackage{fancyhdr}
\usepackage{ifthen}
\pagestyle{fancy}
\fancyhf{}
\renewcommand{\headrulewidth}{0pt}
\fancyfoot[L]{\ifthenelse{\value{page}=1}{\today, \currenttime{} Uhr}{}}
\begin{document}
\begin{table}[ht]
\begin{minipage}[t]{0.5\linewidth}
\small
\begin{center}*D
\end{center}
\begin{tabular}{rl}
\textbf{659} & \textbf{\begin{large}H\end{large}êrre}, sît iwer \textbf{starkiu} nôt\\ 
 & ist \textbf{worden wendec} âne tôt,\\ 
 & \textbf{sîn} gâbe stêt in iwer hant:\\ 
 & \textbf{dise} burc unt diz gemezzen lant,\\ 
5 & er \textbf{en}kêrt sich nimer mêr nû dran.\\ 
 & \textit{i}r solt ouch vride \textbf{von} im hân,\\ 
 & des jach er offenbâre\\ 
 & - er ist mit rede der \textbf{wâre} -,\\ 
 & swer dise âventiure erlite,\\ 
10 & daz dem \textbf{sîn} gâbe wonte mite.\\ 
 & Swaz er gesach der werden\\ 
 & ûf kristenlîcher erden,\\ 
 & ez wære magt, wîb oder man,\\ 
 & der ist \textbf{iu} hie vil undertân.\\ 
15 & manec heiden unt heidenîn\\ 
 & \textbf{muose} \textbf{ouch} \textbf{bî uns hie ûffe} sîn.\\ 
 & Nû lât \textbf{diz} volc wider komen,\\ 
 & dâ nâch \textbf{uns sorge sî} \textbf{vernomen}.\\ 
 & ellende \textbf{vrumt} \textbf{mirz} herze kalt.\\ 
20 & der die sterne hât gezalt,\\ 
 & der müeze iuch helfe lêren\\ 
 & unt uns gein vreuden kêren.\\ 
 & Ein muoter ir vruht gebirt,\\ 
 & diu vruht sîner muoter \textbf{muoter} wirt.\\ 
25 & von dem wazzer kumt daz îs,\\ 
 & daz læt denne \textbf{niht} \textbf{decheinen gewîs},\\ 
 & daz wazzer \textbf{en}kum ouch wider von im.\\ 
 & swenne ich gedanke an mich \textbf{nim},\\ 
 & daz ich ûz vreuden bin \textbf{erborn},\\ 
30 & wirt vreude \textbf{noch} \textbf{an} mir erkorn.\\ 
\end{tabular}
\scriptsize
\line(1,0){75} \newline
D \newline
\line(1,0){75} \newline
\textbf{1} \textit{Initiale} D  \textbf{11} \textit{Majuskel} D  \textbf{17} \textit{Majuskel} D  \textbf{23} \textit{Majuskel} D  \newline
\line(1,0){75} \newline
\textbf{6} ir] er D \newline
\end{minipage}
\hspace{0.5cm}
\begin{minipage}[t]{0.5\linewidth}
\small
\begin{center}*m
\end{center}
\begin{tabular}{rl}
 & sît \textbf{nû} iuwer \textbf{starkiu} nôt\\ 
 & ist \textbf{verendet} âne tôt,\\ 
 & \textbf{sîn} gâbe stât in iuwer hant:\\ 
 & \textbf{dise} burc und diz gemezzen lant,\\ 
5 & er kêret sich nimer mêr nû dran.\\ 
 & ir solt ouch vride \textbf{vor} im hân,\\ 
 & des jach e\textit{r} offenbâre\\ 
 & - er ist mit rede der \textbf{wâre} -,\\ 
 & wer \textbf{hie} dise âventiure erlite,\\ 
10 & daz dem \textbf{sô} gâbe wonte mite.\\ 
 & waz er gesach der werden\\ 
 & ûf \textit{k}riste\textit{n}lîcher erden,\\ 
 & ez wær ma\textit{get}, \textit{w}îp oder ma\textit{n},\\ 
 & der ist \textbf{iu} hie vil un\textit{d}e\textit{rt}â\textit{n}.\\ 
15 & manic heiden und heidenîn\\ 
 & \textbf{muoste} \textbf{ouch} \textbf{bî uns hie ûf} sîn.\\ 
 & nû lât \textbf{daz} volc wider komen,\\ 
 & dâ nâch \textbf{uns \textit{s}o\textit{r}ge sî} \textbf{vernomen}.\\ 
 & ellende \textbf{vrumt} \textbf{mirz} herze kalt.\\ 
20 & der di\textit{e} sterne het gezalt,\\ 
 & der müeze iuch helfe lêren\\ 
 & und uns gegen vröude kêren.\\ 
 & ein muoter ir vruht gebirt,\\ 
 & diu vruht sîner muoter \textbf{muoter} w\textit{i}rt.\\ 
25 & von dem wazzer kumt daz îs,\\ 
 & daz lât dan \textbf{niht} \textbf{dekein wîs},\\ 
 & daz wazzer kome ouch wider von im.\\ 
 & wan ich gedanke an mich \textbf{nim},\\ 
 & daz ich ûz vröude bin \textbf{erborn},\\ 
30 & wirt vröude \textbf{von} mir erkorn.\\ 
\end{tabular}
\scriptsize
\line(1,0){75} \newline
m n o Fr69 \newline
\line(1,0){75} \newline
\newline
\line(1,0){75} \newline
\textbf{2} tôt] not o \textbf{5} er] ern Fr69  $\cdot$ nimer mêr nû] nuͯ nyemer me n nv́ niemer Fr69 \textbf{6} vor] von Fr69 \textbf{7} des] Das o ::: Fr69  $\cdot$ jach er] johen m \textbf{12} kristenlîcher] risteclicher m \textbf{13} Es wer man oder wip oder maget m \textbf{14} hie] \textit{om.} Fr69  $\cdot$ undertân] vnuerzaget m \textbf{15} Manig heidin vnd heidine o \textbf{18} sorge] froge m \textbf{19} vrumt] vremt Fr69  $\cdot$ herze] herczen o  $\cdot$ kalt] balt Fr69 \textbf{20} die] dir m \textbf{21} müeze] mvͦz Fr69 \textbf{22} vröude] froiden o \textbf{24} wirt] wart m \textbf{26} dekein] do keine n \textbf{29} vröude] freiden n (o) \textbf{30} von] noch von n o \newline
\end{minipage}
\end{table}
\newpage
\begin{table}[ht]
\begin{minipage}[t]{0.5\linewidth}
\small
\begin{center}*G
\end{center}
\begin{tabular}{rl}
 & \textbf{hêrre}, sît iwer \textbf{schar\textit{pf}iu} nôt\\ 
 & ist \textbf{wendic worden} ân \textbf{den} tôt,\\ 
 & \textbf{\begin{large}M\end{large}în} gâbe stêt in iwer hant:\\ 
 & \textbf{disiu} burc unde ditze gemezzen lant;\\ 
5 & er\textbf{ne} kêrt sich nimmer mê nû dran.\\ 
 & ir sult ouch \textit{vride} \textit{\textbf{von}} \textit{im} hân,\\ 
 & des jach er offenbære\\ 
 & - er ist mit rede der \textbf{gewære} -,\\ 
 & swer dise âventiure erlite,\\ 
10 & daz dem \textbf{sîn} gâbe wonte mite.\\ 
 & swaz er gesach der werden\\ 
 & ûf kristenlîche\textit{r} erden,\\ 
 & ez wære maget, wîb oder man,\\ 
 & der ist hie vil undertân.\\ 
15 & manic heiden unde heidenîn\\ 
 & \textbf{muosen} \textbf{hie ûffe bî uns} sîn.\\ 
 & nû lât \textbf{daz} volc wider komen,\\ 
 & dâ nâch \textbf{uns sorge ist} \textbf{vernomen}.\\ 
 & ellende \textbf{vriunt} \textbf{mîn} herze kalt.\\ 
20 & der die sterne hât gezalt,\\ 
 & der müeze iuch helfe lêren\\ 
 & unde uns gein vröuden kêren.\\ 
 & ein muoter ir vruht gebirt,\\ 
 & diu vruht sîner muoter wirt.\\ 
25 & von dem wazzer kumt daz îs,\\ 
 & daz \textbf{en}lât danne \textbf{deheine wîs},\\ 
 & daz wazzer kom ouch wider von im.\\ 
 & swenne ich gedanke an mich \textbf{genim},\\ 
 & daz ich \textit{ûz} vröuden bin \textbf{geborn},\\ 
30 & wirt \textbf{imer} vröude \textbf{an} mir erkorn.\\ 
\end{tabular}
\scriptsize
\line(1,0){75} \newline
G I L M Z Fr48 \newline
\line(1,0){75} \newline
\textbf{3} \textit{Initiale} G I L Z Fr48  \textbf{23} \textit{Initiale} I  \newline
\line(1,0){75} \newline
\textbf{1} scharpfiu] scharhiv G scharfie L \textbf{2} ân den] bisz anden M an dem Fr48 \textbf{3} Mîn] Sin L M Z (Fr48) \textbf{5} erne kêrt] Her karte M  $\cdot$ nimmer mê nû] nu niht mer I nv niemmer mere L Nummer mer M \textbf{6} ir] [Jch]: Jr L  $\cdot$ vride von im] von im fride G \textbf{7} des jach] Das sprach M \textbf{8} der gewære] gevare L \textbf{9} swer] Wer L M \textbf{10} dem] im I  $\cdot$ gâbe] cram I \textbf{11} swaz] Waz L (M) \textbf{12} kristenlîcher] christenlihen G \textbf{13} maget wîb] wip magt Z \textbf{14} hie] ev hie nu I uͯch L uch hie M (Z) \textbf{16} muosen] muͤst I Musten ouch Z \textbf{17} lât] laze I \textbf{18} uns] in Z  $\cdot$ ist] si Z \textbf{19} vriunt] frvmt L (M)  $\cdot$ mîn herze] mit herczen M mirz herze Z \textbf{20} sterne] sternen L \textbf{21} müeze] moͮze G muz Z \textbf{22} vröuden] vroude M \textbf{23} ein] Diu I \textbf{24} sîner] zuͯ siner L ir Z  $\cdot$ muoter] muter muter Z \textbf{25} daz] da M \textbf{26} enlât] lat I L M  $\cdot$ deheine] niht deheine L niht keinen Z \textbf{27} kom] kvmt L qvam Z \textbf{28} swenne] Wanne L (M)  $\cdot$ gedanke] gandanche I  $\cdot$ genim] nime I (L) (Z) geyn yme M \textbf{29} ûz] ze G \textbf{30} imer] myner M \newline
\end{minipage}
\hspace{0.5cm}
\begin{minipage}[t]{0.5\linewidth}
\small
\begin{center}*T
\end{center}
\begin{tabular}{rl}
 & \textbf{hêrre}, sît iuwer \textbf{scharpfiu} nôt\\ 
 & ist \textbf{wen\textit{d}ic worden} âne \textbf{den} tôt,\\ 
 & \textbf{sîn} gâbe stêt in \textit{iuw}er hant:\\ 
 & \textbf{die} burc und diz gemezzen lant,\\ 
5 & er\textbf{n} kêrt sich nimmer mêr nû dran.\\ 
 & ir solt ouch vride \textbf{von} im hân,\\ 
 & des jach er offenbære\\ 
 & - er ist mit rede der \textbf{gewære} -,\\ 
 & wer dise âventiure erlite,\\ 
10 & daz dem \textbf{sîn} gâbe wonte mite.\\ 
 & waz er gesach der werden\\ 
 & ûf kristenlîcher erden,\\ 
 & ez wære maget, wîp oder man,\\ 
 & der ist \textbf{iu} hie vil undertân.\\ 
15 & manec heiden und heidenîn\\ 
 & \textbf{muosten} \textbf{hie ûf bî uns} sîn.\\ 
 & nû lât \textbf{daz} volc wider komen,\\ 
 & dâ nâch \textbf{ist uns sorge} \textbf{benomen}.\\ 
 & ellende \textbf{vriunt} \textbf{mîn} herze kalt.\\ 
20 & der die sterne hât gezalt,\\ 
 & der müez iuch helf\textit{e} lêren\\ 
 & und uns gê\textit{n} vreuden kêren.\\ 
 & ein muoter ir vruht gebirt,\\ 
 & diu vruht sîner muoter wirt.\\ 
25 & von dem wazzer kumt daz îs,\\ 
 & da\textit{z} \textit{l}ât denne \textbf{niht} \textbf{keinen wîs},\\ 
 & daz wazzer k\textit{o}m\textit{e} ouch wider von im.\\ 
 & wann ich gedanke an mich \textbf{nim},\\ 
 & daz ich ûz vreuden bin \textbf{geborn},\\ 
30 & wirt \textbf{immer} vreut \textbf{an} mir erkorn.\\ 
\end{tabular}
\scriptsize
\line(1,0){75} \newline
Q R W V \newline
\line(1,0){75} \newline
\textbf{3} \textit{Initiale} Q W  \textbf{23} \textit{Initiale} V  \newline
\line(1,0){75} \newline
\textbf{2} wendic] wenick Q f::: wendig R [*]: verendet V \textbf{3} iuwer] seiner Q \textbf{4} die] Dise R W V  $\cdot$ gemezzen] gemeine V \textbf{5} kêrt] enkert V  $\cdot$ nimmer] [ni*mer]: niemer V  $\cdot$ nû] \textit{om.} V \textbf{6} von] vor W \textbf{9} wer] [*]: Swer hie V \textbf{10} wonte] [wone]: wonet R \textbf{11} waz] Swaz V \textbf{14} hie vil] vil hie R \textbf{15} heiden] haide W \textbf{16} Mvͤzent [*]: hie oͮch bi vnz hvffe sin V  $\cdot$ muosten] Muͤssen W  $\cdot$ uns] mir W \textbf{17} nû] \textit{om.} W \textbf{18} ist uns sorge] vns sorge ist R W vns sorge [*]: ist V  $\cdot$ benomen] vernomen R (W) (V) \textbf{19} vriunt] frv́mt V  $\cdot$ mîn] [*]: mirz V \textbf{20} sterne] sternen R W [stern*]: sternen V \textbf{21} helfe] helfen Q \textbf{22} gên vreuden] gefrewden Q \textbf{24} muoter] muͦtter muͦtter R (W) (V)  $\cdot$ wirt] [w*]: wirt V \textbf{26} daz lât] Da zu lat Q Das enlat W (V)  $\cdot$ niht] \textit{om.} R V  $\cdot$ keinen] deheine R (W) V \textbf{27} kome] kam Q kvmet V \textbf{28} wann] Swenne V  $\cdot$ nim] genim V \textbf{30} mir] mich R \newline
\end{minipage}
\end{table}
\end{document}
