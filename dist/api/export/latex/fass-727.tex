\documentclass[8pt,a4paper,notitlepage]{article}
\usepackage{fullpage}
\usepackage{ulem}
\usepackage{xltxtra}
\usepackage{datetime}
\renewcommand{\dateseparator}{.}
\dmyyyydate
\usepackage{fancyhdr}
\usepackage{ifthen}
\pagestyle{fancy}
\fancyhf{}
\renewcommand{\headrulewidth}{0pt}
\fancyfoot[L]{\ifthenelse{\value{page}=1}{\today, \currenttime{} Uhr}{}}
\begin{document}
\begin{table}[ht]
\begin{minipage}[t]{0.5\linewidth}
\small
\begin{center}*D
\end{center}
\begin{tabular}{rl}
\textbf{727} & \begin{large}I\end{large}wer niftel Itonje\\ 
 & sol mîme neven gebieten ê,\\ 
 & daz er den kampf \textbf{durch si} verber,\\ 
 & sî, daz er ir minne ger,\\ 
5 & sô wirt vür wâr der kampf vermiten\\ 
 & gar mit \textbf{strîteclîchen} siten,\\ 
 & und \textbf{helfet} ouch dem neven mîn\\ 
 & hulde dâ zer herzogîn."\\ 
 & Artus sprach: "daz wil ich tuon.\\ 
10 & Gawan, mîner swester sun,\\ 
 & ist \textbf{wol} sô gewaldec ir,\\ 
 & daz si \textbf{beidiu} im und \textbf{ouch} mir,\\ 
 & durch \textbf{ir} zuht die schulde gît.\\ 
 & sô scheidet ir \textbf{disehalp} den strît."\\ 
15 & "Ich tuon", sprach Brandelidelin.\\ 
 & si giengen beide wider în.\\ 
 & dô saz der künec von Punturtoys\\ 
 & \textbf{zuo Ginovern, diu was kurtoys}.\\ 
 & anderthalben \textbf{ir} saz Parzival,\\ 
20 & der was ouch sô lieht gemâl,\\ 
 & \textbf{nie ouge ersach sô schœnen man}.\\ 
 & Artus, \textbf{der künec}, \textbf{der} huop sich dan\\ 
 & \textbf{zuo} sîme neven Gawan.\\ 
 & dem was ze wizzene getân,\\ 
25 & der künec Gramoflanz wære komen.\\ 
 & dô wart ouch schiere \textbf{vor} im vernomen,\\ 
 & \textbf{daz} Artus erbeizte vorem gezelt.\\ 
 & gein dem sprang er ûfez velt.\\ 
 & \textit{\begin{large}S\end{large}}i truogen daz ze samne dâ,\\ 
30 & daz diu herzogîn sprach suone jâ,\\ 
\end{tabular}
\scriptsize
\line(1,0){75} \newline
D \newline
\line(1,0){75} \newline
\textbf{1} \textit{Initiale} D  \textbf{15} \textit{Majuskel} D  \textbf{29} \textit{Initiale} D  \newline
\line(1,0){75} \newline
\textbf{1} Itonje] Jtonie D \textbf{17} Punturtoys] Pvntvrtoẏs D \textbf{19} Parzival] Parcifal D \textbf{29} Si] ÷i D \newline
\end{minipage}
\hspace{0.5cm}
\begin{minipage}[t]{0.5\linewidth}
\small
\begin{center}*m
\end{center}
\begin{tabular}{rl}
 & iuwer niftel Itonie\\ 
 & sol mînem neven gebieten ê,\\ 
 & daz er den kampf \textbf{durch si} verber,\\ 
 & sî, daz \textit{er ir} minne ger,\\ 
5 & sô wirt vür wâr der kampf vermiten\\ 
 & gar mit \textbf{strîteclîchem} siten,\\ 
 & und \textbf{helfet} ouch \textbf{ir} dem neven mîn\\ 
 & hulde d\textit{â} zer herzogîn."\\ 
 & Artus sprach: "daz wil ich tuon.\\ 
10 & Gawan, mîner swester sun,\\ 
 & ist \textbf{wol} sô gewaltic ir,\\ 
 & daz si \textbf{beidiu} im und mir\\ 
 & durch \textbf{die} zuht die schulde gît.\\ 
 & sô scheidet ir \textbf{anderhalp} de\textit{n} strît."\\ 
15 & "ich tuon", sprach Brandelidelin.\\ 
 & si giengen beide wider în.\\ 
 & dô saz der künic von Ponturteis\\ 
 & \textbf{zuo Ginoveren, diu was kurteis}.\\ 
 & anderhalp \textbf{ir} saz Parcifal,\\ 
20 & der was ouch sô lieht gemâl,\\ 
 & \textbf{nie ouge ersach sô schœnen man}.\\ 
 & Artus huop sich \textbf{her} dan\\ 
 & \textbf{zuo} sînem neven Gawan.\\ 
 & dem was zuo wi\textit{zz}en getân,\\ 
25 & der künic Gramolantz wær komen.\\ 
 & dô wart ouch schier \textbf{vor} im vernomen,\\ 
 & Artus erbeizte vor dem gezelt.\\ 
 & gegen dem spranc er ûf daz velt.\\ 
 & si truogen daz zuo samen dâ,\\ 
30 & daz diu herzogîn sprach suon jâ,\\ 
\end{tabular}
\scriptsize
\line(1,0){75} \newline
m n o \newline
\line(1,0){75} \newline
\newline
\line(1,0){75} \newline
\textbf{1} Itonie] ithonie n jtonie o \textbf{2} mînem] mynnen o \textbf{4} sî] Sie o  $\cdot$ er ir] ir er m \textbf{5} der] ir n \textbf{6} strîteclîchem] stritteklichen o \textbf{7} ouch] \textit{om.} n \textbf{8} dâ] do m n o \textbf{13} die zuht] ir zucht n (o) \textbf{14} den] der m \textbf{15} Brandelidelin] brandeledelin o \textbf{17} Ponturteis] punturteisz o \textbf{18} Ginoveren] genofern n genofefern o \textbf{22} Artus] Artuͯs o \textbf{24} wizzen] wisen m \textbf{25} Gramolantz] gramolancz o \textbf{26} vor im] >vor im< o \textbf{29} dâ] do n \newline
\end{minipage}
\end{table}
\newpage
\begin{table}[ht]
\begin{minipage}[t]{0.5\linewidth}
\small
\begin{center}*G
\end{center}
\begin{tabular}{rl}
 & iwer niftel Itonie\\ 
 & sol mînem neven gebieten ê,\\ 
 & daz er den kampf \textbf{durch si} verber,\\ 
 & sî, daz er ir minne ger,\\ 
5 & sô wirt vür wâr der kampf vermiten\\ 
 & gar mit \textbf{strîteclîchen} siten,\\ 
 & unde \textbf{helfet} ouch dem neven mîn\\ 
 & hulde dâ zer herzogîn."\\ 
 & Artus sprach: "daz wil ich tuon.\\ 
10 & Gawan, mîner swester sun,\\ 
 & ist sô gewaltec ir,\\ 
 & daz si im unde mir\\ 
 & durch \textbf{ir} zuht die schulde gît.\\ 
 & sô scheidet ir \textbf{disehalp} den strît."\\ 
15 & "ich tuon", sprach Brandelidelin.\\ 
 & si giengen bêde wider în.\\ 
 & dô saz der künec von Ponturteis\\ 
 & \textbf{zuo Schinovern, diu was ouch kurteis}.\\ 
 & anderhalb saz Parcival,\\ 
20 & der was ouch sô lieht gemâl,\\ 
 & \textbf{ezne wart nie rîter baz getân}.\\ 
 & Artus, \textbf{der künec}, huop sich dan\\ 
 & \textbf{gein} sînem neven Gawan.\\ 
 & dem was ze wizzen getân,\\ 
25 & der künec Gramoflanz wære komen.\\ 
 & dô wart ouch schiere \textbf{von} im vernomen,\\ 
 & Artus erbeizt vor dem gezelt.\\ 
 & gein dem spranc er ûf daz velt.\\ 
 & si truogen daz ze samne dâ,\\ 
30 & daz diu herzogîn sprach suone jâ,\\ 
\end{tabular}
\scriptsize
\line(1,0){75} \newline
G I L M Z Fr20 Fr24 \newline
\line(1,0){75} \newline
\textbf{9} \textit{Initiale} I Fr24  \textbf{27} \textit{Initiale} I M  \newline
\line(1,0){75} \newline
\textbf{1} Itonie] Itonîe G Jconie Z \textbf{3} er] si M \textbf{4} sî] sit I \textbf{5} vür wâr] Gar I \textbf{6} strîteclîchen] stritlichen I (M) (Fr20) Fr24 sentlichen L \textbf{7} helfet] helfe L (M) Z Fr24 \textbf{8} dâ zer] der I  $\cdot$ herzogîn] konnigin M \textbf{9} Artus] Artuͯs L \textbf{11} sô] wol so I L M Z Fr24 \textbf{12} mir] auch mir I \textbf{13} die schulde] ir hvlde Z \textbf{15} Brandelidelin] brandalidelin I Branlidelin L Brandlidelin M :::randelin \textit{nachträglich korrigiert zu:} :::randelidelin Fr20 Brandli::: Fr24 \textbf{17} dô] Da M Z  $\cdot$ Ponturteis] ponturtoys I pvntvrtoys L punterteis M pvnturtois Z \textbf{18} zuo] Vnd L M Fr24  $\cdot$ Schinovern] Ginovern G Ginofern I [G*]: Gynover L ginover M gynovern Z :::rn Fr20 Gynouer Fr24  $\cdot$ was ouch] \textit{om.} L \textbf{19} saz] \textit{om.} I  $\cdot$ Parcival] parcifal G Z Parzifal I (L) (M) Parcif::: Fr24 \textbf{20} ouch sô] vil I  $\cdot$ lieht] licht L M \textbf{22} dan] an M Fr20 \textbf{23} Gawan] Gaw::: Fr24 \textbf{25} Gramoflanz] gramoflaz M gramoflantz Z :::amoflanz Fr20 Gramofla::: Fr24 \textbf{26} dô] Da M Z  $\cdot$ ouch] \textit{om.} M  $\cdot$ von im] von in I vor in L \textit{om.} M vor im Z \textbf{27} erbeizt] erbaizte I (L) (M) \textbf{30} herzogîn] chunginne I  $\cdot$ suone] zesuͦne I son M \newline
\end{minipage}
\hspace{0.5cm}
\begin{minipage}[t]{0.5\linewidth}
\small
\begin{center}*T
\end{center}
\begin{tabular}{rl}
 & iuwer niftele Itonie\\ 
 & sol mîme neven gebieten ê,\\ 
 & daz er den kampf \textbf{gein im} verber,\\ 
 & s\textit{î}, daz er ir minne ger,\\ 
5 & sô wirt vür wâr der kampf vermiten\\ 
 & gar mit \textbf{strîtlîchen} siten,\\ 
 & und \textbf{helfe} ouch dem neven mîn\\ 
 & hulde dâ zuo der herzogîn."\\ 
 & Artus sprach: "daz wil ich tuon.\\ 
10 & Gawan, mîner swester sun,\\ 
 & ist \textbf{wol} sô gewaltic ir,\\ 
 & daz si im und mir\\ 
 & durch \textbf{ir} zuht die schulde gît.\\ 
 & sô scheidet ir \textbf{disehalp} den strît."\\ 
15 & "ich tuon", sprach Brandelidelin.\\ 
 & si giengen beide wider în.\\ 
 & d\textit{â} saz \textit{der künec von Puntertoys},\\ 
 & \textbf{d\textit{â} bî der künec von Brituneis},\\ 
 & anderhalp saz Parcifal,\\ 
20 & der was ouch sô lieht gemâl,\\ 
 & \textbf{ez enwart nie rîter baz getân}.\\ 
 & Artus, \textbf{der künec}, huop sich dan\\ 
 & \textbf{gein} sîme neven Gawan.\\ 
 & dem was \textbf{ouch} zuo wizzen getân,\\ 
25 & der künec Gramoflanz wære komen.\\ 
 & dô wart ouch schiere \textbf{von} im vernomen,\\ 
 & Artus erbeizte vor dem gezelt.\\ 
 & gein dem sprang er ûf daz velt.\\ 
 & si truogen daz zuo samen dâ,\\ 
30 & daz diu herzoginne sprach suone jâ,\\ 
\end{tabular}
\scriptsize
\line(1,0){75} \newline
U V W Q R \newline
\line(1,0){75} \newline
\textbf{9} \textit{Initiale} R  \textbf{17} \textit{Initiale} W  \newline
\line(1,0){75} \newline
\textbf{1} Itonie] Jtonie U R ẏconie V ytonie W Q \textbf{2} ê] \textit{om.} Q \textbf{3} gein im] durch sv́ V (Q) (R)  $\cdot$ verber] erwerb Q verler R \textbf{4} sî] So U Seit W (R) \textbf{5} vür wâr] \textit{om.} R \textbf{6} strîtlîchen] stritlichem W R streiteklichē Q \textbf{7} helfet] helfe U W Q [helfe*]: helfet oͮch ir V helffent och R \textbf{8} Jmme hulde zuͦr herzogin V  $\cdot$ dâ] do W \textbf{9} ich] irh W \textbf{10} Gawan] Gawan sprach W Gawin R \textbf{12} si im] [sim *]: sú beide V \textbf{13} die] vnd Q  $\cdot$ gît] ergit R \textbf{14} scheidet] schidt Q \textbf{15} Brandelidelin] brandelidelein W blandlidelin Q \textbf{16} wider] vnder R \textbf{17} dâ] Do U V (W) Q R  $\cdot$ der künec von] Artus der U  $\cdot$ Puntertoys] Brituͦneis U pvntertoẏs V ponturtoyß W puntertois Q puͯntruͯteis R \textbf{18} Zvͦ Gynovern (gynouern Q ) die (der R ) waz oͮch kvrtoys V (Q) (R)  $\cdot$ Vnd artus der britunoyß W  $\cdot$ dâ] Do U  $\cdot$ Brituneis] Brituͦneis U \textbf{19} saz] [*]: ir saz V saß herr W  $\cdot$ Parcifal] Parzifal U (V) partzifal W Q Parczifal R \textbf{20} lieht] licht Q \textbf{22} dan] [*]: her dan V \textbf{24} ouch] \textit{om.} W Q R \textbf{25} Gramoflanz] Gramaflanz V gramoflantz W Q Gramoflancz R  $\cdot$ wære] e were Q werre Zu hoffe R \textbf{26} dô] Daz V  $\cdot$ von im] \textit{om.} Q  $\cdot$ vernomen] [ver*en]: vernomen V \textbf{29} dâ] do W \textbf{30} suone] zur sune Q \newline
\end{minipage}
\end{table}
\end{document}
