\documentclass[8pt,a4paper,notitlepage]{article}
\usepackage{fullpage}
\usepackage{ulem}
\usepackage{xltxtra}
\usepackage{datetime}
\renewcommand{\dateseparator}{.}
\dmyyyydate
\usepackage{fancyhdr}
\usepackage{ifthen}
\pagestyle{fancy}
\fancyhf{}
\renewcommand{\headrulewidth}{0pt}
\fancyfoot[L]{\ifthenelse{\value{page}=1}{\today, \currenttime{} Uhr}{}}
\begin{document}
\begin{table}[ht]
\begin{minipage}[t]{0.5\linewidth}
\small
\begin{center}*D
\end{center}
\begin{tabular}{rl}
\textbf{220} & \begin{large}D\end{large}az \textbf{Brobarzære} vrouwen lîp\\ 
 & mit ir hulden wære mîn wîp,\\ 
 & sô daz ich si umbevienge,\\ 
 & swie ez mir dar nâch \textbf{ergienge}.\\ 
5 & ir minne ist leider verre\\ 
 & dem künege von Iserterre.\\ 
 & mîn lant untz volc ze Brandigan\\ 
 & müezens immer \textbf{jâmer hân}.\\ 
 & mînes veter\textit{n} sun, Mabonagrin,\\ 
10 & leit ouch dâ ze langen pîn.\\ 
 & Nû bin ich, künec Artus,\\ 
 & her geriten in \textbf{dîn} hûs,\\ 
 & betwungen von ritters hant.\\ 
 & \textbf{dû weist} wol, \textbf{daz in} mîn lant\\ 
15 & \textbf{dir manec} laster \textbf{ist} getân.\\ 
 & des vergiz nû, werder man.\\ 
 & die wîle ich hie gevangen sî,\\ 
 & lâz mich sölhes hazzes vrî.\\ 
 & Mich sol vrou Cunneware\\ 
20 & \textbf{ouch} scheiden von dem vâre,\\ 
 & diu mîne sicherheit enpfienc,\\ 
 & dô ich \textbf{gewâpent} vür si gienc."\\ 
 & Artuses \textbf{vil} \textbf{getriwer} munt\\ 
 & verkôs die schulde sân zestunt.\\ 
25 & \begin{large}D\end{large}ô vriesch wîb unt man,\\ 
 & \textbf{daz} der künec von Brandigan\\ 
 & \textbf{was} geriten \textbf{ûf} den rinc.\\ 
 & nû dar nâher, dringâ drinc!\\ 
 & schiere wart daz mære breit.\\ 
30 & mit \textbf{zühten} iesch gesellecheit\\ 
\end{tabular}
\scriptsize
\line(1,0){75} \newline
D \newline
\line(1,0){75} \newline
\textbf{1} \textit{Initiale} D  \textbf{11} \textit{Majuskel} D  \textbf{19} \textit{Majuskel} D  \textbf{25} \textit{Initiale} D  \newline
\line(1,0){75} \newline
\textbf{9} vetern] veter D \textbf{23} Artuses] Artvs D \textbf{28} nâher] [nacher]: naher D \newline
\end{minipage}
\hspace{0.5cm}
\begin{minipage}[t]{0.5\linewidth}
\small
\begin{center}*m
\end{center}
\begin{tabular}{rl}
 & daz \textbf{Brobarzere} vrouwen lîp\\ 
 & mit ir hulden wære mîn wîp,\\ 
 & sô daz ich si umbevienge,\\ 
 & wie ez mir dar nâch \textbf{gienge}.\\ 
5 & ir m\textit{inn}e ist leider verre\\ 
 & dem künic von Iserterre.\\ 
 & mîn lant und daz volc ze Bra\textit{n}digan\\ 
 & m\textit{üe}zens iemer \textbf{jâme\textit{r} \textit{h}ân}.\\ 
 & mînes veteren su\textit{n}, \textit{M}o\textit{b}o\textit{n}agrin,\\ 
10 & leit ouch dâ ze langen pîn.\\ 
 & nû bin ich, künic Artus,\\ 
 & her geriten in \textbf{diz} hûs,\\ 
 & betwungen von ritters hant.\\ 
 & \textbf{dû weist} \textbf{vil} wol, \textbf{daz in} mîn lant\\ 
15 & \textbf{dir manic} laster \textbf{ist} getân.\\ 
 & des vergiz nû, \textbf{vil} werder man.\\ 
 & die wîle ich hie gevangen sî,\\ 
 & lâz mich solhes h\textit{a}zzes vrî.\\ 
 & mich sol vrouwe \textit{C}un\textit{n}ew\textit{a}re\\ 
20 & \textbf{ouch} scheiden von dem vâre,\\ 
 & diu mîne sicherheit enpfienc,\\ 
 & dô ich \textbf{gevangen} vür si gienc."\\ 
 & Artuses \textbf{getriuwelîcher} munt\\ 
 & verkôs die schulde sân zestunt.\\ 
25 & \begin{large}D\end{large}ô vriesch \textbf{d\textit{â}} wîp und man:\\ 
 & der künic von Brandigan\\ 
 & \textbf{was} geriten \textbf{ûf} den rinc.\\ 
 & nû dar nâher, dringâ drinc!\\ 
 & schiere wart d\textit{az m}ære breit.\\ 
30 & mit \textbf{zühten} iesch gesellecheit\\ 
\end{tabular}
\scriptsize
\line(1,0){75} \newline
m n o Fr69 \newline
\line(1,0){75} \newline
\textbf{25} \textit{Initiale} m   $\cdot$ \textit{Capitulumzeichen} n  \newline
\line(1,0){75} \newline
\textbf{1} Brobarzere] brobarcere m brobrartzere n brobarczere o \textbf{2} mîn] [eim]: Min m \textbf{4} Swie es mir ergienge Fr69 \textbf{5} minne] mume m (n) \textbf{6} Iserterre] [iser terer]: iser tere m iser terre n o \textbf{7} ze] \textit{om.} n o  $\cdot$ Brandigan] bradigan m \textbf{8} müezens] Mussens m Muͯssen sin n Mussen sin o  $\cdot$ jâmer hân] iomer iomer han m \textbf{9} sun Mobonagrin] sun von [nobe]: [nobo]: monobagrin m suͦn monobagrin n suͯn monabagrin o sun [Mon]: Mobonagrin Fr69 \textbf{10} dâ] do n o  $\cdot$ langen] lange n o langen hohen Fr69 \textbf{11} Artus] artuͯs o \textbf{12} diz hûs] das [has]: disz huͯs o \textbf{14} mîn] mẏ o \textbf{16} vergiz] vergisse n morgens vergasz o \textbf{17} hie] [sie]: hie o \textbf{18} lâz] Losse n  $\cdot$ hazzes] hesses m \textbf{19} Cunneware] gunewere m guneware n gymewaren o \textbf{23} Artuses] Artusus o  $\cdot$ getriuwelîcher] getruwecklicher m \textbf{24} verkôs] Verkose n \textbf{25} dâ] do m \textit{om.} n o \textbf{26} der] Der den n o \textbf{29} daz mære] des munt mere m \newline
\end{minipage}
\end{table}
\newpage
\begin{table}[ht]
\begin{minipage}[t]{0.5\linewidth}
\small
\begin{center}*G
\end{center}
\begin{tabular}{rl}
 & \textit{daz} \textbf{\textit{B}riubarz} \textbf{der} vrouwen lîp\\ 
 & \textit{mit i}r hulden wære mîn wîp,\\ 
 & \textit{sô} daz ich si umbevienge,\\ 
 & \textit{s}wiez mir dar nâch \textbf{ergienge}.\\ 
5 & \textit{i}r minne ist leider verre\\ 
 & dem künige von Yserterre.\\ 
 & mîn lant unt daz volc ze Brandigan\\ 
 & müezens imer \textbf{jâmeric stân}.\\ 
 & mînes veteren sun, Mobonagrin,\\ 
10 & leit ouch dâ ze lange pîn.\\ 
 & nû bin ich, künic Artus,\\ 
 & her geriten in \textbf{dîn} hûs,\\ 
 & betwungen von rîters hant.\\ 
 & \textbf{dû weist} wol, \textbf{daz in} mîn lant\\ 
15 & \textbf{dir manic} laster \textbf{ist} getân.\\ 
 & des vergiz n\textit{û}, \textit{w}erder man.\\ 
 & die wîle ich hie gevangen sî,\\ 
 & lâ mich solhes hazzes vrî.\\ 
 & mich sol vrô Kuneware\\ 
20 & scheiden von dem vâre,\\ 
 & diu mîne sicherheit enpfie,\\ 
 & dô ich \textbf{gevangen} vür si gie."\\ 
 & Artuses \textbf{vil} \textbf{getriwer} munt\\ 
 & verkôs die schulde sân zestunt.\\ 
25 & dô vriesch wîp unde man,\\ 
 & \textbf{daz} der künic von Brandigan\\ 
 & \textbf{was} geriten \textbf{in} den rinc.\\ 
 & \begin{large}N\end{large}û dar nâher, dringâ drinc!\\ 
 & \textbf{vil} schiere wart daz mære breit.\\ 
30 & mit \textbf{zuht} iesch gesellicheit\\ 
\end{tabular}
\scriptsize
\line(1,0){75} \newline
G I O L M Q R Z Fr21 Fr40 \newline
\line(1,0){75} \newline
\textbf{1} \textit{Initiale} R  \textbf{7} \textit{Initiale} Fr21  \textbf{11} \textit{Initiale} O  \textbf{17} \textit{Initiale} I  \textbf{19} \textit{Initiale} Z  \textbf{25} \textit{Initiale} L  \textbf{28} \textit{Initiale} G  \newline
\line(1,0){75} \newline
\textbf{1} daz] ::: G Daz ze O (M) (Q) (R) (Z) (Fr21)  $\cdot$ Briubarz] :::rivbarz G Brvbarz O (M) Z (Fr21) Bruͯbarz L brubars Q burbarcz R \textbf{2} mit ir] :::r G Sold si in O Mir ist M  $\cdot$ hulden] hulde R  $\cdot$ wære] sin O wen M \textbf{3} sô] ::: G  $\cdot$ ich si] ichz Z  $\cdot$ umbevienge] vienge Z \textbf{4} swiez] :::wiez G Wie ez L (Q) (R) \textbf{5} ir] :::r G \textbf{6} Yserterre] yser terre G isenterre I ysen terre O Jsrertere R iserterre Z Jserterre Fr21 eser terre Fr40 \textbf{7} lant unt daz volc] hant vnde zlant von O  $\cdot$ ze Brandigan] zebradigan G zeprandigan I Brandigan O zcu bradigan M zu [*randigan]: brandigan Q \textbf{8} müezens] Mussem Q  $\cdot$ jâmeric stân] han I schaden han O iamer han L (M) Q (R) Z Fr21 (Fr40) \textbf{9} sun] san M  $\cdot$ Mobonagrin] maboagrin G Mabenagrum I [Mv*onagrin]: Mvbonagrin O Mabonagrin L mabonagryn M Mvbonagrin Z Fr21 mabonagr:: Fr40 \textbf{10} leit] Lidet M  $\cdot$ ouch dâ] alda O (Q) R Fr21 oͯch daz L  $\cdot$ ze lange] zuͯ langen L (Fr21) (Fr40) so lange R \textbf{11} nû] ÷v O  $\cdot$ bin] win Q  $\cdot$ ich] \textit{om.} Fr21  $\cdot$ künic] kvnige Fr21  $\cdot$ Artus] Artuͯs L artuͦs Fr40 \textbf{12} dîn] dem Fr40 \textbf{13} betwungen] Betungen R \textbf{14} dû weist] Daz weistu L  $\cdot$ wol] vil wol O M Q R Z Fr21  $\cdot$ in mîn] ich min O nuͯ myn L in myme M (Q) (R) [i*]: ::: min Fr21 \textbf{15} ist] hat O \textbf{16} des] Der Q  $\cdot$ nû werder] nv vil werder G \textbf{18} hazzes] hassen Q \textbf{19} vrô] \textit{om.} R  $\cdot$ Kuneware] kunnware I kvnware O (M) Cvneware L konware Q Cuͦnware R kvnneware Z Cvnware Fr21 (Fr40) \textbf{20} von dem] vnd dem L vonden Fr21  $\cdot$ vâre] were R \textbf{21} mîne] minne R \textbf{22} dô] Da Z  $\cdot$ gevangen] gewoppet Q (R) (Fr40) \textbf{23} Artuses] Artuͯses L Artus M Q Fr40 Arttus R  $\cdot$ vil] wil Q \textbf{24} verkôs] Vor keusche Q  $\cdot$ sân zestunt] da zestunt I ander stuͯnt L sasestund Q \textbf{25} dô] Da Z  $\cdot$ vriesch] gefriese I freisch L vernam R gefriͤsch Z  $\cdot$ unde] noch I \textbf{26} von] \textit{om.} I  $\cdot$ Brandigan] prandigan I Brandegan L \textbf{28} Nû] \textit{om.} R  $\cdot$ nâher dringâ drinc] dinga ding L \textbf{29} daz mære] disz mere L das R  $\cdot$ breit] bereit R \textbf{30} zuht] zvhten O (L) (M) (Q) (R) Z Fr21 (Fr40)  $\cdot$ iesch] [ieslich]: iesch G iclichen M ieslich Fr21 \newline
\end{minipage}
\hspace{0.5cm}
\begin{minipage}[t]{0.5\linewidth}
\small
\begin{center}*T
\end{center}
\begin{tabular}{rl}
 & daz \textbf{ze Breharz} \textbf{der} vrouwen lîp\\ 
 & mit ir hulden wære mîn wîp,\\ 
 & sô daz ich si umbevienge,\\ 
 & swiez mir dar nâch \textbf{ergienge}.\\ 
5 & ir minne ist leider verre\\ 
 & dem künege von Iserterre.\\ 
 & mîn lant unde daz volc ze Brandigan\\ 
 & müezens iemer \textbf{jâmer hân}.\\ 
 & mînes vetern sun, Mobonagrin,\\ 
10 & leit ouch dâ ze langen pîn.\\ 
 & Nû bin ich, künec Artus,\\ 
 & her geriten in \textbf{dîn} hûs,\\ 
 & betwungen von rîters hant.\\ 
 & \textbf{ich weiz} wol, \textbf{dir} mîn lant\\ 
15 & \textbf{grôz} laster \textbf{hât} getân.\\ 
 & des vergiz nû, werder man.\\ 
 & die wîle ich hie gevangen sî,\\ 
 & lâ mich solhes hazzes vrî.\\ 
 & mich sol vrou Cunneware\\ 
20 & \textbf{ouch} scheiden von dem vâre,\\ 
 & diu mîne sicherheit enpfie,\\ 
 & dô ich \textbf{gevangen} vür si gie."\\ 
 & Artuses \textbf{vil} \textbf{getriuwer} munt\\ 
 & verkôs die schulde sân zestunt.\\ 
25 & \textit{\begin{large}D\end{large}}ô vriesch wîp unde man,\\ 
 & \textbf{daz} der künec von Brandigan\\ 
 & \textbf{wære} geriten \textbf{in} den rinc.\\ 
 & nû dar nâher, dringâ drinc!\\ 
 & schiere wart daz mære breit.\\ 
30 & mit \textbf{zühten} iesch gesellecheit\\ 
\end{tabular}
\scriptsize
\line(1,0){75} \newline
T U V W \newline
\line(1,0){75} \newline
\textbf{11} \textit{Majuskel} T  \textbf{23} \textit{Majuskel} T  \textbf{25} \textit{Initiale} T U V W  \newline
\line(1,0){75} \newline
\textbf{1} daz] daz ich U  $\cdot$ Breharz] Brebarz T brebraz U [*]: probartz V brabars W \textbf{2} hulden] hulde W \textbf{4} swiez mir] Wie mir iz U Swie mirs V Wie es mir W \textbf{5} leider] leidor V \textbf{6} Iserterre] Jserterre T Jsenterre U V ysenterre W \textbf{7} mîn lant] \textit{om.} U \textbf{8} müezens] Muͦz iz U Muͤssent W \textbf{9} Mobonagrin] moabogrin W \textbf{10} dâ] do U W  $\cdot$ ze langen] zuͦ lange U (V) \textbf{12} dîn] dis W \textbf{14} Du weist wol das in mein land W  $\cdot$ ich weiz wol] Duͦ weist wol daz U (V) \textbf{15} Dir groß laster ist getan W  $\cdot$ grôz laster] Grozen schaden U (V) \textbf{18} solhes] selber W \textbf{19} Cunneware] Cvnnewâre T Cuneware U kvnneware V kunnewar W \textbf{20} dem vâre] schwar W \textbf{24} sân] do U V \textbf{25} Dô] No T  $\cdot$ vriesch] vreisch V W \textbf{27} wære] [W*r]: Waz V  $\cdot$ geriten] komen W \textbf{29} daz] dise U dis W \newline
\end{minipage}
\end{table}
\end{document}
