\documentclass[8pt,a4paper,notitlepage]{article}
\usepackage{fullpage}
\usepackage{ulem}
\usepackage{xltxtra}
\usepackage{datetime}
\renewcommand{\dateseparator}{.}
\dmyyyydate
\usepackage{fancyhdr}
\usepackage{ifthen}
\pagestyle{fancy}
\fancyhf{}
\renewcommand{\headrulewidth}{0pt}
\fancyfoot[L]{\ifthenelse{\value{page}=1}{\today, \currenttime{} Uhr}{}}
\begin{document}
\begin{table}[ht]
\begin{minipage}[t]{0.5\linewidth}
\small
\begin{center}*D
\end{center}
\begin{tabular}{rl}
\textbf{96} & man sprach ein urteile zehant:\\ 
 & "swelch ritter \textbf{helm hie} ûf \textbf{gebant},\\ 
 & der her \textbf{nâch} rîterschaft \textbf{ist} komen,\\ 
 & \textbf{hât} \textbf{er} den prîs hie genomen,\\ 
5 & den sol diu küneginne hân."\\ 
 & \textbf{dar nâch diu volge wart} getân.\\ 
 & \textbf{dô sprach si}: "hêrre, nû sît ir mîn.\\ 
 & ich tuon iu dienest nâch hulden schîn\\ 
 & unt vüege iu sölher vröuden teil,\\ 
10 & daz ir nâch jâmer werdet geil."\\ 
 & \begin{large}E\end{large}r hete iedoch \textbf{von} jâmer pîn.\\ 
 & \textbf{dô} was des aberellen schîn\\ 
 & \textbf{zergangen}. dar nâch komen was\\ 
 & \textbf{kurz} kleine, grüene gras.\\ 
15 & daz velt was \textbf{gar vergrüenet},\\ 
 & daz \textbf{blœdiu} herzen küenet\\ 
 & unt in gît hôchgemüete.\\ 
 & vil \textbf{boume} stuont in blüete\\ 
 & von dem süezen lufte des meien.\\ 
20 & sîn art von der feien\\ 
 & muose minnen oder minne gern.\\ 
 & des wolde in vriundîn \textbf{dâ} \textbf{gewern}.\\ 
 & An vron Herzeloyden er \textbf{dô} sach.\\ 
 & sîn \textbf{süezer munt mit} zühten sprach:\\ 
25 & "vrouwe, sol ich mit iu genesen,\\ 
 & sô lât mich âne huote wesen.\\ 
 & wan verlæt mich immer jâmers kraft,\\ 
 & sô tæt ich gerne rîterschaft.\\ 
 & lât ir niht turnieren mich,\\ 
30 & sô kan ich noch den alten slich,\\ 
\end{tabular}
\scriptsize
\line(1,0){75} \newline
D \newline
\line(1,0){75} \newline
\textbf{11} \textit{Initiale} D  \textbf{23} \textit{Majuskel} D  \newline
\line(1,0){75} \newline
\newline
\end{minipage}
\hspace{0.5cm}
\begin{minipage}[t]{0.5\linewidth}
\small
\begin{center}*m
\end{center}
\begin{tabular}{rl}
 & man sprach ein urteil zehant:\\ 
 & "welch ritter \textbf{helm hie} ûf \textbf{bant},\\ 
 & der her \textbf{nâch} ritterschaft \textbf{ist} komen,\\ 
 & \textbf{hât} \textbf{er} den brîs \textbf{al}hie genomen,\\ 
5 & den sol diu küniginne hân."\\ 
 & \textbf{dar nâch diu volge wart} getân.\\ 
 & \textbf{dô sprach si}: "hêrre, nû sît ir mîn.\\ 
 & ich tuon iu dienst nâch hulden schîn\\ 
 & und vüege iu solicher vröude teil,\\ 
10 & daz ir nâch jâmer werdet geil."\\ 
 & er hete iedoch \textbf{noch} jâmerpîn.\\ 
 & \textbf{dô} was des aberellen schîn\\ 
 & \textbf{zergangen}. dar nâch komen was\\ 
 & \dag durch\dag  \textbf{daz} kleine, grüene gras.\\ 
15 & daz velt was \textbf{gar vergrüenet},\\ 
 & daz \textbf{blœdiu} herzen k\textit{üe}net\\ 
 & und in gît hôchgemüete.\\ 
 & vil \textbf{boume} stuont in blüete\\ 
 & von dem süezen luft des meien.\\ 
20 & sîn art von der feien\\ 
 & muose minnen oder minne gern.\\ 
 & des wolte in \dag vriunden\dag  \textbf{d\textit{â}} \textbf{gewern}.\\ 
 & an vrouwen Herczeloiden er \textbf{dô} sach.\\ 
 & sîn \textbf{süezer munt mit} zühten sprach:\\ 
25 & "vrouwe, sol ich mit iu genesen,\\ 
 & sô lât mich âne huote wesen.\\ 
 & wan verlât mich iemer jâmers kraft,\\ 
 & sô tæte ich gerne ritterschaft.\\ 
 & lât ir niht turnieren mich,\\ 
30 & sô kan ich noch den \textit{alten} slich,\\ 
\end{tabular}
\scriptsize
\line(1,0){75} \newline
m n o \newline
\line(1,0){75} \newline
\newline
\line(1,0){75} \newline
\textbf{2} welch] [Weh]: Welh m  $\cdot$ bant] gebant n o \textbf{8} dienst] dienste o  $\cdot$ hulden] hulde n (o) \textbf{9} vröude] freiden n o \textbf{11} hete] hat n \textbf{12} dô was des] Das was das o \textbf{16} herzen] hertze n (o)  $\cdot$ küenet] kroͯnnet m \textbf{18} stuont] stont n (o) \textbf{19} dem] den o \textbf{20} der] dem n \textbf{21} muose] Muͯsz n (o) \textbf{22} vriunden] freiden n (o)  $\cdot$ dâ] do m n \textit{om.} o  $\cdot$ gewern] genern o \textbf{23} vrouwen] froͯuwe m (n) (o)  $\cdot$ Herczeloiden] hertzeleide n herczeleiden o \textbf{30} alten] \textit{om.} m \newline
\end{minipage}
\end{table}
\newpage
\begin{table}[ht]
\begin{minipage}[t]{0.5\linewidth}
\small
\begin{center}*G
\end{center}
\begin{tabular}{rl}
 & man sprach ein urteil \textit{\textbf{dâ}} zehant:\\ 
 & "swelch rîter \textbf{helm hie} ûf \textbf{gebant},\\ 
 & der her \textbf{durch} rîterschaft \textbf{was} komen,\\ 
 & \textbf{hât} \textbf{der} den brîs hie genomen,\\ 
5 & den sol diu küniginne hân."\\ 
 & \textbf{des wart ein urteil} getân.\\ 
 & \textbf{si sprach}: "hêrre, nû sît ir mîn.\\ 
 & ich tuon iu dienst nâch hulden schîn\\ 
 & unde vüege iu solher vröuden teil,\\ 
10 & daz ir nâch jâmere werdet geil."\\ 
 & er het iedoch \textbf{von} jâmer pîn.\\ 
 & \textbf{nû} was des aberellen schîn\\ 
 & \textbf{zergangen}. dar nâch komen was\\ 
 & \textbf{kurz} kleine, grüene gras.\\ 
15 & daz velt was \textbf{gar vergrüenet},\\ 
 & daz \textbf{blœdiu} herze küenet\\ 
 & \multicolumn{1}{l}{ - - - }\\ 
 & \multicolumn{1}{l}{ - - - }\\ 
 & von dem süezen luft des meien.\\ 
20 & sîn art von der feien\\ 
 & muose minnen oder minne gern.\\ 
 & des wolt in vriundîn \textbf{dâ} \textbf{gewern}.\\ 
 & an \textit{vrouwen} \textit{Herzeloide} er sach.\\ 
 & sîn \textbf{süezer munt mit} zühten sprac\textit{h}:\\ 
25 & "\begin{large}V\end{large}rouwe, sul ich mit iu genesen,\\ 
 & sô lât mich âne huote wesen.\\ 
 & \textit{wan} verlât mich imer jâmers kraft,\\ 
 & sô tæte ich gerne rîterschaft.\\ 
 & lât ir niht turnieren mich,\\ 
30 & sô kan ich noch den alten slich,\\ 
\end{tabular}
\scriptsize
\line(1,0){75} \newline
G I O L M Q R Z Fr36 \newline
\line(1,0){75} \newline
\textbf{1} \textit{Initiale} O  \textbf{11} \textit{Initiale} L Q R Z  \textbf{23} \textit{Initiale} I   $\cdot$ \textit{Capitulumzeichen} L  \textbf{25} \textit{Initiale} G  \newline
\line(1,0){75} \newline
\textbf{1} man] ÷an O  $\cdot$ dâ zehant] alzehant G do zu hant Q zehand R \textbf{2} swelch] Welch L (Q) (Z) Welhe R  $\cdot$ helm hie] helm da I hie helm O Q helm R Fr36 helm ie Z  $\cdot$ gebant] haubet bant I bant O M (R) Fr36 haubt gebant Q \textbf{3} der her] Dy her M Der herre Q \textbf{4} hât] Hatte M  $\cdot$ der] er I O (M) Q R Z Fr36 \textbf{5} den] Der L Q  $\cdot$ hân] [gie]: hie han I \textbf{6} Da nach div volge wart (war Q ) getan O (L) (M) (Q) (R) (Z) (Fr36) \textbf{7} si sprach] Da sprach si O (M) Do sprach die kvnigin L Do sprach sie Q (R) (Z)  $\cdot$ nû] so O Q R \textit{om.} M  $\cdot$ sît ir] sit I ir sit M \textbf{8} dienst nâch hulden] dienstes nach hulden R vnd hulde Z \textbf{9} vüege] sucht Q  $\cdot$ iu] \textit{om.} Z  $\cdot$ vröuden] vroide M \textbf{10} jâmere] [vreud]: iamer I \textbf{11} het] hat M  $\cdot$ iedoch] doch L \textbf{12} nû] ez I Do O Q  $\cdot$ aberellen] apriles L \textbf{13} zergangen] Ergangen M [Er]: ZEr gangen Q \textbf{15} daz] Da M  $\cdot$ vergrüenet] gegruͤnt I (L) \textbf{16} blœdiu] blode O (R)  $\cdot$ herze] hertzen L (R)  $\cdot$ küenet] [choͮzet]: choͮnet G \sout{mvd*} chvͦnet O kvlet L erkuͯnet R \textbf{17} \textit{Die Verse 96.17-18 fehlen} G I   $\cdot$ Vnde in geit hoh gemvͦte O (L) (M) (Q) (Z)  $\cdot$ Vnd git Jn hochgemuͯtte R \textbf{18} Vil bavme stvͦnt (stvnden L [ M R ] stent Q ) in (mit L ) blvͦte O (L) (M) (Q) (R) (Z) \textbf{19} dem] des O der M den Q  $\cdot$ süezen] suͤzzem I  $\cdot$ luft] lust R  $\cdot$ meien] [mengin]: meigin M \textbf{21} muose] muͤs I  $\cdot$ minnen] mynne M (R)  $\cdot$ oder] vnd L  $\cdot$ minne] minnen I O Z  $\cdot$ gern] ger M \textbf{22} vriundîn] sin friundinne I (Z) die frewdin Q froͯden R  $\cdot$ dâ] \textit{om.} Q Z \textbf{23} vrouwen] die chungin G vrow L (R)  $\cdot$ Herzeloide] \textit{om.} G herzenlauden I herzenlavden O Hertzeleuͯden L herzeloidin M herzeloúde Q herczenlauden R herzelovden Z  $\cdot$ sach] ersach O do sach L R da sach M Z \textbf{24} sîn] ein I Siner R  $\cdot$ sprach] sprac G \textbf{25} mit iu] euch nicht Q \textbf{26} huote] heute Q \textbf{27} wan] \textit{om.} G  $\cdot$ imer] myner M meines Q  $\cdot$ jâmers] [iamer]: iamers Q \textbf{28} rîterschaft] [riteschaft]: riterschaft G \textbf{30} noch] wol L  $\cdot$ alten] Rechten R  $\cdot$ slich] sic I \newline
\end{minipage}
\hspace{0.5cm}
\begin{minipage}[t]{0.5\linewidth}
\small
\begin{center}*T (U)
\end{center}
\begin{tabular}{rl}
 & man sprach ein urteil \textbf{dô} zehant:\\ 
 & "welch ritter \textbf{hie helm} ûf \textbf{bant},\\ 
 & der her \textbf{durch} ritterschaft \textbf{was} komen,\\ 
 & \textbf{hete} \textbf{der} den prîs \textbf{al} hie genomen,\\ 
5 & den sol diu küniginne hân."\\ 
 & \textbf{dar nâch diu volge wart} getân.\\ 
 & \textbf{\begin{large}D\end{large}ô sprach si}: "hêrre, nû sît ir mîn.\\ 
 & ich tuon iu dienst nâch hulden schîn\\ 
 & und vüege \textit{iu} sol\textit{h}er vreuden teil,\\ 
10 & daz ir nâch jâmer werdet geil."\\ 
 & er hete iedoch \textbf{von} jâmer pîn.\\ 
 & \textbf{nû} was des aberellen schîn\\ 
 & \textbf{ergangen}. dar nâch komen was\\ 
 & \textbf{kurz} kleine, grüene gras.\\ 
15 & daz velt was \textbf{ergrüenet},\\ 
 & daz \textbf{brœdiu} herze küenet\\ 
 & und in gît hôchgemüete.\\ 
 & vil \textbf{bluomen} stuont in blüete\\ 
 & von dem süeze\textit{n} lufte des meien.\\ 
20 & sîn art von der feien\\ 
 & muose minnen oder minnen gern.\\ 
 & des wolt in \textbf{diu} vriundinn\textit{e} \textbf{wern}.\\ 
 & an vrouwen Herzeloyd er \textbf{dô} sach.\\ 
 & sîn \textbf{munt mit süezen} zühten sprach:\\ 
25 & "vrouw\textit{e}, sol ich mit iu genesen,\\ 
 & sô lât mich âne huote wesen.\\ 
 & wan verlât mich imer jâmers kraft,\\ 
 & sô tæte ich gerne ritterschaft.\\ 
 & lâzet ir niht turnieren mich,\\ 
30 & sô kan ich noch den alten slic\textit{h},\\ 
\end{tabular}
\scriptsize
\line(1,0){75} \newline
U V W T \newline
\line(1,0){75} \newline
\textbf{7} \textit{Initiale} U V   $\cdot$ \textit{Majuskel} T  \textbf{11} \textit{Initiale} W   $\cdot$ \textit{Majuskel} T  \textbf{12} \textit{Majuskel} T  \textbf{20} \textit{Majuskel} T  \textbf{23} \textit{Initiale} T  \newline
\line(1,0){75} \newline
\textbf{1} man] vnd T  $\cdot$ dô] sa V (W) o\textit{m. } T \textbf{2} Das ward in allen erkand W  $\cdot$ welch] swelich V (T)  $\cdot$ ûf] uf hoͮbet V \textbf{3} Dy vmb ritterschaft dar warn komen W  $\cdot$ durch ritterschaft] zetvrnei T \textbf{4} hete der] Hat er W (T)  $\cdot$ al] \textit{om.} W T \textbf{8} nâch] an V  $\cdot$ hulden] schvlden T \textbf{9} iu solher vreuden] sol er vreiden U îv vrovde solhen T \textbf{11} er hete iedoch] Jr hetent doch T  $\cdot$ von] nach W \textbf{12} nû] Es W Do T  $\cdot$ des] auch W \textbf{13} ergangen] [*]: Zergangen V zergangen T  $\cdot$ dar nâch] wen doch W \textbf{14} kurz] Maye kurtz W \textbf{15} was] was gar T \textbf{16} brœdiu] bernde W bloͤde T  $\cdot$ küenet] erkuͤnet V W \textbf{17} in gît] gibt im W \textbf{18} bluomen] bôvme T  $\cdot$ stuont] stan W \textbf{19} süezen] suͦze U \textit{om.} W \textbf{20} der] den W \textbf{21} muose] mveser T  $\cdot$ minnen gern] minne gern V W T \textbf{22} diu vriundinne] die vruͦndinnen U da frúndinne V sein frúndinne W vrivndin da T  $\cdot$ wern] gewern T \textbf{23} vrouwen] vroͮ V  $\cdot$ Herzeloyd] [herzele*]: herzeleit U Herzelauden V herzoloyden W herzeloyden T \textbf{24} munt mit süezen] mund mit schoͤnen W svezer mvnt mit T \textbf{25} vrouwe] Vreuͦwen U  $\cdot$ sol ich] súllen wir W \textbf{27} imer] \textit{om.} W \textbf{28} tæte] tâtet T \textbf{29} lâzet] Lassen V \textbf{30} slich] slichin U strich W \newline
\end{minipage}
\end{table}
\end{document}
