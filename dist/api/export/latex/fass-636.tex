\documentclass[8pt,a4paper,notitlepage]{article}
\usepackage{fullpage}
\usepackage{ulem}
\usepackage{xltxtra}
\usepackage{datetime}
\renewcommand{\dateseparator}{.}
\dmyyyydate
\usepackage{fancyhdr}
\usepackage{ifthen}
\pagestyle{fancy}
\fancyhf{}
\renewcommand{\headrulewidth}{0pt}
\fancyfoot[L]{\ifthenelse{\value{page}=1}{\today, \currenttime{} Uhr}{}}
\begin{document}
\begin{table}[ht]
\begin{minipage}[t]{0.5\linewidth}
\small
\begin{center}*D
\end{center}
\begin{tabular}{rl}
\textbf{636} & \textit{\begin{large}G\end{large}}awan hôrte an dem vröuwelîn,\\ 
 & daz si bî minne \textbf{wolde} sîn.\\ 
 & dar zuo was ouch niht ze laz\\ 
 & gein der herzoginne ir haz.\\ 
5 & sus truoc si minne und haz.\\ 
 & ouch het er sich gesundet baz\\ 
 & gein der \textbf{einvaltigen} magt,\\ 
 & diu im ir kumber hât geklagt,\\ 
 & wander ir niht zuo gewuoc,\\ 
10 & daz \textbf{in} unt si ein muoter truoc.\\ 
 & ouch was ir bêder vater Lot.\\ 
 & der meide er sîne helfe bôt,\\ 
 & dâ engein si \textbf{tougenlîchen} neic,\\ 
 & daz er si \textbf{trœsten} niht versweic.\\ 
15 & \textbf{Nû} was \textbf{ouch} zît, daz man dar truoc\\ 
 & \textbf{tischlachen manegez} wîz genuoc\\ 
 & unt\textbf{z brôt} ûf den palas,\\ 
 & dâ manec clâriu vrouwe was.\\ 
 & daz het ein underscheit erkant,\\ 
20 & daz die rîter eine want\\ 
 & heten sunder dort hin dan.\\ 
 & \textbf{den sedel} schuof hêr Gawan.\\ 
 & der Turkote \textbf{zuo z}im \textbf{saz},\\ 
 & Lischoys \textbf{mit} Gawans muoter \textbf{az},\\ 
25 & der clâren \textbf{Sangiven}.\\ 
 & Mit der küneginne Arniven\\ 
 & az diu herzoginne clâr.\\ 
 & sîne swester, \textbf{bêde} wol gevar,\\ 
 & Gawan zuo \textbf{z}im sitzen liez.\\ 
30 & iewederiu tet, als er si hiez.\\ 
\end{tabular}
\scriptsize
\line(1,0){75} \newline
D Z Fr1 \newline
\line(1,0){75} \newline
\textbf{1} \textit{Initiale} D Z  \textbf{15} \textit{Majuskel} D  \textbf{26} \textit{Majuskel} D  \newline
\line(1,0){75} \newline
\textbf{1} Gawan] ÷awan D \textbf{2} minne] minnen Z \textbf{7} einvaltigen] einvalten Z \textbf{13} tougenlîchen] [*]: tovgenlichen Z \textbf{14} trœsten] trostens Z \textbf{22} den] Daz Z \textbf{23} Turkote] Tvrkoite Z Tvrkoẏte Fr1  $\cdot$ zim] im Z \textbf{24} Lischoys] Liscoys D Lishois Z Lẏscois Fr1 \textbf{25} der] Mit der Z  $\cdot$ Sangiven] Sangîven D Fr1 Seyven Z \textbf{26} Arniven] Arnîven D Fr1 \textbf{28} swester bêde] bede Swester Z \textbf{29} zim] im Z \newline
\end{minipage}
\hspace{0.5cm}
\begin{minipage}[t]{0.5\linewidth}
\small
\begin{center}*m
\end{center}
\begin{tabular}{rl}
 & Gawa\textit{n} \textit{h}ôrte an dem vröuwelîn,\\ 
 & daz si bî minne \textbf{solte} sîn.\\ 
 & dar zuo was ouch niht zuo laz\\ 
 & gegen der herzogîn ir haz.\\ 
5 & sus truoc si minne und haz.\\ 
 & ouch het er sich gesundet baz\\ 
 & gegen der \textbf{einvalten} maget,\\ 
 & diu im ir kumber het geklaget,\\ 
 & wan er ir niht zuo gewuoc,\\ 
10 & daz \textbf{si} und  ein muoter truoc.\\ 
 & ouch was ir beider vater Lot.\\ 
 & der megde er sîn helfe bôt,\\ 
 & dâ gegen si \textbf{tougenlîchen} neic,\\ 
 & daz er si \textbf{trœsten} niht versweic.\\ 
15 & \textbf{nû} was \textbf{doch} zît, daz man dar truoc\\ 
 & \textbf{manic tischlachen} wîz genuoc\\ 
 & und \textbf{daz brôt} ûf de\textit{n} palas,\\ 
 & \dag daz\dag  manic clâriu vrowe was.\\ 
 & daz het ein underscheit erkant,\\ 
20 & daz die ritter eine want\\ 
 & heten sunder dort hin dan.\\ 
 & \textbf{den se\textit{d}el} schuof hêr Gawan.\\ 
 & der Turk\textit{o}ite \textbf{mit} im \textbf{az},\\ 
 & Lischois \textbf{bî} Gawans muoter \textbf{saz},\\ 
25 & der clâren \textbf{\textit{S}ang\textit{iv}en}.\\ 
 & mit der künigîn Ar\textit{niv}en\\ 
 & az diu herzogîn clâr.\\ 
 & sîn swester, \textbf{beide} wol gevar,\\ 
 & Gawan zuo im sitzen liez.\\ 
30 & ietwederiu tet, als er si hiez.\\ 
\end{tabular}
\scriptsize
\line(1,0){75} \newline
m n o \newline
\line(1,0){75} \newline
\newline
\line(1,0){75} \newline
\textbf{1} Gawan hôrte] Gawan fuͯrtte vnd horte m \textbf{2} solte] wolte n o \textbf{4} \textit{Vers 636.4 fehlt} n  \textbf{5} sus] Sú n \textbf{6} \textit{Vers 636.6 fehlt} n  \textbf{7} einvalten] einfaltigen o \textbf{8} ir] ẏe n \textbf{9} zuo] \textit{om.} n \textbf{11} was] \textit{om.} o \textbf{13} tougenlîchen] túgenlichen n \textbf{17} brôt] brocht o  $\cdot$ den] dem m \textbf{22} sedel] segel m \textbf{23} Turkoite] turkeitte m turkeite n turckeite o \textbf{24} Lischois] Liscois m o Liscoẏs n  $\cdot$ Gawans] gawanes o \textbf{25} Sangiven] wangwen m sagiwen n saniven o \textbf{26} künigîn] konigen o  $\cdot$ Arniven] Arunen m arniwen n arnwen o \textbf{27} az] Also n \newline
\end{minipage}
\end{table}
\newpage
\begin{table}[ht]
\begin{minipage}[t]{0.5\linewidth}
\small
\begin{center}*G
\end{center}
\begin{tabular}{rl}
 & \begin{large}G\end{large}awan hôrte an dem vröuwelîn,\\ 
 & daz si bî minnen \textbf{wolde} sîn.\\ 
 & dar zuo was ouch niht ze laz\\ 
 & gein der herzogîn ir haz.\\ 
5 & sus truoc si minne unde haz.\\ 
 & ouch het er sich gesundet baz\\ 
 & gein der \textbf{einvaltigen} maget,\\ 
 & diu im ir \textit{kumber} hât geklaget,\\ 
 & wan er ir niht zuo gewuoc,\\ 
10 & daz \textbf{in} unde si ein muoter truoc.\\ 
 & ouch was ir beider vater Lot.\\ 
 & der meide er sîn helfe bôt,\\ 
 & dâ engegen si \textbf{im} \textbf{tougenlîchen} neic,\\ 
 & daz er si \textbf{trôstes} niht versweic.\\ 
15 & \textbf{nû} was \textit{zît}, daz man dar truoc\\ 
 & \textbf{tischlachen manigez} wîz genuoc\\ 
 & unde \textbf{enbôt} ûf den palas,\\ 
 & dâ manic clâriu vrouwe was.\\ 
 & daz het ein underscheit erkant,\\ 
20 & daz die rîter eine want\\ 
 & heten sunder dort hin dan.\\ 
 & \textbf{daz sedel} schuof \textbf{mîn} hêrre Gawan.\\ 
 & der Turkoite \textbf{zuo} im \textbf{saz},\\ 
 & Lishois \textbf{mit} Gawans muoter \textbf{az},\\ 
25 & \textit{\textbf{mit} der} clâren \textbf{Sagiven}.\\ 
 & mit de\textit{r} \textit{k}ünegîn Arniven\\ 
 & az diu herzoginne clâr.\\ 
 & sîne swester wolgevar\\ 
 & Gawan zuo \textbf{z}im sitzen liez.\\ 
30 & ietwederiu tet, als er si hiez.\\ 
\end{tabular}
\scriptsize
\line(1,0){75} \newline
G I L M Z Fr18 Fr51 \newline
\line(1,0){75} \newline
\textbf{1} \textit{Initiale} G I L M Z Fr18 Fr51  \textbf{23} \textit{Initiale} I  \newline
\line(1,0){75} \newline
\textbf{1} hôrte] hort I Fr18 \textbf{2} minnen] mýnne L (M) (Fr18) (Fr51) \textbf{3} ze] so Fr51 \textbf{4} gein der] Ander Fr51 \textbf{6} het] hat M  $\cdot$ er] \textit{om.} I se Fr51  $\cdot$ gesundet] gesundert I gesvndiget Fr51 \textbf{7} gein der] Ander Fr51  $\cdot$ einvaltigen] einvalten L Z \textbf{8} kumber] \textit{om.} G \textbf{11} Lot] tot L \textbf{12} sîn helfe] sinen denest Fr51 \textbf{13} engegen] [vniegen]: vntiegen Fr51  $\cdot$ im] \textit{om.} L M Fr18 Fr51  $\cdot$ tougenlîchen] togentlichen L (Fr51) [*]: tovgenlichen  Z \textbf{14} si] \textit{om.} L  $\cdot$ trôstes] trostens Z  $\cdot$ versweic] besweich Fr51 \textbf{15} zît] \textit{om.} G ouch zit L (M) Z Fr18 Fr51 \textbf{16} Tislachen vnd genvͯch L  $\cdot$ Manich tislachen wis genoch Fr51 \textbf{17} unde enbôt] man gebot I Vnd ein bot L Vnd daz brot Z  $\cdot$ den] dem I \textbf{19} het ein] si heten I \textbf{20} eine] heten eine I \textbf{21} heten] \textit{om.} I \textbf{22} sedel] \textit{om.} I sedes M sitzen Fr51  $\cdot$ mîn] \textit{om.} L M Z Fr18 Fr51 \textbf{23} Turkoite] tvrzot G Turkoys I Tuͯrkoit L Tvrkoẏte Fr18 :::eite Fr51  $\cdot$ im] zim I \textbf{24} Lishois] Liscoys I Lýtschoýs L Lysois M Lẏshoẏs Fr18 :::s Fr51  $\cdot$ mit] [min]: mit G  $\cdot$ Gawans] gawanes Fr51 \textbf{25} mit der] der mit G (M) Er mit Fr18  $\cdot$ clâren] clar Fr51  $\cdot$ Sagiven] sâifen I Segiven L \textit{om.} M Seyven Z Saẏnen Fr18 sangive: Fr51 \textbf{26} Mit der clarin chunegin arniven G  $\cdot$ az diu kuniginne arniue I  $\cdot$ Arniven] ARnẏuen Fr18 \textbf{27} az] \textit{om.} I [As]: Alz L  $\cdot$ clâr] chlare I \textbf{28} sîne swester] mit siner swester I Sine swester bede L (M) (Fr18) Sine bede Swester Z \textbf{29} zim] im L Z yn M \newline
\end{minipage}
\hspace{0.5cm}
\begin{minipage}[t]{0.5\linewidth}
\small
\begin{center}*T
\end{center}
\begin{tabular}{rl}
 & Gawan hôrte an dem vröulîn,\\ 
 & daz si bî minne \textbf{wolte} sîn.\\ 
 & dar zuo was ouch niht zuo laz\\ 
 & gein der herzoginne ir haz.\\ 
 & \hspace*{-.7em}\big| ouch het er sich gesundet baz\\ 
5 & \hspace*{-.7em}\big| - sus truoc si minne und haz -\\ 
 & gein der \textbf{einvalten} maget,\\ 
 & diu im ir kumber hât geklaget,\\ 
 & wan er \textit{ir} niht \textbf{dar} zuo gewuoc,\\ 
10 & daz \textbf{in} und si ein muoter truoc.\\ 
 & ouch \textit{was} ir beider vater Lot.\\ 
 & der megde er sîne helfe bôt,\\ 
 & dâ engein si \textbf{tugentlîche} neic,\\ 
 & daz er si \textbf{trôstes} niht versweic.\\ 
15 & \textbf{\begin{large}D\end{large}ô} was \textbf{ouch} zît, daz man dar truoc\\ 
 & \textbf{tischlachen manegez} wîz genuoc\\ 
 & und \textbf{enbôt} ûf den palas,\\ 
 & d\textit{â} manegiu clâriu vrouwe was.\\ 
 & daz hete ein underscheit erkant,\\ 
20 & daz die rîter eine want\\ 
 & heten sunder dort hin dan.\\ 
 & \textbf{daz gesitze} schuof hêr Gawan.\\ 
 & der Turkoyte \textbf{zuo} im \textbf{saz},\\ 
 & Lyschoys \textbf{mit} Gawanes muoter \textbf{az},\\ 
25 & \textbf{mit} der clâren \textbf{Seyven}.\\ 
 & mit der küneginne Arnyven\\ 
 & az diu herzoginne clâr.\\ 
 & sîne swester, \textbf{beide} wol gevar,\\ 
 & Gawan zuo im sitzen liez.\\ 
30 & ietweder \textbf{sîte} tet, als er si hiez.\\ 
\end{tabular}
\scriptsize
\line(1,0){75} \newline
U V W Q R Fr40 \newline
\line(1,0){75} \newline
\textbf{1} \textit{Initiale} W Q R Fr40  \textbf{15} \textit{Initiale} U V  \newline
\line(1,0){75} \newline
\textbf{1} Gawan] Gawann Q Gawin R  $\cdot$ hôrte] [hor]: hort Q hort R Fr40  $\cdot$ dem] den R \textbf{2} minne] minnen V \textbf{6} \textit{Versfolge 636.5-6} W Q R Fr40   $\cdot$ sus] Als Q \textbf{7} einvalten] einvaltigen V (W) (Q) (R) (Fr40) \textbf{8} hât] het Q \textbf{9} ir] \textit{om.} U  $\cdot$ dar] \textit{om.} V W Q R Fr40 \textbf{10} si] [*]: sv́ V \textbf{11} was] \textit{om.} U \textbf{13} engein] gegen W  $\cdot$ si tugentlîche] sim toͮgenliche V (Q) (R) (Fr40) \textbf{14} si] sins R  $\cdot$ trôstes] troͤstens W \textbf{15} Dô] Nv V (W) (Q) (R) (Fr40)  $\cdot$ ouch] do V \textbf{17} [*]: vnde enbôt vf den palaz V  $\cdot$ den] dem Q Fr40 \textbf{18} dâ] Do U V W Q  $\cdot$ clâriu] clare R \textbf{19} daz] Do V Des W \textbf{22} gesitze] sedel W R Fr40 schadel Q  $\cdot$ hêr] \textit{om.} W \textbf{23} Turkoyte] Turcoite U (W) [kurkoite]: turkoite  Q turkoite R  $\cdot$ zuo im] zvͦzim V \textbf{24} Lyschoys] Lyschois U R Lischois V Lyshoys W Lischoisz Q ::shois Fr40  $\cdot$ Gawanes] gawans V W Q Fr40 Gawins R \textbf{25} \textit{Verse 636.25-26 kontrahiert zu:} Mit der klaren arnyuen Q   $\cdot$ Seyven] Seyuͦen U seiuen V seyuen W (R) \textbf{26} [*v́nigin]: vnd mit der kv́nigin arniuen V  $\cdot$ küneginne] kunginen R  $\cdot$ Arnyven] arneyuen W Arnyuen R [Arni*]: Arniven Fr40 \textbf{28} swester] schwestren R  $\cdot$ beide] \textit{om.} Q beidu R \textbf{29} Gawan] Gawin R :awan Fr40  $\cdot$ zuo im] zvͦz im do V zvzim Fr40  $\cdot$ liez] hies R \textbf{30} ietweder sîte] Jewederv́ V (W) (Q) (Fr40) ettwedre R \newline
\end{minipage}
\end{table}
\end{document}
