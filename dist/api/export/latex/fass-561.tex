\documentclass[8pt,a4paper,notitlepage]{article}
\usepackage{fullpage}
\usepackage{ulem}
\usepackage{xltxtra}
\usepackage{datetime}
\renewcommand{\dateseparator}{.}
\dmyyyydate
\usepackage{fancyhdr}
\usepackage{ifthen}
\pagestyle{fancy}
\fancyhf{}
\renewcommand{\headrulewidth}{0pt}
\fancyfoot[L]{\ifthenelse{\value{page}=1}{\today, \currenttime{} Uhr}{}}
\begin{document}
\begin{table}[ht]
\begin{minipage}[t]{0.5\linewidth}
\small
\begin{center}*D
\end{center}
\begin{tabular}{rl}
\textbf{561} & wande ich strîte selten -\\ 
 & wes möht \textbf{er} danne engelten?\\ 
 & \textbf{\begin{large}H\end{large}êrre}, \textbf{swenn} ir ûf \textbf{hin} kumt,\\ 
 & ein dinc \textbf{iu} zem orse vrumt:\\ 
5 & ein krâmære sitzet vor dem tor,\\ 
 & dem lât daz ors \textbf{hie} vor.\\ 
 & koufet umb in, \textbf{en}\textbf{ruochet} waz;\\ 
 & er behaltet iuz ors deste baz,\\ 
 & ob irz im versetzet.\\ 
10 & werdet ir niht geletzet,\\ 
 & \textbf{ir mugt} daz ors gerne hân."\\ 
 & Dô sprach mîn hêr Gawan:\\ 
 & "sol ich niht zorse \textbf{rîten} în?"\\ 
 & "\textbf{nein}, hêrre, alder vrouwen schîn\\ 
15 & ist \textbf{vor iu} verborgen;\\ 
 & sô næhet ez den sorgen.\\ 
 & den palas vindet ir eine;\\ 
 & weder grôz noch kleine\\ 
 & vindet ir niht, daz dâ lebe.\\ 
20 & sô \textbf{walde}s diu gotes \textbf{gebe},\\ 
 & \textbf{Sô} ir in die kemenâten gêt,\\ 
 & dâ Lit Marvale stêt.\\ 
 & daz bette unt die stollen sîn,\\ 
 & von Marroch \textbf{der} mahmumelîn,\\ 
25 & des \textbf{krône} unt \textbf{al} sîn rîcheit,\\ 
 & \textbf{wære} daz dar gegen \textbf{geleit},\\ 
 & dâ mit ez wære vergolten niht.\\ 
 & dâr an ze lîden iu geschiht,\\ 
 & swaz got an iu wil meinen;\\ 
30 & nâch vreude er\textbf{z} müeze erscheinen.\\ 
\end{tabular}
\scriptsize
\line(1,0){75} \newline
D \newline
\line(1,0){75} \newline
\textbf{3} \textit{Initiale} D  \textbf{12} \textit{Majuskel} D  \textbf{21} \textit{Majuskel} D  \newline
\line(1,0){75} \newline
\newline
\end{minipage}
\hspace{0.5cm}
\begin{minipage}[t]{0.5\linewidth}
\small
\begin{center}*m
\end{center}
\begin{tabular}{rl}
 & wen i\textit{ch} strîte selten.\\ 
 & wes m\textit{ö}h\textit{t} \textbf{ir} dan engelten,\\ 
 & \textbf{hêrre}, \textbf{wen} ir ûf \textbf{în} komet?\\ 
 & ein dinc zuo dem ros vromet:\\ 
5 & ein krâmer sitzet vor dem tor,\\ 
 & dem lât daz ros \textbf{hie} vor.\\ 
 & kouft umb in, \textbf{ir} \textbf{ruochet} waz;\\ 
 & er behaltet iu daz ros deste baz,\\ 
 & ob irz im versetzet.\\ 
10 & werdet ir niht geletzet,\\ 
 & \textbf{ir mogt} daz ros gern hân."\\ 
 & dô sprach mîn hêr Gawan:\\ 
 & "sol ich niht zuo ros \textbf{rîten} în?"\\ 
 & "\textbf{mîn} hêrre, alder vrouwen schîn\\ 
15 & ist \textbf{vor iu} verborgen;\\ 
 & sô nâhet ez den sorgen.\\ 
 & den palas vindet ir \textit{e}ine;\\ 
 & weder grôz noch kleine\\ 
 & vindet ir niht, daz d\textit{â} lebe.\\ 
20 & sô \textbf{walt} es diu gotes \textbf{gebe},\\ 
 & \textbf{sô} ir \textit{in} die kemenâten gât,\\ 
 & d\textit{â} L\textit{e}t Mar\textit{vei}le \textbf{dô} stât.\\ 
 & daz bette und die stollen sîn,\\ 
 & von Ma\textit{r}roch \textbf{der} mahmurmelîn,\\ 
25 & des \textbf{krône} und \textbf{alle} sîn rîcheit,\\ 
 & \textbf{wer} daz dar gegen \textbf{leit},\\ 
 & dâ mit e\textit{z} \textit{w}ær vergolten niht.\\ 
 & dâr an zuo lîden iu geschiht,\\ 
 & waz got an iu wil me\textit{i}nen;\\ 
30 & nâch vröude er müeze erscheinen.\\ 
\end{tabular}
\scriptsize
\line(1,0){75} \newline
m n o \newline
\line(1,0){75} \newline
\newline
\line(1,0){75} \newline
\textbf{1} wen] [Wem]: Wen o  $\cdot$ ich] ist m  $\cdot$ strîte] striten o \textbf{2} möht] mohtte m (o) \textbf{3} wen] wem o  $\cdot$ ûf în] vff hin n >uff< hin o \textbf{9} versetzet] [v*]: verseczen o \textbf{10} geletzet] geleczent o \textbf{12} hêr] herre her n  $\cdot$ Gawan] [*]: gawan m \textbf{14} mîn] Nein n [M]: Nein o  $\cdot$ vrouwen] [frower]: frowen o \textbf{17} eine] reine m \textbf{18} grôz] grosse n \textbf{19} dâ] do m n o \textbf{20} walt] wolt n \textbf{21} in] \textit{om.} m o  $\cdot$ die] d: o \textbf{22} dâ] Do m So n o  $\cdot$ Let Marveile] lot marnaẏle m lot marnaile n lat marnalie o \textbf{23} stollen] [sollen]: stollen o \textbf{24} Marroch] mahroh m marrach n maroch o  $\cdot$ der] de n  $\cdot$ mahmurmelîn] mahimúrmelin n [mar]: mahẏmirmelin o \textbf{26} daz] des o  $\cdot$ leit] geleit n \textbf{27} ez wær] es niht wer m \textbf{29} meinen] mennen m \newline
\end{minipage}
\end{table}
\newpage
\begin{table}[ht]
\begin{minipage}[t]{0.5\linewidth}
\small
\begin{center}*G
\end{center}
\begin{tabular}{rl}
 & \textit{\begin{large}W\end{large}}ande ich strîte selten -\\ 
 & wes m\textit{ö}ht \textbf{er} danne engelten?\\ 
 & \textbf{hêrre}, \textbf{swenne} ir ûf \textbf{hin} kumet,\\ 
 & ein dinc \textbf{iu} ze dem ors vrumet:\\ 
5 & ein krâmer sitzet vor dem tor,\\ 
 & dem lât daz ors \textbf{hie} vor.\\ 
 & koufet umbe in, \textbf{en}\textbf{ruochet} waz;\\ 
 & er behalt iuz ors deste baz,\\ 
 & ob irz im versetzet.\\ 
10 & werdet ir niht geletzet,\\ 
 & \textbf{ir muget} daz ors gerne hân."\\ 
 & dô sprach mîn hêrre Gawan:\\ 
 & "sol ich niht ze orse \textbf{rîten} în?"\\ 
 & "\textbf{nein}, hêrre, al der vrouwen schîn\\ 
15 & ist \textbf{vor iu} verborgen;\\ 
 & sô nâhet ez den sorgen.\\ 
 & den palas vindet ir eine;\\ 
 & weder grôz noch kleine\\ 
 & vindet ir niht, daz dâ lebe.\\ 
20 & sô \textbf{walde}s diu gotes \textbf{pflege},\\ 
 & \textbf{sô} ir in die kemenâten gêt,\\ 
 & dâ Let Marveile stêt.\\ 
 & daz bette unde die stollen sîn,\\ 
 & von Marroch \textbf{de\textit{r}} mahmumelîn,\\ 
25 & des \textbf{êre} unde \textbf{al} sîn rîcheit,\\ 
 & \textbf{wære} daz dar geine \textbf{geleit},\\ 
 & dâ mit ez wære vergolten niht.\\ 
 & dâr an ze lîden iu geschiht,\\ 
 & swaz got an iu wil meinen;\\ 
30 & nâch vröude er\textbf{z} müeze erscheinen.\\ 
\end{tabular}
\scriptsize
\line(1,0){75} \newline
G I L M Z \newline
\line(1,0){75} \newline
\textbf{1} \textit{Initiale} G L  \textbf{3} \textit{Initiale} I Z  \textbf{21} \textit{Initiale} I M  \newline
\line(1,0){75} \newline
\textbf{1} Wande] Nande G \textbf{2} möht er] moht er G L Z moht ich I mochtir M  $\cdot$ engelten] erkelten M \textbf{3} \textit{Versfolge 561.4-3} M   $\cdot$ swenne] wenne L  $\cdot$ ûf hin] hin uff M \textbf{4} iu] \textit{om.} M \textbf{7} enruochet] ir enruͤcht I (M) \textbf{8} iuz ors] ez uͯch L das ros M \textbf{11} daz] ditz I Z \textbf{12} dô] Da M  $\cdot$ hêrre Gawan] ergawan M \textbf{13} în] \textit{om.} Z \textbf{14} der vrouwen] myn M \textbf{16} sô] Do L  $\cdot$ nâhet] nahent I \textbf{20} waldes] wolde isz M  $\cdot$ pflege] gebe L M Z \textbf{21} sô ir] Wer L \textbf{22} Let Marveile] leit marueile I lettemarveile L Lit Marvale Z \textbf{23} daz] Da M \textbf{24} der] de G \textit{om.} I  $\cdot$ mahmumelîn] [mahmuͤlin]: mahnuͤlin I Myramuͯndelin M \textbf{25} êre] krone L (M) Z \textbf{26} geine] an L (M) \textbf{29} swaz] Waz L (M)  $\cdot$ meinen] meyne M \textbf{30} vröude] vreuden I (M) (Z)  $\cdot$ erz müeze] muͤz erz ev I er muͯsz L  $\cdot$ erscheinen] sheinen I erschine M \newline
\end{minipage}
\hspace{0.5cm}
\begin{minipage}[t]{0.5\linewidth}
\small
\begin{center}*T
\end{center}
\begin{tabular}{rl}
 & wand ich strîte selten -\\ 
 & wes m\textit{ö}ht \textbf{er} danne engelten?\\ 
 & \textbf{sô} ir ûf \textbf{die burc hin} komet,\\ 
 & ein dinc \textbf{iu} zuo dem orse vromet:\\ 
5 & \textit{\begin{large}E\end{large}}in krâmer sitzet vorme tor,\\ 
 & dem lât daz ors \textbf{dâr} vor.\\ 
 & koufet umb in, \textbf{iu} \textbf{en}\textbf{ruoche} waz;\\ 
 & er behalt iu daz ors deste baz,\\ 
 & ob irz im versetzet.\\ 
10 & werdet ir niht geletzet,\\ 
 & \textbf{sô muget ir}z ors gerne hân."\\ 
 & Dô sprach mîn hêr Gawan:\\ 
 & "sol ich niht ze ors \textbf{hin} în?"\\ 
 & "\textbf{Nein}, hêrre, alder vrouwen schîn\\ 
15 & ist \textbf{iu vor} verborgen;\\ 
 & sô nâhet ez den sorgen.\\ 
 & den palas vindet ir eine;\\ 
 & weder grôz noch kleine\\ 
 & vindet ir niht, daz dâ lebe.\\ 
20 & sô \textbf{walt} es diu gotes \textbf{pflege},\\ 
 & \textbf{als} ir in die kemenâten gêt,\\ 
 & dâ Let Marvele \textbf{inne} stêt.\\ 
 & daz bette unde die stollen sîn,\\ 
 & Von Marroch mahmurmelîn,\\ 
25 & des \textbf{krône} unde \textbf{al}sîn rîcheit,\\ 
 & \textbf{wære} daz dar gegen \textbf{geleit},\\ 
 & dâ mite ez wære vergolten niht.\\ 
 & dâr an ze lîdenne iu geschiht,\\ 
 & swaz got an iu wil meinen;\\ 
30 & nâch vröuden er\textbf{z} müeze erscheinen.\\ 
\end{tabular}
\scriptsize
\line(1,0){75} \newline
T U V W Q R Fr25 Fr39 Fr40 \newline
\line(1,0){75} \newline
\textbf{1} \textit{Initiale} Fr25 Fr40   $\cdot$ \textit{Capitulumzeichen} R  \textbf{3} \textit{Initiale} Q  \textbf{5} \textit{Initiale} T  \textbf{12} \textit{Majuskel} T  \textbf{14} \textit{Majuskel} T  \textbf{24} \textit{Majuskel} T  \newline
\line(1,0){75} \newline
\textbf{1} \textit{Die Verse 553.1-599.30 fehlen} U   $\cdot$ strîte] stritten R (Fr25) \textbf{2} wes] Was W (Fr39)  $\cdot$ möht er] mohter T (V) (Q) (Fr25) (Fr40) \textbf{3} sô] Wann W \textbf{4} zuo dem] denn zuͦm W (Fr40) do zum Q denne zem R (Fr25) \textbf{5} Ein] ÷in T \textbf{6} daz] \textit{om.} Fr25 ditze Fr40 \textbf{7} koufet] Kavf Fr25  $\cdot$ iu] \textit{om.} V W Q R Fr25 Fr39 Fr40  $\cdot$ enruoche] enrvͦchent V (W) (Q) (R) (Fr25) (Fr39) [ruch]: rucht  Fr40 \textbf{8} er] Der Q (Fr40)  $\cdot$ iu] \textit{om.} Fr25  $\cdot$ daz] \textit{om.} Fr39 \textbf{9} irz im] ir im ez V (W) (Fr25) (Fr39) ir im Q Fr40 \textbf{10} ir] aber ir V Q Fr25 (Fr39) Fr40 aber W ir aber R \textbf{13} sol] So R  $\cdot$ hin] ritten V (W) (Q) (R) (Fr25) (Fr39) (Fr40) \textbf{14} Nein] \textit{om.} Fr25 \textbf{15} iu vor] vor v́ch V (W) (Q) (R) (Fr25) (Fr39) vor ir Fr40 \textbf{19} dâ] do V W Fr39 \textbf{20} diu gotes] gotte R  $\cdot$ pflege] gebe V W Q R (Fr25) Fr39 Fr40 \textbf{21} kemenâten] camer R \textbf{22} dâ Let Marvele inne] [*]: Do lit marveile V Dalet marfeile W Da letmarveile Q (R) Da lit Marveile Fr25 do Let marveile Fr39 da Let [marveile]: marvaile Fr40 \textbf{24} Marroch] maroch T V  $\cdot$ mahmurmelîn] der [*]: mahmvrmelin V der machmúrmelein W der mahmvmelin Q (R) (Fr39) (Fr40) machmelin Fr25 \textbf{25} des] Der Fr25  $\cdot$ alsîn] [aller]: alle sin V des W \textbf{26} gegen geleit] degn geleit Fr25 \textbf{28} iu] ie Fr40 \textbf{29} swaz] Was W Q R  $\cdot$ meinen] nemen Q \textbf{30} erz] er V es Q (Fr40) \newline
\end{minipage}
\end{table}
\end{document}
