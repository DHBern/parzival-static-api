\documentclass[8pt,a4paper,notitlepage]{article}
\usepackage{fullpage}
\usepackage{ulem}
\usepackage{xltxtra}
\usepackage{datetime}
\renewcommand{\dateseparator}{.}
\dmyyyydate
\usepackage{fancyhdr}
\usepackage{ifthen}
\pagestyle{fancy}
\fancyhf{}
\renewcommand{\headrulewidth}{0pt}
\fancyfoot[L]{\ifthenelse{\value{page}=1}{\today, \currenttime{} Uhr}{}}
\begin{document}
\begin{table}[ht]
\begin{minipage}[t]{0.5\linewidth}
\small
\begin{center}*D
\end{center}
\begin{tabular}{rl}
\textbf{157} & des sol \textbf{vil} wênic von mir komen,\\ 
 & ez \textbf{gê} ze schaden oder ze vromen."\\ 
 & daz dûhte wunderlîch genuoc\\ 
 & Iwanet, der was kluoc.\\ 
 & \multicolumn{1}{l}{ - - - }\\ 
 & \multicolumn{1}{l}{ - - - }\\ 
 & \multicolumn{1}{l}{ - - - }\\ 
 & \multicolumn{1}{l}{ - - - }\\ 
5 & \textbf{iedoch} muoser im volgen.\\ 
 & er\textbf{n} was im niht erbolgen.\\ 
 & zwô liehte hosen îserîn\\ 
 & schuoht er \textbf{im} über diu ribbalîn.\\ 
 & sunder leder mit zwein borten\\ 
10 & zwêne sporen dar zuo \textbf{gehôrten}.\\ 
 & \textbf{er spien} im \textbf{an} \textbf{daz} goldes werc.\\ 
 & \textbf{ê er im büte dâr} den halsberc,\\ 
 & \multicolumn{1}{l}{ - - - }\\ 
 & \multicolumn{1}{l}{ - - - }\\ 
 & \textbf{er stricte} im umbe diu schinnelier.\\ 
 & sunder twâle \textbf{vil} harte schier\\ 
15 & von \textbf{vuoz} ûf gewâpent wol\\ 
 & \textbf{wart} Parzival mit gerender dol.\\ 
 & Dô iesch der knappe mære\\ 
 & sînen kochære.\\ 
 & "ich \textbf{en}reiche dir dechein gabylôt.\\ 
20 & diu ritterschaft dir daz verbôt",\\ 
 & sprach Iwanet, der knappe wert.\\ 
 & \textbf{der} gurte im umbe ein \textbf{scharpfez} swert.\\ 
 & daz lêrt ern ûz ziehen\\ 
 & unt widerriet im vliehen.\\ 
25 & dô zôch \textbf{er} im dar nâher sân\\ 
 & des \textbf{tôten mannes} kastelân.\\ 
 & daz truoc bein hôch unt lanc.\\ 
 & \textbf{der gewâpent} in den satel spranc,\\ 
 & er\textbf{n} \textbf{gerte stegreifes} niht,\\ 
30 & \textbf{dem} man noch snelheite giht.\\ 
\end{tabular}
\scriptsize
\line(1,0){75} \newline
D \newline
\line(1,0){75} \newline
\textbf{17} \textit{Majuskel} D  \newline
\line(1,0){75} \newline
\textbf{4} Iwanet] Jwanet D \textbf{21} Iwanet] Jwanet D \newline
\end{minipage}
\hspace{0.5cm}
\begin{minipage}[t]{0.5\linewidth}
\small
\begin{center}*m
\end{center}
\begin{tabular}{rl}
 & \hspace*{-.7em}\big| ez \textbf{gê} ze schaden oder ze vromen,\\ 
 & \hspace*{-.7em}\big| des sol \textbf{vil} wênic vo\textit{n} mir komen."\\ 
 & daz dûhte \textbf{in} wunderlîch genuoc,\\ 
 & Iwaneten, der was kluoc.\\ 
 & \multicolumn{1}{l}{ - - - }\\ 
 & \multicolumn{1}{l}{ - - - }\\ 
 & \multicolumn{1}{l}{ - - - }\\ 
 & \multicolumn{1}{l}{ - - - }\\ 
5 & \textbf{iedoch} \textit{m}uos er ime volgen.\\ 
 & er \textbf{en}was im niht erbolgen.\\ 
 & zwô liehte hosen îserîn\\ 
 & schu\textit{o}het er \textbf{ime} über diu ribbalîn.\\ 
 & sunder leder mit zwein borten\\ 
10 & zwêne sporen dar zuo \textbf{gehôrten}.\\ 
 & \textbf{er spien} ime \textbf{an} \textbf{des} goldes wer\textit{c}.\\ 
 & \textbf{ê er ime dâr b\textit{ü}te} den halsberc,\\ 
 & \multicolumn{1}{l}{ - - - }\\ 
 & \multicolumn{1}{l}{ - - - }\\ 
 & \textbf{er stricket} ime umbe diu schinnelier.\\ 
 & sunder twâle harte schier\\ 
15 & von \textbf{vuoze} ûf gewâpent wol\\ 
 & \textbf{wart} Parcifal mit gernder dol.\\ 
 & dô iesch der knappe mære\\ 
 & sînen kochære.\\ 
 & "i\textbf{n} reiche dir kein gabilôt.\\ 
20 & diu ritterschaft dir daz verbôt",\\ 
 & sprach Iwanet, der knappe wert.\\ 
 & \textbf{er} gurt ime umbe ein \textbf{scharpfez} swert.\\ 
 & daz lêrte er in ûz ziehen\\ 
 & und widerriet ime vliehen.\\ 
25 & dô zôch \textbf{er} im dar nâher sân\\ 
 & des \textbf{tôten mannes} kastelân.\\ 
 & d\textit{a}z truoc bein hôch und lanc.\\ 
 & \textbf{gewâpent er} in den satel spranc,\\ 
 & er \textbf{en}\textbf{gerte stegreifes} niht,\\ 
30 & \textbf{dem} man noch \textit{s}nelheit giht.\\ 
\end{tabular}
\scriptsize
\line(1,0){75} \newline
m n o \newline
\line(1,0){75} \newline
\newline
\line(1,0){75} \newline
\textbf{2} ez] E n o  $\cdot$ ze vromen] fromen n (o) \textbf{1} des] Das n o  $\cdot$ von] vor m \textbf{3} daz] Doch o  $\cdot$ in] \textit{om.} n o \textbf{4} Iwaneten] Jwanetten m Jwaneten n o \textbf{5} muos] [mith]: miuͦs m muͯst n o \textbf{7} liehte] lichte n (o)  $\cdot$ îserîn] [ysen]: yserrin m \textbf{8} schuohet] Schuͦche het m \textbf{11} des] das o  $\cdot$ werc] wert m n o \textbf{12} büte] bitte m n \textbf{13} schinnelier] schiemelier o \textbf{14} twâle] \textit{om.} n \textbf{16} mit] nit o \textbf{17} iesch] hesch n \textbf{19} in reiche] Jch einreche o \textbf{21} Iwanet] jwanet m n o \textbf{24} vliehen] das fliehen o \textbf{25} dar] her n do o \textbf{26} des tôten] Das daten o \textbf{27} daz] des m  $\cdot$ bein] in n o \textbf{29} engerte stegreifes] engert stegereiff n o \textbf{30} man] manne n  $\cdot$ snelheit] slnelheit m manheit n o \newline
\end{minipage}
\end{table}
\newpage
\begin{table}[ht]
\begin{minipage}[t]{0.5\linewidth}
\small
\begin{center}*G
\end{center}
\begin{tabular}{rl}
 & des sol \textbf{vil} wênic von mir komen,\\ 
 & ez \textbf{gê} ze schaden oder ze vrumen."\\ 
 & daz dûhte wunderlîch genuoc\\ 
 & Ywaneten, der was kluoc.\\ 
 & \multicolumn{1}{l}{ - - - }\\ 
 & \multicolumn{1}{l}{ - - - }\\ 
 & \multicolumn{1}{l}{ - - - }\\ 
 & \multicolumn{1}{l}{ - - - }\\ 
5 & \textbf{iedoch} muoser im volgen.\\ 
 & er was im niht erbolgen.\\ 
 & zwuo liehte hosen îserîn\\ 
 & schuohter über diu ribbalîn.\\ 
 & sunder leder mit zwein borten\\ 
10 & zwêne sporen dar zuo \textbf{gehôrten}.\\ 
 & \textbf{er spien} im \textbf{umbe} \textbf{daz} goldes werc.\\ 
 & \textbf{ê er büte im} den halsberc,\\ 
 & \multicolumn{1}{l}{ - - - }\\ 
 & \multicolumn{1}{l}{ - - - }\\ 
 & \textbf{er stricte} im umbe diu tschillier.\\ 
 & sunder twâl \textbf{wart} harte schier\\ 
15 & \begin{large}V\end{large}on \textbf{vuoze} ûf gewâpent wol\\ 
 & Parzival mit gernder dol.\\ 
 & dô iesch der knappe mære\\ 
 & sînen kochære.\\ 
 & "ich reiche dir nehein gabilôt.\\ 
20 & diu ritterschaft dir daz verbôt",\\ 
 & sprach Ywanet, der knappe wert.\\ 
 & \textbf{er} gurte im umbe ein \textbf{scharfez} swert.\\ 
 & daz lêrt ern ûz ziehen\\ 
 & unde widerriet im vliehen.\\ 
25 & dô zôch \textbf{er} im dar nâher sân\\ 
 & des \textbf{tôten mannes} kastelân.\\ 
 & daz truoc bein hôch und lanc.\\ 
 & \textbf{dô er gewâpent} in den satel spranc,\\ 
 & er \textbf{gert stegereife} niht,\\ 
30 & \textbf{dem} man noch snelheite giht.\\ 
\end{tabular}
\scriptsize
\line(1,0){75} \newline
G I O L M Q R Z Fr36 \newline
\line(1,0){75} \newline
\textbf{15} \textit{Initiale} G  \textbf{17} \textit{Initiale} I  \newline
\line(1,0){75} \newline
\textbf{1} \textit{Versfolge 157.2-1} L   $\cdot$ des] Das R  $\cdot$ vil] \textit{om.} O L M Fr36 \textbf{2} ez] Das R  $\cdot$ gê] si I erge O (Fr36) gen R \textbf{3} dûhte] duht in I (R) \textbf{4} Ywaneten] ẏwaneten G Juuanet I Jwaneten O M Jwanet L Ywanen Q Jwan R iwanet Fr36  $\cdot$ der was] der was so O den knappen L wan er was Q der wan R \textbf{6} was] en was M (Q) \textbf{7} liehte] lýchte L (M) (Q)  $\cdot$ hosen îserîn] hosen isenin I (O) (L) (R) isen hosen isner:: Fr36 \textbf{8} schuohter] shuht er I (R) (Fr36) Schvrtzter Z  $\cdot$ über] im vber O L (M) (R) Z in vber Q  $\cdot$ diu ribbalîn] dy rabilin M das Rabilin R \textbf{9} sunder] Vnter Q \textbf{10} sporen] spor I  $\cdot$ dar zuo] druber I die dar zu R  $\cdot$ gehôrten] horten I O Fr36 \textbf{11} umbe] an L \textit{om.} M  $\cdot$ daz] des R  $\cdot$ goldes werc] Goltwerc I (O) (Q) goldes wert R \textbf{12} ê er büte im] er bot im dar I (L) (M) (Q) Vnde bot im dar O E er Im but R E er bvte im dar Z \textbf{13} stricte] stræich O  $\cdot$ umbe] \textit{om.} I an O L Q da dy M  $\cdot$ tschillier] schinckelier L schallier R \textbf{14} wart] \textit{om.} L \textbf{15} von] vor I  $\cdot$ vuoze] vuͤzen I fleisze Q  $\cdot$ ûf] \textit{om.} M \textbf{16} Parzival] Parzifal I R Parcifal O Z Wart Parcifal L Partzifal Q \textbf{17} dô] Da M Doch Z  $\cdot$ iesch] hiesz Q \textbf{19} ich] ichn I (L) (M) (Z) \textbf{20} dir] dy M  $\cdot$ verbôt] Gebot I (O) (R) (Z) \textbf{21} Ywanet] ẏwanet G iuuanet I Jwanet O (L) ywan Q Jwan R \textbf{22} er] Der O Q R Z Dy M  $\cdot$ gurte] Gurt I (O) (Q) (R)  $\cdot$ im] \textit{om.} Q \textbf{23} lêrt ern] lert in I lerte er in L (M) (R) \textbf{24} Vnd wider in die scheid ziechen R \textbf{25} dô] Da M (Z)  $\cdot$ im] in I  $\cdot$ dar] do R  $\cdot$ nâher] nach I R nahet Q [*hen]: san Z \textbf{26} Den Rotten castilan R  $\cdot$ des] Dest O \textbf{28} dô er gewâpent] do er I Gewaffent er O (L) (M) (Q) (R) Der gewappent Z \textbf{29} gert] en gerte M (Q) (Z) \textbf{30} dem] den I  $\cdot$ noch] \textit{om.} Q nach Z \newline
\end{minipage}
\hspace{0.5cm}
\begin{minipage}[t]{0.5\linewidth}
\small
\begin{center}*T (U)
\end{center}
\begin{tabular}{rl}
 & \hspace*{-.7em}\big| ez \textbf{ergê} zuo schaden oder zuo vromen,\\ 
 & \hspace*{-.7em}\big| des sol wênec von mir komen\\ 
 & \multicolumn{1}{l}{ - - - }\\ 
 & \multicolumn{1}{l}{ - - - }\\ 
 & durch iemannes dröuwen oder bete."\\ 
 & der vil stolze Ywanete,\\ 
 & er wunderte sich der rede dô\\ 
 & und wart mit Parcifale vrô.\\ 
5 & \textbf{sus} muos er im volgen.\\ 
 & er was im niht erbolgen.\\ 
 & zwô liehte hosen îserîn\\ 
 & schuoht er über diu ribbalîn.\\ 
 & sunder leder mit zwein borten\\ 
10 & zwêne sporn, \textbf{die} dâ zuo \textbf{hôrten},\\ 
 & \textbf{sp\textit{i}en er} im \textbf{umb} \textbf{di\textit{u}} \textit{g}oldes \textit{werc}.\\ 
 & \textbf{dar nâch bôt er im} den halsberc;\\ 
 & dar in sloufte sich der werde.\\ 
 & dar nâch, als er gerde,\\ 
 & \textbf{striht er} im umb diu schillier.\\ 
 & sunder twâl \textbf{war\textit{t}} harte schier\\ 
15 & von \textbf{vuozen} ûf gewâpent wol\\ 
 & Parcifal mit gernder dol.\\ 
 & dô iesch der knappe mære\\ 
 & \textbf{den} sînen kochære.\\ 
 & "ich \textbf{en}reiche di\textit{r} dekein gabilôt.\\ 
20 & diu rîterschaft dir daz verbôt",\\ 
 & sprach Ywanet, der knappe wert.\\ 
 & \textbf{er} gurt im umb ein swert.\\ 
 & daz lêrte er in ûz ziehen\\ 
 & und widerriet im vliehen.\\ 
25 & dô zôch \textbf{man} im dar nâher sân\\ 
 & des \textbf{rôten rîters} kastelân.\\ 
 & daz truoc bein hôch und lanc.\\ 
 & \textbf{gewâpent er} in den satel spranc,\\ 
 & \textbf{daz} er \textbf{stegereifes gerte} niht,\\ 
30 & \textbf{dâ} man noch snellekeite giht.\\ 
\end{tabular}
\scriptsize
\line(1,0){75} \newline
U V W T \newline
\line(1,0){75} \newline
\textbf{3} \textit{Initiale} T  \textbf{7} \textit{Majuskel} T  \textbf{11} \textit{Majuskel} T  \textbf{25} \textit{Majuskel} \textsuperscript{1}\hspace{-1.3mm} T  \newline
\line(1,0){75} \newline
\textbf{2} \textit{Versfolge 157.1-2} T   $\cdot$ ergê] gê T \textbf{1} wênec] vil wenic T \textbf{3} \textit{Die Verse 157.3-4 fehlen} U W   $\cdot$ Daz dvhte in (\textit{om.} T ) wunderlich (wunderliclich T ) gnvͦg V (T) \textbf{4} ywanete (jweneten T ) der waz klvͦg V (T) \textbf{4} \textit{Die Verse 157.4\textasciicircum1-4\textasciicircum4 fehlen} T   $\cdot$ dröuwen] [dro]: dreuwen U droͮ V (W) \textbf{4} vil] \textit{om.} W  $\cdot$ Ywanete] ẏwanete V ywanette W \textbf{4} wunderte] wundert V \textbf{4} Parcifale] Parzifale U (V) partzifale W \textbf{5} sus] ie doch T \textbf{7} liehte] liehten V  $\cdot$ îserîn] eisenin W \textbf{8} schuoht er] [striht*]: strihter im V schvͦhterm T  $\cdot$ diu] die T \textbf{10} die] \textit{om.} T  $\cdot$ hôrten] gehorten W \textbf{11} spien er] Splen er U Er spien T  $\cdot$ diu goldes werc] die von goldes U [*]: die goldez werg V das goldes werg W (T) \textbf{12} er bôt im dâr den halsperc T  $\cdot$ den] das W \textbf{12} \textit{Die Verse 157.12\textasciicircum1-12\textasciicircum2 fehlen} T  \textbf{12} gerde] begerde V \textbf{13} striht er im] er strictim T  $\cdot$ diu] deie V die T \textbf{14} wart] war U  $\cdot$ harte] [*]: do V \textit{om.} W \textbf{15} vuozen] fuͦße W (T) \textbf{16} Parcifal] parzifal V Partzifal W \textbf{17} iesch] hies V  $\cdot$ mære] vire W \textbf{18} den] \textit{om.} T \textbf{19} dir] die U \textbf{20} dir] [*]: dir V \textit{om.} W  $\cdot$ verbôt] [*]: verbot V \textbf{22} gurt im] gurte im W (T)  $\cdot$ swert] scharpfes swert V (T) \textbf{25} \textit{Versdoppelung 157.25-158.10 (²T) nach 158.10; Fassungstext *T nach ²T mit Lesarten der vorausgehenden Verse (¹T) im Apparat} T   $\cdot$ man] er \textsuperscript{1}\hspace{-1.3mm} T U V W  $\cdot$ im] in W  $\cdot$ nâher] [nahen]: naher V \textbf{26} rôten] toten \textsuperscript{1}\hspace{-1.3mm} T \textbf{27} lanc] land W \textbf{29} ern gerte stegereffes niht \textsuperscript{1}\hspace{-1.3mm} T  $\cdot$ stegereifes] stegereife U \textbf{30} dâ] dem \textsuperscript{1}\hspace{-1.3mm} T (V) Do U W  $\cdot$ noch] oúch W \newline
\end{minipage}
\end{table}
\end{document}
