\documentclass[8pt,a4paper,notitlepage]{article}
\usepackage{fullpage}
\usepackage{ulem}
\usepackage{xltxtra}
\usepackage{datetime}
\renewcommand{\dateseparator}{.}
\dmyyyydate
\usepackage{fancyhdr}
\usepackage{ifthen}
\pagestyle{fancy}
\fancyhf{}
\renewcommand{\headrulewidth}{0pt}
\fancyfoot[L]{\ifthenelse{\value{page}=1}{\today, \currenttime{} Uhr}{}}
\begin{document}
\begin{table}[ht]
\begin{minipage}[t]{0.5\linewidth}
\small
\begin{center}*D
\end{center}
\begin{tabular}{rl}
\textbf{415} & ich hôrt ie sagen, swâ ez sô gezôch,\\ 
 & daz man gein wîbes scherme vlôch,\\ 
 & dâ solde ellenthaftez jagen\\ 
 & an sîme strîte gar verzagen,\\ 
5 & ob dâ wære manlîchiu zuht.\\ 
 & hêr Vergulaht, iwers gastes vluht,\\ 
 & die er gein mir tet vür den tôt,\\ 
 & lêret iwern prîs noch lasters nôt."\\ 
 & Kingrimursel dô sprach:\\ 
10 & "hêrre, ûf iwern trôst geschach,\\ 
 & daz ich hêrn Gawane\\ 
 & ûf \textbf{dem} Plimizœls plâne\\ 
 & gap vride \textbf{her} in iwer lant.\\ 
 & iwer sicherheit \textbf{was} pfant,\\ 
15 & ob in sîn ellen trüege her,\\ 
 & daz \textbf{ich} \textbf{des} vür iuch würde wer,\\ 
 & i\textit{n} bestüende \textbf{hie} \textbf{niht} wan \textbf{einec} man.\\ 
 & hêrre, dâ bin ich \textbf{bekrenket} an.\\ 
 & hie sehen mîne genôze zuo.\\ 
20 & \textbf{diz} laster ist uns gar ze vruo.\\ 
 & kunnet ir niht vürsten schônen,\\ 
 & wir krenken ouch die krônen.\\ 
 & sol man iuch bî zühten sehen,\\ 
 & sô muoz \textbf{des iwer zuht} \textbf{verjehen},\\ 
25 & daz sippe reichet ab iu an mich.\\ 
 & wære daz ein kebslîcher slich\\ 
 & mînhalp, swâ \textbf{uns} diu \textbf{wirt} gezilt,\\ 
 & ir hetet iuch gâhes \textbf{an} mir bevilt.\\ 
 & \begin{large}W\end{large}ande ich bin ein ritter doch,\\ 
30 & an dem nie valsch wart vunden noch.\\ 
\end{tabular}
\scriptsize
\line(1,0){75} \newline
D \newline
\line(1,0){75} \newline
\textbf{29} \textit{Initiale} D  \newline
\line(1,0){75} \newline
\textbf{6} Vergulaht] Vergvlaht D \textbf{12} Plimizœls] Plimizoͤls D \textbf{17} in] ich D \newline
\end{minipage}
\hspace{0.5cm}
\begin{minipage}[t]{0.5\linewidth}
\small
\begin{center}*m
\end{center}
\begin{tabular}{rl}
 & ich hôrte ie sagen, w\textit{â} ez sô gezôch,\\ 
 & daz man gegen wîbes schirme vlôch,\\ 
 & d\textit{â} solte ellenthaftez jagen\\ 
 & an sînem strîte gar verzagen,\\ 
5 & ob dâ wære manlîch\textit{iu} zuht.\\ 
 & hêr Vergulaht, i\textit{wer}s gastes vluht,\\ 
 & die er gegen mir tet vür den tôt,\\ 
 & lêret iuwern prîs n\textit{o}ch lasters nôt."\\ 
 & \textit{\begin{large}K\end{large}}in grimursel dô sprach:\\ 
10 & "hêrre, ûf iuwern trôst gesch\textit{a}ch,\\ 
 & daz ich \textbf{dem} hêrren Gawan\\ 
 & ûf \textbf{dem} Plimizols plân\\ 
 & gap vride \textbf{her} in iuwer lant.\\ 
 & iuwer sicherheit \textbf{gap} pfant,\\ 
15 & ob in sîn ellen trüege her,\\ 
 & daz \textbf{ich} \textbf{es} vür iuch würde wer,\\ 
 & i\textit{n} bestüende \textbf{niht} wanne \textbf{einic} man.\\ 
 & hêrre, dâ bin ich \textbf{bekrenket} an.\\ 
 & hie sehen mîne genôzen zuo.\\ 
20 & \textbf{diz} laster ist uns gar ze vruo.\\ 
 & kunnet ir niht vürsten schônen,\\ 
 & wir krenken ouch die krônen.\\ 
 & sol man iuch bî zühten sehen,\\ 
 & sô muoz \textbf{des iuwer zuht} \textbf{verjehen},\\ 
25 & daz sippe reichet ab iu an mich.\\ 
 & wære daz ein kebeslîcher slich\\ 
 & mînhalp, wâ \textbf{uns} diu \textbf{wirt} gezilt,\\ 
 & ir het iuch gâhes \textbf{gegen} mir bevilt.\\ 
 & wan ich bin ein ritter doch,\\ 
30 & an dem nie valsch wart vunden noch.\\ 
\end{tabular}
\scriptsize
\line(1,0){75} \newline
m n o \newline
\line(1,0){75} \newline
\textbf{9} \textit{Initiale} m n  \newline
\line(1,0){75} \newline
\textbf{1} ie] E n (o)  $\cdot$ wâ] was m \textbf{2} man] man es o  $\cdot$ schirme] schẏrmes o \textbf{3} dâ] Do m n o \textbf{4} sînem] synen o \textbf{5} ob] Ob der o  $\cdot$ dâ] do n o  $\cdot$ manlîchiu] manliche m manlich n o \textbf{6} Vergulaht] vergulacht n  $\cdot$ iwers] ires m  $\cdot$ vluht] fluͯt o \textbf{7} vür] frur o \textbf{8} noch] nach m o \textbf{9} Kin grimursel] Ein grimursel m KJngrumúrsel n Kingrimmursel o \textbf{10} geschach] gescheich m \textbf{12} Plimizols] plimczolcz o \textbf{14} gap] gap so n \textbf{16} es] das n o  $\cdot$ iuch] \textit{om.} o  $\cdot$ würde] werde n \textbf{17} in] Jch m  $\cdot$ wanne] wan o \textbf{18} bekrenket] gekrekencket n gekrencket o \textbf{19} sehen] sehent o · mîne ] min n (o) \textbf{21} schônen] schowen o \textbf{22} wir] Wie o  $\cdot$ ouch] úch n (o) \textbf{23} man] nach n \textbf{25} ab] an o \textbf{27} gezilt] gezelt n \textbf{28} het] hetten n  $\cdot$ bevilt] gezelt n \textbf{29} wan] Wenne n \textbf{30} valsch] falsch wort n falch o \newline
\end{minipage}
\end{table}
\newpage
\begin{table}[ht]
\begin{minipage}[t]{0.5\linewidth}
\small
\begin{center}*G
\end{center}
\begin{tabular}{rl}
 & ich hôrte ie sagen, swâ ez sô gezôch,\\ 
 & daz man gein wîbes scherme vlôch,\\ 
 & dâ solt ellenthaftez jagen\\ 
 & an sînem strîte gar verzagen,\\ 
5 & op dâ wære manlîch zuht.\\ 
 & hêr Vergulaht, iwers gastes vluht,\\ 
 & dier gein mir tet vür den tôt,\\ 
 & lêrt iwern brîs noch lasters nôt."\\ 
 & Kingrimursel dô sprach:\\ 
10 & "hêrre, ûf iweren trôst ges\textit{ch}ach,\\ 
 & daz ich \textbf{dem} hêrn Gawan\\ 
 & ûf \textbf{dem} Blimzoles plân\\ 
 & gap vride in iwer lant.\\ 
 & iwer sicherheit \textbf{was} pfant,\\ 
15 & obe in sîn ellen trüege her,\\ 
 & daz \textbf{ich} \textbf{des} vür iuch würde wer,\\ 
 & in bestüende \textbf{hie} \textbf{niht} wan \textbf{einec} man.\\ 
 & hêrre, dâ bin ich \textbf{gekrenkt} an.\\ 
 & hie sehen mîne genôze zuo.\\ 
20 & \textbf{diz} laster ist uns gar ze vruo.\\ 
 & \begin{large}K\end{large}unnet ir niht vürsten schônen,\\ 
 & wir krenken ouch die krônen.\\ 
 & sol man iuch bî zühten sehen,\\ 
 & sô muoz \textbf{des iwer zuht} \textbf{vergehen},\\ 
25 & daz sippe reicht abe iu an mich.\\ 
 & wære daz ein kebeslîcher slich\\ 
 & mînhalp, swâ \textbf{uns} diu \textbf{wære} gezilt,\\ 
 & ir hetet iuch gâhens \textbf{gein} mir befilt.\\ 
 & wan ich bin ein rîter doch,\\ 
30 & an dem nie valsch wart vunden noch.\\ 
\end{tabular}
\scriptsize
\line(1,0){75} \newline
G I O L M Q R Z \newline
\line(1,0){75} \newline
\textbf{1} \textit{Initiale} O L Z   $\cdot$ \textit{Capitulumzeichen} R  \textbf{9} \textit{Initiale} I  \textbf{21} \textit{Initiale} G  \newline
\line(1,0){75} \newline
\textbf{1} ich] ÷ch O  $\cdot$ swâ ez] swa ez sich I waz L wa esz M (Q) (R) (Z) \textbf{2} gein] zuͤ I  $\cdot$ wîbes scherme] schirmes wibe R  $\cdot$ vlôch] ie vloͤch I zoch R \textbf{3} dâ] Du Q R  $\cdot$ solt] soldin M  $\cdot$ ellenthaftez] erenthafftes Q allenthafftts R ellentez Z \textbf{5} dâ] do Q  $\cdot$ manlîch] manlichiv O (L) (M) \textbf{6} Vergulaht] virgaluht I vergvlaht O (L) vorgulacht M vergulacht Q R  $\cdot$ iwers] úwer R  $\cdot$ vluht] [fuht]: fluht G \textbf{7} tet vür] tet durch I fvͤr O (M) \textbf{8} lêrt] Leit Q  $\cdot$ iwern] u˒wer R \textbf{9} Kingrimursel] Kyngrimvrsel O Kyngrymursel M Kúngrumursel R  $\cdot$ dô] da M \textbf{10} geschach] gesach G daz geshach I \textbf{11} ich dem] ich den I (L) ich O Q Z \textit{om.} R  $\cdot$ hêrn] Her R  $\cdot$ Gawan] Gawin R \textbf{12} Blimzoles] plimizoles I plimizol O Q plẏmizolles L primizol M plimizols R Z \textbf{13} vride] fride her O M (R) (Z) her L fride er Q  $\cdot$ lant] hant L \textbf{14} pfant] gepfant Z \textbf{15} ellen] ere Q \textbf{16} würde wer] wer gewer I werde wer L M ward wer R \textbf{17} in] \textit{om.} R  $\cdot$ hie] \textit{om.} M  $\cdot$ einec] ein I O L (M) Q R \textbf{18} dâ] do Q  $\cdot$ gekrenkt] bechrenchet O (L) (M) bekumber R bekrenken Z \textbf{19} mîne] min I O  $\cdot$ genôze] genozen I (L) (M) (Q) (Z) \textbf{20} diz] Daz O L (Q) (R) \textbf{23} iuch] mich I \textbf{24} des] mir I dz R  $\cdot$ vergehen] des veriehen I \textbf{25} daz] daz diu I  $\cdot$ reicht] rechit M reiche Q Reiget R  $\cdot$ abe] von R \textbf{26} kebeslîcher] chebblicher I  $\cdot$ slich] schein Q \textbf{27} swâ] wa L M Q (Z) was R  $\cdot$ wære] wirt I O L M Q R Z  $\cdot$ gezilt] bezilt I geczelt R \textbf{28} hetet iuch] \textit{om.} I hat vnsz M  $\cdot$ gâhens] gahen I Q  $\cdot$ befilt] bewilt Q \textbf{29} wan] wande I  $\cdot$ ein] \textit{om.} O L \textbf{30} dem] den Q  $\cdot$ nie] \textit{om.} I  $\cdot$ valsch wart] wart valsch L  $\cdot$ noch] dach M (R) \newline
\end{minipage}
\hspace{0.5cm}
\begin{minipage}[t]{0.5\linewidth}
\small
\begin{center}*T
\end{center}
\begin{tabular}{rl}
 & ich hôrte ie sagen, swâz \textbf{sich} sô gezôch,\\ 
 & daz man gegen wîbes schirme vlôch,\\ 
 & dâ solte ellenthaftez jagen\\ 
 & an sînem strîte gar verzagen,\\ 
5 & ob dâ wære manlîche zuht.\\ 
 & hêr Vergulaht, iuwers gastes vluht,\\ 
 & die er gegen mir tet vür den tôt,\\ 
 & lêret iuwern prîs noch lasters nôt."\\ 
 & \begin{large}K\end{large}yngrimursel dô sprach:\\ 
10 & "hêrre, ûf iuwern trôst geschach,\\ 
 & daz ich hêrn Gawan\\ 
 & ûf \textbf{des} Plymizols plân\\ 
 & gap vride \textbf{her} in iuwer lant.\\ 
 & iuwer sicherheit \textbf{was} pfant,\\ 
15 & ob in sîn ellen trüege her,\\ 
 & daz \textbf{des} vür iu würde wer,\\ 
 & in bestüende \textbf{hie} \textbf{nieman} wan \textbf{ein} man.\\ 
 & hêrre, dâ bin ich \textbf{bekrenket} an.\\ 
 & hie sehen mîne genôze zuo.\\ 
20 & \textbf{daz} laster ist uns gar ze vruo.\\ 
 & kunnet ir niht vürsten schônen,\\ 
 & wir krenken ouch die krônen.\\ 
 & sol man iu bî zühten sehen,\\ 
 & sô muoz \textbf{iuwer zuht des} \textbf{jehen},\\ 
25 & daz sippe reichet ab iu an mich.\\ 
 & wære daz ein kebeslîcher slich\\ 
 & mînhalp, swâ di\textit{u} \textbf{wirt} gezilt,\\ 
 & ir het iu gâhens \textbf{gein} mir bevilt.\\ 
 & wan ich bin ein rîter doch,\\ 
30 & an dem nie valsch wart \textit{v}unden noch.\\ 
\end{tabular}
\scriptsize
\line(1,0){75} \newline
T U V W \newline
\line(1,0){75} \newline
\textbf{9} \textit{Initiale} T U V W  \newline
\line(1,0){75} \newline
\textbf{1} swâz] swenne ez V wo es W \textbf{3} dâ] Do U W \textbf{5} dâ] do U V W  $\cdot$ manlîche] mannlich W \textbf{6} Vergulaht] Vergvlaht T vergulacht U W virgulaht V \textbf{8} noch] wol V \textbf{9} Kyngrimursel] Kẏngrimursel V \textbf{10} geschach] daz geschach V \textbf{11} Gawan] Gawân T \textbf{12} des] dem U V W  $\cdot$ Plymizols] plimizols V W \textbf{13} vride] vriden U \textbf{14} was] die was U V \textbf{16} des vür iu würde] ich des vor vch worde U (V) ich das fúr in wird W \textbf{17} nieman] nit W  $\cdot$ wan] denne V \textbf{18} dâ] do U V W  $\cdot$ bekrenket] gecrenket U \textbf{19} genôze] genossen V W \textbf{22} ouch] eúch W \textbf{24} iuwer zuht des jehen] des eúwer zucht veriehen W \textbf{25} ab] von U \textbf{26} kebeslîcher] buͤbischlicher W \textbf{27} swâ die] [swas]: swa die T wo vns die U W swa vnz die V \textbf{28} gâhens] gahen W \textbf{30} nie valsch wart] valch nie ward W  $\cdot$ vunden] wunden T \newline
\end{minipage}
\end{table}
\end{document}
