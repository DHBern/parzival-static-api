\documentclass[8pt,a4paper,notitlepage]{article}
\usepackage{fullpage}
\usepackage{ulem}
\usepackage{xltxtra}
\usepackage{datetime}
\renewcommand{\dateseparator}{.}
\dmyyyydate
\usepackage{fancyhdr}
\usepackage{ifthen}
\pagestyle{fancy}
\fancyhf{}
\renewcommand{\headrulewidth}{0pt}
\fancyfoot[L]{\ifthenelse{\value{page}=1}{\today, \currenttime{} Uhr}{}}
\begin{document}
\begin{table}[ht]
\begin{minipage}[t]{0.5\linewidth}
\small
\begin{center}*D
\end{center}
\begin{tabular}{rl}
\textbf{88} & diu liefen elliu driu vür in\\ 
 & \textbf{und} sprâchen: "hêrre, hâstû sin,\\ 
 & dir zelt \textbf{regin} de Franze\\ 
 & der werden minne \textbf{sô ganze},\\ 
5 & sô mahtû spiln sunder pfant.\\ 
 & \textbf{dîn} \textbf{vreude} ist kumbers ledic zehant."\\ 
 & \textbf{dô} diu botschaft was vernomen,\\ 
 & Kaylet, der ê was komen,\\ 
 & saz \textbf{der künegîn under} \textbf{ir} mandels ort.\\ 
10 & \textbf{hin zim sprach si} disiu wort:\\ 
 & "\begin{large}S\end{large}ag an, ist dir \textbf{iht} mêr geschehen?\\ 
 & ich hân slege an dir \textbf{gesehen}."\\ 
 & dô begreif \textbf{im} diu gehiure\\ 
 & sîne quaschiure\\ 
15 & mit ir linden handen wîz.\\ 
 & \textbf{dâr an} lac \textbf{der} gotes vlîz.\\ 
 & dô was im geamsieret\\ 
 & unt \textbf{zerquaschieret}\\ 
 & hiufel, kinne unt \textbf{an} der nasen.\\ 
20 & er hete der küneginne basen,\\ 
 & diu dise êre an im begienc,\\ 
 & daz si in \textbf{mit handen} z\textbf{ir} gevienc.\\ 
 & Si sprach \textbf{nâch} zühte lêre\\ 
 & \textbf{hin} ze Gahmurete mêre:\\ 
25 & "iu biutet vaste ir minne\\ 
 & diu werde Franzoysinne.\\ 
 & \textbf{nû} êret an mir elliu wîp\\ 
 & unt lât ze rehte mînen lîp.\\ 
 & sît hie, unz ich mîn reht genem.\\ 
30 & ir lâzet anders mich in schem."\\ 
\end{tabular}
\scriptsize
\line(1,0){75} \newline
D \newline
\line(1,0){75} \newline
\textbf{11} \textit{Initiale} D  \textbf{23} \textit{Majuskel} D  \newline
\line(1,0){75} \newline
\textbf{17} geamsieret] gæmsieret D \textbf{24} hin ze] hinz D  $\cdot$ Gahmurete] Gahmvrete D \newline
\end{minipage}
\hspace{0.5cm}
\begin{minipage}[t]{0.5\linewidth}
\small
\begin{center}*m
\end{center}
\begin{tabular}{rl}
 & diu liefen alliu driu vür in.\\ 
 & \textbf{si} sprâchen: "hêrre, hâstû sin,\\ 
 & dir zelt \textbf{regine} d\textit{e} Franze\\ 
 & der werde\textit{n} minne \textbf{schanze},\\ 
5 & sô maht dû spiln sunder pfant.\\ 
 & \textbf{dîn} \textbf{munt} ist kumbers ledic zehant."\\ 
 & \textbf{\begin{large}D\end{large}ô} diu botschaft was vernomen,\\ 
 & Kailet, der ê was komen,\\ 
 & saz \textbf{der künigîn under} \textbf{ir} mantels ort.\\ 
10 & \textbf{hin zuo im sprach si} disiu wort:\\ 
 & "sage an, ist dir \textbf{iht} mêr geschehen?\\ 
 & ich hân slege an dir \textbf{ersehen}."\\ 
 & dô begreif \textbf{in} diu gehiure,\\ 
 & sîne quaschiure,\\ 
15 & mit ir linden henden wîz.\\ 
 & \textbf{an den} lac gotes vlîz.\\ 
 & dô was ime ge\textit{am}sieret\\ 
 & und \textbf{sêre} \textbf{zerquaschieret}\\ 
 & hiufel, kinne und \textbf{an} der nasen.\\ 
20 & er het \textit{der} küniginne basen,\\ 
 & diu dise êre an ime begienc,\\ 
 & daz si in \textbf{sô nâhe} zuo \textbf{ir} gevienc.\\ 
 & si sprach \textbf{nâch} zühte lêre\\ 
 & \textbf{hin} zuo Gahmurete mêre:\\ 
25 & "iu biutet vaste ir minne\\ 
 & diu werde Franzoisinne.\\ 
 & \textbf{nû} êret an mir alliu wîp\\ 
 & und lât ze rehte mînen lîp.\\ 
 & sît hie, unz ich mîn reht geneme.\\ 
30 & ir lâzet anders mich in scheme."\\ 
\end{tabular}
\scriptsize
\line(1,0){75} \newline
m n o \newline
\line(1,0){75} \newline
\textbf{7} \textit{Initiale} m n o  \newline
\line(1,0){75} \newline
\textbf{1} vür in] do hin o \textbf{3} dir] Die n o  $\cdot$ zelt] zelte o  $\cdot$ de Franze] die francze m o defrantze n \textbf{4} werden] werde m  $\cdot$ minne] mẏnnen o \textbf{5} pfant] [schancz]: pfant o \textbf{7} was] wart n \textbf{8} Kailet] Kaẏles n \textbf{9} saz] Das o  $\cdot$ künigîn] konig o  $\cdot$ ir mantels ort] jrem mantelsort o \textbf{11} geschehen] beschehen o \textbf{14} quaschiure] do atscúre n doat scuͯre o \textbf{15} henden] hende n o \textbf{17} geamsieret] gemasieret m (n) \textbf{18} zerquaschieret] coascieret n zuͦ toascieret o \textbf{19} kinne] kome n \textbf{20} der] \textit{om.} m \textbf{22} sô] zuͦ o  $\cdot$ ir] ie o  $\cdot$ gevienc] fing n \textbf{24} Gahmurete] gamiret n gamúret o  $\cdot$ mêre] hohe mere n \textbf{25} iu] Jch n \textbf{26} Franzoisinne] franczossinnne m frantzosinne n franczosinne o \textbf{29} hie] sie o  $\cdot$ geneme] gememe o \textbf{30} lâzet] lossent es n \newline
\end{minipage}
\end{table}
\newpage
\begin{table}[ht]
\begin{minipage}[t]{0.5\linewidth}
\small
\begin{center}*G
\end{center}
\begin{tabular}{rl}
 & diu liefen elliu driu vür in.\\ 
 & \textbf{si} sprâchen: "hêrre, hâstû sin,\\ 
 & \begin{large}D\end{large}ir zelt \textbf{roy} de Franze\\ 
 & der werden minne \textbf{schanze},\\ 
5 & sô mahtû spilen sunder pfant.\\ 
 & \textbf{dîn} \textbf{vröude} ist kumbers ledec zehant."\\ 
 & diu botschaft was \textbf{ouch} vernomen.\\ 
 & Kailet, der ê was komen,\\ 
 & saz \textbf{der künigîn under}\textbf{s} mandeles ort.\\ 
10 & \textbf{si sprach hin ze im} disiu wort:\\ 
 & "sage an, ist dir \textbf{iht} mê geschehen?\\ 
 & ich hân slege an dir \textbf{ersehen}."\\ 
 & dô begreif \textbf{im} diu gehiure\\ 
 & sîne quatschiure\\ 
15 & mit ir linden handen wîz.\\ 
 & \textbf{an den} lac \textbf{der} gotes vlîz.\\ 
 & dô was im geamisieret\\ 
 & unde \textbf{sêre} \textbf{zerquatschieret}\\ 
 & hiufel, kinne und \textbf{an} der nasen.\\ 
20 & er het der küniginne basen,\\ 
 & diu dise êre an im begienc,\\ 
 & daz sin \textbf{mit handen} z\textbf{ir} gevienc.\\ 
 & si sprach \textbf{mit} zühte lêre\\ 
 & ze Gahmurete mêre:\\ 
 & \hspace*{-.7em}\big| "diu werde Franzoisinne\\ 
25 & \hspace*{-.7em}\big| iu biut vaste ir minne.\\ 
 & \textbf{nû} êret an mir elliu wîp\\ 
 & unde lât ze rehte mînen lîp.\\ 
 & sît hie, \textit{un}z ich mîn reht geneme.\\ 
30 & ir lât anders mich in scheme."\\ 
\end{tabular}
\scriptsize
\line(1,0){75} \newline
G I O L M Q R Z Fr21 \newline
\line(1,0){75} \newline
\textbf{3} \textit{Initiale} G  \textbf{7} \textit{Initiale} I L R Z Fr21  \textbf{27} \textit{Initiale} I  \newline
\line(1,0){75} \newline
\textbf{1} diu] Sie M \textbf{2} sprâchen] sprach I  $\cdot$ hâstû] vnd hastu R \textbf{3} Dir] Diesz M Die R  $\cdot$ zelt] tailet L  $\cdot$ roy] Royn O (L) (Q) (R) (Fr21) rein Z  $\cdot$ de] der O \textit{om.} L zu Q von Fr21  $\cdot$ Franze] francze R frantze Z \textbf{4} der] die I  $\cdot$ minne] minnen Z  $\cdot$ schanze] tantze Q [lanze]: tscanze Fr21 \textbf{5} sunder] Sundern M \textbf{6} kumbers ledec] ledic kummers Z \textbf{7} ouch] da L \textbf{8} Kailet] Gahilet I Kaylet O L Q R Fr21 Gailet Z  $\cdot$ ê was] was ê O \textbf{9} saz] Vnde saz O Das R  $\cdot$ künigîn] kúng R  $\cdot$ unders] vnder ir L  $\cdot$ mandeles] mantess I wandels Q \textbf{10} sprach] sprachin M  $\cdot$ hin] zuͤ I \textbf{11} iht] nicht Q \textbf{12} \textit{Vers 88.12 fehlt} R   $\cdot$ ich] [H]: Jch O  $\cdot$ slege] flege O  $\cdot$ ersehen] gesehen I O L Z \textbf{13} dô] Da M Z  $\cdot$ im] in O (M) (R) \textbf{14} quatschiure] quaskure I \textbf{15} linden] liehten I  $\cdot$ handen] hende R \textbf{16} den] der Q  $\cdot$ gotes] todes Q \textbf{17} dô] Da O M R Z  $\cdot$ im] in R  $\cdot$ geamisieret] zemasciert I gemisieret O gemusieret M R \textbf{18} zerquatschieret] quatschieret O (Fr21) geqwatschieret L \textbf{20} het] hat M  $\cdot$ küniginne] kúnginnen R \textbf{21} im] Jn R \textbf{22} sin] sein Q \textit{om.} R in Fr21  $\cdot$ gevienc] gewinck Q \textbf{23} mit] nach Z  $\cdot$ zühte] zúchten R \textbf{24} Gahmurete] Gahmureten I Gamvreten O Gahmuͯreten L gamurete M Q Z Gahmuretten R Gahmoreten Fr21 \textbf{26} Franzoisinne] fronzoisinne I franzoysinne O Q franzossẏnne L franciosynne M franzosinne R frantzoisinne Z francisinne Fr21 \textbf{25} biut vaste] vaste biutet I \textbf{29} unz] biz G waz Fr21  $\cdot$ ich] \textit{om.} Q  $\cdot$ mîn] mit Q mine Z \textbf{30} ir] ê ir I  $\cdot$ anders mich in] mich in rehter I andirs mich M  $\cdot$ scheme] erschem M schein Q \newline
\end{minipage}
\hspace{0.5cm}
\begin{minipage}[t]{0.5\linewidth}
\small
\begin{center}*T (U)
\end{center}
\begin{tabular}{rl}
 & diu liefen alliu driu vür in.\\ 
 & \textbf{si} sprâchen: "hêrre, hâst dû sin,\\ 
 & dir zelt \textbf{royn} d\textit{e} Franze\\ 
 & der werden minne \textbf{schanze},\\ 
5 & sô maht dû spiln sunder pfant.\\ 
 & \textbf{diu} \textbf{vreude} ist kumbers ledic zuohant."\\ 
 & \begin{large}D\end{large}iu botschaft was \textbf{ouch} vernomen.\\ 
 & Kaylet, der ê was komen,\\ 
 & saz \textbf{under der küneginne} mantel\textit{s} ort.\\ 
10 & \textbf{si sprach hin zuo im} disiu wort:\\ 
 & "sage an, ist dir \textbf{êt} mê geschehen?\\ 
 & ich hân slege an dir \textbf{gesehen}."\\ 
 & dô begreif \textbf{im} diu gehiure\\ 
 & sîne quatschiure\\ 
15 & mit ir linden henden wîz.\\ 
 & \textbf{an den} lac \textbf{der} gotes vlîz.\\ 
 & dô was im geamisieret\\ 
 & und \textbf{sêre} \textbf{gequatschieret}\\ 
 & hiufel, kinne und der nasen.\\ 
20 & er hât\textit{e} \textit{der} küneginne basen,\\ 
 & diu dise êre an im begienc,\\ 
 & daz sin \textbf{mit henden} zuo gevienc.\\ 
 & si sprach \textbf{mit} zühte lêre\\ 
 & zuo Gahmurete mêre:\\ 
 & \hspace*{-.7em}\big| "diu werde Franzoysinne\\ 
25 & \hspace*{-.7em}\big| iu biutet vaste ir minne.\\ 
 & êret an mir alliu wîp\\ 
 & und lât zuo rehte mînen lîp.\\ 
 & s\textit{î}t hie, unz ich mîn reht geneme.\\ 
30 & ir lâzet anders mich in scheme."\\ 
\end{tabular}
\scriptsize
\line(1,0){75} \newline
U V W T \newline
\line(1,0){75} \newline
\textbf{2} \textit{Majuskel} T  \textbf{3} \textit{Majuskel} T  \textbf{7} \textit{Initiale} U V W T  \textbf{13} \textit{Majuskel} T  \textbf{17} \textit{Majuskel} T  \newline
\line(1,0){75} \newline
\textbf{1} diu] die T \textbf{3} royn] roy W  $\cdot$ de] der U  $\cdot$ Franze] frantze W \textbf{4} der] die V \textbf{5} sunder] \textit{om.} W \textbf{6} diu] din V (W) T  $\cdot$ vreude] frauw W  $\cdot$ kumbers ledic zuohant] von kvmbers bant T \textbf{8} Kaylet] Kalet U Kaẏlet V Gaylet W \textbf{9} under der küneginne mantels] vnder der kuͦneginne mantel U der kúnigin vnder des mantels W (T) \textbf{10} hin] \textit{om.} W T \textbf{11} êt] it U iht V (W) T \textbf{12} slege] schlege vil W \textbf{13} im] in W \textit{om.} T \textbf{14} quatschiure] micheln quatschúre W \textbf{16} den] der T  $\cdot$ der gotes] goldes so grosser W \textbf{17} im geamisieret] er gegasieret W \textbf{18} gequatschieret] zerqvarscieret T \textbf{19} Húlffe kúnne vnd nasen W  $\cdot$ und] vnde an T  $\cdot$ der] die V \textbf{20} hâte] hat U bat W  $\cdot$ der] \textit{om.} U \textbf{21} im] ir W \textbf{22} mit] mit irn W  $\cdot$ zuo gevienc] zir [gefienge]: gefieng V zuͦ ir fing W zir gevienc T \textbf{23} zühte] zúhtiger V \textbf{24} Gahmurete] Gahmuͦrete U Gamurete V gamuret W \textbf{26} \textit{Versfolge 88.25-26} W   $\cdot$ Franzoysinne] frantzoẏssinne V franzosinne W \textbf{25} iu biutet vaste ir] Im erbeútet vast ir W îv enbîvtet werde T \textbf{27} êret] nv êrent T \textbf{28} Sprach ir minnenclicher lip V  $\cdot$ lât] lond mich W lânt comen T  $\cdot$ mînen] mynne W \textbf{29} sît] Sist U  $\cdot$ unz] bit U  $\cdot$ geneme] neme W \newline
\end{minipage}
\end{table}
\end{document}
