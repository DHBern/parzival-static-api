\documentclass[8pt,a4paper,notitlepage]{article}
\usepackage{fullpage}
\usepackage{ulem}
\usepackage{xltxtra}
\usepackage{datetime}
\renewcommand{\dateseparator}{.}
\dmyyyydate
\usepackage{fancyhdr}
\usepackage{ifthen}
\pagestyle{fancy}
\fancyhf{}
\renewcommand{\headrulewidth}{0pt}
\fancyfoot[L]{\ifthenelse{\value{page}=1}{\today, \currenttime{} Uhr}{}}
\begin{document}
\begin{table}[ht]
\begin{minipage}[t]{0.5\linewidth}
\small
\begin{center}*D
\end{center}
\begin{tabular}{rl}
\textbf{431} & \begin{large}D\end{large}ô Gawan enbizzen was\\ 
 & - ich sage iu, als Kyot las -,\\ 
 & durch herzenlîche triwe\\ 
 & huop sich dâ grôziu riwe.\\ 
5 & Er sprach zer küneginne:\\ 
 & "vrouwe, \textbf{hân ich} sinne\\ 
 & \textbf{unt} sol mir got den lîp bewaren,\\ 
 & sô muoz ich dienstlîchez varen\\ 
 & unt rîterlîch gemüete\\ 
10 & iwer wîplîchen güete\\ 
 & ze dienste immer kêren.\\ 
 & wande \textbf{iuch kan} sælde lêren,\\ 
 & daz ir habt valsche angesigt.\\ 
 & iwer prîs vür alle prîse wigt.\\ 
15 & gelücke \textbf{iuch müeze} sælden wern.\\ 
 & vrouwe, ich wil urloubes gern.\\ 
 & den gebt mir und lât mich varn.\\ 
 & iwer zuht müeze iwern prîs bewarn."\\ 
 & Ir was sîn danscheiden leit.\\ 
20 & dô \textbf{weinden} durch gesellecheit\\ 
 & mit ir manec \textbf{juncvrouwe} clâr.\\ 
 & diu künegîn sprach ân allen vâr:\\ 
 & "het ir mîn genozzen mêr,\\ 
 & mîn vröude \textbf{wære gein sorgen} hêr.\\ 
25 & nû moht iwer vride niht bezzer sîn.\\ 
 & \textbf{des} geloubet \textbf{aber}, swenne ir lîdet pîn,\\ 
 & ob iuch vertreit ritterschaft\\ 
 & in riwebæren kumbers kraft,\\ 
 & sô wizzet, mîn hêr Gawan,\\ 
30 & des sol mîn herze pflihte hân\\ 
\end{tabular}
\scriptsize
\line(1,0){75} \newline
D \newline
\line(1,0){75} \newline
\textbf{1} \textit{Initiale} D  \textbf{5} \textit{Majuskel} D  \textbf{19} \textit{Majuskel} D  \newline
\line(1,0){75} \newline
\newline
\end{minipage}
\hspace{0.5cm}
\begin{minipage}[t]{0.5\linewidth}
\small
\begin{center}*m
\end{center}
\begin{tabular}{rl}
 & \begin{large}D\end{large}ô Gawan enb\textit{i}zzen was\\ 
 & - ich sage iu, als Kiot \dag was\dag  -,\\ 
 & durch herzenlîche triuwe\\ 
 & huop sich dô grôz\textit{iu} riuwe.\\ 
5 & er sprach zer küniginne:\\ 
 & "vrouwe, \textbf{hân ich} sinne\\ 
 & \textbf{und} sol mir got den lîp bewarn,\\ 
 & sô muoz ich dienstlîchez varn\\ 
 & und ritterlîch gemüete\\ 
10 & i\textit{w}e\textit{r} wîplîchen güete\\ 
 & ze dieneste iemer kêren.\\ 
 & want \textbf{iuch kan} sælde lêren,\\ 
 & daz ir habet valsch angesiget.\\ 
 & iuwer prîs vür alle prîse wiget.\\ 
15 & glücke \textbf{müeze iuch} sælden wern.\\ 
 & vrouwe, ich wil urloubes gern.\\ 
 & den gebet mir und lât mich varn.\\ 
 & iuwe\textit{r} zuht müeze iuwern prîs bewarn."\\ 
 & ir was sîn danscheiden leit.\\ 
20 & dô \textbf{weineten} durch gesellecheit\\ 
 & mit ir manic \textbf{juncvrouwe} klâr.\\ 
 & diu künigîn sprach âne allen vâr:\\ 
 & "het ir mîn genozzen mêr,\\ 
 & mîn vröude \textbf{wære gegen sorgen} hêr.\\ 
25 & nû \textit{mohte} iuwer vride niht bezzer sîn.\\ 
 & geloubet \textbf{aber}, wenne ir lîdet pîn,\\ 
 & ob iuch vertreit ritterschaft\\ 
 & in \dag riuwebærem\dag  kumbers kraft,\\ 
 & sô wizzet, mîn hêr Gawan,\\ 
30 & des sol mîn herze pflihte hân\\ 
\end{tabular}
\scriptsize
\line(1,0){75} \newline
m n o \newline
\line(1,0){75} \newline
\textbf{1} \textit{Initiale} m   $\cdot$ \textit{Capitulumzeichen} n  \newline
\line(1,0){75} \newline
\textbf{1} enbizzen] enbeissen m \textbf{2} Kiot] kẏot n o \textbf{3} herzenlîche] hertzecliche n \textbf{4} grôziu] grosse m n o \textbf{6} sinne] sin n o \textbf{7} bewarn] verwarn o \textbf{8} dienstlîchez] dienstlich n dienstlichen o \textbf{10} iwer] Jre m  $\cdot$ wîplîchen] wipliche n o \textbf{12} want] Wenne n Wan o  $\cdot$ kan] sol n \textbf{15} müeze] muͯsz n o  $\cdot$ sælden] selde n o \textbf{18} iuwer] V́wern m  $\cdot$ müeze] muͯsz n o \textbf{22} allen] alle n o \textbf{24} hêr] wer n \textbf{25} mohte] \textit{om.} m moͯchte n  $\cdot$ vride] freide n (o) \textbf{26} wenne] wan o \textbf{27} vertreit] vertreat o \textbf{28} riuwebærem] vnberendem n ruberndemm o  $\cdot$ kraft] craff o \textbf{29} Gawan] gawann o \newline
\end{minipage}
\end{table}
\newpage
\begin{table}[ht]
\begin{minipage}[t]{0.5\linewidth}
\small
\begin{center}*G
\end{center}
\begin{tabular}{rl}
 & dô Gawan enbizzen was\\ 
 & - ich sage iu, als Kiot las -,\\ 
 & durch herzenlîche triwe\\ 
 & huop sich dâ grôziu riwe.\\ 
5 & er sprach zer küneginne:\\ 
 & "vrouwe, \textbf{ich hân die} sinne.\\ 
 & sol mir got den lîp bewaren,\\ 
 & sô muoz ich dienstlîchez varen\\ 
 & unt rîterlîch gemüete\\ 
10 & iwerre wîplîchen güete\\ 
 & ze dienste imer kêren.\\ 
 & wan \textbf{iuch kan} sælde lêren,\\ 
 & daz ir habet valsche angesiget.\\ 
 & iwer brîs vür alle brîse wiget.\\ 
15 & gelücke \textbf{iuch muoze} sælden weren.\\ 
 & vrouwe, ich wil urloubes geren.\\ 
 & den gebt mir und lât mich varen.\\ 
 & iwer zuht muoze iweren brîs bewaren."\\ 
 & ir was sîn danscheiden leit.\\ 
20 & dô \textbf{weinde} durch gesellecheit\\ 
 & mit ir manec \textbf{vrouwe} clâr.\\ 
 & diu künegîn sprach âne allen vâr:\\ 
 & "het ir mîn genozzen mêr,\\ 
 & mîn vröude \textbf{wære gein sorgen} hêr.\\ 
25 & nû maht iwer vride niht bezzer sîn\\ 
 & \textbf{unde} geloubet, swenne ir lîdet pîn,\\ 
 & obe iuch vertreit rîterschaft\\ 
 & in riwebære kumbers kraft,\\ 
 & sô wizzet, mîn hêr Gawan,\\ 
30 & des sol mîn herze pflihte hân\\ 
\end{tabular}
\scriptsize
\line(1,0){75} \newline
G I O L M Q R Z Fr21 \newline
\line(1,0){75} \newline
\textbf{1} \textit{Überschrift:} Hie fvr her gawan von tschanfanzvn Vnd wolde nach dem gral varn wie ez im dar nach gienge wer daz wizzen welle der lese vort Z   $\cdot$ \textit{Initiale} I O L M R Z  \textbf{17} \textit{Initiale} I  \textbf{19} \textit{Initiale} M  \newline
\line(1,0){75} \newline
\textbf{1} dô] ÷o O Da Z  $\cdot$ Gawan] Gawain R \textbf{2} sage] sagz I sagt O  $\cdot$ als] als ez I  $\cdot$ Kiot] Kyot O (Q) (R) (Z) kẏot L \textbf{4} dâ] do Q R  $\cdot$ grôziu] groz O ganczú R \textbf{6} ich hân die] han ich O L M (Q) vnd hab ich R \textbf{7} sol] vnd wil I Vnd sol O L (M) (Q) (R) (Z) \textbf{8} ich] \textit{om.} M  $\cdot$ dienstlîchez] dienstlichen O (Q) dienstliche L R \textbf{9} rîterlîch] ritterliche L \textbf{10} iwerre] ewer I (L) (M) (Q) Ẃwer R  $\cdot$ wîplîchen] weipliche Q wiplich R  $\cdot$ güete] \textit{om.} Z \textbf{11} kêren] chere I (R) \textbf{12} wan] wande I  $\cdot$ iuch kan] evch I ir kond R  $\cdot$ sælde] seldin M (Q)  $\cdot$ lêren] lere I R \textbf{13} habet valsche] falsche habt R  $\cdot$ angesiget] an gesicht Q \textbf{15} \textit{nach 431.15:} Wan uch kan selde kerren R   $\cdot$ iuch muoze sælden] ev muͤz selden I uch musze selde M euch selde mússen Q muͯsz úch selde R  $\cdot$ weren] leren R mern Z \textbf{16} wil] muͤz I \textbf{18} ewer pris muͤz ewer zuht bewarn I  $\cdot$ Got muͯs úwer bewarn R  $\cdot$ muoze] musz Q  $\cdot$ iweren] ewr Q  $\cdot$ bewaren] beiagen O \textbf{19} ir] Dr M  $\cdot$ danscheiden] hein scheiden R \textbf{20} dô] Vnd R  $\cdot$ weinde] weinden O (M) (Z) weinet R  $\cdot$ gesellecheit] gesellenheit R \textbf{21} vrouwe] ivnchfrowe O (L) (M) (Q) (R) (Z) \textbf{22} allen] alle M R  $\cdot$ vâr] war Q \textbf{23} mîn] mir Z \textbf{24} mîn] Mit Q  $\cdot$ vröude] frewden Q sorge R  $\cdot$ wære gein sorgen] gein sorgen were L wer gen froude R \textbf{25} nû] nune I (O)  $\cdot$ maht] maht aber I moͯchtte R \textbf{26} unde] Daz O L (Q) (R) (Z) Der M  $\cdot$ swenne] aber swenne I (O) (M) (Z) aber wenne L (Q) R  $\cdot$ lîdet] liden R \textbf{28} riwebære] riwebarn I triwe were O riber M \textbf{29} sô] Das R  $\cdot$ hêr Gawan] ergawan M her Gawain R \textbf{30} des] Wy desz M  $\cdot$ hân] \textit{om.} O \newline
\end{minipage}
\hspace{0.5cm}
\begin{minipage}[t]{0.5\linewidth}
\small
\begin{center}*T
\end{center}
\begin{tabular}{rl}
 & \begin{large}D\end{large}ô Gawan enbizzen was\\ 
 & - ich sagiu, als\textbf{ez} Kyot las -\\ 
 & durch herzeclîche triuwe\\ 
 & huop sich dô grôze riuwe.\\ 
5 & er sprach zer küneginne:\\ 
 & "vrouwe, \textbf{hân ich} sinne\\ 
 & \textbf{und} sol mir got den lîp bewarn,\\ 
 & sô muoz ich dienstlîchez varn\\ 
 & und rîterlîch gemüete\\ 
10 & iuwer wîplîchen güete\\ 
 & ze dienste iemer kêren.\\ 
 & wan \textbf{diu kam} sælde lêren,\\ 
 & daz ir habt valsche angesiget.\\ 
 & iuwer prîs vür alle prîse wiget.\\ 
15 & glücke \textbf{müeziu} sælden wern.\\ 
 & vrouwe, ich wil urloubes gern.\\ 
 & den gebt mir und lât mich varn.\\ 
 & iuwer zuht müeze iuwern prîs bewarn."\\ 
 & Ir was sîn danscheiden leit.\\ 
20 & dô \textbf{weinde} durch gesellecheit\\ 
 & mit ir manec \textbf{ju\textit{n}cvrouwe} clâr.\\ 
 & diu künegîn sprach âne allen vâr:\\ 
 & "het ir mîn genozzen mêr,\\ 
 & mîn vröude \textbf{gegen sorgen wære} hêr.\\ 
25 & nû moht iuwer vride niht bezzer sîn.\\ 
 & \textbf{daz} geloubet \textbf{aber}, swennir lîdet pîn,\\ 
 & ob iu vertreget rîterschaft\\ 
 & in riuwebære kumbers kraft,\\ 
 & sô wizzet, mîn hêr Gawan,\\ 
30 & des sol mîn herze pflihte hân\\ 
\end{tabular}
\scriptsize
\line(1,0){75} \newline
T U V W \newline
\line(1,0){75} \newline
\textbf{1} \textit{Initiale} T U V  \textbf{19} \textit{Versal} T  \newline
\line(1,0){75} \newline
\textbf{1} Gawan] gawans W \textbf{2} sagiu] sage U  $\cdot$ alsez] als W \textbf{3} herzeclîche triuwe] hertzenliche reúwe W \textbf{8} varn] waren W \textbf{10} iuwer wîplîchen] Vwer wipliche U [*]: Vwerre wipliche V Eúwer weiblich W \textbf{12} diu kam] vch kan U (V) ich kan W \textbf{14} alle prîse] ander preis W \textbf{15} müeziu] muͦze U muͦße ich W  $\cdot$ sælden] selde V W \textbf{16} vrouwe ich wil] Erawe ich W \textbf{21} manec] manege U  $\cdot$ juncvrouwe] îuvcvrouwe T \textbf{22} diu künegîn] Die W  $\cdot$ allen] alle W \textbf{23} het] Hetten W \textbf{24} gegen sorgen wære] wer gegen sorgen V (W) \textbf{25} moht] moͤcht W  $\cdot$ vride] froͤd W \textbf{26} swennir] wan ir U wenn ir W \textbf{27} iu] eúchs W \textbf{28} riuwebære] reúweberm W \newline
\end{minipage}
\end{table}
\end{document}
