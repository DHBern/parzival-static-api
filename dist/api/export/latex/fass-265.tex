\documentclass[8pt,a4paper,notitlepage]{article}
\usepackage{fullpage}
\usepackage{ulem}
\usepackage{xltxtra}
\usepackage{datetime}
\renewcommand{\dateseparator}{.}
\dmyyyydate
\usepackage{fancyhdr}
\usepackage{ifthen}
\pagestyle{fancy}
\fancyhf{}
\renewcommand{\headrulewidth}{0pt}
\fancyfoot[L]{\ifthenelse{\value{page}=1}{\today, \currenttime{} Uhr}{}}
\begin{document}
\begin{table}[ht]
\begin{minipage}[t]{0.5\linewidth}
\small
\begin{center}*D
\end{center}
\begin{tabular}{rl}
\textbf{265} & \begin{large}D\end{large}â ergienc diu scharpfe herte:\\ 
 & iewederer vaste werte\\ 
 & sînen prîs vor dem ander.\\ 
 & \textbf{der herzoge} Orilus de Lalander\\ 
5 & streit nâch sîme gelêrten site.\\ 
 & ich wæne, ieman sô \textbf{vil} gestrite.\\ 
 & er hete kunst unde kraft,\\ 
 & des wart er dicke sigehaft\\ 
 & an maneger stat, swie ez dâ ergienc.\\ 
10 & durch den trôst \textbf{zuo zim er} vienc\\ 
 & den jungen, starken Parzival.\\ 
 & der begreif ouch \textit{in} dô sunder twâl\\ 
 & und zucten ûz dem satel sîn\\ 
 & als eine garben hebrîn\\ 
15 & Vast er in under \textbf{die arme} swanc,\\ 
 & mit im \textbf{er von dem orse} spranc\\ 
 & unt dructen über einen ronen.\\ 
 & dâ muose schumpfentiure wonen,\\ 
 & der sölher nœte niht was gewent.\\ 
20 & "dû gearnest, daz sich hât versent\\ 
 & disiu vrouwe von \textbf{dîme} zorne.\\ 
 & \textbf{nû} bistû der verlorne,\\ 
 & dû\textbf{ne} lâzest si \textbf{dîne} hulde hân."\\ 
 & "daz enwirt sô \textbf{gâhes} niht getân",\\ 
25 & sprach der herzoge Orilus,\\ 
 & "ich bin \textbf{noch} \textbf{unbetwungen} sus."\\ 
 & Parzival, der werde degen,\\ 
 & dructen an sich, daz bluotes regen\\ 
 & spranc durch die barbiere.\\ 
30 & dâ wart der vürste schiere\\ 
\end{tabular}
\scriptsize
\line(1,0){75} \newline
D \newline
\line(1,0){75} \newline
\textbf{1} \textit{Initiale} D  \textbf{15} \textit{Majuskel} D  \newline
\line(1,0){75} \newline
\textbf{4} Lalander] Lalandr D \textbf{12} in] \textit{om.} D \textbf{25} Orilus] Ôrilvs D \newline
\end{minipage}
\hspace{0.5cm}
\begin{minipage}[t]{0.5\linewidth}
\small
\begin{center}*m
\end{center}
\begin{tabular}{rl}
 & \begin{large}D\end{large}\textit{â} ergienc diu scharpfe herte:\\ 
 & ietweder vaste werte\\ 
 & sîne\textit{n} prîs vor dem andern.\\ 
 & \textbf{duc} \textit{O}rilus de Lalander\textit{n}\\ 
5 & streit nâch sînem gelêrten site.\\ 
 & ich wæne, ieman sô \textbf{vil} gestrite.\\ 
 & er hete kunst und kraft,\\ 
 & des wart er dicke sigehaft\\ 
 & an maniger stat, wie ez d\textit{â} ergienc.\\ 
10 & durch den trôst \textbf{zuom er} vienc\\ 
 & den jungen, starken Parcifal.\\ 
 & der begrei\textit{f} ouch in dô sunder twâl\\ 
 & und zucte in ûz dem satele sîn\\ 
 & alsô ein garb heberîn\\ 
15 & vaste \textit{e}r\textit{n} under \textbf{die arme} swanc,\\ 
 & mit ime \textbf{er von dem rosse} spranc\\ 
 & und druhte in über einen ronen.\\ 
 & dô muose schumpfentiure wonen,\\ 
 & der solher nœte niht was gewenet.\\ 
20 & "\dag die arnasch\dag , daz sich hât versenet\\ 
 & disiu vrouwe von \textbf{dînem} zorne.\\ 
 & \textbf{nû} bistû der verlorne,\\ 
 & dû \textbf{en}lâzest \textit{si} \textbf{die} hulden hân."\\ 
 & "daz enwirt sô \textbf{gâhes} niht getân",\\ 
25 & sprach der herzoge Orilus,\\ 
 & "ich \textbf{en}bin \textbf{noch} \textbf{niht betwungen} sus."\\ 
 & Parcifal, der werde degen,\\ 
 & d\textit{r}uht in an sich, daz bluotes regen\\ 
 & spranc durch die barbiere.\\ 
30 & dô wart der vürste schiere\\ 
\end{tabular}
\scriptsize
\line(1,0){75} \newline
m n o Fr69 \newline
\line(1,0){75} \newline
\textbf{1} \textit{Initiale} m   $\cdot$ \textit{Capitulumzeichen} n  \newline
\line(1,0){75} \newline
\textbf{1} Dâ] DO m (n) (o) \textbf{3} sînen] Sinem m  $\cdot$ dem] den o \textbf{4} duc] Vntze n (o)  $\cdot$ Orilus] arilus m orẏlus n  $\cdot$ de Lalandern] de [lalanden]: lalander m de lalanderen n der lalander o delalandern Fr69 \textbf{5} site] sitten o \textbf{6} gestrite] stritt o \textbf{7} kunst] kunfft n \textbf{9} stat] \textit{om.} n  $\cdot$ dâ] do m n o  $\cdot$ ergienc] ging n o \textbf{12} begreif] begreiffe m  $\cdot$ ouch in dô] in ouch n \textbf{14} heberîn] lebelin n o \textbf{15} ern] arne m  $\cdot$ swanc] twang n o \textbf{17} druhte] durchte o \textbf{18} muose] musse m muͯste n \textbf{20} arnasch] ernst n [erst]: ernst o  $\cdot$ daz] des n \textbf{22} bistû] bist o \textbf{23} dû enlâzest] Du lossest n Des sassest o  $\cdot$ si] \textit{om.} m sú denne n sie dan o  $\cdot$ die] din Fr69  $\cdot$ hulden] hulde n o \textbf{24} enwirt] wurt n (o) \textbf{25} Orilus] orelus o \textbf{26} enbin] bin n o \textbf{27} werde] wede o \textbf{28} druht] Duht m (n) (o)  $\cdot$ sich] sin o  $\cdot$ daz] des n o  $\cdot$ regen] wegen n regensz o \newline
\end{minipage}
\end{table}
\newpage
\begin{table}[ht]
\begin{minipage}[t]{0.5\linewidth}
\small
\begin{center}*G
\end{center}
\begin{tabular}{rl}
 & \begin{large}D\end{large}â ergie diu scherpfe herte:\\ 
 & ietweder vaste werte\\ 
 & sînen brîs vor dem ander.\\ 
 & Orillus de Lalander\\ 
5 & streit nâch sînem gelêrten site.\\ 
 & ich wæne, ie man sô \textbf{wol} gestrite.\\ 
 & er hete kunst unde kraft,\\ 
 & des wart er dicke sigehaft\\ 
 & an maniger stat, swiez dâ ergienc.\\ 
10 & durch den trôst \textbf{zuo \textit{i}m er} vienc\\ 
 & den jungen, starken Parzival.\\ 
 & der begreif ouch in dô sunder twâl\\ 
 & unt zucte in ûz dem satel sîn\\ 
 & als eine garbe heberîn\\ 
15 & vaster in under \textbf{die arme} swanc,\\ 
 & mit im \textbf{von dem orse er} spranc\\ 
 & unde dructe in über eine ronen.\\ 
 & dô muose\textbf{r} schumpfentiure wonen,\\ 
 & der solher nôt niht was gewent.\\ 
20 & "dû gearnest, daz sich hât versent\\ 
 & disiu vrouwe von \textbf{dînem} zorne.\\ 
 & \textit{\textbf{nû}} bistû der verlorne,\\ 
 & dû\textbf{ne} lâzest si \textbf{dîne} hulde hân."\\ 
 & "dazne wirt sô \textbf{schiere} niht getân",\\ 
25 & sprach der herzoge Orillus,\\ 
 & "ich\textbf{ne} bin \textbf{doch} \textbf{niht betwungen} sus."\\ 
 & Parzival, der werde degen,\\ 
 & dructe in \textit{an sich}, da\textit{z b}luotes regen\\ 
 & spranc durch die barbiere.\\ 
30 & dô wart der vürste schiere\\ 
\end{tabular}
\scriptsize
\line(1,0){75} \newline
G I O L M Q R Z Fr21 \newline
\line(1,0){75} \newline
\textbf{1} \textit{Initiale} G M  \textbf{27} \textit{Überschrift:} Wie parczifal mit orilus streit R   $\cdot$ \textit{Initiale} I O L Q R Z Fr21  \newline
\line(1,0){75} \newline
\textbf{1} Dâ] Do Q  $\cdot$ scherpfe herte] harte scherffe M \textbf{2} werte] ferte Q \textbf{3} sînen] Deinen Q  $\cdot$ vor] von I  $\cdot$ ander] andern L (M) (Q) \textbf{4} Orillus] orilus I (O) (M) (Q) (R) (Z) (Fr21)  $\cdot$ de Lalander] delalander G Q Z Fr21 delander I der lalander O de lalandir M \textbf{5} sînem gelêrten site] sinen gelerten sitten L (Q) syme gelernden siten M sim gelertem site Fr21 \textbf{6} wol] vil O L Q R Fr21  $\cdot$ gestrite] gestriten Q \textbf{9} swiez] wie ez L (M) (Q) (R) Z  $\cdot$ dâ] do Q \textit{om.} R \textbf{10} zuo im er vienc] zoͮ zim er viench G er zvͦ im viench O (L) (Fr21) zcu yme ergevinc M er zuzim ging Q zu Im er do fieng R er zv im gevienc Z \textbf{11} \textit{Die Verse 265.11-12 fehlen} I   $\cdot$ den jungen starken] Der junge starke M  $\cdot$ Parzival] Parcifal O (L) (Z) (Fr21) Parzifal M partzifal Q parczifal R \textbf{12} in] den in R  $\cdot$ dô] [d*]: div G \textit{om.} O L Q R Fr21 da Z \textbf{13} zucte] zucht I (O) (Z) (Fr21)  $\cdot$ dem] den R \textbf{14} garbe] garben M graffe Q  $\cdot$ heberîn] hube er jn M herberin Z \textbf{15} under die arme] vnder den arm O L (Fr21) vnder den arme Q \textbf{16} von dem orse er] vnder daz ors er I er von dem orse O (L) (M) (Q) (Z) Fr21 er von den Rose R \textbf{17} dructe] truct I (Z) (Fr21) drucke M (Q)  $\cdot$ eine] einen L \textbf{18} dô] Da O Z Fr21 Des M  $\cdot$ muoser] muͯste L (M) (Q) (R) (Fr21) \textbf{19} was] ist I \textbf{20} dû] Da Fr21  $\cdot$ gearnest] garst I harrist M \textbf{21} disiu] Div O \textbf{22} nû bistû] des bistv G nu bistuz I (O) Nun bist R Dv bist nv Fr21 \textbf{23} dûne] Du R  $\cdot$ lâzest] lastest Q  $\cdot$ si] sy denne R  $\cdot$ dîne] dan I \textbf{24} dazne] Daz O (R) Da zu Q \textbf{25} Orillus] orilus I (O) M Q R Z (Fr21) \textbf{26} ichne] Jch O L Q R  $\cdot$ doch] noch I L Q \textbf{27} Parzival] Parzifal I (L) M ÷Arcifal O Parcifal Q Z Fr21 PArczifal R  $\cdot$ werde] chuͤne I \textbf{28} dructe] truct I (O) (Z) (Fr21)  $\cdot$ an sich daz bluotes] daz der bloͮtes G \textbf{29} durch] nuͯ >in< Q \textbf{30} dô] Da M Z  $\cdot$ wart] war Fr21 \newline
\end{minipage}
\hspace{0.5cm}
\begin{minipage}[t]{0.5\linewidth}
\small
\begin{center}*T
\end{center}
\begin{tabular}{rl}
 & \begin{large}D\end{large}â ergienc diu scharpfe herte:\\ 
 & ietweder vaste werte\\ 
 & sînen prîs vor dem ander.\\ 
 & \textbf{Duc} Orilus de Lalander\\ 
5 & streit nâch sînem gelêrten site.\\ 
 & ich wæne, ieman sô \textbf{vil} gestrite.\\ 
 & er hete kunst unde kraft,\\ 
 & des wart er dicke sigehaft\\ 
 & an maneger stat, swiez dâ ergienc.\\ 
10 & durch den trôst \textbf{er zuo zim} vienc\\ 
 & den jungen, starken Parcifal.\\ 
 & der begreif ouch in dô sunder twâl\\ 
 & unde zuhtin ûz dem satel sîn\\ 
 & als eine garbe heberîn\\ 
15 & vaste ern under\textbf{n arm} swanc,\\ 
 & mit im \textbf{er von dem orse} spranc\\ 
 & unde druhtin über einen ronen.\\ 
 & dô muose schumpfentiure wonen,\\ 
 & der solher nôt niht was gewent.\\ 
20 & "Dû gearnest, daz sich hât versent\\ 
 & dis\textit{iu} vrouwe von \textbf{disem} zorne.\\ 
 & \textbf{des} bist dû der verlorne,\\ 
 & dû lâze\textit{st} si \textbf{dîne} hulde hân."\\ 
 & "Daz enwirt sô \textbf{gâhes} niht getân",\\ 
25 & sprach der herzoge Orilus,\\ 
 & "i\textbf{ne} bin \textbf{noch} \textbf{niht betwungen} sus."\\ 
 & \begin{large}P\end{large}arcifal, der werde degen,\\ 
 & druhten an sich, daz bluotes regen\\ 
 & \textbf{im} spranc durch die barbiere.\\ 
30 & dô wart der vürste schiere\\ 
\end{tabular}
\scriptsize
\line(1,0){75} \newline
T U V W \newline
\line(1,0){75} \newline
\textbf{1} \textit{Initiale} T U V W  \textbf{4} \textit{Majuskel} T  \textbf{20} \textit{Majuskel} T  \textbf{24} \textit{Majuskel} T  \textbf{27} \textit{Initiale} T U W  \newline
\line(1,0){75} \newline
\textbf{1} Dâ] [Pa]: Da T DO V (W)  $\cdot$ scharpfe] [schar*]: scharpfe T \textbf{3} ander] andern U W [and*]: ander V \textbf{4} Duc] Duͦt U Auch W  $\cdot$ Lalander] lalandern W \textbf{5} gelêrten] gelertem V \textbf{6} ieman] daz nie man W  $\cdot$ vil] wol W \textbf{9} swiez] wie iz U (W)  $\cdot$ dâ] do U V \textit{om.} W \textbf{10} den] \textit{om.} U  $\cdot$ er zuo zim vienc] zu im er vinc U (V) er zuͦ im geuieng W \textbf{11} Parcifal] parzifal T V partzifaln W \textbf{12} ouch in dô] oͮch in V in auch W  $\cdot$ sunder twâl] sundertwaln W \textbf{13} zuhtin] zoch in U W \textbf{15} undern arm] vndern armen U vnder [*]: die arme V \textbf{16} von] ab W \textbf{18} muose] mvese T \textbf{19} der] Do U  $\cdot$ nôt] \textit{om.} W  $\cdot$ niht] \textit{om.} U [*]: nút V \textbf{20} gearnest daz] arnest des W \textbf{21} disiu] dise T Die U [D*]: Dise V  $\cdot$ disem] dime U V (W) \textbf{22} des] [*]: Nv V \textbf{23} dû lâzest] dv lazez T [D*]: Dv enlazest V \textbf{24} enwirt] wirt W  $\cdot$ gâhes] gahens U W \textbf{26} noch] \textit{om.} W \textbf{27} Parcifal] Parzifal T V PArtzifal W \textbf{28} druhten] Durck in W \textbf{29} im] Vz U Hin V \textit{om.} W  $\cdot$ barbiere] baniere U \newline
\end{minipage}
\end{table}
\end{document}
