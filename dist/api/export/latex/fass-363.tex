\documentclass[8pt,a4paper,notitlepage]{article}
\usepackage{fullpage}
\usepackage{ulem}
\usepackage{xltxtra}
\usepackage{datetime}
\renewcommand{\dateseparator}{.}
\dmyyyydate
\usepackage{fancyhdr}
\usepackage{ifthen}
\pagestyle{fancy}
\fancyhf{}
\renewcommand{\headrulewidth}{0pt}
\fancyfoot[L]{\ifthenelse{\value{page}=1}{\today, \currenttime{} Uhr}{}}
\begin{document}
\begin{table}[ht]
\begin{minipage}[t]{0.5\linewidth}
\small
\begin{center}*D
\end{center}
\begin{tabular}{rl}
\textbf{363} & \begin{large}D\end{large}az spilwîp zem vürsten sprach\\ 
 & al \textbf{daz} sîn tohter dar verjach.\\ 
 & \textbf{swer} ie urliuges \textbf{pflac},\\ 
 & dem was vil nôt, ob er bejac\\ 
5 & \textbf{m\textit{ö}hte an rîcher koste} hân.\\ 
 & Lyppauten, den getriwen man,\\ 
 & überlesten soldiere,\\ 
 & daz er gedâhte schiere:\\ 
 & "ich sol \textbf{daz} guot gewinnen\\ 
10 & mit zorne oder \textbf{aber} mit minnen."\\ 
 & die nâchreise er niht vermeit.\\ 
 & Scherules im widerreit.\\ 
 & \textbf{er} vrâgte, war im wære sô gâch.\\ 
 & "ich rîte \textbf{dem} \textbf{trügenære} nâch.\\ 
15 & von dem sagt man mir mære,\\ 
 & \textbf{ez} sî ein \textbf{valschære}."\\ 
 & Unschuldec was hêr Gawan;\\ 
 & ez \textbf{en}\textbf{hete} niht wan diu ors getân\\ 
 & unt ander, daz er vuorte.\\ 
20 & Scherulesen lachen ruorte.\\ 
 & \textbf{Dô sprach er}: "hêrre, ir sît betrogen.\\ 
 & swer\textbf{z} iu sagete, \textbf{er} hât gelogen,\\ 
 & ez \textbf{wære} magt, \textbf{man} oder wîp.\\ 
 & unschuldec ist mînes gastes lîp.\\ 
25 & ir sult in anders prîsen,\\ 
 & er \textbf{en}gewan nie münzîsen.\\ 
 & welt ir der rehten mære losen,\\ 
 & sîn lîp getruoc nie wehselpfosen.\\ 
 & seht sîne gebære \textbf{unt} hœret sîniu wort.\\ 
30 & in mîme hûs liez ich in dort.\\ 
\end{tabular}
\scriptsize
\line(1,0){75} \newline
D Fr3 Fr4 \newline
\line(1,0){75} \newline
\textbf{1} \textit{Initiale} D Fr4  \textbf{17} \textit{Initiale} Fr4   $\cdot$ \textit{Majuskel} D  \textbf{21} \textit{Majuskel} D  \newline
\line(1,0){75} \newline
\textbf{4} bejac] beiage Fr4 \textbf{5} an richer coste mochte han Fr4  $\cdot$ möhte] mohte D \textbf{6} Lyppauten] Lyppaoten D lippaothin Fr4 \textbf{10} aber] \textit{om.} Fr4 \textbf{12} Scherules] Scervles D tserules Fr4 \textbf{17} hêr] er Fr4 \textbf{18} enhete] en hetin Fr4 \textbf{19} ander] anders Fr4 \textbf{20} Scherulesen] Scervlesn D tserulesin Fr4 \textbf{22} er] der Fr4 \textbf{29} sîne] sîn Fr3  $\cdot$ gebære] geberde Fr4 \textbf{30} mîme hûs] mînen hof Fr3 \newline
\end{minipage}
\hspace{0.5cm}
\begin{minipage}[t]{0.5\linewidth}
\small
\begin{center}*m
\end{center}
\begin{tabular}{rl}
 & daz spilwîp zem vürsten sprach\\ 
 & al \textbf{daz} sî\textit{n t}ohter dar verjach.\\ 
 & \textbf{\begin{large}D\end{large}er} ie urliuges \textbf{gepflac},\\ 
 & dem was vil nôt, ob er bejac\\ 
5 & \textbf{m\textit{ö}hte an rîcher koste} hân.\\ 
 & Lipp\textit{ou}ten, den getriuwen man,\\ 
 & überl\textit{e}sten soldiere,\\ 
 & daz er gedâhte schiere:\\ 
 & "ich sol \textbf{diz} guot gewinnen\\ 
10 & mit zorne oder mit minnen."\\ 
 & die nâchreise er niht vermeit.\\ 
 & Scherules im widerreit.\\ 
 & \textbf{er} vrâgete \textbf{in}, war \textit{im} wær sô gâch.\\ 
 & "ich rîte \textbf{einem} \textbf{trügenære} nâch.\\ 
15 & von dem sagt man mir mære,\\ 
 & \textbf{er} sî ein \textbf{valschære}."\\ 
 & unschuldic was hêr Gawan;\\ 
 & ez \textbf{heten} niwan diu ros getân\\ 
 & und ander, daz er vuorte.\\ 
20 & Scher\textit{u}lesen lachen ruorte.\\ 
 & \textbf{er sprach}: "hêrre, ir sît betrogen.\\ 
 & wer \textbf{daz} iu sagete, \textbf{er} hât gelogen,\\ 
 & ez \textbf{sî} maget, \textbf{man} oder wîp.\\ 
 & unschuldic ist mîn\textit{es} gastes lîp.\\ 
25 & ir sult in anders prîsen,\\ 
 & er gewan nie \textit{m}ünzîsen.\\ 
 & welt ir der rehte\textit{n} mære losen,\\ 
 & sîn lîp getruoc nie wehselpfosen.\\ 
 & seht sîne gebærde, hœret sîniu wort.\\ 
30 & in mînem hûse liez ich in dort.\\ 
\end{tabular}
\scriptsize
\line(1,0){75} \newline
m n o \newline
\line(1,0){75} \newline
\textbf{3} \textit{Initiale} m   $\cdot$ \textit{Capitulumzeichen} n  \newline
\line(1,0){75} \newline
\textbf{2} sîn tohter] sin ros vnd tohter m  $\cdot$ dar] do n o \textbf{4} ob] ob ob o \textbf{5} möhte] Mohte m (o) \textbf{6} Lippouten] Lippoatten m Lippaoten n Lippooten o \textbf{7} überlesten] V̂berleczsten m Vber lester o \textbf{8} gedâhte] gedechte n gedeckte o \textbf{9} gewinnen] gewunnen o \textbf{10} oder] vnd n \textbf{12} Scherules] Scerules m Sterules n Strengels o \textbf{13} im] vmbe m (n) \textbf{14} einem] eẏner o \textbf{15} sagt] sagete n (o) \textbf{19} daz] dis o \textbf{20} Scherulesen] Scerelesen m Sterulesen n Stern lesen o \textbf{22} daz iu] úch das n (o)  $\cdot$ sagete] seit n o \textbf{24} mînes] min m (o) \textbf{26} münzîsen] in vnzisen m man mvnzisen n manzisen o \textbf{27} rehten] rehte m (o) \textbf{28} getruoc] truͯg n  $\cdot$ nie] \textit{om.} n o  $\cdot$ wehselpfosen] wehsels zuͯ erkosen n wessels kosen o \textbf{29} sîne] sin n o \textbf{30} hûse] [libe]: huse m \newline
\end{minipage}
\end{table}
\newpage
\begin{table}[ht]
\begin{minipage}[t]{0.5\linewidth}
\small
\begin{center}*G
\end{center}
\begin{tabular}{rl}
 & daz spilwîp zem vürsten sprach\\ 
 & al \textbf{des} sîn tohter dar verjach.\\ 
 & \textbf{swer} ie urliuges \textbf{pflac},\\ 
 & dem was vil nôt, ob er bejac\\ 
5 & \textbf{an rîcher koste m\textit{ö}hte} hân.\\ 
 & Libauten, den getriwen man,\\ 
 & überlesten soldiere,\\ 
 & daz er gedâhte schiere:\\ 
 & "ich sol \textbf{diz} guot gewinnen\\ 
10 & mit zorne ode\textit{r m}it minnen."\\ 
 & die nâchreiser niht vermeit.\\ 
 & Tscherules im widerreit\\ 
 & \textbf{unde} vrâgte \textit{\textbf{in}}, war im wære sô gâch.\\ 
 & "ich rîte \textbf{einem} \textbf{triegære} nâch.\\ 
15 & von dem saget man mir mære,\\ 
 & \textbf{er} sî ein \textbf{triegære}."\\ 
 & unschuldic was hêr Gawan;\\ 
 & ez \textbf{heten} niwan diu ors getân\\ 
 & unde ander, daz er vuorte.\\ 
20 & Tscherulesen lachen ruorte.\\ 
 & \textbf{er sprach}: "hêrre, ir sît betrogen.\\ 
 & swer\textbf{z} iu saget, \textbf{der} hât gelogen,\\ 
 & ez \textbf{sî} maget, \textbf{man} oder wîp.\\ 
 & unschuldic ist mînes gastes lîp.\\ 
25 & \begin{large}I\end{large}r sult in anders brîsen,\\ 
 & er gewan nie münzîsen.\\ 
 & welt ir der rehten \textit{mære} losen,\\ 
 & sîn lîp getruoc nie wehselpfosen.\\ 
 & seht sîne gebærde, hœret sîniu wort.\\ 
30 & in mînem hûse \textit{l}i\textit{ez ich} in dort.\\ 
\end{tabular}
\scriptsize
\line(1,0){75} \newline
G I O L M Q R Z Fr38 \newline
\line(1,0){75} \newline
\textbf{1} \textit{Initiale} I O L Q Z Fr38   $\cdot$ \textit{Capitulumzeichen} R  \textbf{17} \textit{Initiale} I  \textbf{25} \textit{Initiale} G  \newline
\line(1,0){75} \newline
\textbf{1} daz] ÷Az O \textbf{2} al] als I (Q) (Z) (Fr38)  $\cdot$ dar] do Q R \textbf{3} swer] Wer L M Q R \textbf{4} ob] daz Z  $\cdot$ bejac] [mac]: maht I biwac M (Z) \textbf{5} möhte] mohte G I O L (M) (Q) Z (Fr38) \textbf{6} Libauten] Lybavten O L Fr38 Lẏbanten R Lybarten Z  $\cdot$ getriwen] trúwen R \textbf{7} überlesten] ob er loste I \textbf{8} er gedâhte] ergidechte M  $\cdot$ schiere] sicher R \textbf{9} Wie er dizze gewinne O \textbf{10} ab im im mit zorn oder mit minnen I  $\cdot$ oder] olde abe G aber Z  $\cdot$ minnen] minne O \textbf{12} Tscherules] Scrules I Tschervles O Tshervles L Scherulus M Terulus R Tschervl:s Fr38 \textbf{13} unde] er I  $\cdot$ vrâgte] fragt I O Q Z  $\cdot$ in] \textit{om.} G  $\cdot$ war im] \textit{om.} L \textbf{14} rîte] reit O Q  $\cdot$ triegære] valschære O (L) (M) (Q) (R) (Fr38) \textbf{15} saget] sagite M \textbf{16} er] ez I  $\cdot$ triegære] valskere I (Z) :::ere Fr38 \textbf{17} hêr Gawan] irgawan M \textbf{18} heten] enheten L (M) Z  $\cdot$ niwan] nymant Q num an R  $\cdot$ diu] sinev I \textbf{19} ander] anders R  $\cdot$ er] er da Fr38 \textbf{20} Tscherulesen] Scrulesen I Tschervlesen O Fr38 [Tshervlesnlachen]: Tshervlesn lachen L Scherulesen M Scherelesen R  $\cdot$ lachen] \textit{om.} O \textbf{21} er sprach] \textit{om.} R \textbf{22} swerz iu] Werz uͯch L (M) (R) Wer euchs Q  $\cdot$ saget] sagite M (R)  $\cdot$ der] er I O  $\cdot$ hât] hatz O (Q) \textbf{23} sî] \textit{om.} Q  $\cdot$ maget man] man magt I Z man L (M) \textbf{24} mînes] min Z mine Fr38 \textbf{26} er gewan] ern wart I Er engewan L (M) (Z) (Fr38) \textbf{27} rehten mære] rehten warheit G rechte mere Q mere recht R \textbf{28} getruoc] trvch O  $\cdot$ wehselpfosen] valshen phosen I \textbf{29} seht] Horet L  $\cdot$ hœret] vnde hort O  $\cdot$ sîne gebærde] sin geberde I (O) (Q) Z sin gebare L sine geberde R \textbf{30} Er ist in minem huse dort R  $\cdot$ mînem] minen I  $\cdot$ liez ich] ih liez G \newline
\end{minipage}
\hspace{0.5cm}
\begin{minipage}[t]{0.5\linewidth}
\small
\begin{center}*T
\end{center}
\begin{tabular}{rl}
 & \multicolumn{1}{l}{ - - - }\\ 
 & \multicolumn{1}{l}{ - - - }\\ 
 & \textbf{\begin{large}D\end{large}er} ie urliuges \textbf{gepflac},\\ 
 & dem was vil nôt, ob er bejac\\ 
5 & \textbf{an rîcher koste m\textit{ö}hte} hân.\\ 
 & Lybaut, den getriuwen man,\\ 
 & überlesten soldiere,\\ 
 & daz er gedâhte schiere:\\ 
 & "ich sol \textbf{diz} guot gewinnen\\ 
10 & mit zorne oder mit minnen."\\ 
 & Die nâchreise er niht vermeit.\\ 
 & Tscherules im widerreit\\ 
 & \textbf{unde} vrâgete \textbf{in}, war im wære sô gâch.\\ 
 & "Ich rîte \textbf{einem} \textbf{trieger} nâch.\\ 
15 & von dem saget man mir mære,\\ 
 & \textbf{er} sî ein \textbf{valschære}."\\ 
 & Unschuldic was hêr Gawan;\\ 
 & ez \textbf{heten} niht wan diu ors getân\\ 
 & unde ander, daz er vuorte.\\ 
20 & Tscherules \textbf{ein} lachen ruo\textit{r}te.\\ 
 & \textbf{er sprach}: "hêrre, ir sît betrogen.\\ 
 & swer\textbf{z} iu saget, \textbf{er} hât gelogen,\\ 
 & ez \textbf{sî} maget oder wîp.\\ 
 & unschuldic ist mînes gastes lîp.\\ 
25 & ir sult in anders prîsen,\\ 
 & er gewan nie münzîsen.\\ 
 & welt ir der rehten mære losen,\\ 
 & sîn lîp getruoc nie wehselpfosen.\\ 
 & seht sîne geberde, hœrt sîniu wort.\\ 
30 & in mînem hûse liez ich in dort.\\ 
\end{tabular}
\scriptsize
\line(1,0){75} \newline
T V W \newline
\line(1,0){75} \newline
\textbf{2} \textit{Initiale} V  \textbf{3} \textit{Initiale} T W  \textbf{11} \textit{Majuskel} T  \textbf{14} \textit{Majuskel} T  \textbf{17} \textit{Majuskel} T  \newline
\line(1,0){75} \newline
\textbf{1} \textit{Die Verse 363.1-2 fehlen} T W   $\cdot$ Daz spilwip zvͦme fv́rsten sprach V \textbf{2} Alles dez sin tohter do veriach V \textbf{3} Der] [*]: Swer V WEr W  $\cdot$ gepflac] pflag W \textbf{4} bejac] úberwag W \textbf{5} an] Was er an W  $\cdot$ möhte] mohte T (W) \textbf{6} Lybaut] [L*aoten]: Lẏppaoten V Lybout W  $\cdot$ den getriuwen] der getreúwe W \textbf{7} überlesten] Vber laster W \textbf{10} oder] oder [*]: aber V \textbf{11} nâchreise] nachtraise W \textbf{12} Tscherules] Tscerules T Schervles V De scherules W \textbf{13} vrâgete] fraget V (W)  $\cdot$ sô] \textit{om.} W \textbf{15} mir] \textit{om.} W \textbf{17} Gawan] Gawân T \textbf{18} heten] enhetten V \textbf{19} ander] anders V \textbf{20} Tscherules] Tscerules T Schervles V Descherules W  $\cdot$ ruorte] ruͦte T \textbf{22} swerz iu] Wer eúchs W  $\cdot$ er] der V W \textbf{23} maget] maget man V W \textbf{27} der] die W \textbf{28} wehselpfosen] wehselsphosen T \textbf{29} Sehent sein geberd vnd losent sein wort W \newline
\end{minipage}
\end{table}
\end{document}
