\documentclass[8pt,a4paper,notitlepage]{article}
\usepackage{fullpage}
\usepackage{ulem}
\usepackage{xltxtra}
\usepackage{datetime}
\renewcommand{\dateseparator}{.}
\dmyyyydate
\usepackage{fancyhdr}
\usepackage{ifthen}
\pagestyle{fancy}
\fancyhf{}
\renewcommand{\headrulewidth}{0pt}
\fancyfoot[L]{\ifthenelse{\value{page}=1}{\today, \currenttime{} Uhr}{}}
\begin{document}
\begin{table}[ht]
\begin{minipage}[t]{0.5\linewidth}
\small
\begin{center}*D
\end{center}
\begin{tabular}{rl}
\textbf{242} & \begin{large}I\end{large}ch \textbf{wil} iu \textbf{doch baz} bediuten\\ 
 & von disen \textbf{jâmerbæren} liuten,\\ 
 & \textbf{dar} \textbf{kom geriten} Parzival:\\ 
 & man sach dâ selten vreuden schal,\\ 
5 & ez wære bûhurt oder tanz.\\ 
 & ir klagendiu stæte was sô ganz,\\ 
 & si\textbf{ne} kêrten sich an \textbf{schimpfen} niht.\\ 
 & swâ man noch \textbf{min} \textbf{volkes} siht,\\ 
 & den tuot etswenne vreude wol;\\ 
10 & \textbf{dort} wâren die winkel alle vol\\ 
 & \textbf{unt ouch ze hove, dâ} man \textbf{si} sach.\\ 
 & der wirt ze sîme gaste sprach:\\ 
 & "Ich wæne, man iu gebettet hât.\\ 
 & sît ir müede, sô ist mîn rât,\\ 
15 & daz ir gêt, \textbf{leit iuch} slâfen."\\ 
 & nû solt ich schrîen 'wâfen'\\ 
 & \textbf{umb} ir scheiden, daz si tuont.\\ 
 & \textbf{ez} \textbf{wirt} grôz \textbf{schade} \textbf{in} beiden kunt.\\ 
 & Vo\textbf{me spanbette} trat\\ 
20 & ûf\textbf{en} teppech an eine stat\\ 
 & Parzival, der wol geslaht.\\ 
 & \textbf{der wirt bôt im} guote naht.\\ 
 & diu rîterschaft dô gar ûf spranc.\\ 
 & ein teil ir im dar nâher dranc.\\ 
25 & \textbf{dô} vuorten \textbf{si} den jungen man\\ 
 & in eine kemenâten sân,\\ 
 & diu was \textbf{wol} gehêret,\\ 
 & mit einem bette geêret,\\ 
 & daz mich mîn armuot \textbf{immer} müet,\\ 
30 & sît \textbf{d}erde \textbf{al}sölhe rîcheit blüet.\\ 
\end{tabular}
\scriptsize
\line(1,0){75} \newline
D \newline
\line(1,0){75} \newline
\textbf{1} \textit{Initiale} D  \textbf{13} \textit{Majuskel} D  \textbf{19} \textit{Majuskel} D  \newline
\line(1,0){75} \newline
\newline
\end{minipage}
\hspace{0.5cm}
\begin{minipage}[t]{0.5\linewidth}
\small
\begin{center}*m
\end{center}
\begin{tabular}{rl}
 & ich \textbf{wil} iu \textbf{doch baz} bediuten\\ 
 & von disen \textbf{jâmerbæren} liuten,\\ 
 & \textbf{ze de\textit{n}} \textbf{kam geriten} Parcifal:\\ 
 & man sach d\textit{â} selten vröuden schal,\\ 
5 & ez wære bûhurt oder tanz.\\ 
 & ir klagendiu stæte was sô ganz,\\ 
 & si kêrten sich an \textbf{schimpfen} niht.\\ 
 & wâ man noch \textbf{minre} \textbf{volkes} siht,\\ 
 & den t\textit{u}ot etwenne vröude wol;\\ 
10 & \textbf{dort} wâren die winkel alle vol\\ 
 & \textbf{und ze hove, d\textit{â}} man \textbf{si} sach.\\ 
 & der wirt ze sîne\textit{m} gaste sprach:\\ 
 & "ich wæne, man iu gebettet hât.\\ 
 & sît ir müede, sô ist mîn rât,\\ 
15 & daz ir gêt \textbf{und} \textbf{leget iuch} slâfen."\\ 
 & nû solt ich sch\textit{r}îe\textit{n} 'wâfen'\\ 
 & \textbf{umb} ir scheiden, daz si tuont.\\ 
 & \textbf{ez} \textbf{wart} grôz \textbf{schade} beiden kunt.\\ 
 & von\textbf{\textit{m}e spanbette} trat\\ 
20 & ûf \textbf{eine\textit{n}} teppich an eine stat\\ 
 & Parcifal, der wol geslah\textit{t}.\\ 
 & \textbf{der wirt bôt ime} guote nah\textit{t}.\\ 
 & diu ritterschaft dô gar ûf spranc.\\ 
 & ein teil ir ime dâ nâher dranc.\\ 
25 & \textbf{dô} vuorten \textbf{si} den jungen man\\ 
 & in eine kemenâten sân,\\ 
 & diu was \textbf{alsô} gehêr\textit{e}t,\\ 
 & mit einem bette geêret,\\ 
 & daz mich mîn armuot \textbf{\textit{i}mer} \textit{mü}et,\\ 
30 & sît \textbf{diu} erde \textbf{al}soliche \textit{rîcheit} blüet.\\ 
\end{tabular}
\scriptsize
\line(1,0){75} \newline
m n o Fr69 \newline
\line(1,0){75} \newline
\newline
\line(1,0){75} \newline
\textbf{1} doch] \textit{om.} n \textbf{3} den] der m \textbf{4} dâ] do m n o \textbf{7} schimpfen] schẏmpfes o \textbf{9} tuot] tot m \textbf{10} dort] Den dort n  $\cdot$ alle] beide n o \textbf{11} und] Vnd ouch n (o) (Fr69)  $\cdot$ dâ] do m n o \textbf{12} sînem] sinen m \textbf{15} leget iuch] ligent n o \textbf{16} schrîen] schier m \textbf{17} umb ir] V́ber Fr69 \textbf{18} ez] Er o  $\cdot$ wart] wirt Fr69  $\cdot$ beiden] in beiden n o (Fr69) \textbf{19} vonme] Vonne m  $\cdot$ trat] er do trat Fr69 \textbf{20} einen] einem m (o) ein ein Fr69 \textbf{21} geslaht] geschlahte m \textbf{22} naht] nahte m \textbf{24} dâ] do n o \textbf{25} vuorten] fuͯrte o \textbf{27} gehêret] geherert m \textbf{29} imer] nẏmer m  $\cdot$ müet] ruͯwet m (n) (o) \textbf{30} alsoliche] also solliche n  $\cdot$ rîcheit blüet] [bouwet]: bluwet m \newline
\end{minipage}
\end{table}
\newpage
\begin{table}[ht]
\begin{minipage}[t]{0.5\linewidth}
\small
\begin{center}*G
\end{center}
\begin{tabular}{rl}
 & ich \textbf{muoz} iu \textbf{mêre} bediuten\\ 
 & von disen \textbf{jâmerbernden} liuten,\\ 
 & \textbf{dar} \textbf{kom geriten} Parzival:\\ 
 & man sach dâ selten vröuden schal,\\ 
5 & ez wære bûhurt oder tanz.\\ 
 & ir klagendiu stæte was sô ganz,\\ 
 & si\textbf{ne} kêrten sich an \textbf{schimpfen} niht.\\ 
 & swâ man noch \textbf{minner} \textbf{volkes} siht,\\ 
 & \begin{large}D\end{large}en tuot etwenne vröude wol;\\ 
10 & \textbf{dort} wâren die winkel alle vol,\\ 
 & \textbf{ein teil} man \textbf{ir} \textbf{ze hove} sach.\\ 
 & der wirt ze sînem gaste sprach:\\ 
 & "ich wæne, man iu gebettet hât.\\ 
 & sît ir müede, sô ist mîn rât,\\ 
15 & daz ir gêt, \textbf{leit iuch} slâfen."\\ 
 & nû solt ich schrîen 'wâfen'\\ 
 & \textbf{von} ir scheiden, daz si tuont.\\ 
 & \textbf{des} \textbf{wirt} grôz \textbf{schade} \textbf{in} beiden kunt.\\ 
 & von \textbf{dem spanbette} trat\\ 
20 & ûf \textbf{den} tepch an eine stat\\ 
 & Parzival, der wol geslaht.\\ 
 & \textbf{der wirt bôt im} guote naht.\\ 
 & diu rîterschaft dô gar ûf spranc.\\ 
 & ein teil ir im dar nâher dranc.\\ 
25 & \textbf{dô} vuorten \textbf{si} den jungen man\\ 
 & in eine kemenâten sân,\\ 
 & diu was \textbf{alsô} gehêret,\\ 
 & mit einem bette geêret,\\ 
 & daz mich mîn armuot \textbf{iemer} müet,\\ 
30 & sît \textbf{diu} erde solche rîcheit blüet.\\ 
\end{tabular}
\scriptsize
\line(1,0){75} \newline
G I O L M Q R Z Fr54 \newline
\line(1,0){75} \newline
\textbf{1} \textit{Initiale} I L M Q Z  \textbf{9} \textit{Überschrift:} Hie siczet die kúngin ze tische mit den Junkfrowen vnd mit dem gesind Hie kam parczifal an ein sew vnd fand den wirt in einem schiff siczen der den gral in hett den fragt er vmb herberg der wist in zu der burg daruff der gral was mit dem gesind dar kam garczifal vnd ward wol empfangen vnd sach gros richeit dar In vnd menig wunderlich geschicht Da ist der sal da der wirt mit sinen Rittern In sas mit hundert tischen R   $\cdot$ \textit{Initiale} G R  \textbf{19} \textit{Initiale} I  \textbf{23} \textit{Initiale} M  \textbf{29} \textit{Initiale} O  \newline
\line(1,0){75} \newline
\textbf{1} iu] \textit{om.} L  $\cdot$ mêre] iamer I doch baz Z  $\cdot$ bediuten] tuten I \textbf{2} von] An Z  $\cdot$ disen] disem Q  $\cdot$ jâmerbernden] iamer I iamerbarn O (L) (M) (Q) (Z) Jamerhafftten R \textbf{3} Parzival] Parzifal I (M) Parcifal O (L) (Z) partzifal Q parczifal R \textbf{4} man] Nan Q  $\cdot$ sach] shach I  $\cdot$ dâ] do Q  $\cdot$ schal] mal L \textbf{6} klagendiu] chlangnde O \textbf{7} sine] Si O (L) (R) \textbf{8} swâ] Wo L (M) Q (R)  $\cdot$ volkes] volch I folke R \textbf{9} Den] De R  $\cdot$ vröude] ein froͯde R \textbf{10} die winkel alle] die winchel iamers L alle winkel Z \textbf{11} ein teil] Vnd wo L Vnd auch Q (R) (Z)  $\cdot$ man ir ze hove] man sie zuͯ hoffe L zu hore do man sie Q zu houe da man sy R (Z)  $\cdot$ sach] [sach]: ersach O esach Z \textbf{14} müede] mute Q \textbf{15} leit] vnde legt I legen R \textbf{17} von] Vmbe L  $\cdot$ tuont] nv tvͯnt L \textbf{18} des] Wan es L Das M  $\cdot$ wirt grôz schade] zuͯ schaden L \textbf{20} den] einen I daz L deme M ein Q R \textbf{21} Parzival] parzifal I Parcifal O L Z Parziual M Partzifal Q Parczifal R \textbf{22} bôt] der bot O  $\cdot$ im] im do I \textbf{23} dô] da O M Z \textit{om.} Q \textbf{24} ein teil ir] Einer O (M) Jegelicher L Ein teil Z  $\cdot$ im] hie I \textit{om.} L  $\cdot$ dar] do Q  $\cdot$ dranc] tranck Q \textbf{25} dô] Vnd L Da Z  $\cdot$ vuorten] wisten I (O) (Q) (R) Z wusten M  $\cdot$ si] \textit{om.} L \textbf{26} sân] dan R \textbf{28} geêret] kert O geerret R gertit Fr54 \textbf{29} daz] ÷az O  $\cdot$ iemer] diche I \textbf{30} erde] rede R  $\cdot$ solche] al solich Z asol ich Fr54  $\cdot$ rîcheit] richen Fr54 \newline
\end{minipage}
\hspace{0.5cm}
\begin{minipage}[t]{0.5\linewidth}
\small
\begin{center}*T
\end{center}
\begin{tabular}{rl}
 & Ich \textbf{wil} iu \textbf{noch baz} betiuten\\ 
 & von disen \textbf{jâmerbær\textit{e}n} liuten,\\ 
 & \textbf{dar} \textbf{geriten kom} Parcifal:\\ 
 & man sach dâ selten vröuden schal,\\ 
5 & ez wære bûhurt oder tanz.\\ 
 & ir klagend\textit{iu} stæte was sô ganz,\\ 
 & si kêrten sich an \textbf{schimpf} niht.\\ 
 & swâ man noch \textbf{minre} \textbf{liute} siht,\\ 
 & den tuot \textbf{ouch} etswenne vröude wol;\\ 
10 & \textbf{dâ} wâren die winkele alle vol\\ 
 & \textbf{unde ouch ze hove, swâ} man \textbf{si} sach.\\ 
 & Der wirt ze sînem gaste sprach:\\ 
 & "ich wæne, man iu gebettet hât.\\ 
 & sît ir müede, sô ist mîn rât,\\ 
15 & daz ir gêt \textbf{iuch legen} slâfen."\\ 
 & Nû soltich schrîen 'wâfen'\\ 
 & \textbf{umb}ir scheiden, daz si tuont.\\ 
 & \textbf{des} \textbf{wirt} grôz \textbf{trûren} \textbf{in} beiden kunt.\\ 
 & von \textbf{der hertstat} \textbf{er} trat\\ 
20 & ûf \textbf{einen} tepich an eine stat,\\ 
 & Parcifal, der wol geslaht,\\ 
 & \textbf{bôt dem wirte} guote naht.\\ 
 & \begin{large}D\end{large}iu rîterschaft dô gar ûf spranc.\\ 
 & ein \textit{teil} ir im dar nâher dranc\\ 
25 & \textbf{unde} vuorten den jungen man\\ 
 & in eine kemenâte sân,\\ 
 & diu was \textbf{alsô} gehêret,\\ 
 & mit einem bette geêret,\\ 
 & daz mich mîn armuot müet,\\ 
30 & sît \textbf{di\textit{u}} erde \textbf{al}solhe rîcheit blüet.\\ 
\end{tabular}
\scriptsize
\line(1,0){75} \newline
T U V W \newline
\line(1,0){75} \newline
\textbf{1} \textit{Initiale} W  · Majuskel T  \textbf{12} \textit{Majuskel} T  \textbf{16} \textit{Majuskel} T  \textbf{23} \textit{Initiale} T  \newline
\line(1,0){75} \newline
\textbf{1} wil] wils V  $\cdot$ noch] [*]: Doch V \textit{om.} W \textbf{2} jâmerbæren] iamerbêreren T iamerlichen U \textbf{3} dar] [*]: Zvͦ den V  $\cdot$ geriten kom] kam geritten W  $\cdot$ Parcifal] Parzifal T U (V) partzifal W \textbf{4} dâ] do V W \textbf{6} klagendiu] clagende T klage W \textbf{7} kêrten] in kerten U  $\cdot$ schimpf] schimpffen W \textbf{8} swâ] Wa U (W)  $\cdot$ minre] mine U  $\cdot$ liute] [*]: volkez V \textbf{9} tuot] truͦc U  $\cdot$ ouch] \textit{om.} U V W \textbf{10} dâ] Do U [D*]: Dort V Dort W  $\cdot$ alle] iamers V \textbf{11} hove swâ] houe wa U (W) [*]: hove swa  V \textbf{15} iuch] îv T  $\cdot$ slâfen] slagen U \textbf{17} tuont] thuͦnt hie W \textbf{18} Das wirt schad ir beuindet wol wie W  $\cdot$ trûren] \textit{om.} U schade V \textbf{19} der hertstat] der herter [st*]: stat U [den*]: dem spanbete V dem wirte W  $\cdot$ trat] do trat W \textbf{20} einen] ein U V \textbf{21} Parcifal] parzifal T (U) (V) Partzifal W  $\cdot$ der] \textit{om.} W \textbf{22} bôt dem wirte] Der bot dem wirte U W [D*]: Der wurt bot im V \textbf{23} dô] \textit{om.} W \textbf{24} teil] \textit{om.} T  $\cdot$ ir] \textit{om.} W \textbf{25} unde vuorten] [*]: Do fuͦrten sv́ V Do fuͦrten sy W \textbf{26} kemenâte] kemmenaten W \textbf{29} müet] ymer muͤt W \textbf{30} diu] die T  $\cdot$ erde] erden U [*de]: rede V \newline
\end{minipage}
\end{table}
\end{document}
