\documentclass[8pt,a4paper,notitlepage]{article}
\usepackage{fullpage}
\usepackage{ulem}
\usepackage{xltxtra}
\usepackage{datetime}
\renewcommand{\dateseparator}{.}
\dmyyyydate
\usepackage{fancyhdr}
\usepackage{ifthen}
\pagestyle{fancy}
\fancyhf{}
\renewcommand{\headrulewidth}{0pt}
\fancyfoot[L]{\ifthenelse{\value{page}=1}{\today, \currenttime{} Uhr}{}}
\begin{document}
\begin{table}[ht]
\begin{minipage}[t]{0.5\linewidth}
\small
\begin{center}*D
\end{center}
\begin{tabular}{rl}
\textbf{63} & vil schilde sach er schînen.\\ 
 & die hellen pusînen\\ 
 & \textbf{mit krache vor im} gâben dôz.\\ 
 & von \textbf{würfen unt mit slegen} grôz\\ 
5 & zwêne tambûre gâben schal.\\ 
 & der galm über al die stat erhal.\\ 
 & der dôn iedoch gemischet wart\\ 
 & mit floitieren \textbf{an der} vart.\\ 
 & eine reisenote si bliesen.\\ 
10 & nû sulen wir niht verliesen,\\ 
 & wie ir hêrre komen sî.\\ 
 & dem riten videlære bî.\\ 
 & Dô leite der degen wert\\ 
 & \textbf{ein} bein vür sich ûfez pfert,\\ 
15 & zwêne stivâle über \textbf{blôziu} bein.\\ 
 & sî\textit{n} munt als ein rubîn schein\\ 
 & \textbf{von} \textbf{der} rœte, als ob er brünne.\\ 
 & \textbf{der} was dicke \textbf{unt} niht ze dünne.\\ 
 & sîn lîp was allenthalben clâr.\\ 
20 & lieht reideloht \textbf{was im} \textbf{sîn} hâr,\\ 
 & \textbf{swâ} man \textbf{daz} vor dem huote sach.\\ 
 & \textbf{der} was ein tiwer houbetdach.\\ 
 & grüene samît was der mandel sîn.\\ 
 & ein zobel dâ vor gap \textbf{swarzen} schîn\\ 
25 & ob einem hemde, daz was blanc.\\ 
 & \textbf{von schouwen wart dâ} grôz gedranc.\\ 
 & \begin{large}V\end{large}il dicke \textbf{al dâ} gevrâget wart,\\ 
 & wer wære der \textbf{ritter} âne bart,\\ 
 & der \textbf{vuorte} \textbf{alsölhe} rîcheit.\\ 
30 & \textbf{vil} schiere wart daz mære breit.\\ 
\end{tabular}
\scriptsize
\line(1,0){75} \newline
D Fr9 \newline
\line(1,0){75} \newline
\textbf{13} \textit{Majuskel} D  \textbf{27} \textit{Initiale} D  \newline
\line(1,0){75} \newline
\textbf{10} sulen] ne sule Fr9 \textbf{11} ir] der Fr9 \textbf{16} sîn] si D  $\cdot$ als] im als Fr9  $\cdot$ rubîn] ruͦbin Fr9 \textbf{17} von] Vuͦr Fr9 \textbf{18} der] Her Fr9 \textbf{21} dem] der Fr9 \textbf{24} vor] vuͦre Fr9 \textbf{25} ob] Vber Fr9 \textbf{29} vuorte alsölhe] da vuͦrte sulche Fr9 \newline
\end{minipage}
\hspace{0.5cm}
\begin{minipage}[t]{0.5\linewidth}
\small
\begin{center}*m
\end{center}
\begin{tabular}{rl}
 & vil schilte sach er schînen.\\ 
 & die hellen busînen\\ 
 & \textbf{mit krachen vor ime} gâben d\textit{ô}z.\\ 
 & von \textbf{würfen und mit slegen} grôz\\ 
5 & z\textit{w}êne tambûre gâben schal.\\ 
 & der ga\textit{lm} über alle die stat erhal.\\ 
 & der d\textit{ô}n iedoch gemischet wart\\ 
 & mit floitieren \textbf{an der} \textit{v}a\textit{r}t.\\ 
 & eine reisenote si bliesen.\\ 
10 & nû sullen wir niht verliesen,\\ 
 & wie ir hêrre komen sî.\\ 
 & dem riten videlære bî.\\ 
 & dô leite der degen wert\\ 
 & \textbf{ein} bein vür sich ûf daz pfert,\\ 
15 & zwêne stivel über \textbf{blôziu} bein.\\ 
 & sîn munt als ein rubîn schein\\ 
 & \textbf{von} rœte, als ob er brünne.\\ 
 & \textbf{der} was dicke, niht ze dünne.\\ 
 & sîn lîp was allenthalben clâr.\\ 
20 & lieht reideloht \textbf{was im} \textbf{daz} hâr,\\ 
 & \textbf{wâ} man\textbf{z} vor dem huote sach.\\ 
 & \textbf{der} was ein tiure houbetdach.\\ 
 & grüene samît was der mantel sîn.\\ 
 & ein zobel dâ vor gap \textbf{swarzen} schîn\\ 
25 & ob einem hemde, daz was blanc.\\ 
 & \textbf{von schouwen wart d\textit{â}} grôz gedranc.\\ 
 & \begin{large}V\end{large}il dicke \textbf{aldâ} gevrâget wart,\\ 
 & wer wære der \textbf{ritter} âne bart,\\ 
 & der \textbf{vuorte} \textbf{alsolich\textit{e}} rîcheit.\\ 
30 & \textbf{vil} schiere wart daz mære breit.\\ 
\end{tabular}
\scriptsize
\line(1,0){75} \newline
m n o \newline
\line(1,0){75} \newline
\textbf{27} \textit{Initiale} m n o  \newline
\line(1,0){75} \newline
\textbf{3} dôz] das \textit{nachträglich korrigiert zu:} dasz m \textbf{5} zwêne] Zenne m  $\cdot$ gâben] die gobent n geben o \textbf{6} galm] gaben m  $\cdot$ erhal] hal n erschal o \textbf{7} dôn] den m \textbf{8} vart] stat \textit{nachträglich korrigiert zu:} vart m \textbf{13} leite] leit n o \textbf{14} \textit{Vers 63.14 fehlt} n  \textbf{15} stivel] stiffeln n \textbf{16} rubîn] robin n \textbf{18} niht] vnd nit n  $\cdot$ dünne] stund: o \textbf{19} sîn] Der n o \textbf{22} tiure] tor o \textbf{26} dâ] do m o \textit{om.} n \textbf{28} wære der ritter] der ritter were n o \textbf{29} alsoliche] alsollichv m also sollich n \textbf{30} mære breit] mere geseit n jnen geseit o \newline
\end{minipage}
\end{table}
\newpage
\begin{table}[ht]
\begin{minipage}[t]{0.5\linewidth}
\small
\begin{center}*G
\end{center}
\begin{tabular}{rl}
 & vil schilde sach er schînen.\\ 
 & die hellen pusînen\\ 
 & \textbf{vor im mit krache} gâben dôz.\\ 
 & von \textbf{slegen und mit würfen} grôz\\ 
5 & zwêne tambûre gâben schal.\\ 
 & der galm über al die stat erhal.\\ 
 & der dôn iedoch gemischet wart\\ 
 & mit floitierene \textbf{ûf die} vart.\\ 
 & ein reisenote si bliesen.\\ 
10 & nû sulen wir niht ver\textit{l}iesen,\\ 
 & wie ir hêrre komen sî.\\ 
 & dem riten videlære bî.\\ 
 & dô leite der degen wert\\ 
 & \textbf{ein} bein vür sich ûf daz pfert,\\ 
15 & zwêne stivâl über \textbf{blôzez} bein.\\ 
 & sîn munt als ein rubîn schein\\ 
 & \textbf{vor} rœte, als ober brünne.\\ 
 & \textbf{er} was dicke, niht ze dünne.\\ 
 & sîn lîp was allenthalben klâr,\\ 
20 & lieht reideloht \textbf{sîn} hâr,\\ 
 & \textbf{swâ} man\textbf{z} vor dem huote sach.\\ 
 & \textbf{der} was ein tiwere houbetdach.\\ 
 & grüene samît was der mandel sîn.\\ 
 & ein zobel dâ vor gap \textbf{liehten} schîn\\ 
25 & obe einem hemde, daz was blanc.\\ 
 & \textbf{dâ wart von s\textit{ch}ouwene} grôz gedranc.\\ 
 & \textit{vil} dicke \textbf{dô} gevrâget wart,\\ 
 & wer wære der \textbf{junge} âne bart,\\ 
 & der \textbf{vuorte} \textbf{solhe} rîcheit.\\ 
30 & \begin{large}S\end{large}chiere wart daz mære breit.\\ 
\end{tabular}
\scriptsize
\line(1,0){75} \newline
G I O L M Q R Z Fr37 Fr44 \newline
\line(1,0){75} \newline
\textbf{1} \textit{Initiale} O  \textbf{15} \textit{Initiale} I  \textbf{27} \textit{Initiale} L R Z Fr37 Fr44  \textbf{30} \textit{Initiale} G  \newline
\line(1,0){75} \newline
\textbf{1} \textit{Die Verse 58.9-63.24 fehlen (Blattverlust)} R   $\cdot$ vil] Die Z Fr37  $\cdot$ sach] \textit{om.} Q  $\cdot$ er] er da I man Fr37 \textbf{3} \textit{nach 63.3:} Zwene tambure Q   $\cdot$ krache] kracke L chache Fr37  $\cdot$ dôz] [don]: dasz Q \textbf{4} Mit (Von Fr44 ) wuͦrfen vnd >von< (uon Fr44 ) slegen groz O (Fr44)  $\cdot$ Mit wuͯrffen vnd mit slegen groz L (M) (Q) (Z)  $\cdot$ Mit wrfen mit ::n slingen groz Fr37 \textbf{5} tambûre] tambvren O tambvrte Z \textbf{6} galm] gal M gaml Q  $\cdot$ über al] al dvrch O (L) (Q) (Fr37)  $\cdot$ erhal] hal L Q Z irschal M \textbf{7} dôn] [von]: don Q \textbf{8} floitierene] volitieren L  $\cdot$ die] der O L M Q Z Fr37 Fr44 \textbf{9} reisenote] reisen Fr44 \textbf{10} sulen] ensvl Z (Fr44)  $\cdot$ verliesen] verchiesen G veliesen I \textbf{11} ir hêrre] er her O (L) (M) Fr37 vnszer ritter Q \textbf{12} dem] Jm L (Fr37) \textbf{13} dô] Da M Z  $\cdot$ leite] leit O (Z) (Fr37)  $\cdot$ degen] helt Fr37 \textbf{15} zwêne] Zwei Fr44  $\cdot$ blôzez] blozev I (L) (M) (Q) (Z) (Fr44)  $\cdot$ bein] [vel]: bein O \textbf{16} als] sam Q  $\cdot$ rubîn] rubein Q \textbf{17} vor] von I Von der Z  $\cdot$ rœte als] rate sam Q  $\cdot$ ober] er O L (Q) Fr37 \textbf{18} er] Ez L (Q) Der Z (Fr37)  $\cdot$ was] enwas ze Fr44  $\cdot$ niht] vnde niht O (M) (Q) (Z) noch Fr44 \textbf{19} \textit{Vers 63.19 fehlt} M  \textbf{20} \textit{Versdoppelung 62.9-10 nach 63.20} Q   $\cdot$ lieht] Lyech L Licht M Q Sleht Fr44  $\cdot$ reideloht] raide loc I raideloht was O (Z) (Fr37) (Fr44) rodelecht L gel was M redlech wasz Q \textbf{21} swâ] Waz L (Q) Wo M Swaz Fr44  $\cdot$ manz] man L (Fr44) man sein Q  $\cdot$ vor] ob I davor L \textbf{22} der] daz I (Q) (Fr44) \textbf{23} grüene] ein gruner I \textbf{24} ein] einen I  $\cdot$ dâ vor gap] der gab vor O do vor der gab Q  $\cdot$ liehten] swarzen I O (L) (M) (Z) (Fr37) (Fr44) lichten Q \textbf{25} obe] Vbir M (Fr44) \textbf{26} dâ wart von schouwene] da wart von soͮwene G Von schawen wart do O (M) (R) Von schowen wart da L Z (Fr37) Fr44 Vor schowen wasz Q  $\cdot$ gedranc] geranck Q gedank R \textbf{27} vil] \textit{om.} G Wie dicke R Al dicke Fr44  $\cdot$ dô] da M R Z \textbf{28} wer] Wer da Z  $\cdot$ wære der junge] wer der ritter O (L) (M) (Q) (Z) (Fr37) (Fr44) der Ritter were R \textbf{29} der] Der da Z Er Fr44  $\cdot$ solhe] al solhe O (L) (M) (Q) (Fr37) (Fr44) also hoche R \textbf{30} Schiere] vil schier I (O) (L) (M) (Z) Fr37 (Fr44)  $\cdot$ daz] die L  $\cdot$ breit] bereit L R albreit M \newline
\end{minipage}
\hspace{0.5cm}
\begin{minipage}[t]{0.5\linewidth}
\small
\begin{center}*T (U)
\end{center}
\begin{tabular}{rl}
 & vil schilte sach er schînen.\\ 
 & die hel\textit{l}e\textit{n} busînen\\ 
 & \textbf{vor im mit krache} gâben dôz.\\ 
 & von \textbf{slegen und würfen} grôz\\ 
5 & zwêne tambûre gâben schal,\\ 
 & \textbf{daz} der galm über al die stat erhal.\\ 
 & der dôn iedoch gemischet wart\\ 
 & mit floitierne \textbf{ûf der} vart.\\ 
 & eine reisenote si bliesen.\\ 
10 & nû soln wir niht verliesen,\\ 
 & wie ir hêrre komen sî.\\ 
 & dem riten videlære bî.\\ 
 & dô leit der degen wert\\ 
 & \textbf{sîn} bein vür sich ûf daz pfert,\\ 
15 & zwêne stivâl über \textbf{blôz} bein.\\ 
 & sîn munt als ein rubîn schein\\ 
 & \textbf{von} rœte, als \textit{o}b er brünne.\\ 
 & \textbf{er} was dicke \textbf{und} niht zuo dünne.\\ 
 & sîn lîp was allenthalben clâr.\\ 
20 & lieht reideloht \textbf{was im} \textbf{daz} hâr,\\ 
 & \textbf{sô} man \textbf{ez} vor de\textit{m} huote sach.\\ 
 & \textbf{daz} was ein tiure houbetdach.\\ 
 & grüener samît was der mantel sîn.\\ 
 & ein zobel dâ vor gap \textbf{liehten} schîn\\ 
25 & ob eime hemede, daz was blanc.\\ 
 & \textbf{von schouwen was dâ} grôz ged\textit{r}anc.\\ 
 & vil dicke \textbf{dâ} gevrâget wart,\\ 
 & wer wære der \textbf{ritter} âne bart,\\ 
 & der \textbf{dâ} \textbf{vüeret} \textbf{sô grôze} rîcheit.\\ 
30 & schiere wart daz mære \textit{b}reit.\\ 
\end{tabular}
\scriptsize
\line(1,0){75} \newline
U V W T \newline
\line(1,0){75} \newline
\textbf{2} \textit{Majuskel} T  \textbf{5} \textit{Majuskel} T  \textbf{10} \textit{Majuskel} T  \textbf{13} \textit{Majuskel} T  \textbf{15} \textit{Majuskel} T  \textbf{23} \textit{Majuskel} T  \textbf{27} \textit{Initiale} V W   $\cdot$ \textit{Majuskel} T  \textbf{30} \textit{Majuskel} T  \newline
\line(1,0){75} \newline
\textbf{2} hellen] helde U helme vnd W \textbf{3} vor im mit krache] Mit crache vor im V \textbf{4} slegen und würfen] slegen und mit (von W ) wúrfen V (W) wurfen vnde slegen T \textbf{6} daz] \textit{om.} V W T  $\cdot$ al] \textit{om.} T \textbf{9} reisenote] reisehorn V \textbf{11} ir] der V (T) \textbf{13} leit] leite V (W) T \textbf{14} sîn] [*in]: Ein V ein T \textbf{15} zwêne] Zwo V  $\cdot$ . blôz] blosze V (W) blôziv T \textbf{16} als] alsam V  $\cdot$ rubîn] Ruͦbin U robin V \textbf{17} von] Vor W  $\cdot$ ob er] vber U \textbf{18} er] Der V  $\cdot$ dicke und niht] dicke niht V ze dicke noch T \textbf{20} reideloht] vnde crv̂z T  $\cdot$ im daz] sin V im sein W \textbf{21} sô] Swo V (T)  $\cdot$ man ez] men daz V  $\cdot$ dem] der U \textbf{22} daz] [D*r]: Der V  $\cdot$ tiure] duͦrre U \textbf{23} grüener] Ein gruͤner W \textbf{24} liehten] swarzen T \textbf{26} von] vor T  $\cdot$ schouwen] schoͮwende V  $\cdot$ was] ward W (T)  $\cdot$ dâ] do V W  $\cdot$ gedranc] gedanc U \textbf{27} dâ] do V W T \textbf{29} dâ] do V \textit{om.} W T  $\cdot$ vüeret] fuͦrte W (T)  $\cdot$ sô grôze] selhe T \textbf{30} schiere] Vil schiere V (T)  $\cdot$ breit] bereit U \newline
\end{minipage}
\end{table}
\end{document}
