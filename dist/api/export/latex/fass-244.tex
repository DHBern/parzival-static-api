\documentclass[8pt,a4paper,notitlepage]{article}
\usepackage{fullpage}
\usepackage{ulem}
\usepackage{xltxtra}
\usepackage{datetime}
\renewcommand{\dateseparator}{.}
\dmyyyydate
\usepackage{fancyhdr}
\usepackage{ifthen}
\pagestyle{fancy}
\fancyhf{}
\renewcommand{\headrulewidth}{0pt}
\fancyfoot[L]{\ifthenelse{\value{page}=1}{\today, \currenttime{} Uhr}{}}
\begin{document}
\begin{table}[ht]
\begin{minipage}[t]{0.5\linewidth}
\small
\begin{center}*D
\end{center}
\begin{tabular}{rl}
\textbf{244} & \begin{large}D\end{large}urch uns \textbf{noch} eine wîle."\\ 
 & ein spil mit der île\\ 
 & het er unz \textbf{an} den ort gespilt.\\ 
 & daz \textbf{man} gein liehter varwe \textbf{zilt},\\ 
5 & daz begunde ir ougen süezen,\\ 
 & ê si enpfiengen sîn grüezen.\\ 
 & ouch vuogten in gedanke nôt,\\ 
 & daz im \textbf{sîn} m\textit{un}t was \textbf{sô rôt}\\ 
 & unt daz \textbf{vor} jugende niemen dran\\ 
10 & kôs \textbf{gein einer} halben gran.\\ 
 & \textbf{Dise} vier juncvrouwen kluoc,\\ 
 & hœret, wa\textit{z} ieslîchiu truoc:\\ 
 & môraz, wîn \textbf{unt} lûtertranc\\ 
 & \textbf{truogen} drî ûf henden blanc.\\ 
15 & diu vierde juncvrouwe wîs\\ 
 & truog obez der art von pardîs\\ 
 & \textbf{ûf} einer tweheln \textbf{blanc} gevar.\\ 
 & diu selbe kniete \textbf{ouch} vür in dar.\\ 
 & er bat \textbf{die vrouwen} sitzen.\\ 
20 & \textbf{si sprach}: "lât mich \textbf{bî} witzen.\\ 
 & sô \textbf{wæret} ir \textbf{dienstes} \textbf{ungewert},\\ 
 & als \textbf{mîn her vür iuch ist} gegert."\\ 
 & Süezer rede er \textbf{gein in} niht vergaz.\\ 
 & der hêrre tranc, ein teil er az.\\ 
25 & mit urloube si \textbf{giengen} wider.\\ 
 & Parzival \textbf{sich leite} nider.\\ 
 & \textbf{ouch} sazten \textbf{juncvröuwelîn}\\ 
 & ûfen teppech die kerzen sîn.\\ 
 & dô si in \textbf{slâfen} sâhen,\\ 
30 & si begunden dannen gâhen.\\ 
\end{tabular}
\scriptsize
\line(1,0){75} \newline
D \newline
\line(1,0){75} \newline
\textbf{1} \textit{Initiale} D  \textbf{11} \textit{Majuskel} D  \textbf{23} \textit{Majuskel} D  \newline
\line(1,0){75} \newline
\textbf{8} munt] mvͦt D \textbf{12} waz] was D \textbf{21} dienstes] diens D \newline
\end{minipage}
\hspace{0.5cm}
\begin{minipage}[t]{0.5\linewidth}
\small
\begin{center}*m
\end{center}
\begin{tabular}{rl}
 & durch uns \textbf{noch} eine wîle."\\ 
 & ein spil mit der île\\ 
 & het er unz \textbf{an} den ort gespilt.\\ 
 & daz \textbf{man} gegen liehter varw\textit{e} \textbf{zilt},\\ 
5 & daz begunde ir ougen süezen,\\ 
 & ê si enpfiengen sîn grüezen.\\ 
 & ouch vuo\textit{c}ten in gedanke nôt,\\ 
 & daz im \textbf{sîn} munt was \textbf{sô rôt}\\ 
 & und daz \textbf{vor} \textit{ju}gende nieman drane\\ 
10 & kôs \textbf{gegen einer} halben grane.\\ 
 & \textbf{dise} vier juncvrouwen kluoc,\\ 
 & hœret, waz ieglîchiu truoc:\\ 
 & môraz, wîn \textbf{und} lûtertranc\\ 
 & \textbf{truogen} drîe ûf henden blanc.\\ 
15 & diu vierde juncvrouwe wîs\\ 
 & truoc \dag ob\dag  der art von paradîs\\ 
 & \textbf{ûf} einer twehelen \textbf{blanc} gevar.\\ 
 & diu selbe kniete \textbf{ouch} vür in dar.\\ 
 & er bat \textbf{die juncvrouwen} sitzen.\\ 
20 & \textbf{si sprach}: "lât mich \textbf{mit} witzen.\\ 
 & sô \textbf{wæret} ir \textbf{dienstes} \textbf{ungewert},\\ 
 & als \textbf{mîn here vür iuch ist} gegert."\\ 
 & süezer rede er \textbf{gegen ir} niht vergaz.\\ 
 & der hêrre tranc, ein teil er az.\\ 
25 & mit u\textit{r}loube si \textbf{giengen} wider.\\ 
 & Parcifal \textbf{sich leite} nider.\\ 
 & \textbf{ouch} sasten \textbf{junchêrrelîn}\\ 
 & ûfen teppich die kerzen sîn.\\ 
 & dô sin \textbf{entslâfen} sâhen,\\ 
30 & si begunden \textit{dannen} gâhen.\\ 
\end{tabular}
\scriptsize
\line(1,0){75} \newline
m n o Fr69 \newline
\line(1,0){75} \newline
\newline
\line(1,0){75} \newline
\textbf{3} unz] bitz n (o)  $\cdot$ ort] orten o \textbf{4} varwe] varwa m  $\cdot$ zilt] zelt o \textbf{5} begunde ir ougen] begunden ir oren o \textbf{6} si enpfiengen] siner enpfing o \textbf{7} ouch vuocten] Ouch furten m Ouch fuͦrte n Echt vuͦgt Fr69 \textbf{9} Vnden nieman trang o  $\cdot$ jugende] ymegende m \textbf{10} kôs] Kose n \textbf{12} ieglîchiu] igliches o \textbf{14} truogen] Trigen o \textbf{15} juncvrouwe] jungfrouwen m \textbf{19} juncvrouwen] jungfrouwe n (o) \textbf{20} mit] bẏ n o (Fr69) \textbf{21} sô] Sú n Do o \textbf{22} here] herre m n (o) \textbf{24} tranc] er trang n \textbf{25} urloube] vnlobe m \textbf{27} sasten] satten die n (o) satzte jegslich Fr69 \textbf{28} ûfen] Vff n o  $\cdot$ die] des o \textbf{29} entslâfen] sloffen n (o) (Fr69) \textbf{30} dannen] \textit{om.} m \newline
\end{minipage}
\end{table}
\newpage
\begin{table}[ht]
\begin{minipage}[t]{0.5\linewidth}
\small
\begin{center}*G
\end{center}
\begin{tabular}{rl}
 & durch uns \textbf{noch} eine wîle."\\ 
 & ein spil mit der île\\ 
 & het er unze \textbf{an} den ort gespilt.\\ 
 & daz \textbf{man} gên liehter varwe \textbf{zilt},\\ 
5 & daz begunde ir ougen süezen,\\ 
 & ê si enpfiengen sîn grüezen.\\ 
 & ouch vuogten i\textit{n} gedanke nôt,\\ 
 & daz i\textit{m} \textbf{d\textit{er}} mun\textit{t} wa\textit{s} \textbf{sô rôt}\\ 
 & unde daz \textbf{vor} jugende niemen dran\\ 
10 & kôs \textbf{gein einer} halben gran.\\ 
 & \textbf{die} vier juncvrouwen kluoc,\\ 
 & \textbf{nû} hœret, waz iegelîchiu truoc:\\ 
 & môraz, wîn, lûtertranc\\ 
 & \textbf{truogen} drî ûf henden blanc.\\ 
15 & diu vierde juncvrouwe wîs\\ 
 & truoc obez der art von pardîs\\ 
 & \textbf{in} einer twehelen \textbf{wîz} gevar.\\ 
 & diu selbe kniete vür in dar.\\ 
 & er bat \textbf{si alle} sitzen.\\ 
20 & \textbf{si sprach}: "lât \textit{mich} \textbf{bî} witzen.\\ 
 & sô w\textit{ær}et ir \textbf{dienstlîch} \textbf{gewert},\\ 
 & als \textbf{\textit{mî}n \textit{h}er vür iuch ist} gegert."\\ 
 & süezer rede er \textit{\textbf{gein ir}} niht vergaz.\\ 
 & der hêrre tranc, ein teil er az.\\ 
25 & mit urloube si \textbf{schieden} wider.\\ 
 & Parzival, \textbf{der leit sich} nider.\\ 
 & \textbf{dô} sazten \textbf{diu} \textbf{junchêrrelîn}\\ 
 & ûf den tepch die kerzen sîn.\\ 
 & dô \textit{si} in \textbf{slâfen} sâhen,\\ 
30 & si begunden dannen gâhen.\\ 
\end{tabular}
\scriptsize
\line(1,0){75} \newline
G I O L M Q R Z Fr54 \newline
\line(1,0){75} \newline
\textbf{3} \textit{Initiale} I  \textbf{11} \textit{Initiale} L R  \textbf{15} \textit{Initiale} I O  \newline
\line(1,0){75} \newline
\textbf{2} ein] Deme M Din Z  $\cdot$ spil] pfil R \textbf{3} er] ersz Q  $\cdot$ an] in R \textbf{4} man gên] mangen G mangem I  $\cdot$ liehter] lýchter L (M) (Q) \textbf{5} daz begunde] Da bigonden M \textbf{6} ê] ê daz I Daz M  $\cdot$ enpfiengen] enpfvnden L  $\cdot$ sîn] seine Q \textbf{7} vuogten] Gefugte I fvͦgt O fuͯgte L fuctem Q  $\cdot$ in] im G I O Q (R)  $\cdot$ gedanke] gedankin M (Q) (Z) \textbf{8} im der munt was] in die mvnde waren G \textbf{9} jugende] iugenden I \textbf{11} vier] vur M \textbf{13} wîn] win vnde M (R) (Z) \textbf{14} \textit{Vers 244.14 fehlt} R   $\cdot$ truogen] trungen Q \textbf{15} diu] ÷iv O  $\cdot$ vierde] vierd u R \textbf{16} obez der art] des obezs I ober der art M  $\cdot$ von] vom R  $\cdot$ pardîs] paris L \textbf{17} in] Vf L  $\cdot$ twehelen] taveln O truchen R  $\cdot$ wîz] lieht O lýcht L (Q) \textbf{18} kniete] chniet I O (Q) \textbf{19} si alle] sie zuͯ ým L sie nider Q die Junckfrowen R (Z) \textbf{20} sprach] sprachen I sprach herre L  $\cdot$ mich] vns G  $\cdot$ bî] mit O \textbf{21} wæret] werdet G  $\cdot$ ir] in Q  $\cdot$ dienstlîch] dienstes O M (Q) R Z dinst L  $\cdot$ gewert] vngewert O L (M) Q R Z \textbf{22} mîn her] vnser G mir herre L  $\cdot$ vür iuch ist] ist fvr evch Z \textbf{23} süezer] Susze M Der Q  $\cdot$ gein ir] \textit{om.} G er gen in Q (Z) \textbf{26} Parzival] mit vrlaube I Parcifal O L Z Parzifal M Partzifal Q Parczifal R  $\cdot$ der leit sich] er leit sich I sich leite O L (Q) (R) \textbf{27} dô] Avch O (L) (M) Q (R) (Z)  $\cdot$ sazten] [sazen]: sazten O sassen R  $\cdot$ diu] \textit{om.} O  $\cdot$ junchêrrelîn] jvngfrowelin L (Q) \textbf{28} den] daz L in Q dem R \textbf{29} dô] Da M Z  $\cdot$ si] \textit{om.} G \textbf{30} dannen] wider I \newline
\end{minipage}
\hspace{0.5cm}
\begin{minipage}[t]{0.5\linewidth}
\small
\begin{center}*T
\end{center}
\begin{tabular}{rl}
 & durch uns eine wîle."\\ 
 & ein spil mit der île\\ 
 & het er unz \textbf{in} den ort gespilt.\\ 
 & daz \textbf{was} gegen liehter varwe \textbf{gezilt}:\\ 
5 & \multicolumn{1}{l}{ - - - }\\ 
 & \multicolumn{1}{l}{ - - - }\\ 
 & ouch vuocten in gedanke nôt,\\ 
 & daz im \textbf{sîn} munt was \textbf{rôserôt}\\ 
 & unde daz \textbf{von} jugende niemen dran\\ 
10 & kôs \textbf{einen} halben gran.\\ 
 & \textbf{\begin{large}D\end{large}is\textit{e}} \textit{v}ier juncvrouwen kluoc,\\ 
 & \textbf{nû} hœret, waz ieglîch\textit{iu} truoc:\\ 
 & môraz, wîn, lûtertranc\\ 
 & \textbf{brâhten} drîe ûf handen blanc.\\ 
15 & diu vierde juncvrouwe wîs\\ 
 & truoc obez der art von paradîs\\ 
 & \textbf{ûf} einer tweheln \textbf{blanc} gevar.\\ 
 & diu selbe kniete vür in dar.\\ 
 & er bat \textbf{die vrouwen} sitzen.\\ 
20 & "\textbf{Nein, hêrre}, lât mich \textbf{bî} witzen.\\ 
 & sô \textbf{wæret} ir \textbf{dienstes} \textbf{ungewert},\\ 
 & als\textbf{ich bin vür iuch} gegert."\\ 
 & Süezer rede er \textbf{gegen ir} niht vergaz.\\ 
 & der hêrre tranc, ein teil er az.\\ 
25 & Mit urloube si \textbf{giengen} wider.\\ 
 & Parcifal \textbf{sich leite} nider.\\ 
 & \textbf{Dô} sazten \textbf{die} \textbf{junchêrrelîn}\\ 
 & ûf den tepich die kerzen sîn.\\ 
 & dô sin \textbf{slâfen} sâhen,\\ 
30 & si begunden dannen gâhen.\\ 
\end{tabular}
\scriptsize
\line(1,0){75} \newline
T U V W \newline
\line(1,0){75} \newline
\textbf{11} \textit{Initiale} T U V W  \textbf{20} \textit{Majuskel} T  \textbf{23} \textit{Majuskel} T  \textbf{25} \textit{Majuskel} T  \textbf{27} \textit{Majuskel} T  \newline
\line(1,0){75} \newline
\textbf{1} uns] not U [*]: vns noch V vns noch W \textbf{2} [*]: Ein spil mit der ile V \textbf{3} het] Hat W  $\cdot$ unz in] mit an U  $\cdot$ den] daz U (W) [*]: daz  V \textbf{4} was] [*]: man V  $\cdot$ gezilt] [*]: zilt V \textbf{5} \textit{Die Verse 244.5-6 sind am Rand nachgetragen und später radiert:} Daz begunde ir :::sen / E s: enpfiengen s:: g:::en V   $\cdot$ [*]: Daz begunde ir oͮgen suͤssen V \textbf{6} [*]: E sú enphiengen sin gruͤssen V \textbf{7} vuocten] [*det]: det U [*]: fuͤgete V  $\cdot$ in] im U W  $\cdot$ gedanke] gedancken W \textbf{8} sîn] [*]: der V  $\cdot$ rôserôt] so rot U W [*]: so rot V \textbf{9} von] vor U W [*]: vor V \textbf{10} einen] [*]: gegen einer V gen einer W \textbf{11} Dise vier] Dise vie vier T DIse W \textbf{12} ieglîchiu] iegliche T \textbf{13} lûtertranc] [*]: vnde lúter trang V \textbf{14} brâhten] [*]: Truͦgen V  $\cdot$ handen] armen U [*]: henden V \textbf{16} obez] ob iz U  $\cdot$ der art] der der art V \textit{om.} W  $\cdot$ von] [von*]: vomme V vom W \textbf{17} blanc] weiß W \textbf{18} kniete] kniete auch U oͮch knv́wete V \textbf{19} vrouwen] [*]: iuncfrowen V \textbf{20} Nein hêrre] [*]: Sv́ sprach V  $\cdot$ mich] \textit{om.} W  $\cdot$ witzen] wizzen V \textbf{21} sô wæret ir] Mich ir werent W \textbf{22} Als [*]: min herre fúr v́ch ist gegert V  $\cdot$ Als mein herre hin zuͦ eúch gert W  $\cdot$ vür iuch] vur îv T her vor vch U \textbf{23} ir] in U [in]: ir V \textbf{24} tranc] twanc U \textbf{26} Parcifal] [*]: Parzefal V Partzifal W \textbf{27} dô] [*]: Oͮch V  $\cdot$ sazten] sageten U  $\cdot$ die junchêrrelîn] die Juͦncvreuwelin U iúnckerlin W \textbf{29} dô] Sie do U \newline
\end{minipage}
\end{table}
\end{document}
