\documentclass[8pt,a4paper,notitlepage]{article}
\usepackage{fullpage}
\usepackage{ulem}
\usepackage{xltxtra}
\usepackage{datetime}
\renewcommand{\dateseparator}{.}
\dmyyyydate
\usepackage{fancyhdr}
\usepackage{ifthen}
\pagestyle{fancy}
\fancyhf{}
\renewcommand{\headrulewidth}{0pt}
\fancyfoot[L]{\ifthenelse{\value{page}=1}{\today, \currenttime{} Uhr}{}}
\begin{document}
\begin{table}[ht]
\begin{minipage}[t]{0.5\linewidth}
\small
\begin{center}*D
\end{center}
\begin{tabular}{rl}
\textbf{428} & \begin{large}D\end{large}ô sprach der werde, süeze man:\\ 
 & "daz tuon ich, swester, ob ich kan.\\ 
 & dar zuo gib selbe dînen rât.\\ 
 & dich dunket, daz mir missetât\\ 
5 & werdecheit habe underswungen,\\ 
 & von prîse mich gedrungen.\\ 
 & waz t\textit{ö}ht ich denne ze bruoder dir?\\ 
 & wan dienden alle krône mir,\\ 
 & \textbf{der} stüende ich ab durch dîn gebot.\\ 
10 & dîn hazzen wære mîn \textbf{hœhstiu} nôt.\\ 
 & mir ist unmære vreude unt êre\\ 
 & \textbf{niht} wan nâch dîner lêre.\\ 
 & Hêr Gawan, ich wil iuch \textbf{des} biten:\\ 
 & ir kômt durch prîs dâ her geriten.\\ 
15 & nû tuot\textbf{z} durch prîses hulde,\\ 
 & helfet mir, daz \textbf{mîne} schulde\\ 
 & mîn swester ûf mich verkiese.\\ 
 & ê daz ich si verliese,\\ 
 & ich verkiuse ûf iuch mîn herzeleit,\\ 
20 & welt ir mir geben sicherheit,\\ 
 & daz ir mir werbet sunder twâl\\ 
 & mit guoten triwen umben Grâl."\\ 
 & Dâ wart diu suone geendet\\ 
 & unt Gawan gesendet\\ 
25 & an dem selben mâle\\ 
 & durch strîten nâch dem Grâle.\\ 
 & Kyngrimursel ouch verkôs\\ 
 & ûf den künec, der in dâ \textbf{vor} verlôs,\\ 
 & daz er im sîn geleite brach.\\ 
30 & vor alden vürsten daz geschach.\\ 
\end{tabular}
\scriptsize
\line(1,0){75} \newline
D Fr1 Fr5 Fr68 \newline
\line(1,0){75} \newline
\textbf{1} \textit{Initiale} D Fr1 Fr5 Fr68  \textbf{13} \textit{Capitulumzeichen} Fr5   $\cdot$ \textit{Majuskel} D  \textbf{23} \textit{Capitulumzeichen} Fr5   $\cdot$ \textit{Versal} Fr1   $\cdot$ \textit{Majuskel} D  \textbf{27} \textit{Initiale} Fr68  \newline
\line(1,0){75} \newline
\textbf{3} selbe] selbiv Fr5 \textbf{4} mir] [dir]: mir Fr5 \textbf{5} werdecheit habe] habe werdecheit Fr1 (Fr68)  $\cdot$ underswungen] vnderswunden Fr68 \textbf{6} Vnd trivwe vndir drungin Fr5 vnde von prise mih verdrungen Fr68 \textbf{7} töht ich] toht ich D (Fr1) (Fr5) (Fr68) \textbf{8} Krône] cronen Fr68 \textbf{10} hazzen] haz Fr1  $\cdot$ mîn] mir Fr5 \textbf{11} unmære] vmbe Fr68 \textbf{13} Gawan] Gauwan Fr5  $\cdot$ iuch] iv Fr5 \textbf{14} dâ her] [her*]: da her Fr68 \textbf{15} tuotz] tuͦnzt Fr5 \textbf{16} helfet] helf Fr68 \textbf{22} guoten triwen] gvͦtem willen Fr1 \textbf{23} Dâ] Do Fr1 (Fr68)  $\cdot$ diu] di Fr68  $\cdot$ geendet] verendet Fr1 \textbf{24} Gawan] her Gawan Fr1 Gauwan Fr5 \textbf{26} strîten] vorscen Fr1  $\cdot$ dem] den Fr68 \textbf{27} Kyngrimursel] Kẏngrimvrsel Fr1 Kingrimursel Fr68 \textbf{28} dâ vor verlôs] da virlos Fr5 verlos Fr68 \newline
\end{minipage}
\hspace{0.5cm}
\begin{minipage}[t]{0.5\linewidth}
\small
\begin{center}*m
\end{center}
\begin{tabular}{rl}
 & \begin{large}D\end{large}ô sprach der werde, süeze man:\\ 
 & "daz tuon ich, swester, ob ich kan.\\ 
 & dar zuo gip selber dînen rât.\\ 
 & dich dunket, daz mir missetât\\ 
5 & werdecheit habe underswungen,\\ 
 & von prîse mich gedrungen.\\ 
 & waz t\textit{ö}ht ich danne ze bruoder dir?\\ 
 & wanne dieneten alle krônen mir,\\ 
 & \textbf{der} stüende ich abe durch dîn gebot.\\ 
10 & dîn hazzen wære mîn \textbf{hœhestiu} nôt.\\ 
 & mir ist unmære vröude und êre\\ 
 & \textbf{niht} wanne nâch dîner lêre.\\ 
 & hêr Gawan, ich wil iuch biten:\\ 
 & ir kômet durch prîs dâ her geriten.\\ 
15 & nû tuot \textbf{daz} durch prîses hulde,\\ 
 & helfet mir, daz \textbf{mîne} schulde\\ 
 & mîn swester ûf mich verkiese.\\ 
 & ê daz ich si verliese,\\ 
 & ich verkiuse ûf iuch mîn herzeleit,\\ 
20 & wellet ir mir geben sicherheit,\\ 
 & daz ir mir werbet sunder twâl\\ 
 & mit guoten triuwen umb de\textit{n} Grâl."\\ 
 & d\textit{â} wart diu suone geendet\\ 
 & und Gawan gesendet\\ 
25 & an dem selben mâle\\ 
 & durch strîten nâch dem Grâle.\\ 
 & Kingri\textit{m}ursel ouch verkôs\\ 
 & ûf den künic, der in dâ \textbf{vor} verlôs,\\ 
 & daz er ime sîn geleite brach.\\ 
30 & vor allen den vürsten daz geschach.\\ 
\end{tabular}
\scriptsize
\line(1,0){75} \newline
m n o \newline
\line(1,0){75} \newline
\textbf{1} \textit{Initiale} m n  \newline
\line(1,0){75} \newline
\textbf{3} gip] gip mir n \textbf{4} missetât] missetan o \textbf{5} habe] hat n \textbf{7} töht] doht m (n) (o) \textbf{9} abe durch dîn] ob dẏme o \textbf{11} unmære] so ẏemer n \textbf{13} iuch] úch des n (o) \textbf{15} tuot daz] das duͯnt n \textbf{16} schulde] [hulde]: schulde o \textbf{17} mîn swester ûf] Vff min swester n \textbf{18} si] sie des o \textbf{19} iuch] [mich]: uͯch m \textbf{20} geben] \textit{om.} n \textbf{21} werbet] werben n \textbf{22} den] dem m \textbf{23} dâ] Do m n o  $\cdot$ geendet] verendet n o \textbf{26} dem] \textit{om.} n \textbf{27} Kingrimursel] Kingringvrsel m Kingrumúrsel n Kyngrumuͯrsel o  $\cdot$ verkôs] vergosz n verkuͯsz o \textbf{28} dâ vor] do n \textbf{30} den] \textit{om.} n o \newline
\end{minipage}
\end{table}
\newpage
\begin{table}[ht]
\begin{minipage}[t]{0.5\linewidth}
\small
\begin{center}*G
\end{center}
\begin{tabular}{rl}
 & dô sprach der werde, süeze man:\\ 
 & "daz tuon ich, swester, obe ich kan.\\ 
 & dar zuo gip selbe dînen rât.\\ 
 & dich dunket, daz mir missetât\\ 
5 & werdicheit habe underswungen,\\ 
 & von brîse mich gedrungen.\\ 
 & waz t\textit{ö}hte ich dane ze bruoder dir?\\ 
 & wan dienden alle krône mir,\\ 
 & \textbf{der} stüende ich abe durch dîn gebot.\\ 
10 & dîn hazzen wære mîn \textbf{meistiu} nôt.\\ 
 & mirst unmære vröude unde êre\\ 
 & wan nâch dîner lêre.\\ 
 & hêr Gawan, ich wil iuch \textbf{des} biten:\\ 
 & ir kômet durch prîs dâ her geriten.\\ 
15 & nû tuot \textbf{ez} durch brîses hulde,\\ 
 & helfet mir, daz \textbf{mîne} schulde\\ 
 & mîn swester ûf mich verkiese.\\ 
 & ê da\textit{z} ich si verliese,\\ 
 & \begin{large}I\end{large}ch verkiuse ûf iuch mîn herzeleit,\\ 
20 & welt ir mir geben sicherheit,\\ 
 & daz ir mir werbet sunder twâl\\ 
 & mit guoten triwen umbe den Grâl."\\ 
 & dâ wart diu suone geendet\\ 
 & unde Gawan gesendet\\ 
25 & an dem selben mâle\\ 
 & durch strîten nâch dem Grâle.\\ 
 & Kingrimursel ouch verkôs\\ 
 & ûf den künic, der in dâ \textbf{vor} verlôs,\\ 
 & daz er im sîn geleite brach.\\ 
30 & vor al den vürsten daz geschach.\\ 
\end{tabular}
\scriptsize
\line(1,0){75} \newline
G I O L M Q R Z Fr21 \newline
\line(1,0){75} \newline
\textbf{1} \textit{Initiale} I O L Q Z Fr21   $\cdot$ \textit{Capitulumzeichen} R  \textbf{15} \textit{Initiale} I  \textbf{19} \textit{Initiale} G  \newline
\line(1,0){75} \newline
\textbf{1} dô] ÷o O Da M  $\cdot$ werde] werlt M \textbf{3} selbe] selbin M selber Q R \textbf{4} dich] Mich L Dick Q Das dich R  $\cdot$ missetât] misse stat L R \textbf{5} underswungen] vnder vnderswvngen L vnderschwingen R vndersungen Z (Fr21) \textbf{6} gedrungen] verdrungen I (Q) ze dringen R \textbf{7} töhte] tohte G O L (M) (Q) (R) (Z) Fr21 toͮc I  $\cdot$ ze bruoder] brudir zcu M  $\cdot$ dir] dri Q \textbf{8} wan] Vnd R  $\cdot$ krône] Cronen L (M) conen R \textbf{9} stüende] stuͤnt I (M) (Q) (Fr21)  $\cdot$ ich] \textit{om.} M \textbf{10} hazzen] haz Z  $\cdot$ meistiu] grostev I (R) hostiv O (L) (M) (Z) Fr21 grosse Q \textbf{11} unde] ane L (R) vn Fr21 \textbf{12} wan nâch] Nîe wan nach O Nv wan nach L Nicht wanne durch M Nicht wan nach Q (R) (Z) (Fr21) \textbf{13} Gawan] Gawain R  $\cdot$ des] \textit{om.} R \textbf{14} durch prîs dâ her] da her dvrch pris O (M) durch pris her Z \textbf{15} tuot ez] tvͦz O tuntz R  $\cdot$ hulde] [ere]: hvlde O \textbf{16} helfet] Vnd helffent R  $\cdot$ mir] \textit{om.} M \textbf{18} daz] dane G \textit{om.} Q \textbf{19} verkiuse] verkuͯse E L  $\cdot$ herzeleit] herleid R \textbf{20} sicherheit] sicheit Q \textbf{21} werbet] werben R [werben]: werbet Z \textbf{23} dâ] do I (Q) (R)  $\cdot$ wart] was M wirt Fr21  $\cdot$ diu] \textit{om.} M der R disiv Fr21 \textbf{24} Gawan] Gawane L Gawain R  $\cdot$ gesendet] versendet I \textbf{25} selben] selbe Q sellen R \textbf{26} durch] \textit{om.} I \textbf{27} Kingrimursel] Kyngrimvrsel O Q Kingrymursel M Kyngrumursel R  $\cdot$ ouch] auch da I \textbf{28} in] \textit{om.} M  $\cdot$ dâ vor] da I (O) L M Fr21 do Q R \textbf{29} geleite] geleit do R \textbf{30} al den] allen den I (O) (Z) den M alten Q allen R  $\cdot$ geschach] beschach R g geschach Fr21 \newline
\end{minipage}
\hspace{0.5cm}
\begin{minipage}[t]{0.5\linewidth}
\small
\begin{center}*T
\end{center}
\begin{tabular}{rl}
 & Dô sprach der werde, süeze man:\\ 
 & "daz tuon ich, swester, ob ich kan.\\ 
 & dar zuo gip selbe dînen rât.\\ 
 & dich dunket, daz mir missetât\\ 
5 & werdecheit habe underswungen,\\ 
 & von prîse mich gedrungen.\\ 
 & waz t\textit{ö}htich danne ze bruoder dir?\\ 
 & wan dienten alle krônen mir,\\ 
 & \textbf{den} st\textit{üe}ndich abe durch dîn gebot.\\ 
10 & dîn hazzen wære mîn \textbf{hœhst\textit{iu}} nôt.\\ 
 & mirst unmære vröude unde êre\\ 
 & \textbf{niht} wan nâch dîner lêre.\\ 
 & Hêr Gawan, ich wil iuch \textbf{des} biten:\\ 
 & ir kômet durch prîs dâ her geriten.\\ 
15 & nû tuot\textbf{z} durch prîses hulde,\\ 
 & helfet mir, daz \textbf{mîner} schulde\\ 
 & mîn swester ûf mich verkiese.\\ 
 & ê daz ich si verliese,\\ 
 & ich verkiuse ûf iuch mîn herzeleit,\\ 
20 & welt ir mir geben sicherheit,\\ 
 & daz ir mir werbet sunder twâl\\ 
 & mit guoten triuwen umbe den Grâl."\\ 
 & \begin{large}D\end{large}â wart di\textit{u} suone geendet\\ 
 & unde Gawan gesendet\\ 
25 & an dem selben mâle\\ 
 & durch strîten nâch dem Grâle.\\ 
 & Kyngrimursel ouch verkôs\\ 
 & ûf den künec, der in dâ verlôs,\\ 
 & daz er im sîn geleite brach.\\ 
30 & vor alden vürsten daz geschach.\\ 
\end{tabular}
\scriptsize
\line(1,0){75} \newline
T U V W \newline
\line(1,0){75} \newline
\textbf{1} \textit{Initiale} V W   $\cdot$ \textit{Majuskel} T  \textbf{13} \textit{Majuskel} T  \textbf{23} \textit{Initiale} T  \textbf{27} \textit{Initiale} U  \newline
\line(1,0){75} \newline
\textbf{3} selbe] [*]: selbe V selber W \textbf{4} missetât] messestat U \textbf{6} gedrungen] verdrungen W \textbf{7} töhtich] dohtich T (U) (V) (W) \textbf{9} den stüendich] den stvͦndich T Do stuͦnt ich U Der stvͤnd ich V (W)  $\cdot$ dîn] \textit{om.} W \textbf{10} hœhstiu] hôhste T hoͤchte W \textbf{13} iuch] îv T \textbf{16} mîner] mine U V (W) \textbf{19} iuch] iv T \textbf{20} geben] [*]: geben V \textbf{23} Dâ] Do U W  $\cdot$ diu] die T \textbf{25} an] Zuͦ W \textbf{27} Kyngrimursel] Kyngrimuͦrsel U Kingrimursel W \textbf{28} dâ] do U V W \textbf{29} daz] Do W \textbf{30} geschach] beschach V \newline
\end{minipage}
\end{table}
\end{document}
