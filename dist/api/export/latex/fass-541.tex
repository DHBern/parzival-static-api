\documentclass[8pt,a4paper,notitlepage]{article}
\usepackage{fullpage}
\usepackage{ulem}
\usepackage{xltxtra}
\usepackage{datetime}
\renewcommand{\dateseparator}{.}
\dmyyyydate
\usepackage{fancyhdr}
\usepackage{ifthen}
\pagestyle{fancy}
\fancyhf{}
\renewcommand{\headrulewidth}{0pt}
\fancyfoot[L]{\ifthenelse{\value{page}=1}{\today, \currenttime{} Uhr}{}}
\begin{document}
\begin{table}[ht]
\begin{minipage}[t]{0.5\linewidth}
\small
\begin{center}*D
\end{center}
\begin{tabular}{rl}
\textbf{541} & \textbf{\begin{large}D\end{large}er} gab ez Gawane\\ 
 & ûf dem Plimizœles plâne.\\ 
 & hie kom sîn trûrec güete\\ 
 & \textbf{aber} wider in hôchgemüete,\\ 
5 & wan daz in twanc ein riwe\\ 
 & unt \textbf{dienstbæriu} triwe,\\ 
 & die er nâch sîner vrouwen truoc,\\ 
 & diu im \textbf{doch} \textbf{smæhe erbôt} genuoc.\\ 
 & nâch der \textbf{jaget in} sîn gedanc.\\ 
10 & Innen des \textbf{der stolze} Lischoys spranc,\\ 
 & dâ er ligen sach sîn \textbf{eigen} swert,\\ 
 & daz Gawan, der degen wert,\\ 
 & mit strîte ûz sîner hende brach.\\ 
 & manec vrouwe ir ander \textbf{strîten} sach.\\ 
15 & Die schilde wâren sô gedigen,\\ 
 & ieweder lie den sînen ligen\\ 
 & und \textbf{gâheten} sus ze strîte.\\ 
 & ietweder kom bezîte\\ 
 & mit herzenlîcher mannes wer.\\ 
20 & ob \textbf{in} saz vrouwen ein her\\ 
 & \textbf{in den} venstern ûf dem palas\\ 
 & unt sâhen kampf, der vor in was.\\ 
 & Dô huop sich êrste niwer zorn.\\ 
 & ietweder was sô hôch geborn,\\ 
25 & daz sîn prîs unsanfte \textbf{leit},\\ 
 & ob in der ander überstreit.\\ 
 & Helme unt \textbf{ir} swerte liten nôt;\\ 
 & \textbf{diu wâren} ir schilde vür \textbf{den} tôt.\\ 
 & swer \textbf{dâ} \textbf{der} helde strîten sach,\\ 
30 & ich wæne, er\textbf{s in} vür kumber jach.\\ 
\end{tabular}
\scriptsize
\line(1,0){75} \newline
D Fr31 \newline
\line(1,0){75} \newline
\textbf{1} \textit{Initiale} D  \textbf{10} \textit{Majuskel} D  \textbf{15} \textit{Majuskel} D  \textbf{23} \textit{Majuskel} D  \textbf{27} \textit{Majuskel} D  \newline
\line(1,0){75} \newline
\textbf{1} Der] Er Fr31 \textbf{2} Plimizœles] Plimizoͤls D Plimyzols Fr31 \textbf{10} Lischoys] Liscoẏs D \newline
\end{minipage}
\hspace{0.5cm}
\begin{minipage}[t]{0.5\linewidth}
\small
\begin{center}*m
\end{center}
\begin{tabular}{rl}
 & \textbf{der} gap ez Gawane\\ 
 & ûf dem Plimizoles plâne.\\ 
 & hie kam sîn \dag trurwic\dag  güete\\ 
 & \textbf{aber} wider in hôchgemüete,\\ 
5 & wen daz in twanc ein riuwe\\ 
 & und \textbf{dienstbæriu} triuwe,\\ 
 & die er nâch sîner vrouwen truoc,\\ 
 & diu ime \textbf{erbôt smæhe} genuoc.\\ 
 & nâch der \textbf{jagte in} sîn gedanc.\\ 
10 & innen des Lischois spranc,\\ 
 & d\textit{â} er ligen sach sîn swert,\\ 
 & daz \textbf{im} Gawan, der degen wert,\\ 
 & mit strîte ûz sîner hende \textit{b}rach.\\ 
 & manic vrouwe ir ander \textbf{strîte} sach.\\ 
15 & die schilt wâren sô gedigen,\\ 
 & ietweder liez den sînen ligen\\ 
 & und \textbf{gâhete} sus zuo strîte.\\ 
 & iet\textit{w}eder kam bî zîte\\ 
 & mit herzelîcher mannes wer.\\ 
20 & ob \textbf{in} saz vrouwen ein her\\ 
 & \textbf{in den} venstern ûf dem palas\\ 
 & und sâhen kampf, der vor in was.\\ 
 & dô huop sich êrste niuwer zorn.\\ 
 & ietweder w\textit{a}s sô hôch geborn,\\ 
25 & daz sîn prîs unsanfte \textbf{leit},\\ 
 & ob in der ander überstr\textit{e}it.\\ 
 & helm und swert, \textbf{die} liten nôt:\\ 
 & \textbf{si buten} ir schilt vür \textbf{i\textit{r}} tôt.\\ 
 & wer \textbf{d\textit{â}} \textbf{der} helde strîten sach,\\ 
30 & ich wæne, er \textbf{sîn} \textit{vür} kumber jach.\\ 
\end{tabular}
\scriptsize
\line(1,0){75} \newline
m n o \newline
\line(1,0){75} \newline
\newline
\line(1,0){75} \newline
\textbf{2} Plimizoles] plimizols m n plimzol o \textbf{3} trurwig] truwig n o \textbf{8} ime erbôt] erbot jme n \textbf{10} Lischois] liscois m n o \textbf{11} dâ er] Do er m n Der do o \textbf{13} brach] sprach m o \textbf{14} ir ander] er andem n \textbf{15} die] So o \textbf{17} gâhete] gahate o \textbf{18} ietweder] Yetteder m \textbf{23} dô] So o \textbf{24} was] wes m \textbf{26} in] ir o  $\cdot$ überstreit] uͯberstrit m \textbf{27} nôt] noit o \textbf{28} ir schilt] den schilt n  $\cdot$ ir tôt] in tot m den dot o \textbf{29} dâ] do m n o  $\cdot$ helde] helle o \textbf{30} vür] \textit{om.} m  $\cdot$ kumber] [komenn]: komen o \newline
\end{minipage}
\end{table}
\newpage
\begin{table}[ht]
\begin{minipage}[t]{0.5\linewidth}
\small
\begin{center}*G
\end{center}
\begin{tabular}{rl}
 & \textbf{\begin{large}E\end{large}r} gab ez Gawane\\ 
 & ûf dem Blimzoles plâne.\\ 
 & hie ko\textit{m} sîn trûric güete\\ 
 & \textbf{aber} wider in hôchgemüete,\\ 
5 & wan daz in twanc ein riuwe\\ 
 & unde \textbf{dienstbæriu} triuwe,\\ 
 & die er nâch sîner vrouwen truoc,\\ 
 & diu im \textbf{doch} \textbf{smæhe erbôt} genuoc.\\ 
 & nâch der \textbf{jaget in} sîn gedanc.\\ 
10 & inne des \textbf{der stolze} Lishois spranc,\\ 
 & dâ er ligen sach sîn \textbf{eigen} swert,\\ 
 & daz Gawan, der degen wert,\\ 
 & mit strîte ûz sîner hende brach.\\ 
 & man\textit{ic} vrouwe ir ander \textbf{strîten} sach.\\ 
15 & die schilde wâren sô gedigen,\\ 
 & ietweder lie den sînen ligen\\ 
 & unde \textbf{gâhten} sus ze strîte.\\ 
 & ietweder kom bezîte\\ 
 & mit herzenlîcher mannes wer.\\ 
20 & obe \textbf{in} saz vrouwen ein her\\ 
 & \textbf{in den} venstern ûf dem palas\\ 
 & unde sâhen kampf, der vor in was.\\ 
 & dô huop sich êrste niuwer zorn.\\ 
 & ietweder was sô hôch geborn,\\ 
25 & daz sîn brîs unsamfte \textbf{leit},\\ 
 & obe in der ander überstreit.\\ 
 & helm unde \textbf{ir} swert, \textbf{die} liten nôt;\\ 
 & \textbf{di\textit{u} wâren} ir schilde vür \textbf{den} tôt.\\ 
 & swer \textbf{dâ} \textbf{der} helde strîten sach,\\ 
30 & ich wæne, er\textbf{s in} vür kumber jach.\\ 
\end{tabular}
\scriptsize
\line(1,0){75} \newline
G I L M Z \newline
\line(1,0){75} \newline
\textbf{1} \textit{Initiale} G L Z  \textbf{3} \textit{Initiale} I  \textbf{21} \textit{Initiale} I  \newline
\line(1,0){75} \newline
\textbf{1} Er] DEr L (M) (Z)  $\cdot$ Gawane] Gawan I \textbf{2} Blimzoles] blimzols G blimizols I plymizolles L plimizcols M plimizols Z \textbf{3} kom] chome G \textbf{5} in twanc] entwanc M \textbf{8} doch] \textit{om.} L  $\cdot$ erbôt] bot I \textbf{9} jaget in] lagen I iagiten M \textbf{10} der stolze] \textit{om.} M  $\cdot$ Lishois] lishoys G liscoys I Lýtschoys L lisois M \textbf{12} Gawan] [Gewan]: GAwan I \textbf{14} manic] Man G \textbf{15} die] ir I \textbf{16} ietweder] ir ietdweder I (M) \textbf{17} \textit{Die Verse 541.17-18 fehlen} L  \textbf{19} herzenlîcher] hertzeliches L herteclicher Z \textbf{21} \textit{Versfolge 541.22-21} M   $\cdot$ den venstern] dem venster I  $\cdot$ ûf] von M \textbf{22} kampf] den champh I (M)  $\cdot$ in] in da I \textbf{23} dô] Da Z  $\cdot$ niuwer] Nuwe M \textbf{27} ir] \textit{om.} I L M  $\cdot$ die] \textit{om.} I \textbf{28} diu] die G I  $\cdot$ schilde] shilt I \textbf{29} swer] swa I Wer L M  $\cdot$ der] die I  $\cdot$ helde] heldin M \textbf{30} in] ev Z \newline
\end{minipage}
\hspace{0.5cm}
\begin{minipage}[t]{0.5\linewidth}
\small
\begin{center}*T
\end{center}
\begin{tabular}{rl}
 & \textbf{der} gab ez Gawane\\ 
 & ûf dem Plymizoles plâne.\\ 
 & \textit{\begin{large}H\end{large}}ie kom sîn trûric güete\\ 
 & wider in hôchgemüete,\\ 
5 & wan daz in twanc ein riuwe\\ 
 & unde \textbf{ein} \textbf{dienstberndiu} triuwe,\\ 
 & die er nâch sîner vrouwen truoc,\\ 
 & diu im \textbf{smæhe bôt} genuoc.\\ 
 & nâch der \textbf{in jagete} sîn gedanc.\\ 
10 & Indes \textbf{der stolze} Lyschoys \textbf{ûf} spranc,\\ 
 & dâ er ligen sach sîn swert,\\ 
 & daz Gawan, der degen wert,\\ 
 & mit strîte ûz sîner hende brach.\\ 
 & manec vrouwe ir ander \textbf{strîten} sach.\\ 
15 & Die schilte wâren sô gedigen,\\ 
 & ietweder lie den sînen ligen\\ 
 & unde \textbf{gâhten} sus ze strîte.\\ 
 & ietweder kom bezîte\\ 
 & mit herzenlîcher mannes wer.\\ 
20 & ob \textbf{im} saz vrouwen ein her\\ 
 & \textbf{zen} venstern ûf dem palas\\ 
 & unde sâhen \textbf{den} kampf, der vor in was.\\ 
 & Dô huop sich êrst niuwer zorn.\\ 
 & ietweder was sô hôch geborn,\\ 
25 & daz sîn prîs unsanfte \textbf{erleit},\\ 
 & ob in der ander überstreit.\\ 
 & helme unde swert, \textbf{die} liten nôt;\\ 
 & \textbf{di\textit{u} wâren} ir schilte vür \textbf{den} tôt.\\ 
 & \textit{\begin{large}S\end{large}}wer \textbf{die} helde strîten sach,\\ 
30 & ich wæne, er\textbf{\textit{s} in} vür kumber jach.\\ 
\end{tabular}
\scriptsize
\line(1,0){75} \newline
T U V W O Q R Fr40 \newline
\line(1,0){75} \newline
\textbf{3} \textit{Initiale} T U V O R Fr40  \textbf{10} \textit{Majuskel} T  \textbf{15} \textit{Majuskel} T  \textbf{23} \textit{Majuskel} T  \textbf{29} \textit{Initiale} T U  \newline
\line(1,0){75} \newline
\textbf{1} Gawane] Gawaine R \textbf{2} Plymizoles] Plymizols T (U) plimizols V W Q Fr40 brimizols O (R) \textbf{3} Hie] ÷ie T O Do U  $\cdot$ güete] gemuͯte R \textbf{6} dienstberndiu] dinstbrende Q dientstbere R \textbf{8} smæhe] leides O \textbf{9} der] den Q  $\cdot$ in jagete] iagt in O ye gachte Q \textbf{10} Indes] Inne W (Q) inner des Fr40  $\cdot$ stolze] stoltzen W  $\cdot$ Lyschoys] Lyscois T Lyschois U lishoys W Lẏhoẏs O lischois Q R [lisl*]: lisklois Fr40 \textbf{11} dâ] Do U V W Q  $\cdot$ swert] schwerr W \textbf{12} Gawan] Gewin R  $\cdot$ der] den W  $\cdot$ degen] degent Fr40 \textbf{13} ûz] vsser R  $\cdot$ brach] brachte Q \textbf{14} ander] anders Q aber Fr40 \textbf{15} schilte] scheide Fr40 \textbf{16} den] di Fr40 \textbf{17} gâhten sus] gachte als Q \textbf{18} kom] site quam U \textbf{19} herzenlîcher] herlicher Q \textbf{20} im] in U V O Q (R) Fr40  $\cdot$ her] grosses her W \textbf{22} vor] ob W \textbf{23} niuwer] ein neúwer W \textbf{25} erleit] leit Q \textbf{26} überstreit] wider streit Q (Fr40) \textbf{27} die] \textit{om.} O R  $\cdot$ liten] [liden]: leiden Q \textbf{28} diu] die T (O)  $\cdot$ wâren] frawen Q \textbf{29} Swer] ÷wer T Wer U W Q R  $\cdot$ die] der W O Q R Fr40  $\cdot$ helde] helden R [helden]: helde Fr40  $\cdot$ strîten] [*]: strîten V stritte R \textbf{30} ers in] erz in T er ins W (O) \newline
\end{minipage}
\end{table}
\end{document}
