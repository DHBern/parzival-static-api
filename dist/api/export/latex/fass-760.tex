\documentclass[8pt,a4paper,notitlepage]{article}
\usepackage{fullpage}
\usepackage{ulem}
\usepackage{xltxtra}
\usepackage{datetime}
\renewcommand{\dateseparator}{.}
\dmyyyydate
\usepackage{fancyhdr}
\usepackage{ifthen}
\pagestyle{fancy}
\fancyhf{}
\renewcommand{\headrulewidth}{0pt}
\fancyfoot[L]{\ifthenelse{\value{page}=1}{\today, \currenttime{} Uhr}{}}
\begin{document}
\begin{table}[ht]
\begin{minipage}[t]{0.5\linewidth}
\small
\begin{center}*D
\end{center}
\begin{tabular}{rl}
\textbf{760} & mit grôzer rîchen hôchgezît.\\ 
 & mich müet iwer beider strît.\\ 
 & dâ sult ir \textbf{bî mir ruowen} nâch.\\ 
 & sît aber strît von iu \textbf{geschach},\\ 
5 & ir erkennet ein ander deste baz.\\ 
 & nû kieset vriwentschaft vür den haz."\\ 
 & Gawan des \textbf{âbents} az dest ê,\\ 
 & \textbf{daz} sîn neve von Thasme,\\ 
 & Feirefiz Anschevin,\\ 
10 & dennoch vaste, unt der bruoder sîn.\\ 
 & \textbf{Matraze dicke} unde lanc,\\ 
 & der wart ein wîter \textbf{umbevanc}.\\ 
 & kultern maneger künne\\ 
 & von balmâte, niht ze dünne,\\ 
15 & \textbf{wurden} dô der matraze dach.\\ 
 & tiwern pfelle man drûf gesteppet sach,\\ 
 & beidiu lanc und breit.\\ 
 & diu \textbf{Clinschors} rîcheit\\ 
 & wart dâ ze schouwen vür getragen.\\ 
20 & dô sluoc man ûf, sus hôrt ich sagen,\\ 
 & von pfelle \textbf{vier} \textbf{ruclachen}\\ 
 & mit rîlîchen sachen,\\ 
 & gein ein ander \textbf{viersîte},\\ 
 & dâr unde \textbf{senfte} plûmîte\\ 
25 & mit kultern verdecket,\\ 
 & ruclachen drüber \textbf{gestecket}.\\ 
 & der rinc begreif \textbf{sô wît ein} velt,\\ 
 & dâ \textbf{wæren} gestân \textbf{sehs} gezelt\\ 
 & \begin{large}Â\end{large}ne gedrenge der snüere.\\ 
30 & unbescheidenlîche ich vüere,\\ 
\end{tabular}
\scriptsize
\line(1,0){75} \newline
D \newline
\line(1,0){75} \newline
\textbf{11} \textit{Majuskel} D  \textbf{29} \textit{Initiale} D  \newline
\line(1,0){75} \newline
\textbf{9} Anschevin] Anscevin D \textbf{18} Clinschors] Clinscors D \newline
\end{minipage}
\hspace{0.5cm}
\begin{minipage}[t]{0.5\linewidth}
\small
\begin{center}*m
\end{center}
\begin{tabular}{rl}
 & mit grôzer rîcher hôchzît.\\ 
 & mich müejet iuwer beider strît.\\ 
 & d\textit{â} solt ir \textbf{bî mir ruowen} nâch.\\ 
 & sît aber strît von iu \textbf{beschach},\\ 
5 & ir e\textit{r}kenne\textit{t} ein ander deste baz.\\ 
 & nû kieset vriuntschaft vür den haz."\\ 
 & \begin{large}G\end{large}awan des \textbf{âbendes} az deste ê,\\ 
 & \textbf{daz} sîn neve von Thas\textit{m}e,\\ 
 & Ferefiz A\textit{n}schevin,\\ 
10 & dannoch vast, und der bruoder sîn.\\ 
 & \textbf{matraz dic} und lanc,\\ 
 & der wart ein wîter \textbf{umbevanc}.\\ 
 & kulter maniger künne\\ 
 & von palmât, niht zuo dünne,\\ 
15 & \textbf{wâren} dô der matraz dach.\\ 
 & tiur pfelle \textit{man} dâr ûf ges\textit{t}ep\textit{p}et sach,\\ 
 & beidiu lanc und breit.\\ 
 & diu \textbf{Clinsors} rîcheit\\ 
 & wart d\textit{â} ze schouwen vür getragen.\\ 
20 & dô sluoc man ûf, sus hôrt ich sagen,\\ 
 & von pfelle \textbf{vier} \textbf{ruclachen}\\ 
 & mit rîchlîchen sachen,\\ 
 & gegen ein ander \textbf{viersîte},\\ 
 & dâr under pl\textit{û}m\textit{î}te\\ 
25 & mit kulter\textit{n} verdecket,\\ 
 & ruclachen dâr über \textbf{gestecket}.\\ 
 & der rinc begreif \textbf{ein wîtez} velt,\\ 
 & d\textit{â} \textbf{wæren} gestanden \textbf{sehs} gezelt\\ 
 & âne gedrenge der snüer.\\ 
30 & unbescheidenlîch ich vüer,\\ 
\end{tabular}
\scriptsize
\line(1,0){75} \newline
m n o V V' \newline
\line(1,0){75} \newline
\textbf{7} \textit{Initiale} m V   $\cdot$ \textit{Capitulumzeichen} n  \newline
\line(1,0){75} \newline
\textbf{1} \textit{Die Verse 759.29-760.30 fehlen} V'   $\cdot$ hôchzît] hochgezit n (o) V \textbf{3} dâ] Do m n o V \textbf{4} beschach] geschach V \textbf{5} erkennet] enkennen m erkennen n o \textbf{8} Thasme] thasine m o \textbf{9} Ferefiz] ferefis m o Ferrefis n Ferevis V  $\cdot$ Anschevin] auscevin m ausce vin n ansce vin o \textbf{10} vast] fastent n \textbf{11} matraz] Martrag o \textbf{13} kulter] Kultern n o (V) \textbf{15} wâren] Wurden n (V) Werden o \textbf{16} tiur] Dv́ren V  $\cdot$ man] \textit{om.} m n o  $\cdot$ gesteppet] geschepphet m  $\cdot$ sach] [wasz]: sach o \textbf{19} dâ] do m n o V \textbf{20} sagen] [clagen]: sagen o \textbf{24} plûmîte] plimutte m pluͯmte o \textbf{25} kultern] kultter m \textbf{26} über] vnder n  $\cdot$ gestecket] gestrecket o \textbf{27} ein wîtez] so wit ein n (o) (V) \textbf{28} dâ] Do m n o V \newline
\end{minipage}
\end{table}
\newpage
\begin{table}[ht]
\begin{minipage}[t]{0.5\linewidth}
\small
\begin{center}*G
\end{center}
\begin{tabular}{rl}
 & \begin{large}M\end{large}it grôzer rîcher hôchzît.\\ 
 & mich müet iwer bêder strît.\\ 
 & dâ sült ir \textbf{sîn mit triwen} nâch.\\ 
 & sît aber strît von iu \textbf{geschach},\\ 
5 & ir erkennt ein ander deste baz.\\ 
 & nû kieset vriuntschaft vür den haz."\\ 
 & Gawan des \textbf{tages} az deste ê,\\ 
 & \textbf{dô} sîn neve von Tasme,\\ 
 & Feirafiz Antschevin,\\ 
10 & dannoch vaste, unde der bruoder sîn.\\ 
 & \textbf{von palmât wît} unde lanc\\ 
 & d\textit{e}r wart ein wît \textbf{umbehanc}.\\ 
 & kulter maniger künne\\ 
 & von palmât, niht ze dünne,\\ 
15 & \textbf{wurden} dâ der matraz dach.\\ 
 & tiwer pfelle man drûf gesteppet sach,\\ 
 & beidiu lanc unde breit.\\ 
 & diu \textbf{hêrlîche} rîcheit\\ 
 & wart dâ ze schouwene vür getragen.\\ 
20 & dô sluoc man ûf, sus hôrte ich sagen,\\ 
 & von pfelle \textbf{niwe} \textbf{ruclachen}\\ 
 & mit rîchlîchen sachen,\\ 
 & gein ein ander \textbf{niwe site},\\ 
 & dâr under \textbf{senfte} pf\textit{l}ûmîte\\ 
25 & mit kultern verdecket,\\ 
 & ruclachen dâr über \textbf{gestrecket}.\\ 
 & der rinc begreif \textbf{sô wît ein} velt,\\ 
 & dâ \textbf{wâren} gestanden \textbf{vier} gezelt\\ 
 & âne gedrenge der snüere.\\ 
30 & unbescheidenlîche ich vüere,\\ 
\end{tabular}
\scriptsize
\line(1,0){75} \newline
G I L M Z Fr48 Fr70 \newline
\line(1,0){75} \newline
\textbf{1} \textit{Initiale} G L Z Fr48  \newline
\line(1,0){75} \newline
\textbf{1} rîcher] richeit L  $\cdot$ hôchzît] hohgezit L hocheit M hohezit Fr70 \textbf{3} sîn mit triwen] by mir ruwen M (Z) (Fr48) (Fr70) \textbf{4} iu] ir Fr48 \textbf{5} ein] an ein Fr48 \textbf{6} den] \textit{om.} L \textbf{7} tages] abens L (M) (Z) (Fr48) (Fr70) \textbf{8} dô] Da M Z  $\cdot$ Tasme] Tasine L thasme M \textbf{9} Feirafiz] feirefiz G (Z) Ferefiz L Fereviesz M Feirefiez Fr48 Ferasiz Fr70  $\cdot$ Antschevin] Anschevin G (M) antsheuin I anshevin L Z ansheuin Fr48 Anzevin Fr70 \textbf{11} von palmât wît] Von palmat dicke L (M) (Fr70) Matraz dicke Z (Fr48) \textbf{12} der] dar G \textit{om.} Fr70  $\cdot$ wît] \textit{om.} I  $\cdot$ umbehanc] vmbeuanc I (M) (Z) (Fr48) (Fr70) \textbf{13} kulter] koltren Fr70 \textbf{15} wurden] Vorden M  $\cdot$ dâ] \textit{om.} L do Fr48 Fr70 \textbf{16} tiwer] Tiuren I Thuru M \textit{om.} Fr70  $\cdot$ pfelle] pfellet Fr70  $\cdot$ gesteppet] steppin M \textbf{18} diu] By M  $\cdot$ hêrlîche] clinsors M Clingshors Z Chlinshors Fr48 Clinsores Fr70 \textbf{19} dâ] do Fr48  $\cdot$ ze] durch Fr70 \textbf{20} dô] Da L M Z  $\cdot$ sus] \textit{om.} L  $\cdot$ hôrte] hor Fr70 \textbf{21} niwe] vier M Z (Fr48) Fr70 \textbf{22} mit] mit vil Fr70 \textbf{23} niwe site] vier site M Z Fr48 vor sniten Fr70 \textbf{24} senfte] slehte I  $\cdot$ pflûmîte] phvmite G pulmit I plumiten Fr70 \textbf{25} kultern] Gulter I \textbf{26} über] vnder I (Z)  $\cdot$ gestrecket] gitreckit M (Fr70) gestecket Z Fr48 \textbf{27} sô wît ein] ein so wit Fr70 \textbf{28} dâ] dan Fr70  $\cdot$ wâren] wern I (M) (Fr70)  $\cdot$ vier] vierzech I vie Fr48 \textbf{29} \textit{Vers 760.29 fehlt} Fr70  \textbf{30} unbescheidenlîche] vnbeshadenlichen I \newline
\end{minipage}
\hspace{0.5cm}
\begin{minipage}[t]{0.5\linewidth}
\small
\begin{center}*T
\end{center}
\begin{tabular}{rl}
 & mit grôzer rîcher hôchgezît.\\ 
 & mich müet iuwer beider strît.\\ 
 & d\textit{â} sult ir \textbf{bî mir ruowen} nâch.\\ 
 & sît aber strît von iu \textbf{geschach},\\ 
5 & ir erkennet ein ander deste baz.\\ 
 & nû kieset vriuntschaft vür den haz."\\ 
 & Gawan des \textbf{âbendes} az deste ê,\\ 
 & \textbf{dô} sîn neve von Tasme,\\ 
 & Ferefis \textbf{von} Anschevin,\\ 
10 & dannoch vast\textit{e}, und der bruoder sîn.\\ 
 & \textbf{von palmât dicke} und lanc\\ 
 & der wart ein wît \textbf{umbehanc}.\\ 
 & kultern maneger künne\\ 
 & von palmât, niht zuo dünne,\\ 
15 & \textbf{wurden} dô der matraz dach.\\ 
 & tiure pfelle man drûf gesteppet sach,\\ 
 & beidiu lanc und \textbf{beidiu} breit.\\ 
 & diu \textbf{Clynsors} rîcheit\\ 
 & wart d\textit{â} zuo schouwene vür getragen.\\ 
20 & dô s\textit{l}uoc man ûf, sus hôrt ich sagen,\\ 
 & von pfelle \textbf{vier} \textbf{kurze lachen}\\ 
 & mit rîchlîchen sachen,\\ 
 & gein ein ander \textbf{viersîte},\\ 
 & dâr under \textbf{sanfte} plûmîte\\ 
25 & mit kultern verdecket,\\ 
 & rückelachen dâr über \textbf{gestecket}.\\ 
 & der rinc begreif \textbf{sô wît ein} velt,\\ 
 & d\textit{â} \textbf{wæren} gestanden \textbf{vier} gezelt\\ 
 & âne gedrenge der snüere.\\ 
30 & un\textit{b}escheidenlîche ich vüere,\\ 
\end{tabular}
\scriptsize
\line(1,0){75} \newline
U W Q R \newline
\line(1,0){75} \newline
\newline
\line(1,0){75} \newline
\textbf{1} hôchgezît] hochtzeit Q (R) \textbf{2} müet] reúwer W \textbf{3} dâ] Do U W Q \textbf{5} erkennet] kennet W \textbf{7} Gawan] Gawin R  $\cdot$ âbendes] aubent R \textbf{8} sîn] sine R  $\cdot$ Tasme] thasme R \textbf{9} Ferefis] Ferafis W feirefisz Q Feirefis R  $\cdot$ von] \textit{om.} W Q R  $\cdot$ Anschevin] antscheuin W anschevein Q Anshevin R \textbf{10} vaste] [v*er]: vaster U \textbf{11} \textit{Die Verse 760.11-13 fehlen} Q   $\cdot$ palmât] palmant R \textbf{12} der] Da R  $\cdot$ umbehanc] vmbeuang W (R) \textbf{13} kultern] Kulter W (R) \textbf{15} dô] do da R \textbf{16} gesteppet] \textit{om.} W \textbf{17} beidiu lanc] Gesteppet baide lang W  $\cdot$ beidiu breit] brait W (Q) (R) \textbf{18} Clynsors] klinshors W clinszhor Q Clinshor R \textbf{19} dâ] do U W Q R \textbf{20} sluoc] suͦc U  $\cdot$ sus] daz W als Q \textbf{21} kurze lachen] rúckelachen W (Q) (R) \textbf{23} ander] an::en R  $\cdot$ viersîte] vier sitten R \textbf{24} plûmîte] pulmeite W [pfulm*]: pfulmatten R \textbf{26} rückelachen] kuckelachen Q  $\cdot$ gestecket] gestrecket Q (R) \textbf{28} dâ] Do U W Q  $\cdot$ wæren] waren R \textbf{30} unbescheidenlîche] Vngescheidenliche U Gar vnbeschaidenlich W \newline
\end{minipage}
\end{table}
\end{document}
