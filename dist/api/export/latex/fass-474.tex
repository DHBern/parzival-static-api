\documentclass[8pt,a4paper,notitlepage]{article}
\usepackage{fullpage}
\usepackage{ulem}
\usepackage{xltxtra}
\usepackage{datetime}
\renewcommand{\dateseparator}{.}
\dmyyyydate
\usepackage{fancyhdr}
\usepackage{ifthen}
\pagestyle{fancy}
\fancyhf{}
\renewcommand{\headrulewidth}{0pt}
\fancyfoot[L]{\ifthenelse{\value{page}=1}{\today, \currenttime{} Uhr}{}}
\begin{document}
\begin{table}[ht]
\begin{minipage}[t]{0.5\linewidth}
\small
\begin{center}*D
\end{center}
\begin{tabular}{rl}
\textbf{474} & \textit{\begin{large}H\end{large}}êrre, sît irz, Læhelin?\\ 
 & sô stêt in dem stalle mîn\\ 
 & den orsen \textbf{ein ors} gelîch gevar,\\ 
 & diu dâ hœren\textit{t} \textbf{an}s Grâles schar.\\ 
5 & ame satel ein turteltûbe stêt.\\ 
 & daz ors von Munsalvæsche gêt.\\ 
 & diu wâpen gap \textbf{in} Anfortas,\\ 
 & dô er der vreuden hêrre was.\\ 
 & ir schilte sint von alter sô.\\ 
10 & Titurel si brâhte dô\\ 
 & an sînen sun, \textbf{rois} Frimutel.\\ 
 & dâr unde vlôs \textbf{der} degen snel\\ 
 & von \textbf{einer} tjoste ouch sînen lîp.\\ 
 & der minnet sîn selbes wîp,\\ 
15 & daz nie von manne mêre\\ 
 & wîp geminnet wart sô sêre;\\ 
 & ich meine, mit rehten triwen.\\ 
 & sîne site sult ir niwen\\ 
 & und minnet von herzen iwer konen.\\ 
20 & \textbf{sîner site sult ir} wonen.\\ 
 & iwer varwe \textbf{im treit} gelîchiu mâl.\\ 
 & \textbf{der} was ouch hêrre übern Grâl.\\ 
 & owî, hêrre, wannen ist iwer vart?\\ 
 & nû ruochet mir prüeven iwern art."\\ 
25 & ieweder vaste an den andern sach.\\ 
 & Parzival zem wirte sprach:\\ 
 & "ich bin von einem man \textbf{erborn},\\ 
 & der mit tjoste hât den lîp verlorn\\ 
 & unt durch rîterlîch gemüete.\\ 
30 & hêrre, durch iwer güete\\ 
\end{tabular}
\scriptsize
\line(1,0){75} \newline
D \newline
\line(1,0){75} \newline
\textbf{1} \textit{Initiale} D  \newline
\line(1,0){75} \newline
\textbf{1} Hêrre] ÷erre \textit{nachträglich korrigiert zu:} Herre D \textbf{4} hœrent] horen D \textbf{6} Munsalvæsche] Mvnsælvæsche D \textbf{10} Titurel] Tytvrel D \textbf{11} Frimutel] Frimvtel D \textbf{26} Parzival] Parcifal D \newline
\end{minipage}
\hspace{0.5cm}
\begin{minipage}[t]{0.5\linewidth}
\small
\begin{center}*m
\end{center}
\begin{tabular}{rl}
 & hêrre, sît irz, Lehelin?\\ 
 & sô stât in dem stalle mîn\\ 
 & den rossen \textbf{ein ros} glîch gevar,\\ 
 & diu d\textit{â} hœrent \textbf{an} des Grâles schar.\\ 
5 & an dem satel ein turt\textit{e}ltûbe stât.\\ 
 & daz ros von Muntsalvasche gât.\\ 
 & diu wâpen gap \textbf{in} Anfortas,\\ 
 & dô er der vröu\textit{d}en hêrre was.\\ 
 & ir schilte \textit{sint von} alter sô.\\ 
10 & Titurel si brâ\textit{h}te dô\\ 
 & an sînen sun, \textbf{rois} Fr\textit{i}mutel.\\ 
 & dâr under vlôs \textbf{der} degen snel\\ 
 & von \textbf{einer} juste ouch sînen lîp.\\ 
 & der minnete sîn selbes wîp,\\ 
15 & daz nie von mann\textit{e} \textit{m}êre\\ 
 & wîp geminnet wart sô sêre;\\ 
 & ich mein, mit rehten triuwen.\\ 
 & sînen site solt ir niuwen\\ 
 & und minnet von herzen iuwere konen.\\ 
20 & \textbf{sîner site solt ir} wonen.\\ 
 & \textit{i}uwer varwe \textbf{treit im} glîchiu mâl.\\ 
 & \textbf{der} was ouch hêrre über den Grâl.\\ 
 & ouwê, hêrre, wannen ist iuwer vart?\\ 
 & nû ruochet mir prüeven iuwer art."\\ 
25 & ietweder vast an den andern sach.\\ 
 & Parcifal zuo dem wirte sprach:\\ 
 & "ich bin von einem man \textbf{erborn},\\ 
 & der mit juste het den lîp verlorn\\ 
 & und durch ritterlîch gemüete.\\ 
30 & hêrre, durch iuwer güete\\ 
\end{tabular}
\scriptsize
\line(1,0){75} \newline
m n o \newline
\line(1,0){75} \newline
\newline
\line(1,0){75} \newline
\textbf{2} stât] stant n \textbf{4} dâ] do m n \textbf{5} turteltûbe] turtultube m \textbf{6} Muntsalvasche] [munt]: muntsaluasce m muntsaluasce n montsaluasce o \textbf{7} Anfortas] an fortas n \textbf{8} vröuden] frouwen m \textbf{9} sint von] von sint m  $\cdot$ sô] kommen [sint]: so o \textbf{10} Titurel] Titturel n Tyturel o  $\cdot$ brâhte] brabtte m  $\cdot$ dô] [fur]: do o \textbf{11} Frimutel] frumuͯtel m frimuͯtel n frunutel o \textbf{15} manne mêre] manne were vnd mere m \textbf{16} wîp] Wart o \textbf{18} sînen] Sinem o  $\cdot$ solt] soltent n solte o \textbf{19} iuwere] ire m o ir n \textbf{20} sîner] Sinen o  $\cdot$ solt] solte o \textbf{21} iuwer] Vuͯwer m  $\cdot$ varwe] froͧwe n \textbf{24} prüeven iuwer] pruͯffent uwern n \textbf{28} verlorn] do verlorn n \newline
\end{minipage}
\end{table}
\newpage
\begin{table}[ht]
\begin{minipage}[t]{0.5\linewidth}
\small
\begin{center}*G
\end{center}
\begin{tabular}{rl}
 & \begin{large}H\end{large}êrre, sît irz, Lehelin?\\ 
 & sô stêt in dem stalle mîn\\ 
 & den orsen \textbf{ein ors} gelîch gevar,\\ 
 & diu dâ hœrent \textbf{in}s Grâles schar.\\ 
5 & ame satel eine turteltûbe stêt.\\ 
 & daz ors von Muntsalvatsche gêt.\\ 
 & diu wâpen gap \textbf{in} Anfortas,\\ 
 & dô er der vröuden hêrre was.\\ 
 & ir schilde sint von alter sô.\\ 
10 & \begin{large}T\end{large}iturel si brâhte dô\\ 
 & an sînen sun, \textbf{roy} Frimutel.\\ 
 & dâr under vlôs \textbf{der} degen snel\\ 
 & von \textbf{einer} tjoste ouch sînen lîp.\\ 
 & der minnete sînes selbes wîp,\\ 
15 & daz nie von manne mêre\\ 
 & wîp geminnet wart sô sêre;\\ 
 & ich meine, mit rehten triuwen.\\ 
 & sîne site sult ir niuwen\\ 
 & unde minnet von herzen iuwer konen.\\ 
20 & \textbf{sîner site sult ir} wonen.\\ 
 & iuwer varwe \textit{\textbf{im treit}} gelîchiu mâl.\\ 
 & \textbf{er} was ouch hêrre über den Grâl.\\ 
 & owê, hêrre, wannen ist iuwer vart?\\ 
 & nû ruochet mir prüeven iuwern art."\\ 
25 & ietweder vaste an den andern sach.\\ 
 & Parzival ze dem wirte sprach:\\ 
 & "ich bin von eine\textit{m} man \textbf{\textit{ge}born},\\ 
 & der mit tjost hât den lîp verlorn\\ 
 & unt durch rîterlîch gemüete.\\ 
30 & hêrre, durch iuwer güete\\ 
\end{tabular}
\scriptsize
\line(1,0){75} \newline
G I O L M Z Fr18 Fr49 \newline
\line(1,0){75} \newline
\textbf{1} \textit{Initiale} G I O L Z Fr18  \textbf{10} \textit{Initiale} G  \textbf{17} \textit{Initiale} I  \newline
\line(1,0){75} \newline
\textbf{1} Hêrre] ÷erre O  $\cdot$ Lehelin] lehehelin I læhelin O lehelein Fr49 \textbf{3} den orsen] Dem oͤrs O  $\cdot$ gevar] var O \textbf{4} diu] die I Fr49  $\cdot$ ins] ans O (L) Z Fr18 usz M \textbf{5} eine turteltûbe] [eich]: eine turtel tube G ein tvͯrteltuͯbelin L \textbf{6} Muntsalvatsche] muntsalualgce I Munsalvatsche M montsalvatsche Z muntschalualsch Fr49 \textbf{7} in] im I Z Fr49 \textit{om.} M  $\cdot$ Anfortas] Amfortas L \textbf{8} dô] Da O M Z  $\cdot$ vröuden] vreude I (Fr49) vrouwen M \textbf{9} von] vor I Fr49  $\cdot$ sô] ho L \textbf{10} Titurel] Tytvrel O Tituͯrel L TẏtuRel Fr18  $\cdot$ dô] da M \textbf{11} sînen] sinem L  $\cdot$ Frimutel] frimvtel G Fr18 frimuntel I (O) Frýmvtel L frymuͯtel M \textbf{13} ouch] \textit{om.} I O Fr18 Fr49 \textbf{14} minnete] mýnnet L (Z) meynete M \textbf{15} manne] minnen I minne Fr49 \textbf{16} geminnet wart] wart geminnet I (Fr49) \textbf{18} sîne] sin I (Fr18) Sit Z \textbf{19} \textit{Die Verse 474.19-20 fehlen} L  \textbf{20} sîner site] sinen sit I Siner M \textbf{21} im treit] treit im G  $\cdot$ gelîchiu] gelichen O \textbf{22} er] Der O L M Z Fr18  $\cdot$ den] \textit{om.} Z \textbf{23} owê] Awi O Owý L (Z) (Fr18)  $\cdot$ wannen] von wannen I (O) \textbf{24} nû] \textit{om.} I  $\cdot$ ruochet] helffit M  $\cdot$ iuwern] iwer O (L) (M) \textbf{25} ietweder] ir ietdweder I Ja wilchir M  $\cdot$ vaste] \textit{om.} M  $\cdot$ den andern] ein ander O an andern Fr18 \textbf{26} Parzival] parzifal I (M) Parcifal O Z (Fr18)  $\cdot$ sprach] sparch L \textbf{27} einem] ainen G  $\cdot$ man] [wirt]: man O  $\cdot$ geborn] erborn G \newline
\end{minipage}
\hspace{0.5cm}
\begin{minipage}[t]{0.5\linewidth}
\small
\begin{center}*T
\end{center}
\begin{tabular}{rl}
 & hêrre, sît irz, Lehelin?\\ 
 & sô stât in dem stalle mîn\\ 
 & den orsen \textbf{einez} glîch gevar,\\ 
 & di\textit{u} dâ hœrent \textbf{an} des Grâles schar.\\ 
5 & an dem satele ein turteltûbe stêt.\\ 
 & daz ors von Munsalvasche gêt.\\ 
 & diu wâpen gap \textbf{im} Anfortas,\\ 
 & dô er der vröuden hêrre was.\\ 
 & ir schilte sint von alter sô.\\ 
10 & Tyturel si brâhte dô\\ 
 & an sînen sun Frimutel.\\ 
 & dâr under verlôs \textbf{ein} degen snel\\ 
 & von \textbf{sîner} tjost ouch sînen lîp.\\ 
 & der minnete sîn se\textit{l}bes wîp,\\ 
15 & daz nie von manne mêre\\ 
 & wîp geminnet wart sô sêre;\\ 
 & ich meine, mit rehten triuwen.\\ 
 & sîne site sult ir niuwen\\ 
 & unde minnet von herzen iuwer ko\textit{n}en.\\ 
20 & \textbf{Ir sult in sînen siten} wonen.\\ 
 & iuwer va\textit{r}we \textbf{treget} glîch\textit{iu} mâl.\\ 
 & \textbf{der} was ouch hêrre übern Grâl.\\ 
 & ouwê, hêrre, wannen ist iuwer vart?\\ 
 & nû ruochet mir prüeven iuwern art."\\ 
25 & \begin{large}I\end{large}etweder vaste an den andern sach.\\ 
 & Parcifal zem wirte sprach:\\ 
 & "ich bin von ein\textit{e}m man \textbf{geborn},\\ 
 & der mit tjost hât den lîp verlorn\\ 
 & unde durch rîterlîch gemüete.\\ 
30 & hêrre, durch iuwer güete\\ 
\end{tabular}
\scriptsize
\line(1,0){75} \newline
T U V W Q R Fr42 \newline
\line(1,0){75} \newline
\textbf{1} \textit{Initiale} Q   $\cdot$ \textit{Capitulumzeichen} R  \textbf{20} \textit{Majuskel} T  \textbf{25} \textit{Initiale} T  \textbf{26} \textit{Initiale} R  \newline
\line(1,0){75} \newline
\textbf{1} \textit{Die Verse 453.1-502.30 fehlen} U   $\cdot$ irz] ir roys W  $\cdot$ Lehelin] lehalein W lechelin R \textbf{2} mîn] [mit]: min Fr42 \textbf{3} einez] ein ors V (Q) (R) ein W Fr42 \textbf{4} diu] die T  $\cdot$ dâ] do W Q  $\cdot$ an des] als Q \textbf{5} an] In W \textbf{6} von] vom Q  $\cdot$ Munsalvasche] [munts*]: muntsalvasche V montsaluatschs W muntsalvasche Q Munsauashe R Mvnsalvahse Fr42 \textbf{7} im] [i*]: im T in V W Q R Fr42 \textbf{9} sint] sein Q  $\cdot$ sô] do W \textbf{10} Tyturel] Titurel Q \textbf{11} sînen] sinem R  $\cdot$ Frimutel] [*]: kv́nig frimvntel V roys frimitel W roy friműtel Q Roys frimutel R \textbf{12} verlôs] schlos R  $\cdot$ ein] [*]: der V der W Q R \textbf{13} Von einer tyost verlos den lib R  $\cdot$ sîner] [siner]: einer V einer W Q \textbf{14} sîn selbes] sin sebes T selbs sein Q \textbf{15} Das mer vor manne were Q \textbf{16} wart] war R \textbf{17} rehten triuwen] rechter trúwe R \textbf{18} site] sitten W R \textbf{19} konen] comen T \textbf{20} Jr soͤllent in sime sitten wonen V  $\cdot$ Seine sitte súlt ir wonen W  $\cdot$ Seiner site solt ir wonen Q (R) \textbf{21} varwe] varvwe T frawe Q (R)  $\cdot$ treget] im treit V (W) R im teyl Q  $\cdot$ glîchiu] gliche T R \textbf{22} der] Das W  $\cdot$ ouch hêrre] herre auch Q \textbf{24} iuwern] v́wer V (Q) (R) \textbf{25} \textit{Vers 474.25 fehlt} R  \textbf{26} Parcifal] Parzifal V R Partzifal W Q \textbf{27} einem] einenem T \textbf{28} Der mit tyost hat daz leben verlorn W  $\cdot$ Der mit strit den lib het verlorn R \textbf{30} hêrre] Nerre W \newline
\end{minipage}
\end{table}
\end{document}
