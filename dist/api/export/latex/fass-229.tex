\documentclass[8pt,a4paper,notitlepage]{article}
\usepackage{fullpage}
\usepackage{ulem}
\usepackage{xltxtra}
\usepackage{datetime}
\renewcommand{\dateseparator}{.}
\dmyyyydate
\usepackage{fancyhdr}
\usepackage{ifthen}
\pagestyle{fancy}
\fancyhf{}
\renewcommand{\headrulewidth}{0pt}
\fancyfoot[L]{\ifthenelse{\value{page}=1}{\today, \currenttime{} Uhr}{}}
\begin{document}
\begin{table}[ht]
\begin{minipage}[t]{0.5\linewidth}
\small
\begin{center}*D
\end{center}
\begin{tabular}{rl}
\textbf{229} & \begin{large}S\end{large}în harnasch was von im getragen.\\ 
 & daz \textbf{begunder} \textbf{sider} \textbf{sêre} klagen,\\ 
 & dâ er sich schimpfes niht versan:\\ 
 & ze hove ein \textbf{redespæher} man\\ 
5 & \textbf{bat} komen \textbf{ze} vrevellîche\\ 
 & den gast ellens rîche\\ 
 & zem wirte, als ob im wære zorn.\\ 
 & des het er nâch den lîp verlorn\\ 
 & von dem jungen Parzival.\\ 
10 & dô er sîn swert wol gemâl\\ 
 & \textbf{ninder bî im ligen} vant,\\ 
 & \textbf{zer viuste} twang er \textbf{sus} \textbf{die} hant,\\ 
 & daz daz bluot ûz\textbf{en nagelen} schôz\\ 
 & unt im den ermel gar begôz.\\ 
15 & "Nein, hêrre", sprach diu ritterschaft,\\ 
 & "ez ist ein man, der schimpfes kraft\\ 
 & hât, swie trûrec wir \textbf{anders} sîn.\\ 
 & tuot iwer zuht \textbf{gein} im schîn.\\ 
 & ir \textbf{en}sultz niht anders hân vernomen,\\ 
20 & wan daz der vischære \textbf{sî} komen.\\ 
 & \textbf{dar} gêt - ir sît im \textbf{werder} gast -\\ 
 & unt \textbf{schüttet} \textbf{ab} iu zornes last."\\ 
 & Si giengen \textbf{ûf einen} palas.\\ 
 & \textbf{hundert krône dâ} gehangen was,\\ 
25 & vil kerzen drûf gestôzen\\ 
 & ob den hûsgenôzen,\\ 
 & \textbf{kleine} \textbf{kerzen} \textbf{al} umbe an der want.\\ 
 & hundert bette er ligen vant.\\ 
 & \textbf{daz} schuofen, die\textbf{s} \textbf{dâ} pflâgen.\\ 
30 & hundert kulter drûffe lâgen,\\ 
\end{tabular}
\scriptsize
\line(1,0){75} \newline
D \newline
\line(1,0){75} \newline
\textbf{1} \textit{Initiale} D  \textbf{15} \textit{Majuskel} D  \textbf{23} \textit{Majuskel} D  \newline
\line(1,0){75} \newline
\textbf{13} daz daz] daz [des]: dez D \newline
\end{minipage}
\hspace{0.5cm}
\begin{minipage}[t]{0.5\linewidth}
\small
\begin{center}*m
\end{center}
\begin{tabular}{rl}
 & sîn harnasch was von ime getragen.\\ 
 & daz \textbf{begund er} \textbf{sider} klagen,\\ 
 & dô er sich schimpfes niht versan:\\ 
 & zuo hove \dag kam der werde man\dag ,\\ 
5 & \dag den enpfienc man minneclîche\dag ,\\ 
 & den gast ellen\textit{s} rîche,\\ 
 & zuo dem w\textit{i}r\textit{t}e, als ob im wær zorn.\\ 
 & des het er \textit{nâch} den lîp verlorn\\ 
 & von dem jungen Parcifal.\\ 
10 & dô er sîn swert wol gemâl\\ 
 & \textbf{niender bî ime ligen} vant,\\ 
 & \textbf{der vürste}, twanc er \textbf{sus} \textbf{die} hant,\\ 
 & daz \textbf{ime} daz bluot ûz \textbf{dem nagel} s\textit{chôz}\\ 
 & und ime den ermel gar begôz.\\ 
15 & "nein, hêrre", sprach diu ritterschaft,\\ 
 & "ez ist ein man, der schimpfes kraft\\ 
 & hât, wie trûric wir \textbf{anders} sîn.\\ 
 & tuot iuwer zuht \textbf{gegen} ime schîn.\\ 
 & ir solletz niht anders hân vernomen,\\ 
20 & wanne daz der vischære \textbf{sî} komen.\\ 
 & \textbf{dar} gêt - ir sît ime \textbf{ein werder} gast -\\ 
 & und \textbf{lât} \textbf{von} iu \textbf{den} zornes last."\\ 
 & si giengen \textbf{ûf einen} palas,\\ 
 & \textbf{d\textit{â} hundert krône} gehangen was\\ 
25 & \textbf{und} vil kerzen drûf gestôzen\\ 
 & ob den hûsgenôzen,\\ 
 & \textbf{kleine} \textbf{kerzen} umbe an der want.\\ 
 & hundert bette er ligen vant.\\ 
 & \textbf{daz} schuofen, die \textbf{es} pflâgen.\\ 
30 & hundert kultern drûf lâgen,\\ 
\end{tabular}
\scriptsize
\line(1,0){75} \newline
m n o Fr69 \newline
\line(1,0){75} \newline
\newline
\line(1,0){75} \newline
\textbf{2} sider] sit her n sit o \textbf{3} sich schimpfes] schimppfes sich n \textbf{6} ellens rîche] ellentriche m (n) (o) \textbf{7} wirte] wurde m  $\cdot$ im wær zorn] er verzorn o \textbf{8} nâch] \textit{om.} m \textbf{10} sîn] sint o  $\cdot$ gemâl] gemaln o \textbf{11} niender] Niergent n  $\cdot$ vant] [fart]: fant o \textbf{12} vürste] fúst Fr69 \textbf{13} daz bluot] bluͦt Fr69  $\cdot$ dem nagel schôz] dem nagelsang m den nagelen schosz n (Fr69) den nageln sang o \textbf{17} trûric] [truͯt]: truͯrig o \textbf{20} vischære] ritter n \textbf{21} ir sît ime] in vnd sit jme n ẏm sit o \textbf{23} einen] ein n (o) \textbf{24} dâ] Do m n o \newline
\end{minipage}
\end{table}
\newpage
\begin{table}[ht]
\begin{minipage}[t]{0.5\linewidth}
\small
\begin{center}*G
\end{center}
\begin{tabular}{rl}
 & \textit{sîn harnasch was von im getragen.}\\ 
 & \textit{\textbf{daz begunde er}} \textit{\textbf{sider}} \textit{klagen,}\\ 
 & \textit{dô} \textit{er sich schimpfes niht versan:}\\ 
 & \textit{ze hove ein} \textit{\textbf{wortspæher}} \textit{man}\\ 
5 & \textit{\textbf{bat}} \textit{komen vrevellîchen}\\ 
 & \textit{den gast ellens} \textit{rîchen}\\ 
 & \textit{zuo dem wirt, als ob im wære zorn.}\\ 
 & \textit{des} \textit{het} \textit{er nâch den lîp verlorn}\\ 
 & \textit{von dem jungen Parzival.}\\ 
10 & \textit{dô er sî}n \textit{swert wol gemâl}\\ 
 & \textit{\textbf{ninder bî im ligen}} \textit{vant,}\\ 
 & \textit{\textbf{zuo der vûst}} \textit{twanc er} \textit{\textbf{sô}} \textbf{\textit{die}} \textit{hant,}\\ 
 & \textit{daz} \textit{\textbf{ime}} \textit{daz bluot ûz} \textit{\textbf{den nagelen}} \textit{schôz}\\ 
 & \textit{und im den ermel gar begôz.}\\ 
15 & \textit{"\begin{large}N\end{large}ein, hêrre", sprach diu rîterschaft,}\\ 
 & \textit{"ez ist ein man, der schimpfes kraft}\\ 
 & \textit{hât,} \textit{swie} \textit{trûrec wir} \textit{\textbf{anders}} \textit{sîn.}\\ 
 & \textit{tuot iuwer zuht} \textit{\textbf{an}} \textit{im schîn.}\\ 
 & \textit{ir sult ez niht anders hân} vernomen,\\ 
20 & \textit{wan} daz der vischære \textbf{wære} komen.\\ 
 & \textbf{\begin{large}Z\end{large}uo dem} g\textit{êt - ir sît im} \textbf{\textit{ein} werde\textit{r}} gast -\\ 
 & \textit{und \textbf{lât} \textbf{ab} iu \textbf{den} zornes l}ast."\\ 
 & si giengen \textbf{ûf ein} palas.\\ 
 & \textbf{\textit{h}undert krône dâ} gehangen was,\\ 
25 & vil kerzen drûf gestôzen\\ 
 & obe den hûsgenôzen,\\ 
 & \textbf{vil} \textbf{kleiner} \textbf{kerzen} umbe an der want.\\ 
 & \textit{h}undert better ligen vant.\\ 
 & \textit{\textbf{ez} schuofen, die \textbf{sîn} \textbf{dâ} pflâgen.}\\ 
30 & \textit{hundert kulter drûffe lâgen,}\\ 
\end{tabular}
\scriptsize
\line(1,0){75} \newline
G I O L M Q R Z Fr21 Fr54 \newline
\line(1,0){75} \newline
\textbf{1} \textit{Initiale} L Z  \textbf{9} \textit{Initiale} R  \textbf{15} \textit{Initiale} I  \textbf{21} \textit{Initiale} G  \textbf{23} \textit{Initiale} M Fr54  \newline
\line(1,0){75} \newline
\textbf{1} \textit{Die Verse 228.27-229.18 fehlen} G   $\cdot$ was] wart M Q Fr21 \textbf{2} er] \textit{om.} R \textbf{3} dô] Da Z  $\cdot$ sich schimpfes] schýmpfes sich L (Q) schimpfes Z  $\cdot$ versan] versach O \textbf{4} hove] hore Q  $\cdot$ wortspæher] rede speher O (L) (M) Q Z (Fr21) (Fr54) raͯt specher R  $\cdot$ man] man sprach O \textbf{5} bat] Er bat O Hiez L Fr54  $\cdot$ vrevellîchen] zv freveliche Z ze urævelich Fr54 \textbf{6} ellens] eren Q  $\cdot$ rîchen] riche O L M Z Fr21 Fr54 \textbf{7} ob] \textit{om.} L M R \textbf{8} nâch den lîp] den lip nahe L  $\cdot$ verlorn] verlon R \textbf{9} Parzival] [parzifal]: Parzifal I Parcifal O (L) (Z) (Fr21) parzifal M partzifal Q Fr54 parczifal R \textbf{10} dô] Da Z  $\cdot$ sîn] si I  $\cdot$ wol] so wol Fr54 \textbf{11} ninder bî im] Bi im ninder O (Z) Fr21 (Fr54) Bie ym nedir M (Q) By Im niendet R  $\cdot$ ligen] ligende Z \textbf{12} Der fᵫrste zwang also sin hand R  $\cdot$ twanc] twack Q  $\cdot$ sô die] sine Fr54 \textbf{13} daz ime] \textit{om.} O M Q R Z Fr21  $\cdot$ daz bluot] bluͯt L Do bluet Q  $\cdot$ den] dem Q \textbf{14} im] \textit{om.} Fr54  $\cdot$ den] die O L Q R Z Fr21  $\cdot$ ermel] ermeln M \textbf{16} ez] Er Fr54  $\cdot$ der] dy M des R  $\cdot$ kraft] erhaft Fr21 \textbf{17} swie] wie L (M) (Q) R Z  $\cdot$ wir anders] wie wir alle R wir alle Fr54 \textbf{18} zuht] niht Fr54  $\cdot$ an] gein Fr54  $\cdot$ schîn] schind R \textbf{19} nwart oͮch schiere do vernomen G  $\cdot$ ir sult ez niht anders] Jrn ensolt anders nit R Jrn svlt fur anders niht Fr54 \textbf{20} wan] \textit{om.} G  $\cdot$ vischære] \textit{om.} M  $\cdot$ wære] ist I si O (L) (M) (Q) Z Fr21 Fr54 \textbf{21} Zoͮ dem gie der werde gast G  $\cdot$ Zuo dem] dar Fr54  $\cdot$ ir sît im] ir sît O (M) ir im sind ir R  $\cdot$ ein werder] ein lieber L lieber Fr54 \textbf{22} an dem des wunsches niht gebrast G  $\cdot$ lât] schuttit Fr54  $\cdot$ ab iu] von uch M ab ir R  $\cdot$ den] \textit{om.} L M Q R Z Fr54 \textbf{23} si] ÷i Fr54  $\cdot$ ûf ein] [*]: vf ein G in ein I O (M) Z Fr21 in einen L uff einen Q vff den R \textbf{24} hundert] wol hundert G  $\cdot$ gehangen] gahen Q \textbf{26} obe] Vff M \textbf{27} kerzen] herczin M  $\cdot$ der] dy Q \textbf{28} hundert] wol hundert G  $\cdot$ better] bette man da L bette M Fr21 \textbf{29} geriht herliche G  $\cdot$ sîn] des O ez L (M) (Q) (R) (Fr21) \textbf{30} Mit gulteren riche G \newline
\end{minipage}
\hspace{0.5cm}
\begin{minipage}[t]{0.5\linewidth}
\small
\begin{center}*T
\end{center}
\begin{tabular}{rl}
 & Sîn harnasch was von im getragen.\\ 
 & daz \textbf{er iedoch begunde} klagen,\\ 
 & dô er sich schimpfes niht versan:\\ 
 & Ze hove ein \textbf{redespæher} man\\ 
5 & \textbf{hiez} komen \textbf{ze} vrevellîche\\ 
 & den gast ellens rîche\\ 
 & zem wirte, als ob im wære zorn.\\ 
 & des hâter nâch den lîp verlorn\\ 
 & von dem jungen Parcifal.\\ 
10 & dô er sîn swert wol gemâl\\ 
 & \textbf{bî im ligen niender} vant,\\ 
 & \textbf{zeiner vûst} twang er \textbf{sîne} hant,\\ 
 & daz daz bluot ûz \textbf{den nageln} schôz\\ 
 & unde im den ermel gar begôz.\\ 
15 & "Nein, hêrre", sprach diu rîterschaft,\\ 
 & "ez ist ein man, der schimpfes kraft\\ 
 & hât, swie trûric wir \textbf{selbe} sîn.\\ 
 & tuot iuwer zuht \textbf{gegen} im schîn.\\ 
 & ir sultz niht anders hân vernomen,\\ 
20 & wan daz der vischære \textbf{sî} komen.\\ 
 & \textbf{dar} gât - ir sît im \textbf{ein lieber} gast -\\ 
 & und \textbf{schütet} \textbf{ab} iu zornes last."\\ 
 & \begin{large}S\end{large}i giengen \textbf{gegen dem} palas.\\ 
 & \textbf{hundert krônen dâ} gehangen was,\\ 
25 & vil kerzen drûf gestôzen\\ 
 & ob den hûsgenôzen,\\ 
 & \textbf{kleine} \textbf{lieht} umb an der want.\\ 
 & hundert bette er \textbf{dâ} ligen vant.\\ 
 & \textbf{ez} schuofen, die\textbf{s} \textbf{dâ} pflâgen.\\ 
30 & hundert kulter drûfe lâgen,\\ 
\end{tabular}
\scriptsize
\line(1,0){75} \newline
T U V W \newline
\line(1,0){75} \newline
\textbf{1} \textit{Majuskel} T  \textbf{4} \textit{Majuskel} T  \textbf{15} \textit{Majuskel} T  \textbf{23} \textit{Initiale} T U V W  \newline
\line(1,0){75} \newline
\textbf{2} iedoch] doch U doch sit V (W) \textbf{3} sich] \textit{om.} U W  $\cdot$ niht] úbel W \textbf{4} redespæher] [redesp*]: redesprecher  V \textbf{5} hiez] [*]: Bat V  $\cdot$ ze] \textit{om.} V den gast zuͦ W \textbf{6} ellens] allen U \textbf{7} zem] Zuͦ eime U \textbf{8} des] [*]: Dez V  $\cdot$ nâch] [*]: noch V nahet W \textbf{9} Parcifal] Parzifal T [*]: parzifal V partzifal W \textbf{11} bî] Niergent bi V  $\cdot$ ligen niender] nider ligen U ligen V \textbf{13} daz daz] Daz [*]: imz V Das im daz W \textbf{14} im den ermel] die ermeln U [*]: im den ermel V  $\cdot$ begôz] vergos W \textbf{17} swie] wie U W  $\cdot$ wir selbe] wir selber U [wi*]: wir anders V wir alle W \textbf{20} der] die W \textbf{21} ein lieber] werder W \textbf{22} schütet] schútenz W  $\cdot$ ab] von U \textbf{23} gegen dem] [*]: vf einen V \textbf{24} dâ] do V W \textbf{27} kleine lieht] [*eine*]: vil kleiner kerzen V \textbf{28} dâ] do V W \textbf{29} ez] [*]: Daz V  $\cdot$ dâ] \textit{om.} V W \textbf{30} kulter] kultern U kvtern V \newline
\end{minipage}
\end{table}
\end{document}
