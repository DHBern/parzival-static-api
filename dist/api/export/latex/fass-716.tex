\documentclass[8pt,a4paper,notitlepage]{article}
\usepackage{fullpage}
\usepackage{ulem}
\usepackage{xltxtra}
\usepackage{datetime}
\renewcommand{\dateseparator}{.}
\dmyyyydate
\usepackage{fancyhdr}
\usepackage{ifthen}
\pagestyle{fancy}
\fancyhf{}
\renewcommand{\headrulewidth}{0pt}
\fancyfoot[L]{\ifthenelse{\value{page}=1}{\today, \currenttime{} Uhr}{}}
\begin{document}
\begin{table}[ht]
\begin{minipage}[t]{0.5\linewidth}
\small
\begin{center}*D
\end{center}
\begin{tabular}{rl}
\textbf{716} & \begin{large}A\end{large}rtus sprach: "niftel, dû hâst wâr,\\ 
 & der künec dich grüezet âne vâr.\\ 
 & \textbf{dirre} brief tuot \textbf{mir} \textbf{mære} kunt,\\ 
 & daz ich sô \textbf{wunderlîchen} vunt\\ 
5 & gein minne nie gemezzen sach.\\ 
 & dû solt im sîn ungemach\\ 
 & wenden, alsô sol er dir.\\ 
 & lât ir daz beidiu her ze mir,\\ 
 & ich wil den kampf undervarn.\\ 
10 & die wîle soltû weinen sparn.\\ 
 & Nû wære dû doch gevangen.\\ 
 & sage mir, wie ist daz ergangen,\\ 
 & daz ir ein ander wurdet holt?\\ 
 & dû solt im dîner minne solt\\ 
15 & teilen; dâ wil er dienen nâch."\\ 
 & Itonje, \textbf{Artuses niftel}, sprach:\\ 
 & "si ist hie, diu \textbf{daz} zesamne truoc.\\ 
 & unser \textbf{enwederiu} es nie gewuoc.\\ 
 & welt ir, si vüeget wol, daz ich in sihe,\\ 
20 & dem ich mînes herzen gihe."\\ 
 & Artus sprach: "die \textbf{zeige} mir.\\ 
 & mag ich, sô vüege ich im unt dir,\\ 
 & daz iwer wille dran gestêt\\ 
 & und iwer beider vreude ergêt."\\ 
25 & Itonje sprach: "ez ist Bene.\\ 
 & ouch sint sîner knappen zwêne\\ 
 & alhie. muget ir versuochen,\\ 
 & welt ir mînes lebens ruochen,\\ 
 & ob mich der künec welle sehen,\\ 
30 & dem ich muoz mîner vreuden jehen?"\\ 
\end{tabular}
\scriptsize
\line(1,0){75} \newline
D \newline
\line(1,0){75} \newline
\textbf{1} \textit{Initiale} D  \textbf{11} \textit{Majuskel} D  \newline
\line(1,0){75} \newline
\textbf{16} Itonje] Jtonie D  $\cdot$ Artuses] Artvs D \textbf{25} Itonje] Jtonîe D \newline
\end{minipage}
\hspace{0.5cm}
\begin{minipage}[t]{0.5\linewidth}
\small
\begin{center}*m
\end{center}
\begin{tabular}{rl}
 & Artus sprach: "niftel, dû hâst wâr,\\ 
 & der künic dich grüezet âne vâr.\\ 
 & \textbf{der} brief tuot \textbf{mir} \textbf{mê} kunt,\\ 
 & daz ich sô \textbf{wunderlîchen} vunt\\ 
5 & gegen minne nie gemezzen sach.\\ 
 & dû solt im sîn ungemach\\ 
 & wenden, als sol er dir.\\ 
 & lât ir daz beidiu her zuo mir,\\ 
 & ich wil den kampf undervarn.\\ 
10 & die wîle soltû weinen sparn.\\ 
 & nû wære dû doch gevangen.\\ 
 & sage mir, wie ist daz ergangen,\\ 
 & daz ir ein ander wurdet holt?\\ 
 & dû solt \dag in\dag  dîner minne solt\\ 
15 & teilen; d\textit{â} wil er dienen nâch."\\ 
 & Ithonie \textbf{zuo Artuse} sprach:\\ 
 & "si ist hie, diu \textbf{daz} zuosamen truoc.\\ 
 & unser \textbf{iet\textit{w}eder}s nie ge\textit{w}uoc.\\ 
 & wolt ir, si vüeget wol, daz ich in sihe,\\ 
20 & dem ich mînes herzen gihe."\\ 
 & Artus sprach: "die \textbf{zouge} mir.\\ 
 & mac ich, sô vüege ich im und dir,\\ 
 & daz iuwer wil\textit{l}e dâr an gestêt\\ 
 & und iuwer beider vröude ergêt."\\ 
25 & Ithonie sprach: "ez ist Bene.\\ 
 & ouch sint sîner \textit{k}nappen zwêne\\ 
 & alhie. müget \textit{ir} versuochen,\\ 
 & wolt ir mînes lebens ruochen,\\ 
 & ob mich der künic welle sehen,\\ 
30 & dem ich muoz mîner vröude jehen?"\\ 
\end{tabular}
\scriptsize
\line(1,0){75} \newline
m n o Fr69 \newline
\line(1,0){75} \newline
\newline
\line(1,0){75} \newline
\textbf{3} mê] mere n (o) \textbf{4} wunderlîchen] wuͯnderlicher o \textbf{12} daz] disz o \textbf{15} dâ] do m n o \textbf{16} Ithonie] Jthonie m Jthonẏe n Jtonie o \textbf{17} zuosamen] samen n zuͦ samenen o \textbf{18} ietweders] yetteders m  $\cdot$ gewuoc] genuͯg m \textbf{21} zouge] zeige n \textbf{23} wille] wile m  $\cdot$ gestêt] stet n \textbf{25} Ithonie] Jthonie m n o Ytonie Fr69 \textbf{26} sîner knappen] siner kanappen m sine kappen Fr69 \textbf{27} ir] \textit{om.} m n o \newline
\end{minipage}
\end{table}
\newpage
\begin{table}[ht]
\begin{minipage}[t]{0.5\linewidth}
\small
\begin{center}*G
\end{center}
\begin{tabular}{rl}
 & \begin{large}A\end{large}rtus sprach: "niftel, dû hâst wâr,\\ 
 & der künec dich grüezt âne vâr.\\ 
 & \textbf{dirre} brief tuot \textbf{mêre} kunt,\\ 
 & daz ich sô \textbf{werdeclîchen} vunt\\ 
5 & gein minne nie gemezzen sach.\\ 
 & dû solt im sîn ungemach\\ 
 & wenden, als sol er dir.\\ 
 & lât ir daz beidiu her ze mir,\\ 
 & ich wil den kampf undervarn.\\ 
10 & die wîle soltû weinen sparn.\\ 
 & nû wære dû doch gevangen.\\ 
 & sage mir, wie ist daz ergangen,\\ 
 & daz ir ein ander wurdet holt?\\ 
 & dû solt im dîner minne solt\\ 
15 & teilen; dâ wil er dienen nâch."\\ 
 & Itonie \textbf{ze Artuse} sprach:\\ 
 & "si ist hie, diu \textbf{ez} zesamne truoc.\\ 
 & unser \textbf{dewe\textit{de}riu} es nie \textbf{mê} gewuoc.\\ 
 & welt ir, si vüeget wol, daz ich in sihe,\\ 
20 & dem ich mînes herzen gihe."\\ 
 & Artus sprach: "die \textbf{zeige} mir.\\ 
 & mac ich, sô vüege ich im unde dir,\\ 
 & daz iwer wille dran gestêt\\ 
 & unde iwer bêder vröude ergêt."\\ 
25 & Itonie sprach: "ez ist Bene.\\ 
 & ouch sint sîner knappen zwêne\\ 
 & al hie. müget ir versuochen,\\ 
 & welt ir mînes lebens ruochen,\\ 
 & op mich der künec welle sehen,\\ 
30 & dem ich muoz mîner vröude jehen?"\\ 
\end{tabular}
\scriptsize
\line(1,0){75} \newline
G I L M Z Fr24 \newline
\line(1,0){75} \newline
\textbf{1} \textit{Initiale} G L Z Fr24  \textbf{11} \textit{Initiale} I  \newline
\line(1,0){75} \newline
\textbf{3} tuot] tvt mir Z \textbf{6} im] nu Z \textbf{8} ir] dir I \textbf{10} weinen] din wainen I  $\cdot$ sparn] spar M \textbf{14} minne] mynnen L (M) (Z) \textbf{15} dienen] dir dinen M \textbf{16} Itonie] Jtonîe G Jconie Z  $\cdot$ ze Artuse sprach] Artuses (artus M Artusen Fr24 ) niftel sprach L (M) Z (Fr24) \textbf{17} ez] daz L (M) Z Fr24 \textbf{18} dewederiu es] deweriv es G dewederz sin I dewidirs noch M entwedere ez Z  $\cdot$ mê] \textit{om.} M \textbf{19} Wolders sy gevugt wol das yn gese M  $\cdot$ sihe] gesihe Z Fr24 \textbf{23} gestêt] gischet M \textbf{24} vröude] wille M \textbf{25} Itonie] Jtonîe G Jconie Z \textbf{30} vröude] freuden I (M) (Z) \newline
\end{minipage}
\hspace{0.5cm}
\begin{minipage}[t]{0.5\linewidth}
\small
\begin{center}*T
\end{center}
\begin{tabular}{rl}
 & \begin{large}A\end{large}rtus sprach: "niftel, dû hâst wâr,\\ 
 & der künec dich grüezet âne vâr.\\ 
 & \textbf{dirre} brief tuot \textbf{mir} \textbf{mære} kunt,\\ 
 & daz ich sô \textbf{wunderlîchen} vunt\\ 
5 & gein minne nie gemezzen sach.\\ 
 & dû solt im sîn ungemach\\ 
 & wenden, alsô sol er dir.\\ 
 & lât ir daz beidiu her zuo mir,\\ 
 & ich wil den kampf undervarn.\\ 
10 & die wîle soltû weinen sparn.\\ 
 & nû wære dû doch gevangen.\\ 
 & sage mir, wie ist daz ergangen,\\ 
 & daz ir ein ander wurdet holt?\\ 
 & dû solt im dîner minne solt\\ 
15 & teilen; dâ wil \textit{er} \textbf{dir} dienen nâch."\\ 
 & Itonie, \textbf{Artuses niftel}, sprach:\\ 
 & "si ist hie, diu \textbf{daz} zuosamen truoc.\\ 
 & unser \textbf{\textit{d}e\textit{we}der} es nie \textbf{mê} gewuoc.\\ 
 & wolt ir, si vüeget wol, daz \textit{ich} in sihe,\\ 
20 & dem ich mînes herzen gihe."\\ 
 & Artus sprach: "die \textbf{zeige} mir.\\ 
 & mag ich, sô vüege ich im und dir,\\ 
 & daz iuwer wille dâr an gestêt\\ 
 & und iuwer beider vreude ergêt."\\ 
25 & Itonie sprach: "ez ist Bene.\\ 
 & ouch sint sîne\textit{r} knappen zwêne\\ 
 & alhie. moget ir versuochen,\\ 
 & wolt ir mînes lebens ruochen,\\ 
 & o\textit{b} mich der künec wolle sehen,\\ 
30 & dem ich muoz mîner vreuden jehen?"\\ 
\end{tabular}
\scriptsize
\line(1,0){75} \newline
U V W Q R \newline
\line(1,0){75} \newline
\textbf{1} \textit{Initiale} U V W Q R  \newline
\line(1,0){75} \newline
\textbf{1} Artus] aRtuß W  $\cdot$ wâr] gancz war R \textbf{2} dich grüezet] gruͯset dich R \textbf{3} dirre] Der R  $\cdot$ mir] \textit{om.} W \textbf{4} wunderlîchen] [w*lichen]: wúnderlichen V wunnigliche W \textbf{7} er] er oͮch V \textbf{12} daz] es V R \textbf{13} ander wurdet] andren wurden R \textbf{14} im] in R  $\cdot$ minne] minnen V meinen Q \textbf{15} dâ] do V W Q des R  $\cdot$ er] [*]: ich U  $\cdot$ dir] \textit{om.} Q R \textbf{16} Ytonie [*]: zvͦ artvse sprach V  $\cdot$ Itonie] Jtonie U Q R Ytonie W  $\cdot$ Artuses] kúnig artus W artus Q (R) \textbf{17} diu daz] [d*]: die daz V das die Q die es R \textbf{18} deweder] beider U  $\cdot$ es] des R  $\cdot$ mê] \textit{om.} R  $\cdot$ gewuoc] [*]: gewuͦg V \textbf{19} ich] \textit{om.} U R  $\cdot$ sihe] gesihe V (W) Q (R) \textbf{22} im] [in]: im U \textbf{23} gestêt] erget R \textbf{24} iuwer] ᵫwe R  $\cdot$ beider] bede Q  $\cdot$ ergêt] gestet R \textbf{25} Itonie] Jtonie U Q R Jconie V Ytonie W \textbf{26} sîner] sine U  $\cdot$ knappen] knabe Q \textbf{28} ruochen] geruͯchen R \textbf{29} ob] Oder U  $\cdot$ wolle] wolde Q  $\cdot$ sehen] gesehen V W (R) \textbf{30} vreuden] froͮede V (Q) \newline
\end{minipage}
\end{table}
\end{document}
