\documentclass[8pt,a4paper,notitlepage]{article}
\usepackage{fullpage}
\usepackage{ulem}
\usepackage{xltxtra}
\usepackage{datetime}
\renewcommand{\dateseparator}{.}
\dmyyyydate
\usepackage{fancyhdr}
\usepackage{ifthen}
\pagestyle{fancy}
\fancyhf{}
\renewcommand{\headrulewidth}{0pt}
\fancyfoot[L]{\ifthenelse{\value{page}=1}{\today, \currenttime{} Uhr}{}}
\begin{document}
\begin{table}[ht]
\begin{minipage}[t]{0.5\linewidth}
\small
\begin{center}*D
\end{center}
\begin{tabular}{rl}
\textbf{262} & \begin{large}P\end{large}arzival was ouch bereit:\\ 
 & \textbf{sîn} ors mit walap er reit\\ 
 & gein Orilus de Lalander.\\ 
 & ûf des schilde vander\\ 
5 & einen trachen, als er lebte.\\ 
 & ein ander trache strebte\\ 
 & ûf sîme helme gebunden,\\ 
 & an den selben stunden\\ 
 & manec guldîn trache kleine\\ 
10 & - mit manegem edelen \textbf{steine}\\ 
 & \textbf{muosen die} gehêrt sîn;\\ 
 & ir ougen wâren rubîn -\\ 
 & ûf der decke und \textbf{ame} kursît.\\ 
 & \textbf{dâ} wart genomen der poynder wît\\ 
15 & von den zwein helden unverzagt.\\ 
 & \textbf{newederhalp} wart widersagt:\\ 
 & si wâren doch ledec ir triwe.\\ 
 & trunzûne sta\textit{r}c al niwe\\ 
 & von in wæten gein den lüften.\\ 
20 & ich wolde mich des güften,\\ 
 & het ich eine sölhe tjost gesehen,\\ 
 & als mir \textbf{diz} mære hât verjehen.\\ 
 & Dâ wart von rabbîne geriten,\\ 
 & ein sölch tjoste niht vermiten.\\ 
25 & vroun Jeschuten muot verjach,\\ 
 & \textbf{schœner tjost si nie} gesach.\\ 
 & diu hielt dâ, want ir hende.\\ 
 & si vreuden ellende\\ 
 & gunde \textbf{enwederm} \textbf{helde} schaden.\\ 
30 & diu ors in sweize muosen baden.\\ 
\end{tabular}
\scriptsize
\line(1,0){75} \newline
D \newline
\line(1,0){75} \newline
\textbf{1} \textit{Initiale} D  \textbf{23} \textit{Majuskel} D  \newline
\line(1,0){75} \newline
\textbf{3} Orilus] Ôrilvs D  $\cdot$ Lalander] Lalandr D \textbf{12} rubîn] Rvbin D \textbf{18} starc] stach D \textbf{25} Jeschuten] Jescv̂ten D \newline
\end{minipage}
\hspace{0.5cm}
\begin{minipage}[t]{0.5\linewidth}
\small
\begin{center}*m
\end{center}
\begin{tabular}{rl}
 & \begin{large}P\end{large}arcifal was ouch bereit:\\ 
 & \textbf{sîn} ros mit walap er reit\\ 
 & gegen Oriluse de Lalander.\\ 
 & ûf des schilt vander\\ 
5 & einen trachen, alsô er lebete.\\ 
 & ein ander trache strebete\\ 
 & ûf sînem helme \textit{g}e\textit{b}unden,\\ 
 & an den selben stunden\\ 
 & manic guldîn trache kleine\\ 
10 & - mit manigem edelem \textbf{gesteine}\\ 
 & \textbf{muosen die} gehêret sîn;\\ 
 & ir ougen wâren rubîn -\\ 
 & ûf der decke unt \textbf{ame} kursît.\\ 
 & \textbf{d\textit{â}} wart genomen der poinder wît\\ 
15 & von den zwein helden unverzaget.\\ 
 & \textbf{entwederhalp} wart widersaget:\\ 
 & si wâren doch ledic ir triuwe.\\ 
 & trunz\textit{ûn}e starc alniuwe\\ 
 & von in wæten gegen den lüften.\\ 
20 & ich wolte mich des güften,\\ 
 & hete ic\textit{h} \textit{e}in solich\textit{e} just gesehen,\\ 
 & alsô mir \textbf{des} mære hât verje\textit{h}e\textit{n}.\\ 
 & d\textit{â} wart von rabîne geriten,\\ 
 & ein soliche just niht vermiten.\\ 
25 & vrouwen Jeschuten muot \textbf{des} verjach,\\ 
 & \textbf{schœner just si nie} gesach.\\ 
 & diu hielt dâ, wan\textit{t} ir hende.\\ 
 & si vröuden ellende\\ 
 & \textbf{en}gunde \textbf{enwederm} \textbf{helde} schaden.\\ 
30 & diu ros in sweize muosen baden.\\ 
\end{tabular}
\scriptsize
\line(1,0){75} \newline
m n o Fr69 \newline
\line(1,0){75} \newline
\textbf{1} \textit{Initiale} m o   $\cdot$ \textit{Capitulumzeichen} n  \newline
\line(1,0){75} \newline
\textbf{2} Sin ros ros nit warlap er reit o \textbf{3} Oriluse] [oris]: oriluse o \textbf{7} sînem] sinen n  $\cdot$ gebunden] begunden m \textbf{11} muosen] Mussent m Muͯstent n (o) \textbf{12} rubîn] robin n \textbf{13} der decke] die decken n  $\cdot$ unt] \textit{om.} o \textbf{14} dâ] Do m n o \textbf{15} unverzaget] vnuerzogen o \textbf{16} entwederhalp] Etwederhalp m Jetweder halp n Jetwieder halp o  $\cdot$ widersaget] versaget n \textbf{18} trunzûne] Truncz ẏme m Truntzenẏe n Struncznie o Trvnzen Fr69  $\cdot$ alniuwe] [a*]: alle nuͯwe o \textbf{19} den] dien Fr69 \textbf{21} ich ein soliche] ich ich ein sollichv m \textbf{22} des] dise n disz o  $\cdot$ verjehen] verierret m \textbf{23} dâ] Do m n o \textbf{24} ein] Also n o \textbf{25} vrouwen] Frouwe m (n) (o)  $\cdot$ Jeschuten] jescutte m jescuten n jescuͯten o \textbf{26} gesach] versach Fr69 \textbf{27} dâ] do n o  $\cdot$ want] wan m \textbf{28} vröuden] froide o \textbf{29} enwederm helde] enttwederm helden m ẏetwederem heilte n ietwiedern hielte o \textbf{30} muosen] muͯssen m muͯsten n o \newline
\end{minipage}
\end{table}
\newpage
\begin{table}[ht]
\begin{minipage}[t]{0.5\linewidth}
\small
\begin{center}*G
\end{center}
\begin{tabular}{rl}
 & Parzival was ouch bereit:\\ 
 & \textbf{daz} ors \textit{mit w}a\textit{lap} er reit\\ 
 & gein Orillus de Lalander.\\ 
 & ûf des schilte vander\\ 
5 & einen trachen, alser lebte.\\ 
 & ein ander trache strebte\\ 
 & ûf sîne\textit{m} helm gebunden,\\ 
 & an den selben stunden\\ 
 & manic guldîn trache kleine\\ 
10 & - mit manigem edelen \textbf{steine}\\ 
 & \textbf{die muosen} \textbf{wol} gehêret sîn;\\ 
 & ir ougen wâren rubîn -\\ 
 & ûf der decke unde \textbf{\textit{an} dem} kursît.\\ 
 & \textbf{dâ} wart genomen der ponder wît\\ 
15 & von den zwein helden unverzaget.\\ 
 & \textbf{newederhalp} wart widersaget:\\ 
 & si wâren doch ledic ir triuwe.\\ 
 & trunzûne starc al niuwe\\ 
 & von in wæten gein den lüften.\\ 
20 & ich wolte mich des güften,\\ 
 & het ich eine solhe tjost gesehen,\\ 
 & als mir \textbf{diz} m\textit{æ}re hât verjehen.\\ 
 & dâ wart von rabîne geriten,\\ 
 & ein solch tjost niht vermiten.\\ 
25 & vroun Jeschuten muot verjach,\\ 
 & \textbf{daz si nie schœner tjost} gesach.\\ 
 & diu hielt dâ, want ir hende.\\ 
 & si vröuden ellende\\ 
 & gunde \textbf{dewe\textit{de}rem} \textbf{rîter} schaden.\\ 
30 & diu ors in sweize muosen baden.\\ 
\end{tabular}
\scriptsize
\line(1,0){75} \newline
G I O L M Q R Z Fr21 \newline
\line(1,0){75} \newline
\textbf{1} \textit{Initiale} L  \textbf{3} \textit{Initiale} I  \textbf{23} \textit{Initiale} I  \textbf{27} \textit{Initiale} O Z Fr21  \newline
\line(1,0){75} \newline
\textbf{1} Parzival] parzifal I (M) Parcifal O (L) Z Fr21 Partzifal Q Parczifal R  $\cdot$ was] ward R \textbf{2} daz] Ditze O (Fr21) Sein Q (R)  $\cdot$ mit walap] von rabine G mit gewalt R  $\cdot$ er] ir M \textbf{3} gein] Ein O  $\cdot$ Orillus] [o*]: orillus G orilus I (O) M Q (R) Z (Fr21)  $\cdot$ de Lalander] delalander G L R Z de lander I der lalander O de lalandir M de la lander Q \textbf{5} lebte] lebt O \textbf{6} strebte] [swebte]: strebte G swebete I (R) der strebt O \textbf{7} sînem] sinen G L (Q) \textbf{10} manigem] \textit{om.} L  $\cdot$ edelen] edelem I (O) (L) (Q) (Fr21)  $\cdot$ steine] gesteine I L (Q) Z \textbf{11} wol] \textit{om.} Q \textbf{12} ougen] ovgem Z  $\cdot$ rubîn] rubein Q \textbf{13} an] vf G \textbf{14} dâ] do I (Q) (R)  $\cdot$ ponder] ban so R \textbf{15} zwein] \textit{om.} L Fr21 \textbf{16} newederhalp] dewederthalpt I (O) (L) (Z) (Fr21) Jclicher halb M Entwederthap Q ettwedrem R \textbf{17} ledic] \textit{om.} L \textbf{18} trunzûne starc] Die schiltte R \textbf{19} Die sper stuben in die lúfften R  $\cdot$ wæten] wete M  $\cdot$ gein] [von]: gein G \textbf{20} güften] gvfen O \textbf{21} eine solhe] die selbe O ein selle Fr21 \textbf{22} diz] daz I (Q)  $\cdot$ mære] marere G  $\cdot$ hât] het Q \textbf{23} dâ] Do Q R \textbf{24} solch] solhev I \textbf{25} vroun] Vrow L (M) (Q) (R)  $\cdot$ Jeschuten] ieschvten G ieskuten I Jeschvͦten O Jescuͯten L ieschuten M iescuten Q Z Juscuten R  $\cdot$ muot] munt Z (Fr21) \textbf{26} Schoner tyost si nîe gesach O (L) (Q) (Fr21) Schoner tiost sy ny gischach M Schoner Ritten sy nie gesach R Schoner tiost sie niht gesach Z \textbf{27} diu] ÷iv O  $\cdot$ hielt] helt M  $\cdot$ dâ] do Q \textit{om.} Z  $\cdot$ want] vnd want I (L) (Z) bar O wart M von Q sy wand R \textbf{28} si] Div Fr21  $\cdot$ vröuden] frouweten M forcht R \textbf{29} gunde] Chvnde O (Fr21) Engvnde L (Q) Sy gunde R  $\cdot$ dewederem] dewerem G ir dewedrem I enwederm L icwederynne M  $\cdot$ rîter schaden] riters schaden I \textbf{30} in sweize muosen] in swaize muͤsten I (L) (Q) Im schweis muͦstent R musten in sweiz Z \newline
\end{minipage}
\hspace{0.5cm}
\begin{minipage}[t]{0.5\linewidth}
\small
\begin{center}*T
\end{center}
\begin{tabular}{rl}
 & \begin{large}P\end{large}arzifal was ouch bereit:\\ 
 & \textbf{sîn} ors mit walap er reit\\ 
 & gegen Oriluse de Lalander.\\ 
 & ûf des schilte vande\textit{r}\\ 
5 & einen trachen, als er lebete.\\ 
 & ein ander trache strebete\\ 
 & ûf sînem helme gebunden,\\ 
 & an den selben stunden\\ 
 & manec guldîn trache kleine\\ 
10 & - mit manegem edeln \textbf{steine}\\ 
 & \textbf{muosen die} gehêret sîn;\\ 
 & ir ougen wâren rubîn -\\ 
 & ûf der decke unde \textbf{ûf dem} kursît.\\ 
 & \textbf{hie} wart genomen der poynder wît\\ 
15 & von den zwein he\textit{l}den unverzaget.\\ 
 & \textbf{dewederhalp} wart widersaget:\\ 
 & si wâren doch ledic ir triuwe.\\ 
 & trunzûne starc al niuwe\\ 
 & von in wæten gegen den lüften.\\ 
20 & ich wolte mich des güften,\\ 
 & hetich ein solhe tjost gesehen,\\ 
 & als mir \textbf{diz} mære hât verjehen.\\ 
 & dâ wart von rabîne geriten,\\ 
 & ein sölch tjost niht vermiten.\\ 
25 & vroun Jeschuten muot verjach,\\ 
 & \textbf{schœner tjost si nie} gesach.\\ 
 & di\textit{u} hielt dâ \textbf{unde} want ir hende,\\ 
 & si vröuden ellende,\\ 
 & \textbf{unde} \textbf{en}gunde \textbf{dewederme} \textbf{helde} schaden.\\ 
30 & diu ors in sweize muosen baden.\\ 
\end{tabular}
\scriptsize
\line(1,0){75} \newline
T U V W \newline
\line(1,0){75} \newline
\textbf{1} \textit{Initiale} T U V W  \newline
\line(1,0){75} \newline
\textbf{1} Parzifal] PArcifal U PArtzifal W \textbf{2} er] er do W \textbf{3} Oriluse de Lalander] Orilus de lalander U (W) orilus [delaland*]: delalander  V \textbf{4} des schilte] die schilte da U des schiltes W  $\cdot$ vander] vande T wander U [*]: vanter V \textbf{7} sînem] sinen U (V) \textbf{10} edeln] edelme U (V) (W)  $\cdot$ steine] gesteine U (W) \textbf{11} muosen] mvesen T Muͦsse W \textbf{12} rubîn] Ruͦbin U rubein W \textbf{13} der decke] dem gedecke U  $\cdot$ ûf dem] dem U [*]: amme V an dem W \textbf{14} hie] Da V \textbf{15} helden] helnden T \textbf{16} dewederhalp] Jequeder site U Entwederthalb W \textbf{18} \textit{Vers 262.18 fehlt} U   $\cdot$ trunzûne starc] Truntzen starcke W \textbf{21} ein] \textit{om.} U \textbf{23} dâ] Do U V W \textbf{25} vroun] Vreuͦwe U (W)  $\cdot$ Jeschuten] Jescvten T iescuten U V iestuten W  $\cdot$ verjach] [*]: dez veriach V \textbf{27} diu] die T Die do W  $\cdot$ dâ] die W \textbf{28} si] Die arme an W \textbf{29} Wann sy kunde keinem helde schaden W  $\cdot$ engunde] guͦnde U (V)  $\cdot$ dewederme] [de*]: dewederme T Jequeder U \textbf{30} muosen] mvesen T mvͤstent V \newline
\end{minipage}
\end{table}
\end{document}
