\documentclass[8pt,a4paper,notitlepage]{article}
\usepackage{fullpage}
\usepackage{ulem}
\usepackage{xltxtra}
\usepackage{datetime}
\renewcommand{\dateseparator}{.}
\dmyyyydate
\usepackage{fancyhdr}
\usepackage{ifthen}
\pagestyle{fancy}
\fancyhf{}
\renewcommand{\headrulewidth}{0pt}
\fancyfoot[L]{\ifthenelse{\value{page}=1}{\today, \currenttime{} Uhr}{}}
\begin{document}
\begin{table}[ht]
\begin{minipage}[t]{0.5\linewidth}
\small
\begin{center}*D
\end{center}
\begin{tabular}{rl}
\textbf{663} & bî einem clârem, snellem bach,\\ 
 & \textbf{dâ} man schiere \textbf{ûf} geslagen sach\\ 
 & \textit{\begin{large}M\end{large}}anec \textbf{zelt} wolgetân.\\ 
 & dem künege sunder dort hin dan\\ 
5 & wart manec wîter rinc genomen\\ 
 & \textbf{unt} rîteren, die dâ wâren komen.\\ 
 & die hete\textit{n} âne vrâge\\ 
 & ûf \textbf{ir} reise grôze slâge.\\ 
 & Gawan bî Benen hin ab enbôt\\ 
10 & \textbf{sîme wirte} Plippalinot,\\ 
 & Kocken, ussiere,\\ 
 & daz er \textbf{die} \textbf{slüzze} schiere,\\ 
 & sô daz \textbf{vor} sîner übervart\\ 
 & daz her des tage\textit{s} wære bewart.\\ 
15 & Vrou Bene ûz Gawans \textbf{hende} nam\\ 
 & die êrsten gâbe ûz \textbf{sîme rîchen} krâm,\\ 
 & Swalwen, diu noch zEngellant\\ 
 & zeiner tiwern harpfen ist erkant.\\ 
 & Bene vuor mit \textbf{vreuden} dan.\\ 
20 & dô hiez mîn hêr Gawan\\ 
 & besliezen die ûzern porten.\\ 
 & alt unt junge hôrten,\\ 
 & wes er si zühteclîchen bat:\\ 
 & "\textbf{dâ derhalben} an den stat\\ 
25 & \textbf{sich leget} ein alsô grôzez her,\\ 
 & \textbf{weder} \textbf{ûf} lande noch \textbf{in} dem mer\\ 
 & gesach ich \textbf{rotte nie} gevarn\\ 
 & mit \textbf{alsus} krefteclîchen scharn.\\ 
 & wellent si uns \textbf{hie} suochen mit \textbf{ir} kraft,\\ 
30 & helfet mir, ich gib in rîterschaft."\\ 
\end{tabular}
\scriptsize
\line(1,0){75} \newline
D \newline
\line(1,0){75} \newline
\textbf{3} \textit{Initiale} D  \textbf{11} \textit{Majuskel} D  \textbf{15} \textit{Majuskel} D  \textbf{17} \textit{Majuskel} D  \newline
\line(1,0){75} \newline
\textbf{3} Manec] ÷anech D \textbf{7} heten] hete D \textbf{14} tages] tage D \newline
\end{minipage}
\hspace{0.5cm}
\begin{minipage}[t]{0.5\linewidth}
\small
\begin{center}*m
\end{center}
\begin{tabular}{rl}
 & bî einem clâren, snellen bach,\\ 
 & \textbf{daz} man schier \textbf{ûz} geslagen sach\\ 
 & manic \textbf{gezelt} wol getân.\\ 
 & dem künige sunder dort hin dan\\ 
5 & wart manic wîter rinc genomen\\ 
 & \textbf{und} rittern, die d\textit{â} wâren komen.\\ 
 & die heten âne vrâge\\ 
 & ûf reise grôze slâge.\\ 
 & \begin{large}G\end{large}awan bî Benen hin ab enbôt\\ 
10 & \textbf{ir vater} Pli\textit{pp}al\textit{in}ot,\\ 
 & \hspace*{-.7em}\big| daz er \textbf{beslüzze} schier\\ 
 & \hspace*{-.7em}\big| \dag kochey\dag , ussier,\\ 
 & sô daz \textbf{vor} sîner übervart\\ 
 & daz her des tages wær bewart.\\ 
15 & vrowe Bene ûz Gawanes \textbf{henden} nam\\ 
 & die êrsten gâbe ûz \textbf{sîne\textit{m} rîche\textit{m}} krâm,\\ 
 & Swal\textit{we}n, diu noch zuo Engelant\\ 
 & zuo e\textit{i}n\textit{er} tiuren harpfen ist erkant.\\ 
 & \textbf{vrowe} Bene vuor mit \textbf{vröuden} dan.\\ 
20 & dô hie mîn hêr Gawan\\ 
 & besliezen die ûzern porten.\\ 
 & alt und junc hôrten,\\ 
 & wes er si zühteclîchen bat.\\ 
 & \textbf{er sprach}: "\textbf{jenhalp} an den stat\\ 
25 & \textbf{leget sich} \textit{ei}n alsô grôzez her,\\ 
 & \textbf{weder} \textbf{ûf dem} lande noch \textbf{in} dem mer\\ 
 & gesach ich \textbf{nie rote} gevarn\\ 
 & mit \textbf{alsô} krefteclîchen scharn.\\ 
 & wellent si uns \textbf{hie} suochen mit \textbf{ir} kraft,\\ 
30 & helfet mir, ich gib in ritterschaft."\\ 
\end{tabular}
\scriptsize
\line(1,0){75} \newline
m n o Fr69 \newline
\line(1,0){75} \newline
\textbf{9} \textit{Initiale} m   $\cdot$ \textit{Capitulumzeichen} n  \newline
\line(1,0){75} \newline
\textbf{1} snellen] smallen o \textbf{3} gezelt] gezelte n \textbf{6} dâ] do m n o \textbf{8} reise] ir reise n o \textbf{9} Benen] bene n \textbf{10} Plippalinot] plimalniot m plippamot o \textbf{11} kochey] Kockẏ n Kocher o  $\cdot$ ussier] vschier m (n) (o) \textbf{14} daz] Des o  $\cdot$ wær] wart o \textbf{15} henden] hende n o \textbf{16} êrsten] erste m n o  $\cdot$ sînem rîchem] sinen richen m \textbf{17} Swalwen] Swalman m  $\cdot$ Engelant] engellant n \textbf{18} einer] end m  $\cdot$ harpfen] harppfant n \textbf{20} hêr] herre her n \textbf{21} ûzern] vsser m (o) \textbf{22} junc hôrten] jungen horten n (o) vngehorten Fr69 \textbf{23} si] \textit{om.} Fr69 \textbf{24} jenhalp] ienhabe Fr69  $\cdot$ den] der n die o \textbf{25} ein] an m  $\cdot$ grôzez] groszer o \textbf{26} in] uͯff o \textbf{27} nie rote] roͯte nẏe n (o)  $\cdot$ gevarn] gefar o \textbf{28} alsô] alsus n (o) \textbf{30} in] in mit Fr69 \newline
\end{minipage}
\end{table}
\newpage
\begin{table}[ht]
\begin{minipage}[t]{0.5\linewidth}
\small
\begin{center}*G
\end{center}
\begin{tabular}{rl}
 & bî einem clâren, snellen bach,\\ 
 & \textbf{dâ} man schiere \textbf{ûf} geslagen sach\\ 
 & \begin{large}M\end{large}anic \textbf{gezelt} wolgetân.\\ 
 & dem künige sunder dort hin dan\\ 
5 & wart manic wîter rinc genomen\\ 
 & \textbf{von} rîtern, die dâ wâren komen.\\ 
 & die heten ân vrâge\\ 
 & ûf \textbf{ir} reise grôze slâge.\\ 
 & Gawan bî Benen hin abe enbôt\\ 
10 & \textbf{sîne\textit{m} wirt} Pliplalinot,\\ 
 & kocken \textbf{unde} \textit{urs}siere,\\ 
 & daz er \textbf{diu} \textbf{slüzze} schiere,\\ 
 & sô daz \textbf{dâ von} sîner übervart\\ 
 & daz her des tages wære bewart.\\ 
15 & vrô Bene ûz Gawanes \textbf{hende} nam\\ 
 & die êrsten gâbe ûz \textbf{sîner} krâm,\\ 
 & Swalwen, diu noch ze Engellant\\ 
 & zeiner tiuren harpfen ist erkant.\\ 
 & \textbf{vrô} Bene vuor mit \textbf{vröuden} dan.\\ 
20 & dô hiez mîn hêr Gawan\\ 
 & besliezen die ûzern porten.\\ 
 & alte unde junge \textbf{dâ} hôrten,\\ 
 & wes er si zühticlîchen bat:\\ 
 & "\textbf{dô anderhalben} an den stat\\ 
25 & \textbf{sich leget} ein als grôz her,\\ 
 & \textbf{daz} \textbf{ûf dem} lande noch \textbf{in} dem mer\\ 
 & gesach ich \textbf{roten nie} gevarn\\ 
 & mit \textbf{alsô} krefticlîchen scharn.\\ 
 & wellent si uns suochen mit \textbf{hers} kraft,\\ 
30 & helfet mir, ich gibe in rîterschaft."\\ 
\end{tabular}
\scriptsize
\line(1,0){75} \newline
G I L M Z Fr45 \newline
\line(1,0){75} \newline
\textbf{3} \textit{Initiale} G I L Z  \textbf{15} \textit{Initiale} I  \newline
\line(1,0){75} \newline
\textbf{1} einem] einen Z  $\cdot$ clâren] chlarem I clarē M \textbf{2} geslagen] slahen I \textbf{6} von] Vnd Z \textbf{8} slâge] lage I \textbf{9} Gawan hin abe bi bene enbot I  $\cdot$ Benen] bênen G bene M \textbf{10} sînem] Sinen G  $\cdot$ Pliplalinot] plibalinot I plipalinot L M Z \textbf{11} kocken] koch I  $\cdot$ urssiere] visiere G vrshiere I \textbf{12} slüzze] bereite L \textbf{13} dâ] \textit{om.} L \textbf{14} her] er I \textbf{15} Gawanes] Gawans I (M) (Z) (Fr45) Gawanz L  $\cdot$ hende] henden I (M) \textbf{16} sîner] sinem Fr45 \textbf{17} Swalwen] swaliwen I Swalben L Swalffin M  $\cdot$ ze Engellant] ze Engelle G ze engenlant I usz engillant M zengelant Fr45 \textbf{20} dô] Da M Z  $\cdot$ hêr Gawan] ergawan M \textbf{21} porten] phorten M (Fr45) \textbf{22} junge] jungen M  $\cdot$ dâ] \textit{om.} Z \textbf{23} si] \textit{om.} Z \textbf{24} dô] Da M Z  $\cdot$ den] daz I L \textbf{25} leget] legite M  $\cdot$ ein] an I \textbf{26} daz] Dar M Weder Z  $\cdot$ noch] \textit{om.} M \textbf{27} roten] rotte I L \textbf{28} krefticlîchen] creftigen I \textbf{29} wellent] wellen I \newline
\end{minipage}
\hspace{0.5cm}
\begin{minipage}[t]{0.5\linewidth}
\small
\begin{center}*T
\end{center}
\begin{tabular}{rl}
 & bî einem clâren, snellen bach,\\ 
 & \textbf{d\textit{â}} man schier \textbf{ûf} geslagen sach\\ 
 & manic \textbf{gezelt} wol getân.\\ 
 & dem künige sunder dort hin dan\\ 
5 & wart manic wîter rinc genomen\\ 
 & \textbf{von} rittern, die dar wâren komen.\\ 
 & die heten âne vrâge\\ 
 & ûf \textbf{ir} reise grôze slâge.\\ 
 & Gawan bî Bene hin abe enbôt\\ 
10 & \textbf{sînem wirte} Plipalinot,\\ 
 & kocken \textbf{und} ussier\textit{e},\\ 
 & daz er \textbf{die} \textbf{slüzze} schiere,\\ 
 & sô daz \textbf{vor} sîner übervart\\ 
 & daz \textit{h}er des tages wær bewart.\\ 
15 & vrou Bene ûz Gawans \textbf{hende} nam\\ 
 & die êrsten gâbe ûz \textbf{sînem r\textit{î}c\textit{he}m} krâm,\\ 
 & Swalwen, diu noch zEngellant\\ 
 & zuo einer tiuren harpfen ist erkant.\\ 
 & \textbf{vrou} Bene vuor mit \textbf{werden} dan.\\ 
20 & dô hiez mîn hêr Gawan\\ 
 & besliezen die ûzeren pforten.\\ 
 & alte und junge \textbf{d\textit{â}} hôrten,\\ 
 & wes er si zühticlîch \textbf{dô} bat:\\ 
 & "\textbf{d\textit{â} anderthalben} an daz stat\\ 
25 & \textbf{sich legte} ein alsô grôzez her,\\ 
 & \textbf{weder} \textbf{in dem} lande noch \textbf{ûffe}\textit{m} mer\\ 
 & gesach ich \textbf{rotten nie} g\textit{e}varen\\ 
 & mit \textbf{alsô} krefticlîchen scharen.\\ 
 & wellent si uns suochen mit \textbf{hers} kraft,\\ 
30 & helfet mir, ich gib in ritterschaft."\\ 
\end{tabular}
\scriptsize
\line(1,0){75} \newline
Q R W V \newline
\line(1,0){75} \newline
\textbf{9} \textit{Initiale} V  \newline
\line(1,0){75} \newline
\textbf{1} einem] einē Q einen R  $\cdot$ clâren snellen] clarē snellē Q klarē schnellen W snellen claren V \textbf{2} dâ] Do Q W V \textbf{3} wol] schon W \textbf{6} dar] do V  $\cdot$ wâren] sein W \textbf{7} âne] an alle W \textbf{8} slâge] schage R \textbf{9} Gawan] Gawann Q Gawin R  $\cdot$ Bene] benen R W V \textbf{10} Plipalinot] plypalinot W \textbf{11} ussiere] vssiren Q \textbf{12} slüzze] schússe R [*]: beslv́zze V \textbf{13} vor] da von R (W) [von]: vor  V  $\cdot$ übervart] vart W \textbf{14} her] er Q  $\cdot$ wær] do were W \textbf{15} Gawans] gawins R \textbf{16} êrsten] erste V  $\cdot$ ûz] vsser R  $\cdot$ sînem rîchem] seinem rechtem Q sinen Richen R Seinē reichē W [sim*]: sime richem V \textbf{17} Swalwen] Swalben Q Welhem R Schwalben W  $\cdot$ zEngellant] ze engenland R \textbf{18} tiuren] trúwen R \textbf{19} werden] froͯden R (W) (V) \textbf{21} pforten] porten R W V \textbf{22} dâ] do Q R W V \textbf{23} si zühticlîch] zv́hteklichen [*]: sv́ V  $\cdot$ dô] \textit{om.} R W V \textbf{24} [*]: Er sprach ienhalb an den stat V  $\cdot$ dâ] Do Q W \textbf{25} sich] \textit{om.} W  $\cdot$ legte] [*]: leget V  $\cdot$ ein alsô] als ein R [*]: ein also V \textbf{26} in] vff R (W) (V)  $\cdot$ ûffem] uffen Q in dem R [*]: in dem V \textbf{27} gevaren] gavaren Q \textbf{28} alsô] [al*]: alsuz V  $\cdot$ krefticlîchen] mehteklicher V \textbf{29} suochen] suͯch R [*]: hie suͦchen V  $\cdot$ hers] herres macht vnd R [*]: hers V \textbf{30} in] im W \newline
\end{minipage}
\end{table}
\end{document}
