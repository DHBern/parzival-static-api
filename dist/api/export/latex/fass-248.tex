\documentclass[8pt,a4paper,notitlepage]{article}
\usepackage{fullpage}
\usepackage{ulem}
\usepackage{xltxtra}
\usepackage{datetime}
\renewcommand{\dateseparator}{.}
\dmyyyydate
\usepackage{fancyhdr}
\usepackage{ifthen}
\pagestyle{fancy}
\fancyhf{}
\renewcommand{\headrulewidth}{0pt}
\fancyfoot[L]{\ifthenelse{\value{page}=1}{\today, \currenttime{} Uhr}{}}
\begin{document}
\begin{table}[ht]
\begin{minipage}[t]{0.5\linewidth}
\small
\begin{center}*D
\end{center}
\begin{tabular}{rl}
\textbf{248} & \begin{large}N\end{large}âch den mæren schrei der gast;\\ 
 & gegenrede im gar gebrast.\\ 
 & Swie vil er nâch geriefe,\\ 
 & Reht als er gênde sliefe,\\ 
5 & warp der knappe unt sluoc die porten zuo.\\ 
 & dô was sîn \textbf{danscheiden} ze vruo\\ 
 & an \textbf{der} vlustbæren zît\\ 
 & dem, der nû zins von vreuden gît.\\ 
 & diu \textbf{ist} an im verborgen.\\ 
10 & umbe den wurf der sorgen\\ 
 & \textbf{wart} getoppelt, dô er den Grâl vant\\ 
 & mit sînen ougen, âne hant\\ 
 & unt âne würfels ecke.\\ 
 & ob in nû kumber \textbf{wecke},\\ 
15 & des was er dâ vor \textbf{niht gewent};\\ 
 & er enhete sich niht vil gesent.\\ 
 & Parzival, der huop sich nâch\\ 
 & \textbf{vaste} ûf \textbf{die} slâ, die er dâ sach.\\ 
 & er dâhte: "die vor mir \textbf{riten},\\ 
20 & \textbf{ich wæne, die} hiute \textbf{striten}\\ 
 & manlîche umbe mînes wirtes dinc.\\ 
 & \textbf{ruochten} si\textbf{s}, sô wære ir rinc\\ 
 & mit mir \textbf{niht verkrenket}.\\ 
 & \textbf{al}dâ würde niht gewenket:\\ 
25 & ich hülfe in \textbf{in} der selben nôt,\\ 
 & daz ich gediende mîn brôt\\ 
 & unt ouch \textbf{diz} wünneclîche swert,\\ 
 & daz mir gap ir hêrre wert."\\ 
 & \multicolumn{1}{l}{ - - - }\\ 
30 & \multicolumn{1}{l}{ - - - }\\ 
\end{tabular}
\scriptsize
\line(1,0){75} \newline
D \newline
\line(1,0){75} \newline
\textbf{1} \textit{Initiale} D  \textbf{3} \textit{Majuskel} D  \textbf{4} \textit{Majuskel} D  \newline
\line(1,0){75} \newline
\newline
\end{minipage}
\hspace{0.5cm}
\begin{minipage}[t]{0.5\linewidth}
\small
\begin{center}*m
\end{center}
\begin{tabular}{rl}
 & nâch den mæren schrei der gast;\\ 
 & gegenrede im gar gebrast.\\ 
 & wie vil er nâch geriefe,\\ 
 & reht als er gênde sliefe,\\ 
5 & warp der knabe und sluoc die porten zuo.\\ 
 & dô was sîn \textbf{scheiden danne} ze vruo\\ 
 & an \textbf{der} vlustbæren zît\\ 
 & dem, der nû zins von vröuden gît.\\ 
 & diu \textbf{ist} an ime verborgen.\\ 
10 & umbe den wurf der sorgen\\ 
 & \textbf{wart} getoppelt, dô er den Grâl vant\\ 
 & mit sînen ougen, âne hant\\ 
 & und âne würfels ecken.\\ 
 & ob in nû kumber \textbf{wecken},\\ 
15 & des was er dâ vor \textbf{niht gewenet};\\ 
 & er enhete sich niht vil gesenet.\\ 
 & Parcifal, der huop sich nâch\\ 
 & \textbf{vaste} ûf \textbf{die} slâ, die er dâ sach.\\ 
 & er dâhte: "die vor m\textit{ir} \textbf{rîtent},\\ 
20 & \textbf{ich wæne, die} hiute \textbf{strîtent}\\ 
 & manlîch umb mînes wirtes dinc.\\ 
 & \textbf{geruochten} si\textbf{s}, sô \textbf{en}wære ir rinc\\ 
 & mit mir \textbf{niht verkrenket}.\\ 
 & \textbf{al}dâ würde niht gewenket:\\ 
25 & ich hülfe in \textbf{an} der selben nôt,\\ 
 & daz \textit{ich} gedienete mîn brôt\\ 
 & und ouch \textbf{d\textit{az}} wünneclîch swert,\\ 
 & daz mir gap ir hêrre wert.\\ 
 & ungedienet ich daz trage.\\ 
30 & si wænent lîhte, ich sî ein zage."\\ 
\end{tabular}
\scriptsize
\line(1,0){75} \newline
m n o Fr69 \newline
\line(1,0){75} \newline
\textbf{2} \textit{Capitulumzeichen} Fr69  \newline
\line(1,0){75} \newline
\textbf{1} schrei] schre n \textbf{3} wie] Swie Fr69  $\cdot$ nâch geriefe] noch gerieff n o nahriefe Fr69 \textbf{4} sliefe] slieff n o \textbf{10} wurf] ruff o \textbf{13} ecken] ecke o \textbf{14} wecken] [*]: wecke o \textbf{15} des] Das o  $\cdot$ vor] fur o \textbf{18} dâ] do n o  $\cdot$ sach] [vant]: sach Fr69 \textbf{19} mir] muͯt m  $\cdot$ rîtent] ritten Fr69 \textbf{20} strîtent] striten o \textbf{22} enwære] were n (o) \textbf{24} aldâ] Also n \textbf{25} hülfe] hilff o \textbf{26} ich] \textit{om.} m \textbf{27} daz] die m  $\cdot$ wünneclîch] mẏnneclich n (o) \textbf{30} wænent] wenet o Fr69 \newline
\end{minipage}
\end{table}
\newpage
\begin{table}[ht]
\begin{minipage}[t]{0.5\linewidth}
\small
\begin{center}*G
\end{center}
\begin{tabular}{rl}
 & nâch den mæren \textit{sch}r\textit{ei} der gast;\\ 
 & gegenrede im gar gebrast.\\ 
 & swie vil er nâch geriefe,\\ 
 & reht alser gênde sliefe,\\ 
5 & warp der knappe unde sluoc die por\textit{t}e zuo.\\ 
 & dô was sîn \textbf{scheiden dan} ze vruo\\ 
 & an \textbf{der} vlustbæren zît\\ 
 & dem, der nû zins von vröuden gît.\\ 
 & d\textit{iu} \textbf{was} an im verborgen.\\ 
10 & umbe den wurf der sorgen\\ 
 & \textbf{was} getopelt, dô er den Grâl vant\\ 
 & mit sînen ougen, âne hant\\ 
 & \begin{large}U\end{large}nd âne würfels ecke.\\ 
 & obe in nû kumber \textbf{wecke},\\ 
15 & des was er dâ vor \textbf{ungewent};\\ 
 & erne het sich niht vil gesent.\\ 
 & Parzival, der huop sich nâch\\ 
 & ûf \textbf{die} slâ, dier dâ sach.\\ 
 & er dâhte: "di\textit{e} \textit{v}or mir \textbf{riten},\\ 
20 & \textbf{die wæne ich} hiut\textit{e} \textbf{\textit{s}triten}\\ 
 & manlîch umbe mînes wirtes dinc.\\ 
 & \textbf{geruohten} si\textbf{s}, sô\textbf{ne} wære ir rinc\\ 
 & mit mir \textbf{ungekrenket}.\\ 
 & dâ\textbf{ne} würde niht gewenket:\\ 
25 & ich hülfe in \textbf{an} der selben nôt,\\ 
 & daz ich gediente mîn brôt\\ 
 & unde ouch \textbf{daz} wünniclîche swert,\\ 
 & daz mir gap ir hêrre wert.\\ 
 & ungedient ich daz trage.\\ 
30 & si wænent lîhte, ich sî ein zage."\\ 
\end{tabular}
\scriptsize
\line(1,0){75} \newline
G I O L M Q R Z Fr21 Fr36 \newline
\line(1,0){75} \newline
\textbf{9} \textit{Überschrift:} Awentewr wy partzifal von der búrge reit vnd nicht gefraget hott vmb den gral Q  · Wie parcifal schiet von der bvrge da der gral vf was Z   $\cdot$ \textit{Initiale} I O L Q Z   $\cdot$ \textit{Capitulumzeichen} R  \textbf{13} \textit{Initiale} G  \newline
\line(1,0){75} \newline
\textbf{1} den mæren] dem mere M  $\cdot$ schrei] rief G sprach R \textbf{2} gar gebrast] zcu brast M gar zvbrast Z \textbf{3} swie] Wie L (Q) R \textbf{4} Er tet sam er sliefe O  $\cdot$ alser gênde] als ob der in der R \textbf{5} warp] Vnde warf O Warf L (M) (Q) Fr36  $\cdot$ der] dy M  $\cdot$ unde sluoc] \textit{om.} O L M Q Fr36  $\cdot$ porte] por:e G porten I O (L) (M) (Q) (R) Z (Fr36) \textbf{6} dô] Da Z  $\cdot$ dan] Gar I (O) (M) (Fr36) \textbf{7} vlustbæren] fluhtbarn I \textbf{8} nû] \textit{om.} I den O  $\cdot$ gît] geit Fr36 \textbf{9} diu] do G (I) (L) (M) (Fr36) ÷a O \textbf{10} wurf der] \textit{om.} I \textbf{11} dô er] da er M R Z vmb Fr36 \textbf{12} hant] [bant]: hant I bant M \textbf{13} würfels] wuͯrffes L wortels M \textbf{15} des] Dez L  $\cdot$ er] \textit{om.} I  $\cdot$ vor] \textit{om.} O \textbf{16} erne] Er O L R Es en Q  $\cdot$ gesent] gewent R \textbf{17} Parzival] [parzifal]: Parzifal I Parcifal O L Z Parzifal M Partzifal Q Parczifal R  $\cdot$ der] \textit{om.} I \textbf{18} ûf] Vast vff Q (R) (Z)  $\cdot$ slâ] straze I  $\cdot$ dâ] \textit{om.} O do Q \textbf{19} er] E er L  $\cdot$ dâhte] gedacht R  $\cdot$ die vor] die hie vor G \textbf{20} wæne ich] wane ich noch L wond noch R  $\cdot$ hiute striten] hivte morgen striten G hivte stritent O (L) (M) (Z) \textbf{21} mînes] sins I des Fr21 \textbf{22} geruohten] ruͤchten I (O) (Q) (Z) (Fr21) ruͦhtens Fr36  $\cdot$ sône] so I L Fr21 Fr36 ich R  $\cdot$ ir] an Jr R ich Fr36 \textbf{23} ungekrenket] niht verchrenchet O (L) (M) (Q) (R) (Z) (Fr21) (Fr36) \textbf{24} dâne] Da O L (Fr36)  $\cdot$ gewenket] vor krenkrit M \textbf{25} in] [im]: in G  $\cdot$ an] vff Q al Fr36 \textbf{27} Des wunenklichen schwert R  $\cdot$ ouch] \textit{om.} I O L M Fr21 Fr36 \textbf{29} ungedient] Vnde gedient O (Fr36) \textbf{30} si wænent] si went I (Fr21) So wenent si O  $\cdot$ lîhte] \textit{om.} O lich R  $\cdot$ ich] [iz]: ich O [sich]: ich Q \newline
\end{minipage}
\hspace{0.5cm}
\begin{minipage}[t]{0.5\linewidth}
\small
\begin{center}*T
\end{center}
\begin{tabular}{rl}
 & Nâch den mæren schrei der gast;\\ 
 & \textbf{der} gegenrede im gar gebrast.\\ 
 & swie vil er nâch geriefe,\\ 
 & rehte alser gânde sliefe,\\ 
5 & warp der knappe unde sluoc die porten zuo.\\ 
 & dô was sîn \textbf{sch\textit{ei}den dan} ze vruo\\ 
 & an \textbf{dirre} vlustbæren zît\\ 
 & dem, der nû zins von vröuden gît.\\ 
 & diu \textbf{was} an im verborgen.\\ 
10 & umb den wurf der sorgen\\ 
 & \textbf{wart} getopelt, dô er den Grâl vant\\ 
 & mit sînen ougen, âne hant\\ 
 & unde âne würfels ecke.\\ 
 & obin nû kumber \textbf{wecke},\\ 
15 & des was er dâ vor \textbf{ungewent};\\ 
 & ern hete sich niht vil gesent.\\ 
 & Parcifal, der huop sich nâch\\ 
 & \textbf{vaste} ûf \textbf{ir} slâ, dier dâ sach.\\ 
 & er dâhte: "die vor mir \textbf{rîtent},\\ 
20 & \textbf{ich wæn, si} \textbf{noch} hiute \textbf{strîtent}\\ 
 & manlîche umbe mînes wirtes dinc.\\ 
 & \textbf{geruohten} si\textbf{z}, sô wære ir rinc\\ 
 & mit mir \textbf{niht verkrenket}.\\ 
 & dâ\textbf{ne} würde niht gewenket:\\ 
25 & i\textbf{ne} hülfin \textbf{an} der selben nôt,\\ 
 & daz ich gediende mîn brôt\\ 
 & unde ouch \textbf{diz} wünneclîche swert,\\ 
 & daz mir gab ir hêrre wert.\\ 
 & ungedient ich daz trage.\\ 
30 & si wænent lîhte, ich \textit{sî} ein zage."\\ 
\end{tabular}
\scriptsize
\line(1,0){75} \newline
T U V W \newline
\line(1,0){75} \newline
\textbf{1} \textit{Majuskel} T  \newline
\line(1,0){75} \newline
\textbf{2} der gegenrede] Gein der rede U Gegen rede W  $\cdot$ gar] \textit{om.} U V \textbf{3} swie] Wie U W  $\cdot$ geriefe] [ge*]: gerieffe V geliefe U \textbf{5} porten] porte V \textbf{6} scheiden] scieden T  $\cdot$ vruo] vro U \textbf{7} dirre] der W  $\cdot$ vlustbæren] verlvstberenden V \textbf{15} er] \textit{om.} W \textbf{17} Parcifal] Parzifal T V Partzifal W \textbf{18} ir] [*]: die V  $\cdot$ dâ] do U V W \textbf{19} dâhte] gedachte W  $\cdot$ rîtent] súllen reiten W \textbf{20} ich wæn si] Die wellen W \textbf{21} manlîche] Nemlich W \textbf{24} dâne] Dannen W \textbf{25} ine hülfin] Jn hulfe U Jch hv́lf in V (W) \textbf{26} gediende] gediende wol W \textbf{28} ir hêrre] mein wirt so W \textbf{30} wænent] wênt T  $\cdot$ sî] \textit{om.} T \newline
\end{minipage}
\end{table}
\end{document}
