\documentclass[8pt,a4paper,notitlepage]{article}
\usepackage{fullpage}
\usepackage{ulem}
\usepackage{xltxtra}
\usepackage{datetime}
\renewcommand{\dateseparator}{.}
\dmyyyydate
\usepackage{fancyhdr}
\usepackage{ifthen}
\pagestyle{fancy}
\fancyhf{}
\renewcommand{\headrulewidth}{0pt}
\fancyfoot[L]{\ifthenelse{\value{page}=1}{\today, \currenttime{} Uhr}{}}
\begin{document}
\begin{table}[ht]
\begin{minipage}[t]{0.5\linewidth}
\small
\begin{center}*D
\end{center}
\begin{tabular}{rl}
\textbf{235} & \textbf{\begin{large}S\end{large}i} nigen. \textbf{ir zwô dô truogen} dar\\ 
 & ûf die taveln wol gevar\\ 
 & \textbf{daz silber} unt leiten\textbf{z} nider.\\ 
 & \textbf{dô giengen si mit zühten} wider\\ 
5 & zuo den êrsten zwelven \textbf{sân}.\\ 
 & ob ich geprüevet rehte hân,\\ 
 & hie sulen ahzehen vrouwen stên.\\ 
 & Âvoy, nû siht man sehse gên\\ 
 & in wæte, die man tiure galt.\\ 
10 & \textbf{daz} was halbez plîalt,\\ 
 & daz ander pfelle von Ninnive.\\ 
 & dise unt die êrsten sehse ê\\ 
 & truogen zwelf röcke geteilet,\\ 
 & gein \textbf{tiwerer} kost geveilet.\\ 
15 & Nâch den \textbf{kom} diu künegîn.\\ 
 & \textbf{ir} antlütze gap den schîn,\\ 
 & si wânden alle, ez wolde tagen.\\ 
 & man sach die maget an ir tragen\\ 
 & pfellel von Arabi.\\ 
20 & ûf einem grüenem achmardî\\ 
 & truoc si den wunsch \textbf{von} pardîs,\\ 
 & bêde \textbf{wurzeln} unt rîs:\\ 
 & daz was ein dinc, daz \textbf{hiez} der Grâl,\\ 
 & erden wunsches überwal.\\ 
25 & \textbf{Repanse de schoye} \textbf{si} hiez,\\ 
 & die \textbf{sich der} Grâl tragen liez.\\ 
 & der Grâl was von sölher art:\\ 
 & wol \textbf{muose ir} kiusche sîn bewart,\\ 
 & diu sîn \textbf{ze} rehte solde pflegen.\\ 
30 & \textbf{diu} muose valsches sich bewegen.\\ 
\end{tabular}
\scriptsize
\line(1,0){75} \newline
D \newline
\line(1,0){75} \newline
\textbf{1} \textit{Initiale} D  \textbf{8} \textit{Majuskel} D  \textbf{15} \textit{Majuskel} D  \newline
\line(1,0){75} \newline
\textbf{11} Ninnive] Ninnivê D \textbf{25} Repanse de schoye] Repanse de scoye D \newline
\end{minipage}
\hspace{0.5cm}
\begin{minipage}[t]{0.5\linewidth}
\small
\begin{center}*m
\end{center}
\begin{tabular}{rl}
 & \textbf{si} nigen, \textit{\textbf{ir zwô. dô truogen}} \textbf{si} \textit{dar}\\ 
 & \textit{ûf die tavelen wol gevar}\\ 
 & \textbf{diu mezzer} und leiten \textbf{si} nider\\ 
 & \textbf{\textit{und giengen dô} mit zühten} wider\\ 
5 & zuo den êrsten zwelven \textbf{stân}.\\ 
 & ob ich\textbf{z} gebrüefet rehte hân,\\ 
 & hie sullen ahzehen vrouwen stân.\\ 
 & â\textit{v}oy, nû siht man sehse gân\\ 
 & i\textit{n} wæte, die man tiure galt.\\ 
10 & \textbf{daz} was halbez bl\textit{î}alt,\\ 
 & daz ander pfelle von Ni\textit{niv}e.\\ 
 & dise und die êrsten sehse ê\\ 
 & truogen zwelf röcke geteilet,\\ 
 & gegen \textbf{tiuren} kost gev\textit{ei}let.\\ 
15 & nâch den \textbf{kam} diu künigîn.\\ 
 & \textbf{ir} antlitze gap den schîn,\\ 
 & si wânden al, ez wolte \textit{t}agen.\\ 
 & man sach die mage\textit{t} an ir tragen\\ 
 & pfelle \textbf{guot} von Arabi.\\ 
20 & û\textit{f} einem grüe\textit{n}en achmardî\\ 
 & truoc si den wunsch \textbf{vom} paradîs,\\ 
 & beide \textbf{wurzeln} und rîs:\\ 
 & daz was ein dinc, daz \textbf{hiez} der G\textit{r}âl,\\ 
 & erden wunsches über\textit{w}al.\\ 
25 & \textbf{Repa\textit{n}se de schoye} \textbf{si} hiez,\\ 
 & die \textbf{sich} \dag den\dag  Grâl tragen liez.\\ 
 & der Grâl was von solher art:\\ 
 & wol \textbf{muose ir} kiusche sîn bewart,\\ 
 & diu sî\textit{n} \textbf{ze} rehte solde pflegen.\\ 
30 & \textbf{si} muose valsches sich bewegen.\\ 
\end{tabular}
\scriptsize
\line(1,0){75} \newline
m n o Fr69 \newline
\line(1,0){75} \newline
\newline
\line(1,0){75} \newline
\textbf{1} \textit{Verse 235.1 und 235.4 kontrahiert zu:} Si nigen si mit zuhten wider m   $\cdot$ nigen] nyder n (o) \textbf{2} \textit{Vers 235.2 fehlt} m   $\cdot$ tavelen] tauel Fr69 \textbf{5} zwelven] zwolff o \textbf{6} gebrüefet] gepruͯfent o \textbf{7} ahzehen] aczehen o \textbf{8} âvoy nû] Anoi nuͯ m Owe nuͯ n Owe mẏner o \textbf{9} in wæte] Jwete m \textbf{10} halbez] helbes o  $\cdot$ blîalt] blalt m \textbf{11} Ninive] nime m jme n (o) \textbf{12} êrsten] [erster]: ersten m [sehsten]: ersten o \textbf{14} tiuren] túrer n (o) Fr69  $\cdot$ geveilet] geuiellet m \textbf{16} antlitze] anczlitz o \textbf{17} al ez] alles m  $\cdot$ tagen] iagen m jagen n o \textbf{18} maget] magen m \textbf{19} pfelle] Gefelle o  $\cdot$ Arabi] araby n arabẏ o arabj Fr69 \textbf{20} ûf] Vs m n  $\cdot$ grüenen] grumen m gruͤnem Fr69  $\cdot$ achmardî] archamadj Fr69 \textbf{21} vom] von n o \textbf{22} beide wurzeln] Beidv wrzen Fr69 \textbf{23} Grâl] gual m [gnal]: gral o \textbf{24} überwal] v̂ber al m \textbf{25} Repanse de schoye] Reppasse de schoi m Ropanse de schoẏe n [Res]: Repanse do schoie o \textbf{26} liez] liesse o \textbf{28} muose] muͯsse m muͯste n \textbf{29} sîn] sie m \textbf{30} si] So o  $\cdot$ muose] muͯsse m muͯste n muͤse Fr69 \newline
\end{minipage}
\end{table}
\newpage
\begin{table}[ht]
\begin{minipage}[t]{0.5\linewidth}
\small
\begin{center}*G
\end{center}
\begin{tabular}{rl}
 & \textbf{ez} nigen \textbf{z\textit{w}ô unde truogen} dar\\ 
 & ûf die tavelen wol gevar\\ 
 & \textbf{daz silber} unde leiten\textbf{z} nider.\\ 
 & \textbf{dô giengen si mit zühten} wider\\ 
5 & \textit{z}uo den êrsten \textit{zwelven} \textbf{stân}.\\ 
 & obe ich\textbf{z} geprüevet rehte hân,\\ 
 & hie sulen ahzehen vrouwen stên.\\ 
 & âvoy, nû siht man sehse gên\\ 
 & in wæte, die man tiure galt.\\ 
10 & \textbf{ez} was halbez plîalt,\\ 
 & daz ander pfelle von Ninve.\\ 
 & dise unt die êrsten sehse ê\\ 
 & truogen zwelf röcke geteilt,\\ 
 & gein \textbf{tiur} kost geveilt.\\ 
15 & nâch den \textbf{gie} diu künigîn.\\ 
 & \textbf{der} antlitze gap den schîn,\\ 
 & si wânden alle, ez wolde tagen.\\ 
 & man sach die maget an ir tragen\\ 
 & pfelle von Arabis.\\ 
20 & ûf einem grüenen achmardîs\\ 
 & truoc si den wunsch \textbf{von} pardîs,\\ 
 & beidiu \textbf{wurz} unde rîs:\\ 
 & daz was ein dinc, \textit{daz} \textbf{hiez} der Grâl,\\ 
 & erden wunsch\textit{es} über\textit{w}al.\\ 
25 & \textbf{Urrepanschoye} hiez,\\ 
 & die \textbf{man den} Grâl tragen liez.\\ 
 & der Grâl was von solher art:\\ 
 & wol \textbf{muoser} kiusche sîn bewart.\\ 
 & diu sîn \textbf{ze} reht solte pflegen,\\ 
30 & \textbf{diu} muose valsches sich bewegen.\\ 
\end{tabular}
\scriptsize
\line(1,0){75} \newline
G I O L M Q R Z Fr21 Fr51 \newline
\line(1,0){75} \newline
\textbf{1} \textit{Initiale} Z Fr21  \textbf{15} \textit{Initiale} I O  \textbf{27} \textit{Initiale} Fr51  \newline
\line(1,0){75} \newline
\textbf{1} ez] Jr Fr51  $\cdot$ zwô] zoͮ G ýe zwo L  $\cdot$ unde] die L  $\cdot$ truogen] tragen M \textbf{2} die] den R \textbf{3} leitenz] legtes Q \textbf{4} dô giengen si] Da gingen sie M (Z) Vnd gingen do Q  $\cdot$ wider] \textit{om.} M \textbf{5} zuo] aber zoͮ G  $\cdot$ zwelven] \textit{om.} G  $\cdot$ stân] san O L M Q Z Fr51 \textbf{6} ichz] ich Fr51  $\cdot$ geprüevet rehte] reht Geparliert I \textbf{8} âvoy] Awe O  $\cdot$ sehse] [sie]: sisze L spise R \textbf{9} man] ma I  $\cdot$ galt] Gap I \textbf{10} halbez] half Fr51 \textbf{11} Ninve] Niniue I (Q) ninive O (M) Z Nýnive L nyniue R nẏnẏue Fr51 \textbf{12} dise] Diesen Q  $\cdot$ êrsten] andern O (L) o\textit{m. } Fr51  $\cdot$ sehse] spise R \textbf{13} röcke] choche O  $\cdot$ geteilt] geteyl Q \textbf{14} gein tiur] Gein tivrr O (L) (R) Getivwer Q Mit túrrer R Zo durer Fr51 \textbf{15} nâch] ÷ach O  $\cdot$ den] dem R \textbf{16} der] Jr Q Fr51 \textbf{17} wânden] wolden O wondet R \textbf{18} an ir] abir M \textbf{19} Arabis] arabi O M R Z Arabý L araby Q Fr51 \textbf{20} einem] einen O Fr51  $\cdot$ grüenen] grúnem Q  $\cdot$ achmardîs] Achmardi O (L) (M) (Q) (R) (Z) (Fr51) \textbf{21} truoc si] Si truͤc I  $\cdot$ wunsch] wusch M  $\cdot$ von] \textit{om.} R \textbf{22} wurz] wuͯrtzen L vourze Fr51 \textbf{23} Daz dinch heiz der gral Fr51  $\cdot$ daz hiez] hiez G \textbf{24} erden wunsches] erden wunsch G (O) der het den wunsch I Erden wúnschen Q  $\cdot$ überwal] vber al G (I) Q (Z) vberval O \textbf{25} Urrepanschoye] Luuirpanschi de schoy I Repanse des hoie O Vrrepensade schoie L Vrrepansade schoie M Repansse de tshoye Q Rapanse dehoie R Vrrepanse de tschoie Z Vrrepanze de schoẏe Fr51  $\cdot$ hiez] sie hiez L (Q) \textbf{26} die] Den Fr51  $\cdot$ man den] sich der L R sie der Q sich den Z  $\cdot$ tragen] da tragen I  $\cdot$ liez] leẏz Fr51 \textbf{27} was] wasz so Q \textbf{28} wol muoser] er muͤst I Wol mvͦse ir O (L) (M) (Q) (R)  $\cdot$ bewart] gewart Q \textbf{29} solte] solten Q \textbf{30} diu] Der M  $\cdot$ muose] muͤst I (L) (M) (Q) (Z) (Fr51)  $\cdot$ valsches] [valschen]: valschez L  $\cdot$ sich] [mich]: sich O  $\cdot$ bewegen] verwegen R \newline
\end{minipage}
\hspace{0.5cm}
\begin{minipage}[t]{0.5\linewidth}
\small
\begin{center}*T
\end{center}
\begin{tabular}{rl}
 & \textbf{si} nigen. \textbf{dô truogen zwô} dar\\ 
 & ûf die taveln wol gevar\\ 
 & \textbf{daz silber} unde legeten\textbf{s} nider.\\ 
 & \textbf{mit grôzer zuht si giengen} wider\\ 
5 & zuo den êrsten zwelven \textbf{stân}.\\ 
 & \multicolumn{1}{l}{ - - - }\\ 
 & \multicolumn{1}{l}{ - - - }\\ 
 & Âvoy, nû siht man sehse gân\\ 
 & in wæte, die man tiure galt.\\ 
10 & \textbf{ez} was halbez blîalt,\\ 
 & daz ander \textbf{was} pfelle von Ninive.\\ 
 & dise unde die êrsten sehse ê\\ 
 & truogen zwelf röcke geteilet,\\ 
 & gegen \textbf{tiurer} koste gev\textit{ei}let.\\ 
15 & Nâch den \textbf{gie} diu künegîn.\\ 
 & \textbf{der} antlitze gap den schîn,\\ 
 & si wânden alle, ez wolte tagen.\\ 
 & man sach die maget an ir tragen\\ 
 & pfelle von Arabi.\\ 
20 & ûf einem grüenen achmardî\\ 
 & truoc si den wunsch \textbf{von} paradîs,\\ 
 & beidiu \textbf{wurzeln} unde rîs:\\ 
 & daz was ein dinc, daz \textbf{heizet} der Grâl,\\ 
 & erden wunsches über\textit{w}al.\\ 
25 & \textbf{Repanse de joie} \textbf{si} hiez,\\ 
 & die \textbf{sich der} Grâl tragen liez.\\ 
 & Der Grâl was von solher art:\\ 
 & wol \textbf{muose ir} kiusche sîn bewart,\\ 
 & di\textit{u} sîn rehte solte pflegen.\\ 
30 & \textbf{di\textit{u}} muose valsches sich bewegen.\\ 
\end{tabular}
\scriptsize
\line(1,0){75} \newline
T U V W \newline
\line(1,0){75} \newline
\textbf{8} \textit{Majuskel} T  \textbf{15} \textit{Majuskel} T  \textbf{23} \textit{Überschrift:} [*]: Hie sizet parzival v́ber tisch zvͦme grole bi anfortaz V  \textbf{27} \textit{Majuskel} T  \newline
\line(1,0){75} \newline
\textbf{1} dô] vnd W  $\cdot$ truogen] drungen U  $\cdot$ zwô dar] [*]: ir zwo dar V dar W \textbf{3} [*]: Die messer vnde leiten si nider V  $\cdot$ silber] selbe U \textbf{4} si giengen] gingen sie U (V) \textbf{5} zwelven] zwayen W \textbf{6} \textit{Versfolge 235.8-6-7} U V   $\cdot$ Ob ich iz gepruͦvet rechte han (recht geprúfet han W ) U (V) (W) \textbf{7} Hie soln ebern vreuwen stan U Hje soͤllent ahzehen juncfrowen stan (frawen gan W ) V (W) \textbf{8} Âvoy] Ach W  $\cdot$ gân] stan W \textbf{10} Plyat waz es ist mir gezalt W  $\cdot$ blîalt] phẏalt V \textbf{11} daz ander was] Vnd guͦt W \textbf{12} die] \textit{om.} W \textbf{14} geveilet] gevielet T \textbf{19} pfelle] Pfellor [*]: guͦt V  $\cdot$ Arabi] Araby T \textbf{20} achmardî] achẏmardi V \textbf{21} von] \textit{om.} W \textbf{22} wurzeln] wurtzel W \textbf{23} daz heizet der Grâl] daz [hei*]: heis der gral V vnd haist der gral W \textbf{24} Der erde wunsch úber al W  $\cdot$ überwal] vberal T (U) \textbf{25} Repanse de joie] Repanse de îoie T Repanse de ioie V Vtrepans de schoye W \textbf{26} die] [D*]: Do U  $\cdot$ der] [*n]: der V \textbf{28} muose] mvese T  $\cdot$ ir] er W \textbf{29} diu sîn] die sin T [D*]: Die sin ze V Die sein zuͦ W \textbf{30} diu] die T Sv́ V  $\cdot$ muose] mvese T muͦzen U  $\cdot$ sich] sein W \newline
\end{minipage}
\end{table}
\end{document}
