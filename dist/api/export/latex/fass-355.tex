\documentclass[8pt,a4paper,notitlepage]{article}
\usepackage{fullpage}
\usepackage{ulem}
\usepackage{xltxtra}
\usepackage{datetime}
\renewcommand{\dateseparator}{.}
\dmyyyydate
\usepackage{fancyhdr}
\usepackage{ifthen}
\pagestyle{fancy}
\fancyhf{}
\renewcommand{\headrulewidth}{0pt}
\fancyfoot[L]{\ifthenelse{\value{page}=1}{\today, \currenttime{} Uhr}{}}
\begin{document}
\begin{table}[ht]
\begin{minipage}[t]{0.5\linewidth}
\small
\begin{center}*D
\end{center}
\begin{tabular}{rl}
\textbf{355} & \begin{large}S\end{large}childes ambet zeige,\\ 
 & mîn \textbf{bestiu} zuht \textbf{ist} veige.\\ 
 & ez hülfe mich unt stüende ouch baz\\ 
 & sîn hulde den sîn grôzer haz.\\ 
5 & wie stêt ein tjost durch mînen schilt\\ 
 & mit sîner hende dar gezilt\\ 
 & oder ob versnîden sol mîn swert\\ 
 & sînen schilt, mînes hêrren wert?\\ 
 & \textbf{gelobt} daz iemer wîse wîp,\\ 
10 & diu \textbf{hât} \textbf{al ze} lôsen lîp.\\ 
 & \textbf{Nû} lât \textbf{mich} mînen hêrren hân\\ 
 & in mîme turne. ich m\textit{üe}ste in lân\\ 
 & unt mit im in den sînen.\\ 
 & swâr \textbf{an} er mich wil pînen,\\ 
15 & des stên ich \textbf{im} gar ze sînem gebote.\\ 
 & doch \textbf{sol} ich\textbf{s} gerne danken gote,\\ 
 & daz er mich niht gevangen hât,\\ 
 & sît in sîn \textbf{zürnen} niht erlât,\\ 
 & er enwelle mich hie besitzen.\\ 
20 & nû râtet mir mit witzen",\\ 
 & sprach er zen burgæren,\\ 
 & "gein disen strengen mæren."\\ 
 & Dô sprach dâ manec wîse man:\\ 
 & "m\textit{ö}ht ir unschult genozzen hân,\\ 
25 & ez \textbf{en}wære niht komen an disiu zil."\\ 
 & si gâben im des râtes vil,\\ 
 & daz er \textbf{sîne} porten ûf tæte\\ 
 & unt al die besten bæte\\ 
 & ûz gein der tjoste rîten.\\ 
30 & si jâhen: "wir mugen sô \textbf{strîten},\\ 
\end{tabular}
\scriptsize
\line(1,0){75} \newline
D \newline
\line(1,0){75} \newline
\textbf{1} \textit{Initiale} D  \textbf{11} \textit{Majuskel} D  \textbf{23} \textit{Majuskel} D  \newline
\line(1,0){75} \newline
\textbf{12} müeste] mvͦste D \textbf{24} möht] moht D \newline
\end{minipage}
\hspace{0.5cm}
\begin{minipage}[t]{0.5\linewidth}
\small
\begin{center}*m
\end{center}
\begin{tabular}{rl}
 & schiltes ambet z\textit{ei}ge,\\ 
 & mîn \textbf{beste} zuht \textbf{ich} veige.\\ 
 & ez hülfe mich und stüende ouch baz\\ 
 & sîn hulde danne sîn grôzer haz.\\ 
5 & wie stât ein just durch mînen schilt\\ 
 & mit sîner hende dar gezilt\\ 
 & oder ob versnîden sol mîn swert\\ 
 & sînen schilt, mînes hêrren wert?\\ 
 & \textbf{gelobet} daz iemer wîse wîp,\\ 
10 & diu \textbf{treit} \textbf{al ze} lôsen lîp.\\ 
 & \textbf{nû} lât \textbf{mich} mînen hêrren hân\\ 
 & in mînem turne. ich müese \textit{i}n lân\\ 
 & und mit ime in den sînen.\\ 
 & wâr \textbf{an} er mich wil pînen,\\ 
15 & des stân ich gar ze sînem gebote.\\ 
 & doch \textbf{sol} ich gerne danken gote,\\ 
 & daz er mich niht gevangen hât,\\ 
 & sît in sîn \textbf{zorn} \textbf{des} niht erlât,\\ 
 & er enwelle mich hie besitzen.\\ 
20 & nû râtet mir mit witzen",\\ 
 & sprach er zen burgæren,\\ 
 & "gegen disen strengen mæren."\\ 
 & \begin{large}D\end{large}ô sprach d\textit{â} manic wîse man:\\ 
 & "m\textit{ö}ht ir unschult genozzen hân,\\ 
25 & ez \textbf{en}wære niht komen an disiu zil."\\ 
 & si gâben ime des râtes vil,\\ 
 & daz er \textbf{sîne} porten ûf tæte\\ 
 & und alle die besten bæte\\ 
 & ûz gegen der juste rîten.\\ 
30 & si jâhen: "wir mügen sô \textbf{strîten},\\ 
\end{tabular}
\scriptsize
\line(1,0){75} \newline
m n o \newline
\line(1,0){75} \newline
\textbf{23} \textit{Initiale} m   $\cdot$ \textit{Capitulumzeichen} n  \newline
\line(1,0){75} \newline
\textbf{1} zeige] zoúge m \textbf{2} ich] ist n o \textbf{4} danne] don o \textbf{6} mit] [Min]: Mit m  $\cdot$ gezilt] gezelt o \textbf{10} al ze] alzuͯ hant n \textbf{12} müese] muͯsse m musz n muͦs o  $\cdot$ in] sin m mich o \textbf{14} wil] músz o \textbf{17} \textit{Versfolge 356.11-26 (¹m), 355.17-22 (¹m) (Bl. 228v), Versdoppelung 355.17-22 (²m), dann 355.23-356.10 (Bl. 229r), Versdoppelung 356.11-14 (²m), dann 356.27-357.14 (Bl. 229v)} m  \textbf{18} sît] Sin \textsuperscript{2}\hspace{-1.3mm} m  $\cdot$ sîn zorn] sin zit vnd zorn \textsuperscript{2}\hspace{-1.3mm} m \textbf{19} enwelle] welle n o \textbf{21} zen] zurn o \textbf{22} strengen] trengen o \textbf{23} dâ] do m \textit{om.} n o \textbf{24} möht] Moht m (n) Mucht o \textbf{25} enwære] were n (o) \textbf{27} porten] porte n o \textbf{28} alle] er n o \textbf{29} rîten] reten o \newline
\end{minipage}
\end{table}
\newpage
\begin{table}[ht]
\begin{minipage}[t]{0.5\linewidth}
\small
\begin{center}*G
\end{center}
\begin{tabular}{rl}
 & schiltes ambet zeige,\\ 
 & mîn \textbf{bestiu} zuht \textbf{ist} veige.\\ 
 & \textit{ez hülfe mich und stüende ouch baz}\\ 
 & \textit{sîn hulde dan sîn grôzer haz}\\ 
5 & wie stêt ein tjost durch mînen schilt\\ 
 & mit sîner hende dar gezilt\\ 
 & oder obe versnîden sol mîn swert\\ 
 & sînen schilt, mînes hêrren wert?\\ 
 & \textbf{gelobet} daz imer wîse wîp,\\ 
10 & diu \textbf{treit} \textbf{alze} lôsen lîp.\\ 
 & \textbf{nû} lât \textbf{mich} mînen hêrren hân\\ 
 & in mînem turne. ich m\textit{üe}se in lân\\ 
 & unde mit im in den sînen.\\ 
 & swâr er mich wil pî\textit{n}en,\\ 
15 & des stên ich g\textit{ar} ze sînem gebote.\\ 
 & doch \textbf{wil} ich \textit{gerne} danken gote,\\ 
 & daz er mich niht gevangen hât,\\ 
 & sît in sîn \textbf{zürnen} niht erlât,\\ 
 & er enwelle mich hie besitzen.\\ 
20 & nû râtet mir mit witzen",\\ 
 & sprach er zen burgæren,\\ 
 & "gein disen strengen mæren."\\ 
 & dô sprach dâ manic wîse man:\\ 
 & "m\textit{ö}ht ir unschult genozzen hân,\\ 
25 & ez wære niht komen an disiu zil."\\ 
 & si gâben im des râtes vil,\\ 
 & \begin{large}D\end{large}az \textit{er} \textbf{\textit{sîn}e} porte\textit{n} ûf tæte\\ 
 & unde aldie besten bæte\\ 
 & ûz gein der tjoste rîten.\\ 
30 & si jâhen: "wir mugen sô \textbf{strîten},\\ 
\end{tabular}
\scriptsize
\line(1,0){75} \newline
G I O L M Q R Z Fr39 \newline
\line(1,0){75} \newline
\textbf{5} \textit{Initiale} I O L Z Fr39   $\cdot$ \textit{Capitulumzeichen} R  \textbf{23} \textit{Initiale} I  \textbf{27} \textit{Initiale} G  \newline
\line(1,0){75} \newline
\textbf{1} zeige] erczeige M gelde zeige Q \textbf{2} bestiu] beste R Fr39 \textbf{3} \textit{Die Verse 355.3-4 fehlen} G  \textbf{4} hulde] schulden R \textbf{5} wie] ÷ie O Hie L Q  $\cdot$ stêt] stuͤnde I  $\cdot$ ein] sin M \textbf{6} dar] gar M \textbf{7} oder] olde G \textbf{8} sînen] Den L Fr39  $\cdot$ schilt] shiht I \textbf{9} wîse] weises Q \textbf{10} treit] tret R  $\cdot$ alze] als ze R \textbf{12} müese] moͮse G (O) (Q) (Z) musz M \textbf{14} swâr er] Swar an er O Z Fr39 War [oner]: an er L War an er M R Swor er an Q  $\cdot$ pînen] pinenen G \textbf{15} stên] wil I  $\cdot$ gar] gerne G gesten I \textbf{16} gerne] imer G \textbf{18} sît] Sein Q  $\cdot$ sîn] \textit{om.} I \textbf{19} enwelle] well I (O) (R) enwellen Fr39  $\cdot$ hie] \textit{om.} Z \textbf{21} zen] zűm Q  $\cdot$ burgæren] buͯrgare L \textbf{22} gein] Zu R \textbf{23} dô] Da M So Fr39  $\cdot$ dâ] do O \textit{om.} L Q R Fr39 \textbf{24} möht] moht G I (O) (L) (M) (Q) (Z) Fr39 \textbf{25} ez] ezn I (L) (M) (Fr39) So Q  $\cdot$ niht] nye Q (R)  $\cdot$ disiu] daz R \textbf{27} er sîne porten] man die porte G er sine porte L Fr39 er sine tor R \textbf{29} tjoste] tiostin M \textbf{30} jâhen] sprachin M  $\cdot$ wir] sie Z \newline
\end{minipage}
\hspace{0.5cm}
\begin{minipage}[t]{0.5\linewidth}
\small
\begin{center}*T
\end{center}
\begin{tabular}{rl}
 & schiltes ambet zeige,\\ 
 & mîn \textbf{bestiu} zuht \textbf{ist} veige.\\ 
 & \multicolumn{1}{l}{ - - - }\\ 
 & \multicolumn{1}{l}{ - - - }\\ 
5 & wie stêt ein tjost durch mînen schilt\\ 
 & mit sîner hende dar gezilt\\ 
 & oder ob versnîden sol mîn swert\\ 
 & sînen schilt, mînes hêrren wert?\\ 
 & \textbf{gelobe} daz iemer wîse wîp,\\ 
10 & diu \textbf{treit} \textbf{einen} lôsen lîp.\\ 
 & Lât mînen hêrren hân\\ 
 & in mînem turne. ich müesin lân\\ 
 & unde mit im in den sînen.\\ 
 & swâr \textbf{an} er mich wil pînen,\\ 
15 & des stân ich gar ze sînem gebote.\\ 
 & doch \textbf{wil} ich gerne danken gote,\\ 
 & daz er mich niht gevangen hât,\\ 
 & sît in sîn \textbf{zorn} niht erlât,\\ 
 & er enwelle mich hie besitzen.\\ 
20 & nû râtet mir mit witzen",\\ 
 & sprach er zen burgæren,\\ 
 & "gegen disen strengen mæren."\\ 
 & \begin{large}D\end{large}ô sprach dâ manec wîse man:\\ 
 & "m\textit{ö}htir unschulde genozzen hân,\\ 
25 & ez wære niht komen an disiu zil."\\ 
 & si gâben im des râtes vil,\\ 
 & daz er \textbf{die} porten ûf tæte\\ 
 & unde aldie besten bæte\\ 
 & ûz gegen der tjost rîten.\\ 
30 & si jâhen: "wir mugen sô \textbf{gestrîten},\\ 
\end{tabular}
\scriptsize
\line(1,0){75} \newline
T V W \newline
\line(1,0){75} \newline
\textbf{11} \textit{Majuskel} T  \textbf{23} \textit{Initiale} T  \newline
\line(1,0){75} \newline
\textbf{2} ist] ich V \textbf{3} \textit{Die Verse 355.3-4 fehlen} T W   $\cdot$ Ez húlfe mich vnde stúnde oͮch bas V \textbf{4} Sin hulde denne sin grosser has V \textbf{6} Mit seiner hand durch meinen hilt W \textbf{10} einen lôsen] alzelosen V (W) \textbf{11} Lât] Nv lant mich V Lont mich W \textbf{12} mînem] meinen W  $\cdot$ müesin] muͦß in W \textbf{13} in den] die W \textbf{14} swâr an] War W \textbf{15} stân] stand W \textbf{17} er] ir W \textbf{18} niht] dez niht V \textbf{19} enwelle] welle W \textbf{23} dâ] do V \textit{om.} W \textbf{24} möhtir] mohtir T Moͤchten ir W \textbf{25} wære] enwere V  $\cdot$ disiu] diß W \textbf{26} des] \textit{om.} W \textbf{27} die porten] sine porte V (W) \textbf{30} gestrîten] streiten W \newline
\end{minipage}
\end{table}
\end{document}
