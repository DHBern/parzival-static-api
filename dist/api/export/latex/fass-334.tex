\documentclass[8pt,a4paper,notitlepage]{article}
\usepackage{fullpage}
\usepackage{ulem}
\usepackage{xltxtra}
\usepackage{datetime}
\renewcommand{\dateseparator}{.}
\dmyyyydate
\usepackage{fancyhdr}
\usepackage{ifthen}
\pagestyle{fancy}
\fancyhf{}
\renewcommand{\headrulewidth}{0pt}
\fancyfoot[L]{\ifthenelse{\value{page}=1}{\today, \currenttime{} Uhr}{}}
\begin{document}
\begin{table}[ht]
\begin{minipage}[t]{0.5\linewidth}
\small
\begin{center}*D
\end{center}
\begin{tabular}{rl}
\textbf{334} & \textbf{\begin{large}D\end{large}ô} \textbf{vuor} der messenîe vil\\ 
 & gein dem arbeitlîchem zil,\\ 
 & ein âventiure ze schouwen,\\ 
 & dâ vier hundert juncvrouwen\\ 
5 & unt vier küneginne\\ 
 & gevangen wâren inne,\\ 
 & \textbf{ze} Schastel Marveile.\\ 
 & swaz in \textbf{dâ} \textbf{wart} ze teile,\\ 
 & daz haben âne mînen haz;\\ 
10 & ich bin \textbf{doch} vrouwen lônes laz.\\ 
 & \textbf{Dô} sprach der Krieche Clias:\\ 
 & "ich bin der dâ versûmet was."\\ 
 & vor in allen er des jach.\\ 
 & "\textbf{der} Turkote mich dâ stach\\ 
15 & hinder\textit{z} ors, ich muoz mich schamen.\\ 
 & \textbf{doch sagt er} mir vier vrouwen namen,\\ 
 & die dâ krônebære sint.\\ 
 & zwô sint alt, zwô sint \textbf{noch} kint.\\ 
 & der heizet einiu Itonje,\\ 
20 & diu ander \textbf{heizet} Cundrie,\\ 
 & diu dritte \textbf{Arnive},\\ 
 & diu vierde \textbf{Sangive}."\\ 
 & daz wolt ieslîcher \textbf{dâ} \textbf{besehen}.\\ 
 & \textbf{ez enmoht ir reise} niht vol spehen:\\ 
25 & si muosten schaden dâ bejagen,\\ 
 & den sol ouch ich ze mâzen klagen,\\ 
 & \textbf{Wan} swer durch wîp hât arbeit,\\ 
 & \textbf{daz} gît \textbf{im} vreude, etswenne \textbf{ouch} leit\\ 
 & an dem orte vürbaz wigt.\\ 
30 & sus \textbf{dicke minne} ir lônes pfligt.\\ 
\end{tabular}
\scriptsize
\line(1,0){75} \newline
D \newline
\line(1,0){75} \newline
\textbf{1} \textit{Initiale} D  \textbf{11} \textit{Majuskel} D  \textbf{27} \textit{Majuskel} D  \newline
\line(1,0){75} \newline
\textbf{7} Schastel] Scastel D \textbf{11} Krieche] chrîeche D \textbf{14} Turkote] Tvrkoẏte D \textbf{15} hinderz] hinders D \textbf{19} Itonje] Jtonîe D \textbf{20} Cundrie] Cvndrîe D \textbf{21} Arnive] Arnîve D \textbf{22} Sangive] Sangîve D \newline
\end{minipage}
\hspace{0.5cm}
\begin{minipage}[t]{0.5\linewidth}
\small
\begin{center}*m
\end{center}
\begin{tabular}{rl}
 & \textbf{\begin{large}D\end{large}ô} \textbf{vuor} der massenîe vil\\ 
 & gegen dem arbe\textit{it}lîch\textit{en} zil,\\ 
 & ein âventiure ze schouwen,\\ 
 & d\textit{â} vier hundert juncvrouwen\\ 
5 & und vier küniginne\\ 
 & gevangen wâren inne,\\ 
 & \textbf{ze} Schahtel Marv\textit{ei}le.\\ 
 & waz in \textbf{dâr} \textbf{wart} ze teile,\\ 
 & daz haben \dag in\dag  âne mîne\textit{n} haz;\\ 
10 & ich bin \textbf{doch} vrouwen \dag lônen\dag  laz.\\ 
 & \textbf{dô} sprach der Krieche Klias:\\ 
 & "ich bin der d\textit{â} versûmet was."\\ 
 & vor in allen er des jach.\\ 
 & "\textbf{der} Turkoite mich d\textit{â} stach\\ 
15 & hinder daz ros, ich muoz mich sch\textit{am}en.\\ 
 & \textbf{doch sagete er} mir vier vrouwen na\textit{m}en,\\ 
 & die dâ krônebære sint.\\ 
 & zwô sint alt, zwô sint \textbf{noch} kint.\\ 
 & der heizet einiu Ithonie,\\ 
20 & diu ander \textbf{heizet} C\textit{o}ndrie,\\ 
 & diu dritte \textbf{heizet} \textbf{Sa\textit{n}gi\textit{v}e},\\ 
 & diu vierde \textbf{Ar\textit{niv}e}."\\ 
 & daz wolte etlîcher \textbf{d\textit{â}} \textbf{besehen}.\\ 
 & \textbf{ez enm\textit{o}hte ir reise} niht volle spehen:\\ 
25 & si muosen schaden d\textit{â} bejagen,\\ 
 & den sol ouch ich ze mâze klagen,\\ 
 & \textbf{wanne} wer durch wîp hât arbeit,\\ 
 & \textbf{daz} gît \textbf{ime} vröude, etwenne leit\\ 
 & an dem orte vürbaz \textbf{mêre} wiget.\\ 
30 & sus \textbf{minne dicke} ir lô\textit{n}es pfliget.\\ 
\end{tabular}
\scriptsize
\line(1,0){75} \newline
m n o \newline
\line(1,0){75} \newline
\textbf{1} \textit{Initiale} m   $\cdot$ \textit{Capitulumzeichen} n  \newline
\line(1,0){75} \newline
\textbf{1} vuor] vor o  $\cdot$ vil] [vier]: vil o \textbf{2} gegen] Gegein o  $\cdot$ arbeitlîchen] arbechlich m \textbf{3} ze] \textit{om.} n \textbf{4} dâ] Do m n o  $\cdot$ vier] vie o \textbf{7} Zeschahtel mar viele m  $\cdot$ Zuͯ schahteil marfeil n  $\cdot$ Zuͦ schalteẏl marfeil o \textbf{8} dâr] do n o \textbf{9} in] an o  $\cdot$ mînen] mine m \textbf{11} Krieche] krieg n kriech o  $\cdot$ Klias] klẏas n [klichen]: klias o \textbf{12} dâ] do m n o \textbf{13} allen er des] aller er dasz o \textbf{14} Turkoite] turckoite m trurekeite n tuͯrkeite o  $\cdot$ dâ] do m n o \textbf{15} muoz] muͯs m (o)  $\cdot$ schamen] schouwen m \textbf{16} doch] Do n  $\cdot$ sagete] saget n (o)  $\cdot$ namen] nanen m \textbf{17} dâ] do n o \textbf{18} zwô] Zw o \textbf{19} Ithonie] itonie m ẏtonẏe n ytonie o \textbf{20} heizet] \textit{om.} n o  $\cdot$ Condrie] cundrie m condrẏe n \textbf{21} Sangive] Sagie m sangie n sagine o \textbf{22} vierde] \textit{om.} o  $\cdot$ Arnive] aruͯwe m armẏe n \textbf{23} dâ] do m n o \textbf{24} enmohte] enmoͯhte m moͯchte n mochte o  $\cdot$ volle] wolte o \textbf{25} muosen] mussen m muͯssen n o  $\cdot$ dâ] do m n o \textbf{26} ouch] uch o  $\cdot$ mâze] mossen n (o) \textbf{28} etwenne] ouch etwenne n etwen auch o \textbf{29} mêre] \textit{om.} n o \textbf{30} ir] \textit{om.} o  $\cdot$ lônes] lontes m \newline
\end{minipage}
\end{table}
\newpage
\begin{table}[ht]
\begin{minipage}[t]{0.5\linewidth}
\small
\begin{center}*G
\end{center}
\begin{tabular}{rl}
 & \textbf{ouch} \textbf{begunde} der massenîe vil\\ 
 & gein dem arbeitlîchen zil,\\ 
 & \textit{ein} âventiure ze schouwen,\\ 
 & dâ vier hundert juncvrouwen\\ 
5 & \begin{large}U\end{large}nde vier küniginne\\ 
 & gevangen wâren inne,\\ 
 & \textbf{ûf} Tschastel Marveile.\\ 
 & swaz in \textbf{dâ} \textbf{wart} ze teile,\\ 
 & daz haben âne mînen haz;\\ 
10 & ich bin \textbf{doch} vrouwen lônes laz.\\ 
 & \textbf{ouch} sprach der \textit{Kr}ie\textit{che} Clias:\\ 
 & "ich bin der dâ versûmet was."\\ 
 & vor in allen er des jach.\\ 
 & "\textbf{ein} Turkoite mich dâ stach\\ 
15 & hinderz ors, ich muoz mich schamen.\\ 
 & \textbf{er seite} mir vier vrouwen namen,\\ 
 & die dâ krônebære sint.\\ 
 & zwô sint alt, zwô sint \textbf{noch} kint.\\ 
 & d\textit{er h}ei\textit{zt e}i\textit{niu} Itonie,\\ 
20 & diu ander \textit{\textbf{heizt}} Gundrie,\\ 
 & diu dritte \textbf{heizt} \textbf{Arnive},\\ 
 & diu vierde \textbf{Sagive}."\\ 
 & daz wolt ieslîcher \textbf{sehen}.\\ 
 & \textbf{ir reise moht ez} niht vol spehen:\\ 
25 & si muosen schaden dâ bejagen,\\ 
 & den sol ouch ich ze mâze klagen,\\ 
 & \textbf{wan} swer durch wîp hât arbeit,\\ 
 & \textbf{ez} gît vröude, etswenne \textbf{ouch} leit.\\ 
 & an dem orte \textbf{ez} vürbaz wiget.\\ 
30 & sus \textbf{dicke minne} ir lônes pfliget.\\ 
\end{tabular}
\scriptsize
\line(1,0){75} \newline
G I O L M Q R Z Fr21 Fr27 Fr39 \newline
\line(1,0){75} \newline
\textbf{1} \textit{Initiale} L Q Fr21 Fr39   $\cdot$ \textit{Capitulumzeichen} R  \textbf{5} \textit{Initiale} G  \textbf{11} \textit{Initiale} I  \textbf{13} \textit{Initiale} O  \newline
\line(1,0){75} \newline
\textbf{1} ouch] Svst O (Q) (R) (Fr21)  $\cdot$ begunde] chom I chert O (Q) (R) (Z) (Fr21) kerte L (M) Fr39 \textbf{2} dem] der R  $\cdot$ arbeitlîchen] [arbaitliche*]: arbaitlichem I arbeitlichem O (M) Z arbeitsame L arbeit samen Fr39 \textbf{3} ein] \textit{om.} G Gen Q \textbf{4} dâ] Do Q R (Fr39) \textbf{7} Tschastel Marveile] tschater marveile G shatelmarvelde I tshahtez Mar veile O thahtel marveile L schachtel marfeile M schatel marueile Q Tschahtel marweile R tschatel marveile Z tschatel m::: Fr21 :::vaile Fr27 tsahtel marveile Fr39 \textbf{8} swaz] Waz L (M) (Q) (R)  $\cdot$ in] im I  $\cdot$ dâ] do Q R Fr39 \textbf{9} âne] sie ane M \textbf{10} laz] basz Q \textbf{11} ouch] Do I Q R Z  $\cdot$ Krieche] fier G crie I krige Q  $\cdot$ Clias] klias I heilias M klyas Q R \textbf{12} dâ] do Q Fr39  $\cdot$ versûmet] vorsunet M versmehet Q \textbf{13} vor] ÷or O  $\cdot$ er] e er R  $\cdot$ des jach] das sprach M das Jach R \textbf{14} ein] Der Q R Z  $\cdot$ Turkoite] turkoyte G (R) turkoide I tvrkorte O tuͯrkoýte L Tvrkoẏte Fr21  $\cdot$ dâ] do Q R Fr39 \textbf{16} er seite] er seit I Doch seit er O L M Q (R) Z Fr21 (Fr39)  $\cdot$ vrouwen] ivncfroͮn Fr21  $\cdot$ namen] nam Q \textbf{17} dâ] do Q Fr39 \textbf{18} Wo seint alt wo sint noch kint Q  $\cdot$ sint noch] sint I noch O \textbf{19} der heizt einiu] div ein heizt G Der hiez einev O  $\cdot$ Itonie] Jhtvͦnîe O Jtonie L (M) Z ytonie Q Nytonie R J::: Fr21 Jtunie Fr39 \textbf{20} ander] andriv Fr21 Fr39  $\cdot$ heizt] \textit{om.} G  $\cdot$ Gundrie] kundrie I (L) M Z (Fr39) Gvndrîe O Kondrie R :::rie Fr21 \textbf{21} \textit{Verse 334.20-21 kontrahiert zu:} Die anderre heisset arniue Q   $\cdot$ Arnive] arniue I Arnyue R :::e Fr21 \textbf{22} diu] die Fr39  $\cdot$ vierde] vierde haizet I (O)  $\cdot$ Sagive] saivie G saliue I saive O M Seýve L seyre Q Seyue R Seive Z Savͥ: Fr21 Seyve Fr39 \textbf{23} daz] Die L (Fr39)  $\cdot$ sehen] da besehen O L (M) Z (Fr21) Fr39 do beschawen Q do besechen R \textbf{24} vol] wol I L M R Z Fr39 \textbf{25} schaden] schanden Q  $\cdot$ dâ] do Q Fr39 \textbf{26} ouch ich] ich I ovch O ich auch Q  $\cdot$ ze mâze] zemazzen I (M) (Z) \textbf{27} swer] wer O L M Q \textit{om.} R \textbf{28} ez gît] Er hat M  $\cdot$ vröude] vroide vnde M  $\cdot$ ouch] \textit{om.} I L M Fr39 \textbf{30} dicke minne] minne diche I (L) (Q) (R) (Fr39) \newline
\end{minipage}
\hspace{0.5cm}
\begin{minipage}[t]{0.5\linewidth}
\small
\begin{center}*T
\end{center}
\begin{tabular}{rl}
 & \textbf{\begin{large}O\end{large}uch} \textbf{kêrte} der massenîe vil\\ 
 & gegen dem arbeitlîchem zil,\\ 
 & ein âventiure ze schouwen,\\ 
 & dâ vier hundert junc\textit{vrouw}en\\ 
5 & unde vier küneginne\\ 
 & gevangen wâren inne,\\ 
 & \textbf{ûf} Tschahtel Marveile.\\ 
 & waz in \textbf{würde} ze teile,\\ 
 & daz haben âne mînen haz;\\ 
10 & ich bin \textbf{durch} vrouwen lônes laz.\\ 
 & \textbf{Ouch} sprach der Krieche Clyas:\\ 
 & "ich bin der dâ versûmet was."\\ 
 & vor in allen er des jach.\\ 
 & "\textbf{ein} Turkoyt mich dâ stach\\ 
15 & hinder\textit{z} ors, ich muoz mich schamen.\\ 
 & \textbf{doch saget er} mir vier vrouwen namen,\\ 
 & die dâ krônbære sint.\\ 
 & Zwô sint alt, zwô sint kint.\\ 
 & Der heizet ein\textit{iu} Itonie,\\ 
20 & diu ander Kuondrie,\\ 
 & diu dritte \textbf{heizet} \textbf{Seyve}\\ 
 & \textbf{unde} diu vierde \textbf{Arnyve}."\\ 
 & daz wolt \textbf{ir} ieglîcher \textbf{sehen}.\\ 
 & \textbf{ir reise mohte\textit{z}} niht volle spehen:\\ 
25 & si muosen schaden dâ bejagen,\\ 
 & den sol ouch ich ze mâze klagen.\\ 
 & Swer durch wîp hât arbeit,\\ 
 & \textbf{ez} gît vröude \textbf{unde} etswenne leit.\\ 
 & An dem orte \textbf{ez} vürbaz wiget.\\ 
30 & sus \textbf{dicke minn\textit{e}} \textit{i}r lônes pfliget.\\ 
\end{tabular}
\scriptsize
\line(1,0){75} \newline
T U V W \newline
\line(1,0){75} \newline
\textbf{1} \textit{Initiale} T U W  \textbf{11} \textit{Majuskel} T  \textbf{18} \textit{Majuskel} T  \textbf{19} \textit{Majuskel} T  \textbf{27} \textit{Majuskel} T  \textbf{29} \textit{Majuskel} T  \newline
\line(1,0){75} \newline
\textbf{1} Ouch] DO W \textbf{3} ze schouwen] [*schowen]: ze schowen V \textbf{4} dâ] do U V W  $\cdot$ juncvrouwen] ivngen T \textbf{5} vier] vier iunge W \textbf{7} Tschahtel Marveile] Tscahtel marveile T Tschatel marveile U Scatelmarveile V kastel marfeile W \textbf{8} würde] [*]: do wart V \textbf{9} âne] [*]: in ane V \textbf{10} durch] doch V \textbf{11} Ouch] [*h]: Do V  $\cdot$ Krieche] Crieche T U [*]: krieche V kriech W  $\cdot$ Clyas] Clias V lyas W \textbf{12} dâ] do U V \textit{om.} W \textbf{13} vor] Von W  $\cdot$ jach] veriach W \textbf{14} ein] [*]: Der V  $\cdot$ Turkoyt] Tvrkoẏt V teúre haiden W  $\cdot$ dâ] do U V W \textbf{15} hinderz] hinders T Hinder U \textbf{16} namen] [na*]: namen V \textbf{17} dâ] do V \textit{om.} W \textbf{18} sint alt] alt W  $\cdot$ sint kint] sint [*]: noch kint V noch kint W \textbf{19} Die eine heisset [*onie]: ẏtonie V  $\cdot$ einiu] eine T  $\cdot$ Itonie] Jtonie T ytonie W \textbf{20} diu] Druͦ U  $\cdot$ Kuondrie] kvndrie T V kuͦndrie U haisset kiriadrie W \textbf{21} Seyve] Sêyue T Arnyve U [aruiue]: arnine V arnie W \textbf{22} unde diu] Die W  $\cdot$ vierde] viere V  $\cdot$ Arnyve] Seyve U seẏue V haisset saffie W \textbf{23} [D*]: Daz wolte etlicher do besehen V  $\cdot$ wolt ir] wolte W \textbf{24} mohtez] mohtes T moͤht ez V moͤchte W  $\cdot$ volle] wol W \textbf{25} muosen] mvesen T muͦzen U mvͤsten V (W)  $\cdot$ dâ] do V W \textbf{26} ouch ich] man auch W \textbf{27} Swer] Wer U W \textbf{28} [*]: Daz git im froͤide etzwenne oͮch leit V  $\cdot$ gît] gibt auch W  $\cdot$ etswenne leit] hertzelait W \textbf{29} An] In W  $\cdot$ ez] er W \textbf{30} minne ir] minne dicke ir T \newline
\end{minipage}
\end{table}
\end{document}
