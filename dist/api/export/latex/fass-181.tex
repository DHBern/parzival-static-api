\documentclass[8pt,a4paper,notitlepage]{article}
\usepackage{fullpage}
\usepackage{ulem}
\usepackage{xltxtra}
\usepackage{datetime}
\renewcommand{\dateseparator}{.}
\dmyyyydate
\usepackage{fancyhdr}
\usepackage{ifthen}
\pagestyle{fancy}
\fancyhf{}
\renewcommand{\headrulewidth}{0pt}
\fancyfoot[L]{\ifthenelse{\value{page}=1}{\today, \currenttime{} Uhr}{}}
\begin{document}
\begin{table}[ht]
\begin{minipage}[t]{0.5\linewidth}
\small
\begin{center}*D
\end{center}
\begin{tabular}{rl}
\textbf{181} & sint, sô si armbrustes span\\ 
 & mit senewen \textbf{swanke} trîbet dan.\\ 
 & dâr \textbf{umbe} gie ein brückenslac,\\ 
 & dâ manec hurt ûffe lac.\\ 
5 & \textbf{ez} vlôz aldâ reht inz mer.\\ 
 & Pelrapeire \textbf{stuont} \textbf{wol} ze wer.\\ 
 & Seht, wie kint ûf schocken varn,\\ 
 & die man \textbf{schockes} niht wil sparn:\\ 
 & sus vuor diu brücke âne seil.\\ 
10 & \textbf{diu} was vor jugende niht sô geil.\\ 
 & Dort anderthalben stuonden\\ 
 & mit helmen ûf gebunden\\ 
 & sehzec ritter oder mêr.\\ 
 & \textbf{die} riefen alle: "kêrâ kêr!"\\ 
15 & mit ûf geworfen swerten\\ 
 & die kranken strîtes gerten.\\ 
 & \textbf{\begin{large}D\end{large}urch daz si in} dicke sâhen ê,\\ 
 & si wânden, ez wære Clamide,\\ 
 & wander sô \textbf{küneclîchen} reit\\ 
20 & gein der brücke ûf dem velde breit.\\ 
 & dô si disen jungen man\\ 
 & sus mit schalle riefen an,\\ 
 & swie \textbf{vil} erz ors mit sporen versneit,\\ 
 & durch vorhte ez doch die brücken meit.\\ 
25 & \textbf{den} rehtiu zagheit ie \textbf{vlôch},\\ 
 & der erbeizte nider unde zôch\\ 
 & sîn ors \textbf{ûf} der brücken swanc.\\ 
 & eines zagen muot wære al ze kranc,\\ 
 & solt er gein sölhem strîte varn.\\ 
30 & dar zuo muoser ein dinc bewarn,\\ 
\end{tabular}
\scriptsize
\line(1,0){75} \newline
D Fr15 \newline
\line(1,0){75} \newline
\textbf{11} \textit{Majuskel} D  \textbf{17} \textit{Initiale} D  \textbf{23} \textit{Initiale} Fr15  \newline
\line(1,0){75} \newline
\textbf{5} \textit{Versfolge 181.6-5} Fr15  \textbf{6} Pelrapeire] Pelrapeir::: Fr15 \textbf{18} Clamide] Chlamide D \newline
\end{minipage}
\hspace{0.5cm}
\begin{minipage}[t]{0.5\linewidth}
\small
\begin{center}*m
\end{center}
\begin{tabular}{rl}
 & sint, sô \dag sin\dag  armbrustes span\\ 
 & mit senwen \textbf{swanze} trîbet dan.\\ 
 & \dag haraber giengen im\dag  brückenslac,\\ 
 & dâ manic \dag hütten\dag  ûf lac.\\ 
5 & \textbf{ez} vlôz aldâ reht in daz mer.\\ 
 & P\textit{e}l\textit{r}aperie \textbf{stuont} ze wer.\\ 
 & sehet, wie kint ûf schocken varn,\\ 
 & die man \textbf{schockes} niht wil sparn:\\ 
 & sus vu\textit{or} diu br\textit{ü}cke âne seil.\\ 
10 & \textbf{si} was vor jugende niht sô geil.\\ 
 & \begin{large}D\end{large}ort anderhalben stuonden\\ 
 & mit helmen ûf gebunden\\ 
 & sehzic ritter oder mêr.\\ 
 & \textbf{die} riefen alle: "kêrâ kêr!"\\ 
15 & mit ûf geworfenen swerten\\ 
 & die kranken strîtes gerten.\\ 
 & \hspace*{-.7em}\big| si wânden, ez wære Clamide,\\ 
 & \hspace*{-.7em}\big| \textbf{den si vil} dicke sâhen ê,\\ 
 & wan er sô \textbf{küniclîchen} reit\\ 
20 & gegen der brücke ûf dem velde breit.\\ 
 & dô si disen jungen man\\ 
 & sus mit schalle riefen an,\\ 
 & \dag swig\dag  \textbf{vil} erz ros mit sporen versneit,\\ 
 & durch vorhte ez doch die brücke meit.\\ 
25 & \textbf{den} rehtiu zagheit ie \textbf{geflôch},\\ 
 & der erbeizete nider und zôch\\ 
 & sîn ros \textbf{ûf} der brücke swanc.\\ 
 & eines zage\textit{n} m\textit{uo}t wær al ze kranc,\\ 
 & solt er gegen solichem strîte varn.\\ 
30 & dar zuo muos er ein dinc bewarn,\\ 
\end{tabular}
\scriptsize
\line(1,0){75} \newline
m n o Fr69 \newline
\line(1,0){75} \newline
\textbf{11} \textit{Initiale} m  \newline
\line(1,0){75} \newline
\textbf{3} haraber] Her abe n (o) \textbf{5} vlôz] slos o \textbf{6} Pelraperie] Pil apperie m Penlapier n Peẏlapeir o  $\cdot$ ze] wol zuͯ n (o) \textbf{7} kint] [kunig]: kinnt m \textbf{8} schockes] schockens n o  $\cdot$ niht] \textit{om.} o \textbf{9} vuor] fure m  $\cdot$ brücke] brake m \textbf{11} anderhalben] an der selben o \textbf{13} sehzic] Seczig o \textbf{15} geworfenen] gewoffen o \textbf{17} si] sehent sú n \textbf{19} küniclîchen] kuͤnlichen Fr69 \textbf{23} swig vil] swigen n o  $\cdot$ versneit] sneit n o \textbf{24} vorhte] frocht o  $\cdot$ meit] \textit{om.} Fr69 \textbf{25} den] Denne n Deren Fr69  $\cdot$ rehtiu] rechten o  $\cdot$ geflôch] floch n o (Fr69) \textbf{26} der] Die o \textbf{28} zagen] zagens m  $\cdot$ muot] mit m nit n \textbf{30} muos] muͯste n \newline
\end{minipage}
\end{table}
\newpage
\begin{table}[ht]
\begin{minipage}[t]{0.5\linewidth}
\small
\begin{center}*G
\end{center}
\begin{tabular}{rl}
 & sint, sô si \textbf{des} armbrustes span\\ 
 & mit senwe \textbf{swanke} trîbet dan.\\ 
 & dâr \textbf{über} gienc ein brückenslac,\\ 
 & dâ manic hurt ûffe lac.\\ 
5 & \textbf{ez} vlôz aldâ reht in daz mer.\\ 
 & Peilrapeire \textbf{was} \textbf{wol} ze wer.\\ 
 & \textbf{nû} seht, wie kint ûf schocken varen,\\ 
 & die man \textbf{schockes} niht wil sparen:\\ 
 & sus vuor diu brücke âne seil.\\ 
10 & \textbf{si}\textbf{ne} was vor jugende niht sô geil.\\ 
 & dort anderhalben stuonden\\ 
 & mit helmen ûf gebunden\\ 
 & sehze\textit{c} rîter oder mêr.\\ 
 & \textbf{die} riefen alle: "kêrâ kêr!"\\ 
15 & mit ûf gewo\textit{r}fenen swerten\\ 
 & die kranken strîtes gerten.\\ 
 & \textbf{durch daz sin} dicke sâhen ê,\\ 
 & si wânden, ez wære Clamide,\\ 
 & wan er sô \textbf{künsticlîche} reit\\ 
20 & gein der brücke ûf dem velde breit.\\ 
 & dô si disen jungen man\\ 
 & \textbf{al}sus mit schalle riefen an,\\ 
 & swie erz ors mit sporen versneit,\\ 
 & durch vorht ez doch die brücke meit.\\ 
25 & \textbf{\begin{large}D\end{large}er} rehte zageheit ie \textbf{vlôch},\\ 
 & der erbeizte nider unde zôch\\ 
 & sîn ors \textbf{ûf} der brücken swanc.\\ 
 & eines zagen muot wære al ze kranc,\\ 
 & solter gein solhem strîte varen.\\ 
30 & dar zuo muoser ein dinc bewaren,\\ 
\end{tabular}
\scriptsize
\line(1,0){75} \newline
G I O L M Q R Z Fr40 \newline
\line(1,0){75} \newline
\textbf{17} \textit{Initiale} I O L Q R Fr40  \textbf{25} \textit{Initiale} G  \newline
\line(1,0){75} \newline
\textbf{1} sô si des] des des O so si das M so sis Q R (Fr40)  $\cdot$ armbrustes] armbrust M (Q) armbrosters R \textbf{2} senwe] senuen I (L) (M) (Z) (Fr40) senwes O einem Q  $\cdot$ swanke] swanken Z  $\cdot$ trîbet] triben L tribe Q  $\cdot$ dan] an O kan L san R \textbf{3} dâr] Das M  $\cdot$ brückenslac] [bru*hen]: bruchen slach G brukken [swa]: slac I bruͯche slach L (Q) (R) \textbf{4} lac] gelach O \textbf{5} ez] Ein M  $\cdot$ reht] \textit{om.} I O Q R Fr40  $\cdot$ mer] rechtte mer R \textbf{6} Peilrapeire] pelrapeire G (L) (Q) (R) pailrapier I Pelrapeir O Pelrapire M Peltapeire Z pelr:peire Fr40  $\cdot$ was wol] was was M stunt wol Z  $\cdot$ ze] alczu M \textbf{7} seht] schecht R  $\cdot$ kint] die kint L  $\cdot$ schocken] kochen O sch:en Fr40 \textbf{8} schockes] shokkens I (M) (Q) (R) (Z) [sch*]: schokes  O sch:hens Fr40 \textbf{9} sus] Vff Q  $\cdot$ brücke] bruckin M \textbf{10} sine] dun I Div O (L) Dy en M (Z) Deme Q Dú ye R  $\cdot$ vor] von I O  $\cdot$ sô] zu Q (R)  $\cdot$ geil] [gel]: geil G \textbf{11} \textit{Versfolge 181.12-11} L   $\cdot$ stuonden] sy stuͦnden R \textbf{12} helmen] helm O (L) Fr40 \textbf{13} sehzec] sehzeh G Wol sechczig M \textbf{15} geworfenen] gewofenen G \textbf{16} kranken] cranches I (L) (Q) (R) kranche O \textbf{17} durch daz] ÷vrch daz O Wande M  $\cdot$ sin] si O (L) sein Q  $\cdot$ ê] \textit{om.} Q \textbf{18} ez] er R  $\cdot$ Clamide] chlamide I \textbf{19} sô] \textit{om.} I O  $\cdot$ künsticlîche] chvnichlichen O (L) (M) (Q) (R) (Z) (Fr40) \textbf{20} gein] Vor M  $\cdot$ brücke] bruckin M \textbf{21} dô] Da Z \textbf{22} alsus] sus I (O) (L) (Q) (R) (Z) (Fr40) So M \textbf{23} swie] Wie L (M) (Q) R  $\cdot$ erz] vil er daz I L (M) sere erz O (R) (Z) ser er daz Fr40  $\cdot$ versneit] sneit L M \textbf{24} \textit{nach 181.24:} \sout{der erbeizte nider vnde zoch / der [nîe]: îe zageheit gefloch} G   $\cdot$ Dy brucke osz dach durch vorchte meit M  $\cdot$ durch vorht] vor vorhte I  $\cdot$ brücke] bruchen I \textbf{25} Der] Den O L M Q R Z (Fr40)  $\cdot$ rehte] rehtev O Fr40  $\cdot$ ie] y R \textbf{26} erbeizte] erbeicze R erbeizzet Z  $\cdot$ nider] wider Q (Fr40) \textbf{27} sîn] Das M  $\cdot$ ûf der] vber der I ubir dy M er úff der Q (Fr40)  $\cdot$ brücken] brvke O (L) (Q) (R)  $\cdot$ swanc] wanc Fr40 \textbf{28} al ze] zuͯ L (R) also Q \textbf{29} solhem] soͯmlichen R \textbf{30} muoser] muͤster I mvͦse O \newline
\end{minipage}
\hspace{0.5cm}
\begin{minipage}[t]{0.5\linewidth}
\small
\begin{center}*T
\end{center}
\begin{tabular}{rl}
 & sint, sô si armbrustes span\\ 
 & mit senwe \textbf{swanke} trîbet dan.\\ 
 & dâr \textbf{über} gienc ein brückenslac,\\ 
 & dâ manec hurt ûffe lac.\\ 
5 & \textbf{daz} vlôz aldâ rehte in daz mer.\\ 
 & Peilrapere \textbf{stuont} \textbf{wol} ze wer.\\ 
 & seht, wie kint ûf schocken varn,\\ 
 & die man \textbf{schockens} niht wil sparn:\\ 
 & sus vuor diu brücke âne seil.\\ 
10 & \textbf{diu} was vor jugende niht sô geil.\\ 
 & Dort anderhalben stuonden\\ 
 & mit helmen ûf gebunden\\ 
 & \textbf{wol} sehzic rîter oder mêr.\\ 
 & \textbf{si} riefen alle: "kêrâ kêr!"\\ 
15 & mit ûf geworfen swerten\\ 
 & die kranken strîtes gerten.\\ 
 & \textbf{durch daz si} dicke sâhen ê,\\ 
 & si wânden, ez wære Clamide,\\ 
 & wander sô \textbf{küneclîche} reit\\ 
20 & gegen der brücke ûf dem velde breit.\\ 
 & \begin{large}D\end{large}ô si disen jungen man\\ 
 & \textbf{al}sus mit schalle riefen an,\\ 
 & swie \textbf{vil} erz ors mit sporn versneit,\\ 
 & durch vorhte ez doch die brücke meit.\\ 
25 & \textbf{den} reht\textit{iu} zageheit ie \textbf{vlôch},\\ 
 & der erbeizete nider unde zôch\\ 
 & sîn ors \textbf{über} der brücken swanc.\\ 
 & eines zagen muot wære al ze kranc,\\ 
 & solter gegen solhem strîte varn.\\ 
30 & dar zuo muose er ein dinc bewarn,\\ 
\end{tabular}
\scriptsize
\line(1,0){75} \newline
T U V W \newline
\line(1,0){75} \newline
\textbf{11} \textit{Majuskel} T  \textbf{13} \textit{Majuskel} T  \textbf{17} \textit{Initiale} W  \textbf{21} \textit{Initiale} T U V  \newline
\line(1,0){75} \newline
\textbf{1} sint] sein W \textbf{2} senwe swanke] seneweln swenken U senwen [*]: swanke V sennen schwancke W \textbf{3} brückenslac] bruͦnken slac U \textbf{5} vlôz aldâ rehte] volz al do rechte U vloz reht al V floß aldo W \textbf{6} Peilrapere] Belrepere V Pelrapier W \textbf{7} kint] die kint U \textbf{8} schockens] schoͤken V schoches W \textbf{10} was vor] in was vor U \textit{om.} W  $\cdot$ sô] \textit{om.} U [*]: so V zuͦ W \textbf{11} Do sach er an den selben stunden W \textbf{12} mit] Ritter mit W \textbf{13} wol sehzic rîter] Anderthalb sechtzig W \textbf{14} si] Die U V W  $\cdot$ kêrâ] kera her W \textbf{17} \textit{Versfolge 181.18-17} V   $\cdot$ durch daz si] [*]: Den sv́ vil V  $\cdot$ dicke sâhen] in dicke sehen W \textbf{18} ez] [*]: ez V  $\cdot$ Clamide] Clamidê T Clamede U klanide W \textbf{19} wander] Wan der U  $\cdot$ sô küneclîche] [*]: so kv́niclichen V \textbf{20} brücke] brucken U W \textbf{21} Do dise den sahen úber den plan W \textbf{22} riefen] ritten sy W \textbf{23} swie] Wie W  $\cdot$ versneit] versnite U vesneit W \textbf{24} ez] er W  $\cdot$ brücke] bruken W \textbf{25} den] Die U  $\cdot$ rehtiu] rehte T \textbf{27} über] auff W \textbf{28} al] do W \textbf{30} muose] mvese T muͦz U (W) \newline
\end{minipage}
\end{table}
\end{document}
