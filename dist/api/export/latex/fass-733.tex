\documentclass[8pt,a4paper,notitlepage]{article}
\usepackage{fullpage}
\usepackage{ulem}
\usepackage{xltxtra}
\usepackage{datetime}
\renewcommand{\dateseparator}{.}
\dmyyyydate
\usepackage{fancyhdr}
\usepackage{ifthen}
\pagestyle{fancy}
\fancyhf{}
\renewcommand{\headrulewidth}{0pt}
\fancyfoot[L]{\ifthenelse{\value{page}=1}{\today, \currenttime{} Uhr}{}}
\begin{document}
\begin{table}[ht]
\begin{minipage}[t]{0.5\linewidth}
\small
\begin{center}*D
\end{center}
\begin{tabular}{rl}
\textbf{733} & \textbf{\begin{large}E\end{large}r dâhte}: "sît ich mangel hân,\\ 
 & daz den sældehaften undertân\\ 
 & ist, ich meine die minne,\\ 
 & diu \textbf{maneges trûrigen} sinne\\ 
5 & mit vreuden helfe \textbf{ergeilet},\\ 
 & sît ich \textbf{des bin} verteilet,\\ 
 & ich enruoche, waz mir geschiht.\\ 
 & got wil mîner vreude niht.\\ 
 & Diu mich twinget minnen gir,\\ 
10 & stüende unser minne, mîn und ir,\\ 
 & daz scheiden dar zuo hôrte,\\ 
 & sô daz uns zwîvel stôrte,\\ 
 & ich m\textit{ö}hte wol zanderer minne komen.\\ 
 & nû hât ir minne mir benomen\\ 
15 & \textbf{ander minne und vreudebæren} trôst.\\ 
 & ich bin trûrens unerlôst.\\ 
 & Gelücke müeze vreude wern\\ 
 & die endehafter vreude gern.\\ 
 & got gebe vreude al disen scharn,\\ 
20 & ich wil ûz disen vreuden varn."\\ 
 & Er greif dâ sîn harnasch lac,\\ 
 & des er dicke al eine pflac,\\ 
 & daz er sich balde wâpende drîn.\\ 
 & \textbf{nû wil er} werben \textbf{niwen} pîn.\\ 
25 & Dô der vreuden vlühtec man\\ 
 & \textbf{het al sîn harnasch} an,\\ 
 & er sateltz ors mit sîner hant.\\ 
 & schilt unt \textbf{sper} bereit er vant.\\ 
 & man hôrte sîne reise \textbf{s}morgens klagen.\\ 
30 & dô er dannen schiet, dô begundez tagen.\\ 
\end{tabular}
\scriptsize
\line(1,0){75} \newline
D \newline
\line(1,0){75} \newline
\textbf{1} \textit{Initiale} D  \textbf{9} \textit{Majuskel} D  \textbf{17} \textit{Majuskel} D  \textbf{21} \textit{Majuskel} D  \textbf{25} \textit{Majuskel} D  \newline
\line(1,0){75} \newline
\textbf{13} möhte] mohte D \newline
\end{minipage}
\hspace{0.5cm}
\begin{minipage}[t]{0.5\linewidth}
\small
\begin{center}*m
\end{center}
\begin{tabular}{rl}
 & \textbf{er dâht}: "sît ich mangel hân,\\ 
 & daz den s\textit{æ}ldehaften undertân\\ 
 & ist, ich meine die minne,\\ 
 & diu \textbf{trûriges mannes} sinne\\ 
5 & mit vröuden helf \textbf{ergeilet},\\ 
 & sît ich \textbf{des bin} verteilet,\\ 
 & ich enruoch \textbf{nû}, waz mir geschiht.\\ 
 & got wil mîner vröude niht.\\ 
 & diu mich twinget minnen gir,\\ 
10 & stüende unser minne, mîn und ir,\\ 
 & daz scheiden dar zuo hôrte,\\ 
 & sô daz uns zwîvel stôrte,\\ 
 & ich m\textit{ö}hte wol zuo ander minne komen.\\ 
 & nû hât ir minne mir benomen\\ 
15 & \textbf{ander minne und vröudebæren} trôst.\\ 
 & ich bin t\textit{r}û\textit{r}ens \textbf{noch} unerlôst.\\ 
 & glücke müez vröude weren\\ 
 & die endehafter vröude geren.\\ 
 & got gebe vröude al disen scharn,\\ 
20 & ich wil ûz disen vröuden varn."\\ 
 & er greif d\textit{â} sîn harnasch lac,\\ 
 & des er dicke aleine pflac,\\ 
 & daz er sich balde wâpent drîn:\\ 
 & "\textbf{nû wil ich} werben \textbf{niht wan} pîn."\\ 
25 & dô der vröude vlühtic man\\ 
 & \textbf{geleit al sîn harnasch} an,\\ 
 & er satelt daz ros mit sîner hant.\\ 
 & schilt und \textbf{sper} bereit er vant.\\ 
 & man hôrte sîn reise morgens klagen.\\ 
30 & dô er dannen schie\textit{t}, dô begunde \textit{ez} tagen.\\ 
\end{tabular}
\scriptsize
\line(1,0){75} \newline
m n o Fr69 \newline
\line(1,0){75} \newline
\newline
\line(1,0){75} \newline
\textbf{1} dâht] gedocht n \textbf{2} sældehaften] soldehaften m \textbf{4} trûriges] trurigens n  $\cdot$ sinne] suͯne o \textbf{9} minnen] mẏner o \textbf{13} möhte] mohtte m moch Fr69 \textbf{14} hât] habent n \textbf{16} trûrens] turnenens m  $\cdot$ noch] \textit{om.} n o \textbf{19} al] allen n \textbf{21} dâ] do m n o \textbf{24} wan] wenne n \textbf{25} vlühtic] flúchtige n flochete o \textbf{26} al sîn] allen sinen n \textbf{28} bereit er] er bereit n \textbf{29} sîn] myn o \textbf{30} schiet] schie m  $\cdot$ ez] \textit{om.} m \newline
\end{minipage}
\end{table}
\newpage
\begin{table}[ht]
\begin{minipage}[t]{0.5\linewidth}
\small
\begin{center}*G
\end{center}
\begin{tabular}{rl}
 & \textbf{\begin{large}D\end{large}ô dâhter}: "sît ich mangel hân,\\ 
 & daz den sældehaften undertân\\ 
 & ist, ich meine \textit{die} minne,\\ 
 & diu \textbf{maneges trûregen} sinne\\ 
5 & mit vröude helfe \textbf{geilt},\\ 
 & sît ich \textbf{der bin} verteilt,\\ 
 & ich enruoche \textbf{niht}, waz mir geschiht.\\ 
 & got wil mîner vröuden niht.\\ 
 & diu mich twinget minne gir,\\ 
10 & stüende unser minne, mîn unde ir,\\ 
 & daz scheiden dar zuo hôrte,\\ 
 & sô daz uns zwîvel stôrte,\\ 
 & ich m\textit{ö}hte wol ze andere minne komen.\\ 
 & nû hât ir minne mir benomen\\ 
15 & \textbf{ander minne unde aller vröuden} trôst.\\ 
 & ich bin trûrens unerlôst.\\ 
 & gelücke muoze vröude wern\\ 
 & die endehafter vröude gern.\\ 
 & got gebe vröude al disen scharn,\\ 
20 & ich wil ûz disen vröuden varn."\\ 
 & er greif \textbf{hin}, dâ sîn harnasch lac,\\ 
 & des er dicke aleine pflac,\\ 
 & daz er sich balde wâpent drîn.\\ 
 & \textbf{er wil nû} werben \textbf{niwen} pîn.\\ 
25 & dô der vröuden vlühtic man\\ 
 & \textbf{het al sîn harnasch} an,\\ 
 & er satelte \textit{daz} ors mit sîner hant.\\ 
 & schilte unde \textbf{sper} bereit er vant.\\ 
 & man hôrte sîn reise \textbf{des} morgens klagen.\\ 
30 & dô er danne schiet, dô begunde ez tagen.\\ 
\end{tabular}
\scriptsize
\line(1,0){75} \newline
G I L M Z Fr18 Fr24 \newline
\line(1,0){75} \newline
\textbf{1} \textit{Initiale} G I Z Fr18 Fr24  \textbf{21} \textit{Überschrift:} Disiu auentiuuer sait wie Parzifal mit sinem brvͦder ferafiz [strat]: strait I   $\cdot$ \textit{Initiale} I  \newline
\line(1,0){75} \newline
\textbf{1} Dô] Da M \textbf{2} den] dem L  $\cdot$ sældehaften] seldehafftir M  $\cdot$ undertân] wider tan::: Fr18 \textbf{3} die] \textit{om.} G \textbf{4} maneges trûregen] manige trvrige L \textbf{5} vröude] vroiden M (Z) (Fr18)  $\cdot$ geilt] er geileit L ergeilit M (Z) (Fr18) (Fr24) \textbf{6} der] \textit{om.} L M \textbf{7} niht] \textit{om.} I nv L (M) Z Fr18 Fr24 \textbf{8} vröuden] frovde L (Fr24) \textbf{9} minne] minnen I \textbf{10} Stvnde nv mýn mýnne vnd ir L  $\cdot$ unser] ir I nv vnser Fr24 \textbf{13} möhte] mohte G I (L) (M) Z (Fr18) (Fr24)  $\cdot$ wol] \textit{om.} Fr18 \textbf{15} aller] ander Fr18  $\cdot$ vröuden] vroide M \textbf{17} muoze] muz vns I musz M \textbf{18} die] daz chan I \textbf{21} hin] alhin I \textbf{22} er] er ê Fr18 \textbf{23} daz] Wasz M  $\cdot$ wâpent] wapinde M \textbf{24} werben] erwerben L  $\cdot$ niwen] \textit{om.} I \textbf{25} dô] Da M Z  $\cdot$ vröuden] frowen L \textbf{27} satelte daz] satlte G satelt daz I L (Z) (Fr18) satelt Fr24 \textbf{29} klagen] clage M \textbf{30} dô er] Da er M Z  $\cdot$ danne] dannan I (M) dan Fr24  $\cdot$ dô begunde ez] da bigondez M (Z) \newline
\end{minipage}
\hspace{0.5cm}
\begin{minipage}[t]{0.5\linewidth}
\small
\begin{center}*T
\end{center}
\begin{tabular}{rl}
 & \textbf{\begin{large}D\end{large}ô dâht er}: "sît ich mangel hân,\\ 
 & daz den sældehaften undertân\\ 
 & ist, ich meine die minne,\\ 
 & diu \textbf{maneges trûrigen} sinne\\ 
5 & mit vreuden helfe \textbf{ergeilet},\\ 
 & sît ich \textbf{bin der} verteilet,\\ 
 & ich enruoche \textbf{nû}, waz mir geschiht.\\ 
 & got wil mîner vreuden niht.\\ 
 & diu mich twinget \textbf{mit} minnen gir,\\ 
10 & stüende unser minne, mîn und ir,\\ 
 & daz scheiden, \textbf{daz} dar zuo hôrte,\\ 
 & sô daz uns zwîvel stôrte,\\ 
 & ich möhte wol zuo ander minne komen.\\ 
 & nû h\textit{â}t ir minne mir benomen\\ 
15 & \textbf{vreude und ander minne} trôst.\\ 
 & ich bin trûrens unerlôst,\\ 
 & glücke müeze \textbf{mich} vreude wern.\\ 
 & die endehafter vreuden gern,\\ 
 & got g\textit{e}b\textit{e} \textit{v}reude al disen scharn.\\ 
20 & ich wil ûz disen vreuden varn."\\ 
 & er greif \textbf{hin}, dâ sîn harnasch lac,\\ 
 & des er dicke aleine pflac,\\ 
 & daz er sich balde wâpente drîn.\\ 
 & \textbf{er wil nû} werben \textbf{niuwe} pîn.\\ 
25 & dô der vreuden vlühtic man\\ 
 & \textbf{a\textit{l} sîn harnasch het} an,\\ 
 & er satelte daz ors mit sîner hant.\\ 
 & schilt und \textbf{glêne} bereit er vant.\\ 
 & man hôrte sîn reise \textbf{des} morges klagen.\\ 
30 & dô er dannen schiet, dô begunde ez tagen.\\ 
\end{tabular}
\scriptsize
\line(1,0){75} \newline
U V W Q R \newline
\line(1,0){75} \newline
\textbf{1} \textit{Initiale} U W Q  \newline
\line(1,0){75} \newline
\textbf{1} mangel] manchel Q \textbf{4} maneges trûrigen] mannes [trurige*]: trurigez V \textbf{5} vreuden] [frou*]: froͤuden V bloͤder W froͯde R \textbf{6} bin der] der bin V (Q) \textbf{7} enruoche] ruge Q Ruͯchen R  $\cdot$ nû waz mir] nv swaz mir V was mir nun W was mir R \textbf{8} vreuden] froͯde R \textbf{9} mit] \textit{om.} V Q R  $\cdot$ minnen] minne R \textbf{10} mîn] nun Q mir R \textbf{11} daz dar] das do W do Q R \textbf{13} möhte] mochte U (Q) (R)  $\cdot$ wol] ich W \textbf{14} nû] [*]: Nv V So W  $\cdot$ hât] hant U het V habt W Q  $\cdot$ benomen] genomen R \textbf{15} minne] minnen V (R) \textbf{17} müeze] muͦß W  $\cdot$ mich] \textit{om.} W Q R  $\cdot$ vreude] froͯden R \textbf{18} endehafter] [endehaften]: endehafter V endehafften W  $\cdot$ vreuden] froͤde W (Q) froͯden wern vnd och R \textbf{19} gebe] gab ir V  $\cdot$ al disen scharn] [*]: al disen scharn V allen disen scharn W an [diser schar]: disen scharen Q alle disen scharn R \textbf{20} ûz] [ausz]: vnsz Q \textbf{21} greif] ginck Q  $\cdot$ dâ] do V W Q \textbf{23} wâpente] wapent W Q (R) \textbf{24} er] Jch R  $\cdot$ niuwe] nuwen V (Q) hochen R \textbf{25} vlühtic] fluchent R \textbf{26} al] Alle U Allen W Als Q R  $\cdot$ harnasch het an] [harnes*]: harnesch geleite an V harnach hat an R \textbf{27} satelte daz] satteltz V (R) sattelt W (Q) \textbf{28} glêne] sper V W Q R \textbf{29} morges] morgens Von mengem R \textbf{30} dannen] dann W  $\cdot$ schiet] [sch*]: scheid R  $\cdot$ ez] er R \newline
\end{minipage}
\end{table}
\end{document}
