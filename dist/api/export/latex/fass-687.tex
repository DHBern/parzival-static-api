\documentclass[8pt,a4paper,notitlepage]{article}
\usepackage{fullpage}
\usepackage{ulem}
\usepackage{xltxtra}
\usepackage{datetime}
\renewcommand{\dateseparator}{.}
\dmyyyydate
\usepackage{fancyhdr}
\usepackage{ifthen}
\pagestyle{fancy}
\fancyhf{}
\renewcommand{\headrulewidth}{0pt}
\fancyfoot[L]{\ifthenelse{\value{page}=1}{\today, \currenttime{} Uhr}{}}
\begin{document}
\begin{table}[ht]
\begin{minipage}[t]{0.5\linewidth}
\small
\begin{center}*D
\end{center}
\begin{tabular}{rl}
\textbf{687} & \begin{large}M\end{large}an truog \textbf{im} zimierde dar\\ 
 & von \textbf{tiwerre} koste \textbf{alsô} gevar,\\ 
 & \textbf{swem} \textbf{diu minne ie} des betwanc,\\ 
 & daz er nâch \textbf{wîbe} lône ranc,\\ 
5 & ez wære Gahmuret oder Galoes\\ 
 & oder der künec \textbf{Kyllicrates},\\ 
 & der decheiner dorfte sînen lîp\\ 
 & \textbf{nie} baz \textbf{gezieren} durch diu wîp.\\ 
 & von Ipopotiticon\\ 
10 & \textbf{oder ûz} der wîten Acraton\\ 
 & \textbf{oder} von Kalomidente\\ 
 & \textbf{oder} von Agatyrsjente\\ 
 & wart nie bezzer pfelle \textbf{brâht},\\ 
 & denne \textbf{dâ} \textbf{zer} zimierde \textbf{wart} \textbf{erdâht}.\\ 
15 & Dô \textbf{kuste}r daz vingerlîn,\\ 
 & daz Itonje, diu junge künegîn,\\ 
 & im durch minne sande.\\ 
 & ir triwe \textbf{er} \textbf{sô} bekande,\\ 
 & \textbf{swâ im kumbers wære} bevilt,\\ 
20 & dâ was ir minne \textbf{vür} ein schilt.\\ 
 & Der künec \textbf{was gewâpent} \textbf{nuo}.\\ 
 & zwelf \textbf{vrouwen} \textbf{griffen zuo}\\ 
 & ûf \textbf{starken} runzîden,\\ 
 & \textbf{si}ne \textbf{solten} \textbf{daz} niht \textbf{mîden},\\ 
25 & diu clâre geselleschaft,\\ 
 & ieslîchiu hete an einen schaft\\ 
 & \textbf{den} tiwern pfelle genomen,\\ 
 & dar unde der künec wolde komen;\\ 
 & den vuorten si durch schate dan\\ 
30 & ob dem strît gernden man.\\ 
\end{tabular}
\scriptsize
\line(1,0){75} \newline
D \newline
\line(1,0){75} \newline
\textbf{1} \textit{Initiale} D  \textbf{15} \textit{Majuskel} D  \textbf{21} \textit{Majuskel} D  \newline
\line(1,0){75} \newline
\textbf{5} Galoes] Galôes D \textbf{6} Kyllicrates] Kyllicratês D \textbf{9} Ipopotiticon] Jpoptiticon D \textbf{11} Kalomidente] Kaloytvdênte D \textbf{12} Agatyrsjente] Agatyrsîente D \textbf{16} Itonje] Jtonîe D \newline
\end{minipage}
\hspace{0.5cm}
\begin{minipage}[t]{0.5\linewidth}
\small
\begin{center}*m
\end{center}
\begin{tabular}{rl}
 & man truoc \textbf{i\textit{m}} z\textit{imier}de dar\\ 
 & von \textbf{diser} koste \textbf{alsô} gevar,\\ 
 & \dag wan\dag  \textbf{diu minne ie} des betwanc,\\ 
 & daz er nâch \textbf{wîbe} lône ranc,\\ 
5 & ez wære Gahmuret oder Galoes\\ 
 & oder der künic \textbf{Killicrates},\\ 
 & der dekeiner dorft sînen lîp\\ 
 & \textbf{nie} baz \textbf{gezieren} durch diu wîp.\\ 
 & von Ipo\textit{po}ti\textit{ti}c\textit{o}n\\ 
10 & \textbf{oder ûz} der wîten Acrat\textit{o}n\\ 
 & \textbf{oder} von Kalomidente\\ 
 & \textbf{oder} von Ag\textit{a}thirs\textit{i}en\textit{te}\\ 
 & wart nie bezzer  \textbf{verbrâht},\\ 
 & dan \textbf{sîn} z\textit{i}m\textit{ie}rde \textbf{was} \textbf{erdâht}.\\ 
15 & dô \textbf{kust} er daz vingerlîn,\\ 
 & daz Ithonie, diu junge künigîn,\\ 
 & im durch minne sante.\\ 
 & ir triuwe \textbf{in} \textbf{sô} bekante,\\ 
 & \textbf{wâ in kumbers wær} bevilt,\\ 
20 & d\textit{â} was ir minne \textbf{vür} ein schilt.\\ 
 & \textbf{\begin{large}N\end{large}û daz} der künic \textbf{gewâfent wart},\\ 
 & zwelf \textbf{juncvrowen} \textbf{an der vart}\\ 
 & \textbf{bescheiden wâr\textit{en}} ûf \textbf{schœnen} runzîden.\\ 
 & \textbf{dô} en\textbf{solte} niht \textbf{vermîden}\\ 
25 & diu clâre geselleschaft,\\ 
 & ieglîchiu het an einen schaft\\ 
 & \textbf{den} tiuren pfelle genomen,\\ 
 & dar unde der künic wolte komen;\\ 
 & den vuorten si durch schate dan\\ 
30 & ob dem strît gernden man.\\ 
\end{tabular}
\scriptsize
\line(1,0){75} \newline
m n o Fr69 \newline
\line(1,0){75} \newline
\textbf{1} \textit{Initiale} Fr69  \textbf{21} \textit{Initiale} m   $\cdot$ \textit{Capitulumzeichen} n  \newline
\line(1,0){75} \newline
\textbf{1} im] in m n o  $\cdot$ zimierde] zwurnde m zirmúrde o  $\cdot$ dar] [dan]: dar n \textbf{2} von] Vor o  $\cdot$ diser] tv́ren Fr69 \textbf{5} Gahmuret] gamúret n \textbf{6} Killicrates] killikrattes m \textbf{7} dekeiner] do keiner n  $\cdot$ dorft] bedorfft n dúrff o \textbf{9} Ipopotiticon] ipotikan m patiticon n ipaticon o \textbf{10} Acraton] akratan m acratun o \textbf{11} Kalomidente] kallomident m kalomident n o \textbf{12} Agathirsiente] agethirsen m agathirsient n agatúrsient o \textbf{13} verbrâht] volbracht n (o) \textbf{14} zimierde] zmurde m wúrde n \textbf{17} im] Jn o vnt Fr69 \textbf{20} dâ] Do m n o \textbf{22} vart] art n \textbf{23} wâren] wart m  $\cdot$ runzîden] rinziden n (o) \textbf{24} dô] Die o \textbf{26} einen] einen einen o \textbf{30} gernden] gerende n \newline
\end{minipage}
\end{table}
\newpage
\begin{table}[ht]
\begin{minipage}[t]{0.5\linewidth}
\small
\begin{center}*G
\end{center}
\begin{tabular}{rl}
 & \begin{large}M\end{large}an truoc \textbf{im} zimier dar\\ 
 & von \textbf{rîcher} kost \textbf{unde sô} gevar,\\ 
 & \textbf{swen} \textbf{ie diu minne} des betwanc,\\ 
 & daz er nâch \textbf{wîbe} lône ranc,\\ 
5 & ez wære Gahmuret oder Galoes\\ 
 & oder der künic \textbf{Galicrates},\\ 
 & der deheiner dorfte sînen lîp\\ 
 & baz \textbf{gezieren} durch diu wîp.\\ 
 & von Ipopotiticon\\ 
10 & \textbf{noch von} der wîten Acraton\\ 
 & \textbf{noch} von Kalimodente\\ 
 & \textbf{noch} von Accratirsiente\\ 
 & wart nie bezzer pfelle \textbf{brâht},\\ 
 & danne \textbf{dâ} \textbf{zer} zimiere \textbf{wart} \textbf{erdâht}.\\ 
15 & dô \textbf{kêrte} er daz vingerlîn,\\ 
 & daz Itonie, diu junge künigîn,\\ 
 & im durch minne sande.\\ 
 & ir triwe \textbf{er} \textbf{wol} \textit{be}kande:\\ 
 & \textbf{het iener kumbers in} bevilt,\\ 
20 & dâ \textbf{engegene} was ir minne ein schilt.\\ 
 & der künic \textbf{was gewâpent} \textbf{nuo}.\\ 
 & zwelf \textbf{juncvrouwen} \textbf{griffen zuo}\\ 
 & ûf \textbf{schœnen} runzîden,\\ 
 & \textbf{die}ne \textbf{solden} \textbf{d\textit{es}} niht \textbf{mîden},\\ 
25 & diu clâre geselleschaft,\\ 
 & ieslîchiu het an einen schaft\\ 
 & \textbf{einen} tiuren pfelle genomen,\\ 
 & dar under der künic wolde komen;\\ 
 & den vuorten si durch schate dan\\ 
30 & obe dem strît gernden man.\\ 
\end{tabular}
\scriptsize
\line(1,0){75} \newline
G I L M Z Fr18 Fr20 Fr52 \newline
\line(1,0){75} \newline
\textbf{1} \textit{Initiale} G L Fr18 Fr20  \textbf{15} \textit{Initiale} I  \newline
\line(1,0){75} \newline
\textbf{1} Man] ÷an Fr20 \textbf{2} rîcher] richaite I  $\cdot$ sô gevar] wolgeuar I \textbf{3} swen] Wen L (M) \textbf{4} wîbe] wibes M (Fr18) \textbf{5} wære] ware ware L  $\cdot$ Gahmuret] gahmv̂ret G Gahmvret L (Fr18) Gamuͯret M gamuret Z  $\cdot$ oder] olde G (Fr20) vnd I Z \textbf{6} Galicrates] Galigrades I Gallicrates M kylicrates Z Galẏ crates Fr18 \textbf{8} gezieren] gezimiern I zieren L gezieret sin Fr18 \textbf{9} Ipopotiticon] lipiponticon I ypopo Titicon L Jpipotiticon Z Ipopotẏticon Fr18 jpopotticon Fr20 \textbf{10} der] \textit{om.} I  $\cdot$ Acraton] agreton I atraton Z :::craton Fr52 \textbf{11} Kalimodente] kalimogente I kalimodante M kalẏmodente Fr18 \textbf{12} Accratirsiente] Aocratirsente I Arcratir siente L acratirsigente M attratirsiente Z Accratẏrsiente Fr18 [acc*]: accratirsiente Fr20 \textbf{13} brâht] brah I \textbf{14} dâ zer] zuͤ dem I dez zuͯ der L zcu der M  $\cdot$ wart] waz L  $\cdot$ erdâht] gedah I gedaht L (M) Z Fr18 (Fr52) \textbf{15} dô] Da M Z  $\cdot$ kêrte er] chert er I (Fr18) kvst er L (Z) \textbf{16} Itonie] Jtonîe G ytonie I Jthonie M Jconie Z Jtonẏe Fr18  $\cdot$ junge] \textit{om.} L \textbf{17} sande] sanden Fr18 \textbf{18} wol] so L M Z Fr18  $\cdot$ bekande] erkande G :::te Fr52 \textbf{19} iener] der I irgen M \textbf{20} dâ] dan L  $\cdot$ ein] \textit{om.} I :::ine Fr52 \textbf{24} diene] die I (L)  $\cdot$ des] daz G ::: Fr52 \textbf{26} an] \textit{om.} Z \textbf{27} einen] Den L (M) Z Fr18 \textbf{30} gernden] gerndem I (L) \newline
\end{minipage}
\hspace{0.5cm}
\begin{minipage}[t]{0.5\linewidth}
\small
\begin{center}*T
\end{center}
\begin{tabular}{rl}
 & man truoc \textbf{sîn} zimierde dar\\ 
 & von \textbf{rîcher} kost \textbf{und sô} gevar,\\ 
 & \textbf{wen} \textbf{ie diu minne} des betwanc,\\ 
 & daz er nâch \textbf{wîbes} lône ranc,\\ 
5 & ez wære Gahmuret oder Galoes\\ 
 & oder der künec \textbf{Galicrates},\\ 
 & der dekeiner dorfte sînen lîp\\ 
 & baz \textbf{zimieren} durch diu wîp.\\ 
 & von Hypipotiticon\\ 
10 & \textbf{oder ûz} der wîten Acraton\\ 
 & \textbf{oder} von Galomidente\\ 
 & \textbf{oder} von Agatirsente\\ 
 & wart nie bezzer pfelle \textbf{brâht},\\ 
 & dan \textbf{des} \textbf{zuo der} zimierde \textbf{wart} \textbf{gedâht}.\\ 
15 & dô \textbf{kuste}r daz vingerlîn,\\ 
 & daz Itonie, diu junge künegîn,\\ 
 & im durch minne sante.\\ 
 & ir triuwe \textbf{er} \textbf{sô} bekante:\\ 
 & \textbf{hæte iener kumber in} bevilt,\\ 
20 & dâ \textbf{engein} was ir minne ein schilt.\\ 
 & \begin{large}D\end{large}er künec \textbf{was gewâpent} \textbf{nuo}.\\ 
 & zwelf \textbf{juncvrouwen} \textbf{griffen zuo}\\ 
 & ûf \textbf{schœnen} runzîden,\\ 
 & \textbf{die} en\textbf{solten} \textbf{des} niht \textbf{mîden},\\ 
25 & diu clâre geselleschaft,\\ 
 & ieglîchiu het an einen schaft\\ 
 & \textbf{den} tiuren pfelle genomen,\\ 
 & dar under der künec wolte komen;\\ 
 & den vuorten si durch schaten \textit{d}an\\ 
30 & ob dem strît gernden man.\\ 
\end{tabular}
\scriptsize
\line(1,0){75} \newline
U V W Q R \newline
\line(1,0){75} \newline
\textbf{1} \textit{Initiale} Q R  \textbf{21} \textit{Initiale} U  \newline
\line(1,0){75} \newline
\textbf{1} sîn] im sine V die W im Q ir R  $\cdot$ zimierde] [*]: zimierde V \textbf{3} wen] Swen V \textbf{5} Gahmuret] Gahmuͦret U gamvret V gamuret W gamúret Q  $\cdot$ Galoes] Galoez V \textbf{6} Galicrates] [*licrates]: kylicrates V kalicrates W (R) \textbf{8} zimieren] zuniren Q \textbf{9} Hypipotiticon] [*]: Jpotikon V hyppipoticion W hippipoticiton Q hippipotiticon R \textbf{11} Galomidente] kalcomidente W kalomidende Q kalomidente R \textbf{12} Agatirsente] agatirsiente V W Q agatrisiente R \textbf{14} des zuo der] [d*]: die V des W der zer Q  $\cdot$ wart] [w*]: was V  $\cdot$ gedâht] erdaht V (W) (Q) (R) \textbf{15} kuster] kvst [*]: er V kust er Q R \textbf{16} Itonie] Jtonie U R [ẏ*onie]: ẏtonie V ytonie W Q  $\cdot$ junge] \textit{om.} W \textbf{18} er] \textit{om.} Q \textbf{19} iener kumber in] in kúmmers ye Q \textbf{20} dâ] Das R  $\cdot$ engein] gen Q \textbf{24} ensolten] wolten W soltten R \textbf{26} einen] einē V Q einem W R \textbf{29} schaten] [schat*]: schate V schatte W R  $\cdot$ dan] dran U [*]: dan V \textbf{30} gernden] gernde Q (R) \newline
\end{minipage}
\end{table}
\end{document}
