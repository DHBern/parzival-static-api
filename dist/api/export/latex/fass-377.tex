\documentclass[8pt,a4paper,notitlepage]{article}
\usepackage{fullpage}
\usepackage{ulem}
\usepackage{xltxtra}
\usepackage{datetime}
\renewcommand{\dateseparator}{.}
\dmyyyydate
\usepackage{fancyhdr}
\usepackage{ifthen}
\pagestyle{fancy}
\fancyhf{}
\renewcommand{\headrulewidth}{0pt}
\fancyfoot[L]{\ifthenelse{\value{page}=1}{\today, \currenttime{} Uhr}{}}
\begin{document}
\begin{table}[ht]
\begin{minipage}[t]{0.5\linewidth}
\small
\begin{center}*D
\end{center}
\begin{tabular}{rl}
\textbf{377} & \begin{large}S\end{large}waz hers anderhalp der \textbf{brücken} lac,\\ 
 & daz zogete über, ê kœm der tac,\\ 
 & ze Bearosche in die stat,\\ 
 & \textbf{als si} Lyppaut, der vürste, bat.\\ 
5 & \textbf{dô} wâren die von \textbf{Jamor}\\ 
 & geriten über die brücken vor.\\ 
 & man bevalch ieslîche porten sô,\\ 
 & daz si werlîche dô\\ 
 & \textbf{stuonden}, \textbf{dô} der tac erschein.\\ 
10 & Scherules, der kôs im ein,\\ 
 & die er unt \textbf{mîn} \textbf{hêr} Gawan\\ 
 & niht unbehuot wolden lân.\\ 
 & Man hôrt dâ von den gesten\\ 
 & - ich wæne, \textbf{daz} wâren die besten -,\\ 
15 & \textbf{die} klageten, daz dâ \textbf{wære} geschehen\\ 
 & ritterschaft gar ân ir sehen\\ 
 & unt daz diu vesperîe ergienc,\\ 
 & daz ir deheiner tjost \textbf{dâ} enpfienc.\\ 
 & diu klage was gar ân nôt,\\ 
20 & ungezalt man\textbf{s} in dâ bôt,\\ 
 & allen \textbf{den}, dies \textbf{geruochent}\\ 
 & unt \textbf{es} \textbf{ûz} ze velde \textbf{suochent}.\\ 
 & in den gazzen kôs man grôze slâ.\\ 
 & ouch sach man \textbf{her} unt dâ\\ 
25 & manege baniere \textbf{zogen} în,\\ 
 & allez bî \textbf{des mânen schîn},\\ 
 & unt manegen helm von rîcher kost\\ 
 & - \textbf{man} wolde si vüeren gein der tjost -\\ 
 & unt manec sper wol gemâl.\\ 
30 & ein Regenspurger zindâl\\ 
\end{tabular}
\scriptsize
\line(1,0){75} \newline
D \newline
\line(1,0){75} \newline
\textbf{1} \textit{Initiale} D  \textbf{13} \textit{Majuskel} D  \newline
\line(1,0){75} \newline
\textbf{3} Bearosche] Bearôsce D \textbf{4} Lyppaut] Lyppaot D \textbf{10} Scherules] Scervles D \newline
\end{minipage}
\hspace{0.5cm}
\begin{minipage}[t]{0.5\linewidth}
\small
\begin{center}*m
\end{center}
\begin{tabular}{rl}
 & waz hers anderhalp der \textbf{brücke} lac,\\ 
 & daz zougete \textbf{ouch} über, ê kœme der tac,\\ 
 & ze Bearosche in die stat,\\ 
 & \textbf{als si} Lippout, der vürste, bat.\\ 
5 & \textbf{dô} wâren die von \textbf{Jamor}\\ 
 & g\textit{e}riten über di\textit{e} \textit{b}rücken \textit{v}o\textit{r}.\\ 
 & \textit{man bevalch iegelîche porten} sô,\\ 
 & daz si werlîchen dô\\ 
 & \textbf{stuonden}, \textbf{dô} der tac erschein.\\ 
10 & Scherules, der kôs im ein,\\ 
 & die er und Gawan\\ 
 & niht unbehuot wolten lân.\\ 
 & man hôrte d\textit{â} von den gesten\\ 
 & - ich wæne, \textbf{daz} wâren die besten -,\\ 
15 & \textbf{die} klageten, daz d\textit{â} \textbf{was} geschehen\\ 
 & ritterschaft gar âne i\textit{r} sehen\\ 
 & und daz diu vesperîe ergienc,\\ 
 & daz ir dekeiner just \textbf{dâ} entvienc.\\ 
 & diu klage was gar âne nôt,\\ 
20 & ungezalt man\textbf{s} in d\textit{â} bôt,\\ 
 & allen \textbf{den}, die es \textbf{dô} \textbf{geruochten}\\ 
 & und \textbf{es} \textbf{ûz} ze velde \textbf{suochten}.\\ 
 & in d\textit{en} gazzen kôs man grôze slâ.\\ 
 & ouch sach man \textbf{hie} und dâ\\ 
25 & manige baniere \textbf{zogen} în,\\ 
 & allez bî \textbf{dem mâneschîn},\\ 
 & und manigen helm von rîcher \textit{k}ost\\ 
 & - \textbf{man} wolte si vüeren gegen der jost -\\ 
 & und manic sper wol gemâl.\\ 
30 & ein Regensburger zindâl\\ 
\end{tabular}
\scriptsize
\line(1,0){75} \newline
m n o \newline
\line(1,0){75} \newline
\newline
\line(1,0){75} \newline
\textbf{1} brücke] brucken n (o) \textbf{2} daz] Do n  $\cdot$ zougete] zeigete n zogite o  $\cdot$ kœme] kam n o \textbf{3} Bearosche] bearosce m [ber*]: bearosc n bearosc o \textbf{4} Lippout] lippaot n o \textbf{5} dô] Da o  $\cdot$ Jamor] jamer o \textbf{6} \textit{Verse 377.6-7 (mit Anteil aus 377.3) kontrahiert zu:} Getritten v̂ber die stat brucken so m   $\cdot$ brücken] brúcke n o \textbf{7} bevalch] befalsch o  $\cdot$ iegelîche] ẏegelichen n iglicher o \textbf{8} werlîchen] verlichen o \textbf{10} Scherules] Scerules m Sterules n Sterles o  $\cdot$ ein] eÿns o \textbf{12} lân] [han]: lan o \textbf{13} dâ] do m n o \textbf{14} daz wâren] es weren n o \textbf{15} dâ] do m n o \textbf{16} ir] ire m \textbf{17} ergienc] ergeg o \textbf{18} dekeiner] do keiner n  $\cdot$ dâ] \textit{om.} n o \textbf{19} klage] clagte die o \textbf{20} in] >in< o  $\cdot$ dâ] do m n o \textbf{21} den] \textit{om.} n \textbf{23} den] das m \textbf{25} manige] Manigen o  $\cdot$ zogen] zeigen n zoigen o \textbf{27} von] vor o  $\cdot$ kost] trost m \textbf{28} wolte] walt o \textbf{29} gemâl] [gefar]: gefal n \textbf{30} Regensburger] regens burger m n o  $\cdot$ zindâl] [zal]: zindal o \newline
\end{minipage}
\end{table}
\newpage
\begin{table}[ht]
\begin{minipage}[t]{0.5\linewidth}
\small
\begin{center}*G
\end{center}
\begin{tabular}{rl}
 & swaz hers anderhalp der \textbf{brücke} lac,\\ 
 & daz zogete \textbf{ouch} über, ê kœme der tac,\\ 
 & ze Bearotsche in die stat.\\ 
 & Libaut, der vürste, \textbf{si des} bat.\\ 
5 & \textbf{ouch} wâren die von \textbf{Amor}\\ 
 & geriten über die brücke vor.\\ 
 & \begin{large}M\end{large}an bevalch ieslîche porte sô,\\ 
 & daz si werlîche dô\\ 
 & \textbf{stuont}, \textbf{als} der tac erschein.\\ 
10 & Tscherules, der kôs im ein,\\ 
 & die er unde \textbf{hêr} Gawan\\ 
 & niht unbehuot wolten lân.\\ 
 & man hôrte dâ von den gesten\\ 
 & - ich wæne, \textbf{daz} wâren die besten -,\\ 
15 & \textbf{si} klagten, daz dâ \textbf{was} geschehen\\ 
 & rîterschaft gar âne ir sehen\\ 
 & unt daz d\textit{iu} vesperîe ergienc,\\ 
 & daz ir deheiner tjost enpfienc.\\ 
 & diu klage was gar âne nôt,\\ 
20 & \textbf{wan} ungezalt man\textbf{s} in dâ bôt,\\ 
 & allen, dies \textbf{geruochten}\\ 
 & unde \textbf{si} ze velde \textbf{suochten}.\\ 
 & in den gazzen kôs man grôze slâ.\\ 
 & ouch sach man \textbf{her} unde dâ\\ 
25 & manige banier \textbf{trecken} în,\\ 
 & allez bî \textbf{des mânen schîn},\\ 
 & unde manigen helm von rîcher kost\\ 
 & - \textbf{man} wolt si vüeren gein der tjost -\\ 
 & unt manic sper wol gemâl.\\ 
30 & ein Regenspurgære zendâl\\ 
\end{tabular}
\scriptsize
\line(1,0){75} \newline
G I O L M Q R Z Fr21 Fr38 \newline
\line(1,0){75} \newline
\textbf{1} \textit{Initiale} O L M R Z  \textbf{3} \textit{Initiale} I  \textbf{7} \textit{Initiale} G  \textbf{15} \textit{Initiale} I  \newline
\line(1,0){75} \newline
\textbf{1} \textit{Die Verse 370.13-412.12 fehlen} Q   $\cdot$ swaz] ÷waz O Waz L (M) (R)  $\cdot$ anderhalp] ennet R  $\cdot$ brücke] pruccen I (M) (Z) bvrch O \textbf{2} zogete] zoget I (L) (R) (Z) zcoych M  $\cdot$ ouch] ê O \textit{om.} M  $\cdot$ kœme] chom I (R) \textbf{3} ze Bearotsche] zebearotsche G Zebeatrotsce I Ze Bearots O Zuͯ Bearotsch L Ze Bearose R Ze Bearoshe Fr38 \textbf{4} Libaut] Lẏbavt O Lybavt L Z Fr38 Libait M Libant R \textbf{5} Amor] Jamor O R Z Fr38 lamor L iamor M \textbf{6} brücke] prukken I (M)  $\cdot$ vor] [enbor]: for G \textbf{7} ieslîche] iclicheme M (R)  $\cdot$ porte] porten I O (L) (M) Z  $\cdot$ sô] da M sa R \textbf{8} si] \textit{om.} I  $\cdot$ dô] da M R \textbf{9} stuont] Stuͦndent R \textbf{10} Tscherules] Scrules I Tschervles O Tshervles L Scerules M Scherules R Tschervl::: Fr38  $\cdot$ der kôs] erchos I der erkos R \textbf{11} hêr Gawan] ergawan M \textbf{13} hôrte] erhort I (L) (M) (R) (Z) enh::: Fr38  $\cdot$ dâ] do R \textbf{14} daz] da R  $\cdot$ wâren] wern I (O) (R) \textbf{15} si] Die Fr38  $\cdot$ daz dâ] do das R \textbf{17} diu] da G  $\cdot$ vesperîe] vesper M \textbf{18} ir deheiner] dehainer I er icheiner M ir deheiner dehein R ir Z  $\cdot$ enpfienc] da enpfienc Z \textbf{19} âne] ir M \textbf{20} wan] \textit{om.} Fr21  $\cdot$ mans in] man [inz]: ins O man yns M (Fr21)  $\cdot$ dâ] do R \textbf{21} allen] allen den I (O) (L) (Z) (Fr21) (Fr38) Alle den R  $\cdot$ dies] die sin I des R \textbf{22} unde si] vnde si vz O (L) (M) (Z) (Fr21) Das sys vsz R Vnd sie ::: Fr38 \textbf{23} grôze] grozen O  $\cdot$ slâ] scla R \textbf{24} sach] kos R  $\cdot$ her] hie R \textbf{25} trecken] treche O strechen L schreken R \textbf{26} des mânen] dem mane O (M) (R) (Fr21) dem manen Z \textbf{27} von] mit L \textbf{28} der] \textit{om.} I \textbf{29} gemâl] gamal R \textbf{30} Regenspurgære] regenbvrger I [Regenpvrgær]: Regenspvrgær O regens buͯrgare L regensburger M bogen spurger R Regenspurger Z regenspvrgær Fr21 ::: Fr38 \newline
\end{minipage}
\hspace{0.5cm}
\begin{minipage}[t]{0.5\linewidth}
\small
\begin{center}*T
\end{center}
\begin{tabular}{rl}
 & Swaz hers anderhalp der \textbf{bürge} lac,\\ 
 & daz zogete \textbf{ouch} über, ê kœme der tac,\\ 
 & ze Bearosche in die stat.\\ 
 & Lybaut, der vürste, \textbf{si des} bat.\\ 
5 & \textbf{Ouch} wâren die von \textbf{Jammor}\\ 
 & geriten über die brücke vor.\\ 
 & man bevalch iegelîche porten sô,\\ 
 & daz si werlîche dô\\ 
 & \textbf{stuont}, \textbf{alse} der tac erschein.\\ 
10 & Tscherules, der kôs im ein,\\ 
 & dier unde \textbf{hêr} Gawan\\ 
 & niht u\textit{n}behuot wolten lân.\\ 
 & man hôrte dâ von den gesten\\ 
 & - ich wæne, \textbf{dâ} wâren die besten -,\\ 
15 & \textbf{si} klageten, daz dâ \textbf{was} geschehen\\ 
 & rîterschaft gar âne ir sehen\\ 
 & unde daz diu vesperîe ergienc,\\ 
 & daz ir deheiner tjost enpfienc.\\ 
 & Diu klage was gar âne nôt,\\ 
20 & \textbf{wand} ungezalt man\textbf{z} in dâ bôt\\ 
 & \textbf{unde} allen, dies \textbf{geruochten}\\ 
 & unde \textbf{ez} ze velde \textbf{suochten}.\\ 
 & In den gazzen kôs man grôze slâ.\\ 
 & ouch sach man \textbf{her} unde dâ\\ 
25 & manege baniere \textbf{trecken} în,\\ 
 & allez bî \textbf{dem mânen schîn},\\ 
 & unde manegen helm von rîcher kost\\ 
 & - \textbf{er} wolte si \textit{v}üeren gegen der tjost -\\ 
 & unde manec sper wol gemâl.\\ 
30 & ein Regespurgære zindâl\\ 
\end{tabular}
\scriptsize
\line(1,0){75} \newline
T V W \newline
\line(1,0){75} \newline
\textbf{1} \textit{Majuskel} T  \textbf{5} \textit{Majuskel} T  \textbf{19} \textit{Majuskel} T  \textbf{23} \textit{Majuskel} T  \newline
\line(1,0){75} \newline
\textbf{1} Swaz] Was W  $\cdot$ bürge] [*]: brucke V brucken W \textbf{2} kœme] keine W \textbf{3} Bearosche] bearotsche V betroische W \textbf{4} [*]: alse sv́ lippaot der fúrste bat V  $\cdot$ Lybaut] Lybout W \textbf{5} Jammor] JaminoR T Jamor V iamor W \textbf{9} stuont alse] [S*]: Stunden da V \textbf{10} Tscherules] Schervles V Scherules W  $\cdot$ der kôs] erkos W \textbf{12} unbehuot] vbehvͦt T  $\cdot$ wolten] wolte V \textbf{13} hôrte] erhorte W  $\cdot$ dâ] do V W \textbf{14} dâ] daz V (W) \textbf{15} si] [*]: Die V  $\cdot$ dâ] do V W \textbf{18} enpfienc] [*]: da enphieng V \textbf{20} manz in] man ins W  $\cdot$ dâ] do V W \textbf{21} unde] \textit{om.} V W  $\cdot$ geruochten] da geruͦchten V \textbf{22} ez ze] [*]: ez vf ze V sy es aus zuͦ W \textbf{24} her] hie V \textbf{25} trecken] zogen V \textbf{26} mânen] mane V manes W \textbf{28} er] [*]: Man V Man W  $\cdot$ vüeren] :üeren T \textbf{29} sper] \textit{om.} W \textbf{30} Regespurgære] regensbvrger V regenspurgere W \newline
\end{minipage}
\end{table}
\end{document}
