\documentclass[8pt,a4paper,notitlepage]{article}
\usepackage{fullpage}
\usepackage{ulem}
\usepackage{xltxtra}
\usepackage{datetime}
\renewcommand{\dateseparator}{.}
\dmyyyydate
\usepackage{fancyhdr}
\usepackage{ifthen}
\pagestyle{fancy}
\fancyhf{}
\renewcommand{\headrulewidth}{0pt}
\fancyfoot[L]{\ifthenelse{\value{page}=1}{\today, \currenttime{} Uhr}{}}
\begin{document}
\begin{table}[ht]
\begin{minipage}[t]{0.5\linewidth}
\small
\begin{center}*D
\end{center}
\begin{tabular}{rl}
\textbf{708} & \begin{large}A\end{large}rtus ze Parzivale sprach:\\ 
 & "neve, sît dir sus geschach,\\ 
 & daz dû des kampfes bæte\\ 
 & unt manlîche tæte\\ 
5 & unt Gawan dirz versagete,\\ 
 & daz dîn munt \textbf{dô} \textbf{sêre} klagete,\\ 
 & nû hâstû den kampf \textbf{iedoch} gestriten\\ 
 & \textbf{gein} im, der sîn dâ het erbiten,\\ 
 & ez wære \textbf{uns} leit oder liep.\\ 
10 & \textbf{dû sliche von uns} als ein diep.\\ 
 & wir heten anders dîne hant\\ 
 & dises kampfes wol erwant.\\ 
 & nû darf Gawan \textbf{des} zürnen niht,\\ 
 & swaz man dir \textbf{drumbe prîses} giht."\\ 
15 & Gawan sprach: "mir ist niht leit\\ 
 & mînes neven hôhiu werdecheit.\\ 
 & mir ist \textbf{dennoch morgen} al ze vruo,\\ 
 & sol ich kampfes grîfen zuo.\\ 
 & wolte michs der künec erlâzen,\\ 
20 & \textbf{des jæhe} ich im \textbf{gein} mâzen."\\ 
 & Daz her \textbf{reit în} mit maneger schar.\\ 
 & man sach dâ vrouwen wol gevar\\ 
 & unt manegen zimierten man,\\ 
 & daz nie dechein her mêr gewan\\ 
25 & \textbf{solher} zimierde wunder.\\ 
 & die von der tavelrunder\\ 
 & unt diu messenîe der herzogîn,\\ 
 & ir wâpenröcke gâben schîn\\ 
 & \textbf{mit} pfelle von \textbf{Cynidunte}\\ 
30 & und brâht von \textbf{Pelpiunte};\\ 
\end{tabular}
\scriptsize
\line(1,0){75} \newline
D Fr66 \newline
\line(1,0){75} \newline
\textbf{1} \textit{Initiale} D  \textbf{21} \textit{Majuskel} D  \newline
\line(1,0){75} \newline
\textbf{1} Parzivale] Parcifale D \textbf{29} Cynidunte] :::nt Fr66 \textbf{30} Pelpiunte] :::elpivnt Fr66 \newline
\end{minipage}
\hspace{0.5cm}
\begin{minipage}[t]{0.5\linewidth}
\small
\begin{center}*m
\end{center}
\begin{tabular}{rl}
 & \begin{large}A\end{large}rtus zuo Parcifal sprach:\\ 
 & "neve, sît dir sus geschach,\\ 
 & daz dû des kampfes bæte\\ 
 & und manlîche tæte\\ 
5 & und Gawan dirz versagte,\\ 
 & \textit{daz dîn munt \textbf{sô} \textbf{sêre} klagte},\\ 
 & nû hâstû den kampf \textbf{doch} gestriten\\ 
 & \textbf{gegen} im, der sîn d\textit{â} het erbiten,\\ 
 & ez wær \textbf{uns} leit oder liep.\\ 
10 & \textbf{dû sliche von uns} als ein diep.\\ 
 & wir heten anders dîne hant\\ 
 & dises kampfes wo\textit{l} er\textit{w}ant.\\ 
 & nû darf Gawan \textbf{des} z\textit{ü}rn\textit{en} niht,\\ 
 & waz man dir \textbf{dar umbe prîses} giht."\\ 
15 & Gawan sprach: "mir ist \dag noch\dag  leit\\ 
 & mînes neven hôhiu wirdicheit.\\ 
 & mir ist \textbf{dannoch morgen} alzuo vruo,\\ 
 & sol ich kampf\textit{es} grîfe\textit{n} zuo.\\ 
 & wolt michs der künic erlâzen,\\ 
20 & \textbf{daz gihe} ich im \textbf{gegen} mâzen."\\ 
 & daz her \textbf{reit în} mit maniger schar.\\ 
 & man sach d\textit{â} vrowen wol gevar\\ 
 & und manigen gezimierten man,\\ 
 & daz nie kein her mê gewan\\ 
25 & \textbf{solich} zimierde wunder.\\ 
 & die von der tavelrunder\\ 
 & und di\textit{u} massenîe der herzogîn,\\ 
 & ir wâpenröcke gâben schîn\\ 
 & \textbf{mit} pfelle von \textbf{Zunidu\textit{n}te}\\ 
30 & und brâhte von \textbf{Pelp\textit{i}unte};\\ 
\end{tabular}
\scriptsize
\line(1,0){75} \newline
m n o Fr69 \newline
\line(1,0){75} \newline
\textbf{1} \textit{Initiale} m   $\cdot$ \textit{Capitulumzeichen} n  \newline
\line(1,0){75} \newline
\textbf{1} Artus] Artuͯs o \textbf{6} \textit{Vers 708.6 fehlt} m  \textbf{8} dâ] do m n o \textbf{10} sliche] slich: o \textbf{12} wol] wo m  $\cdot$ erwant] erkant m o \textbf{13} Gawan] gawanen n  $\cdot$ des] dasz o  $\cdot$ zürnen] zorns m \textbf{15} noch] nuͯ n o \textbf{18} sol ich] Sollich m (o) Solliches n  $\cdot$ kampfes grîfen] kampf griffens m [kamff]: kampff griffes o \textbf{19} künic] [tag]: konig o \textbf{20} im] >im< o \textbf{21} în] ein m \textbf{22} dâ] do m n o \textbf{25} solich] Sollicher n \textbf{27} diu] dis m o \textbf{29} Zunidunte] zúnidúnitte m zundunte n ziemete o \textbf{30} Pelpiunte] pelptuntte m \newline
\end{minipage}
\end{table}
\newpage
\begin{table}[ht]
\begin{minipage}[t]{0.5\linewidth}
\small
\begin{center}*G
\end{center}
\begin{tabular}{rl}
 & \begin{large}A\end{large}rtus ze Parcivale sprach:\\ 
 & "neve, sît dir sus geschach,\\ 
 & daz dû des kampfes bæte\\ 
 & unde manlîche tæte\\ 
5 & unde Gawan dirz versagte,\\ 
 & daz dîn munt \textbf{dô} \textbf{sêre} klagte,\\ 
 & \textit{nû} hâstû den kampf \textbf{iedoch} gestriten\\ 
 & \textbf{mit} im, der sîn dâ het erbiten,\\ 
 & ez wære \textbf{uns} leit ode liep.\\ 
10 & \textbf{dô ersliche dûn} als ei\textit{n} diep.\\ 
 & wir heten anders dîne hant\\ 
 & disses kampfes wol erwant.\\ 
 & nû\textbf{ne} darf Gawan \textbf{daz} zürne\textit{n} niht,\\ 
 & swaz man dir \textbf{brîses drumbe} giht."\\ 
15 & Gawan sprach: "mir\textbf{n} ist niht leit\\ 
 & mînes neven hôhiu werdecheit.\\ 
 & mir ist \textbf{dannoch morgen} al ze vruo,\\ 
 & sol ich kampfes grîfen zuo.\\ 
 & wolde michs der künic erlâzen,\\ 
20 & \textbf{des jæhe} ich im \textbf{ze} mâzen."\\ 
 & daz her \textbf{reit în} mit maniger schar.\\ 
 & man sach dâ vrouwen wolgevar\\ 
 & unde manigen gezimierten man,\\ 
 & daz nie dehein her mê gewan\\ 
25 & \textbf{sölher} zimierde wunder.\\ 
 & die von der tavelrunder\\ 
 & unde diu messen\textit{îe} der herzogîn,\\ 
 & ir wâpenröcke gâben schîn\\ 
 & \textbf{von} pfell\textit{e} \textit{v}on \textbf{Zididunt}\\ 
30 & unde brâht von \textbf{Pelimunt};\\ 
\end{tabular}
\scriptsize
\line(1,0){75} \newline
G I L M Z Fr18 \newline
\line(1,0){75} \newline
\textbf{1} \textit{Initiale} G I L Z Fr18  \textbf{19} \textit{Initiale} I  \newline
\line(1,0){75} \newline
\textbf{1} Parcivale] parzivale G parzifale I parzifal L M (Fr18) parcifaln Z \textbf{2} \textit{Verse 708.2-3 kontrahiert zu:} neue sit du des champhes bete I  \textbf{6} dô] so I M da Z \textbf{7} nû] unde G  $\cdot$ iedoch] doch M Fr18 \textbf{8} mit] Gein L (M) Z Fr18  $\cdot$ sîn] \textit{om.} I \textbf{9} ez wære] Ewær Fr18  $\cdot$ uns] im L (M) Fr18 \textbf{10} dô] [Dv]: Do L Dv Z Da M  $\cdot$ dûn] vnsen Z  $\cdot$ ein] einen G Z \textbf{11} dîne] diner I \textbf{12} disses kampfes] disen champh I \textbf{13} Gawan] Gauwan I  $\cdot$ daz] des M Z Fr18  $\cdot$ zürnen] zv̂rne G \textbf{14} swaz] Waz L (M) Z \textbf{15} mirn] mir I L \textbf{16} hôhiu] \textit{om.} I \textbf{17} dannoch morgen] morgen dannoch L M (Fr18) \textbf{19} michs] mich sin I \textbf{20} jæhe] sege M \textbf{21} daz] Des Fr18  $\cdot$ her] er L  $\cdot$ reit] \textit{om.} I  $\cdot$ în] hin L \textbf{23} gezimierten] geczirten M \textbf{27} messenîe] messen G \textbf{28} wâpenröcke] wapenrichen I \textbf{29} von] Mit L M Z Fr18  $\cdot$ pfelle von] phelle vnde von G  $\cdot$ Zididunt] zitzidunt I [*]: Cimdvnt L cinadunt M zvndunt Z cẏnadvnt Fr18 \textbf{30} Pelimunt] belimvnt L (M) pelpiunt Z pelẏpimvnt Fr18 \newline
\end{minipage}
\hspace{0.5cm}
\begin{minipage}[t]{0.5\linewidth}
\small
\begin{center}*T
\end{center}
\begin{tabular}{rl}
 & \begin{large}A\end{large}rtus zuo Parcifal sprach:\\ 
 & "neve, sît dir sus geschach,\\ 
 & daz dû des kampfes bæte\\ 
 & und manlîchen tæte\\ 
5 & und Gawan dir ez versagete,\\ 
 & daz dîn munt \textbf{dô} klagete,\\ 
 & nû hâst dû den kampf \textbf{dô} gestriten\\ 
 & \textbf{gein} im, der sîn d\textit{â} hete erbiten,\\ 
 & ez wære \textbf{im} leit oder liep.\\ 
10 & \textbf{doch ersliche dû in} als ein diep.\\ 
 & wir heten anders dîne hant\\ 
 & dis kampfes wol erwant.\\ 
 & nû \textbf{en}darf Gawan \textbf{des} zürnen niht,\\ 
 & waz man dir \textbf{dar umb prîses} giht."\\ 
15 & Gawan sprach: "mir ist niht leit\\ 
 & mînes neven hôhiu wirdecheit.\\ 
 & mir ist \textbf{morne dannoch} alzuo vruo,\\ 
 & sol ich kampfes grîfen zuo.\\ 
 & wolte mich es der künec erlâzen,\\ 
20 & \textbf{des jæhe} ich im \textbf{zuo} mâzen."\\ 
 & daz her \textbf{în reit} mit maneger schar.\\ 
 & man sach dâ vrouwen wol gevar\\ 
 & und manegen gezimierten man,\\ 
 & daz nie kein her mêr gewan\\ 
25 & \textbf{solich} zimierde wunder.\\ 
 & die von der tavelrunder\\ 
 & und diu massenîe der herzogîn,\\ 
 & ir wâpenröcke gâben schîn\\ 
 & \textbf{mit} pfelle von \textbf{Zinidunt}\\ 
30 & und brâhte von \textbf{Pelpunt};\\ 
\end{tabular}
\scriptsize
\line(1,0){75} \newline
U V W Q R \newline
\line(1,0){75} \newline
\textbf{1} \textit{Initiale} U V Q R  \newline
\line(1,0){75} \newline
\textbf{1} Parcifal] Parzifal U Parzefale V partzifalie W partzifale Q parczifaln R \textbf{2} sus] als Q \textbf{5} Gawan] Gawin R  $\cdot$ versagete] gesagte Q \textbf{6} dô] doch sere V do sere W so sere Q R \textbf{7} dô] [*]: doch V \textit{om.} W doch Q R \textbf{8} dâ] do U V W Q \textbf{9} im] [*]: vnz V  $\cdot$ leit oder liep] lieb oder leid R \textbf{10} [*]: Dv sliche von vnz als ein diep V  $\cdot$ Doch erschlichtu als sin leid R  $\cdot$ ein] einen W \textbf{12} erwant] ermant R \textbf{13} endarf] endarff daz W darff R  $\cdot$ Gawan] Gawin R  $\cdot$ des] \textit{om.} W \textbf{14} waz] Swaz V  $\cdot$ dar umb prîses] preises drumbe W (Q) (R) \textbf{15} Gawan] Gawin R \textbf{16} neven] neue Q  $\cdot$ hôhiu] hoche R \textbf{18} kampfes] kampfe R \textbf{19} wolte mich es] Wolcz mich R \textbf{21} her] er Q  $\cdot$ în reit] reit in V (W) (Q) (R) \textbf{22} dâ] do V W Q \textbf{23} gezimierten] zimiertten R \textbf{24} nie] \textit{om.} W  $\cdot$ mêr] nie W \textbf{25} solich] Sollicher W (Q) (R) \textbf{27} massenîe] massanie W \textbf{28} gâben] gab W \textbf{29} Zinidunt] ziniduͦnt U [*dvnt]: zendvnt V Cidinûnt Q \textbf{30} Pelpunt] Pelpuͦnt U pelpiunt V W (R) palptúnt Q \newline
\end{minipage}
\end{table}
\end{document}
