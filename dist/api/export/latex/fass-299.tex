\documentclass[8pt,a4paper,notitlepage]{article}
\usepackage{fullpage}
\usepackage{ulem}
\usepackage{xltxtra}
\usepackage{datetime}
\renewcommand{\dateseparator}{.}
\dmyyyydate
\usepackage{fancyhdr}
\usepackage{ifthen}
\pagestyle{fancy}
\fancyhf{}
\renewcommand{\headrulewidth}{0pt}
\fancyfoot[L]{\ifthenelse{\value{page}=1}{\today, \currenttime{} Uhr}{}}
\begin{document}
\begin{table}[ht]
\begin{minipage}[t]{0.5\linewidth}
\small
\begin{center}*D
\end{center}
\begin{tabular}{rl}
\textbf{299} & der \textbf{noch dort} ûze unvlühtec habt,\\ 
 & weder ern schûft noch \textbf{en}drabt.\\ 
 & \textbf{\begin{large}O\end{large}uch} \textbf{en}ist hie ninder vrouwen hâr\\ 
 & weder sô mürwe noch sô clâr,\\ 
5 & ez enwære \textbf{doch} ein veste bant\\ 
 & ze \textbf{wern} strîtes iwer hant.\\ 
 & Swelch man tuot solhe diemüete schîn,\\ 
 & der êret \textbf{ouch} die muoter sîn.\\ 
 & vaterhalben solt er ellen hân.\\ 
10 & kêrt muoterhalp, hêr Gawan,\\ 
 & sô werdet ir swertes blicke bleich\\ 
 & unt manlîcher herte weich."\\ 
 & Sus was der wol gelobte man\\ 
 & gerant ze\textbf{r blôzen} sîten an\\ 
15 & mit rede. \textbf{er kunde ir} gelten niht,\\ 
 & als wol gezogenem man geschiht,\\ 
 & dem \textbf{scham versliuzet} sînen munt,\\ 
 & daz dem verschamtem ist unkunt.\\ 
 & Gawan ze Keien sprach:\\ 
20 & "swâ man \textbf{sluog oder stach},\\ 
 & swaz des \textbf{gein mir ist} geschehen,\\ 
 & \textbf{swer} mîne varwe \textbf{wolde} spehen,\\ 
 & diu, wæne ich, ie \textbf{erbliche}\\ 
 & von \textbf{slage} \textbf{oder} von stiche.\\ 
25 & dû zürnest mit mir ân nôt.\\ 
 & ich bin, der dir ie dienst bôt."\\ 
 & Ûz\textbf{em} poulûn gienc hêr Gawan.\\ 
 & sîn ors hiez er bringen sân.\\ 
 & sunder swert unt âne sporn\\ 
30 & saz drûf der \textbf{degen} wol geborn.\\ 
\end{tabular}
\scriptsize
\line(1,0){75} \newline
D \newline
\line(1,0){75} \newline
\textbf{3} \textit{Initiale} D  \textbf{7} \textit{Majuskel} D  \textbf{13} \textit{Majuskel} D  \textbf{27} \textit{Majuskel} D  \newline
\line(1,0){75} \newline
\newline
\end{minipage}
\hspace{0.5cm}
\begin{minipage}[t]{0.5\linewidth}
\small
\begin{center}*m
\end{center}
\begin{tabular}{rl}
 & der \textbf{noch dort} ûze unvlühtic habt,\\ 
 & weder er enschûf\textit{t}et noch drabt.\\ 
 & \textbf{doch} ist hie niender vrouwen hâr\\ 
 & weder sô mürwe noch sô clâr,\\ 
5 & ez enwære \textbf{iedoch} \textit{e}in veste bant\\ 
 & ze \textbf{wern} strîte\textit{s} iuwer hant.\\ 
 & welh man tuot soliche diemuot schîn,\\ 
 & der êret \textbf{ouch} die muoter sîn.\\ 
 & vaterhalp solt er ellen hân.\\ 
10 & kêrt muoterhalp, hêr G\textit{a}wan,\\ 
 & sô werdet ir swertes blicke bleich\\ 
 & und manlîcher herte weich."\\ 
 & sus was der wolgelobete man\\ 
 & gerant ze \textbf{blôzer} sîten an.\\ 
15 & mit \textit{rede} \textbf{enkunde er} gelten niht,\\ 
 & als wolgezogenem man geschiht,\\ 
 & dem \textbf{scham versliuzet} sînen munt,\\ 
 & daz dem verschamten ist unkunt.\\ 
 & \begin{large}G\end{large}\textit{a}wan ze Keien sprach:\\ 
20 & "wâ man \textbf{sluoc oder stach},\\ 
 & waz des \textbf{gegen mir ist} geschehen,\\ 
 & \textbf{wer} mîne varwe \textbf{wolte} spehen,\\ 
 & diu, wæne ich, ie \textbf{erbliche}\\ 
 & von \textbf{slage} \textbf{oder} von stiche.\\ 
25 & dû zürnest mit mir âne nôt.\\ 
 & ich bin, der di\textit{r} ie dienest bôt."\\ 
 & ûz\textbf{er} pavelûn gienc hêr Gawa\textit{n}.\\ 
 & sîn ros hiez er \textbf{ime} bringen sâ\textit{n}.\\ 
 & sunder swert und âne sporn\\ 
30 & saz drûf der \textbf{degen} wolgeborn.\\ 
\end{tabular}
\scriptsize
\line(1,0){75} \newline
m n o \newline
\line(1,0){75} \newline
\textbf{19} \textit{Initiale} m n  \newline
\line(1,0){75} \newline
\textbf{2} enschûftet] entschuffet m \textbf{3} hie] \textit{om.} o  $\cdot$ niender] nẏergent n \textbf{4} mürwe] nuwe n o \textbf{5} enwære] were n (o)  $\cdot$ ein] min m \textbf{6} strîtes] stritten m  $\cdot$ iuwer] ir m n er o \textbf{9} hân] [sin]: han o \textbf{10} hêr] er n o  $\cdot$ Gawan] gewan m n o \textbf{12} herte] herter n \textbf{15} rede] \textit{om.} m  $\cdot$ enkunde] erkuͯnde o \textbf{16} wolgezogenem] wol gezogen o \textbf{19} Gawan] Gowan m Gewan o  $\cdot$ Keien] keẏen n o \textbf{21} waz] Wan o  $\cdot$ ist] \textit{om.} n  $\cdot$ geschehen] beschehen o \textbf{22} varwe] frouwe n \textbf{23} erbliche] rebbeliche m (n) (o)  $\cdot$ ie] \textit{om.} n \textbf{26} der] der der n  $\cdot$ dir] die m o \textbf{27} ûzer] Vs n o  $\cdot$ Gawan] gawant m gewan o \textbf{28} sân] sant m \textbf{30} saz] Das o  $\cdot$ wolgeborn] hoch geborn n \newline
\end{minipage}
\end{table}
\newpage
\begin{table}[ht]
\begin{minipage}[t]{0.5\linewidth}
\small
\begin{center}*G
\end{center}
\begin{tabular}{rl}
 & der \textbf{dort noch} \textbf{d\textit{â}} ûze unvlühtic habet,\\ 
 & weder er enschûftet noch \textbf{en}drabet.\\ 
 & \textbf{ouch} \textbf{en}ist hie ninder vrouwen hâr\\ 
 & weder sô mürwe noch sô clâr,\\ 
5 & ez enwære \textbf{iu} \textbf{doch} ein veste bant\\ 
 & ze \textbf{bewarne} strîtes iwer hant.\\ 
 & swelch man tuot solhe diemuot schîn,\\ 
 & der êret \textbf{iedoch} die muoter sîn.\\ 
 & vaterhalp solter ellen hân.\\ 
10 & kêrt muoterhalp, hêr Gawan,\\ 
 & sô werdet ir swertes blicke bleich\\ 
 & unde manlîcher herte weich."\\ 
 & sus was der wolgelobte man\\ 
 & gerant ze\textbf{r blôzen} sîten an.\\ 
15 & mit rede \textbf{er kunde} gelten niht,\\ 
 & als wolgezogenem man geschiht.\\ 
 & dem \textbf{versliuzet schame} sînen munt,\\ 
 & daz dem verschamten ist unkunt.\\ 
 & Gawan \textbf{iedoch} ze Kayn sprach:\\ 
20 & "swâ man \textbf{mich strîten ie gesach},\\ 
 & swaz des \textbf{ist von mir} geschehen,\\ 
 & \textbf{der} mîne varwe \textbf{kunde} spehen,\\ 
 & diu, wæne ich, ie \textbf{erbliche}\\ 
 & von \textbf{slegen} \textbf{noch} von stiche.\\ 
25 & dû zü\textit{r}nest mit mir âne nôt.\\ 
 & ich bin, der dir ie dienst bôt."\\ 
 & ûz \textbf{dem} pavelûn gie hêr Gawan.\\ 
 & sîn ors hiez er bringen sân.\\ 
 & sunder swert unde âne sporn\\ 
30 & saz drûf der \textbf{helt} wolgeborn.\\ 
\end{tabular}
\scriptsize
\line(1,0){75} \newline
G I O L M Q R Z \newline
\line(1,0){75} \newline
\textbf{1} \textit{Capitulumzeichen} L  \textbf{3} \textit{Initiale} Z  \textbf{9} \textit{Initiale} Q  \textbf{13} \textit{Initiale} L  \textbf{17} \textit{Initiale} I  \textbf{19} \textit{Capitulumzeichen} L  \textbf{27} \textit{Initiale} L Z  \newline
\line(1,0){75} \newline
\textbf{1} dort noch dâ ûze] dort noch devze G dor vze noch I dort noch O týost dort L noch dort M noch dort ausz Q (R) (Z) \textbf{2} weder] Vnd weder R  $\cdot$ er enschûftet] shuffet I stapffet R er enschevfte Z  $\cdot$ endrabet] trabt R \textbf{3} enist] ist I R Z  $\cdot$ hie ninder] ninder hie dehainer I ninder O nirgen M \textbf{4} weder sô] So weder R  $\cdot$ mürwe] núwe R  $\cdot$ sô clâr] zu clar Q \textbf{5} enwære] wer I (O) (L) (Q) (R) \textbf{6} Jwerm strît zewerr hant O  $\cdot$ bewarne] wern L (Q) R (Z) o\textit{m. } M \textbf{7} swelch] Welch L (M) Q (R)  $\cdot$ solhe] solher I sulchen M \textbf{8} êret] andert L  $\cdot$ iedoch] doch R \textbf{9} ellen] eren Q \textbf{10} kêrt] Ker R  $\cdot$ Gawan] gewan R \textbf{11} ir] ir von L  $\cdot$ swertes] schwerte R  $\cdot$ blicke] blickes M \textbf{13} was] wart I Z \textbf{15} rede] redin M  $\cdot$ er] ern Q  $\cdot$ gelten] im gelten L ir geldin M (Q) (Z) in gelten R \textbf{16} als] als noch I  $\cdot$ wolgezogenem] wol gezcogene M  $\cdot$ geschiht] nach geschiht Z \textbf{17} versliuzet] vor vluzet M \textbf{18} dem versampten ist vnchunt I  $\cdot$ verschamten] vnuersamten Z \textbf{19} Gawan] Gewan Q R  $\cdot$ Kayn] kain G I keyn O M Z kayen L kay Q keẏ R \textbf{20} swâ] swan I Wo L (M) Q (R)  $\cdot$ mich] \textit{om.} O L M Q R Z  $\cdot$ strîten ie] ie instrite I ie gestrîten O (Q) (R) ýe stritten L (M) ie slvc Z  $\cdot$ gesach] sach I O L M Q R oder stach Z \textbf{21} swaz] Waz L (M) (Q) Wes R  $\cdot$ ist von mir] ist >von< mir G von mir ist I Q R Z \textbf{22} der] swer I (Z)  $\cdot$ mîne] minne R  $\cdot$ varwe] fraw Q (R) \textbf{23} erbliche] der bliche I vorbliche M (R) \textbf{24} slegen] slege I (O) (M) (Q) (R)  $\cdot$ noch] oder O Z \textbf{25} zürnest] zunst G zu nest Q \textbf{26} der] der der R  $\cdot$ dienst] hilffe R \textbf{27} ûz dem] Zdem O  $\cdot$ hêr] \textit{om.} L  $\cdot$ Gawan] gewan Q R \textbf{28} er] >er ym< O er im Q (R) \newline
\end{minipage}
\hspace{0.5cm}
\begin{minipage}[t]{0.5\linewidth}
\small
\begin{center}*T
\end{center}
\begin{tabular}{rl}
 & der \textbf{noch dort} ûze unvl\textit{ü}ht\textit{ic} habt,\\ 
 & weder er enschûft noch \textbf{en}drabt.\\ 
 & \textbf{ouch} ist hie niender vrowen hâr\\ 
 & weder sô mürwe noch sô clâr,\\ 
5 & ez enwære \textbf{iedoch} ein veste bant\\ 
 & ze \textbf{wern} strîtes iuwer hant.\\ 
 & swelch man tuot solch diemuot schîn,\\ 
 & der êret \textbf{iedoch} die muoter sîn.\\ 
 & vaterhalp solter ellen hân.\\ 
10 & kêret muoterhalp, hêr Gawan,\\ 
 & sô werdet ir \textbf{von} swertes blicke bleich\\ 
 & unde manlîcher herte weich."\\ 
 & Sus was der wol gelobete man\\ 
 & gerant ze \textbf{blôzen} sîten an\\ 
15 & mit rede. \textbf{er kundir} gelten niht,\\ 
 & als wol gezogenen man geschiht.\\ 
 & dem \textbf{besliuzet \textit{schame}} sînen munt,\\ 
 & daz dem verschameten ist unkunt.\\ 
 & \begin{large}G\end{large}awan \textbf{dô} ze Key sprach:\\ 
20 & "swâ man \textbf{mich ie gestrîten sach},\\ 
 & swaz des \textbf{ist von mir} geschehen,\\ 
 & \textbf{der} mîne varwe \textbf{kunde} spehen,\\ 
 & di\textit{u}, wænich, ie \textbf{verbliche}\\ 
 & von \textbf{slage} \textbf{noch} von stiche.\\ 
25 & dû zürnest mit mir âne nôt.\\ 
 & ich bin, der dir ie dienst bôt."\\ 
 & Ûz \textbf{dem} pavelûn gie hêr Gawan.\\ 
 & sîn ors hiez er \textbf{im} bringen sân.\\ 
 & sunder swert unde âne sporn\\ 
30 & saz drûf der \textbf{helt} wol geborn.\\ 
\end{tabular}
\scriptsize
\line(1,0){75} \newline
T U V W \newline
\line(1,0){75} \newline
\textbf{13} \textit{Majuskel} T  \textbf{19} \textit{Initiale} T U V W  \textbf{27} \textit{Majuskel} T  \newline
\line(1,0){75} \newline
\textbf{1} ûze] auß W  $\cdot$ unvlühtic] vn vluht T \textbf{2} weder] Enweder W  $\cdot$ enschûft] in schuͦf U schúpffet W \textbf{3} niender] nider U \textbf{5} ez enwære iedoch] [E*]: Ez emwere v́ch doch V Es wer eúch doch W  $\cdot$ veste] vestes W \textbf{6} Wan ir sint doch suͦs gewant U  $\cdot$ ze wern] [*]: Ze werende V Gegen werden W  $\cdot$ strîtes] streit W  $\cdot$ iuwer] zuͦ diser W \textbf{7} swelch] Welch U W  $\cdot$ diemuot] muͦt U \textbf{8} iedoch] [*]: oͮuch V \textbf{9} solter] so solt er V \textbf{10} Gawan] gewan U \textbf{11} blicke] blicken W \textbf{12} unde] An W \textbf{14} sîten] [*]: siten V \textbf{15} kundir] kuͦnde U (V) (W) \textbf{16} wol gezogenen] wol [*]: gezogenem V wolgezognē W  $\cdot$ geschiht] beschith V \textbf{17} schame] \textit{om.} T \textbf{18} dem] den W \textbf{19} Gawan] GAban W  $\cdot$ Key] keẏn V \textbf{20} swâ] Wan U Wo W  $\cdot$ mich ie gestrîten sach] [*]: slvͦg oder stach V  $\cdot$ gestrîten] streiten W \textbf{21} swaz] Waz U (W)  $\cdot$ ist von mir] von mir ist W \textbf{22} kunde] [*]: koͤnde V \textbf{23} diu] die T  $\cdot$ ie] nie W  $\cdot$ verbliche] [*]: verbliche T V erbliche W \textbf{26} der dir ie] ie der dir W \textbf{27} pavelûn] gezelt V  $\cdot$ Gawan] gewan U \textbf{28} im] \textit{om.} W \newline
\end{minipage}
\end{table}
\end{document}
