\documentclass[8pt,a4paper,notitlepage]{article}
\usepackage{fullpage}
\usepackage{ulem}
\usepackage{xltxtra}
\usepackage{datetime}
\renewcommand{\dateseparator}{.}
\dmyyyydate
\usepackage{fancyhdr}
\usepackage{ifthen}
\pagestyle{fancy}
\fancyhf{}
\renewcommand{\headrulewidth}{0pt}
\fancyfoot[L]{\ifthenelse{\value{page}=1}{\today, \currenttime{} Uhr}{}}
\begin{document}
\begin{table}[ht]
\begin{minipage}[t]{0.5\linewidth}
\small
\begin{center}*D
\end{center}
\begin{tabular}{rl}
\textbf{314} & \begin{large}R\end{large}ûch was ir antlütze erkant.\\ 
 & einen geisel \textbf{si vuorte} in der hant.\\ 
 & \textbf{dem} \textbf{wâren die} swenkel sîdîn\\ 
 & unt der stil ein rubîn.\\ 
5 & gevar als \textbf{eines affen hût}\\ 
 & truoc hende \textbf{diz} gæbe \textbf{trût}.\\ 
 & \textbf{die} nagele wâren niht ze lieht,\\ 
 & \textbf{wan} mir diu âventiure giht,\\ 
 & si stüenden als \textbf{eines} lewen klân.\\ 
10 & nâch ir minne was selten tjost getân.\\ 
 & Sus kom geriten in den rinc\\ 
 & \textbf{trûrens} urhap, vreuden twinc.\\ 
 & si \textbf{kêrte}, \textbf{al} dâ si den \textbf{wirt} vant.\\ 
 & vrou Cunneware de Lalant\\ 
15 & az mit Artuse,\\ 
 & diu küneginne von \textbf{Janfuse}\\ 
 & mit vroun Ginoveren az.\\ 
 & \textbf{Artus, der künec}, schône saz.\\ 
 & Cundrie hielt \textbf{vür den} Bertenoys.\\ 
20 & si sprach hin zim \textbf{en} franzoys\\ 
 & - ob ichz \textbf{iu} tiuschen sagen sol,\\ 
 & \textbf{mir} \textbf{tuont} ir mære niht ze wol -:\\ 
 & "Fillu roy Utepandragun,\\ 
 & dich \textbf{selben} unt manegen Bertun\\ 
25 & hât \textbf{dîn} gewerp \textbf{al hie} geschant.\\ 
 & die besten über elliu lant\\ 
 & sæzen hie mit werdecheit,\\ 
 & wan daz ein galle \textbf{ir} prîs versneit.\\ 
 & Tavelrunde ist entnihtet,\\ 
30 & der valsch hât dran gepflihtet.\\ 
\end{tabular}
\scriptsize
\line(1,0){75} \newline
D \newline
\line(1,0){75} \newline
\textbf{1} \textit{Initiale} D  \textbf{11} \textit{Majuskel} D  \textbf{23} \textit{Majuskel} D  \textbf{29} \textit{Majuskel} D  \newline
\line(1,0){75} \newline
\textbf{4} rubîn] Rvbbin D \textbf{16} diu] de D \textbf{20} en franzoys] enfranzoẏs D \textbf{21} tiuschen] tîvscen D \textbf{23} Utepandragun] Vͦtepandragv̂n D \newline
\end{minipage}
\hspace{0.5cm}
\begin{minipage}[t]{0.5\linewidth}
\small
\begin{center}*m
\end{center}
\begin{tabular}{rl}
 & rûch was ir antlitze erkant.\\ 
 & eine\textit{n} geisel \textbf{vuorte si} in der hant.\\ 
 & \textbf{dem} \textbf{wâren die} swenkel sîdîn\\ 
 & und der s\textit{ti}l ein rubîn.\\ 
5 & ge\textit{v}ar als \textbf{ein affen hût}\\ 
 & truoc hende \textit{\textbf{diz}} \textit{gæbe} \textbf{trût}.\\ 
 & \textbf{die} nagele wâren niht ze lieht,\\ 
 & \textbf{wand} mir diu âventiure giht,\\ 
 & si stüenden als \textbf{ein} lewen kl\textit{â}n.\\ 
10 & nâch ir min\textit{n}e was selten just getân.\\ 
 & sus kam geriten in den rinc\\ 
 & \textbf{trûrens} urhap, vröuden tw\textit{i}nc.\\ 
 & si \textbf{kêrte}, \textbf{al}dâ si den \textbf{wirt} vant.\\ 
 & vrouwe C\textit{u}nn\textit{e}w\textit{a}re de Lalant\\ 
15 & az mit Artuse,\\ 
 & diu künigîn von \textbf{Janfuse}\\ 
 & mit vrouwen Ginoveren az.\\ 
 & \textbf{Artus, der künic}, schône saz.\\ 
 & Con\textit{d}r\textit{ie} hielt \textbf{vür den} Britunois.\\ 
20 & si sprach hin zuo im \textbf{\textit{i}n} franzois\\ 
 & - ob \textit{ichz} \textbf{iu} tiusc\textit{h}en sagen sol,\\ 
 & \textbf{sô} \textbf{tuont} ir mære niht ze wol -:\\ 
 & "fili rois Utrapandr\textit{ag}un,\\ 
 & dich \textbf{selben} und manigen Britun\\ 
25 & hât \textbf{dîn} gewer\textit{p} \textbf{alhie} geschant.\\ 
 & die besten über alliu lant,\\ 
 & \textbf{die} sæzen hie mit werdicheit,\\ 
 & wan daz ein galle \textbf{den} prîs vers\textit{n}eit.\\ 
 & tavelrunde ist en\textit{t}nihtet,\\ 
30 & \textit{der valsch hât dâr an gepflihtet.}\\ 
\end{tabular}
\scriptsize
\line(1,0){75} \newline
m n o \newline
\line(1,0){75} \newline
\newline
\line(1,0){75} \newline
\textbf{1} antlitze] anczlit o \textbf{2} einen] Eine m Ein n o  $\cdot$ vuorte] fuͯrt o \textbf{3} swenkel] [swencken]: swenckell m (o) \textbf{4} stil] sal m  $\cdot$ rubîn] robin n \textbf{5} gevar] Gewar m \textbf{6} diz gæbe] als ein m \textbf{8} wand] Do wenne n \textbf{9} ein] eins n  $\cdot$ lewen] beren n (o)  $\cdot$ klân] clagen m \textbf{10} minne] mine m  $\cdot$ getân] gran o \textbf{12} twinc] twang m \textbf{13} aldâ] do n o  $\cdot$ den] \textit{om.} o \textbf{14} Cunneware] konuwere m conneware n Conne waren o  $\cdot$ de Lalant] delalant n \textbf{15} az] Alles n  $\cdot$ Artuse] artuͯse o \textbf{16} künigîn] konigen o  $\cdot$ Janfuse] jamfuse m o jamfusen o \textbf{17} vrouwen] frouwe m (n)  $\cdot$ Ginoveren az] Ginouereas m ginoferen asz n (o) \textbf{19} Condrie hielt] Conder hielt m Conde hielt n Condehielt o  $\cdot$ Britunois] brittunois m britonis n britaneisz o \textbf{20} hin] \textit{om.} n o  $\cdot$ in] hin m \textit{om.} o  $\cdot$ franzois] franczois m frantzois n franczeis o \textbf{21} ichz iu tiuschen] uͯch tuͯschuczen m ichs úch tútsche n ichez uch tusche o \textbf{23} Utrapandragun] vttrapandrigevn m uter pandragun n usser pandragon o \textbf{24} manigen] manige n  $\cdot$ Britun] brittvn m britẏm o \textbf{25} gewerp] gewerpt m  $\cdot$ geschant] beschant o \textbf{28} den] iren n (o)  $\cdot$ versneit] versmeit m \textbf{29} entnihtet] [en*]: enihtet m entnichet o \textbf{30} \textit{Vers 314.30 fehlt} m  \newline
\end{minipage}
\end{table}
\newpage
\begin{table}[ht]
\begin{minipage}[t]{0.5\linewidth}
\small
\begin{center}*G
\end{center}
\begin{tabular}{rl}
 & rûch was ir antlütze erkant.\\ 
 & eine geiselen \textbf{vuorte si} in der hant.\\ 
 & \textbf{der} \textbf{was der} swenkel sîdîn\\ 
 & unt der stil ein rubîn.\\ 
5 & \begin{large}G\end{large}evar als \textbf{ei\textit{n} affen hût}\\ 
 & truoc hende \textbf{diz} gæbe \textbf{trût}.\\ 
 & \textbf{ir} nagele w\textit{â}ren niht ze lieht;\\ 
 & \textbf{als} mir diu âventiure giht,\\ 
 & si stüenden als \textbf{eines} lewen klân.\\ 
10 & nâch ir minne was selten tjost getân.\\ 
 & sus kom geriten \textit{i}n den rinc\\ 
 & \textbf{trûren} urhap, vröuden twinc.\\ 
 & si \textbf{kêrte}, dâ si den \textbf{wirt} vant.\\ 
 & vrou Kuneware de Lalant\\ 
15 & az mit Artuse,\\ 
 & diu künigîn von \textbf{Lanfuse}\\ 
 & mit vroun Schinoveren az.\\ 
 & \textbf{der künic Artus} schône saz.\\ 
 & Gundrie hielt \textbf{vür den} Britaneis.\\ 
20 & si sprach hin ze im \textbf{en} franzeis\\ 
 & - obe ich ez \textbf{iu} tiutschen sagen sol,\\ 
 & \textbf{mir} \textbf{tuot} ir mære niht ze wol -:\\ 
 & "filiroys Utpandragun,\\ 
 & dich unde manigen Britûn\\ 
25 & hât \textbf{ein} gewerp \textbf{gar} geschant.\\ 
 & die besten über elliu lant\\ 
 & s\textit{æ}zen hie mit werdicheit,\\ 
 & wan daz ein galle \textbf{ir} brîs versneit.\\ 
 & tavelrunder ist entnihtet,\\ 
30 & der valsch hât dranne gepflihtet.\\ 
\end{tabular}
\scriptsize
\line(1,0){75} \newline
G I O L M Q R Z Fr39 Fr64 \newline
\line(1,0){75} \newline
\textbf{5} \textit{Initiale} G O L  \textbf{19} \textit{Überschrift:} Heb an dem vorderm blat an so vindes [t*]: dv geschriben wie ein wildez wip qvam geriten vf kvnic artus hof die weil er saz vnd az Z   $\cdot$ \textit{Initiale} Q R Z Fr39  \textbf{23} \textit{Initiale} I  \textbf{27} \textit{Illustration mit Überschrift:} Wie parczifal geschuldiget ward R  \newline
\line(1,0){75} \newline
\textbf{1} rûch] Rucke Q  $\cdot$ ir] irem Q \textbf{2} eine geiselen] ein gaisel I (Z) Eine geisel O Eyne geilis M Einen geisel Q  $\cdot$ vuorte si] vurt si I (O) (Z) fuͯrtes L (Q) furste M  $\cdot$ in der] an ir I L Z an der O Q yn ir M \textbf{3} der was der] Dem warn die Z \textbf{4} unt der stil ein] der stil was ein I Vnde der stilin M  $\cdot$ rubîn] rubein Q \textbf{5} Gevar] ÷e var O  $\cdot$ ein] eines G  $\cdot$ affen hût] hauen huͤt I \textbf{7} wâren] wæren G warn crimp vnd I warn ir O (L) (M) (R) Z (Fr39) wurden ir Q  $\cdot$ ze] \textit{om.} I \textbf{8} mir] \textit{om.} L  $\cdot$ giht] seit M \textbf{9} stüenden] stoͮnden G (O) (M) (Q) (Z) (Fr39) stuͯdent R  $\cdot$ eines] ein O Z \textit{om.} L \textbf{10} nâch] vmb I  $\cdot$ tjost] trost M \textbf{11} in] an G \textbf{12} trûren] Trvrens O (L) (M) (Q) (R) (Z) (Fr39)  $\cdot$ urhap] erhuͤp I  $\cdot$ twinc] troinc Q \textbf{13} kêrte] chert I (O) (Z) fuͦrtte R  $\cdot$ dâ] do O Q Fr39 alda M Z \textbf{14} vrou] Frovn O (Z)  $\cdot$ Kuneware] kunware I M kvnawarn O Cvneware L Conware Q Kuͦnware R kvnewaren Z Cuͦnware Fr39  $\cdot$ de Lalant] [delant]: delalant G der lalant O labant R \textbf{15} Artuse] artus I artusze Q \textbf{16} Lanfuse] lanfus I Lamfvse O Lanfuͯse L lanfusze Q Jamfuse Z \textbf{17} vroun] vrou L (Q) (R)  $\cdot$ Schinoveren] Ginofern I kvnovern O Gýnovern L ginovern M (Z) Gynouern Q R Gynovern Fr39 \textbf{18} Artus] artusze Q \textbf{19} Gundrie] Kvndrie O Q Z Fr39 Kondrie M R  $\cdot$ Britaneis] britoneẏs G pritoneis I britoneis O Brittanoisz L briteneis Q britonis R britvneis Z britoneys Fr39 \textbf{20} hin] \textit{om.} I  $\cdot$ ze im] zu dem Q (R)  $\cdot$ en franzeis] enfranzoys G in fronzeis I franzeis O Frantzois L franczeis M frantzoisz Q francoẏs R in franzeis Z franzoys Fr39 \textbf{21} iu] in R Fr39  $\cdot$ tiutschen] [tuschen]: tutschen G Tuhshen I tevschen O tuͯtschen L dutsch M deutzin Q tútsch R devtsch Z tuͥtschen Fr39 \textbf{22} mir] Mich L  $\cdot$ tuot] tvnt O (M) (Z) (Fr39) \textbf{23} Utpandragun] [u]: vtrepandaGrun I vrpandragvn O Vterpandragun M vszpandragun Q vtpandagrun R \textbf{24} dich] eͮch I Jch O Dich selben Z  $\cdot$ manigen] mange O  $\cdot$ Britun] pritun I Brittvn L (Fr39) brittuͯn Q Birtun R \textbf{25} ein] din O L M (Q) R Z Fr39  $\cdot$ gar] \textit{om.} I al hie O (L) (M) (Q) (R) (Z) (Fr39)  $\cdot$ geschant] gesant I \textbf{27} sæzen] sazen G I (O) (L) (M) (Z) Setzen Q (R)  $\cdot$ hie] da I \textbf{28} ir] da ir I \textbf{29} tavelrunder] Tavelrvnde O L (Z) (Fr64)  $\cdot$ ist] sint I  $\cdot$ entnihtet] eyn amichtet M euchnichtet Q entwichet R \textbf{30} dranne] drin L da an Fr64 \newline
\end{minipage}
\hspace{0.5cm}
\begin{minipage}[t]{0.5\linewidth}
\small
\begin{center}*T
\end{center}
\begin{tabular}{rl}
 & rûch was ir antlitze erkant.\\ 
 & ein geisel \textbf{vuorte s}in der hant.\\ 
 & \textbf{der} \textbf{wâren die} swenkel sîdîn\\ 
 & unde der stil ein rubîn.\\ 
5 & gevar als \textbf{ein affen hût}\\ 
 & truoc hende \textbf{dis\textit{iu}} gæbe \textbf{brût}.\\ 
 & \textbf{ir} nagele wâren niht ze lieht,\\ 
 & \textbf{wand} mir diu âventiure giht,\\ 
 & si st\textit{üe}nden als \textbf{eines} lewen klân.\\ 
10 & nâch ir minne was selten tjost getân.\\ 
 & sus kom geriten in den rinc\\ 
 & \textbf{trûrens} urhap, vröuden twinc.\\ 
 & si \textbf{hielt}, \textbf{al} dâ si de\textit{n} \textbf{künec} vant.\\ 
 & Vrou Cunneware de Lalant\\ 
15 & az mit Artuse,\\ 
 & diu künegîn von \textbf{Janfuse}\\ 
 & mit vroun Gynovern az.\\ 
 & \textbf{Artus, der künec}, schône saz.\\ 
 & Kundrie hielt \textbf{vor dem} Brituneis.\\ 
20 & si sprach hin zim franzeis\\ 
 & - ob ichz \textbf{in} tiuschen sagen sol,\\ 
 & \textbf{mir} \textbf{tuont} ir mære niht ze wol -:\\ 
 & "Filly rois Utpandragun,\\ 
 & dich unde manec Britun\\ 
25 & hât \textbf{dîn} gewerp \textbf{alhie} geschant.\\ 
 & die besten über alliu lant\\ 
 & sæzen hie mit werdecheit,\\ 
 & wan daz ein galle \textbf{ir} prîs versneit.\\ 
 & tavelrunder ist entnihtet,\\ 
30 & der valsch hât dran gepflihtet.\\ 
\end{tabular}
\scriptsize
\line(1,0){75} \newline
T U V W \newline
\line(1,0){75} \newline
\textbf{1} \textit{Initiale} W  \textbf{14} \textit{Majuskel} T  \textbf{23} \textit{Majuskel} T  \newline
\line(1,0){75} \newline
\textbf{2} sin] sy an W \textbf{4} rubîn] ruͦbin U \textbf{5} \textit{nach 314.5:} Dise schoͤne obentraut W   $\cdot$ ein] eins V (W) \textbf{6} \textit{nach 314.6:} Als ein gerchtes hartz W   $\cdot$ Truͦg hende die warn schwartz W  $\cdot$ disiu] dise T  $\cdot$ brût] [*vt]: trvt V \textbf{7} niht ze lieht] zelieht niht V \textbf{8} wand] Von U \textbf{9} stüenden] stvͦnden T (U) (W) \textbf{10} was] ist W \textbf{12} urhap] an hap U  $\cdot$ twinc] schwing W \textbf{13} hielt] kerte V  $\cdot$ al dâ] do W  $\cdot$ den] der T \textbf{14} Vrou] Erawe W  $\cdot$ Cunneware] kvnnewar T (U) (W) kvneware V  $\cdot$ de Lalant] delalant V \textbf{16} Janfuse] ianfuse V W \textbf{17} vroun] vrow U (W)  $\cdot$ Gynovern] Genovern T U schinouern W \textbf{19} Kundrie] Kuͦndrie U  $\cdot$ dem] den V W  $\cdot$ Brituneis] Brituͦneiz U brittunois V britonis W \textbf{20} hin] \textit{om.} W  $\cdot$ franzeis] Franzeiz T Franzez U franzois V frantzois W \textbf{21} ichz] [*]: ich es v́ch V ich eúchs W  $\cdot$ in tiuschen] in tv̂scen T tuͦsschen in U in túschen V in teúschen W \textbf{22} mir] So V  $\cdot$ tuont] duͦt U (W)  $\cdot$ ze] \textit{om.} U \textbf{23} Utpandragun] Vtpandragv̂n T vtpantraguͦn U [vterpandra*]: vterpandragvn V vterpandragun W \textbf{24} Britun] brituͦn U brittvn V \textbf{25} gewerp] gewer W \textbf{27} sæzen] Sazen U (W) Die sesen V \textbf{29} tavelrunder] Davelruͦnne U (V) Die tauelrunde W \textbf{30} valsch hât] valsche hat sich U \newline
\end{minipage}
\end{table}
\end{document}
