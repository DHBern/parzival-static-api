\documentclass[8pt,a4paper,notitlepage]{article}
\usepackage{fullpage}
\usepackage{ulem}
\usepackage{xltxtra}
\usepackage{datetime}
\renewcommand{\dateseparator}{.}
\dmyyyydate
\usepackage{fancyhdr}
\usepackage{ifthen}
\pagestyle{fancy}
\fancyhf{}
\renewcommand{\headrulewidth}{0pt}
\fancyfoot[L]{\ifthenelse{\value{page}=1}{\today, \currenttime{} Uhr}{}}
\begin{document}
\begin{table}[ht]
\begin{minipage}[t]{0.5\linewidth}
\small
\begin{center}*D
\end{center}
\begin{tabular}{rl}
\textbf{519} & \begin{large}D\end{large}och \textbf{en}\textbf{gezwîvelte} nie sîn wille.\\ 
 & diu küneginne Secundille,\\ 
 & die Feirefiz mit rîters hant\\ 
 & erwarp, ir lîp unt ir lant,\\ 
5 & diu hete in ir rîche\\ 
 & hart unlougenlîche\\ 
 & von alter dar der liute vil\\ 
 & mit verkêrtem \textbf{antlützes} zil;\\ 
 & si truogen vremdiu, wilden mâl.\\ 
10 & dô sagete man ir umbe den Grâl,\\ 
 & daz ûf erde niht sô rîches was\\ 
 & \textbf{unt} \textbf{des} pflæge ein künec, hiez Anfortas.\\ 
 & daz dûhte si wunderlîch genuoc,\\ 
 & wan vil wazzer in ir lant truoc\\ 
15 & vür den griez edel \textbf{gesteine}.\\ 
 & grôz, niht ze kleine,\\ 
 & hete si gebirge guldîn.\\ 
 & dô dâhte diu edele künegîn:\\ 
 & "wie gewinne ich künde disses man,\\ 
20 & dem der Grâl ist undertân?"\\ 
 & Si sant ir kleinœde dar,\\ 
 & zwei mensch wunderlîch gevar,\\ 
 & Cundrien unt ir bruoder clâr.\\ 
 & si sante im mêr dennoch vür wâr,\\ 
25 & daz niemen m\textit{ö}hte vergelten;\\ 
 & man vündz veile selten.\\ 
 & Dô sande der süeze Anfortas,\\ 
 & wander êt ie vil milte was,\\ 
 & Orgelusen de Logrois\\ 
30 & disen knappen kurtois.\\ 
\end{tabular}
\scriptsize
\line(1,0){75} \newline
D \newline
\line(1,0){75} \newline
\textbf{1} \textit{Initiale} D  \textbf{21} \textit{Majuskel} D  \textbf{27} \textit{Majuskel} D  \newline
\line(1,0){75} \newline
\textbf{25} möhte] mohte D \textbf{29} Orgelvͦsen de Logroys D \newline
\end{minipage}
\hspace{0.5cm}
\begin{minipage}[t]{0.5\linewidth}
\small
\begin{center}*m
\end{center}
\begin{tabular}{rl}
 & doch \textbf{gezwîvelt} nie sîn wille.\\ 
 & diu k\textit{ü}nigîn Secundille,\\ 
 & die Ferefi\textit{z} mit ritters hant\\ 
 & erwarp, ir lîp und ir lant,\\ 
5 & diu het in ir rîche\\ 
 & harte unlo\textit{u}genlîche\\ 
 & von alter dar der liute vil\\ 
 & mit verkêrtem \textbf{antlitze} zil;\\ 
 & si truogen vrömdiu, wildiu mâl.\\ 
10 & dô sagt man ir umb den Grâl,\\ 
 & daz ûf erde nih\textit{t} \textit{s}ô rîches was,\\ 
 & \textbf{de\textit{s}} pflæge ein künic, hiez Anfortas.\\ 
 & daz d\textit{û}hte si wunderlîch genuoc,\\ 
 & wan vil wazzer in ir lant truoc\\ 
15 & v\textit{ür} den grie\textit{z} edel \textbf{gestein}.\\ 
 & grôz, niht zuo klein,\\ 
 & het si gebirge guldîn.\\ 
 & dô dâhte diu edel künigîn:\\ 
 & "wie gew\textit{i}nne ich künde dises man,\\ 
20 & dem der Grâl ist undertân?"\\ 
 & si sante ir kleinœte dar,\\ 
 & zwei mensch wunderlîch gevar,\\ 
 & Condrien und ir bruoder clâr.\\ 
 & si sante im mê dannoch vür wâr,\\ 
25 & daz niemen möhte vergelten;\\ 
 & man \textit{v}ü\textit{n}de ez v\textit{e}il selten.\\ 
 & dô sante der süeze Anfortas,\\ 
 & wan er eht ie vil milte was,\\ 
 & Urgeluse de Logrois\\ 
30 & disen knappen kurtois.\\ 
\end{tabular}
\scriptsize
\line(1,0){75} \newline
m n o \newline
\line(1,0){75} \newline
\newline
\line(1,0){75} \newline
\textbf{1} wille] sinne oder wille n \textbf{2} künigîn] knigin m  $\cdot$ Secundille] secundile m secúndille o \textbf{3} Ferefiz] ferre fir m o ferre fuͯr n \textbf{6} unlougenlîche] vnlouogeliche m vngelogelich o \textbf{8} antlitze] antlitzes n anczlit o \textbf{9} wildiu] [milde]: wilde o \textbf{10} sagt] sagete n \textbf{11} erde] erden o  $\cdot$ niht sô] nit nit so m \textbf{12} des] Der m  $\cdot$ Anfortas] anforttas m [anfr]: anfortas o \textbf{13} dûhte] duͯhtte m (o) \textbf{15} vür] Von m  $\cdot$ griez] grier m o grior n \textbf{18} dô] Da o \textbf{19} gewinne] gewunne m  $\cdot$ künde dises] kunne des o \textbf{20} Grâl] grole n \textbf{22} \textit{Versfolge 519.23-22} n   $\cdot$ mensch] [m*]: mensch m [me*]: menschen n \textbf{23} Condrien] [Cond*]: Condrie n  $\cdot$ bruoder] ponder n \textbf{24} mê dannoch] dannoch mer n (o) \textbf{25} möhte] mochte o \textbf{26} Man suͯnede es vil seltten m \textbf{27} Anfortas] anforttas m \textbf{28} er] \textit{om.} o  $\cdot$ eht] \textit{om.} n \textbf{29} Vrgelúge die logrois o \newline
\end{minipage}
\end{table}
\newpage
\begin{table}[ht]
\begin{minipage}[t]{0.5\linewidth}
\small
\begin{center}*G
\end{center}
\begin{tabular}{rl}
 & \begin{large}D\end{large}och \textbf{gezwî\textit{v}elte} nie sîn wille.\\ 
 & di\textit{u} künegîn Secundille,\\ 
 & die Feirafiz mit rîters hant\\ 
 & erwarp, ir lîp unde ir lant,\\ 
5 & diu het in ir rîche\\ 
 & harte unlougenlîche\\ 
 & von alter dô der liute vil\\ 
 & mit verkêrtem \textbf{antlützes} zil;\\ 
 & si truogen vrömd\textit{iu}, wild\textit{iu} mâl.\\ 
10 & dô sagete man ir umben Grâl,\\ 
 & daz ûf erden niht sô rîches was\\ 
 & \textbf{unde} \textbf{es} pflæge ein künic, hiez Anfortas.\\ 
 & daz dûhte s\textit{i} wunderlîch genuoc,\\ 
 & wan vil wazzer in ir lant truoc\\ 
15 & vür den griez edel \textbf{gesteine}.\\ 
 & grôz, niht ze kleine,\\ 
 & het si gebirge guldîn.\\ 
 & dô dâhte diu edel künigîn:\\ 
 & "wie gewinne ich kün\textit{d}e dises man,\\ 
20 & dem der Grâl ist undertân?"\\ 
 & si sande ir kleinœde dar,\\ 
 & zwei mensch\textit{e} wunderlîch gevar,\\ 
 & Gundrien unde ir bruoder clâr.\\ 
 & si sande im mêr dannoch vür wâr,\\ 
25 & daz niemen m\textit{ö}hte vergelten;\\ 
 & man vünd\textit{e ez} veile selten.\\ 
 & dô sande der süeze Anfortas,\\ 
 & wan er êt ie vil milte was,\\ 
 & Orgelusen de Logrois\\ 
30 & disen knappen kurtois.\\ 
\end{tabular}
\scriptsize
\line(1,0){75} \newline
G I L M Z \newline
\line(1,0){75} \newline
\textbf{1} \textit{Initiale} G I L Z  \textbf{9} \textit{Initiale} M  \textbf{23} \textit{Initiale} I  \newline
\line(1,0){75} \newline
\textbf{1} gezwîvelte] gezwischelte G  $\cdot$ sîn] \textit{om.} I \textbf{2} diu] die G  $\cdot$ Secundille] segundille G secuntille I Secuͯndille L \textbf{3} Feirafiz] fetefiz G ferafeiz I feirefiz L ferrafisz M ferefiz Z \textbf{4} lant] [hant]: lant Z \textbf{6} unlougenlîche] [vngelonliche]: vnlogenliche G [vngeliche]: vnlugeliche M vngelogenliche Z \textbf{7} dô] da I L (Z) o\textit{m. } M \textbf{8} verkêrtem] verkerten I (M) \textbf{9} Si truegen fromden wilden mal G \textbf{10} dô] Da L M  $\cdot$ sagete] seit I (L) (Z) \textbf{11} erden] erde I (Z) er dez L der erdin M  $\cdot$ rîches] tures M \textbf{12} unde] \textit{om.} M  $\cdot$ es] sin I dez L (M) (Z)  $\cdot$ hiez] der hiez L  $\cdot$ Anfortas] Amfortas L \textbf{13} si] sih G \textbf{14} lant] lande I \textbf{15} gesteine] steyne M \textbf{16} niht ze] [*]: vnde M \textbf{18} dô] Da L M  $\cdot$ dâhte] gedaht I \textbf{19} gewinne ich künde] gewinne ih chunne G chunde ich gewinnen I gewýnne ich kvndes L gewunne kvnde Z  $\cdot$ dises] disen I dez L (M) \textbf{22} mensche] menschen G \textbf{23} Gundrien] Kvndrien L Cvndrien Z \textbf{24} mêr] \textit{om.} M  $\cdot$ dannoch] noch I L \textbf{25} möhte] mohte G I L (M) Z \textbf{26} Man vindet osz vil Seldin M  $\cdot$ vünde ez] vunden G vindet ez I \textbf{27} dô] Da M Z  $\cdot$ Anfortas] Amfortas L \textbf{28} êt] \textit{om.} L \textbf{29} Orgelusen] Orgulus I Orgelýsen L Orgeluse M  $\cdot$ Logrois] logrôys G lorgois I Logroýs L logroÿs M \newline
\end{minipage}
\hspace{0.5cm}
\begin{minipage}[t]{0.5\linewidth}
\small
\begin{center}*T
\end{center}
\begin{tabular}{rl}
 & Doch \textbf{en}\textbf{zwîvelte} nie sîn wille.\\ 
 & diu künegîn Secundille,\\ 
 & die Ferefis mit rîters hant\\ 
 & erwarp, ir lîp unde ir lant,\\ 
5 & diu hete i\textit{n} \textit{i}r rîche\\ 
 & harte unlougenlîche\\ 
 & von alter dâ der liute vil\\ 
 & mit verkêrten \textbf{antlitzes} zil;\\ 
 & si truogen vremdiu, wilt mâl.\\ 
10 & dô seite man ir umbe den Grâl,\\ 
 & daz ûf erde niht sô rîches was\\ 
 & \textbf{unde} \textbf{des} pflæge ein künec, hiez Anfortas.\\ 
 & \begin{large}D\end{large}az dûht si wunderlîch genuoc,\\ 
 & wande vil wazzer in ir lande truoc\\ 
15 & vür den griez edel \textbf{steine}.\\ 
 & grôze, niht ze kleine,\\ 
 & hete si gebirge guldîn.\\ 
 & dô dâhte diu edele künegîn:\\ 
 & "wie gewinnich künde disses man,\\ 
20 & dem der Grâl ist undertân?"\\ 
 & si sante ir kleinœte dar,\\ 
 & zwei menschen wunderlîch gevar,\\ 
 & Kundrien unde ir bruoder clâr.\\ 
 & si santim mêr dannoch vür wâr,\\ 
25 & daz nieman m\textit{ö}hte vergelten;\\ 
 & man vündez veile selten.\\ 
 & Dô sante der süeze Anfortas,\\ 
 & wander eht ie vil milte was,\\ 
30 & \hspace*{-.7em}\big| disen knappen kurtois\\ 
 & \hspace*{-.7em}\big| Orgelusen de Logrois.\\ 
\end{tabular}
\scriptsize
\line(1,0){75} \newline
T U V W O Q R \newline
\line(1,0){75} \newline
\textbf{1} \textit{Initiale} O   $\cdot$ \textit{Majuskel} T  \textbf{13} \textit{Initiale} T U  \textbf{21} \textit{Initiale} W  \textbf{27} \textit{Majuskel} T  \newline
\line(1,0){75} \newline
\textbf{1} Doch] ÷vch Q  $\cdot$ enzwîvelte] ingezwivelte U gezwifelte V zweifelte W zwifelt O (Q) gezwiffelt R \textbf{2} Secundille] Setuͦndille U \textbf{3} Ferefis] fereifiz U (O) vereuis V ferafis W feirefisz Q feirefis R  $\cdot$ rîters] ir O \textbf{4} ir lîp] irn leib W  $\cdot$ unde] vnd auch W \textbf{5} hete] hat W (R)  $\cdot$ in ir] in in ir T \textbf{6} unlougenlîche] vngelogenliche U V Q vngelvgeliche O vngegenliche R \textbf{7} dâ] \textit{om.} O \textbf{8} mit] \textit{om.} O  $\cdot$ verkêrten] verkerteme U (V) (W) (O)  $\cdot$ antlitzes] [antl*]: antlitze V antlút R \textbf{9} \textit{Versfolge 519.10-9} O   $\cdot$ truogen vremdiu] truͦg fremder W truͦgent froͯmde R  $\cdot$ wilt] [wil]: wilde V \textbf{10} seite] sagt O (Q) \textbf{11} erde] erden W R \textbf{12} unde] \textit{om.} W  $\cdot$ des] sein Q  $\cdot$ hiez] hieze U \textbf{13} si] sein Q \textbf{14} wazzer in ir] wazzers in irm U (W) wasser [*]: in ir  V \textbf{15} edel steine] edel gesteine U (V) (W) O (Q) R \textbf{17} hete] Hettent V  $\cdot$ gebirge] gebrig R \textbf{18} dô] Da R  $\cdot$ dâhte] gedaht O gedecht Q duchte R  $\cdot$ edele] edelen W \textbf{19} gewinnich] gewúnne ich V (Q) (R) gewnne ich O  $\cdot$ künde] kúnne Q \textbf{23} Kundrien] kvndryen T Kuͦndrien U  $\cdot$ unde] dar zuͦ W \textbf{24} mêr dannoch] dannoch mer O \textbf{25} daz] Die W  $\cdot$ möhte] mohte T (U) (W) (O) (Q) R \textbf{26} vündez] vinde es W vindet ez O  $\cdot$ selten] gar selten W \textbf{27} sante] sant e Q  $\cdot$ der] \textit{om.} U \textbf{28} wander] Wan der U Wann fúrwar W  $\cdot$ eht] \textit{om.} U V W O Q R  $\cdot$ ie] \textit{om.} W \textbf{30} \textit{Versfolge 519.29-30} V W O Q R  \textbf{29} Orgelusen] Orgolusen V  $\cdot$ de] der O von R  $\cdot$ Logrois] Logroys T (U) logroẏs V \newline
\end{minipage}
\end{table}
\end{document}
