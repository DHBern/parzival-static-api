\documentclass[8pt,a4paper,notitlepage]{article}
\usepackage{fullpage}
\usepackage{ulem}
\usepackage{xltxtra}
\usepackage{datetime}
\renewcommand{\dateseparator}{.}
\dmyyyydate
\usepackage{fancyhdr}
\usepackage{ifthen}
\pagestyle{fancy}
\fancyhf{}
\renewcommand{\headrulewidth}{0pt}
\fancyfoot[L]{\ifthenelse{\value{page}=1}{\today, \currenttime{} Uhr}{}}
\begin{document}
\begin{table}[ht]
\begin{minipage}[t]{0.5\linewidth}
\small
\begin{center}*D
\end{center}
\begin{tabular}{rl}
\textbf{277} & Keie erwarp dô \textbf{niwen} haz\\ 
 & von rittern \textbf{unt} \textbf{von} vrouwen, \textbf{swer} dâ saz\\ 
 & \textbf{ame} stade bî dem Plimizœl.\\ 
 & Gawan unt Jofreit fiz Idœl\\ 
5 & \textbf{unt} des nôt ir habt gehœret ê,\\ 
 & der gevangene \textbf{künec} Clamide,\\ 
 & unt \textbf{anders manec} werder man\\ 
 & - \textbf{ir namen ich wol genennen} kan,\\ 
 & wan daz ichz niht wil lengen -,\\ 
10 & die begunden sich \textbf{dô} mengen.\\ 
 & \textbf{Ir} dienst mit \textbf{zühten} wart gedolt.\\ 
 & vrou Jeschute wart geholt\\ 
 & ûf ir pferde, \textbf{al} dâ si saz.\\ 
 & \textbf{der künec Artus} niht vergaz,\\ 
15 & \textbf{unt ouch} diu küneginne, sîn wîp,\\ 
 & \textbf{si} \textbf{enpfiengen} Jeschuten lîp.\\ 
 & von vrouwen dâ manec kus geschach.\\ 
 & Artus ze Jeschuten sprach:\\ 
 & "\textbf{iwern vater}, den künec von Karnant,\\ 
20 & \textbf{Lacken}, hân ich \textbf{des} erkant,\\ 
 & daz ich iwern kumber klagete,\\ 
 & sît man \textbf{mire\textit{n} zem êrsten} sagete.\\ 
 & ouch sît ir selbe sô wolgetân,\\ 
 & \textbf{es} solt iuch vriwent erlâzen hân,\\ 
25 & wan iwer minneclîcher blic\\ 
 & behielt den prîs ze Kanadic.\\ 
 & durch iwer schœne mære\\ 
 & beleib iu der sparwære;\\ 
 & \textbf{\begin{large}I\end{large}wer} hant er dannen reit.\\ 
30 & swie mir von Orilus leit\\ 
\end{tabular}
\scriptsize
\line(1,0){75} \newline
D \newline
\line(1,0){75} \newline
\textbf{11} \textit{Majuskel} D  \textbf{29} \textit{Initiale} D  \newline
\line(1,0){75} \newline
\textbf{1} Keie] kêie D \textbf{3} Plimizœl] Plimizoͤl D \textbf{4} Idœl] Jdoͤl D \textbf{6} Clamide] Clamidê D \textbf{12} Jeschute] Jescv̂te D \textbf{16} Jeschuten] Jescv̂ten D \textbf{18} Jeschuten] Jescvten D \textbf{20} Lacken] lachen D \textbf{22} miren] mirem D \textbf{26} Kanadic] kanedîch D \newline
\end{minipage}
\hspace{0.5cm}
\begin{minipage}[t]{0.5\linewidth}
\small
\begin{center}*m
\end{center}
\begin{tabular}{rl}
 & \begin{large}K\end{large}eie erwarp dô \textbf{niuwen} haz\\ 
 & von rittern, \textbf{von} vrouwen, \textbf{wer} d\textit{â} saz\\ 
 & \textbf{an} stade bî dem Pli\textit{mi}zo\textit{l}.\\ 
 & Gawan und Jof\textit{r}e\textit{i}t fi\textit{z} Ido\textit{l}\\ 
 & \hspace*{-.7em}\big| \textbf{und} der \textit{gevangen \textbf{künic}} \textit{C}lamide,\\ 
5 & \hspace*{-.7em}\big| des nôt ir habt gehœret ê,\\ 
 & und \textbf{anders manic} werder man,\\ 
 & \textbf{der namen ich wol genennen} kan,\\ 
 & wand daz ichz niht wil lengen,\\ 
10 & die begunden sich \textbf{dô} \textit{m}en\textit{g}en.\\ 
 & \textbf{ir} dienst mit \textbf{zühten} wart gedolt.\\ 
 & vrouwe Jeschute wart geholt\\ 
 & ûf ir pferde, \textbf{al}dâ si saz.\\ 
 & \textbf{der künic Artus} niht vergaz,\\ 
15 & \textbf{und ouch} diu künigîn, sîn wîp,\\ 
 & \textbf{si} \textbf{enpfienc} Jeschuten lîp.\\ 
 & von vrouwen d\textit{â} manic kus geschach.\\ 
 & Artus ze Jeschuten sprach:\\ 
 & "den künic \textbf{Lac} von Karnant,\\ 
20 & \textbf{iuwern vater}, hân ich \textbf{sô} erkant,\\ 
 & daz ich iuwern kumber klagete,\\ 
 & sît man \textbf{von êrst mir in} sagete.\\ 
 & ouch sît ir selbe sô wol getân,\\ 
 & \textbf{es} solt iuch \dag vriunde\dag  erlâzen hân,\\ 
25 & wand iuwer minneclîche\textit{r} bli\textit{c}\\ 
 & behielt den prîs ze Kanedi\textit{c}.\\ 
 & durch iuwer schœne mære\\ 
 & beleip iu der sperwære;\\ 
 & \textbf{in der} hant er dannen reit.\\ 
30 & wie mir von Orilus leit\\ 
\end{tabular}
\scriptsize
\line(1,0){75} \newline
m n o \newline
\line(1,0){75} \newline
\textbf{1} \textit{Initiale} m n  \newline
\line(1,0){75} \newline
\textbf{1} Keie] KEẏe n Keine o \textbf{2} dâ] do m n o \textbf{3} stade] dem staden n dem stade o  $\cdot$ Plimizol] plinezolt m plúnzol n plunzol o \textbf{4} Gawan] Gewan n o  $\cdot$ Jofreit] jofert m gotfrit n gotfritz o  $\cdot$ fiz Idol] fir idolt m firtzidol n friczedol o \textbf{6} Vnd der kunig gefangen von clamide m  $\cdot$ Clamide] klamide o \textbf{5} habt] hab o \textbf{8} der] Den o  $\cdot$ genennen] genen m \textbf{9} ichz] ich o \textbf{10} begunden] begunde o  $\cdot$ sich dô mengen] sich do nennen m do sich mengen n \textbf{12} Jeschute] jescutte m jescute n juscete o \textbf{13} aldâ si] sú do n (o) \textbf{15} diu künigîn] die konige o \textbf{16} enpfienc] enpfingen n (o)  $\cdot$ Jeschuten] jescutten m jescúten n juscuten o \textbf{17} dâ] do m n o  $\cdot$ kus] kúsch o \textbf{18} Jeschuten] jescutten m jescuten n o \textbf{19} Lac] lag m n o  $\cdot$ von] \textit{om.} n \textbf{23} selbe] selbes n \textbf{24} solt] sint so n sint o  $\cdot$ erlâzen] erloͯset o \textbf{25} minneclîcher] minnekliche m  $\cdot$ blic] blicke m \textbf{26} Kanedic] Kanedige m konedick n konigdig o \textbf{27} iuwer] ire m ir n o \textbf{28} iu] \textit{om.} n o \textbf{30} Orilus] orelus o \newline
\end{minipage}
\end{table}
\newpage
\begin{table}[ht]
\begin{minipage}[t]{0.5\linewidth}
\small
\begin{center}*G
\end{center}
\begin{tabular}{rl}
 & Kay erwarp dô \textbf{niwan} haz\\ 
 & von rîtæren \textbf{unde} vrouwen, \textbf{swaz ir} dâ saz\\ 
 & \textbf{an dem} stade bî dem Blimzol.\\ 
 & Gawan und Jofreit fis Idol\\ 
5 & \textbf{unt} des nôt ir habet gehœret ê,\\ 
 & der gevangene Clamide,\\ 
 & unde \textbf{manic ander} werder man,\\ 
 & \textbf{den ich genennen wol} kan,\\ 
 & wan daz ich ez niht wil lengen,\\ 
10 & die begunden sich mengen.\\ 
 & \textbf{ê} dienst mit \textbf{zuht} wart gedolt,\\ 
 & vrou Jeschute wart geholt\\ 
 & ûf ir pferde, dâ si saz.\\ 
 & \textbf{Artus, der künic}, niht vergaz,\\ 
15 & \textbf{unde ouch} diu künigîn, sîn wîp,\\ 
 & \textbf{die} \textbf{enpfiengen} Jeschuten lîp.\\ 
 & von vrouwen dâ manic kus geschach.\\ 
 & Artus ze Jeschuten sprach:\\ 
 & "\textbf{iuweren vater}, den künic von Karnant,\\ 
20 & \textbf{lange} hân ich \textbf{den} erkant,\\ 
 & daz ich iuwern kumber klagte,\\ 
 & sît man \textbf{mirn von êrste} sagte.\\ 
 & ouch sît ir selbe sô wolgetân,\\ 
 & \textbf{es} solt iuch vriunt erlâzen hân,\\ 
25 & wan iuwer minniclîcher blic\\ 
 & behielt den brîs ze Kanadic.\\ 
 & durch iuwer schœne mære\\ 
 & beleip iu der sparwære;\\ 
 & \textbf{iuwer} hant er dannen reit.\\ 
30 & swie mir von Orillus leit\\ 
\end{tabular}
\scriptsize
\line(1,0){75} \newline
G I O L M Q R Z Fr30 Fr36 \newline
\line(1,0){75} \newline
\textbf{1} \textit{Initiale} I O L Q  \textbf{24} \textit{Initiale} I  \textbf{25} \textit{Initiale} R  \textbf{27} \textit{Initiale} Z  \newline
\line(1,0){75} \newline
\textbf{1} Kay] kai G Kayn I ÷ey O Key M R Z  $\cdot$ erwarp] erwar R  $\cdot$ dô] da I O L M Z  $\cdot$ niwan] nyman M newen Q (R)  $\cdot$ haz] basz M \textbf{2} Von rittern vrowen vnd waz daz waz L  $\cdot$ rîtæren] hern M  $\cdot$ unde] vnd von I Q R Z  $\cdot$ swaz] was M (Q) R wer Z  $\cdot$ ir] \textit{om.} Z  $\cdot$ dâ] das R do Q  $\cdot$ saz] was R \textbf{3} dem] den M  $\cdot$ Blimzol] blimizol I (O) (Fr36) plýmýzol L blimiczol M plimizol Q R Z \textbf{4} Gawan] gæwan G gewan M (Q) (R) Fr36 Gawein Z  $\cdot$ Jofreit] tschofreit G schoffrit I Tschovrit O Jofrait L \textit{om.} M Jofreiet Q Lofreit Z Joͤfrit Fr36  $\cdot$ fis Idol] visidol G O M Z Fr36 fisidol I Q R Fizedol L \textbf{5} unt] von I (O)  $\cdot$ nôt] not vnd I  $\cdot$ habet gehœret] hortent L haupt gehort Q \textbf{6} gevangene] gevangen chvnich O (L) (M) (Q) (R) (Z)  $\cdot$ Clamide] Glamide O \textbf{7} manic ander werder] ander manger werder I manich werde ander O manig wert ander L (M) (Q) (R) (Z) (Fr36) \textbf{8} der namen ich wol genennen chan I (Z)  $\cdot$ Die ich wol genennen (genemen R ) chan O (L) (M) (Q) (R) (Fr36) \textbf{9} ich ez] ich I osz M \textbf{10} begunden] bendent R  $\cdot$ sich] sich da L  $\cdot$ mengen] lengen R nennen Fr30 \textbf{11} ê] \textit{om.} I L Er M Jr Z  $\cdot$ zuht] zuhten I (L) (R) (Z) \textbf{12} vrou] Vrder L  $\cdot$ Jeschute] ieschute G ieskute I Jeschvͦte O Jescuͯte L iescute M Z Fr36 Jescute Q Jscute R iescuten Fr30  $\cdot$ geholt] gedolt Fr30 \textbf{13} ûf] Vnd Fr36  $\cdot$ dâ] alda O (L) M (Q) R Z (Fr36) \textbf{14} der künic] do I  $\cdot$ niht vergaz] vorgasz M [ver]: niht vergar Z \textbf{15} ouch] \textit{om.} O L M Q Fr36 \textbf{16} die] Si O (L) (M) (Q) (R) (Z) (Fr36) sine Fr30  $\cdot$ Jeschuten] ieschuten G ieskuten I Jeschvͦten O Jescuͯten L [Jecuten]: Jescuten M iescuten Q Jscuten R Jescuten Z iescvͦten Fr30 iscuten Fr36 \textbf{17} vrouwen] freuden I vrowe L  $\cdot$ dâ] do O Q R \textit{om.} Fr30 \textbf{18} Artus] Artuͯs L  $\cdot$ ze Jeschuten] ze ieschuten G zuͤ froͮn ieskuten I ze Jescvͦten O zuͯ Jescuͯten L zcu iescuten M (Z) zu Jescuten Q zu Juscuten R zeiescvͦten Fr30 \textbf{19} iuweren] ewer I  $\cdot$ den] der I R \textbf{20} lange] Lachen O Lacken L (R) (Z)  $\cdot$ den] des Z \textbf{21} daz] Des R  $\cdot$ iuwern] uwer L den Fr30  $\cdot$ klagte] clage L \textbf{22} mirn von] mir O mir den L myrn M (Q) (Z) mirs von R mir iz Fr30  $\cdot$ sagte] clagite M \textbf{23} selbe] \textit{om.} I selbern M selber Q \textbf{24} es] Diz I Er O  $\cdot$ iuch] uͯcher L  $\cdot$ vriunt] fromde O \textbf{25} wan] War Q \textbf{26} den] [der]: den G  $\cdot$ ze Kanadic] zechanadich G zechanadic I ze [kar]: kamadich O zuͯ Kanadich L zcu kanedic M zu kanadick Q ze kanadik R zv kanadic Z zekandiz Fr30 \textbf{30} swie] Wie L (Q) R  $\cdot$ Orillus] Orilus I (Q) R (Z) Orilvse O (M) Orylles Fr30 \newline
\end{minipage}
\hspace{0.5cm}
\begin{minipage}[t]{0.5\linewidth}
\small
\begin{center}*T
\end{center}
\begin{tabular}{rl}
 & \begin{large}K\end{large}ey erwarp dô \textbf{niwen} haz\\ 
 & von rîtern \textbf{unde} vrouwen, \textbf{swer} dâ saz\\ 
 & \textbf{an dem} stade bî dem Plymizol.\\ 
 & Gawan unde Jofreit fis Idol\\ 
5 & \multicolumn{1}{l}{ - - - }\\ 
 & \multicolumn{1}{l}{ - - - }\\ 
 & unde \textbf{anders manec} werder man\\ 
 & - \textbf{ir namen ich wol genennen} kan,\\ 
 & wan daz ichz niht wil lengen -,\\ 
10 & die begunden sich mengen.\\ 
 & \textbf{ir} dienst mit \textbf{zühten} wart gedolt.\\ 
 & Vrou Jeschute wart geholt\\ 
 & ûf ir pferde, \textbf{al}dâ si saz.\\ 
 & \textbf{Artus, der künec}, niht vergaz,\\ 
15 & \textbf{Als tet} diu künegîn, sîn wîp,\\ 
 & \textbf{si} \textbf{enpfiengen} \textbf{vroun} Jeschuten lîp.\\ 
 & von vrouwen dâ manec kus geschach.\\ 
 & Artus ze \textbf{vroun} Jeschuten sprach:\\ 
 & "\textbf{Iuwern vater}, den künec von Garnant,\\ 
20 & \textbf{Lacken}, hân ich \textbf{des} erkant,\\ 
 & daz ich iuwern kumber klagete,\\ 
 & sît man \textbf{mirn zem êrsten} sagete.\\ 
 & ouch sît ir selbe sô wol getân,\\ 
 & \textbf{ez} solt i\textit{uch} vriunt erlâzen hân,\\ 
25 & wand iuwer minneclîcher blic\\ 
 & behielt den prîs ze Kanadic.\\ 
 & durch iuwer schœne mære\\ 
 & bleip iu der sperwære;\\ 
 & \textbf{iuwer} hant er dannen reit.\\ 
30 & swie mir von Oriluse leit\\ 
\end{tabular}
\scriptsize
\line(1,0){75} \newline
T U V W \newline
\line(1,0){75} \newline
\textbf{1} \textit{Initiale} T U V W  \textbf{12} \textit{Majuskel} T  \textbf{15} \textit{Majuskel} T  \textbf{19} \textit{Majuskel} T  \newline
\line(1,0){75} \newline
\textbf{1} Key] Keyn V \textbf{2} unde] vnd von U (V) o\textit{m. } W  $\cdot$ swer] wer U vnd wer W  $\cdot$ dâ] do U V W \textbf{3} Plymizol] plimizol V W \textbf{4} Gawan] [J]: Gawein T U Gawen V  $\cdot$ Jofreit] Jofêit T Jofrit U (W) iofrid V  $\cdot$ fis Idol] fisidol T U [*]: visidol V frizidol W \textbf{5} \textit{Die Verse 277.5-6 sind am Rand nachgetragen und später radiert:} D::: gehoͤret ::: / vnde der :::g :::de V   $\cdot$ [*]: vnde dez not ir hant gehoͤret e V \textbf{6} [*]: Der gevangene kv́nig clamide V \textbf{7} [*]: vnde manig wert ander man V  $\cdot$ anders] ander W \textbf{8} ir namen] [*]: Die V Der namen W  $\cdot$ genennen] nennen W \textbf{9} niht] \textit{om.} U \textbf{11} ir] [*]: E V \textbf{12} Jeschute] Jescvte T Jescuten U iescute V iestute W \textbf{14} [*]: Artus der kv́nig niht vergas V \textbf{15} Als tet diu] Als die U [*]: vnde oͮch die V Als thet dir W \textbf{16} si] [*]: Sú V  $\cdot$ vroun] \textit{om.} W  $\cdot$ Jeschuten] Jescvten T (U) iescuten V iestuten W \textbf{17} dâ] do V W \textbf{18} [*]: Artus zuͦ froͮn iescuten sprach V  $\cdot$ vroun] \textit{om.} W  $\cdot$ Jeschuten] Jescvten T (U) iestuten W \textbf{19} [*]: Vwern vatter han ich so erkant V  $\cdot$ Garnant] sarnant W \textbf{20} [*]: Den kv́nig lag von karnant V  $\cdot$ Lacken] Laken U Lachen W \textbf{22} mirn] mir W  $\cdot$ zem êrsten] ze mersten T zuͦ me erste U [*]: von erst V \textbf{24} iuch vriunt] in vrivnt T eúwer frúnt úch W \textbf{26} Kanadic] Canadic U [*]: kanadig V benedic W \textbf{30} swie] Wie U W  $\cdot$ Oriluse] Orilus U (V) (W) \newline
\end{minipage}
\end{table}
\end{document}
