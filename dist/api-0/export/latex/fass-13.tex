\documentclass[8pt,a4paper,notitlepage]{article}
\usepackage{fullpage}
\usepackage{ulem}
\usepackage{xltxtra}
\usepackage{datetime}
\renewcommand{\dateseparator}{.}
\dmyyyydate
\usepackage{fancyhdr}
\usepackage{ifthen}
\pagestyle{fancy}
\fancyhf{}
\renewcommand{\headrulewidth}{0pt}
\fancyfoot[L]{\ifthenelse{\value{page}=1}{\today, \currenttime{} Uhr}{}}
\begin{document}
\begin{table}[ht]
\begin{minipage}[t]{0.5\linewidth}
\small
\begin{center}*D
\end{center}
\begin{tabular}{rl}
\textbf{13} & \textbf{sîniu werc}, dâ er vremde wære.\\ 
 & sô geloupte man\textbf{z} mære.\\ 
 & \textit{\begin{large}G\end{large}}ahmuret der site pflac,\\ 
 & \textbf{den rehtiu} mâze widerwac,\\ 
5 & und ander schanze enkeine:\\ 
 & sîn rüemen, daz was kleine,\\ 
 & grôze êre er lîdenlîche leit,\\ 
 & der \textbf{lôse} wille in gar vermeit.\\ 
 & doch wânde der gevüege,\\ 
10 & daz \textbf{nieman} krône trüege,\\ 
 & \textbf{künec}, keiser, keiserîn,\\ 
 & \textbf{des} messenîe er wolde sîn,\\ 
 & \textbf{niwan} \textbf{eines}, \textbf{der} die hœhsten hant\\ 
 & \textbf{trüege} ûf erde über \textbf{elliu} lant.\\ 
15 & der wille in sînem herzen lac.\\ 
 & im \textbf{wart} gesagt, ze Baldac\\ 
 & wære ein sô gewaltic man,\\ 
 & daz im der erde undertân\\ 
 & \textbf{diu zwei teil} wæren oder mêr.\\ 
20 & sîn nam\textit{e} heidensch was sô hêr,\\ 
 & daz man in hiez '\textbf{der} bâruc'.\\ 
 & er hete an \textbf{krefte} \textbf{al} solhen \textbf{zuc},\\ 
 & vil künege wâren sîne man,\\ 
 & mit \textbf{krôntem} lîbe undertân.\\ 
25 & daz bâruc ambet \textbf{hiute stêt}.\\ 
 & seht, wie man kristen ê begêt\\ 
 & ze Rome, als uns der touf vergiht.\\ 
 & heidensch orden man dort siht:\\ 
 & ze Baldac \textbf{nement} \textbf{si} ir bâbestreht;\\ 
30 & \textbf{daz} dunket si âne \textbf{krumben} sleht.\\ 
\end{tabular}
\scriptsize
\line(1,0){75} \newline
D \newline
\line(1,0){75} \newline
\textbf{3} \textit{Initiale} D  \newline
\line(1,0){75} \newline
\textbf{3} Gahmuret] ÷Ahmvͦret \textit{nachträglich korrigiert zu:} GAhmvͦret D \textbf{16} Baldac] Baldach D \textbf{20} name] namen D \textbf{29} Baldac] Baldach D \newline
\end{minipage}
\hspace{0.5cm}
\begin{minipage}[t]{0.5\linewidth}
\small
\begin{center}*m
\end{center}
\begin{tabular}{rl}
 & \textbf{sîne were}, d\textit{â} er vrömde wære.\\ 
 & sô gloubte man \textbf{der} mære.\\ 
 & \textit{G}ahmuret der site pfla\textit{c},\\ 
 & \textbf{diu rehter} mâze widerwac,\\ 
5 & und ander schanze keine:\\ 
 & sîn rüemen, daz was kleine,\\ 
 & grôze êre er lîdenlîchen leit,\\ 
 & der \textbf{bôse} wille in gar vermeit.\\ 
 & doch wânde der gevüege,\\ 
10 & daz \textbf{niemen} krône trüege,\\ 
 & \textbf{künic}, keiser, keiserîn,\\ 
 & \textbf{des} massenîe er wolte sîn,\\ 
 & \textbf{wenne} \textbf{eines}, \textbf{der} die hœheste\textit{n} \dag lant\dag \\ 
 & \textbf{trüege} ûf erden über \textbf{allez} lant.\\ 
15 & der wille in sînem herzen lac.\\ 
 & ime \textbf{wart} geseit, zuo Baldac\\ 
 & wære ein sô gewaltic man,\\ 
 & daz ime der erden undertân\\ 
 & \textbf{daz zweiteil} wæren oder mêre.\\ 
20 & sîn name heidensch was sô hêre,\\ 
 & daz man in hiez \textbf{den} b\textit{âr}uc.\\ 
 & er het an \textbf{kreften} solichen \textbf{zuc},\\ 
 & vil künige wâren sîne man,\\ 
 & mit \textbf{krônete\textit{m}} lîbe undertân.\\ 
25 & daz bâruc ambaht \textbf{hiel\textit{t} stæt}.\\ 
 & seht, wie man kristen ê begêt\\ 
 & zuo Rome, als uns der touf vergiht.\\ 
 & heidensche orden man dort siht:\\ 
 & zuo Baldac \textbf{nement} \textbf{si} ir bâbest\textit{reht};\\ 
30 & \textbf{daz} dunket si ân \textbf{krumbe} sleht.\\ 
\end{tabular}
\scriptsize
\line(1,0){75} \newline
m n o \newline
\line(1,0){75} \newline
\newline
\line(1,0){75} \newline
\textbf{1} dâ] do m n o  $\cdot$ vrömde] fromder o \textbf{2} der] des n das o \textbf{3} Gahmuret] Sahmuret \textit{nachträglich korrigiert zu:} gahmuret m Gamiret n o  $\cdot$ pflac] pflage m \textbf{6} daz] \textit{om.} n \textbf{11} keiser] \textit{om.} n \textbf{13} hœhesten] hohiste m  $\cdot$ lant] lant \textit{nachträglich korrigiert zu:} bant m \textbf{14} über allez lant] wit erkant n o \textbf{16} Baldac] baldack m baldag n o \textbf{19} daz] \textit{om.} n \textbf{21} den] denne n  $\cdot$ bâruc] brauck \textit{nachträglich korrigiert zu:} baruͯck m \textbf{22} kreften] treffent o \textbf{23} vil] [Wil]: Vil o \textbf{24} krônetem] kronetten m gekrontem n (o) \textbf{25} hielt] hielte m  $\cdot$ stæt] stet m stete n o \textbf{26} seht] Sehen n  $\cdot$ begêt] begete n begot o \textbf{27} Rome] Roͯme o \textbf{28} orden] erden m erde n o \textbf{29} Baldac] baldack m baldag n o  $\cdot$ bâbestreht] baͯbest \textit{nachträglich korrigiert zu:} baͯbest recht m \textbf{30} ân] eẏn o \newline
\end{minipage}
\end{table}
\newpage
\begin{table}[ht]
\begin{minipage}[t]{0.5\linewidth}
\small
\begin{center}*G
\end{center}
\begin{tabular}{rl}
 & \textbf{sîniu werc}, dâ er vrömde wære.\\ 
 & sô geloubte man \textbf{daz} mære.\\ 
 & Gahmuret der site pflac,\\ 
 & \textbf{den rehtiu} mâze widerwac,\\ 
5 & unde ander schanze deheine:\\ 
 & sîn rüemen, daz was kleine,\\ 
 & grôz êre er lîdeclîchen leit,\\ 
 & der \textbf{lôse} wille in gar vermeit.\\ 
 & doch wânde der gevüege,\\ 
10 & daz \textbf{iemen} krône trüege,\\ 
 & \textbf{künige}, keiser, keiserîn,\\ 
 & \textbf{der} messenîe er wolte sîn,\\ 
 & \textbf{\begin{large}W\end{large}an} \textbf{der bînamen} die hœhesten hant\\ 
 & \textbf{trüege} ûf erde über \textbf{elliu} lant.\\ 
15 & der wille in sînem herzen lac.\\ 
 & im \textbf{wart} gesaget, ze Baldac\\ 
 & wære ein sô gewaltic man,\\ 
 & daz im der erde undertân\\ 
 & \textbf{diu zwei teil} wæren oder mêr.\\ 
20 & sîn name heidensch was sô hêr,\\ 
 & daz man in hiez \textbf{den} bâruc.\\ 
 & er hete an \textbf{krefte} solhen \textbf{zuc},\\ 
 & vil künige wâren sîn man,\\ 
 & mit \textbf{gekrôntem} lîbe undertân.\\ 
25 & daz bâruc ambet \textbf{hiute stêt}.\\ 
 & seht, wie man kristen ê begêt\\ 
 & ze Rome, als uns der touf vergiht.\\ 
 & heidensch orden man dort siht:\\ 
 & ze Baldac \textbf{nement} \textbf{si} ir bâbestreht;\\ 
30 & \textbf{ez} dunket si âne \textbf{krumbe} sleht.\\ 
\end{tabular}
\scriptsize
\line(1,0){75} \newline
G O L M Q R W Z Fr29 Fr32 Fr36 \newline
\line(1,0){75} \newline
\textbf{1} \textit{Initiale} O M  \textbf{3} \textit{Überschrift:} Hie fuͦr gamuret auß seinem lande vnd kam gen baldag W   $\cdot$ \textit{Platz für Illustration ausgespart} W   $\cdot$ \textit{Initiale} W Fr29  \textbf{5} \textit{Versal} Fr32  \textbf{9} \textit{Initiale} L R Z  \textbf{13} \textit{Initiale} G  \textbf{15} \textit{Versal} Fr32  \newline
\line(1,0){75} \newline
\textbf{1} sîniu] Sine R (Fr32)  $\cdot$ werc] wert W  $\cdot$ dâ] do Q W das R  $\cdot$ vrömde] frome R \textbf{2} geloubte] geloubet L (Q) (Z) (Fr32)  $\cdot$ daz] der M \textbf{3} Gahmuret] Gamvret O (Fr32) Gahmuͯret L Gamurat M Gaműert Q GAmuret W (Z) Gahmvͦret Fr29  $\cdot$ site] siten Q Z stete W  $\cdot$ pflac] pfagk Q \textbf{4} den] Der Q W  $\cdot$ rehtiu] Rechtte R (Fr32) rechten W \textbf{5} ander schanze] anderhande L ander schantzen W \textbf{6} rüemen] Ruͯme R \textbf{7} êre] not W  $\cdot$ er lîdeclîchen] vnd lidenlichev O lidenklich er R \textbf{8} der lôse] Der bose L Boͤser W  $\cdot$ vermeit] vemeit R \textbf{9} doch] Noch L Q Auch W \textbf{10} iemen] man M nymant Q iemans W kein Z \textbf{11} künige] Kung R (W)  $\cdot$ keiserîn] kvnigin L (Q) \textbf{12} wolte] solde O (L) M W Z \textbf{13} Wan] Wann eines Q (R) (Z) (Fr32)  $\cdot$ der] er O L M W  $\cdot$ bînamen] \textit{om.} Q R Z Fr32 \textbf{14} trüege] Truͦch O (Q) (W)  $\cdot$ erde] erdin M der erde W  $\cdot$ elliu] alle R \textbf{15} herzen] hertze Q \textbf{16} Baldac] baldach G O (L) baldagk Q Baldag R (W) \textbf{17} ein sô] so ein L \textbf{18} erde] erde wer O erdin M (W) (Z) \textbf{19} Were die czweyteil vnd mer M  $\cdot$ diu] Der Q  $\cdot$ wæren] \textit{om.} O waren L Z  $\cdot$ oder] vnd L \textbf{20} heidensch was] was haidenischen O heidens was M was Fr32 Fr36 \textbf{21} bâruc] graúck \textit{nachträglich korrigiert zu:} barúck Q \textbf{22} er hete] Es hat R der hete Fr32  $\cdot$ krefte] chreften O (L) (Q) (W) (Z) (Fr32)  $\cdot$ solhen] al solhen O an sulen M  $\cdot$ zuc] tuck Q muͦt R \textbf{23} wâren] worden Q \textbf{24} gekrôntem] gekrontin M (R) gekroͤntē W krontem Z  $\cdot$ lîbe] haubten W \textbf{25} daz] Das dasz Q Da des R (Fr32)  $\cdot$ bâruc ambet] ammecht baruch M g*auck ampt \textit{nachträglich korrigiert zu:} baruck ampt Q  $\cdot$ stêt] stette R \textbf{27} Rome] Rôme O  $\cdot$ der touf] div tavffe O \textbf{28} heidensch] Hedenschen L (Z) Heidens M Heydennische Q Heidesch R Haidenschen W Heidenischen Z heidensc Fr32 Fr36  $\cdot$ man dort] man do Q dort man W \textbf{29} Baldac] baldach G (O) (L) baldack Q baldag R  $\cdot$ ir] irn R \textbf{30} ez] Das W (Z) \newline
\end{minipage}
\hspace{0.5cm}
\begin{minipage}[t]{0.5\linewidth}
\small
\begin{center}*T
\end{center}
\begin{tabular}{rl}
 & \textbf{sîn\textit{iu} werc}, dâ er vremde wære.\\ 
 & sô geloubete man \textbf{der} mære.\\ 
 & Gahmuret der site pflac,\\ 
 & \textbf{ein rehte} mâze widerwac\\ 
5 & und ander schanze deheine:\\ 
 & sîn rüemen, daz was kleine,\\ 
 & \hspace*{-.7em}\big| der \textbf{lôse} wille in gar vermeit,\\ 
 & \hspace*{-.7em}\big| grôz êre er lîdenlîche leit.\\ 
 & \begin{large}D\end{large}och wânde der gevüege,\\ 
10 & daz \textbf{ieman} krône trüege,\\ 
 & \textbf{künege}, keiser, keiserîn,\\ 
 & \textbf{der} massenîe er wolte sîn,\\ 
 & \textbf{wande} \textbf{benamen der} \textit{die} hœhesten hant\\ 
 & \textbf{truoc} ûf erde über \textbf{alliu} lant.\\ 
15 & der wille in sînem herzen lac.\\ 
 & im \textbf{was} gesaget, ze Baldac\\ 
 & wære ein sô gewaltic man,\\ 
 & daz im der erden undertân\\ 
 & \textbf{d\textit{iu} zwei teil} wæren oder mêr.\\ 
20 & sîn nam heidensch was sô hêr,\\ 
 & daz man in hiez \textbf{den} bâruc.\\ 
 & er hete an \textbf{krefte} solhen \textbf{ruc},\\ 
 & vil künege wâren sîne man,\\ 
 & mit \textbf{gekrônetem} lîbe undertân.\\ 
25 & \begin{large}D\end{large}az bâruc ambet \textbf{hiute stât}.\\ 
 & seht, wie man kristen ê begât\\ 
 & ze Rome, als uns der touf vergiht.\\ 
 & heidensch orden man dort siht\\ 
 & ze Baldac \textbf{nemen} ir bâbestreht;\\ 
30 & \textbf{ez} dunket si âne \textbf{krumbe} sleht.\\ 
\end{tabular}
\scriptsize
\line(1,0){75} \newline
T U V \newline
\line(1,0){75} \newline
\textbf{9} \textit{Initiale} T  \textbf{25} \textit{Initiale} T U V  \newline
\line(1,0){75} \newline
\textbf{1} sîniu werc] sine werc T Sin werc U [S*]: Sine werg V \textbf{3} Gahmuret] Gahmvret T Gahmuͦret U Gamuret V \textbf{4} ein] Den V \textbf{5} schanze] schanzte V \textbf{6} daz] \textit{om.} V \textbf{8} \textit{Versfolge 13.7-8} U V  \textbf{7} êre] [*]: ere V \textbf{9} Doch] [Doch]: Ovch V  $\cdot$ gevüege] \textit{om.} U \textbf{12} wolte] solte U V \textbf{13} Wan der bieder man die hohest hant U · [Wand*]: Wande eines der die hohesten hand V  $\cdot$ die] \textit{om.} T \textbf{14} truoc] [tr*]: trvͤg V  $\cdot$ erde] erden V \textbf{16} Baldac] baldag U V \textbf{19} diu] die T \textbf{22} krefte] creften U (V)  $\cdot$ ruc] gezuͦc U zug V \textbf{24} Mit gekroͤneten [*vndertan]: hoͮbeten vndertan V  $\cdot$ gekrônetem] gecrotem U \textbf{25} hiute] hete U \textbf{27} Rome] Rôme T \textbf{29} Zuͦ baldac nemen ir daz beste reth U · Zuͦ Baldag [nement*]: nementz ir [babest*]: babestes reht V \newline
\end{minipage}
\end{table}
\end{document}
