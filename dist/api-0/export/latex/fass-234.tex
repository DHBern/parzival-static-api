\documentclass[8pt,a4paper,notitlepage]{article}
\usepackage{fullpage}
\usepackage{ulem}
\usepackage{xltxtra}
\usepackage{datetime}
\renewcommand{\dateseparator}{.}
\dmyyyydate
\usepackage{fancyhdr}
\usepackage{ifthen}
\pagestyle{fancy}
\fancyhf{}
\renewcommand{\headrulewidth}{0pt}
\fancyfoot[L]{\ifthenelse{\value{page}=1}{\today, \currenttime{} Uhr}{}}
\begin{document}
\begin{table}[ht]
\begin{minipage}[t]{0.5\linewidth}
\small
\begin{center}*D
\end{center}
\begin{tabular}{rl}
\textbf{234} & \begin{large}M\end{large}it \textbf{zuht} si kunden wider gên,\\ 
 & zuo den êrsten vieren stên.\\ 
 & an \textbf{disen} aht vrouwen was\\ 
 & röcke grüener denn ein gras,\\ 
5 & von \textbf{Azagouc} samît,\\ 
 & gesniten \textbf{wol} lanc und wît.\\ 
 & dâ mitten si zesamne twanc\\ 
 & \textbf{gürteln tiure}, \textbf{smal} unt lanc.\\ 
 & \textbf{dise} aht \textbf{vrouwen} kluoc,\\ 
10 & ieslîchiu ob ir hâre truoc\\ 
 & ein kleine \textbf{blüemîn} schapel.\\ 
 & der grâve \textbf{Iwan} von \textbf{Nonel}\\ 
 & unt \textbf{Jernis} von Ril,\\ 
 & \textbf{jâ was} über manege mîl\\ 
15 & \textbf{ze dienste} ir tohter dar genomen.\\ 
 & \textbf{man sach die zwô vürstîn} komen\\ 
 & in harte \textbf{wünneclîcher} wât.\\ 
 & zwei mezzer \textbf{snîdende} als ein grât\\ 
 & \textbf{brâhten} si durch wunder\\ 
20 & \textbf{ûf} zwein twehelen \textbf{al}besunder;\\ 
 & \textbf{daz was} \textbf{silber} \textbf{hert} \textbf{unt} wîz.\\ 
 & dâr an lag ein spæher vlîz:\\ 
 & \textbf{im} was \textbf{solch} \textbf{scherpfen} niht vermiten,\\ 
 & \textbf{ez hete} stahel wol \textbf{versniten}.\\ 
25 & Vor\textbf{em silber} kômen vrouwen wert,\\ 
 & der \textbf{dar ze dienste was} gegert,\\ 
 & die truogen lieht \textbf{dem silber} bî,\\ 
 & vier kint vor missewende vrî.\\ 
 & \textbf{sus} giengen \textbf{si} alle \textbf{sehse} zuo.\\ 
30 & nû hœret, waz ieslîchiu tuo:\\ 
\end{tabular}
\scriptsize
\line(1,0){75} \newline
D \newline
\line(1,0){75} \newline
\textbf{1} \textit{Initiale} D  \textbf{25} \textit{Majuskel} D  \newline
\line(1,0){75} \newline
\textbf{5} Azagouc] Azagoͮch D \textbf{12} Iwan] Jwan D \textbf{13} Ril] Rîl D \textbf{29} zuo] >zvͦ< D \newline
\end{minipage}
\hspace{0.5cm}
\begin{minipage}[t]{0.5\linewidth}
\small
\begin{center}*m
\end{center}
\begin{tabular}{rl}
 & mit \textbf{zuht} si kunden wider gên,\\ 
 & zuo den êrsten vieren stên.\\ 
 & an \textbf{den} ahte vrouwen was\\ 
 & röcke grüener \textit{denne} ein g\textit{r}as,\\ 
5 & von \textbf{Adamach} samît\\ 
 & gesniten \textbf{und} lanc und wît.\\ 
 & dâ mitten si zesamen twanc\\ 
 & \textbf{tiure gürtel} \textbf{smal} und lanc.\\ 
 & \textbf{dise} ahte \textbf{juncvrouwen} kluoc,\\ 
10 & ieglîchiu obe ir hâre truoc\\ 
 & ein kleine \textbf{bluomen} schapel.\\ 
 & der grâve \textbf{Iwan} von \textbf{Nouel}\\ 
 & und \textbf{Gernis} von R\textit{i}le,\\ 
 & \textbf{j\textit{â} was} über manige mîle\\ 
15 & \textbf{zuo dienste} ir tohter dâ genomen.\\ 
 & \textbf{die zwô vürstinne sach ma\textit{n}} komen\\ 
 & in harte \textbf{minniclîcher} wât.\\ 
 & zwei mezzer \textbf{schindende} als ein grât\\ 
 & \textbf{brâhten} si durch wunder\\ 
20 & \textbf{ûf} zwe\textit{i}en twehelen besunder;\\ 
 & \textbf{diu wâren} \textbf{silberîn}, \textbf{herte} \textbf{und} wîz.\\ 
 & dâr an lac ein spæher vlîz:\\ 
 & \textbf{in} was \textbf{solichiu} \textbf{scherpfe} niht vermiten,\\ 
 & \textbf{si heten} stahe\textit{l} wol \textbf{gesniten}.\\ 
25 & vor \textbf{den mezzeren} kômen vrouwen wert,\\ 
 & der \textbf{dâ ze dienste was} gegert,\\ 
 & die truogen lieht \textbf{den mezzeren} bî,\\ 
 & vier kint vor \dag ime iwende\dag  vrî.\\ 
 & \textbf{sus} giengen alle \textbf{sehs\textit{e}} zuo.\\ 
30 & nû hœret, waz ieglîchiu tuo:\\ 
\end{tabular}
\scriptsize
\line(1,0){75} \newline
m n o Fr69 \newline
\line(1,0){75} \newline
\newline
\line(1,0){75} \newline
\textbf{4} röcke] Rock o  $\cdot$ denne] \textit{om.} m  $\cdot$ gras] glas m \textbf{5} Adamach] dem adamas n adamas o \textbf{6} und lanc] wol lang n o \textbf{7} dâ] Do n o \textbf{9} juncvrouwen] jungfrowe o \textbf{10} obe] an o \textbf{11} bluomen] bluͦmelin o \textbf{12} Iwan] jwan m n ie wan o  $\cdot$ Nouel] nauel n [a]: nouel o \textbf{13} Gernis] jernis n iernis o  $\cdot$ Rile] rale m n o \textbf{14} jâ was] Jwas m \textbf{15} dâ] do n o \textbf{16} vürstinne] furstan o  $\cdot$ man] mach m \textbf{17} minniclîcher] wunneclicher n (o) \textbf{18} schindende] snydende n (o) \textbf{19} durch wunder] besunder n \textbf{20} zweien twehelen] zweinen twehelen m zwey twehelin n zwehelin o \textbf{21} diu] Su n (o)  $\cdot$ und] \textit{om.} n \textbf{24} stahel] stahen m \textbf{25} kômen] kuͯne n o \textbf{26} dâ] do n o  $\cdot$ ze] \textit{om.} o  $\cdot$ gegert] begert n o \textbf{28} vor] von o  $\cdot$ iwende] yewende n iewenden o \textbf{29} sehse] sehssen m \newline
\end{minipage}
\end{table}
\newpage
\begin{table}[ht]
\begin{minipage}[t]{0.5\linewidth}
\small
\begin{center}*G
\end{center}
\begin{tabular}{rl}
 & mit \textbf{züht\textit{en}} si ku\textit{n}den wider gên,\\ 
 & zuo den êrsten vieren stên.\\ 
 & an \textbf{den} ahte vrouwen was\\ 
 & röcke grüener danne ein gras,\\ 
5 & von \textbf{Azagouch} samît,\\ 
 & gesniten lanc unde wît.\\ 
 & dâ enmitten si zesamne twanc\\ 
 & \textbf{gürtel \textit{tiur}}, \textbf{smal} unde lanc.\\ 
 & \textbf{di\textit{e}} ahte \textbf{juncvrouwen} kluoc,\\ 
10 & ieslîchiu obe ir hâre truoc\\ 
 & ein kleine \textbf{blüemîn} schapel.\\ 
 & der grâve \textbf{Iwein} von \textbf{Nonel}\\ 
 & unde \textbf{Kernis} von Rile,\\ 
 & \textbf{ez was} über manige mîle\\ 
15 & \textbf{ze dienst} ir tohter dar genomen.\\ 
 & \textbf{man sach die z\textit{w}ô vürstinne} komen\\ 
 & in harte \textbf{wünniclîcher} wât.\\ 
 & zwei mezzer \textbf{snîdende} als ein grât\\ 
 & \textbf{truogen} si durch wunder\\ 
20 & \textbf{in} zwein twehelen \textit{\textbf{al}} \textit{be}sunder;\\ 
 & \textbf{diu wâren} \textbf{von} \textbf{silber} wîz.\\ 
 & dâr an lag ein spæher vlîz:\\ 
 & \textbf{\begin{large}I\end{large}n} was \textbf{ir} \textbf{scherpfe} niht vermiten,\\ 
 & \textbf{si heten} stâl wol \textbf{\textit{v}e\textit{r}sniten}.\\ 
25 & vor \textbf{dem silber} kômen vrouwen wert,\\ 
 & der \textbf{dar ze dienste was} gegert,\\ 
 & die truogen lieht \textbf{dem silber} bî,\\ 
 & vier kint vor missewende vrî.\\ 
 & \textbf{die} giengen alle \textbf{viere} zuo.\\ 
30 & nû hœret, waz iegelîchiu tuo:\\ 
\end{tabular}
\scriptsize
\line(1,0){75} \newline
G I O L M Q R Z Fr21 Fr40 Fr51 \newline
\line(1,0){75} \newline
\textbf{1} \textit{Initiale} I Q Fr21 Fr40  \textbf{3} \textit{Initiale} O Z  \textbf{22} \textit{Initiale} I  \textbf{23} \textit{Initiale} G  \textbf{29} \textit{Initiale} L  \newline
\line(1,0){75} \newline
\textbf{1} zühten] zuht G  $\cdot$ si kunden] si chuden G kvnden sie L (Z) \textbf{3} an] ÷n O  $\cdot$ den] \textit{om.} I  $\cdot$ ahte vrouwen] aht jvngfrowen L vrouwen achte M achten frawen Q (Fr51) \textbf{4} grüener danne] gruͤn als I gruͯnner won R \textbf{5} Azagouch] azagoͮch G azagoͮc I Azagvͦch O azogoúe Q asagouc R azagovc Z Fr21 azago:: Fr40 azagowe Fr51 \textbf{6} lanc] wol lang L (M) (Q) R (Z) (Fr21) (Fr40) \textbf{7} \textit{Versfolge 234.8-7} R   $\cdot$ dâ enmitten] Da mitten O (M) Z (Fr21) Daze siten Q da enreiten Fr40 Dar midden Fr51 \textbf{8} tiur] \textit{om.} G  $\cdot$ smal] \textit{om.} O  $\cdot$ lanc] [lant]: lanc I \textbf{9} die] div G  $\cdot$ juncvrouwen] frowen O \textbf{10} obe] vf I (M) (Fr51) \textbf{11} blüemîn] bluͦmen I (M) (Q) (Fr40) (Fr51) bluͦmi R  $\cdot$ schapel] schaͯplin R zappel Fr51 \textbf{12} der] Die Fr51  $\cdot$ Iwein] iuvain I zwelf O Jwain L yweyn M ywan Q Fr40 Iwan R Jwein Z Lvuel Fr21 ywen Fr51  $\cdot$ Nonel] lonel I (L) novel M nouel Q Nouellin R Nollel Z Nvnel Fr21 navel Fr40 \textbf{13} Kernis] kennis I Gernis O Z (M) (Fr40) Garnis L gerius Q Gerins R gernys Fr51  $\cdot$ Rile] kile G I Fr21 Kẏle O (Fr51) Kýle L kyle Q R Rile Z keyle Fr40 \textbf{14} ez] er I (R)  $\cdot$ was] wart L  $\cdot$ über] aber Q dar vber Fr51  $\cdot$ manige] \textit{om.} O M Fr21 eẏne Fr51 \textbf{15} dar] waren Fr51  $\cdot$ genomen] chomen I \textbf{16} sach] \textit{om.} L  $\cdot$ die] dar Fr51  $\cdot$ zwô] zoͮ G (Fr21)  $\cdot$ vürstinne] jvngfrowen L fᵫrstinen R (Fr51) \textbf{18} zwei] [Z*]: Zwelf O  $\cdot$ mezzer] metzes Fr51  $\cdot$ snîdende] sniten I (M) scarf Fr51  $\cdot$ als] alsam Fr51  $\cdot$ ein] \textit{om.} O \textbf{20} in] Vff Q R An Fr51  $\cdot$ twehelen] taueln I (O) twehel Q tewehlhen Fr21  $\cdot$ al besunder] sunder G besvnder O ab besvnder L al besundern M \textbf{21} \textit{Versfolge 234.22-21} I   $\cdot$ diu wâren] ir snide warn I Daz was O (L) (M) (Q) (R) Z (Fr21) (Fr51)  $\cdot$ von silber] siber harte O silber harte L M (Q) R (Z) (Fr51) \textbf{22} ein spæher] kvnst vnde Fr51 \textbf{23} In] Jm O M Q R Z Fr21 Fr51  $\cdot$ ir] si O solich L Z sin M (Q) R Fr21 Fr51  $\cdot$ niht vermiten] vn vergniten M \textbf{24} si heten] Jz het O (L) (M) (R) (Z) (Fr21) (Fr51) Er het Q  $\cdot$ stâl wol] wol stal I (Fr51)  $\cdot$ versniten] gesniten G geschmitten R \textbf{26} dar] do Q  $\cdot$ was] wart O \textbf{27} \textit{Die Verse 234.27-28 fehlen} I   $\cdot$ die] Sie Fr51  $\cdot$ lieht] lýcht L (M) (Q)  $\cdot$ dem] den Fr51 \textbf{28} vor] \textit{om.} Fr21 \textbf{29} die] Sie O (L) M Z (Fr21) Fr51  $\cdot$ viere] sehse O L (M) Z Fr21 (Fr51) sesthe Q sechssú R  $\cdot$ zuo] do Fr51 \newline
\end{minipage}
\hspace{0.5cm}
\begin{minipage}[t]{0.5\linewidth}
\small
\begin{center}*T
\end{center}
\begin{tabular}{rl}
 & Mit \textbf{zühten} si kunden wider gân,\\ 
 & zuo den êrsten vieren stân.\\ 
 & \hspace*{-.7em}\big| röcke grüener dannein gras\\ 
 & \hspace*{-.7em}\big| an \textbf{disen} ahte vrouwen was,\\ 
5 & von \textbf{Azagouc} samît,\\ 
 & gesniten lanc unde wît.\\ 
 & dâ mitten si zesamne twanc\\ 
 & \textbf{gürtele tiure} unde lanc.\\ 
 & \textbf{dise} ahte \textbf{juncvrouwen} kluoc,\\ 
10 & ieslîch\textit{iu} ob ir hâre truoc\\ 
 & ein kleine \textbf{blüemîn} schapel.\\ 
 & Der grâve \textbf{Iwan} von \textbf{Jonel}\\ 
 & unde \textbf{Scharius} von Rile,\\ 
 & \textbf{ze dienste} über manege mîle\\ 
15 & \textbf{wâren} ir tohtere dar genomen.\\ 
 & \textbf{man sach die zwô vürstinne} komen\\ 
 & in harte \textbf{wünneclîcher} wât.\\ 
 & Zwei mezzer \textbf{snîdende} als ein grât\\ 
 & \textbf{brâhten} si durch wunder\\ 
20 & \textbf{ûf} zwein tweheln \textbf{wîz} besunder;\\ 
 & \textbf{ez was} \textbf{silber} \textbf{harte} wîz.\\ 
 & dâr an lac ein spæher vlîz:\\ 
 & \textbf{im} was \textbf{solh} \textbf{scherpf\textit{e}} niht vermiten,\\ 
 & \textbf{ez hete} stahel wol \textbf{gesniten}.\\ 
25 & Vor \textbf{dem silber} kômen vrouwen wert,\\ 
 & der \textbf{was ze dienste dar} gegert:\\ 
 & \hspace*{-.7em}\big| vier kint vor missewende vrî,\\ 
 & \hspace*{-.7em}\big| die truogen lieht \textbf{dem silber} bî.\\ 
 & \textbf{Sus} giengen \textbf{s}alle \textbf{sehse} zuo.\\ 
30 & nû hœret, waz ieglîchiu tuo:\\ 
\end{tabular}
\scriptsize
\line(1,0){75} \newline
T U V W \newline
\line(1,0){75} \newline
\textbf{1} \textit{Majuskel} T  \textbf{12} \textit{Majuskel} T  \textbf{18} \textit{Majuskel} T  \textbf{25} \textit{Majuskel} T  \textbf{29} \textit{Initiale} W   $\cdot$ \textit{Majuskel} T  \newline
\line(1,0){75} \newline
\textbf{1} zühten] zvht V (W) \textbf{2} êrsten vieren] vieren ersten W \textbf{5} Azagouc] Azagôvc T azaguͦc U azagovg V \textbf{6} lanc] [*]: wol lang V wol lang W \textbf{7} dâ mitten] Do enmitten W \textbf{8} [*]: Túre gúrtele smal vnde lang V  $\cdot$ tiure] teúre schmal W \textbf{9} dise] Die W  $\cdot$ juncvrouwen] frawen W \textbf{10} ieslîchiu] iesliche T \textbf{11} kleine] \textit{om.} W  $\cdot$ blüemîn] bluͦmen U W [blvͦm*]: blvͦmin V \textbf{12} grâve] \textit{om.} W  $\cdot$ Iwan] Jwân T (U) (V) ywan W  $\cdot$ Jonel] Jovel U lonel W \textbf{13} Scharius] scharuͦs U scharios V scharniß W \textbf{16} die] \textit{om.} W \textbf{18} snîdende] schnieden W \textbf{20} tweheln] [twêhel*]: twêhellen V tafeln W  $\cdot$ wîz] \textit{om.} W \textbf{21} [*]: Die waren silberin herte vnde wiz V \textbf{23} im] [Jm]: Jn V  $\cdot$ scherpfe] scherpfiv T [scherph*]: scherphe V \textbf{24} ez hete] [*]: Sv́ hetten V  $\cdot$ gesniten] versniten U (W) \textbf{25} dem silber] den selbin U [der*]: den meseren V \textbf{26} was] [*z]: dar V  $\cdot$ dar] [*]: waz V do W \textbf{27} truogen] tregen U  $\cdot$ dem silber] [*]: den messern V \textbf{29} salle] alle W \textbf{30} ieglîchiu] ieclicher U \newline
\end{minipage}
\end{table}
\end{document}
