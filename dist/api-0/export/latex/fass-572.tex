\documentclass[8pt,a4paper,notitlepage]{article}
\usepackage{fullpage}
\usepackage{ulem}
\usepackage{xltxtra}
\usepackage{datetime}
\renewcommand{\dateseparator}{.}
\dmyyyydate
\usepackage{fancyhdr}
\usepackage{ifthen}
\pagestyle{fancy}
\fancyhf{}
\renewcommand{\headrulewidth}{0pt}
\fancyfoot[L]{\ifthenelse{\value{page}=1}{\today, \currenttime{} Uhr}{}}
\begin{document}
\begin{table}[ht]
\begin{minipage}[t]{0.5\linewidth}
\small
\begin{center}*D
\end{center}
\begin{tabular}{rl}
\textbf{572} & \begin{large}\textbf{I}\end{large}me schilde beleip der vierde vuoz.\\ 
 & \textbf{mit} bluote gab er solhen guz,\\ 
 & daz Gawan \textbf{mohte vaste} stên.\\ 
 & her unt dar begunde\textbf{z} gên.\\ 
5 & Der lewe spranc dicke an den gast;\\ 
 & durch die nasen manegen pfnast\\ 
 & tet er mit bleckenden zenen.\\ 
 & wolte man in solher spîse wenen,\\ 
 & daz er guote liute \textbf{geæze},\\ 
10 & ungern ich bî im sæze.\\ 
 & Ez was ouch Gawane leit,\\ 
 & der ûf den lîp \textbf{dâ mit im} streit.\\ 
 & er hete in sô geletzet,\\ 
 & mit bluote wart \textbf{benetzet}\\ 
15 & al diu kemenâte gar.\\ 
 & mit zorne spranc der lewe dar\\ 
 & \textbf{unt} wolt in zücken under sich.\\ 
 & Gawan \textbf{tet} im einen stich\\ 
 & durch die brust unz an die hant,\\ 
20 & dâ von des lewen zorn verswant,\\ 
 & wander strûchte nider tôt.\\ 
 & Gawan het \textbf{die} grôze nôt\\ 
 & mit strîte überwunden.\\ 
 & \textbf{in} den selben stunden\\ 
25 & dâhter: "waz ist \textbf{mir} nû guot?\\ 
 & ich sitze ungern in ditze bluot.\\ 
 & Ouch sol ich mich des wol bewarn\\ 
 & - diz bette kan sô umbe varn -,\\ 
 & daz ich \textbf{dran} sitze oder lige,\\ 
30 & ob ich rehter \textbf{wîsheit} pflige."\\ 
\end{tabular}
\scriptsize
\line(1,0){75} \newline
D \newline
\line(1,0){75} \newline
\textbf{1} \textit{Initiale} D  \textbf{5} \textit{Majuskel} D  \textbf{11} \textit{Majuskel} D  \textbf{27} \textit{Majuskel} D  \newline
\line(1,0){75} \newline
\newline
\end{minipage}
\hspace{0.5cm}
\begin{minipage}[t]{0.5\linewidth}
\small
\begin{center}*m
\end{center}
\begin{tabular}{rl}
 & \textbf{in} dem schilt bleip der vierde vuoz.\\ 
 & \textbf{mit} bluote gap er solichen guz,\\ 
 & daz Gawan \textbf{vaste mohte} stân.\\ 
 & her und dar begunde \textbf{ez} gân.\\ 
5 & der lewe spranc dicke an den gast;\\ 
 & durch die nase manigen pfnast\\ 
 & tet er mit bleckenden zenen.\\ 
 & wolt man in solicher spîse wenen,\\ 
 & daz er guote liute \textbf{geæze},\\ 
10 & ungerne ich bî im sæze.\\ 
 & ez was ouch Gawan leit,\\ 
 & der ûf den lîp \textbf{d\textit{â} mit im} streit.\\ 
 & er het in sô geletzet,\\ 
 & mit bluote wart \textbf{genetzet}\\ 
15 & aldiu kemenâte gar.\\ 
 & mit \textit{zorne} spranc der lewe dar:\\ 
 & \textbf{er} wolte in zücken under sich.\\ 
 & Gawan \textbf{tet} im einen stich\\ 
 & durch die brust unz an die hant,\\ 
20 & dâ von des lewen zorn verswant,\\ 
 & wan er strûhte nider tôt.\\ 
 & Gawan het \textbf{die} grôze nôt\\ 
 & mit strîte überwunden.\\ 
 & \textbf{in} den selben stunden\\ 
25 & dâht er: "waz ist \textbf{mir} nû guot?\\ 
 & ich sitze ungerne in diz bluot.\\ 
 & ouch sol ich mich des wol bewarn\\ 
 & - di\textit{z} bette kan sô umbe varn -,\\ 
 & daz ich \textbf{dâr} sitz oder lige,\\ 
30 & ob ich rehter \textbf{witze} pflige."\\ 
\end{tabular}
\scriptsize
\line(1,0){75} \newline
m n o \newline
\line(1,0){75} \newline
\newline
\line(1,0){75} \newline
\textbf{1} vierde] werde n \textbf{2} guz] gos o \textbf{3} mohte] moͯchte n \textbf{4} dar] dan n  $\cdot$ begunde] beguͯnde o  $\cdot$ gân] began n \textbf{7} tet] [Der]: Det m  $\cdot$ bleckenden] bleckenen o \textbf{8} wenen] gewenen n \textbf{12} dâ] do m n o \textbf{15} kemenâte] kemnaten n (o) \textbf{16} zorne] \textit{om.} m \textbf{20} dâ] \textit{om.} n \textbf{28} diz] Die m \textbf{29} dâr] dar an n o \newline
\end{minipage}
\end{table}
\newpage
\begin{table}[ht]
\begin{minipage}[t]{0.5\linewidth}
\small
\begin{center}*G
\end{center}
\begin{tabular}{rl}
 & \textbf{\begin{large}A\end{large}n}me schilte beleip der vierde vuoz.\\ 
 & \textbf{mit} bluot gab er solhen guz,\\ 
 & daz Gawan \textbf{begunde vaste} stên.\\ 
 & her unde dar begunde\textbf{z} gên.\\ 
5 & der lewe spranc dicke an den gast;\\ 
 & durch die nase manigen pfnast\\ 
 & tet er mit bleckenden zenen.\\ 
 & wolt man in solher spîse wenen,\\ 
 & daz er guote liute \textbf{geæze},\\ 
10 & ungerne ich bî im sæze.\\ 
 & ez was ouch Gawane leit,\\ 
 & der ûf den lîp \textbf{dâ mit im} streit.\\ 
 & er het in sô geletzet,\\ 
 & mit bluote wart \textbf{benetzet}\\ 
15 & al diu kemenâte gar.\\ 
 & mit zorne spranc der lewe dar\\ 
 & \textbf{unde} wold in zücken under sich.\\ 
 & Gawan \textbf{tet} im einen stich\\ 
 & durch die brust unze an die hant,\\ 
20 & dâ von des lewen zorn verswant,\\ 
 & wan er strûchete nider tôt.\\ 
 & Gawan het \textbf{die} grôzen nôt\\ 
 & mit strîte überwunden.\\ 
 & \textbf{an} den selben stunden\\ 
25 & dâht er: "waz ist \textbf{mir} nû guot?\\ 
 & ich sitze ungerne in ditze bluot.\\ 
 & ouch sol ich mich des wol bewarn\\ 
 & - ditze bette kan sô umbe varn -,\\ 
 & daz ich \textbf{dran} sitze oder lige,\\ 
30 & obe ich rehter \textbf{sinne} pflige."\\ 
\end{tabular}
\scriptsize
\line(1,0){75} \newline
G I L M Z Fr23 \newline
\line(1,0){75} \newline
\textbf{1} \textit{Initiale} G I L Z  \textbf{5} \textit{Initiale} M  \textbf{21} \textit{Initiale} I  \newline
\line(1,0){75} \newline
\textbf{1} Anme] Jn deme M (Z) (Fr23) \textbf{2} mit] von I  $\cdot$ guz] vluͦz I duz Z \textbf{3} begunde] moht L (M) (Z) Fr23 \textbf{4} her unde dar] Dar vnd her Fr23  $\cdot$ begundez] begunde I (L)  $\cdot$ gên] vast gen Fr23 \textbf{5} lewe] \textit{om.} L  $\cdot$ an den] Gein dem I \textbf{6} nase] nasen I L (M) Z \textbf{8} man] er Fr23 \textbf{9} geæze] asze L eze Fr23 \textbf{10} ich bî im] by yme ich M \textbf{11} ez] Er L Fr23  $\cdot$ Gawane] Gawan I Gawanen Fr23 \textbf{12} dâ] \textit{om.} M Z \textbf{13} het] \textit{om.} L \textbf{14} wart] \textit{om.} I  $\cdot$ benetzet] genezzit M [gel*etzet]: genetzet Z \textbf{15} al diu kemenâte] wart al diu kemenate I Die kemenate al L \textbf{17} wold] wol Z \textbf{18} tet] gap Z \textbf{19} unze] bisz M \textbf{24} an] Jn M Z \textbf{25} nû] \textit{om.} M \textbf{26} ditze] disem I \textbf{27} ouch sol ich mich des] ich sol mich des I Oͮh sol ih Fr23 \textbf{29} dran] dryn M  $\cdot$ sitze oder lige] nih sitze noch enlige I sicze adir lege M \textbf{30} obe] Do Fr23  $\cdot$ sinne] witze L wisheit M Z wishei Fr23  $\cdot$ pflige] \textit{om.} Fr23 \newline
\end{minipage}
\hspace{0.5cm}
\begin{minipage}[t]{0.5\linewidth}
\small
\begin{center}*T
\end{center}
\begin{tabular}{rl}
 & \textbf{in}me schilte bleib der vierde vuoz.\\ 
 & \textbf{von} bluote gap er sölhen guoz,\\ 
 & daz Gawan \textbf{mohte vaste} stân.\\ 
 & her unde dar begund \textbf{er} gân.\\ 
5 & \textit{\begin{large}D\end{large}}er lewe spranc dicke an den gast;\\ 
 & durch die nasen manegen pfnast\\ 
 & tet \textit{er} mit bleckenden zenen.\\ 
 & wolte man in sölher spîse wenen,\\ 
 & daz er guote l\textit{iu}te \textbf{æze},\\ 
10 & ungerne ich bî im sæze.\\ 
 & ez was ouch Gawane leit,\\ 
 & der ûf den lîp \textbf{mit im dâ} streit.\\ 
 & er het in sô geletzet,\\ 
 & mit bluote wart \textbf{be\textit{n}etzet}\\ 
15 & al diu kemenâte gar.\\ 
 & mit zorne spranc der lewe dar\\ 
 & \textbf{unde} wolt in zücken under sich.\\ 
 & Gawan \textbf{gap} im einen stich\\ 
 & durch die brust unz an die hant,\\ 
20 & dâ von des lewen zor\textit{n} verswant,\\ 
 & wander strûhte nider tôt.\\ 
 & Gawan hete grôze nôt\\ 
 & mit strîte überwunden.\\ 
 & \textbf{an} den selben stunden\\ 
25 & dâhter: "waz ist \textbf{dir} nû guot?\\ 
 & ich sitze ungerne in diz bluot.\\ 
 & ouch sol ich mich des wol bewarn\\ 
 & - diz bette kan sus umbe varn -,\\ 
 & daz ich \textbf{dran} sitze \textit{oder} lige,\\ 
30 & ob ich rehter \textbf{wîsheit} pflige."\\ 
\end{tabular}
\scriptsize
\line(1,0){75} \newline
T U V W Q R Fr39 \newline
\line(1,0){75} \newline
\textbf{1} \textit{Initiale} Fr39   $\cdot$ \textit{Capitulumzeichen} R  \textbf{5} \textit{Initiale} T  \newline
\line(1,0){75} \newline
\textbf{1} \textit{Die Verse 553.1-599.30 fehlen} U   $\cdot$ \textit{Vers 572.1 ist am Rand nachgetragen und später radiert:} Jn dem schilde bleip der fuz V   $\cdot$ [*]: Jm schilte bleip der fierde fus V \textbf{2} von] [*]: Von V Vor Q  $\cdot$ guoz] [*]: gus V flusz Q \textbf{3} Gawan] Gawin R  $\cdot$ mohte] moͤchte W \textbf{4} begund er] begond ez V (Q) (R) (Fr39) \textbf{5} Der] ÷er T  $\cdot$ an] [*]: an V \textbf{6} nasen] nahen Q  $\cdot$ pfnast] pfanst Q \textbf{7} er] \textit{om.} T  $\cdot$ mit] Jm R  $\cdot$ bleckenden] bleken die R \textbf{8} spîse] spisen R \textbf{9} guote liute] gvͦte l::te T helden R \textbf{11} \textit{Die Verse 572.11-12 fehlen} W   $\cdot$ Gawane] Gawine R G::: Fr39 \textbf{12} mit im dâ] mit im V (R) do mit im Q \textbf{14} benetzet] beswetzet T geneczet R \textbf{15} kemenâte] kamere W \textbf{18} Gawan] Gawin R  $\cdot$ gap] tet V (W) Q R \textbf{20} zorn] zor: T krafft W  $\cdot$ verswant] beswant Q \textbf{21} wander] Wan der R \textbf{22} Gawan] Gawin R  $\cdot$ grôze] grossen W R \textbf{24} an] [*]: An V Jn Q (Fr39) \textbf{25} waz] das Q  $\cdot$ dir nû] [*]: mir nv V mir nun W (Q) nun mir R \textbf{26} ungerne] [*]: ungerne V vnd gern R  $\cdot$ diz] [d*]: daz V \textbf{27} \textit{Vers 572.27 ist am Rand nachgetragen und später radiert:} :l bewarn V   $\cdot$ [*]: Oͮch sol ich mich dez wol bewarn V \textbf{28} \textit{Vers 572.28 ist am Rand nachgetragen und später radiert:} bette :n s: V   $\cdot$ diz] [*]: Diuz V Das Q R d:s Fr39  $\cdot$ sus] als Q  $\cdot$ umbe varn] vmmewarn Q \textbf{29} dran] [*]: dran V  $\cdot$ oder lige] ::: lige T [*]: oder lige V noch enlige Q \textbf{30} wîsheit] [*]: wisheit V wicze R \newline
\end{minipage}
\end{table}
\end{document}
