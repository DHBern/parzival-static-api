\documentclass[8pt,a4paper,notitlepage]{article}
\usepackage{fullpage}
\usepackage{ulem}
\usepackage{xltxtra}
\usepackage{datetime}
\renewcommand{\dateseparator}{.}
\dmyyyydate
\usepackage{fancyhdr}
\usepackage{ifthen}
\pagestyle{fancy}
\fancyhf{}
\renewcommand{\headrulewidth}{0pt}
\fancyfoot[L]{\ifthenelse{\value{page}=1}{\today, \currenttime{} Uhr}{}}
\begin{document}
\begin{table}[ht]
\begin{minipage}[t]{0.5\linewidth}
\small
\begin{center}*D
\end{center}
\begin{tabular}{rl}
\textbf{593} & \begin{large}S\end{large}i \textbf{kômen} die strâzen durch daz muor,\\ 
 & als Lischoys, der stolze, vuor,\\ 
 & den er entschumpfierte.\\ 
 & diu vrouwe condwierte\\ 
5 & \textbf{den} rîter mit dem zoume her;\\ 
 & tjustieren was sîn ger.\\ 
 & Gawan sich umbe kêrte,\\ 
 & sînen kumber er gemêrte;\\ 
 & in dûhte, diu sûl het in betrogen.\\ 
10 & dô sach er vür ungelogen\\ 
 & Orgelusen de Logroys\\ 
 & und einen rîter kurtoys\\ 
 & gein \textbf{dem urvar} ûf \textbf{den} wasen.\\ 
 & Ist \textbf{diu} nieswürze in der nasen\\ 
15 & dræte unde strenge,\\ 
 & \textbf{durch} sîn herze enge\\ 
 & kom alsus diu herzogîn\\ 
 & durch sîniu ougen oben în.\\ 
 & Gein minne helflôs ein man,\\ 
20 & ouwê, daz ist hêr Gawan!\\ 
 & zuo sîner meisterinne er sprach,\\ 
 & dô er den rîter \textbf{komen} sach:\\ 
 & "Vrouwe, dort vert ein \textit{ritter} her\\ 
 & mit ûf gerihtem sper;\\ 
25 & der wil suochens niht erwinden.\\ 
 & \textbf{ouch} sol sîn suochen vinden,\\ 
 & sît er rîterschefte gert,\\ 
 & \textbf{strîtes} \textbf{ist er von mir} gewert.\\ 
 & sagt \textbf{mir}, wer mac diu vrouwe sîn."\\ 
30 & si sprach: "daz ist diu herzogîn\\ 
\end{tabular}
\scriptsize
\line(1,0){75} \newline
D Z Fr7 \newline
\line(1,0){75} \newline
\textbf{1} \textit{Initiale} D Z  \textbf{14} \textit{Majuskel} D  \textbf{19} \textit{Majuskel} D  \textbf{23} \textit{Majuskel} D  \newline
\line(1,0){75} \newline
\textbf{1} strâzen] strazze Z \textbf{2} Lischoys] Liscoys D Lishois Z \textbf{10} sach] gesach Z \textbf{11} Logroys] logrois Z \textbf{16} durch] Jn Z \textbf{19} helflôs] helfelose Z \textbf{20} hêr] min her Z \textbf{21} zuo] Hin zv Z \textbf{22} dô] Da Z \textbf{23} dort] da Z  $\cdot$ ritter] \textit{om.} D \textbf{29} mir] \textit{om.} Z \newline
\end{minipage}
\hspace{0.5cm}
\begin{minipage}[t]{0.5\linewidth}
\small
\begin{center}*m
\end{center}
\begin{tabular}{rl}
 & si \textbf{kômen} die strâzen durch daz muor,\\ 
 & als Lischois, der stolze, vuor,\\ 
 & den er enschumpfieret.\\ 
 & diu vrouwe condwieret\\ 
5 & \textbf{den} ritter mit dem zoume her;\\ 
 & justieren was sîn ger.\\ 
 & Gawan sich umbe kêrte,\\ 
 & sînen kumber er gemêrte;\\ 
 & in dûhte, diu sûle het in betrogen.\\ 
10 & dô sach er vür ungelogen\\ 
 & Urgelusen de Logrois\\ 
 & und eine\textit{n} ritter kurtois\\ 
 & gegen \textbf{dem urvar} ûf \textbf{den} wasen.\\ 
 & ist \textbf{diu} niese\textit{würze} in der nasen\\ 
15 & dræte und strenge,\\ 
 & \textbf{durch} sîn herz enge\\ 
 & kam alsus diu herzogîn\\ 
 & durch sîniu ougen oben în.\\ 
 & gegen minne helflôs ein man,\\ 
20 & owê, daz ist hêr Gawan!\\ 
 & zuo sîner meisterîn er sprach,\\ 
 & dô er den ritter \textbf{wider} sach:\\ 
 & "vrouwe, dort vert ein ritter her\\ 
 & mit ûf geri\textit{h}tem sper;\\ 
25 & der wil suochens niht erwinden.\\ 
 & \textbf{ouch} sol sîn suochen vinden,\\ 
 & sît er ritterschefte gert,\\ 
 & \textbf{strît} \textbf{von mir wirt er} gewert.\\ 
 & saget \textbf{mir}, wer mac diu vrowe sîn."\\ 
30 & si sprach: "daz ist diu herzogîn\\ 
\end{tabular}
\scriptsize
\line(1,0){75} \newline
m n o \newline
\line(1,0){75} \newline
\newline
\line(1,0){75} \newline
\textbf{1} die strâzen] durch stossen o  $\cdot$ muor] mir n \textbf{2} Lischois] liscois m n o \textbf{4} condwieret] conduwiere n Conwieret o \textbf{11} de Logrois] do logroisz n do begrusz o \textbf{12} einen] einem m \textbf{13} urvar ûf den] vnfar vff dem o \textbf{14} niesewürze] niese m \textbf{15} dræte] Detre o  $\cdot$ und] vnd ouch n \textbf{20} Gawan] ga:an o \textbf{21} er] >er< o \textbf{22} ritter wider] ritter komen n komen o \textbf{23} vrouwe] Frowen o \textbf{24} gerihtem] gerittem m gerecktem o \textbf{28} strît] Srit o \newline
\end{minipage}
\end{table}
\newpage
\begin{table}[ht]
\begin{minipage}[t]{0.5\linewidth}
\small
\begin{center}*G
\end{center}
\begin{tabular}{rl}
 & si \textbf{kôm\textit{en}} die strâze durch daz muor,\\ 
 & alsô Lishois, der stolze, vuor,\\ 
 & den er ens\textit{c}hum\textit{pf}ierte.\\ 
 & diu vrouwe cundewierte\\ 
5 & \textbf{einen} rîter mit dem zoume her;\\ 
 & tjostieren was sîn ger.\\ 
 & Gawan sich umbe kêrte,\\ 
 & sînen kumber er gemêrte;\\ 
 & in dûhte, diu sûl het in betrogen.\\ 
10 & dô sach er vür ungelogen\\ 
 & Orgelusen de Logroys\\ 
 & unde einen rîter kurtoys\\ 
 & gein \textbf{dem \textit{u}r\textit{va}r} \textit{û}f \textbf{dem} wasen.\\ 
 & ist \textbf{iu} nieswurz in der nasen\\ 
15 & dræte unde strenge,\\ 
 & \textbf{in} sîn herze enge\\ 
 & kom alsus diu herzogîn\\ 
 & durch sîniu ougen oben în.\\ 
 & gein minne helfelôs ein man,\\ 
20 & owê, daz \textit{ist} hêr Gawan!\\ 
 & \textbf{hin} ze sîner meisterinne er sprach,\\ 
 & dô er den rîter \textbf{komen} sach:\\ 
 & "vrouwe, dort vert ein rîter her\\ 
 & mit ûf gerihtem sper;\\ 
25 & der wil suochens niht erwinden.\\ 
 & \textbf{er} sol sîn suochen vinden,\\ 
 & sît er rîterschefte gert,\\ 
 & \textbf{strîtes} \textbf{ist er von mir} gewert.\\ 
 & saget, wer mac di\textit{u} vrouwe sîn."\\ 
30 & si sprach: "daz ist diu herzogîn\\ 
\end{tabular}
\scriptsize
\line(1,0){75} \newline
G I L M Z \newline
\line(1,0){75} \newline
\textbf{1} \textit{Initiale} L Z  \textbf{7} \textit{Initiale} I  \textbf{27} \textit{Initiale} I  \newline
\line(1,0){75} \newline
\textbf{1} kômen] chom G (L) (M)  $\cdot$ die] ir M  $\cdot$ daz] dy M \textbf{2} Lishois] Liscoys I lýtschoýs L lisois M  $\cdot$ stolze] da I \textbf{3} enschumpfierte] ensthunchierte G \textbf{4} cundewierte] in conduwierte I \textbf{5} einen] Den L (M) Z  $\cdot$ zoume] chom I \textbf{6} tjostieren] Tiostierns I \textbf{7} umbe] kvme vmbe L \textbf{10} dô] Da M  $\cdot$ sach] gesach L Z geschach M \textbf{11} Orgelusen] Orgulusen I Orgelisen L Orgelusin M  $\cdot$ de Logroys] delogrois G de Logroýs L de logrois M \textbf{12} einen] \textit{om.} M \textbf{13} urvar ûf] fuͦr er G  $\cdot$ dem] den I dē M \textbf{14} iu] diu I (L) (Z) \textbf{16} sîn herze] sines herzen I \textbf{18} oben] obnen I \textbf{19} helfelôs ein] eyn hilfelos M helfelose ein Z \textbf{20} daz ist] daz G daz >ist< I  $\cdot$ hêr] myn her M (Z) \textbf{21} hin] \textit{om.} L \textbf{22} dô] Da M Z \textbf{23} dort] da Z  $\cdot$ vert] \textit{om.} L \textbf{24} ûf gerihtem] wol vf gerichtem L syme uff gerichten M \textbf{26} er] Ouch Z  $\cdot$ sîn] \textit{om.} I \textbf{28} strîtes] Stritens L  $\cdot$ ist er] er ist L \textbf{29} diu] die G \newline
\end{minipage}
\hspace{0.5cm}
\begin{minipage}[t]{0.5\linewidth}
\small
\begin{center}*T
\end{center}
\begin{tabular}{rl}
 & si \textbf{kam} die strâze durch daz muor,\\ 
 & als Lyschoys, der stolze, vuor,\\ 
 & den er entschumpfierte.\\ 
 & diu vrouwe condwierte\\ 
5 & \textbf{den} ritter mit dem zoume her;\\ 
 & tjostieren was sîn ger.\\ 
 & Gawan sich umbe kêrte,\\ 
 & sînen kumber er gemêrte;\\ 
 & in dûhte, diu sûle het in betrogen.\\ 
10 & dô sach er vür ungelogen\\ 
 & Orgelusen de Logrois\\ 
 & und einen ritter kurtois\\ 
 & gên \textbf{der var} ûf \textbf{den} wasen.\\ 
 & ist \textbf{diu} ni\textit{es}wurz in der nasen\\ 
15 & dræte und strenge,\\ 
 & \textbf{durch} sîn herze enge\\ 
 & kom alsus diu herzogîn\\ 
 & durch sîniu ougen oben în.\\ 
 & gên minne helfelôs ein man,\\ 
20 & owî, daz ist \textbf{mîn} hêrre Gawan!\\ 
 & \textbf{hin} zuo sîner meisterinne er sprach,\\ 
 & dô er den ritter \textbf{komen} sach:\\ 
 & "vrouwe, dort vert ein ritter her\\ 
 & mit ûf gerihtem sper;\\ 
25 & der wil suochens niht erwinden.\\ 
 & \textbf{er} sol sîn suochen vinden,\\ 
 & sît er ritterschefte gert,\\ 
 & \textbf{strîtes} \textbf{ist er von mir} gewert.\\ 
 & sagt \textbf{mir}, wer mac diu vrouwe sîn."\\ 
30 & si sprach: "daz ist diu herzogîn\\ 
\end{tabular}
\scriptsize
\line(1,0){75} \newline
Q R W V U \newline
\line(1,0){75} \newline
\textbf{1} \textit{Capitulumzeichen} R  \textbf{21} \textit{Initiale} W V  \newline
\line(1,0){75} \newline
\textbf{1} \textit{Die Verse 553.1-599.30 fehlen} U   $\cdot$ kam] [ka*]: kamen V \textbf{2} Lyschoys] lishois Q Lẏshois R lishoys W lischoys V \textbf{5} dem] \textit{om.} R \textbf{6} tjostieren] Stritten R Zuͦ tyostieren W \textbf{7} Gawan] Gawin R \textbf{9} diu sûle het in] in hett die sul R \textbf{11} Orgelusen] Orgulusen R  $\cdot$ de Logrois] delogroy͑s Q de logris R de logroẏs V \textbf{13} Gen dem vrfar (var W ) vff dem wassen R (W) (V) \textbf{14} nieswurz] nienzwurtz Q \textbf{15} und] vnd dar zuͦ W \textbf{18} ougen] \textit{om.} R \textbf{19} helfelôs ein] ein rechtlosen R \textbf{20} owî] Owe R W (V)  $\cdot$ mîn] mine R \textbf{21} hin zuo] ZV W \textbf{23} vrouwe] \textit{om.} R \textbf{24} ûf gerihtem] vffgerichtten R \textbf{25} suochens] suͦchen W \textbf{27} sît] Seit das W \textbf{28} von] vom R \textbf{30} sprach] fprach W \newline
\end{minipage}
\end{table}
\end{document}
