\documentclass[8pt,a4paper,notitlepage]{article}
\usepackage{fullpage}
\usepackage{ulem}
\usepackage{xltxtra}
\usepackage{datetime}
\renewcommand{\dateseparator}{.}
\dmyyyydate
\usepackage{fancyhdr}
\usepackage{ifthen}
\pagestyle{fancy}
\fancyhf{}
\renewcommand{\headrulewidth}{0pt}
\fancyfoot[L]{\ifthenelse{\value{page}=1}{\today, \currenttime{} Uhr}{}}
\begin{document}
\begin{table}[ht]
\begin{minipage}[t]{0.5\linewidth}
\small
\begin{center}*D
\end{center}
\begin{tabular}{rl}
\textbf{560} & \begin{large}W\end{large}elt ir niht erwinden,\\ 
 & mir unt mînen kinden\\ 
 & geschach sô rehte leide nie,\\ 
 & ob ir den \textbf{lîp} verlieset hie.\\ 
5 & Sult \textbf{aber ir} prîs behalten\\ 
 & unt \textbf{dises} landes walten,\\ 
 & sô hât mîn armuot ende.\\ 
 & ich getrouwe \textbf{des} iwerer hende,\\ 
 & \textbf{si hœhe mich mit} rîcheit.\\ 
10 & mit vreuden liep âne leit\\ 
 & mac iwer prîs \textbf{hie} erwerben,\\ 
 & sult ir niht ersterben.\\ 
 & Nû wâpent iuch gein kumber grôz."\\ 
 & dennoch \textbf{was} Gawan alblôz;\\ 
15 & er sprach: "\textbf{traget} mir mînen harnasch her."\\ 
 & der bete was der wirt sîn wer:\\ 
 & von vuoz ûf wâpende in dô gar\\ 
 & diu süeze magt wol gevar.\\ 
 & Der wirt nâch dem orse gienc.\\ 
20 & ein schilt an sîner wende hienc,\\ 
 & der dicke unt \textbf{alsô} herte was,\\ 
 & dâ von doch Gawan sît genas.\\ 
 & schilt und ors im \textbf{wâren} brâht.\\ 
 & der wirt \textbf{was alsô} bedâht,\\ 
25 & daz er wider vür in stuont;\\ 
 & \textbf{Dô sprach er}: "hêrre, \textbf{ich tuon iu} kunt,\\ 
 & wie ir sult gebâren\\ 
 & gein iwers verhes vâren:\\ 
 & mînen schilt sult ir tragen,\\ 
30 & der\textbf{n} ist durchstochen noch \textbf{zerslagen},\\ 
\end{tabular}
\scriptsize
\line(1,0){75} \newline
D \newline
\line(1,0){75} \newline
\textbf{1} \textit{Initiale} D  \textbf{5} \textit{Majuskel} D  \textbf{13} \textit{Majuskel} D  \textbf{19} \textit{Majuskel} D  \textbf{26} \textit{Majuskel} D  \newline
\line(1,0){75} \newline
\newline
\end{minipage}
\hspace{0.5cm}
\begin{minipage}[t]{0.5\linewidth}
\small
\begin{center}*m
\end{center}
\begin{tabular}{rl}
 & welt ir niht erw\textit{i}nden,\\ 
 & mir und mînen kinden\\ 
 & geschach sô rehte leide nie,\\ 
 & ob ir den \textbf{lîp} verlieset \textit{h}ie.\\ 
5 & solt \textbf{aber ir} prîs behalten\\ 
 & und \textbf{des} landes walten,\\ 
 & sô het mîn armuot ende.\\ 
 & ich getriuwe \textbf{des} iuwer hende,\\ 
 & \textbf{si hœhe mich mit} rîcheit.\\ 
10 & mit vröuden liep \dag oder\dag  leit\\ 
 & mac iuwer prîs \textbf{hie} erwerben,\\ 
 & solt ir niht ersterben.\\ 
 & nû wâpent iuch gegen kumber grôz."\\ 
 & dannoch \textbf{was} Gawan alblôz;\\ 
15 & er sprach: "\textbf{t\textit{r}agt} mir mîn harnasch her."\\ 
 & der bete was der wirt sîn wer:\\ 
 & von vuoz ûf wâpent in dô gar\\ 
 & diu süeze maget wol gevar.\\ 
 & der wirt nâch dem ros gienc.\\ 
20 & ein schilt an sîner wende hienc,\\ 
 & der dicke und \textbf{alsô} her\textit{t}e was,\\ 
 & dâ von doch Gawan sît genas.\\ 
 & schilt und ros im \textbf{wurden} brâht.\\ 
 & der wirt \textbf{alsô was} bedâht,\\ 
25 & daz er wider vür in stuont;\\ 
 & \textbf{er sprach}: "hêrre, \textbf{daz tuon ich} kunt,\\ 
 & wie ir sult geb\textit{â}ren\\ 
 & gegen iuwers verhes vâren:\\ 
 & mînen schilt sult ir tragen,\\ 
30 & der ist durchstochen \textit{noch} \textbf{durchslagen},\\ 
\end{tabular}
\scriptsize
\line(1,0){75} \newline
m n o \newline
\line(1,0){75} \newline
\newline
\line(1,0){75} \newline
\textbf{1} erwinden] erwenden m \textbf{3} geschach] Gechach o  $\cdot$ leide] beide o \textbf{4} hie] ẏe m \textbf{5} aber] \textit{om.} n \textbf{6} des] disz n \textbf{8} des] das m \textbf{9} si] So o \textbf{10} oder] vnd n o \textbf{13} wâpent] woppen n \textbf{15} tragt] tagt m  $\cdot$ mîn] minen n \textbf{16} der wirt] sin wirt o \textbf{17} dô] >do< o \textbf{21} herte] herre m her n o  $\cdot$ und] \textit{om.} o \textbf{22} sît] dick o \textbf{24} alsô] alda o \textbf{26} daz tuon ich] ich duͯn úch n (o) \textbf{27} sult] schilt solt o  $\cdot$ gebâren] geberen m \textbf{28} verhes] ferre o \textbf{29} sult] den súllent n \textbf{30} noch] vnd m n \newline
\end{minipage}
\end{table}
\newpage
\begin{table}[ht]
\begin{minipage}[t]{0.5\linewidth}
\small
\begin{center}*G
\end{center}
\begin{tabular}{rl}
 & \begin{large}W\end{large}elt ir niht erwinden,\\ 
 & mir unde mînen kinden\\ 
 & geschach sô \textit{rehte} leide nie,\\ 
 & ob ir den \textbf{lîp} verlieset hie.\\ 
5 & sult \textbf{aber ir} brîs behalten\\ 
 & unde \textbf{dises} landes walten,\\ 
 & sô hât mîn armuot ende.\\ 
 & ich getrouwe \textbf{des} iuwerre hende,\\ 
 & \textbf{sô hœhet sich mîn} rîcheit.\\ 
10 & mit vröuden lieb âne leit\\ 
 & mac iuwer brîs \textbf{hie} erwerben,\\ 
 & sult ir niht ersterben.\\ 
 & nû wâpent iuch gein kumber grôz."\\ 
 & dannoch \textbf{stuont} Gawan al blôz;\\ 
15 & er sprach: "\textbf{traget} mir mîn harnasch her."\\ 
 & der \textit{bete} was der \textit{wirt} sîn wer:\\ 
 & von vuoze ûf wâpent in dô gar\\ 
 & diu süeze maget wol gevar.\\ 
 & der wirt nâch dem orse gienc.\\ 
20 & ein schilt an sîner \textit{w}ende hienc,\\ 
 & der dicke unde \textbf{als} herte was,\\ 
 & dâ von doch Gawan sît genas.\\ 
 & schilt unde ors im \textbf{wâren} brâht.\\ 
 & der wirt \textbf{was alsô} bedâht,\\ 
25 & daz er wider vür in stuont;\\ 
 & \textbf{dô sprach er}: "hêrre, \textbf{ich tuon iu} kunt,\\ 
 & wie ir sult gebâren\\ 
 & gein iuwers verhes vâren:\\ 
 & mînen schilt sult ir tragen,\\ 
30 & der ist durchstochen noch \textbf{\textit{durch}slagen},\\ 
\end{tabular}
\scriptsize
\line(1,0){75} \newline
G I L M Z \newline
\line(1,0){75} \newline
\textbf{1} \textit{Initiale} G L Z  \textbf{11} \textit{Initiale} I  \newline
\line(1,0){75} \newline
\textbf{3} rehte] \textit{om.} G \textbf{9} Si hohe mich mit richeit L (Z)  $\cdot$ Sie hohet mich mit richeit M \textbf{11} hie] \textit{om.} I \textbf{14} stuont] waz L (M) (Z)  $\cdot$ al] \textit{om.} M \textbf{15} traget] nu traget I  $\cdot$ mîn] \textit{om.} L \textbf{16} bete] wirt G  $\cdot$ der] sin M  $\cdot$ wirt] bete G  $\cdot$ wer] [ger]: wer I gewer Z \textbf{17} vuoze] fuͤzen I  $\cdot$ in dô] do in L yn da M (Z) \textbf{20} wende] hende G \textbf{21} herte] gerecht M \textbf{22} doch] \textit{om.} I  $\cdot$ Gawan] gawas Z \textbf{23} im] \textit{om.} L M  $\cdot$ wâren] was I wurden Z \textbf{24} was alsô] also was I \textbf{26} dô] Da M  $\cdot$ hêrre] \textit{om.} L \textbf{28} iuwers] uwir M \textbf{30} ist] en ist L M  $\cdot$ durchstochen noch durchslagen] durch stochen noch zerslagin G zerhawen durc stochen noch durc slagen I \newline
\end{minipage}
\hspace{0.5cm}
\begin{minipage}[t]{0.5\linewidth}
\small
\begin{center}*T
\end{center}
\begin{tabular}{rl}
 & welt ir niht erwinden,\\ 
 & mir unde mînen kinden\\ 
 & geschach sô rehte leide nie,\\ 
 & ob ir den \textbf{prîs} verlieset hie.\\ 
5 & sult \textbf{ir aber} \textbf{den} prîs behalten\\ 
 & unde \textbf{disses} landes walten,\\ 
 & sô het mîn armuot ende.\\ 
 & ich getriuwe \textit{\textbf{es}} iuwerre hende,\\ 
 & \textbf{si hœhe mich mit} rîcheit.\\ 
10 & mit vröuden liep âne leit\\ 
 & mac iuwer prîs erwerben,\\ 
 & sult ir niht ersterben.\\ 
 & Nû wâpent iuch gegen kumber grôz."\\ 
 & dannoch \textbf{saz} Gawan alblôz;\\ 
15 & er sprach: "\textbf{nû} \textbf{bringet} mir mîn harnasch her."\\ 
 & der bete was der wirt sîn wer:\\ 
 & von vuoze ûf wâpent in dô gar\\ 
 & diu süeze magt wol gevar.\\ 
 & Der wirt nâch dem orse gienc.\\ 
20 & ein schilt an sîner wende hienc,\\ 
 & der dicke unde herte was,\\ 
 & dâ von doch Gawan sît genas.\\ 
 & Schilt unde ors im \textbf{wurden} brâht.\\ 
 & der wirt \textbf{was alsô} bedâht,\\ 
25 & da\textit{z} er wider vür in stuont;\\ 
 & \textbf{er sprach}: "hêrre, \textbf{ich tuon iu} kunt,\\ 
 & wie ir sult gebâren\\ 
 & gegen iuwers verhes vâren:\\ 
 & mînen schilt sult ir tragen,\\ 
30 & der \textbf{en}ist durchstochen noch \textbf{durchslagen},\\ 
\end{tabular}
\scriptsize
\line(1,0){75} \newline
T U V W Q R Fr25 Fr39 Fr40 \newline
\line(1,0){75} \newline
\textbf{1} \textit{Initiale} Fr25  \textbf{13} \textit{Initiale} W Fr40   $\cdot$ \textit{Majuskel} T  \textbf{19} \textit{Majuskel} T  \textbf{23} \textit{Majuskel} T  \newline
\line(1,0){75} \newline
\textbf{1} \textit{Die Verse 553.1-599.30 fehlen} U   $\cdot$ erwinden] erwenden W \textbf{3} nie] mie R \textbf{4} prîs] lip V (W) (Q) (R) (Fr25) Fr39 (Fr40) \textbf{5} ir aber] aber ir V W Q R Fr25 Fr39 Fr40  $\cdot$ prîs] [*]: pris V leib W  $\cdot$ behalten] bebalten W [beiagen]: behalden Fr25 \textbf{8} es] \textit{om.} T \textbf{9} hœhe] hohet Q \textbf{11} iuwer] hie v́wer V (W) (Q) (R) (Fr39) (Fr40) iwer hant hîe Fr25 \textbf{12} ersterben] sterben W verderben Fr25 \textbf{13} iuch] iv T \textbf{14} Gawan] Gawin R \textbf{15} nû] \textit{om.} V W Q R Fr25 Fr39 Fr40 \textbf{16} der wirt] sin wúrt V \textbf{17} \textit{Die Verse 560.17-18 fehlen} R   $\cdot$ ûf] \textit{om.} W \textbf{20} wende] hende Q \textbf{22} doch] \textit{om.} V  $\cdot$ Gawan] Gawin R  $\cdot$ sît] \textit{om.} Fr40 \textbf{23} im] \textit{om.} R \textbf{24} was alsô] [*]: also waz V \textbf{25} daz] dar T \textbf{26} iu] ivch T \textbf{28} gegen] \textit{om.} Fr25 :::in Fr39  $\cdot$ verhes] werckes Q (R) (Fr40) \textbf{29} sult ir] ir súlt W den schvlt ir Fr25 \textbf{30} enist] ist Q R Fr25 (Fr40)  $\cdot$ noch] vnd Q  $\cdot$ durchslagen] zerslagen W gesclagen R \newline
\end{minipage}
\end{table}
\end{document}
