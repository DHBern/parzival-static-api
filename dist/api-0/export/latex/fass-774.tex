\documentclass[8pt,a4paper,notitlepage]{article}
\usepackage{fullpage}
\usepackage{ulem}
\usepackage{xltxtra}
\usepackage{datetime}
\renewcommand{\dateseparator}{.}
\dmyyyydate
\usepackage{fancyhdr}
\usepackage{ifthen}
\pagestyle{fancy}
\fancyhf{}
\renewcommand{\headrulewidth}{0pt}
\fancyfoot[L]{\ifthenelse{\value{page}=1}{\today, \currenttime{} Uhr}{}}
\begin{document}
\begin{table}[ht]
\begin{minipage}[t]{0.5\linewidth}
\small
\begin{center}*D
\end{center}
\begin{tabular}{rl}
\textbf{774} & \begin{large}D\end{large}ie vrouwen rûnten \textbf{dâ}, swelch wîp\\ 
 & dâ mite zierte sînen lîp,\\ 
 & het \textbf{er} gein \textbf{ir} gewenket,\\ 
 & sô wære \textbf{sîn} prîs verkrenket.\\ 
5 & etslîchiu was im doch sô holt,\\ 
 & si hete sîn dienst wol gedolt,\\ 
 & ich wæne durch sîn\textit{iu} \textbf{werdiu} mâl.\\ 
 & Gramoflanz, Artus unt Parzival\\ 
 & unt der wirt Gawan,\\ 
10 & die viere giengen \textbf{sunder} dan.\\ 
 & den vrouwen wart bescheiden\\ 
 & in ir pflege der rîche heiden.\\ 
 & Artus warp eine hôchgezît,\\ 
 & daz diu des morgens âne strît\\ 
15 & ûf dem velde ergienge,\\ 
 & daz man dâ mite enpfienge\\ 
 & sînen neven Feirefiz.\\ 
 & "an den \textbf{gewerp kêrt} iwern vlîz\\ 
 & unt iwer besten witze,\\ 
20 & daz er mit uns besitze\\ 
 & ob der tavelrunder."\\ 
 & Si lobten al besunder,\\ 
 & si wurbenz, wærez im niht leit.\\ 
 & dô lobt \textbf{in} gesellecheit\\ 
25 & Feirefiz, der rîche.\\ 
 & daz volc vuor al gelîche,\\ 
 & dô man \textbf{geschancte}, an \textbf{ir} gemach.\\ 
 & \textbf{maneges} vreude aldâ geschach.\\ 
 & smorgens, ob ich \textbf{sô} sprechen mac,\\ 
30 & dô erschein der süeze \textbf{mære tac}.\\ 
\end{tabular}
\scriptsize
\line(1,0){75} \newline
D Fr2 \newline
\line(1,0){75} \newline
\textbf{1} \textit{Initiale} D  \textbf{22} \textit{Majuskel} D  \newline
\line(1,0){75} \newline
\textbf{7} sîniu] [sint]: sine D \textbf{8} Parzival] Parcifal D \textbf{24} lobt] gelobt er Fr2 \newline
\end{minipage}
\hspace{0.5cm}
\begin{minipage}[t]{0.5\linewidth}
\small
\begin{center}*m
\end{center}
\begin{tabular}{rl}
 & die vrowen rûnden \textbf{d\textit{â}}, welich wîp\\ 
 & dâ mit zierte sînen lîp,\\ 
 & het \textbf{er} gegen \textbf{\textit{de}r} gewenket,\\ 
 & sô wær \textbf{sîn} prîs verkrenket.\\ 
5 & etlîchiu was im doch sô holt,\\ 
 & si het sîn dienst wol gedolt,\\ 
 & ich wæne durch sîn\textit{iu} \textbf{vremdiu} mâl.\\ 
 & Gramolanz, Artus und Parcifal\\ 
 & und \textbf{ouch} der wirt \textbf{hêr} Gawan,\\ 
10 & die vier giengen \textbf{wider} dan.\\ 
 & den vrowen wart bescheiden\\ 
 & i\textit{n} ir pflege der rîche heiden.\\ 
 & Artus warp ein hôchgezît,\\ 
 & daz diu des morgens âne strît\\ 
15 & ûf dem velde ergieng\textit{e},\\ 
 & daz man dâ mit enpfieng\textit{e}\\ 
 & sînen neven Ferefiz.\\ 
 & "an den \textbf{kêret} iuwern vlîz\\ 
 & und iuwer besten witz,\\ 
20 & daz er mit uns besitz\\ 
 & o\textit{b} der tavelrunder."\\ 
 & si lobten alle besunder,\\ 
 & si wurbenz, wær ez im niht leit.\\ 
 & dô lobte \textbf{in} gesellicheit\\ 
25 & Ferefiz, der rîch.\\ 
 & daz volc vuor al glîch,\\ 
 & dô man \textbf{geschancte}, an \textbf{ir} gemach.\\ 
 & \textbf{manigiu} vröude aldâ geschach.\\ 
 & des morgens, ob ich \textbf{ez} sprech\textit{en} mac,\\ 
30 & dô erschein der süeze \textbf{mære tac}.\\ 
\end{tabular}
\scriptsize
\line(1,0){75} \newline
m n o V V' W Fr6 \newline
\line(1,0){75} \newline
\textbf{1} \textit{Majuskel} Fr6  \textbf{11} \textit{Initiale} W  \newline
\line(1,0){75} \newline
\textbf{1} \textit{Die Verse 774.1-21 fehlen} V'   $\cdot$ rûnden] fromden n run:en o ruͦmten W  $\cdot$ dâ] do m n o V W  $\cdot$ welich] swelich V (Fr6) \textbf{2} zierte] [zierten]: zierte V zymierte W \textbf{3} der] mir m \textbf{4} sîn] myn her o  $\cdot$ verkrenket] gekrencket W \textbf{5} etlîchiu] etsliche Fr6  $\cdot$ im] s: o \textbf{6} sîn] sinen V (W) \textbf{7} sîniu] sinen m (o) sine Fr6  $\cdot$ vremdiu] fremden o (Fr6) \textbf{8} Gramolanz Artus] Gramolantz artus m n Gramolancz artusz o Artus Gramaflanz V Gramoflantz artus W Gramovlanz artvs Fr6  $\cdot$ Parcifal] parzefal V partzifal W \textbf{9} ouch] \textit{om.} W \textbf{10} vier] werden vie o  $\cdot$ wider] sunder n o (V) W (Fr6)  $\cdot$ dan] [wan]: dan V wan W \textbf{11} wart] war o \textbf{12} in] Jr m \textbf{13} Artus] Artuͯs o \textbf{15} ergienge] ergingen m n o \textbf{16} enpfienge] enpfingen m n o \textbf{17} Ferefiz] ferefis m o ferrefis n ferevis V ferafis W \textbf{18} den] den gewerp V Fr6 \textbf{19} besten] beste V W \textbf{21} ob der] Oder m Ob o \textbf{22} \textit{statt 774.22-27:} Sie hetten die nacht freuden vil / Tamburen vnd seiten spil / Treip man do allegeliche / Ferevis den riche / Man do furte an sin gemach V'   $\cdot$ lobten] lobetens V \textbf{23} wurbenz] wúrbens n wurben o  $\cdot$ wær ez im niht leit] ferres ẏm sin leit o \textbf{24} lobte] lehte W \textbf{25} Ferefiz] Ferefis m o Ferrefis n Ferevis V Ferafiß W  $\cdot$ der] den V' \textbf{26} daz] Do W  $\cdot$ al] alle n \textbf{28} manigiu] Maniges V  $\cdot$ aldâ] do V' W \textbf{29} ob] als W  $\cdot$ ez] \textit{om.} W  $\cdot$ sprechen] sprech m \textbf{30} süeze] suͤszen V liechte V'  $\cdot$ mære tac] meyetag W \newline
\end{minipage}
\end{table}
\newpage
\begin{table}[ht]
\begin{minipage}[t]{0.5\linewidth}
\small
\begin{center}*G
\end{center}
\begin{tabular}{rl}
 & \begin{large}D\end{large}ie vrouwen rûnden \textbf{dâ}, swelch wîp\\ 
 & dâ mit zierte sînen lîp,\\ 
 & het \textbf{der} gein \textbf{ir} gewenket,\\ 
 & sô wære \textbf{in ir} brîs verkrenket.\\ 
5 & etslîchiu was im doch sô holt,\\ 
 & si het sînen dienst wol gedolt,\\ 
 & ich wæne durch sîniu \textbf{vrömdiu} mâl.\\ 
 & Gramoflanz, Artus unde Parcival\\ 
 & unde der wirt Gawan,\\ 
10 & die viere giengen \textbf{sunder} dan.\\ 
 & den vrouwen wart bescheiden\\ 
 & in ir pflege der rîche heiden.\\ 
 & Artus warp eine hôchzît,\\ 
 & daz diu des morgens âne strît\\ 
15 & ûf dem velde ergienge,\\ 
 & daz man dâ mit enpfienge\\ 
 & sînen neven Feirafiz.\\ 
 & "an den \textbf{gewerp leget} iuren vlîz\\ 
 & unde iuwer beste witze,\\ 
20 & daz er mit uns besitze\\ 
 & obe der tavelrunder."\\ 
 & si lobten\textbf{z} alle besunder,\\ 
 & si wurbenz, wære ez im niht leit.\\ 
 & dô lobte \textbf{in} gesellecheit\\ 
25 & Feirafiz, der rîche.\\ 
 & daz volc vuor al gelîche,\\ 
 & dô man \textbf{geschancte}, an \textbf{sîn} gemach.\\ 
 & \textbf{maniges} vröude al dâ geschach.\\ 
 & des morgens, obe ich \textbf{sô} sprechen mac,\\ 
30 & dô erschein der süeze \textbf{sumertac}.\\ 
\end{tabular}
\scriptsize
\line(1,0){75} \newline
G I L M Z Fr18 \newline
\line(1,0){75} \newline
\textbf{1} \textit{Initiale} G I L M Z Fr18  \textbf{19} \textit{Initiale} I  \newline
\line(1,0){75} \newline
\textbf{1} swelch] welch L \textbf{2} Beide ir muͯt vnd ir lip L \textbf{3} der] er I (M) Z  $\cdot$ ir] \textit{om.} L \textbf{4} in ir] ir I ez in ir L sin M Z Fr18  $\cdot$ brîs] prise L \textbf{5} doch] da L \textbf{7} sîniu vrömdiu] sinen fromden I \textbf{8} Gramoflanz] Gramoflantz Z Fr18  $\cdot$ unde] \textit{om.} L  $\cdot$ Parcival] parcifal G Z Fr18 parzifal I L M \textbf{10} giengen] gienge M \textbf{12} rîche] \textit{om.} L  $\cdot$ heiden] heide M \textbf{15} dem] dein L \textbf{17} Feirafiz] feirefiz G Z ferefiz L feirefisz M feẏrefẏz Fr18 \textbf{19} beste] besten M \textbf{20} daz] vnd daz I  $\cdot$ besitze] gesýtze L \textbf{22} lobtenz alle] loptan al I \textbf{23} leit] zcu leit M \textbf{24} dô] Da M Z  $\cdot$ lobte] lopt I L (Z) Fr18  $\cdot$ in] in da I yme M \textbf{25} Feirafiz] Feirefiz G Ferefiz L Z Feirefisz M Feẏrefẏz Fr18 \textbf{27} dô] Da M \textbf{28} maniges] manc I (M)  $\cdot$ al] \textit{om.} I L \textbf{29} sô] daz L \textbf{30} dô] Da M Z  $\cdot$ süeze] svzzen Z  $\cdot$ sumertac] dach L mere tac M Z (Fr18) \newline
\end{minipage}
\hspace{0.5cm}
\begin{minipage}[t]{0.5\linewidth}
\small
\begin{center}*T
\end{center}
\begin{tabular}{rl}
 & \begin{large}D\end{large}ie vrouwen rûnden, welch wîp\\ 
 & dâ mite zierte sînen lîp,\\ 
 & het \textbf{er} gein \textbf{ir} gewenket,\\ 
 & sô wære \textbf{sîn} prîs verkrenket.\\ 
5 & etslîchiu was im doch sô holt,\\ 
 & si hete sînen dienst wol gedolt,\\ 
 & ich wæne durch sîniu \textbf{vremediu} mâl.\\ 
 & Gramoflanz, Artus und Parcifal\\ 
 & und der wirt Gawan,\\ 
10 & die viere giengen \textbf{sunder} dan.\\ 
 & den vrouwen wart bescheiden\\ 
 & in ir pflege der rîche heiden.\\ 
 & Artus warp eine hôchzît,\\ 
 & daz diu des morgens âne strît\\ 
15 & ûf dem velde ergienge,\\ 
 & daz man dâ mit enpfienge\\ 
 & sînen neven Ferefis.\\ 
 & "an den \textbf{gewerp leget} iuwern vlîz\\ 
 & und iuwer beste witze,\\ 
20 & daz er mit uns besitze\\ 
 & ob der tavelrunder."\\ 
 & si lobeten \textbf{ez} alle besunder,\\ 
 & si wurben ez, wære ez im niht leit.\\ 
 & dô lobet \textbf{im} gesellecheit\\ 
25 & Ferefis, der rîche.\\ 
 & daz volc vuor a\textit{l} gelîche,\\ 
 & dô man \textbf{gesante}, an \textbf{sîn} gemach.\\ 
 & \textbf{manegiu} vreude al dâ geschach.\\ 
 & des morgens, ob ich sprechen mac,\\ 
30 & dô erschein der süeze \textbf{mære tac}.\\ 
\end{tabular}
\scriptsize
\line(1,0){75} \newline
U Q R Fr53 \newline
\line(1,0){75} \newline
\textbf{1} \textit{Initiale} U  \newline
\line(1,0){75} \newline
\textbf{1} \textit{Die Verse 764.13-774.30 fehlen} R   $\cdot$ Die] Eya Q  $\cdot$ rûnden] rúnden da Q prvͤften Fr53  $\cdot$ welch] swelch Fr53 \textbf{2} sînen] iren Q \textbf{4} verkrenket] gekrencket Q \textbf{8} Gramoflanz] Gramoflantz Q  $\cdot$ Parcifal] partzifal Q \textbf{10} giengen] geingen Q \textbf{13} warp] wart Q \textbf{15} ergienge] ergingen Q \textbf{17} Ferefis] feirefisz Q \textbf{22} alle] all Q \textbf{23} wurben ez] wurbes Q \textbf{24} im] in Q \textbf{25} Ferefis] feirefisz Q \textbf{26} al] alle U \textbf{27} gesante] geschanet Q \textbf{28} manegiu] Manigen Q \textbf{29} ich] ich so Q \textbf{30} der süeze mære] des suszen meres Q \newline
\end{minipage}
\end{table}
\end{document}
