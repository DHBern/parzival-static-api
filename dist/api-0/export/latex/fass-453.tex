\documentclass[8pt,a4paper,notitlepage]{article}
\usepackage{fullpage}
\usepackage{ulem}
\usepackage{xltxtra}
\usepackage{datetime}
\renewcommand{\dateseparator}{.}
\dmyyyydate
\usepackage{fancyhdr}
\usepackage{ifthen}
\pagestyle{fancy}
\fancyhf{}
\renewcommand{\headrulewidth}{0pt}
\fancyfoot[L]{\ifthenelse{\value{page}=1}{\today, \currenttime{} Uhr}{}}
\begin{document}
\begin{table}[ht]
\begin{minipage}[t]{0.5\linewidth}
\small
\begin{center}*D
\end{center}
\begin{tabular}{rl}
\textbf{453} & \begin{large}S\end{large}wer mich \textbf{dar von} \textbf{ê} vrâgte\\ 
 & unt dâr umbe mit mir bâgte,\\ 
 & ob ichs im niht sagte,\\ 
 & unprîs \textbf{der} dran bejagte.\\ 
5 & mich batz heln Kyot,\\ 
 & wand \textbf{im} diu âventiure gebôt,\\ 
 & daz es immer \textbf{man} gedæhte,\\ 
 & ê ez diu âventiure bræhte\\ 
 & mit worten an der mære gruoz,\\ 
10 & daz man dar von \textbf{doch} sprechen muoz.\\ 
 & Kyot, der meister wol bekant,\\ 
 & ze Dolet verworfen ligen vant\\ 
 & in heidenscher schrifte\\ 
 & dirre âventiure \textbf{gestifte}.\\ 
15 & \textbf{der} karacter a b c\\ 
 & muoser hân gelernet ê,\\ 
 & âne den list von nigromanzî.\\ 
 & ez half, daz im der touf was bî.\\ 
 & anders \textbf{wære diz mære} \textbf{noch} unvernumen.\\ 
20 & dehein heidensch list moht gevrumen,\\ 
 & ze künden umbes Grâles art,\\ 
 & wie man sîner tougen innen wart.\\ 
 & ein heiden Flegetanis\\ 
 & bejagte an \textbf{künste} hôhen prîs.\\ 
25 & der selbe fisîôn\\ 
 & was geborn von Salomon,\\ 
 & ûz israhelischer \textbf{sippe} \textbf{erzilt}\\ 
 & von alter her, unz unser schilt\\ 
 & der touf \textbf{wart} \textbf{vürz helle} viur.\\ 
30 & der schreip von\textbf{s} Grâles âventiur.\\ 
\end{tabular}
\scriptsize
\line(1,0){75} \newline
D Fr5 \newline
\line(1,0){75} \newline
\textbf{1} \textit{Initiale} D Fr5  \textbf{11} \textit{Capitulumzeichen} Fr5  \newline
\line(1,0){75} \newline
\textbf{2} Vnd mit mir darvmbe bagite Fr5 \textbf{3} ichs] ich ez Fr5 \textbf{4} der] er Fr5 \textbf{7} es] ers Fr5 \textbf{8} ez] [eh]: ez D  $\cdot$ bræhte] brahte Fr5 \textbf{9} worten] worte Fr5 \textbf{10} doch] nv Fr5 \textbf{11} Kyot] Kyoth Fr5  $\cdot$ der] ein Fr5 \textbf{12} Dolet] doleth Fr5 \textbf{13} schrifte] gischrifte Fr5 \textbf{15} der] Den Fr5 \textbf{16} muoser] Muͦes er Fr5 \textbf{19} diz] daz Fr5  $\cdot$ noch unvernumen] niht virnomin Fr5 \textbf{20} dehein] Gein Fr5  $\cdot$ moht] moht vns Fr5 \textbf{21} künden] kunde Fr5 \textbf{23} heiden] heidin hiez Fr5  $\cdot$ Flegetanis] Flegetanîs D \textbf{26} von] \sout{was} von Fr5 \textbf{27} Von israhelishir diet irzilt Fr5 \textbf{29} wart] was Fr5 \textbf{30} vons] von Fr5 \newline
\end{minipage}
\hspace{0.5cm}
\begin{minipage}[t]{0.5\linewidth}
\small
\begin{center}*m
\end{center}
\begin{tabular}{rl}
 & wer mich \textbf{dâ von} vrâgete\\ 
 & und dâr umb mit mir bâgete,\\ 
 & ob ich es ime niht \textbf{en}sagete,\\ 
 & unprîs \textbf{er} dâr an bejagete.\\ 
5 & mich bat ez heln Kiot,\\ 
 & wan \textbf{im} diu âventiure gebôt,\\ 
 & daz es iemer \textbf{man} gedæht,\\ 
 & ê ez diu âventiure bræht\\ 
 & mit worten an der mære gruoz,\\ 
10 & daz man dâ von \textbf{doch} sprech\textit{en} muoz.\\ 
 & Kiot, der meister wol bekant,\\ 
 & zuo Dolet verworfen ligen vant\\ 
 & in heidenischer schrift\\ 
 & \textit{diser âventiure \textbf{gestift}}.\\ 
15 & \textbf{den} karacter a b c\\ 
 & muos er hân gelernet ê,\\ 
 & âne den list von nigromanzî.\\ 
 & ez half, daz im der touf was bî.\\ 
 & anders \textbf{diz mær wær} unvernomen.\\ 
20 & kein heidensch list m\textit{o}ht \textbf{uns} \textbf{niht} gevromen,\\ 
 & zuo künden umb des Grâles \textit{a}rt,\\ 
 & wie man sîner tougen innen wart.\\ 
 & ein heiden \textbf{hiez} Flegetanis,\\ 
 & \textbf{der} bejagete an \textbf{kunst} hôhen prîs.\\ 
25 & der selbe \textbf{wîse} fisîôn\\ 
 & was geborn von Sal\textit{o}mon,\\ 
 & û\textit{z} isra\textit{he}lischer \textbf{sippe} \textbf{gez\textit{i}lt}\\ 
 & von alter her, unz unser schilt\\ 
 & der touf  \textbf{vor dem helschen} viur.\\ 
30 & der schreip von Grâles âventiur.\\ 
\end{tabular}
\scriptsize
\line(1,0){75} \newline
m n o \newline
\line(1,0){75} \newline
\textbf{1} \textit{Illustration mit Überschrift:} Also parcifal gen treuriende dem einsidel kam m  Also parcifal gon treurizende zuͯ dem einsydel kam in den walt geritten n   $\cdot$ \textit{Großinitiale} n   $\cdot$ \textit{Initiale} o  \textbf{23} \textit{Illustration mit Überschrift:} Also ein heiden genant flegetanis eyn kalp fur sinen got anbettet o   $\cdot$ \textit{Überschrift:} Also ein heiden genant flegetanis ein kalp fúr sinen got an bettete n   $\cdot$ \textit{Initiale} n o  \newline
\line(1,0){75} \newline
\textbf{1} wer] DEr n o  $\cdot$ vrâgete] E frogete n (o) \textbf{4} unprîs] An pris o \textbf{5} Kiot] kẏot m n o \textbf{8} ê] Je o \textbf{10} sprechen] sprech m \textbf{11} Kiot] Kyot m Kẏot n o \textbf{14} \textit{Vers 453.14 fehlt} m   $\cdot$ diser] Dise o \textbf{16} muos] Muͯsz n \textbf{19} diz] dise n  $\cdot$ unvernomen] vernomen n \textbf{20} moht] moͯht m (n)  $\cdot$ niht] \textit{om.} n o  $\cdot$ gevromen] [gefrowen]: gefromen o \textbf{21} art] ort m \textbf{26} Salomon] salamon m o \textbf{27} ûz] Vff m  $\cdot$ israhelischer] ysralscher m ysrahelscher n israhelsche o  $\cdot$ gezilt] gezelt m erzelt n \textbf{29} vor dem] fúr das n (o)  $\cdot$ helschen] helle n helse o \textbf{30} Grâles] des groles n \newline
\end{minipage}
\end{table}
\newpage
\begin{table}[ht]
\begin{minipage}[t]{0.5\linewidth}
\small
\begin{center}*G
\end{center}
\begin{tabular}{rl}
 & \begin{large}S\end{large}wer mich \textbf{drumbe} vrâgete\\ 
 & unt drumbe mit mir bâgete,\\ 
 & ob ichs im niht sagete,\\ 
 & unbrîs \textbf{er} dran bejagete.\\ 
5 & \textit{m}ich bat ez helen Kiot,\\ 
 & wande \textbf{im} diu âventiure gebôt,\\ 
 & daz es immer \textbf{man} ged\textit{æ}hte,\\ 
 & ê ez diu âventiure bræhte\\ 
 & mit worten an der mære gruoz,\\ 
10 & daz man dar von \textbf{nû} sprechen muoz.\\ 
 & Kiot, der meister wol bekant,\\ 
 & ze Dolet verworfen l\textit{i}gen vant\\ 
 & in heidenischer schrifte\\ 
 & dirre âventiure \textbf{stifte}.\\ 
15 & \textbf{der} karacter a b c\\ 
 & muose er haben gelernet ê,\\ 
 & âne den list von nigromanzî.\\ 
 & ez half, daz im der touf was bî.\\ 
 & anders \textbf{wære ditze mære} \textbf{noch} unvernomen.\\ 
20 & nehein heidenischer list moht \textbf{uns} gevromen\\ 
 & ze künden umb de\textit{s} Grâles art,\\ 
 & wie man sîner tougen innen wart.\\ 
 & ein heiden Fleigetanis\\ 
 & bejagete an \textbf{künste} hôhen brîs.\\ 
25 & der selbe fisîôn\\ 
 & was geborn von Salmon,\\ 
 & ûz israhelischer \textbf{diet} \textbf{erz\textit{i}lt}\\ 
 & von alter her, unze unser schilt\\ 
 & der touf \textbf{was} \textbf{vür daz helle} viur.\\ 
30 & der schreip von\textbf{s} Grâles âventiur.\\ 
\end{tabular}
\scriptsize
\line(1,0){75} \newline
G I O L M Z \newline
\line(1,0){75} \newline
\textbf{1} \textit{Überschrift:} Hie ist parcifal zv dem klosener zv fontane komen der sagt im alle gelegenheit vmb den gral wie er dar zv mvzze oder komen mvge Z   $\cdot$ \textit{Initiale} G O L Z  \textbf{15} \textit{Initiale} I  \newline
\line(1,0){75} \newline
\textbf{1} Swer] ÷wer O WEr L (M)  $\cdot$ drumbe] da von ê O (L) (Z) da vone y M  $\cdot$ vrâgete] vragit M \textbf{2} \textit{Vers 453.2 fehlt} O   $\cdot$ vnd mit mir dar vmb bagete (E bagte L ) I (L)  $\cdot$ bâgete] bâget G (M) \textbf{3} ichs im] ich ims O ich yme M (Z)  $\cdot$ niht] icht M \textbf{4} unbrîs] Vnde pris M  $\cdot$ er] ich I der L Z \textit{om.} M \textbf{5} mich] Dich G  $\cdot$ ez] des I  $\cdot$ Kiot] kyot O (L) M Z \textbf{7} es] ers O er L  $\cdot$ immer] myner M  $\cdot$ man gedæhte] [g]: man gedahte G \textbf{8} ê] E daz L Er M  $\cdot$ bræhte] brachte L \textbf{9} gruoz] [groch]: groͮz G \textbf{10} daz] da I  $\cdot$ dar von nû] nu von I da von doch O (L) (M) Z \textbf{11} Kiot] Kyot O M Z Kýot L \textbf{12} Dolet] dolêt G Tolet L  $\cdot$ verworfen] verworfenz I  $\cdot$ ligen] liegen G \textbf{14} stifte] Geshihte I (L) gestifte O (M) Z \textbf{15} der] Die I \textbf{16} muose] Muͯste L  $\cdot$ haben gelernet] [gelert]: gelernt haben I \textbf{17} list von] listen M \textbf{18} ez] in I  $\cdot$ der] dy M \textbf{19} ditze] daz I O  $\cdot$ unvernomen] vornommen M \textbf{20} nehein heidenischer] der haidnisch I Dehein heidenichs O Deheinsz L Nyhein heidinsch M (Z) \textbf{21} des] den G \textbf{22} tougen] togint M \textbf{23} Fleigetanis] [flegetanis]: fleigetanis G flegitenis I Flegetanis O (M) (Z) Flegetanisz L \textbf{24} bejagete] beiagt I (O)  $\cdot$ künste] kunsten Z \textbf{26} Salmon] salomon I (L) (M) Salemon O \textbf{27} israhelischer] israhels O ýszrahels L israheilschir M  $\cdot$ diet] sippe O L M Z  $\cdot$ erzilt] [erzat]: erzalt G gezcilt M \textbf{28} unze] \textit{om.} I vntz an Z \textbf{29} was] wart O L M Z  $\cdot$ vür daz] vur I furbasz M \textbf{30} vons] von Z \newline
\end{minipage}
\hspace{0.5cm}
\begin{minipage}[t]{0.5\linewidth}
\small
\begin{center}*T
\end{center}
\begin{tabular}{rl}
 & \begin{Large}S\end{Large}wer mich \textbf{dar von} \textbf{ê} vrâgete\\ 
 & unde drumbe mit mir bâgete,\\ 
 & ob ich\textit{s} im niht \textbf{en}sagete,\\ 
 & unprîs \textbf{er} dran bejagete.\\ 
5 & mich bat ez heln Kyot,\\ 
 & wand\textbf{in} diu âventiure gebôt,\\ 
 & daz \textbf{er}s iemer \textbf{manne} gedæhte,\\ 
 & ê ez diu âventiure bræhte\\ 
 & mit worten an der mære gruoz,\\ 
10 & daz man\textbf{z} dâ von \textbf{doch} sprechen muoz.\\ 
 & Kyot, der meister wol bekant,\\ 
 & ze Dolet verworfen ligen vant\\ 
 & in heidenscher schrifte\\ 
 & dirre âventiure \textbf{gestifte}.\\ 
15 & \textbf{den} karacter a b c\\ 
 & muoser hân gelernet ê,\\ 
 & âne den list von nigromanzî.\\ 
 & ez half \textbf{i\textit{m}}, daz im der touf was bî.\\ 
 & anders \textbf{wære diz mære} unvernomen.\\ 
20 & dehein heidensch list moht \textbf{uns} gevromen,\\ 
 & \hspace*{-.7em}\big| wie man sîner tougen innen wart\\ 
 & \hspace*{-.7em}\big| ze kündenne umbe des Grâles art.\\ 
 & \textit{\begin{large}E\end{large}}in heiden \textbf{hiez} Flegetanis,\\ 
 & \textbf{der} bejagete an \textbf{künsten} hôhen prîs.\\ 
25 & der selbe fisîôn\\ 
 & was geborn von Salomon,\\ 
 & ûz israhelischer \textbf{sippe} \textbf{gezilt}\\ 
 & von alter her, unz unser schilt\\ 
 & der touf \textbf{wart} \textbf{vür daz helle} viur.\\ 
30 & der schreip von \textbf{des} Grâles âventiur.\\ 
\end{tabular}
\scriptsize
\line(1,0){75} \newline
T U V W Q R \newline
\line(1,0){75} \newline
\textbf{1} \textit{Überschrift:} Awentewr wie partzifal bericht wart vmb den gral Q   $\cdot$ \textit{Großinitiale} T Q R   $\cdot$ \textit{Initiale} V W  \textbf{23} \textit{Initiale} T  \newline
\line(1,0){75} \newline
\textbf{1} \textit{Die Verse 453.1-502.30 fehlen} U   $\cdot$ Swer] WEr W (Q) (R)  $\cdot$ von] vor Q  $\cdot$ ê] \textit{om.} R \textbf{2} bâgete] bagate R \textbf{3} ichs im] ichz im T ichs nun Q ich ims R  $\cdot$ ensagete] sagte W Q R \textbf{4} er dran] der dar an W (R) dran ich Q \textbf{5} Jch wil nomen kyot R  $\cdot$ Kyot] Kŷot T \textbf{6} Wan der nun durch der auentúr gebot R  $\cdot$ wandin] Wan im V (W) Wann Q \textbf{7} ers] [*]: ers V es W \textit{om.} R  $\cdot$ manne] man V W R  $\cdot$ gedæhte] gedachte Q \textbf{8} bræhte] brachte Q \textbf{10} manz] men V (W) (Q) (R)  $\cdot$ doch] \textit{om.} Q \textbf{11} Kyot] Koyt Q \textbf{12} ligen] \textit{om.} W \textbf{13} schrifte] geschrifftte R \textbf{14} dirre] Dise Q Der R  $\cdot$ gestifte] gifftte R \textbf{15} den] Der W Q R \textbf{16} muoser] mveser T Mvͤst er V Mustu Q  $\cdot$ gelernet] geleret Q \textbf{18} im] in T \textit{om.} V W Q R  $\cdot$ touf] tod R \textbf{19} unvernomen] [vernomen]: vnvernomen T noch vnvernomen V noch nit vernomen R \textbf{20} dehein] den hein T  $\cdot$ list] munt W  $\cdot$ moht] moͤht V (W)  $\cdot$ gevromen] fromen W \textbf{22} \textit{Versfolge 453.21-22} W Q R  \textbf{21} ze kündenne] Es kunden Q  $\cdot$ Grâles] [grass]: grals T \textbf{23} Ein] Sin T  $\cdot$ Flegetanis] flegetaneis W flegentanis Q flegatanis R \textbf{24} Beiagt an kunste hoher preysz Q  $\cdot$ künsten] kunst W \textbf{25} fisîôn] [*]: wise fision V fasion R \textbf{26} Salomon] salamon Q (R) \textbf{27} ûz] Vser R  $\cdot$ israhelischer] israhelscer T israhelscher V ysrahelschen R  $\cdot$ gezilt] erzilt W (Q) (R) \textbf{29} daz] der W \textbf{30} des] \textit{om.} Q \newline
\end{minipage}
\end{table}
\end{document}
