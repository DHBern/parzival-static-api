\documentclass[8pt,a4paper,notitlepage]{article}
\usepackage{fullpage}
\usepackage{ulem}
\usepackage{xltxtra}
\usepackage{datetime}
\renewcommand{\dateseparator}{.}
\dmyyyydate
\usepackage{fancyhdr}
\usepackage{ifthen}
\pagestyle{fancy}
\fancyhf{}
\renewcommand{\headrulewidth}{0pt}
\fancyfoot[L]{\ifthenelse{\value{page}=1}{\today, \currenttime{} Uhr}{}}
\begin{document}
\begin{table}[ht]
\begin{minipage}[t]{0.5\linewidth}
\small
\begin{center}*D
\end{center}
\begin{tabular}{rl}
\textbf{283} & \begin{large}S\end{large}ît ich dir hie \textbf{gelîchez} vant.\\ 
 & geêrt sî diu gotes hant\\ 
 & unt al diu crêatûre sîn.\\ 
 & Condwiramurs, hie lît dîn schîn,\\ 
5 & sît der snê dem bluote wîze bôt\\ 
 & unt ez den snê \textbf{sus} machet rôt.\\ 
 & Condwiramurs,\\ 
 & dem gelîchet sich dîn bêâcurs;\\ 
 & des enbistû niht erlâzen."\\ 
10 & des heldes ougen mâzen,\\ 
 & \textbf{alse}z dort was ergangen,\\ 
 & zwêne zaher an ir wangen,\\ 
 & den dritten an ir kinne.\\ 
 & er pflac der wâren minne\\ 
15 & gein ir \textbf{gar} âne wenken.\\ 
 & sus begunder sich verdenken,\\ 
 & \textbf{unze} daz er unversunnen hielt.\\ 
 & diu starke minne sîn dâ wielt.\\ 
 & \textbf{Sölhe} nôt \textbf{vuogte} im sîn \textit{w}î\textit{p}.\\ 
20 & dirre varwe truoc gelîchen lîp\\ 
 & von Pelrapeire diu künegîn;\\ 
 & diu zucte im \textbf{wizzenlîchen} \textbf{sin}.\\ 
 & sus hielt er, als er \textbf{slie\textit{f}}.\\ 
 & wer dâ zuo \textbf{z}im \textbf{lief}?\\ 
25 & Cunnewaren garzûn was gesant;\\ 
 & der solde gegen Lalant.\\ 
 & \textbf{der} sach an \textbf{den} stunden\\ 
 & einen helm mit \textbf{maneger} wunden\\ 
 & unt einen schilt \textbf{gar} \textbf{verhouwen}\\ 
30 & in dienste des knappen vrouwen.\\ 
\end{tabular}
\scriptsize
\line(1,0){75} \newline
D \newline
\line(1,0){75} \newline
\textbf{1} \textit{Initiale} D  \textbf{19} \textit{Majuskel} D  \newline
\line(1,0){75} \newline
\textbf{4} Condwiramurs] Condwir amvrs D \textbf{7} Condwiramurs] Cvndwir amvrs D \textbf{19} wîp] pris D \textbf{23} slief] slieft D \newline
\end{minipage}
\hspace{0.5cm}
\begin{minipage}[t]{0.5\linewidth}
\small
\begin{center}*m
\end{center}
\begin{tabular}{rl}
 & sît ich dir hie \textbf{glîchez} vant.\\ 
 & \textit{g}eêret sî diu got\textit{e}s hant\\ 
 & und alliu diu crêatûre sîn.\\ 
 & Con\textit{d}wieramurs, hie lît dîn schîn,\\ 
5 & sît der snê dem bluote wîze bôt\\ 
 & und ez den snê \textbf{sus} machet rôt.\\ 
 & Con\textit{d}wieramurs,\\ 
 & dem glîchet sich dîn bêâcurs;\\ 
 & des enbistû niht erlâzen."\\ 
10 & des heldes ougen mâzen,\\ 
 & \textbf{als} ez dort was ergangen,\\ 
 & zwêne zeher an ir wangen,\\ 
 & den dritten an ir kinne.\\ 
 & er pflac der wâren minne\\ 
15 & gegen ir \textbf{gare} \textit{âne} wenken.\\ 
 & sus begunde er sich verdenken,\\ 
 & \textbf{vür} daz er unversunnen hielt.\\ 
 & diu starke minne sîn dâ wielt.\\ 
 & \textbf{soliche} nôt \textbf{vuogete} ime sîn wîp.\\ 
20 & dirre v\textit{ar}we truoc glîchen lîp\\ 
 & von Pelraperie diu küniginne;\\ 
 & diu zuckete ime \textbf{wizzeclîch} \textbf{sinne}.\\ 
 & sus hielt e\textit{r}, \textit{a}ls er \textbf{sliefe}.\\ 
 & wer d\textit{â} \textit{z}u\textit{o} ime \textbf{liefe}?\\ 
25 & \textit{C}u\textit{nn}ewaren garzûn was gesa\textit{n}t;\\ 
 & der solte gegen Lalant.\\ 
 & \textbf{der} sach an \textbf{den} stunden\\ 
 & einen helm mit \textbf{maniger} wunden\\ 
 & und einen schilt \textbf{gar} \textbf{verhouwen}\\ 
30 & in dienste des knappen vrouwen.\\ 
\end{tabular}
\scriptsize
\line(1,0){75} \newline
m n o Fr8 \newline
\line(1,0){75} \newline
\textbf{25} \textit{Initiale} Fr8  \newline
\line(1,0){75} \newline
\textbf{1} glîchez] geliche Fr8 \textbf{2} geêret] Eeret m  $\cdot$ gotes] gottens m \textbf{3} alliu diu] alle n o \textbf{4} Condwieramurs] Condewier amurs m Cundewúr amúrsz n Cun wir amúrez o :onduweramurs Fr8  $\cdot$ hie lît dîn] dirre Fr8 \textbf{7} Condwieramurs] Condewier amurs m Condiwir amuris o \textbf{8} \textit{nach 283.8:} :l sulhe varewe hat min wib Fr8   $\cdot$ dem] \textit{om.} Fr8  $\cdot$ bêâcurs] beahurs n brahurs o schoner lib Fr8 \textbf{9} erlâzen] zuͦ erlósen o \textbf{12} ir] ire m \textbf{15} gare âne] gare m one n (Fr8) \textbf{17} vür] :nz Fr8 \textbf{18} sîn] sint o  $\cdot$ dâ] do n o \textit{om.} Fr8  $\cdot$ wielt] gewielt Fr8 \textbf{19} vuogete] fuͯget n o (Fr8) \textbf{20} varwe] frouwe m farwen n  $\cdot$ truoc] muͦz Fr8 \textbf{21} von] Haben von Fr8  $\cdot$ Pelraperie] pelrapeir n pelrapier o Pelrapeire Fr8 \textbf{22} zuckete] vntzuct Fr8  $\cdot$ wizzeclîch sinne] wisseclichen sin n (o) (Fr8) \textbf{23} er als] er alsus hielt er als m er als ob Fr8  $\cdot$ sliefe] slieff n o \textbf{24} dâ] do m n o  $\cdot$ zuo] susse m sus n o  $\cdot$ ime] jnne n jnn o  $\cdot$ liefe] lieff n o \textbf{25} Cunnewaren] Gumewaren m Gunne waren o  $\cdot$ was gesant] was gesat m von Lalant Fr8 \textbf{26} Den hette div vrowe da gesant Fr8 \textbf{27} an] in an Fr8 \textbf{28} maniger] manegen Fr8 \textbf{29} verhouwen] zuͦ howen Fr8 \newline
\end{minipage}
\end{table}
\newpage
\begin{table}[ht]
\begin{minipage}[t]{0.5\linewidth}
\small
\begin{center}*G
\end{center}
\begin{tabular}{rl}
 & sît ich dir hie \textbf{gelîchez} vant.\\ 
 & geêret sî diu gotes hant\\ 
 & unde al diu crêatûre sîn.\\ 
 & Condwiramurs, hie lît dîn schîn,\\ 
5 & sît der snê dem b\textit{l}uote \textbf{die} wîze bôt\\ 
 & unde ez den snê \textbf{sô} machet rôt.\\ 
 & \textbf{süeziu} Condwiramurs,\\ 
 & dem gelîchet sich dîn bêâcurs;\\ 
 & des enbistû niht erlâzen."\\ 
10 & des heldes ougen mâzen,\\ 
 & \textbf{wie}z dort was ergangen,\\ 
 & zwêne zeher an ir wangen,\\ 
 & den dritten an ir kinne.\\ 
 & er pflac der wâren minne\\ 
15 & gein ir âne wenken.\\ 
 & sus begunder sich verdenken,\\ 
 & \textbf{sô} daz er unversunnen hielt.\\ 
 & diu starke minne sîn dâ wielt.\\ 
 & \textbf{al}\textbf{solher} nôt \textbf{half} im sîn wîp.\\ 
20 & \textit{d}irre varwe truoc gelîchen lîp\\ 
 & von Pelrapeire diu künigîn;\\ 
 & diu zuct im \textbf{wizzenlîchen} \textbf{sin}.\\ 
 & sus hiel\textit{t} er, alser \textbf{sliefe}.\\ 
 & wer dâ zuo im \textbf{liefe}?\\ 
25 & Kunewaren garzûn was gesant;\\ 
 & der solt gein Lalant.\\ 
 & \textbf{er} sach an \textbf{den} stunden\\ 
 & einen helm mit \textbf{manigen} wunden\\ 
 & unde einen schilt \textbf{verhouwen}\\ 
30 & in dienste des knappen vrouwen.\\ 
\end{tabular}
\scriptsize
\line(1,0){75} \newline
G I O L M Q R Z Fr30 Fr40 Fr60 \newline
\line(1,0){75} \newline
\textbf{1} \textit{Initiale} O L Q Z Fr60  \textbf{7} \textit{Initiale} G  \textbf{21} \textit{Initiale} I  \textbf{25} \textit{Initiale} R  \newline
\line(1,0){75} \newline
\textbf{1} sît] ÷it O Mit M  $\cdot$ gelîchez] gelichen O \textbf{2} diu] in Q \textbf{4} Condwiramurs] Gondwiramurs I Kvndwiramvrs O Kvndwir Amuͯrs L Kondwir Amuͯrs M Kundwiraműrs Q Kondwiramurs R Z :::uͦrs Fr30 :::viramurs Fr40 Kvndwir amvrs Fr60 \textbf{5} bluote] boͮte G  $\cdot$ die] \textit{om.} O L M Q R Z Fr40 \textbf{6} sô] \textit{om.} M Q Fr40 \textbf{7} süeziu] froͮe I \textit{om.} O L M Q R Z Fr40 Fr60  $\cdot$ Condwiramurs] conwiramurs G Gondwiramurs I [Kv*]: Kvndwir amvͦrs O [Ko*]: Kvndwirs Amuͯrs L Kondwir Amuͯrs M Kundwiraműrs Q Kondwiramuͦrs R Kondwiramurs Z :::iramurs Fr40 Kvndwir amvrs Fr60 \textbf{8} sich] \textit{om.} L \textbf{9} enbistû] bistv O (R) (Fr60) \textbf{10} heldes] helden R \textbf{11} wiez] Waz O Fr60 Als es Q R  $\cdot$ was] [wart]: was Z \textbf{13} kinne] kniene kinne R \textbf{14} pflac] sprach O Fr60  $\cdot$ der] [den]: der M  $\cdot$ wâren] warer Q \textbf{15} âne] gar ane O M (Q) (R) Z Fr60 \textbf{16} begunder] begunde I (L) (Fr40) [b*gvnde]: bgvnde  O  $\cdot$ verdenken] bedenchen O (Q) (Z) \textbf{17} sô] Vntz Z  $\cdot$ unversunnen] vnuersinen R \textbf{18} dâ] do Q R \textbf{19} alsolher] Solch Z  $\cdot$ half] gehalf I fugt Z \textbf{20} dirre varwe] div dirre varwe G Dise frawe Q \textbf{21} Pelrapeire] pailrapier I pelraperre Q :::rapevre Fr40 pelrapeir Fr60 \textbf{22} diu] dv G Die R  $\cdot$ zuct] enzucht I (O) (L) (M) (Q) (R) (Z)  $\cdot$ wizzenlîchen] witzeklichen Q (Fr60)  $\cdot$ sin] schin L [*hin]: sin Z \textbf{23} hielt] hiel G hilft Q  $\cdot$ alser] sam er Q  $\cdot$ sliefe] schlieff R \textbf{24} dâ] do Q  $\cdot$ zuo im] zuͯ L zvzim Fr40  $\cdot$ liefe] lieff R \textbf{25} Kunewaren] kunuwarn I Kvnawaren O Vrow Cvnewaren L Kunwaren M (Fr60) Cűnwaren Q Cunwaren R Cvnnewaren Z  $\cdot$ garzûn] eyn garczun M  $\cdot$ was] wart I  $\cdot$ gesant] genant O Fr60 \textbf{26} gein] ge R  $\cdot$ Lalant] lalan O la lant Q \textbf{27} er] Der Q Den Z  $\cdot$ an] gein M  $\cdot$ den] der Z \textbf{28} manigen] maniger O (M) (Q) (R) (Z) Fr40 Fr60 \textbf{29} unde] \textit{om.} L  $\cdot$ verhouwen] gar verhawen O (L) (M) (Q) (Z) (Fr40) (Fr60) gar zerhowen R \newline
\end{minipage}
\hspace{0.5cm}
\begin{minipage}[t]{0.5\linewidth}
\small
\begin{center}*T
\end{center}
\begin{tabular}{rl}
 & sît ich dir hie \textbf{wol} \textbf{glîches} vant.\\ 
 & geêret sî diu gotes hant\\ 
 & unde al di\textit{u} crêatûre sîn.\\ 
 & Cundewiramurs, hie lît dîn schîn,\\ 
5 & sît der snê dem bluote wîze bôt\\ 
 & unde ez den snê \textbf{sô} machet rôt.\\ 
 & Cundewiramurs,\\ 
 & dem glîchet sich dîn bêâ\textit{c}urs;\\ 
 & des enbistû niht erlâzen."\\ 
10 & des heldes ougen mâzen,\\ 
 & \textbf{wie}z dort was ergangen,\\ 
 & zwêne zeher an ir wangen\\ 
 & \textbf{unde} den dritten an ir kinne.\\ 
 & er pflac der wâren minne\\ 
15 & gegen ir âne wenken.\\ 
 & sus begunder sich verdenken,\\ 
 & \textbf{sô} daz er unversunnen hielt.\\ 
 & diu starke minne sîn dâ wielt.\\ 
 & \textbf{al}\textbf{so\textit{l}her} nôt \textbf{half} im sîn wîp.\\ 
20 & dirre varwe truoc glîchen lîp\\ 
 & von Peilraper diu künegîn;\\ 
 & di\textit{u} zuctim \textbf{wizzenden} \textbf{sin}.\\ 
 & Sus hielt er, alser \textbf{sliefe}.\\ 
 & wer dar zuo \textbf{z}im \textbf{riefe}?\\ 
25 & \begin{large}C\end{large}unnewaren garzûn was gesant;\\ 
 & der solte gegen Lalant.\\ 
 & \textbf{der} sach an \textbf{der} stunden\\ 
 & einen helm mit \textbf{manegen} wunden\\ 
 & unde einen schilt \textbf{gar} \textbf{zerhouwen}\\ 
30 & in dienste des knappen vrouwen.\\ 
\end{tabular}
\scriptsize
\line(1,0){75} \newline
T U V W \newline
\line(1,0){75} \newline
\textbf{9} \textit{Initiale} W  \textbf{23} \textit{Majuskel} T  \textbf{25} \textit{Initiale} T U  \newline
\line(1,0){75} \newline
\textbf{1} wol] \textit{om.} W \textbf{3} diu] die T \textbf{4} Cundewiramurs] Kvndewiramvrs T Kuͦndewiramurs U Gundwiramurs W  $\cdot$ lît] leit W \textbf{5} wîze] [*]: wisse V weise W \textbf{6} ez] er W \textbf{7} Cundewiramurs] Kvndewiramurs T Kuͦndewiramuͦrs U Cvndiwiramvrs V Ahy gundwiramurs W \textbf{8} bêâcurs] beamvrs T Beamuͦrs U [bea]: beakurs V \textbf{9} des] DEr W  $\cdot$ enbistû] bistu W \textbf{12} zeher] tropfen V  $\cdot$ ir] irn U \textbf{14} pflac] plat U  $\cdot$ wâren] [wârer]: wâren T \textbf{16} verdenken] bedencken W \textbf{18} sîn] in W  $\cdot$ dâ] do U W [*]: do V \textbf{19} alsolher] alsoher T  $\cdot$ half] [fuͦct*]: fuͦct V \textbf{20} varwe] varwen U \textbf{21} Peilraper] peilrapere V pelrapier W \textbf{22} diu] die T  $\cdot$ wizzenden] witze vnde oͮch den V ee die witze W \textbf{23} alser] als ob er W \textbf{24} zuo zim] zuͦ im U (V) W  $\cdot$ riefe] lieffe V W \textbf{25} Cunnewaren] [Kvnner]: Kvnnewâren T Zuͦ ime waren U Kunnewarn W \textbf{27} sach] gesach W  $\cdot$ der] den U V W \textbf{28} manegen] manger W \textbf{29} schilt] schit U  $\cdot$ zerhouwen] dvrch howen V verhawen W \textbf{30} knappen] knappe U \newline
\end{minipage}
\end{table}
\end{document}
