\documentclass[8pt,a4paper,notitlepage]{article}
\usepackage{fullpage}
\usepackage{ulem}
\usepackage{xltxtra}
\usepackage{datetime}
\renewcommand{\dateseparator}{.}
\dmyyyydate
\usepackage{fancyhdr}
\usepackage{ifthen}
\pagestyle{fancy}
\fancyhf{}
\renewcommand{\headrulewidth}{0pt}
\fancyfoot[L]{\ifthenelse{\value{page}=1}{\today, \currenttime{} Uhr}{}}
\begin{document}
\begin{table}[ht]
\begin{minipage}[t]{0.5\linewidth}
\small
\begin{center}*D
\end{center}
\begin{tabular}{rl}
\textbf{590} & \begin{large}W\end{large}az diu \textbf{wunders} mohte hân.\\ 
 & durch schouwen gienc hêr Gawan\\ 
 & ûf daz warthûs eine\\ 
 & zuo manegen \textbf{tiwerem} steine.\\ 
5 & dâ vander solch wunder grôz,\\ 
 & des in ze sehen niht verdrôz.\\ 
 & In dûhte, \textbf{daz} im al \textbf{diu} lant\\ 
 & in der grôzen sûl wæren bekant\\ 
 & unt \textbf{daz} diu lant \textbf{al} umbe giengen\\ 
10 & unt daz mit hurte enpfiengen\\ 
 & die grôzen berge ein ander.\\ 
 & in der sûl vander\\ 
 & liute rîten unde gên,\\ 
 & \textbf{disen} loufen, \textbf{jenen} stên.\\ 
15 & In \textbf{ein} venster er gesaz,\\ 
 & \textbf{er wolt daz wunder} prüeven baz.\\ 
 & dô kom diu alte Arnive\\ 
 & und \textbf{ir tohter} Sangive\\ 
 & und ir tohter \textbf{tohter} zwuo;\\ 
20 & die giengen alle viere zuo.\\ 
 & \textbf{Gawan} spranc ûf, dô er si sach.\\ 
 & diu küneginne Arnive sprach:\\ 
 & "Hêrre, ir \textbf{soltet} noch slâfes pflegen.\\ 
 & habt ir ruowens iuch bewegen,\\ 
25 & dar zuo sît ir \textbf{ze} sêre wunt,\\ 
 & sol iu ander ungemach sîn kunt."\\ 
 & Dô sprach er: "vrouwe unde meisterîn,\\ 
 & mir hât kraft unde \textbf{sin}\\ 
 & iwer helfe alsô gegeben,\\ 
30 & daz ich gediene, \textbf{muoz} ich leben."\\ 
\end{tabular}
\scriptsize
\line(1,0){75} \newline
D Z \newline
\line(1,0){75} \newline
\textbf{1} \textit{Initiale} D  \textbf{7} \textit{Majuskel} D  \textbf{15} \textit{Majuskel} D  \textbf{23} \textit{Majuskel} D  \textbf{27} \textit{Majuskel} D  \newline
\line(1,0){75} \newline
\textbf{4} Zv manigem edeln gesteine Z \textbf{7} daz] wie Z \textbf{9} al] \textit{om.} Z \textbf{12} sûl] grozzen sevl Z \textbf{16} Daz wunder wold er prvfen baz Z \textbf{17} dô] Da Z  $\cdot$ Arnive] Arnîve D \textbf{18} Sangive] Sangîve D Seyve Z \textbf{19} tohter tohter] toͤhter Z \textbf{21} Er spranc vf da er sie komen sach Z \textbf{22} Arnive] Arnîve D \textbf{23} soltet] svlt Z \textbf{30} muoz] sol Z \newline
\end{minipage}
\hspace{0.5cm}
\begin{minipage}[t]{0.5\linewidth}
\small
\begin{center}*m
\end{center}
\begin{tabular}{rl}
 & \textit{w}az diu \textbf{wunder} mohte hân.\\ 
 & durch schouwen gienc hêr Gawan\\ 
 & ûf daz warthûs eine\\ 
 & zuo manige\textit{m} \textbf{\textit{t}iure\textit{n}} \textit{s}teine.\\ 
5 & d\textit{â} vant er solich wunder grôz,\\ 
 & des in zuo sehen niht verdrôz.\\ 
 & in dûhte, \textbf{daz} im alliu \textbf{diu} lant\\ 
 & in der grôzen sûl wære\textit{n} bekant\\ 
 & und \textbf{daz} diu lant umbe giengen\\ 
10 & und daz mit h\textit{u}rte enpfiengen\\ 
 & die grôzen berge ein ander.\\ 
 & in der sûle vander\\ 
 & liute rîten und gân,\\ 
 & \textbf{disen} loufen, \textbf{jenen} stân.\\ 
15 & in \textbf{ein} venster er gesaz,\\ 
 & \textbf{er wolt daz wunder} brüefen baz.\\ 
 & dô kam diu alte Arni\textit{v}e\\ 
 & und \textbf{ir tohter} Sa\textit{n}g\textit{iv}e\\ 
 & und ir tohter \textbf{tohter} zwô;\\ 
20 & die giengen alle vier zuo.\\ 
 & \textbf{Gawan} spranc ûf, dô er si sach.\\ 
 & diu künigîn Arn\textit{iv}e sprach:\\ 
 & "hêrre, ir \textbf{solt} noch slâfes pflegen.\\ 
 & habt ir ruowens iuch b\textit{e}wegen,\\ 
25 & dar zuo sît ir \textbf{zuo} sêre wunt,\\ 
 & sol iu ander ungemach sîn kunt."\\ 
 & dô sprach er: "vrouwe und meisterîn,\\ 
 & mir hât kraft und \textbf{sin}\\ 
 & iuwer helf alsô gegeben,\\ 
30 & daz ich gediene, \textbf{sol} ich leben."\\ 
\end{tabular}
\scriptsize
\line(1,0){75} \newline
m n o \newline
\line(1,0){75} \newline
\newline
\line(1,0){75} \newline
\textbf{1} waz] DWas m  $\cdot$ wunder] wunders n (o)  $\cdot$ mohte] moͯchte n \textbf{2} hêr] der o \textbf{4} Zuͯ mangen sturen cleine steine m \textbf{5} dâ] Do m n o \textbf{6} des] Das o \textbf{7} dûhte] duͯchte o  $\cdot$ alliu] do alle n \textbf{8} wæren] were m \textbf{10} hurte] herte m n [berte]: herte o \textbf{11} die grôzen] Do grosse o \textbf{13} gân] got n \textbf{17} Arnive] Arniwe m [n]: arniwe n arnuͯwe o \textbf{18} tohter] docker o  $\cdot$ Sangive] sagwie m saigwe n sagiwe o \textbf{19} tohter tohter] docker dochter o \textbf{22} künigîn] konige o  $\cdot$ Arnive] arnuͯwe m arniwe n o \textbf{23} slâfes] slaffen o \textbf{24} bewegen] betwegen m \textbf{25} sît] so sint n  $\cdot$ sêre] se::e o \textbf{28} hât] >het< o \newline
\end{minipage}
\end{table}
\newpage
\begin{table}[ht]
\begin{minipage}[t]{0.5\linewidth}
\small
\begin{center}*G
\end{center}
\begin{tabular}{rl}
 & waz diu \textbf{wunders} mohte hân.\\ 
 & durch schouwen gienc hêr Gawan\\ 
 & ûf daz warthûs eine\\ 
 & zuo manigem \textbf{edelen} steine.\\ 
5 & dâ vant er solich wunder grôz,\\ 
 & des in ze sehen niht verdrôz.\\ 
 & in dûhte, \textbf{wie} im al \textbf{diu} lant\\ 
 & in der grôzen sûle wæren bekant\\ 
 & \begin{large}U\end{large}nde diu lant umbe giengen\\ 
10 & unt daz mit h\textit{u}rt enpfienge\textit{n}\\ 
 & die grôzen berge ein ander.\\ 
 & in der sûle vant er\\ 
 & liut rîten unde gên,\\ 
 & \textbf{dise} loufen, \textbf{jene} stên.\\ 
15 & in \textbf{einem} venster er \textbf{dô} gesaz,\\ 
 & \textbf{daz wunder wold er} prüeven baz.\\ 
 & dô kom diu alte Arnive\\ 
 & unde Sagive\\ 
 & unde ir tohter zwô;\\ 
20 & die giengen alle vier zuo.\\ 
 & \textbf{er} spranc ûf, dô er si \textbf{komen} sach.\\ 
 & diu künegîn Arnive sprach:\\ 
 & "hêrre, ir \textbf{solt} noch slâfes pflegen.\\ 
 & habet ir ruowens iuch bewegen,\\ 
25 & dar zuo sît ir \textbf{ze} sêre wunt,\\ 
 & sol iu ander ungemach sîn kunt."\\ 
 & dô sprach er: "vrouwe unde meisterîn,\\ 
 & mir hât kraft unde \textbf{sin}\\ 
 & iuwer helfe alsô gegeben,\\ 
30 & daz ich gediene, \textbf{muoz} ich leben."\\ 
\end{tabular}
\scriptsize
\line(1,0){75} \newline
G I L M Fr23 \newline
\line(1,0){75} \newline
\textbf{9} \textit{Initiale} G L  \textbf{13} \textit{Initiale} I  \textbf{29} \textit{Initiale} I  \newline
\line(1,0){75} \newline
\textbf{1} mohte] mochten M \textbf{2} hêr Gawan] ergawan M \textbf{4} edelen] shonem I \textit{om.} M \textbf{5} dâ] do I  $\cdot$ solich] solhe I \textbf{6} verdrôz] er droz L bedroz Fr23 \textbf{7} diu] \textit{om.} L M \textbf{8} grôzen] \textit{om.} L  $\cdot$ wæren] waren L :::arn Fr23 \textbf{9} Unde] Vnde daz L Nu das M \textbf{10} daz] \textit{om.} M  $\cdot$ hurt] hort G  $\cdot$ enpfiengen] enphienge G \textbf{12} in] Vnde M  $\cdot$ sûle] groszin sule M \textbf{14} dise] iene I Disen L  $\cdot$ jene] dise I ýenen L ::nen Fr23 \textbf{15} einem] ein I L  $\cdot$ dô gesaz] do saz I gesaz L (M) \textbf{17} dô] Da M  $\cdot$ Arnive] Arniue I [Arn*]: Arnyve L \textbf{18} Sagive] sâide G diu shone shawe I ir tochter Seýve L ir tochter seive M \textbf{19} ir] mit der ir L  $\cdot$ tohter zwô] tohter auch zwoͮ I zcwu tochter Nu M \textbf{20} die] da I  $\cdot$ giengen] gienge Fr23 \textbf{21} dô] da M  $\cdot$ komen] \textit{om.} I \textbf{22} Arnive] Arniue I Anive L arnÿve M arniv Fr23 \textbf{23} solt] soltet L \textbf{24} habet] hab I \textbf{25} wunt] chunt I \textbf{26} ander] anders M  $\cdot$ sîn] sy M \textbf{27} dô] Da M  $\cdot$ vrouwe] vrouwen M  $\cdot$ unde] \textit{om.} I \textbf{30} daz ich] Daz ichz L (M) Fr23  $\cdot$ muoz] sol L (M) Fr23 \newline
\end{minipage}
\hspace{0.5cm}
\begin{minipage}[t]{0.5\linewidth}
\small
\begin{center}*T
\end{center}
\begin{tabular}{rl}
 & waz diu \textbf{wunders} mohte hân.\\ 
 & durch schouwen gienc hêr Gawan\\ 
 & ûf daz warthûs eine\\ 
 & zuo manegem \textbf{edelem} steine.\\ 
5 & d\textit{â} vant er solch wunder grôz,\\ 
 & d\textit{e}s in zuo sehen niht verdrôz.\\ 
 & in dûhte, \textbf{wie} im alliu lant\\ 
 & in der grôzen sûl wæren bekant\\ 
 & und \textbf{daz} diu lant umbe giengen\\ 
10 & \textit{u}nd daz mit hurte enpfiengen\\ 
 & die grôzen berge ein ander.\\ 
 & \textit{in} der sûle vander\\ 
 & liute rîten und gên,\\ 
 & \textbf{disen} loufen \textbf{und} \textbf{jenen} stên.\\ 
15 & in \textbf{einem} venster er gesaz,\\ 
 & \textbf{daz wunder wolt er} prüeven baz.\\ 
 & dô kam diu alte Arnive\\ 
 & und \textbf{ir tohter} Seyve\\ 
 & und ir tohter zwô;\\ 
20 & die giengen alle viere zuo.\\ 
 & \textbf{er} spranc ûf, dô er si \textbf{komen} sach.\\ 
 & diu küniginne Arnive sprach:\\ 
 & "hêrre, ir \textbf{solt} noch slâfes pflegen.\\ 
 & habt ir ruowens iuch bewegen,\\ 
25 & dâ zuo sît ir sêre wunt,\\ 
 & sol iu ander ungemach sîn kunt."\\ 
 & dô sprach er: "vrouwe und meisterinne,\\ 
 & mir hât kraft und \textbf{sinne}\\ 
 & iuwer helfe alsô gegeben,\\ 
30 & daz ich gediene, \textbf{sol} ich leben."\\ 
\end{tabular}
\scriptsize
\line(1,0){75} \newline
Q R W V U \newline
\line(1,0){75} \newline
\textbf{7} \textit{Capitulumzeichen} R  \textbf{9} \textit{Initiale} Q  \textbf{17} \textit{Initiale} W V  \newline
\line(1,0){75} \newline
\textbf{1} \textit{Die Verse 553.1-599.30 fehlen} U   $\cdot$ mohte] moͤchte W (V) \textbf{3} eine] alleine W \textbf{4} edelem] [*]: tv́ren V  $\cdot$ steine] gesteine R W \textbf{5} dâ] Do Q R W V \textbf{6} des] Das Q W  $\cdot$ sehen] sechent R (V) \textbf{7} im] \textit{om.} R \textbf{8} grôzen] grosse R  $\cdot$ wæren] wer R (V) \textbf{10} und] nd Q \textbf{12} in der] Nider Q  $\cdot$ vander] do vant er V \textbf{13} liute] Lútten R \textbf{14} disen] Die W  $\cdot$ und] den R \textit{om.} W V  $\cdot$ jenen] andren R iene W \textbf{17} alte Arnive] altu arnẏue R \textbf{18} Seyve] seyue Q (R) siue W \textbf{19} und] Dar zuͦ W  $\cdot$ tohter] tochtter tochttren R (V) \textbf{20} zuo] do R [zvͦo]: do V \textbf{21} [*]: Gawan sprang vf do er sv́ sach V \textbf{22} Arnive] aronyue R \textbf{23} solt] solten R \textbf{24} bewegen] erwegen R \textbf{25} ir] ir so R ir zuͦ W (V) \textbf{30} gediene] gedinge R \newline
\end{minipage}
\end{table}
\end{document}
