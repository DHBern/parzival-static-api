\documentclass[8pt,a4paper,notitlepage]{article}
\usepackage{fullpage}
\usepackage{ulem}
\usepackage{xltxtra}
\usepackage{datetime}
\renewcommand{\dateseparator}{.}
\dmyyyydate
\usepackage{fancyhdr}
\usepackage{ifthen}
\pagestyle{fancy}
\fancyhf{}
\renewcommand{\headrulewidth}{0pt}
\fancyfoot[L]{\ifthenelse{\value{page}=1}{\today, \currenttime{} Uhr}{}}
\begin{document}
\begin{table}[ht]
\begin{minipage}[t]{0.5\linewidth}
\small
\begin{center}*D
\end{center}
\begin{tabular}{rl}
\textbf{552} & \begin{large}K\end{large}unde Gawan guoten willen zern,\\ 
 & des m\textit{ö}ht er sich dâ wol \textbf{nern}.\\ 
 & \textbf{nie} muoter gunde ir kinde baz\\ 
 & denn im der wirt, des brôt er az.\\ 
5 & Dô man den tisch \textbf{hin} dan enpfienc\\ 
 & unt \textbf{dô} diu wirtîn ûz \textbf{gegienc},\\ 
 & vil bette man dar \textbf{ûf dô} treit;\\ 
 & diu wurden Gawane geleit.\\ 
 & \textbf{eine\textit{z}} was ein pflûmît,\\ 
10 & \textbf{des} zieche ein \textbf{grüener} samît,\\ 
 & \textbf{des} niht vo\textit{n} der \textbf{hôhen} art:\\ 
 & ez was ein samît bastart.\\ 
 & Ein kulter \textbf{wart} des bettes dach,\\ 
 & niht wan durch Gawans gemach,\\ 
15 & mit einem pfellel sunder golt,\\ 
 & verre \textbf{in} heidenschaft geholt,\\ 
 & gesteppet \textbf{ûf} palmât.\\ 
 & dar über zôch man \textbf{linde wât},\\ 
 & zwei lîlachen snêvar.\\ 
20 & man leite \textbf{ein} wanküssen dar\\ 
 & unt der meide mantel einen,\\ 
 & härmîn, niwe, reinen.\\ 
 & Mit urloube er\textbf{z} \textbf{undervienc},\\ 
 & der wirt, ê \textbf{daz} er slâfen gienc.\\ 
25 & Gawan al eine, \textbf{ist mir gesagt},\\ 
 & \textbf{beleip} \textbf{al} dâ, \textbf{mit im diu magt}.\\ 
 & het er iht \textbf{hin zir} \textbf{gegert},\\ 
 & ich wæne, si hetes in gewert.\\ 
 & er sol ouch slâfen, ob er mac.\\ 
30 & got hüete \textbf{sîn}, sô kom der tac.\\ 
\end{tabular}
\scriptsize
\line(1,0){75} \newline
D \newline
\line(1,0){75} \newline
\textbf{1} \textit{Initiale} D  \textbf{5} \textit{Majuskel} D  \textbf{13} \textit{Majuskel} D  \textbf{23} \textit{Majuskel} D  \newline
\line(1,0){75} \newline
\textbf{2} möht] moht D \textbf{9} einez] eines D \textbf{11} von] vor D \newline
\end{minipage}
\hspace{0.5cm}
\begin{minipage}[t]{0.5\linewidth}
\small
\begin{center}*m
\end{center}
\begin{tabular}{rl}
 & künde Gawan guoten willen zern,\\ 
 & des m\textit{ö}hte er sich d\textit{â} wol \textbf{genern}.\\ 
 & \textbf{nie} muoter g\textit{u}nde ir k\textit{i}nde baz\\ 
 & dan im der wirt, des brôt er az.\\ 
5 & dô man den tisch \textbf{hin} dan enpfienc\\ 
 & und diu wirtîn ûz \textbf{gegienc},\\ 
 & vil bette man dar \textbf{ûf dô} treit;\\ 
 & diu wurden Gawan geleit.\\ 
 & \textbf{einez} was ein plû\textit{m}ît,\\ 
10 & \textbf{diu} ziech ein \textbf{grüener} samît.\\ 
 & \hspace*{-.7em}\big| ez was ein samît bastart,\\ 
 & \hspace*{-.7em}\big| \textbf{daz} niht von der \textbf{hôhen} art.\\ 
 & ein kulte\textit{r} \textbf{\textit{w}art} des bettes dach,\\ 
 & niht wan durch Gawans gemach,\\ 
15 & mit einem pfelle sunder golt,\\ 
 & verre \textbf{in} heidenschaft geholt,\\ 
 & gesteppet \textbf{ûf} \textbf{dem} palmât.\\ 
 & dar über zôch man \textbf{linde wât},\\ 
 & zwei lîlacher snêvar.\\ 
20 & man leit \textbf{ein} \textit{w}anküssen dar\\ 
 & und der megde mantel einen,\\ 
 & her\textit{m}în, niuwen, reinen.\\ 
 & mit urloube er\textbf{z} \textbf{gevienc},\\ 
 & der wirt, ê er slâfen gienc.\\ 
25 & Gawan alein, \textbf{ist mir gesaget},\\ 
 & \textbf{bleip} dâ, \textbf{bî im diu maget}.\\ 
 & het er iht \textbf{hin zuo ir} \textbf{begert},\\ 
 & ich wæne, si het es in gewert.\\ 
 & er sol ouch slâfen, ob er mac.\\ 
30 & got hüete, sô kum der tac.\\ 
\end{tabular}
\scriptsize
\line(1,0){75} \newline
m n o \newline
\line(1,0){75} \newline
\newline
\line(1,0){75} \newline
\textbf{2} möhte] mohtte m (o)  $\cdot$ dâ] do m n des o \textbf{3} gunde] guͯnde m (n) o  $\cdot$ kinde] kunde m \textbf{4} des] daz o \textbf{7} dar] do n \textbf{8} diu] Eines die o \textbf{9} plûmît] plunit m \textbf{10} samît] samuͦt o \textbf{12} samît] samet o  $\cdot$ bastart] beschart o \textbf{13} kulter wart] kultter was vnd wart m \textbf{14} Gawans] gawanes n gewanes o \textbf{17} dem] den n \textbf{19} lîlacher] lilachen n o  $\cdot$ snêvar] sne gefar n \textbf{20} wanküssen] man kuͯssen m (n) (o) \textbf{22} hermîn] Hernẏn m \textbf{23} Mit vrlaubes er: o  $\cdot$ gevienc] enpfing n \textbf{24} ê] E das n (o) \textbf{26} dâ bî] aldo mit n (o) \textbf{30} kum] kompt n kam o \newline
\end{minipage}
\end{table}
\newpage
\begin{table}[ht]
\begin{minipage}[t]{0.5\linewidth}
\small
\begin{center}*G
\end{center}
\begin{tabular}{rl}
 & \begin{large}K\end{large}unde Gawan guoten willen zeren,\\ 
 & des m\textit{ö}ht er sich dâ wol \textbf{erneren}.\\ 
 & \textbf{nie} muoter gunde ir kinde baz\\ 
 & danne \textit{im} der wirt, des brôt er az.\\ 
5 & dô man den tisch \textbf{hin} dan enpfienc\\ 
 & unde \textbf{dô} diu wirtinne ûz \textbf{gienc},\\ 
 & vil bette man dar \textbf{ûf dô} tr\textit{eit};\\ 
 & diu wurden Gawanen geleit.\\ 
 & \textbf{einez} was ein pf\textit{l}ûmît,\\ 
10 & \textbf{des} zieche ein \textbf{grüener} samît,\\ 
 & \textbf{des} niht von der \textbf{hôhen} art:\\ 
 & ez was ein samît bastart.\\ 
 & ein kulter \textbf{was} des bettes dach,\\ 
 & niht wan durch Gawanes gemach,\\ 
15 & mit einem pfelle sunder golt,\\ 
 & verre \textbf{ûz} he\textit{i}denschaft geholt,\\ 
 & gesteppet \textbf{ûf} palmât.\\ 
 & dar über zôch man \textbf{linde wât},\\ 
 & zwei lîlachen snêvar.\\ 
20 & man leit \textbf{ein} wanküsse dar\\ 
 & unde der meide mandel einen,\\ 
 & hermîn, niuwen, reinen.\\ 
 & mit urloube er\textbf{z} \textbf{undervienc},\\ 
 & der wirt, ê \textbf{daz} er slâfen gienc.\\ 
25 & Gawan al eine, \textbf{ist mir gesaget},\\ 
 & \textbf{beleip} \textbf{al} dâ, \textbf{mit im diu maget}.\\ 
 & het er iht \textbf{hin ze ir} \textbf{gegert},\\ 
 & ich wæne, si het\textit{es} in gewert.\\ 
 & er sol ouch slâfen, ob er mac.\\ 
30 & got hüete \textbf{sîn}, sô kom der tac.\\ 
\end{tabular}
\scriptsize
\line(1,0){75} \newline
G I L M Z \newline
\line(1,0){75} \newline
\textbf{1} \textit{Initiale} G I L Z  \textbf{23} \textit{Initiale} I  \newline
\line(1,0){75} \newline
\textbf{1} Kunde] Kvͤnde Z \textbf{2} möht] moht G I (L) (M) Z \textbf{3} gunde] kunde M  $\cdot$ kinde] kinden M \textbf{4} im] \textit{om.} G \textbf{5} dô] Da M Z \textbf{6} dô] \textit{om.} L da M Z  $\cdot$ gienc] gegie L (M) (Z) \textbf{7} dô] \textit{om.} I da Z  $\cdot$ treit] trvch G \textbf{8} Gawanen] Gawan I (Z) Gawane L (M) \textbf{9} pflûmît] pfvmit G \textbf{10} ein] was ein I \textbf{11} des niht] Nýht des L \textbf{12} ein] \textit{om.} M \textbf{13} was] wart L M \textbf{14} Gawanes] Gawans I (Z) Gawanz L \textbf{16} ûz] in L (M) Z  $\cdot$ heidenschaft] heindenschaft G \textbf{17} ûf] [v*]: vsz L \textbf{18} linde] lynen M \textbf{20} wanküsse] wan kuͯssen L (M) banckvsse Z \textbf{21} unde] Von Z  $\cdot$ einen] eyne M \textbf{22} niuwen] niwe I (M) (Z)  $\cdot$ reinen] [re*]: rein I \textbf{25} Gawan] Gawane M \textbf{26} al dâ] alein I (Z)  $\cdot$ diu] da die Z \textbf{27} het er] Ehir M  $\cdot$ iht hin ze ir] hinz ir iht L (M)  $\cdot$ gegert] gigerte M \textbf{28} si] [er]: si G  $\cdot$ hetes] het G \textbf{30} kom] kome M \newline
\end{minipage}
\hspace{0.5cm}
\begin{minipage}[t]{0.5\linewidth}
\small
\begin{center}*T
\end{center}
\begin{tabular}{rl}
 & kunde Gawan guoten willen zern,\\ 
 & des m\textit{ö}hter sich dâ wol \textbf{nern},\\ 
 & \textbf{wande kein} muoter gunde\textbf{s} ir kinde baz\\ 
 & dan \textit{im} der wirt, des brôt er az.\\ 
5 & Dô man den tisch \textbf{her} dan enpfienc\\ 
 & unde diu wirtinne ûz \textbf{gienc},\\ 
 & vil bette man dar \textbf{nâher} treit;\\ 
 & diu wurden Gawane geleit.\\ 
 & \textbf{daz eine} was ein plûmît,\\ 
10 & \textbf{des} zieche \textbf{was} ein samît,\\ 
 & \textbf{des} niht von der \textbf{edeln} art:\\ 
 & ez was ein samît bastart.\\ 
 & Ein kulter \textbf{was} des bettes dach,\\ 
 & niuwan durch Gawanes gemach,\\ 
15 & mit einem pfellel sunder golt,\\ 
 & verre \textit{\textbf{ûz}} heidenschaft geholt,\\ 
 & gesteppet \textbf{ûz} palmât.\\ 
 & dar über zôch man \textbf{lînwât},\\ 
 & zwei lîlachen snêvar.\\ 
20 & man leite wan\textit{k}üssen dar\\ 
 & unde der megede mantel einen,\\ 
 & hermîn, niuwen, reinen.\\ 
 & mit urloube er \textbf{si} \textbf{undervienc},\\ 
 & der wirt, ê \textbf{daz} er slâfen gienc.\\ 
25 & Gawan aleine \textbf{unde diu maget}\\ 
 & \textbf{bliben} dâ, \textbf{wart mir gesaget}.\\ 
 & het er iht \textbf{an si} \textbf{gegert},\\ 
 & ich wæne, si hetes in gewert.\\ 
 & er sol ouch slâfen, ob er mac.\\ 
30 & got hüete \textbf{sîn}, sô kome der tac.\\ 
\end{tabular}
\scriptsize
\line(1,0){75} \newline
T U V W O Q R Fr39 \newline
\line(1,0){75} \newline
\textbf{1} \textit{Initiale} O Q Fr39   $\cdot$ \textit{Capitulumzeichen} R  \textbf{5} \textit{Majuskel} T  \textbf{13} \textit{Majuskel} T  \newline
\line(1,0){75} \newline
\textbf{1} kunde] ÷vnde O  $\cdot$ Gawan] Gawin R \textbf{2} möhter] mohter T (U) (V) (O) (Q) (Fr39) moͯch er R  $\cdot$ dâ] do U V W Q Fr39  $\cdot$ nern] ernern V (W) R Fr39 \textbf{3} gundes] gonde V (W) (O) (Q) (R) (Fr39) \textbf{4} im] \textit{om.} T R  $\cdot$ des] das R \textbf{5} Dô] Da O  $\cdot$ her] hin V \textit{om.} R \textbf{7} dar nâher] dar vf do V darnahen R \textbf{8} Gawane] an Gawan V Gawin R \textbf{9} plûmît] pfulmit W R \textbf{10} des] Div O  $\cdot$ samît] gruͤne semit V \textbf{12} ein] \textit{om.} O  $\cdot$ bastart] von pasthart V botschart Q \textbf{13} Ein] Eine U  $\cdot$ kulter] kvter V \textbf{14} Gawanes] Gawans U W O (Q) Fr39 Gawins R \textbf{16} verre] Were R  $\cdot$ ûz] \textit{om.} T auß der W (R) \textbf{17} ûz] vssen V vff Q \textbf{18} lînwât] linde wat V \textbf{20} leite] legt im Q  $\cdot$ wanküssen] wantkvssin T ein wankussen U (V) (R) ein banckússein W ein wanckûsse Fr39 \textbf{21} einen] einem R \textbf{22} hermîn] Herre min Q Hernin R  $\cdot$ niuwen] nuwe vnde V \textbf{23} er si] erz O ers Q R Fr39 \textbf{24} der wirt] \textit{om.} V  $\cdot$ er] der wurt V \textbf{25} Gawan] Gawin R  $\cdot$ unde diu maget] ist mir gesaget V \textbf{26} Bleip aldo mit im die maget V  $\cdot$ dâ] do W Q Fr39 alda R \textbf{27} iht] iht anders V ihtes O (Q) (R) Fr39 \textbf{30} sô] do O R bisz Q  $\cdot$ kome] kam W R \newline
\end{minipage}
\end{table}
\end{document}
