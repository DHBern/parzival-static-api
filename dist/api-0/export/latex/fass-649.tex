\documentclass[8pt,a4paper,notitlepage]{article}
\usepackage{fullpage}
\usepackage{ulem}
\usepackage{xltxtra}
\usepackage{datetime}
\renewcommand{\dateseparator}{.}
\dmyyyydate
\usepackage{fancyhdr}
\usepackage{ifthen}
\pagestyle{fancy}
\fancyhf{}
\renewcommand{\headrulewidth}{0pt}
\fancyfoot[L]{\ifthenelse{\value{page}=1}{\today, \currenttime{} Uhr}{}}
\begin{document}
\begin{table}[ht]
\begin{minipage}[t]{0.5\linewidth}
\small
\begin{center}*D
\end{center}
\begin{tabular}{rl}
\textbf{649} & \begin{large}D\end{large}iu botschaft den knappen \textbf{twanc},\\ 
 & daz ern ruochte, wer in \textbf{dranc},\\ 
 & unz in der \textbf{künec} selbe \textbf{sach},\\ 
 & der \textbf{sîn grüezen} gein im sprach.\\ 
5 & \textbf{Der knappe} gab im einen brief,\\ 
 & der Artuse in sîn herze rief,\\ 
 & dô er von im \textbf{wart} gelesen,\\ 
 & dô muoser bî beiden wesen:\\ 
 & daz eine was vreude \textbf{unt}z ander klage.\\ 
10 & Er sprach: "wol disem \textbf{süezem} tage,\\ 
 & bî des liehte ich hân vernomen:\\ 
 & \textbf{mir sint diu wâren mære} komen\\ 
 & umbe \textbf{mîner} werden swester sun.\\ 
 & kan ich \textbf{manlîch} dienst tuon\\ 
15 & durch sippe unt durch geselleschaft,\\ 
 & ob triwe \textbf{an mir gewan ie} kraft,\\ 
 & sô leist ich, daz mir Gawan\\ 
 & hât enboten, ob ich kan."\\ 
 & Hin zem knappen sprach er dô:\\ 
20 & "\textbf{nû} sage \textbf{mir}, ist Gawan vrô?"\\ 
 & "jâ, hêrre, ob ir wellet,\\ 
 & ze\textbf{r vreude} er sich gesellet",\\ 
 & \textbf{sus} sprach der knappe wîse.\\ 
 & "er schiede gar von prîse,\\ 
25 & ob ir in liezet under wegen.\\ 
 & wer solt ouch dâ bî vreuden pflegen?\\ 
 & iuwer trôst im zücket vreude enbor,\\ 
 & unz ûzerhalp der riuwe tor\\ 
 & \textbf{von} sîme herzen \textbf{kumber} jagt,\\ 
30 & daz ir \textbf{an} im \textbf{iht} sît verzagt.\\ 
\end{tabular}
\scriptsize
\line(1,0){75} \newline
D \newline
\line(1,0){75} \newline
\textbf{1} \textit{Initiale} D  \textbf{5} \textit{Majuskel} D  \textbf{10} \textit{Majuskel} D  \textbf{19} \textit{Majuskel} D  \newline
\line(1,0){75} \newline
\newline
\end{minipage}
\hspace{0.5cm}
\begin{minipage}[t]{0.5\linewidth}
\small
\begin{center}*m
\end{center}
\begin{tabular}{rl}
 & diu botschaft den knappen \textbf{dranc},\\ 
 & daz er enruochte, wer in \textbf{twanc},\\ 
 & unz in der \textbf{künic} selber \textbf{sach},\\ 
 & der \textbf{sîn grüezen} gegen im sprach.\\ 
5 & \textbf{der knappe} gap im einen brief,\\ 
 & der Artuse in sîn herze rief,\\ 
 & dô er von im \textbf{wart} gele\textit{s}en,\\ 
 & dô muost er bî \textbf{in} beiden wesen:\\ 
 & daz ein was vröude, daz ander klage.\\ 
10 & er sprach: "wol disem tage,\\ 
 & bî des lieht ich hân vernomen\\ 
 & \textbf{liebiu mære, diu mir sint} komen\\ 
 & umb \textbf{mînen} werden swester sun.\\ 
 & kan ich \textbf{im} \textbf{manlîch} diens\textit{t} tuon\\ 
15 & durch sippe und durch geselleschaft,\\ 
 & ob triuwe \textbf{an mir gewan ie} kraft,\\ 
 & sô leist ich, daz mir Gawan\\ 
 & het enboten, ob ich kan."\\ 
 & hin zem knappen sprach er dô:\\ 
20 & "\textbf{nû} sage \textbf{mir}, ist Gawan vrô?"\\ 
 & "jâ, hêrre, ob ir wellet,\\ 
 & ze\textbf{r vröude} er sich gesellet",\\ 
 & \textbf{sus} sprach der knappe wîse.\\ 
 & "er schiede gar von prîse,\\ 
25 & ob ir in liezet under wegen.\\ 
 & wer solte ouch dâ bî vröude pflegen?\\ 
 & iuwer trôst im zücket vröude enbor,\\ 
 & unz ûzerhalp der riuwe tor\\ 
 & \textbf{von} sînem herzen \textbf{kumber} jaget,\\ 
30 & daz ir \textbf{an} im \textbf{iht} sît verzaget.\\ 
\end{tabular}
\scriptsize
\line(1,0){75} \newline
m n o Fr69 \newline
\line(1,0){75} \newline
\newline
\line(1,0){75} \newline
\textbf{1} dranc] twang o \textbf{2} twanc] trang o \textbf{5} brief] [pries]: prieff m \textbf{6} Artuse] artuͯse o \textbf{7} gelesen] gelestet senden m \textbf{8} in] den n dem o \textbf{10} tage] suͯssen tage n (o) \textbf{12} mære] mir o \textbf{14} manlîch dienst] manlich dienste m manlichen dienst n \textbf{16} Ob drú g:wan an mir wol craft o \textbf{18} kan] gan o \textbf{24} von] vnd o \textbf{25} liezet] liessen n \textbf{27} zücket] zúchtet o \newline
\end{minipage}
\end{table}
\newpage
\begin{table}[ht]
\begin{minipage}[t]{0.5\linewidth}
\small
\begin{center}*G
\end{center}
\begin{tabular}{rl}
 & \begin{large}D\end{large}iu botschaft den knappen \textbf{twanc},\\ 
 & daz ern ruochte, wer in \textbf{dranc},\\ 
 & unze in der \textbf{wirt} selbe \textbf{ersach},\\ 
 & der \textbf{sînen gruoz} gein im sprach.\\ 
5 & \textbf{in die hant} gab \textbf{er} im einen brief,\\ 
 & der Artus in sî\textit{n} herze rief,\\ 
 & dô er von im \textbf{was} gelesen,\\ 
 & dô muos er bî \textbf{den} beiden wesen:\\ 
 & daz ein was vröude, daz ander klage.\\ 
10 & er sprach: "wol disem \textbf{süezem} tage,\\ 
 & bî des liehte ich hân vernomen:\\ 
 & \textbf{mir sint diu wâren mære} komen\\ 
 & umbe \textbf{mînen} werden swester sun.\\ 
 & kan ich \textbf{manlîchen} dienst tuon\\ 
15 & durch sippe unde durch geselleschaft,\\ 
 & ob triuwe \textbf{gewan ie an mir} kraft,\\ 
 & sô leist ich, daz mir Gawan\\ 
 & hât enboten, ob ich kan."\\ 
 & hin ze dem knappen sprach er dô:\\ 
20 & "sag \textbf{an}, ist Gawan vrô?"\\ 
 & "jâ, hêrre, ob ir wellet,\\ 
 & ze \textbf{vröuden} er sich gesellet",\\ 
 & sprach der knappe wîse.\\ 
 & "er schiede \textbf{ouch} gar von brîse,\\ 
25 & ob ir in liezet under wegen.\\ 
 & wer solde ouch dâ bî vröude pflegen?\\ 
 & iuwer trôst im zücket vröude enbor,\\ 
 & unze ûzerhalp der riuwen tor\\ 
 & \textbf{ûz} sînem herzen \textbf{sorge} jaget,\\ 
30 & daz ir \textbf{an} im \textbf{niht} sît verzaget.\\ 
\end{tabular}
\scriptsize
\line(1,0){75} \newline
G I L M Z \newline
\line(1,0){75} \newline
\textbf{1} \textit{Initiale} G L Z  \textbf{5} \textit{Initiale} I  \textbf{19} \textit{Initiale} I  \newline
\line(1,0){75} \newline
\textbf{2} ern ruochte] en ruchte M er enruht Z \textbf{3} ersach] sach I \textbf{6} Artus] Artuͯse L artuse M  $\cdot$ sîn] sine G sinen Z \textbf{7} dô] Da M Z \textbf{8} dô] Da M Z  $\cdot$ muos] muͤster I  $\cdot$ den] \textit{om.} Z \textbf{9} klage] was chlage I \textbf{10} süezem] suͯszen L (Z) susszē M \textbf{11} liehte] lýht L \textbf{12} wâren] rehten I ware M \textbf{14} manlîchen] manlich I L (Z) \textbf{16} ie an mir] an mir ie L (M) Z \textbf{17} ich] \textit{om.} Z \textbf{19} dô] da M \textbf{22} ze] Zcu der M  $\cdot$ gesellet] stellet I gellet L \textbf{24} schiede] schiet M (Z) \textbf{26} vröude] freuden I (L) (Z) \textbf{28} der] \textit{om.} Z  $\cdot$ riuwen] triwen I (L) (M) \textbf{29} ûz] Von Z \newline
\end{minipage}
\hspace{0.5cm}
\begin{minipage}[t]{0.5\linewidth}
\small
\begin{center}*T
\end{center}
\begin{tabular}{rl}
 & diu botschaft den knaben \textbf{twanc},\\ 
 & daz ern ruochte, wer in \textbf{dranc},\\ 
 & unz in der \textbf{künic} selbe \textbf{ersach},\\ 
 & der \textbf{sînen gruoz} gên im \textbf{ouch} sprach.\\ 
5 & \textbf{der knabe} gap im einen brief,\\ 
 & der Artuse in sîn herze rief,\\ 
 & dô er von im \textbf{was} gelesen,\\ 
 & dô muost \textit{er} bî \textbf{den} beiden wesen:\\ 
 & daz ein was vreude, daz ander klage.\\ 
10 & er sprach: "wol disem \textbf{süezen} tage,\\ 
 & bî des lieht ich hân vernomen,\\ 
 & \textbf{mir sîn diu wâren mære} komen\\ 
 & umb \textbf{mînen} werden swester sun.\\ 
 & kan ich \textbf{menlîchen} dienst  tuon\\ 
15 & durch sippe und durch geselleschaft,\\ 
 & ob triuwe \textbf{an mir gewan ie} kraft,\\ 
 & sô leist ich, daz mir Gawan\\ 
 & hât enboten, ob ich kan."\\ 
 & hin zuom knaben sprach er dô:\\ 
20 & "\textbf{nû} sag \textbf{mir}, ist Gawan vrô?"\\ 
 & "jâ, hêrre, ob ir wellet,\\ 
 & zuo \textbf{der vreude} er sich gesellet",\\ 
 & sprach der knabe wîse.\\ 
 & "er schiede \textbf{ouch} gar von prîse,\\ 
25 & ob ir in liezet under wegen.\\ 
 & wer solt ouch dâ bî vreuden pflegen?\\ 
 & iwer trôst im zücket vreude enbor,\\ 
 & unz ûzerhalp der riwen tor\\ 
 & \textbf{von} sînem herzen \textbf{kumber} jaget,\\ 
30 & daz ir \textbf{von} im \textbf{niht} sît verzaget.\\ 
\end{tabular}
\scriptsize
\line(1,0){75} \newline
Q R W V \newline
\line(1,0){75} \newline
\textbf{1} \textit{Initiale} V   $\cdot$ \textit{Capitulumzeichen} R  \newline
\line(1,0){75} \newline
\textbf{1} botschaft] botschfft W \textbf{2} in dranc] im dankt R \textbf{3} selbe ersach] selber sach V \textbf{4} ouch] \textit{om.} V  $\cdot$ sprach] iach R \textbf{6} Artuse] Artus R  $\cdot$ rief] lieff W \textbf{8} er] \textit{om.} Q \textbf{9} was] \textit{om.} R W  $\cdot$ ander] andere waz V \textbf{10} süezen] [svͤze*]: svͤzem V \textbf{12} sîn] sind R (W) (V)  $\cdot$ diu wâren] waren W [*]: liebe V \textbf{14} ich] [*]: ich im V  $\cdot$ menlîchen dienst] manliche dienste R \textbf{15} sippe] sige R [*]: sippe V \textbf{20} Gawan] Gawin R \textbf{22} zuo der vreude] Zer froͯden R Zuͦ froͤden W [Zr]: Ze froͤiden V \textbf{24} [E*]: Er schiede gar von prise V \textbf{26} solt] sol R  $\cdot$ ouch] \textit{om.} W  $\cdot$ vreuden] froͯde R \textbf{28} riuwen] trúwen R reúwe W [trv́we]: rv́we V \textbf{29} kumber] ez kvnber V \textbf{30} von] an R W V  $\cdot$ niht] [*iht]: niht V \newline
\end{minipage}
\end{table}
\end{document}
