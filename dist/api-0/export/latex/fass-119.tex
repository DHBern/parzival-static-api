\documentclass[8pt,a4paper,notitlepage]{article}
\usepackage{fullpage}
\usepackage{ulem}
\usepackage{xltxtra}
\usepackage{datetime}
\renewcommand{\dateseparator}{.}
\dmyyyydate
\usepackage{fancyhdr}
\usepackage{ifthen}
\pagestyle{fancy}
\fancyhf{}
\renewcommand{\headrulewidth}{0pt}
\fancyfoot[L]{\ifthenelse{\value{page}=1}{\today, \currenttime{} Uhr}{}}
\begin{document}
\begin{table}[ht]
\begin{minipage}[t]{0.5\linewidth}
\small
\begin{center}*D
\end{center}
\begin{tabular}{rl}
\textbf{119} & si wolt ir schal verk\textit{r}enken.\\ 
 & ir bûliute unt ir enken,\\ 
 & \textbf{die} hiez si \textbf{vaste} gâhen,\\ 
 & vogele \textbf{würgen} und vâhen.\\ 
5 & \textit{die} vogele wâren baz geriten;\\ 
 & \textbf{etslîches} sterben wart vermiten.\\ 
 & \textbf{der} beleip dâ lebendic ein teil,\\ 
 & die sît mit \textbf{sange} \textbf{wurden} geil.\\ 
 & \textit{\begin{large}D\end{large}}er knappe sprach zer künegîn:\\ 
10 & "waz wîzet man den vogelîn?"\\ 
 & er gert \textbf{in} vrides \textbf{sân} zestunt.\\ 
 & \textbf{sîn} muoter \textbf{kust in} an den munt.\\ 
 & \textbf{Diu} sprach: "wes wende ich sîn gebot,\\ 
 & der doch ist der hœhste got?\\ 
15 & sulen vogele durch mich vreude lân?"\\ 
 & der knappe sprach zer muoter sân:\\ 
 & "owê muoter, wa\textit{z} ist got?"\\ 
 & "sun, ich sage dirz âne spot:\\ 
 & er ist noch liehter denne \textbf{der} tac,\\ 
20 & der antlützes sich bewac\\ 
 & nâch \textbf{menschen} antlitze.\\ 
 & su\textit{n}, merke eine witze\\ 
 & unt vlêhe \textbf{in} umb dîne nôt.\\ 
 & sîn triwe der \textbf{werlde} ie helfe bôt.\\ 
25 & Sô heizet einer der hellewirt:\\ 
 & der ist swarz, untriwe in niht verbirt.\\ 
 & von dem kêre dîne gedanke\\ 
 & unt ouch von zwîvels wanke."\\ 
 & \textbf{Sîn muoter underschiet} im gar\\ 
30 & daz vinster unt daz lieht gevar.\\ 
\end{tabular}
\scriptsize
\line(1,0){75} \newline
D \newline
\line(1,0){75} \newline
\textbf{9} \textit{Initiale} D  \textbf{13} \textit{Majuskel} D  \textbf{25} \textit{Majuskel} D  \textbf{29} \textit{Majuskel} D  \newline
\line(1,0){75} \newline
\textbf{1} verkrenken] verchenchen D \textbf{5} die] \textit{om.} D \textbf{9} Der] ÷er D \textbf{17} waz] was D \textbf{22} sun] sv D \newline
\end{minipage}
\hspace{0.5cm}
\begin{minipage}[t]{0.5\linewidth}
\small
\begin{center}*m
\end{center}
\begin{tabular}{rl}
 & si wolte ir schal verkrenken.\\ 
 & ir bûliute und ir enken\\ 
 & hiez si \textbf{vaste} gâhen,\\ 
 & vogele \textbf{würgen} und vâhen.\\ 
5 & die vogele wâren b\textit{az} geriten;\\ 
 & \textbf{etlîchen} sterben wart vermiten.\\ 
 & \textbf{der} beleip dâ lebendic ein teil,\\ 
 & die s\textit{î}t mit \textbf{sangen} \textit{\textbf{wâren}} geil.\\ 
 & \begin{large}D\end{large}er knabe sprach zuo der künigîn:\\ 
10 & "waz wî\textit{z}et man den vogelîn?"\\ 
 & er gerte vrides \textbf{sô} zestunt.\\ 
 & \textbf{sîn} muoter \textbf{kuste in} an den munt\\ 
 & \textbf{und} sprach: "wes wendich sîn gebot,\\ 
 & der doch ist der hœheste got?\\ 
15 & sullen vogele durch mich vröude lân?"\\ 
 & der k\textit{n}abe sprach zer muoter sân:\\ 
 & "owê muoter, waz ist got?"\\ 
 & "sun, ich sage dir ez âne spot:\\ 
 & er ist noch liehter denne tac,\\ 
20 & der antlitzes sich bewac\\ 
 & nâch \textbf{m\textit{a}nnes} antlitze.\\ 
 & sun, merke ein witze\\ 
 & und vlêhe \textbf{ime} umb dîne nôt.\\ 
 & sîn triuwe der \textbf{wer\textit{d}e} ie helfe bôt.\\ 
25 & sô heizet einer der hellewirt:\\ 
 & der ist swarz \textbf{und} untriuwe in niht verbirt.\\ 
 & von dem kêre dîne gedanke\\ 
 & und ouch von zwîvels wanke."\\ 
 & \textbf{alsus undersch\textit{ie}t si} im gar\\ 
30 & daz vinster und daz lieht gevar.\\ 
\end{tabular}
\scriptsize
\line(1,0){75} \newline
m n o \newline
\line(1,0){75} \newline
\textbf{9} \textit{Initiale} m   $\cdot$ \textit{Capitulumzeichen} n  \newline
\line(1,0){75} \newline
\textbf{3} hiez si] Sú hiesse n \textbf{5} baz geriten] begeritten m \textbf{6} etlîchen] Etliches n (o) \textbf{7} der] Den o  $\cdot$ dâ] do n o \textbf{8} sît] sint m  $\cdot$ sangen] sange n  $\cdot$ wâren] \textit{om.} m \textbf{10} wîzet] wiset m zihet n o  $\cdot$ den] die n o \textbf{11} gerte] gert in n o \textbf{12} kuste] kust n (o) \textbf{13} wes] was o  $\cdot$ sîn] din n o \textbf{16} knabe] [spr]: kabe m \textbf{19} tac] der tag n o \textbf{20} antlitzes] antlitze es n anczliczes o \textbf{21} mannes] minnes m  $\cdot$ antlitze] ancztlit o \textbf{24} sîn] Din o  $\cdot$ werde] werdie m  $\cdot$ ie helfe] nie heffe o \textbf{26} swarz] swart n  $\cdot$ und] \textit{om.} n o  $\cdot$ in] er n des o \textbf{29} underschiet] vnderscheid m \newline
\end{minipage}
\end{table}
\newpage
\begin{table}[ht]
\begin{minipage}[t]{0.5\linewidth}
\small
\begin{center}*G
\end{center}
\begin{tabular}{rl}
 & si wolt ir schal verkrenken.\\ 
 & ir bûliute und ir enken\\ 
 & hiez si \textbf{balde} gâhen,\\ 
 & vogele \textbf{würgen} und vâhen.\\ 
5 & die vogele wâren baz geriten;\\ 
 & \textbf{etslîches} sterben wart vermiten.\\ 
 & \textbf{ir} beleip dâ lebendic ein teil,\\ 
 & die sît mit \textbf{sange} \textbf{wurden} geil.\\ 
 & der knappe sprach zer künigîn:\\ 
10 & "waz wîzet man den vogelîn?"\\ 
 & er gert \textbf{in} vrides \textbf{sân} zestunt.\\ 
 & \textbf{diu} muoter \textbf{kuste in} an den munt.\\ 
 & \textbf{si} sprach: "wes wende ich sîn gebot,\\ 
 & der doch ist der hœheste got?\\ 
15 & sulen vogele durch mich vröude lân?"\\ 
 & der knappe sprach zer muoter sân:\\ 
 & "owê muoter, waz ist got?"\\ 
 & "sun, ich sage dirz âne spot:\\ 
 & er ist noch liehter dane \textbf{der} tac,\\ 
20 & der antlützes sich bewac\\ 
 & \begin{large}N\end{large}âch \textbf{menschen} antlitze.\\ 
 & sun, merke eine witze\\ 
 & \textit{und} vlêge \textbf{in} umbe dîne nôt.\\ 
 & sîn triuwe der \textbf{werl\textit{t}} \textit{i}e helfe bôt.\\ 
25 & sô heizet einer der hellewirt:\\ 
 & der ist swarz, untriwe in niht verbirt.\\ 
 & von dem kêre dîne gedanke\\ 
 & unde ouch von zwîvels wanke."\\ 
 & \textbf{sîn muoter underschiet} im gar\\ 
30 & daz vinster un\textit{d} \textit{d}az lieht gevar.\\ 
\end{tabular}
\scriptsize
\line(1,0){75} \newline
G I O L M Q R Z \newline
\line(1,0){75} \newline
\textbf{1} \textit{Initiale} O M  \textbf{9} \textit{Initiale} I L Q R Z  \textbf{21} \textit{Initiale} G  \newline
\line(1,0){75} \newline
\textbf{1} si] ÷i O  $\cdot$ wolt] en wolde M  $\cdot$ verkrenken] crenchen I \textbf{2} ir bûliute] Sir belute M Jr blvte Z \textbf{3} balde] bade R \textbf{4} vogele] Die vogel Q  $\cdot$ vâhen] hahen I L \textbf{5} \textit{Versfolge 119.6-5} Z   $\cdot$ die] Wan die Z  $\cdot$ wâren] worden Q  $\cdot$ baz] wol Z  $\cdot$ geriten] beriten Q \textbf{6} etslîches] Etslicher O Etlichen Q \textbf{7} ir] Er Q  $\cdot$ dâ] do Q \textbf{8} sît] sein Q  $\cdot$ mit] \textit{om.} L \textbf{10} wîzet] wirret M \textbf{11} gert] gerte L bat M  $\cdot$ sân] so L \textbf{12} diu] Sin O M Z  $\cdot$ kuste] kust I (Q) R Z \textbf{13} wes] we wes O Q R \textit{om.} Z \textbf{14} got] hort Q \textbf{15} vogele] voglin Z  $\cdot$ mich] in Z  $\cdot$ vröude] ir singen I \textbf{16} zer] zvͤ siner I \textbf{17} owê] Awe O Q Owý L  $\cdot$ waz] wor M \textbf{18} dirz] dir O M  $\cdot$ âne] svnder L \textbf{19} liehter] lýchter L (M) (Q) \textbf{20} der] Desz Q  $\cdot$ sich] sic M  $\cdot$ bewac] bewand Q \textbf{21} Nâch] Duͯrch L  $\cdot$ menschen] mannes O (L) (M) (Q) R Z \textbf{22} eine] meyn Q \textbf{23} und] \textit{om.} G \textbf{24} wand er ie der werlt helfe bot I  $\cdot$ werlt ie] werlt werlt ie G werlde hie Z \textbf{25} sô] Do Q \textbf{27} von] Vnd L \textbf{28} zwîvels] des zwiuel I zwiuel L  $\cdot$ wanke] Gewanc I wenk R \textbf{29} underschiet] vnter scheidet Q \textbf{30} daz vinster] Die vinster I Daz vinster vinster Z  $\cdot$ und daz] vnde och daz G  $\cdot$ lieht] lýcht L (M) (Q)  $\cdot$ gevar] var M \newline
\end{minipage}
\hspace{0.5cm}
\begin{minipage}[t]{0.5\linewidth}
\small
\begin{center}*T (U)
\end{center}
\begin{tabular}{rl}
 & si wolte ir schal verkrenken.\\ 
 & ir bûliute und ir enken\\ 
 & hiez si \textbf{balde} gâhen,\\ 
 & vogele \textbf{werfen} und vâhen.\\ 
5 & die vogele wâren baz geriten;\\ 
 & \textbf{etslîcher} sterben wart vermiten.\\ 
 & \textbf{ir} beleip dâ lebendic ein teil,\\ 
 & die sît mit \textbf{sange} \textbf{wurden} geil.\\ 
 & der knabe sprach zuo der künegîn:\\ 
10 & "waz wîzet man den vogelîn?"\\ 
 & er gerte \textbf{in} vrides \textbf{dô} zestunt.\\ 
 & \textbf{sîn} muoter \textbf{in kuste} an den munt.\\ 
 & \textbf{si} sprach: "wes wende ich sîn gebot,\\ 
 & der doch ist der hœheste got?\\ 
15 & soln \textbf{die} vogele durch \textit{mich} vre\textit{u}de lân?"\\ 
 & der knabe sprach zuo der muoter sân:\\ 
 & "owê muoter, waz ist got?"\\ 
 & "sun, ich sage dir ez âne spot:\\ 
 & er ist noch liehter dan \textbf{der} tac,\\ 
20 & de\textit{r} antlitzes sich bewac\\ 
 & nâch \textbf{mannes} antlitze.\\ 
 & sun, merke eine witze\\ 
 & und vlêhe \textbf{im} umb dîn nôt.\\ 
 & sîn triuwe der \textbf{werlde} ie helfe bôt.\\ 
25 & sô heizet einer der hellewirt:\\ 
 & der ist swarz, untriuwe in niht verbirt.\\ 
 & von dem kêre dîne gedenke\\ 
 & und ouch von zwîvels wenke."\\ 
 & \textbf{sîn muoter underschiet} im gar\\ 
30 & daz vinster und daz lieht gevar.\\ 
\end{tabular}
\scriptsize
\line(1,0){75} \newline
U V W T \newline
\line(1,0){75} \newline
\textbf{5} \textit{Majuskel} T  \textbf{9} \textit{Initiale} W T  \textbf{12} \textit{Majuskel} T  \textbf{16} \textit{Majuskel} T  \textbf{17} \textit{Majuskel} T  \textbf{25} \textit{Majuskel} T  \textbf{29} \textit{Majuskel} T  \newline
\line(1,0){75} \newline
\textbf{2} bûliute] buͦwen leúte W \textbf{3} hiez] Die hies V (W) \textbf{5} baz] auch bas W \textbf{6} etslîcher] Etteliches V  $\cdot$ wart] waz V \textbf{7} dâ] do V W  $\cdot$ ein] doch ein W \textbf{10} wîzet] wirret W  $\cdot$ den] die W \textbf{11} in] ir V  $\cdot$ dô] so W sa T \textbf{12} in kuste] kvste in V (W) (T) \textbf{13} wes] was W \textbf{14} hœheste] oberste T \textbf{15} die] \textit{om.} T  $\cdot$ vogele] voͤgellin V  $\cdot$ durch mich vreude] durch vrede U froͤd durch mich W \textbf{17} muoter] frauwe W \textbf{18} ez] \textit{om.} W \textbf{20} der] Des U  $\cdot$ antlitzes] antlútze W \textbf{22} eine] ein ê U \textbf{23} im] in W [im]: in T \textbf{24} ie] \textit{om.} W \textbf{25} hellewirt] hellen wirt W \newline
\end{minipage}
\end{table}
\end{document}
