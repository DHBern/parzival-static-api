\documentclass[8pt,a4paper,notitlepage]{article}
\usepackage{fullpage}
\usepackage{ulem}
\usepackage{xltxtra}
\usepackage{datetime}
\renewcommand{\dateseparator}{.}
\dmyyyydate
\usepackage{fancyhdr}
\usepackage{ifthen}
\pagestyle{fancy}
\fancyhf{}
\renewcommand{\headrulewidth}{0pt}
\fancyfoot[L]{\ifthenelse{\value{page}=1}{\today, \currenttime{} Uhr}{}}
\begin{document}
\begin{table}[ht]
\begin{minipage}[t]{0.5\linewidth}
\small
\begin{center}*D
\end{center}
\begin{tabular}{rl}
\textbf{726} & \begin{large}V\end{large}ür die küneginne man \textbf{dô} truoc\\ 
 & daz trinken. trunken si genuoc,\\ 
 & die rîter unt die vrouwen gar,\\ 
 & si wurden deste baz gevar.\\ 
5 & Man truog ouch trinken \textbf{dort} hin în\\ 
 & Artus unde Brandelidelin.\\ 
 & der schenke gienc \textbf{her} wider dan,\\ 
 & Artus sîne rede alsus huob an:\\ 
 & "hêr künec, nû lât \textbf{si}z alsô \textbf{tuon},\\ 
10 & daz \textbf{der künec, iwerre swester sun},\\ 
 & \textbf{mîner swester sun mir het} erslagen.\\ 
 & wolt er denne minne tragen\\ 
 & gein \textbf{mîner niftel}, der magt,\\ 
 & diu im ir kumber ouch dort klagt,\\ 
15 & dâ wir si liezen sitzen,\\ 
 & vüere si denne mit witzen,\\ 
 & si würde im \textbf{nimmer drumbe} holt\\ 
 & und teilte im \textbf{solhen hazzes} solt,\\ 
 & dês \textbf{den künec} m\textit{ö}hte \textbf{erdriezen},\\ 
20 & wolt er ir \textbf{iht} geniezen.\\ 
 & swâ haz die minne undervert,\\ 
 & dem stætem herzen \textbf{vreude er} wert."\\ 
 & Dô sprach der künec \textbf{von} Punturtoys\\ 
 & zArtuse dem Bertenoys:\\ 
25 & "hêrre, si sint unserer swester kint,\\ 
 & die gein ein ander in hazze sint.\\ 
 & wir sulen den \textbf{kampf} understên,\\ 
 & \textbf{dâ}\textbf{ne} mac niht anders \textbf{an} ergên,\\ 
 & wan daz \textbf{si} ein ander minnen\\ 
30 & mit \textbf{herzenlîchen} sinnen.\\ 
\end{tabular}
\scriptsize
\line(1,0){75} \newline
D \newline
\line(1,0){75} \newline
\textbf{1} \textit{Initiale} D  \textbf{5} \textit{Majuskel} D  \textbf{23} \textit{Majuskel} D  \newline
\line(1,0){75} \newline
\textbf{19} möhte] mohte D \newline
\end{minipage}
\hspace{0.5cm}
\begin{minipage}[t]{0.5\linewidth}
\small
\begin{center}*m
\end{center}
\begin{tabular}{rl}
 & vür die künigîn \textit{man} \textit{\textbf{dô}} truoc\\ 
 & daz trinken. trunken si genuoc,\\ 
 & die ritter und die vrowen gar,\\ 
 & si wurden deste baz gevar.\\ 
5 & man truoc ouch trinken \textbf{dort} hin în\\ 
 & Artuse und Brandelin.\\ 
 & der schenk gienc \textbf{hin} wider dan,\\ 
 & Artus sîn rede alsus huop an:\\ 
 & "hêr künic, nû lât daz alsô \textbf{sîn},\\ 
10 & daz \textbf{iuwer neve den neven mîn}\\ 
 & \textbf{in einem kreize habe} erslagen.\\ 
 & wolte er danne minne tragen\\ 
 & gegen \textbf{mîne\textit{r} \textit{n}iftel}, der maget,\\ 
 & diu im ir kumber ouch dort klaget,\\ 
15 & d\textit{â} wir si liezen sitzen,\\ 
 & \dag unêre\dag  si dan mit witzen,\\ 
 & si würde im \textbf{nimmer dâr umb} holt\\ 
 & und teilte im \textbf{solichen hazzes} solt,\\ 
 & dês \textbf{in} m\textit{ö}hte \textbf{verdriezen},\\ 
20 & wol\textit{t} er ir \textbf{iht} geniezen.\\ 
 & wâ haz die minne undervert,\\ 
 & dem stæten herzen \textbf{vröude ez} wert."\\ 
 & dô sprach der künic \textbf{von} Ponturteis\\ 
 & zuo Artuse dem Brittuneis:\\ 
25 & "hêrre, si sint unser swester kint,\\ 
 & die gegen ein ander in hazze sint.\\ 
 & wir sollen den \textbf{kampf} understân,\\ 
 & \textbf{dan} \textbf{ez} mac niht anders ergân,\\ 
 & wan daz \textbf{si} ein ander minnen\\ 
30 & mit \textbf{herzelîchen} sinnen.\\ 
\end{tabular}
\scriptsize
\line(1,0){75} \newline
m n o Fr69 \newline
\line(1,0){75} \newline
\newline
\line(1,0){75} \newline
\textbf{1} man dô] do man m \textbf{4} gevar] gewar o \textbf{6} Artuse] Artusen n  $\cdot$ Brandelin] brandelidelin n brandeledelin o \textbf{7} hin] \textit{om.} n \textbf{11} in einem] Minem Fr69 \textbf{12} wolte] [wol]: wolt Fr69 \textbf{13} niftel] mýnne niftel m nẏfftelen n (o)  $\cdot$ der] de Fr69 \textbf{14} im ir] mir Fr69 \textbf{15} dâ] Do m n o \textbf{16} unêre] Vnd ere o \textbf{17} dâr umb] [von]: darummb o \textbf{19} möhte] mohtte m (o) \textbf{20} wolt] Wol m \textbf{22} wert] gert o \textbf{23} Ponturteis] punterteisz o \textbf{24} Artuse] artuͯse o  $\cdot$ Brittuneis] brituneise n britaneis o \textbf{25} unser] vwer o \textbf{29} wan] Wenne n \textbf{30} herzelîchen] hertzeclichen n (o) \newline
\end{minipage}
\end{table}
\newpage
\begin{table}[ht]
\begin{minipage}[t]{0.5\linewidth}
\small
\begin{center}*G
\end{center}
\begin{tabular}{rl}
 & vür die küngîn man truoc\\ 
 & daz trinken. trunken si genuoc,\\ 
 & die rîter unde die vrouwen gar,\\ 
 & si wurden deste baz gevar.\\ 
5 & man truoc ouch trinken hin în\\ 
 & Artus unde Brandelidelin.\\ 
 & der schenke \textit{gie} \textbf{her} wider dan,\\ 
 & Artus sîn rede alsus huop an:\\ 
 & "hêr künec, nû lât \textbf{si}z alsô \textbf{tuon},\\ 
10 & daz \textbf{der künec, iwer swester sun},\\ 
 & \textbf{mîner swester sun mir het} erslagen,\\ 
 & \textbf{unde} wolde er dane minne tragen\\ 
 & gein \textbf{sîner swester}, der maget,\\ 
 & diu im ir kumber ouch dort klaget,\\ 
15 & dâ wir si liezen sitzen,\\ 
 & vüere si danne mit witzen,\\ 
 & \begin{large}S\end{large}i\textbf{ne} würde im \textbf{drumbe nimmer} holt\\ 
 & unde teilte im \textbf{sölchen hazzes} solt,\\ 
 & deis \textbf{den künec} m\textit{ö}ht \textbf{erdriezen},\\ 
20 & wolde er ir \textbf{iht} geniezen.\\ 
 & swâ haz die minne undervert,\\ 
 & dem stæten herzen \textbf{ez vröude} wert."\\ 
 & dô sprach der künec \textbf{ûz} Ponturteis\\ 
 & ze Artus dem Britaneis:\\ 
25 & "hêrre, si sint unser swester kint,\\ 
 & die gein ein ander in hazze sint.\\ 
 & wir sülen den \textbf{kampf} understên,\\ 
 & \textbf{dâ}\textbf{ne} mac niht anders \textbf{an} ergên,\\ 
 & wan daz \textbf{si} ein ander minnen\\ 
30 & mit \textbf{herzen unde mit} sinnen.\\ 
\end{tabular}
\scriptsize
\line(1,0){75} \newline
G I L M Z Fr20 Fr24 \newline
\line(1,0){75} \newline
\textbf{9} \textit{Initiale} Fr24  \textbf{17} \textit{Initiale} G I L Z  \newline
\line(1,0){75} \newline
\textbf{1} die] \textit{om.} Fr20 \textbf{2} daz trinken] trinchen do I  $\cdot$ genuoc] in Genuͤc I (L) \textbf{5} ouch] \textit{om.} L M Z Fr24 \textbf{6} Artus] Artusen I  $\cdot$ Brandelidelin] brandilidelin G prandalidelin I Branlidelin L brandlidelin M (Fr24) brandideli::: Fr20 \textbf{7} gie] \textit{om.} G  $\cdot$ her wider] hin I \textbf{8} sîn rede alsus huop] huͤp die rede alsus I hub sin rede svs L \textbf{9} hêr künec] DeR kvnich >sprach< Fr24 \textbf{10} der künec] Gramoflanz I \textbf{11} mir hete er slagin Fr20  $\cdot$ mir het] myr hat M het Z \textbf{12} minne] \textit{om.} M \textbf{13} sîner] miner Fr20  $\cdot$ der] dy M (Z) [s*]: dirre  Fr20 \textbf{14} im ir] ir Z mir Fr20  $\cdot$ dort] \textit{om.} M \textbf{17} Sine] Sie Fr24  $\cdot$ drumbe nimmer] nimmer drvmbe Z (Fr24)  $\cdot$ holt] \textit{om.} Fr20 \textbf{18} teilte] tailt I (L) (M) (Z)  $\cdot$ sölchen] solhes I selben L  $\cdot$ hazzes] hazen M \textbf{19} möht] moht G (I) (L) (M) (Z) Fr20 Fr24  $\cdot$ erdriezen] verdriezen I (M) (Z) bedrîezen Fr24 \textbf{20} wolde er ir] Woldir M \textbf{21} swâ] Wa L M \textbf{22} stæten] stetem I  $\cdot$ ez] ist I \textbf{23} dô] Da M  $\cdot$ ûz] von I  $\cdot$ Ponturteis] ponturtoys I pvntvrtoýs L poterteys M pvntvrtois Z Pvntvrteis Fr24 \textbf{24} ze] [her]: zuͤ I  $\cdot$ Artus] artuse I (L)  $\cdot$ dem] zuͯ dem L  $\cdot$ Britaneis] britanoeis G pritonoys I Brittanoýs L britaneys M britunois Z pritanoeis Fr20 \textbf{27} kampf] haz L (M) Fr24 \textbf{28} mac] solt I \textbf{30} unde] vnd auch I \newline
\end{minipage}
\hspace{0.5cm}
\begin{minipage}[t]{0.5\linewidth}
\small
\begin{center}*T
\end{center}
\begin{tabular}{rl}
 & vür die küneginne man truoc\\ 
 & daz trinken. trunken si genuoc,\\ 
 & die rîter und die vrouwen gar,\\ 
 & si wurden deste baz gevar.\\ 
5 & man truoc ouch trinken \textbf{dort} hin în\\ 
 & Artuse und Brandelidelin.\\ 
 & der schenke gienc \textbf{her} wider dan,\\ 
 & Artus sîne rede alsus huop an:\\ 
 & "hêr künec, nû lât \textbf{s\textit{i}} ez alsô \textbf{tuon},\\ 
10 & daz \textbf{der künec, iuwer swester sun},\\ 
 & \textbf{mîner swester sun habe} erslagen,\\ 
 & \textbf{und} wolt er dan minne tragen\\ 
 & gein \textbf{mîner nifteln}, der maget,\\ 
 & diu im irn kumber ouch dort klaget,\\ 
15 & d\textit{â} wir si liezen sitzen,\\ 
 & vüere si danne mit witzen,\\ 
 & si \textbf{en}würde im \textbf{dâr umb niemer} holt\\ 
 & und teilt im \textbf{hazzes solichen} solt,\\ 
 & daz ez \textbf{den künec} m\textit{ö}hte \textbf{erdriezen},\\ 
20 & wolt er ir geniezen.\\ 
 & \begin{large}W\end{large}â haz die minne undervert,\\ 
 & dem stæten herzen \textbf{ez vreude} wert."\\ 
 & dô sprach der künec \textbf{von} Punterteis\\ 
 & zuo Artuse dem Brituneis:\\ 
25 & "hêrre, si sint unser swester kint,\\ 
 & die gein ein ander in hazze sint.\\ 
 & wir soln den \textbf{haz} understân,\\ 
 & \textbf{dâ} \textbf{en}mac niht anders \textbf{ane} ergân\\ 
 & wan daz ein ander minnen\\ 
30 & mit \textbf{herzen und mit} sinnen.\\ 
\end{tabular}
\scriptsize
\line(1,0){75} \newline
U V W Q R \newline
\line(1,0){75} \newline
\textbf{1} \textit{Initiale} R  \textbf{21} \textit{Initiale} U W  \newline
\line(1,0){75} \newline
\textbf{2} \textit{statt 726.2-3 (korrigierende Versdoppelung):} Das trinken trunken mag gesin / Der Ritter vnd die frowen fin / Die Ritter vnd der frowen schar R  \textbf{3} gar] schar W \textbf{4} gevar] gafar Q gewar R \textbf{5} hin în] hin Q her In R \textbf{6} Artuse] Artus W Q R  $\cdot$ Brandelidelin] brandelidelein W brandlidelin Q \textbf{7} her] gieng R \textbf{9} [*er]: Her kv́nig nv [*]: lant sús also tvn V  $\cdot$ nû] \textit{om.} Q  $\cdot$ lât si ez] lat sich iz U losz sisz Q \textbf{10} [*]: Daz der kv́nig uwer swester svn V \textbf{11} [*]: Minre swester svn mir hette erslagen V  $\cdot$ habe] het W mir het Q R \textbf{14} im irn] [*]: im irn V Jm ir R  $\cdot$ klaget] treit Q \textbf{15} dâ] Do U V W Q \textbf{16} danne] den R \textbf{17} enwürde] wurde R  $\cdot$ dâr umb niemer] niemer drumbe V (R) nymer W \textbf{18} teilt] teilte Q R  $\cdot$ hazzes] hassens W \textbf{19} erdriezen] [*riessen]: erdriezzen V vertrissen Q (R) \textbf{20} ir] icht Q R \textbf{21} Wâ] Swo V dO W \textbf{22} dem] Den V  $\cdot$ herzen] bertzen W \textbf{23} Punterteis] Puͦnterteis U ponterteys V ponturtoys W puntertois Q puͯntruͯeis R \textbf{24} \textit{Vers 726.24 fehlt} Q   $\cdot$ Artuse] Artuͦse U artus W (R)  $\cdot$ Brituneis] Brituͦneis U brittuneys V britunoys W pritonois R \textbf{28} Denne mag anders nit ergen R  $\cdot$ dâ] Do V W Q \textbf{29} ein ander] sv́ einander V sy ein andren R \newline
\end{minipage}
\end{table}
\end{document}
