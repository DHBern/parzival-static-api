\documentclass[8pt,a4paper,notitlepage]{article}
\usepackage{fullpage}
\usepackage{ulem}
\usepackage{xltxtra}
\usepackage{datetime}
\renewcommand{\dateseparator}{.}
\dmyyyydate
\usepackage{fancyhdr}
\usepackage{ifthen}
\pagestyle{fancy}
\fancyhf{}
\renewcommand{\headrulewidth}{0pt}
\fancyfoot[L]{\ifthenelse{\value{page}=1}{\today, \currenttime{} Uhr}{}}
\begin{document}
\begin{table}[ht]
\begin{minipage}[t]{0.5\linewidth}
\small
\begin{center}*D
\end{center}
\begin{tabular}{rl}
\textbf{646} & \textbf{\textit{\begin{large}D\end{large}}az} Gawan von Artuse reit,\\ 
 & \textbf{sît} hât sorge und leit\\ 
 & mit \textbf{krache} ûf mich geleit ir \textbf{vlîz}.\\ 
 & mir sagete Meljanz von Liz,\\ 
5 & er sæhe in sît ze Parbigœl.\\ 
 & owî", sprach si, "Plimizœl,\\ 
 & daz dich mîn ouge ie gesach!\\ 
 & waz \textbf{mir doch leides dâ} \textbf{geschach}!\\ 
 & Cunneware de Lalant\\ 
10 & wart mir nimmer \textbf{mêr} bekant,\\ 
 & mîn \textbf{süeziu, werdiu} gespil.\\ 
 & tavelrunde wart dâ vil\\ 
 & mit rede ir reht gebrochen.\\ 
 & vünftehalp jâr unt sehs wochen\\ 
15 & ist, daz der werde Parzival\\ 
 & von dem Plimizœl nâch dem Grâl\\ 
 & reit. dô kêrt ouch Gawan\\ 
 & gein Ascalun, der werde man.\\ 
 & Jeschute und Eckuba\\ 
20 & schieden sich von mir \textbf{al} dâ.\\ 
 & grôz jâmer nâch der werden diet\\ 
 & mich \textbf{sît} von stæten vreuden schiet."\\ 
 & diu künegîn trûrens vil verjach.\\ 
 & hin zem knappen si \textbf{dô} sprach:\\ 
25 & "Nû volge mîner lêre:\\ 
 & verholne von mir kêre,\\ 
 & unze sich \textbf{erhebe hôch} der tac,\\ 
 & \textbf{daz} daz volc ze hove wesen mac,\\ 
 & rîter, sarjande,\\ 
30 & \textbf{diu grôze} mahinande.\\ 
\end{tabular}
\scriptsize
\line(1,0){75} \newline
D \newline
\line(1,0){75} \newline
\textbf{1} \textit{Initiale} D  \textbf{25} \textit{Majuskel} D  \newline
\line(1,0){75} \newline
\textbf{1} Daz] ÷az D \textbf{4} Melianz von Lîz D \textbf{5} Parbigœl] Parbigoͤl D \textbf{6} Plimizœl] Plimizoͤl D \textbf{15} Parzival] Parcival D \textbf{16} Plimizœl] Plimizoͤl D \textbf{19} Jescvte vnd Êckvba D \newline
\end{minipage}
\hspace{0.5cm}
\begin{minipage}[t]{0.5\linewidth}
\small
\begin{center}*m
\end{center}
\begin{tabular}{rl}
 & \textbf{sît} Gawan von Artuse reit,\\ 
 & \textbf{sô} het sorge und l\textit{ei}t\\ 
 & mit \textbf{kraft} ûf mich geleit ir \textbf{prîs}.\\ 
 & mir saget Melianz von Liz,\\ 
5 & er sæhe in sît zuo \textit{B}arbigol.\\ 
 & owê", sprach si, "Plimizol,\\ 
 & daz dich mîn ouge ie gesach!\\ 
 & waz \textbf{leides mir bî dir} \textbf{geschach}!\\ 
 & C\textit{u}nn\textit{e}ware de Lalant\\ 
10 & wart mir nimer \textbf{sît} bekant,\\ 
 & mîn \textbf{süeziu, werdiu} gespil.\\ 
 & \textbf{der} tavelrunder wart d\textit{â} vil\\ 
 & mit rede ir reht gebrochen.\\ 
 & vünftehalp jâr und sehs wochen\\ 
15 & ist, daz der werde Parcifal\\ 
 & von dem Plimizol nâch dem Grâl\\ 
 & reit. dô kêrte ouch Gawan\\ 
 & gegen Ascalun, der werde man.\\ 
 & Jeschute und Ecuba\\ 
20 & schieden sich \textbf{ouch} von mir dâ.\\ 
 & grôz jâmer nâch der werden diet\\ 
 & mich \textbf{sît} von stæten vröuden schiet."\\ 
 & diu künigîn t\textit{r}ûre\textit{n}s vil verjach.\\ 
 & hin zem knappen si \textbf{dô} sprach:\\ 
25 & "nû volge mîner lêre:\\ 
 & verholn von mir kêre,\\ 
 & unz sich \textbf{hôch erhebe} der tac,\\ 
 & \textbf{daz} daz volc zuo hove wesen mac,\\ 
 & ritter \textbf{und} sarjant,\\ 
30 & \textbf{die grôzen} mahinant.\\ 
\end{tabular}
\scriptsize
\line(1,0){75} \newline
m n o Fr69 \newline
\line(1,0){75} \newline
\newline
\line(1,0){75} \newline
\textbf{1} Artuse] artúse o \textbf{2} leit] liet m \textbf{3} kraft] [r*]: craft m roch n (o) \textbf{4} Melianz] meliantz m n meleancz o  $\cdot$ Liz] lis m o lisz n \textbf{5} Barbigol] parbẏgol m parbigol n o \textbf{6} Plimizol] plúmizol n \textbf{9} Cunneware] Conware m n Cunware o \textbf{12} tavelrunder] tafelrunde n (o) ::: Fr69  $\cdot$ dâ] do m n o ::: Fr69 \textbf{13} gebrochen] were gebrochen Fr69 \textbf{14} sehs] sech Fr69 \textbf{15} Parcifal] ::: Fr69 \textbf{16} Plimizol] plúmzol n \textbf{18} Ascalun] ascalún m ascaluͯn o \textbf{19} Jeschute] Jescutte m Jescute n Jescúte o  $\cdot$ Ecuba] ocuba n ocúba o \textbf{21} grôz] Grosses n \textbf{23} trûrens] ture vs m \textbf{24} zem] zwem o \textbf{27} erhebe] hebe Fr69 \textbf{28} \textit{Vers 646.28 fehlt} o  \textbf{30} mahinant] machemant n \newline
\end{minipage}
\end{table}
\newpage
\begin{table}[ht]
\begin{minipage}[t]{0.5\linewidth}
\small
\begin{center}*G
\end{center}
\begin{tabular}{rl}
 & \textbf{\begin{large}D\end{large}az} Gawan von Artuse reit,\\ 
 & \textbf{sît} hât sorge unde leit\\ 
 & mit \textbf{kraft} ûf mich geleit ir \textbf{vlîz}.\\ 
 & mir sagete Melianz von Liz,\\ 
5 & er s\textit{æ}h\textit{e} in sît ze Barbigol.\\ 
 & owê", sprach si, "Blimzol,\\ 
 & daz dich mîn ouge ie gesach!\\ 
 & waz \textbf{mir doch leides dâ} \textbf{geschach}!\\ 
 & Kuneware de Lalant\\ 
10 & wart mir nimer \textbf{mê} bekant,\\ 
 & mîn \textbf{süez\textit{iu}, werd\textit{iu}} gespil.\\ 
 & tavelrunder wart dâ vil\\ 
 & mit rede ir reht gebrochen.\\ 
 & vünftehalp jâr unde sehs wochen\\ 
15 & ist, daz der werde Parcival\\ 
 & von dem Blimzol nâch dem Grâl\\ 
 & reit. dô kêrt ouch Gawan\\ 
 & gein Aschalun, der werde man.\\ 
 & Jeschute unde Ekuba\\ 
20 & schieden sich von mir \textbf{al} dâ.\\ 
 & grôz jâmer nâch der werden diet\\ 
 & mich von stæten vröuden schiet."\\ 
 & diu künegîn trûrens vil verjac\textit{h}.\\ 
 & hin ze dem knappen si \textbf{dô} sprach:\\ 
25 & "nû volge mîner lêre:\\ 
 & verholne von mir kêre,\\ 
 & unze sich \textbf{erhebe hôch} der tac,\\ 
 & \textbf{dâ} daz volc ze hove wesen mac,\\ 
 & rîter \textbf{unde} sarjande\\ 
30 & \textbf{unde} \textbf{diu guote} mahinande.\\ 
\end{tabular}
\scriptsize
\line(1,0){75} \newline
G I L M Z \newline
\line(1,0){75} \newline
\textbf{1} \textit{Initiale} G I L Z  \textbf{21} \textit{Initiale} I  \newline
\line(1,0){75} \newline
\textbf{1} Daz] DO L  $\cdot$ Artuse] artus I M Z Artuͯse L \textbf{2} sorge] sorgen I \textbf{3} ir] den I \textbf{4} sagete] seit I (L)  $\cdot$ Melianz] Meliantze Z  $\cdot$ Liz] lisz M \textbf{5} sæhe] sah G  $\cdot$ sît] sitzen L  $\cdot$ ze Barbigol] zebarbigol G \textbf{6} Blimzol] plimizol I L M Z \textbf{9} Kuneware] kunware I (M) Cvneware L Kvnneware Z  $\cdot$ de Lalant] der lalant M delalant Z \textbf{11} süeziu werdiu] sueze werde G \textbf{12} wart] wer I \textbf{13} rede] reden M \textbf{14} sehs] sehse G \textbf{15} Parcival] parzival G parzifal I L M parcifal Z \textbf{16} Blimzol] plimizol I L M Z \textbf{17} dô] da da M da Z  $\cdot$ kêrt] karte M \textbf{18} Aschalun] ashalun I ascalvn L (M) (Z) \textbf{19} Jeschute] Ieschvte G ieskute I Jescuͯte L Jescute M Z  $\cdot$ Ekuba] trebvca G tribuca I Ecuba L eckuba M (Z) \textbf{20} sich] \textit{om.} I  $\cdot$ al] \textit{om.} L \textbf{22} mich] Mich sit Z  $\cdot$ stæten vröuden] starcken [fro*]: frovde L \textbf{23} verjach] viriac G \textbf{24} hin] \textit{om.} I  $\cdot$ dô] da M Z \textbf{25} nû] \textit{om.} I \textbf{26} von mir verholn chere I \textbf{28} dâ] So L \textbf{29} unde] \textit{om.} I L M Z \textbf{30} diu guote] die guͤten I \newline
\end{minipage}
\hspace{0.5cm}
\begin{minipage}[t]{0.5\linewidth}
\small
\begin{center}*T
\end{center}
\begin{tabular}{rl}
 & \textbf{daz} Gawan von Artusen reit,\\ 
 & \textbf{sît} hât sorge und leit\\ 
 & mit \textbf{kraft} ûf mich geleit ir \textbf{vlîz}.\\ 
 & mir sagte Melyanz von Liz,\\ 
5 & er sæhe in sît zuo Barbigol.\\ 
 & owê", sprach si, "Plymizol,\\ 
 & daz dich mîn oug\textit{e} ie gesach!\\ 
 & waz \textbf{mir leides doch ie} \textbf{schach}!\\ 
 & Cunneware de Lalant\\ 
10 & wart mir nimmer \textbf{mêr} bekant,\\ 
 & mîn \textbf{werdiu, süeziu} gespil.\\ 
 & tavelrunder wart d\textit{â} vil\\ 
 & mit rede ir reht gebrochen.\\ 
 & vünftehalp jâr und sehs wochen\\ 
15 & ist, daz der werde Parcifal\\ 
 & von dem Plymizol nâch dem Grâl\\ 
 & reit. dô kêrt ouch Gawan\\ 
 & gên Ascalun, der werde man.\\ 
 & Jeschute und Eckuba\\ 
20 & \textit{s}chieden sich von mir \textbf{al}dâ.\\ 
 & grôz jâmer nâch der werden diet\\ 
 & mic\textit{h} \textbf{sît} von stæten vreuden schiet."\\ 
 & diu künigîn trûrens vil verjach.\\ 
 & hin zuom knaben si \textbf{doch} sprach:\\ 
25 & "nû volge mîner lêre:\\ 
 & verholne von mir kêre,\\ 
 & unz sich \textbf{erhebe hôch} der tac,\\ 
 & \textbf{dô} daz volc zuo hove wesen mac,\\ 
 & ritter \textbf{und} sarjante\\ 
30 & \textbf{und} \textbf{diu grôze} mahinante.\\ 
\end{tabular}
\scriptsize
\line(1,0){75} \newline
Q R W V \newline
\line(1,0){75} \newline
\textbf{23} \textit{Initiale} W  \newline
\line(1,0){75} \newline
\textbf{1} daz] Do W  $\cdot$ Gawan] gawin R  $\cdot$ Artusen] Artuse R (W) (V) \textbf{2} sît] Sy R  $\cdot$ und] vnd auch W \textbf{3} vlîz] pris R \textbf{4} Melyanz] melianze Q Meliancz R melianz W V  $\cdot$ Liz] liesz Q leis W lis V \textbf{6} sprach] iach W  $\cdot$ Plymizol] plimizol Q R V \textbf{7} ouge] augen Q \textbf{8} leides doch ie] leides da R doch laides do W do leides [*]: bi dir V  $\cdot$ schach] geschach R W [*]: geschach V \textbf{9} Cunneware] Conware Q Cuͦnware R Kunnewar W  $\cdot$ de] von R \textbf{10} mêr] [*]: sît V \textbf{11} werdiu süeziu] werde suͯsze R \textbf{12} tavelrunder] [*auelrvnder]: der tauelrvnder V  $\cdot$ wart] wazd R  $\cdot$ dâ] do Q R W V \textbf{13} gebrochen] gesprochen R \textbf{15} Parcifal] partzifal Q W parczifal R parzifal V \textbf{16} Plymizol] plimizol Q R V \textbf{17} kêrt] kerte R W V  $\cdot$ Gawan] gewan R \textbf{18} Ascalun] aschalún Q ascelun R astalun W aschalun V \textbf{19} Jeschute] Jescute Q R Iestute W  $\cdot$ und] de W  $\cdot$ Eckuba] ekuba Q (R) (V) \textbf{20} schieden] Sichiden Q  $\cdot$ aldâ] dan alda R [*da]: alda V \textbf{22} mich] Micht Q  $\cdot$ stæten] \textit{om.} R \textbf{24} doch] do R V \textit{om.} W \textbf{27} erhebe] erhabe R erhebet W  $\cdot$ hôch] \textit{om.} V \textbf{28} dô] Daz V  $\cdot$ hove] hoffen R \textbf{29} und] \textit{om.} R W V \textbf{30} mahinante] machmande W \newline
\end{minipage}
\end{table}
\end{document}
