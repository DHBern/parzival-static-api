\documentclass[8pt,a4paper,notitlepage]{article}
\usepackage{fullpage}
\usepackage{ulem}
\usepackage{xltxtra}
\usepackage{datetime}
\renewcommand{\dateseparator}{.}
\dmyyyydate
\usepackage{fancyhdr}
\usepackage{ifthen}
\pagestyle{fancy}
\fancyhf{}
\renewcommand{\headrulewidth}{0pt}
\fancyfoot[L]{\ifthenelse{\value{page}=1}{\today, \currenttime{} Uhr}{}}
\begin{document}
\begin{table}[ht]
\begin{minipage}[t]{0.5\linewidth}
\small
\begin{center}*D
\end{center}
\begin{tabular}{rl}
\textbf{92} & si habent ir schildes breite\\ 
 & nâch jâmers geleite\\ 
 & \textbf{ze}r erden gekêret.\\ 
 & grôz trûren si daz lêret.\\ 
5 & \textbf{alsus} tuont si r\textit{î}terschaft.\\ 
 & \textbf{si} sint \textbf{verladen} mit jâmers kraft,\\ 
 & sît Galoes, mîner muomen sun,\\ 
 & nâch minnen \textbf{dienest niht solde} tuon."\\ 
 & \begin{large}D\end{large}ô er \textbf{vernam} des bruoder tôt,\\ 
10 & daz was sîn ander herzenôt.\\ 
 & mit jâmer sprach \textit{er} disiu wort:\\ 
 & "wie hât \textbf{nû} mînes ankers ort\\ 
 & in riwe ergriffen \textbf{landes} habe!"\\ 
 & \textbf{der} wâpen tet er sich \textbf{doch} abe.\\ 
15 & \textbf{sîn} riwe im hertes kumbers jach.\\ 
 & der \textbf{helt mit wâren} triwen sprach:\\ 
 & "von Anschouwe Galoes,\\ 
 & vürbaz darf niemen vrâgen des,\\ 
 & ez enwart nie manlîcher zuht\\ 
20 & geborn. der \textbf{wâren} \textbf{milte} vruht\\ 
 & ûz dîme herzen blüete.\\ 
 & nû erbarmet mich dîn güete."\\ 
 & Er sprach ze Kaylete:\\ 
 & "wie gehabt sich \textbf{Schoette},\\ 
25 & mîn muoter vreuden arme?"\\ 
 & "sô daz \textbf{ez} got erbarme!\\ 
 & dô ir erstarp Gandin\\ 
 & unt Galoes, der bruoder dîn,\\ 
 & unt \textbf{dô} si \textbf{dîn} bî \textbf{ir} niht sach,\\ 
30 & der tôt \textbf{ouch ir daz} herze brach."\\ 
\end{tabular}
\scriptsize
\line(1,0){75} \newline
D \newline
\line(1,0){75} \newline
\textbf{9} \textit{Initiale} D  \textbf{23} \textit{Majuskel} D  \newline
\line(1,0){75} \newline
\textbf{5} rîterschaft] reterschaft D \textbf{11} er] \textit{om.} D \textbf{17} Anschouwe] Anscoͮwe D \textbf{23} Kaylete] kaylette D \textbf{24} Schoette] Scoette D \newline
\end{minipage}
\hspace{0.5cm}
\begin{minipage}[t]{0.5\linewidth}
\small
\begin{center}*m
\end{center}
\begin{tabular}{rl}
 & si habent ir schiltes breite\\ 
 & nâch jâmers geleite\\ 
 & \textbf{ze}r erden gekêret.\\ 
 & grôz trûren si daz lêret.\\ 
5 & \textbf{alsus} tuont si ritterschaft.\\ 
 & \textbf{si} sint \textbf{beladen} mit jâmers kraft,\\ 
 & sît Gal\textit{oe}s, mîne\textit{r} muomen sun,\\ 
 & nâch \dag mînem\dag  \textbf{dienst niht solte} tuon."\\ 
 & dô er \textbf{vernam} des bruoder tôt,\\ 
10 & \begin{large}D\end{large}az was sîn ander herzenôt.\\ 
 & mit jâmer sprach er disiu wort:\\ 
 & "wie hât \textbf{nû} mînes ankers ort\\ 
 & in riuwe ergriffen \textbf{landes} habe!"\\ 
 & \textbf{der} wâpen tet er sich \textbf{dô} abe.\\ 
15 & \textbf{sîn} riuwe ime hertes kumbers jach.\\ 
 & der \textbf{helt mit wâren} triuwen sprach:\\ 
 & "von Anschouwe Galoes,\\ 
 & vürbaz darf niemen vrâgen des,\\ 
 & ez enwart nie manlîcher zuht\\ 
20 & geborn. der \textbf{minne} vruht\\ 
 & ûz dînem herzen blüete.\\ 
 & nû erbarmet mich dîn güete."\\ 
 & er sprach zuo Kailete:\\ 
 & "wie gehabet sich \textbf{Schoiete},\\ 
25 & mîn muoter vröuden arme?"\\ 
 & "sô daz \textbf{ez} got erbarme!\\ 
 & dô ir erstarp Gandin\\ 
 & und Galoes, der bruoder dîn,\\ 
 & und \textbf{dô} si \textbf{den} bî \textbf{dir} niht sach,\\ 
30 & der t\textit{ô}t \textbf{ouch ir daz} herze brach."\\ 
\end{tabular}
\scriptsize
\line(1,0){75} \newline
m n o \newline
\line(1,0){75} \newline
\textbf{10} \textit{Initiale} m  \newline
\line(1,0){75} \newline
\textbf{3} Der zedel erdeden gekeren o \textbf{4} trûren] trúten o \textbf{7} Galoes] galeus m goloes o  $\cdot$ mîner] minne m \textbf{9} er] \textit{om.} n \textbf{10} sîn] sin sin n \textbf{13} landes] lande n [lande]: landes o \textbf{17} Anschouwe] aschowe o \textbf{19} enwart] wart n o \textbf{20} minne] woren mẏnne n woren mynnen o \textbf{21} dînem] deme o \textbf{23} Kailete] kailette m kailet n o \textbf{24} Schoiete] schoẏet n schaiete o \textbf{28} Galoes] galaes o \textbf{29} sach] gesach n \textbf{30} tôt] tet m \newline
\end{minipage}
\end{table}
\newpage
\begin{table}[ht]
\begin{minipage}[t]{0.5\linewidth}
\small
\begin{center}*G
\end{center}
\begin{tabular}{rl}
 & si hânt ir schiltes breite\\ 
 & nâch jâmers geleite\\ 
 & \textbf{ze}r erden gekêret.\\ 
 & grôz \textit{trûr}e\textit{n} si daz lêret.\\ 
5 & \textbf{alsô} tuont si rîterschaft\\ 
 & \textbf{unt} sint \textbf{verladen} mit jâmers kraft,\\ 
 & sît Galoes, mîner muomen sun,\\ 
 & nâch minnen \textbf{dienst niht sol} tuon."\\ 
 & dô er \textbf{gevriesch} des bruoder tôt,\\ 
10 & daz was sîn ander herzenôt.\\ 
 & mit jâmer sprach er disiu wort:\\ 
 & "wie hât \textbf{sus} mînes ankers ort\\ 
 & in riwe ergriffen \textbf{landes} habe!"\\ 
 & \textbf{der} wâpen teter sich \textbf{dô} abe.\\ 
15 & \textbf{\begin{large}G\end{large}rôz} riwe im hertes kumbers jach.\\ 
 & der \textbf{hêrre ûz grôzen} triwen sprach:\\ 
 & "von Antschouwe Galoes,\\ 
 & vürbaz darf niemen vrâgen des,\\ 
 & ez enwart nie manlîcher zuht\\ 
20 & geboren. der \textbf{rehten} \textbf{milte} vruht\\ 
 & ûz dînem herzen blüete.\\ 
 & nû erbarmet mich dîn güete."\\ 
 & er sprach ze Kailet:\\ 
 & "wie gehabet sich \textbf{Tuschet},\\ 
25 & mîn muoter vröuden arme?"\\ 
 & "sô daz \textbf{ez} got erbarme!\\ 
 & dô ir erstarp Gandin\\ 
 & unde Galoes, der bruoder dîn,\\ 
 & unde si \textbf{dîn} bî \textbf{ir} niht sach,\\ 
30 & der tôt \textbf{ir ouch ir} herze brach."\\ 
\end{tabular}
\scriptsize
\line(1,0){75} \newline
G I O L M Q R Z Fr21 \newline
\line(1,0){75} \newline
\textbf{1} \textit{Initiale} O  \textbf{9} \textit{Initiale} I L R Z Fr21  \textbf{15} \textit{Initiale} G   $\cdot$ \textit{Capitulumzeichen} L  \newline
\line(1,0){75} \newline
\textbf{1} si] ÷i O  $\cdot$ breite] brete R \textbf{2} jâmers] iamer Q \textbf{3} zer] Zu Q  $\cdot$ erden] erde O L Q R Fr21 [erde*]: erde Z  $\cdot$ gekêret] gecherte O koret R \textbf{4} trûren] iamer G truten M trúwe R  $\cdot$ daz] \textit{om.} I  $\cdot$ lêret] lerte O lerent Q \textbf{5} alsô] Asvs O Alsus L M Q R Z (Fr21)  $\cdot$ tuont] tut Q \textbf{6} unt] Sie Z  $\cdot$ verladen] úberladen R \textbf{7} \textit{nach 92.7:} Das er nicht lenger solt vertún Q   $\cdot$ sît] Mit Q  $\cdot$ Galoes] Gaoles L \textbf{8} \textit{nach 92.8:} Das was auch ir aller clage Q   $\cdot$ nach minne dienst (Nach mynne L Nach minnen Z Nach minnen \sout{reht} Fr21 ) niht solte (sol dinst L dinst sol Fr21 ) tuͤn I (O) (L) (M) (R) (Z) (Fr21) · Nach minne dinst seiner tage Q \textbf{9} dô] Da Z ÷O Fr21  $\cdot$ er gevriesch] vornam M er clagt R grefriesch Fr21  $\cdot$ des] sins R \textbf{10} was sîn] die Z  $\cdot$ herzenôt] not I [not]: hertzen not Z \textbf{11} er] \textit{om.} I \textbf{12} sus] vnsz Q nu Z  $\cdot$ ort] not M \textbf{13} riwe] trewe Q  $\cdot$ landes] lendes I \textbf{14} teter] der e tet er Q  $\cdot$ dô] da M doch Z \textbf{15} Grôz] Sein Q (R)  $\cdot$ riwe] trew Q  $\cdot$ im hertes] im grozes O (L) Fr21 syme herczin M im Z im grossers Q  $\cdot$ kumbers] kvmber Fr21 \textbf{16} hêrre] helt O L M Q (R) Z Fr21  $\cdot$ grôzen] waren O (L) (M) (Q) (R) [ian]: waren  Z  $\cdot$ triwen] riwen O \textbf{17} Antschouwe] anschoͮwe G antschoͮe I anschawe O Anschowe L (Q) (R) anscowe M antschowe Z :::schoͮwe Fr21  $\cdot$ Galoes] der Galoes O Gaoles L \textbf{18} darf] endarf I durff M tar R \textbf{19} enwart] wart Q  $\cdot$ nie] \textit{om.} Z  $\cdot$ manlîcher] mynneclicher L \textbf{20} der] \textit{om.} I de O  $\cdot$ rehten milte] rehter milte I warn milte O (L) (Q) (R) (Z) (Fr21) milde warin M \textbf{21} dînem] einem O sinem Z  $\cdot$ herzen] hertze Q \textbf{22} erbarmet] erbarmde O  $\cdot$ mich dîn] dich min Z mich dev Fr21 \textbf{23} \textit{Versfolge 92.24-23} O   $\cdot$ Kailet] Gahilete I kaylet O L Q R (Fr21) Gailet Z \textbf{24} gehabet] gehap L  $\cdot$ Tuschet] ieskutte I tschvet O Joet L tscowet M ahoet Q Schoet R thschvet Z Tsvet Fr21 \textbf{26} sô] \textit{om.} I L \textbf{27} dô] daz I Da M Z  $\cdot$ ir] \textit{om.} M R Fr21  $\cdot$ Gandin] candin I gaudin Q \textbf{28} Galoes] Gaoles L Goloes R  $\cdot$ dîn] min I [min din]: din R \textbf{29} si dîn] sit ich I do si dich O (Q) (R) Fr21 do sie din L da sie dyn M (Z)  $\cdot$ bî ir niht] nit by ir R  $\cdot$ sach] enshach I \textbf{30} ir ouch] avch O (L) (M) (Q) (R) (Z) (Fr21)  $\cdot$ ir herze] herze I (M) \newline
\end{minipage}
\hspace{0.5cm}
\begin{minipage}[t]{0.5\linewidth}
\small
\begin{center}*T (U)
\end{center}
\begin{tabular}{rl}
 & si hânt ir schiltes breite\\ 
 & nâch jâmers geleite\\ 
 & \textbf{gein} der erden gekêret.\\ 
 & grôz trûren si daz \textit{lêret}.\\ 
5 & \textbf{alsus} tuont si ritterschaft\\ 
 & \textbf{und} sint \textbf{beladen} mit jâmers kraft,\\ 
 & sît Galoes, mîner muomen suon,\\ 
 & nâch minnen \textbf{niht dienst \textit{solte}} tuon."\\ 
 & dô er \textbf{ervriesch} des bruoder tôt,\\ 
10 & daz was sîn ander herzenôt.\\ 
 & mit jâmere sprach er disiu wort:\\ 
 & "wie hât \textbf{sus} mînes ankers ort\\ 
 & in riuwe ergriffen \textbf{leides} habe!"\\ 
 & \textbf{diu} wâpen tet er sich \textbf{dô} abe.\\ 
15 & \textbf{grôz} riuwe im herte\textit{s} kumbers jach.\\ 
 & der \textbf{helt ûz wâren} triuwen sprach:\\ 
 & "von Anschouwe Galoes,\\ 
 & vürbaz darf nieman vrâgen des,\\ 
 & ez enwart nie menlîcher zuht\\ 
20 & geborn. der \textbf{wâren} \textbf{minnen} vruht\\ 
 & ûz dîme herzen blüete.\\ 
 & nû erbarmet mich dîn güete."\\ 
 & er sprach zuo Kaylete:\\ 
 & "wie gehabt sich \textbf{Tscheuwete},\\ 
25 & mîn muoter vreuden arme?"\\ 
 & "sô daz \textbf{sich} got erbarme!\\ 
 & dô \textit{i}r erstarp Gandin\\ 
 & und Galoes, der bruoder dîn,\\ 
 & und \textbf{dô} si \textbf{den} bî \textbf{ir} niht sach,\\ 
30 & der tôt \textbf{ir ouch daz} herze brach."\\ 
\end{tabular}
\scriptsize
\line(1,0){75} \newline
U V W T \newline
\line(1,0){75} \newline
\textbf{9} \textit{Initiale} V W T  \textbf{23} \textit{Majuskel} T  \textbf{30} \textit{Majuskel} T  \newline
\line(1,0){75} \newline
\textbf{1} si hânt] [*]: An V  $\cdot$ schiltes] scilte T  $\cdot$ breite] gebraite W \textbf{3} gein der erden] Gegen der erde W zerde T \textbf{4} trûren] truwe V (T)  $\cdot$ si] si doch T  $\cdot$ lêret] \textit{om.} U lerte T \textbf{6} beladen] úberladen W verladen T  $\cdot$ jâmers] leides T \textbf{7} Galoes] Gales U \textbf{8} Nach meinem dienst nit sol thun W · nach dienste niht solte minne tvͦn T  $\cdot$ minnen] minne V  $\cdot$ solte] \textit{om.} U \textbf{9} ervriesch] erfuͦr V W vriesc T \textbf{10} herzenôt] hertzen not W \textbf{13} riuwe] ruwen V (W)  $\cdot$ leides] in leides V landes W T \textbf{14} diu] der V (W) T  $\cdot$ sich dô] doch sich V \textbf{15} im hertes kumbers] im hertez kuͦmers U sin hertze kumbers V in seinem hertzen kummer W im kvmbers T \textbf{16} ûz] mit T \textbf{17} Anschouwe] Anscheuwe U Anschowe V antschowe W  $\cdot$ Galoes] Gales U \textbf{19} enwart nie] ward mit W \textbf{20} minnen] [mil*]: milte V minne W T \textbf{23} zuo] sa ze T  $\cdot$ Kaylete] kalete U kaẏlete V gaylete W kaylet T \textbf{24} Tscheuwete] Tschewete U Schewete V deschawete W Tscêuwet T \textbf{25} vreuden] an froͤden V \textbf{26} sich] [*]: es V \textbf{27} ir erstarp] er erstarp U erstarp V ir starb W (T)  $\cdot$ Gandin] Gaudin U (W) \textbf{28} Galoes] Gales U  $\cdot$ dîn] [*]: din V min W T \textbf{29} \textit{Versfolge 92.30-29} U   $\cdot$ den bî ir niht] [*]: dich niht bi ir V den nit bei ir W (T)  $\cdot$ sach] vand W \textbf{30} Von dem ir freúde gar verschwand W  $\cdot$ daz] ir T \newline
\end{minipage}
\end{table}
\end{document}
