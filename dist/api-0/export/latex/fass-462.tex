\documentclass[8pt,a4paper,notitlepage]{article}
\usepackage{fullpage}
\usepackage{ulem}
\usepackage{xltxtra}
\usepackage{datetime}
\renewcommand{\dateseparator}{.}
\dmyyyydate
\usepackage{fancyhdr}
\usepackage{ifthen}
\pagestyle{fancy}
\fancyhf{}
\renewcommand{\headrulewidth}{0pt}
\fancyfoot[L]{\ifthenelse{\value{page}=1}{\today, \currenttime{} Uhr}{}}
\begin{document}
\begin{table}[ht]
\begin{minipage}[t]{0.5\linewidth}
\small
\begin{center}*D
\end{center}
\begin{tabular}{rl}
\textbf{462} & \begin{large}G\end{large}ot \textbf{müeze} uns helfen beiden.\\ 
 & hêrre, ir sult \textbf{mich} bescheiden\\ 
 & - \textbf{ruochet} alrêrst sitzen -,\\ 
 & sagt mir mit kiuschen witzen,\\ 
5 & wie der zorn sich \textbf{an gevienc},\\ 
 & dâ von got iwern haz enpfienc.\\ 
 & durch iwer zuht gedult\\ 
 & vernemt von mir sîn unschult,\\ 
 & ê daz ir mir von im iht klagt.\\ 
10 & sîn helfe ist immer unverzagt.\\ 
 & doch ich ein leie wære,\\ 
 & der wâren buoche mære\\ 
 & künd ich lesen und schrîben,\\ 
 & wie der mensch sol belîben\\ 
15 & mit dienste gein des helfe grôz,\\ 
 & den der stæten helfe nie verdrôz\\ 
 & vür der \textbf{sêle} senken.\\ 
 & \textbf{sît getriwe} âne \textbf{allez} wenken,\\ 
 & sît got selbe ein triwe ist.\\ 
20 & dem \textbf{was} unmære \textbf{ie} valscher list.\\ 
 & wir suln in des geniezen lân:\\ 
 & er hât vil durch uns getân,\\ 
 & sît sîn edel hôher art\\ 
 & durch uns ze menschen bilde wart.\\ 
25 & got heizet unt ist \textbf{ein} wârheit.\\ 
 & dem was ie valschiu vuore leit.\\ 
 & daz sult ir gar bedenken.\\ 
 & er\textbf{n} kan an niemen wenken.\\ 
 & nû lêret iwer gedanke,\\ 
30 & hüetet iuch gein im an wanke.\\ 
\end{tabular}
\scriptsize
\line(1,0){75} \newline
D \newline
\line(1,0){75} \newline
\textbf{1} \textit{Initiale} D  \newline
\line(1,0){75} \newline
\newline
\end{minipage}
\hspace{0.5cm}
\begin{minipage}[t]{0.5\linewidth}
\small
\begin{center}*m
\end{center}
\begin{tabular}{rl}
 & got \textbf{muoz} un\textit{s} helfen beiden.\\ 
 & hêrre, ir sullet \textbf{mich} bescheiden\\ 
 & - \textbf{und} \textbf{ruocht} allerêrst sitzen -,\\ 
 & saget mir mit kiuschen witzen,\\ 
5 & wie der zorn sich \textbf{an gevienc},\\ 
 & dâ von got iuwern haz enpfienc.\\ 
 & durch iuwer zuht gedult\\ 
 & vernemet von mir sîn unschult,\\ 
 & ê daz ir mir von im ih\textit{t} klaget.\\ 
10 & sîn helf ist iemer unverzag\textit{e}t.\\ 
 & doch ich ein \dag lœsære\dag ,\\ 
 & der wâren buoch mære\\ 
 & künde ich lesen und schrîben,\\ 
 & wie der mensch sol blîben\\ 
15 & mit dienst gegen des he\textit{l}fe grôz,\\ 
 & den der stæte\textit{n} helfe nie verdrôz\\ 
 & vür der \textbf{sêl\textit{e}} \textit{s}enken.\\ 
 & \textbf{sî\textit{t} getriuwe} âne \textbf{allez} wenken,\\ 
 & sît got selbe ein triuwe ist.\\ 
20 & dem \textbf{ist} unmær \textbf{ein} valscher list.\\ 
 & wir sullen in des geniezen lân:\\ 
 & er het vil durch uns getân,\\ 
 & sît sîn edel hôher art\\ 
 & durch uns ze menschen b\textit{il}d\textit{e} wart.\\ 
25 & got heizet und ist \textbf{diu} wârheit.\\ 
 & dem was ie valschiu vuore leit.\\ 
 & daz solt ir ga\textit{r} bedenken.\\ 
 & er kan an niem\textit{an} wenken.\\ 
 & nû lêret iuwer gedanke,\\ 
30 & hüetet iuch gegen im an wanke.\\ 
\end{tabular}
\scriptsize
\line(1,0){75} \newline
m n o \newline
\line(1,0){75} \newline
\newline
\line(1,0){75} \newline
\textbf{1} uns] vnd m \textbf{5} an gevienc] anefing o \textbf{6} got] gotte n  $\cdot$ iuwern] vwer o \textbf{7} iuwer] uch o \textbf{8} vernemet] Vernemen n  $\cdot$ sîn] myn o \textbf{9} iht] ich m o \textbf{10} unverzaget] vnuerzagent m \textbf{13} künde] Kunne n \textbf{15} helfe] heff m \textbf{16} stæten] stette m (n) o \textbf{17} sêle senken] sele sele sencken m \textbf{18} sît] Sin m o \textbf{19} ein] >eyn< o \textbf{21} sullen] sol o \textbf{23} edel hôher] edele hohe n \textbf{24} uns] zuͦ o  $\cdot$ bilde] bliden m \textbf{25} \textit{Die Verse 462.25-464.30 fehlen} o  \textbf{27} gar] garb m \textbf{28} nieman] nyem m \textbf{30} hüetet] Huͯtten m (n) \newline
\end{minipage}
\end{table}
\newpage
\begin{table}[ht]
\begin{minipage}[t]{0.5\linewidth}
\small
\begin{center}*G
\end{center}
\begin{tabular}{rl}
 & \begin{large}G\end{large}ot \textbf{müeze} uns helfen beiden.\\ 
 & hêrre, ir sult \textbf{mich} bescheiden\\ 
 & - \textbf{ruochet} alrêrste sitzen -,\\ 
 & saget mir mit kiuschen witzen,\\ 
5 & wie der zorn sich \textbf{an gevienc},\\ 
 & dâ von got iuwern haz enpfienc.\\ 
 & durch iuwer zühte gedult\\ 
 & vernemet von mir sîn unschult,\\ 
 & ê daz ir mir von im iht klaget.\\ 
10 & sîn helfe ist immer unverzaget.\\ 
 & doch ich ein leige wære,\\ 
 & der wâren buoche mære\\ 
 & künde ich lesen unde schrîben,\\ 
 & wie der \textit{mensche} sol belîben\\ 
15 & mit dienste gein des helfe grôz,\\ 
 & den der stæten helfe nie verdrôz\\ 
 & vür der \textbf{sêle} senken.\\ 
 & \textbf{sît getriu} ân \textbf{allez} we\textit{n}ken,\\ 
 & sît got \textit{selbe} ein triuwe ist.\\ 
20 & dem \textbf{was} unmære \textbf{ie} valscher list.\\ 
 & wir suln in des geniezen lân:\\ 
 & er hât vil durch uns getân,\\ 
 & sît sîn edel hôher art\\ 
 & durch uns ze menschen bilde wart.\\ 
25 & got heizet unde ist \textbf{diu} wârheit.\\ 
 & dem was ie valschiu vuore leit.\\ 
 & daz sult ir gar bedenken.\\ 
 & er\textbf{ne} kan an niemen wenken.\\ 
 & nû lêrt iuwer gedanke,\\ 
30 & hüet iuch gein im an wanke.\\ 
\end{tabular}
\scriptsize
\line(1,0){75} \newline
G I O L M Z Fr22 Fr61 \newline
\line(1,0){75} \newline
\textbf{1} \textit{Initiale} G I O L Z Fr22  \textbf{15} \textit{Initiale} I  \newline
\line(1,0){75} \newline
\textbf{1} Got] ÷ot O  $\cdot$ müeze] muͤz I (O) (L) (M) (Z) \textbf{2} mich] mir baz O mir L Fr22 \textbf{3} ruochet] Geruͤcht Fr61 \textbf{4} mit] [mir]: mit G  $\cdot$ kiuschen] chvnste Fr61 \textbf{5} der zorn sich] sich der zorn L  $\cdot$ an gevienc] anevienc I \textbf{6} dâ von got iuwern] Daz got von uͯch L Daz got îvwirn Fr22  $\cdot$ haz] [zorn]: haz I zorn O  $\cdot$ enpfienc] gevienc M \textbf{8} sîn] uwir M \textbf{9} mir] \textit{om.} M  $\cdot$ klaget] sagt Z \textbf{10} immer] myner M \textbf{11} leige] lerær Fr61 \textbf{12} buoche] buchere M \textbf{13} künde] Konde M \textbf{14} mensche] \textit{om.} G \textbf{15} des] der L (Fr61)  $\cdot$ grôz] grvͦz O \textbf{16} der stæten] stater L (Fr61) \textbf{17} der] die L \textbf{18} allez] \textit{om.} I O L M  $\cdot$ wenken] wechen G \textbf{19} selbe] \textit{om.} G selbern M selber Fr61 \textbf{20} valscher] [falc]: falscher G \textbf{21} \textit{Die Verse 462.21-26 fehlen} L  \textbf{23} edel] edler Fr61  $\cdot$ hôher] hohe Z \textbf{24} menschen] menchen Z \textbf{26} dem] im I \textbf{27} gar] paz Fr61 \textbf{28} erne] er I (O)  $\cdot$ an] \textit{om.} L \textbf{29} lêrt] cheret I \textbf{30} hüet iuch gein] Hvͦten gein O (L) sein germe Fr61 \newline
\end{minipage}
\hspace{0.5cm}
\begin{minipage}[t]{0.5\linewidth}
\small
\begin{center}*T
\end{center}
\begin{tabular}{rl}
 & got \textbf{müeze} uns helfen beiden.\\ 
 & hêrre, ir sult \textbf{mir} bescheiden\\ 
 & - \textbf{geruochet} alrêrst sitzen -,\\ 
 & saget mir mit kiuschen witzen,\\ 
5 & wie der zorn sich \textbf{angehüebe} \textbf{ie},\\ 
 & dâ von got iuwern haz enpfie.\\ 
 & durch iuwer zühte gedult\\ 
 & vernemet von mir sîn unschult,\\ 
 & ê daz ir mir von im iht klaget.\\ 
10 & sîn helfe ist iemer unverzaget.\\ 
 & doch ich ein le\textit{ie} wære,\\ 
 & der wâren buoche mære\\ 
 & kündich lesen unde schrîben,\\ 
 & wie der mensche sol blîben\\ 
15 & mit dienste gegen des helf\textit{e} \textit{g}rôz,\\ 
 & den der stæten helfe nie verdrôz\\ 
 & vür der \textbf{helle} senken.\\ 
 & \textbf{sîn triuwe ist} âne wenken,\\ 
 & sît got selbe ein triuwe ist.\\ 
20 & dem \textbf{was} unmære \textbf{ie} valscher list.\\ 
 & wir suln in des geniezen lân:\\ 
 & er hât vil durch uns getân,\\ 
 & sît sîn edel hôher art\\ 
 & durch uns ze menschen bilde wart.\\ 
25 & got heizet unde ist \textbf{diu} wârheit.\\ 
 & dem was ie valsch\textit{iu} vuore leit.\\ 
 & daz sult ir gar bedenken.\\ 
 & er\textbf{n} kan an niemanne wenken.\\ 
 & nû lêret iuwer gedanke,\\ 
30 & hüet iuch gegen im an wanke.\\ 
\end{tabular}
\scriptsize
\line(1,0){75} \newline
T U V W Q R \newline
\line(1,0){75} \newline
\textbf{1} \textit{Initiale} Q  \textbf{23} \textit{Überschrift:} [*]: Hie bihtet parzefal Trefrischenten V  \newline
\line(1,0){75} \newline
\textbf{1} \textit{Die Verse 453.1-502.30 fehlen} U   $\cdot$ müeze] musz Q muͦs R \textbf{2} ir] [is]: ir R  $\cdot$ mir] mir aber W R min aber Q \textbf{3} geruochet] Rvͦchent V (W) (Q) (R) \textbf{4} mit] \textit{om.} R \textbf{5} wie sich der zorn an [geh*]: gevieng V  $\cdot$ Wie sich der zorne angefingk Q  $\cdot$ wie der zorn sich ane vieng W R \textbf{8} sîn] [*]: Sine V \textbf{11} ich] wie ich Q  $\cdot$ leie] lerer T [le*e]: lege V ler W [leibe]: leie Q \textbf{12} der] Mir Q  $\cdot$ buoche] buchen R \textbf{13} kündich] Jch konde Q \textbf{15} des] der W  $\cdot$ helfe grôz] helfe ist grôz T \textbf{16} den] Dem Q  $\cdot$ der stæten] der stete W stet Q der strit R \textbf{17} der] die R  $\cdot$ helle] sele Q \textbf{18} Seit (Mit Q ) getreúw ane wencken W (Q) (R) \textbf{19} ein] die R \textbf{20} unmære ie] ye vnmerre R \textbf{23} hôher] hohe V W \textbf{24} menschen] mensche R \textbf{25} ist] \textit{om.} W \textbf{26} valschiu vuore] valsce vuͤre T falsche R \textbf{28} ern] Er V W R \textbf{29} lêret] lerent V W Q R  $\cdot$ iuwer] úwern R  $\cdot$ gedanke] gedancken Q \textbf{30} iuch] îv T \newline
\end{minipage}
\end{table}
\end{document}
