\documentclass[8pt,a4paper,notitlepage]{article}
\usepackage{fullpage}
\usepackage{ulem}
\usepackage{xltxtra}
\usepackage{datetime}
\renewcommand{\dateseparator}{.}
\dmyyyydate
\usepackage{fancyhdr}
\usepackage{ifthen}
\pagestyle{fancy}
\fancyhf{}
\renewcommand{\headrulewidth}{0pt}
\fancyfoot[L]{\ifthenelse{\value{page}=1}{\today, \currenttime{} Uhr}{}}
\begin{document}
\begin{table}[ht]
\begin{minipage}[t]{0.5\linewidth}
\small
\begin{center}*D
\end{center}
\begin{tabular}{rl}
\textbf{192} & \textbf{Daz} kom, als ich iu sagen wil.\\ 
 & ez brach niht wîplîchiu zil.\\ 
 & mit stæte kiusche truoc diu magt,\\ 
 & von der ein teil \textbf{hie} wirt gesagt.\\ 
5 & \textbf{die} twanc urliuges nôt\\ 
 & unt lieber helfære tôt\\ 
 & ir herze an sölhez krachen,\\ 
 & daz ir ougen muosen wachen.\\ 
 & \textbf{\begin{large}D\end{large}ô} gienc diu küneginne\\ 
10 & niht nâch sölher minne,\\ 
 & diu \textbf{sölhen} namen reizet,\\ 
 & \textbf{der} meide wîp heizet.\\ 
 & si suoch\textit{t}e helfe und \textbf{vriwents} rât.\\ 
 & an ir was werlîchiu wât,\\ 
15 & ein hemde wîz sîdîn.\\ 
 & waz m\textit{ö}hte kampflîcher sîn\\ 
 & dan gein dem man sus komende ein wîp?\\ 
 & \textbf{ouch} swanc diu \textbf{vrouwe} umb ir lîp\\ 
 & von samîte einen mantel lanc.\\ 
20 & \textbf{si} gie, als si der kumber twanc.\\ 
 & Juncvrouwen, kamerære,\\ 
 & swaz der \textbf{dâ} bî ir wære,\\ 
 & die lie si slâfen überal.\\ 
 & dô sleich si \textbf{lîse} ân allen schal\\ 
25 & in eine kemenâten.\\ 
 & \textbf{daz} schuofen, die ez \textbf{dâ} tâten,\\ 
 & daz Parzival al eine lac.\\ 
 & \textbf{von} kerzen lieht \textbf{sô} der tac\\ 
 & was \textbf{vor} sîner slâfstat.\\ 
30 & gein sînem bette gie ir pfat.\\ 
\end{tabular}
\scriptsize
\line(1,0){75} \newline
D \newline
\line(1,0){75} \newline
\textbf{1} \textit{Majuskel} D  \textbf{9} \textit{Initiale} D  \textbf{21} \textit{Majuskel} D  \newline
\line(1,0){75} \newline
\textbf{13} suochte] svͦche D \textbf{16} möhte] mohte D \newline
\end{minipage}
\hspace{0.5cm}
\begin{minipage}[t]{0.5\linewidth}
\small
\begin{center}*m
\end{center}
\begin{tabular}{rl}
 & \textbf{daz} kam, als ich iu sagen wil.\\ 
 & ez brach niht wîplîchiu zil.\\ 
 & mit stæte kiusche truoc diu maget,\\ 
 & von der ein teil \textbf{hie} wirt gesaget.\\ 
5 & \textbf{der} twanc urliuges nôt\\ 
 & und lieber helfære tôt\\ 
 & ir herze an solich krachen,\\ 
 & daz ir ougen muosen wachen.\\ 
 & \textbf{ez} gienc diu küniginne\\ 
10 & niht nâc\textit{h} \textit{s}olher minne,\\ 
 & diu \textbf{solh\textit{e}n} namen reizet,\\ 
 & \textbf{der} megde wîp heizet.\\ 
 & si suochte helfe und \textbf{vriundes} rât.\\ 
 & an ir was werlîchiu wât,\\ 
15 & ein hemde wîz sîdîn.\\ 
 & waz möhte kampflîcher sîn\\ 
 & danne gegen dem man sus komende ein wîp?\\ 
 & \textbf{ouch} \textit{swanc} diu \textbf{vrouwe} umb ir lîp\\ 
 & von sam\textit{ît}e einen mantel lanc.\\ 
20 & \textbf{si} gienc, als si der kumber twanc.\\ 
 & juncvrouwen, kamerære,\\ 
 & waz der \textbf{d\textit{â}} bî ir wære,\\ 
 & die liez si slâfen überal.\\ 
 & dô sleich si \textbf{lîse} ân allen schal\\ 
25 & in eine kemenâten.\\ 
 & \textbf{daz} schuofen, die ez tâten,\\ 
 & daz Parcifal \textit{a}l eine lac.\\ 
 & \textbf{von} kerzen lieht \textbf{sô} der tac\\ 
 & was \textbf{vor} sîner slâfstat.\\ 
30 & gegen sînem bette gienc ir pfat.\\ 
\end{tabular}
\scriptsize
\line(1,0){75} \newline
m n o Fr69 \newline
\line(1,0){75} \newline
\newline
\line(1,0){75} \newline
\textbf{2} brach] braht o \textbf{5} der] Die Fr69 \textbf{6} helfære] helffe o  $\cdot$ tôt] dut o \textbf{8} muosen] muͯssen m muͯsten n o \textbf{10} nâch solher] nach siner solher m \textbf{11} diu solhen namen] Die solhan namen m Der die magd Fr69  $\cdot$ reizet] reiset o \textbf{12} megde] die megde Fr69 \textbf{13} si] Er n o  $\cdot$ helfe und vriundes] frunt vnd helffes o \textbf{16} Was mochte kompffelich sin o \textbf{17} komende] komen n  $\cdot$ ein] \textit{om.} n o \textbf{18} swanc] \textit{om.} m \textbf{19} samîte] samende m \textbf{21} kamerære] Cammere o \textbf{22} der] er o  $\cdot$ dâ] do m n o \textbf{24} dô] Vnd n o  $\cdot$ si] \textit{om.} n o  $\cdot$ schal] s schal o \textbf{25} eine] ern o \textbf{27} al eine] an leine m \textbf{28} sô] alsam n \textbf{30} ir] ein n (o) \newline
\end{minipage}
\end{table}
\newpage
\begin{table}[ht]
\begin{minipage}[t]{0.5\linewidth}
\small
\begin{center}*G
\end{center}
\begin{tabular}{rl}
 & \textbf{hie} kom, als ich iu sagen wil.\\ 
 & ez brach niht wîplîchiu zil.\\ 
 & mit stæte kiusche truoc diu maget,\\ 
 & von der ein teil \textbf{hie} wirt gesaget.\\ 
5 & \textbf{si} twanc urliuges nôt\\ 
 & unde lieber helfære tôt\\ 
 & ir herze an solhez krachen,\\ 
 & daz ir ougen muosen wachen.\\ 
 & \textbf{dô} gienc diu küniginne\\ 
10 & niht nâch solher minne,\\ 
 & diu \textbf{solhen} namen reizet,\\ 
 & \textbf{der} meide wîp heizet.\\ 
 & si suochte helfe unde \textbf{vriundes} rât.\\ 
 & an ir was werlîchiu wât,\\ 
15 & ein hemde wîz sîdîn.\\ 
 & waz m\textit{ö}hte kampflîcher sîn\\ 
 & danne gein dem man sus komende ein wîp?\\ 
 & \textbf{dô} swanc diu \textbf{vrouwe} umbe ir lîp\\ 
 & von samît einen mandel lanc.\\ 
20 & \textbf{si} gie, alsi der kumber dwanc.\\ 
 & juncvrouwen, kamerære,\\ 
 & swaz der \textbf{dâ} bî ir wære,\\ 
 & die lie si slâfen überal.\\ 
 & dô sleich si \textbf{eine} ân allen schal\\ 
25 & in eine kemenâten.\\ 
 & \textbf{ez} schuofen, \textit{d}iez \textbf{dâ} tâten,\\ 
 & daz Parcival \textit{alein}e lac.\\ 
 & \textbf{von} kerzen lieht \textbf{alsam} der tac\\ 
 & was \textbf{vor} sîner slâfstat.\\ 
30 & gein sînem b\textit{e}te gieng ir pfat.\\ 
\end{tabular}
\scriptsize
\line(1,0){75} \newline
G I O L M Q R Z Fr47 \newline
\line(1,0){75} \newline
\textbf{1} \textit{Initiale} I O M  \textbf{9} \textit{Initiale} L  \textbf{13} \textit{Initiale} I  \newline
\line(1,0){75} \newline
\textbf{1} hie] ÷ie O Do L M \textbf{2} ez] Herzcin M Esen Q (R)  $\cdot$ wîplîchiu] wipliche I \textbf{3} stæte] steter Q  $\cdot$ truoc] trugit M \textbf{4} ein teil hie] ein teil O Z ein teil uͯch L uch eyn teil hie M  $\cdot$ wirt] wider M wirt >hie< O \textbf{6} lieber] [liebere]: lieber Z [liebert]: lieber Fr47 \textbf{7} an] ein I L gab Fr47 \textbf{8} wachen] lachen Q \textbf{9} dô] Da M Z  $\cdot$ gienc] \textit{om.} Q \textbf{10} minne] \textit{om.} I \textbf{11} solhen] solher G diese L sulen M selbenn Q  $\cdot$ namen] iamer Z \textbf{13} suochte] svͦht O (Z)  $\cdot$ vriundes] \textit{om.} I R  $\cdot$ rât] tag Q \textbf{14} werlîchiu] [herlichev]: werlichev I [weliche]: werliche L wetliche Q  $\cdot$ wât] tat L \textbf{16} möhte] mohte G I O L (M) (Q) (Z) \textbf{17} dem] \textit{om.} O  $\cdot$ komende] chomendem I chom O komt R  $\cdot$ ein] \textit{om.} I Z \textbf{18} dô] Da Z  $\cdot$ ir] den I \textbf{19} von samît einen] einen samit I \textbf{22} swaz] Waz L (Q) (R)  $\cdot$ dâ bî ir] da I bý ir da L do bey ir Q \textbf{23} slâfen] schalaffen R \textbf{24} dô] Da O M Z  $\cdot$ sleich] gieng R  $\cdot$ eine] \textit{om.} I lise O L M (Q) (R) Z  $\cdot$ allen] \textit{om.} R \textbf{26} diez] :iez G  $\cdot$ dâ] \textit{om.} I L \textbf{27} daz] Der R  $\cdot$ Parcival] parzival G M Parzifal I (L) (R) Parcifal O (Z) partzifal Q  $\cdot$ aleine] :::e G eine I \textbf{28} \textit{Vers 192.28 fehlt} Q   $\cdot$ lieht] lýcht L (M) liech R  $\cdot$ alsam] also O L (R) \textbf{30} bete] b:te G \newline
\end{minipage}
\hspace{0.5cm}
\begin{minipage}[t]{0.5\linewidth}
\small
\begin{center}*T
\end{center}
\begin{tabular}{rl}
 & \textbf{daz} kom, als ich iu sagen wil.\\ 
 & ez \textbf{en}brach niht wîplîchiu zil.\\ 
 & mit stæte kiusche truo\textit{c} diu maget,\\ 
 & von der ein teil \textbf{iu} wir\textit{t} gesaget.\\ 
5 & \textbf{si} twanc urliuges nôt\\ 
 & unde lieber helfære tôt\\ 
 & ir herze an solhe\textit{z} krachen,\\ 
 & daz ir ougen muosen wachen.\\ 
 & \textbf{\begin{large}D\end{large}ô} gie diu küneginne\\ 
10 & niht nâch solher minne,\\ 
 & diu \textbf{disen} namen reizet,\\ 
 & \textbf{daz} maget wîp heizet.\\ 
 & si suochte helfe unde rât.\\ 
 & an ir was werlîchiu wât,\\ 
15 & ein hemede wîz sîdîn.\\ 
 & waz m\textit{ö}hte kampflîcher sîn\\ 
 & danne gegen dem man sus komende ein wîp?\\ 
 & \textbf{dô} swanc diu \textbf{maget} umbe ir lîp\\ 
 & von samîde einen mantel lanc\\ 
20 & \textbf{unde} gie, alse si der kumber twanc.\\ 
 & Juncvrouwen \textbf{unde} kamerære,\\ 
 & swaz der bî ir wære,\\ 
 & die lie si slâfen überal.\\ 
 & dô sleich si \textbf{lîse} âne allen schal\\ 
25 & in eine kemenâten.\\ 
 & \textbf{ez} schuofen, diez \textbf{dâ} tâten,\\ 
 & daz Parcifal aleine lac.\\ 
 & kerzen lieht \textbf{sô} der tac\\ 
 & was \textbf{bî} sîner slâfstat.\\ 
30 & gegen sînem bette gienc ir pfat.\\ 
\end{tabular}
\scriptsize
\line(1,0){75} \newline
T U V W \newline
\line(1,0){75} \newline
\textbf{9} \textit{Initiale} T U V W  \textbf{21} \textit{Majuskel} T  \newline
\line(1,0){75} \newline
\textbf{3} stæte] steter W  $\cdot$ truoc] trvͦt T \textbf{4} der] dir U [dem]: der V  $\cdot$ wirt] wir T \textbf{5} urliuges] orluͦgens U \textbf{7} an] gewan V  $\cdot$ solhez] solhes T \textbf{8} muosen] mvesen T (V) \textbf{11} disen] [*]: solhen V \textbf{12} daz maget] [D*]: Der megede V Der megde W \textbf{13} rât] frv́ndes rat V \textbf{16} möhte] mohte T (U) V \textbf{17} danne] Da V Wann W  $\cdot$ sus] \textit{om.} W \textbf{18} ir lîp] [ir l*]: irn lip V \textbf{22} swaz] Waz U (W)  $\cdot$ der] da U  $\cdot$ bî ir] do bei W  $\cdot$ wære] [*]: do were V \textbf{23} slâfen] slagen V \textbf{24} lîse] dan U \textbf{25} kemenâten] kemenate U \textbf{27} Parcifal] parzifal V partzifal W \textbf{28} kerzen lieht sô] Kerzen liech so U [*]: Von kerzen lieht so V Kertzen liecht sam W \textbf{29} was] Warn W  $\cdot$ bî] [*]: vor V \newline
\end{minipage}
\end{table}
\end{document}
