\documentclass[8pt,a4paper,notitlepage]{article}
\usepackage{fullpage}
\usepackage{ulem}
\usepackage{xltxtra}
\usepackage{datetime}
\renewcommand{\dateseparator}{.}
\dmyyyydate
\usepackage{fancyhdr}
\usepackage{ifthen}
\pagestyle{fancy}
\fancyhf{}
\renewcommand{\headrulewidth}{0pt}
\fancyfoot[L]{\ifthenelse{\value{page}=1}{\today, \currenttime{} Uhr}{}}
\begin{document}
\begin{table}[ht]
\begin{minipage}[t]{0.5\linewidth}
\small
\begin{center}*D
\end{center}
\begin{tabular}{rl}
\textbf{730} & \begin{large}D\end{large}en ouch ir minne lêrte pîn,\\ 
 & \textbf{den} herzogen von Gowerzin,\\ 
 & \textbf{Lischoyse} wart Cundrie gegeben.\\ 
 & âne vreude stuont sîn leben,\\ 
5 & unz er ir werden minne \textbf{enpfant}.\\ 
 & \textbf{dem Turkoten} Florant\\ 
 & \textbf{\textbf{Sangiven} Artus} ze wîbe bôt.\\ 
 & die het dâ vor der künec Lot.\\ 
 & der vürste ouch si vil gerne nam.\\ 
10 & diu gâbe minne wol gezam.\\ 
 & Artus was vrouwen milte,\\ 
 & sölher gâbe in niht bevilte.\\ 
 & \textbf{des} was mit râte vor \textbf{erdâht}.\\ 
 & \textbf{nû} disiu rede \textbf{wart} volbrâht,\\ 
15 & Dô sprach diu herzoginne,\\ 
 & daz Gawan het ir minne\\ 
 & \textbf{gedient mit prîse} hôch erkant,\\ 
 & daz er ir \textbf{lîbes} und über ir lant\\ 
 & von rehte hêrre wære.\\ 
20 & diu rede dûhte swære\\ 
 & ir soldiere, die manec sper\\ 
 & \textbf{ê} brâchen durch ir minne ger.\\ 
 & Gawan unt die gesellen \textit{s}în,\\ 
 & Arnive \textbf{und} diu \textbf{herzogîn}\\ 
25 & und \textbf{manec vrouwe} lieht gemâl\\ 
 & und \textbf{ouch} der werde Parzival,\\ 
 & \textbf{Sangive} und Cundrie\\ 
 & \textbf{nam} urloup. Itonje\\ 
 & beleip bî Artuse dâ.\\ 
30 & nû darf niemen \textbf{sprechen}, wâ\\ 
\end{tabular}
\scriptsize
\line(1,0){75} \newline
D \newline
\line(1,0){75} \newline
\textbf{1} \textit{Initiale} D  \textbf{15} \textit{Majuskel} D  \newline
\line(1,0){75} \newline
\textbf{3} Lischoyse] Liscoyse D \textbf{7} Sangiven] Sangîven D \textbf{23} sîn] min D \textbf{27} Sangive] Sangîve D  $\cdot$ Cundrie] Cvndrîe D \textbf{28} Itonje] Jtonîe D \newline
\end{minipage}
\hspace{0.5cm}
\begin{minipage}[t]{0.5\linewidth}
\small
\begin{center}*m
\end{center}
\begin{tabular}{rl}
 & den ouch ir minne lêrte pîn,\\ 
 & \textbf{dem} herzogen von Gowertzin\\ 
 & wart Condrie \textbf{dô} gegeben.\\ 
 & âne vröude stuont sîn leben,\\ 
5 & unz er ir werden minne \textbf{enpfant}.\\ 
 & \textbf{dem Turcoiten} Florant\\ 
 & \textbf{Sang\textit{iv}en Artus} zuo wîbe bôt.\\ 
 & die het d\textit{â} vor der künic Lot.\\ 
 & der vürste ouch si vil gerne nam.\\ 
10 & diu gâbe minne wol gezam.\\ 
 & \begin{large}A\end{large}rtus was vrowen milte,\\ 
 & solicher gâbe in niht bevilte.\\ 
 & \textbf{daz} was mit râte \textbf{d\textit{â}} vor \textbf{bedâht}.\\ 
 & \textbf{nû} disiu rede \textbf{wart} volbrâht,\\ 
15 & dô sprach diu herzoginn,\\ 
 & daz Gawan het ir minn\\ 
 & \textbf{gedienet mit prîse} hôch erkant,\\ 
 & daz er \textbf{über} ir \textbf{lîp} und über \textit{ir} lant\\ 
 & von rehte hêrre wære.\\ 
20 & diu rede dûhte swære\\ 
 & ir soldie\textit{r}, die manic sper\\ 
 & \textbf{ê} brâchen durch ir minne ger.\\ 
 & Gawan und die gesellen sîn,\\ 
 & A\textit{r}n\textit{iv}e \textbf{und} diu \textbf{herzogîn}\\ 
25 & und \textbf{manic vrowe} li\textit{e}ht gemâl\\ 
 & und \textbf{ouch} der werde Parcifal,\\ 
 & \textbf{Sangive} und Condrie\\ 
 & \textbf{nâmen} urloup. Itonie\\ 
 & beleip bî Artuse dâ.\\ 
30 & nû darf niemen \textbf{sprechen}, wâ\\ 
\end{tabular}
\scriptsize
\line(1,0){75} \newline
m n o Fr69 \newline
\line(1,0){75} \newline
\textbf{11} \textit{Initiale} m   $\cdot$ \textit{Capitulumzeichen} n  \newline
\line(1,0){75} \newline
\textbf{1} den] Denne n  $\cdot$ ouch] ich o \textbf{2} Gowertzin] gowortzin n gowerczuͦn o \textbf{3} Condrie] kvndrie n cundrie o \textbf{5} werden] werde n \textbf{6} Turcoiten] túrcoiten o \textbf{7} Sangiven] Sangwen m n Sanwen o  $\cdot$ Artus] artuͯs o \textbf{8} dâ] do m n o \textbf{13} dâ] do m n o \textbf{18} ir] \textit{om.} m o \textbf{19} von] Vor o \textbf{21} soldier] soldie m \textbf{24} Arnive] Arune m Arniwe n \textbf{25} lieht] liht m \textbf{27} Sangive] sangiwe n Sanwe o Sangiue Fr69  $\cdot$ Condrie] kondrie n kuͯndrie o :::drie Fr69 \textbf{28} Itonie] jtonie m o ithonẏe n \textbf{29} Artuse] ar::: Fr69  $\cdot$ dâ] do n \textbf{30} darf] bedarff n \newline
\end{minipage}
\end{table}
\newpage
\begin{table}[ht]
\begin{minipage}[t]{0.5\linewidth}
\small
\begin{center}*G
\end{center}
\begin{tabular}{rl}
 & \begin{large}D\end{large}en ouch ir minne lêrte pîn,\\ 
 & \textbf{den} herzogen von Gowerzin,\\ 
 & \textbf{Lishoise} wart Gundrie gegeben.\\ 
 & âne vröude stuont sîn leben,\\ 
5 & unz er ir werden minne \textbf{enpfant},\\ 
 & \textbf{der Turkoite} Florant:\\ 
 & \textbf{Artus Sagiven} ze wîbe bôt.\\ 
 & die het dâ vor der künec Lot.\\ 
 & der vürste ouch si vil gerne nam.\\ 
10 & diu gâbe minne wol gezam.\\ 
 & Artus was vrouwen milte,\\ 
 & sölher gâbe in niht bevilte.\\ 
 & \textbf{des} was mit râte vor \textbf{erdâht}.\\ 
 & \textbf{dô} disiu rede \textbf{wart} volbrâht,\\ 
15 & dô sprach diu herzoginne,\\ 
 & daz Gawan het ir minne\\ 
 & \textbf{mit prîse gedient}, \textbf{sô} hôch erkant,\\ 
 & daz er ir \textbf{lîbes} unde über ir lant\\ 
 & von rehte hêrre wære.\\ 
20 & diu rede dûhte swære\\ 
 & ir soldier, die manec sper\\ 
 & brâchen durch ir minne ger.\\ 
 & Gawan unde die gesellen sîn,\\ 
 & Arnive, diu \textbf{künigîn},\\ 
 & \hspace*{-.7em}\big| unde der werde Parcival\\ 
25 & \hspace*{-.7em}\big| unde \textbf{diu herzoginne} lieht gemâl,\\ 
 & \textbf{Sagive} unde Gundrie\\ 
 & \textbf{nâmen} urloup. Itonie\\ 
 & beleip bî Artus dâ.\\ 
30 & nû\textbf{ne} darf \textbf{mich} niemen \textbf{vrâgen}, wâ\\ 
\end{tabular}
\scriptsize
\line(1,0){75} \newline
G I L M Z Fr18 Fr24 \newline
\line(1,0){75} \newline
\textbf{1} \textit{Initiale} G L Z  \textbf{23} \textit{Initiale} Fr24  \newline
\line(1,0){75} \newline
\textbf{2} den] Dem Z  $\cdot$ Gowerzin] Gowerczin M \textbf{3} Lishoise] Liscoysen I Lytschoẏse L Lisoie M Lyshois Fr24  $\cdot$ Gundrie] kvndrie G L (M) (Z) Fr24 \textbf{5} ein phant vnz er ir werder minne vant I  $\cdot$ unz] Suz M \textbf{6} der] [der]: dem I Dem L (M) Z Fr24  $\cdot$ Turkoite] Turchoyde I Tuͯrkoýten L turkoiten M [Lyrhoiten]: Tvrkoiten Z Tvrkoyten Fr24  $\cdot$ Florant] floriant I Z \textbf{7} Artus] Artuͯs L Artusz M  $\cdot$ Sagiven] saifen I saiven M Seyven Z Sayven Fr24  $\cdot$ ze wîbe] zeminne I \textbf{8} der] den I (L) (Z) \textbf{10} minne wol] wol mynne M \textbf{11} Artus] Artusz M  $\cdot$ was] [gab]: wasz M \textbf{13} vor] da vor L Fr24  $\cdot$ erdâht] gidacht M \textbf{14} dô] Da M Z  $\cdot$ wart] was Z \textbf{15} dô] Da M \textbf{17} gedient] gediente Fr24 \textbf{18} er] \textit{om.} M  $\cdot$ ir lîbes] vber ir lip I  $\cdot$ ir lant] lant M \textbf{19} von rehte] Vnd rechter L \textbf{22} durch] dvrchen duͯrch L \textbf{23} die] \textit{om.} I \textbf{24} Arnive] Arniua I ARnẏue Fr18  $\cdot$ diu] vnde dy M \textbf{26} Parcival] parcifal G Z (Fr18) (Fr24) Parzifal I (L) (M) \textbf{25} lieht] licht L M \textbf{27} Sagive] saife I Saive M Seyve Z Saẏue Fr18 Sayve Fr24  $\cdot$ Gundrie] kvndrię G kvndrie L (M) Z Fr18 Fr24 \textbf{28} Itonie] Itonîe G Jconie Z Itonẏe Fr18 \textbf{29} bî] mit I  $\cdot$ Artus] artuse I (L) M (Fr18) Artvˢ Fr24 \textbf{30} darf] bidarff M (Fr24)  $\cdot$ mich niemen] nieman I (L) (M) [niht mich nieman]: mich nieman  Z \newline
\end{minipage}
\hspace{0.5cm}
\begin{minipage}[t]{0.5\linewidth}
\small
\begin{center}*T
\end{center}
\begin{tabular}{rl}
 & den ouch ir minne lêrte pîn,\\ 
 & \textbf{dem} herzogen von Gowerzin,\\ 
 & \textbf{Lyschoyen} wart Kundri\textit{e} gegeben.\\ 
 & âne vreude stuont sîn leben,\\ 
5 & unz er ir werde minne \textbf{ervant}.\\ 
 & \textbf{dem Turkoiten} Florant\\ 
 & \textbf{Artus Seyven} zuo wîbe bôt.\\ 
 & die hete d\textit{â} vor der künec Lot.\\ 
 & der vürste ouch si vil gerne nam.\\ 
10 & diu gâbe minne wol gezam.\\ 
 & Artus was vrouwen milte,\\ 
 & solicher gâbe in niht bevilte.\\ 
 & \textbf{daz} was mit râte \textbf{d\textit{â}} vor \textbf{erdâht}.\\ 
 & \textbf{dô} disiu rede \textbf{was} volbrâht,\\ 
15 & dô sprach diu herzoginne,\\ 
 & daz Gawan hete ir minne\\ 
 & \textbf{mit prîse gedienet}, \textbf{sô} hôhe erkant,\\ 
 & daz er ir \textbf{lîbes} und über ir lant\\ 
 & von rehte hêrre wære.\\ 
20 & diu rede dûhte swære\\ 
 & ir soldier, die manec sper\\ 
 & brâchen durch ir minne ger.\\ 
 & \begin{large}G\end{large}awan und die gesellen sîn,\\ 
 & Arnyve, diu \textbf{künegîn},\\ 
 & \hspace*{-.7em}\big| und der werde Parcifal\\ 
25 & \hspace*{-.7em}\big| und \textbf{die herzoginne} lieht gemâl,\\ 
 & \textbf{Seyve} und Kundrie\\ 
 & \textbf{nâmen} urloup. Itonie\\ 
 & bleip bî Artuse dâ.\\ 
30 & nû \textbf{en}darf \textbf{mich} nieman \textbf{vrâgen}, wâ\\ 
\end{tabular}
\scriptsize
\line(1,0){75} \newline
U V W Q R \newline
\line(1,0){75} \newline
\textbf{23} \textit{Initiale} U V  \newline
\line(1,0){75} \newline
\textbf{2} dem] Den W  $\cdot$ Gowerzin] gowerzein W kawerzin Q Gowerczin R \textbf{3} [L*]: Wart kvndrie do gegeben V  $\cdot$ Lyschoyen] Lysoien U Lyshoyen W Lishoise Q Lyschoyse R  $\cdot$ Kundrie] kuͦndrien U kundrien W kundrie Q kondire R \textbf{4} vreude] froͤden W \textbf{5} unz] Mit U  $\cdot$ ir werde] ir werden V W irr Q  $\cdot$ ervant] enpfant V (W) (Q) empfieng R \textbf{6} turkoiten] Turcoyten U turkoẏten V turkoite Q (R) \textbf{7} Seyven] Seẏuen V seyuen W Q (R) \textbf{8} dâ] do U V W Q R  $\cdot$ der] den V \textit{om.} Q  $\cdot$ Lot] lott Q \textbf{12} gâbe] \textit{om.} Q \textbf{13} dâ vor] do vor U V (W) do er Q vor R  $\cdot$ erdâht] gedacht R \textbf{14} dô] [*]: Nv V \textbf{15} dô] Da V \textbf{16} Gawan] herr gawan W \textbf{17} gedienet] verdienet W \textbf{18} er] \textit{om.} R \textbf{22} brâchen] [*chen]: E brachen V  $\cdot$ ir] irre U \textbf{23} Gawan] Herr gawan W \textbf{24} arnẏve [*]: die kv́negin V  $\cdot$ Arnyve] Vnd arnyue W Arniue Q Arnyue R \textbf{26} Parcifal] Parzifal U parzefal V partzifal W Q parczifal R \textbf{25} herzoginne] [*]: manige vrowe V  $\cdot$ lieht] licht Q \textbf{27} Seyve] Seẏue V Seyue W Q Seuye R  $\cdot$ Kundrie] kuͦndrie U kvndrye V kúndrie W kondrye R \textbf{28} Itonie] Jtonie U Q R Jconie V ytonie W \textbf{29} Artuse] artus R  $\cdot$ dâ] do V W \textbf{30} endarf] darff W bedarff R \newline
\end{minipage}
\end{table}
\end{document}
