\documentclass[8pt,a4paper,notitlepage]{article}
\usepackage{fullpage}
\usepackage{ulem}
\usepackage{xltxtra}
\usepackage{datetime}
\renewcommand{\dateseparator}{.}
\dmyyyydate
\usepackage{fancyhdr}
\usepackage{ifthen}
\pagestyle{fancy}
\fancyhf{}
\renewcommand{\headrulewidth}{0pt}
\fancyfoot[L]{\ifthenelse{\value{page}=1}{\today, \currenttime{} Uhr}{}}
\begin{document}
\begin{table}[ht]
\begin{minipage}[t]{0.5\linewidth}
\small
\begin{center}*D
\end{center}
\begin{tabular}{rl}
\textbf{225} & \begin{large}W\end{large}elt ir nû hœren, wie ez im \textbf{gestê}?\\ 
 & er kom des âbents an einen sê.\\ 
 & dâ heten geankert weideman,\\ 
 & den was daz wazzer undertân.\\ 
5 & dô si in rîten \textbf{sâhen},\\ 
 & \textbf{si} wâren dem stade sô nâhen,\\ 
 & daz si wol hôrten, swaz er sprach.\\ 
 & einen er ime schiffe \textbf{sach},\\ 
 & der hete an im \textbf{al}solch gewant,\\ 
10 & ob im dienden elliu lant,\\ 
 & daz \textbf{ez} niht bezzer \textbf{m\textit{ö}hte} sîn;\\ 
 & gefurriert sîn huot \textbf{was} pfâwîn.\\ 
 & den selben \textit{v}ischære\\ 
 & begunder vrâgen mære,\\ 
15 & daz er im riete durch got\\ 
 & unt durch sîner zühte gebot,\\ 
 & wâ er herberge m\textit{ö}hte hân.\\ 
 & \textbf{sus} antwurte im der trûrige man.\\ 
 & \textbf{Er sprach}: "hêrre, mir ist \textbf{niht bekant},\\ 
20 & daz weder wazzer \textbf{oder} lant\\ 
 & \textbf{inre} drîzec mîlen erbouwen sî,\\ 
 & wan ein \textbf{hûs} \textbf{lît} \textbf{uns} hie bî.\\ 
 & mit triwen ich iu râte dar.\\ 
 & war m\textit{ö}ht ir tâlanc anderswar?\\ 
25 & Dort an des \textbf{velses} \textbf{ende},\\ 
 & dâ kêrt zer zeswen hende.\\ 
 & sô ir \textbf{ûf hin komt an} den graben,\\ 
 & ich wæne, dâ müezet ir stille haben.\\ 
 & \textbf{bittet} \textbf{iu die brücken} nider lâzen\\ 
30 & unt \textbf{offen iu} die strâzen."\\ 
\end{tabular}
\scriptsize
\line(1,0){75} \newline
D \newline
\line(1,0){75} \newline
\textbf{1} \textit{Initiale} D  \textbf{19} \textit{Majuskel} D  \textbf{25} \textit{Majuskel} D  \newline
\line(1,0){75} \newline
\textbf{11} möhte] mohte D \textbf{13} vischære] wiscære D \textbf{17} möhte] mohte D \textbf{24} möht] moht D \newline
\end{minipage}
\hspace{0.5cm}
\begin{minipage}[t]{0.5\linewidth}
\small
\begin{center}*m
\end{center}
\begin{tabular}{rl}
 & \begin{large}W\end{large}ellet ir nû hœren, wie ez ime \textbf{ergê}?\\ 
 & er kam des âbendes an einen sê.\\ 
 & d\textit{â} heten g\textit{e}a\textit{n}kert weideman,\\ 
 & den was da\textit{z} \textit{w}azzer undertân.\\ 
 & \hspace*{-.7em}\big| \textbf{die} wâren dem stade sô nâhen,\\ 
5 & \hspace*{-.7em}\big| dô si in \textbf{dâ} rîten \textbf{gesâhen},\\ 
 & daz si wol hôrten, waz er sprach.\\ 
 & einen er imme schiffe \textbf{ersach},\\ 
 & der hete an ime \textbf{al} solich gewant,\\ 
10 & ob ime dienten alliu lant,\\ 
 & daz \textbf{er} niht bezzer \textbf{dörfte} sîn;\\ 
 & gefurrieret sîn huot pfæwîn.\\ 
 & den selben vischære\\ 
 & begunde er vrâgen mære,\\ 
15 & daz er ime riete durch got\\ 
 & und \textbf{ouch} durch sîner zühte gebot,\\ 
 & wâ er herberge möhte hân.\\ 
 & \textbf{dô} antwurte ime der trûrige man.\\ 
 & \textbf{er sprach}: "hêrre, mir ist \textbf{niht erkant},\\ 
20 & daz weder wazzer \textbf{noch} lant\\ 
 & \textbf{in} drîzic mîlen erbûwen sî,\\ 
 & wanne ein \textbf{hûs}, \textbf{daz} \textbf{lît} hie bî.\\ 
 & mit triuwen ich i\textit{u} \textit{r}âte dar.\\ 
 & war m\textit{ö}hte\textit{t} ir tâlanc anderswar?\\ 
25 & dort an des \textbf{velses} \textbf{ende},\\ 
 & d\textit{â} kêret zer zeswen hende.\\ 
 & sô ir \textbf{ûf \textit{h}in kumt an} den graben,\\ 
 & ich wæne, dâ müezet ir stille haben.\\ 
 & \textbf{heizet} \textbf{iu die brücke} nider lâzen\\ 
30 & und \textbf{offene\textit{n} iu} die strâzen."\\ 
\end{tabular}
\scriptsize
\line(1,0){75} \newline
m n o Fr69 \newline
\line(1,0){75} \newline
\textbf{1} \textit{Initiale} m   $\cdot$ \textit{Capitulumzeichen} n  \newline
\line(1,0){75} \newline
\textbf{1} ergê] ge o \textbf{2} des] das o  $\cdot$ einen] eẏn o \textbf{3} dâ] Das m Do n o  $\cdot$ geankert] gar kert m \textbf{4} daz wazzer] das ouge vnd wasser m \textbf{6} stade] staden n o  $\cdot$ nâhen] naher o \textbf{5} dâ] \textit{om.} n o  $\cdot$ gesâhen] sohen n [n]: sohen o \textbf{9} al] \textit{om.} n o \textbf{11} dörfte] dorffte n o \textbf{12} sîn] sine n \textbf{18} antwurte] antwurt n o  $\cdot$ ime] in o \textbf{20} noch] oder n o (Fr69) \textbf{21} mîlen] muͯlen o \textbf{22} daz] \textit{om.} n  $\cdot$ lît] ist Fr69 \textbf{23} iu râte] uͯch sage rote m \textbf{24} möhtet] mochtten m moͯchten n (o)  $\cdot$ ir] wir o \textbf{25} velses] felsches n (o) veltz Fr69 \textbf{26} dâ] Do m o  $\cdot$ zer] zuͦ o \textbf{27} hin] in m  $\cdot$ an] in o \textbf{28} dâ] do n duͯ o \textbf{30} offenen] offenet m \newline
\end{minipage}
\end{table}
\newpage
\begin{table}[ht]
\begin{minipage}[t]{0.5\linewidth}
\small
\begin{center}*G
\end{center}
\begin{tabular}{rl}
 & welt ir nû hœren, wiez im \textbf{gestê}?\\ 
 & er kom des âbendes an einen sê.\\ 
 & dâ heten geankert weideman,\\ 
 & den was daz wazzer undertân.\\ 
5 & dô si in rîten \textbf{sâhen},\\ 
 & \textbf{si} wâren dem stade sô nâhen,\\ 
 & daz si wol hôrten, waz er sprach.\\ 
 & einen er in dem schiffe \textbf{sach},\\ 
 & der het an im solch gewant,\\ 
10 & obe im dienten elliu lant,\\ 
 & daz \textbf{ez} niht bezzer \textbf{m\textit{ö}hte} sîn;\\ 
 & gefurriert sîn huot \textbf{was} pfâwîn.\\ 
 & den selben vischære\\ 
 & begunder vrâgen mære,\\ 
15 & daz er im riete durch got\\ 
 & unde \textbf{ouch} durch sîner zuht gebot,\\ 
 & wâ er \textbf{die} herberge m\textit{ö}ht hân.\\ 
 & \textbf{des} antwurte im der trûrige man.\\ 
 & \textbf{er sprach}: "hêrre, mirst \textbf{unerkant},\\ 
20 & daz weder wazzer \textbf{noch} lant\\ 
 & \textbf{in} drîzic mîlen erbûwen sî,\\ 
 & wan ein \textbf{burc}, \textbf{diu} \textbf{lît} hie bî.\\ 
 & mit triuwen ich iu râte dar.\\ 
 & \begin{large}W\end{large}ar m\textit{ö}ht ir tâlanc anderswar?\\ 
25 & dort an des \textbf{veldes} \textbf{ende},\\ 
 & dâ kêrt ze der zeswen hende.\\ 
 & sôr \textbf{ûf hin komet an} den graben,\\ 
 & ich wæne, dâ müezt ir stille haben.\\ 
 & \textbf{bit} \textbf{die brücke iu} nider lâzen\\ 
30 & unde \textbf{offen iu} die strâzen."\\ 
\end{tabular}
\scriptsize
\line(1,0){75} \newline
G I O L M Q R Z Fr21 Fr23 \newline
\line(1,0){75} \newline
\textbf{1} \textit{Initiale} O L M Z Fr21  \textbf{7} \textit{Initiale} I  \textbf{9} \textit{Capitulumzeichen} R  \textbf{24} \textit{Initiale} G  \newline
\line(1,0){75} \newline
\textbf{1} welt] ÷elt O Uvel Fr21  $\cdot$ nû] \textit{om.} R  $\cdot$ im] \textit{om.} I Q  $\cdot$ gestê] ste L M erge Fr23 \textbf{2} einen sê] eyn sehe M (Fr23) \textbf{3} dâ] Do Q  $\cdot$ heten] hattett R h::: Fr21  $\cdot$ geankert] geackert Q  $\cdot$ weideman] werde man Q ein weide::: Fr21 \textbf{4} den] Dem Fr21  $\cdot$ undertân] tan Q \textbf{5} dô] Da M Z \textbf{6} stade] staden L \textbf{7} waz] swaz O Fr21 \textbf{8} er] der Q  $\cdot$ sach] shach I \textbf{9} solch] alsolch O (L) (M) (Fr21) alsolhe Z  $\cdot$ gewant] gevant Fr21 \textbf{10} dienten] dieten Fr21 \textbf{11} daz] Dar M  $\cdot$ ez] er Q  $\cdot$ möhte] mohte G I O L (M) (Q) Z (Fr21)  $\cdot$ sîn] gesÿ M \textbf{12} gefurriert] gevorniret M Gefurrit Q  $\cdot$ sîn huot was] was sin huͤt I  $\cdot$ pfâwîn] pfellin Z \textbf{14} begunder] beGunder er I Begondin M Begunde Q (R)  $\cdot$ vrâgen] [tragen]: fragen O \textbf{15} durch] darch M \textbf{16} sîner] sine M \textbf{17} die] \textit{om.} I O M Q R Z Fr21  $\cdot$ herberge] herbreche Q  $\cdot$ möht] moht G (I) (O) (L) (M) (Q) Z Fr21  $\cdot$ hân] gehan M Q \textbf{18} des] Daz L  $\cdot$ antwurte] Antwurt I (O) (Q) (R) (Z) (Fr21) \textbf{19} er sprach] \textit{om.} L  $\cdot$ hêrre] \textit{om.} Z  $\cdot$ unerkant] vmbe chant O (Q) (R) (Z) (Fr21) niht bekant L [bekant]: vnbekant M \textbf{21} in] Jnner O L (R) Fr21 Jn er Q  $\cdot$ erbûwen] erbowet Q (Z) \textbf{22} wan] Niwen I (O) (M) (Fr21) Nyemad Q Nument R  $\cdot$ burc diu] hus daz I (Q) (R) (Z) (Fr21) hvs O (M)  $\cdot$ lît] \textit{om.} L  $\cdot$ bî] nahe bý L (Fr21) \textbf{23} râte] raten M \textbf{24} möht] moht G O (L) (M) (Q) Z Fr21  $\cdot$ tâlanc] helt M  $\cdot$ anderswar] anders varn I anderschwa R andersvar Fr21 \textbf{25} veldes] velses I M Z wellea Q wildes R \textbf{26} dâ] Do O L Q Das M  $\cdot$ zeswen] rechtin M zserwen Q  $\cdot$ hende] henden Q \textbf{27} ûf hin] hin vf I O L (M) (Q) Fr21  $\cdot$ den] dem Q \textbf{28} dâ] do Q  $\cdot$ haben] halden M \textbf{29} bit die brücke iu] bitet ev die bruchge I (L) (Q) betet (Biz Z ) die bruckin uch M (Z) Bitiv die brvke Fr21 \textbf{30} offen] vffene M  $\cdot$ iu] \textit{om.} O L M Q R Fr21  $\cdot$ strâzen] strasse R \newline
\end{minipage}
\hspace{0.5cm}
\begin{minipage}[t]{0.5\linewidth}
\small
\begin{center}*T
\end{center}
\begin{tabular}{rl}
 & \begin{large}W\end{large}elt ir nû hœren, wiez im \textbf{gestê}?\\ 
 & er kom des âbendes an einen sê.\\ 
 & dâ heten geankert weideman,\\ 
 & den was daz wazzer undertân.\\ 
5 & dô sin rîten \textbf{sâhen},\\ 
 & \textbf{si} wâren dem stade sô nâhen,\\ 
 & daz si wol hôrten, swaz er sprach.\\ 
 & einen er in dem schiffe \textbf{sach},\\ 
 & der hete an im solch gewant,\\ 
10 & ob im dienten elliu lant,\\ 
 & daz \textbf{ez} niht bezzer \textbf{m\textit{ö}hte} sîn;\\ 
 & gefurriert sîn huot \textbf{was} pfæwîn.\\ 
 & den selben vischære\\ 
 & begunder vrâgen mære,\\ 
15 & daz er im riete durch got\\ 
 & unde durch sîner zühte gebot,\\ 
 & wâ er \textbf{die} herberge m\textit{ö}hte hân.\\ 
 & \textbf{Sus} antwurtim der trûrige man:\\ 
 & "hêrre, mir ist \textbf{niht bekant},\\ 
20 & daz weder wazzer \textbf{oder} lant\\ 
 & \textbf{in} drîzic mîle\textit{n} erbûwen sî,\\ 
 & wan ein \textbf{burc}, \textbf{diu} \textbf{stêt} hie bî.\\ 
 & mit triuwen ich iu râte dar.\\ 
 & war m\textit{ö}htir tâlanc anderswar?\\ 
25 & dort an des \textbf{velses} \textbf{wende},\\ 
 & dâ kêret zer zesewen hende.\\ 
 & sô ir \textbf{komet hin ûf} den graben,\\ 
 & ich wæne, dâ müezet ir stille haben.\\ 
 & \textbf{bitet} \textbf{iu die brücke} nider lâzen\\ 
30 & unde \textbf{iu offenen} die strâzen."\\ 
\end{tabular}
\scriptsize
\line(1,0){75} \newline
T U V W \newline
\line(1,0){75} \newline
\textbf{1} \textit{Überschrift:} Hie kvmet parzifal zem ersten male zvͦm grale do er von pelrepere schiet V   $\cdot$ \textit{Initiale} T U V W  \textbf{18} \textit{Majuskel} T  \newline
\line(1,0){75} \newline
\textbf{1} nû] \textit{om.} W  $\cdot$ gestê] [*]: erge V ergie W \textbf{3} dâ] Do U V W  $\cdot$ weideman] werde man W \textbf{5} dô] So U  $\cdot$ sâhen] v́rsahen V \textbf{6} stade] gaste U [*]: staden V \textbf{7} swaz] waz U V (W) \textbf{9} solch] solches W \textbf{10} dienten] diente U \textbf{11} möhte] mohte T (U) \textbf{12} gefurriert] Geformieret U \textbf{15} riete] seite V \textbf{17} möhte] mohte T (U) \textbf{18} antwurtim] antwirtim T antwurt im V  $\cdot$ trûrige] [trvrig]: trvrige V \textbf{21} mîlen] mile T  $\cdot$ erbûwen] in baw W \textbf{22} diu] \textit{om.} W  $\cdot$ hie] hie nahe W \textbf{23} râte] raten U \textbf{24} möhtir] mohtir T (U)  $\cdot$ anderswar] [*]: war U andeswar V W \textbf{25} wende] ende U V W \textbf{26} dâ] Do U V W  $\cdot$ zer] zuͦ ir U  $\cdot$ zesewen] rehten V \textbf{27} ir komet] komet ir U [kv́ment]: koment ir V  $\cdot$ hin] \textit{om.} W  $\cdot$ den] de U \textbf{28} dâ] do W \textbf{29} iu die brücke] die bruͦcke vch U (W) \textbf{30} iu offenen] offenen vch U (W) offenen [*]: v́ch  V \newline
\end{minipage}
\end{table}
\end{document}
