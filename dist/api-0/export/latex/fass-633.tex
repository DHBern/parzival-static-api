\documentclass[8pt,a4paper,notitlepage]{article}
\usepackage{fullpage}
\usepackage{ulem}
\usepackage{xltxtra}
\usepackage{datetime}
\renewcommand{\dateseparator}{.}
\dmyyyydate
\usepackage{fancyhdr}
\usepackage{ifthen}
\pagestyle{fancy}
\fancyhf{}
\renewcommand{\headrulewidth}{0pt}
\fancyfoot[L]{\ifthenelse{\value{page}=1}{\today, \currenttime{} Uhr}{}}
\begin{document}
\begin{table}[ht]
\begin{minipage}[t]{0.5\linewidth}
\small
\begin{center}*D
\end{center}
\begin{tabular}{rl}
\textbf{633} & \begin{large}D\end{large}ô sprach mîn hêr Gawan:\\ 
 & "ir sult sîn vürbaz künde hân,\\ 
 & sît er sich prîse nâhet\\ 
 & unt des mit \textbf{willen} gâhet.\\ 
5 & von sînem munde ich hân vernomen,\\ 
 & daz er herzenlîche ist komen\\ 
 & mit dienste, ob irs \textbf{geruochet},\\ 
 & sô daz er helfe suochet\\ 
 & durch trôst an iwer minne.\\ 
10 & künec durch küneginne\\ 
 & sol billîche enpfâhen nôt.\\ 
 & vrouwe, hiez iwer vater Lot,\\ 
 & sô sît irz, die er meinet,\\ 
 & nâch der sîn herze weinet.\\ 
15 & Unt heizet ir Itonje,\\ 
 & sô tuot ir im von herzen wê.\\ 
 & ob ir triwe kunnet tragen,\\ 
 & sô sult ir \textbf{wenden im} \textbf{sîn} klagen.\\ 
 & \textbf{beidenthalp} wil ich des bote sîn.\\ 
20 & vrouwe, nemt diz vingerlîn;\\ 
 & daz sant iu der clâre.\\ 
 & ouch wirb ich\textbf{z} âne vâre,\\ 
 & vrouwe, daz lât albalde an mich."\\ 
 & Si begunde \textbf{al rôt} \textbf{värwen} sich;\\ 
25 & als \textbf{ê} was gevar ir munt,\\ 
 & wart al dem antlütze kunt.\\ 
 & dar nâch \textbf{schiere} wart si anders var.\\ 
 & si greif \textbf{al} \textbf{blûweclîche} dar;\\ 
 & daz vingerlîn wart schiere erkant,\\ 
30 & si enpfieng ez mit ir clâren hant.\\ 
\end{tabular}
\scriptsize
\line(1,0){75} \newline
D Z Fr63 \newline
\line(1,0){75} \newline
\textbf{1} \textit{Initiale} D Z Fr63  \textbf{15} \textit{Majuskel} D  \textbf{24} \textit{Majuskel} D  \newline
\line(1,0){75} \newline
\textbf{4} willen] trewen Z \textbf{11} sol] So Z \textbf{13} sô] Do Fr63 \textbf{14} weinet] meinet Z \textbf{15} Itonje] Jtonie D Fr63 Jconie Z \textbf{17} kunnet] kvndet Z (Fr63) \textbf{22} ichz] ichs Fr63 \textbf{25} ê was gevar] gevar was Z \textbf{28} blûweclîche] blodclichen Z \textbf{30} enpfieng ez] enpfienges Fr63 \newline
\end{minipage}
\hspace{0.5cm}
\begin{minipage}[t]{0.5\linewidth}
\small
\begin{center}*m
\end{center}
\begin{tabular}{rl}
 & dô sprach mîn hêr Gawan:\\ 
 & "ir solt sîn vürbaz künde hân,\\ 
 & sît er sich prîse nâhet\\ 
 & und des mit \textbf{willen} gâhet.\\ 
5 & von sînem munde ich hân vernomen,\\ 
 & daz er \dag Herczeloide\dag  ist komen\\ 
 & mit dienst, ob irs \textbf{ruochet},\\ 
 & sô daz \dag ir\dag  helfe suochet\\ 
 & durch trôst an i\textit{uwe}r minne.\\ 
10 & künic durch küniginne\\ 
 & sol billîch enpfâhen nôt.\\ 
 & vrowe, hiez iuwer vater Lot,\\ 
 & sô sît irz, die er meinet,\\ 
 & nâch der sîn herz weinet.\\ 
15 & und heizet ir Ithonie,\\ 
 & sô tuot ir im von herzen wê.\\ 
 & ob ir triuwe kunnet tragen,\\ 
 & sô solt ir \textbf{wenden im} \textbf{sîn} klagen.\\ 
 & \textbf{beidenthalp} wil ich des bote sîn.\\ 
20 & vrouwe, nemt diz vingerlîn;\\ 
 & daz sant iu der clâre.\\ 
 & ouch wirb ich\textbf{z} âne vâre,\\ 
 & vrowe, daz lât albalde an mich."\\ 
 & si begunde \textbf{allerêrst} \textbf{vröwen} sich.\\ 
25 & als \textbf{ê} was gevar ir munt,\\ 
 & wart aldem antlitz kunt.\\ 
 & dar nâch \textbf{schier} wart si anders var.\\ 
 & si greif \textbf{al}\textbf{blœdeclîchen} dar;\\ 
 & daz vingerlîn wart schier erkant,\\ 
30 & si enpfienc ez mit ir clâren hant.\\ 
\end{tabular}
\scriptsize
\line(1,0){75} \newline
m n o \newline
\line(1,0){75} \newline
\textbf{1} \textit{Capitulumzeichen} n  \newline
\line(1,0){75} \newline
\textbf{1} hêr] herre her n \textbf{6} Herczeloide] hertzoloiden n herczeleit o \textbf{7} ruochet] geruͦchent n (o) \textbf{9} an] \textit{om.} n  $\cdot$ iuwer] ir m \textbf{11} enpfâhen] enpfohen hohe n enpfhohen o \textbf{13} er] ir o \textbf{15} Ithonie] jtonie m n (o) \textbf{17} kunnet] konnen m (n) (o) \textbf{19} wil] \textit{om.} n \textbf{20} diz] das n \textbf{21} iu] úch úch n \textbf{24} begunde] begúnden o  $\cdot$ vröwen] ferwen n (o) \textbf{25} ê] es n \textbf{26} antlitz] anczlit o \textbf{27} dar nâch] Dannach o  $\cdot$ schier] schie o \textbf{28} alblœdeclîchen] alles bloͯdeclichen n \textbf{29} vingerlîn] fingelin o \newline
\end{minipage}
\end{table}
\newpage
\begin{table}[ht]
\begin{minipage}[t]{0.5\linewidth}
\small
\begin{center}*G
\end{center}
\begin{tabular}{rl}
 & \begin{large}D\end{large}ô sprach mîn hêr Gawan:\\ 
 & "ir sult sîn vürbaz künde hân,\\ 
 & sît er sich brîse nâ\textit{h}et\\ 
 & unt des mit \textbf{triuwen} gâhet.\\ 
5 & von sînem munde ich hân vernomen,\\ 
 & daz er herzelîche ist komen\\ 
 & mit dienst, obe irs \textbf{geruochet},\\ 
 & sô daz er helfe suochet\\ 
 & durch trôst an iuwer minne.\\ 
10 & künic durch küneginne\\ 
 & sol billîchen enpfâhen nôt.\\ 
 & vrouwe, hiez iuwer vater Lot,\\ 
 & sô sît irz, die er m\textit{ei}net,\\ 
 & nâch der sîn herze weinet.\\ 
15 & unde heizet ir Itonie,\\ 
 & sô tuot ir im von herzen wê.\\ 
 & ob ir triuwe kunnet tragen,\\ 
 & sô sult ir \textbf{im wenden} \textbf{sîn} klagen.\\ 
 & \textbf{bêdenthalben} wil ich des bote sîn.\\ 
20 & vrouwe, nemet diz vingerlîn;\\ 
 & daz sande iu der clâre.\\ 
 & o\textit{uch} wirbe ich ân vâre,\\ 
 & vrouwe, daz lât albalde an mich."\\ 
 & si begunde \textbf{alrôt} \textbf{värwen} sich;\\ 
25 & alsô was gevar ir munt,\\ 
 & wart al dem antlütze kunt.\\ 
 & dar nâch wart si anders var.\\ 
 & si greif \textbf{al} \textbf{bl\textit{ûc}lîchen} dar;\\ 
 & daz vingerlîn wart schier erkant,\\ 
30 & si enpfieng ez mit ir clâren hant.\\ 
\end{tabular}
\scriptsize
\line(1,0){75} \newline
G I L M Z Fr51 \newline
\line(1,0){75} \newline
\textbf{1} \textit{Initiale} G L Z  \textbf{15} \textit{Initiale} I  \newline
\line(1,0){75} \newline
\textbf{1} Dô] Da M  $\cdot$ mîn] \textit{om.} Fr51  $\cdot$ hêr Gawan] ergawan M \textbf{3} nâhet] nabet G \textbf{4} triuwen] truwe Fr51 \textbf{5} sînem] sinen Fr51  $\cdot$ ich] \textit{om.} M \textbf{6} er] \textit{om.} M \textbf{7} geruochet] ruͯchet L (Fr51) \textbf{11} sol] so Z  $\cdot$ enpfâhen] ein phahen L \textbf{12} hiez] heiz Fr51  $\cdot$ vater] vater der kunc I \textbf{13} irz] ir Fr51  $\cdot$ meinet] minnet G  $\cdot$ er] ir M \textbf{14} weinet] meinet Z \textbf{15} Itonie] ytonie G Jtonie I (L) Jthonie M Jconie Z eltonie Fr51 \textbf{16} tuot] do Fr51 \textbf{17} kunnet] kvndet L (M) Z kvnnen Fr51 \textbf{18} im wenden] wenden L Fr51 wenden im Z  $\cdot$ sîn klagen] sine clage M sines clagen Fr51 \textbf{19} bêdenthalben wil ich des] beidenthalben wil ich I Beidenthalb willic des M Des willich en Fr51 \textbf{22} ouch wirbe ich] ob wirbe ih G ich wirb ez I Ouch wirb ichz Z \textbf{23} albalde] gar Fr51 \textbf{24} alrôt] rot Fr51  $\cdot$ värwen] weren L \textbf{25} was gevar] gevar was Z \textbf{26} wart al dem] [da*]: daz wart allem ir I Wart al den Fr51 \textbf{27} wart si] was si I wart sie schire L schiere wart sy M (Z) wart Fr51  $\cdot$ var] geuar I \textbf{28} al blûclîchen] al bluͦchech lichen G al blodeclichen M (Z) blotlichen Fr51 \textbf{29} wart schier erkant] wart schier kant L was ir bekant Fr51 \textbf{30} enpfieng] enphienge G nam Fr51  $\cdot$ clâren] \textit{om.} I \newline
\end{minipage}
\hspace{0.5cm}
\begin{minipage}[t]{0.5\linewidth}
\small
\begin{center}*T
\end{center}
\begin{tabular}{rl}
 & \begin{large}D\end{large}ô sprach mîn hêr Gawan:\\ 
 & "ir solt sîn vürbaz künde hân,\\ 
 & sît er sich prîse nâhet\\ 
 & und des mit \textbf{triuwen} gâhet.\\ 
5 & von sîme munde ich hân vernomen,\\ 
 & daz er herzeclîche ist komen\\ 
 & mit dienste, ob ir es \textbf{geruochet},\\ 
 & sô daz er helfe suochet\\ 
 & durch trôst an iuwer minne.\\ 
10 & künec durch küneginne\\ 
 & sol billîche entvâhen nôt.\\ 
 & vrouwe, hiez iuwer vater Lot,\\ 
 & sô sît ir ez, die er meinet,\\ 
 & nâch der sîn herze weinet.\\ 
15 & und heizet ir Itonie,\\ 
 & sô tuot ir im von herzen wê.\\ 
 & ob ir triuwe kunnet tragen,\\ 
 & sô solt ir \textbf{wenden} \textbf{sîne} klagen.\\ 
 & \textbf{beidersîte} wil ich des bote sîn.\\ 
20 & vrouwe, nemet \textbf{hin} diz vingerlîn;\\ 
 & daz sante iu der clâre.\\ 
 & ouch w\textit{i}rbe ich âne vâre,\\ 
 & vrouwe, daz lât al balde \textit{an} mich."\\ 
 & si begunde \textbf{alrôt} \textbf{värwen} sich;\\ 
25 & alsô was gevar ir munt,\\ 
 & wart al dem antlitze kunt.\\ 
 & dar nâch \textbf{schiere} wart si anders var.\\ 
 & si greif \textbf{gar} \textbf{blœdeclîche} dar;\\ 
 & \begin{large}D\end{large}az vingerlîn wart schiere erkant,\\ 
30 & si entvieng ez mit ir clâren hant.\\ 
\end{tabular}
\scriptsize
\line(1,0){75} \newline
U V W Q R \newline
\line(1,0){75} \newline
\textbf{1} \textit{Initiale} U V W R  \textbf{29} \textit{Initiale} U V  \newline
\line(1,0){75} \newline
\textbf{2} solt] sv́ln V \textbf{3} prîse] [p*]: prise V preises W \textbf{5} ich hân] han Jch R \textbf{6} ist] sige R \textbf{7} dienste] dienste >an úch< R \textbf{8} er] ir W \textbf{10} küneginne] [*]: kv́niginne V \textbf{13} ez] \textit{om.} V R  $\cdot$ er] er do V  $\cdot$ meinet] minnet R \textbf{14} der] der sich R  $\cdot$ weinet] [seunet]: seÿnet R \textbf{15} Itonie] Jtonie U [y*onie]: ytonie V ytonie W Q Jtonye R \textbf{16} tuot] thuͦn W \textbf{17} kunnet] kúndet W kunnen R \textbf{18} sîne] [sine]: sin V sein W \textbf{19} beidersîte] Beidenthalp V (W) (Q) (R)  $\cdot$ wil ich] wir euch Q  $\cdot$ des] der W (R) o\textit{m. } Q \textbf{20} diz] das Q (R) \textbf{22} wirbe] worbe U wúrb V W (R)  $\cdot$ ich] [ich *]: ichz V \textbf{23} al] als Q  $\cdot$ an] \textit{om.} U \textbf{26} al] an V W  $\cdot$ antlitze] [antl*]: antlitze V [al]: antlutzen Q \textbf{28} gar] al V (W) Q (R)  $\cdot$ blœdeclîche] blv́cklichen V (R) blodichen Q \textbf{30} ez] es schier Q  $\cdot$ clâren] \textit{om.} W Q \newline
\end{minipage}
\end{table}
\end{document}
