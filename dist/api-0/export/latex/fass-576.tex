\documentclass[8pt,a4paper,notitlepage]{article}
\usepackage{fullpage}
\usepackage{ulem}
\usepackage{xltxtra}
\usepackage{datetime}
\renewcommand{\dateseparator}{.}
\dmyyyydate
\usepackage{fancyhdr}
\usepackage{ifthen}
\pagestyle{fancy}
\fancyhf{}
\renewcommand{\headrulewidth}{0pt}
\fancyfoot[L]{\ifthenelse{\value{page}=1}{\today, \currenttime{} Uhr}{}}
\begin{document}
\begin{table}[ht]
\begin{minipage}[t]{0.5\linewidth}
\small
\begin{center}*D
\end{center}
\begin{tabular}{rl}
\textbf{576} & \textbf{\textit{\begin{large}I\end{large}}n} der jugende an sîn ende.\\ 
 & diu magt mit ir hende\\ 
 & des zobels \textbf{roufte} und \textbf{habt} in dar\\ 
 & vür \textbf{sîne nasen}. dô nam si war,\\ 
5 & \textbf{ob} der âtem daz hâr sô regete,\\ 
 & daz \textbf{er} sich \textbf{inder} wegete.\\ 
 & der âtem \textbf{wart dâ} vunden.\\ 
 & an den selben stunden\\ 
 & hiez si balde springen,\\ 
10 & ein lûter wazzer bringen.\\ 
 & ir gespil wol gevar\\ 
 & \textbf{brâht ir daz} snellîche dar.\\ 
 & Diu magt schoub ir vingerlîn\\ 
 & zwischen die zene sîn;\\ 
15 & mit \textbf{grôzen vuogen} daz geschach.\\ 
 & dô gôz si daz wazzer nâch,\\ 
 & sanfte und aber mêre.\\ 
 & si\textbf{ne} gôz \textbf{iedoch} niht sêre,\\ 
 & unze \textbf{daz} er diu ougen \textbf{ûf} swanc.\\ 
20 & er \textbf{bôt in dienst unt sagete in} danc,\\ 
 & den zwein süezen kinden:\\ 
 & "daz ir mich soldet vinden\\ 
 & sus ungezogenlîche ligen,\\ 
 & ob daz wirt von iu verswigen,\\ 
25 & daz prüeve ich \textbf{iu} vür güete.\\ 
 & iwer zuht iuch dran behüete."\\ 
 & Si \textbf{jâhen}: "ir lâget und liget\\ 
 & als der des hœhsten prîses pfliget.\\ 
 & ir habt den prîs alhie bezalt,\\ 
30 & \textbf{des} ir mit vreuden werdet alt;\\ 
\end{tabular}
\scriptsize
\line(1,0){75} \newline
D Fr7 Fr59 \newline
\line(1,0){75} \newline
\textbf{1} \textit{Initiale} D  \textbf{7} \textit{Initiale} Fr7  \textbf{13} \textit{Majuskel} D  \textbf{27} \textit{Majuskel} D  \newline
\line(1,0){75} \newline
\textbf{1} In] ÷N D \textbf{3} habt in] habeten Fr7 \textbf{6} er] es Fr7  $\cdot$ wegete] vegete Fr7 \textbf{7} dâ] do Fr59 \textbf{10} wazzer] wasserz Fr7 \textbf{13} ir] ein Fr7 \textbf{18} sine] si Fr7  $\cdot$ sêre] zesere Fr7 \textbf{19} unze daz] Wen Fr7  $\cdot$ diu] die Fr7 \textbf{24} wirt] were Fr7 \textbf{25} iu vür güete] nv verguͤte Fr7 \newline
\end{minipage}
\hspace{0.5cm}
\begin{minipage}[t]{0.5\linewidth}
\small
\begin{center}*m
\end{center}
\begin{tabular}{rl}
 & \textbf{in} der jugent an sîn ende.\\ 
 & diu maget mit ir hende\\ 
 & des zobels \textbf{rouft} und \textbf{habt} in dar\\ 
 & vür \textbf{sîn nasen}. dô nam \textit{si} war,\\ 
5 & \textbf{ob} der âtem daz hâr sô reget,\\ 
 & daz \textbf{ez} sich \textbf{indert} weget.\\ 
 & der âtem \textbf{d\textit{â} wart} vunden.\\ 
 & an den selben stunden\\ 
 & hiez si balde springen,\\ 
10 & ein lûter wazzer bringen.\\ 
 & \hspace*{-.7em}\big| \textbf{daz brâht} snelleclîch dar\\ 
 & \hspace*{-.7em}\big| ir gespil wol gevar.\\ 
 & diu maget schoup ir vingerlîn\\ 
 & enzwischen die zene sîn;\\ 
15 & mit \textbf{grôze\textit{n} vuogen} daz geschach.\\ 
 & dô gôz si daz wazzer nâch,\\ 
 & sanft und aber mêre.\\ 
 & si gôz \textbf{ie dô} niht sêre,\\ 
 & unz \textbf{daz} er diu ougen swanc.\\ 
20 & er \textbf{bôt in dienst und sagt in} danc,\\ 
 & den zwein süezen kinden:\\ 
 & "daz ir mich soltet vinden\\ 
 & sus ungez\textit{o}g\textit{e}lîche ligen,\\ 
 & ob daz wirt von iu verswigen,\\ 
25 & daz brüefe ich \textbf{iu} vür güete.\\ 
 & iuwer zuht iuch dâr an behüete."\\ 
 & si \textbf{sprach}: "ir lâget und liget\\ 
 & als der des hœhesten prîses pflig\textit{e}t.\\ 
 & ir habt den prîs al hie bezalt,\\ 
30 & \textbf{daz} ir mit vröuden werdet alt;\\ 
\end{tabular}
\scriptsize
\line(1,0){75} \newline
m n o \newline
\line(1,0){75} \newline
\newline
\line(1,0){75} \newline
\textbf{3} rouft] rauff o \textbf{4} nam] man o  $\cdot$ si] er m \textbf{6} indert] iergen n \textbf{7} dâ] do m n o \textbf{11} ir] Jre n \textbf{15} grôzen] grosse m \textbf{18} ie dô] ẏedoch n (o) \textbf{19} unz daz] Vff vntze [er]: das n \textbf{20} sagt] sagete n \textbf{22} soltet] solten n \textbf{23} ungezogelîche] vngezugecliche m \textbf{25} brüefe] pruͯse o  $\cdot$ güete] guͦt o \textbf{28} pfliget] pfligent m \textbf{30} werdet] werden n o \newline
\end{minipage}
\end{table}
\newpage
\begin{table}[ht]
\begin{minipage}[t]{0.5\linewidth}
\small
\begin{center}*G
\end{center}
\begin{tabular}{rl}
 & \textbf{\begin{large}V\end{large}on} der jugent an sîn ende.\\ 
 & diu maget mit ir hende\\ 
 & des zobels \textbf{brach} unde \textbf{habet} \textit{in dar}\\ 
 & vür \textbf{sînem munt}. dô nam si war,\\ 
5 & \textbf{ob} der âtem daz hâr sô regete,\\ 
 & daz \textbf{ez} sich \textbf{iender} wegete.\\ 
 & der âtem \textbf{wart dâ} vunden.\\ 
 & an den selben stunden\\ 
 & hiez si balde springen,\\ 
10 & ein lûter wazzer bringen.\\ 
 & ir gespil wolgevar\\ 
 & \textbf{brâhte ir daz} snellîchen dar.\\ 
 & diu maget schoup ir vingerlîn\\ 
 & zwischen die zene sîn;\\ 
15 & mit \textbf{grôzer vuoge} daz geschach.\\ 
 & dô gôz si daz wazzer nâch,\\ 
 & sanfte unde aber mêre.\\ 
 & si gôz \textbf{iedoch} niht sêre,\\ 
 & unze \textbf{daz} er diu ougen \textbf{ûf} swanc.\\ 
20 & er \textbf{saget in} \textbf{genâde unde} danc,\\ 
 & den zwein süezen kinden:\\ 
 & "daz ir mich soldet vinden\\ 
 & sus ungezogenlîchen ligen,\\ 
 & ob daz wirt von iu verswigen,\\ 
25 & daz prüeve ich \textbf{iu} vür güete.\\ 
 & iuwer zuht iuch dran behüete."\\ 
 & si \textbf{jâhen}: "ir lâget unde liget\\ 
 & alse de\textit{r} \textit{d}es hœhesten brîses pfliget.\\ 
 & ir habet den brîs al hie bezalt,\\ 
30 & \textbf{des} ir mit vröuden werdet alt;\\ 
\end{tabular}
\scriptsize
\line(1,0){75} \newline
G I L M Z \newline
\line(1,0){75} \newline
\textbf{1} \textit{Initiale} G Z  \textbf{3} \textit{Initiale} L  \textbf{7} \textit{Initiale} I  \newline
\line(1,0){75} \newline
\textbf{1} Von] Jn L (M) (Z) \textbf{2} hende] gebende L \textbf{3} des zobels brach] nam des zobels balc I Des zobels roͮfte L (M) (Z)  $\cdot$ habet] bracht L hatte M  $\cdot$ in dar] mit ir hende G \textbf{4} sînem munt] sinen munt I sin nasen L (M) Z  $\cdot$ dô] da M Z \textbf{6} ez] er L  $\cdot$ iender] nider I irgen M iendert Z \textbf{9} balde] baldin M \textbf{12} daz] dar Z  $\cdot$ snellîchen] snelli M \textbf{13} schoup] schuff M \textbf{14} zene] zende I \textbf{15} grôzer vuoge] groszir gefuge M grozzen fvgen Z \textbf{16} dô] Da M Z \textbf{19} unze] Bisz M  $\cdot$ daz] \textit{om.} I L M  $\cdot$ swanc] geswanc M \textbf{20} saget in genâde unde] bot in dinst vnd sagt in L (M) (Z) \textbf{22} daz] der I \textbf{25} iu] \textit{om.} I L \textbf{27} jâhen] sprachen M  $\cdot$ lâget] lit M \textbf{28} der des] der der des G der [der]: des Z \textbf{29} al] \textit{om.} I \textbf{30} werdet] werdin M \newline
\end{minipage}
\hspace{0.5cm}
\begin{minipage}[t]{0.5\linewidth}
\small
\begin{center}*T
\end{center}
\begin{tabular}{rl}
 & \textbf{In} der jugent an sînen ende.\\ 
 & diu magt mi\textit{t} ir hende\\ 
 & des zobels \textbf{roufte} und \textbf{brâht} in dar\\ 
 & vür \textbf{sîne nase}. dô nam si war,\\ 
5 & \textbf{dô} der âtem daz hâr sô regte,\\ 
 & daz \textbf{ez} sich \textbf{wider} wegte.\\ 
 & der âtem \textbf{wart d\textit{â}} vunden.\\ 
 & an den selben stunden\\ 
 & hiez si balde springen,\\ 
10 & ein lûter wazzer bringen.\\ 
 & ir gespi\textit{l} wolgevar\\ 
 & \textbf{brâht in daz} snelleclîch dar.\\ 
 & diu magt schoup ir vingerlîn\\ 
 & zwischen die zene sîn;\\ 
15 & mit \textbf{grôzer gevuoge} daz geschach.\\ 
 & dô gôz si daz wazzer nâch,\\ 
 & sanfte und aber mêre.\\ 
 & si \textit{gôz} \textbf{iedoch} niht sêre,\\ 
 & unz er diu ougen \textbf{ûf} swanc.\\ 
20 & er \textbf{bôt in dienst und sagt in} danc,\\ 
 & den zweien süezen kinden:\\ 
 & "daz ir \textit{m}ic\textit{h} soldet vinden\\ 
 & sus ungezogenlîchen ligen,\\ 
 & ob daz wirt von iu verswigen,\\ 
25 & daz prüeve ich vür güete.\\ 
 & iuwer zuht iuch dran behüete."\\ 
 & si \textbf{jâhen}: "ir l\textit{â}get und ligt\\ 
 & als der des hœhsten prîses pfligt.\\ 
 & ir habt den prîs al hie bezalt,\\ 
30 & \textbf{des} ir mit vreuden werdet alt;\\ 
\end{tabular}
\scriptsize
\line(1,0){75} \newline
Q R W V U \newline
\line(1,0){75} \newline
\textbf{1} \textit{Initiale} Q   $\cdot$ \textit{Capitulumzeichen} R  \textbf{27} \textit{Initiale} W  \newline
\line(1,0){75} \newline
\textbf{1} \textit{Die Verse 553.1-599.30 fehlen} U   $\cdot$ sînen] sin R V (W) \textbf{2} magt] Junckfrow R  $\cdot$ mit] mir Q \textbf{3} brâht] habt R (V) huͦb W \textbf{4} nase] nasen R W V \textbf{5} dô] Ob R W V  $\cdot$ sô] \textit{om.} R \textbf{6} wider] yendert R (W) iergent V \textbf{7} dâ] do Q R W V \textbf{10} lûter wazzer] lauters wasser dar W \textbf{11} gespil] gespilen Q  $\cdot$ wolgevar] schon geuar W \textbf{12} in] ir R (V) o\textit{m. } W \textbf{13} schoup] schopt R \textbf{15} grôzer gevuoge] grossen fuͯgen R (W) (V) \textbf{16} daz wazzer] des wassers R \textbf{18} gôz] \textit{om.} Q \textbf{19} unz] Vncz das R (V)  $\cdot$ diu ougen ûf] vf die oͮgen V \textbf{22} mich] nicht Q  $\cdot$ soldet] solten V \textbf{23} sus] Als Q \textbf{25} ich] ich úch R (W) (V)  $\cdot$ güete] guͦtet W \textbf{26} behüete] behuͦtet W \textbf{27} jâhen] sprachen R  $\cdot$ lâget] liget Q \textbf{29} al hie] [*]: alhie V \newline
\end{minipage}
\end{table}
\end{document}
