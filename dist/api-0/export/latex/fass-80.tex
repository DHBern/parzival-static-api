\documentclass[8pt,a4paper,notitlepage]{article}
\usepackage{fullpage}
\usepackage{ulem}
\usepackage{xltxtra}
\usepackage{datetime}
\renewcommand{\dateseparator}{.}
\dmyyyydate
\usepackage{fancyhdr}
\usepackage{ifthen}
\pagestyle{fancy}
\fancyhf{}
\renewcommand{\headrulewidth}{0pt}
\fancyfoot[L]{\ifthenelse{\value{page}=1}{\today, \currenttime{} Uhr}{}}
\begin{document}
\begin{table}[ht]
\begin{minipage}[t]{0.5\linewidth}
\small
\begin{center}*D
\end{center}
\begin{tabular}{rl}
\textbf{80} & doch læse ich samfter süeze birn,\\ 
 & swie die ritter \textbf{vor im} \textbf{nider} rirn.\\ 
 & Der krîe dô vil maneger wielt,\\ 
 & swer \textbf{vor} sîner tjoste hielt:\\ 
5 & "hie kumt der anker, fîâ fî!"\\ 
 & \textbf{zegegen kom im} gehurtet bî\\ 
 & \begin{large}E\end{large}in vürste \textbf{von} Anschouwe\\ 
 & - diu riwe was sîn vrouwe -\\ 
 & mit ûf kêrter \textbf{schildes} spitze.\\ 
10 & daz lêrt in jâmers witze.\\ 
 & \textbf{diu} wâpen er \textbf{erkande}.\\ 
 & wâr umbe er von im wande?\\ 
 & welt ir, ich bescheide iuch des.\\ 
 & si gap der stolze Galoes,\\ 
15 & filiiroy Gandin,\\ 
 & der vil \textbf{getriwe} bruoder sîn,\\ 
 & dâ vor, \textbf{unz} im diu minne erwarp,\\ 
 & daz er an einer tjost \textbf{erstarp}.\\ 
 & \textbf{Dô bant er} ab sînen helm.\\ 
20 & weder\textbf{z} \textbf{gras} noch den melm\\ 
 & sîn strît dâ niht mêre bante.\\ 
 & grôz jâmer in des mante.\\ 
 & mit sîme sinne er bâgete,\\ 
 & \textbf{daz} er \textbf{niht} \textbf{dicke} \textbf{en}\textbf{vrâgete}\\ 
25 & Kayleten, sîner muomen sun,\\ 
 & waz sîn bruoder wolde tuon,\\ 
 & daz er niht turnierte hie.\\ 
 & \textbf{daz} en\textbf{wesse}r leider, wie\\ 
 & er starp vo\textit{r} Muntori.\\ 
30 & dâ \textbf{vor} was im ein kumber bî.\\ 
\end{tabular}
\scriptsize
\line(1,0){75} \newline
D \newline
\line(1,0){75} \newline
\textbf{3} \textit{Majuskel} D  \textbf{7} \textit{Initiale} D  \textbf{19} \textit{Majuskel} D  \newline
\line(1,0){75} \newline
\textbf{7} Anschouwe] Anscoͮwe D \textbf{29} vor] von D  $\cdot$ Muntori] Mvnthôri D \newline
\end{minipage}
\hspace{0.5cm}
\begin{minipage}[t]{0.5\linewidth}
\small
\begin{center}*m
\end{center}
\begin{tabular}{rl}
 & doch læse ich sanfter süeze birn,\\ 
 & wie die ritter \textbf{vor ime} \textbf{nider} \textit{ri}rn.\\ 
 & der krîge dô vil maniger wielt,\\ 
 & wer \textbf{vor} sîner juste hielt:\\ 
5 & "hie kumt der anker, fîâ fî!"\\ 
 & \textbf{zegegen kam im} gehurtet bî\\ 
 & ein vürste \textbf{ûz} \textit{An}sch\textit{ou}we\\ 
 & - diu riuwe was sîn vrouwe -\\ 
 & mit ûf gekêrter spitze.\\ 
10 & daz lêrt in jâmers witze.\\ 
 & \textbf{diu} wâpen er \textbf{erkante}.\\ 
 & wâr umbe er von im wante?\\ 
 & wellet ir, ich bescheide iuch des.\\ 
 & si gap der stolze Galoes,\\ 
15 & fili rois Gandin,\\ 
 & der vil \textbf{getriuwe} bruoder sîn,\\ 
 & dâ vor, \textbf{unz} ime diu minne erwarp,\\ 
 & daz e\textit{r} \textit{a}n einer just \textbf{erstarp}.\\ 
 & \textbf{dô bant er} abe sînen helm.\\ 
20 & w\textit{e}der\textbf{z} \textbf{gras} noch de\textit{n} melm\\ 
 & sîn strît dô niht mêre bante.\\ 
 & grôz jâmer in des mante.\\ 
 & mit sînem s\textit{i}nne er bâgete,\\ 
 & \textbf{daz} er \textbf{niht} \textbf{ê} \textbf{gevrâgete}\\ 
25 & Kaileten, sîne\textit{r} muomen sun,\\ 
 & waz sî\textit{n} bruoder wolte tuon,\\ 
 & daz er niht turnierte hie.\\ 
 & \textbf{daz} en\textbf{weiz} er l\textit{ei}der \textbf{reht}, wie\\ 
 & er starp vor Munnthori.\\ 
30 & dâ \textbf{vor} was im ein kumber bî.\\ 
\end{tabular}
\scriptsize
\line(1,0){75} \newline
m n o \newline
\line(1,0){75} \newline
\newline
\line(1,0){75} \newline
\textbf{2} rirn] farn m \textbf{3} maniger] manigen o \textbf{7} Anschouwe] [schwen]: schwe m enschowe o \textbf{8} riuwe was sîn] ruwessin n \textbf{9} mit] \textit{om.} n  $\cdot$ gekêrter spitze] gekerte spise o \textbf{10} lêrt] lerte n o \textbf{11} wâpen] knappen o \textbf{12} wante] rante n (o) \textbf{14} si] Des o \textbf{15} rois] rosz n \textbf{16} der] Die o \textbf{17} dâ] Jr o \textbf{18} er an] er [an]: Me an m \textbf{20} wederz] Wenders m Wider das n o  $\cdot$ den] dem m o  $\cdot$ melm] melnn o \textbf{21} bante] bant n o \textbf{22} grôz] Grosses n  $\cdot$ des] das o  $\cdot$ mante] mant n (o) \textbf{23} sinne] sunne m \textbf{24} ê] me n \textbf{25} Kaileten] Kailetten m Kaẏleten n o  $\cdot$ sîner] sinen \textit{nachträglich korrigiert zu:} siner m \textbf{26} sîn] siner \textit{nachträglich korrigiert zu:} sin m \textbf{28} daz] Des n o  $\cdot$ enweiz] weisz n (o)  $\cdot$ leider] lieder m \textbf{29} Munnthori] munnthorÿ m monthori n muͯntori o \textbf{30} ein] eẏ o \newline
\end{minipage}
\end{table}
\newpage
\begin{table}[ht]
\begin{minipage}[t]{0.5\linewidth}
\small
\begin{center}*G
\end{center}
\begin{tabular}{rl}
 & doch læse ich sanfter süeze biren,\\ 
 & swi\textit{e} \textit{d}ie rîter \textbf{vor im} riren.\\ 
 & der crîge dâ vil maniger wielt,\\ 
 & swer \textbf{ie gein} sîner tjoste hielt:\\ 
5 & "hie kumt der anker, phîâ phî!"\\ 
 & \textbf{gein im kom} gehurt bî\\ 
 & ein vürste \textbf{ûz} Anschouwe\\ 
 & - diu riwe was sîn vrouwe -\\ 
 & mit ûf gekêrter spitze.\\ 
10 & daz lêrt in jâmers witze.\\ 
 & \textbf{diu} wâpen er \textbf{erkande}.\\ 
 & wâr umber von im wande?\\ 
 & welt ir, ich bescheide iuch des.\\ 
 & si gap der stolze Galoes,\\ 
15 & fily rois Gandin,\\ 
 & der vil \textbf{liebe} bruoder sîn,\\ 
 & dâ vor, \textbf{ê} im diu minne erwarp,\\ 
 & daz er an einer tjoste \textbf{starp}.\\ 
 & \textbf{man bant im} abe sînen helm.\\ 
20 & weder \textbf{daz} \textbf{ors} noch den melm\\ 
 & sîn strît \textit{dâ} nimer bante.\\ 
 & grôz jâmer in des mante.\\ 
 & \begin{large}M\end{large}it sînem sinner bâgte,\\ 
 & \textbf{wâr umb}er \textbf{nine} \textbf{vrâgte}\\ 
25 & Kaileten, sîner muomen sun,\\ 
 & waz sîn bruoder wolte tuon,\\ 
 & daz er niht turnierte hie.\\ 
 & \textbf{dô}ne \textbf{wesse}r leider \textbf{reht}, wie\\ 
 & er starp vor Muntori.\\ 
30 & dâ \textbf{vor} was im ein kumber bî.\\ 
\end{tabular}
\scriptsize
\line(1,0){75} \newline
G I O L M Q R Z Fr50 \newline
\line(1,0){75} \newline
\textbf{1} \textit{Initiale} O  \textbf{3} \textit{Initiale} L Q R Z  \textbf{5} \textit{Initiale} I  \textbf{23} \textit{Initiale} G  \textbf{27} \textit{Initiale} I  \newline
\line(1,0){75} \newline
\textbf{1} doch] ÷och O  $\cdot$ læse] lêr I aͯse R  $\cdot$ ich] ir Z  $\cdot$ sanfter süeze biren] sansster suͯszen bri R \textbf{2} swie die] swie da die G Wie die L Q R Wy dy vor deme M  $\cdot$ vor im] vor in I nedir \sout{ir} M vor mir Q  $\cdot$ riren] nider riern O (Z) nit sind fry R \textbf{3} der] [de]: der I Den R  $\cdot$ crîge] groier I krigk Q  $\cdot$ dâ] do O L Q  $\cdot$ maniger] mannig M \textbf{4} swer] Wer L M Q R  $\cdot$ ie gein] hie gein I vor Z  $\cdot$ sîner tjoste] seiner trost Q sinen Iostieren R \textbf{5} kumt] kvnet Z \textbf{6} Engegen chom hie im gehvrtet hi O  $\cdot$ En gegin qwam om gehurtet by M (R) (Z) \textbf{7} ein] ern I  $\cdot$ Anschouwe] anschoͮwe G antschauwe I anschowe O Q R Anschowen L aschouwe M antschowe Z \textbf{8} riwe] trúwe R (Z) \textbf{9} ûf gekêrter] vfgekerten I of gerichter L vf kerter Z  $\cdot$ spitze] schiltes spitze L spiczen Q \textbf{10} lêrt] larte M \textbf{11} diu] Disiv O (L) (M) (Q) (Z)  $\cdot$ erkande] bechande I irkanten M kante Q (R) \textbf{12} wâr] Dar L (R)  $\cdot$ umber] her M  $\cdot$ wande] wantin M \textbf{13} ich] icht M  $\cdot$ des] das M [das]: des R \textbf{14} si] Sa M  $\cdot$ der stolze] deme stolczin M  $\cdot$ Galoes] Gaoles L \textbf{15} Gandin] chandin I gaudin Q (R) \textbf{16} liebe] getriwe O (L) (M) (Q) (R) (Z) \textbf{17} ê im] im I O (M) Q ým E L im E R e Z  $\cdot$ erwarp] im erwarp Z \textbf{18} starp] erstarp I (O) L Q (R) \textbf{20} Weder daz ors] Wederz gras O (M) (Q) (R) Z Weder grasz L  $\cdot$ den] der I \textbf{21} sîn] sinen I  $\cdot$ dâ nimer] nimer G da ninder I da niht L da nicht mer M (R) do nicht mer Q  $\cdot$ bante] bant I \textbf{22} des] daz L  $\cdot$ mante] mant I \textbf{23} sînem sinner] sinen sinnen er I (R)  $\cdot$ bâgte] beiagite M (Z) dachte R \textbf{24} wâr umber] Daz er L Dar vmbe er Z  $\cdot$ nine] niht dicher I O (L) (Q) nit R (Z) \textbf{25} Kaileten] Gahileten I Kayleten O M Q Kaýleten L Kaylitten R Gaileten Z \textbf{28} dône wesser] Da on wuste er M Wone wust er R Da enweste [lr]: er Z daz en welle er Fr50  $\cdot$ reht] \textit{om.} O L Fr50 nit R \textbf{29} er] Wie er L  $\cdot$ starp] stap Fr50  $\cdot$ vor] [von]: vor I von Q  $\cdot$ Muntori] muntorie I monthori O Q monchorẏ L Montori M manchori R mvnthori Z mvntehori Fr50 \textbf{30} vor] von Q \newline
\end{minipage}
\hspace{0.5cm}
\begin{minipage}[t]{0.5\linewidth}
\small
\begin{center}*T (U)
\end{center}
\begin{tabular}{rl}
 & doch læse ich sanfter süeze birn,\\ 
 & wie die rîter \textbf{\textit{ni}der} rirn.\\ 
 & \begin{large}D\end{large}er crîen dô vil maneger wielt,\\ 
 & wer \textbf{ie gein} sîner jost hielt:\\ 
5 & "hie kumet der anker, fîa fî!"\\ 
 & \textbf{gein im kam} gehurtet bî\\ 
 & ein vürste \textbf{ûz} Anschouwe\\ 
 & - die riuwe was sîn vrouwe -\\ 
 & mit ûf gekêrter spitze.\\ 
10 & daz lêrte in jâmers witze.\\ 
 & \textbf{disiu} wâpen er \textbf{bekante}.\\ 
 & wâr umb er von im wante?\\ 
 & wolt ir, ich bescheide iuch des.\\ 
 & si gap der stolze Galoes,\\ 
15 & fillyro\textit{y}s Gandin,\\ 
 & der vil \textbf{getriuwe} bruoder sîn,\\ 
 & dâ vor, \textbf{ê} im diu minne erwarp,\\ 
 & daz er an einer jost \textbf{starp}.\\ 
 & \textbf{man bant im} abe sînen helm.\\ 
20 & weder \textbf{gras} noch den melm\\ 
 & sîn strît dâ niemer bant.\\ 
 & grôz jâmer in des \textbf{nû} mant.\\ 
 & mit sîne\textit{m} sinne er bâgete,\\ 
 & \textbf{wâr umb} er \textbf{en}\textbf{vrâgete}\\ 
25 & Kayleten, sîner muomen sun,\\ 
 & waz sîn bruoder wolte tuon,\\ 
 & daz er niht turnierte hie.\\ 
 & \textbf{dô} en\textbf{weiz} er leider \textbf{reht}, wie\\ 
 & er starp vor Muntori.\\ 
30 & dô was im ein kumber bî.\\ 
\end{tabular}
\scriptsize
\line(1,0){75} \newline
U V W T \newline
\line(1,0){75} \newline
\textbf{1} \textit{Majuskel} T  \textbf{3} \textit{Initiale} U V W   $\cdot$ \textit{Majuskel} T  \textbf{6} \textit{Majuskel} T  \textbf{14} \textit{Majuskel} T  \textbf{15} \textit{Majuskel} T  \textbf{22} \textit{Majuskel} T  \textbf{30} \textit{Majuskel} T  \newline
\line(1,0){75} \newline
\textbf{1} læse] ließ W \textbf{2} Wie dicke ritterschafft von in riern W  $\cdot$ wie] [*]: swie V swie T  $\cdot$ nider] in der U [*]: von im nider V von im nider T \textbf{3} crîen dô vil] crîe do T \textbf{4} wer] swer V T \textbf{5} hie] Der sper hie W \textbf{6} kam] kam ainer W \textbf{7} vürste] fúrstin W  $\cdot$ ûz] von V W T  $\cdot$ Anschouwe] Anschowe U antschowe W anscôuwe T \textbf{9} gekêrter] [gekertem*]: gekerteme V gekertem W [geker*]: gekertem T  $\cdot$ spitze] [*]: schiltes spitze V schiltes spitze W \textbf{10} lêrte] lert V lertin T \textbf{11} disiu] dise T \textbf{13} gervͦchent irz ich sagîv ez T \textbf{14} Galoes] Gales U Galoez T \textbf{15} fillyroys] fillyros U  $\cdot$ Gandin] Gaudin U (W) \textbf{17} ê im] im e V im T \textbf{18} starp] erstarp V T \textbf{20} weder] wederz T  $\cdot$ den] \textit{om.} V der W \textbf{21} Seinen gedanck do nieman beuand W  $\cdot$ niemer] niht mer T \textbf{22} grôz jâmer] Sines iamers craft T  $\cdot$ des] \textit{om.} W  $\cdot$ nû mant] nv mante V mante T \textbf{23} sînem] siner U  $\cdot$ er bâgete] erwaget W \textbf{24} envrâgete] nie nv́t fragete V nicht enfraget W niht ê vragete T \textbf{25} Kayleten] Kaẏleten V Gayleten W \textbf{28} dô enweiz er] da enwuster V Do enwezzer W (T)  $\cdot$ reht] \textit{om.} V W T \textbf{29} Muntori] Muͦnthori U munthori V W Mvntorj T \textbf{30} dô was] da waz V Do vor was W (T)  $\cdot$ ein] \textit{om.} T \newline
\end{minipage}
\end{table}
\end{document}
