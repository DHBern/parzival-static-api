\documentclass[8pt,a4paper,notitlepage]{article}
\usepackage{fullpage}
\usepackage{ulem}
\usepackage{xltxtra}
\usepackage{datetime}
\renewcommand{\dateseparator}{.}
\dmyyyydate
\usepackage{fancyhdr}
\usepackage{ifthen}
\pagestyle{fancy}
\fancyhf{}
\renewcommand{\headrulewidth}{0pt}
\fancyfoot[L]{\ifthenelse{\value{page}=1}{\today, \currenttime{} Uhr}{}}
\begin{document}
\begin{table}[ht]
\begin{minipage}[t]{0.5\linewidth}
\small
\begin{center}*D
\end{center}
\begin{tabular}{rl}
\textbf{501} & \textit{\begin{large}D\end{large}}în œheim gap dir ouch ein swert,\\ 
 & dâ mit dû sünden bist gewert,\\ 
 & sît daz dîn wol \textbf{redender} munt\\ 
 & dâ leider niht tet vrâge kunt.\\ 
5 & die sünde lâ bî den andern stên!\\ 
 & wir ouch tâlanc \textbf{ruowen} gên!"\\ 
 & \textbf{wênec wart in bette und kulter} brâht;\\ 
 & si giengen êt ligen \textbf{ûf} ein bâht.\\ 
 & daz leger was \textbf{ir} hôhen art\\ 
10 & gelîche ninder dâ bewart.\\ 
 & sus was er dâ \textit{v}ünfzehen tage.\\ 
 & der wirt sîn pflac, als ich iu sage:\\ 
 & \textbf{krût} unde \textbf{würzelîn},\\ 
 & \textbf{daz} muose ir bestiu spîse sîn.\\ 
15 & Parzival \textbf{die} swære\\ 
 & truoc durch \textbf{süeze} mære,\\ 
 & wand in der wirt von sünden schiet\\ 
 & unt im doch rîterlîchen riet.\\ 
 & \textbf{Eines tages} \textbf{vrâgt in} Parzival:\\ 
20 & "wer was ein man, lac vorme Grâl?\\ 
 & der \textbf{was} al grâ bî liehtem vel."\\ 
 & der wirt sprach: "daz \textbf{was} Titurel.\\ 
 & der selbe \textbf{ist} dîner muoter an.\\ 
 & dem \textbf{wart} alrêst des Grâles van\\ 
25 & bevolhe\textit{n} durch \textbf{schermens} rât.\\ 
 & ein \textbf{siechtuom}, heizet pôgrât,\\ 
 & \textbf{treit er}, \textbf{die lem} helfelôs.\\ 
 & sîne varwe er \textbf{iedoch} nie verlôs,\\ 
 & wander den Grâl sô dicke siht;\\ 
30 & dâ von \textbf{er mac} ersterben niht.\\ 
\end{tabular}
\scriptsize
\line(1,0){75} \newline
D Fr11 \newline
\line(1,0){75} \newline
\textbf{1} \textit{Initiale} D  \textbf{19} \textit{Majuskel} D  \newline
\line(1,0){75} \newline
\textbf{1} Dîn] Min D \textbf{8} êt ligen ûf] ligen in Fr11 \textbf{9} hôhen] hoͯcherivͯ Fr11 \textbf{11} vünfzehen] [svs*]: svnfcehen D \textbf{15} Parzival] Parcifal D \textbf{19} Parzival] Parcifal D \textbf{22} Titurel] tyturel Fr11 \textbf{23} selbe] selber Fr11 \textbf{25} bevolhen] bevolhens D  $\cdot$ schermens] schermes Fr11 \textbf{27} die] divͯ Fr11 \textbf{28} iedoch] doch Fr11 \textbf{30} er mac] mag er Fr11 \newline
\end{minipage}
\hspace{0.5cm}
\begin{minipage}[t]{0.5\linewidth}
\small
\begin{center}*m
\end{center}
\begin{tabular}{rl}
 & dîn œheim gap dir ouch ein swert,\\ 
 & dâ mit dû sünden bist gewert,\\ 
 & sît daz dîn wol \textbf{geborner} munt\\ 
 & d\textit{â} leider niht tet vrâgen kunt.\\ 
5 & die sünde lâz bî den andern stên!\\ 
 & wir \textbf{sullen} ouch \textbf{dar umb} tâlanc \textbf{slâfen} gên!"\\ 
 & \textbf{bette oder kultern wart in wênic} brâht;\\ 
 & si giengen eht ligen \textbf{ûf} ein bâht.\\ 
 & daz leger was hôher art\\ 
10 & gelîch nindert d\textit{â} bewart.\\ 
 & sus was er d\textit{â} vünfzehen tage.\\ 
 & der wirt sîn pflac, als ich iu sage:\\ 
 & \textbf{krût} und \textbf{w\textit{ü}r\textit{z}elîn},\\ 
 & \textbf{daz} muose ir bestiu spîse sîn.\\ 
15 & Parcifal \textbf{der} swære\\ 
 & truoc durch \textbf{süeze} mære,\\ 
 & wan in der wirt von sünden schiet\\ 
 & und im doch ritterlîchen riet.\\ 
 & \textbf{\begin{large}E\end{large}ines tages} \textbf{sprach} Parcifal:\\ 
20 & "wer was ein man, lac vor dem Grâl?\\ 
 & der \textbf{was} al grâ bî liehtem vel."\\ 
 & der wirt sprach: "daz \textbf{was} Titurel.\\ 
 & der selbe \textbf{ist} dîner muoter ane.\\ 
 & dem \textbf{wart} allerêrst des Grâles vane\\ 
25 & bevolhen durch \textbf{schirmens} rât.\\ 
 & ein \textbf{siechtage} heizet pôgrât,\\ 
 & \textbf{die leme t\textit{r}e\textit{i}t der} helfelôs.\\ 
 & sîn varwe er \textbf{iedoch} nie verlôs,\\ 
 & wan er den Grâl sô dicke siht;\\ 
30 & dâ von \textbf{mac er} ersterben niht.\\ 
\end{tabular}
\scriptsize
\line(1,0){75} \newline
m n o \newline
\line(1,0){75} \newline
\textbf{19} \textit{Initiale} m n  \newline
\line(1,0){75} \newline
\textbf{3} geborner] redender n reddener o \textbf{4} dâ] Do m n o \textbf{6} dar umb] \textit{om.} n o  $\cdot$ tâlanc] talig m toling n (o) \textbf{9} hôher] ir hoher n (o) \textbf{10} dâ] do m n o \textbf{11} dâ] do m n o \textbf{13} würzelîn] wirttelin m \textbf{19} sprach] frogete in n (o) \textbf{22} Titurel] tútturel n \textbf{25} bevolhen] Beholffen o  $\cdot$ schirmens] [schirme*]: schirmens m [schrmes]: schirmes n schẏrmes o \textbf{27} treit] tet m \textbf{28} nie] nit o \textbf{29} Grâl] grole n \textbf{30} mac] so mag n \newline
\end{minipage}
\end{table}
\newpage
\begin{table}[ht]
\begin{minipage}[t]{0.5\linewidth}
\small
\begin{center}*G
\end{center}
\begin{tabular}{rl}
 & \begin{large}D\end{large}în œheim gap dir ouch ein swert,\\ 
 & dâ mit dû sünden bist gewert,\\ 
 & sît daz dîn wol \textbf{redender} munt\\ 
 & dâ leider niht tet vrâgen kunt.\\ 
5 & die sünde lâ bî den andern stên!\\ 
 & wir \textit{\textbf{suln}} ouch tâlanc \textbf{ruowen} gên!"\\ 
 & \textbf{wênic wart in bette unde kulter} brâht;\\ 
 & si giengen êt ligen \textbf{in} ein bâht.\\ 
 & daz leger was \textbf{ir} hôhen art\\ 
10 & gelîche niender dâ bewart.\\ 
 & sus was er dâ vünfzehen tage.\\ 
 & der wirt sîn pflac, als ich iu sage:\\ 
 & \textbf{krût} unde \textbf{würzelîn},\\ 
 & \textbf{daz} muose ir beste spîse sîn.\\ 
15 & Parzival \textbf{di\textit{e}} swære\\ 
 & truoc durch \textbf{süeziu} mære,\\ 
 & wan in der wirt von sünden schiet\\ 
 & unde im doch rîterlîchen riet.\\ 
 & \textbf{eines tages} \textbf{vrâget in} Parzival:\\ 
20 & "wer was ein man, \textbf{der} lac vorme Grâl?\\ 
 & der \textbf{was} al grâ bî liehtem vel."\\ 
 & der wirt sprach: "daz \textbf{was} Titurel.\\ 
 & der selbe \textbf{was} dîner muoter ane.\\ 
 & dem \textbf{wart} alrêrst des Grâles vane\\ 
25 & bevolhen durch \textbf{schermes} rât.\\ 
 & ein \textbf{siechtuom}, heizet pôgrât,\\ 
 & \textbf{treit er}, \textbf{die lem} helfelôs.\\ 
 & sîne varwe er \textbf{iedoch} nie verlôs,\\ 
 & wan er den Grâl sô dicke siht;\\ 
30 & dâ von \textbf{mag er} ersterben niht.\\ 
\end{tabular}
\scriptsize
\line(1,0){75} \newline
G I L M Z \newline
\line(1,0){75} \newline
\textbf{1} \textit{Initiale} G I L Z  \textbf{15} \textit{Initiale} I M  \newline
\line(1,0){75} \newline
\textbf{4} vrâgen] vrage I L \textbf{6} suln] \textit{om.} G  $\cdot$ tâlanc] talangen G \textbf{7} wênic] Wen ich L  $\cdot$ in] im Z  $\cdot$ bette unde kulter] beiden I bete vur M \textbf{8} si giengen êt ligen] noch gulter si leiten sich I Sie giengen uort ligen M  $\cdot$ in] vf Z \textbf{9} leger] ligen I  $\cdot$ hôhen] hoher M \textbf{10} niender] nirgen M \textbf{12} als] alle M \textbf{14} beste] beder I \textbf{15} Parzival] Parzifal I L M [D*]: Parcifal Z  $\cdot$ die] din G \textbf{19} vrâget] vragite M  $\cdot$ Parzival] parziual G Parzifal I (L) (M) parcifal Z \textbf{20} der] \textit{om.} L M Z \textbf{21} al] \textit{om.} I  $\cdot$ liehtem] lichtem L (M) \textbf{22} Titurel] Týtuͯrel L tyturel M (Z) \textbf{23} was] ist L M \textbf{24} dem] Der L  $\cdot$ wart alrêrst] wartet allez I \textbf{25} wan er wart im beuolhen durc shermes rat I  $\cdot$ schermes] schirmens M (Z) \textbf{27} helfelôs] heflos I \textbf{28} iedoch] doch I L \textbf{29} siht] sihte G \textbf{30} er] \textit{om.} M \newline
\end{minipage}
\hspace{0.5cm}
\begin{minipage}[t]{0.5\linewidth}
\small
\begin{center}*T
\end{center}
\begin{tabular}{rl}
 & Dîn œheim gap dir ouch ein swert,\\ 
 & dâ mit dû sünden bist gewert,\\ 
 & sît daz dîn wol \textbf{redender} munt\\ 
 & dâ leider niht tet vrâge kunt.\\ 
5 & die sünde lâ bî den andern stên!\\ 
 & wir \textbf{suln} ouch tâlanc \textbf{ruowen} gên!"\\ 
 & \textbf{wênec wart in kultern unde bette} brâht;\\ 
 & si giengen eht ligen \textbf{in} ein bâht.\\ 
 & daz leger was \textbf{ir} hôher art\\ 
10 & glîche niender dâ bewart.\\ 
 & Sus was er dâ vünfzehen tage.\\ 
 & der wirt sîn pflac, als ich iu sage:\\ 
 & \textbf{würze} unde \textbf{krütelîn}\\ 
 & muose ir best\textit{iu} spîse sîn.\\ 
15 & Parcifal \textbf{die} swære\\ 
 & truoc durch \textbf{liebiu} mære,\\ 
 & wand in der wirt von sünden schiet\\ 
 & unde im doch rîterlîchen riet.\\ 
 & \textbf{Aber} \textbf{sprach} \textbf{dô} Parcifal:\\ 
20 & "wer was ein man, \textbf{der} lac vorme Grâl?"\\ 
 & \hspace*{-.7em}\big| Der wirt sprach: "daz \textbf{ist} Tyturel,\\ 
 & \hspace*{-.7em}\big| der \textbf{ist} al grâ bî liehtem vel.\\ 
 & der selbe \textbf{ist} dîner muoter ane.\\ 
 & dem \textbf{was} alrêrst des Grâles vane\\ 
25 & bevolhen durch \textbf{schirmes} rât.\\ 
 & ein \textbf{siechtage}, heizet pôgrât,\\ 
 & \textbf{treit er}, \textbf{die leme} helfelôs.\\ 
 & sîne varwe er \textbf{doch} nie verlôs,\\ 
 & wander den Grâl sô dicke siht;\\ 
30 & dâ von \textbf{mag er} ersterben niht.\\ 
\end{tabular}
\scriptsize
\line(1,0){75} \newline
T U V W O Q R Fr39 \newline
\line(1,0){75} \newline
\textbf{1} \textit{Initiale} O   $\cdot$ \textit{Majuskel} T  \textbf{7} \textit{Initiale} Fr39  \textbf{11} \textit{Majuskel} T  \textbf{15} \textit{Initiale} W  \textbf{19} \textit{Majuskel} T  \textbf{22} \textit{Majuskel} T  \newline
\line(1,0){75} \newline
\textbf{1} \textit{Die Verse 453.1-502.30 fehlen} U  \textbf{4} dâ] Do W Q R  $\cdot$ vrâge] fragen O R \textbf{5} lâ] lan Q \textbf{6} Wir welt talanc ruͦwe han R  $\cdot$ ouch tâlanc ruowen] talen ruͦgen W auch rueven Q \textbf{7} wênec] Donoch Fr39  $\cdot$ kultern] kvtern V kulter W (O) (R) (Fr39) kelter Q \textbf{8} eht] auch Q lecht R  $\cdot$ in] vf V (W) (O) (Q) (R) Fr39  $\cdot$ ein] einen W (Fr39) \textbf{9} hôher] hohen V W O [hoher]: hohen Q \textbf{10} niender] niergent V  $\cdot$ dâ] do V W Fr39 \textit{om.} O \textbf{11} Sus] Als Q  $\cdot$ dâ] do V W O Q \textit{om.} Fr39  $\cdot$ vünfzehen] funtzehen Q (Fr39) \textbf{13} würze] [Wur*]: Wurzelen V Wurczen R  $\cdot$ unde] vnd auch W \textbf{14} muose] mvese T  $\cdot$ bestiu] beste T R \textbf{15} Parcifal] Parzifal V PArtzifal W (Q) Parczifal R  $\cdot$ die] div O Fr39 der R \textbf{16} Durch libe durch mere Q  $\cdot$ liebiu] liebe R \textbf{18} im] \textit{om.} O \textbf{19} dô] da R  $\cdot$ Parcifal] parzifal V partzifal W Q parczifal R \textbf{20} der] \textit{om.} V W O Q R \textbf{22} \textit{Versfolge 501.21-22} V W O Q R Fr39   $\cdot$ ist] waz V (W) (O) (Q) (R) (Fr39)  $\cdot$ Tyturel] tirurel Q \textbf{21} ist] waz V (W) (O) (Q) (R) (Fr39)  $\cdot$ liehtem] lichtem Q liechtten R \textbf{23} muoter] muͦmen R \textbf{24} dem was] [*]: Dem wart V Dem ward W (O) (Q) (R) (Fr39) \textbf{25} rât] art Q \textbf{26} siechtage] siechtuͦm W (O) (Q) (R) (Fr39)  $\cdot$ heizet] der heisset V  $\cdot$ pôgrât] prograt V podagrat Q \textbf{28} sîne varwe] Sein frawe Q \textbf{29} sô] \textit{om.} R  $\cdot$ dicke] \textit{om.} W \textbf{30} mag er] so mag er V \newline
\end{minipage}
\end{table}
\end{document}
