\documentclass[8pt,a4paper,notitlepage]{article}
\usepackage{fullpage}
\usepackage{ulem}
\usepackage{xltxtra}
\usepackage{datetime}
\renewcommand{\dateseparator}{.}
\dmyyyydate
\usepackage{fancyhdr}
\usepackage{ifthen}
\pagestyle{fancy}
\fancyhf{}
\renewcommand{\headrulewidth}{0pt}
\fancyfoot[L]{\ifthenelse{\value{page}=1}{\today, \currenttime{} Uhr}{}}
\begin{document}
\begin{table}[ht]
\begin{minipage}[t]{0.5\linewidth}
\small
\begin{center}*D
\end{center}
\begin{tabular}{rl}
\textbf{485} & \begin{large}S\end{large}i bêde wâren \textbf{mit} herzen klage.\\ 
 & dô nâhtez dem mittem tage.\\ 
 & der wirt sprach: "gê wir nâch der nar,\\ 
 & dîn ors \textbf{ist} unberâten gar;\\ 
5 & ich mac uns selben niht gespîsen,\\ 
 & \textbf{es} \textbf{en}welle uns got \textbf{bewîsen}.\\ 
 & mîn küche riuchet selten;\\ 
 & des muostû hiute engelten\\ 
 & unt al die wîle dû bî mir bist.\\ 
10 & ich \textbf{solde} dich hiute lêren list\\ 
 & an den würzen, lieze uns der snê.\\ 
 & got gebe, daz \textbf{der} schiere zergê.\\ 
 & nû brechen die wîle \textbf{îwîn graz}.\\ 
 & ich wæne, dîn ors dicke \textbf{geaz}\\ 
15 & ze Munsalvæsche \textbf{baz} denne hie.\\ 
 & dû noch ez ze wirte nie\\ 
 & kômet, der iwer \textbf{gerner} pflæge,\\ 
 & ob ez hie \textbf{bereitez læge}."\\ 
 & si giengen ûz umb ir bejac.\\ 
20 & Parzival des vuoters pflac,\\ 
 & der wirt gruop \textbf{im} würzelîn;\\ 
 & daz muose ir beste spîse sîn.\\ 
 & der wirt sîner orden niht vergaz:\\ 
 & swie vil er gruop, decheine er az\\ 
25 & der würze vor der nône.\\ 
 & an die stûden schône\\ 
 & hienc ers unt suochte mêre.\\ 
 & durch die gotes êre\\ 
 & manegen tac \textbf{ungâz er} gienc,\\ 
30 & sô er vermiste, \textbf{dâ} sîn spîse hienc.\\ 
\end{tabular}
\scriptsize
\line(1,0){75} \newline
D \newline
\line(1,0){75} \newline
\textbf{1} \textit{Initiale} D  \newline
\line(1,0){75} \newline
\textbf{15} Munsalvæsche] Mvnsælvæsche D \textbf{20} Parzival] Parcifal D \newline
\end{minipage}
\hspace{0.5cm}
\begin{minipage}[t]{0.5\linewidth}
\small
\begin{center}*m
\end{center}
\begin{tabular}{rl}
 & \begin{large}S\end{large}i beide wâren \textbf{in} herzeklage.\\ 
 & dô nâhet ez dem mitten tage.\\ 
 & der wirt sprach: "gân wir nâch der \textit{n}ar,\\ 
 & dîn ros \textbf{ist} unberâten gar;\\ 
5 & ich mac uns selben niht gespîsen,\\ 
 & \textbf{es} welle uns got \textbf{bewîsen}.\\ 
 & mîn k\textit{ü}chen riuchet selten;\\ 
 & des muost\textit{û} hiute engelten\\ 
 & und alle die wîle d\textit{û} bî mir bist.\\ 
10 & ich \textbf{solt} dich hiute lêren list\\ 
 & an den würzen, liez uns der snê.\\ 
 & got gebe, daz \textbf{er} schier zergê.\\ 
 & nû brechen die wîle \textbf{îwîn graz}.\\ 
 & ich wæne, dîn ros dicke \textbf{geaz}\\ 
15 & zuo Muntsalvasche \textbf{baz} dan hie.\\ 
 & dû noch ez zuo wirte nie\\ 
 & \textbf{ê} kômet, der iuwer \textbf{gerne} pflæge,\\ 
 & ob ez hie \textbf{bereitez læge}."\\ 
 & si giengen ûz umb ir bejac.\\ 
20 & Parcifal des vuo\textit{t}e\textit{r}s pflac,\\ 
 & der wirt gruop \textbf{der} würzelîn;\\ 
 & daz muost ir bestiu spîse sîn.\\ 
 & der wirt sîner orden niht vergaz:\\ 
 & wie vil er gruop, dekein er az\\ 
25 & der wurz vor der nône.\\ 
 & an die st\textit{û}den schône\\ 
 & hienc er si und suohte mêre.\\ 
 & durch die gotes êre\\ 
 & manigen tac \textbf{\textit{er} ungeezzen} gienc,\\ 
30 & sô er vermi\textit{s}te, \textbf{wâ} sîn spîse hienc.\\ 
\end{tabular}
\scriptsize
\line(1,0){75} \newline
m n o \newline
\line(1,0){75} \newline
\textbf{1} \textit{Illustration mit Überschrift:} Also der wirt vnd parcifal mit einander Lange ze (zit n o  ) retten (rehten o  ) vmb den gral m (n) (o)   $\cdot$ \textit{Initiale} m n o  \newline
\line(1,0){75} \newline
\textbf{2} nâhet] nohete n  $\cdot$ mitten] mittem n \textbf{3} nar] var m \textbf{4} unberâten] vnbereiten o \textbf{5} selben] selb n \textbf{6} bewîsen] gewisen o \textbf{7} küchen] kuschen m o \textbf{8} des] Das o  $\cdot$ muostû] muͯste m (o) \textbf{9} dû] do m \textbf{11} würzen] wurczeln o \textbf{12} \textit{Vers 485.12 fehlt} n  \textbf{13} îwîn] win o \textbf{15} Muntsalvasche] muntsaluasce m muntsaluasc n munt saluasce o \textbf{16} dû] Do o  $\cdot$ zuo] zir o \textbf{17} ê] \textit{om.} n o \textbf{20} Parcifal] Parcifals n  $\cdot$ vuoters] fures m furtes o \textbf{21} der würzelîn] do wurtzelin n (o) \textbf{23} sîner] \sout{sprach} siner o \textbf{24} dekein] do keine n die kein o \textbf{26} stûden] stunden m \textbf{29} er] \textit{om.} m \textbf{30} vermiste] vermischtte m \newline
\end{minipage}
\end{table}
\newpage
\begin{table}[ht]
\begin{minipage}[t]{0.5\linewidth}
\small
\begin{center}*G
\end{center}
\begin{tabular}{rl}
 & \begin{large}S\end{large}i bêde wâre\textit{n} \textbf{mit} herzen klage.\\ 
 & dô nâhet ez dem mittem tage.\\ 
 & der wirt sprach: "gê wir nâch der nar,\\ 
 & dîn ors \textbf{stêt} unberâten gar;\\ 
5 & ich\textbf{ne} mac uns sel\textit{b}en niht gespîsen,\\ 
 & \textbf{ez}\textbf{ne} welle uns got \textbf{wîsen}.\\ 
 & mîn küche riuchet selten;\\ 
 & des muostû hiute engelten\\ 
 & unt al die wîle dû bî mir bist.\\ 
10 & ich \textbf{solde} dich hiute lêren list\\ 
 & an den würzen, lieze uns der snê.\\ 
 & got gebe, daz \textbf{der} schier zergê.\\ 
 & nû brechen die wîle \textbf{wîngraz}.\\ 
 & ich wæne, dîn ors dicke \textbf{geaz}\\ 
15 & ze Muntsalvatsche \textbf{baz} danne hie.\\ 
 & dû noch ez ze wirte nie\\ 
 & kômet, der iuwer \textbf{gerner} pflæge,\\ 
 & ob ez hie \textbf{bereitez læge}."\\ 
 & si giengen ûz umbe ir bejac.\\ 
20 & Parzival des vuoters pflac,\\ 
 & der wirt gruop \textbf{i\textit{m}} würzelîn;\\ 
 & daz muose ir bestiu spîse sîn.\\ 
 & der wirt sîner orden niht vergaz:\\ 
 & swie vil er gruop, dehein er az\\ 
25 & der würze vor der nône.\\ 
 & an die stûden schône\\ 
 & hienc ers unde suohte mêre.\\ 
 & durch die gotes êre\\ 
 & manigen tac \textbf{er ungâz} gienc,\\ 
30 & sô er vermiste, \textbf{wâ} sîn spîse hienc.\\ 
\end{tabular}
\scriptsize
\line(1,0){75} \newline
G I O L M Z \newline
\line(1,0){75} \newline
\textbf{1} \textit{Initiale} G O L M Z  \textbf{3} \textit{Initiale} I  \textbf{13} \textit{Initiale} I  \newline
\line(1,0){75} \newline
\textbf{1} Si] ÷i O  $\cdot$ wâren] ware G  $\cdot$ herzen] herze O (M) \textbf{2} dô] Da M  $\cdot$ nâhet] nahent I O \textbf{3} der wirt sprach] Do sprach er I  $\cdot$ der] de O \textbf{4} stêt] ist O L M Z \textbf{5} ichne] ich I (O) (Z)  $\cdot$ selben] [sei]: selhin G \textit{om.} O selber L \textbf{6} ezne] Es O  $\cdot$ uns got] got vns sin I  $\cdot$ wîsen] bewisen O L M Z \textbf{7} küche] chuchel I \textbf{9} wîle dû] duͯ wil L \textbf{10} solde] sol I O L  $\cdot$ hiute] \textit{om.} Z \textbf{11} würzen] witzen L worze M \textbf{12} der] er I O \textbf{13} Nv brechen wir di wile gras O  $\cdot$ wîle wîngraz] wil ein wenic gras I wile nuͯwen graz L wile ywin gras M ewerm ross gras Z \textbf{15} ze Muntsalvatsche] zemvntsalvatsche G (O) zemuntshaluasche I Zuͯ mvntsalvatsche L (M) Zv montsalvatsche Z \textbf{16} noch ez] mehtist I \textbf{17} kômet] komen I Kome L  $\cdot$ iuwer] din I L  $\cdot$ gerner] gerne M  $\cdot$ pflæge] wol phlege I \textbf{18} bereitez] herczitel M  $\cdot$ læge] [ware]: lage G \textbf{20} Parzival] Parzifal I L M Barcifal O Parcifal Z \textbf{21} im] in G \textbf{22} Daz muese ir bestiv spise sin G  $\cdot$ Daz ir beste spise solte sin L \textbf{23} sîner] sines L \textbf{24} swie] Wie L M  $\cdot$ gruop] grube M \textbf{25} der würze] die wurzen I \textbf{26} stûden] stunden M  $\cdot$ schône] schone \sout{hie} Z \textbf{27} hienc] Hienge G \textbf{29} er ungâz] vngaz er O Z vngaszen er L (M) \textbf{30} wâ] da O M Z \newline
\end{minipage}
\hspace{0.5cm}
\begin{minipage}[t]{0.5\linewidth}
\small
\begin{center}*T
\end{center}
\begin{tabular}{rl}
 & \begin{large}S\end{large}i beide wâren \textbf{mit} herzeklage.\\ 
 & dô nâhetez dem mitten tage.\\ 
 & der wirt sprach: "gên wir nâch der nar,\\ 
 & dîn ors \textbf{ist} unberâten gar;\\ 
5 & i\textbf{ne} mac uns selbe niht gespîsen,\\ 
 & \textbf{ez} \textbf{en}welle uns got \textbf{bewîsen}.\\ 
 & mîn küchen riuchet selten;\\ 
 & des muostû hiute engelten\\ 
 & unde aldie wîle dû bî mir bist.\\ 
10 & ich \textbf{sol} dich hiute lêren \textbf{einen} list\\ 
 & an den würzen, lieze uns der snê.\\ 
 & got gebe, daz \textbf{der} schiere zergê.\\ 
 & nû brechen die wîl \textbf{îwîn gra\textit{z}}.\\ 
 & ich wæne, dîn ors dicke \textbf{baz}\\ 
15 & \textbf{geaz} ze Munsalvasche danne hie.\\ 
 & dû noch ez ze wirte nie\\ 
 & kômet, der iuwer \textbf{gerner} \textbf{wol} pflæge,\\ 
 & ob ez hie \textbf{bereitet wære}."\\ 
 & Si giengen ûz umbir bejac.\\ 
20 & Parcifal des vuoters pflac,\\ 
 & der wirt gruop \textbf{in} würzelîn;\\ 
 & daz muose ir best\textit{iu} spîse sîn.\\ 
 & Der wirt sîner orden niht vergaz:\\ 
 & swie vil er \textbf{würze} gruop, deheine er az\\ 
25 & der würze vor der nône.\\ 
 & an die stûden schône\\ 
 & hienc er si unde suochte mêre.\\ 
 & durch die gotes êre\\ 
 & manegen tac \textbf{ungâz er} gienc,\\ 
30 & sô er vermissete, \textbf{wâ} sîn spîse hienc.\\ 
\end{tabular}
\scriptsize
\line(1,0){75} \newline
T U V W Q R Fr40 \newline
\line(1,0){75} \newline
\textbf{1} \textit{Initiale} T V W Q   $\cdot$ \textit{Capitulumzeichen} R  \textbf{19} \textit{Majuskel} T  \textbf{23} \textit{Initiale} W   $\cdot$ \textit{Majuskel} T  \newline
\line(1,0){75} \newline
\textbf{1} \textit{Die Verse 453.1-502.30 fehlen} U   $\cdot$ Si beide] Dú beidu R  $\cdot$ herzeklage] herzen clage V (R) herter klage W hertzen clagen Q \textbf{2} dô] Da V  $\cdot$ nâhetez] nahet ez V (W) (Q) (R)  $\cdot$ mitten tage] [mitten]: mittem tage V mittē tagen Q mittē tage R \textbf{3} wirt sprach] sprach der wirt Q \textbf{5} Jch kan vnd mag mich sellen nicht gespissen R  $\cdot$ selbe] selber W selben Q \textbf{6} uns got] got vns den R  $\cdot$ bewîsen] [geweysen]: beweysen Q \textbf{10} sol] solte W  $\cdot$ einen] \textit{om.} V W Q den R \textbf{11} an] In W  $\cdot$ lieze] liez V (W) litz Q \textbf{12} gebe] geb vnß W welle R  $\cdot$ der schiere] er bald R  $\cdot$ zergê] zu gen Q \textbf{13} brechen] brech wir Q  $\cdot$ die wîl îwîn] die weile úch ein W úwerm ros R  $\cdot$ graz] gras T V W R \textbf{14} dîn ors] es R  $\cdot$ baz] gas W (Fr40) grasz Q bas gas R \textbf{15} geaz] \textit{om.} W Q R Fr40  $\cdot$ ze Munsalvasche] zemvlsalvasce T ze [munts*]: muntsalvasche V Zuͦ montsaluatschs W Zu [muntsalbsche]: muntsalvsche Q :::emunsalvashe Fr40  $\cdot$ danne hie] bas dann hie W (R) (Fr40) han bas do [hi*]: hie  Q \textbf{16} dû] Do W \textbf{17} komen der úwern gerner woͯllte pflegen R  $\cdot$ gerner wol] [ger* *l]: gerner V gerner W gerne Q  $\cdot$ pflæge] pflegen Q \textbf{18} bereitet wære] bereites lege V (W) (Fr40) bereite lege Q bereittes were gelegen R \textbf{20} Parcifal] Parzifal V (Fr40) Partzifal W Q Parczifal R  $\cdot$ vuoters] [furste]: furstes Q furtes Fr40 \textbf{21} in] im V \textit{om.} W ein Q \textbf{22} ir bestiu] irbeste T (R) \textbf{23} sîner orden] sins ordens V (W) R seiner order Q \textbf{24} swie] Wie V W Q R  $\cdot$ vil] wol R  $\cdot$ würze] \textit{om.} V W Q R Fr40 \textbf{26} die stûden] die [stv*den]: stvden T die [stv*]: stvden V den stunden R \textbf{27} er si] ers auff W \textbf{28} die] de T \textbf{29} ungâz] vngessen Q \textbf{30} sô] Do Q (Fr40)  $\cdot$ vermissete] [vermisset]: vermissete T vermischte R \newline
\end{minipage}
\end{table}
\end{document}
