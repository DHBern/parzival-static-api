\documentclass[8pt,a4paper,notitlepage]{article}
\usepackage{fullpage}
\usepackage{ulem}
\usepackage{xltxtra}
\usepackage{datetime}
\renewcommand{\dateseparator}{.}
\dmyyyydate
\usepackage{fancyhdr}
\usepackage{ifthen}
\pagestyle{fancy}
\fancyhf{}
\renewcommand{\headrulewidth}{0pt}
\fancyfoot[L]{\ifthenelse{\value{page}=1}{\today, \currenttime{} Uhr}{}}
\begin{document}
\begin{table}[ht]
\begin{minipage}[t]{0.5\linewidth}
\small
\begin{center}*D
\end{center}
\begin{tabular}{rl}
\textbf{101} & teilte Gahmuretes hant\\ 
 & unt \textbf{ouch} swaz er dâ vürsten vant.\\ 
 & \multicolumn{1}{l}{ - - - }\\ 
 & \multicolumn{1}{l}{ - - - }\\ 
5 & lât \textbf{si} rîten, \textbf{swer} dâ geste sîn!\\ 
 & den gap urloup der Anschevin.\\ 
 & \textbf{Daz} pantel, daz sîn vater truoc,\\ 
 & von zobele ûf \textbf{sînen schilt} man sluoc.\\ 
 & al kleine, wîz, sîdîn\\ 
10 & ein hemde der künegîn,\\ 
 & als ez ruorte ir \textbf{blôzer} lîp,\\ 
 & diu nû \textbf{worden was} sîn wîp,\\ 
 & daz was sînes halsberges dach.\\ 
 & ahzeheniu man durchstochen sach\\ 
15 & unt mit swerten gar \textbf{zerhouwen},\\ 
 & ê er \textbf{schied} von \textbf{der} vrouwen.\\ 
 & daz leit \textbf{ouch} si an blôze hût,\\ 
 & sô \textbf{von ritterschaft kom} ir trût,\\ 
 & der manegen schilt vil \textbf{dürkel stach}.\\ 
20 & ir zweier minne \textbf{man} triwen jach.\\ 
 & \begin{large}E\end{large}r hete werdecheit genuoc,\\ 
 & dô in sîn manlîch ellen truoc\\ 
 & hin über gein der herte.\\ 
 & mich jâmert sîner verte.\\ 
25 & im kom \textbf{diu} wâre botschaft,\\ 
 & sîn hêrre, der bâruc, \textbf{wære} mit kraft\\ 
 & überriten von \textbf{dem} Babylon.\\ 
 & einer hiez Ipomidon,\\ 
 & der ander \textbf{Pompeius}.\\ 
30 & den nennet diu âventiure \textbf{alsus}.\\ 
\end{tabular}
\scriptsize
\line(1,0){75} \newline
D \newline
\line(1,0){75} \newline
\textbf{7} \textit{Majuskel} D  \textbf{21} \textit{Initiale} D  \newline
\line(1,0){75} \newline
\textbf{1} Gahmuretes] Gahmvretes D \textbf{3} \textit{Die Verse 101.3-4 fehlen} D  \textbf{6} Anschevin] Anscevin D \textbf{28} Ipomidon] Jpomydon D \newline
\end{minipage}
\hspace{0.5cm}
\begin{minipage}[t]{0.5\linewidth}
\small
\begin{center}*m
\end{center}
\begin{tabular}{rl}
 & teilte Gahmuretes hant\\ 
 & und \textit{\textbf{ouch} waz} er d\textit{â} vürsten vant.\\ 
 & \multicolumn{1}{l}{ - - - }\\ 
 & \multicolumn{1}{l}{ - - - }\\ 
5 & \begin{large}L\end{large}ât rîten, \textbf{wer} dâ geste sîn!\\ 
 & den gap urloup der A\textit{n}schevin.\\ 
 & \textbf{daz} pantel, daz sîn \textit{vater} truoc,\\ 
 & von zobele ûf \textbf{sînem schilte} man sluoc.\\ 
 & al klein, wîz, sîdîn\\ 
10 & ein hemede der künigîn,\\ 
 & als ez ruorte ir \textbf{blôzen} lîp,\\ 
 & diu nû \textbf{was worden} sîn wîp,\\ 
 & daz was sînes halsberges dach.\\ 
 & ahzeheniu man durchstochen sach\\ 
15 & und mit swerten gar \textbf{zerhouwen},\\ 
 & ê \textbf{daz} er \textbf{scheide} von \textbf{sîner} vrouwen.\\ 
 & daz le\textit{i}te \textbf{ouch} si an blôze hût,\\ 
 & sô \textbf{kam von ritterschaft} ir trût,\\ 
 & der manigen schilt vil \textbf{durchstach}.\\ 
20 & ir zweier minne triuwen jach.\\ 
 & \begin{large}E\end{large}r hete wirdicheit genuoc,\\ 
 & dô in sîn manlîch ellen truoc\\ 
 & hin über gegen der herte.\\ 
 & mich jâmert sîner verte.\\ 
25 & im ko\textit{m} \textbf{ê} \textbf{diu} wâre botschaft,\\ 
 & sîn hêrre, der bâruc, \textbf{wære} mit kraft\\ 
 & überriten von \textbf{den} Babilon.\\ 
 & einer hiez Ypom\textit{e}don,\\ 
 & der ander \textbf{Pompeius}.\\ 
30 & den nennet diu âventiure \textbf{alsus}.\\ 
\end{tabular}
\scriptsize
\line(1,0){75} \newline
m n o \newline
\line(1,0){75} \newline
\textbf{5} \textit{Initiale} m  \textbf{21} \textit{Initiale} m n o  \newline
\line(1,0){75} \newline
\textbf{1} teilte] Teilet n (o)  $\cdot$ Gahmuretes] gahmurettes m gamiretes n gamuͯres o \textbf{2} ouch waz] was och m  $\cdot$ dâ] do m n o \textbf{3} \textit{Die Verse 101.3-4 fehlen} m n o  \textbf{5} dâ] do n o \textbf{6} Anschevin] ausceuin m anscevin n ansce win o \textbf{7} vater] \textit{om.} m \textbf{8} zobele] zebel o  $\cdot$ ûf] auͯch o  $\cdot$ sînem schilte] sin schilt n (o) \textbf{9} al] Alle n  $\cdot$ wîz] wisse n \textbf{12} wîp] [lip]: wip o \textbf{13} sînes] by n \textbf{14} ahzeheniu] Aczehen o  $\cdot$ sach] [was]: sach o \textbf{16} scheide] schiede n o  $\cdot$ sîner] der n o \textbf{17} leite] lette m leit n o  $\cdot$ blôze] bluͯsse o \textbf{19} vil] do vil n \textbf{22} manlîch] manig o \textbf{23} herte] huͯrte o \textbf{24} sîner verte] sin geferte n o \textbf{25} kom] kome m  $\cdot$ ê] \textit{om.} n o \textbf{26} bâruc] banug o \textbf{28} Ypomedon] ÿppomadon m ẏpomidon n ẏpomidens o \textbf{30} den nennet] Die noment n o \newline
\end{minipage}
\end{table}
\newpage
\begin{table}[ht]
\begin{minipage}[t]{0.5\linewidth}
\small
\begin{center}*G
\end{center}
\begin{tabular}{rl}
 & teilte Gahmuretes hant\\ 
 & unde swaz er dâ vürsten vant.\\ 
 & dâ wart daz varnde volc vil geil.\\ 
 & \textbf{die} enpfiengen rîcher gâbe teil.\\ 
5 & \begin{large}L\end{large}ât rîten, \textbf{die} dâ geste sîn!\\ 
 & den gap urloup der Antschevin.\\ 
 & \textbf{daz} pantel, daz sîn vater truoc,\\ 
 & von zobele ûf \textbf{sînen schilt} man\textbf{z} sluoc.\\ 
 & al kleine, wîz, sîdîn\\ 
10 & ein hemde der künigîn,\\ 
 & als ez ruo\textit{r}te ir \textbf{blôzen} lîp,\\ 
 & diu nû \textbf{worden was} sîn wîp,\\ 
 & daz was sînes halsberges dach.\\ 
 & ahzeheniu man \textbf{er} durchstochen sach\\ 
15 & unde mit swerten gar \textbf{zerhouwen},\\ 
 & ê er \textbf{schiede} von \textbf{der} vrouwen.\\ 
 & daz leite si an \textbf{ir} blôze hût,\\ 
 & sô \textbf{kom von rîte\textit{r}schaft} ir trût,\\ 
 & der manigen schilt vil \textbf{dürkel stach}.\\ 
20 & ir zweier minne triwen jach.\\ 
 & er hete werdicheit genuoc,\\ 
 & dô in sîn manlîch ellen truoc\\ 
 & hin über gein der herte.\\ 
 & mich jâmert sîner verte.\\ 
25 & im kom \textbf{ein} wâriu botschaft,\\ 
 & sîn hêrre, der bâruc, \textbf{wære} mit kraft\\ 
 & überriten von Babilon.\\ 
 & einer hiez Ipomidon,\\ 
 & der ander \textbf{Ponpeirus}.\\ 
30 & den nennet diu âventiure \textbf{sus}.\\ 
\end{tabular}
\scriptsize
\line(1,0){75} \newline
G I O L M Q R Z Fr21 Fr36 Fr48 \newline
\line(1,0){75} \newline
\textbf{1} \textit{Initiale} O  \textbf{5} \textit{Initiale} G I L  \textbf{21} \textit{Überschrift:} Wie gamuret fvr vber mer vnd da starp Z   $\cdot$ \textit{Initiale} L R Z Fr21 Fr48  \textbf{25} \textit{Initiale} I   $\cdot$ \textit{Capitulumzeichen} L  \newline
\line(1,0){75} \newline
\textbf{1} teilte] Geilte M Teẏlt Fr48  $\cdot$ Gahmuretes] Gamvretes O (Q) (Z) Gahmuͯretes L gamuretes M Ghamathamurs R Gachmuretes Fr48 \textbf{2} unde] vnd auch I (O) (Q) (R) (Z) (Fr48)  $\cdot$ swaz] waz L (M) (Q) (R)  $\cdot$ dâ] do Q \textbf{3} dâ] Do Q  $\cdot$ vil] \textit{om.} I L Z \textbf{4} die] Si O (L) (M) (Q) (R) (Z) Fr21 (Fr36) Fr48  $\cdot$ rîcher] Riche R  $\cdot$ gâbe] habe Q \textbf{5} Lât] Lat si I (M) (Q) (R) (Z) (Fr21) Fr36 (Fr48) Lat nv O  $\cdot$ die dâ] die O do die Q da die R (Z)  $\cdot$ geste] gesten R  $\cdot$ sîn] synt M \textbf{6} der] \textit{om.} O R  $\cdot$ Antschevin] anschevin G Z antsheuin I anshevin O (L) Fr48 anscevin M anshevein Q aschwevin R :::evin Fr21 :::in Fr36 \textbf{8} ûf sînen schilt manz] vf sinen schilt man I (L) (M) (Q) (Fr48) man vf sinen schilt O vnd sinen schilt man R vf sine schilt man Z man im vf sinen schilt Fr21 mans auf sin schilt Fr36 \textbf{9} al kleine wîz] ein cleinwis I Alle clein [v*]: wis R \textbf{10} der] daz diu I \textbf{11} als] al da I  $\cdot$ ez] irs M er Q  $\cdot$ ruorte] roͮte G rurt I (Fr21) Fr48  $\cdot$ ir] [sine]: iren Z \textbf{12} diu] Den Fr48 \textbf{13} halsberges] halspers R halsperge Fr21 \textbf{14} ahzeheniu] Al zechnú R  $\cdot$ man er] man ir I man O (M) (Q) R Z Fr21 Fr48 stuͯnt man in L \textbf{15} unde mit] Von L  $\cdot$ gar] [d]: gar R  $\cdot$ zerhouwen] durch hawen O (Q) (R) (Fr21) \textbf{16} ê] \textit{om.} O M Q Z Fr21  $\cdot$ schiede] schied I (O) R (Z) (Fr21) schilde Q \textbf{17} daz leite si] si leit ez ie I Das leit sie M Das legt er Q Das legttes R (Fr21) Daz legt ouch sie Z (Fr48)  $\cdot$ ir] \textit{om.} I Z Fr48  $\cdot$ blôze] blossen R \textbf{18} kom von rîterschaft] chom von riteschaft G ie von dem strit chom I von ritterschefte kom L (Q) \textbf{19} vil] \textit{om.} I L Q  $\cdot$ dürkel] durch Q dunkel R \textbf{20} triwen] trurens I truͯwe L (Q) (R) \textbf{21} hete] had M \textbf{22} dô in] Doch in L Da en M (Z) (Fr48) o\textit{m. } Q  $\cdot$ manlîch] melich M  $\cdot$ ellen] ere Q \textbf{23} über] vnder Q \textbf{25} ein] div O (L) (M) (Q) (R) (Z) Fr21 (Fr48) \textbf{26} der] \textit{om.} Fr21  $\cdot$ bâruc] brauck Q \textbf{27} überriten] Vbir geretin M  $\cdot$ von] von dem Q von den R Z Fr48  $\cdot$ Babilon] ypomidone L babylon Z \textbf{28} Dem stoltzen Babilone L  $\cdot$ hiez] heizet Fr21  $\cdot$ Ipomidon] ypomedon I ypomidon O M Q Jhpomido R Ihpomidon Fr21 jhpomidon Fr48 \textbf{29} der ander] Dez bruder hisz L  $\cdot$ Ponpeirus] ponpeius I Pompeirus O (Z) Pompeiuͯs L pomperus M pompeius R \textbf{30} sus] alsus I \newline
\end{minipage}
\hspace{0.5cm}
\begin{minipage}[t]{0.5\linewidth}
\small
\begin{center}*T (U)
\end{center}
\begin{tabular}{rl}
 & teilte Gahmuretes hant\\ 
 & und waz er dâ vürsten vant.\\ 
 & d\textit{â} wart daz varnde volc vil geil.\\ 
 & \textbf{si} entviengen rîcher gâbe teil.\\ 
5 & lât \textbf{si} rîten, \textbf{die} dâ geste sîn!\\ 
 & den gap urloup der Anschevin.\\ 
 & \textbf{ein} pantel, daz sîn vater truoc,\\ 
 & von zobele ûf \textbf{sînen schilt} man sluoc.\\ 
 & a\textit{l} kleine, wîz, sîdîn\\ 
10 & ein hemede der künegîn,\\ 
 & als ez ruorte ir \textbf{blôzen} lîp,\\ 
 & diu nû \textbf{worden was} sîn wîp,\\ 
 & daz was sînes halsberges dach.\\ 
 & ahzehen man durchstochen sach\\ 
15 & und mit swerten gar \textbf{verhouwen},\\ 
 & ê er \textbf{scheide} von \textbf{der} vrouwen.\\ 
 & daz leit si an \textbf{ir} blôze hût,\\ 
 & sô \textbf{kom von ritterschaft} ir trût,\\ 
 & der manegen schilt vil \textbf{dürkel stach}.\\ 
20 & ir zw\textit{e}ier minne triuwe jach.\\ 
 & \begin{large}E\end{large}r hete wirdecheit genuoc,\\ 
 & dô in sîn manlîch ellen truoc\\ 
 & hin über gein der herte.\\ 
 & mich jâmert sîner verte.\\ 
25 & im \textit{kom} \textbf{diu} wâre botschaft,\\ 
 & sîn hêrre, der bâruc, mit kraft\\ 
 & \textbf{wære} überriten von \textbf{den von} Babilon.\\ 
 & einer hiez Ihpomidon,\\ 
 & der ander \textbf{Pompeius}.\\ 
30 & den nemmet diu âventiure \textbf{sus}.\\ 
\end{tabular}
\scriptsize
\line(1,0){75} \newline
U V W T \newline
\line(1,0){75} \newline
\textbf{3} \textit{Majuskel} T  \textbf{5} \textit{Majuskel} T  \textbf{7} \textit{Majuskel} T  \textbf{9} \textit{Majuskel} T  \textbf{21} \textit{Initiale} U V W T  \textbf{28} \textit{Majuskel} T  \textbf{29} \textit{Majuskel} T  \newline
\line(1,0){75} \newline
\textbf{1} Gahmuretes] Gahmuͦretes U Gamuretes V (W) \textbf{2} waz] swaz V (T)  $\cdot$ dâ] do W \textbf{3} dâ] Do U V W  $\cdot$ volc] vock W \textbf{4} rîcher gâbe] groß gab ein michel W \textbf{5} lât] Nun lant W  $\cdot$ si] \textit{om.} T  $\cdot$ dâ] \textit{om.} W \textbf{6} Anschevin] Anscheuin V antscheuin W anscevin T \textbf{7} ein] Das W [*]: Dc T \textbf{8} ûf sînen schilt man] man auff sein schilt W \textbf{9} Al] Alle U Als W \textbf{12} nû] neúwe W  $\cdot$ worden was] was worden W (T) \textbf{14} ahzehen man] [*h*h]: ahzehen stunt mans V Achtzehen man er W \textbf{15} verhouwen] zerhowen V (W) (T) \textbf{16} ê] e daz V  $\cdot$ scheide] schiede V (T) \textbf{17} leit si] leite si V (W) leiter T \textbf{18} sô] Wann W \textbf{19} vil dürkel stach] vil tunkel stach V durchstach W dvrkel stach T \textbf{20} zweier] zwerir U  $\cdot$ minne] minnen W  $\cdot$ triuwe] truwen V (W) (T) \textbf{22} in] sy W \textbf{25} kom] \sout{lam} U \textbf{26} mit] were mit V (W) T \textbf{27} wære] \textit{om.} V W T  $\cdot$ den von] [d*]: den V \textit{om.} W  $\cdot$ Babilon] babylon U babilone W \textbf{28} einer] [De*s*ol*]: Einer V  $\cdot$ Ihpomidon] Jpomidon U [*bylone]: ypomidon V ypomidone W Jhpomidon T \textbf{29} Pompeius] [*]: ponpeus V pompeyus W \textbf{30} nemmet] nemet U  $\cdot$ sus] alsus W \newline
\end{minipage}
\end{table}
\end{document}
