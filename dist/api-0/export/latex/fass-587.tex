\documentclass[8pt,a4paper,notitlepage]{article}
\usepackage{fullpage}
\usepackage{ulem}
\usepackage{xltxtra}
\usepackage{datetime}
\renewcommand{\dateseparator}{.}
\dmyyyydate
\usepackage{fancyhdr}
\usepackage{ifthen}
\pagestyle{fancy}
\fancyhf{}
\renewcommand{\headrulewidth}{0pt}
\fancyfoot[L]{\ifthenelse{\value{page}=1}{\today, \currenttime{} Uhr}{}}
\begin{document}
\begin{table}[ht]
\begin{minipage}[t]{0.5\linewidth}
\small
\begin{center}*D
\end{center}
\begin{tabular}{rl}
\textbf{587} & sine müesen dienst gein iu tragen.\\ 
 & \textbf{nû welt ir} prîs an \textbf{im} bejagen?\\ 
 & ir \textbf{soltet} kraft gein \textbf{kreften} geben\\ 
 & und liezet Gawanen leben\\ 
5 & siech mit sînen wunden\\ 
 & unt \textbf{twinget} die gesunden.\\ 
 & Maneger hât von \textbf{minnen sanc},\\ 
 & den \textbf{nie} diu minne \textbf{alsô} \textbf{getwanc}.\\ 
 & Ich m\textit{ö}hte nû wol stille dagen,\\ 
10 & \textbf{ez solten minnære} klagen,\\ 
 & waz dem von Norwæge was,\\ 
 & dô er der âventiwer genas,\\ 
 & daz in bestuont der minnen schûr\\ 
 & âne helfe gar ze sûr.\\ 
15 & \textbf{Er sprach}: "\textbf{owê}, daz ich \textbf{ie} erkôs\\ 
 & \textbf{disiu} bette \textbf{ruowelôs}!\\ 
 & \textbf{einez} hât mich versêret\\ 
 & \textbf{unt} daz ander mir gemêret\\ 
 & \textbf{gedanke} nâch minne.\\ 
20 & Orgeluse, diu herzoginne,\\ 
 & muoz genâde an mir begên,\\ 
 & ob ich \textbf{bî} \textbf{vreuden} sol bestên."\\ 
 & \textbf{Vor} ungedult er sich \textbf{sô} want,\\ 
 & daz brast etslîch sîn \textbf{wunden bant}.\\ 
25 & in solhem ungemache er lac.\\ 
 & nû seht, dô schein ûf in der tac;\\ 
 & des het er \textbf{unsanfte} erbiten.\\ 
 & er hete \textbf{dâ} \textbf{vor} \textbf{dicke} erliten\\ 
 & mit swerten manegen \textbf{scharpfen} strît\\ 
30 & sanfter danne \textbf{die} \textbf{ruowens} zît.\\ 
\end{tabular}
\scriptsize
\line(1,0){75} \newline
D \newline
\line(1,0){75} \newline
\textbf{7} \textit{Majuskel} D  \textbf{9} \textit{Majuskel} D  \textbf{15} \textit{Majuskel} D  \textbf{23} \textit{Majuskel} D  \newline
\line(1,0){75} \newline
\textbf{9} möhte] mohte D \newline
\end{minipage}
\hspace{0.5cm}
\begin{minipage}[t]{0.5\linewidth}
\small
\begin{center}*m
\end{center}
\begin{tabular}{rl}
 & si \dag müeste\dag  dienst gegen iu tragen.\\ 
 & \textbf{nû wolt ir} prîs an \textbf{im} bejagen?\\ 
 & ir \textbf{sullet} kraft gegen \textbf{krefte} geben\\ 
 & und liezet Gawan leben\\ 
5 & siech mit sînen wunden\\ 
 & und \textbf{tw\textit{i}ngen} die gesunden.\\ 
 & maniger het von \textbf{einem gesanc},\\ 
 & den \textbf{nie} diu minne \textbf{alsô} \textbf{betwanc}.\\ 
 & ich m\textit{ö}hte nû wol stille dagen,\\ 
10 & \textbf{ez solten minner} klagen,\\ 
 & waz dem von \textit{N}orwæge was,\\ 
 & dô er der âventiur genas,\\ 
 & daz in bestuont der minne\textit{n} \textit{s}chûr\\ 
 & âne helf gar zuo \textit{s}ûr.\\ 
15 & \textbf{er sprach}: "\textbf{owê}, daz ich erkôs\\ 
 & \textbf{diz} bette \textbf{ruowelôs}!\\ 
 & \textbf{einez} het mich versêret,\\ 
 & daz ander mir gemêret\\ 
 & \textbf{gedanc} nâch minne.\\ 
20 & Urgeluse, diu herzoginne,\\ 
 & muoz gnâde an mir begân,\\ 
 & ob ich \textbf{bî} \textbf{gnâden} sol bestân."\\ 
 & \textbf{vor} ungedult er sich want,\\ 
 & daz brast etlîch sîn \textbf{wuntgebant}.\\ 
25 & in solichem ungemach er lac.\\ 
 & nû seht, dô sch\textit{ei}n ûf in der tac;\\ 
 & des heter \textbf{unsanft} erbiten.\\ 
 & er hât \textbf{dô} \textbf{vil} \textbf{dicke} erliten\\ 
 & mit swerten manigen \textbf{herten} strît\\ 
30 & sanfter dan \textbf{die} \textbf{ruowens} zît.\\ 
\end{tabular}
\scriptsize
\line(1,0){75} \newline
m n o \newline
\line(1,0){75} \newline
\newline
\line(1,0){75} \newline
\textbf{1} iu] \textit{om.} o \textbf{2} ir] in o  $\cdot$ im] in o \textbf{3} krefte] crefften n o \textbf{4} Gawan] gawanen n o \textbf{6} twingen] twungen m twuͯngen o \textbf{7} einem] mẏnnen n enẏnngen o  $\cdot$ gesanc] sang n o \textbf{8} betwanc] getwang n o \textbf{9} möhte] mohte m (o) \textbf{11} Norwæge] orwege m norwege n [norge]: norwege o \textbf{13} minnen schûr] mynnen las schuͯr m \textbf{14} sûr] fruͯr m \textbf{17} het] hette o \textbf{21} muoz] Mus m (o) \textbf{23} want] do want n o \textbf{24} wuntgebant] wundehant n wunde bant o \textbf{26} schein] schin m \textbf{27} des] Das o \textbf{28} hât] hette n  $\cdot$ vil] vor n o \textbf{29} herten] scharppfen n (o) \newline
\end{minipage}
\end{table}
\newpage
\begin{table}[ht]
\begin{minipage}[t]{0.5\linewidth}
\small
\begin{center}*G
\end{center}
\begin{tabular}{rl}
 & sine m\textit{üe}sen dienst gein i\textit{u} tragen.\\ 
 & \textbf{welt ir nû} prîs an \textbf{im} bejagen?\\ 
 & ir \textbf{m\textit{ö}ht} kraft gein \textbf{krefte} geben\\ 
 & unde liezet Gawanen leben\\ 
5 & siech mit sînen wunden\\ 
 & unde \textbf{wundet} die gesunden.\\ 
 & maniger hât von \textbf{minnen sanc},\\ 
 & den \textbf{doch} diu minne \textbf{nie} \textbf{bedwanc}.\\ 
 & ich m\textit{ö}hte nû wol stille dagen\\ 
10 & \textbf{unde liez mîn} klagen,\\ 
 & waz dem von Norwæge was,\\ 
 & dô er der âventiure genas,\\ 
 & daz in bestuont der minne schûr\\ 
 & âne helfe gar ze sûr.\\ 
15 & \textbf{dô sprach er}: "\textbf{wê}, daz ich erkôs\\ 
 & \textbf{dis\textit{iu}} bette \textbf{riuwelôs}!\\ 
 & \textbf{daz eine} hât mich versêret,\\ 
 & daz ander mir gemêret\\ 
 & \textbf{gedanke} nâch minne.\\ 
20 & Orgeluse, diu herzoginne,\\ 
 & muoz genâde an mir begên,\\ 
 & ob ich \textbf{bî} \textbf{vröuden} sol bestên."\\ 
 & \textbf{von} ungedult er sich want,\\ 
 & daz brast eteslîch sîn \textbf{wunden ba\textit{n}t}.\\ 
25 & in solhem ungemache er lac.\\ 
 & nû sehet, dô schein ûf in der tac;\\ 
 & des het er \textbf{nû} \textbf{samfte} erbiten.\\ 
 & er het \textbf{ouch} \textbf{dâ} \textbf{vor} erliten\\ 
 & mit swerten manigen \textbf{herten} strît,\\ 
30 & \textbf{doch} senfte\textit{r} denne \textbf{diu} \textbf{trûrens} zît.\\ 
\end{tabular}
\scriptsize
\line(1,0){75} \newline
G I L M Z Fr19 Fr23 \newline
\line(1,0){75} \newline
\textbf{5} \textit{Initiale} I L Z Fr19 Fr23  \textbf{23} \textit{Initiale} I  \newline
\line(1,0){75} \newline
\textbf{1} sine müesen] Sine muͦsin G (Fr19) Sý muͯsen L Sie musten M (Fr23) Sie enmusten Z  $\cdot$ gein] \textit{om.} Fr23  $\cdot$ iu] in G \textbf{2} im] in L Fr23 \textbf{3} möht] moht G (I) (L) (M) Z Fr19 Fr23  $\cdot$ krefte] kreftin M (Z) (Fr19)  $\cdot$ geben] wegen L \textbf{4} liezet] liezzen Z  $\cdot$ Gawanen] Gauwanen I \textbf{5} siech] Mich M Siechen Fr23  $\cdot$ wunden] von den M \textbf{6} wundet] twuͤnget Z  $\cdot$ die] in I \textbf{7} minnen] mýnne L (Z) libe M \textbf{8} Den nie die minne svs getwanc Z  $\cdot$ minne] libe M  $\cdot$ nie] nie suͯsz L sus M \textbf{9} ich möhte] Ich mohte G (I) (M) (Z) (Fr23) Jr mochtet L \textbf{10} Ez solten minnere clagen Z  $\cdot$ mîn] minne I \textbf{11} von] \textit{om.} I  $\cdot$ Norwæge] norwage G norwengen I Norwege L M (Fr23) norwe Z \textbf{12} dô] Da M Z  $\cdot$ der] \textit{om.} I M \textbf{13} minne] minnen I libe M \textbf{14} âne] ân alle I  $\cdot$ sûr] fvr Z \textbf{15} dô sprach er] Da sprach her M Ersprach Z  $\cdot$ wê] \textit{om.} Z \textbf{16} disiu bette] dise bete G dise rede I Dise bette Fr23 \textbf{17} mich] mir L \textbf{18} daz] Vnd daz Z  $\cdot$ mir] mich M (Fr23)  $\cdot$ gemêret] vorkerit M \textbf{19} gedanke] Gedane Fr23  $\cdot$ minne] libe M \textbf{20} Orgeluse] Orguluse I Orgelýse L Orgelose M Orgillus Fr23 \textbf{22} bestên] gesten I \textbf{23} ungedult] vngewalte Z  $\cdot$ want] do want L so want M (Z) \textbf{24} daz] daz da I  $\cdot$ eteslîch sîn] ieglicher I  $\cdot$ bant] bast G \textbf{25} er lac] erbelac M \textbf{26} dô] nu I \textit{om.} L isz M da Z \textbf{27} nû samfte] vnsanifte L (M) (Z) \textbf{28} ouch] \textit{om.} Z  $\cdot$ erliten] dicke erliten Z \textbf{30} Doh sanft er dise ruͦ::: Fr23  $\cdot$ doch] noch I (M) (Z)  $\cdot$ senfter] senfte er G  $\cdot$ diu trûrens] diu dorrens I dise ruͯwens L (M) die ruwens Z \newline
\end{minipage}
\hspace{0.5cm}
\begin{minipage}[t]{0.5\linewidth}
\small
\begin{center}*T
\end{center}
\begin{tabular}{rl}
 & si müesten dienst gên iu tragen.\\ 
 & \textbf{wolt ir nû} prîs an \textbf{in} bejagen?\\ 
 & ir \textbf{m\textit{öh}t} kraft gên \textbf{kreften} geben\\ 
 & und liezet Gawanen leben\\ 
5 & \textit{s}iech mit sînen wunden\\ 
 & und \textbf{wundet} die gesunden.\\ 
 & maneger hât von \textbf{minnen sanc},\\ 
 & den \textbf{doch} diu minne \textbf{nie} \textbf{betwanc}.\\ 
 & ich m\textit{ö}hte nû wol stille dagen\\ 
10 & \textbf{und liez mî\textit{n}e} klagen,\\ 
 & waz dem von Norwæge was,\\ 
 & d\textit{ô} er der âventiur genas,\\ 
 & daz in bestuont der minne schûr\\ 
 & âne hilfe gar zuo sûr.\\ 
15 & \textbf{dô sprach er}: "\textbf{wê}, daz ich erkôs\\ 
 & \textbf{disiu} bette \textbf{ruowelôs}!\\ 
 & \textbf{daz ein} hât mich versêret,\\ 
 & daz ander mir gemêret\\ 
 & \textbf{gedanke} nâch minne.\\ 
20 & Orgeluse, diu herzoginne,\\ 
 & muoz gnâde an mir begên,\\ 
 & ob ich \textbf{mit} \textbf{vreuden} sol bestên."\\ 
 & \textbf{von} ungedult er sich \textbf{sô} want,\\ 
 & daz brast etslîch sîn \textbf{wunden bant}.\\ 
25 & in solchem ungemache er lac.\\ 
 & nû seht, dô schein ûf in der tac;\\ 
 & des het er \textbf{unsanfte} erbiten.\\ 
 & er het \textbf{ouch} \textbf{vor} erliten\\ 
 & mit swerten manegen \textbf{herten} strît,\\ 
30 & \textbf{doch} senfter danne \textbf{dise} \textbf{ruowens} zît.\\ 
\end{tabular}
\scriptsize
\line(1,0){75} \newline
Q R W V U \newline
\line(1,0){75} \newline
\textbf{5} \textit{Capitulumzeichen} R  \textbf{7} \textit{Initiale} W V  \newline
\line(1,0){75} \newline
\textbf{1} \textit{Die Verse 553.1-599.30 fehlen} U   $\cdot$ müesten] enmvͤsten V \textbf{2} in] Jm R [*]: in V  $\cdot$ bejagen] eriagen R \textbf{3} möht] mogt Q  $\cdot$ kreften] Jm krefftt R \textbf{4} Gawanen] Gawin R gawaneu W \textbf{5} siech] Mich Q \textbf{6} wundet] twingent V \textbf{7} von minnen] von minne R gesvngen der minen V \textbf{8} diu] \textit{om.} W  $\cdot$ nie] nie so W nie svs V \textbf{9} möhte] mochte Q (V) \textbf{10} [E*]: Vnde liesse min clagen V  $\cdot$ mîne] minne Q (W)  $\cdot$ klagen] clage R \textbf{11} dem] den W  $\cdot$ Norwæge] norwele Q norwegen R (V) norwege W \textbf{12} dô] Dor Q \textbf{13} minne] minnen V \textbf{15} [D*]: Er sprach [*kos]: owe daz ich erkos V  $\cdot$ wê] \textit{om.} R  $\cdot$ erkôs] verkos W \textbf{16} disiu] Dise R \textbf{19} nâch] nach der V \textbf{20} Orgeluse] Orgelusze Q Orgulus R \textbf{21} muoz] Die mvͦz V \textbf{22} mit] by R (W) (V) \textbf{23} von] Vor V  $\cdot$ sô] do V \textbf{24} daz] Do V  $\cdot$ brast] brach R  $\cdot$ sîn] seine W \textbf{26} \textit{Die Verse 588.5-6 sind am Rand nachgetragen und später radiert:} Daz tuoͧt im lihte alse we / Alse in minne kvmber e V   $\cdot$ schein] schied R  $\cdot$ in] \textit{om.} R \textbf{28} ouch] och da R (V) (W) \textbf{30} danne] won R  $\cdot$ dise] \textit{om.} W \newline
\end{minipage}
\end{table}
\end{document}
