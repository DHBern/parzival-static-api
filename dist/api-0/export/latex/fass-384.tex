\documentclass[8pt,a4paper,notitlepage]{article}
\usepackage{fullpage}
\usepackage{ulem}
\usepackage{xltxtra}
\usepackage{datetime}
\renewcommand{\dateseparator}{.}
\dmyyyydate
\usepackage{fancyhdr}
\usepackage{ifthen}
\pagestyle{fancy}
\fancyhf{}
\renewcommand{\headrulewidth}{0pt}
\fancyfoot[L]{\ifthenelse{\value{page}=1}{\today, \currenttime{} Uhr}{}}
\begin{document}
\begin{table}[ht]
\begin{minipage}[t]{0.5\linewidth}
\small
\begin{center}*D
\end{center}
\begin{tabular}{rl}
\textbf{384} & \textit{\begin{large}M\end{large}}elyanze \textbf{er} helfe sich bewac.\\ 
 & \textbf{der erwarb} ouch \textbf{im} von Semblidac\\ 
 & zwelf knappen, die sîn nâmen war\\ 
 & \textbf{an der} tjoste unt \textbf{an} der poynder schar.\\ 
5 & swaz sper gebieten moht ir hant,\\ 
 & diu wurden gar von im verswant.\\ 
 & sîne tjoste wâren \textbf{mit} hurte \textbf{hel},\\ 
 & wander den künec Schirniel\\ 
 & unt sînen bruoder dâ vienc.\\ 
10 & dennoch \textbf{dâ mêr von im} ergienc:\\ 
 & sicherheit er niht erliez\\ 
 & den herzogen Marangliez.\\ 
 & die wâren des ortes herte.\\ 
 & ir volc sich dennoch werte.\\ 
15 & Meljanz, der künec, \textbf{dâ} selbe streit.\\ 
 & swem er lieb \textbf{unt} herzeleit\\ 
 & hete getân, \textbf{die muosen} jehen,\\ 
 & daz selten \textbf{mêre wære} geschehen\\ 
 & von decheinem alsô jungen man,\\ 
20 & alsez \textbf{dâ von im wart} getân.\\ 
 & sîn hant vil vester schilde kloup.\\ 
 & waz starker sper \textbf{vor} im \textbf{zerstoup},\\ 
 & dâ sich poynder in \textbf{den} poynder slôz.\\ 
 & sîn jungez herze was sô grôz,\\ 
25 & daz er strîtes muose gern.\\ 
 & des \textbf{en}moht in niemen dâ gewern\\ 
 & volleclîche, daz was ein nôt,\\ 
 & unz er Gawane tjustieren bôt.\\ 
 & \textbf{\begin{large}G\end{large}awan ze sîme knappen} nam\\ 
30 & der \textbf{zwelf} sper einez von Angram,\\ 
\end{tabular}
\scriptsize
\line(1,0){75} \newline
D \newline
\line(1,0){75} \newline
\textbf{1} \textit{Initiale} D  \textbf{29} \textit{Initiale} D  \newline
\line(1,0){75} \newline
\textbf{1} Melyanze] ÷elyanze D \textbf{2} Semblidac] Semblidach D \textbf{8} Schirniel] Scirniel D \textbf{12} Marangliez] Maranglîez D \textbf{15} Meljanz] Melianz D \newline
\end{minipage}
\hspace{0.5cm}
\begin{minipage}[t]{0.5\linewidth}
\small
\begin{center}*m
\end{center}
\begin{tabular}{rl}
 & Mel\textit{i}anz \textbf{er} helfe sich bewac.\\ 
 & \textbf{der erwarp} ouch \textbf{ime} von Se\textit{m}blidac\\ 
 & zwelf knappen, die sîn nâmen war\\ 
 & \textbf{an der} juste und \textbf{an} der poinder schar.\\ 
5 & waz sper gebieten moht\textit{e} ir hant,\\ 
 & diu wurden gar von ime verswant.\\ 
 & sîne juste wâren \textbf{von} h\textit{u}rt\textit{e} \textbf{hel},\\ 
 & want er den künic Schirniel\\ 
 & und sînen bruoder d\textit{â} vienc.\\ 
10 & dannoch \textbf{d\textit{â} mêr von ime} ergienc:\\ 
 & sicherheit er niht erliez\\ 
 & den herzogen Marangliez.\\ 
 & die wâren des ortes herte.\\ 
 & ir volc sich dannoch werte.\\ 
15 & Mel\textit{i}anz, der künic, \textbf{d\textit{â}} selbe streit.\\ 
 & wem er liep \textbf{oder} herzeleit\\ 
 & hete getân, \textbf{die muosen} jehen,\\ 
 & daz selten \textbf{mêre wære} geschehen\\ 
 & von dekeinem als jungen man,\\ 
20 & als \textit{ez} \textbf{dâ von ime wart} getân.\\ 
 & sîn hant vil vester schilte kloup.\\ 
 & waz starker sper \textbf{von} ime \textbf{stoup},\\ 
 & d\textit{â} sich poinder in poinder slôz.\\ 
 & sîn jungez herze was sô grôz,\\ 
25 & daz er strîtes muose gern.\\ 
 & des \textbf{en}m\textit{o}ht in \textit{n}ieman d\textit{â} gewern\\ 
 & volleclîche, daz was ein nôt,\\ 
 & unz er Gawane justieren bôt.\\ 
 & \textbf{Gawan ze sînen knappen} nam\\ 
30 & der \textbf{zwelf} sper einez von Agram,\\ 
\end{tabular}
\scriptsize
\line(1,0){75} \newline
m n o \newline
\line(1,0){75} \newline
\newline
\line(1,0){75} \newline
\textbf{1} Melianz] Melancz m Meliantz n Meleancz o  $\cdot$ er] ir n o  $\cdot$ sich bewac] mich beiaiag o \textbf{2} Semblidac] senblidag m semblidig n semblidag o \textbf{4} der juste] den juste o \textbf{5} mohte] mohtten m moͯchte n \textbf{6} wurden] wúrdent n \textbf{7} hurte] herten m herte o \textbf{8} Schirniel] scirniel m surmel n sẏrmel o \textbf{9} dâ] do m n wol do o \textbf{10} dâ] do m n o  $\cdot$ von] noch n o \textbf{12} [Ajezanlies]: herczogen Ajezanlies o  $\cdot$ Marangliez] maranglies m mezangliesz n \textbf{13} die] \textit{om.} n o  $\cdot$ ortes] orten o \textbf{14} volc] folge o \textbf{15} Melianz] Meleancz m o Meliantz n  $\cdot$ dâ] do m \textit{om.} n o  $\cdot$ selbe] selbes o \textbf{17} hete] Hetten n  $\cdot$ die] den o  $\cdot$ muosen] muͯssen m muͯsten n \textbf{18} geschehen] beschehen o \textbf{20} ez] \textit{om.} m er o  $\cdot$ dâ] do n o \textbf{21} kloup] houp n \textbf{22} stoup] zercloup n (o) \textbf{23} dâ] Do m n o  $\cdot$ in] in den n o \textbf{24} sô] also o \textbf{25} muose] muͯsse m muͯste n (o) \textbf{26} enmoht] enmoͯht m moͯchte n mochte o  $\cdot$ nieman] ieman m  $\cdot$ dâ] do m n o \textbf{27} volleclîche daz was] Ffelleclich dasz wacz o \textbf{28} Gawane] gawan n o  $\cdot$ bôt] gebot n \textbf{30} einez] \textit{om.} n \newline
\end{minipage}
\end{table}
\newpage
\begin{table}[ht]
\begin{minipage}[t]{0.5\linewidth}
\small
\begin{center}*G
\end{center}
\begin{tabular}{rl}
 & \begin{large}M\end{large}elianz \textbf{der} helfe sich bewac.\\ 
 & \textbf{de\textit{r} erwarp} \textit{ouch} \textbf{im} von Semlidac\\ 
 & zwelf knappen, die sîn nâmen war\\ 
 & \textbf{zer} tjost unde \textbf{in} der ponder schar.\\ 
5 & swaz sper gebieten moht ir hant,\\ 
 & diu wurden gar v\textit{on} im verswant.\\ 
 & sîne tjoste wâren \textbf{mit} hurte \textbf{snel},\\ 
 & wan er den künic Tschirnel\\ 
 & unde sînen bruoder dâ vienc.\\ 
10 & dannoch \textbf{mêr von im dâ} ergienc:\\ 
 & \textit{s}icherheit er niht erliez\\ 
 & den herzogen Marangliez.\\ 
 & die wâren des ortes herte.\\ 
 & ir volc sich dannoch werte.\\ 
15 & Melianz, der künic, selbe streit.\\ 
 & swem er liep \textbf{oder} herzeleit\\ 
 & hete getân, \textbf{der muose} jehen,\\ 
 & daz selten \textbf{ê was} geschehen\\ 
 & von deheinem als jungen man,\\ 
20 & als ez \textbf{dâ wart von im} getân.\\ 
 & sîn hant vil \textit{vest}er schilde kloup.\\ 
 & waz \textit{stark}er sper \textbf{von} i\textit{m} \textbf{zerstoup},\\ 
 & dâ sich ponder in \textbf{den} ponder slôz.\\ 
 & sîn jungez herze was sô grôz,\\ 
25 & daz er strîtes muose geren.\\ 
 & des\textbf{ne} mohte in niemen dâ geweren\\ 
 & volliclîche, daz was ein nôt,\\ 
 & unzer Gawane tjostieren bôt.\\ 
 & \textbf{von sînen knappen er dô} nam\\ 
30 & der \textbf{zwelf} sper einez von Angram,\\ 
\end{tabular}
\scriptsize
\line(1,0){75} \newline
G I O L M Q R Z Fr21 Fr41 \newline
\line(1,0){75} \newline
\textbf{1} \textit{Initiale} G I L Z Fr21  \textbf{21} \textit{Initiale} I  \textbf{29} \textit{Initiale} O L Z Fr41   $\cdot$ \textit{Capitulumzeichen} R  \newline
\line(1,0){75} \newline
\textbf{1} \textit{Die Verse 370.13-412.12 fehlen} Q   $\cdot$ Melianz] Melyanze O Melianze M Fr41 Meliancze R Meliantze Z  $\cdot$ der] \textit{om.} I er Z \textbf{2} der] [der]: dem G dem I (O) (L) (M) (R) (Fr21)  $\cdot$ ouch] \textit{om.} G  $\cdot$ im] er I O L (M) R Fr21  $\cdot$ Semlidac] semlidach G O Z senblitac I Somlidach L Semblidag R \textbf{4} zer] Der R an der Fr41  $\cdot$ in] \textit{om.} L \textbf{5} swaz] Waz L (M) (R)  $\cdot$ gebieten moht] moch gebietten R \textbf{6} diu] die I (O)  $\cdot$ gar von im] gar vil im G von im gar I alle R \textbf{7} wâren mit] waren von O L (R) die waren von Fr21  $\cdot$ hurte] hurten Z tiost Fr21  $\cdot$ snel] hel O L M R Z Fr21 \textbf{8} er] \textit{om.} O  $\cdot$ den] dem R der Z (Fr21)  $\cdot$ Tschirnel] scrinel I [schr*]: schirmel O Schirmel L (M) schirniel R Fr21 Tschirniel Z \textbf{9} sînen] sine M  $\cdot$ dâ] \textit{om.} Fr21 \textbf{10} dâ] \textit{om.} I L Z \textbf{11} sicherheit] der sicherheit G  $\cdot$ erliez] verliez I \textbf{12} Marangliez] [maragliez]: marangliez G Moratigliez O Marangliesz M Maranglies R Merangliez Z \textbf{13} ortes] orten Z \textbf{14} sich] sie M sin Z \textbf{15} Melianz] Melyanz O Meliancz R Meliantz Z  $\cdot$ selbe] selben M da selbe Z \textbf{16} swem er] Wem L Wenn er R  $\cdot$ herzeleit] leit I hercen leit O (Z) \textbf{17} getân] geto R \textbf{18} :::m::: Fr41  $\cdot$ selten] selbe L  $\cdot$ ê was] was ê O \textbf{19} jungen] iungem I \textbf{20} dâ wart von im] von im da wart I (Fr41) da von im wart L (M) Z do von Jm ward R \textbf{21} vester] starcher G  $\cdot$ kloup] zerchlaup I (R) \textbf{22} starker] vester G  $\cdot$ sper] stangen R  $\cdot$ im] in G \textbf{23} dâ] Do R  $\cdot$ sich] sin I  $\cdot$ ponder in den ponder] poynder in die poynder I poyndir inden inden poyndir M strit gen strit R \textbf{26} desne mohte] Der enmohte L Des mocht R \textbf{27} ein] en M \textbf{28} unzer] Vnz her O (L) (Z) Vncz im R  $\cdot$ Gawane] Gawan I O L (M) R (Z)  $\cdot$ tjostieren] Tioste I vollenklichen R \textbf{29} :::ze sinem knappen nam Fr41  $\cdot$ von] ÷on O  $\cdot$ dô] da M Z \textbf{30} einez] ein M Fr41  $\cdot$ Angram] angaram I angil M agram R (Fr41) \newline
\end{minipage}
\hspace{0.5cm}
\begin{minipage}[t]{0.5\linewidth}
\small
\begin{center}*T
\end{center}
\begin{tabular}{rl}
 & \begin{large}M\end{large}elyanz \textbf{der} helfe sich bewac,\\ 
 & \textbf{dem er wart} ouch vo\textit{n} Semblidac\\ 
 & zwelf knappen, die sîn nâmen war\\ 
 & \textbf{zer} tjost unde \textbf{in} der poynder schar.\\ 
5 & swaz sper gebieten mohte ir hant,\\ 
 & di\textit{u} wurden gar von im verswant.\\ 
 & sîne tjoste wâren \textbf{mit} hurte \textbf{hel},\\ 
 & wander den künec Tschirniel\\ 
 & unde sînen bruoder dâ vienc.\\ 
10 & dannoch \textbf{von im noch mêr} ergienc:\\ 
 & sicherheite er\textbf{n} niht erliez\\ 
 & den herzogen Marangliez.\\ 
 & die wâren des ortes herte.\\ 
 & ir volc sich dannoch werte.\\ 
15 & Melyanz, der künec, \textbf{dâ} selbe streit.\\ 
 & swem er liep \textbf{oder} herzeleit\\ 
 & hete getân, \textbf{der muose} jehen,\\ 
 & daz selten \textbf{ê was} geschehen\\ 
 & von deheinem alsô jungen man,\\ 
20 & alsez \textbf{von im dâ wart} getân.\\ 
 & sîn hant vil vester schilte kloup.\\ 
 & waz starker \textit{sper} \textbf{von} im \textbf{zerstoup},\\ 
 & dâ sich poynder in poynder slôz.\\ 
 & sîn jungez herze was sô grôz,\\ 
25 & daz er strîtes muose gern.\\ 
 & des mohtin niema\textit{n} dâ gewern\\ 
 & volleclîche, daz was ein nôt,\\ 
 & unz er Gawane tjostieren bôt.\\ 
 & \textbf{\textit{\begin{large}V\end{large}}on sînen knappen er dô} nam\\ 
30 & der sper einez von Angram,\\ 
\end{tabular}
\scriptsize
\line(1,0){75} \newline
T V W \newline
\line(1,0){75} \newline
\textbf{1} \textit{Initiale} T W  \textbf{29} \textit{Initiale} T W  \newline
\line(1,0){75} \newline
\textbf{1} Melyanz] [Melian*]: Melianze V MElianz W  $\cdot$ der] [*]: er V \textbf{2} [De*]: Der erwarp [*]: im oͮch von semplidag V  $\cdot$ Dem erwarb ochir von semblidag W  $\cdot$ von] vo T \textbf{4} unde] \textit{om.} V  $\cdot$ poynder] poyndier T \textbf{5} swaz] Was W \textbf{6} diu] die T \textbf{7} mit] [*]: von V \textit{om.} W  $\cdot$ hel] [*]: snel V \textbf{8} Tschirniel] Tscirniel T schirniel V tschirmel W \textbf{9} dâ] do V W \textbf{10} von im noch mêr] mer von im V mer do von im W \textbf{11} ern] er W \textbf{12} Marangliez] maranglies W \textbf{15} Melyanz] Melẏanz V Melians W  $\cdot$ dâ] do V \textit{om.} W \textbf{16} swem] Wem W \textbf{20} von im dâ wart] von im wart [*]: do V do von im ward W \textbf{22} waz] [*]: Jo waz V  $\cdot$ sper] \textit{om.} T  $\cdot$ von] vor W \textbf{23} dâ] Das W  $\cdot$ sich poynder] sich poyndier T  $\cdot$ in] in [*]: den V in dem W \textbf{26} des mohtin] [D*]: Do enmoͤhte in V  $\cdot$ nieman] niema T  $\cdot$ dâ] do V W \textbf{28} er] her W  $\cdot$ Gawane] gawan W \textbf{29} [*]: Gawan von sinen knappen nam V  $\cdot$ Von] Don T \textbf{30} sper] zwelf sper V (W) \newline
\end{minipage}
\end{table}
\end{document}
