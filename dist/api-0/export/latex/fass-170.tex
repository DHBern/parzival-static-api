\documentclass[8pt,a4paper,notitlepage]{article}
\usepackage{fullpage}
\usepackage{ulem}
\usepackage{xltxtra}
\usepackage{datetime}
\renewcommand{\dateseparator}{.}
\dmyyyydate
\usepackage{fancyhdr}
\usepackage{ifthen}
\pagestyle{fancy}
\fancyhf{}
\renewcommand{\headrulewidth}{0pt}
\fancyfoot[L]{\ifthenelse{\value{page}=1}{\today, \currenttime{} Uhr}{}}
\begin{document}
\begin{table}[ht]
\begin{minipage}[t]{0.5\linewidth}
\small
\begin{center}*D
\end{center}
\begin{tabular}{rl}
\textbf{170} & umbe\textbf{z} vingerlîn unt \textbf{umbe}\textbf{z} vürspan\\ 
 & unt wie erz harnasch gewan.\\ 
 & der wirt erkante den ritter rôt.\\ 
 & \textbf{der} \textbf{ersiufte} unt erbarmete in sîn \textbf{nôt}.\\ 
5 & \textbf{sînen} \textbf{gast} \textbf{des namen er} niht \textbf{erliez},\\ 
 & den rôten ritter er in hiez.\\ 
 & \textit{\begin{large}D\end{large}}ô man \textbf{den tisch} \textbf{hin} dan genam,\\ 
 & dar nâch wart \textbf{wilder muot} vil zam.\\ 
 & der wirt sprach zem gaste sîn:\\ 
10 & "ir reit als ein kindelîn.\\ 
 & wan \textbf{geswîget} ir iwerer muoter gar\\ 
 & unt nemt \textbf{anderer} mære war?\\ 
 & \textbf{habt} iuch an mînen rât;\\ 
 & \textbf{der} scheidet iuch \textbf{von} missetât.\\ 
15 & Sus heb ich an, lât iuch gezemen,\\ 
 & ir sult niemer iuch verschemen.\\ 
 & \textbf{verschamter} lîp, waz \textbf{touc} der mêr?\\ 
 & \textbf{der} wont in der mûze rêr,\\ 
 & dâ im werdecheit entrîset\\ 
20 & unt \textbf{in} gein der helle wîset.\\ 
 & \textbf{Ir tragt geschickede} unt schîn,\\ 
 & ir mugt wol \textbf{volkes} hêrre sîn.\\ 
 & ist hôch \textbf{unt} hœhet sich iwer art,\\ 
 & lât iweren willen des bewart,\\ 
25 & i\textit{u}ch sol erbarmen nôtec her.\\ 
 & gein des kumber sît ze wer\\ 
 & mit milte unt mit güete.\\ 
 & vlîzet iuch diemüete.\\ 
 & der kumberhafte werde man\\ 
30 & wol mit schame ringen kan\\ 
\end{tabular}
\scriptsize
\line(1,0){75} \newline
D \newline
\line(1,0){75} \newline
\textbf{7} \textit{Initiale} D  \textbf{15} \textit{Majuskel} D  \textbf{21} \textit{Majuskel} D  \newline
\line(1,0){75} \newline
\textbf{7} Dô] ÷o \textit{nachträglich korrigiert zu:} Do D \textbf{25} iuch] ich D \newline
\end{minipage}
\hspace{0.5cm}
\begin{minipage}[t]{0.5\linewidth}
\small
\begin{center}*m
\end{center}
\begin{tabular}{rl}
 & umb \textbf{ein} vingerlîn und \textbf{umb} \textbf{ein} vürspa\textit{n}\\ 
 & und wie er daz harna\textit{s}ch gewan.\\ 
 & der wirt erkante den ritter rôt.\\ 
 & \textbf{er} \textbf{siufzete} und erbarmete in sîn \textbf{nôt}.\\ 
5 & \textbf{sînen} \textbf{gast} \textbf{des namen er} niht \textbf{enliez},\\ 
 & den rôten ritter er in hiez.\\ 
 & \begin{large}D\end{large}ô man \textbf{die tische} dan genam,\\ 
 & dar nâch wart \textbf{widermuot} vil zam.\\ 
 & der wirt sprach zem gaste sîn:\\ 
10 & "ir redet als ein kindelîn.\\ 
 & wanne \textbf{geswîget} ir iuwer muoter gar\\ 
 & und nemet \textbf{andere} mære war?\\ 
 & \textbf{haltet} iuch an mînen rât;\\ 
 & \textbf{er} scheidet iuch \textbf{von} missetât.\\ 
15 & sus hebe ich an, lât \textbf{es} iuch gezemen,\\ 
 & ir sollet niemer iuch verschemen.\\ 
 & \textbf{verscham\textit{t}er} lîp, waz \textbf{sol} der \textit{m}êr?\\ 
 & \textbf{der} wonet in der mûze rêr,\\ 
 & d\textit{â} ime werdicheit entrîst\\ 
20 & und \textbf{in} gegen der helle wîst.\\ 
 & \textbf{ir traget geschickede} und schîn,\\ 
 & ir muget wol \textbf{volkes} hêrre sîn.\\ 
 & i\textit{st} hôch \textbf{und} hœhet sich iuwer art,\\ 
 & lât iuwern willen des bewart,\\ 
25 & \textit{i}uch sol erb\textit{a}rmen nôtic her.\\ 
 & gegen des kumber sît ze wer\\ 
 & mit milte und mi\textit{t g}üete\\ 
 & \textbf{und} vlîzet iuch diemüete.\\ 
 & der kumberhafte werde man\\ 
30 & wol mit schame ringen kan\\ 
\end{tabular}
\scriptsize
\line(1,0){75} \newline
m n o Fr69 \newline
\line(1,0){75} \newline
\textbf{7} \textit{Initiale} m n o Fr69  \newline
\line(1,0){75} \newline
\textbf{1} umb ein] Vmbes Fr69  $\cdot$ und umb ein] vnd vmbs Fr69  $\cdot$ vürspan] furspang m o \textbf{2} daz harnasch] das harnach m das harners o scharnasch Fr69 \textbf{3} den] der o \textbf{4} siufzete] súfftzet n (o)  $\cdot$ erbarmete] erbarmet n \textbf{5} des namen er] er des name Fr69  $\cdot$ niht enliez] nit liesz n [in hies]: nit lies o \textbf{11} iuwer] ir m \textbf{17} verschamter] Verschamper m  $\cdot$ mêr] ner m mire o \textbf{18} wonet] wonte n (o)  $\cdot$ mûze] masse o \textbf{19} dâ] Do m n o \textbf{20} in] \textit{om.} n o  $\cdot$ helle] hellen n \textbf{21} ir traget] Mit n o \textbf{22} ir] \textit{om.} n o  $\cdot$ volkes] wockes o \textbf{23} ist hôch] Jch hoch m Zoch n Doch o \textbf{24} iuwern] vwer o \textbf{25} iuch] [V]: Ouch m  $\cdot$ erbarmen] erbermen m \textbf{26} kumber] kammer n \textbf{27} milte] fliffe n  $\cdot$ mit güete] mit wer vnd guͦte m \textbf{29} kumberhafte] komber n \newline
\end{minipage}
\end{table}
\newpage
\begin{table}[ht]
\begin{minipage}[t]{0.5\linewidth}
\small
\begin{center}*G
\end{center}
\begin{tabular}{rl}
 & umbe\textbf{z} vinge\textit{r}lîn unde \textbf{umbe}\textbf{z} vürspan\\ 
 & unde wie erz harnasch gewan.\\ 
 & der wirt erkande den rîter rôt.\\ 
 & \textbf{er} \textbf{ersûfte} unde erbarmet in sîn \textbf{tôt}.\\ 
5 & \textbf{den} \textbf{gast} \textbf{ers namen} niht \textbf{erliez},\\ 
 & den rôten rîter er in hiez.\\ 
 & \begin{large}D\end{large}ô man \textbf{den tisch} \textbf{her} dane ge\textit{n}a\textit{m},\\ 
 & dar nâch wart \textbf{wilder muot} vil zam.\\ 
 & der wirt sprach zem gaste sîn:\\ 
10 & "ir redet als ein kindelîn.\\ 
 & wan \textbf{swîget} ir iwerre muoter gar\\ 
 & unde nemet \textbf{andere} mære war?\\ 
 & \textbf{halt} iuch an mînen rât;\\ 
 & \textbf{der} scheidet iuch \textbf{von} missetât.\\ 
15 & sus hebe ich an, lât\textbf{s} iuch gezemen,\\ 
 & ir sult nimer iuch verschemen.\\ 
 & \textbf{verschamter} lîp, waz \textbf{touc} der mêre?\\ 
 & \textbf{der} \textit{won}t in der mûze rêre,\\ 
 & dâ im werdicheit entrîset\\ 
20 & unde \textbf{in} gein der helle wîset.\\ 
 & \textbf{mich entriege gesiht} unde schîn,\\ 
 & ir muget wol \textbf{volkes} hêrre sîn.\\ 
 & ist hôch \textbf{oder} hœhet sich iwer art,\\ 
 & lât iweren willen des bewart,\\ 
25 & iuch sol erbarmen nôtec her.\\ 
 & gein des kumber sît ze wer\\ 
 & mit milte unde mit güete.\\ 
 & vlîzet iuch diemüete.\\ 
 & der kumberhafte werde man\\ 
30 & wol mit schame ringen kan\\ 
\end{tabular}
\scriptsize
\line(1,0){75} \newline
G I O L M Q R Z Fr21 \newline
\line(1,0){75} \newline
\textbf{1} \textit{Initiale} Q  \textbf{7} \textit{Initiale} G O R Z Fr21  \textbf{15} \textit{Initiale} I L M  \newline
\line(1,0){75} \newline
\textbf{1} umbez vingerlîn] vmbez vingelin G vnd vmb daz vingerlin I  $\cdot$ unde umbez vürspan] vnde daz fvrspan O vnd die fuͯrspang L vnde vurspan M (Q) vnd den fᵫrspang R \textbf{2} erz] er den L R  $\cdot$ harnasch] harnach R \textbf{4} er] Der Z  $\cdot$ ersûfte] sufte I (L) (M) (Q) (Fr21) ersunfcze R  $\cdot$ erbarmet] irbarmite M (Q)  $\cdot$ in] im I O Fr21  $\cdot$ tôt] not M Q R Fr21 \textbf{5} den] Sein Q  $\cdot$ niht] da niht Z  $\cdot$ erliez] en liesz Q \textbf{6} er] man L \textbf{7} Dô] ÷o O Da M Z  $\cdot$ genam] gewan G \textbf{8} wart] wort Fr21  $\cdot$ vil] \textit{om.} R Fr21 \textbf{10} redet] redet reht I \textbf{11} swîget] geswiget O L M Q Z (Fr21) geschwig R  $\cdot$ ir] \textit{om.} M Z \textbf{13} halt] hapt I (O) (L) (Q) (Fr21) \textbf{15} ich] ichz I  $\cdot$ lâts iuch] lat evch sin I lat ivchs O Fr21 last euch Q lacz úch R \textbf{16} ir] Jr en M (Z) (Fr21)  $\cdot$ nimer iuch] evch nimmer I (Q) uch mynner M  $\cdot$ verschemen] vor sene M schemen Q \textbf{17} touc] sol L  $\cdot$ der] daz I deme M \textbf{18} der wont] der lebet G da wont I Der itzúnt Q  $\cdot$ mûze] Mure M  $\cdot$ rêre] were Q mære Fr21 \textbf{19} dâ] Do Q \textbf{20} in] \textit{om.} L \textbf{21} Jr tragt geschichede vnd schin Z  $\cdot$ entriege] en truge M  $\cdot$ gesiht] geshiht I (M) (Q) geschickete L (R)  $\cdot$ schîn] geschyn M sin Q \textbf{23} oder] vnd Z  $\cdot$ hœhet] hohe M \textbf{24} willen] \textit{om.} Z  $\cdot$ des] sin des I des sin O \textbf{25} iuch] Noch Q  $\cdot$ nôtec] notie R \textbf{26} kumber] kumers R (Z) \textbf{27} \textit{Versfolge 170.28-27} I   $\cdot$ unde mit] vnd mit vnd mit Z \textbf{28} vlîzet] vnd flizt I (L) (M) (Z) \textbf{29} werde] werder I \textbf{30} wol] Wolt Q  $\cdot$ schame] armuͤt I >scamen< M [schaw*]: schamen Q schein Z \newline
\end{minipage}
\hspace{0.5cm}
\begin{minipage}[t]{0.5\linewidth}
\small
\begin{center}*T
\end{center}
\begin{tabular}{rl}
 & umbe \textbf{daz} vingerlîn unde vürspan\\ 
 & unde wierz harnasch gewan.\\ 
 & Der wirt erkande den rîter rôt.\\ 
 & \textbf{er} \textbf{sûfte} unde erbarmete in sîn \textbf{nôt}.\\ 
5 & \textbf{den}, \textbf{des namen er} niht \textbf{erliez},\\ 
 & den rôten rîter er in \textbf{dô} hiez.\\ 
 & \begin{large}D\end{large}ô man \textbf{den tisch} \textbf{her} dan genam,\\ 
 & dar nâch wart \textbf{sîn} \textbf{wilder muot} vil zam.\\ 
 & der wirt sprach zem gaste sîn:\\ 
10 & "ir redet als ein kindelîn.\\ 
 & wan \textbf{geswîget} ir iuwerre muoter gar\\ 
 & unde nemet \textbf{anderre} mære war?\\ 
 & \textbf{haltet} iuch an mînen rât;\\ 
 & \textbf{er} scheidet iuch \textbf{vor} missetât.\\ 
15 & Sus hebich an, lât\textbf{z} iuch gezemen,\\ 
 & ir sult niemer iuch verschemen.\\ 
 & \textbf{versmâhter} lîp, waz \textbf{touc} der mêr?\\ 
 & \textbf{er} wont in der mûze rêr,\\ 
 & dâ im werdecheit entrîset\\ 
20 & unde gegen der helle wîset.\\ 
 & \textbf{ir traget geschicke} unde schîn,\\ 
 & ir muget wol \textbf{landes} hêrre sîn.\\ 
 & ist hôch \textbf{unde} hœhet sich iuwer art,\\ 
 & lât iuwern willen des bewart,\\ 
25 & iuch sol erbarmen nôtic her.\\ 
 & gegen des kumber sît ze wer\\ 
 & mit milte unde mit güete\\ 
 & \textbf{unde} vlîzet iuch diemüete.\\ 
 & der kumberhafte werde man\\ 
30 & wol mit schame ringen kan\\ 
\end{tabular}
\scriptsize
\line(1,0){75} \newline
T U V W \newline
\line(1,0){75} \newline
\textbf{3} \textit{Majuskel} T  \textbf{7} \textit{Initiale} T U  \textbf{9} \textit{Initiale} W  \textbf{15} \textit{Majuskel} T  \newline
\line(1,0){75} \newline
\textbf{1} vürspan] das fúrspan W \textbf{2} wierz] wie [*]: er V wie er den W \textbf{4} sûfte] ersuͦfzete U (V) seúfftzet W  $\cdot$ erbarmete] erbarmet V W  $\cdot$ nôt] dot U [*]: tot V \textbf{5} den] [*]: Sinen gast V Den gast W \textbf{6} dô hiez] [*]: hiez V \textbf{7} her dan] hindan V her ab W \textbf{8} wilder] wider W  $\cdot$ vil] \textit{om.} U \textbf{11} geswîget] swigent V  $\cdot$ iuwerre] der W \textbf{12} anderre] an dirre U \textbf{14} er] ez V  $\cdot$ vor] von U V W \textbf{15} an] \textit{om.} U  $\cdot$ lâtz] lant V (W) \textbf{16} sult] soͤlt es W  $\cdot$ verschemen] [*schêmen]: verschêmen V geschemen W \textbf{17} versmâhter] Verschameter U V (W)  $\cdot$ touc] thuͦt W  $\cdot$ mêr] mir U \textbf{18} wont] wonte U  $\cdot$ mûze] mv́nzere V \textbf{19} dâ] Do U W \textbf{20} helle] hellen U W  $\cdot$ wîset] [*]: in wiset V \textbf{22} landes] ein landes W \textbf{24} Lond eúwer iugent sein bas bewart W \textbf{26} kumber] kuͦmers U \textbf{27} unde mit] vnd W \textbf{28} diemüete] gemuͦte U [*]: demuͤte V \textbf{30} schame] schamen V \newline
\end{minipage}
\end{table}
\end{document}
