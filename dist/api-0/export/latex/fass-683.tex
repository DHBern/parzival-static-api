\documentclass[8pt,a4paper,notitlepage]{article}
\usepackage{fullpage}
\usepackage{ulem}
\usepackage{xltxtra}
\usepackage{datetime}
\renewcommand{\dateseparator}{.}
\dmyyyydate
\usepackage{fancyhdr}
\usepackage{ifthen}
\pagestyle{fancy}
\fancyhf{}
\renewcommand{\headrulewidth}{0pt}
\fancyfoot[L]{\ifthenelse{\value{page}=1}{\today, \currenttime{} Uhr}{}}
\begin{document}
\begin{table}[ht]
\begin{minipage}[t]{0.5\linewidth}
\small
\begin{center}*D
\end{center}
\begin{tabular}{rl}
\textbf{683} & mit im dâ komen wâren,\\ 
 & die vîende kunden vâren.\\ 
 & Sus wolte der künec Gramoflanz\\ 
 & mit kampfe rechen sînen kranz,\\ 
5 & daz ez vil liute sæhe,\\ 
 & wem man \textbf{dâ} prîses jæhe.\\ 
 & \textbf{die} vürsten ûz sîme rîche\\ 
 & mit rîtern \textbf{werlîche}\\ 
 & \textbf{wâren dâ} unt \textbf{ouch} mit vrouwen schar.\\ 
10 & man sach dâ liute wol gevar.\\ 
 & \textbf{Artuses boten kômen} hie;\\ 
 & die vunden den künec, \textbf{nû} hœret wie:\\ 
 & \textbf{Palmâtes} \textbf{ein dicke} matraz\\ 
 & lag under dem künege, \textbf{al} dâ \textbf{der} saz,\\ 
15 & dar ûf \textbf{gesteppet} ein pfelle breit.\\ 
 & juncvrouwen clâre unt \textbf{ouch} gemeit\\ 
 & schuohten \textbf{îserîne kolzen}\\ 
 & an den künec stolzen.\\ 
 & Ein pfelle gap kostlîchen prîs,\\ 
20 & geworht in \textbf{Ecidemonis},\\ 
 & \textbf{beidiu} breit unde lanc,\\ 
 & hôhe ob im durch schate swanc,\\ 
 & an \textbf{zwelf} schefte genomen.\\ 
 & Artuses boten wâren komen;\\ 
25 & \textbf{gein} dem, der hôchverte hort\\ 
 & truoc, si sprâchen disiu wort:\\ 
 & "\textit{\begin{large}H\end{large}}êrre, uns hât \textbf{dâ her} gesant\\ 
 & Artus, der dâ vür erkant\\ 
 & was, daz er prîs etswenne truoc.\\ 
30 & er het ouch werdecheit genuoc,\\ 
\end{tabular}
\scriptsize
\line(1,0){75} \newline
D \newline
\line(1,0){75} \newline
\textbf{3} \textit{Majuskel} D  \textbf{13} \textit{Majuskel} D  \textbf{19} \textit{Majuskel} D  \textbf{27} \textit{Initiale} D  \newline
\line(1,0){75} \newline
\textbf{11} Artuses] Artvss D \textbf{20} Ecidemonis] Ecidemonîs D \textbf{24} Artuses] Artvs D \textbf{27} Hêrre] ÷erre D \newline
\end{minipage}
\hspace{0.5cm}
\begin{minipage}[t]{0.5\linewidth}
\small
\begin{center}*m
\end{center}
\begin{tabular}{rl}
 & mit im d\textit{â} komen wâren,\\ 
 & die vîende kunden vâren.\\ 
 & sus wolte der künic Gramolanz\\ 
 & mit kampfe rechen sînen kranz,\\ 
5 & daz ez vil liute sæhe,\\ 
 & wem man \textbf{d\textit{â}} prîses jæhe.\\ 
 & \textbf{die} vürsten ûz s\textit{î}ne\textit{m} rîche\\ 
 & mit rittern \textbf{werlîche}\\ 
 & \textbf{w\textit{â}r\textit{en} d\textit{â}} und \textbf{ouch} mit vrowen schar.\\ 
10 & man sach d\textit{â} liute wol gevar.\\ 
 & \textbf{\begin{large}N\end{large}û kômen Artuses boten} hie;\\ 
 & die vunden den künic, hœret wie:\\ 
 & \textbf{pa\textit{l}mâtes} \textbf{ein dicke} matraz\\ 
 & lac under dem künige, \textbf{al}dâ \textbf{er} saz,\\ 
15 & dar ûf \textbf{geschepft} ein pfelle breit.\\ 
 & juncvrowen clâre und gemeit\\ 
 & schuoheten \textbf{îserîne kolzen}\\ 
 & an den künic stolzen.\\ 
 & ein pfelle gap kostlîchen prîs,\\ 
20 & geworht in \textbf{E\textit{c}ide\textit{m}onis},\\ 
 & \textbf{b\textit{ei}d\textit{iu}} breit und lanc,\\ 
 & hôhe ob im durch schate swanc,\\ 
 & an \textbf{zwelf} schefte genomen.\\ 
 & Artuses boten wâren komen;\\ 
25 & \textbf{gegen} dem, der hôchvart hort\\ 
 & truoc, si sprâch\textit{en} disiu wort:\\ 
 & "\textit{h}êrre, uns het \textbf{dâ her} gesant\\ 
 & Artus, der dâ vür erkant\\ 
 & was, daz er prîs et\textit{e}wan truoc.\\ 
30 & er het ouch wirdicheit genuoc,\\ 
\end{tabular}
\scriptsize
\line(1,0){75} \newline
m n o \newline
\line(1,0){75} \newline
\textbf{11} \textit{Initiale} m   $\cdot$ \textit{Capitulumzeichen} n  \newline
\line(1,0){75} \newline
\textbf{1} dâ] do m n o \textbf{3} Gramolanz] gramolantz m n gramolanncz o \textbf{6} dâ] do m n o \textbf{7} sînem] sunnen m \textbf{9} wâren] Worant m  $\cdot$ dâ] do m n o \textbf{10} dâ] do m n o \textbf{11} Artuses] artúses o \textbf{13} palmâtes] Paltmatz m Palmat o \textbf{15} geschepft] [gescherft]: geschepft m gesteppfet n g:::fet o \textbf{20} Ecidemonis] ethidenonis m ecidemanis o \textbf{21} beidiu] Bade m o \textbf{22} ob] ab m \textbf{24} Artuses] Artus m n \textbf{25} der] here n \textbf{26} sprâchen] sprach m n o \textbf{27} hêrre] Lere m n \textbf{28} Artus] Artuͯs o \textbf{29} etewan] etiwan m \newline
\end{minipage}
\end{table}
\newpage
\begin{table}[ht]
\begin{minipage}[t]{0.5\linewidth}
\small
\begin{center}*G
\end{center}
\begin{tabular}{rl}
 & mit im dâ komen wâren,\\ 
 & die vîende kunden vâren.\\ 
 & \begin{large}S\end{large}us wolde der künic Gramoflanz\\ 
 & mit kampfe rechen sînen kranz,\\ 
5 & daz ez vil liute sæhe,\\ 
 & wem man \textbf{des} brîses jæhe.\\ 
 & vürsten ûz sînem rîche\\ 
 & mit rîtern \textbf{werdeclîche}\\ 
 & \textbf{dâ wâren} \textit{und} mit vrouwen schar.\\ 
10 & man sach dâ liute wol gevar.\\ 
 & \textbf{Artuses boten kômen} hie;\\ 
 & die vunden den künic, \textbf{nû} hœrt wie:\\ 
 & \textbf{von palmât} \textbf{dicke ein} matraz\\ 
 & lac under dem künige, dâ \textbf{er} saz,\\ 
15 & dar ûf \textbf{gesteppet} ein pfelle breit.\\ 
 & juncvrouwen clâr unde gemeit,\\ 
 & \textbf{die} schuohten \textbf{îsen golzen}\\ 
 & an den künic stolzen.\\ 
 & ein pfelle gab kosticlîchen  prîs,\\ 
20 & geworht in \textbf{Ezzedemonis},\\ 
 & breit unde lanc,\\ 
 & hôch obe im durch schate swanc,\\ 
 & an \textbf{die} schefte genomen.\\ 
 & Artuses boten wâren komen;\\ 
25 & \textbf{zuo} dem, der hôchverte hort\\ 
 & truoc, si sprâchen disiu wort:\\ 
 & "hêrre, uns hât \textbf{der künic} gesant,\\ 
 & Artus, der dâ vür erkant\\ 
 & was, daz er brîs etswenne truoc.\\ 
30 & er het ouch werdecheit genuoc,\\ 
\end{tabular}
\scriptsize
\line(1,0){75} \newline
G I L M Z Fr18 Fr20 Fr52 \newline
\line(1,0){75} \newline
\textbf{3} \textit{Initiale} G I L Z Fr18 Fr52  \textbf{21} \textit{Initiale} I  \newline
\line(1,0){75} \newline
\textbf{2} vîende] den vienden Fr52 \textbf{3} Gramoflanz] gramorflanz M gramoflantz Z Fr52 \textbf{4} rechen] rethen Z \textbf{6} des] da L M Z Fr18 Fr52 \textbf{7} vürsten] Die fvrsten Z \textbf{9} und] \textit{om.} G \textbf{10} man sach dâ] Da waren Z \textbf{11} Artuses] Artus G Z Artuͯses L  $\cdot$ kômen] \textit{om.} L \textbf{12} den] an den Fr18  $\cdot$ nû] \textit{om.} I M Z \textbf{13} von] Den M \textbf{14} er] der L Z Fr18 \textbf{15} pfelle] pflelle L \textbf{16} gemeit] breit M \textbf{17} schuohten] schichten M \textbf{19} kosticlîchen] kunstechlichen I \textbf{20} Ezzedemonis] ezedomonis I Ezzidemonis L Z Zidemonis M ezẏdemonis Fr18 \textbf{21} lanc] da zv lanc Z \textbf{22} obe im] vnde M  $\cdot$ schate] shates I state M \textbf{23} die] zwelf Z \textbf{24} Artuses] Artus G M Z Fr52 \textbf{25} dem] \textit{om.} I \textbf{26} si] si ir wec si I \textbf{27} der künic] her Z \textbf{28} der] \textit{om.} I  $\cdot$ dâ] \textit{om.} M \textbf{29} brîs] \textit{om.} Fr52 \textbf{30} vnd wirdikeit hette genvc Fr52  $\cdot$ het] hat M \newline
\end{minipage}
\hspace{0.5cm}
\begin{minipage}[t]{0.5\linewidth}
\small
\begin{center}*T
\end{center}
\begin{tabular}{rl}
 & mit im dar komen wâren,\\ 
 & die vîende kunden vâren.\\ 
 & sus wolte der künec Gramoflanz\\ 
 & mit kampfe rechen sînen kranz,\\ 
5 & \textit{daz} ez vil liute sæh\textit{e}\\ 
 & \textbf{und} wem man \textbf{d\textit{â}} prîses jæh\textit{e}.\\ 
 & vürsten ûz sîne\textit{m} rîche\\ 
 & mit rîtern \textbf{wirdeclîche}\\ 
 & \textbf{d\textit{â} wâren} und mit vrouwen schar.\\ 
10 & man sach d\textit{â} liute wol gevar.\\ 
 & \textbf{Artuses boten kômen} hie;\\ 
 & die vunden den künec, \textbf{nû} hœret wie:\\ 
 & \textbf{von palmât} \textbf{dicke ein} matraz\\ 
 & lac under dem künege, d\textit{â} \textbf{er} saz,\\ 
15 & dar ûf \textbf{gesteppet} ein pfelle breit.\\ 
 & juncvrouwen clâr und gemeit,\\ 
 & \textbf{die} schuoheten \textbf{îserkolzen}\\ 
 & an den künec stolzen.\\ 
 & ein pfelle gap kostlîchen prîs,\\ 
20 & geworht in \textbf{Coydomis},\\ 
 & breit und \textbf{dar zuo} lanc,\\ 
 & hôhe o\textit{b} im durch schate swanc,\\ 
 & an \textbf{zwelf} schefte genomen.\\ 
 & Artuses boten wâren komen;\\ 
25 & \textbf{zuo} dem, der hôchvert\textit{e} hort\\ 
 & truoc, si sprâchen disiu wort:\\ 
 & "\begin{large}H\end{large}êrre, uns hât \textbf{d\textit{â} her} gesant\\ 
 & Artus, der dâ vür erkant\\ 
 & was, daz er prîs etswan truoc.\\ 
30 & er hete ouch wirdecheit genuoc,\\ 
\end{tabular}
\scriptsize
\line(1,0){75} \newline
U V W Q R \newline
\line(1,0){75} \newline
\textbf{3} \textit{Initiale} V Q  \textbf{11} \textit{Initiale} W R  \textbf{27} \textit{Initiale} U V W  \newline
\line(1,0){75} \newline
\textbf{1} im] in W  $\cdot$ dar komen] [*]: do kvmmen V do komen W (Q) den kamen R \textbf{3} sus] Als Q  $\cdot$ Gramoflanz] gramaflanz V gramoflantz W gramoflansz Q Gramoflancz R \textbf{5} daz] Vnd U  $\cdot$ sæhe] sehen U Q \textbf{6} und] \textit{om.} W Q R  $\cdot$ dâ prîses] do prises U (W) (Q) [*]: pris do V  $\cdot$ jæhe] iehen U Q \textbf{7} vürsten] [*]: die fúrsten V  $\cdot$ sînem] sinen U seinē Q \textbf{9} dâ] Do U V W Q  $\cdot$ und] vnd auch W \textbf{10} dâ] do U V Q do vil W \textbf{11} Artuses] KVnig artus W Artus Q (R) \textbf{12} nû] \textit{om.} W \textbf{13} dicke] k dike R \textbf{14} dâ] do U V W Q  $\cdot$ er] der Q \textbf{16} clâr] claren Q \textbf{17} îserkolzen] Iserkloczen R \textbf{20} Coydomis] [*]: ecidemonis V assidemonis W ecidomonis Q Ecidemonis R \textbf{22} ob] of U  $\cdot$ schate] schades Q \textbf{23} schefte] schiffen W schefftten R \textbf{24} Artuses] Artus W Q Arttus R \textbf{25} hôchverte] hochvertige U \textbf{26} si] die Q \textbf{27} dâ] \textit{om.} V \textbf{29} er] der R  $\cdot$ prîs etswan] etwenne preiß W \textbf{30} hete] hatt R \newline
\end{minipage}
\end{table}
\end{document}
