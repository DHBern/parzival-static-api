\documentclass[8pt,a4paper,notitlepage]{article}
\usepackage{fullpage}
\usepackage{ulem}
\usepackage{xltxtra}
\usepackage{datetime}
\renewcommand{\dateseparator}{.}
\dmyyyydate
\usepackage{fancyhdr}
\usepackage{ifthen}
\pagestyle{fancy}
\fancyhf{}
\renewcommand{\headrulewidth}{0pt}
\fancyfoot[L]{\ifthenelse{\value{page}=1}{\today, \currenttime{} Uhr}{}}
\begin{document}
\begin{table}[ht]
\begin{minipage}[t]{0.5\linewidth}
\small
\begin{center}*D
\end{center}
\begin{tabular}{rl}
\textbf{397} & \textbf{\begin{large}W\end{large}er} machte si vor der diet sô balt?\\ 
 & daz tet diu minne junc unt alt.\\ 
 & Lyppaut \textbf{sînen willen dô} sach,\\ 
 & wande im sô liebe nie geschach,\\ 
5 & \textbf{sît} \textbf{got der êren in} niht erliez,\\ 
 & sîne tohter er dô vrouwe hiez.\\ 
 & Wie diu hôchzît ergîenc,\\ 
 & des vrâget den, der \textbf{dâ} gâbe enpfienc,\\ 
 & unt \textbf{war} \textbf{dô mannegelîch} rite,\\ 
10 & er hete gemach oder \textbf{er} strite,\\ 
 & des mag ich \textbf{niht ein} ende hân.\\ 
 & man sagte mir, daz Gawan\\ 
 & \textbf{urloup nam} ûf dem palas,\\ 
 & dar er durch urloup komen was.\\ 
15 & \textbf{Obilot des weinde vil}.\\ 
15 & \multicolumn{1}{l}{ - - - }\\ 
15 & \multicolumn{1}{l}{ - - - }\\ 
 & \textbf{si sprach}: "\textbf{nû vüeret mich mit iu hin}."\\ 
 & dô wart der \textbf{jungen}, \textbf{süezen} magt\\ 
 & diu bete von Gawane \textbf{versagt}.\\ 
 & ir muoter si kûme von im \textbf{brach}.\\ 
20 & urloup er dô zin allen sprach.\\ 
 & Lyppaut im \textbf{dienstes bôt} genuoc,\\ 
 & wander im holdez herze truoc.\\ 
 & Scherules, sîn stolzer wirt,\\ 
 & mit alden sînen niht verbirt,\\ 
25 & ern rîte ûz mit dem degen balt.\\ 
 & Gawans strâze \textbf{ûf} einen walt\\ 
 & gienc. \textbf{dar} sant er weideman\\ 
 & \textbf{unt} spîse verre mit \textbf{in} dan.\\ 
 & urloup nam der werde helt.\\ 
30 & Gawan gein kumber was verselt.\\ 
\end{tabular}
\scriptsize
\line(1,0){75} \newline
D \newline
\line(1,0){75} \newline
\textbf{1} \textit{Initiale} D  \textbf{7} \textit{Majuskel} D  \newline
\line(1,0){75} \newline
\textbf{3} Lyppaut] Lyppaot D \textbf{21} Lyppaut] Lyppaot D  $\cdot$ dienstes] diens D \textbf{23} Scherules] Scervles D \newline
\end{minipage}
\hspace{0.5cm}
\begin{minipage}[t]{0.5\linewidth}
\small
\begin{center}*m
\end{center}
\begin{tabular}{rl}
 & \textbf{wer} mahte si vor der diet sô balt?\\ 
 & daz tet diu minne junc und alt.\\ 
 & Lippo\textit{u}t \textbf{dô sînen willen} sach,\\ 
 & wand ime sô liebe nie geschach,\\ 
5 & \dag si\dag  \textbf{got der êre in} niht erliez,\\ 
 & sîne tohter er dô vrouwen hiez.\\ 
 & \textit{\begin{large}W\end{large}}ie diu hôchzît ergienc,\\ 
 & des vrâget den, der \textbf{dâ} gâbe enpfienc,\\ 
 & und \textbf{war} \textbf{mannegelîch dô} rite,\\ 
10 & er hete gemach oder strite,\\ 
 & des \textbf{en}mac ich \textit{\textbf{nû}} \textbf{kein} ende hân.\\ 
 & man sagete mir, daz Gawan\\ 
 & \textbf{urloup nam} ûf dem palas,\\ 
 & dar er durch urloup komen was.\\ 
15 & \textbf{Obilot d\textit{e}s weinde vil}.\\ 
15 & \multicolumn{1}{l}{ - - - }\\ 
15 & \multicolumn{1}{l}{ - - - }\\ 
 & \textbf{si sprach}: "\textbf{mit iu ich hinnen varn wil}."\\ 
 & dô wart der \textbf{sældebæren} maget\\ 
 & diu bete von Gawane \textbf{versaget}.\\ 
 & ir muoter si kûme von ime \textbf{gebrach}.\\ 
20 & urloup er dô zuo in allen sprach.\\ 
 & Lippo\textit{u}t ime \textbf{dienestes bôt} genuoc,\\ 
 & wand er ime holdez herze truoc.\\ 
 & Scherules, sîn stolzer wirt,\\ 
 & mit alden sînen niht verbirt,\\ 
25 & er enrîte ûz mit dem degen balt.\\ 
 & Gawans strâze \textbf{ûf} einen walt\\ 
 & gienc. \textbf{dar} sant er weideman\\ 
 & \textbf{und} spîse verre mit \textbf{im} dan.\\ 
 & urloup nam der werde helt.\\ 
30 & Gawan gegen kumber was verselt.\\ 
\end{tabular}
\scriptsize
\line(1,0){75} \newline
m n o \newline
\line(1,0){75} \newline
\textbf{7} \textit{Initiale} m n  \newline
\line(1,0){75} \newline
\textbf{1} mahte] macht n (o)  $\cdot$ vor] von o \textbf{3} Lippout] Lippoat m Lippaot n o \textbf{4} geschach] beschach n o \textbf{5} in] sie o \textbf{7} Wie] Die m  $\cdot$ hôchzît] hochgezit n o \textbf{8} dâ] do n o \textbf{9} mannegelîch] menlich n o \textbf{11} enmac] mag n o  $\cdot$ nû] min m \textbf{15} des] das m (o) \textbf{16} varn] \textit{om.} n o \textbf{18} von] \textit{om.} o  $\cdot$ Gawane] gawan n o \textbf{19} ime] ir n o \textbf{21} Lippout] Lippoat m Lippaot n o \textbf{23} Scherules] Scerules m Sterules n o  $\cdot$ stolzer] stoltzen n \textbf{24} alden sînen] aldem sẏnem o \textbf{25} enrîte] reit n o  $\cdot$ degen] gegen n \textbf{27} er] \textit{om.} o  $\cdot$ weideman] werde man n (o) \newline
\end{minipage}
\end{table}
\newpage
\begin{table}[ht]
\begin{minipage}[t]{0.5\linewidth}
\small
\begin{center}*G
\end{center}
\begin{tabular}{rl}
 & \textbf{waz} machte si vor der diet sô balt?\\ 
 & daz tet diu minne junc unde alt.\\ 
 & Libaut \textbf{nû sînen willen} sach,\\ 
 & wan im sô liebe nie geschach,\\ 
5 & \textbf{daz} \textbf{in got der êren} niht erliez,\\ 
 & sîne tohter er dô vrouwen hiez.\\ 
 & wie diu hôchzît ergienc,\\ 
 & des vrâget den, der \textbf{die} gâbe enpfienc,\\ 
 & unde \textbf{war} \textbf{mannegelîch} rite,\\ 
10 & er hete gemach oder \textbf{er} strite,\\ 
 & des mag ich \textbf{niht ein} ende hân.\\ 
 & man sagete mir, daz Gawan\\ 
 & \textbf{nam urloup} ûf dem palas,\\ 
 & dar er durch urloup komen was.\\ 
15 & \textbf{daz was Obilote leit},\\ 
15 & wan si grôz weinen niht vermeit.\\ 
15 & dô sprach si: "hêrre, sît ich bin\\ 
 & \textbf{iwer}, \textbf{sô vüeret mich mit iu hin}."\\ 
 & dô wart der \textbf{junge\textit{n}}, \textbf{süezen} maget\\ 
 & diu bet von Gawane \textbf{versaget}.\\ 
 & ir muoter si kûme von im \textbf{gebrach}.\\ 
20 & urloup er dô zin allen sprach.\\ 
 & Libaut im \textbf{dankte} genuoc,\\ 
 & wan er im holdez herze truoc.\\ 
 & Tscherules, sîn stolzer wirt,\\ 
 & mit al den sînen niht verbirt,\\ 
25 & er enrîte ûz mit dem degene balt.\\ 
 & Gawanes strâze \textbf{in} einen walt\\ 
 & gienc. \textbf{dar} sander weideman\\ 
 & \textbf{unde} spîse verre mit \textbf{im} dan.\\ 
 & urloup nam der werde helt.\\ 
30 & Gawan gein kumber was verselt.\\ 
\end{tabular}
\scriptsize
\line(1,0){75} \newline
G I O L M Q R Z Fr28 \newline
\line(1,0){75} \newline
\textbf{1} \textit{Initiale} I L Z   $\cdot$ \textit{Capitulumzeichen} R  \textbf{2} \textit{Initiale} O  \textbf{15} \textit{Initiale} I   $\cdot$ \textit{Capitulumzeichen} R  \newline
\line(1,0){75} \newline
\textbf{1} \textit{Die Verse 370.13-412.12 fehlen} Q   $\cdot$ waz] Daz I (M)  $\cdot$ machte] maht I (O) L (M) (R) (Z) \textbf{2} daz] ÷Daz O \textbf{3} Libaut] Lybavt O Z Lýbavt L Lybant R Lẏbavt Fr28 \textbf{5} êren] ere Z  $\cdot$ erliez] enlies R \textbf{6} dô] da M Z  $\cdot$ vrouwen] frowe R Z \textbf{7} hôchzît] hochgezit I \textbf{8} den] den ob ir welt R  $\cdot$ die] da I L M Z  $\cdot$ enpfienc] da empfieng R \textbf{9} war] wer O M  $\cdot$ mannegelîch] manlichen I (M) (R) menniclich da Z \textbf{10} gemach] gimachit M \textbf{11} mag] en mac M (Z)  $\cdot$ ich] \textit{om.} I ich hie Z  $\cdot$ ein] \textit{om.} I \textbf{12} sagete] seit I (O) (L) (M) (Z) \textbf{14} dar] Das R  $\cdot$ durch] truͯch L \textbf{15} Obilote] Obilot I (Z) obylot O obilete R \textbf{15} groz weinen si nih vermeit I  $\cdot$ grôz] \textit{om.} M \textbf{15} dô] \textit{om.} L Da M  $\cdot$ sprach si] Sý sprach L \textbf{16} mit] \textit{om.} Z \textbf{17} dô] Da O M  $\cdot$ jungen] ivnge G  $\cdot$ süezen] \textit{om.} M \textbf{18} Gawane] Gauwan I Gawanen O gawan M (R) Z  $\cdot$ versaget] gar versagt O L (M) (R) Z \textbf{19} von] \textit{om.} R \textbf{20} dô] da M Z  $\cdot$ zin] zu R \textbf{21} Libaut] Lybavt O Z Libavt L M Lybant R  $\cdot$ dankte] danchet O (L) (R) (Z) \textbf{22} Wan er im danket genuc Z \textbf{23} Tscherules] Schurles I Tschervles O Tsheruͯles L Scerules M Scherules R \textbf{24} al den] aldin M  $\cdot$ niht] er niht O \textbf{25} enrîte] rite L (R)  $\cdot$ dem] \textit{om.} R \textbf{26} Gawanes] Gawans I M R Z Gawan O Gawansz L  $\cdot$ in] vf Z \textbf{27} Gienc dar sante der werde man R \textbf{28} im] in I [in]: im L \newline
\end{minipage}
\hspace{0.5cm}
\begin{minipage}[t]{0.5\linewidth}
\small
\begin{center}*T
\end{center}
\begin{tabular}{rl}
 & \textbf{waz} mahte si vor der diet sô balt?\\ 
 & daz tet diu minne junc unde alt.\\ 
 & Lybaut \textbf{nû sînen willen} sach,\\ 
 & wand im sô liebe nie geschach,\\ 
5 & \textbf{sît} \textbf{in got der êren} niht erliez,\\ 
 & \textit{sîne tohter er dô vrouwe hiez}.\\ 
 & \textit{\begin{large}W\end{large}ie} diu hôchgezît ergienc,\\ 
 & des vrâget den, der gâbe enpfienc,\\ 
 & unde \textbf{wer} \textbf{manlîch dô} rite,\\ 
10 & er hete gemach oder \textbf{er} strite,\\ 
 & des \textbf{en}mag ich \textbf{dehein} ende hân.\\ 
 & Man sagete mir, daz Gawan\\ 
 & \textbf{nam urloup} ûf dem palas,\\ 
 & dar er durch urloup komen was.\\ 
15 & \textbf{daz was Obylote leit},\\ 
15 & wand si grôz weinen niht vermeit.\\ 
15 & Dô sprach si: "hêrre, sît ich bin\\ 
 & \textbf{iuwer}, \textbf{sô vüeret mich mit iu hin}."\\ 
 & Dô wart der \textbf{jungen}, \textbf{süezen} maget\\ 
 & diu bete von Gawane \textbf{widersaget}.\\ 
 & ir muoter si kûme von im \textbf{gebrach}.\\ 
20 & urloup er dô zin allen sprach.\\ 
 & Lybaut im \textbf{dankete} genuoc,\\ 
 & wander im holdez herze truoc.\\ 
 & Tscherules, sîn stolzer wirt,\\ 
 & mit alden sînen niht verbirt,\\ 
25 & ern rîte ûz mit dem degene balt.\\ 
 & Gawans strâze \textbf{in} einen walt\\ 
 & gienc. \textbf{dô} santer weideman\\ 
 & \textbf{mit} spîse verre mit \textbf{im} dan.\\ 
 & urloup nam der werde \textit{helt}\\ 
30 & Gawan, \textbf{der} gegen kumber was verselt.\\ 
\end{tabular}
\scriptsize
\line(1,0){75} \newline
T V W \newline
\line(1,0){75} \newline
\textbf{1} \textit{Initiale} W  \textbf{7} \textit{Initiale} T  \textbf{15} \textit{Majuskel} T  \textbf{17} \textit{Majuskel} T  \textbf{23} \textit{Majuskel} T  \newline
\line(1,0){75} \newline
\textbf{1} waz] DAs W \textbf{3} Lybaut] Libaut V Lybout W  $\cdot$ nû] do V \textbf{5} sît] Das W \textbf{6} \textit{Vers 397.6 fehlt} T  \textbf{7} Wie] Do T  $\cdot$ hôchgezît] hochzit W  $\cdot$ ergienc] zergieng W \textbf{8} gâbe] gabe do V do gabe W \textbf{9} dô] \textit{om.} W \textbf{11} dehein] nit ein W  $\cdot$ hân] nv han V \textbf{15} was] \textit{om.} W  $\cdot$ Obylote] obiloten V obylot W \textbf{17} jungen süezen] suͤssen iungen V \textbf{18} Gawane] gawan W  $\cdot$ widersaget] versaget V gar versaget W \textbf{21} Lybaut] Lẏppaut V Lybout W \textbf{23} Tscherules] Tscervles T Scherules V \textbf{25} ern rîte ûz] Er ritte W  $\cdot$ dem] den W \textbf{26} in] vf V \textbf{27} dô santer] dar sante V der panter W \textbf{28} mit] Vnde V (W) \textbf{29} helt] \textit{om.} T \textbf{30} der] \textit{om.} V W \newline
\end{minipage}
\end{table}
\end{document}
