\documentclass[8pt,a4paper,notitlepage]{article}
\usepackage{fullpage}
\usepackage{ulem}
\usepackage{xltxtra}
\usepackage{datetime}
\renewcommand{\dateseparator}{.}
\dmyyyydate
\usepackage{fancyhdr}
\usepackage{ifthen}
\pagestyle{fancy}
\fancyhf{}
\renewcommand{\headrulewidth}{0pt}
\fancyfoot[L]{\ifthenelse{\value{page}=1}{\today, \currenttime{} Uhr}{}}
\begin{document}
\begin{table}[ht]
\begin{minipage}[t]{0.5\linewidth}
\small
\begin{center}*D
\end{center}
\begin{tabular}{rl}
\textbf{387} & \begin{large}D\end{large}es kom Melyacanz in nôt,\\ 
 & \textbf{daz} im der werde Lanzilot\\ 
 & nie sô vaste zuo getrat,\\ 
 & dô er von der swertbrücke pfat\\ 
5 & kom \textbf{unt} dâ nâch mit im streit.\\ 
 & im was \textbf{gevancnüsse} leit,\\ 
 & die vrou Ginover dolte,\\ 
 & die er dâ mit strîte holte.\\ 
 & Dô punierte Lotes sun.\\ 
10 & waz \textbf{mohte} Meljacanz nû tuon,\\ 
 & erne \textbf{tribe} \textbf{ouch} daz ors mit sporn dar?\\ 
 & vil liute nam \textbf{der} tjoste war.\\ 
 & wer \textbf{dâ} hinderm orse \textbf{læge}?\\ 
 & den der von Norwæge\\ 
15 & gevellet hete ûf \textbf{die ouwe}.\\ 
 & manec ritter unt vrouwe\\ 
 & dise tjost ersâhen,\\ 
 & \textbf{die} Gawane prîses jâhen.\\ 
 & den vrouwen ez guot ze sehen was\\ 
20 & her nider von dem palas:\\ 
 & Melyacanz wart getretet,\\ 
 & durch \textbf{sîn kursît} gewetet\\ 
 & maneg ors, daz \textbf{sît} nie gruose enbeiz,\\ 
 & \textbf{ez reis} ûf in der bluotec sweiz.\\ 
25 & dâ ergienc der orse schelmtac,\\ 
 & dâr nâch den gîren ir bejac.\\ 
 & dô nam der herzoge Astor\\ 
 & Melyacanzen \textbf{von} \textbf{Jamor}.\\ 
 & \textbf{der was vil} nâch gevangen.\\ 
30 & der turnei was ergangen.\\ 
\end{tabular}
\scriptsize
\line(1,0){75} \newline
D \newline
\line(1,0){75} \newline
\textbf{1} \textit{Initiale} D  \textbf{9} \textit{Majuskel} D  \newline
\line(1,0){75} \newline
\textbf{9} Lotes] Lots D \textbf{10} Meljacanz] Meliacanz D \textbf{15} die] de D \newline
\end{minipage}
\hspace{0.5cm}
\begin{minipage}[t]{0.5\linewidth}
\small
\begin{center}*m
\end{center}
\begin{tabular}{rl}
 & des kam Melia\textit{ga}nz in nôt,\\ 
 & \textbf{daz} ime der werde Lanzelot\\ 
 & \textit{\begin{large}N\end{large}}\textit{i}e sô vaste zuo getrat,\\ 
 & dô er von der swertbrücke pfat\\ 
5 & kam \textbf{und} dâ nâch mit im streit.\\ 
 & im was \textbf{gevancnüsse} leit,\\ 
 & die vrouwe Ginove\textit{r} dolte,\\ 
 & \textit{d}ie er d\textit{â} mi\textit{t} strîte holte.\\ 
 & d\textit{ô} punierte Lotes sun.\\ 
10 & waz \textbf{mohte} Mel\textit{i}a\textit{ga}nz nû tuon,\\ 
 & er en\textbf{tribe} \textbf{noch} daz ros mit sporn dar?\\ 
 & vil liute nam \textbf{der} juste war.\\ 
 & wer \textbf{d\textit{â}} hinder dem rosse \textbf{læge}?\\ 
 & den der von Norwæge\\ 
15 & gevellet hete ûf \textbf{die ouwe}.\\ 
 & manic ritter und vrouwe\\ 
 & dise just ersâhen,\\ 
 & \textbf{die} Gawane prîses jâhen.\\ 
 & \multicolumn{1}{l}{ - - - }\\ 
20 & \multicolumn{1}{l}{ - - - }\\ 
 & Mel\textit{i}a\textit{ga}nz wart getret,\\ 
 & durch \textbf{sîn kursît} gewet\\ 
 & manic ros, daz nie gruose enbeiz.\\ 
 & \textbf{e\textit{z} reis} ûf in der bluotic sweiz.\\ 
25 & d\textit{â} ergienc der rosse schelmentac,\\ 
 & dâr nâch den gîren ir bejac.\\ 
 & dô nam der herzoge Astor\\ 
 & Meliagan\textit{z} \textbf{de} \textbf{Jamor}.\\ 
 & \textbf{der was vil} nâc\textit{h g}evangen.\\ 
30 & der turnei was ergangen.\\ 
\end{tabular}
\scriptsize
\line(1,0){75} \newline
m n o \newline
\line(1,0){75} \newline
\textbf{3} \textit{Initiale} m  \newline
\line(1,0){75} \newline
\textbf{1} Meliaganz] meliancz m o meliantz n \textbf{2} Lanzelot] lancze lot m lantzelot n lanczelot o \textbf{3} Nie] ME m (n) \textbf{5} dâ nâch] dannoch n o \textbf{7} Ginover] ginofere m genofere n o  $\cdot$ dolte] dolt n o \textbf{8} die] Sẏe m  $\cdot$ dâ] do m n o  $\cdot$ mit] mitte m  $\cdot$ holte] holt n o \textbf{9} dô] Du m Da o \textbf{10} mohte] moͯchte n  $\cdot$ Meliaganz] meleancz m meliantz n meliancze o \textbf{11} entribe] treip n (o) \textbf{13} dâ] do m n o \textbf{14} Norwæge] norwege m n o \textbf{17} ersâhen] sohen n \textbf{18} die] Vnd n o \textbf{19} \textit{Die Verse 387.19-20 fehlen} m n o  \textbf{21} Meliaganz] Meleancz m Meliantz n Melancz o \textbf{23} gruose] grasz n \textbf{24} ez] Er m  $\cdot$ in der] an die o  $\cdot$ bluotic] [brutigen]: blutigen o \textbf{25} dâ] Do m n o \textbf{26} ir] iren n \textbf{27} herzoge] herczeleide o \textbf{28} Meliaganz] Meliagancze m Meliantz n Meliancz o  $\cdot$ Jamor] iamor m \textbf{29} nâch gevangen] nach gevarn vnd gefangen m \newline
\end{minipage}
\end{table}
\newpage
\begin{table}[ht]
\begin{minipage}[t]{0.5\linewidth}
\small
\begin{center}*G
\end{center}
\begin{tabular}{rl}
 & des kom Meliahganz in nôt,\\ 
 & \textbf{wan} im der werde Lanzelot\\ 
 & nie sô vaste zuo getrat,\\ 
 & dô er von der swertbrücke pfat\\ 
5 & \begin{large}K\end{large}om, dar nâch mit im streit.\\ 
 & im was \textbf{vancnisse} leit,\\ 
 & die vrô Schinovere dolte,\\ 
 & die er dâ mit strîte holte.\\ 
 & dô pungierte Lotes sun.\\ 
10 & waz \textbf{mac} Meliahganz nû tuon,\\ 
 & ern \textbf{trîbe} \textbf{ouch} daz ors mit sporen dar?\\ 
 & vil liute nam \textbf{ir} tjoste war.\\ 
 & wer hinderm orse \textbf{gelæge}?\\ 
 & den der von Norwæge\\ 
15 & gevalt het ûf \textbf{die ouwe}.\\ 
 & manic rîter unde vrouwe\\ 
 & dise tjoste \textit{er}sâhen.\\ 
 & Gawane \textbf{si} prîses jâhen.\\ 
 & den vrouwen ez guot ze sehene was\\ 
20 & her nider von dem palas,\\ 
 & \textbf{daz} Meliahganz wart getretet,\\ 
 & durch \textbf{sînen wâpenroc} gewetet\\ 
 & manic ors, daz \textbf{sît} nie gruose enbeiz.\\ 
 & \textbf{dâ viel} ûf in der bluotic sweiz,\\ 
25 & dâ ergienc der orse schelmetac,\\ 
 & dâr nâch den gîren ir bejac.\\ 
 & dô nam der herzoge Astor\\ 
 & Meliahkanzen \textbf{den von} \textbf{Amor}.\\ 
 & \textbf{die heten in} nâch gevangen.\\ 
30 & der turnei was ergangen.\\ 
\end{tabular}
\scriptsize
\line(1,0){75} \newline
G I O L M Q R Z \newline
\line(1,0){75} \newline
\textbf{1} \textit{Initiale} I L Z   $\cdot$ \textit{Capitulumzeichen} R  \textbf{5} \textit{Initiale} G  \newline
\line(1,0){75} \newline
\textbf{1} \textit{Die Verse 370.13-412.12 fehlen} Q   $\cdot$ Meliahganz] melianz I Melyahkanz O Meliahkanz L (Z) Meliachkancz M Meleakancz R \textbf{2} wan] Daz Z  $\cdot$ im] in L  $\cdot$ Lanzelot] Lanzilot R lantzilot Z \textbf{4} dô] Da Z  $\cdot$ der] \textit{om.} L  $\cdot$ swertbrücke] swere pruche I \textbf{5} Kom] Chom vnde O (L) (M) (R) (Z)  $\cdot$ dar nâch] dannoch O  $\cdot$ im] in O  $\cdot$ streit] stritten R \textbf{6} vancnisse] die vangenisze L gevengnisse M (R) (Z)  $\cdot$ leit] mitten R \textbf{7} die] diu I (O) Do R  $\cdot$ Schinovere] tschinovere G Ginoferre I Ginover O (M) Gynoviere L Gynouer R Gynofere Z  $\cdot$ dolte] dulte M \textbf{8} dâ] do R  $\cdot$ strîte] prise I \textbf{9} dô] Da M Z  $\cdot$ Lotes] lodes M \textbf{10} mac] mohte Z  $\cdot$ Meliahganz] Melianz I Melyahkanz O Meliahkanz L (Z) Meliachkancz M Meleachancz R \textbf{11} ern] Er O R  $\cdot$ ouch] \textit{om.} M R \textbf{12} nam] namen O (L) (M) Z  $\cdot$ ir] der R \textbf{13} wer] Wer da O L M R Z  $\cdot$ gelæge] gelegen M \textbf{14} den der] Von der O Wer den R  $\cdot$ Norwæge] norwage G norwege I (O) (L) (M) R (Z) \textbf{16} manic] vil manc I  $\cdot$ rîter] striter M \textbf{17} dise] Die L M  $\cdot$ ersâhen] sahen G sy sachen R \textbf{18} Gawane] Gawan I O L R Z  $\cdot$ jâhen] gaben R \textbf{19} ze sehene] zesechent R \textbf{20} her] Hin R Z \textbf{21} daz] \textit{om.} Z  $\cdot$ Meliahganz] mæliahkanz G Meliaganz I melyakanz O Meliahkanz L Meliachkanz M Meliachancz R Meliahkantz Z  $\cdot$ getretet] getreit M \textbf{23} manic] mange G  $\cdot$ gruose] gras I grᵫsch R \textbf{24} dâ] So R  $\cdot$ bluotic] bluͤtes I \textbf{25} schelmetac] suͤn tac I schelmig tag R \textbf{26} bejac] bach O \textbf{27} dô] Da O L M Z  $\cdot$ Astor] Castor R \textbf{28} Meliahkanzen] [Melianzen]: Meliganzen I Melyakanzen O Meliachkanczen M Meleachkanczen R Meliahkantzen Z  $\cdot$ den] \textit{om.} O  $\cdot$ Amor] Jamor O R Z Lamor L iamor M \textbf{29} Der was vilnach gevangen Z  $\cdot$ nâch] \textit{om.} L noch M \textbf{30} ergangen] zergangen R \newline
\end{minipage}
\hspace{0.5cm}
\begin{minipage}[t]{0.5\linewidth}
\small
\begin{center}*T
\end{center}
\begin{tabular}{rl}
 & des kom Melyahganz in nôt,\\ 
 & \textbf{wand}im der werde Lanzelot\\ 
 & nie sô vaste zuo getrat,\\ 
 & d\textit{ô}r von der swertbrücke pfat\\ 
5 & kom \textbf{unde} dâ nâch mit im streit.\\ 
 & im was \textbf{di\textit{u}} \textbf{gevancnisse} leit,\\ 
 & di\textit{e} vrou Gynover dolte,\\ 
 & dier dâ mit strîte holte.\\ 
 & \begin{large}D\end{large}ô punierte Lotes suon.\\ 
10 & waz \textbf{mac} Melyahganz nû tuon,\\ 
 & ern \textbf{trîbe} \textbf{ouch} daz ors mit sporn dar?\\ 
 & vil liute nam \textbf{ir} tjost war.\\ 
 & wer hinderm orse \textbf{læge}?\\ 
 & den der von Norwæge\\ 
15 & \textit{gevellet} hete ûf \textbf{dem touwe}.\\ 
 & manec rîter unde vrouwe\\ 
 & dise tjost ersâhen.\\ 
 & Gawane \textbf{si} prîses jâhen.\\ 
 & den vrouwen ez guot ze sehenne was\\ 
20 & her nider von dem palas,\\ 
 & \textbf{daz} Melyahganz wart getret,\\ 
 & durch \textbf{sînen wâpenroc} gewet\\ 
 & \hspace*{-.7em}\big| - \textbf{dô viel} ûf in der bluotic sweiz -\\ 
 & \hspace*{-.7em}\big| manec ors, daz \textbf{sît} nie gruose enbeiz.\\ 
25 & dâ ergie der orse schelmetac,\\ 
 & dâ nâch den gîren ir bejac.\\ 
 & Dô nam der herzoge Astor\\ 
 & Melyahganzen \textbf{den vo\textit{n}} \textbf{Jammor}.\\ 
 & \textbf{der was vil} nâch gevangen.\\ 
30 & der turnei was ergangen.\\ 
\end{tabular}
\scriptsize
\line(1,0){75} \newline
T V W \newline
\line(1,0){75} \newline
\textbf{1} \textit{Initiale} W  \textbf{9} \textit{Initiale} T  \textbf{27} \textit{Majuskel} T  \newline
\line(1,0){75} \newline
\textbf{1} Melyahganz] melẏanz V meliaganz W \textbf{2} wandim] [*]: Daz im V  $\cdot$ Lanzelot] lanzelôt T \textbf{4} dôr von] der von T Do er der W \textbf{5} dâ nâch] dannoch W \textbf{6} diu] die T \textit{om.} V W  $\cdot$ gevancnisse leit] wene misselait W \textbf{7} die] div T  $\cdot$ Gynover] Gẏnouer V gynouer W \textbf{8} dâ] do V W \textbf{10} mac] [*]: moͤhte V  $\cdot$ Melyahganz] melẏanz V meliaganz W \textbf{11} ern trîbe ouch] Er trieb W \textbf{12} ir] oͮch ir V \textbf{13} Wer do hinderm ros gelege W  $\cdot$ læge] [*]: do lege V \textbf{14} Norwæge] Norwege T (V) (W) \textbf{15} gevellet] \textit{om.} T  $\cdot$ hete] hat W  $\cdot$ dem touwe] die oͮwe V (W) \textbf{18} Gawane] Gawe T [*]: Die gawane V  $\cdot$ si] \textit{om.} V \textbf{21} Melyahganz] melẏaganz V meliaganz W \textbf{24} \textit{Versfolge 387.23-24} W   $\cdot$ dô viel] [*]: Ez reis V \textbf{23} sît nie] nie seit W \textbf{25} dâ] [D*]: Do V Do W  $\cdot$ schelmetac] schelmeslag V \textbf{27} Astor] kastor W \textbf{28} Melyahganzen] melyahgansen T Melianzen V Meliaganzen W  $\cdot$ von] vo T  $\cdot$ Jammor] [*]: iamor V iamor W \textbf{29} der was vil nâch] Die hetten in nahe W \newline
\end{minipage}
\end{table}
\end{document}
