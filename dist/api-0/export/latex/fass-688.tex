\documentclass[8pt,a4paper,notitlepage]{article}
\usepackage{fullpage}
\usepackage{ulem}
\usepackage{xltxtra}
\usepackage{datetime}
\renewcommand{\dateseparator}{.}
\dmyyyydate
\usepackage{fancyhdr}
\usepackage{ifthen}
\pagestyle{fancy}
\fancyhf{}
\renewcommand{\headrulewidth}{0pt}
\fancyfoot[L]{\ifthenelse{\value{page}=1}{\today, \currenttime{} Uhr}{}}
\begin{document}
\begin{table}[ht]
\begin{minipage}[t]{0.5\linewidth}
\small
\begin{center}*D
\end{center}
\begin{tabular}{rl}
\textbf{688} & \begin{large}N\end{large}iht ze kranc zwei \textbf{vröuwelîn},\\ 
 & diu truogen \textbf{êt} \textbf{dâ} den besten schîn,\\ 
 & unders küneges \textbf{starken} armen riten.\\ 
 & dô\textbf{ne} wart niht langer \textbf{dâ} gebiten:\\ 
5 & Artuses boten vuoren dan\\ 
 & unt kômen \textbf{dar}, dâ Gawan\\ 
 & ûf ir widerreise streit.\\ 
 & dô wart den kinden \textbf{nie sô} leit,\\ 
 & \textbf{si} schrîten lûte \textbf{umbe sîne} nôt,\\ 
10 & wande in ir triwe daz gebôt.\\ 
 & Ez was vil nâch alsô komen,\\ 
 & daz den sig \textbf{hete al dâ} genomen\\ 
 & Gawans kampfgenôz.\\ 
 & des kraft was über in sô grôz,\\ 
15 & \textbf{daz} Gawan, der werde degen,\\ 
 & des \textbf{siges} \textbf{hete nâch} verpflegen,\\ 
 & wan daz in klagende nanten\\ 
 & kint, \textbf{diu} in \textbf{bekanten}.\\ 
 & Der \textbf{ê} \textbf{des} was \textbf{sîn} strîtes wer,\\ 
20 & verbar \textbf{dô} gein im strîtes ger.\\ 
 & \textbf{verre ûz der hant er} warf daz swert.\\ 
 & "unsælec unt unwert\\ 
 & bin ich", sprach der \textbf{weinde} gast.\\ 
 & "aller sælden mir gebrast,\\ 
25 & daz mîner \textbf{geunêrten} hant\\ 
 & dirre strît ie wart bekant.\\ 
 & \textbf{des} was \textbf{mit unvuoge ir} ze vil.\\ 
 & schuldec ich mich geben wil.\\ 
 & hie trat mîn ungelücke vür\\ 
30 & unt schiet mich von der sælden kür.\\ 
\end{tabular}
\scriptsize
\line(1,0){75} \newline
D \newline
\line(1,0){75} \newline
\textbf{1} \textit{Initiale} D  \textbf{11} \textit{Majuskel} D  \textbf{19} \textit{Majuskel} D  \newline
\line(1,0){75} \newline
\textbf{5} Artuses] Artvs D \newline
\end{minipage}
\hspace{0.5cm}
\begin{minipage}[t]{0.5\linewidth}
\small
\begin{center}*m
\end{center}
\begin{tabular}{rl}
 & niht zuo kranc zwei \textbf{vröuwelîn},\\ 
 & diu truogen den besten schîn,\\ 
 & under des küniges \textbf{starken} armen riten.\\ 
 & dô \textbf{en}w\textit{a}rt niht lenger gebiten:\\ 
5 & Artuses boten vuoren dan\\ 
 & und kômen \textbf{dar}, d\textit{â} Gawan\\ 
 & ûf i\textit{r w}ider\textit{reise} streit.\\ 
 & dô wart de\textit{n} k\textit{ind}e\textit{n} leit,\\ 
 & \textbf{si} schrîten lûte \textbf{umb sîne} nôt,\\ 
10 & wan in ir triuwe daz gebôt.\\ 
 & ez was vil nâch alsô komen,\\ 
 & daz den sic \textbf{het aldâ} genomen\\ 
 & Gawanes kampfgenôz.\\ 
 & des kraft was über in sô grôz,\\ 
15 & \textbf{daz} Gawan, der werde degen,\\ 
 & des \textbf{sigens} \textbf{hete nâch} verpflegen,\\ 
 & wan daz in klagend\textit{e} nanten\\ 
 & \textbf{diu} kint, \textbf{diu} in \textbf{bekanten}.\\ 
 & der \textbf{ê} \textbf{des} was \textbf{sînes} strîtes wer,\\ 
20 & verbar \textbf{dô} gegen im strîtes ger.\\ 
 & \textbf{verre ûz der hant er} warf daz swert.\\ 
 & "unsælic und unwert\\ 
 & bin ich", sprach der \textbf{vremde} gast.\\ 
 & "aller sælden mir gebrast,\\ 
25 & daz mîner \textbf{geunêrten} hant\\ 
 & diser strît ie wart bekant.\\ 
 & \textbf{daz} was \textbf{mit unvuoge ir} zuo vil.\\ 
 & schuldic ich mich geben wil.\\ 
 & hie trat mîn ungelücke vür\\ 
30 & und schiet mich von der \dag selben\dag  kür.\\ 
\end{tabular}
\scriptsize
\line(1,0){75} \newline
m n o \newline
\line(1,0){75} \newline
\newline
\line(1,0){75} \newline
\textbf{1} kranc] krancz o \textbf{3} armen riten] arme retten n \textbf{4} enwart] enwert m  $\cdot$ gebiten] gebetten n \textbf{5} Artuses] Artúses o \textbf{6} dâ] do m n o \textbf{7} Vff ir streit wider streit m  $\cdot$ streit] strit o \textbf{8} den kinden] dem kunige m den kinde o \textbf{9} lûte] luͦten o \textbf{10} in] \textit{om.} n \textbf{15} Gawan] gawen n \textbf{16} sigens] siges n tegen o \textbf{17} klagende] clagendes m \textbf{18} bekanten] bekante o \textbf{19} ê] >e< o \textbf{20} im] yme do n in o \textbf{21} ûz] uͯff o \textbf{27} mit] nit o \newline
\end{minipage}
\end{table}
\newpage
\begin{table}[ht]
\begin{minipage}[t]{0.5\linewidth}
\small
\begin{center}*G
\end{center}
\begin{tabular}{rl}
 & \begin{large}N\end{large}iht ze kranc zwei \textbf{junchêrrelîn},\\ 
 & diu truogen \textbf{êt} \textbf{dâ} den besten schîn,\\ 
 & unders küniges \textbf{starken} armen riten.\\ 
 & dô\textbf{ne} wart niht langer \textbf{dâ} gebiten:\\ 
5 & Artuses boten vuoren dan\\ 
 & unde kômen \textbf{hin}, dâ Gawan\\ 
 & ûf ir widerreise streit.\\ 
 & dô\textbf{ne} wart den kinden \textbf{nie sô} leit,\\ 
 & \textbf{die} schrîten lûte \textbf{sîner} nôt,\\ 
10 & wan in ir triwe daz gebôt.\\ 
 & ez was \textbf{êt} vil nâch alsô komen,\\ 
 & daz den sic \textbf{dâ hete} genomen\\ 
 & Gawans kampfgenôz.\\ 
 & des kraft was über in sô grôz,\\ 
15 & \textbf{dô} Gawan, der werde degen,\\ 
 & des \textbf{siges} \textbf{nâch hete} ve\textit{r}pflegen,\\ 
 & wan daz in klagende nanden\\ 
 & kint, \textbf{dô si}n \textbf{erkanden}.\\ 
 & der \textbf{ê} was \textbf{sînes} strîtes wer,\\ 
20 & \textbf{der} verbar \textit{\textbf{dô}} \textit{gein im} strîtes ger.\\ 
 & \textbf{ûz der hende er verre} warf daz swert.\\ 
 & "unsælic unde unwert\\ 
 & bin ich", sprach der \textbf{werde} gast.\\ 
 & "aller sælden mir gebrast,\\ 
25 & daz mîner \textbf{sigelôsen} hant\\ 
 & dirre strît ie wart bekant.\\ 
 & \textbf{dâ} was \textbf{mîner ungevuoge} ze vil.\\ 
 & schuldic ich mich geben wil.\\ 
 & hie trat mîn ungelücke vür\\ 
30 & unde schiet mich von der sælden kür.\\ 
\end{tabular}
\scriptsize
\line(1,0){75} \newline
G I L M Z Fr20 Fr52 \newline
\line(1,0){75} \newline
\textbf{1} \textit{Initiale} G I L Z Fr20  \textbf{23} \textit{Initiale} I  \newline
\line(1,0){75} \newline
\textbf{1} Niht] ÷iht Fr20  $\cdot$ junchêrrelîn] frewelin Z jvncfrowelin Fr20 \textbf{2} diu] die I  $\cdot$ êt] \textit{om.} M Z  $\cdot$ den] \textit{om.} L Fr20 \textbf{3} starken] strachin Fr20  $\cdot$ armen riten] arm geriten I \textbf{4} dône wart] Da enwart M Z  $\cdot$ dâ] [*]: do L do Fr20 \textbf{5} Artuses] Artus G M Z (Fr20) Artuͯs L \textbf{6} hin] da hin I \textbf{8} dône wart] Da en wart M (Z) \textbf{9} schrîten] sriren I  $\cdot$ sîner] von siner I vber sin L (M) vmb sine Z \textbf{10} in] \textit{om.} M Z \textbf{11} êt] \textit{om.} M Z  $\cdot$ komen] \textit{om.} I \textbf{13} Gawans] Gawanz L \textbf{14} des] Der M  $\cdot$ grôz] grozze Z \textbf{15} dô] Da M Z \textbf{16} hete] \textit{om.} M  $\cdot$ verpflegen] vephlegen G \textbf{17} in] si in I  $\cdot$ klagende] [clagenden]: clagen den L \textbf{18} kint] diu chint I (L)  $\cdot$ dô sin] do si in wol I in do L da sy en M (Z) \textbf{20} dô gein im] gein im do G da geyn yme M \textbf{21} er verre] verr er I er L Fr52 verre Z \textbf{23} werde] fremde L M Z \textit{om.} Fr52 \textbf{24} sælden] salde L (Z) \textbf{25} sigelôsen] gvnerten Z \textbf{26} bekant] erchant I (M) \textbf{27} Des was mit vnfuge ir zv vil Z  $\cdot$ dâ] daz I (L) (M)  $\cdot$ ungevuoge] vnfuge L M (Fr20)  $\cdot$ ze] \textit{om.} Fr20 \textbf{28} geben] selben L \textbf{29} trat] treit Fr20 \textbf{30} sælden] salde L selben Z  $\cdot$ kür] tor M \newline
\end{minipage}
\hspace{0.5cm}
\begin{minipage}[t]{0.5\linewidth}
\small
\begin{center}*T
\end{center}
\begin{tabular}{rl}
 & niht zuo kranc zwei \textbf{vröuwelîn},\\ 
 & diu truogen \textbf{d\textit{â}} den besten schîn,\\ 
 & under des küneges \textbf{stracken} armen riten.\\ 
 & dô wart niht langer \textbf{d\textit{â}} gebiten:\\ 
5 & Artuses boten vuoren dan\\ 
 & und kâmen \textbf{hin}, dâ Gawan\\ 
 & ûf ir widerreise streit.\\ 
 & dô \textbf{en}wart den kinden \textbf{ni\textit{e sô}} \textit{leit},\\ 
 & \textbf{die} schrouwen lûte \textbf{umb sîn} nôt,\\ 
10 & wan in ir triuwe daz gebôt.\\ 
 & ez was vil nâch alsô komen,\\ 
 & daz den sic \textbf{d\textit{â} hete} genomen\\ 
 & Gawans kampfgenôz.\\ 
 & des k\textit{r}a\textit{ft} was über in sô grôz,\\ 
15 & \textbf{daz} Gawan, der werde degen,\\ 
 & des \textbf{siges} \textbf{sich} \textbf{hete nâch} verpflegen,\\ 
 & wan daz in klagende nanten\\ 
 & kint, \textbf{dô si} in \textbf{erkanten}.\\ 
 & \begin{large}D\end{large}er was \textbf{sînes} strîtes wer,\\ 
20 & \textbf{der} verbar gein im strîtes ger.\\ 
 & \textbf{ûz der hende er verre} warf daz swert.\\ 
 & "unsælic und unwert\\ 
 & bin ich", sprach der \textbf{vremede} gast.\\ 
 & "aller sælden mir gebrast,\\ 
25 & daz mîner \textbf{geunêrter} hant\\ 
 & dirre strît ie wart bekant.\\ 
 & \textbf{daz} was \textbf{mîner ungevuoge} zuo vil.\\ 
 & schuldic ich mich geben wil.\\ 
 & hie trat mîn ungelücke vür\\ 
30 & und schiet mich von der sælden kür.\\ 
\end{tabular}
\scriptsize
\line(1,0){75} \newline
U V W Q R \newline
\line(1,0){75} \newline
\textbf{1} \textit{Initiale} W  \textbf{5} \textit{Initiale} R  \textbf{19} \textit{Initiale} U V  \newline
\line(1,0){75} \newline
\textbf{2} dâ] do U V W auch do Q ech do R \textbf{3} stracken] starken V (Q) R \textit{om.} W  $\cdot$ armen riten] armeriten Q \textbf{4} wart] enward W (Q)  $\cdot$ langer] lang R  $\cdot$ dâ] do U V Q \textit{om.} W  $\cdot$ gebiten] gebieten Q \textbf{5} Artuses] Kúnig artus W Artus Q (R) \textbf{6} hin] [*]: dar V  $\cdot$ dâ] do V W Q  $\cdot$ Gawan] herr gawan W \textbf{8} enwart] ward R  $\cdot$ nie sô leit] nit U \textbf{9} die] Sy R  $\cdot$ schrouwen] schruͦwen U schreyen W scritten R  $\cdot$ umb] v́ber R \textbf{11} was] was auch Q was echt R \textbf{12} dâ] do U V W Q R \textbf{13} Gawans] Gawanes V Her gawans W Gaweins R  $\cdot$ kampfgenôz] kampfen gnos R \textbf{14} kraft] kamp U [*]: kraft V  $\cdot$ über in] im W in R \textbf{15} Gawan] herr gawan W Gawin R  $\cdot$ werde] stoltze W \textbf{16} sich] \textit{om.} W Q R  $\cdot$ hete nâch] nahe hette W (Q) (R)  $\cdot$ verpflegen] verwegen R \textbf{17} in] sy in W  $\cdot$ nanten] manden W \textbf{18} kint] Die kint V (W) Kund R  $\cdot$ dô si] do [si*]: sie U [*]: die V  $\cdot$ in] \textit{om.} R \textbf{19} Der] Der e V (W) (Q) (R)  $\cdot$ was sînes] [*]: dez waz sin V \textbf{20} der] \textit{om.} V  $\cdot$ verbar] [*]: verbar do V verbarg W verbar do Q R \textbf{21} ûz] Vser R  $\cdot$ er verre warf] verre warff er W verwarff do R  $\cdot$ daz] sin R \textbf{24} aller] Aller der W \textbf{25} daz] Do R  $\cdot$ geunêrter] gevnerten V (W) Q \textbf{26} ie] \textit{om.} W Q R \textbf{27} ungevuoge] vnvuͦge niht V vnfuͦge W (Q) (R) \textbf{29} trat] tranck Q \textbf{30} kür] schúr R \newline
\end{minipage}
\end{table}
\end{document}
