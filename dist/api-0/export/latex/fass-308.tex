\documentclass[8pt,a4paper,notitlepage]{article}
\usepackage{fullpage}
\usepackage{ulem}
\usepackage{xltxtra}
\usepackage{datetime}
\renewcommand{\dateseparator}{.}
\dmyyyydate
\usepackage{fancyhdr}
\usepackage{ifthen}
\pagestyle{fancy}
\fancyhf{}
\renewcommand{\headrulewidth}{0pt}
\fancyfoot[L]{\ifthenelse{\value{page}=1}{\today, \currenttime{} Uhr}{}}
\begin{document}
\begin{table}[ht]
\begin{minipage}[t]{0.5\linewidth}
\small
\begin{center}*D
\end{center}
\begin{tabular}{rl}
\textbf{308} & \begin{large}D\end{large}ô truoc der junge Parzival\\ 
 & âne \textbf{vlügel} engels mâl\\ 
 & sus g\textit{eb}lüet ûf der erden.\\ 
 & Artus mit den werden\\ 
5 & enpfieng in \textbf{minneclîche}.\\ 
 & guotes willen wâren rîche\\ 
 & alle, die in \textbf{gesâhen} dâ.\\ 
 & ir herzen volge sprâchen jâ,\\ 
 & gein sîme lobe sprach niemen nein,\\ 
10 & sô rehte minneclîch er schein.\\ 
 & Artus sprach z\textbf{em Waleise} sân:\\ 
 & "ir habt mir lieb und leit getân,\\ 
 & doch habt ir mir der êre\\ 
 & brâht unt gesendet mêre,\\ 
15 & denne ich ir ie von manne enpfienc.\\ 
 & \textbf{dâ engein mîn dienst} noch kleine gienc,\\ 
 & Het ir prîses \textbf{nimer} getân,\\ 
 & wan daz diu herzogîn sol hân,\\ 
 & vrou Jeschute, die hulde.\\ 
20 & ouch wære iu Keien schulde\\ 
 & gewandelt \textbf{unt gerochen},\\ 
 & het ich iuch ê gesprochen."\\ 
 & Artus sagete im, wes er bat,\\ 
 & warumbe er an die selben stat\\ 
25 & unt \textbf{ouch} mêr landes was geriten.\\ 
 & si begunden in dô alle biten,\\ 
 & daz er gelobte \textbf{sunder}\\ 
 & den von der tavelrunder\\ 
 & sîne rîterlîch gesellecheit.\\ 
30 & im was ir bete niht ze leit.\\ 
\end{tabular}
\scriptsize
\line(1,0){75} \newline
D \newline
\line(1,0){75} \newline
\textbf{1} \textit{Initiale} D  \textbf{17} \textit{Majuskel} D  \newline
\line(1,0){75} \newline
\textbf{3} geblüet] glvͤt D \textbf{19} Jeschute] Jescv̂te D \newline
\end{minipage}
\hspace{0.5cm}
\begin{minipage}[t]{0.5\linewidth}
\small
\begin{center}*m
\end{center}
\begin{tabular}{rl}
 & \begin{large}D\end{large}ô truoc der junge Parcifal\\ 
 & âne \dag vluges\dag  engels mâl\\ 
 & sus geblüet ûf der erden.\\ 
 & Artus mit den werden\\ 
5 & enpfienc in \textbf{minneclîche}.\\ 
 & guotes willen w\textit{â}ren rîche\\ 
 & alle, die in \textbf{gesâhen} dâ.\\ 
 & ir herzen volge sprâchen jâ,\\ 
 & gegen sînem lobe sprach niemen \textit{n}ein,\\ 
10 & sô rehte minneclîch er schein.\\ 
 & Artus sprach z\textbf{em Waleise} sân:\\ 
 & "ir habt mir liep und leit getân,\\ 
 & doch habt ir mir der êre\\ 
 & brâht und gesen\textit{d}e\textit{t} mêre,\\ 
15 & danne ich ir ie von manne enpfienc.\\ 
 & \textbf{dâ gegen mîn dienst} noch kleine gienc,\\ 
 & \dag hete ir prîs\dag  \textbf{niemêre} getân,\\ 
 & wanne daz diu herzogîn sol hân,\\ 
 & vrouwe Jeschute, die hulde.\\ 
20 & ouch wære iu Keien schulde\\ 
 & gewandelt \textbf{ungerochen},\\ 
 & hete ich iuch ê gesprochen."\\ 
 & Artus sagete ime, wes er bat,\\ 
 & warumbe er an d\textit{ie} selben stat\\ 
25 & und \textbf{ou\textit{ch}} \textit{m}ê landes was geriten.\\ 
 & si begunden in dô alle biten,\\ 
 & daz er \dag gloubete\dag  \textbf{s\textit{u}nder}\\ 
 & den von der tavelrunder\\ 
 & sîne ritterlîch gesellicheit.\\ 
30 & im was ir bete niht ze leit.\\ 
\end{tabular}
\scriptsize
\line(1,0){75} \newline
m n o \newline
\line(1,0){75} \newline
\textbf{1} \textit{Initiale} m   $\cdot$ \textit{Capitulumzeichen} n  \newline
\line(1,0){75} \newline
\textbf{1} truoc] truͦge n \textbf{2} vluges] fluͯgens n sluͦgez o \textbf{6} wâren] weren m \textbf{7} gesâhen] gesehen o  $\cdot$ dâ] do n \textbf{8} herzen] hertze jme n (o)  $\cdot$ sprâchen] gesprochen o \textbf{9} nein] mein m \textbf{10} er schein] erscheẏ o \textbf{11} Artus] Artuͯs o  $\cdot$ Waleise] waleisse m \textbf{14} gesendet] gesenten m \textbf{15} manne] \textit{om.} n \textbf{17} Hette ir [mẏnnre g]: pris mynre getan n  $\cdot$ niemêre] minre o \textbf{19} Jeschute] jescutte m jescute n o \textbf{20} Keien] keẏen n keyen o \textbf{21} ungerochen] vnd gerochen n o \textbf{23} Artus sagete ime] Sagete jme do artus n  $\cdot$ wes] was o \textbf{24} die] der m  $\cdot$ selben] selbe n o \textbf{25} ouch mê] ous nie m \textbf{27} sunder] sinder m \textbf{29} sîne] An o  $\cdot$ gesellicheit] geselleclich o \newline
\end{minipage}
\end{table}
\newpage
\begin{table}[ht]
\begin{minipage}[t]{0.5\linewidth}
\small
\begin{center}*G
\end{center}
\begin{tabular}{rl}
 & dô truoc der junge Parzival\\ 
 & âne \textbf{vlüge} engels mâl\\ 
 & sus geblü\textit{e}t ûf der erden.\\ 
 & Artus mit den werden\\ 
5 & enpfienc in \textbf{rîterlîche}.\\ 
 & guotes willen wâren rîche\\ 
 & alle, die in \textbf{sâhen} dâ.\\ 
 & ir herzen volg\textit{e} \textit{s}prâch\textit{en} jâ,\\ 
 & gein sînem lobe sprach niemen nein,\\ 
10 & sô rehte minniclîch er schein.\\ 
 & Artus sprach ze \textbf{im} sân:\\ 
 & "ir habet mir liep unde leit getân,\\ 
 & doch habet ir mir der êre\\ 
 & brâht unde gesendet mêre,\\ 
15 & danne ich ir ie von manne enpfienc.\\ 
 & \textbf{mîn dienst dâ gein} noch kleine gienc,\\ 
 & het ir brîses \textbf{niht mê} getân,\\ 
 & wan daz diu herzogîn sol hân,\\ 
 & vrou Jeschute, die hulde.\\ 
20 & ouch wære iu Kay\textit{n} schulde\\ 
 & gewandelt \textbf{ungerochen},\\ 
 & het ich iuch ê gesprochen."\\ 
 & Artus saget im, wes er bat,\\ 
 & warumbe er an die selben stat\\ 
25 & unde \textbf{ouch} mê landes was geriten.\\ 
 & si begunden in dô alle biten,\\ 
 & \begin{large}D\end{large}az er gelobte \textbf{sunder}\\ 
 & den von der tavelrunder\\ 
 & sîne rîterlîche \textit{gesellec}heit.\\ 
30 & im was ir bete niht ze leit.\\ 
\end{tabular}
\scriptsize
\line(1,0){75} \newline
G I O L M Q R Z \newline
\line(1,0){75} \newline
\textbf{1} \textit{Initiale} L  \textbf{11} \textit{Überschrift:} Hie kompt Artus parczifaln mit deren Tauelrunder vnd mit sinem Ingesind der messunge Vnd die kungin Wie artus vnd sin wib parczifaln empfiengen mit sinem volk / Wie artus vnd sin wib parczifaln empfiengen mit sinem volk R   $\cdot$ \textit{Initiale} O R Z  \textbf{17} \textit{Initiale} I  \textbf{27} \textit{Initiale} G  \newline
\line(1,0){75} \newline
\textbf{1} dô] So Q Da Z  $\cdot$ Parzival] parzifal I M Barcifal O parcifal L Z partzifal Q parczifal R \textbf{2} fluͦgengels mal I  $\cdot$ vlüge] flvgel O (L) (M) (Q) (R) Z \textbf{3} sus] Sv L  $\cdot$ geblüet] gebloͮmet G gelobet L  $\cdot$ ûf] vz M \textbf{6} willen] willes Q willens R  $\cdot$ wâren] weren M gein im Z \textbf{7} sâhen] gesahen O L (M) (Q) (R) Z  $\cdot$ dâ] do Q \textbf{8} ir] Jrs Q  $\cdot$ herzen] herze I (O) (L) (Z)  $\cdot$ volge sprâchen] volge div sprach G volge sprach en I volgen sprachen L Z \textbf{9} sprach] \textit{om.} I \textbf{10} sô] sus I (O) (L) (M) (Q) (Z)  $\cdot$ minniclîch] manlich O wunneclichen M \textbf{11} Artus] ÷rtvs O Artuͯs L  $\cdot$ sprach ze im] zvͦ im sprach O \textbf{13} der] die R \textbf{14} brâht] Gebracht Q  $\cdot$ gesendet] sendet M gesendet erre R \textbf{15} ir] \textit{om.} M sy R  $\cdot$ ie von manne] von manne ye Q (R) \textbf{16} gein] engein I (L) (Q) (R)  $\cdot$ noch] \textit{om.} L \textbf{17} het ir] Hette er M  $\cdot$ brîses niht] niht prises L pris dannoch nicht M \textbf{18} daz] da Z \textbf{19} Jeschute] ieschvte G ieskute I Jescvte O Jescuͯten L iescuten M iescute Q Z Jscuten R  $\cdot$ die] \textit{om.} I L or M  $\cdot$ hulde] hulden M \textbf{20} wære] wene M  $\cdot$ iu] \textit{om.} I ir L  $\cdot$ Kayn] kaẏ G kain I key O Z kaýen L keine M kein Q R \textbf{21} ungerochen] [vnd]: vn gerochen I vnd gerochen Z \textbf{22} gesprochen] [gebrochen]: gesbrochen I \textbf{23} saget] sagte L (M)  $\cdot$ wes] swes O waz L (R) \textbf{25} ouch mê landes] mere landes ouch L mere landes R  $\cdot$ was] \textit{om.} Q wer R \textbf{26} si] Was sie Q  $\cdot$ begunden] bundent R  $\cdot$ dô alle] alle do I da alle M Z alle Q \textbf{27} gelobte] giloibite M gelobter Q \textbf{28} den] die I \textbf{29} gesellecheit] sicherheit G geselheit R \textbf{30} bete] bit R  $\cdot$ ze] \textit{om.} R \newline
\end{minipage}
\hspace{0.5cm}
\begin{minipage}[t]{0.5\linewidth}
\small
\begin{center}*T
\end{center}
\begin{tabular}{rl}
 & Dô truoc der junge Parcifal\\ 
 & âne \textbf{vlüge} engels mâl\\ 
 & sus geblüet ûf der erden.\\ 
 & Artus mit den werden\\ 
5 & enpfienc in \textbf{rîterlîche}.\\ 
 & guotes willen wâren rîche\\ 
 & alle, die in \textbf{gesâhen} dâ.\\ 
 & ir herzen volge sprâchen jâ,\\ 
 & gegen sînem lobe sprach nieman nein,\\ 
10 & sô rehte minneclîch er schein.\\ 
 & \begin{large}A\end{large}rtus sprach z\textbf{im} sân:\\ 
 & "ir habt mir liep unde leit getân,\\ 
 & doch hât ir mir der êre\\ 
 & brâht unde gesant mêre,\\ 
15 & dannich ir ie von manne enpfie.\\ 
 & \textbf{dâ engegen mîn dienst} noch kleine gie,\\ 
 & hetet ir prîses \textbf{niht mêr} getân,\\ 
 & wan daz diu herzogîn sol hân,\\ 
 & vrou Jeschute, die hulde.\\ 
20 & Ouch wære iu Keys schulde\\ 
 & gewandelt \textbf{unde gerochen},\\ 
 & het ich iuch ê gesprochen."\\ 
 & Artus saget im, we\textit{s} er bat,\\ 
 & warumber an die selben stat\\ 
25 & unde mê landes was geriten.\\ 
 & Si begunden in dô alle biten,\\ 
 & daz er gelobete \textbf{besunder}\\ 
 & den von der tavelrunder\\ 
 & sîne rîterlîche gesellecheit.\\ 
30 & Im was ir bete niht ze leit.\\ 
\end{tabular}
\scriptsize
\line(1,0){75} \newline
T U V W \newline
\line(1,0){75} \newline
\textbf{1} \textit{Majuskel} T  \textbf{4} \textit{Majuskel} T  \textbf{11} \textit{Initiale} T U V W  \textbf{20} \textit{Majuskel} T  \textbf{26} \textit{Majuskel} T  \textbf{30} \textit{Majuskel} T  \newline
\line(1,0){75} \newline
\textbf{1} Parcifal] parzifal T V partzifal W \textbf{2} vlüge] flúgel V fliegen W \textbf{3} geblüet] [*]: geblvͤt T gebluͤmet V gelobet W \textbf{5} rîterlîche] minnencliche V \textbf{6} rîche] gleiche W \textbf{7} dâ] do U V W \textbf{8} herzen] herze U V  $\cdot$ sprâchen] sprach W \textbf{9} sprach nieman] niemant sprach W  $\cdot$ nein] [m*]: nein T \textbf{10} sô] Sus W \textbf{11} zim] zem waleise V zuͦ im do W \textbf{14} brâht unde gesant] Brachte vnd gesante U \textbf{15} ir] \textit{om.} W \textbf{16} engegen] gein U (W)  $\cdot$ noch] \textit{om.} W \textbf{17} prîses niht] niht prises V \textbf{19} Jeschute] Jescvte T (U) iescute V iestute W \textbf{20} Keys] keyns V es kein W \textbf{21} unde gerochen] [*]: vngerochen V \textbf{22} iuch] îv T  $\cdot$ ê] es W \textbf{23} wes] wez T \textbf{25} mê] oͮch me V \textbf{26} dô] \textit{om.} W \textbf{27} gelobete] gelobt W \newline
\end{minipage}
\end{table}
\end{document}
