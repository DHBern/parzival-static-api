\documentclass[8pt,a4paper,notitlepage]{article}
\usepackage{fullpage}
\usepackage{ulem}
\usepackage{xltxtra}
\usepackage{datetime}
\renewcommand{\dateseparator}{.}
\dmyyyydate
\usepackage{fancyhdr}
\usepackage{ifthen}
\pagestyle{fancy}
\fancyhf{}
\renewcommand{\headrulewidth}{0pt}
\fancyfoot[L]{\ifthenelse{\value{page}=1}{\today, \currenttime{} Uhr}{}}
\begin{document}
\begin{table}[ht]
\begin{minipage}[t]{0.5\linewidth}
\small
\begin{center}*D
\end{center}
\begin{tabular}{rl}
\textbf{37} & \begin{large}D\end{large}âr an ich liuge niemen.\\ 
 & sîne schiltriemen,\\ 
 & swaz der dâ zuo gehôrte,\\ 
 & was ein unverblichen borte\\ 
5 & mit \textbf{gesteine} harte tiure.\\ 
 & gelûtert in dem viure\\ 
 & was sîn buckel rôtgolt.\\ 
 & sîn dienest nam der minne solt.\\ 
 & ein scharpfer strît \textbf{in} ringe wac.\\ 
10 & diu künegîn in \textbf{dem venster} lac.\\ 
 & \textbf{bî ir sâzen} vrouwen mêr.\\ 
 & nû seht, dort hielt \textbf{ouch} Hiuteger,\\ 
 & al dâ im \textbf{ê} der prîs geschach.\\ 
 & dô er \textbf{disen} rîter komen sach\\ 
15 & \textbf{zuo z}im \textbf{kalopieren} hie,\\ 
 & \textbf{dô} \textbf{dâht} er: "wenne oder wie\\ 
 & kom dirre Franzois in\textbf{z} lant?\\ 
 & wer \textbf{hât} den stolzen her gesant?\\ 
 & het ich den vür einen môr,\\ 
20 & sô wære mîn bester sin ein tôr."\\ 
 & \textbf{Diu} \textbf{doch} von \textbf{sprüngen} niht beliben,\\ 
 & ir ors, \textbf{mit sporen si bêde} triben\\ 
 & ûzem walap in die rabîn.\\ 
 & si tâten rîters ellen schîn,\\ 
25 & der tjost \textbf{ein ander si} niht lugen.\\ 
 & die sprîzen gein den \textbf{luft et} vlugen\\ 
 & von des \textbf{küenen} Hiutegers sper.\\ 
 & \textbf{ouch} velte in sînes strîtes wer\\ 
 & hinder\textit{z} ors ûfz gras.\\ 
30 & vil ungewent er des was.\\ 
\end{tabular}
\scriptsize
\line(1,0){75} \newline
D \newline
\line(1,0){75} \newline
\textbf{1} \textit{Initiale} D  \textbf{21} \textit{Majuskel} D  \newline
\line(1,0){75} \newline
\textbf{12} Hiuteger] Hvteger D \textbf{27} Hiutegers] Hvtegers D \textbf{29} hinderz] hinders D \newline
\end{minipage}
\hspace{0.5cm}
\begin{minipage}[t]{0.5\linewidth}
\small
\begin{center}*m
\end{center}
\begin{tabular}{rl}
 & dâr an ich liuge niemen.\\ 
 & sîne schiltriemen,\\ 
 & waz der dar zuo gehôrte,\\ 
 & was ein unverbliche\textit{n} borte\\ 
5 & mit \textbf{gesteine} harte tiure.\\ 
 & gelûtert in dem viure\\ 
 & was sîn buckel rôtgolt.\\ 
 & sîn dienst nam der minne solt.\\ 
 & \textit{e}in scharfer strît \textbf{in} ringe wa\textit{c}.\\ 
10 & diu künigîn in \textbf{dem venster} lac\\ 
 & \textbf{und saz ouch bî ir} vrouwen mêr.\\ 
 & nû sehet, dort hielt \textbf{ouch} Huteger,\\ 
 & aldâ im \textbf{ê} der prîs geschach.\\ 
 & dô er \textbf{disen} ritter komen sach\\ 
15 & \textbf{zuo} im \textbf{galopieren} hie,\\ 
 & \textbf{dô} \textbf{dâht} er: "wenne oder wie\\ 
 & kam dirre Franzois in \textbf{daz} lant?\\ 
 & wer \textbf{hât} den stolze\textit{n} her gesant?\\ 
 & hete ich den vür einen môr,\\ 
20 & sô wær mîn bester sin ein tôr."\\ 
 & \textbf{\textit{\begin{large}D\end{large}}i\textit{u}} \textbf{doch} von \textbf{springen} niht beliben,\\ 
 & ir ros, \textbf{mit sporen beide si} tr\textit{i}ben\\ 
 & ûzem walap in die rabîn.\\ 
 & si tâten ritters ellen schîn,\\ 
25 & der just \textbf{ein ander si} niht lugen.\\ 
 & d\textit{ie} sprîzen gegen de\textit{n} \textbf{l\textit{ü}f\textit{t}en} vlugen\\ 
 & von des \textbf{küenen} H\textit{u}t\textit{e}gers sper.\\ 
 & \textbf{\textit{ou}ch} valte in sînes strîtes wer\\ 
 & hinder\textit{z} ros ûf daz gras.\\ 
30 & vil ungewent er des was.\\ 
\end{tabular}
\scriptsize
\line(1,0){75} \newline
m n o W \newline
\line(1,0){75} \newline
\textbf{9} \textit{Initiale} W  \textbf{21} \textit{Initiale} m  \newline
\line(1,0){75} \newline
\textbf{3} waz der] Vnd was n o W \textbf{4} unverblichen] vnuerbliche m  $\cdot$ borte] wort W \textbf{6} in] by n (o) (W) \textbf{7} buckel] bickel n o \textbf{8} nam] name m  $\cdot$ minne] mynnen n (o) (W) \textbf{9} ein] Eein m SEin W  $\cdot$ scharfer] starcker W  $\cdot$ wac] was \textit{nachträglich korrigiert zu:} wak m \textbf{11} bî] \textit{om.} o \textbf{12} dort hielt ouch] auch dort hilt o  $\cdot$ Huteger] hutteger m huttiger n huͯttiger o hútiger W \textbf{13} aldâ] Als W \textbf{16} dâht] gedocht n (o) (W) \textbf{17} daz] dis n o W  $\cdot$ Franzois] franzosz m frantzosz n frantzos o frantzoys W \textbf{18} stolzen her] stolczeliher \textit{nachträglich korrigiert zu:} stolczenher m \textbf{20} tôr] kor o \textbf{21} Diu] Wie m  $\cdot$ springen] sprungen n \textbf{22} mit sporen beide si] mit sporn sú beide n (o) zuͦ beiden sporn mit W  $\cdot$ triben] treiben m \textbf{23} ûzem] Vs ein n (o) Auß eim W  $\cdot$ die] ein W  $\cdot$ rabîn] rubin o \textbf{24} ritters ellen] ritters allen o ellens ritters W \textbf{25} ein ander si] sy einander W  $\cdot$ niht] mit o \textbf{26} die] D m  $\cdot$ sprîzen] spritzen \textit{nachträglich korrigiert zu:} spriszen m spreisse W  $\cdot$ den lüften] dem luffen m den lúfften lúfften n \textbf{27} Hutegers] hiettingers m húttigers n W huͯttigers o \textbf{28} ouch] Voch m  $\cdot$ sînes] sein W \textbf{29} hinderz] huͯnder \textit{nachträglich korrigiert zu:} huͯndersz m Húnder n o  $\cdot$ ûf] vnd vff n \textbf{30} vil ungewent er] Wol er gewenet n o \newline
\end{minipage}
\end{table}
\newpage
\begin{table}[ht]
\begin{minipage}[t]{0.5\linewidth}
\small
\begin{center}*G
\end{center}
\begin{tabular}{rl}
 & dâr an ich liuge niemen.\\ 
 & sîne schiltriemen,\\ 
 & swaz der dar zuo gehôrte,\\ 
 & was ein unverblichen borte\\ 
5 & mit \textbf{gesteine} harte tiure.\\ 
 & gelûtert in dem viure\\ 
 & was sîn buckel rôtgolt.\\ 
 & sîn dienst nam der minnen solt.\\ 
 & ein scharfer strît \textbf{in} ringe wac.\\ 
10 & diu künigîn in \textbf{den vensteren} lac.\\ 
 & \textbf{bî ir sâzen} vrouwen mêr.\\ 
 & nû seht, dort hielt \textbf{ouch} Huteger,\\ 
 & al dâ im \textbf{ê} der brîs geschach.\\ 
 & dô er \textbf{disen} rîter komen sach\\ 
15 & \textbf{zuo} im \textbf{gewalopiert} hie,\\ 
 & \textbf{nû} \textbf{dâhte}r: "wenne oder wie\\ 
 & \begin{large}K\end{large}om dirre Franzoise in \textbf{diz} lant?\\ 
 & wer \textbf{hât} den stolzen her gesant?\\ 
 & het ich den vür einen môr,\\ 
20 & sô wære mîn bester sin ein tôr."\\ 
 & \textbf{iedoch} von \textbf{sprüngen} niht beliben\\ 
 & ir ors. \textbf{mit sporen si bêde} triben\\ 
 & ûz dem walap in die rabîn.\\ 
 & si tâten rîters ellen schîn,\\ 
25 & der tjost \textbf{ein ander si} niht lugen.\\ 
 & die spr\textit{î}zen gein den \textbf{lüften} vlugen\\ 
 & von des \textbf{stolzen} Hutegers sper.\\ 
 & \textbf{doch} valt in sînes strîtes wer\\ 
 & hinderz ors ûf dez gras.\\ 
30 & vil ungewent er des was.\\ 
\end{tabular}
\scriptsize
\line(1,0){75} \newline
G O L M Q R Z Fr21 Fr32 \newline
\line(1,0){75} \newline
\textbf{1} \textit{Initiale} O M Fr21  \textbf{9} \textit{Initiale} Q R Z Fr32  \textbf{17} \textit{Initiale} G  \newline
\line(1,0){75} \newline
\textbf{1} dâr an] ÷aran O Dar nach R  $\cdot$ ich liuge] enlivge ich O (L) sich levgen Q lúge ich R erlevg ih Fr21 \textbf{2} sîne] Seinen Q  $\cdot$ schiltriemen] schilt reinen Q \textbf{3} swaz] Waz L (M) (Q) (R) Fr21  $\cdot$ der] \textit{om.} M Z  $\cdot$ gehôrte] horte L \textbf{4} was] Daz was O (R) Z  $\cdot$ unverblichen] vnerblichen L \textbf{5} mit] Min Z \textbf{8} minnen] mynne Q (R)  $\cdot$ solt] \textit{om.} Z \textbf{9} ein] Sin R Min Fr32  $\cdot$ scharfer] starker Z \textbf{10} den vensteren] dem venster O L (M) (Q) R Z Fr21 \textbf{11} bî] Da pi O (M) (Q) (R) (Z) (Fr21) (Fr32)  $\cdot$ ir] \textit{om.} Q Z \textbf{12} seht] \textit{om.} O  $\cdot$ dort hielt] dor helt M  $\cdot$ Huteger] hvͦteger O hvtteger L Z Nuteger M huͯttiger R Hvͦtiger Fr21 Hivteger Fr32 \textbf{13} ê] \textit{om.} O Fr21 \textbf{14} dô] Da M Z  $\cdot$ disen] [dem]: den L  $\cdot$ rîter] \textit{om.} M \textbf{15} im] zim O Fr21 Fr32  $\cdot$ gewalopiert] galopiern O (L) (M) (Q) (R) (Z) (Fr21) (Fr32) \textbf{16} nû] Da Z  $\cdot$ dâhter] gedahte er L (Q) (R) \textbf{17} Kom] Komen Q  $\cdot$ dirre] dise Q [dir*]: dir Fr21  $\cdot$ Franzoise] franzoys O Fr21 Frantzois L (Z) fronzoys M frantzosen Q francos R frantzoẏs Fr32  $\cdot$ in diz] Jndas R \textbf{19} den] danne M  $\cdot$ Môr] mosz M \textbf{21} \textit{Versfolge 37.23-24-22-21} Q   $\cdot$ iedoch] Die Q Die doch R Z (Fr32)  $\cdot$ sprüngen] springen Q (R) \textbf{23} ûz dem] Ausz Q Vsser dem R  $\cdot$ die] den Z  $\cdot$ rabîn] rubin R \textbf{24} tâten] tatens O  $\cdot$ rîters ellen] ritterlichin M ritters ellen \textit{nachträglich korrigiert zu:} erren Q \textbf{25} ein ander si] sy ein andren R sie ein ander Z \textbf{26} sprîzen] spriezen G sprizel O spitzen L (M) (Q) (R) (Fr32)  $\cdot$ gein den] in die Q  $\cdot$ lüften] luffte Q (R)  $\cdot$ vlugen] [*]: stuben R \textbf{27} Hutegers] hvͦtegeres O Huͯttegers L hutegers M hutegeres Q hútingers R Hvtigers Fr21 Hivtegerz Fr32 \textbf{28} doch] Avch O (L) (M) (Q) (R) (Z) (Fr21) (Fr32)  $\cdot$ sînes] \sout{valt} sines O sin R  $\cdot$ wer] ger R \textbf{30} ungewent] vngewon O L (Fr21)  $\cdot$ des] das M \newline
\end{minipage}
\hspace{0.5cm}
\begin{minipage}[t]{0.5\linewidth}
\small
\begin{center}*T (U)
\end{center}
\begin{tabular}{rl}
 & dâr an ich liuge niemen.\\ 
 & sîne schiltriemen,\\ 
 & swaz der dâ zuo gehôrte,\\ 
 & was ein unverblichen borte\\ 
5 & mit \textbf{golde} harte \textit{tiu}re.\\ 
 & gelûtert in dem viure\\ 
 & was sîn buckel rôtgolt.\\ 
 & sîn dienst nam der minnen solt.\\ 
 & ein scharpfer strît \textbf{im} ringe wac.\\ 
10 & d\textit{iu} künegîn in \textbf{den vensteren} lac.\\ 
 & \textbf{dâ bî ir sâzen} vrouwen mêr.\\ 
 & nû seht, dort h\textit{i}elt Huteger,\\ 
 & aldâ im der prîs geschach.\\ 
 & dô \textit{er} \textbf{einen} ritter komen sach\\ 
15 & \textbf{gein} im \textbf{galopierende} hie,\\ 
 & \textbf{dô} \textbf{gedâhte} er: "wan oder wie\\ 
 & kom dirre Franzoyser in \textbf{diz} lant?\\ 
 & wer \textbf{hete} den stolzen her gesant?\\ 
 & het ich den vür einen môr,\\ 
20 & sô wære mîn bester sin ein tôr."\\ 
 & \textbf{iedoch} von \textbf{springen} \textbf{si} niht bliben,\\ 
 & ir ors, \textbf{si \textit{m}it den spo\textit{r}en beide} triben\\ 
 & ûz dem walap in d\textit{ie} rabîn.\\ 
 & si tâten ritters \textit{e}llen schîn,\\ 
25 & der tjost \textbf{si ein ander} niht lugen.\\ 
 & die sp\textit{r}îzen gein den \textbf{lüften} vlugen\\ 
 & von des \textbf{stolzen} Hutegers sper.\\ 
 & \textbf{dô} valt in sînes strîtes wer\\ 
 & hinder\textit{z} ors ûf daz gras.\\ 
30 & vil ungewent er des was.\\ 
\end{tabular}
\scriptsize
\line(1,0){75} \newline
U V T \newline
\line(1,0){75} \newline
\textbf{10} \textit{Majuskel} T  \textbf{12} \textit{Majuskel} T  \textbf{21} \textit{Majuskel} T  \newline
\line(1,0){75} \newline
\textbf{3} swaz] Waz U  $\cdot$ gehôrte] horte T \textbf{5} golde] gesteine V T  $\cdot$ tiure] were U \textbf{9} im] in T \textbf{10} diu] Des U  $\cdot$ den vensteren] dem venster T [den venstern]: dem venster V \textbf{12} hielt] helt U  $\cdot$ Huteger] hútiger V \textbf{13} im] [*]: im e V im ê T \textbf{14} er] \textit{om.} U  $\cdot$ einen] ginen V disen T \textbf{15} gein im] zvͦ zim T \textbf{16} gedâhte er] dahter T \textbf{17} Franzoyser] frantzois V Franzoys T \textbf{18} hete] hat V T \textbf{19} het] hat T \textbf{21} springen] sprúngen V (T)  $\cdot$ si niht] [*]: sv́ niht V niht T \textbf{22} ir] in T  $\cdot$ si mit den sporen beide] si nit den spoen beide U mit sporn [s*]: sv́ beide V si beide mit sporn T \textbf{23} die] den U \textbf{24} ellen] allen U ellens T \textbf{25} si ein ander] [s*]: sv́ enander V ein ander si T \textbf{26} sprîzen] speizen U [spr*]: sprizen V \textbf{27} Hutegers] Huthigers V \textbf{28} dô] ouch T \textbf{29} hinderz] Hinder U \newline
\end{minipage}
\end{table}
\end{document}
