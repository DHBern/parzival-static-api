\documentclass[8pt,a4paper,notitlepage]{article}
\usepackage{fullpage}
\usepackage{ulem}
\usepackage{xltxtra}
\usepackage{datetime}
\renewcommand{\dateseparator}{.}
\dmyyyydate
\usepackage{fancyhdr}
\usepackage{ifthen}
\pagestyle{fancy}
\fancyhf{}
\renewcommand{\headrulewidth}{0pt}
\fancyfoot[L]{\ifthenelse{\value{page}=1}{\today, \currenttime{} Uhr}{}}
\begin{document}
\begin{table}[ht]
\begin{minipage}[t]{0.5\linewidth}
\small
\begin{center}*D
\end{center}
\begin{tabular}{rl}
\textbf{20} & sus warb ie, der ungerne vlôch.\\ 
 & vil orse man \textbf{im} wider zôch,\\ 
 & durchstochen unt \textbf{verhouwen}.\\ 
 & manege tunkele vrouwen\\ 
5 & sach er bêdenthalben sîn.\\ 
 & nâch rabens varwe was ir schîn.\\ 
 & sîn wirt in minneclîche enpfienc.\\ 
 & daz im \textbf{nâch} vröuden sît ergienc.\\ 
 & \textbf{daz} was ein \textbf{ellens rîcher} man.\\ 
10 & mit sîner hant \textbf{het er} getân\\ 
 & manegen stich unt slac,\\ 
 & wand \textbf{er} einer porten pflac.\\ 
 & \textbf{bî dem} er manegen rîter vant,\\ 
 & \textbf{die} \textbf{ir hende} \textbf{hiengen} in diu bant\\ 
15 & unt \textbf{den ir} houbet \textbf{wâren} verbunden.\\ 
 & die heten sölhe wunden,\\ 
 & daz si doch tâten rîterschaft;\\ 
 & si \textbf{heten} \textbf{lâzen niht ir} kraft.\\ 
 & \begin{large}D\end{large}er burcgrâve von der stat\\ 
20 & sînen gast \textbf{dô} minneclîchen bat,\\ 
 & daz er niht verbære\\ 
 & \textbf{al} daz sîn wille wære\\ 
 & über \textbf{sîn} guot unt über den lîp.\\ 
 & er vuorten, dâ er vant sîn wîp,\\ 
25 & diu Gahmureten kuste,\\ 
 & \textbf{des} in doch \textbf{wênec} \textbf{geluste}.\\ 
 & dâr nâch vuor er enbîzen sân.\\ 
 & dô diz \textbf{alsus was} getân,\\ 
 & der marschalc \textbf{vuor} von im zehant,\\ 
30 & \textbf{al} dâ er die küneginne vant,\\ 
\end{tabular}
\scriptsize
\line(1,0){75} \newline
D Fr9 Fr14 \newline
\line(1,0){75} \newline
\textbf{19} \textit{Initiale} D Fr9 Fr14  \newline
\line(1,0){75} \newline
\textbf{10} sîner] [sinen]: siner Fr9  $\cdot$ er] her ouch Fr9 \textbf{13} dem] [dem]: deR Fr14  $\cdot$ er] her ouch Fr9 \textbf{18} Wante in nicht war an ir kraft Fr9  $\cdot$ heten] hete Fr14 \textbf{23} über den] sẏnen Fr9 \textbf{25} Gahmureten] gamvreten Fr9  $\cdot$ kuste] [*]: chvste D \textbf{26} doch wênec] weẏnich doch Fr9 \textbf{28} alsus was] was alsus Fr9 Fr14 \textbf{29} im] in Fr9 \textbf{30} al] \textit{om.} Fr9 \newline
\end{minipage}
\hspace{0.5cm}
\begin{minipage}[t]{0.5\linewidth}
\small
\begin{center}*m
\end{center}
\begin{tabular}{rl}
 & su\textit{s w}arp ie, der ungerne vlôch.\\ 
 & vil ros man \textbf{im} wider zôch,\\ 
 & durchstochen und \textbf{erhouwen}.\\ 
 & menige dunkele vrowen\\ 
5 & \textit{s}ach er beidenthalben sîn.\\ 
 & nâch r\textit{a}be\textit{n}s varwe was ir schîn.\\ 
 & sîn wirt in minneclîchen enpfienc.\\ 
 & daz im \textbf{nâch} vröuden sît ergienc.\\ 
 & \textbf{daz} was ein \textbf{ellen\textit{t}lîcher} man.\\ 
10 & mit sîner hant \textbf{het er} getân\\ 
 & menigen stich und slac,\\ 
 & wenne \textbf{er} einer porten pflac.\\ 
 & \textbf{bî dem} er menigen ritter vant,\\ 
 & \textbf{der} \textbf{sîne hende} \textbf{hienc} in diu bant\\ 
15 & und \textbf{dem sîn} houbt \textbf{was} verbunden.\\ 
 & die hâten \textbf{alle} soliche wun\textit{d}en,\\ 
 & daz si \dag durch toren\dag  ritterschaft;\\ 
 & si \textbf{heten} \textbf{macht und} kraft.\\ 
 & \textbf{dô} der burcgrâve von der stat\\ 
20 & sînen gast \textbf{sô} minneclîchen bat,\\ 
 & daz er niht verbære\\ 
 & \textbf{allez} daz sîn wille wære\\ 
 & über \textbf{sîn} guot und über den lîp.\\ 
 & er vuorte in, d\textit{â} er vant sîn wîp,\\ 
25 & diu Gahmureten kuste,\\ 
 & \textbf{daz} in doch \textbf{wênic} \textbf{luste}.\\ 
 & \hspace*{-.7em}\big| dô diz \textbf{alsus was} getân,\\ 
 & \hspace*{-.7em}\big| dâr nâch \textit{v}uor er enbîzen sân.\\ 
 & der marschalc \textbf{v\textit{u}or} von ime zehant,\\ 
30 & d\textit{â} er die küniginn\textit{e} vant,\\ 
\end{tabular}
\scriptsize
\line(1,0){75} \newline
m n o \newline
\line(1,0){75} \newline
\textbf{28} \textit{Initiale} n  \newline
\line(1,0){75} \newline
\textbf{1} sus warp] Sus was warb \textit{nachträglich korrigiert zu:} Sus warb m \textbf{3} erhouwen] verhouwen n o \textbf{5} sach] Stach \textit{nachträglich korrigiert zu:} Sach m \textbf{6} rabens] robes \textit{nachträglich korrigiert zu:} roes m Rubins o \textbf{9} ellentlîcher] ellentschlicher m ellenthafftiger o \textbf{10} het] hat n \textbf{12} pflac] pflan o \textbf{14} hienc] [i*]: hinge n  $\cdot$ in] an o \textbf{15} verbunden] berbunden n \textbf{16} alle] \textit{om.} n o  $\cdot$ wunden] wunben m \textbf{17} toren] die n o \textbf{23} den] sin n o \textbf{24} dâ] do m o do do n \textbf{25} Gahmureten] Gahmuretten m gahimireten n garuͯmreten o  $\cdot$ kuste] kuͯsche o \textbf{26} luste] geluste n (o) \textbf{27} vuor] fruͦr \textit{nachträglich korrigiert zu:} fuͦr m do fúre n  $\cdot$ enbîzen] beissen n  $\cdot$ sân] sin o \textbf{29} vuor] vor m  $\cdot$ von] [yn]: von m vor n o \textbf{30} dâ] Do m n o  $\cdot$ küniginne] koniginnen m \newline
\end{minipage}
\end{table}
\newpage
\begin{table}[ht]
\begin{minipage}[t]{0.5\linewidth}
\small
\begin{center}*G
\end{center}
\begin{tabular}{rl}
 & sus warp ie, der ungerne vlôch.\\ 
 & vil orse man \textbf{im} wider zôch,\\ 
 & durchstochen und \textbf{verhouwen}.\\ 
 & manige tunkele vrouwen\\ 
5 & sach er bêdenthalben sîn.\\ 
 & nâch rabenes varwe was ir schîn.\\ 
 & sîn wirt in minniclîche enpfie.\\ 
 & daz im \textbf{ze} vröuden sît ergie.\\ 
 & \textbf{er} was ein \textbf{ellens rîcher} man.\\ 
10 & mit sîner hant \textbf{het er} getân\\ 
 & manigen stich und slac,\\ 
 & wan \textbf{er} einer borte pflac.\\ 
 & \textbf{bî der} er manigen rîter vant,\\ 
 & \textbf{die} \textbf{die arme} \textbf{hiengen} in diu bant\\ 
15 & unt \textbf{diu} houbet \textbf{wâren} verbunden.\\ 
 & die heten solhe wunden,\\ 
 & daz si doch tâten rîterschaft;\\ 
 & si \textbf{hete} \textbf{lâzen niht ir} kraft.\\ 
 & der burcgrâve von der stat\\ 
20 & sînen gast \textbf{dô} minniclîchen bat,\\ 
 & daz er niht verbære\\ 
 & \textbf{al} daz sîn wille wære\\ 
 & über \textbf{sîn} guot und über den lîp.\\ 
 & er vuorte in, dâ er vant sîn wîp,\\ 
25 & diu Gahmureten kuste,\\ 
 & \textbf{des} in doch \textbf{wênic} \textbf{luste}.\\ 
 & dâr nâch vuor er enbîzen sân.\\ 
 & dô diz \textbf{allez was} getân,\\ 
 & der marschalc \textbf{reit} von im zehant,\\ 
30 & dâ er die küniginne vant.\\ 
\end{tabular}
\scriptsize
\line(1,0){75} \newline
G O L M Q R W Z Fr29 Fr32 Fr36 Fr55 Fr71 \newline
\line(1,0){75} \newline
\textbf{1} \textit{Initiale} O M  \textbf{7} \textit{Initiale} Fr71  \textbf{15} \textit{Versal} Fr32  \textbf{19} \textit{Initiale} L Q R W Z Fr32 Fr36  \textbf{29} \textit{Initiale} Fr55  \newline
\line(1,0){75} \newline
\textbf{1} sus] ÷vs O  $\cdot$ warp] warff Q erwarp Z  $\cdot$ ie der] der ie O W Fr32 (Fr71)  $\cdot$ ungerne] gerne Fr32 \textbf{2} vil orse] Manig roß W  $\cdot$ im] in L (M) Q (R) Z (Fr29) Fr32 \textbf{3} verhouwen] duͯrch hauwen L (Q) (Fr32) zer howen R \textbf{6} rabenes varwe] raben varwe O L (M) (Q) Z rappen varbe W \textbf{7} in minniclîche] mît frævden in Fr71 \textbf{8} ze] nach O (M) Q R Z (Fr29) Fr32  $\cdot$ vröuden] frewnden Q \textbf{9} er] Der O L Q Z (Fr32) Dar M Das R  $\cdot$ ellens rîcher] ellent richer O Fr32 ellens riche L (M) ellen* richer n\textit{achträglich korrigiert zu: }erent richer Q \textbf{10} hant] [had]: hant M \textbf{11} manigen] Vil mangen O (M) (Q) (R) (Z) (Fr32) (Fr36) (Fr71) \textbf{12} borte] porten O L R Z Fr29 Fr32 Fr36 Fr71 phorten M (Q)  $\cdot$ pflac] pfagk Q \textbf{13} bî der] Bi dem O L (Q) (R) Z (Fr32) Sider Fr36 Wider Fr71  $\cdot$ rîter] rittern M \textbf{14} die die arme] Deme die arme M Den die arm R (W) Die ir hende Z die armen Fr32 (Fr71)  $\cdot$ hiengen] hiegnen Fr36  $\cdot$ in diu] anden M vnd die R  $\cdot$ bant] hand R \textbf{15} diu] den ir O L Q R Z Fr32 Fr36 den die W  $\cdot$ wâren] [warem]: waren O worden Q  $\cdot$ verbunden] verpnnden W \textbf{16} die] Sy W \textbf{18} si hete] Sie on hetten M (Q) (R) (Z) sine hete Fr32 Sy hetten W  $\cdot$ lâzen niht] verlazen niht O (L) gelaszin nicht M (W) (Fr32) nit glassen R niht lazzen Z \textbf{20} dô] da M Z den R er W er do Fr71  $\cdot$ minniclîchen] Jnenklchen R \textbf{22} al] Ald R Was W \textbf{23} sîn] \textit{om.} O  $\cdot$ den] \textit{om.} O sin L R (W) Z sy M \textbf{24} dâ] do Q W \textbf{25} Gahmureten] Gamvreten O (Fr32) Gahmuͯreten L gammaraten M gaműreten Q Gahmareten R gamureten W Z :::mvͦreten Fr29 \textbf{26} in doch] doch in Z  $\cdot$ wênic] luzel O (L) (R) (Fr29) (Fr32) nútzel n\textit{achträglich korrigiert zu }lútzel Q vil wenic Z  $\cdot$ luste] geluste R \textbf{27} vuor] fuͦrt W  $\cdot$ enbîzen sân] enbisen dan R in in bizensan W erbeizzen san Z \textbf{28} dô] Da M Z  $\cdot$ diz] daz L (W)  $\cdot$ allez was] was alsus O (M) Z (Fr29) alles alsus was R alsus wart Fr32 \textbf{29} von im] all Q \textbf{30} dâ] Al da O (R) (Z) (Fr32) (Fr71) Do Q W  $\cdot$ küniginne] kvnigynnen L (R) \newline
\end{minipage}
\hspace{0.5cm}
\begin{minipage}[t]{0.5\linewidth}
\small
\begin{center}*T
\end{center}
\begin{tabular}{rl}
 & sus warp ie, der ungerne vlôch.\\ 
 & vil orse man \textbf{hin} wider zôch,\\ 
 & durchstochen und \textbf{verhouwen}.\\ 
 & manec tunkele vrouwen\\ 
5 & sach er beidenthalben sîn.\\ 
 & nâch rabens varwe was ir schîn.\\ 
 & Sîn wirt in minneclîche enpfienc.\\ 
 & daz im \textbf{nâch} vröuden sît ergienc.\\ 
 & \textbf{der} was ein \textbf{ellens rîcher} man.\\ 
10 & mit sîner hant \textbf{er hete} getân\\ 
 & \textbf{vil} manegen stich und slac,\\ 
 & wan \textbf{der} einer porte pflac.\\ 
 & \textbf{dâ bî} er manegen rîter vant,\\ 
 & \textbf{die} \textbf{die arme} \textbf{hiengen} in diu bant\\ 
15 & und \textbf{den ir} houbet \textbf{was} verbunden.\\ 
 & die heten solhe wunden,\\ 
 & daz si doch tâten rîterschaft;\\ 
 & si \textbf{heten} \textbf{gelâzen niht ir} kraft.\\ 
 & \begin{large}D\end{large}er burcgrâve von der stat\\ 
20 & sînen gast \textbf{dô} minneclîche bat,\\ 
 & daz er niht verbære\\ 
 & \textbf{al}daz sîn wille wære\\ 
 & über \textbf{daz} guot und über den lîp.\\ 
 & er vuortin, dâ er vant sîn wîp,\\ 
25 & diu Gahmureten kuste,\\ 
 & \textbf{des} in doch \textbf{lützel} \textbf{luste}.\\ 
 & dâr nâch vuor er enbîzen sân.\\ 
 & dô diz \textbf{was alsus} getân,\\ 
 & der marschalc \textbf{reit} von im zehant,\\ 
30 & \textbf{al}dâ er die küneginne vant.\\ 
\end{tabular}
\scriptsize
\line(1,0){75} \newline
T U V \newline
\line(1,0){75} \newline
\textbf{7} \textit{Majuskel} T  \textbf{19} \textit{Initiale} T V  \newline
\line(1,0){75} \newline
\textbf{1} ie der] er ie der V \textbf{7} enpfienc] enplent \textit{nachträglich korrigiert zu:} enplenc U \textbf{8} ergienc] ergrent \textit{nachträglich korrigiert zu:} ergrenc U \textbf{9} ellens] ellendes U \textbf{10} er hete] het er U V \textbf{12} wan der] Wan er U (V)  $\cdot$ porte] porten U V \textbf{14} arme] armen U \textbf{23} daz] [*]: sin V  $\cdot$ den] [*]: sinen V \textbf{24} vuortin] vuͦrte in hin U (V)  $\cdot$ dâ] do V \textbf{25} Gahmureten] Gahmvreten T Gahmuͦreten U Gamureten V \textbf{26} luste] geluͦste U \textbf{27} vuor er enbîzen] fuͦrte er [*]: in enbissen V \textbf{28} alsus] alles V \textbf{29} von] [mit]: von T \newline
\end{minipage}
\end{table}
\end{document}
