\documentclass[8pt,a4paper,notitlepage]{article}
\usepackage{fullpage}
\usepackage{ulem}
\usepackage{xltxtra}
\usepackage{datetime}
\renewcommand{\dateseparator}{.}
\dmyyyydate
\usepackage{fancyhdr}
\usepackage{ifthen}
\pagestyle{fancy}
\fancyhf{}
\renewcommand{\headrulewidth}{0pt}
\fancyfoot[L]{\ifthenelse{\value{page}=1}{\today, \currenttime{} Uhr}{}}
\begin{document}
\begin{table}[ht]
\begin{minipage}[t]{0.5\linewidth}
\small
\begin{center}*D
\end{center}
\begin{tabular}{rl}
\textbf{698} & \begin{large}D\end{large}ô wart ouch si nâch jâmer var,\\ 
 & ir süezer munt meit ezzen gar.\\ 
 & si \textbf{dâhte}: "waz tuot Bene hie?\\ 
 & \textbf{ich hete} \textbf{iedoch} gesendet sie\\ 
5 & ze dem, der \textbf{dort} mîn herze tregt,\\ 
 & daz mich hie \textbf{gar} unsanfte regt.\\ 
 & \textbf{waz} ist an mir gerochen?\\ 
 & hât der künec \textbf{widersprochen}\\ 
 & mîn dienst unt mîne minne?\\ 
10 & sîne \textbf{getriwe}, manlîche sinne\\ 
 & mugen hie niht mêr erwerben,\\ 
 & wan \textbf{dar umbe} muoz ersterben\\ 
 & mîn armer lîp, den ich hie trage,\\ 
 & nâch im mit herzelîcher klage."\\ 
15 & Dô man ezzens \textbf{dâ} verpflac,\\ 
 & dô was ez \textbf{ouch} \textbf{über den} \textbf{mitten tac}.\\ 
 & Artus unt daz wîp sîn,\\ 
 & vrou Gynover, diu künegîn,\\ 
 & mit \textbf{rîtern} unt mit vrouwen schar\\ 
20 & riten, dâ der wol gevar\\ 
 & saz bî werder vrouwen diet.\\ 
 & \textbf{Parzivals} \textbf{antvanc} \textbf{dô} geriet;\\ 
 & manege clâre vrouwen\\ 
 & muoser sich küssen schouwen.\\ 
25 & Artus bôt im êre\\ 
 & unt dankte im des sêre,\\ 
 & daz sîn hôhiu werdecheit\\ 
 & \textbf{wære} sô lanc unt \textbf{ouch} \textbf{sô} breit,\\ 
 & daz er den prîs vür alle man\\ 
30 & von rehten schulden solte hân.\\ 
\end{tabular}
\scriptsize
\line(1,0){75} \newline
D \newline
\line(1,0){75} \newline
\textbf{1} \textit{Initiale} D  \textbf{15} \textit{Majuskel} D  \newline
\line(1,0){75} \newline
\textbf{22} Parzivals] Parcivals D \newline
\end{minipage}
\hspace{0.5cm}
\begin{minipage}[t]{0.5\linewidth}
\small
\begin{center}*m
\end{center}
\begin{tabular}{rl}
 & dô wart ouch \textit{si} nâch jâmer var,\\ 
 & ir süezer munt meit ezzen gar.\\ 
 & si \textbf{dâhte}: "waz tuot Bene hie?\\ 
 & \textbf{ich het} \textbf{iedoch} gesendet sie\\ 
5 & zuo dem, der \textbf{doch} mîn herze treget,\\ 
 & daz mich hie \textbf{gar} unsanfte reget,\\ 
 & \textbf{daz} ist an mir gerochen.\\ 
 & het der künic \textbf{widersprochen}\\ 
 & mîn dienst und mîne minne?\\ 
10 & sîn \textbf{getriuwe}, manlîch sinne\\ 
 & mogen hie niht mê erwerben,\\ 
 & wan \textbf{daz} \textbf{d\textit{ar} umbe} muoz ersterben\\ 
 & mîn armer lîp, den ich hie trage,\\ 
 & nâch im mit herzelîcher klage."\\ 
15 & dô man ezzens verpflac,\\ 
 & d\textit{ô} was ez \textbf{ouch} \textbf{wol} \textbf{mitter tac}.\\ 
 & Artus und daz wîp sîn,\\ 
 & vrowe G\textit{i}nover, diu künigîn,\\ 
 & mit \textbf{ritter} und mit vrowen schar\\ 
20 & riten, d\textit{â} der wol gevar\\ 
 & saz bî werde\textit{r} vrowen diet.\\ 
 & \textbf{Parcifals} \textbf{anpf\textit{a}n\textit{c}} \textbf{dô} geriet;\\ 
 & manige clâre vrouwen\\ 
 & muost er si\textit{ch} küssen schouwen.\\ 
25 & Artus bôt im êre\\ 
 & und danket im des sêre,\\ 
 & daz sîn hôhiu wirdicheit\\ 
 & \textbf{was} sô lanc und breit,\\ 
 & daz er den prîs vür alle man\\ 
30 & von rehten schulden solte hân.\\ 
\end{tabular}
\scriptsize
\line(1,0){75} \newline
m n o Fr69 \newline
\line(1,0){75} \newline
\textbf{15} \textit{Initiale} Fr69  \newline
\line(1,0){75} \newline
\textbf{1} si] mich m \textbf{2} ezzen] essens n \textbf{6} mich] [mir]: mich m \textbf{8} het] Hette n \textbf{12} dar] des m \textbf{15} verpflac] da verpflac Fr69 \textbf{16} dô] Da m o  $\cdot$ mitter tac] mit tag n \textbf{18} Ginover] genover m n ginofer o Ginovier vnd Fr69 \textbf{19} mit vrowen] frouwen n (o) \textbf{20} dâ] do m n o \textbf{21} werder] werden m  $\cdot$ diet] [hie]: diet o \textbf{22} Parcifals] Parcifal o  $\cdot$ anpfanc] enpfings m enpfing o \textbf{23} clâre] claren o \textbf{24} sich] sẏ m \textbf{25} Artus] Artús o \textbf{28} breit] so breit n o \newline
\end{minipage}
\end{table}
\newpage
\begin{table}[ht]
\begin{minipage}[t]{0.5\linewidth}
\small
\begin{center}*G
\end{center}
\begin{tabular}{rl}
 & \begin{large}D\end{large}ô wart ouch si nâch jâmer var,\\ 
 & ir süezer munt meit ezzen gar.\\ 
 & si \textbf{dâhte}: "waz tuot Bene hie?\\ 
 & \textbf{nû het ich} \textbf{doch} gesendet sie\\ 
5 & zuo dem, der \textbf{dort} mîn herze treit,\\ 
 & daz mich \textbf{doch} hie unsanfte reit.\\ 
 & \textbf{waz} ist an mir gerochen?\\ 
 & hât der künic \textbf{versprochen}\\ 
 & mîn dienst unde mîne minne?\\ 
10 & sîne \textbf{getriwe}, manlîche sinne\\ 
 & mügen hie niht mê erwerben,\\ 
 & wan \textbf{daz} muoz ersterben\\ 
 & mîn armer lîp, den ich hie trage,\\ 
 & nâch im mit herzenlîcher klage."\\ 
15 & dô man \textbf{des} ezzens verpflac,\\ 
 & dô was ez \textbf{ouch} \textbf{über den} \textbf{mitten tac}.\\ 
 & Artus unde daz wîp sîn,\\ 
 & vrou Schinover, diu künigîn,\\ 
 & mit \textbf{rîtern} unde mit vrouwen schar\\ 
20 & riten, dâ der wolgevar\\ 
 & saz bî werder vrouwen diet.\\ 
 & \textbf{Parcival} \textbf{enpfâhen} \textbf{dô} geriet;\\ 
 & manige clâre vrouwen\\ 
 & muoser sich küssen schouwen.\\ 
25 & Artus bôt im êre\\ 
 & unde danket im des \textbf{vil} sêre,\\ 
 & \hspace*{-.7em}\big| daz er den brîs vür alle man\\ 
30 & \hspace*{-.7em}\big| von rehten schulden solde hân\\ 
 & \hspace*{-.7em}\big| \textbf{unde} daz sîn hôhiu werdecheit\\ 
 & \hspace*{-.7em}\big| \textbf{was} sô lanc unde \textbf{sô} breit.\\ 
\end{tabular}
\scriptsize
\line(1,0){75} \newline
G I L M Z \newline
\line(1,0){75} \newline
\textbf{1} \textit{Initiale} G I Z  \textbf{3} \textit{Initiale} L  \textbf{17} \textit{Initiale} I  \newline
\line(1,0){75} \newline
\textbf{1} Dô] Da M Z  $\cdot$ ouch si] si ouch M \textbf{4} sie] die I \textbf{6} hie] ie I \textbf{9} mîn] minen Z \textbf{10} getriwe] getriwelich I \textbf{12} ersterben] ersterber Z \textbf{15} dô] Da M  $\cdot$ des] daz L \textbf{16} dô] Da M  $\cdot$ den] \textit{om.} L Z \textbf{18} Schinover] kinover G Ginouere I Gynover L (Z) ginover M \textbf{21} werder] weder L \textbf{22} Er hat werdes gegen biet L  $\cdot$ Parcival] Parcifal G Parzifal I M Parcifals Z  $\cdot$ dô] \textit{om.} M \textbf{24} sich küssen] sich chusshende I kuͯszen L (M) (Z) \textbf{25} Artus] Artuͯs L \textbf{26} danket] danc I dankete M (Z)  $\cdot$ des] \textit{om.} M \textbf{29} \textit{Versfolge 698.27-28-29-30} Z  \textbf{30} schulden solde] schulde M \textbf{27} unde] \textit{om.} Z \textbf{28} was] Wer Z  $\cdot$ unde] vnd ouch Z \newline
\end{minipage}
\hspace{0.5cm}
\begin{minipage}[t]{0.5\linewidth}
\small
\begin{center}*T
\end{center}
\begin{tabular}{rl}
 & dô wart ouch si nâ\textit{ch jâ}mer var,\\ 
 & ir süezer munt meit ezzen gar.\\ 
 & si \textbf{gedâhte}: "waz tuot Bene hie?\\ 
 & \textbf{nû hete ich} \textbf{doch} gesendet sie\\ 
5 & zuo dem, der \textbf{dort} mîn herze treget,\\ 
 & daz mich \textbf{doch} hie unsanfte reget.\\ 
 & \textbf{waz} ist an mir gerochen?\\ 
 & hât der künec \textbf{versprochen}\\ 
 & mîn dienst und mîne minne?\\ 
10 & sîn \textbf{getriuwen}, manlîche sinne\\ 
 & mogen hie niht mê erwerben,\\ 
 & wan \textbf{daz} muoz ersterben\\ 
 & mîn arme\textit{r} lîp, den ich hie trage,\\ 
 & nâch im \textit{mi}t herzeclîche\textit{r} klage".\\ 
15 & \begin{large}D\end{large}ô man \textbf{des} ezzens \textbf{d\textit{â}} verpflac,\\ 
 & dô was ez \textbf{über den} \textbf{mittentac}.\\ 
 & Artus und daz wîp sîn,\\ 
 & vrou Gynover, diu künegîn,\\ 
 & mit \textbf{rîtern} und mit vrouwen schar\\ 
20 & riten, d\textit{â} der wol gevar\\ 
 & saz bî werde\textit{r} vrouwen diet.\\ 
 & \textbf{Parcifals} \textbf{entvâhen} geriet;\\ 
 & manege clâre vrouwen\\ 
 & muos er sich küssen schouwen.\\ 
25 & Artus bôt im êre\\ 
 & und danket im des \textbf{vil} sêre,\\ 
 & \hspace*{-.7em}\big| daz er den prîs vür alle man\\ 
30 & \hspace*{-.7em}\big| von rehten schulden solte hân\\ 
 & \hspace*{-.7em}\big| \textbf{und} daz sîne hôhiu wirdecheit\\ 
 & \hspace*{-.7em}\big| \textbf{was} sô lanc und \textbf{sô} breit.\\ 
\end{tabular}
\scriptsize
\line(1,0){75} \newline
U V W Q R \newline
\line(1,0){75} \newline
\textbf{1} \textit{Capitulumzeichen} R  \textbf{15} \textit{Initiale} U V W  \newline
\line(1,0){75} \newline
\textbf{1} dô] Da R  $\cdot$ ouch si] sie auch Q (R)  $\cdot$ nâch jâmer] namer U \textbf{4} hete] hatt R \textbf{5} der] \textit{om.} W  $\cdot$ dort] doch R  $\cdot$ mîn herze treget] treit meyn hertze Q \textbf{6} Vnd ich hie lebe mit smertze Q  $\cdot$ hie unsanfte reget] vnsanfftt hie wegt R \textbf{8} versprochen] gesprochen W \textbf{10} getriuwen] guͤtlich V getrewe Q trᵫwe R  $\cdot$ manlîche] manlichen W \textbf{12} ersterben] [*]: drvmbe ersterben V \textbf{13} armer] armen U [armen]: armer V \textbf{14} mit] ist U [*]: mit V  $\cdot$ herzeclîcher] herzecliche U \textbf{15} des ezzens] daz essen R  $\cdot$ dâ] do U V W Q R  $\cdot$ verpflac] verpflege Q \textbf{16} ez] auch Q es wol R  $\cdot$ den] \textit{om.} R  $\cdot$ mittentac] mitte tage Q \textbf{17} Artus] Kúnig artus W \textbf{18} Gynover] Gynovere V tschinouer W ginouer Q Gynower R \textbf{19} mit] \textit{om.} R \textbf{20} dâ] do U V W Q \textbf{21} saz] Das Q  $\cdot$ werder] werden U R \textbf{22} Parcifals] Parzifals U Parzefal V Partzifals W Partzifales Q Parczifals R  $\cdot$ geriet] [*]: do geriet V \textbf{24} muos] Muͤst V (Q)  $\cdot$ küssen] kussende Q \textbf{25} Artus] Kúnig artus W  $\cdot$ êre] groß ere W [mere]: ere Q \textbf{26} Vnd dankete des [imm* vil sere]: imme sere V  $\cdot$ vil] \textit{om.} R \textbf{29} daz] Dar Q \textbf{27} hôhiu] hoche R \textbf{28} was] Wer Q \newline
\end{minipage}
\end{table}
\end{document}
