\documentclass[8pt,a4paper,notitlepage]{article}
\usepackage{fullpage}
\usepackage{ulem}
\usepackage{xltxtra}
\usepackage{datetime}
\renewcommand{\dateseparator}{.}
\dmyyyydate
\usepackage{fancyhdr}
\usepackage{ifthen}
\pagestyle{fancy}
\fancyhf{}
\renewcommand{\headrulewidth}{0pt}
\fancyfoot[L]{\ifthenelse{\value{page}=1}{\today, \currenttime{} Uhr}{}}
\begin{document}
\begin{table}[ht]
\begin{minipage}[t]{0.5\linewidth}
\small
\begin{center}*D
\end{center}
\begin{tabular}{rl}
\textbf{732} & \begin{large}N\end{large}û dâhte aber Parzival\\ 
 & an sîn wîp, die lieht gemâl,\\ 
 & und an ir kiuschen süeze.\\ 
 & ob er kein ander grüeze,\\ 
5 & \textbf{daz} er dienst nâch minnen biete\\ 
 & unt sich unstæte niete?\\ 
 & solch minne \textbf{wirt} vom im gespart.\\ 
 & grôz triwe het \textbf{im} sô bewart\\ 
 & sîn manlîch herze und \textbf{ouch} \textbf{den} lîp,\\ 
10 & \textbf{daz} \textbf{vür wâr} \textbf{nie} ander wîp\\ 
 & \textbf{wart} gewaldec sîner minne\\ 
 & niwan diu küneginne\\ 
 & Condwiramurs,\\ 
 & diu geflôrierte bêâflûrs.\\ 
15 & \textbf{Er dâhte}: "sît ich minnen kan,\\ 
 & wie hât \textbf{diu} minne an mir getân?\\ 
 & nû bin ich doch ûz minne \textbf{erborn}:\\ 
 & wie hân ich \textbf{minne alsus} verlorn?\\ 
 & \multicolumn{1}{l}{ - - - }\\ 
20 & \multicolumn{1}{l}{ - - - }\\ 
 & \multicolumn{1}{l}{ - - - }\\ 
 & \multicolumn{1}{l}{ - - - }\\ 
 & Sol ich mit den ougen vreude sehen\\ 
 & und muoz \textbf{mîn} herze jâmers jehen,\\ 
25 & diu werc stênt ungelîche.\\ 
 & hôhes muotes rîche\\ 
 & wirt niemen solher pflihte.\\ 
 & gelücke mich berihte,\\ 
 & waz mir\textbf{z} wægeste drumbe sî."\\ 
30 & \textbf{im lac sîn harnasch} nâhe bî.\\ 
\end{tabular}
\scriptsize
\line(1,0){75} \newline
D \newline
\line(1,0){75} \newline
\textbf{1} \textit{Initiale} D  \textbf{15} \textit{Majuskel} D  \textbf{23} \textit{Majuskel} D  \newline
\line(1,0){75} \newline
\textbf{1} Parzival] Parcival D \textbf{13} Condwiramurs] Condwir amvrs D \textbf{19} \textit{Die Verse 732.19-22 fehlen} D  \newline
\end{minipage}
\hspace{0.5cm}
\begin{minipage}[t]{0.5\linewidth}
\small
\begin{center}*m
\end{center}
\begin{tabular}{rl}
 & nû dâht aber Parcifal\\ 
 & an sîn wîp, die lieht gemâl,\\ 
 & und an ir kiuschen süeze.\\ 
 & ob er kein ander grüeze,\\ 
5 & \textbf{der} er diens\textit{t} nâch minnen biete\\ 
 & und sich unstæte niete?\\ 
 & solich minne \textbf{wirt} von ime gespart.\\ 
 & grôz triuwe het \textbf{in} sô bewart,\\ 
 & sîn manlîch herz und \textbf{ouch} \textbf{den} lîp,\\ 
10 & \textbf{daz} \textbf{vür wâr} \textbf{nie} ande\textit{r} wîp\\ 
 & \textbf{wart} gewaltic sîner minne\\ 
 & niht wan diu küniginne,\\ 
 & \textbf{diu reine} Condwieramurs,\\ 
 & diu geflôriert bêâflûrs.\\ 
15 & \textbf{er dâht}: "sît ich minnen kan,\\ 
 & wie hât minne an mir getân?\\ 
 & nû bin ich doch ûz minne \textbf{erborn}:\\ 
 & wie hân ich \textbf{minne alsus} verlorn?\\ 
 & \textbf{sol} ich nâch dem Grâl ringen,\\ 
20 & \textbf{sô} \textbf{muoz} mich iemer twingen\\ 
 & ir \textbf{kiuschlîcher} umbevanc,\\ 
 & von der ich schiet, \textbf{daz} ist \textbf{sô} lanc.\\ 
 & sol ich mit den ougen vröude sehen\\ 
 & und muoz \textbf{mîn} herz jâmers jehen,\\ 
25 & diu werc stânt ungelîch.\\ 
 & hôhes muotes rîch\\ 
 & wirt niemen solicher pfliht.\\ 
 & glück mich beriht,\\ 
 & waz mir \textit{\textbf{daz}} wægest dâr umb sî."\\ 
30 & \textbf{sîn harnasch lac im} nâhe bî.\\ 
\end{tabular}
\scriptsize
\line(1,0){75} \newline
m n o Fr69 \newline
\line(1,0){75} \newline
\newline
\line(1,0){75} \newline
\textbf{1} dâht] gedochte n \textbf{3} kiuschen] kussen o \textbf{4} ob er] Aber o \textbf{5} dienst] dienste m o  $\cdot$ biete] bieten o \textbf{8} het in] hat n hett im o \textbf{9} den] sin n \textbf{10} ander] anders m \textbf{12} wan] wenne n \textbf{13} Condwieramurs] condiwier amirs n kuͯnwier amurs o \textbf{14} geflôriert] floriert de o \textbf{15} dâht] gedacht n  $\cdot$ ich] in o \textbf{17} erborn] erkorn o \textbf{19} sol] Solh n  $\cdot$ Grâl] grole n \textbf{21} kiuschlîcher] kústlicher o \textbf{24} jâmers] kombers n \textbf{27} pfliht] pflicte Fr69 \textbf{29} mir daz] mir m mriz Fr69 \newline
\end{minipage}
\end{table}
\newpage
\begin{table}[ht]
\begin{minipage}[t]{0.5\linewidth}
\small
\begin{center}*G
\end{center}
\begin{tabular}{rl}
 & \begin{large}N\end{large}û dâhte aber Parcival\\ 
 & an sîn wîp, die lieht gemâl,\\ 
 & unde an ir kiusche süeze.\\ 
 & ob er deheine ander grüeze,\\ 
5 & \textbf{daz} er dienst nâch minne biete\\ 
 & unde sich unstæte niete?\\ 
 & solch minne \textbf{wirt} von im gespart.\\ 
 & grôz triwe het \textbf{in} sô bewart,\\ 
 & sîn manlîch herze und \textbf{sînen} lîp.\\ 
10 & \textbf{vür wâr, ezne wart} ander wîp\\ 
 & gewaltec sîner minne\\ 
 & niwan diu küniginne\\ 
 & Condwiramurs,\\ 
 & diu geflô\textit{r}ierte bêâflûrs.\\ 
15 & \textbf{dô dâhter}: "sît ich minnen kan,\\ 
 & wie \textit{hât} \textbf{diu} minne an mir getân?\\ 
 & nû bin ich doch ûz minne \textbf{erkorn}:\\ 
 & wie hân ich \textbf{minne sus} verlorn?\\ 
 & \textbf{muoz} ich nâch dem Grâle ringen,\\ 
20 & \textbf{doch} \textbf{sol} mich imer twingen\\ 
 & ir \textbf{minneclîcher} umbevanc,\\ 
 & von der ich schiet, \textbf{es} ist \textbf{ze} lanc.\\ 
 & sol ich mit den ougen vröude sehen\\ 
 & unde muoz \textbf{mirz} herze jâmers jehen,\\ 
25 & diu werc stênt ungelîche.\\ 
 & hôhes muotes rîche\\ 
 & wirt niemen sölher pflihte.\\ 
 & gelücke mich berihte,\\ 
 & waz mir\textbf{z} wægest drumbe sî."\\ 
30 & \textbf{im lac sîn harnasch} nâhen bî.\\ 
\end{tabular}
\scriptsize
\line(1,0){75} \newline
G I L M Z Fr18 Fr24 \newline
\line(1,0){75} \newline
\textbf{1} \textit{Initiale} G Z Fr18 Fr24  \textbf{7} \textit{Initiale} I  \newline
\line(1,0){75} \newline
\textbf{1} Nû] do I  $\cdot$ dâhte] gidachte M  $\cdot$ Parcival] parcifal G Z (Fr18) (Fr24) parzifal I L M \textbf{2} die] \textit{om.} L  $\cdot$ lieht] licht L M \textbf{4} deheine] ichein M kein Z \textbf{5} er] er ir I  $\cdot$ dienst] dienste Fr24  $\cdot$ biete] bete M \textbf{7} minne] libe M  $\cdot$ wirt] wart L  $\cdot$ gespart] gesprach I \textbf{8} in] yme M (Z) (Fr18) \textbf{10} Ez enwart nie ander wibe L  $\cdot$ Ez en wart vur war ny ander wip M (Z) (Fr18) (Fr24)  $\cdot$ ezne wart] ez wart nie I \textbf{11} minne] libe M \textbf{12} niwan] Wanne M \textbf{13} Condwiramurs] Guntwir Amuͯrsz M Kvndwiramvrs Z Kondwir Amvrs Fr18 Gvndwir Amvrs Fr24 \textbf{14} geflôrierte] gefloierte G Gefloriert I (M)  $\cdot$ bêâflûrs] beakvrs L \textbf{15} dô] Da M  $\cdot$ dâhter] gedaht er Z \textbf{16} hât] \sout{dah} G \textbf{17} ich] \textit{om.} I  $\cdot$ doch ûz] vz der L  $\cdot$ erkorn] erborn Z Fr18 \textbf{18} sus] alsus I (Fr18) \textbf{19} Nach dem gral muz ich ringen Z  $\cdot$ muoz] muͤz I (L)  $\cdot$ Grâle] gral Z (Fr18) (Fr24) \textbf{20} imer] Nummer M \textbf{21} ir minneclîcher] Kvschlicher L (Fr18) Jr kussinlicher M (Z) (Fr24) \textbf{22} es] des I (L) (M) (Z) (Fr18) (Fr24)  $\cdot$ ze] \textit{om.} I \textbf{23} den] \textit{om.} L  $\cdot$ vröude] vroiden M \textbf{24} muoz] sol I  $\cdot$ mirz] mir Z \textbf{27} niemen] nimmer Z \textbf{28} berihte] des berihte I \newline
\end{minipage}
\hspace{0.5cm}
\begin{minipage}[t]{0.5\linewidth}
\small
\begin{center}*T
\end{center}
\begin{tabular}{rl}
 & \begin{large}N\end{large}û dâhte aber Parcifal\\ 
 & an sîn wîp, die lieht gemâl,\\ 
 & und an ir kiusche süeze.\\ 
 & o\textit{b} er dekein ander grüeze,\\ 
5 & \textbf{daz} er dienst nâch minne biete\\ 
 & und sich unstæte niete?\\ 
 & soliche minne \textbf{wart} von im gespart.\\ 
 & grôze triuwe hete sô bewart\\ 
 & sîn manlîch herze und \textbf{sîn} lîp.\\ 
10 & \textbf{ez enwart vür wâr} \textbf{nie} ander wîp\\ 
 & gewaltic sîner minne\\ 
 & niht wan diu küneginne\\ 
 & Kundewiramurs,\\ 
 & diu geflôrie\textit{r}te bêâflûrs.\\ 
15 & \textbf{dô dâht er}: "sît ich minnen kan,\\ 
 & wie hât \textbf{diu} minne an mir getân?\\ 
 & nû bin \textit{i}ch doch ûz minne \textbf{geborn}:\\ 
 & wie hân ich \textbf{sus minne} verlorn?\\ 
 & \textbf{muoz} ich nâch dem Grâle ringen,\\ 
20 & \textbf{doch} \textbf{muoz} mich iemer twingen\\ 
 & ir \textbf{kiuschlîcher} umbevanc,\\ 
 & von \textit{der} ich schiet, \textbf{daz} ist \textbf{zuo} lanc.\\ 
 & sol ich mit den ougen vreude sehen\\ 
 & und muoz \textbf{mîn} herze \textbf{mir} jâmers jehen,\\ 
25 & diu werc stênt ungelîche.\\ 
 & hôhes muotes rîche\\ 
 & wirt nieman solicher pflihte.\\ 
 & gelücke mich berichte,\\ 
 & waz mir \textbf{aller} wægeste dâr umb sî."\\ 
30 & \textbf{im lac sîn harnasch} nâhe bî.\\ 
\end{tabular}
\scriptsize
\line(1,0){75} \newline
U V W Q R \newline
\line(1,0){75} \newline
\textbf{1} \textit{Initiale} U V W Q R  \newline
\line(1,0){75} \newline
\textbf{1} Nû] dO W (Q)  $\cdot$ dâhte] gedachte W  $\cdot$ Parcifal] parzefal V partzifal W Q parczifal R \textbf{2} lieht] licht Q R \textbf{4} ob] Oder U  $\cdot$ dekein] deheine V (R) kein W keinen Q  $\cdot$ ander] anderrn Q \textbf{5} daz] [D*z]: Der V \textbf{6} niete] mite Q \textbf{7} wart] wúrt V (W) (Q) (R) \textbf{8} hete] hat [*]: im V het im W (Q) het Jn R \textbf{9} sîn lîp] [*]: och den lip V seinen leib W \textbf{10} enwart] ward R \textbf{12} niht wan] Nun wan Q \textbf{13} Kundewiramurs] Kuͦndewiramuͦrs U [*]: Die reine kvndewiramurs V Kondwiramúrs Q Kúndwir amuͯs R \textbf{14} geflôrierte] geflurete U  $\cdot$ bêâflûrs] [Bea*]: Beaflurs V \textbf{15} dô] Da R  $\cdot$ dâht] gedacht W R \textbf{17} ich] doch U  $\cdot$ geborn] erboren W erkorn Q R \textbf{18} sus minne] minne svz V (W) (R) minne als Q \textbf{19} muoz] [*]: Sol V  $\cdot$ Grâle] gral Q \textbf{20} doch] [*ch]: So V  $\cdot$ iemer] Jamer R \textbf{21} kiuschlîcher] kússenlicher W \textbf{22} der] \textit{om.} U  $\cdot$ daz] es W R des Q \textbf{24} muoz] sol R  $\cdot$ mîn herze mir] mirs herze V (W) (Q) (R)  $\cdot$ jâmers] iamer W \textbf{27} nieman] nymmer W \textbf{29} mir aller] mirs V W Q mir daz R  $\cdot$ wægeste dâr umb] [*]: drúmme wegest Q \newline
\end{minipage}
\end{table}
\end{document}
