\documentclass[8pt,a4paper,notitlepage]{article}
\usepackage{fullpage}
\usepackage{ulem}
\usepackage{xltxtra}
\usepackage{datetime}
\renewcommand{\dateseparator}{.}
\dmyyyydate
\usepackage{fancyhdr}
\usepackage{ifthen}
\pagestyle{fancy}
\fancyhf{}
\renewcommand{\headrulewidth}{0pt}
\fancyfoot[L]{\ifthenelse{\value{page}=1}{\today, \currenttime{} Uhr}{}}
\begin{document}
\begin{table}[ht]
\begin{minipage}[t]{0.5\linewidth}
\small
\begin{center}*D
\end{center}
\begin{tabular}{rl}
\textbf{1} & \begin{Large}I\end{Large}st zwîvel herzen nâchgebûr,\\ 
 & daz muoz der sêle werden sûr.\\ 
 & gesmæhet unde gezieret\\ 
 & \textbf{ist}, swâ sich parrieret\\ 
5 & \textbf{unverzaget} mannes muot,\\ 
 & als agelstern varwe tuot.\\ 
 & \textbf{der} mac dennoc\textit{h} wesen geil,\\ 
 & \textbf{wand} an \textbf{im} sint \textit{b}eidiu teil,\\ 
 & des himels und der helle.\\ 
10 & der unstæte gesell\textit{e}\\ 
 & hât die swarzen va\textit{rwe} gar\\ 
 & und wirt ouch nâch der vinster var.\\ 
 & sô habt sich an die blanken\\ 
 & der mit stæten gedanken.\\ 
15 & Diz vliegende bîspel\\ 
 & ist tumben liuten gar \textbf{ze} snel.\\ 
 & sine mugen\textbf{s} niht erdenken,\\ 
 & wand ez kan vor in wenk\textit{e}n\\ 
 & rehte \textbf{alsam} ein \textbf{schel\textit{l}ic} \textit{ha}se.\\ 
20 & \textbf{zin} anderhalb \textbf{an dem} glase\\ 
 & \textbf{gelîchent} und des blinden \textit{tro}um,\\ 
 & die gebent antlützes roum.\\ 
 & \textbf{doch} mac mit stæte \textit{nih}t gesîn\\ 
 & dirre \textbf{trüebe lîhte} schîn.\\ 
25 & \textbf{er} \textbf{machet} kurze vröude al wâ\textit{r}.\\ 
 & wer roufet mich, dâ nie dehein h\textit{âr}\\ 
 & gewuohs, innen \textbf{an} m\textit{în}er han\textit{t}?\\ 
 & der hât vil \textbf{nâhe griffe} erkant.\\ 
 & \textit{s}priche ich gein den \textbf{vorhten}: "\textbf{och}!",\\ 
30 & daz glîchet \textbf{mîner witze} \textbf{iedoch}.\\ 
\end{tabular}
\scriptsize
\line(1,0){75} \newline
D \newline
\line(1,0){75} \newline
\textbf{1} \textit{Großinitiale} D  \textbf{15} \textit{Versal} D  \newline
\line(1,0){75} \newline
\textbf{7} \textit{Nachkonturierungen von späterer Hand (1.7-4.8)} D   $\cdot$ dennoch] dennoc: \textit{nachträglich korrigiert zu:} dennoch D \textbf{8} beidiu] :eidiv \textit{nachträglich korrigiert zu:} beidiv D \textbf{10} geselle] gesell: \textit{nachträglich korrigiert zu:} geselle D \textbf{11} varwe] va::: \textit{nachträglich korrigiert zu:} varwe D \textbf{12} var] var \textit{nachträglich korrigiert zu:} varn D \textbf{14} stæten] steten \textit{nachträglich korrigiert zu:} stëten D \textbf{18} wenken] wenk:n \textit{nachträglich korrigiert zu:} wenken D \textbf{19} schellic] schelbich D  $\cdot$ hase] ::se \textit{nachträglich korrigiert zu:} hase D \textbf{21} troum] :::vm \textit{nachträglich korrigiert zu:} trovm D \textbf{23} niht] :::t \textit{nachträglich korrigiert zu:} niht D \textbf{25} wâr] wa: \textit{nachträglich korrigiert zu:} war D \textbf{26} hâr] h:: \textit{nachträglich korrigiert zu:} har D \textbf{27} mîner hant] m::er han: \textit{nachträglich korrigiert zu:} miner hant D \textbf{29} spriche] :priche \textit{nachträglich korrigiert zu:} spriche D \newline
\end{minipage}
\hspace{0.5cm}
\begin{minipage}[t]{0.5\linewidth}
\small
\begin{center}*m
\end{center}
\begin{tabular}{rl}
 & \begin{Large}I\end{Large}st zwîvel herzen nâchgebûr,\\ 
 & daz muoz der sêlen werden sûr.\\ 
 & gesmâhet und gezieret\\ 
 & \textbf{ist}, wâ sich par\textit{r}ieret\\ 
5 & \textbf{in eines verzageten} mannes muot,\\ 
 & alsô agelstern varwe tuot.\\ 
 & \textbf{der} mac dannoch wesen geil.\\ 
 & ane \textbf{im} sint bêdiu teil,\\ 
 & de\textit{s} himel\textit{s} und der helle.\\ 
10 & der unstæte geselle\\ 
 & hât die swarzen varwe gar\\ 
 & und wirt ouch nâch der vinster var.\\ 
 & sô habt sich an die blanken\\ 
 & der mit stæten gedanken.\\ 
15 & \begin{large}D\end{large}iz vliegende bîspel\\ 
 & ist tumben liuten gar snel.\\ 
 & si enmugen \textbf{es} niht erdenken,\\ 
 & wenne e\textit{z} kan vor in wenken,\\ 
 & reht \textbf{alsô} ein \textbf{schalkehter} hase\\ 
20 & \textbf{zuo} anderhalb \textbf{dem} g\textit{l}ase\\ 
 & \textbf{glîchet} und des blinden troum,\\ 
 & die gebent antlitzes roum.\\ 
 & \textbf{doch} mac mit stæte niht gesîn\\ 
 & d\textit{i}rre \textbf{trüebe liehte} schîn.\\ 
25 & \textbf{er} \textbf{machet} kurze vröude al wâr.\\ 
 & wer roufet mich, d\textit{â} nie kein hâr\\ 
 & gewuohs, innen \textbf{in} mîner hant?\\ 
 & der hât vil \textbf{nâhe griffe} er\textit{k}ant.\\ 
 & sprich ich gegen den \textbf{vorhten}: "\textbf{och}!",\\ 
30 & daz glîchet \textbf{mîner witze} \textbf{doch}.\\ 
\end{tabular}
\scriptsize
\line(1,0){75} \newline
m n o W \newline
\line(1,0){75} \newline
\textbf{1} \textit{Großinitiale} m n o W  \textbf{15} \textit{Initiale} m  \newline
\line(1,0){75} \newline
\textbf{4} sich] sy W  $\cdot$ parrieret] parnieret m \textbf{6} agelstern] aglester n (o) (W) \textbf{7} dannoch] darnach W  $\cdot$ geil] [gel]: geil o \textbf{8} ane] Wenne an n (o) (W)  $\cdot$ sint] sein W \textbf{9} des himels] Der himel m  $\cdot$ helle] hellen W \textbf{10} unstæte] vnstendige W  $\cdot$ geselle] gesellen W \textbf{11} hât] Hette n  $\cdot$ swarzen] swartze m n (o) (W) \textbf{12} wirt] ist W  $\cdot$ ouch nâch der] ouch dar noch o nach der W \textbf{13} habt] hebent W \textbf{16} snel] >snel< m zuͦ snel n o \textbf{17} si] Die n o W  $\cdot$ enmugen] mogent n (o) (W) \textbf{18} ez] er m n o \textbf{19} alsô] [alser]: also so n  $\cdot$ schalkehter] schalkechter \textit{nachträglich korrigiert zu:} schillechter m schilechter n (o) erschelter W \textbf{20} glase] [g*]: grase m grase o \textbf{22} antlitzes] alle antlútz W \textbf{24} dirre] Drre m  $\cdot$ trüebe liehte] truͯbelechter n trubelehte o (W) \textbf{25} al wâr] alwer o \textbf{26} roufet] ropffet W  $\cdot$ dâ] do m n o W  $\cdot$ nie] \textit{om.} n \textbf{28} vil] so W  $\cdot$ nâhe] nohe uͯff o  $\cdot$ erkant] erhant m \textbf{29} vorhten] fú:::en n fursten o (W)  $\cdot$ och] noch n o hoch W \newline
\end{minipage}
\end{table}
\newpage
\begin{table}[ht]
\begin{minipage}[t]{0.5\linewidth}
\small
\begin{center}*G
\end{center}
\begin{tabular}{rl}
 & \begin{Large}I\end{Large}st zwîvel herzen nâchgebûr,\\ 
 & daz muoz der sêle werden sûr.\\ 
 & \textbf{jâ} \textbf{ist} gesmæhet und gezieret,\\ 
 & swâ sich parrieret\\ 
5 & \textbf{unverzaget} mannes muot,\\ 
 & als agelsternen varwe tuot.\\ 
 & \textbf{der} mac dannoch wesen geil,\\ 
 & \textbf{wan} an \textbf{dem} sint beidiu teil,\\ 
 & des himeles und der helle.\\ 
10 & der unstæte geselle,\\ 
 & \textbf{der} hât die swarzen varwe gar\\ 
 & unde wirt ouch nâch der vinster var.\\ 
 & sô habet sich an die blanken\\ 
 & der mit stæten gedanken.\\ 
15 & diz vliegende bîspel\\ 
 & ist tumben liuten gar \textbf{ze} snel.\\ 
 & sine mugen\textbf{s} \textbf{in} niht erdenken,\\ 
 & wan ez kan vor in wenken\\ 
 & rehte \textbf{alsam} ein \textbf{schellec} hase.\\ 
20 & \textbf{zin} anderhalp \textbf{ame} glase\\ 
 & \textbf{gelîchet} und des blinden troum,\\ 
 & die gebent antlützes roum\\ 
 & \multicolumn{1}{l}{ - - - }\\ 
 & \multicolumn{1}{l}{ - - - }\\ 
25 & \textbf{unde} \textbf{machent} kurze vröude al wâr.\\ 
 & wer roufet mich, dâ nie nehein hâr\\ 
 & gewuohs, innen \textbf{an} mîner hant?\\ 
 & der hât vil \textbf{nâhen grif} erkant.\\ 
 & sprich ich gein den \textbf{vorhten} "\textbf{och}!",\\ 
30 & daz gelîchet \textbf{mînen witzen} \textbf{doch}.\\ 
\end{tabular}
\scriptsize
\line(1,0){75} \newline
G O L M Q Z Fr58 \newline
\line(1,0){75} \newline
\textbf{1} \textit{Überschrift:} Hie hebet an das buͯch von Gahmuͯret der waz parcifals vatter L  Hye hebet sich an die aűentewr von parcifal vnd Gaműret Q   $\cdot$ \textit{Großinitiale} G  \textsuperscript{1}\hspace{-1.3mm} O  \textsuperscript{2}\hspace{-1.3mm} O L M Q Z Fr58  \newline
\line(1,0){75} \newline
\textbf{1} \textit{Versdoppelung 1.1.3.7 (²O) nach 42.20 und 1.2.4.6 (²O) nach 43.25 radiert; Lesarten der vorausgehenden Verse mit ¹O bezeichnet} O   $\cdot$ Ist] ÷ST \textsuperscript{2}\hspace{-1.3mm} O  $\cdot$ Ist] Sit L ÷::: Fr58  $\cdot$ herzen] des herczin M \textbf{2} muoz] mag Q  $\cdot$ sêle] selen M \textbf{3} jâ ist] \textit{om.} M Q Z Fr58  $\cdot$ gesmæhet] Gemachet Q  $\cdot$ gezieret] ge erit ist M \textbf{4} swâ] Wa L Jst wo Q  $\cdot$ parrieret] vor irret list M partiret Q \textbf{5} unverzaget] vnd verzagt \textsuperscript{2}\hspace{-1.3mm} O Man saget >verzagten< L Vnde vnuorczagites M Vnverzagetes Z (Fr58) \textbf{6} agelsternen] agelster O L (M) (Z) (Fr58) der agelester Q \textbf{8} dem] im O L (M) Q Z Fr58 \textbf{9} helle] erde \textit{nachträglich korrigiert zu:} helle O \textbf{10} unstæte] vnstetige M \textbf{11} der] \textit{om.} Q  $\cdot$ swarzen] Swarcze M (Q)  $\cdot$ varwe] farben Q \textbf{12} nâch der] darnach O noch Q der Fr58  $\cdot$ var] [gar]: var M \textbf{13} habet] habent O (M) \textbf{14} \textit{Versdoppelung 1.18 nach 1.14:} Wann esz kan vor in wencken Q   $\cdot$ der] Die O \textit{om.} Z Fr58  $\cdot$ mit] mit den O M \textbf{15} diz] Daz Z Fr58  $\cdot$ vliegende] fligenden Z \textbf{17} sine mugens in] Sin enmugen sin in O Sie muͯgen es L Sine muge * \textit{nachträglich korrigiert zu:} Sine muge jnns M Sie enmúgens Q Sie enmugent ins Z (Fr58) \textbf{18} vor] \textit{om.} M von Q  $\cdot$ wenken] entwenkin M \textbf{19} alsam] als O L (M)  $\cdot$ ein] \textit{om.} O \textbf{20} zin] Schin L Czihen M Zwen Q \textbf{21} gelîchet] Gelicket Q  $\cdot$ und] sich L  $\cdot$ blinden] [blinden]: blindem M \textbf{22} gebent] gaubent Q  $\cdot$ antlützes] anders M \textbf{23} \textit{Die Verse 1.23-24 fehlen} G   $\cdot$ \textit{Versfolge 1.24-23} Fr58   $\cdot$ Auch (Doch Q ) mach mit stete niht gesein O (L) (M) (Z) (Fr58) \textbf{24} Dirre trvͤbe lîhte (liechte Q liehter Z [ Fr58 ]) schein O (L) (M) (Q) (Z) (Fr58) \textbf{25} unde machent] Der machet O (M) Z Fr58 Er machet L Q  $\cdot$ al] alle M Fr58 \textbf{26} dâ] do Q  $\cdot$ nie nehein] nikeyn M (Q) (Z) \textbf{27} innen] inn O (M) ýnnerhalp L nye Q \textbf{28} hât] \textit{om.} L  $\cdot$ grif] griffe Q Z \textbf{29} vorhten] worten L  $\cdot$ och] ouch M \textbf{30} gelîchet] [gelichen]: gelichet Z  $\cdot$ mînen witzen] meiner witze Q  $\cdot$ doch] och L ydoch Q \newline
\end{minipage}
\hspace{0.5cm}
\begin{minipage}[t]{0.5\linewidth}
\small
\begin{center}*T
\end{center}
\begin{tabular}{rl}
 & \begin{Large}I\end{Large}st zwîvel herzen nâchgebûr,\\ 
 & daz muoz der sêle werden sûr.\\ 
 & \textbf{Jâ} gesmæhet und gezieret\\ 
 & \textbf{Ist}, swâ sich parrieret\\ 
5 & \textbf{unverzagetes} mannes muot,\\ 
 & als agelstern varwe tuo\textit{t}.\\ 
 & \textbf{er} mac dannoch wesen geil,\\ 
 & \textbf{wan} an \textbf{im} sint beidiu teil,\\ 
 & des himels und der helle.\\ 
10 & der unstæte geselle\\ 
 & hât die swarzen varwe gar\\ 
 & und wirt ouch nâch der vinster var.\\ 
 & Sô habt sich an die blanken\\ 
 & der mit \textbf{den} stæten gedanken.\\ 
15 & Diz vliegende bîspel\\ 
 & ist tumben liuten gar \textbf{ze} snel.\\ 
 & sine mugen\textbf{z} niht erdenken,\\ 
 & wan ez kan vor in wenken\\ 
 & rehte \textbf{alsam} ein \textbf{schellic} hase.\\ 
20 & \textbf{zin} anderhalp \textbf{anme} glase\\ 
 & \textbf{gelicket} und des blinden troum,\\ 
 & die gebent antlitzes roum.\\ 
 & \textbf{Ouch} mac mit stæte niht gesîn\\ 
 & dirre \textbf{liehte trüebe} schîn.\\ 
25 & \textbf{der} \textbf{machet} kurze vröude al wâr.\\ 
 & Swer roufet mich, dâ nie kein hâr\\ 
 & gewuohs, innen \textbf{an} mîner hant?\\ 
 & der hât vil \textbf{nâhen grif} erkant.\\ 
 & sprich ich gegen den \textbf{worten} \textbf{ouch},\\ 
30 & daz glîchet \textbf{mînen witzen} \textbf{doch}.\\ 
\end{tabular}
\scriptsize
\line(1,0){75} \newline
T U V Fr32 \newline
\line(1,0){75} \newline
\textbf{1} \textit{Großinitiale} T U V Fr32  \textbf{3} \textit{Majuskel} T  \textbf{4} \textit{Majuskel} T  \textbf{15} \textit{Initiale} Fr32   $\cdot$ \textit{Majuskel} T  \textbf{23} \textit{Majuskel} T  \textbf{26} \textit{Majuskel} T  \newline
\line(1,0){75} \newline
\textbf{2} sêle] selen U (V) \textbf{4} swâ] wo U \textbf{5} unverzagetes] [Vn*]: Vnverzaget V \textbf{6} agelstern] ageleister V  $\cdot$ tuot] tvͦ: T \textbf{7} er] Der V \textbf{9} helle] hellen Fr32 \textbf{10} unstæte geselle] vnstergeselle Fr32 \textbf{11} swarzen] swarze U \textbf{13} habt] habent V hebt Fr32 \textbf{14} der] Die V \textbf{16} tumben] tuͦmbe U  $\cdot$ gar] \textit{om.} U \textbf{17} sine mugenz] Sine muͦgens U (V) si mvgen ez Fr32 \textbf{19} alsam] als U  $\cdot$ schellic] schelketh U \textbf{20} \textit{Vers 1.20 fehlt} U   $\cdot$ zin] Vnd Fr32 \textbf{21} gelicket] Glichet V (Fr32)  $\cdot$ troum] trôume Fr32 \textbf{22} roum] zôume Fr32 \textbf{23} mac] \textit{om.} Fr32  $\cdot$ gesîn] sin U \textbf{24} dirre] dirrre Fr32  $\cdot$ liehte trüebe] truͦebe lichte U truͤbe liehte V (Fr32) \textbf{26} Swer] Wer U (Fr32)  $\cdot$ roufet] ruͦft U  $\cdot$ dâ] do V \textbf{27} an] in Fr32 \textbf{29} ich] \textit{om.} U  $\cdot$ den worten ouch] [d*]: den forhten noch V \textbf{30} mînen] miner V  $\cdot$ doch] [*]: iedoch V \newline
\end{minipage}
\end{table}
\end{document}
