\documentclass[8pt,a4paper,notitlepage]{article}
\usepackage{fullpage}
\usepackage{ulem}
\usepackage{xltxtra}
\usepackage{datetime}
\renewcommand{\dateseparator}{.}
\dmyyyydate
\usepackage{fancyhdr}
\usepackage{ifthen}
\pagestyle{fancy}
\fancyhf{}
\renewcommand{\headrulewidth}{0pt}
\fancyfoot[L]{\ifthenelse{\value{page}=1}{\today, \currenttime{} Uhr}{}}
\begin{document}
\begin{table}[ht]
\begin{minipage}[t]{0.5\linewidth}
\small
\begin{center}*D
\end{center}
\begin{tabular}{rl}
\textbf{608} & \textit{\begin{large}I\end{large}}ch wil iwer bote sîn.\\ 
 & gebt mir her \textbf{daz} vingerlîn\\ 
 & unt lât mich iwern dienst sagen\\ 
 & unt iwern kumber niht \textbf{verdagen}."\\ 
5 & Der künec des dankte sêre.\\ 
 & Gawan \textbf{vrâgte in} mêre:\\ 
 & "sît iu versmæhet gein mir strît,\\ 
 & \textbf{nû} sagt mir, hêrre, wer ir sît."\\ 
 & "ir \textbf{en}sultz niht vür laster doln",\\ 
10 & Sprach der künec. "mîn name ist \textbf{iu verholn}?\\ 
 & mîn vater, \textbf{der} hiez Irot.\\ 
 & den \textbf{ersluoc} der künec Lot.\\ 
 & ich binz, der künec Gramoflanz.\\ 
 & mîn hôhez herze \textbf{ie was} sô ganz,\\ 
15 & daz ich ze keinen zîten\\ 
 & nimmer wil \textbf{gestrîten},\\ 
 & swaz mir tæte ein man,\\ 
 & wan einer, heizet Gawan,\\ 
 & von dem \textit{ich} prîs hân vernomen,\\ 
20 & daz ich \textbf{gerne} gein im wolte komen\\ 
 & ûf \textbf{strît} durch mîne riwe.\\ 
 & sîn vater, \textbf{der} brach triwe:\\ 
 & \textbf{ime} gruoze \textbf{er mînen vater} sluoc.\\ 
 & ich hân ze \textbf{sprechen} dâr genuoc.\\ 
25 & Nû ist Lot erstorben\\ 
 & und hât Gawan erworben\\ 
 & solhen prîs vor ûz besunder,\\ 
 & daz ob der tavelrunder\\ 
 & im prîses niemen gelîchen mac.\\ 
30 & ich geleb noch gein im strîtes tac."\\ 
\end{tabular}
\scriptsize
\line(1,0){75} \newline
D Z \newline
\line(1,0){75} \newline
\textbf{1} \textit{Großinitiale} Z   $\cdot$ \textit{Initiale} D  \textbf{5} \textit{Majuskel} D  \textbf{10} \textit{Majuskel} D  \textbf{25} \textit{Majuskel} D  \newline
\line(1,0){75} \newline
\textbf{1} Ich] ÷ch D \textbf{6} vrâgte in] der fragte Z \textbf{7} gein] nu gein Z \textbf{8} nû] So Z \textbf{9} ensultz] svltz Z \textbf{10} iu verholn] vnverholn Z \textbf{11} Irot] Jrot D Gyrot Z \textbf{13} Gramoflanz] Gramoͮlanz D Gramoflantz Z \textbf{14} ie was] was ie Z \textbf{18} heizet] heizzen Z \textbf{19} ich] \textit{om.} D \textbf{21} strît] strite Z \textbf{29} prîses niemen gelîchen] gelichen nieman prises Z \newline
\end{minipage}
\hspace{0.5cm}
\begin{minipage}[t]{0.5\linewidth}
\small
\begin{center}*m
\end{center}
\begin{tabular}{rl}
 & ich wil iuwer bote sîn.\\ 
 & gebt mir her \textbf{daz} vingerlîn\\ 
 & und lât mich iuwern dienst sagen\\ 
 & und iuwer\textit{en} kumber niht \textbf{vertragen}."\\ 
5 & der künic des dankte sêre.\\ 
 & Gawan \textbf{vrâge\textit{te} in} mêre:\\ 
 & "sît iu versmâhet gegen mir strît,\\ 
 & \textbf{nû} sagt mir, hêrre, wer ir sît."\\ 
 & "ir solt ez niht vür laster do\textit{l}n",\\ 
10 & sprach der künic. "mîn nam ist \textbf{verholn}?\\ 
 & mîn vater, \textbf{der} hiez Irot.\\ 
 & de\textit{n} \textbf{ersluoc} de\textit{r} künic Lot.\\ 
 & ich binz, der künic Gram\textit{o}lanz.\\ 
 & mîn hôhez herz \textbf{ie was} sô ganz,\\ 
15 & daz ich zuo keinen zîten\\ 
 & niemer wil \textbf{strîten},\\ 
 & waz mir \textbf{halt} tæte ein man,\\ 
 & wan einer, heizet Gawan,\\ 
 & von dem ich prîs hân vernomen,\\ 
20 & daz \textit{ich} \textbf{gern} gegen im \textbf{noch} wolt komen\\ 
 & ûf \textbf{strît} durch mîn riuwe.\\ 
 & sîn vater brach \textbf{sîn} triuwe:\\ 
 & \textbf{in dem} gruoz \textbf{er mînen vater} sluoc.\\ 
 & ich hân zuo \textbf{rechen} dâr genuoc.\\ 
25 & nû ist Lot erstorben\\ 
 & und het Gawan erworben\\ 
 & solichen prîs vor ûz besunder,\\ 
 & daz ob der tavelrunder\\ 
 & im prîses niemen glîchen mac.\\ 
30 & ich gelebe noch gegen ime strîtes tac."\\ 
\end{tabular}
\scriptsize
\line(1,0){75} \newline
m n o \newline
\line(1,0){75} \newline
\newline
\line(1,0){75} \newline
\textbf{1} bote] gotte o \textbf{2} her] \textit{om.} n \textbf{4} iuweren] uͯwer m (o)  $\cdot$ vertragen] vertagen o \textbf{6} vrâgete] froge m \textbf{9} doln] dorn m \textbf{10} verholn] verholn \textit{nachträglich korrigiert zu:} vnverholn m \textbf{11} der] [des]: der o  $\cdot$ Irot] jrot m o \textbf{12} den] der m  $\cdot$ ersluoc] elsluͦg o  $\cdot$ der] den m \textbf{13} Gramolanz] gramonlancz m o gramonlantz n \textbf{16} strîten] gestriten n (o) \textbf{17} halt] halte o \textbf{18} \textit{Die Verse 608.18-609.30 fehlen (Blattverlust)} o  \textbf{20} daz ich] daz \textit{nachträglich korrigiert zu:} daz ich m  $\cdot$ noch] \textit{om.} n \textbf{22} sîn triuwe] die truwe n \textbf{29} prîses niemen] nẏeman prises n \newline
\end{minipage}
\end{table}
\newpage
\begin{table}[ht]
\begin{minipage}[t]{0.5\linewidth}
\small
\begin{center}*G
\end{center}
\begin{tabular}{rl}
 & ich wil iuwer bote sîn.\\ 
 & gebet mir her \textbf{ditze} vingerlîn\\ 
 & unde lât mich iuwern dienst sagen\\ 
 & unde iuwern kumber niht \textbf{verdagen}."\\ 
5 & der künic des danket sêre.\\ 
 & Gawan \textbf{in vrâ\textit{g}t\textit{e}} mêre:\\ 
 & "sît iu versmâhet gein mi\textit{r} strît,\\ 
 & \textbf{sô} saget mir, hêrre, wer ir sît.\\ 
 & ir\textbf{n} sult ez niht vür laster doln."\\ 
10 & \textbf{dô} sprach der künic: "mîn nam ist \textbf{unverhol\textit{n}}.\\ 
 & mîn vater hiez Gyrot.\\ 
 & den \textbf{sluoc} der künic Lot.\\ 
 & ich binz, der künic Gramoflanz.\\ 
 & mîn hôhez herze \textbf{was ie} sô ganz,\\ 
15 & daz ich ze \textit{de}heinen zîten\\ 
 & nimmer wil \textbf{strîten},\\ 
 & swaz mir tæte ein man,\\ 
 & wan einer, heizet Gawan,\\ 
 & von dem ich prîs hân vernomen,\\ 
20 & daz ich gein im w\textit{o}l\textit{d}e komen\\ 
 & ûf \textbf{kampf} durch mîn riuwe.\\ 
 & sîn vater brach \textbf{sî\textit{n}} triuwe:\\ 
 & \textit{\textbf{in einem}} gruoze \textbf{mînen vater \textit{er}} sluoc.\\ 
 & ich hân ze \textbf{sprechen} dâr genuoc.\\ 
25 & nû ist Lot erstorben\\ 
 & unde hât Gawan erworben\\ 
 & solhen prîs vor ûz besunder,\\ 
 & daz obe der tavelrunder\\ 
 & im brîses niemen gelîchen mac.\\ 
30 & ich gelebe noch gein ime strîtes tac."\\ 
\end{tabular}
\scriptsize
\line(1,0){75} \newline
G I L M Z Fr51 \newline
\line(1,0){75} \newline
\textbf{1} \textit{Initiale} L Z Fr51  \textbf{7} \textit{Initiale} I  \newline
\line(1,0){75} \newline
\textbf{2} ditze] daz I L (M) (Fr51) Z \textbf{3} mich] \textit{om.} I \textbf{5} des] im I  $\cdot$ danket] dancte I M Z (Fr51) \textbf{6} in] der Z im Fr51  $\cdot$ vrâgte] fragit G \textbf{7} gein] nu gein Z  $\cdot$ mir] min G \textbf{8} mir] mir doch I \textbf{9} irn sult ez] Jr svltz L Z Jr suͦlns Fr51 \textbf{10} dô] \textit{om.} L M Z Fr51  $\cdot$ ist] ister I  $\cdot$ unverholn] unverholnen G verholn I M vnverstoln Fr51 \textbf{11} hiez] der hiesz L M (Z) (Fr51)  $\cdot$ Gyrot] gẏrot G Fr51 Girot I M \textbf{12} sluoc] ersluͯg L (Z) \textbf{13} binz] bin I \textit{om.} Fr51  $\cdot$ Gramoflanz] grimoflanz G Gramorflanz M Gramoflantz Z gramoflans Fr51 \textbf{14} mîn hôhez herze] Mit hoen herzen Fr51 \textbf{15} ze deheinen] zeheinen G \textbf{16} nimmer wil] Nimber ne wille Fr51  $\cdot$ strîten] gestriten I (L) (M) Z \textbf{17} swaz] Waz L (M) (Fr51)  $\cdot$ tæte] tuͤt I getat L \textbf{18} wan] Svnder Fr51  $\cdot$ heizet] heizzen Z der hezet Fr51 \textbf{19} ich prîs hân] han ich pris Fr51 \textbf{20} ich] ich gerne I (L) (M) (Z)  $\cdot$ gein] enkeyn M  $\cdot$ wolde] welne G \textbf{21} ûf kampf] Vf champhe G Vf strite Z Zo kamf Fr51  $\cdot$ riuwe] truͯwe L \textbf{22} brach] duͯrch L der durch M der brach Z der droch Fr51  $\cdot$ sîn triuwe] sinen triͮwe G ruͯwe L getruwe M \textbf{23} in einem] ::e G Jn L (M) Jn dem Z (Fr51)  $\cdot$ mînen] miner M er minen Z  $\cdot$ er] \textit{om.} G L M Z \textbf{24} zesprechen han ich dar genuͤc I  $\cdot$ dâr] \textit{om.} M Fr51 \textbf{26} unde] nu I  $\cdot$ hât Gawan] Gawan hat L (Fr51) \textbf{27} vor] \textit{om.} L  $\cdot$ ûz] vns M \textbf{28} obe] obin M \textbf{29} im an prise nieman gelichen mac I  $\cdot$ Jme nymant prises geliche mac M  $\cdot$ Jm gelichen nieman prises mac Z \textbf{30} gelebe] lebe Fr51  $\cdot$ noch] \textit{om.} Fr51 \newline
\end{minipage}
\hspace{0.5cm}
\begin{minipage}[t]{0.5\linewidth}
\small
\begin{center}*T
\end{center}
\begin{tabular}{rl}
 & ich wil iuwer bote sîn.\\ 
 & gebet mir her \textbf{daz} vingerlîn\\ 
 & und lât mich iuwern dienst sagen\\ 
 & und iuwern kumber niht \textbf{vertragen}."\\ 
5 & der künec des dankete sêre.\\ 
 & Gawan \textbf{in vrâgete} mêre:\\ 
 & "sît iu versmâhet gein mir strît,\\ 
 & \textbf{sô} saget mir, hêrre, wer ir sît."\\ 
 & "ir solt ez niht vür laster doln",\\ 
10 & sprach der künec. "mîn name ist \textbf{unverstoln}.\\ 
 & mîn vater, \textbf{der} hiez Irot.\\ 
 & den \textbf{sluoc} der künec Lot.\\ 
 & ich bin ez, der künec Gramoflanz.\\ 
 & mîn hôhez herze \textbf{was ie} sô ganz,\\ 
15 & daz ich zuo dekeinen zîten\\ 
 & niemer wil \textbf{gestrîten},\\ 
 & waz mir tæte ein man,\\ 
 & wan einer, heizet Gawan,\\ 
 & von dem ich prîs hân vernomen,\\ 
20 & daz ich \textbf{gerne} gein im wolte komen\\ 
 & ûf \textbf{strît} durch mîne riuwe.\\ 
 & sîn vater, \textbf{der} brach triuwe:\\ 
 & \textbf{i\textit{me}} gruoze \textbf{er mînen vater} sluoc.\\ 
 & ich hân zuo \textbf{sprechene} dâr genuoc.\\ 
25 & nû ist Lot erstorben\\ 
 & und hât Gawan erworben\\ 
 & solichen prîs vor û\textit{z} besunder,\\ 
 & daz ob der tavelrunder\\ 
 & i\textit{m} prîses nieman glîchen mac.\\ 
30 & ich gelebe noch gein im strîtes tac."\\ 
\end{tabular}
\scriptsize
\line(1,0){75} \newline
U V W Q R \newline
\line(1,0){75} \newline
\textbf{1} \textit{Großinitiale} Q  \newline
\line(1,0){75} \newline
\textbf{3} iuwern] eúwen W \textbf{4} iuwern] [*]: v́wern V  $\cdot$ vertragen] verdagen V (W) (R) \textbf{6} Gawan] Gawin R \textbf{7} mir] mir úwer R \textbf{9} ir solt ez] Jr soͤllenz V Jren solt es Q \textbf{10} mîn name] daz R  $\cdot$ unverstoln] vnverholn V (Q) (R) \textbf{11} Irot] yrot V \textbf{12} sluoc] erschluͦg W (Q) (R) \textbf{13} ez] \textit{om.} W  $\cdot$ künec] \textit{om.} R  $\cdot$ Gramoflanz] gramaflanz V gramoflantz W Q gramoflancz R \textbf{17} waz] Swaz V  $\cdot$ ein] ein einzig V \textbf{18} einer] einer der V \textbf{19} ich] ichs Q  $\cdot$ hân] [*]: han V \textbf{20} gein im wolte] wolt gen im W ym wolt Q \textbf{21} riuwe] [*]: ruwe V \textbf{22} \textit{Vers 608.22 fehlt} R   $\cdot$ [Sim vatte*]: Sin vatter [*]: brach die trvwe V  $\cdot$ brach] bracht W \textbf{23} ime] in U \textbf{24} zuo sprechene] [*]: zerechenne V \textbf{26} Gawan] gawann Q Gawin R \textbf{27} solichen] [So*]: Solichen V Solch W  $\cdot$ ûz] vns U \textbf{29} im prîses] Jn prises U [J*]: Jm prises V Jm [pri*]: prise R  $\cdot$ mac] [kan]: mag R \textbf{30} noch gein im] gen Jm noch R \newline
\end{minipage}
\end{table}
\end{document}
