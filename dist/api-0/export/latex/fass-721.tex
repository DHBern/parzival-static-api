\documentclass[8pt,a4paper,notitlepage]{article}
\usepackage{fullpage}
\usepackage{ulem}
\usepackage{xltxtra}
\usepackage{datetime}
\renewcommand{\dateseparator}{.}
\dmyyyydate
\usepackage{fancyhdr}
\usepackage{ifthen}
\pagestyle{fancy}
\fancyhf{}
\renewcommand{\headrulewidth}{0pt}
\fancyfoot[L]{\ifthenelse{\value{page}=1}{\today, \currenttime{} Uhr}{}}
\begin{document}
\begin{table}[ht]
\begin{minipage}[t]{0.5\linewidth}
\small
\begin{center}*D
\end{center}
\begin{tabular}{rl}
\textbf{721} & \textbf{\begin{large}E\end{large}r} sprach, er wolte gerne komen.\\ 
 & dâ wart geselleschaft genomen:\\ 
 & sînes landes vürsten drî\\ 
 & riten dem künege \textbf{dannen} bî.\\ 
5 & \textbf{alsus} \textbf{vuor} ouch der œheim sîn,\\ 
 & der künec Brandelidelin,\\ 
 & \textbf{Bernout} \textbf{de} Riviers\\ 
 & und \textbf{Affinamus} \textbf{von} Clitiers,\\ 
 & ieweder \textbf{einen} gesellen nam,\\ 
10 & der ûf die reise wol gezam.\\ 
 & zwelfe wâren ir überal.\\ 
 & \textbf{junchêrrelîn} vil âne zal\\ 
 & und manec starker sarjant\\ 
 & ûf die reise \textbf{wart} \textbf{benant}.\\ 
15 & Welch der rîter kleider \textbf{mohten} sîn?\\ 
 & pfellel, \textbf{der} vil liehten schîn\\ 
 & \textbf{gap} von \textbf{des goldes} swære.\\ 
 & des küneges valkenære\\ 
 & mit im dan durch beizen riten.\\ 
20 & Nû het \textbf{ouch} Artus niht vermiten,\\ 
 & Beacursen, den lieht gevar,\\ 
 & sander ze halbem wege \textbf{al} dar\\ 
 & dem künege zeime geleite.\\ 
 & über des gevildes breite,\\ 
25 & ez wære tîch oder bach,\\ 
 & swâ \textbf{er die passâschen} \textbf{sach},\\ 
 & dâ reit der künec beizen her\\ 
 & und mêr durch der minne ger.\\ 
 & Beacors in dâ enpfienc,\\ 
30 & \textbf{sô} daz ez mit vreude \textbf{ergienc}.\\ 
\end{tabular}
\scriptsize
\line(1,0){75} \newline
D \newline
\line(1,0){75} \newline
\textbf{1} \textit{Initiale} D  \textbf{15} \textit{Majuskel} D  \textbf{20} \textit{Majuskel} D  \newline
\line(1,0){75} \newline
\textbf{7} Bernout] Bernoͮt D  $\cdot$ Riviers] Rivîers D \textbf{8} Affinamus] Affinamv̂s D  $\cdot$ Clitiers] Clitîers D \textbf{21} Beacursen] Beahcvrsen D \textbf{29} Beacors] Beahcors D \newline
\end{minipage}
\hspace{0.5cm}
\begin{minipage}[t]{0.5\linewidth}
\small
\begin{center}*m
\end{center}
\begin{tabular}{rl}
 & \textbf{er} sprach, er wolte gerne komen.\\ 
 & dô wart geselleschaft genomen:\\ 
 & sîn\textit{es} landes vürsten drî\\ 
 & riten dem künic \textbf{nâhe} bî.\\ 
5 & \dag ouch\dag  \textbf{tet} ouch der œheim sîn,\\ 
 & der künic Brandelidel\textit{i}n.\\ 
 & \textbf{Bernout} \textbf{de} Riviers\\ 
 & und \textbf{Offinamus} \textbf{von} Clitiers,\\ 
 & ietweder \textbf{eine\textit{n}} gesellen nam,\\ 
10 & der ûf die reise wol gezam.\\ 
 & zwelf wâren ir überal.\\ 
 & \textbf{junchêrren} vil âne zal\\ 
 & und manic starker sarjant\\ 
 & ûf die reise \textbf{wart} \textbf{benant}.\\ 
15 & welich der ritter kleider \textbf{mohte} sîn?\\ 
 & pfelle, \textbf{der} vil liehten schîn\\ 
 & \textbf{gap} von \textbf{golde} swære.\\ 
 & des küniges valkenære\\ 
 & mit im dan durch beizen riten.\\ 
20 & nû het \textbf{ouch} Artus niht vermiten,\\ 
 & Beachurs, den lieht gevar,\\ 
 & sant er zuo halbe\textit{m} wege dar\\ 
 & dem künige zuo einem geleite.\\ 
 & über d\textit{es} gevilde\textit{s} breite,\\ 
25 & ez wær tîch oder bach,\\ 
 & wâ \textbf{er die passâschen} \textbf{sach},\\ 
 & d\textit{â} reit der künic beizen her\\ 
 & und mê durch der minne ger.\\ 
 & Be\textit{a}kurs in d\textit{â} enpfienc,\\ 
30 & \textbf{sô} daz ez mit vröuden \textbf{gienc}.\\ 
\end{tabular}
\scriptsize
\line(1,0){75} \newline
m n o Fr69 \newline
\line(1,0){75} \newline
\newline
\line(1,0){75} \newline
\textbf{3} sînes] Sin m \textbf{4} riten] Rittent n \textbf{6} Brandelidelin] brandelidelidin m brandeledelin o \textbf{7} Bernout] Bernont n  $\cdot$ Riviers] rivirs o \textbf{8} Offinamus] affamanius n affinamo o  $\cdot$ Clitiers] cliciers o \textbf{9} einen] einem m  $\cdot$ nam] [man]: nam o \textbf{11} ir] ir [b]: do n \textbf{14} reise] [welt]: reise m \textbf{20} Artus] artuͯs o \textbf{21} Beachurs] Beacurs n Beaturs o \textbf{22} zuo halbem] zuͯ halben m zehalbe Fr69 \textbf{23} dem künige] Den konig o  $\cdot$ einem] einen o \textbf{24} des gevildes] das gefilde m (n) o  $\cdot$ breite] breit o \textbf{25} wær tîch] were diche n weret uch Fr69 \textbf{26} wâ] Swa Fr69  $\cdot$ passâschen] paschahen Fr69 \textbf{27} dâ] Do m n o  $\cdot$ beizen] beiser o \textbf{28} minne] mẏns o \textbf{29} Beakurs] Bekurs m Beaturs o  $\cdot$ dâ] do m n o \textbf{30} gienc] erging n o \newline
\end{minipage}
\end{table}
\newpage
\begin{table}[ht]
\begin{minipage}[t]{0.5\linewidth}
\small
\begin{center}*G
\end{center}
\begin{tabular}{rl}
 & \textbf{\begin{large}D\end{large}er} sprach, er wolde gerne komen.\\ 
 & dô wart geselleschaft genomen:\\ 
 & sînes landes vürsten drî,\\ 
 & \textbf{die} riten dem künige bî.\\ 
5 & \textbf{alsô} \textbf{tet} ouch der œheim sîn,\\ 
 & der künec Brandelidelin.\\ 
 & \textbf{Gernout} \textbf{de} Rivirs\\ 
 & unde \textbf{Affinamus} \textbf{de} Cletirs,\\ 
 & ietweder \textbf{einen} gesellen nam,\\ 
10 & der ûf die reise wol gezam.\\ 
 & zwelfe wâren ir überal.\\ 
 & \textbf{junchêrren} vil \textbf{gar} âne zal\\ 
 & unde manec starc sarjant\\ 
 & ûf die reise \textbf{wart} \textbf{gesant}.\\ 
15 & welch der rîter kleider \textbf{möhte} sîn?\\ 
 & pfelle, \textbf{der} vil liehten schîn\\ 
 & \textbf{gab} von \textbf{des goldes} swære.\\ 
 & des küniges valkenære\\ 
 & mit im dan durch beizen riten.\\ 
20 & nû \textbf{en}het Artus niht vermiten,\\ 
 & Beakurs, den lieht gevar,\\ 
 & sander ze halbem wege \textbf{al} dar\\ 
 & dem künige zeinem geleite.\\ 
 & über des gevildes breite,\\ 
25 & ez wære tîch ode bach,\\ 
 & swâ \textbf{man den fasân} \textbf{gesach},\\ 
 & dar reit der künic \textbf{durch} beizen her\\ 
 & unde mêre durch der minne ger.\\ 
 & Beakurs in dâ enpfienc,\\ 
30 & \textbf{alsô} daz ez mit vröuden \textbf{ergienc}.\\ 
\end{tabular}
\scriptsize
\line(1,0){75} \newline
G I L M Z Fr20 Fr45 \newline
\line(1,0){75} \newline
\textbf{1} \textit{Initiale} G I L Z Fr20  \textbf{21} \textit{Initiale} I  \newline
\line(1,0){75} \newline
\textbf{1} Der] Er I L M Z ÷r Fr20 \textbf{2} dô] Da L M Z (Fr45)  $\cdot$ geselleschaft] selleschaft Fr45 \textbf{4} die] Si L  $\cdot$ bî] da bi Z \textbf{6} Brandelidelin] brandalidelin I Branlidelin L brændelidelin Fr20 Brandelidel::: Fr45 \textbf{7} Gernout] Gernovt G L Fr20 Gernut I Gernot M Bernovt Z :ernovt Fr45  $\cdot$ de] von I Fr45  $\cdot$ Rivirs] riviͤrs G riuers I Riviers L (M) Z (Fr20) Riuiers Fr45 \textbf{8} Affinamus] Affinamvs G (L) Afinamus I affmamvs Z :::finamus Fr45  $\cdot$ de] von Z Fr45  $\cdot$ Cletirs] cletiͤrs G chliters I Cletiers L Z ditiers M cleitiers Fr20 klitiers Fr45 \textbf{9} ietweder] Jclichir M \textbf{11} wâren ir] ir warn I  $\cdot$ überal] beider al M \textbf{12} vil gar] \textit{om.} L vil M Fr45 gar vil Z \textbf{14} gesant] genant M (Fr45) benant Z \textbf{15} der rîter] div Fr20 ir ::: Fr45  $\cdot$ möhte] mohte I mochten L (M) (Z) (Fr20) (Fr45) \textbf{16} pfelle] Pflelle L  $\cdot$ der] dy M (Fr45)  $\cdot$ liehten] lichten L (M) \textbf{17} gab] Gaben M (Fr45)  $\cdot$ des] \textit{om.} L \textbf{19} dan] da L  $\cdot$ riten] retten M \textbf{20} nû enhet] min het I Nv het L (Fr45)  $\cdot$ Artus] ouch Artus L (M) (Fr45) artus ouch Z Fr20 \textbf{21} Beakurs] Beakv̂rs G Beacurs I Beakuͯrs L beachvrs Fr20 Beakuͦrs Fr45  $\cdot$ lieht] lieh I lichten M  $\cdot$ gevar] var M \textbf{22} halbem] halben Z  $\cdot$ al] \textit{om.} I L M Z Fr45 \textbf{23} zeinem] zuͯ L \textbf{25} ode] alder Fr20 \textbf{26} swâ] Wa L M (Fr45)  $\cdot$ man] her M (Fr45)  $\cdot$ den fasân] die passassen L (M) (Z) (Fr45)  $\cdot$ gesach] sach L M Z \textbf{27} dar] do I \textbf{28} minne] minnen Fr45 \textbf{29} Beakurs] beacurs I (M) Beakuͯrs L beachurs Fr20 Beacuͦrs Fr45 \textbf{30} alsô] So L  $\cdot$ daz ez] des M daz Fr20  $\cdot$ vröuden] frovde L Fr20 \newline
\end{minipage}
\hspace{0.5cm}
\begin{minipage}[t]{0.5\linewidth}
\small
\begin{center}*T
\end{center}
\begin{tabular}{rl}
 & \textbf{\begin{large}E\end{large}r} sprach, er wolte gerne komen.\\ 
 & dô wart geselleschaft genomen:\\ 
 & sînes landes vürsten drî,\\ 
 & \textbf{die} riten dem künege bî.\\ 
5 & \textbf{alsô} \textbf{tet} ouch der œheim sîn,\\ 
 & der künec Brandelidelin,\\ 
 & \hspace*{-.7em}\big| und \textbf{Affinamour} \textbf{de} Cletiers\\ 
 & \hspace*{-.7em}\big| \textbf{und} \textbf{Bernout} \textbf{von} Riviers,\\ 
 & ietweder \textbf{sînen} gesellen nam,\\ 
10 & de\textit{r} ûf die reise wol gezam.\\ 
 & zwelfe wâren ir überal.\\ 
 & \textbf{junchêrren} vil âne zal\\ 
 & und manec starc sarjant\\ 
 & ûf die reise \textbf{wâren} \textbf{benant}.\\ 
15 & welich der rîter kleider \textbf{mohten} sîn?\\ 
 & pfelle, \textbf{die} vil liehten schîn\\ 
 & \textbf{gâben} von \textbf{golde} \textbf{grôze} swære.\\ 
 & des küneges valkenære\\ 
 & mit im dan durch beizen riten.\\ 
20 & nû hete Artus niht vermiten,\\ 
 & Beakurs, den lieht gevar,\\ 
 & santer zuo halbe\textit{m} wege dar\\ 
 & dem künege zuo eime geleite.\\ 
 & über des gevildes breite,\\ 
25 & ez wære tîch oder bach,\\ 
 & wâ \textbf{er die passâschen} \textbf{sach},\\ 
 & d\textit{â} reit der künec \textbf{durch} beizen her\\ 
 & und mê durch der minnen ger.\\ 
 & Beakurs in dâ entvienc,\\ 
30 & \textbf{alsô} daz ez mit vreuden \textbf{ergienc}.\\ 
\end{tabular}
\scriptsize
\line(1,0){75} \newline
U V W Q R \newline
\line(1,0){75} \newline
\textbf{1} \textit{Initiale} U V Q R  \newline
\line(1,0){75} \newline
\textbf{2} genomen] an sich genomen R \textbf{4} riten] ritter Q \textbf{6} Brandelidelin] brandelidelein W brandlidelin Q \textbf{8} \textit{Versfolge 721.7-8} V W Q R   $\cdot$ Affinamour] Affinamuͦrg U affinamus V W Q R  $\cdot$ Cletiers] Clitiers V kleitiers W Cletirs Q \textbf{7} und] \textit{om.} V W Q R  $\cdot$ Bernout] Berciot U Bernvt V Bernuͦt W Genaut Q Bernowt R  $\cdot$ von] de V W Q R  $\cdot$ Riviers] Riuiers U (V) riwers W Cliuirs Q Rivers R \textbf{9} sînen] einen V Q (R) \textbf{10} der] Den U \textbf{11} wâren ir] [*]: ir waren V \textbf{15} welich] [*h]: welich V Welhe R  $\cdot$ mohten] moͤhten V mochte Q \textbf{16} die] [d*]: der V der R  $\cdot$ liehten] lichten Q \textbf{17} gâben] [Gab*]: Gab V Gab R  $\cdot$ golde grôze] des goldes V Q dem goldes R \textbf{19} riten] reiten Q \textbf{20} hete] enhette oͮch V en het Q \textbf{21} Beakurs] Beakuͦrs U Beakurß W beakursz Q  $\cdot$ lieht] licht Q liechtten R \textbf{22} santer] Santte der R  $\cdot$ zuo halbem] zuͦ halben U (V) zúm halbe Q zum halben R \textbf{24} des] das R  $\cdot$ breite] gebreite V brete breite R \textbf{26} wâ] Swa V  $\cdot$ passâschen] bassanien Q \textbf{27} dâ] Do U V W Q R  $\cdot$ durch] \textit{om.} Q  $\cdot$ beizen] beiczencz willen R \textbf{28} mê] wer Q \textbf{29} Beakurs] Beakuͦrs U Beakurß W  $\cdot$ dâ] do V W Q R \textbf{30} vreuden] froͮede V (W)  $\cdot$ ergienc] ging Q \newline
\end{minipage}
\end{table}
\end{document}
