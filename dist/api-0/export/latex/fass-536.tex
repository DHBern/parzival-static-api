\documentclass[8pt,a4paper,notitlepage]{article}
\usepackage{fullpage}
\usepackage{ulem}
\usepackage{xltxtra}
\usepackage{datetime}
\renewcommand{\dateseparator}{.}
\dmyyyydate
\usepackage{fancyhdr}
\usepackage{ifthen}
\pagestyle{fancy}
\fancyhf{}
\renewcommand{\headrulewidth}{0pt}
\fancyfoot[L]{\ifthenelse{\value{page}=1}{\today, \currenttime{} Uhr}{}}
\begin{document}
\begin{table}[ht]
\begin{minipage}[t]{0.5\linewidth}
\small
\begin{center}*D
\end{center}
\begin{tabular}{rl}
\textbf{536} & "ir \textbf{en}komt niht zuo mir \textbf{dâ} her în.\\ 
 & ir müezet pfant \textbf{dort} ûze sîn."\\ 
 & Er \textbf{sprach} ir trûreclîchen nâch:\\ 
 & "vrouwe, wie ist iu von mir sô gâch?\\ 
5 & sol ich iuch immer mêr gesehen?"\\ 
 & Si sprach: "iu mac der prîs geschehen,\\ 
 & ich \textbf{state} iu sehens noch an mich.\\ 
 & ich wæne, daz sêre lenget sich."\\ 
 & diu vrouwe \textbf{schiet von im} alsus.\\ 
10 & hie kom Lischoys Gwelljus.\\ 
 & sagte ich \textbf{iu} nû, daz der vlüge,\\ 
 & mit der rede ich iuch betrüge.\\ 
 & er gâhete aber anders sêre,\\ 
 & daz es daz ors het êre\\ 
15 & - wan \textbf{daz} erzeigete snelheit -,\\ 
 & über den grüenen anger breit.\\ 
 & Dô dâhte mîn hêr Gawan:\\ 
 & "wie sol ich \textbf{beiten} dises man?\\ 
 & wederz mac daz wæger sîn:\\ 
20 & ze vuoz oder ûf dem \textbf{pferdelîn}?\\ 
 & wil er \textbf{vollic} \textbf{an} mich varn,\\ 
 & daz er den poinder \textbf{niht kan} sparn,\\ 
 & er sol mich nider rîten.\\ 
 & wes \textbf{mac} sîn ors \textbf{dâ} bîten,\\ 
25 & ez enstrûche \textbf{ouch} über daz runzît?\\ 
 & wil er \textbf{mir} \textbf{denne bieten} strît,\\ 
 & \textbf{al dâ} wir bêde sîn ze vuoz,\\ 
 & ob mir halt nimmer würde ir gruoz,\\ 
 & diu mich disse strîtes hât gewert,\\ 
30 & ich gib im strît, ob er des gert."\\ 
\end{tabular}
\scriptsize
\line(1,0){75} \newline
D Fr31 \newline
\line(1,0){75} \newline
\textbf{3} \textit{Majuskel} D  \textbf{6} \textit{Majuskel} D  \textbf{17} \textit{Majuskel} D  \newline
\line(1,0){75} \newline
\textbf{1} zuo mir dâ] \textit{om.} Fr31 \textbf{3} trûreclîchen] getriͮweliche Fr31 \textbf{4} vrouwe] Oͮwe Fr31 \textbf{7} state] gestat Fr31  $\cdot$ noch] \textit{om.} Fr31 \textbf{9} alsus] svs Fr31 \textbf{10} hie kom] \textit{om.} Fr31  $\cdot$ Lischoys] Liscoys D Lyschoys Fr31  $\cdot$ Gwelljus] gwellivs D (Fr31) \textbf{11} der] er Fr31 \textbf{14} es] ez Fr31 \newline
\end{minipage}
\hspace{0.5cm}
\begin{minipage}[t]{0.5\linewidth}
\small
\begin{center}*m
\end{center}
\begin{tabular}{rl}
 & "\dag er\dag  \textbf{en}komt niht zuo mir her în.\\ 
 & ir müezet pfant \textbf{hie} ûz sîn."\\ 
 & er \textbf{sprach} ir trûreclîchen nâch:\\ 
 & "vrouwe, wie ist iu von mir sô gâch?\\ 
5 & sol ich iuch iemer mê gesehen?"\\ 
 & si sprach: "iu mac der prîs geschehen,\\ 
 & ich \textbf{state} iu sehens noch an mich.\\ 
 & ich wæne, da\textit{z} \textit{s}êre l\textit{e}nget sich."\\ 
 & diu vrouwe \textbf{schiet von im} alsus.\\ 
10 & hie kam Lischois \textit{G}wellius.\\ 
 & saget ich nû, daz der \textit{v}lüge,\\ 
 & mit der rede ich iuch be\textit{t}rüge.\\ 
 & er gâhete aber ander\textit{s} \textbf{sô} sêre,\\ 
 & daz es daz ros het êre\\ 
15 & - wan \textbf{daz} erzöugte snelheit -,\\ 
 & über den grüenen anger breit.\\ 
 & dô dâht mîn hêr Gawan:\\ 
 & "wie sol ich \textbf{beiten} dises man?\\ 
 & \dag wider daz\dag  mac daz wæ\textit{g}er sîn:\\ 
20 & zuo vuoz oder ûf dem \textbf{\textit{p}fe\textit{r}d\textit{e}lîn}?\\ 
 & wil er \textbf{volleclîchen} \textbf{an} mich varn,\\ 
 & daz er den ponder \textbf{kan niht} sp\textit{ar}n,\\ 
 & er sol mich nider rîten.\\ 
 & wes \textbf{mac} sîn ros \textbf{d\textit{â}} bîten,\\ 
25 & ez e\textit{n}s\textit{t}rûch \textbf{ouch} über daz runzît?\\ 
 & wil er \textbf{mir} \textbf{dan bieten} strît,\\ 
 & \textbf{aldâ} wir beide sîn zuo vuoz,\\ 
 & ob mir halt nimmer würde ir gruoz,\\ 
 & diu mich dises strîtes het gewert,\\ 
30 & ich gib \textit{ime} strît, ob er des gert."\\ 
\end{tabular}
\scriptsize
\line(1,0){75} \newline
m n o \newline
\line(1,0){75} \newline
\newline
\line(1,0){75} \newline
\textbf{1} er] Es o  $\cdot$ zuo] her zuͯ n \textbf{2} müezet] muͯst m o  $\cdot$ hie ûz] dort vssen n (o) \textbf{3} trûreclîchen nâch] trurichen not o \textbf{4} iu von mir] von >mir< o \textbf{5} sol] Solt o \textbf{6} geschehen] beschehen n (o) \textbf{7} state] gestatte n \textbf{8} daz sêre] das ich sere m das das sere o  $\cdot$ lenget] langet m \textbf{10} Lischois Gwellius] liscois giwellius m n liscois gewellens o \textbf{11} vlüge] sluͯge m \textbf{12} betrüge] beruͯge m \textbf{13} anders] an der m \textbf{15} erzöugte] er zeigete n er zcigete o \textbf{17} hêr Gawan] hergawan o \textbf{18} beiten] beiden o  $\cdot$ dises] dissen m (o) \textbf{19} wæger] weher m \textbf{20} pferdelîn] federlin m \textbf{21} wil er] Wider o \textbf{22} sparn] spran m \textbf{24} dâ] do m denne n (o) \textbf{25} enstrûch] entschruch m entstruch n \textbf{26} bieten] [stritten]: bietten strit m \textbf{29} diu] Sie o \textbf{30} ime] mich m  $\cdot$ er] ers o \newline
\end{minipage}
\end{table}
\newpage
\begin{table}[ht]
\begin{minipage}[t]{0.5\linewidth}
\small
\begin{center}*G
\end{center}
\begin{tabular}{rl}
 & "\textit{\begin{large}I\end{large}}r\textbf{ne} komet niht zuo mir her în.\\ 
 & ir müezet pfant \textbf{dort} ûze sîn."\\ 
 & er \textbf{sprach} ir trûreclîchen nâch:\\ 
 & "vrouwe, wie ist iu von mir sô gâch?\\ 
5 & sol ich iuch immer mê gesehen?"\\ 
 & si sprach: "iu mac der brîs geschehen,\\ 
 & ich \textbf{state} iu sehens noch an mich.\\ 
 & ich wæne, daz sêre lenget sich."\\ 
 & diu vrouwe \textbf{von im schiet} alsus.\\ 
10 & hie ko\textit{m} Lishois Gewellius.\\ 
 & sagete ich \textbf{iu} nû, daz der vlüge,\\ 
 & mit der rede ich iuch betrüge.\\ 
 & er gâhte aber anders sêre,\\ 
 & daz es daz ors het êre\\ 
15 & - wan \textbf{daz} erzeigte snelheit -,\\ 
 & über den grüenen anger breit.\\ 
 & dô dâhte mîn hêrre Gawan:\\ 
 & "wie sol ich \textbf{beiten} dises man?\\ 
 & wederz mac d\textit{az} wæger sîn:\\ 
20 & ze vuoze oder \textit{ûf} dem \textbf{pf\textit{er}d\textit{e m}în}?\\ 
 & wil er \textbf{volleclîchen} \textbf{an} mich varn,\\ 
 & daz er den poynder \textbf{niht kan} sparn,\\ 
 & er sol mich nider rîten.\\ 
 & wes \textbf{mac} sîn ors \textbf{dâ} bîten,\\ 
25 & ez enstrûche \textbf{ouch} über daz runzît?\\ 
 & wil er \textbf{mir} \textbf{danne bieten} strît,\\ 
 & \textbf{al dâ} wir bêde sîn ze vuoz,\\ 
 & ob mir halt nimer würde ir gruoz,\\ 
 & diu mich dises strîtes hât gewert,\\ 
30 & ich gibe im strît, ob er des gert."\\ 
\end{tabular}
\scriptsize
\line(1,0){75} \newline
G I L M Z Fr19 \newline
\line(1,0){75} \newline
\textbf{1} \textit{Initiale} G L Z Fr19  \textbf{9} \textit{Initiale} I  \newline
\line(1,0){75} \newline
\textbf{1} Irne] ER ne G Ir L  $\cdot$ zuo mir] mit mir I da L M Fr19 \textbf{2} dort ûze sîn] lan dort vszin L \textbf{3} sprach] Spranc M \textbf{5} immer] niemmer L \textbf{7} state] gestat I stat L tet M  $\cdot$ sehens] selhes Fr19 \textbf{8} daz] ez Z  $\cdot$ lenget] lenget lenget I lenge Z \textbf{9} von im schiet] schit von yme M (Fr19) \textbf{10} hie] do I  $\cdot$ kom] chome G  $\cdot$ Lishois] liscois I Lýtschoýs L lisheis M Lishoys Fr19  $\cdot$ Gewellius] gwelluͯs L girullius M gwellius Z (Fr19) \textbf{11} der] her M  $\cdot$ vlüge] flug G slvge Fr19 \textbf{12} mit] Mt Fr19 \textbf{13} aber] \textit{om.} M \textbf{14} es daz ors] sin orse des I sin daz orss Z \textbf{15} daz] das is M \textbf{16} den] der L  $\cdot$ grüenen] grozen I \textbf{17} dô] Da L M \textbf{18} beiten dises man] biten disz man L \textbf{19} wederz] Wedir M  $\cdot$ daz] der G  $\cdot$ wæger] wegiste I (Z) \textbf{20} vuoze] vuͤzen I  $\cdot$ ûf] ze G  $\cdot$ pferde mîn] pharidin G pfardelin L (M) (Z) (Fr19) \textbf{21} volleclîchen] williklichen M  $\cdot$ an] vf I \textbf{22} poynder] poýn L  $\cdot$ kan] wil L \textbf{24} dâ] \textit{om.} L M \textbf{25} ez enstrûche] Ern struͯche L  $\cdot$ ouch] \textit{om.} I \textbf{26} bieten] vuͦgen I biten M Z \textbf{27} al dâ] so I  $\cdot$ ze vuoz] zefueze G \textbf{28} Ob ::: nie mer wͦrde ir grvͦz Fr19  $\cdot$ halt] danne I  $\cdot$ würde] [wurt]: wurde G werde M  $\cdot$ ir gruoz] buͤz I ir buͯsz L \textbf{29} dises] \textit{om.} M \textbf{30} gibe] bin L  $\cdot$ des] sin I \newline
\end{minipage}
\hspace{0.5cm}
\begin{minipage}[t]{0.5\linewidth}
\small
\begin{center}*T
\end{center}
\begin{tabular}{rl}
 & "ir komet niht zuo mir \textbf{dâ} her în.\\ 
 & ir müezet pfant \textbf{dort} ûze sîn."\\ 
 & Er \textbf{sach} ir trûreclîche nâch:\\ 
 & "vrouwe, wie ist iu von mir sô gâch?\\ 
5 & sol ich iuch iemer mêr gesehen?"\\ 
 & Si sprach: "iu mac der prîs geschehen,\\ 
 & ich \textbf{gestate} iu sehens noch an mich.\\ 
 & ich wæne, daz sêre lenget sich."\\ 
 & \textit{\begin{large}D\end{large}}iu vrouwe \textbf{schiet von im} alsus.\\ 
10 & Hie kom Lyschoys Gewellius.\\ 
 & saget ich \textbf{iu} nû, daz der vlüge,\\ 
 & mit der rede ich iuch betrüge.\\ 
 & er gâhete aber anders sêre,\\ 
 & daz e\textit{s} daz ors hete êre\\ 
15 & - wand\textbf{e\textit{z}} \textit{e}rzeigete snelheit -,\\ 
 & über den grüenen anger breit.\\ 
 & Dô dâhte mîn hêr Gawan:\\ 
 & "wie sol ich \textbf{erbeiten} disses man?\\ 
 & wederz mac daz wæger sîn:\\ 
20 & ze vuoz oder ûf dem \textbf{pferdelîn}?\\ 
 & wil er \textbf{volleclîch} \textbf{ûf} mich varn,\\ 
 & daz er den poynder \textbf{niht kan} sparn,\\ 
 & er sol mich nider rîten.\\ 
 & wes \textbf{sol} \textit{s}în ors \textbf{danne} bîten,\\ 
25 & ez enstrûche über daz runzît?\\ 
 & wil er \textbf{bieten denne} strît,\\ 
 & \textbf{sô} wir beide sîn ze vuoz,\\ 
 & ob mir halt niemer würde ir gruoz,\\ 
 & diu mich disses strîtes hât gewert,\\ 
30 & ich gibim strît, ob er des gert."\\ 
\end{tabular}
\scriptsize
\line(1,0){75} \newline
T U V W O Q R \newline
\line(1,0){75} \newline
\textbf{1} \textit{Initiale} O Q  \textbf{3} \textit{Majuskel} T  \textbf{6} \textit{Majuskel} T  \textbf{8} \textit{Überschrift:} Hie streit her gawan mit lyshoys gewellius vnd ving in vnd gewan sein ros wider W  \textbf{9} \textit{Initiale} T U V W  \textbf{10} \textit{Majuskel} T  \textbf{17} \textit{Majuskel} T  \newline
\line(1,0){75} \newline
\textbf{1} ir] Jr in U (V) (Q) ÷r O  $\cdot$ zuo mir] \textit{om.} O  $\cdot$ dâ] do U W \textit{om.} V R \textbf{2} dort] da R \textbf{3} sach] sprach U V W O Q R \textbf{4} vrouwe] Owe O  $\cdot$ von] vom O  $\cdot$ gâch] iach Q \textbf{5} iuch] iv T  $\cdot$ iemer] nuͦmer U (W) (R)  $\cdot$ mêr] \textit{om.} Q \textbf{6} geschehen] [*]: geschehen V beschehe W \textbf{7} gestate] state U (V) (W) (O) Q staten R  $\cdot$ sehens] sehen Q \textbf{9} Diu] ÷iv T \textbf{10} Lyschoys] Lyscois T lichois U lychois V lyshois W Lishoẏs O lischois Q Lyschois R  $\cdot$ Gewellius] gewelleius T gewellus U Gwellyvs O \textbf{11} iu nû] eúch W nun úch R  $\cdot$ der] er W O R \textbf{12} iuch] iv T \textbf{13} gâhete] gedachte Q  $\cdot$ anders] ander W \textbf{14} es] ez T \textbf{15} wandez] wandez es T Wan daz U  $\cdot$ erzeigete] [erz*]: erzeigete T \textbf{17} dâhte] gedachte W (R) \textbf{18} wie] So R  $\cdot$ erbeiten] beitten R \textbf{19} wederz] Weder U V W Q R  $\cdot$ daz] des Q \textbf{20} vuoz] fvͦzen O  $\cdot$ pferdelîn] roße mein W \textbf{21} wil] Wie R  $\cdot$ ûf] an U V W O Q R \textbf{22} poynder] Poyndier T stich R  $\cdot$ kan] wil V R \textbf{23} er] Es R \textbf{24} wes] Waz V (W)  $\cdot$ sol] mag V W (O) Q R  $\cdot$ sîn] min T \textit{om.} W \textbf{25} ez] Er U R Zu Q  $\cdot$ enstrûche] strauche Q  $\cdot$ über] [*]: uͮch v́ber V \textbf{26} bieten denne] bieten dan bieten U aber mir denne bieten V mir danne bieten W O Q (R) \textbf{27} ze vuoz] zefvͦzen O \textbf{28} halt] halp V  $\cdot$ ir gruoz] buͦsse W ir grvͦzen O \textbf{30} gibim] gen im U  $\cdot$ ob] wenn W  $\cdot$ er des] er sein W (O) ers R \newline
\end{minipage}
\end{table}
\end{document}
