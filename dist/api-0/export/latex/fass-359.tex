\documentclass[8pt,a4paper,notitlepage]{article}
\usepackage{fullpage}
\usepackage{ulem}
\usepackage{xltxtra}
\usepackage{datetime}
\renewcommand{\dateseparator}{.}
\dmyyyydate
\usepackage{fancyhdr}
\usepackage{ifthen}
\pagestyle{fancy}
\fancyhf{}
\renewcommand{\headrulewidth}{0pt}
\fancyfoot[L]{\ifthenelse{\value{page}=1}{\today, \currenttime{} Uhr}{}}
\begin{document}
\begin{table}[ht]
\begin{minipage}[t]{0.5\linewidth}
\small
\begin{center}*D
\end{center}
\begin{tabular}{rl}
\textbf{359} & \begin{large}D\end{large}ô sprach Poydiconjunz\\ 
 & zem herzogen von Lanverunz:\\ 
 & "geruochet ir mîn niht bîten,\\ 
 & \textbf{sô} ir vart \textbf{durch rüemen} strîten,\\ 
5 & \textbf{sô wænet ir}, daz sî guot getân?\\ 
 & hie ist der \textbf{werde} Laheduman\\ 
 & \textbf{unt} \textbf{ouch} Meljacanz, mîn sun.\\ 
 & swaz die bêde solden tuon\\ 
 & unt ich selbe, ir m\textit{ö}htet \textbf{dâ} strîten sehen,\\ 
10 & ob ir strîten \textbf{kündet} spehen.\\ 
 & i\textbf{ne} kum niemer von dirre stat,\\ 
 & i\textbf{ne} \textbf{mache} \textbf{uns} alle \textbf{strîtes} sat,\\ 
 & oder mir \textbf{gebent} man unt wîp\\ 
 & her ûz gevangen \textbf{ir bêder} lîp."\\ 
15 & Dô sprach der herzoge Astor:\\ 
 & "hêrre, iwer neve was \textbf{dâ} vor,\\ 
 & der künec, unt al sîn her von Liz.\\ 
 & solt iwer her an \textbf{slâfes} vlîz\\ 
 & die wîle \textbf{sich} hân gekêret?\\ 
20 & habt ir uns daz gelêret,\\ 
 & sô slâf ich, \textbf{dâ} man strîten sol.\\ 
 & ich kan bî strîte slâfen wol.\\ 
 & \textbf{doch gloubet mir} \textbf{daz}, wære ich niht komen,\\ 
 & Die burgære heten \textbf{dâ} genomen\\ 
25 & vrumen unt prîs zir handen.\\ 
 & ich \textbf{bewar} iuch dâ vor schanden.\\ 
 & durch got, \textbf{nû} senftet iwern zorn!\\ 
 & \textbf{dâ} ist mêr gewunnen dan verlorn\\ 
 & von iwerre messenîe,\\ 
30 & wils jehen \textbf{vrou} Obie."\\ 
\end{tabular}
\scriptsize
\line(1,0){75} \newline
D \newline
\line(1,0){75} \newline
\textbf{1} \textit{Initiale} D  \textbf{15} \textit{Majuskel} D  \textbf{24} \textit{Majuskel} D  \newline
\line(1,0){75} \newline
\textbf{1} Poydiconjunz] Poydiconivnz D \textbf{7} Meljacanz] Meliacanz D \textbf{9} möhtet] mohtet D \textbf{17} Liz] Lyz D \textbf{30} Obie] Obîe D \newline
\end{minipage}
\hspace{0.5cm}
\begin{minipage}[t]{0.5\linewidth}
\small
\begin{center}*m
\end{center}
\begin{tabular}{rl}
 & dô sprach Poid\textit{i}c\textit{o}niunz\\ 
 & zem herzogen von Laverunz:\\ 
 & "geruochet ir mîn niht bîten,\\ 
 & \textbf{sô} ir vart \textbf{durch rüemen} strîten,\\ 
5 & \textbf{ir wænet}, daz \textbf{ez} sî guot getân?\\ 
 & hie ist der \textbf{werde} Laheduman\\ 
 & \textbf{und} \textbf{ouch} Melia\textit{g}anz, mîn sun.\\ 
 & waz die beide solten tuon\\ 
 & und ich selbe, ir m\textit{ö}ht \textbf{d\textit{â}} strîten sehen,\\ 
10 & ob ir strîten \textbf{kündet} spehen.\\ 
 & ich kume niemer von dirre stat,\\ 
 & ich \textbf{mach} \textbf{uns} alle \textbf{strîtes} sat,\\ 
 & oder mir \textbf{gebent} man und wîp\\ 
 & her ûz gevangen \textbf{beider} lîp."\\ 
15 & dô sprach der herzoge Astor:\\ 
 & "hêrre, iuwer neve was \textbf{dâ} vor,\\ 
 & der künic, und allez sîn her von Liz.\\ 
 & solt iuwer her an \textbf{slâfens} vlîz\\ 
 & die wîle \textbf{sich} hân gekêret?\\ 
20 & habet ir uns da\textit{z} gelêret,\\ 
 & sô slâf ich, \textbf{d\textit{â}} man strîten sol.\\ 
 & ich kan bî strîte slâfen wol.\\ 
 & \textbf{doch gloubet mir}, wær ich \textit{niht} komen,\\ 
 & die burgære heten \textbf{d\textit{â}} genomen\\ 
25 & vromen und prîs zuo ir handen.\\ 
 & ich \textbf{bewar} iuch d\textit{â} vor schanden.\\ 
 & durch got, \textit{\textbf{nû}} senftet iuwern zorn!\\ 
 & \textbf{dâ} ist mê gewunnen danne verlorn\\ 
 & von iuwerre massenîe,\\ 
30 & wil es jehen \textbf{vrow\textit{e}} Obie."\\ 
\end{tabular}
\scriptsize
\line(1,0){75} \newline
m n o \newline
\line(1,0){75} \newline
\newline
\line(1,0){75} \newline
\textbf{1} \textit{Die Verse 359.1-4 fehlen} n o   $\cdot$ Poidiconiunz] poidocunnivnz m \textbf{2} herzogen] herczoge m  $\cdot$ Laverunz] laueruncz m \textbf{6} werde] werder n \textbf{7} Meliaganz] meliatanz m meliacontz n meliacancz o \textbf{9} möht] moht m moͯchte n mochte o  $\cdot$ dâ] do m \textit{om.} n o \textbf{10} kündet] kundet m (n) \textbf{12} ich] Ach o  $\cdot$ strîtes] stites n [strite]: strites o  $\cdot$ sat] mat n \textbf{15} Astor] astar o \textbf{17} Liz] lisz n o \textbf{18} slâfens] sloffes n (o) \textbf{19} sich] ich n o \textbf{20} daz] da m \textbf{21} dâ] do m n o \textbf{23} niht] \textit{om.} m \textbf{24} burgære] burgen o  $\cdot$ dâ] do m n o \textbf{26} dâ] do m n o \textbf{27} nû senftet] missenfftent m nuͯ senfften n ẏm senfftern o  $\cdot$ iuwern] vwer o \textbf{28} dâ] Do n o \textbf{29} iuwerre] yre m ir n o \textbf{30} vrowe] froͯwew m  $\cdot$ Obie] obye n \newline
\end{minipage}
\end{table}
\newpage
\begin{table}[ht]
\begin{minipage}[t]{0.5\linewidth}
\small
\begin{center}*G
\end{center}
\begin{tabular}{rl}
 & \begin{large}D\end{large}ô sprach Poydekoniunz\\ 
 & zem herzogen von Lanvarunz:\\ 
 & "geruocht ir mîn niht bîten,\\ 
 & \textbf{sô} ir vart \textbf{durch rüemen} strîten,\\ 
5 & \textbf{sô wænet ir}, daz sî guot getân?\\ 
 & hie ist \textit{der} \textbf{grâve} Lachdoman.\\ 
 & \textbf{hie ist} \textbf{ouch} Meliahganz, \textit{m}în sun.\\ 
 & swaz die bêde solden tuon\\ 
 & unde ich selbe, ir m\textit{ö}ht \textbf{\textit{do}ch} strîten sehen,\\ 
10 & obe ir strîten \textbf{kündet} spehen.\\ 
 & ich kum nimer von dirre stat,\\ 
 & ich \textbf{en}\textbf{gemache} \textbf{iuch} alle \textbf{vehtens} sat,\\ 
 & oder mir \textbf{gît} man unde wîp\\ 
 & her ûz gevangen \textbf{bêde ir} lîp."\\ 
15 & dô sprach der herzoge Astor:\\ 
 & "hêrre, iwer neve was \textbf{dâ} vor,\\ 
 & der künic, unde al sîn her von Liz.\\ 
 & solt iwer her an \textbf{slâfes} vlîz\\ 
 & die wîle \textbf{sich} haben gekêrt?\\ 
20 & habet ir uns daz gelêrt,\\ 
 & sô slâfe ich, \textbf{swâ} man strîten sol.\\ 
 & ich kan bî strîte slâfen wol.\\ 
 & \textbf{nû wizzet} \textbf{daz}, wære ich niht komen,\\ 
 & die burgære heten genomen\\ 
25 & vrum unde brîs ze ir handen.\\ 
 & ich \textbf{bewarte} iuch dâ vor schanden.\\ 
 & durch got, \textbf{nû} senftet iwern zorn!\\ 
 & \textbf{hie}st mê gewunnen dane verlorn\\ 
 & von iwere messenîe,\\ 
30 & wils jehen Obie."\\ 
\end{tabular}
\scriptsize
\line(1,0){75} \newline
G I O L M Q R Z Fr22 Fr39 \newline
\line(1,0){75} \newline
\textbf{1} \textit{Initiale} G M  \textbf{11} \textit{Initiale} I O Q Z   $\cdot$ \textit{Capitulumzeichen} R  \newline
\line(1,0){75} \newline
\textbf{1} Poydekoniunz] poydokomunz I Poydekomvnz O Poy de Conivnz L (Fr39) poide kvnivnz M poydekonvnz Q poẏdekomivnz R poidekonivnz Z \textbf{2} \textit{Vers 359.2 fehlt} Q   $\cdot$ Lanvarunz] lauarunz I Lanveronz O Lvnvarvnz L (Fr39) longvarvnz R lonvarvnz Z \textbf{3} ir] \textit{om.} Z \textbf{4} sô] swen I (O) (Z) Wenne L (M) (Q) (R)  $\cdot$ vart] ritet I vurt M  $\cdot$ rüemen] ruͤmes I rvͯm L (Q) (R) (Z) \textbf{5} sô wænet ir] ir went I So weninc ir M  $\cdot$ daz] isz M daz ich R \textbf{6} der] \textit{om.} G  $\cdot$ Lachdoman] lahdoman I (O) (L) Q (Z) Lachtoman R \textbf{7} \textit{Versfolge 359.8-7} Q   $\cdot$ ouch] \textit{om.} I M  $\cdot$ Meliahganz] Meliaganz I Melyakanz O Meliahkanz L Meliachkancz M meliachkans Q Meliakancz R meliahkantz Z  $\cdot$ mîn] sin G I \textbf{8} swaz] Waz L (M) (Q) (R)  $\cdot$ bêde solden] beden sollen Q \textbf{9} selbe] selben M  $\cdot$ möht] moht G O (L) (M) (Q) Z muset I moͯch R  $\cdot$ doch] oͮch G \textbf{10} kündet] chundet G (I) (O) (M) (Q) (R) (Z) \textbf{11} ich] ÷ch O  $\cdot$ kum] enkvm L (M) (Q) (Z)  $\cdot$ dirre] der L \textbf{12} ich engemache] ich gemach I Jch mache O R Z Jch enmache L (M) (Q)  $\cdot$ iuch] vns O (L) M Q R Z  $\cdot$ alle] \textit{om.} I  $\cdot$ vehtens] rechtes R \textbf{13} mir] man M \textbf{14} her] Er M  $\cdot$ bêde] bedú R \textbf{15} dô] Da M Z  $\cdot$ Astor] castor Q R \textbf{17} al] alle O  $\cdot$ Liz] lýsz L lisz M Q Lẏcz R \textbf{18} slâfes] slaffen I L falsches R \textbf{19} haben] hat L  $\cdot$ gekêrt] chert I \textbf{20} daz] \textit{om.} O L \textbf{21} swâ] wo L M Q (R) Z  $\cdot$ strîten] \textit{om.} L \textbf{22} strîte] striten M \textbf{23} nû] \textit{om.} I  $\cdot$ daz] \textit{om.} I R  $\cdot$ wære ich niht] ich niht wer O \textbf{25} vrum] Frvmen O (Q) (R) (Z) (Fr22)  $\cdot$ brîs] preises Q  $\cdot$ ze ir] zuͯr L zu Q dar Fr22  $\cdot$ handen] landen Z \textbf{26} ich bewarte] ir bewart I Jch bewar O (L) (M) Z (Fr22) Vnd be want Q  $\cdot$ iuch] auch Q  $\cdot$ dâ] do Q R  $\cdot$ vor] von I \textbf{27} nû] \textit{om.} I  $\cdot$ senftet iwern] senftet ewerm I senfte sinen L senftrent úwern R \textbf{30} wils] wil sin I Wil ichs Q  $\cdot$ jehen] liegen O  $\cdot$ Obie] obi I Obye O (R) oblie Q frow obye Z Obîe Fr22 \newline
\end{minipage}
\hspace{0.5cm}
\begin{minipage}[t]{0.5\linewidth}
\small
\begin{center}*T
\end{center}
\begin{tabular}{rl}
 & \begin{large}D\end{large}ô sprach Poydekuniunz\\ 
 & zem herzogen von Lunverunz:\\ 
 & "geruochet ir mîn niht bîten,\\ 
 & \textbf{swenn} ir vart \textbf{ze} strîten,\\ 
5 & \textbf{sô wænt ir}, daz sî guot getân?\\ 
 & hie ist der \textbf{grâve} Lachdoman\\ 
 & \textbf{unde} Melyahganz, mîn suon.\\ 
 & swaz die beide solten tuon\\ 
 & unde ich selbe, ir m\textit{ö}ht \textbf{doch} strîten sehen,\\ 
10 & ob ir strîten \textbf{künnet} spehen.\\ 
 & ich\textbf{n} kume niemer von dirre stat,\\ 
 & i\textbf{ne} \textbf{mache} \textbf{uns} alle \textbf{vehtens} sat,\\ 
 & oder mir \textbf{gît} man unde wîp\\ 
 & her ûz gevangen \textbf{ir beider} lîp."\\ 
15 & Dô sprach der herzoge Astor:\\ 
 & "hêrre, iuwer neve was \textbf{hie} vor,\\ 
 & der künec, unde alsîn her von Liz.\\ 
 & solte iuwer her an \textbf{slâfes} vlîz\\ 
 & die wîle hân gekêret?\\ 
20 & habt ir uns daz gelêret,\\ 
 & sô slâfich, \textbf{swâ} man strîten sol.\\ 
 & ich kan bî strîte slâfen wol.\\ 
 & \textbf{unde wizzet} \textbf{daz}, wære ich niht komen,\\ 
 & die burgære heten genomen\\ 
25 & vromen unde prîs zir handen.\\ 
 & ich \textbf{bewart}iuch dâ vor schanden.\\ 
 & durch got, senftet iuwern zorn!\\ 
 & \textbf{hie} ist mêr gewunnen danne verlorn\\ 
 & von iuwerre massenîe,\\ 
30 & wil es jehen Obie."\\ 
\end{tabular}
\scriptsize
\line(1,0){75} \newline
T V W \newline
\line(1,0){75} \newline
\textbf{1} \textit{Initiale} T W  \textbf{15} \textit{Majuskel} T  \newline
\line(1,0){75} \newline
\textbf{1} Poydekuniunz] poydekvmvns V poyde guniuns W \textbf{2} Lunverunz] lvnuernvns V lumiueruns W \textbf{3} geruochet] Ruͦchent W  $\cdot$ niht] \textit{om.} W \textbf{4} swenn] Wenn W  $\cdot$ ze] durch ruͤmen V durch ruͦm W \textbf{5} daz] das es V \textbf{6} grâve] werde V  $\cdot$ Lachdoman] Lohdoman T (W) laheduman V \textbf{7} unde] Hie ist auch W  $\cdot$ Melyahganz] meliaganz V meliagans W \textbf{8} swaz] Was W \textbf{9} möht] moht T múgent W  $\cdot$ doch] do V \textit{om.} W \textbf{10} künnet] kvndent V \textbf{11} ichn kume] Ich kum W \textbf{12} ine] Jch V (W)  $\cdot$ vehtens] streites W \textbf{16} hie] [*]: do V \textbf{17} Liz] Lyz T \textbf{19} hân] sich han V W \textbf{21} swâ] so W \textbf{22} ich kan] Hie kam W \textbf{23} daz] \textit{om.} W \textbf{24} genomen] [d*]: da genomen V \textbf{25} vromen unde prîs] Frome vnd preises W \textbf{26} bewartiuch dâ] bewartiv da T bewar eúch do W \textbf{27} senftet] [*]: nv senftent V \textbf{28} hie] [*]: Da V \textbf{30} es] sy es W  $\cdot$ jehen] iehen vrowe V  $\cdot$ Obie] Obŷe T \newline
\end{minipage}
\end{table}
\end{document}
