\documentclass[8pt,a4paper,notitlepage]{article}
\usepackage{fullpage}
\usepackage{ulem}
\usepackage{xltxtra}
\usepackage{datetime}
\renewcommand{\dateseparator}{.}
\dmyyyydate
\usepackage{fancyhdr}
\usepackage{ifthen}
\pagestyle{fancy}
\fancyhf{}
\renewcommand{\headrulewidth}{0pt}
\fancyfoot[L]{\ifthenelse{\value{page}=1}{\today, \currenttime{} Uhr}{}}
\begin{document}
\begin{table}[ht]
\begin{minipage}[t]{0.5\linewidth}
\small
\begin{center}*D
\end{center}
\begin{tabular}{rl}
\textbf{743} & \begin{large}D\end{large}er heiden truoc \textbf{zwô} geselleschaft,\\ 
 & dâr an \textbf{doch} lac sîn meistiu kraft.\\ 
 & einiu, daz er minne pflac,\\ 
 & diu mit stæte \textbf{in} sîme herzen lac.\\ 
5 & daz ander wâren steine,\\ 
 & \textbf{die} mit edelem arde reine\\ 
 & \textbf{in hôchgemüete} lêrten\\ 
 & unt sîne kraft \textbf{gemêrten}.\\ 
 & Mich \textbf{müet}, daz der getoufte\\ 
10 & an strîte und an loufte\\ 
 & \textbf{sus} \textbf{müedet} unt an starken slegen.\\ 
 & ob im nû niht gehelfen megen\\ 
 & Condwiramurs \textbf{noch} der Grâl,\\ 
 & werlîcher Parzival,\\ 
15 & \textbf{sô} \textbf{müezest} einen trôst \textbf{doch} haben,\\ 
 & daz die clâren, süezen knaben\\ 
 & \textbf{sus} vruo niht \textbf{verweiset} sîn,\\ 
 & Kardeiz unt Loherangrin,\\ 
 & die bêde \textbf{lebendic} truoc \textbf{sîn} wîp,\\ 
20 & dô er jungest umbevienc ir lîp.\\ 
 & mit rehter kiusche erworben kint,\\ 
 & ich wæne, diu smannes sælde sint.\\ 
 & Der getoufte nam an kreften zuo.\\ 
 & er \textbf{dâhte} - \textbf{des} was im niht ze vruo -\\ 
25 & an sîn wîp, die küneginne,\\ 
 & unt an ir werden minne,\\ 
 & die er mit swertes schimpfe erranc,\\ 
 & \textbf{dâ} viwer \textbf{von slegen ûz helmen} spranc,\\ 
 & \textbf{vor} Pelrapeire \textbf{an} Clamide.\\ 
30 & Thabronit unde Thasme,\\ 
\end{tabular}
\scriptsize
\line(1,0){75} \newline
D \newline
\line(1,0){75} \newline
\textbf{1} \textit{Initiale} D  \textbf{9} \textit{Majuskel} D  \textbf{23} \textit{Majuskel} D  \newline
\line(1,0){75} \newline
\textbf{14} Parzival] Parcifal D \textbf{18} Loherangrin] Loherangrîn D \textbf{29} Clamide] Chlamidê D \newline
\end{minipage}
\hspace{0.5cm}
\begin{minipage}[t]{0.5\linewidth}
\small
\begin{center}*m
\end{center}
\begin{tabular}{rl}
 & der heiden truoc \textbf{zwô} geselleschaft,\\ 
 & dâr an \textbf{doch} lac sîn meistiu kraft.\\ 
 & einiu, daz er minne pflac,\\ 
 & diu mit stæte \textbf{i\textit{n}} \textit{s}înem herzen lac.\\ 
5 & daz ander wâren steine,\\ 
 & \textbf{die} mit edelm arde reine\\ 
 & \textbf{hôchgemüete in} lêrten\\ 
 & und sîn kraft \textbf{gemêrten}.\\ 
 & mich \textbf{müet}, daz der getoufte\\ 
10 & an strîte und \textbf{ouch} an \textit{l}oufte\\ 
 & \textbf{sus} \textbf{müedet} und an starken slegen.\\ 
 & ob im nû niht gehelfen megen\\ 
 & Condwieramurs \textbf{noch} der Grâl,\\ 
 & \textbf{vil} we\textit{r}lîcher Parcifal,\\ 
15 & \textbf{sô} \textbf{muost dû} einen trôst \textbf{doch} haben,\\ 
 & daz die clâren, süezen knaben\\ 
 & \textbf{sus} vruo niht \textbf{verweiset} sîn,\\ 
 & Cardeiz und Loh\textit{e}lan\textit{g}rin,\\ 
 & die beide \textbf{lebendic} truoc \textbf{sîn} wîp,\\ 
20 & dô er jungst umbevienc ir lîp.\\ 
 & mit rehter kiusche erworben kint,\\ 
 & ich wæne, \textit{diu} d\textit{e}s mannes sælde sint.\\ 
 & der getoufte nam an \textit{k}re\textit{f}ten zuo.\\ 
 & er \textbf{gedâht} - \textbf{daz} was im niht ze vruo -\\ 
25 & an sîn wîp, die küniginne,\\ 
 & und an ir werden minne,\\ 
 & die \textit{er} mit swertes schimpf erranc,\\ 
 & \textbf{d\textit{â}} viur \textbf{ûz helmen} spranc,\\ 
 & \textbf{vor} Pelraperie \textbf{an} Clamide.\\ 
30 & \textit{T}abronit und Thas\textit{m}e,\\ 
\end{tabular}
\scriptsize
\line(1,0){75} \newline
m n o V V' Fr69 \newline
\line(1,0){75} \newline
\textbf{23} \textit{Initiale} V V'  \newline
\line(1,0){75} \newline
\textbf{2} doch lac] lag doch n lag V (V') \textbf{4} mit] \textit{om.} V'  $\cdot$ in sînem] in in sinem m \textbf{5} \textit{statt 743.5-8:} Daz ander worent steine / Die gaben ime craft vnd myne gemeine V'  \textbf{6} arde] arden Fr69 \textbf{8} gemêrten] merten V \textbf{9} \textit{Die Verse 743.9-22 fehlen} V'  \textbf{10} loufte] anlouft m luͯfft o \textbf{11} müedet] midet m n o \textbf{13} Condwieramurs] Condewier amurs n Cundwier amurs o Kvndewiramurs V \textbf{14} werlîcher] wellicher m  $\cdot$ Parcifal] parzefol V \textbf{15} muost dû] muͯste m muͤssest V  $\cdot$ einen trôst doch] doch einen trost n \textbf{17} sîn] sint o \textbf{18} Cardeiz] Kardeis m o Kordeis n Cardeis V  $\cdot$ Lohelangrin] loholangdrin m loholangrin n \textbf{20} jungst] zuͯ [jun*]: jungest n \textbf{22} diu des] dis m n (o) \textbf{23} kreften] rechten m \textbf{24} daz] des V (V') \textbf{26} werden] werde V \textbf{27} er] \textit{om.} m \textbf{28} dâ] Do m n o V V'  $\cdot$ ûz] \textit{om.} o von slegen vs V (V') \textbf{29} \textit{Die Verse 743.29-744.6 fehlen} V'   $\cdot$ vor] Von n  $\cdot$ Pelraperie] pelrapier n o Belreper V  $\cdot$ Clamide] clanide o \textbf{30} Tabronit] Da bronit m Thabronit n V  $\cdot$ Thasme] thasine m n o \newline
\end{minipage}
\end{table}
\newpage
\begin{table}[ht]
\begin{minipage}[t]{0.5\linewidth}
\small
\begin{center}*G
\end{center}
\begin{tabular}{rl}
 & \begin{large}D\end{large}er heiden truoc geselleschaft,\\ 
 & dâr an \textbf{doch} lac sîn meistiu kraft.\\ 
 & einiu, daz er minne pflac,\\ 
 & diu mit stæte \textbf{in} sînem herzen \textit{l}ac.\\ 
5 & daz ander wâren steine\\ 
 & mit edelm arde reine\\ 
 & \textbf{in hôchgemüete} lêrten\\ 
 & unde sîne kraft \textbf{mêrten}.\\ 
 & mich \textbf{müet}, daz der getoufte\\ 
10 & an strîte unde an loufte\\ 
 & \textbf{müedet} unde an starken slegen.\\ 
 & obe im nû niht gehelfen megen\\ 
 & Condwiramurs \textbf{noch} der Grâl,\\ 
 & werlîcher Parzival,\\ 
15 & \textbf{nû} \textbf{muostû} einen trôst \textbf{doch} haben,\\ 
 & daz die clâren, süezen knaben\\ 
 & \dag~\dag\ ,\\ 
 & Karadeiz unde Loherangrin,\\ 
 & die bêde \textbf{leben\textit{d}e} truoc \textbf{sîn} wîp,\\ 
20 & dô er jungest umbevienc ir lîp.\\ 
 & \multicolumn{1}{l}{ - - - }\\ 
 & \multicolumn{1}{l}{ - - - }\\ 
 & der getoufte nam an krefte\textit{n} zuo.\\ 
 & er \textbf{dâhte} - \textbf{ez} was im niht ze vruo -\\ 
25 & an sîn wîp, die kuniginne,\\ 
 & unde an ir werde minne,\\ 
 & die er mit swertes schimpfe erranc,\\ 
 & \textbf{daz} viur \textbf{ûz helmen von slegen} spranc,\\ 
 & \textbf{von} Pelrapeire Clamide.\\ 
30 & Tabrunit unde Tasme,\\ 
\end{tabular}
\scriptsize
\line(1,0){75} \newline
G I L M Z Fr24 Fr50 \newline
\line(1,0){75} \newline
\textbf{1} \textit{Initiale} G Z Fr24 Fr50  \textbf{9} \textit{Initiale} I  \textbf{29} \textit{Initiale} I  \newline
\line(1,0){75} \newline
\textbf{2} doch] ouch L M Z (Fr24) (Fr50)  $\cdot$ kraft] maistershaft I \textbf{4} lac] phlac G \textbf{5} daz] Dar Fr50 \textbf{6} mit] Die mit M  $\cdot$ edelm] edeler I \textbf{7} lêrten] lerte L \textbf{8} kraft] chrafte Fr50  $\cdot$ mêrten] gemerten I (M) \textbf{9} mich müet] mich mvte Z (Fr24) Mit mvt Fr50 \textbf{11} müedet] muͦte I Sus muͯdet M \textbf{13} Condwiramurs] kondwiramvurs G Condwir Amvrs L Gundwir Amuͯrsz M Kvndwiramvrs Z Gvndwir amvrs Fr24 Gvndwiramvrs Fr50  $\cdot$ noch] vnd L (M) Fr24 Fr50 \textbf{14} Parzival] parcifal G Z Fr24 Fr50 parzifal I L M \textbf{15} muostû] mvzzest Z  $\cdot$ doch] noch I \textbf{17} \textit{Vers 743.17 fehlt} G I   $\cdot$ verweiset] verweisent Fr50 \textbf{18} \textit{nach 743.18:} diu vil schonen kindelin I   $\cdot$ Karadeiz] Karadaiz I Karedeiz L Kardeisz M Karadeyz Fr24 karadiez Fr50  $\cdot$ unde] \textit{om.} Fr50  $\cdot$ Loherangrin] lohangrin G M Johangrin I Joherangrin L (Fr24) Lohrangrin Z lohangin Fr50 \textbf{19} lebende truoc] lebene troͮc G lebedende truͤc I truͯch lebende L lebindic truc M (Z) (Fr24) (Fr50) \textbf{20} dô] Da M Z \textbf{21} \textit{Die Verse 743.21-22 fehlen} G I   $\cdot$ Mit rechter kvͯsche erworben (erwerben Fr50 ) kint L (M) (Z) (Fr24) (Fr50) \textbf{22} Jch wane die des mannez salde sint L (M) (Z) (Fr24) (Fr50) \textbf{23} kreften] krefte G \textbf{24} dâhte] Gedaht I (L)  $\cdot$ ez] dez L  $\cdot$ was] enwas Z \textbf{26} werde] werden Fr50 \textbf{27} swertes] swert L  $\cdot$ schimpfe] kamph M \textbf{28} daz] Da M Z  $\cdot$ viur] vor M  $\cdot$ helmen] helme I helinc M den helm Fr50  $\cdot$ von slegen] \textit{om.} Fr50 \textbf{29} von] Vor L Z (Fr50)  $\cdot$ Pelrapeire] peilrapeire G Fr50 Pailrapaier I pelrapere M  $\cdot$ Clamide] an clamide M \textbf{30} Tabrunit] Tanprunit I  $\cdot$ Tasme] Tasine L thasme M \newline
\end{minipage}
\hspace{0.5cm}
\begin{minipage}[t]{0.5\linewidth}
\small
\begin{center}*T
\end{center}
\begin{tabular}{rl}
 & der heiden truoc geselleschaft,\\ 
 & dâr ane \textbf{ouch} lac sîn meistiu kraft.\\ 
 & einiu, daz er minnen pflac,\\ 
 & diu mit stæte \textbf{an} sîme herzen lac.\\ 
5 & daz ander wâren steine,\\ 
 & \textbf{die} mit edelme arde reine\\ 
 & \textbf{in hôchgemüete} lêrten\\ 
 & und sîne kraft \textbf{gemêrten}.\\ 
 & mich \textbf{müete}, daz der getoufte\\ 
10 & an strîte und an loufte\\ 
 & \textbf{sus} \textbf{muote} und an starken slegen.\\ 
 & ob im nû niht gehelfen megen\\ 
 & Kundewiramurs \textbf{und} der Grâl,\\ 
 & werlîcher Parcifal,\\ 
15 & \textbf{nû} \textbf{muo\textit{s}tû} einen trôst haben,\\ 
 & daz die clâren, süezen knaben\\ 
 & \textbf{sô} vrüeje niht \textbf{verwîset} sîn,\\ 
 & Kardeiz und Lohrangrin,\\ 
 & die bêde \textbf{lebendic} truoc \textbf{ein} wîp,\\ 
20 & dô er jungest umbevienc ir lîp.\\ 
 & mit rehter kiusche erworben kint,\\ 
 & ich wæne, diu des mannes sælde sint.\\ 
 & der getoufte nam an kreften zuo.\\ 
 & er \textbf{gedâhte} - \textbf{des} was im niht zuo vruo -\\ 
25 & an sîn wîp, die küneginne,\\ 
 & und an ir werde minne,\\ 
 & die er mit swertes schimpfe er\textit{r}a\textit{nc},\\ 
 & \textbf{d\textit{â}} viur \textbf{ûz den helmen von slegen} spranc,\\ 
 & \textbf{vor} Peilrapere \textbf{an} Clamide.\\ 
30 & Tabrunit und Tasme,\\ 
\end{tabular}
\scriptsize
\line(1,0){75} \newline
U W Q R \newline
\line(1,0){75} \newline
\textbf{1} \textit{Initiale} R  \textbf{21} \textit{Initiale} W  \newline
\line(1,0){75} \newline
\textbf{2} Dar an doch lag gancz sin meisterschafft R  $\cdot$ ouch lac] lag W lack auch Q \textbf{3} einiu] Eini R  $\cdot$ minnen] minne Q R \textbf{4} an] in W Q R \textbf{7} hôchgemüete] hoche gemuͯtte R \textbf{8} kraft] creft R  $\cdot$ gemêrten] gemerte Q merten R \textbf{9} müete] mut Q (R) \textbf{10} loufte] leuffte Q \textbf{11} sus] Als Q  $\cdot$ muote] muͤdet W (Q) (R)  $\cdot$ und] \textit{om.} R \textbf{13} Kundewiramurs] Kuͦndewiramuͦrs U kundwiramurs Q Kondwiramuͦrs R \textbf{14} Parcifal] Parzifal U herre partzifal W partzifal Q parczifal R \textbf{15} Nun muͦstest doch ein trost nun haben R  $\cdot$ nû] \textit{om.} Q  $\cdot$ muostû] muͦzzestuͦ U  $\cdot$ haben] noch haben W im [ha*]: haben Q \textbf{17} Nicht so fruͦ erbeiset sin W · So fruͯmt verwisset sind R  $\cdot$ verwîset] verweiset Q \textbf{18} Kardeiz] Cardeis W Kardetz Q [Kard*]: Kardeis R  $\cdot$ Lohrangrin] Loharngrin U lohelangrin W lochrangrin Q R \textbf{19} die bêde] Die beden Q Dú beidu R  $\cdot$ ein] sein W Q (R) \textbf{20} ir] seyn Q \textbf{21} mit rehter kiusche] sEit rechte keúsch W \textbf{23} nam] mā Q \textbf{24} des] das W Q (R) \textbf{26} werde] werden W R \textbf{27} erranc] erwarp U \textbf{28} dâ] Do U W Q  $\cdot$ den] \textit{om.} W Q R  $\cdot$ von] mit Q \textbf{29} vor] Von W Q  $\cdot$ Peilrapere] Peilraper U pelrapeir W pelrapeire Q pelrapire R  $\cdot$ Clamide] klamide W \textbf{30} Tabrunit] Tabruͦnit U Tabrúnit Q [Thabunit]: Thabuornit thaburnit R  $\cdot$ Tasme] Thasme R \newline
\end{minipage}
\end{table}
\end{document}
