\documentclass[8pt,a4paper,notitlepage]{article}
\usepackage{fullpage}
\usepackage{ulem}
\usepackage{xltxtra}
\usepackage{datetime}
\renewcommand{\dateseparator}{.}
\dmyyyydate
\usepackage{fancyhdr}
\usepackage{ifthen}
\pagestyle{fancy}
\fancyhf{}
\renewcommand{\headrulewidth}{0pt}
\fancyfoot[L]{\ifthenelse{\value{page}=1}{\today, \currenttime{} Uhr}{}}
\begin{document}
\begin{table}[ht]
\begin{minipage}[t]{0.5\linewidth}
\small
\begin{center}*D
\end{center}
\begin{tabular}{rl}
\textbf{105} & dô kom geriten Tampanis,\\ 
 & ir mannes \textbf{meisterknappe} wîs,\\ 
 & unt kleiner \textbf{juncvrouwen} vil.\\ 
 & dâ gie ez ûz der vreuden zil.\\ 
5 & \textbf{die} sageten klagende ir hêrren tôt.\\ 
 & des kom vrou Herzeloyde in nôt.\\ 
 & si viel hin unversunnen.\\ 
 & die ritter sprâchen: "wie ist gewunnen\\ 
 & mîn hêrre in sîme harnas,\\ 
10 & sô wol \textbf{gewâpent, sô er} was?"\\ 
 & Swie den knappen jâmer jagete,\\ 
 & den \textbf{helden} er \textbf{doch} sagete:\\ 
 & "mînen hêrren lebens lenge vlôch.\\ 
 & sîn härse\textit{n}i\textit{er} \textbf{von im er} zôch,\\ 
15 & des twanc in starkiu hitze.\\ 
 & geunêrtiu \textbf{heidensch} witze\\ 
 & hât uns verstoln den helt guot.\\ 
 & ein ritter hete bockes bluot\\ 
 & genomen in ein langez glas.\\ 
20 & daz sluog er \textbf{an} den adamas.\\ 
 & dô wart er weicher danne ein swamp.\\ 
 & den man \textbf{noch} mâlet vür\textbf{z} lamp\\ 
 & unt \textbf{ouch}z kriuze in \textbf{sîne} klân,\\ 
 & \textbf{den} erbarme, daz dâ \textbf{wart} getân!\\ 
25 & dô si mit scharn \textbf{zein ander} riten,\\ 
 & âvoy, wie dâ wart gestriten!\\ 
 & \begin{large}D\end{large}es bâruckes ritterschaft\\ 
 & sich werte wol mit ellens kraft.\\ 
 & \textbf{von} Baldac ûf dem \textbf{gevilde}\\ 
30 & durchstochen wart vil schilde,\\ 
\end{tabular}
\scriptsize
\line(1,0){75} \newline
D \newline
\line(1,0){75} \newline
\textbf{11} \textit{Majuskel} D  \textbf{27} \textit{Initiale} D  \newline
\line(1,0){75} \newline
\textbf{14} härsenier] hærserin D \textbf{29} Baldac] Baldach D \newline
\end{minipage}
\hspace{0.5cm}
\begin{minipage}[t]{0.5\linewidth}
\small
\begin{center}*m
\end{center}
\begin{tabular}{rl}
 & \begin{large}D\end{large}ô kam geriten Tampanis,\\ 
 & ir mannes \textbf{knappen meister} wîs,\\ 
 & und kleiner \textbf{junchêrren} vil.\\ 
 & d\textit{â} gienc ez ûz der vröuden zil.\\ 
5 & \textbf{die} sageten klagende ir hêrren tôt.\\ 
 & \textit{d}es kam vrouwe Herczeloide in nôt.\\ 
 & si viel hin unversunnen.\\ 
 & die ritter sprâchen: "\textit{wie ist g}e\textit{w}unnen\\ 
 & mîn hêrre in sînem harna\textit{s},\\ 
10 & sô wol, \textbf{al\textit{s} \textit{e}r gewâpent} was?"\\ 
 & wie den knappen jâmer jagete,\\ 
 & den \textbf{hêrren} er \textbf{doch} sagete:\\ 
 & "mînen hêr\textit{r}en lebennes \textit{lenge} vlô\textit{ch}.\\ 
 & sîn hersenier \textbf{von ime \textit{er}} zôch,\\ 
15 & des twanc in starkiu hitze.\\ 
 & \dag generte\dag  \textbf{heidensche} witze\\ 
 & hât uns verstoln den helt guot.\\ 
 & ein ritter hete bockes bluot\\ 
 & genomen in ein langez glas.\\ 
20 & daz sluoc er \textbf{û\textit{f}} den adamas.\\ 
 & dô wart er weicher dan ein swamp.\\ 
 & den man \textbf{noch} mâlet vür \textbf{daz} lamp\\ 
 & und \textbf{ouch} daz kriuz in \textbf{sînen} klân,\\ 
 & \textbf{den} erbarme, daz dâ \textbf{war\textit{t}} getân!\\ 
25 & dô si mit scharn \textbf{zen andern} riten,\\ 
 & â\textit{v}oy, wie dâ wart gestriten!\\ 
 & des bâruckes ritterschaft\\ 
 & sich werte wol mit ellens kraft.\\ 
 & \textbf{vor} Baldac ûf dem \textbf{gewilde}\\ 
30 & durchstochen wart vil schilte,\\ 
\end{tabular}
\scriptsize
\line(1,0){75} \newline
m n o \newline
\line(1,0){75} \newline
\textbf{1} \textit{Illustration mit Überschrift:} Also die herren koment geritten mit grosser macht zuͯ der stat in vnd vil frouwen an der zinnen logent n (o)   $\cdot$ \textit{Initiale} m n o  \newline
\line(1,0){75} \newline
\textbf{1} Tampanis] kampanis n o \textbf{2} mannes] mannes dochter n \textbf{3} junchêrren] júncherlin o \textbf{4} dâ] Do m n o  $\cdot$ gienc ez ûz] gins es usser o \textbf{6} des] Dies m  $\cdot$ Herczeloide] [herczeloiedi]: herczeloiede m hertzeloit n herczeleid o \textbf{8} wie ist gewunnen] vnversunen m \textbf{9} harnas] harnach m harnesch sas n \textbf{10} als er] als es er m \textbf{11} jagete] iagt o \textbf{13} hêrren] hertten m  $\cdot$ lenge vlôch] floss m \textbf{14} er] \textit{om.} m \textbf{15} starkiu] manige o \textbf{16} generte] Genert n \textbf{18} hete] hat n \textbf{20} er] der o  $\cdot$ ûf den] vs [er]: den m  $\cdot$ adamas] adamast m \textbf{22} vür daz] also n als das o \textbf{23} kriuz] ruͯcz o \textbf{24} dâ] do n o  $\cdot$ wart] war: m \textbf{25} zen andern] zuͯ einander n (o) \textbf{26} âvoy] Anoi m Any n o  $\cdot$ dâ] do n \textit{om.} o \textbf{27} bâruckes] beruckes n \textbf{29} Baldac] baldack m baldag n o  $\cdot$ gewilde] gefilde n o \newline
\end{minipage}
\end{table}
\newpage
\begin{table}[ht]
\begin{minipage}[t]{0.5\linewidth}
\small
\begin{center}*G
\end{center}
\begin{tabular}{rl}
 & dô kom geriten Tampanis,\\ 
 & ir mannes \textit{\textbf{meister ein knappe}} wîs,\\ 
 & unde kleiner \textbf{junchêrren} vil.\\ 
 & dâ gieng ez ûz der vröuden zil.\\ 
5 & \textbf{si} sagten klagende ir hêrren tôt.\\ 
 & des kom \textit{vrô} \textit{Herzeloide} in nôt.\\ 
 & si viel hin unversunnen.\\ 
 & die rîter sprâchen: "wie ist gewunnen\\ 
 & mîn hêrre in sînem harnas,\\ 
10 & sô wol \textbf{gewâpent er} was?"\\ 
 & swie den knappen jâmer jagte,\\ 
 & den \textbf{helden} er \textbf{doch} sagte:\\ 
 & "mînen hêrren lebens lenge vlôch.\\ 
 & \begin{large}S\end{large}în härsenier \textbf{er von im} zôch,\\ 
15 & des twanc in starkiu hitze.\\ 
 & geunêrtiu \textbf{heldes} witze\\ 
 & hât uns \textit{v}e\textit{rst}o\textit{l}n den helt guot.\\ 
 & ein rîter hete bockes bluot\\ 
 & genomen in ein langez glas.\\ 
20 & daz sluog er \textbf{ûf} den adamas.\\ 
 & dô ward er weicher danne ein swamp.\\ 
 & den man \textbf{noch} mâlet vür \textbf{daz} lamp\\ 
 & unt daz kriuze \textit{in} \textbf{sînen} klân,\\ 
 & \textbf{dem} erbarme, daz dâ \textbf{sî} getân!\\ 
25 & dô si mit scharen \textbf{ze ein ander} riten,\\ 
 & âvoy, wie dâ wart gestriten!\\ 
 & des bâruckes rîterscha\textit{f}t\\ 
 & sich werte wol mit ellens kraft.\\ 
 & \textbf{vor} Baldac ûf dem \textbf{gevilde}\\ 
30 & durchstochen wart vil schilde,\\ 
\end{tabular}
\scriptsize
\line(1,0){75} \newline
G I O L M Q R Z Fr21 Fr36 \newline
\line(1,0){75} \newline
\textbf{1} \textit{Initiale} O  \textbf{14} \textit{Initiale} G  \textbf{25} \textit{Initiale} I L R Z Fr21  \newline
\line(1,0){75} \newline
\textbf{1} dô] ÷o O Da M Z  $\cdot$ kom geriten] quamen geretten M qvam ritten Z  $\cdot$ Tampanis] Tanponeis I [*ampanis]: Tampanis O tamponis M tampaniz Q \textbf{2} meister ein knappe] chnappen meister G meister in knappen L (Fr21) meister knapin M (Z) \textbf{3} junchêrren] iuncherrlin Z \textbf{4} dâ] do I (O) (L) (M) (Q) (R)  $\cdot$ ûz der vröuden zil] vz der freuden spil I vsser kindes spil R \textbf{5} sagten klagende] clagten vast I clageten sagende M  $\cdot$ ir] irn L M irs Q (R) Z \textbf{6} vrô Herzeloide] div chungin G vro herzelaude I vrav herzelavde O frow Hertzelauͯde L froy herczeloide M fraw herzeloude Q fro herczelaude R herzelovde Z froͮ herzelovde Fr21 \textbf{7} si] diu I \textbf{8} ist] sy R \textbf{9} mîn] Vnser Q  $\cdot$ in sînem] vnde sin M \textbf{10} sô] Se M  $\cdot$ er] als er Q \textbf{11} swie] Wie L (M) Q R  $\cdot$ den] die R  $\cdot$ jagte] iagt L (M) Z \textbf{12} helden] herren L  $\cdot$ doch] do I L so M  $\cdot$ sagte] sagt L (M) Z \textbf{13} mînen] Minem R Mins Z Minnen Fr21  $\cdot$ lebens] leben Z  $\cdot$ lenge] lange R \textbf{14} Sîn] daz I  $\cdot$ härsenier] her siner R  $\cdot$ er] \textit{om.} O M  $\cdot$ zôch] er zoch O irczoch M \textbf{15} starkiu] starke R \textbf{16} geunêrtiu] Gvnertie L Si nerte M Gunertú trúwe R  $\cdot$ heldes] hadnisch I heidens O L (M) Z veindes Q hendes R handens Fr21 \textbf{17} verstoln] benomen G  $\cdot$ helt] helden M \textbf{18} hete] hat M \textit{om.} Z \textbf{21} dô] Da O M Z \textbf{22} man] \textit{om.} R  $\cdot$ noch] doch I  $\cdot$ daz] [ein]: daz O ein Z \textbf{23} in] hat vnder G an M  $\cdot$ sînen] sine L M Z seiner Q sinem R sin Fr21  $\cdot$ klân] lan R \textbf{24} dem] Den L Q R Z  $\cdot$ erbarme] erbarnte Q  $\cdot$ dâ] ez hie I do Q das R \textbf{25} dô] ÷o I Da O M Z  $\cdot$ scharen] schren Q  $\cdot$ ze ein ander] zuͯ samen L (R)  $\cdot$ riten] [chomen]: riten O \textbf{26} âvoy] Awe O Ach M  $\cdot$ dâ] do Q R  $\cdot$ wart] wirt M \textbf{27} bâruckes] brauches Q  $\cdot$ rîterschaft] riterschat G \textbf{28} werte wol] werte I wol wert O Fr21  $\cdot$ ellens] erens Q \textbf{29} vor] Ze O  $\cdot$ Baldac] baldach G (O) (L) Baldack R \textbf{30} schilde] der schilde I \newline
\end{minipage}
\hspace{0.5cm}
\begin{minipage}[t]{0.5\linewidth}
\small
\begin{center}*T (U)
\end{center}
\begin{tabular}{rl}
 & dô kam geriten Tampanis,\\ 
 & ir mannes \textbf{meisterknapp\textit{e}} wîs,\\ 
 & und kleiner \textbf{junchêrren} vil.\\ 
 & d\textit{â} gienc ez ûz der vreuden zil.\\ 
5 & \textbf{si} \textit{s}ageten klagende ir hêrren tôt.\\ 
 & des kam vrô Herzeloyde in nôt.\\ 
 & si viel hin unversunnen.\\ 
 & die ritter sprâchen: "wie ist gewunnen\\ 
 & mîn hêrre in sîme harnas,\\ 
10 & sô wol \textbf{gewâpent, sô er} was?"\\ 
 & wie den knappen jâmer jagete,\\ 
 & den \textbf{helden} er \textbf{dô} sagete:\\ 
 & "mînen hêrren lebens lenge vlôch.\\ 
 & sîn hersenier \textbf{er von im} zôch,\\ 
15 & des twanc in starkiu hitze.\\ 
 & geunêrtiu \textbf{heidensch} witze\\ 
 & hât uns verstoln den helt guot.\\ 
 & ein ritter, \textbf{der} hete bockes bluot\\ 
 & genomen in ein langez glas.\\ 
20 & daz sluoc er \textbf{ûf} den adamas.\\ 
 & dô wart er weicher dan ein swamp.\\ 
 & den man mâlet vür \textbf{ein} lamp\\ 
 & und daz kriuze in \textbf{sînen} klân,\\ 
 & \textbf{den} erbarme, daz dâ \textbf{sî} getân!\\ 
25 & \begin{large}D\end{large}ô si mit scharn \textbf{zuo \textit{ein} ander} riten,\\ 
 & âvoy, wie dâ wart gestriten!\\ 
 & \multicolumn{1}{l}{ - - - }\\ 
 & \multicolumn{1}{l}{ - - - }\\ 
 & \textbf{vor} Baldac ûf dem \textbf{gevilde}\\ 
30 & durchstochen wart vil schilde,\\ 
\end{tabular}
\scriptsize
\line(1,0){75} \newline
U V W T \newline
\line(1,0){75} \newline
\textbf{1} \textit{Initiale} T  \textbf{5} \textit{Majuskel} T  \textbf{11} \textit{Majuskel} T  \textbf{25} \textit{Initiale} U V W   $\cdot$ \textit{Majuskel} T  \textbf{27} \textit{Majuskel} T  \newline
\line(1,0){75} \newline
\textbf{1} Tampanis] Tampenis U V tampaneis W Tampanîs T \textbf{2} meisterknappe] meister knappen U (T) meister ein knappe V knappe ein maister W \textbf{3} junchêrren] iunckherlein W \textbf{4} dâ] Do U (V) W (T)  $\cdot$ zil] spil T \textbf{5} sageten klagende] clageten clagende U clageten vaste T  $\cdot$ ir] irn W \textbf{6} Herzeloyde] Herzeleide U Hertzelaude V hertzeloyde W \textbf{7} hin] nider W \textbf{9} mîn] [*]: Vnser V vnser T \textbf{10} sô er] als er V W er T \textbf{11} wie] Swie do V Swie T \textbf{12} dô] doch W \textbf{13} mînen] [M*]: Mines V mins T  $\cdot$ lebens lenge] lenge seines lebens W \textbf{15} starkiu] starke T \textbf{16} geunêrtiu] Geunerten W gevnêrte T  $\cdot$ heidensch] heidens V (W) T \textbf{18} der] \textit{om.} V T \textbf{22} mâlet] noch malet W (T)  $\cdot$ ein] daz V (W) (T) \textbf{23} sînen] sein W \textbf{24} dâ] do W \textbf{25} zuo ein] zuͦ U zuͦn V \textbf{26} dâ] do V W \textbf{27} \textit{Die Verse 105.27-28 fehlen} U W   $\cdot$ Des Barukes ritterschaft V (T) \textbf{28} sich werte wol mit ellens kraft V (T) \textbf{29} Baldac] Baldag V (W) Balde T \newline
\end{minipage}
\end{table}
\end{document}
