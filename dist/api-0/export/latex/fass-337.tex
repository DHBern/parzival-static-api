\documentclass[8pt,a4paper,notitlepage]{article}
\usepackage{fullpage}
\usepackage{ulem}
\usepackage{xltxtra}
\usepackage{datetime}
\renewcommand{\dateseparator}{.}
\dmyyyydate
\usepackage{fancyhdr}
\usepackage{ifthen}
\pagestyle{fancy}
\fancyhf{}
\renewcommand{\headrulewidth}{0pt}
\fancyfoot[L]{\ifthenelse{\value{page}=1}{\today, \currenttime{} Uhr}{}}
\begin{document}
\begin{table}[ht]
\begin{minipage}[t]{0.5\linewidth}
\small
\begin{center}*D
\end{center}
\begin{tabular}{rl}
\textbf{337} & \begin{large}N\end{large}û weiz ich wol, swelch sinnec wîp,\\ 
 & ob si hât getriwen lîp,\\ 
 & diu diz mære geschriben siht,\\ 
 & daz si \textbf{mir} mit wârheit giht,\\ 
5 & ich \textbf{kunde} wîben sprechen baz,\\ 
 & denne als ich sanc gein einer maz.\\ 
 & Diu küneginne Belakane\\ 
 & was \textbf{missewenden} âne\\ 
 & unt aller valscheit laz,\\ 
10 & dô si ein tôter künec besaz.\\ 
 & sît gap vroun Herzeloyden \textbf{troum}\\ 
 & siufzebæren \textbf{herzeroum}.\\ 
 & Welch was vroun Ginovern klage\\ 
 & an Ithers endetage!\\ 
15 & dar zuo was mir ein trûren leit,\\ 
 & daz \textbf{alsô schamlîchen} reit\\ 
 & des küneges kint von Karnant,\\ 
 & vrou Jeschute, \textbf{von kiusche} erkant.\\ 
 & wie wart vrou Cunneware\\ 
20 & \textbf{gealûnet} \textbf{mit} ir hâre!\\ 
 & des sint si vaste wider komen:\\ 
 & ir bêder scham hât prîs genomen.\\ 
 & ze machene nem diz mære ein man,\\ 
 & der âventiure \textbf{prüeven} kan\\ 
25 & unt \textbf{rîme} künne sprechen,\\ 
 & be\textit{i}diu samnen unt \textbf{zerbrechen}.\\ 
 & ich tætez \textbf{iu} gerne vürbaz kunt,\\ 
 & wolt ez gebieten mir ein munt,\\ 
 & den doch ander vüeze tragent,\\ 
30 & denne die mir ze \textbf{stegreifen} wagent.\\ 
\end{tabular}
\scriptsize
\line(1,0){75} \newline
D \newline
\line(1,0){75} \newline
\textbf{1} \textit{Initiale} D  \textbf{7} \textit{Majuskel} D  \textbf{13} \textit{Majuskel} D  \newline
\line(1,0){75} \newline
\textbf{7} Diu] De D  $\cdot$ Belakane] Belakâne D \textbf{14} Ithers] Ĵthers D \textbf{18} Jeschute] Jescv̂te D \textbf{26} beidiu] beldiv D \newline
\end{minipage}
\hspace{0.5cm}
\begin{minipage}[t]{0.5\linewidth}
\small
\begin{center}*m
\end{center}
\begin{tabular}{rl}
 & nû weiz ich wol, welch si\textit{nnic} wîp,\\ 
 & ob si hât getriuwen lîp,\\ 
 & diu diz mære geschriben siht,\\ 
 & daz si \textbf{mir} mit wârheit giht,\\ 
5 & ich \textbf{kunde} wîbe\textit{n} sprechen baz,\\ 
 & dannals ich \dag sante\dag  gegen einer maz.\\ 
 & diu küniginne Belakane\\ 
 & was \textbf{missewen\textit{den}} âne\\ 
 & und aller valscheit laz,\\ 
10 & dô si ein tôter künic besaz.\\ 
 & sît gap vrouwen Herczeloiden \textbf{troum}\\ 
 & siufzebæren \textbf{herzen roum}.\\ 
 & welch was vrouwen Ginovern klag\textit{e}\\ 
 & an I\textit{t}he\textit{r}s endetag\textit{e}!\\ 
15 & dar zuo was mir ein trûren leit,\\ 
 & daz \textbf{alsô schamlîchen} reit\\ 
 & des küniges kint von Karnant,\\ 
 & vrouwe Jeschute erkant.\\ 
 & wie wart vrouwe Cu\textit{nne}w\textit{a}re\\ 
20 & \textbf{gealûnet} \textbf{mit} ir hâre!\\ 
 & des sint si vaste wider komen:\\ 
 & ir beider scham hât prîs genomen.\\ 
 & ze machen neme diz mære ein man,\\ 
 & der âventiure \textbf{prüeven} kan\\ 
25 & und \textbf{rîme} künne spr\textit{e}chen,\\ 
 & beidiu samen und \textbf{br\textit{e}chen}.\\ 
 & ich tæte ez gerne vürbaz kunt,\\ 
 & wolt ez gebieten mir ein munt,\\ 
 & den doch ander \textit{v}üeze tragent,\\ 
30 & danne die mir ze \textbf{stegereifen} wagent.\\ 
\end{tabular}
\scriptsize
\line(1,0){75} \newline
m n o \newline
\line(1,0){75} \newline
\newline
\line(1,0){75} \newline
\textbf{1} weiz] weich n  $\cdot$ sinnic] sime m sonig o \textbf{2} hât] hette n \textbf{3} diz] dise n  $\cdot$ geschriben] beschriben n (o) \textbf{5} ich] Er o  $\cdot$ wîben] wibes m \textbf{6} dannals] Dan o \textbf{7} Belakane] belackane o \textbf{8} missewenden] missewen m \textbf{10} tôter] torter o \textbf{11} vrouwen] frouwe m n (o)  $\cdot$ Herczeloiden] hertzeleiden n herczoleide o \textbf{13} welch] We ach n  $\cdot$ vrouwen] frouw m (n) (o)  $\cdot$ Ginovern] ginoferen n ginofere o  $\cdot$ klage] clagen m (o) \textbf{14} an] Vnd o  $\cdot$ Ithers] Icherens m jthers n thers o  $\cdot$ endetage] ende tagen m \textbf{18} Jeschute] jescutte m jescute n o  $\cdot$ erkant] kúsche erkant n (o) \textbf{19} Cunneware] cumuwere m conneware n Conne waren o \textbf{20} gealûnet] Galuet o \textbf{23} diz] dise n \textbf{25} rîme] rume n (o)  $\cdot$ sprechen] sprochen m \textbf{26} brechen] brochen m \textbf{27} ez] es úch n uͯch o \textbf{29} vüeze] suͯsse m \textbf{30} danne] Die [man]: Dan o  $\cdot$ ze] \textit{om.} o  $\cdot$ stegereifen] stegereiff n o \newline
\end{minipage}
\end{table}
\newpage
\begin{table}[ht]
\begin{minipage}[t]{0.5\linewidth}
\small
\begin{center}*G
\end{center}
\begin{tabular}{rl}
 & \multicolumn{1}{l}{ - - - }\\ 
 & \multicolumn{1}{l}{ - - - }\\ 
 & \multicolumn{1}{l}{ - - - }\\ 
 & \multicolumn{1}{l}{ - - - }\\ 
5 & \multicolumn{1}{l}{ - - - }\\ 
 & \multicolumn{1}{l}{ - - - }\\ 
 & \multicolumn{1}{l}{ - - - }\\ 
 & \multicolumn{1}{l}{ - - - }\\ 
 & \multicolumn{1}{l}{ - - - }\\ 
10 & \multicolumn{1}{l}{ - - - }\\ 
 & \multicolumn{1}{l}{ - - - }\\ 
 & \multicolumn{1}{l}{ - - - }\\ 
 & \multicolumn{1}{l}{ - - - }\\ 
 & \multicolumn{1}{l}{ - - - }\\ 
15 & \multicolumn{1}{l}{ - - - }\\ 
 & \multicolumn{1}{l}{ - - - }\\ 
 & \multicolumn{1}{l}{ - - - }\\ 
 & \multicolumn{1}{l}{ - - - }\\ 
 & \multicolumn{1}{l}{ - - - }\\ 
20 & \multicolumn{1}{l}{ - - - }\\ 
 & \multicolumn{1}{l}{ - - - }\\ 
 & \multicolumn{1}{l}{ - - - }\\ 
 & \multicolumn{1}{l}{ - - - }\\ 
 & \multicolumn{1}{l}{ - - - }\\ 
25 & \multicolumn{1}{l}{ - - - }\\ 
 & \multicolumn{1}{l}{ - - - }\\ 
 & \multicolumn{1}{l}{ - - - }\\ 
 & \multicolumn{1}{l}{ - - - }\\ 
 & \multicolumn{1}{l}{ - - - }\\ 
30 & \multicolumn{1}{l}{ - - - }\\ 
\end{tabular}
\scriptsize
\line(1,0){75} \newline
G I O L M Q R Z Fr39 \newline
\line(1,0){75} \newline
\textbf{1} \textit{Initiale} Z  \newline
\line(1,0){75} \newline
\textbf{1} \textit{Die Verse 336.1-337.30 fehlen} G I O L M Fr39   $\cdot$ Nun weisz ich wol welch (swelich Z ) sinnic weip Q (Z)  $\cdot$ Jch weis wol soͯlich sinnig Wib R \textbf{2} Ob sie hott getrewen leip Q (R) (Z) \textbf{3} Die disz mere geschriben sicht Q (R) (Z) \textbf{4} Das sie mir mit worheit gicht (vergiht R ) Q (R) (Z) \textbf{5} Jch kunde weiben gesprechen (sprechen Z ) basz Q (R) (Z) \textbf{6} Dann das (als Z ) ich sant (sanc Z ) gein einer masz Q (Z)  $\cdot$ Denne das santte gen inen minen has R \textbf{7} Die konigein belakane (Belacone R belankane Z ) Q (R) (Z) \textbf{8} Was missewenden (missewende Z ) ane Q (R) (Z) \textbf{9} Vnd aller (alles R ) falscheit lasz Q (R) (Z) \textbf{10} Do (Da Z ) sie ein toter kûnick besasz Q (R) (Z) \textbf{11} Sint gabt frawen herzeloude trouͯm Q  $\cdot$ Sit gab sy frow herczelauden trom R  $\cdot$ Sit gap frowen hertzenlovden trovm Z \textbf{12} Sufzebern (Súnffczenbere R ) herzen (hertze Z ) roûm Q (R) (Z) \textbf{13} Welch was frawen (fro R ) ginoueren (Signower R gynovern Z ) clage Q (R) (Z) \textbf{14} An ithers (Jhters R Jthers Z ) ende tage Q (R) (Z) \textbf{15} Do zu was mir ein (ir R ) trauren leit Q (R) (Z) \textbf{16} Das also schemelichen reit Q (Z)  $\cdot$ Do sy so schamlich reit R \textbf{17} Des koniges kint von karnant Q (R) (Z) \textbf{18} Fraw iescute (Jscuten R Jescute Z ) keusch erkant Q (R) (Z) \textbf{19} Wie wart fraw conware (Cuͦnware R kuneware Z ) Q (R) (Z) \textbf{20} Gealúnt (Gehandelt R ) mit ir hare Q (R) (Z) \textbf{21} Des seint si faste wider komen Q (R) (Z) \textbf{22} Jr beder scham hot preisz genomen Q (R) (Z) \textbf{23} Czu manchen nemen (Ze machent R Leiht machen nem Z ) diese mere eyn man Q (R) (Z) \textbf{24} Der awentewre prufen (erkenen R ) kon Q (R) (Z) \textbf{25} Vnd reúmme (rime R Z ) kunne (konde R ) gesprechen (sprechen R Z ) Q (R) (Z) \textbf{26} beide samne vnd zu brechen Q (R)  $\cdot$ Beide samenen vnd brechen Z \textbf{27} Jch tetes euch furbas gerne (gern fúrbas R [ Z ]) kunt Q (R) (Z) \textbf{28} Wolt es gebieten (erlovben Z ) mir (\textit{om.} R ) ein (min R ) mûnt Q (R) (Z) \textbf{29} Den doch ander fusze tragent Q (R) (Z) \textbf{30} Den die mir zu stegereife ragent (wagent R Z ) Q (R) (Z) \newline
\end{minipage}
\hspace{0.5cm}
\begin{minipage}[t]{0.5\linewidth}
\small
\begin{center}*T
\end{center}
\begin{tabular}{rl}
 & \begin{large}N\end{large}û weiz ich wol, swelch sinnic wîp,\\ 
 & ob si hât getriuwen lîp,\\ 
 & diu diz mære geschriben siht,\\ 
 & daz si mit wârheite giht,\\ 
5 & ich \textbf{künne} wîben sprechen baz,\\ 
 & danne als ich sanc gegen einer maz.\\ 
 & Diu künegîn Belacane\\ 
 & was \textbf{missewende} âne\\ 
 & unde aller valscheite laz,\\ 
10 & dô si ein tôter künec besaz.\\ 
 & Sît gab vroun Herzeloyden \textbf{tuon}\\ 
 & \textbf{vil} siufzebæren \textbf{herzen ruon}.\\ 
 & Welch was vroun Gynovern klage\\ 
 & An Ithers endetage!\\ 
15 & Dar zuo was mir ein trûren leit,\\ 
 & daz \textbf{diu herzogîn sô nacket} reit,\\ 
 & des küneges kint von Garnant,\\ 
 & vrou Jeschute, \textbf{diu kiusche} erkant.\\ 
 & Wie wart vrou Cunneware\\ 
20 & \textbf{geroufet} \textbf{bî} ir hâre!\\ 
 & des sint si vaste wider komen:\\ 
 & ir beider schame hât prîs genomen.\\ 
 & Ze machene nem diz mære ein man,\\ 
 & der âventiure \textbf{erkennen} kan\\ 
25 & unde \textbf{der} \textbf{schône} künne sprechen,\\ 
 & beid\textit{iu} samnen unde \textbf{brechen}.\\ 
 & ich tætez \textbf{iu} gerne vürbaz kunt,\\ 
 & wolt ez gebieten mir ein munt,\\ 
 & den doch ander vüeze tragent,\\ 
30 & danne die mir ze \textbf{stegreife} wagent.\\ 
\end{tabular}
\scriptsize
\line(1,0){75} \newline
T U V W \newline
\line(1,0){75} \newline
\textbf{1} \textit{Überschrift:} Hie rait her gawan gen schanfesum zuͦ dem kampffe W   $\cdot$ \textit{Platz für Illustration ausgespart} W   $\cdot$ \textit{Initiale} T U W  \textbf{7} \textit{Majuskel} T  \textbf{11} \textit{Majuskel} T  \textbf{13} \textit{Majuskel} T  \textbf{14} \textit{Majuskel} T  \textbf{15} \textit{Majuskel} T  \textbf{19} \textit{Majuskel} T  \textbf{23} \textit{Majuskel} T  \newline
\line(1,0){75} \newline
\textbf{1} wol swelch] welch W \textbf{2} si] sy do W \textbf{3} diz] dise U das W  $\cdot$ siht] siecht U \textbf{4} mit] [m*]: mir mit V  $\cdot$ wârheite] arbait W \textbf{5} künne] kan U [kv́n*]: kv́nde V \textbf{6} als] \textit{om.} W \textbf{7} Belacane] Belacâne T belakane V W \textbf{8} missewende] missewenden V \textbf{10} dô] Di V \textbf{11} Herzeloyden] Herzeleiden U herzelauden V hertzeloyden W  $\cdot$ tuon] troͮm V (W) \textbf{12} siufzebæren herzen] seúfftzeberes hetze W  $\cdot$ ruon] roͮm V roum W \textbf{13} Gynovern] [*]: Gynoueren V schinouern W \textbf{14} Ithers] Jthers T U ẏterns V ythers W \textbf{16} \textit{Vers 337.16 fehlt} U   $\cdot$ Daz also schemelichen reit V (W) \textbf{18} Jeschute] Jescvte T (U) iescute V iestute W  $\cdot$ kiusche] wol W \textbf{19} Cunneware] kvnneware T V (W) Kuͦnneware U \textbf{20} geroufet] Gekauͦfet U \textbf{21} vaste] baide vast W \textbf{23} nem] nein U nam W  $\cdot$ diz] dise U \textbf{24} erkennen] [*]: prvͤfen V erkiesen W \textbf{25} schône] reine U [*]: rime V reine wol W  $\cdot$ künne] [kvn*]: kvnde V kan W \textbf{26} beidiu] beide T \textbf{27} tætez iu gerne] dede iz gehe vch U [*]: tete ez v́ch gerne V thet es eúch gerne W \textbf{29} vüeze] fúre W  $\cdot$ tragent] tragen W \textbf{30} die] \textit{om.} W \newline
\end{minipage}
\end{table}
\end{document}
