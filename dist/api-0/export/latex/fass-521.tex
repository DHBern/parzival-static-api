\documentclass[8pt,a4paper,notitlepage]{article}
\usepackage{fullpage}
\usepackage{ulem}
\usepackage{xltxtra}
\usepackage{datetime}
\renewcommand{\dateseparator}{.}
\dmyyyydate
\usepackage{fancyhdr}
\usepackage{ifthen}
\pagestyle{fancy}
\fancyhf{}
\renewcommand{\headrulewidth}{0pt}
\fancyfoot[L]{\ifthenelse{\value{page}=1}{\today, \currenttime{} Uhr}{}}
\begin{document}
\begin{table}[ht]
\begin{minipage}[t]{0.5\linewidth}
\small
\begin{center}*D
\end{center}
\begin{tabular}{rl}
\textbf{521} & \textit{\begin{large}I\end{large}}ch bin noch ledec \textbf{vor} solhem pîn.\\ 
 & welt ab ir unt diu vrouwe mîn\\ 
 & mir smæhe rede bieten,\\ 
 & ir müezet iuch eine nieten,\\ 
5 & daz ir wol meget vür \textbf{zürnen} hân.\\ 
 & swie \textbf{vreissam} ir sît getân,\\ 
 & ich enbære doch sanfte iwerre  drô."\\ 
 & Gawan in bîme hâre dô\\ 
 & \textbf{begreif} unt swang in underz pfert.\\ 
10 & der knappe wîs und wert\\ 
 & vorhtlîch wider sach.\\ 
 & sîn igelmæzec \textbf{hâr} sich rach;\\ 
 & daz versneit \textbf{Gawane sô die} hant,\\ 
 & \textbf{diu} wart \textbf{von bluote alrôt} \textbf{erkant}.\\ 
15 & Des lachte diu vrouwe;\\ 
 & si sprach: "vil gerne ich schouwe\\ 
 & iuch zwêne sus mit zornes site."\\ 
 & si kêrten \textbf{dan}, daz pfert lief mite.\\ 
 & \textbf{Si} kômen, dâ si vunden\\ 
20 & ligen den rîter wunden.\\ 
 & mit triwen Gawans hant\\ 
 & die würze ûf die wunden bant.\\ 
 & der wunde sprach: "wie ergie ez dir,\\ 
 & sît daz dû schiede \textbf{hie} von mir?\\ 
25 & dû hâst eine vrouwen brâht,\\ 
 & diu \textbf{dînes schaden} \textbf{hât} gedâht.\\ 
 & von \textbf{ir schulden} ist mir sô wê;\\ 
 & \textit{in} \textbf{Una Stroyt Viê}\\ 
 & half si mir scherpfer tjoste\\ 
30 & \textbf{ûf}\textit{\textbf{s}} \textbf{lîbes und guotes} koste.\\ 
\end{tabular}
\scriptsize
\line(1,0){75} \newline
D \newline
\line(1,0){75} \newline
\textbf{1} \textit{Initiale} D  \textbf{15} \textit{Majuskel} D  \textbf{19} \textit{Majuskel} D  \newline
\line(1,0){75} \newline
\textbf{1} Ich] ÷ch D \textbf{28} in] \textit{om.} D \textbf{30} ûfs] v̂fz D \newline
\end{minipage}
\hspace{0.5cm}
\begin{minipage}[t]{0.5\linewidth}
\small
\begin{center}*m
\end{center}
\begin{tabular}{rl}
 & ich bin noch ledic \textbf{mit} solher pîn.\\ 
 & welt aber i\textit{r} \textit{u}nd diu vrouwe mîn\\ 
 & mir smæhe rede bieten,\\ 
 & ir müezet iuch eine nieten,\\ 
5 & daz ir wol muget vür \textbf{zor\textit{n}} \textit{h}ân.\\ 
 & wie \textbf{vreislîch} ir sît getân,\\ 
 & ich enbære doch sanfte iuwer drô."\\ 
 & Gawan in bî dem hâr dô\\ 
 & \textbf{begreif} und swanc \textit{in} under daz pfert.\\ 
10 & der knappe wîse und wert\\ 
 & vorhtlîchen wider sach.\\ 
 & sîn igelmæzic \textbf{hâr} sich rach;\\ 
 & daz versneit \textbf{sô Gawanes} hant,\\ 
 & \textbf{daz si} wart \textbf{rô\textit{t} von bluote} \textbf{erkant}.\\ 
15 & des lachete diu vrouwe;\\ 
 & si sprach: "vil gerne ich schouwe\\ 
 & iuch zwêne sus mit zornes site."\\ 
 & si kêrten - daz pfert lief mite -\\ 
 & \textbf{und} kômen, d\textit{â} si vunden\\ 
20 & ligen den ritter wunden.\\ 
 & mit triuwen Gawanes hant\\ 
 & die würz ûf die wunde bant.\\ 
 & der wunde sprach: "wie ergienc \textit{ez} dir,\\ 
 & sît daz dû schiede von mir?\\ 
25 & dû hâst ein vrouwen brâht,\\ 
 & diu \textbf{dînes schaden} gedâht.\\ 
 & von \textbf{ir schulde} ist mir sô wê;\\ 
 & in \textbf{A\textit{v}estroit Ma\textit{v}oie}\\ 
 & half si mir scharfer joste\\ 
30 & \textbf{ûf} \textbf{lîbes und guotes} koste.\\ 
\end{tabular}
\scriptsize
\line(1,0){75} \newline
m n o \newline
\line(1,0){75} \newline
\newline
\line(1,0){75} \newline
\textbf{1} mit] vor n o \textbf{2} aber ir und] aber ẏr mit ir vnd m ir aber vnd n \textbf{4} müezet] muͯst m (o) muͯsten n \textbf{5} muget] megen o  $\cdot$ zorn hân] zorn sin vnd han m zúrnen han n (o) \textbf{6} Ane [fri*tlich]: fristlich ir suz getan o \textbf{9} in] \textit{om.} m  $\cdot$ under] wider o \textbf{14} rôt] rote m \textbf{18} pfert lief] pferff o \textbf{19} dâ] do m n o \textbf{23} ez] \textit{om.} m \textbf{24} von] hie von n >hie< von o \textbf{25} vrouwen] frouwe m n (o) \textbf{28} Jn anestroit manoie m  $\cdot$ Jn ane stroit [man*]: manoye n  $\cdot$ Jn anestroit manomine o \textbf{29} scharfer] scharffen o \newline
\end{minipage}
\end{table}
\newpage
\begin{table}[ht]
\begin{minipage}[t]{0.5\linewidth}
\small
\begin{center}*G
\end{center}
\begin{tabular}{rl}
 & \begin{large}I\end{large}ch bin noch ledic \textbf{vor} solhem pîn.\\ 
 & welt aber ir unde diu vrouwe mîn\\ 
 & mir smæhe rede bieten,\\ 
 & ir müezet iuch eine nieten,\\ 
5 & daz ir wol müget vür \textbf{zorn} hân.\\ 
 & swie \textbf{vreislîche} ir sît getân,\\ 
 & ich\textbf{ne} enbær doch samfte iuwer drô."\\ 
 & Gawan in bî dem hâre dô\\ 
 & \textbf{begreif} unde swanc in underz pfert.\\ 
10 & der knappe wîse unde wert\\ 
 & vorhtlîche wider sach.\\ 
 & sîn igelmæzic \textbf{hâr} sich rach;\\ 
 & daz versneit \textbf{Gawan alsô die} hant,\\ 
 & \textbf{diu} wart \textbf{von bluote al rôt} \textbf{erkant}.\\ 
15 & des lachete diu vrouwe;\\ 
 & si sprach: "vil gerne ich schouwe\\ 
 & iuch zwêne sus mit zornes site."\\ 
 & si kêrten \textbf{dan}, daz pfert lief mite.\\ 
 & \textbf{si} kômen, dâ si vunden\\ 
20 & ligen den rîter wunden.\\ 
 & mit triuwen Gawans hant\\ 
 & die würz ûf die wunden bant.\\ 
 & der wunde sprach: "wie ergienc ez dir,\\ 
 & sît \textit{daz} dû schiede \textbf{hie} von mir?\\ 
25 & dû hâst ein vrouwen brâht,\\ 
 & diu \textbf{dînes schaden} \textbf{hât} gedâht.\\ 
 & von \textbf{ir schulden} ist mir sô wê;\\ 
 & in \textbf{Avestroit Mavoie}\\ 
 & half si mir sc\textit{harf}er tjoste\\ 
30 & \textbf{ûf} \textbf{lîbes unde ûf guotes} koste.\\ 
\end{tabular}
\scriptsize
\line(1,0){75} \newline
G I L M Z Fr23 Fr62 \newline
\line(1,0){75} \newline
\textbf{1} \textit{Initiale} G L Z Fr62  \textbf{19} \textit{Initiale} I  \newline
\line(1,0){75} \newline
\textbf{1} vor solhem] vor solhen L von Sulchē M von solhen Fr23 von sulhem Fr62 \textbf{2} diu] di Fr62 \textbf{3} mir smæhe rede] mir smachrede I Mit smaher rede M Mir solhe rede Z \textbf{4} ir müezet] so muztir Fr62  $\cdot$ eine] arnen M \textbf{5} zorn] zvrnen L (Z) (Fr23) (Fr62) \textbf{6} swie] Wie L Wo M  $\cdot$ vreislîche] freisam M \textbf{7} ichne] Jch L M Z Fr23 (Fr62)  $\cdot$ enbær] ber I einbare L \textbf{9} begreif] begraif in I \textbf{10} der knappe] [Gawan]: der chnappe I \textbf{12} sich] in Fr23 \textbf{13} Gawan] Gawane L (M) (Fr62) gawanen Z  $\cdot$ alsô] al I so L M Z Fr23 Fr62 \textbf{14} diu wart] Daz sý L (Fr62)  $\cdot$ al] \textit{om.} I wart L Fr62  $\cdot$ erkant] bechant Fr23 \textbf{15} lachete] lachet I (Fr23) erdachte M \textbf{17} zornes site] zornen siten I zorne site Z zors site Fr62 \textbf{18} kêrten dan] kert I karte dan M  $\cdot$ lief] lief ir I \textbf{19} si kômen] Sie kerten L Vnde quamen M (Fr62) \textbf{20} ligen] Legen M \textbf{21} Gawans] Gawansz L gawanes M Z \textbf{22} würz] wurzen I \textbf{24} daz] \textit{om.} G  $\cdot$ hie] hin I \textbf{26} diu] di Fr62 \textbf{27} schulden] schult Fr62  $\cdot$ sô] \textit{om.} M \textbf{28} Avestroit] Auesteroit I evestroit M auenstroit Fr62  $\cdot$ Mavoie] ma:::e G mavie L Mavoe M mauoie Fr62 \textbf{29} \textit{Die Verse 521.29-30 fehlen} M   $\cdot$ half] hals I  $\cdot$ scharfer] sc:::er G starch I \textbf{30} unde ûf] vnd Z (Fr62) \newline
\end{minipage}
\hspace{0.5cm}
\begin{minipage}[t]{0.5\linewidth}
\small
\begin{center}*T
\end{center}
\begin{tabular}{rl}
 & "Ich bin noch ledic \textbf{vor} solhem pîn.\\ 
 & welt aber ir unde diu vrouwe mîn\\ 
 & mir smæhe rede bieten,\\ 
 & ir müezet iuch eine nieten,\\ 
5 & daz ir wol muget vür \textbf{zürnen} hân.\\ 
 & swie \textbf{\textit{en}geslîch\textit{e}} ir sît getân,\\ 
 & ich enbære doch sanfte iuwerre drô."\\ 
 & Gawan in bî dem hâre dô\\ 
 & \textbf{nam} unde swanc in underz pfert.\\ 
10 & Der knappe wîse unde wert\\ 
 & vorhtlîche wider sach.\\ 
 & sîn igelmæzic \textbf{hût} sich rach;\\ 
 & daz versneit \textbf{Gawane dô die} hant,\\ 
 & \textbf{daz si} wart \textbf{von bluote alrôt} \textbf{bekant}.\\ 
15 & Des lachete \textbf{dô} diu vrouwe;\\ 
 & si sprach: "vil gerne ich schouwe\\ 
 & iuch zwêne sus mit zornes site."\\ 
 & Si kêrten \textbf{dan} - daz pfert lief mite -\\ 
 & \textbf{unde} kômen, dâ si vunden\\ 
20 & ligen den rîter wunden.\\ 
 & \textit{\begin{large}M\end{large}}it triuwen Gawanes hant\\ 
 & die würz ûf die wunden bant.\\ 
 & Der wunde sprach: "wie ergieng\textit{e}z dir,\\ 
 & sît daz dû schiede \textbf{hie} von mir?\\ 
25 & dû hâst eine vrouwen brâht,\\ 
 & diu \textbf{dîner schanden} \textbf{hât} gedâht.\\ 
 & von \textbf{der schult} ist mir sô wê;\\ 
 & In \textbf{Avestroyt Mavoie}\\ 
 & half si mir scharpfer tjoste\\ 
30 & \textbf{gegen} \textbf{mînes} \textbf{verhes} koste.\\ 
\end{tabular}
\scriptsize
\line(1,0){75} \newline
T U V W O Q R \newline
\line(1,0){75} \newline
\textbf{1} \textit{Initiale} O Q   $\cdot$ \textit{Majuskel} T  \textbf{10} \textit{Majuskel} T  \textbf{15} \textit{Überschrift:} Hie nam der wunde ritter gawan sein roß mit valschait W   $\cdot$ \textit{Initiale} W   $\cdot$ \textit{Majuskel} T  \textbf{18} \textit{Majuskel} T  \textbf{21} \textit{Initiale} T U  \textbf{23} \textit{Majuskel} T  \textbf{28} \textit{Majuskel} T  \newline
\line(1,0){75} \newline
\textbf{1} Ich] ÷ch O  $\cdot$ vor] von Q  $\cdot$ solhem] soliche U solcher W (Q) soͯmher R \textbf{2} aber ir] ir aber O  $\cdot$ diu] \textit{om.} W \textbf{4} iuch] îv T  $\cdot$ eine] eines U Q einig R \textbf{5} wol] \textit{om.} O  $\cdot$ vür] eúwer W  $\cdot$ zürnen] zorn U (Q) (R)  $\cdot$ hân] lan W \textbf{6} swie] Wie U W (Q) R  $\cdot$ engeslîche] iegeslich T [fr*]: freislich V eischliche W (O) itzliche Q freislich R  $\cdot$ ir sît] seit wol Q \textbf{7} enbære] enbere T (O) [*]: enbere  V bere Q  $\cdot$ sanfte] senffter W  $\cdot$ iuwerre drô] vwer do U \textbf{8} Gawan] Gawin R \textbf{11} vorhtlîche] Frolichen Q \textbf{12} hût] har U (V) (W) O Q R  $\cdot$ rach] stach O \textbf{13} versneit] versert R  $\cdot$ Gawane] gawan W (O) Q Gawin R  $\cdot$ dô] so U V O Q  $\cdot$ die] sein W \textbf{14} daz si] Die W (O) Q (R)  $\cdot$ von] im von W  $\cdot$ alrôt] do R  $\cdot$ bekant] erkant V (O) \textbf{15} lachete] lachet Q  $\cdot$ dô] \textit{om.} R \textbf{16} vil] wie R \textbf{17} iuch] iv T  $\cdot$ sus] als Q \textbf{18} pfert] roß W \textbf{19} dâ] do U V W Q R \textbf{20} den] der Q \textbf{21} Mit] ÷it T  $\cdot$ Gawanes] gawans V W (O) Q Gawines R \textbf{23} wie ergiengez] wier giengenz T wie gieng es W \textbf{24} schiede hie] hie schiedest W \textbf{26} dîner] dines U (W) (Q) (R) dir V [die]: dines O  $\cdot$ schanden] schaden U (V) O R schaidens W schades Q \textbf{27} von] Vor Q \textbf{28} Jn avestroit mavoie U  $\cdot$ Jn auestroit mavoie V  $\cdot$ In auestroit mauoie W (R)  $\cdot$ Jnavestroyt mavôie O  $\cdot$ Jn avestroyt maroye Q \textbf{29} half] Schvͦf O  $\cdot$ scharpfer] starcker W starche O \textbf{30} verhes] verheis R \newline
\end{minipage}
\end{table}
\end{document}
