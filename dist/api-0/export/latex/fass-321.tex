\documentclass[8pt,a4paper,notitlepage]{article}
\usepackage{fullpage}
\usepackage{ulem}
\usepackage{xltxtra}
\usepackage{datetime}
\renewcommand{\dateseparator}{.}
\dmyyyydate
\usepackage{fancyhdr}
\usepackage{ifthen}
\pagestyle{fancy}
\fancyhf{}
\renewcommand{\headrulewidth}{0pt}
\fancyfoot[L]{\ifthenelse{\value{page}=1}{\today, \currenttime{} Uhr}{}}
\begin{document}
\begin{table}[ht]
\begin{minipage}[t]{0.5\linewidth}
\small
\begin{center}*D
\end{center}
\begin{tabular}{rl}
\textbf{321} & \begin{large}I\end{large}ch sol doch nennen, wer \textbf{der} sî.\\ 
 & ach, ich arman, unt owî,\\ 
 & daz er mîn herze \textbf{ie sus} versneit!\\ 
 & mîn \textbf{jâmer} \textbf{ist von im ze} breit.\\ 
5 & daz ist \textbf{hie} hêr Gawan,\\ 
 & der dicke prîs hât getân\\ 
 & unt hôhe werdecheit bezalt.\\ 
 & unprîs sîn het \textbf{aldâ} gewalt,\\ 
 & dô in sîn gir dar zuo \textbf{vertruoc}:\\ 
10 & i\textbf{me} gruoze er \textbf{mînen} hêrren sluoc.\\ 
 & ein kus, den Judas teilte,\\ 
 & im \textbf{solhen} willen veilte.\\ 
 & Ez tuot manec tûsent herzen wê,\\ 
 & daz strenge mortlîche rê\\ 
15 & an mîme hêrren ist getân.\\ 
 & lougent des hêr Gawan,\\ 
 & \textbf{des} antwurte ûf kampfes slac\\ 
 & von hiute über den vierzegisten tac\\ 
 & vor dem künege von Ascalun\\ 
20 & in der \textbf{hôhen stat} ze Schanpfanzun.\\ 
 & ich lade in kampflîche dar,\\ 
 & \textbf{gein mir ze komen} \textbf{in kampfes var}.\\ 
 & kan sîn lîp des niht verzagen,\\ 
 & ern welle dâ schildes ambet tragen,\\ 
25 & sô man ich in dennoch mêre\\ 
 & bî des \textbf{helmes} êre\\ 
 & unt durch \textbf{ritter ordenlîchez} leben.\\ 
 & dem sint zwô rîche urbor gegeben:\\ 
 & rehtiu scham unt \textbf{werdiu} triwe\\ 
30 & gebent prîs alt unt niwe.\\ 
\end{tabular}
\scriptsize
\line(1,0){75} \newline
D \newline
\line(1,0){75} \newline
\textbf{1} \textit{Initiale} D  \textbf{13} \textit{Majuskel} D  \newline
\line(1,0){75} \newline
\textbf{11} Judas] ivdas D \textbf{20} Schanpfanzun] Scanpfanzvn D \newline
\end{minipage}
\hspace{0.5cm}
\begin{minipage}[t]{0.5\linewidth}
\small
\begin{center}*m
\end{center}
\begin{tabular}{rl}
 & ich sol doch nennen, wer \textbf{er} sî.\\ 
 & ach, ich armman, und ouwî,\\ 
 & daz er mîn herze \textbf{alsus} versneit!\\ 
 & mîn \textbf{jâmer} \textbf{ist von ime sô} breit.\\ 
5 & daz ist hêr Gawan,\\ 
 & der dicke prîs hât getân\\ 
 & und hôhe werdicheit bezalt.\\ 
 & unprîs sîn hete gewalt,\\ 
 & dô in sîn gir dar zuo \textbf{vertruoc}:\\ 
10 & in\textbf{me} gruoze er \textbf{mir den} hêrren sluoc.\\ 
 & ein kus, den Judas teilte,\\ 
 & im \textbf{solhen} willen veilte.\\ 
 & ez tuot manic tûsent herzen wê,\\ 
 & daz strenge mortlîche rê,\\ 
15 & \textbf{daz} an mînem hêrren ist getân.\\ 
 & lougent des hêr Gawan,\\ 
 & \textbf{des} antwurte ûf kampfes slac\\ 
 & von hiute über den vierzigisten tac\\ 
 & vor dem künige von Ascalun\\ 
20 & in der \textbf{houbetstat} ze Schanfanzun.\\ 
 & ich lade in kampflîche dar,\\ 
 & \textbf{gegen mir ze komenne} \textbf{kampf\textit{va}r}.\\ 
 & kan sîn lîp des niht verzagen,\\ 
 & er enwelle d\textit{â} schiltes ambet tragen,\\ 
25 & sô man ich \textit{in} dann\textit{o}ch mêre\\ 
 & bî des \textbf{helmes} êre\\ 
 & und durch \textbf{ritter ordenlîchez} leben.\\ 
 & dem sint zwô rîche urbor \textit{geb}en:\\ 
 & rehtiu schame und \textbf{werdiu} triuwe\\ 
30 & gebent prîs al\textit{t} und niuwe.\\ 
\end{tabular}
\scriptsize
\line(1,0){75} \newline
m n o \newline
\line(1,0){75} \newline
\newline
\line(1,0){75} \newline
\textbf{2} armman] armer o \textbf{3} daz] Des m  $\cdot$ alsus] ẏe susz n (o) \textbf{5} Gawan] gewan o \textbf{7} hôhe] hoͯhen o \textbf{8} gewalt] aldo gewalt n (o) \textbf{11} Judas] juͯdas o \textbf{12} solhen] selben n o  $\cdot$ veilte] vielte o \textbf{13} manic] manigem n manigen o \textbf{14} strenge] trenge o \textbf{16} des] das n o  $\cdot$ Gawan] gewan o \textbf{17} antwurte] antwuͯrt o \textbf{18} vierzigisten] xiiij n o \textbf{19} vor] Von n  $\cdot$ künige] konigin o  $\cdot$ Ascalun] ascaluͯn m ascelun n ascalún o \textbf{20} in] Zuͦ o  $\cdot$ Schanfanzun] Scanfanzún m scanpazun n tampacẏm o \textbf{22} kampfvar] campfer m kampfar n kamfar o \textbf{23} des niht verzagen] dasz nit verzagent o \textbf{24} dâ] do m n o \textbf{25} in dannoch] dannach m \textbf{28} geben] worden m \textbf{30} alt] alte m \newline
\end{minipage}
\end{table}
\newpage
\begin{table}[ht]
\begin{minipage}[t]{0.5\linewidth}
\small
\begin{center}*G
\end{center}
\begin{tabular}{rl}
 & ich sol doch nennen, wer \textbf{er} sî.\\ 
 & \textit{ach}, ich arman, unde owî,\\ 
 & daz er mîn herze \textbf{ie sus} versneit!\\ 
 & mîn \textbf{riuwe} \textbf{ist von im a\textit{l}ze} breit.\\ 
5 & \textit{da}z ist \textbf{mîn} hêr Gawan,\\ 
 & der dicke brîs hât getân\\ 
 & \begin{large}U\end{large}nde hôhe werdicheit bezalt.\\ 
 & unbrîs sîn hete \textbf{aldâ} gewalt,\\ 
 & dô in sîn gir dar zuo \textbf{vertruoc}:\\ 
10 & i\textbf{me} gruozer \textbf{mînen} hêrren sluoc.\\ 
 & ein kus, den Judas teilte,\\ 
 & im \textbf{solhen} willen veilte.\\ 
 & ez t\textit{uo}t manic tûsent herzen wê,\\ 
 & daz strenge mortlîche rê\\ 
15 & \textit{a}n mînem hêrren ist getân.\\ 
 & lougent des hêr Gawan,\\ 
 & \textbf{sô} antwurte ûf kampfes slac\\ 
 & von hiute über den vierzigeste\textit{n} tac\\ 
 & vor dem künige von Aschalun\\ 
20 & in der \textbf{houbetstat} ze Tschanfenzun.\\ 
 & ich lade in kampflîche dar,\\ 
 & \textbf{gein mir ze komenne} \textbf{kampfes var}.\\ 
 & \begin{large}K\end{large}an sîn lîp des niht verzagen,\\ 
 & er enwelle dâ schiltes ambet tragen,\\ 
25 & sô mane ich in dannoch mêre\\ 
 & bî des \textbf{helmes} êre\\ 
 & unt durch \textbf{rîters ordenlîchez} leben.\\ 
 & dem sint zwei rîchiu urbor gegeben:\\ 
 & rehtiu schame unde \textbf{rehtiu} triuwe\\ 
30 & gebent prîs alt unde niuwe.\\ 
\end{tabular}
\scriptsize
\line(1,0){75} \newline
G I O L M Q R Z Fr22 Fr39 Fr40 \newline
\line(1,0){75} \newline
\textbf{7} \textit{Initiale} G  \textbf{9} \textit{Initiale} O L M Fr22 Fr39  \textbf{11} \textit{Initiale} I  \textbf{17} \textit{Initiale} Z  \textbf{23} \textit{Initiale} G  \textbf{25} \textit{Initiale} Q Fr40  \textbf{27} \textit{Initiale} I  \newline
\line(1,0){75} \newline
\textbf{1} sol] solt I  $\cdot$ doch nennen] dach Nenne M nennen doch Z  $\cdot$ er] der O Z Fr40 \textbf{2} ach] owe G  $\cdot$ arman] arm O armer man L  $\cdot$ unde] \textit{om.} L  $\cdot$ owî] awe O \textbf{3} er] ir R  $\cdot$ ie sus] alsus M \textbf{4} riuwe] truͯwe L (M) (Fr22) (Fr39) (Fr40) iamer Z  $\cdot$ alze breit] azebreit G worden breit I zuͯ arbeit L (M) (Fr22) zebreit R \textbf{5} daz] ez G  $\cdot$ mîn] hie Z  $\cdot$ hêr Gawan] ergawan M \textbf{6} dicke] dicher O  $\cdot$ brîs] preise Q \textbf{7} hôhe werdicheit] hochwirdikeit M \textbf{8} sîn hete aldâ] het sin da L hette sin alda M (Fr22) sein het do Q aldo sin het R het sin do Fr39  $\cdot$ gewalt] gevalt M \textbf{9} dô] ÷o O Da Z  $\cdot$ sîn] \textit{om.} Q  $\cdot$ gir] ger R  $\cdot$ vertruoc] truck Q (R) \textbf{10} ime] in I (M)  $\cdot$ mînen] sinen M \textbf{11} ein] Einen O Q Eines L (Fr22) Fr39  $\cdot$ kus] kvssis Fr22  $\cdot$ Judas] ivdas G O iudas I Q Z Fr40 iuͯdas L \textbf{12} im] Jn O  $\cdot$ solhen] selben den I selben L (Fr22) Fr39  $\cdot$ veilte] wielte R \textbf{13} ez] Er Z  $\cdot$ tuot] tet G  $\cdot$ manic] mannigen M manigim Fr22  $\cdot$ tûsent] tvsinde Fr22  $\cdot$ herzen] herze M herren Q \textbf{14} daz] e daz Fr40  $\cdot$ strenge] strenger O L (Q) R Z (Fr22) Fr39 (Fr40)  $\cdot$ mortlîche] mortlicher O L M Q (R) Z (Fr22) Fr39 (Fr40)  $\cdot$ rê] ve Q \textbf{15} an] daz an G  $\cdot$ mînem hêrren] mynem herczin M minen herren R \textbf{16} lougent] Lougnet R  $\cdot$ hêr Gawan] irgawan M \textbf{17} sô] Des Z  $\cdot$ antwurte] atwurte Q antwirte Fr39  $\cdot$ ûf] vf des L Fr39 vffem M  $\cdot$ kampfes] kampfer M \textbf{18} über den] an dem I úber R  $\cdot$ vierzigesten] vierzgester G [vierzehenden]: vierzegisten I viertzehenten O virzehnsten L (Fr39) vierzigestes Z  $\cdot$ tac] \textit{om.} Z \textbf{19} von] \textit{om.} R  $\cdot$ Aschalun] Aschalv̂n O [Ascal*]: Ascalvn L ascalun M (R) (Z) (Fr39) asculún Q anscalun Fr40 \textbf{20} houbetstat] hohen stat Z  $\cdot$ ze Tschanfenzun] zeshaphanzuͦn I ze schanfezv̂n O zuͯ tshanfenzvn L zcu schafenzcvn M zu schanpfenzűn Q ze schanfenzun R zv tschanfanzvn Z ze tshanfenzvn Fr39 \textbf{21} kampflîche] camlichen I  $\cdot$ dar] [dar]: dan R \textbf{22} kampfes] schampfes O in kamphes L (Fr39) vff kampfes Q R  $\cdot$ var] plan R \textbf{24} enwelle] welle O R Z  $\cdot$ dâ] d I do Q Fr39 des Z  $\cdot$ schiltes] schilt L Fr39 \textbf{25} sô] Doch Q Do Fr40  $\cdot$ ich] \textit{om.} Fr39  $\cdot$ in] \textit{om.} Z  $\cdot$ dannoch] noch I doch L Fr39 dann Q \textbf{27} rîters] ritter Q Z  $\cdot$ ordenlîchez] ordenlichen R \textbf{28} dem] Den Fr39  $\cdot$ rîchiu] rich R (Fr39)  $\cdot$ urbor gegeben] vrbot geben Q úber geben R \textbf{29} unde rehtiu] vnd warev I (O) vnd werde L (M) (Q) (R) (Z) (Fr40) vnd :::iv Fr39 \textbf{30} gebent] Gebirt O Hoͯchent R \newline
\end{minipage}
\hspace{0.5cm}
\begin{minipage}[t]{0.5\linewidth}
\small
\begin{center}*T
\end{center}
\begin{tabular}{rl}
 & ich sol doch nennen, wer \textbf{der} sî.\\ 
 & ach, ich arm man, unde ouwî,\\ 
 & daz er mîn herze \textbf{ie sus} versneit!\\ 
 & mîn \textbf{riuwe} \textbf{von im ist alze} breit.\\ 
5 & Daz ist \textbf{hie mîn} hêr Gawan,\\ 
 & der dicke prîs hât getân\\ 
 & unde hôhe werdecheit bezalt.\\ 
 & unprîs sîn hete \textbf{aldâ} gewalt,\\ 
 & dô in sîn gir dâ zuo \textbf{getruoc}:\\ 
10 & in gruoze er \textbf{mînen} hêrren sluoc.\\ 
 & ein kus, den Judas teilte,\\ 
 & im \textbf{selben} willen veilte.\\ 
 & ez tuot manec tûsent herzen wê,\\ 
 & daz strenge mortlîche rê\\ 
15 & an mînem hêrren ist getân.\\ 
 & lougent des hêr Gawan,\\ 
 & \textbf{des} antwurte ûf kampfes slac\\ 
 & von hiute über den vierzigesten tac\\ 
 & vorme künege von Ascalun\\ 
20 & in der \textbf{houbetstat} ze Tschamfenzun.\\ 
 & ich ladin kampflîche dar,\\ 
 & \textbf{Ze komene gegen} \textbf{mînes kampfes var}.\\ 
 & kan sîn lîp des niht verzagen,\\ 
 & ern welle dâ schiltes ambet tragen,\\ 
25 & sô man ichn dannoch mêre\\ 
 & bî des \textbf{schiltes} êre\\ 
 & unde durch \textbf{rîterlîches ordens} leben.\\ 
 & dem sint zwei rîchiu urbor geben:\\ 
 & reht\textit{iu} schame unde \textbf{wâr\textit{iu}} triuwe\\ 
30 & gebent prîs alt unde niuwe.\\ 
\end{tabular}
\scriptsize
\line(1,0){75} \newline
T U V W \newline
\line(1,0){75} \newline
\textbf{1} \textit{Initiale} V  \textbf{5} \textit{Initiale} W   $\cdot$ \textit{Majuskel} T  \textbf{22} \textit{Majuskel} T  \newline
\line(1,0){75} \newline
\textbf{1} \textit{Die Verse 320.29-321.2 fehlen} W  \textbf{4} mîn] Von W  $\cdot$ riuwe] ruͦwe U [*]: iamer V  $\cdot$ von im ist] ist von im U W [*]: ist von im V  $\cdot$ alze breit] [*]: zebreit V \textbf{5} hie] \textit{om.} V \textbf{8} unprîs] Vnd pris U [Vn*]: Vnpris V  $\cdot$ sîn hete] hette sein W \textbf{10} in [min]: grvͦze er minen herren slvͦc T  $\cdot$ in] Jn me U (V) (W) \textbf{11} ein] Einen W  $\cdot$ Judas] iudaz V iudas W \textbf{12} im] In dem V  $\cdot$ selben] [*]: solhen V  $\cdot$ willen] wilden W \textbf{13} herzen] herze U \textbf{15} an] [*]: Daz an V Das an W \textbf{17} antwurte] antwúrte W \textbf{19} Ascalun] Ascalv̂n T Aschaluͦn U astalun W \textbf{20} Tschamfenzun] Tscamfenzv̂n T Tsamfenzvn U [*]: Schanfanzvn V schampfenzun W \textbf{22} mînes] mir W  $\cdot$ kampfes] [kamp*]: kampfes T \textbf{23} verzagen] versagen W \textbf{24} ern] Er V  $\cdot$ dâ] do V W \textbf{25} ichn] ich W \textbf{26} schiltes] helmes W \textbf{28} rîchiu urbor] vrbos W \textbf{29} rehtiu] rehte T  $\cdot$ wâriu] ware T rechte W \newline
\end{minipage}
\end{table}
\end{document}
