\documentclass[8pt,a4paper,notitlepage]{article}
\usepackage{fullpage}
\usepackage{ulem}
\usepackage{xltxtra}
\usepackage{datetime}
\renewcommand{\dateseparator}{.}
\dmyyyydate
\usepackage{fancyhdr}
\usepackage{ifthen}
\pagestyle{fancy}
\fancyhf{}
\renewcommand{\headrulewidth}{0pt}
\fancyfoot[L]{\ifthenelse{\value{page}=1}{\today, \currenttime{} Uhr}{}}
\begin{document}
\begin{table}[ht]
\begin{minipage}[t]{0.5\linewidth}
\small
\begin{center}*D
\end{center}
\begin{tabular}{rl}
\textbf{236} & \textit{\begin{large}V\end{large}}orem Grâle kômen lieht,\\ 
 & di\textit{u} wâren \textbf{von} armer koste nieht:\\ 
 & sehs glas \textbf{lanc}, lûter, wolgetân,\\ 
 & dâr inne balsem, der wol bran.\\ 
5 & Dô si kômen \textbf{von} der tür\\ 
 & ze rehter mâze \textbf{alsus} her vür,\\ 
 & mit zühten neic diu künegîn\\ 
 & unt al diu juncvröuwelîn,\\ 
 & die dâ truogen \textbf{balsamvaz}.\\ 
10 & diu küneginne \textbf{valscheite} laz\\ 
 & sazte vür den wirt den Grâl.\\ 
 & \textbf{diz} mære giht, daz Parzival\\ 
 & dicke an si sach unt dâhte,\\ 
 & diu den Grâl dâ brâhte.\\ 
15 & er het \textbf{ouch} ir mantel an.\\ 
 & mit zuht die sibene giengen dan\\ 
 & zuo den ahzehen êrsten.\\ 
 & dô liezen si die hêrsten\\ 
 & zwischen \textbf{sich}, man sagete mir,\\ 
20 & zwelve iewederhalben ir.\\ 
 & Diu maget mit der krône\\ 
 & stuont dâ harte schône.\\ 
 & swaz ritter dô gesezzen was\\ 
 & über al den palas,\\ 
25 & den wâren kamerære\\ 
 & mit guldînen \textbf{becken} swære\\ 
 & ie vieren \textbf{geschaffet einer} dar\\ 
 & unt ein junchêrre wol gevar,\\ 
 & der eine wîze tweheln truoc.\\ 
30 & man sach dâ rîcheit genuoc.\\ 
\end{tabular}
\scriptsize
\line(1,0){75} \newline
D \newline
\line(1,0){75} \newline
\textbf{1} \textit{Initiale} D  \textbf{5} \textit{Majuskel} D  \textbf{21} \textit{Majuskel} D  \newline
\line(1,0){75} \newline
\textbf{1} Vorem] ÷orem \textit{nachträglich korrigiert zu:} Vorem D \textbf{2} diu] die D \newline
\end{minipage}
\hspace{0.5cm}
\begin{minipage}[t]{0.5\linewidth}
\small
\begin{center}*m
\end{center}
\begin{tabular}{rl}
 & vor \textit{dem} Grâle kômen lieht,\\ 
 & diu wâren armer koste niht:\\ 
 & sehs glas lûter, wolgetân,\\ 
 & dâr inne balsam, der wol b\textit{ra}n.\\ 
5 & dô si kômen \textbf{vor} der tür\\ 
 & ze rehter mâze \textbf{als} her vür,\\ 
 & mit züh\textit{t}en neic diu künigîn\\ 
 & und alliu diu juncvröuwelîn,\\ 
 & die dâ truogen \textbf{balsamvaz}.\\ 
10 & diu küniginne \textbf{valscheite} laz\\ 
 & saste vür den wirt de\textit{n} Grâl.\\ 
 & \textbf{diz} mære giht, daz Parcifal\\ 
 & dicke an si sach und dâhte,\\ 
 & di\textit{u} den Grâl dâ brâhte.\\ 
15 & er hete \textbf{ouch} ir mantel an.\\ 
 & mit zuht die sibene giengen dan\\ 
 & zuo den ahzehen êrsten.\\ 
 & dô liezen si die hêrsten\\ 
 & zwischen \textbf{sich}, man sagete mir,\\ 
20 & zwelve ietwederhalben ir.\\ 
 & diu maget mit der krône\\ 
 & stuont d\textit{â} harte schône.\\ 
 & \dag iuwer\dag  ritter dô gesezzen was\\ 
 & über al den palas,\\ 
25 & den wâren kamerære\\ 
 & mit guldînen \textbf{bechern} swære\\ 
 & ie vieren \textbf{geschaffet einer} dar\\ 
 & und ein junchêrre wol gevar,\\ 
 & der eine wîze twehelen truoc.\\ 
30 & man sach d\textit{â} rîcheite genuoc.\\ 
\end{tabular}
\scriptsize
\line(1,0){75} \newline
m n o Fr69 \newline
\line(1,0){75} \newline
\newline
\line(1,0){75} \newline
\textbf{1} vor] [Von]: Vor Fr69  $\cdot$ dem] ẏme m \textbf{2} Dú armer koste waren niecht Fr69 \textbf{3} glas] [gras]: glas o \textbf{4} bran] barn m \textbf{5} der tür] die tor n (o) \textbf{6} als] alsus n o \textbf{7} zühten] zuhttigen m \textbf{9} dâ] do n o \textbf{11} den Grâl] der gral m \textbf{12} diz] Dise n  $\cdot$ giht] git o \textbf{14} diu] Dir m n o  $\cdot$ dâ] do n \textbf{20} ietwederhalben] yetweder haben n (o) \textbf{22} dâ] do m n o \textbf{24} al] allen n o \textbf{25} den] Denne n \textbf{26} guldînen] silbern n sẏlber o \textbf{29} der] Die o  $\cdot$ eine wîze twehelen] ein wisz twehelin n \textbf{30} dâ] do m n o \newline
\end{minipage}
\end{table}
\newpage
\begin{table}[ht]
\begin{minipage}[t]{0.5\linewidth}
\small
\begin{center}*G
\end{center}
\begin{tabular}{rl}
 & vor dem Grâle kômen lieht,\\ 
 & diu wâren \textbf{von} armer koste niht:\\ 
 & sehs glas lûte\textit{r}, \textit{w}olgetân,\\ 
 & dâr inne balsem, der wol bran.\\ 
5 & dô si kômen \textbf{von} der tür\\ 
 & ze rehter mâze her vür,\\ 
 & mit zühten neic diu künigîn\\ 
 & unde al diu juncvröuwelîn,\\ 
 & die dâ truogen \textbf{balsamvaz}.\\ 
10 & diu küniginne \textbf{valsches} laz\\ 
 & sazte vür den wirt den Grâl.\\ 
 & \textbf{daz} mære giht, daz Parzival\\ 
 & dicke an si sach unde dâhte,\\ 
 & diu den Grâl dâ brâhte.\\ 
15 & \textit{\begin{large}E\end{large}}r het \textbf{ouch} ir mandel an.\\ 
 & mit zuht die s\textit{i}ben giengen dan\\ 
 & zuo den ahzehen êrsten.\\ 
 & dô liezen si die hêrsten\\ 
 & zwischen \textbf{in}, man sagte mir,\\ 
20 & zwelve ietw\textit{e}derhalben ir.\\ 
 & diu maget mit der krône\\ 
 & stuont dâ harte schône.\\ 
 & swaz rîter dâ gesezzen was\\ 
 & über al den palas,\\ 
25 & den wâren kamerære\\ 
 & mit guldînen \textbf{becken} swære\\ 
 & ie vieren \textbf{\textit{einer} geschaffet} dar\\ 
 & unde ein junchêrre wolgevar,\\ 
 & der eine wîze tweheln truoc.\\ 
30 & man sach dâ rîcheit genuoc.\\ 
\end{tabular}
\scriptsize
\line(1,0){75} \newline
G I O L M Q R Z Fr51 \newline
\line(1,0){75} \newline
\textbf{1} \textit{Initiale} L Q Z  \textbf{5} \textit{Initiale} I  \textbf{15} \textit{Initiale} G  \textbf{21} \textit{Initiale} I O  \newline
\line(1,0){75} \newline
\textbf{1} dem] der M den Fr51  $\cdot$ lieht] lýcht L (M) (Q) \textbf{2} von] \textit{om.} L \textbf{3} lûter] lvter vnde G lang luͯter L (M) (Q) (Z) (Fr51) lutter lang R \textbf{4} balsem der wol] balsamus der wol M balsine Fr51 \textbf{5} dô] Da M Z  $\cdot$ von der] fur die Q (Fr51) \textbf{6} mâze] maze svs O (L) (M) (Q) (R) (Z)  $\cdot$ her vür] irvor M \textbf{7} neic] neigt Q \textbf{8} diu] \textit{om.} R \textbf{9} dâ] do Q  $\cdot$ balsamvaz] balsham glas I balsem var M \textbf{10} laz] \textit{om.} O lar M lat Fr51 \textbf{11} wirt] kung R (Z) \textbf{12} daz] dizze I (L) (Q) (R)  $\cdot$ mære giht daz] mere seyte das M merkede Fr51  $\cdot$ Parzival] [parzifal]: Parzifal I Barcifal O parcifal L Z partzifal Q parczifal R parzẏual Fr51 \textbf{13} Sach an sich vnde dahte O  $\cdot$ Dicke an sie dachte M  $\cdot$ Dicke sach an sie vnd dachte Q  $\cdot$ Dicke her an sie ge dachte Fr51 \textbf{14} dâ] \textit{om.} Q R \textbf{15} Er] Der G \textbf{16} mit] Miz Fr51  $\cdot$ zuht] zuhten I (M) (R)  $\cdot$ siben] selben G I Fr51 \textbf{17} zuo] zwishen I  $\cdot$ ahzehen] ahzehenen den I ahtzehenden L achten Fr51 \textbf{18} dô] Da Z \textbf{19} in] sie L (M) (Q) Fr51 sich Z  $\cdot$ sagte] seit I (O) \textbf{20} zwelve] Welhe R  $\cdot$ ietwederhalben] ietwerder halben G iwerderhalben O ir icwedir halben M  $\cdot$ ir] er Fr51 \textbf{21} diu] ÷iv O \textbf{22} dâ] do O Q \textbf{23} swaz] Waz L (M) (Q) (R) Do waz Fr51  $\cdot$ dâ] do Q \textbf{24} al den] an den I aldin M \textbf{25} den] Der Q \textbf{27} ie vieren] ie vier I Jr vieren O Jvieren Q [Jeviere]: Je viere Fr51  $\cdot$ einer geschaffet] geschaft G (I) geschaffet einer L (M) Z eyn ge scaffen Fr51 \textbf{28} ein junchêrre] einen iuncherren I ein jvncfrauͯwe L (Fr51)  $\cdot$ wolgevar] wol givarn M \textbf{29} der] Die Fr51  $\cdot$ wîze] wisz sýdin L  $\cdot$ tweheln] [thehel]: thvehel I \textbf{30} dâ rîcheit] do richeit O (Q) richeit da M \newline
\end{minipage}
\hspace{0.5cm}
\begin{minipage}[t]{0.5\linewidth}
\small
\begin{center}*T
\end{center}
\begin{tabular}{rl}
 & \begin{large}V\end{large}or dem Grâle kômen lieht,\\ 
 & di\textit{u} wâren \textbf{von} armer koste nieht:\\ 
 & sehs glas \textbf{lanc}, lûter, wol getân,\\ 
 & dâr inne balsem, der wol bran.\\ 
5 & Dô si kômen \textbf{von} der tür\\ 
 & ze rehter mâze \textbf{alsus} her vür,\\ 
 & mit zühten neic diu künegîn\\ 
 & unde al diu juncvröuwelîn,\\ 
 & die dâ truogen \textbf{balsemen vaz}.\\ 
10 & diu künegîn \textbf{valscheite} laz\\ 
 & sazte vür den wirt den Grâl.\\ 
 & \textbf{Daz} mære giht, daz Parcifal\\ 
 & dicke an si sach unde dâhte,\\ 
 & diu den Grâl dâ brâhte.\\ 
15 & er hete \textbf{iedoch} ir mantel an.\\ 
 & mit zuht die sibene giengen dan\\ 
 & zuo den ahzehen êrsten.\\ 
 & dô liezen si die hêrsten\\ 
 & zwischen \textbf{sich}. man sagete mir,\\ 
20 & \textbf{man sazte} zwelve ietwederhalben ir.\\ 
 & Diu maget mit der krône\\ 
 & stuont dâ harte schône.\\ 
 & swaz rîter dâ gesezzen was\\ 
 & über al den palas,\\ 
25 & den wâren kamerære\\ 
 & mit guldînen \textbf{bec\textit{k}en} swære\\ 
 & ie vieren \textbf{geschaffet einer} dar\\ 
 & unde ein junchêrre wol gevar,\\ 
 & der eine wîze twehele truoc.\\ 
30 & man sach dâ rîcheite genuoc.\\ 
\end{tabular}
\scriptsize
\line(1,0){75} \newline
T U V W \newline
\line(1,0){75} \newline
\textbf{1} \textit{Initiale} T  \textbf{5} \textit{Initiale} U W   $\cdot$ \textit{Majuskel} T  \textbf{12} \textit{Majuskel} T  \textbf{21} \textit{Majuskel} T  \newline
\line(1,0){75} \newline
\textbf{2} diu] die T  $\cdot$ wâren von] von U [*]: waren V \textbf{3} Ses gla luͦter lanc wol getan U  $\cdot$ lanc lûter] lauter vnd W \textbf{4} balsem] [bl*]: balsem T was balsam W \textbf{6} mâze] aliaze U  $\cdot$ her] hin W \textbf{9} die dâ] [D*]: Die do V Die do W  $\cdot$ balsemen] balseme U (V) \textbf{12} Daz] Die U [D*]: Diz V  $\cdot$ daz Parcifal] daz parzifal T Parzifal U [*]: daz parzefal V das partzifal W \textbf{13} si sach] sich U  $\cdot$ dâhte] gedahte V (W) \textbf{14} dâ] do W \textbf{16} zuht] zúchten W \textbf{18} dô] [D*]: Do V  $\cdot$ liezen] lieffen W \textbf{19} sich] sy W \textbf{20} [*]: Zwelfe ietwederthalben ir V Zwelff ietweder halb ir W \textbf{22} dâ] do U V W \textbf{23} swaz] Waz U (W)  $\cdot$ dâ] do V W \textbf{25} den] [D*]: Den V \textbf{26} becken] beckinen T becheren V \textbf{27} geschaffet] geschaffen V  $\cdot$ einer] eine W \textbf{28} junchêrre] [ivncvrouwe]: ivncherre T \textbf{30} Men [sa*eit]: sach do richeit genoͮg V  $\cdot$ dâ] do W \newline
\end{minipage}
\end{table}
\end{document}
