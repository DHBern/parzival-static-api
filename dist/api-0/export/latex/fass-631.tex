\documentclass[8pt,a4paper,notitlepage]{article}
\usepackage{fullpage}
\usepackage{ulem}
\usepackage{xltxtra}
\usepackage{datetime}
\renewcommand{\dateseparator}{.}
\dmyyyydate
\usepackage{fancyhdr}
\usepackage{ifthen}
\pagestyle{fancy}
\fancyhf{}
\renewcommand{\headrulewidth}{0pt}
\fancyfoot[L]{\ifthenelse{\value{page}=1}{\today, \currenttime{} Uhr}{}}
\begin{document}
\begin{table}[ht]
\begin{minipage}[t]{0.5\linewidth}
\small
\begin{center}*D
\end{center}
\begin{tabular}{rl}
\textbf{631} & \textit{\begin{large}D\end{large}}er wirt niht langer wolde stên;\\ 
 & er bat die zwêne sitzen gên\\ 
 & zuo den vrouwen, swâ si wolden.\\ 
 & \textbf{dô} si \textbf{sô} tuon solden,\\ 
5 & diu bete tet \textbf{in} niht \textbf{ze} wê.\\ 
 & "\textbf{Welhez} ist Itonje?",\\ 
 & \textbf{sus} sprach der werde Gawan.\\ 
 & "diu sol mich bî ir sitzen lân."\\ 
 & des vrâgt er Benen stille,\\ 
10 & sît ez was sîn wille,\\ 
 & \textbf{si zeigete} im die magt clâr.\\ 
 & "diu den rôten munt, daz brûne hâr\\ 
 & dort treit bî liehten ougen,\\ 
 & welt ir \textbf{si} \textbf{sprechen} tougen,\\ 
15 & daz tuot gevuoclîche",\\ 
 & sprach \textbf{vrou} Bene, diu zühte rîche.\\ 
 & Diu wesse Itonje minnen nôt\\ 
 & unt daz ir herze \textbf{dienst} bôt\\ 
 & \textbf{der werde künec} Gramoflanz\\ 
20 & mit \textbf{rîterlîchen triwen} ganz.\\ 
 & Gawan saz nider zuo der magt.\\ 
 & ich sag \textbf{iu}, daz mir wart gesagt:\\ 
 & sîner rede er dâ begunde\\ 
 & mit \textbf{vuogen}, wand erz kunde.\\ 
25 & Ouch kunde si gebâren,\\ 
 & daz von sô kurzen jâren,\\ 
 & als Itonje, diu junge, truoc,\\ 
 & \textbf{den} hete \textbf{si} zühte \textbf{gar} genuoc.\\ 
 & er hete sich \textbf{vrâgen} gein ir bewegen,\\ 
30 & ob si noch minne kunde pflegen.\\ 
\end{tabular}
\scriptsize
\line(1,0){75} \newline
D Z Fr16 Fr63 \newline
\line(1,0){75} \newline
\textbf{1} \textit{Initiale} D Z  \textbf{6} \textit{Majuskel} D  \textbf{17} \textit{Majuskel} D  \textbf{25} \textit{Majuskel} D  \newline
\line(1,0){75} \newline
\textbf{1} Der] ÷er D \textbf{4} dô] Da Z ::: Fr16 \textbf{6} Welhez] Welhe Z ::: Fr16  $\cdot$ Itonje] Jtonîe D Jconie Z ::: Fr16 Jtonie Fr63 \textbf{7} werde] \textit{om.} Z ::: Fr16 \textbf{9} Benen] ::: Fr16 Bene Fr63 \textbf{10} sîn] sine Fr63 \textbf{11} zeigete] zeigt Z ::: Fr16 Fr63 \textbf{17} Itonje] Jtonie D Fr63 Jconie Z \textbf{18} herze] hertzen Z \textbf{19} Gramoflanz] gramoflantz Z Gramolanz Fr63 \textbf{24} vuogen] fvge Z \textbf{25} gebâren] bewaren Fr63 \textbf{27} Itonje] Jtonie D Jconie Z Itonie Fr63 \textbf{29} vrâgen] fragens Z \newline
\end{minipage}
\hspace{0.5cm}
\begin{minipage}[t]{0.5\linewidth}
\small
\begin{center}*m
\end{center}
\begin{tabular}{rl}
 & \begin{large}D\end{large}er wirt niht langer wolte stân;\\ 
 & er bat die zwên sitzen gân\\ 
 & zuo den vrowen, wâ si wolten,\\ 
 & \textbf{daz} si \textbf{dô} tuon solten.\\ 
5 & diu bete tet \textbf{in} ni\textit{ht} \textbf{sô} wê.\\ 
 & "\textbf{wel\textit{h}ez} ist Ithonie?",\\ 
 & sprach der werde Gawan.\\ 
 & "diu sol mich bî ir sitzen lân."\\ 
 & des vrâgte e\textit{r} \textit{B}enen stille,\\ 
10 & sît ez was sîn wille,\\ 
 & \textbf{dô zougte si} im die maget clâr.\\ 
 & "diu den rôten munt, daz brûne hâr\\ 
 & dort trei\textit{t b}î liehten ougen,\\ 
 & wolt ir \textbf{sprechen} tougen,\\ 
15 & daz tuot gevuoclîche",\\ 
 & sprach Bene, diu zühterîche.\\ 
 & diu weste Ithonien minne nôt\\ 
 & und daz ir herze \textbf{dienste} bôt\\ 
 & \textbf{der werde künic} Gram\textit{o}lanz\\ 
20 & mit \textbf{ritterlîchen triuwen} ganz.\\ 
 & Gawan saz nider zuo der magt.\\ 
 & ich sage \textbf{iu}, daz mir wart gesagt:\\ 
 & sîner rede er dô begunde\\ 
 & mit \textbf{vuoge}, wan erz kunde.\\ 
25 & ouch kunde si gebâren,\\ 
 & daz von sô kurzen jâren,\\ 
 & als Ithonie, diu junge, truoc,\\ 
 & \textbf{den} het \textbf{si} zuht genuoc.\\ 
 & er het sich \textbf{vrâge} gegen ir bewegen,\\ 
30 & ob si noch minne kunde pflegen.\\ 
\end{tabular}
\scriptsize
\line(1,0){75} \newline
m n o \newline
\line(1,0){75} \newline
\textbf{1} \textit{Illustration mit Überschrift:} Also der wirt nit langer beittet (>beiten< o  ) er hies dien (die n  ) hern zuͯ den frowen sitzen m (n) (o)   $\cdot$ \textit{Initiale} m n o  \newline
\line(1,0){75} \newline
\textbf{4} dô] \textit{om.} o \textbf{5} niht] nẏe m  $\cdot$ sô] zuͯ n (o) \textbf{6} welhez] Welhies m  $\cdot$ Ithonie] jthonie m jtonẏe n itonie o \textbf{9} er Benen] er stille Benen m \textbf{13} dort] Do n  $\cdot$ treit bî] treit der bẏ m \textbf{15} gevuoclîche] gefugluch o \textbf{17} Ithonien] itonyen n itonie o \textbf{18} dienste] mynne o \textbf{19} Gramolanz] gramon lantz m gramonlantz n grymonlancz o \textbf{26} sô] \textit{om.} n \textbf{27} Ithonie] jtonie m n (o)  $\cdot$ truoc] brug o \textbf{29} vrâge] frowe o  $\cdot$ bewegen] bewegent o \newline
\end{minipage}
\end{table}
\newpage
\begin{table}[ht]
\begin{minipage}[t]{0.5\linewidth}
\small
\begin{center}*G
\end{center}
\begin{tabular}{rl}
 & der wirt niht langer wolde stên;\\ 
 & er bat die zwêne sitzen gên\\ 
 & zuo den vrouwen, swâ si wolden.\\ 
 & \textbf{dâ} si \textbf{sô} tuon solden,\\ 
5 & diu bet tet \textbf{in} niht \textbf{ze} wê.\\ 
 & "\textbf{welhiu} ist Itonie?",\\ 
 & sprach d\textit{er} werde Gawan.\\ 
 & "di\textit{u} sol mich bî ir sitzen lân."\\ 
 & des vrâget er Benen stille,\\ 
10 & sît ez was sîn wille,\\ 
 & \textbf{si zeiget} im die maget clâr.\\ 
 & "diu den rôten munt, daz brûne hâr\\ 
 & dort treit bî liehten ougen,\\ 
 & welt ir \textbf{si} \textbf{gesprechen} tougen,\\ 
15 & daz tuot gevuoclîche",\\ 
 & sprach \textbf{vrô} Bene, diu zühte rîche.\\ 
 & diu west Itonien minne nôt\\ 
 & unde daz ir herze \textbf{dienst} bôt\\ 
 & \textbf{der werde künic} Gramoflanz\\ 
20 & mit \textbf{rîterlîcher triuwe} ganz.\\ 
 & Gawan saz nider zuo der maget.\\ 
 & ich sage \textbf{iu}, daz mir wart gesaget:\\ 
 & sîner rede er dâ begunde\\ 
 & mit \textbf{vuoge}, wan erz kunde.\\ 
25 & ouch kunde si gebâren,\\ 
 & daz von \textit{sô} kurzen jâren,\\ 
 & als Itonie, diu junge, truoc,\\ 
 & \textbf{diu} het zuht \textbf{gar} genuoc.\\ 
 & er het sich \textbf{vrâgens} gein ir bewegen,\\ 
30 & ob si noch minne kunde pflegen.\\ 
\end{tabular}
\scriptsize
\line(1,0){75} \newline
G I L M Z Fr51 \newline
\line(1,0){75} \newline
\textbf{1} \textit{Initiale} L Z Fr51  \textbf{15} \textit{Initiale} I  \textbf{19} \textit{Initiale} M  \newline
\line(1,0){75} \newline
\textbf{1} langer] lanchen Fr51 \textbf{2} bat die] bas Fr51 \textbf{3} den] [der]: den M  $\cdot$ swâ] wo L (M) (Fr51) \textbf{4} dâ] do I (L) (Fr51) \textbf{5} tet] diu tet I  $\cdot$ ze] \textit{om.} Fr51 \textbf{6} welhiu] Wilchir M  $\cdot$ Itonie] ẏtonie G jtonie L Jthonie M Jconie Z eltonie Fr51 \textbf{7} sprach] Sus sprach M Z (Fr51)  $\cdot$ der] diu G (M)  $\cdot$ werde] \textit{om.} Z \textbf{8} diu] die G  $\cdot$ mich] mir Fr51  $\cdot$ bî ir sitzen] sýtzen bi ir L zcu ir siczen M \textbf{9} vrâget er] vragite M  $\cdot$ Benen] froͮn bene I Bene L (M) \textbf{11} zeiget] zceigete M zogete Fr51 \textbf{12} rôten] rot I \textbf{13} dort] Dorch Fr51  $\cdot$ liehten] lichten L (M) \textbf{14} welt ir si] welt irz L (M)  $\cdot$ gesprechen] sprechen Z Fr51 \textbf{16} Sprach de vrowe zvchten riche Fr51  $\cdot$ vrô] \textit{om.} L  $\cdot$ zühte] [zu*]: zuhten I \textbf{17} Itonien] Jtonien L Jthonien M Jconie Z eltonien Fr51  $\cdot$ minne] minnen Fr51 \textbf{18} unde] \textit{om.} L  $\cdot$ ir] iren Fr51  $\cdot$ herze] hertzen L (Fr51)  $\cdot$ dienst] [bot]: dienst G mynne L \textbf{19} der werde] dem werdem I Der werdir M  $\cdot$ Gramoflanz] gramorflanz M gramoflantz Z gramoflans Fr51 \textbf{20} mit] bi I  $\cdot$ rîterlîcher] ernstlicher I ritterliche L Fr51 ritterlichen Z  $\cdot$ triuwe] trewen Z  $\cdot$ ganz] [gra]: ganz G \textbf{22} iu] \textit{om.} L M  $\cdot$ mir] mir da L \textbf{24} wan] als Fr51  $\cdot$ erz] [ez]: erz G er daz I her wol Fr51 \textbf{26} sô] \textit{om.} G \textbf{27} als] niemen als I  $\cdot$ Itonie] Jtonie I (L) Jthonie M Jconie Z eltonie Fr51 \textbf{28} diu het] Den het sie Z  $\cdot$ gar] \textit{om.} L Fr51 \textbf{29} het] \textit{om.} I  $\cdot$ vrâgens] vrage L  $\cdot$ ir] ir gar I \textit{om.} Fr51 \textbf{30} noch] noch noch Fr51  $\cdot$ minne kunde] kvnde mýnne L \newline
\end{minipage}
\hspace{0.5cm}
\begin{minipage}[t]{0.5\linewidth}
\small
\begin{center}*T
\end{center}
\begin{tabular}{rl}
 & \begin{large}D\end{large}er wirt niht langer wolte stân;\\ 
 & er bat die zwêne sitzen gân\\ 
 & zuo den vrouwen, wâ si wolten.\\ 
 & \textbf{dô} si \textit{\textbf{sô}} tuon solten,\\ 
5 & diu bete tet \textbf{im} niht \textbf{zuo} wê.\\ 
 & "\textbf{welhiu} ist Itonie?",\\ 
 & \textbf{sus} sprach der werde Gawan.\\ 
 & "diu sol mich bî ir sitzen lân."\\ 
 & des vrâgeter Benen stille,\\ 
10 & sît ez was sîn wille,\\ 
 & \textbf{si zeiget}im die maget clâr.\\ 
 & "diu den rôten munt, daz brûne hâr\\ 
 & dort traget bî liehten ougen,\\ 
 & wolt ir \textbf{gesprechen} to\textit{ug}en,\\ 
15 & daz tuot gevuoclîche",\\ 
 & sprach \textbf{vrou} Bene, diu zuht rîche.\\ 
 & diu wiste Itonien minnen nôt\\ 
 & und daz ir herze \textbf{dienst} bôt\\ 
 & \textbf{dem werden künege} Gramoflanz\\ 
20 & mit \textbf{rîterlîcher triuwen} ganz.\\ 
 & Gawan saz nider zuo der maget.\\ 
 & ich sage, daz mir wart gesaget:\\ 
 & sîner rede er dô begunde\\ 
 & mit \textbf{gevuoge}, wan er ez kunde.\\ 
25 & ouch kunde si gebâren,\\ 
 & daz von sô kurzen jâren,\\ 
 & als Itonie, diu junge, truoc,\\ 
 & \textbf{dannoch} hete \textbf{si} zuht genuoc.\\ 
 & er hete sich \textbf{vrâgens} gein ir bewegen,\\ 
30 & ob si noch minne kunde pflegen.\\ 
\end{tabular}
\scriptsize
\line(1,0){75} \newline
U V W Q R \newline
\line(1,0){75} \newline
\textbf{1} \textit{Initiale} U W Q  \newline
\line(1,0){75} \newline
\textbf{3} wâ] swa V was R \textbf{4} sô] \textit{om.} U \textbf{5} im] in V Q (R) mir W  $\cdot$ niht zuo] nicht so Q zu R \textbf{6} welhiu] [Welch*]: Welche V Welhe R  $\cdot$ Itonie] Jtone U ẏtonie V ytonie W Q R \textbf{7} sus] Als Q \textbf{8} ir] dir Q \textbf{9} vrâgeter] fragt er W (Q) R \textbf{11} zeigetim] [zoiget *]: zoͤiget im do V tzegt im Q \textbf{13} liehten] lichten Q \textbf{14} ir] ir sv́ V (W) (Q) (R)  $\cdot$ gesprechen] sprechen W besprechen Q  $\cdot$ tougen] tauͦwen U \textbf{16} zuht] zv́hten V (Q) (R) \textbf{17} \textit{Verse 631.17-18 kontrahiert zu:} Du west ytonien minnen bot Q   $\cdot$ diu] Du R  $\cdot$ Itonien] Jtonien U R ytonien W  $\cdot$ minnen] minne V (R) \textbf{19} [D*]: Der werde kv́nig gramaflanz V  $\cdot$ Der werde kúnig gramoflantz W (Q)  $\cdot$ Der werde kúng Gramoflancz R \textbf{20} triuwen] treúwe W (Q) (R) \textbf{21} Gawan] Gawin R \textbf{22} sage] sag v́ch V (W) (Q) (R)  $\cdot$ wart] ist W \textbf{24} gevuoge] fvͦge V (W) (Q) fuͦgen R  $\cdot$ er ez] ers wol W R \textbf{25} kunde] kunden W  $\cdot$ gebâren] bewaren Q \textbf{26} sô] \textit{om.} Q \textbf{27} Itonie] Jtonie U ẏtonie V ytonie W Q R  $\cdot$ junge] maget V \textbf{28} dannoch] Den V W Q R  $\cdot$ zuht] zv́hten V  $\cdot$ genuoc] gar genvͦg V (W) (Q) (R) \textbf{29} sich] sy W (Q) \newline
\end{minipage}
\end{table}
\end{document}
