\documentclass[8pt,a4paper,notitlepage]{article}
\usepackage{fullpage}
\usepackage{ulem}
\usepackage{xltxtra}
\usepackage{datetime}
\renewcommand{\dateseparator}{.}
\dmyyyydate
\usepackage{fancyhdr}
\usepackage{ifthen}
\pagestyle{fancy}
\fancyhf{}
\renewcommand{\headrulewidth}{0pt}
\fancyfoot[L]{\ifthenelse{\value{page}=1}{\today, \currenttime{} Uhr}{}}
\begin{document}
\begin{table}[ht]
\begin{minipage}[t]{0.5\linewidth}
\small
\begin{center}*D
\end{center}
\begin{tabular}{rl}
\textbf{398} & \begin{Large}S\end{Large}wer was ze Bearosche komen,\\ 
 & \textbf{doch} hete Gawan dâ genomen\\ 
 & den prîs ze bêder sît al ein,\\ 
 & wan daz \textbf{dâr vor} ein ritter schein,\\ 
5 & bî \textbf{rôten} wâpen unerkant,\\ 
 & des prîs man in die hœhe bant.\\ 
 & Gawan hete êre und heil,\\ 
 & ieweders volleclîchen teil.\\ 
 & nû nâhet ouch sînes kampfes zît.\\ 
10 & der walt was lanc und wît,\\ 
 & dâ durch er muose strîchen,\\ 
 & wolder kampfes niht entwîchen.\\ 
 & \textbf{âne schulde er was dar zuo} erkorn.\\ 
 & nû was ouch Ingliart verlorn,\\ 
15 & sîn ors mit kurzen ôren.\\ 
 & in Thabronit von môren\\ 
 & wart nie bezzer ors ersprenget.\\ 
 & nû wart der walt gemenget,\\ 
 & hie ein \textbf{schache}, dort ein velt,\\ 
20 & etslîchez sô breit, daz ein gezelt\\ 
 & Vil kûme drûffe stüende.\\ 
 & mit sehen gewan er künde\\ 
 & \textbf{erbûwens landes, daz} hiez Ascalun.\\ 
 & dâ vrâgeter gegen Schanpfanzun,\\ 
25 & swaz im \textbf{dâ} volkes widervuor.\\ 
 & hôch gebirge \textbf{unt} manec muor,\\ 
 & des het er vil durchstrichen dar.\\ 
 & dô nam er einer bürge war.\\ 
 & âvoy, diu gap vil werden glast!\\ 
30 & dâ \textbf{kêrte gegen} des landes gast.\\ 
\end{tabular}
\scriptsize
\line(1,0){75} \newline
D \newline
\line(1,0){75} \newline
\textbf{1} \textit{Großinitiale} D  \textbf{21} \textit{Majuskel} D  \newline
\line(1,0){75} \newline
\textbf{1} Bearosche] Bearosce D \textbf{14} Ingliart] Jngliart D \textbf{21} Vil] [*]: Vil D \textbf{23} Ascalun] Ascalvͦn D \textbf{24} Schanpfanzun] Scanpfanzvͦn D \newline
\end{minipage}
\hspace{0.5cm}
\begin{minipage}[t]{0.5\linewidth}
\small
\begin{center}*m
\end{center}
\begin{tabular}{rl}
 & wer was ze Bearosche komen,\\ 
 & \textbf{dô} hete Gawan d\textit{â} genomen\\ 
 & den brîs ze beider sîte alein,\\ 
 & wanne daz \textbf{dâr vor} ein ritter schein,\\ 
5 & bî \textbf{rôten} wâpen unerkant,\\ 
 & des prîs man in die hœhe bant.\\ 
 & Gawan hete êre und heil,\\ 
 & \begin{large}I\end{large}etweder\textit{s} volleclîchen teil.\\ 
 & nû nâhete ouch sînes kampfes zît.\\ 
10 & der wal\textit{t} was lanc und wît,\\ 
 & dâ durch er muose strîchen,\\ 
 & wolte er kampfes niht entwîchen.\\ 
 & \textbf{dar zuo er âne schulde was} erkorn.\\ 
 & nû was ouch Ingliart verlorn,\\ 
15 & sîn ros mit kurzen ôren.\\ 
 & in Tabronit von môren\\ 
 & wart nie bezzer ros ersprenget.\\ 
 & nû wart der walt gemenget,\\ 
 & hie ein \textbf{schache}, dort ein velt,\\ 
20 & etslîche\textit{z} sô breit, daz ein gezelt\\ 
 & vil kûme drûffe stüende.\\ 
 & mit sehene gewan er künde\\ 
 & \textbf{erbûwens landes, daz} hie Ascalun.\\ 
 & dô vrâgete er gegen Schanfanzun,\\ 
25 & waz ime volkes widervuor.\\ 
 & hôch gebirge \textbf{und} manic muor,\\ 
 & des hete er vil durchstrichen dar.\\ 
 & dô nam er einer bürge war.\\ 
 & \textit{âv}oy, diu gap vil werden glast!\\ 
30 & dâ \textbf{kêrte engegen} des landes \textit{g}ast.\\ 
\end{tabular}
\scriptsize
\line(1,0){75} \newline
m n o \newline
\line(1,0){75} \newline
\textbf{7} \textit{Illustration mit Überschrift:} Gawanes not die er uff schauffan zuͯn leit Vnd wie er dannen der gral suͯchen reit m  Also gawan durch einen walt reit vnd zuͯ einer burg kam vnd jme der wurt engegen kam n (o)   $\cdot$ \textit{Initiale} n o  \textbf{8} \textit{Initiale} m  \newline
\line(1,0){75} \newline
\textbf{1} Bearosche] bearosce m n berasce o \textbf{2} dâ] do m \textit{om.} n o \textbf{5} rôten] rotem n o  $\cdot$ unerkant] vnderkant o \textbf{6} prîs] prises n  $\cdot$ in die] jme do n (o) \textbf{8} ietweders] Iyettweder m  $\cdot$ volleclîchen] glichen n o \textbf{10} walt] wal m \textbf{11} muose] musse m múste n o \textbf{12} kampfes] kampff n \textbf{14} was] wars o  $\cdot$ Ingliart] Jngliart m (o) jugliart n \textbf{20} etslîchez] Eczlicher m \textbf{21} stüende] komme o \textbf{22} gewan] gawan o \textbf{23} daz] \textit{om.} n o  $\cdot$ Ascalun] ascalún m astalim o \textbf{24} Schanfanzun] schanfanzún m schamfanzun n schamfantzim o \textbf{25} volkes] [l]: valkes m wolkes o \textbf{27} hete] hetten o \textbf{29} âvoy] Noi m Anoi n (o) \textbf{30} gast] glast m \newline
\end{minipage}
\end{table}
\newpage
\begin{table}[ht]
\begin{minipage}[t]{0.5\linewidth}
\small
\begin{center}*G
\end{center}
\begin{tabular}{rl}
 & \begin{Large}S\end{Large}wer was ze Bearotsche komen,\\ 
 & \textbf{doch} hete Gawan dâ genomen\\ 
 & den brîs ze bêder sît al ein,\\ 
 & wan daz \textbf{dâr vor} ein rîter schein,\\ 
5 & bî \textbf{rôtem} wâpen unerkant,\\ 
 & des prîs man in die hœhe bant.\\ 
 & Gawan het êre unde heil,\\ 
 & ietweders volleclîchen teil.\\ 
 & nû nâhet ouch sînes kampfes zît.\\ 
10 & der walt was lanc unde wît,\\ 
 & dâ durch er muose strîchen,\\ 
 & wolt er kampfes niht entwîchen.\\ 
 & \textbf{âne schulde er was dar zuo} erkoren.\\ 
 & nû was ouch Inguliart verloren,\\ 
15 & sîn ors mit kurzen ôren.\\ 
 & in Tabrunit von môren\\ 
 & wart nie bezzer ors ersprenget.\\ 
 & nû wart der walt gemenget,\\ 
 & hie ein \textbf{schache}, dort ein velt,\\ 
20 & etslîchez sô breit, daz ein gezelt\\ 
 & vil kûme drûffe stüende.\\ 
 & mit sehenne gewan er künde\\ 
 & \textbf{erbûwenes landes, daz} hiez Aschalun.\\ 
 & dô vrâgter gegen Tschanfenzun,\\ 
25 & swaz im volkes widervuor.\\ 
 & hôch gebirge, manic muor,\\ 
 & des het er vil durchstrichen dar.\\ 
 & dô nam er einer bürge war.\\ 
 & âvoy, diu gap vil werden glast!\\ 
30 & dâ \textbf{engêne kêrte} des landes gast.\\ 
\end{tabular}
\scriptsize
\line(1,0){75} \newline
G I O L M Q R Z \newline
\line(1,0){75} \newline
\textbf{1} \textit{Überschrift:} Aventiwer wie Gawan vor bearotsch Melyanz gesiget an vnd hinz Schachtelmarvail chom I  Hie schiet gawan von bearotsch vnd wil zv sinem kampfe varn waz auentevre im wider fvr vf dem wege wer daz welle wizzen der lese fvrbaz Z   $\cdot$ \textit{Großinitiale} R Z   $\cdot$ \textit{Initiale} G I O L  \newline
\line(1,0){75} \newline
\textbf{1} \textit{Die Verse 370.13-412.12 fehlen} Q   $\cdot$ Swer] ÷wer O Eer L Wer nun R  $\cdot$ ze Bearotsche] zebearotsche G ze bearoche I zebearotsch O ze Arosche R zv bearoth Z \textbf{2} dâ genomen] do gewunnen R \textbf{3} den] Des R  $\cdot$ ze bêder sît] zebeden siten O zu beiden sitte R  $\cdot$ al ein] alsein L \textbf{4} daz] \textit{om.} Z  $\cdot$ ein] ein roter I der ein R  $\cdot$ schein] scheine Z \textbf{5} rôtem] roten I (M)  $\cdot$ unerkant] vnbechant I \textbf{6} des] den I  $\cdot$ man] nam M  $\cdot$ in die] do so R  $\cdot$ bant] [hant]: bant M \textbf{7} het] hat R \textbf{8} teil] \sout{ein} teil Z \textbf{9} ouch sînes] auch sin I des O \textbf{11} muose] muͦs R \textbf{12} niht] \textit{om.} O \textbf{13} âne] \sout{Ane Schulde nicht entwichin} Ane M  $\cdot$ er was dar zuo] er dar zvͤ was I waz er dar zuͯ L \textbf{14} ouch Inguliart] auch ingliart I ovch Jngliart O (R) (Z) Jnguͯliart L inguliart ouch M \textbf{15} kurzen] den churzen I (L) (M) Rotten R \textbf{16} in Tabrunit] in tanbrump I Jncapronit O Jn Taburnit R \textbf{17} ersprenget] gesprenget I \textbf{18} \textit{statt 398.18 (mit vorgezogener Versdoppelung 398.19 und Füllvers):} Hie in schachte doͯrt im wal genet / Nun ward der walt gemenget / Vnd der weg geenget R  \textbf{19} schache] sla I schlachtte R \textbf{21} stüende] gestunde I (L) (M) \textbf{22} gewan] gawan M (R) (Z) \textbf{23} erbûwenes] erbuͤwen I  $\cdot$ landes daz] land R  $\cdot$ Aschalun] aschalûn I Ascalvn L (M) (R) (Z) \textbf{24} dô] Da O M Z  $\cdot$ vrâgter] vragt er I (O) (L) (R) (Z)  $\cdot$ Tschanfenzun] shanphazun I schampfazvn O schanffenzuͯn M schafenzun R tschanfanzvn Z \textbf{25} swaz] Waz L (M) (R) \textbf{26} hôch] Hohe O  $\cdot$ manic] vnd manic I (O) (L) (M) Z \textbf{27} durchstrichen] erstrichen M \textbf{28} dô] Da O M \textbf{29} âvoy] Awi O  $\cdot$ diu] diu diu I  $\cdot$ glast] [gast]: glast L \textbf{30} engêne kêrte] kert er gein I (O) kert engegen L R karte gegin M keret gein Z \newline
\end{minipage}
\hspace{0.5cm}
\begin{minipage}[t]{0.5\linewidth}
\small
\begin{center}*T
\end{center}
\begin{tabular}{rl}
 & \begin{Large}S\end{Large}wer was ze Bearosche komen,\\ 
 & \textbf{Doch} hete Gawan dâ genomen\\ 
 & den prîs ze beider sît alein,\\ 
 & wan daz \textbf{vor im} ein rîter schein,\\ 
5 & bî \textbf{rôtem} wâpe\textit{n} unerkant,\\ 
 & des prîs man in die hœhe bant.\\ 
 & Gawan hete êre unde heil,\\ 
 & ietweders volleclîchen teil.\\ 
 & Nû nâhete ouch sînes kampfes zît.\\ 
10 & der walt was lanc unde wît,\\ 
 & dâ durch er muose strîchen,\\ 
 & wolter kampfes niht entwîchen.\\ 
 & \textbf{âne schulde er was dar zuo} erkorn.\\ 
 & Nû was ouch Ingliart verlorn,\\ 
15 & sîn ors mit \textbf{den} kurzen ôren.\\ 
 & In Tabrunit von môren\\ 
 & wart nie bezzer ors ersprenget.\\ 
 & Nû wart der walt gemenget,\\ 
 & hie ein \textbf{slâ}, dort ein velt,\\ 
20 & etslîchez sô breit, daz ein gezelt\\ 
 & vil kûme drûffe stüende.\\ 
 & mit sehene gewan er künde:\\ 
 & \textbf{ein erbûwen lant} hiez Ascalun.\\ 
 & dô vrâgeter gegen Tschampfenzun,\\ 
25 & swaz im volkes widervuor.\\ 
 & hôch gebirge, manec muor,\\ 
 & des heter vil durchstrichen dar.\\ 
 & Dô nam er einer bürge war.\\ 
 & âvoy, diu gap vil werden glast!\\ 
30 & dâ \textbf{kêrte engegene} des landes gast.\\ 
\end{tabular}
\scriptsize
\line(1,0){75} \newline
T U V W \newline
\line(1,0){75} \newline
\textbf{1} \textit{Großinitiale} T U   $\cdot$ \textit{Initiale} V W  \textbf{2} \textit{Majuskel} T  \textbf{9} \textit{Majuskel} T  \textbf{14} \textit{Majuskel} T  \textbf{16} \textit{Majuskel} T  \textbf{18} \textit{Majuskel} T  \textbf{25} \textit{Überschrift:} Hie kvmet Gawan zvͦ schamphanzvn do er kempfen solte V   $\cdot$ \textit{Initiale} V  \textbf{28} \textit{Majuskel} T  \newline
\line(1,0){75} \newline
\textbf{1} Swer] WEr W  $\cdot$ Bearosche] Bearosce T (W) Bearotsche U (V) \textbf{2} Doch] Do W  $\cdot$ dâ] do V W \textbf{3} beider sît] beiden siten V \textbf{4} vor im] [*]: der vor V dar vor W  $\cdot$ schein] sich U \textbf{5} rôtem] roten V  $\cdot$ wâpen] wapem T \textbf{8} ietweders volleclîchen] Jequeder site vollecliche U Yetweders voͤlligliches W \textbf{9} nâhete] nahet U W \textbf{11} muose] mvese T muͤste V \textbf{14} Ingliart] Jngliart T gringalet V \textbf{15} den] \textit{om.} W \textbf{16} Tabrunit] da bruͦnit U tamburnit V tabrumit W \textbf{18} wart] was V \textbf{19} slâ] schache U (V) schachse W \textbf{22} gewan] gawan W \textbf{23} ein erbûwen lant] Ein erbuͦwet lant U Ein gebuwen lant V Erbauwens landes das W  $\cdot$ Ascalun] aschaluͦn U astalun W \textbf{24} Tschampfenzun] Tscampfenzvn T schamphanzvn V \textbf{25} swaz] Waz U (W) \textbf{26} gebirge] gebirge vnd U (V) \textbf{30} des landes] der werde W \newline
\end{minipage}
\end{table}
\end{document}
