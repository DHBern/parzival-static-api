\documentclass[8pt,a4paper,notitlepage]{article}
\usepackage{fullpage}
\usepackage{ulem}
\usepackage{xltxtra}
\usepackage{datetime}
\renewcommand{\dateseparator}{.}
\dmyyyydate
\usepackage{fancyhdr}
\usepackage{ifthen}
\pagestyle{fancy}
\fancyhf{}
\renewcommand{\headrulewidth}{0pt}
\fancyfoot[L]{\ifthenelse{\value{page}=1}{\today, \currenttime{} Uhr}{}}
\begin{document}
\begin{table}[ht]
\begin{minipage}[t]{0.5\linewidth}
\small
\begin{center}*D
\end{center}
\begin{tabular}{rl}
\textbf{210} & \textbf{kumt durch mîne nôt} ze wer."\\ 
 & zwischen dem graben unt dem ûzerem her\\ 
 & wart gestætet dirre vride.\\ 
 & dô wâpenden sich die \textbf{kampfes smide}.\\ 
5 & Dô saz der künec von Brandigan\\ 
 & ûf ein gewâpent kastelân,\\ 
 & daz was geheizen Guverjorz.\\ 
 & von sîme neven Grigorz,\\ 
 & dem künege von Ipotente,\\ 
10 & mit rîcher prîsente\\ 
 & was ez komen Clamide\\ 
 & \textbf{norden} über den Ukerse.\\ 
 & \begin{large}E\end{large}z brâhte \textbf{der} \textbf{künec} Narant\\ 
 & unt dar zuo \textbf{tûsent} scharjant\\ 
15 & mit harnasche, alsunder schilt.\\ 
 & den was \textbf{ir} solt \textbf{alsus} \textbf{gezilt}:\\ 
 & \textit{v}olleclîchen zwei jâr,\\ 
 & \textbf{ob dâventiure seit} \textbf{al} wâr.\\ 
 & Grigorz im sande ritter kluoc\\ 
20 & vünf hundert - ieslîcher truoc\\ 
 & helm ûf houbet gebunden -,\\ 
 & die wol mit strîte kunden.\\ 
 & Dô hete Clamides her\\ 
 & \textbf{ûf dem} lande unt \textbf{in} dem mer\\ 
25 & Pelrapeire alsô belegen,\\ 
 & die burgære \textbf{m\textit{uo}sen kumbers} pflegen.\\ 
 & \textbf{Ûz} kom geriten Parzival\\ 
 & \textbf{an} daz urteillîche wal,\\ 
 & dâ got erzeigen \textbf{solde},\\ 
30 & ob er im lâzen \textbf{wolde}\\ 
\end{tabular}
\scriptsize
\line(1,0){75} \newline
D \newline
\line(1,0){75} \newline
\textbf{5} \textit{Majuskel} D  \textbf{13} \textit{Initiale} D  \textbf{23} \textit{Majuskel} D  \textbf{27} \textit{Majuskel} D  \newline
\line(1,0){75} \newline
\textbf{7} Guverjorz] Gvueriorz D \textbf{9} Ipotente] Jpotente D \textbf{11} Clamide] Chlamide D \textbf{12} Ukerse] Vcher se D \textbf{17} volleclîchen] wolleclichen D \textbf{23} Clamides] Chlamides D \textbf{26} muosen] mvͤsen D \newline
\end{minipage}
\hspace{0.5cm}
\begin{minipage}[t]{0.5\linewidth}
\small
\begin{center}*m
\end{center}
\begin{tabular}{rl}
 & \textbf{kumet durch mîne nôt} ze wer."\\ 
 & zwischen dem graben und dem ûzeren her\\ 
 & wart gestætet dirre vride.\\ 
 & dô wâpeten sich die \textbf{kampfsmide}.\\ 
5 & dô saz der künic von Brandigan\\ 
 & ûf ein gewâpent kastelân,\\ 
 & daz was geheizen Guveri\textit{or}z.\\ 
 & von sînem n\textit{ev}en Grigorz,\\ 
 & dem künic von Ipotente,\\ 
10 & mit rîchem prêsente\\ 
 & was ez komen Clamide\\ 
 & \textbf{von den} über den Ucherse.\\ 
 & ez brâhte \textbf{der} \textbf{grâve} Narant\\ 
 & und dar zuo sarjant\\ 
15 & mit harnasch, alle sunder \dag schalt\dag .\\ 
 & den was \textbf{ir} solt \textbf{aldâr} \textbf{gezalt}\\ 
 & volleclîchen zwei jâr,\\ 
 & \textbf{seit diu âventiure} wâr.\\ 
 & Grigorz im sante ritter kluoc\\ 
20 & vünf hundert - ieclîcher truoc\\ 
 & helm ûf houbet gebunden -,\\ 
 & die wol mit strîte kunden.\\ 
 & dô hete Clamides her\\ 
 & \textbf{ûf dem} lande und \textbf{in} dem mer\\ 
25 & Pelraperie alsô belegen,\\ 
 & die burgære \textbf{muosen kumbers} pflegen.\\ 
 & \textbf{\begin{large}E\end{large}z} kam geriten Parcifal\\ 
 & \textbf{\textit{a}n} daz urteillîche wal,\\ 
 & d\textit{â} got erzöugen \textbf{solte},\\ 
30 & ob er i\textit{m} lâzen \textbf{wolte}\\ 
\end{tabular}
\scriptsize
\line(1,0){75} \newline
m n o Fr69 \newline
\line(1,0){75} \newline
\textbf{27} \textit{Initiale} m   $\cdot$ \textit{Capitulumzeichen} n  \newline
\line(1,0){75} \newline
\textbf{3} gestætet] bestetet n o \textbf{7} Guveriorz] guuerius m gúneriors n genieriors o \textbf{8} neven] namen m  $\cdot$ Grigorz] grigors m n o \textbf{9} Ipotente] ẏpotente o \textbf{12} von den] Von dem n o  $\cdot$ Ucherse] vͯcherse m ichter se n vter se o \textbf{14} sarjant] dusent sariant n (o) \textbf{18} seit] Siet o Ob seit Fr69  $\cdot$ wâr] fúr wor n (o) \textbf{19} Grigorz] Grigors m n o Griorz Fr69 \textbf{23} Clamides] clamidez o \textbf{25} Pelraperie] Palrapeir n Palrapier o \textbf{26} muosen] muͯssen m muͯsten n  $\cdot$ kumbers] kumber m (n) (o) \textbf{27} Ez] S::: Fr69 \textbf{28} an] Aln m \textbf{29} dâ] Do m n o  $\cdot$ erzöugen] erzeigen n Fr69  $\cdot$ solte] wolte n o (Fr69) \textbf{30} im] in m n o  $\cdot$ wolte] solte n o (Fr69) \newline
\end{minipage}
\end{table}
\newpage
\begin{table}[ht]
\begin{minipage}[t]{0.5\linewidth}
\small
\begin{center}*G
\end{center}
\begin{tabular}{rl}
 & \textbf{durch mîne nôt kumt} ze wer."\\ 
 & zwischen dem graben unde de\textit{m} ûzern her\\ 
 & wart gest\textit{æ}tet dirre vride.\\ 
 & dô wâpenten sich die \textbf{kampfes smide}.\\ 
5 & dô saz der künic von Brandigan\\ 
 & ûf ein gewâpent kastelân,\\ 
 & daz was geheizen Guferschurz.\\ 
 & \textit{von s}în\textit{em} neve\textit{n} Gregurz.\\ 
 & \textit{\textbf{von} dem künige von Spotente}\\ 
10 & \textit{mit rîcher prêsente}\\ 
 & \textit{was ez komen Clamide}\\ 
 & \textit{\textbf{norden} über den Ukerse.}\\ 
 & \textit{ez brâht \textbf{der} \textbf{grâve} Narrant}\\ 
 & \textit{und dar zuo \textbf{tûsent} sarjant}\\ 
15 & \textit{mit harnasch, alle sunder schilt.}\\ 
 & \textit{den was solt \textbf{sus} \textbf{bezilt}}:\\ 
 & \textit{volleclîche zwei jâr,}\\ 
 & \textit{\textbf{ob diu âventiure seit} wâr.}\\ 
 & \textit{Gregurz im sande rîter kluoc}\\ 
20 & \textit{vünfhundert - ieglîcher truoc}\\ 
 & \textit{helm ûf houbt gebunden -,}\\ 
 & \textit{die wol mit strîte kunden.}\\ 
 & \textit{dô het Clamides her}\\ 
 & \textit{\textbf{von} lande unde \textbf{ûf} dem mer}\\ 
25 & \textit{Pelrapeire alsô belegen,}\\ 
 & \textit{die burgære \textbf{muosen kumbers} pflegen.}\\ 
 & \textit{\textbf{ûz}} kom \textit{g}er\textit{it}e\textit{n} Parzival\\ 
 & \textbf{an} daz urteillîche wal,\\ 
 & dâ got erzeigen \textbf{wolte},\\ 
30 & ober im lâzen \textbf{solte}\\ 
\end{tabular}
\scriptsize
\line(1,0){75} \newline
G I O L M Q R Z \newline
\line(1,0){75} \newline
\textbf{5} \textit{Initiale} I L  \textbf{19} \textit{Initiale} I  \textbf{21} \textit{Initiale} M  \textbf{23} \textit{Überschrift:} Wie Clamide vnd parcifal mit ein anden kempften Z   $\cdot$ \textit{Initiale} O Q Z  \textbf{27} \textit{Initiale} L  \textbf{29} \textit{Initiale} R  \newline
\line(1,0){75} \newline
\textbf{1} \textit{Versdoppelung 209.26-210.20 (²M) nach 210.20; Lesarten der vorausgehenden Verse mit ¹M bezeichnet} M   $\cdot$ kumt] chom I (R) \textbf{2} zwischen dem] Zcwuschen \textsuperscript{2}\hspace{-1.3mm} M (R) (Z)  $\cdot$ dem ûzern] des vzern G dem vͤzrem I vsszirn \textsuperscript{1}\hspace{-1.3mm} M den vzzirn \textsuperscript{2}\hspace{-1.3mm} M  $\cdot$ her] wer R \textbf{3} gestætet] gestatet G (L) gestecget I \textbf{4} dô] Da M Z  $\cdot$ wâpenten] wauffnotent R  $\cdot$ sich] \textit{om.} L  $\cdot$ kampfes] champf O (L) (M) (Q) (R) (Z) \textbf{5} dô] Da M Z  $\cdot$ Brandigan] prandigan I Brandegan L \textbf{6} ein gewâpent] ein gewappen L eynen gewapeten M \textbf{7} Guferschurz] kuuerscurz I schvfert schvͦrz O [Gvferivs]: Gvferivrs L guffer schurz \textsuperscript{1}\hspace{-1.3mm} M guferschurz \textsuperscript{2}\hspace{-1.3mm} M gurerturtz Q guͦfershuͦrz R Gvneriorz Z \textbf{8} im sandez sin neve gregurz G  $\cdot$ sînem neven] sinem O disen neuen R  $\cdot$ Gregurz] Gregruz I Grehvͦrz O Greguͯrs L gregutz Q Greguͦrz R Grigorz Z \textbf{9} \textit{Die Verse 210.9-26 fehlen} G   $\cdot$ von dem] Dem O L (M) Q R Z  $\cdot$ spotente] ypotente O L \textsuperscript{2}\hspace{-1.3mm} M Q ipotente \textsuperscript{1}\hspace{-1.3mm} M (R) (Z) \textbf{10} rîcher] rechter Q guͯtter R \textbf{11} was] Wart L  $\cdot$ komen] gesendet L  $\cdot$ Clamide] Glamide O \textbf{12} \textit{Vers 210.12 fehlt} Q   $\cdot$ norden] nordern I (O) Northin 2 M Worden R  $\cdot$ Ukerse] vchen se I vker se O vͯcher se L vkirse M Vͦcherse R vcher se Z \textbf{13} ez brâht] Er brachte L Zu brachte Q Bracht es R  $\cdot$ Narrant] Narant L (M) (Q) R (Z) \textbf{14} dar zuo] [daz]: dar zvͦ O das zcu \textsuperscript{1}\hspace{-1.3mm} M [das]: dar zcu \textsuperscript{2}\hspace{-1.3mm} M \textbf{15} harnasch] harnaisk I  $\cdot$ alle] [alls]: alle Q  $\cdot$ sunder] sundern M \textbf{16} den] Des O  $\cdot$ solt] ir solt O L M Q (R) Z  $\cdot$ bezilt] gezilt O L (M) Q (R) Z \textbf{17} volleclîche] Vellecliche L \textbf{18} Ob] Vff M  $\cdot$ wâr] al war Z \textbf{19} Gregurz] GreGruz I [Grege]: Gregruͯs L Gegrusset Q Greguͦrz R Grigorz Z  $\cdot$ kluoc] chlvͦge O \textbf{20} vünfhundert] Fumhudert Q  $\cdot$ ieglîcher] ritter ýglicher L  $\cdot$ truoc] trvͦge O \textbf{21} houbt] sinem havpt O (L) (R) sin houbit M (Q) \textbf{22} kunden] kinden M \textbf{23} dô] ÷o O Da M  $\cdot$ Clamides] klamides I Glamides O Clamidez L \textbf{24} von] Von dem O L (Z) Vff deme M (Q) (R)  $\cdot$ ûf] von O (L) in Q R \textbf{25} Pelrapeire] Pailrapeir I [Perrapere]: Pelrapere O Pelapiere M  $\cdot$ alsô] so M Q \textbf{27} ûz] hie G Ez L  $\cdot$ geriten] och der ivnge G  $\cdot$ Parzival] Parzifal I Paricifal O parcifal L Z parczifal M R partzifal Q \textbf{28} urteillîche] vnteliche M \textbf{29} dâ] Do Q  $\cdot$ erzeigen] erzeuͯgen L  $\cdot$ wolte] solde O (L) M Q (R) \textbf{30} im] ymanns Q  $\cdot$ solte] wolde O (L) (M) Q (R) \newline
\end{minipage}
\hspace{0.5cm}
\begin{minipage}[t]{0.5\linewidth}
\small
\begin{center}*T
\end{center}
\begin{tabular}{rl}
 & \textbf{kumt durch mîne nôt} ze wer."\\ 
 & zwischen dem graben unde dem ûzern her\\ 
 & wart gestætet dirre vride.\\ 
 & Dô wâpenten sich die \textbf{kampfsmide}.\\ 
5 & Dô saz der künec von Brandigan\\ 
 & ûf ein gewâpent kastelân,\\ 
 & daz was geheizen Guverschorz.\\ 
 & von sînem neven Grigorz,\\ 
 & dem künege von Ypotente,\\ 
10 & mit rîcher prîsente\\ 
 & was ez komen Clamide\\ 
 & \textbf{norden} über den Ucker se.\\ 
 & ez brâhte \textbf{grâve} Narant\\ 
 & unde dar zuo \textbf{tûsent} sarjant\\ 
15 & mit harnasche, alle sunder schilt.\\ 
 & den was \textbf{ir} solt \textbf{alsus} \textbf{gezilt}:\\ 
 & volleclîche zwei jâr,\\ 
 & \textbf{ob diu âventiure sagt} wâr.\\ 
 & Grigorz im sante rîter kluoc\\ 
20 & vünfhundert - iegeslîcher truoc\\ 
 & \textbf{einen} helm ûf \textbf{sîn} houbet gebunden -,\\ 
 & die wol mit strîte kunden.\\ 
 & \begin{large}D\end{large}ô hete Clamides her\\ 
 & \textbf{ûf dem} lande unde \textbf{in} dem mer\\ 
25 & Peilrapere alsô belegen,\\ 
 & die burgære \textbf{kumbers muosen} pflegen.\\ 
 & \textbf{ûz} kom geriten Parcifal\\ 
 & \textbf{in} daz urteillîche wal,\\ 
 & dâ got erzeigen \textbf{wolte},\\ 
30 & ob er im lâzen \textbf{solte}\\ 
\end{tabular}
\scriptsize
\line(1,0){75} \newline
T U V W \newline
\line(1,0){75} \newline
\textbf{4} \textit{Majuskel} T  \textbf{5} \textit{Majuskel} T  \textbf{23} \textit{Initiale} T U  \textbf{27} \textit{Überschrift:} Hie streit her partzifal mit dez kúnig klamide von brandigan vnd lediget pelarpier W   $\cdot$ \textit{Initiale} W  \newline
\line(1,0){75} \newline
\textbf{1} kumt] Kvmme V \textbf{3} gestætet] bestedeget U (V) \textbf{6} ein] einen W \textbf{7} daz] Der W  $\cdot$ Guverschorz] Gvͤuerschorz T Guͦverschorz U gvuerschoz V guferschorß W \textbf{8} sînem neven] im nenen U  $\cdot$ Grigorz] grigorfoz V grigorß W \textbf{9} Ypotente] ẏpotente V yporente W \textbf{11} ez] er W  $\cdot$ Clamide] klamide W \textbf{12} norden] Norde U  $\cdot$ Ucker se] acker se U [*se]: vcher se V huker se W \textbf{13} grâve] der graue W  $\cdot$ Narant] karant W \textbf{16} gezilt] bezilt W \textbf{19} Grigorz] Grigors U [Gigors]: Grigors V Gigors W \textbf{21} einen] \textit{om.} W \textbf{23} Clamides] klamides W \textbf{24} in] auff W \textbf{25} Peilrapere] Pelrapier W \textbf{26} die] Do die W  $\cdot$ muosen] mvesen T \textbf{27} Parcifal] parzifal V partzifal W  $\cdot$ ûz] SVß W \textbf{28} daz] [daz]: da U  $\cdot$ urteillîche] vrtailliches W \textbf{29} dâ] Do U Daz V (W)  $\cdot$ erzeigen] v́rzoͤigen V \newline
\end{minipage}
\end{table}
\end{document}
