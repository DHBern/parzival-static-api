\documentclass[8pt,a4paper,notitlepage]{article}
\usepackage{fullpage}
\usepackage{ulem}
\usepackage{xltxtra}
\usepackage{datetime}
\renewcommand{\dateseparator}{.}
\dmyyyydate
\usepackage{fancyhdr}
\usepackage{ifthen}
\pagestyle{fancy}
\fancyhf{}
\renewcommand{\headrulewidth}{0pt}
\fancyfoot[L]{\ifthenelse{\value{page}=1}{\today, \currenttime{} Uhr}{}}
\begin{document}
\begin{table}[ht]
\begin{minipage}[t]{0.5\linewidth}
\small
\begin{center}*D
\end{center}
\begin{tabular}{rl}
\textbf{385} & als er\textbf{z} erwarp zem Plimizœl.\\ 
 & Melyanzes krîe was 'Barbigœl',\\ 
 & diu werde houbtstat in Liz.\\ 
 & Gawan \textbf{nam} sîner tjoste vlîz.\\ 
5 & \textbf{dô} lêrte Melyanzen pîn\\ 
 & von Orastegentesin\\ 
 & der starke rœrîne schaft,\\ 
 & \multicolumn{1}{l}{ - - - }\\ 
 & \multicolumn{1}{l}{ - - - }\\ 
 & durch den schilt in \textbf{dem} arme \textbf{er} \textbf{gehaft}.\\ 
 & ein rîchiu tjost dâ geschach.\\ 
10 & Gawan in vlügelingen stach\\ 
 & \textbf{unden} enzwei sînen hindern satelbogen,\\ 
 & daz die helde vür unbetrogen\\ 
 & hinder den orsen \textbf{stuonden}.\\ 
 & dô tâten si, als si kunden,\\ 
15 & mit den swerten tûren.\\ 
 & dâ \textbf{wære zwein} gebûren\\ 
 & gedroschen mêr denne genuoc.\\ 
 & \textbf{ieweder} des andern garbe truoc.\\ 
 & \textbf{stückeht die wurden} hin geslagen.\\ 
20 & Melyanz ein sper \textbf{ouch} muose tragen,\\ 
 & daz \textbf{steckete dem helde} durch den arm.\\ 
 & \textbf{bluotec} sweiz \textbf{in} machte warm.\\ 
 & Dô zucte in mîn hêr Gawan\\ 
 & in \textbf{Brevigariezer} barbigân\\ 
25 & unt \textbf{betwang} in sicherheite.\\ 
 & der was er im bereite.\\ 
 & wære der junge man niht wunt,\\ 
 & dâ\textbf{ne} wære \textbf{niemen sô gâhes} kunt,\\ 
 & daz er im würde undertân.\\ 
30 & man m\textit{üe}se \textbf{in} langer hân erlân.\\ 
\end{tabular}
\scriptsize
\line(1,0){75} \newline
D \newline
\line(1,0){75} \newline
\textbf{23} \textit{Majuskel} D  \newline
\line(1,0){75} \newline
\textbf{1} Plimizœl] Plymizoͤl D \textbf{2} Barbigœl] Parbygoͤl D \textbf{3} Liz] Lŷz D \textbf{6} Orastegentesin] Orastegentesîn D \textbf{30} müese] mvͦse D \newline
\end{minipage}
\hspace{0.5cm}
\begin{minipage}[t]{0.5\linewidth}
\small
\begin{center}*m
\end{center}
\begin{tabular}{rl}
 & als er\textit{\textbf{z}} erwarp zem Plimizol.\\ 
 & Melianzes krîe was '\textit{B}arbig\textit{o}l',\\ 
 & diu werde houbetstat in L\textit{i}z.\\ 
 & Gawan \textbf{nam} sîner juste vlîz.\\ 
5 & \textbf{dô} lêrte Mel\textit{i}anzen pîn\\ 
 & von Arast\textit{e}gentesin\\ 
 & der starke rœrîne schaft\\ 
 & \multicolumn{1}{l}{ - - - }\\ 
 & \multicolumn{1}{l}{ - - - }\\ 
 & durch den schilt in \textbf{dem} arme \textbf{brast}.\\ 
 & ein rîchiu juste dâ geschach.\\ 
10 & Gawan in vlügelingen \textbf{nider} stach\\ 
 & \textbf{und} in zwei sînen hindern satelbogen,\\ 
 & daz die helde vür unbetrogen\\ 
 & hinder den rossen \textbf{stuonden}.\\ 
 & dô tâten\textit{s}, als si kunden,\\ 
15 & mit den swerten tûren.\\ 
 & dâ \textbf{wæren zwên} gebûren\\ 
 & gedroschen mêre danne genuoc.\\ 
 & \textbf{ietweder} des anderen garbe truoc.\\ 
 & \textbf{s\textit{t}ückeht die wurden} hin geslagen.\\ 
20 & Mel\textit{i}anz ein sper \textbf{ouch} muose tragen,\\ 
 & daz \textbf{steckete dem helde} durch den arm.\\ 
 & \textbf{bluot und} sweiz \textbf{im} mach\textit{te} warm.\\ 
 & dô zuckete in mîn hêr Gawan\\ 
 & in \textbf{Brevig\textit{a}r\textit{i}eze\textit{r}} barbigân\\ 
25 & und \textbf{twanc} in sicherheite.\\ 
 & der was er ime bereite.\\ 
 & wær \textbf{aber} der junge man niht wunt,\\ 
 & d\textit{â} \textbf{en}wære \textbf{niemen sô gâhes} kunt,\\ 
 & daz er ime würde undertân.\\ 
30 & man müese \textbf{ez} \textbf{in} lange\textit{r} hân erlân.\\ 
\end{tabular}
\scriptsize
\line(1,0){75} \newline
m n o \newline
\line(1,0){75} \newline
\newline
\line(1,0){75} \newline
\textbf{1} erz] er m \textbf{2} Melianzes] Melianz m Meliantz n Meliancz o  $\cdot$ krîe] kriegen n  $\cdot$ Barbigol] parbigal m parbigol n o \textbf{3} Liz] liez m lisz n liesz o \textbf{5} Melianzen] Meleanczen m meliantzen n meliancz o \textbf{6} Arastegentesin] arastagente sin m orastegente sin n aregestegente sin o \textbf{7} rœrîne] roͯrinen n rorinan o \textbf{9} dâ geschach] do beschach n o \textbf{10} vlügelingen] flugenligen o \textbf{14} tâtens] tattencz m dotens sú n \textbf{16} dâ] Do n Die o  $\cdot$ wæren zwên gebûren] worent zwen buren o \textbf{18} des anderen] der ander o \textbf{19} \textit{Verse 385.19-21 kontrahiert zu:} Stucket dem helm durch den arm o   $\cdot$ stückeht] Sckuckeht m \textbf{20} Melianz] Meleancz m Meliantz n  $\cdot$ muose] musse m muͯste n \textbf{21} daz steckete] Das stecket n  $\cdot$ helde] helm n \textbf{22} machte] machen m \textbf{23} zuckete] zucke o  $\cdot$ Gawan] gawann o \textbf{24} Brevigariezer] brevigere ze m brevigarie zuͯ der n brebigarie zer o \textbf{26} der] Desz o \textbf{27} aber] >aber< o \textbf{28} dâ] Do m n o \textbf{29} undertân] vnderta o \textbf{30} müese] muͯsse m muste o  $\cdot$ langer] langes m n o \newline
\end{minipage}
\end{table}
\newpage
\begin{table}[ht]
\begin{minipage}[t]{0.5\linewidth}
\small
\begin{center}*G
\end{center}
\begin{tabular}{rl}
 & als er\textbf{z} erwar\textit{p} zem Blimzol.\\ 
 & Melianzes krîe was 'Barbigol',\\ 
 & diu werde houbtstat in Liz.\\ 
 & Gawan \textbf{nam} sîner tjoste vlîz.\\ 
5 & \textbf{diu} lêrte Melianzen pîn.\\ 
 & von Orastegentesin\\ 
 & \begin{large}D\end{large}er starke rœrî\textit{n}e schaft\\ 
 & wart dâ getriben mit hurte kraft.\\ 
 & - daz tet Gawan, der werde gast -,\\ 
 & durch den schilt in \textbf{den} arm \textbf{er} \textbf{brast}.\\ 
 & ein rîchiu tjost \textbf{al} dâ geschach.\\ 
10 & Gawan in vlügelingen stach\\ 
 & \textbf{unde} enzwei sînen hinderen satelbogen,\\ 
 & daz die helde vür unbetrogen\\ 
 & hinder den orsen \textbf{stuonden}.\\ 
 & dô tâten si, als si kunden,\\ 
15 & mit den swerten tûren.\\ 
 & dâ \textbf{wære zwein} gebûren\\ 
 & gedroschen mêr dane genuoc.\\ 
 & \textbf{ietweder} des andern garbe truoc.\\ 
 & \textbf{die wurden stückeht} hin geslagen.\\ 
20 & Melianz ein sper muose tragen,\\ 
 & daz \textbf{stach den helt} durch den arm.\\ 
 & \textbf{bluotic} sweiz \textbf{im} machte warm.\\ 
 & dô zuct in mîn hêr Gawan\\ 
 & in \textbf{Brevegariezære} barbigân\\ 
25 & unt \textbf{twanc} in \textbf{umbe} sicherheit.\\ 
 & der was er im \textbf{dô} bereit.\\ 
 & wære der junge man niht wunt,\\ 
 & dâ\textbf{ne} wære \textbf{sô \textit{g}âhes niemen} kunt,\\ 
 & daz er im würde undertân.\\ 
30 & man m\textit{üe}se \textbf{in} lenger hân erlân.\\ 
\end{tabular}
\scriptsize
\line(1,0){75} \newline
G I O L M Q R Z Fr41 \newline
\line(1,0){75} \newline
\textbf{7} \textit{Initiale} G  \textbf{7} \textit{Initiale} I  \newline
\line(1,0){75} \newline
\textbf{1} \textit{Die Verse 370.13-412.12 fehlen} Q   $\cdot$ erwarp] erwarf G  $\cdot$ zem Blimzol] zemblimzol G zuͤ dem plimizol I (L) (Z) ze plimizol O zcu deme blimizol M von pomizol R zem Plẏmizoͤl Fr41 \textbf{2} Melianzes] Melyanzes O Meliantzes L Z Melianczes R  $\cdot$ Barbigol] babigol I Parbigoͤl Fr41 \textbf{3} in] von I L an Fr41  $\cdot$ Liz] Lisz L (M) Lys R Liez Z \textbf{5} lêrte] lert I  $\cdot$ Melianzen] Melẏanzen O Meliantzen L (Z) Malianczen R :::eli:::azen Fr41 \textbf{6} Orastegentesin] orastegente sin G Oraste gentesin O (L) (M) R (Z) :::gentesin Fr41 \textbf{7} rœrîne] ror ime G \textbf{7} wart dâ] Do ward R  $\cdot$ hurte] hurtes R \textbf{7} Gawan] :::n Fr41 \textbf{8} in den] in dē L (M) im R (Z) (Fr41) \textbf{9} rîchiu] riche R  $\cdot$ al] \textit{om.} I R \textbf{10} vlügelingen] :::enliche Fr41 \textbf{11} ein wenc hinder sinen satelbogen I  $\cdot$ unde] Vnden L  $\cdot$ hinderen] \textit{om.} R \textbf{12} helde] heldin M  $\cdot$ vür] \textit{om.} I R fvͦren O  $\cdot$ unbetrogen] en betrogen M \textbf{13} hinder den orsen] Hinderm rosze L \textbf{14} dô] Da O L M Z  $\cdot$ kunden] da chunden I \textbf{15} tûren] durnen R \textbf{16} wære] wærn O (Z) \textbf{17} gedroschen] Zedroschen O Getrosche R \textbf{18} ietweder] ir ietdweder I  $\cdot$ andern] ander L (R)  $\cdot$ garbe] grabe M R \textbf{20} Melianz] Melyanz O Meliantz L Z Meliancz R  $\cdot$ muose] ouch muste Z \textbf{21} stach] stachte O (L) (R) stecke Z  $\cdot$ den helt] dem helde O L (M) Z dem helden R  $\cdot$ arm] armen I \textbf{22} bluotic] Blutet M  $\cdot$ im] in O L Z \textbf{23} dô] Da M Z  $\cdot$ zuct] zcucte M (R) (Z)  $\cdot$ hêr] er M \textbf{24} in Brevegariezære] in preregariezare G inparbirazare I Jn prevegariezzerte O Jn Brevegarsszare L Jn prefegariezer M Jn Breuegariezere R Jn brevigariezzere Z \textbf{25} twanc] twant Z  $\cdot$ umbe] \textit{om.} O L M R Z \textbf{26} dô] vil I L \textit{om.} O M R Z \textbf{28} dâne] Denne R  $\cdot$ sô gâhes] so hahes G sin fachen R \textbf{29} würde] werre R \textbf{30} müese in] moͮse in G (M) muͦst in da I mvͦses in O (Z) muͯste insz L muͯst in sin R  $\cdot$ lenger hân erlân] lenger ledic lan I lenger han verlon R \newline
\end{minipage}
\hspace{0.5cm}
\begin{minipage}[t]{0.5\linewidth}
\small
\begin{center}*T
\end{center}
\begin{tabular}{rl}
 & \textbf{ê} alser \textbf{si} erwarp zem Plymizol.\\ 
 & Melyanzes krîe was 'Barbigol',\\ 
 & diu werde houbetstat in Liz.\\ 
 & Gawan sîner tjoste vlîz\\ 
5 & \textbf{nam war}. \textbf{diu} lêrte Melyanzen pîn.\\ 
 & von Orestegentesin\\ 
 & der starke rœrîne schaft\\ 
 & wart dar getriben mit hurte kraft.\\ 
 & - daz tet Gawan, der werde gast -,\\ 
 & durch den schilt in \textbf{dem} arme \textbf{si} \textbf{brast}.\\ 
 & ein rîche tjost \textbf{al}dâ gescha\textit{ch}.\\ 
10 & Gawan in vlügelingen stach\\ 
 & \textbf{unde} enzwei sînen hindern satelbogen,\\ 
 & daz die helde vür unbetrogen\\ 
 & hindern orsen \textbf{gestuonden}.\\ 
 & dô tâtens, alse si kunden,\\ 
15 & mit den swerten tûren.\\ 
 & dâ \textbf{wære zwein} gebûren\\ 
 & gedroschen mêr danne genuoc.\\ 
 & \textbf{ietwedern} des andern g\textit{ar}be \textbf{dar} truoc.\\ 
 & \textbf{die wurden stückehte} hin geslagen.\\ 
20 & Melyanz ein sper muose tragen,\\ 
 & daz \textbf{steckte dem helde} durch den arm.\\ 
 & \textbf{bluotic} sweiz \textbf{in} machte warm.\\ 
 & dô zuhtin mîn hêr Gawan\\ 
 & in \textbf{Prevegariez} barbigân\\ 
25 & unde \textbf{twang} in sicherheite.\\ 
 & der was er im bereite.\\ 
 & wære der junge man niht wunt,\\ 
 & dâ wære \textbf{dannoch niemanne} kunt,\\ 
 & daz er i\textit{m} würde undertân.\\ 
30 & man müese \textbf{sîn} lenger hân erlân.\\ 
\end{tabular}
\scriptsize
\line(1,0){75} \newline
T V W \newline
\line(1,0){75} \newline
\newline
\line(1,0){75} \newline
\textbf{1} ê] \textit{om.} V W  $\cdot$ si] [*]: ez V \textit{om.} W  $\cdot$ Plymizol] plẏmizol V plimizol W \textbf{2} Melyanzes] Melẏanzes V Melianzes W \textbf{3} Liz] Lŷz T lẏs V lis W \textbf{4} sîner] nam siner V (W) \textbf{5} nam war] \textit{om.} V W  $\cdot$ Melyanzen] melianzen T W melẏanzen V \textbf{6} Orestegentesin] [*]: orastegentesin V arastegente sein W \textbf{7} \textit{Die Verse 385.7¹-7² fehlen} W   $\cdot$ getriben] [*]: gezilt V \textbf{8} si brast] [*]: er brast V erbrast W \textbf{9} geschach] gescahc T \textbf{10} stach] [*]: nider stach V \textbf{11} sînen] den W \textbf{12} vür] vil V \textbf{13} gestuonden] stunden W \textbf{14} dô] Da V  $\cdot$ tâtens] taten W \textbf{16} dâ] Do V W \textbf{18} ietwedern] Jewederre V (W)  $\cdot$ garbe] grabe T  $\cdot$ dar] \textit{om.} V W \textbf{19} geslagen] getragen W \textbf{20} Melyanz] Melianz V Melyanß W \textbf{22} bluotic] [B*]: Bluͦt vnde V  $\cdot$ in] [*]: im V im W \textbf{24} Prevegariez] [*]: brevigariezer V breuegariessere W \textbf{27} wære] [W*]: Were aber V \textbf{28} [*]: Done were nieman so gohes kvnt V  $\cdot$ Danne wer so gahes niemant kunt W \textbf{29} im] in T \textbf{30} sîn] [*]: ez in V in es W \newline
\end{minipage}
\end{table}
\end{document}
