\documentclass[8pt,a4paper,notitlepage]{article}
\usepackage{fullpage}
\usepackage{ulem}
\usepackage{xltxtra}
\usepackage{datetime}
\renewcommand{\dateseparator}{.}
\dmyyyydate
\usepackage{fancyhdr}
\usepackage{ifthen}
\pagestyle{fancy}
\fancyhf{}
\renewcommand{\headrulewidth}{0pt}
\fancyfoot[L]{\ifthenelse{\value{page}=1}{\today, \currenttime{} Uhr}{}}
\begin{document}
\begin{table}[ht]
\begin{minipage}[t]{0.5\linewidth}
\small
\begin{center}*D
\end{center}
\begin{tabular}{rl}
\textbf{589} & \begin{large}Û\end{large}f durch den palas eine sît\\ 
 & gie ein gewelbe niht ze wît,\\ 
 & gegrêdet \textbf{über} den palas hôch;\\ 
 & sinwel sich daz \textbf{umbe} zôch.\\ 
5 & dâr ûffe stuont ein clâriu sûl,\\ 
 & diu was niht von holze vûl;\\ 
 & si was \textbf{lieht} unde starc,\\ 
 & sô grôz, vroun Camillen sarc\\ 
 & wære drûffe wol gestanden.\\ 
10 & ûz Feirefizes landen\\ 
 & brâht ez der wîse Clinschor,\\ 
 & werc, daz \textbf{hie stuont} enbor.\\ 
 & Sinwel als ein gezelt ez was.\\ 
 & der meister Jeometras,\\ 
15 & solt ez geworht hân \textbf{des} hant,\\ 
 & diu kunst wære im unbekant.\\ 
 & \textbf{Ez} was geworht mit \textbf{liste}:\\ 
 & adamas und \textbf{ametiste}\\ 
 & - diu âventiure uns wizzen lât -,\\ 
20 & \textbf{topâzje} und grânât,\\ 
 & \textbf{krisolîte}, \textbf{rubîne},\\ 
 & \textbf{smareide}, \textbf{sardîne},\\ 
 & sus wâren diu venster rîche.\\ 
 & wît unt hôch gelîche,\\ 
25 & als man der venster sûl sach,\\ 
 & der art was \textbf{obene al} daz dach.\\ 
 & dechein sûl stuont dâr unde,\\ 
 & diu sich gelîchen kunde\\ 
 & \textbf{der grôzen} sûl dâ zwischen stuont.\\ 
30 & uns tuot diu âventiure kunt,\\ 
\end{tabular}
\scriptsize
\line(1,0){75} \newline
D Z \newline
\line(1,0){75} \newline
\textbf{1} \textit{Initiale} D  \textbf{13} \textit{Majuskel} D  \textbf{17} \textit{Initiale} Z   $\cdot$ \textit{Majuskel} D  \newline
\line(1,0){75} \newline
\textbf{3} gegrêdet] Gedret Z \textbf{7} lieht] leiht Z \textbf{8} Camillen] gamillen Z \textbf{10} Feirefizes] feirefiezzes Z \textbf{11} Clinschor] Clinscor D Clinshor Z \textbf{14} Jeometras] Geometras Z \textbf{15} des] sin Z \textbf{16} kunst] kunt Z \textbf{17} Ez] Daz Z  $\cdot$ liste] listen Z \textbf{18} ametiste] amatisten Z \textbf{20} topâzje] Thopazie D Topazien Z \textbf{21} Crisolte Rvbbine D  $\cdot$ Crisoliten vnd rvbine Z \textbf{22} smareide] Smaraide D Smareide vnd Z \textbf{29} dâ zwischen] die da enzwisschen Z \newline
\end{minipage}
\hspace{0.5cm}
\begin{minipage}[t]{0.5\linewidth}
\small
\begin{center}*m
\end{center}
\begin{tabular}{rl}
 & ûf durch den palas ein sîte\\ 
 & gienc ein gewelbe niht zuo wîte,\\ 
 & gegrêdet \textbf{über} den palas hôch;\\ 
 & sinwel sich daz \textbf{umbe} zôch.\\ 
5 & dâr ûf stuont ein clâriu sûl,\\ 
 & diu was niht von holze vûl;\\ 
 & si was \textbf{lieht} und starc,\\ 
 & sô grô\textit{z}, \textit{v}rowen C\textit{a}millen sarc\\ 
 & wær dâr ûf wol gestanden.\\ 
10 & \dag ûf\dag  Ferefizes landen\\ 
 & brâht ez der wîse Clinsor,\\ 
 & werc, daz \textbf{hie stuont} enbor.\\ 
 & sinwel als ein gezelt ez was.\\ 
 & der meister Geo\textit{me}t\textit{r}as,\\ 
15 & solt ez geworht hân \textbf{des} hant,\\ 
 & diu kunst wær im unbekant.\\ 
 & \textbf{e\textit{z}} was geworht mit \textbf{liste}:\\ 
 & adamast und \textbf{a\textit{m}a\textit{t}iste}\\ 
 & - diu âventiure uns wizzen lât -,\\ 
20 & \textbf{topâzje} und grânât,\\ 
 & \textbf{krisolîte} \textbf{und} \textbf{rubîne},\\ 
 & \textbf{smarac} \textbf{und} \textbf{sardîne},\\ 
 & sus wâren diu venster rîche.\\ 
 & wît und hôch gelîche,\\ 
25 & als man der venster sûle sach,\\ 
 & der art was \textbf{oben al} daz dach.\\ 
 & dekein sûl stuont dâr unde,\\ 
 & diu sich glîchen kunde\\ 
 & \textbf{der grôzen} sûl dâ zwischen stuont.\\ 
30 & uns tuot diu âventiure kunt,\\ 
\end{tabular}
\scriptsize
\line(1,0){75} \newline
m n o \newline
\line(1,0){75} \newline
\newline
\line(1,0){75} \newline
\textbf{1} \textit{Die Verse 588.29-589.16 fehlen} n   $\cdot$ den] \textit{om.} o \textbf{7} starc] starckg o \textbf{8} grôz vrowen] grosse sorgen froͯwe m grosse frowe o  $\cdot$ Camillen] comillen m Camúllen o \textbf{10} Ferefizes] fere fizes m ferificzes o \textbf{14} Geometras] geongittas m geomeczas o \textbf{15} geworht hân des] geforcht han das o \textbf{17} ez] Er m \textbf{18} adamast] Adamas n Adomas o  $\cdot$ amatiste] adamiste m \textbf{19} wizzen] wisen m \textbf{20} topâzje] Thopazie m n Thopacie o \textbf{21} krisolîte] Crisolite m n o  $\cdot$ rubîne] robine n rúbẏne o \textbf{22} smarac] Smarag m n o \textbf{27} dekein] Do kein n \textbf{28} glîchen] gelchen o  $\cdot$ kunde] kuͯnde m o \newline
\end{minipage}
\end{table}
\newpage
\begin{table}[ht]
\begin{minipage}[t]{0.5\linewidth}
\small
\begin{center}*G
\end{center}
\begin{tabular}{rl}
 & ûf durch den palas ein sît\\ 
 & gienc ein gewelbe niht ze wît,\\ 
 & gegrêdet \textbf{ûf} den palas hôch;\\ 
 & sinewel sich daz \textbf{über} zôch.\\ 
5 & dâr ûffe stuont ein clâriu sûl,\\ 
 & diu was niht von holze vûl;\\ 
 & si was \textbf{grôz} unde starc,\\ 
 & sô grôz, vroun Camillen sarc\\ 
 & wære drûf \textit{wol} gestanden.\\ 
10 & ûz Feirafizes landen\\ 
 & brâht ez der wîse Clinsor,\\ 
 & werc, daz \textbf{hie stuont} enbor.\\ 
 & sinewel als ein gezelt ez was.\\ 
 & der meister Geometrias,\\ 
15 & solt ez geworht haben \textbf{sîn} hant,\\ 
 & diu kunst wære im unbekant.\\ 
 & \textbf{daz} was geworht mit \textbf{listen}:\\ 
 & adamas unde \textbf{amatisten}\\ 
 & - diu âventiure uns wizzen lât -,\\ 
20 & \textbf{topâzjen} unde grânât,\\ 
 & \textbf{krisolîten} \textbf{unde} \textbf{rubîn},\\ 
 & \textbf{smaragde\textit{n}} \textbf{unde} \textbf{sardîn},\\ 
 & sus wâren diu venster rîche.\\ 
 & wît unde hôch gelîche,\\ 
25 & als man der venster sûle sach,\\ 
 & der art was \textbf{\textit{ob}en\textit{â}n} daz dach.\\ 
 & \multicolumn{1}{l}{ - - - }\\ 
 & \multicolumn{1}{l}{ - - - }\\ 
 & \textbf{dehein} sûl dâ enzwischen stuont.\\ 
30 & uns tuot diu âventiure kunt,\\ 
\end{tabular}
\scriptsize
\line(1,0){75} \newline
G I L M Fr23 \newline
\line(1,0){75} \newline
\textbf{1} \textit{Initiale} I  \textbf{7} \textit{Initiale} L  \textbf{19} \textit{Initiale} I  \newline
\line(1,0){75} \newline
\textbf{2} gewelbe] giwelle M \textbf{3} gegrêdet] gebraitet I  $\cdot$ ûf] vber L (M) \textbf{4} \textit{nach 589.4:} Ein tach [v*]: von richer achte / Als es Clinisor er dachte L   $\cdot$ daz] dar L \textbf{6} was] en was M \textbf{7} si] Dy M \textbf{8} sô grôz] daz I Ez ware L  $\cdot$ vroun] frow L  $\cdot$ Camillen] kanillen I Kamillen L gamillen M \textbf{9} wære] \textit{om.} L  $\cdot$ wol] \textit{om.} G \textbf{10} Feirafizes] ferafizes G feirafeiz I ferefizes L ferrefiszes M \textbf{11} ez] \textit{om.} L  $\cdot$ Clinsor] chlinsor G Clinisor L klinsor M :::nsoht Fr23 \textbf{12} werc daz] Warp des M \textbf{14} Geometrias] :::as Fr23 \textbf{20} topâzjen] Topazien G L (M) Topozien I  $\cdot$ grânât] :::t Fr23 \textbf{21} krisolîten] Crisoliten G (M) Grisoliten I Crisolten L  $\cdot$ rubîn] Ruͯbin L :::e Fr23 \textbf{22} smaragden] Smaragde G Smaragdin M \textbf{24} hôch] auch I \textbf{26} obenân] chenen G oben L (M) \textbf{27} \textit{Die Verse 589.27-28 fehlen} G I L M Fr23  \textbf{29} dehain suͤl da enzwishen was I  $\cdot$ Dehein svl oben dan zwischen stuͯnt L  $\cdot$ Jchein sul obin da entszwuscen stunt M  $\cdot$ :::h zwise stunt Fr23 \textbf{30} diu auenture chundet vns daz I \newline
\end{minipage}
\hspace{0.5cm}
\begin{minipage}[t]{0.5\linewidth}
\small
\begin{center}*T
\end{center}
\begin{tabular}{rl}
 & ûf durch den palas ein sît\\ 
 & gienc ein gewelbe niht ze wît,\\ 
 & gegrêdet \textbf{über} den palas hôch;\\ 
 & sinewel sich daz \textbf{über} zôch.\\ 
5 & dâr ûf stuont ein clâriu sûl,\\ 
 & diu was niht von holze vûl;\\ 
 & si was \textbf{michel} und starc,\\ 
 & sô grôz, vrouwen Camille sarc\\ 
 & wære drûf wol gestanden.\\ 
10 & ûz Ferefises landen\\ 
 & brâht ez der wîse Clynsor,\\ 
 & werc, daz \textbf{stuont hie} enbor.\\ 
 & sinewel als ein gezelt ez was.\\ 
 & der meister Geometrias,\\ 
15 & \textit{solt ez gewürket hân \textbf{sîn} hant},\\ 
 & \textit{diu kunst wær im unbekant}.\\ 
 & \textit{\textbf{daz}} \textit{was} geworht mit \textbf{listen}:\\ 
 & adamas und \textbf{amatisten}\\ 
 & - diu âventiure uns wizzen lât -,\\ 
20 & \textbf{topâzje} und grânât,\\ 
 & \textbf{krisolîte} \textbf{und} \textbf{rubîne},\\ 
 & \textbf{smaragde} \textbf{und} \textbf{sardîne},\\ 
 & sus wâren diu venster rîche.\\ 
 & wît und hôch gelîc\textit{h}e,\\ 
25 & als man der venster sûle sach,\\ 
 & der art was \textbf{oben allez} daz dach.\\ 
 & kein sûl stuont dâr unde,\\ 
 & diu sich glîchen kunde\\ 
 & \textbf{der grôzen} sûl, \textbf{die} dâ enzwischen stuont.\\ 
30 & un\textit{s} tuot diu av̂entiur kunt,\\ 
\end{tabular}
\scriptsize
\line(1,0){75} \newline
Q R W V U \newline
\line(1,0){75} \newline
\textbf{7} \textit{Capitulumzeichen} R  \newline
\line(1,0){75} \newline
\textbf{1} \textit{Die Verse 553.1-599.30 fehlen} U   $\cdot$ durch den palas] den palas durch R \textbf{3} gegrêdet] Gedrenget R \textbf{4} sich] \textit{om.} R  $\cdot$ über zôch] auff gezoch W \textbf{5} clâriu] clare R \textbf{7} michel] [*]: lieht V \textbf{8} Camille] Camillen R kamillen W gamillen V \textbf{10} Ferefises] feirefizzes Q freiefizes R ferafisses W verefis V \textbf{11} Clynsor] clinsor Q clinshor R klinshor W \textbf{12} stuont hie] hie stuͦnd R W (V) \textbf{14} Geometrias] geometras W V \textbf{15} \textit{Die Verse 589.15-16 fehlen} Q  \textbf{17} daz was] Soldes Q \textbf{18} amatisten] amitisten R ametisten W \textbf{20} topâzje] Topazion Q V Tapicen R topazian W \textbf{21} krisolîte] Cristolite Q V Chrisolte R Trisolite W \textbf{22} smaragde] Smaraden R \textbf{23} sus] Als Q \textbf{24} gelîche] gelichte Q \textbf{25} der venster] des vensters W  $\cdot$ sach] gesach W \textbf{29} enzwischen] zwischen W [*]: entzwúschent V  $\cdot$ stuont] waz W \textbf{30} Vnß kúnt die auenteúre das W  $\cdot$ uns] Vnd Q \newline
\end{minipage}
\end{table}
\end{document}
