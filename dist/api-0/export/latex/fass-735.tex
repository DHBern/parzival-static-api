\documentclass[8pt,a4paper,notitlepage]{article}
\usepackage{fullpage}
\usepackage{ulem}
\usepackage{xltxtra}
\usepackage{datetime}
\renewcommand{\dateseparator}{.}
\dmyyyydate
\usepackage{fancyhdr}
\usepackage{ifthen}
\pagestyle{fancy}
\fancyhf{}
\renewcommand{\headrulewidth}{0pt}
\fancyfoot[L]{\ifthenelse{\value{page}=1}{\today, \currenttime{} Uhr}{}}
\begin{document}
\begin{table}[ht]
\begin{minipage}[t]{0.5\linewidth}
\small
\begin{center}*D
\end{center}
\begin{tabular}{rl}
\textbf{735} & \textit{\begin{large}Û\end{large}}f sîner unverzagten reise.\\ 
 & der selbe kurteise\\ 
 & was ein heidenischer man,\\ 
 & der toufes kunde nie gewan.\\ 
5 & Parzival reit balde\\ 
 & gein eime grôzem walde\\ 
 & ûf einer liehten waste\\ 
 & gein eime rîchen gaste.\\ 
 & ez ist wunder, ob ich \textbf{armer man}\\ 
10 & \textbf{die} rîcheit \textbf{iu} gesagen kan,\\ 
 & die der heiden vür zimierde truoc.\\ 
 & sage ich des mêre denne genuoc,\\ 
 & dennoch mac ich\textbf{s} \textbf{iu} mêr wol \textbf{sagen},\\ 
 & \textbf{wil ich} sîner rîcheit niht \textbf{gedagen}.\\ 
15 & Swaz diende Artuses hant\\ 
 & \textbf{ze} Bertane unt \textbf{in} Engellant,\\ 
 & \textbf{daz vergülte} niht die steine,\\ 
 & die mit edelem arde reine\\ 
 & lâgen ûf des \textbf{heldes} wâpenroc.\\ 
20 & \textbf{der was} tiure ân al getroc.\\ 
 & rubîne, calcidône\\ 
 & wâren dâ ze swachem lône.\\ 
 & der wâpenroc gap \textbf{blanken} schîn.\\ 
 & in dem berge zAgremuntin\\ 
25 & die \textbf{würme} salamander\\ 
 & in worhten zein ander\\ 
 & in dem heizen viure.\\ 
 & \textbf{die} \textbf{wâren} steine tiure\\ 
 & lâgen drûf tunkel und lieht,\\ 
30 & ir art mac ich \textbf{benennen} niht.\\ 
\end{tabular}
\scriptsize
\line(1,0){75} \newline
D \newline
\line(1,0){75} \newline
\textbf{1} \textit{Initiale} D  \textbf{15} \textit{Majuskel} D  \newline
\line(1,0){75} \newline
\textbf{1} Ûf] ÷f D \textbf{5} Parzival] Parcifal D \textbf{15} Artuses] Artvs D \textbf{21} rubîne] Rvbbine D \newline
\end{minipage}
\hspace{0.5cm}
\begin{minipage}[t]{0.5\linewidth}
\small
\begin{center}*m
\end{center}
\begin{tabular}{rl}
 & ûf sîner unverzagte\textit{n} reise.\\ 
 & der selbe kurteise\\ 
 & wa\textit{s} ein heidenischer man,\\ 
 & der toufes kunde nie gewan.\\ 
5 & \begin{large}P\end{large}arcifal reit balde\\ 
 & gegen einem grôzen walde\\ 
 & ûf einer liehte\textit{n} waste\\ 
 & gegen einem rîchen gaste.\\ 
 & ez ist wunder, ob ich \textbf{armer man}\\ 
10 & \textbf{die} rîcheit \textbf{iu} gesagen kan,\\ 
 & die der heiden vür zimierde truoc.\\ 
 & sage ich des mêre dan genuoc,\\ 
 & dannoch mac ich\textbf{s} \textbf{iu} mêr wol \textbf{gesagen},\\ 
 & \textbf{ich wil} sîner rîcheit niht \textbf{gedagen}.\\ 
15 & waz diende Artuses hant\\ 
 & \textbf{zuo} Britanie und \textbf{in} Engelant,\\ 
 & \textbf{daz \textit{v}ergülten} niht die steine,\\ 
 & die mit edelm arde reine\\ 
 & lâgen ûf des wâpenroc.\\ 
20 & \textbf{der was} tiur âne alliu getroc.\\ 
 & rubîne \textbf{oder} calcedône\\ 
 & wâren d\textit{â} ze swachem lône.\\ 
 & der wâpenroc gap \textbf{blanken} schîn.\\ 
 & in dem berge zuo Agremontin\\ 
25 & die \textbf{tierlîn} salamander\\ 
 & \textit{in worhten zuo ein ander}\\ 
 & in dem heizen \textit{v}iur.\\ 
 & \textbf{die} \textbf{edelen} stein tiur\\ 
 & lâgen dâr ûf tunkel und lieht,\\ 
30 & ir ar\textit{t} mac ich \textbf{benennen} niht.\\ 
\end{tabular}
\scriptsize
\line(1,0){75} \newline
m n o V V' Fr69 \newline
\line(1,0){75} \newline
\textbf{5} \textit{Initiale} m n V V' Fr69  \newline
\line(1,0){75} \newline
\textbf{1} \textit{Die Verse 734.20-735.4 fehlen} V'   $\cdot$ unverzagten] vnferzagtte m \textbf{2} selbe] selben n \textbf{3} was] wan m  $\cdot$ heidenischer] heidenischen o \textbf{5} Parcifal] Parzefal V Parzifal V' \textbf{6} \textit{Die Verse 735.6-7 fehlen } o   $\cdot$ einem] ime V' \textbf{7} liehten] liehtte m \textbf{9} armer] armen o \textbf{10} iu] \textit{om.} V'  $\cdot$ gesagen] [gesaget]: gesagen n \textbf{11} \textit{Die Verse 735.11-14 fehlen} V'  \textbf{12} ich] ich úch V \textbf{13} ichs iu mêr wol] iches uͯch nie wol o ich úch mere wol V ichz v́ wol mer Fr69  $\cdot$ gesagen] sagen n (o) V (Fr69) \textbf{14} gedagen] vertagen o (Fr69) \textbf{15} waz] Swaz V Fr69  $\cdot$ Artuses] artusz o \textbf{16} Britanie] brittanie m V'  $\cdot$ Engelant] engellant n Engenlant V (V') \textbf{17} vergülten] ergultten m [vergulten]: vergulte o \textbf{19} :::es wapen roch Fr69  $\cdot$ des wâpenroc] des [*]: heldez wopenrog V dez heldes wappenrok V' \textbf{20} der was] [D*]: Die worent V Die worent V'  $\cdot$ alliu] allen V V' :llen Fr69  $\cdot$ getroc] betrok V' \textbf{21} rubîne] Rubin n Robine V  $\cdot$ calcedône] kalttedone m calcidone o calzedone V V' \textbf{22} dâ] do m n o V V' \textbf{24} zuo Agremontin] zuͯ agramontin m n (o) zagremontin V sagremontin V' \textbf{26} \textit{Vers 735.26 fehlt} m   $\cdot$ in] Die V'  $\cdot$ worhten] worten n o \textbf{27} dem] der o  $\cdot$ viur] fruͯr m \textbf{28} tiur] so túre n \textbf{30} art] arat m  $\cdot$ benennen] genennen V' \newline
\end{minipage}
\end{table}
\newpage
\begin{table}[ht]
\begin{minipage}[t]{0.5\linewidth}
\small
\begin{center}*G
\end{center}
\begin{tabular}{rl}
 & \begin{large}Û\end{large}f sîner unverzageten reise.\\ 
 & der selbe kurteise\\ 
 & was ein heidenisch man,\\ 
 & der toufe er kunde nie gewan.\\ 
5 & Parcival reit balde\\ 
 & gein einem grôzen walde\\ 
 & ûf einer liehten waste\\ 
 & gein einem rîchen gaste.\\ 
 & ez ist wunder, obe ich \textbf{arman}\\ 
10 & \textbf{dise} rîcheit gesagen kan,\\ 
 & die der heiden vür zimiere truoc.\\ 
 & sage ich des mê danne genuoc,\\ 
 & dannoch mac ich \textbf{iu} mê wol \textbf{sagen},\\ 
 & \textbf{wil ich} sîner rîcheit niht \textbf{verdagen}.\\ 
15 & swaz dient Artuses hant,\\ 
 & Britanie und Engellant,\\ 
 & \textbf{die vergülten} niht die steine,\\ 
 & die mit edelm arde reine\\ 
 & lâgen ûf des \textbf{heiden} wâpenroc.\\ 
20 & \textbf{die wâren} tiure ân al getroc.\\ 
 & rubîne, calcidône\\ 
 & wâren dâ ze swachem lône.\\ 
 & der wâpenroc gab \textbf{liehten} schîn.\\ 
 & in dem berge ze Agementin\\ 
25 & die \textbf{würme} salamander\\ 
 & in worhten zein ander\\ 
 & in dem heizen viure.\\ 
 & \textbf{die} \textbf{wâren} steine tiure\\ 
 & lâgen drûfe tunkel unde lieht,\\ 
30 & ir art mac ich \textbf{genennen} niht.\\ 
\end{tabular}
\scriptsize
\line(1,0){75} \newline
G I L M Z Fr18 Fr24 \newline
\line(1,0){75} \newline
\textbf{1} \textit{Initiale} G L Z Fr18  \textbf{5} \textit{Initiale} Fr24  \textbf{15} \textit{Initiale} I  \newline
\line(1,0){75} \newline
\textbf{4} der] Dez L  $\cdot$ toufe] tovfes L Z Fr18  $\cdot$ er] \textit{om.} I M Z Fr18  $\cdot$ kunde] chunne I \textbf{5} Parcival] Parcifal G Z Fr18 (Fr24) Parzifal I L M  $\cdot$ reit] der rait I \textbf{6} grôzen] so grozem I grvnem Fr18 grozzem Fr24 \textbf{7} liehten] lichten L (M) (Fr24) \textbf{8} rîchen] richem I Fr18 \textbf{9} ez ist] ez ist ez ist I  $\cdot$ arman] arm man I (L) (M) (Z) (Fr18) (Fr24) \textbf{10} gesagen] ev gesagen I (M) Z (Fr18) (Fr24) \textbf{11} vür] vffe vur I \textbf{12} ich] ich ev I \textbf{13} ich iu mê wol] ich ev me I ich wol mere L ich v me M ichs mer ev wol Z ich wol mer iv Fr18 \textbf{14} sîner] sin I  $\cdot$ verdagen] dagen L gidagen M (Z) (Fr24) \textbf{15} swaz] Waz L (M)  $\cdot$ dient] dinte L (Fr24)  $\cdot$ Artuses] Artus G (Z) (Fr24) Artuͯses L \textbf{16} Britanie] pritanie I Brẏtanie Fr18  $\cdot$ Engellant] engenlant I engillant M (Fr18) \textbf{17} die] dine I \textbf{18} edelm] edeler L edelin M \textbf{19} heiden] [*eidens]: heidens L heldis M (Fr18) \textbf{20} al] \textit{om.} I allen L alle M Z Fr24  $\cdot$ getroc] [troch]: getroch L \textbf{21} rubîne] rv̂bine G Rubin I Rvbẏne Fr18 :::ŷne Fr24  $\cdot$ calcidône] Galcidone G I Calcedonie L kalczidone M Calcẏdone Fr18 \textbf{22} ze] in I \textbf{23} liehten] blanchen L (Z) Fr18 Fr24 swachen M \textbf{24} in] Jm L  $\cdot$ ze Agementin] ze agemetin I Sagremontin L zcaragmeyntin M zv egremontin Z ze Agramentin Fr18 zeAgmentin Fr24 \textbf{25} würme] vuͤrine I \textbf{27} heizen] haizem I herczin M \textbf{28} die wâren steine] Die steine waren L Da waren steyne M Die wurme waren Z \textbf{29} lâgen] Die lagen L Stein lagen Z  $\cdot$ lieht] lýcht L (M) \textbf{30} mac ich] ich mach L \newline
\end{minipage}
\hspace{0.5cm}
\begin{minipage}[t]{0.5\linewidth}
\small
\begin{center}*T
\end{center}
\begin{tabular}{rl}
 & \begin{large}Û\end{large}f sîner unverzageten reise.\\ 
 & der selbe kurteise\\ 
 & was ein heidenscher man,\\ 
 & der toufes kunde \textbf{ê} nie gewan.\\ 
5 & Parcifal reit balde\\ 
 & gein eime grôzen walde\\ 
 & ûf einer liehte\textit{n} waste\\ 
 & gein eime rîchen gaste.\\ 
 & ez ist wunder, ob ich \textbf{armer man}\\ 
10 & \textbf{dise} rîcheit gesagen kan,\\ 
 & die der heiden vür zimierde truoc.\\ 
 & sage ich \textbf{iu} des mêr dan genuoc,\\ 
 & dannoch mag ich \textbf{es} mêr wol \textbf{sagen},\\ 
 & \textbf{wil ich} sîner rîcheit niht \textbf{gedagen}.\\ 
15 & waz dienete Artuses hant\\ 
 & \textbf{zuo} Britanie und \textbf{in} Engellant,\\ 
 & \textbf{daz vergülten} niht die steine,\\ 
 & die mit edelme arde reine\\ 
 & lâgen ûf des \textbf{heldes} wâpenroc.\\ 
20 & \textbf{die wâren} tiure âne alliu getroc.\\ 
 & rubîne, calcidône\\ 
 & wâre\textit{n} d\textit{â} zuo swacheme lône.\\ 
 & der wâpenroc gap \textbf{blanken} schîn.\\ 
 & in dem berge zuo Agremontin\\ 
25 & die \textbf{würme} salamander\\ 
 & in worhten zuo ein ander\\ 
 & in dem heizen viure.\\ 
 & \textbf{daz} \textbf{wâren} steine tiure\\ 
 & lâgen dâr ûfe dunkel und lieht,\\ 
30 & ir art mac ich \textbf{genennen} niht.\\ 
\end{tabular}
\scriptsize
\line(1,0){75} \newline
U W Q R \newline
\line(1,0){75} \newline
\textbf{1} \textit{Initiale} U  \textbf{5} \textit{Großinitiale} R   $\cdot$ \textit{Initiale} W  \newline
\line(1,0){75} \newline
\textbf{1} unverzageten] vnuertzagter Q \textbf{4} der] Des R  $\cdot$ toufes] [tewfels]: tewfes Q  $\cdot$ ê] \textit{om.} W Q \textbf{5} Parcifal] Parzifal U PArtzifal W (Q) (R) \textbf{6} grôzen] grossem W Q \textbf{7} Vff einen liechtte vaste R  $\cdot$ liehten] liechte U \textbf{8} rîchen] reichem W \textbf{10} gesagen] eúch gesagen W (Q) (R) \textbf{11} truoc] schone truͦg R \textbf{12} iu] \textit{om.} Q R \textbf{13} ich es mêr] ich euch Q ich uch mere R \textbf{14} gedagen] bedagen Q vertagen R \textbf{15} Artuses] artus W Q R \textbf{16} Britanie] britania W britange Q  $\cdot$ in Engellant] in engel lant Q mengenland R \textbf{17} vergülten] vergúlte W (Q) (R) \textbf{18} edelme] edlen R \textbf{19} heldes] helden R \textbf{20} die] Der Q  $\cdot$ âne alliu] an alles W an al Q (R) \textbf{21} rubîne] Rubein Q  $\cdot$ calcidône] Calcydone U kalcidone W \textbf{22} wâren] ware U  $\cdot$ dâ] do U Q die W  $\cdot$ zuo] \textit{om.} W \textbf{24} Agremontin] [Agrenontin]: Agranontin U agremontein W agremvntein Q Agromontin R \textbf{28} daz] [D*]: daz U Die W Q R \textbf{29} lieht] licht Q \textbf{30} ir art] Rart Q  $\cdot$ genennen] genomen R \newline
\end{minipage}
\end{table}
\end{document}
