\documentclass[8pt,a4paper,notitlepage]{article}
\usepackage{fullpage}
\usepackage{ulem}
\usepackage{xltxtra}
\usepackage{datetime}
\renewcommand{\dateseparator}{.}
\dmyyyydate
\usepackage{fancyhdr}
\usepackage{ifthen}
\pagestyle{fancy}
\fancyhf{}
\renewcommand{\headrulewidth}{0pt}
\fancyfoot[L]{\ifthenelse{\value{page}=1}{\today, \currenttime{} Uhr}{}}
\begin{document}
\begin{table}[ht]
\begin{minipage}[t]{0.5\linewidth}
\small
\begin{center}*D
\end{center}
\begin{tabular}{rl}
\textbf{213} & mac \textbf{nû belîben vor dir} vrî.\\ 
 & nû lerne, waz sterben sî!"\\ 
 & "Neinâ, \textbf{werder} degen balt!\\ 
 & dîne êre \textbf{wirt} sus drîzecvalt\\ 
5 & \textbf{vast} an mir erzeiget,\\ 
 & sît dû mich hâst geneiget.\\ 
 & \textbf{wâ} \textbf{m\textit{ö}hte} dir hœher prîs geschehen?\\ 
 & Condwiramurs \textbf{mac} \textbf{wol} \textbf{jehen},\\ 
 & daz ich der unsælige bin\\ 
10 & unt dîn gelücke hât gewin.\\ 
 & dîn lant ist erlœset,\\ 
 & als der sîn schif \textbf{erœset}.\\ 
 & \textbf{ez} \textbf{ist} vil deste lîhter.\\ 
 & mîn gewalt ist sîhter.\\ 
15 & reht manlîchiu wünne\\ 
 & ist \textbf{worden an mir} dünne.\\ 
 & durch waz \textbf{soldest}û mich sterben?\\ 
 & ich muoz doch laster erben\\ 
 & ûf alle mîne nâchkumen.\\ 
20 & dû hâst den prîs unt den vrumen.\\ 
 & \textit{\begin{large}T\end{large}}uost\textbf{û} mir mêr, daz ist ân nôt.\\ 
 & ich trage den lebendigen tôt,\\ 
 & sît ich von \textbf{ir} gescheiden bin,\\ 
 & diu \textbf{mir} herze unde sin\\ 
25 & ie mit ir gewalt beslôz,\\ 
 & unt ich des nie gein ir genôz.\\ 
 & des muoz ich unsælic man\\ 
 & ir lîb, ir lant \textbf{dir} ledec lân."\\ 
 & Dô \textbf{dâhte}, der den sig hât,\\ 
30 & sân an Gurnemanzes rât,\\ 
\end{tabular}
\scriptsize
\line(1,0){75} \newline
D \newline
\line(1,0){75} \newline
\textbf{3} \textit{Majuskel} D  \textbf{21} \textit{Initiale} D  \textbf{29} \textit{Majuskel} D  \newline
\line(1,0){75} \newline
\textbf{7} möhte] mohte D \textbf{8} Condwiramurs] Condwir amvrs D \textbf{21} Tuostû] ÷vͦstv \textit{nachträglich korrigiert zu:} Tvͦstv D \textbf{30} Gurnemanzes] Gvrnemanzs D \newline
\end{minipage}
\hspace{0.5cm}
\begin{minipage}[t]{0.5\linewidth}
\small
\begin{center}*m
\end{center}
\begin{tabular}{rl}
 & mac \textbf{\textit{nû} blîben von dir} vrî.\\ 
 & nû lerne, waz sterben sî!"\\ 
 & "neinâ, \textbf{werder} degen balt!\\ 
 & dîn êre \textbf{ist} sus drîzicvalt\\ 
5 & \textbf{mêr} an mir erz\textit{ei}get,\\ 
 & sît dû mich hâst geneiget.\\ 
 & \textbf{wâ} \textbf{möht} dir hœher prîs geschehen?\\ 
 & C\textit{o}ndwieramurs \textbf{mac} \textbf{wol} \textbf{jehen},\\ 
 & daz ich der unsælige bin\\ 
10 & und dîn gelücke hât gewin.\\ 
 & dîn lant ist erlœset,\\ 
 & \multicolumn{1}{l}{ - - - }\\ 
 & \textbf{ez} \textbf{wirt} vil deste lîh\textit{t}er.\\ 
 & mîn gewalt ist sîhter.\\ 
15 & reht manlîch wünne\\ 
 & ist \textbf{worden an mir} dünne.\\ 
 & durch waz \textbf{soltest} dû mich sterben?\\ 
 & ich muoz doch laster erben\\ 
 & ûf alle mîne nâchkumen.\\ 
20 & dû hâst den prîs und den vrumen,\\ 
 & tuost mir \textit{m}êre, d\textit{ê}s âne nôt.\\ 
 & ich trage den lebendigen tôt,\\ 
 & sît ich von \textbf{ir} gescheiden bin,\\ 
 & diu \textbf{mir} herze und sin\\ 
25 & ie mit ir gewalt beslôz,\\ 
 & und ich des \textit{nie} gegen ir genôz.\\ 
 & des muoz ich unsælic man\\ 
 & ir lîp, ir lant \textbf{dir} ledic \textit{lân}."\\ 
 & \begin{large}D\end{large}ô \textbf{dâhte}, der den sige hât,\\ 
30 & sân an G\textit{u}rnemanzes rât,\\ 
\end{tabular}
\scriptsize
\line(1,0){75} \newline
m n o Fr69 \newline
\line(1,0){75} \newline
\textbf{29} \textit{Initiale} m n o  \newline
\line(1,0){75} \newline
\textbf{1} nû] in m  $\cdot$ von] vor n \textbf{2} lerne] lere n o \textbf{5} mêr] Mir o  $\cdot$ erzeiget] erzouget m (o) \textbf{7} geschehen] beschehen n o \textbf{8} Condwieramurs] Candwier amurs m Conduwúr amersz n Condiwir amersz o Cvndewier amurs Fr69 \textbf{9} daz] [Des]: Das n  $\cdot$ der] dir o \textbf{10} \textit{Versdoppelung (mit Anteil aus Vers 213.11):} Din lant gluͯg hat gewin o  \textbf{12} Vnd bin ich entbloͯset n (o) \textbf{13} ez] [Er]: Es Fr69  $\cdot$ lîhter] licher m liechter o \textbf{14} mîn] [Mit]: Min Fr69  $\cdot$ ist] ist worden Fr69  $\cdot$ sîhter] sicher n o \textbf{17} soltest dû] solestuͯ o \textbf{21} mir] \textit{om.} n o  $\cdot$ mêre] onere m mere des aner n mere dasz aner o  $\cdot$ dês] das m \textbf{22} lebendigen] bebendigen n \textbf{25} ie] Die o \textbf{26} nie] mer m n o \textbf{28} lân] \textit{om.} m \textbf{30} sân] Sin n o  $\cdot$ Gurnemanzes] garnemanczes m gúrnemantz n gurmenancz o \newline
\end{minipage}
\end{table}
\newpage
\begin{table}[ht]
\begin{minipage}[t]{0.5\linewidth}
\small
\begin{center}*G
\end{center}
\begin{tabular}{rl}
 & mac \textbf{nû belîben von dir} vrî.\\ 
 & nû lerne, waz sterben sî!"\\ 
 & "neinâ, \textbf{mærer} d\textit{e}gen balt!\\ 
 & dîn êr \textbf{wirt} sus drîzicvalt\\ 
5 & \textbf{vast} an mir erzeiget,\\ 
 & sît dû mich hâst geneiget.\\ 
 & \textbf{wie} \textbf{mac} dir hœher brîs geschehen?\\ 
 & Condwiramurs \textbf{ma\textit{c}} \textbf{\textit{w}ol} \textbf{sehen},\\ 
 & daz ich der unsælige bin\\ 
10 & unt dîn gelücke hât gewin.\\ 
 & dîn lant ist er\textit{l}œset,\\ 
 & als der sîn schif \textbf{erœset}.\\ 
 & \textbf{daz} \textbf{wirt} vil deste lîhter.\\ 
 & mîn gewalt ist \textbf{worden} sîhter.\\ 
15 & reht manlîch wünne\\ 
 & ist \textbf{worden an mir} dünne.\\ 
 & durch waz \textbf{woltst}û mich sterben?\\ 
 & ich muoz doch laster erben\\ 
 & ûf alle mîne nâchkomen.\\ 
20 & dû hâst den brîs unt den vr0men.\\ 
 & tuost\textbf{û} mir mê, deist ân nôt.\\ 
 & ich trage den lebendegen tôt,\\ 
 & sît ich von \textbf{ir} gescheiden bin,\\ 
 & diu \textbf{mîn} herze und\textit{e} \textit{s}in\\ 
25 & ie mit ir gewalt beslôz,\\ 
 & unt ich des nie gein ir genôz.\\ 
 & des muoz ich unsælic man\\ 
 & ir lîp, ir lant \textbf{ir} ledic lân."\\ 
 & dô \textbf{\textit{ge}dâhte}, der den sic hât,\\ 
30 & sân an Gurnomanzes rât,\\ 
\end{tabular}
\scriptsize
\line(1,0){75} \newline
G I O L M Q R Z \newline
\line(1,0){75} \newline
\textbf{3} \textit{Initiale} O   $\cdot$ \textit{Capitulumzeichen} L  \textbf{11} \textit{Initiale} I  \textbf{21} \textit{Initiale} M Z  \textbf{29} \textit{Initiale} I O L R  \newline
\line(1,0){75} \newline
\textbf{1} nû belîben] beliben nu I wol beliben L  $\cdot$ von] vor O L M (Q) R Z  $\cdot$ vrî] [dry]: vry M \textbf{3} neinâ] ÷æine O  $\cdot$ mærer] kvner L werder Q \textit{om.} R  $\cdot$ degen] dedgen G \textbf{4} drîzicvalt] dritzic alt Q \textbf{5} erzeiget] ertzaget Q geczeiget R \textbf{6} hâst] hat R \textbf{7} wie] Wa O M (Q) R (Z) Nv L \textbf{8} Condwiramurs] kondwiramurs G (Q) Gonwiramurs I Kvndwiramvrs O Z Condwir amvrs L Kond wir Amuͯrs M Kúnd wiramuͦs R  $\cdot$ mac wol] mach nv wol G mach O  $\cdot$ sehen] iehen O (L) (M) (R) Z ichn Q \textbf{11} erlœset] eroset G dir vorboset M \textbf{12} erœset] veroset I (O) oset L vor roset M \textbf{13} daz] Ez O L M (R) Z Er Q  $\cdot$ deste] diche O \textbf{14} gewalt] gewant L  $\cdot$ sîhter] lichter Q \textbf{15} reht] rehtev I  $\cdot$ manlîch] manlicher O  $\cdot$ wünne] [truwe]: wunne M \textbf{16} worden an mir] an mir worden I (Q) \textbf{17} woltstû] soldest tu I \textbf{21} tuostû] Mustu M  $\cdot$ mê] ih me I \textit{om.} M  $\cdot$ deist] dasz Q ist R  $\cdot$ ân] ein L \textbf{22} ich] Won ich R \textbf{24} mîn] mir O L M Q R Z  $\cdot$ unde sin] vnde minen sin G \textbf{26} ich] \textit{om.} I  $\cdot$ nie gein ir] gein ir nie M \textbf{27} des] Daz L (M)  $\cdot$ man] [sin]: man O \textbf{28} lîp ir lant] lant ir lip I  $\cdot$ ir ledic] dir ledic I L M Q (R) Z \textbf{29} dô] ÷a O Da M  $\cdot$ gedâhte] dahte G  $\cdot$ der] der der Q  $\cdot$ hât] da hat I \textbf{30} sân] \textit{om.} O  $\cdot$ Gurnomanzes] kurnemanzes I Gvrnemanzes O Gvrnomantz L gurnemanzis M Gurnomantzes Q Guͦrnamanres R gurnemantzes Z \newline
\end{minipage}
\hspace{0.5cm}
\begin{minipage}[t]{0.5\linewidth}
\small
\begin{center}*T
\end{center}
\begin{tabular}{rl}
 & mac \textbf{vor dir nû blîben} vrî.\\ 
 & nû lerne, waz sterben sî!"\\ 
 & "Neinâ, \textbf{werder} degen balt!\\ 
 & dîn êre \textbf{wirt} sus drîzicvalt\\ 
5 & \textbf{vaste} an mir erzeiget,\\ 
 & sît dû mich hâst geneiget.\\ 
 & \textbf{wâ} \textbf{mac} dir hœher prîs geschehen?\\ 
 & Kundewiramurs, \textbf{diu muoz} \textbf{jehen},\\ 
 & daz ich der unsælige bin\\ 
10 & unde dîn gelücke hât gewin.\\ 
 & dîn lant ist erlœset,\\ 
 & alse der sîn schif \textbf{œset}.\\ 
 & \textbf{ez} \textbf{ist} vil deste lîhter.\\ 
 & mîn gewalt ist sîhter.\\ 
15 & rehte manlîch\textit{iu} wünne\\ 
 & ist \textbf{ammir worden} dünne.\\ 
 & durch waz \textbf{woltest}û mich sterben?\\ 
 & ich muoz doch laster erben\\ 
 & ûf alle mîne nâchkomen.\\ 
20 & dû hâst den prîs unde den vromen.\\ 
 & Tuost \textbf{dû} mir mêr, dêst âne nôt.\\ 
 & ich trage den lebendigen tôt,\\ 
 & sît ich von \textbf{der} gescheiden bin,\\ 
 & diu \textbf{mir} herze unde sin\\ 
25 & ie mit ir gewalt beslôz,\\ 
 & unde ich des nie gegen ir genôz.\\ 
 & des muoz ich unsælic man\\ 
 & Ir lîp, ir lant \textbf{ir} ledic lân."\\ 
 & \begin{large}D\end{large}ô \textbf{gedâhte}, der den sigen \textbf{dâ} hât,\\ 
30 & sân an Gurnemanzes rât,\\ 
\end{tabular}
\scriptsize
\line(1,0){75} \newline
T U V W \newline
\line(1,0){75} \newline
\textbf{3} \textit{Majuskel} T  \textbf{21} \textit{Majuskel} T  \textbf{28} \textit{Majuskel} T  \textbf{29} \textit{Initiale} T U V W  \newline
\line(1,0){75} \newline
\textbf{1} vor dir nû] nun von dir W \textbf{2} waz] was auch W \textbf{4} wirt sus] seint nun W \textbf{5} vaste] Harte W  $\cdot$ erzeiget] v́rzoͤiget V \textbf{7} mac] [*]: moͤhte V  $\cdot$ geschehen] beschehen V \textbf{8} Kundewiramurs] Cvdewiramvrs T Cundewideramuͦrs U Cvndewiramurs V Gundwiramurs W  $\cdot$ diu muoz] [*]: mag wol V muͦß des W \textbf{9} unsælige] vnsigber W \textbf{12} œset] er oset U (V) veroͤset W \textbf{13} ez ist] [E*]: Ez wurt V \textbf{14} mîn] Manig W  $\cdot$ ist] [*]: der ist V \textbf{15} rehte manlîchiu] rehte manliche T Rehter manlicher V \textbf{17} woltestû] soltestu W \textbf{20} den prîs] preiß W \textbf{21} mêr] icht mer W \textbf{23} der] dir U ir W \textbf{25} ir] \textit{om.} W \textbf{28} ir ledic lân] ir ledic san U [*]: dir lidig lan V \textbf{29} Do gedach der den \sout{da} U  $\cdot$ Dô] Da V  $\cdot$ sigen dâ] sig do V W \textbf{30} sân] Zehant V \textit{om.} W  $\cdot$ Gurnemanzes] Gurmanzes U gurnemantzes W  $\cdot$ rât] guͦten rat W \newline
\end{minipage}
\end{table}
\end{document}
