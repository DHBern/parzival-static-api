\documentclass[8pt,a4paper,notitlepage]{article}
\usepackage{fullpage}
\usepackage{ulem}
\usepackage{xltxtra}
\usepackage{datetime}
\renewcommand{\dateseparator}{.}
\dmyyyydate
\usepackage{fancyhdr}
\usepackage{ifthen}
\pagestyle{fancy}
\fancyhf{}
\renewcommand{\headrulewidth}{0pt}
\fancyfoot[L]{\ifthenelse{\value{page}=1}{\today, \currenttime{} Uhr}{}}
\begin{document}
\begin{table}[ht]
\begin{minipage}[t]{0.5\linewidth}
\small
\begin{center}*D
\end{center}
\begin{tabular}{rl}
\textbf{350} & \begin{large}E\end{large}r dâhte: "sol ich \textbf{strîten} sehen\\ 
 & unt sol des niht von mir geschehen,\\ 
 & sô ist al mîn prîs \textbf{verloschen} gar.\\ 
 & kum aber ich \textbf{durch} strîten dar\\ 
5 & unt \textbf{wirde} ich dâ geletzet,\\ 
 & \textbf{mit} wârheit ist entsetzet\\ 
 & al mîn \textbf{werltlîcher} prîs.\\ 
 & i\textbf{ne} tuon \textbf{es} niht \textbf{decheinen gewîs},\\ 
 & ich sol \textbf{ê} leisten mînen kampf."\\ 
10 & sîn nôt sich in ein ander klampf.\\ 
 & gegen \textbf{sînes} kampfes verte\\ 
 & was \textbf{ze} belîben al ze herte.\\ 
 & er\textbf{n} moht \textbf{ôt} dô niht vür gevarn.\\ 
 & er sprach: "nû müeze got bewarn\\ 
15 & die kraft an mîner manheit."\\ 
 & Gawan gein Bearosche reit.\\ 
 & burg unt stat sô vor im lac,\\ 
 & daz niemen bezzers hûses pflac.\\ 
 & \textbf{ouch} gleste gein im schône\\ 
20 & aller ander bürge eine krône\\ 
 & mit t\textit{ü}r\textit{n}en wol gezieret.\\ 
 & nû was geloschieret\\ 
 & \textbf{dem her} dar vür ûf den plân.\\ 
 & \textbf{dô} \textbf{marhte} mîn hêr Gawan\\ 
25 & manegen rinc wol gehêret.\\ 
 & dâ was hôchvart gemêret.\\ 
 & wunderlîcher baniere\\ 
 & kôs er dâ manege schiere\\ 
 & unt maneger slahte \textbf{vremden} bovel.\\ 
30 & der zwîvel was sînes herzen hovel,\\ 
\end{tabular}
\scriptsize
\line(1,0){75} \newline
D \newline
\line(1,0){75} \newline
\textbf{1} \textit{Initiale} D  \newline
\line(1,0){75} \newline
\textbf{16} Bearosche] Bearosce D \textbf{21} türnen] trvren D \textbf{23} vür] fvͦr D \newline
\end{minipage}
\hspace{0.5cm}
\begin{minipage}[t]{0.5\linewidth}
\small
\begin{center}*m
\end{center}
\begin{tabular}{rl}
 & er dâhte: "sol ich \textbf{strîte} sehen\\ 
 & und sol des niht von mi\textit{r g}eschehen,\\ 
 & sô ist al mîn prîs \textbf{verloschen} gar.\\ 
 & kume aber ich \textbf{von} strîten dar\\ 
5 & und \textbf{wirde} ich d\textit{â} geletzet,\\ 
 & \textbf{mit} wârheit ist entsetzet\\ 
 & a\textit{l} mîn \textbf{werltlîcher} prîs.\\ 
 & ich tuon \textbf{es} niht \textbf{dekeine wîs},\\ 
 & ich sol \textbf{ê} leisten mînen kampf."\\ 
10 & sîn nôt sich in ein ander klampf.\\ 
 & gegen \textbf{sînes} kampfes verte\\ 
 & was belîben a\textit{l} ze herte.\\ 
 & e\textit{r} \textbf{en}m\textit{o}ht \textbf{eht} dô niht vür gevarn.\\ 
 & er sprach: "nû müeze got bewarn\\ 
15 & die kraft an mîner manheit."\\ 
 & Gawan gegen Bearosche reit.\\ 
 & burc und stat sô vor im lac,\\ 
 & daz niemen bezzers hûses pflac,\\ 
 & \textbf{wanne} \textbf{d\textit{â}} gleste gegen im schône\\ 
20 & aller ander bürge ein krône\\ 
 & mit türnen wol gezieret.\\ 
 & nû was geloschieret\\ 
 & \textbf{dem her} dâ vür ûf den plân.\\ 
 & \textbf{dô} \textbf{marhte} mîn hêrre Gawan\\ 
25 & manigen rinc wol gehêret.\\ 
 & d\textit{â} was hôchvart gemêret.\\ 
 & wunderlîcher baniere\\ 
 & kôs er dâ manige schiere\\ 
 & und maniger slahte \textbf{vr\textit{ö}mede} povel.\\ 
30 & der zwîvel was sînes herzen hovel,\\ 
\end{tabular}
\scriptsize
\line(1,0){75} \newline
m n o \newline
\line(1,0){75} \newline
\newline
\line(1,0){75} \newline
\textbf{1} dâhte] gedochte n (o)  $\cdot$ strîte] striten n o \textbf{2} mir geschehen] mir nit geschehen m mir beschehen o \textbf{4} von] durch n o \textbf{5} wirde] wurde n (o)  $\cdot$ dâ] do m n o \textbf{7} al] Alle m \textbf{8} dekeine] do keine n \textbf{9} ê] \textit{om.} n \textbf{10} in] \textit{om.} o \textbf{12} al] alle m \textbf{13} er] Es m  $\cdot$ enmoht eht] enmoͯht eht m moͯchte n o  $\cdot$ dô] da o \textbf{14} müeze] muse m misse o \textbf{15} mîner] sẏner o \textbf{16} Bearosche] bearosce m n [arasce]: Bearasce o \textbf{19} dâ] do m n o \textbf{22} geloschieret] gelustieret n o \textbf{23} vür] vor n o  $\cdot$ den] dem n o \textbf{26} dâ] Do m n o \textbf{28} dâ] do n o  $\cdot$ manige] manigen o \textbf{29} vrömede] framede m  $\cdot$ povel] pauel m punel n [slahte]: punel o \textbf{30} hovel] honel n hunel o \newline
\end{minipage}
\end{table}
\newpage
\begin{table}[ht]
\begin{minipage}[t]{0.5\linewidth}
\small
\begin{center}*G
\end{center}
\begin{tabular}{rl}
 & er dâhte: "sol ich \textbf{strîten} sehen\\ 
 & unde sol des niht von mir geschehen,\\ 
 & sôst al mîn prîs \textbf{erloschen} gar.\\ 
 & kum aber ich \textbf{durch} strîten dar\\ 
5 & unde \textbf{wirde} ich dâ geletzet,\\ 
 & \textbf{mit} wârheit ist entsetzet\\ 
 & al mîn \textbf{werltlîcher} brîs.\\ 
 & ich \textbf{en}tuon \textit{\textbf{ez}} niht \textbf{neheine wîs},\\ 
 & ich sol \textbf{ê} leisten mînen kampf."\\ 
10 & sîn nôt sich in ein ander klampf.\\ 
 & gein \textbf{sîner} kampfes verte\\ 
 & was belîben al ze herte.\\ 
 & er moht \textbf{ouch} dâ niht vür gevaren.\\ 
 & er sprach: "nû müeze got bewaren\\ 
15 & die kraft an mîner manheit."\\ 
 & Gawan gein Bearotsche reit.\\ 
 & burc unde \textit{stat} sô vor im lac,\\ 
 & daz niemen bezzers hûses pflac.\\ 
 & \textbf{ouch} glaste gein im schône\\ 
20 & aller anderen bürge ein krône\\ 
 & mit tür\textit{n}en wol gezieret.\\ 
 & nû was gelotschieret\\ 
 & \textbf{dem her} dar vür ûf den plân.\\ 
 & \textbf{dâ} \textbf{sach} mîn hêr Gawan\\ 
25 & manigen rinc wol gehêret.\\ 
 & dâ was hôchvart gemêret.\\ 
 & wunderlîcher baniere\\ 
 & kôs er dâ manige schiere\\ 
 & \begin{large}U\end{large}nde maniger slahte \textbf{vrömden} povel.\\ 
30 & der zwîvel was sînes herzen hovel,\\ 
\end{tabular}
\scriptsize
\line(1,0){75} \newline
G I O L M Q R Z Fr39 Fr40 \newline
\line(1,0){75} \newline
\textbf{1} \textit{Initiale} I L R Z Fr39 Fr40  \textbf{17} \textit{Initiale} I  \textbf{29} \textit{Initiale} G  \newline
\line(1,0){75} \newline
\textbf{1} dâhte] gedacht R  $\cdot$ sol] solde I (R)  $\cdot$ strîten] strite R \textbf{2} sol] \textit{om.} Q \textbf{3} al] \textit{om.} M Q R Fr40 \textbf{4} aber ich] ich aber L  $\cdot$ strîten] strit L (R) \textbf{5} wirde] wir I wurd Q R (Fr39)  $\cdot$ dâ] danne O do Q Fr39 \textbf{7} werltlîcher] werdechlicher I (O) (R) (Fr40) ritterlicher L Fr39 werdiger Q \textbf{8} entuon ez] entoͮn sin G tuͤn ez I entvͯns L (M) (Q) (Fr39) tuͦn es R  $\cdot$ neheine] deheinen O (Q) (Z) Fr40 \textbf{10} \textit{Vers 350.10 fehlt} Q R   $\cdot$ klampf] kramph M \textbf{11} gein] Ein R  $\cdot$ sîner] sines I (Q) Z \textbf{13} er] ern I (O) (L) (M) (Q) (Z) (Fr39)  $\cdot$ dâ] do Q Fr39  $\cdot$ gevaren] varn I (Q) (R) \textbf{14} sprach] \textit{om.} L Fr39 \textbf{16} Bearotsche] beatrotsche I Baerotsh L bearoshe Q R bearotsch Fr39 \textbf{17} stat] hus G \textbf{18} hûses] huse R \textbf{19} glaste] glesten O (L) M (Fr39)  $\cdot$ gein] vor O Q R \textbf{20} ein] \textit{om.} O Z \textbf{21} türnen] turen G trúwen R \textbf{22} gelotschieret] Geloisiert I \textbf{23} dem] von dem I  $\cdot$ dar vür ûf] vur vf I dar uff fur M dem fur vff Q vff R  $\cdot$ den] di O dē M dem R \textbf{24} dâ] daz I Do L M Q R (Fr39) \textbf{25} manigen rinc] Maniger rit Q \textbf{26} dâ] Do O Q R (Fr39)  $\cdot$ gemêret] wol gemeret M \textbf{27} wunderlîcher] Wunderliche R (Z) \textbf{28} dâ] do O Q R Fr39  $\cdot$ manige] manger O mannigen M manches Q \textbf{29} slahte] slahter O hande M (Z)  $\cdot$ vrömden] vreuden I fromder O fremde M \textit{om.} R \textbf{30} sînes] sin M  $\cdot$ herzen] \textit{om.} L ::: Fr39  $\cdot$ hovel] [pouel]: houel R \newline
\end{minipage}
\hspace{0.5cm}
\begin{minipage}[t]{0.5\linewidth}
\small
\begin{center}*T
\end{center}
\begin{tabular}{rl}
 & Er dâhte: "sol ich \textbf{strîten} sehen\\ 
 & unde sol des niht von mir geschehen,\\ 
 & sôst almîn prîs \textbf{erloschen} gar.\\ 
 & kum aber ich \textbf{durch} strîten dar\\ 
5 & unde \textbf{würd}ich dâ geletzet,\\ 
 & \textbf{mî\textit{n}} wârheit ist en\textit{ts}e\textit{t}zet\\ 
 & \textbf{unde} almîn \textbf{werdeclîcher} prîs.\\ 
 & i\textbf{n} tuon \textbf{sîn} niht \textbf{deheine wîs},\\ 
 & ich sol leisten mînen kam\textit{pf}."\\ 
10 & sîn nôt sich in ein ander klampf.\\ 
 & gegen \textbf{sînes} kampfe\textit{s} verte\\ 
 & \textit{was blîben alze herte.}\\ 
 & er\textbf{n} mohte \textbf{ouch} dâ niht vür gevarn.\\ 
 & er sprach: "nû müeze got bewarn\\ 
15 & die kraft an mîner manheit."\\ 
 & Gawan gegen Bearosche reit.\\ 
 & Burc unde stat sô vor im lac,\\ 
 & daz nieman bezzers hûses pflac.\\ 
 & \textbf{ouch} gleste gegen im schône\\ 
20 & aller andern bürge ein krône\\ 
 & mit türnen wol gezieret.\\ 
 & nû was \textbf{dem her} gelotschieret\\ 
 & dar vür ûf den plân.\\ 
 & \textbf{nû} \textbf{sach} mîn hêr Gawan\\ 
25 & manegen rinc wol gehêret.\\ 
 & dâ was hôchvart gemêret.\\ 
 & wunderlîcher baniere\\ 
 & kôs er dâ manege schiere\\ 
 & unde maneger slahte \textbf{vremde} povel.\\ 
30 & der zwîvel was sînes herzen hovel,\\ 
\end{tabular}
\scriptsize
\line(1,0){75} \newline
T V W \newline
\line(1,0){75} \newline
\textbf{1} \textit{Initiale} W   $\cdot$ \textit{Majuskel} T  \textbf{17} \textit{Initiale} V   $\cdot$ \textit{Majuskel} T  \newline
\line(1,0){75} \newline
\textbf{1} dâhte] gedachte W \textbf{2} des] es W \textbf{3} erloschen] verloͤschet V \textbf{4} aber ich] [*]: aber V \textbf{5} dâ] do V W \textbf{6} mîn] mit T  $\cdot$ entsetzet] en zezet T \textbf{7} almîn] gar mein W \textbf{8} in] Ich W \textbf{9} leisten] e leisten V  $\cdot$ kampf] kamfp T \textbf{10} sich] \textit{om.} W  $\cdot$ klampf] krampff W \textbf{11} sînes] seiner W  $\cdot$ kampfes] kampfef T \textbf{12} \textit{Vers 350.12 fehlt (Zeile ausgespart)} T  \textbf{13} ern mohte ouch dâ] Er moͤhte eht do V Er moͤcht auch do W  $\cdot$ niht vür] fúr nit W \textbf{15} an] \textit{om.} V \textbf{16} Bearosche] Bearotsce T (V) betrosch W \textbf{17} vor] verre vor W \textbf{20} andern] \textit{om.} W \textbf{22} Nun was gelaizieret W \textbf{23} dar vür] Her fúr V Dem her do fuͦr W \textbf{24} nû] Do V W \textbf{26} dâ] Do V Das W \textbf{28} dâ] do V W \textbf{29} vremde] \textit{om.} W \textbf{30} sînes] sein W  $\cdot$ hovel] [*ovel]: hovel V zouel W \newline
\end{minipage}
\end{table}
\end{document}
