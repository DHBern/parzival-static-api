\documentclass[8pt,a4paper,notitlepage]{article}
\usepackage{fullpage}
\usepackage{ulem}
\usepackage{xltxtra}
\usepackage{datetime}
\renewcommand{\dateseparator}{.}
\dmyyyydate
\usepackage{fancyhdr}
\usepackage{ifthen}
\pagestyle{fancy}
\fancyhf{}
\renewcommand{\headrulewidth}{0pt}
\fancyfoot[L]{\ifthenelse{\value{page}=1}{\today, \currenttime{} Uhr}{}}
\begin{document}
\begin{table}[ht]
\begin{minipage}[t]{0.5\linewidth}
\small
\begin{center}*D
\end{center}
\begin{tabular}{rl}
\textbf{808} & \begin{large}N\end{large}û \textbf{diz} was \textbf{êt} âne strît,\\ 
 & daz hôrte oder spræche ze \textbf{keiner} zît\\ 
 & iemen von schœnerem wîbe.\\ 
 & si \textbf{truoc} ouch an ir lîbe\\ 
5 & pfellel, den ein künstec hant\\ 
 & worhte, als in Sarant\\ 
 & mit \textbf{grôzem liste} erdâht ê\\ 
 & in der stat ze Thasme.\\ 
 & Feirefiz Anschevin\\ 
10 & si brâhte, \textbf{diu} gap liehten schîn,\\ 
 & mitten durch den palas.\\ 
 & driu grôziu viwer gemachet was,\\ 
 & lign âlôe des viwers smac.\\ 
 & vierzec teppeche unt \textbf{gesitze mêr} dâ lac\\ 
15 & denne \textbf{zeiner zît}, dô Parzival\\ 
 & ouch dâ vür \textbf{sach tragen} den Grâl.\\ 
 & Ein gesiz vor ûz gehêrt was,\\ 
 & dâ Feirefiz unt Anfortas\\ 
 & bî dem wirte \textbf{solde} sitzen.\\ 
20 & Dô warp mit \textbf{zühte} witzen,\\ 
 & swer dâ dienen wolde,\\ 
 & sô der Grâl komen solde.\\ 
 & ir habt gehôrt \textbf{ê} des genuoc,\\ 
 & wie man in vür Anfortasen truoc.\\ 
25 & dem siht man nû gelîche tuon\\ 
 & vür des werden Gahmuretes sun\\ 
 & unt ouch vür Tampenteires kint.\\ 
 & Juncvrouwen nû niht langer sint.\\ 
 & ordenlîche \textbf{si} kômen überal,\\ 
30 & vünf unt zweinzec an der zal.\\ 
\end{tabular}
\scriptsize
\line(1,0){75} \newline
D \newline
\line(1,0){75} \newline
\textbf{1} \textit{Initiale} D  \textbf{17} \textit{Majuskel} D  \textbf{20} \textit{Majuskel} D  \textbf{28} \textit{Majuskel} D  \newline
\line(1,0){75} \newline
\textbf{9} Anschevin] Anscivin D \textbf{15} Parzival] Parcifal D \textbf{26} Gahmuretes] Gahmvrets D \newline
\end{minipage}
\hspace{0.5cm}
\begin{minipage}[t]{0.5\linewidth}
\small
\begin{center}*m
\end{center}
\begin{tabular}{rl}
 & nû \textbf{da\textit{z}} \textit{w}as \textbf{eht} âne strît,\\ 
 & daz hôrte oder spræche zuo \textbf{keiner} zît\\ 
 & ieman von schœnerem wîbe.\\ 
 & si \textbf{truoc} ouch an ir lîbe\\ 
5 & pfelle, den ein künsti\textit{c h}ant\\ 
 & worht, als \textit{i}n Sa\textit{r}ant\\ 
 & mit \textbf{grôzem list} erdâht ê\\ 
 & in der stat zuo Thasine.\\ 
 & Ferefiz A\textit{n}schevin\\ 
10 & si brâhte, \textbf{si} gap liehten schîn,\\ 
 & mitten durch den palas.\\ 
 & driu grôziu viur gemachet was,\\ 
 & lingnum \textit{â}l\textit{ô}e des viures smac.\\ 
 & vierzic teppich und \textbf{gesitz mê} d\textit{â} lac\\ 
15 & dan \textbf{zuo einer zît}, dô Parcifal\\ 
 & ouch d\textit{â} vür \textbf{sach tragen} den Grâl.\\ 
 & ein gesiz vor ûz gehêret was,\\ 
 & d\textit{â} Ferefiz und Anfortas\\ 
 & bî dem wirte \textbf{solte} sitzen.\\ 
20 & dô warp mit \textbf{zühtigen} witzen,\\ 
 & wer d\textit{â} dienen wolte,\\ 
 & sô der Grâl komen solte.\\ 
 & ir habt \textbf{her} geh\textit{ô}rt \textbf{ê} des genuoc,\\ 
 & wie man in vür Anfortasen truoc.\\ 
25 & dem sih\textit{t} man nû gl\textit{îch}e tuon\\ 
 & vür des werden Gahmuretes sun\\ 
 & und ouch v\textit{ür} Tampenter\textit{i}es kint.\\ 
 & juncvrowen nû niht langer sint.\\ 
 & ordenlîch \textbf{si} kômen überal,\\ 
30 & vünf und zweinzic an der \textit{z}al.\\ 
\end{tabular}
\scriptsize
\line(1,0){75} \newline
m n V V' W \newline
\line(1,0){75} \newline
\newline
\line(1,0){75} \newline
\textbf{1} \textit{Die Verse 808.1-10 fehlen} V'   $\cdot$ daz was] das es was m n \textbf{2} spræche] sprach W \textbf{5} künstic hant] kunstig man vnd hant m kúnfftig hant W \textbf{6} in Sarant] ein sariant m in sariant n ein sarant W \textbf{7} erdâht] als endacht n \textbf{8} Thasine] Tasme V thasme W \textbf{9} Ferefiz] Ferefis m V Ferrefis n Ferafis W  $\cdot$ Anschevin] auscevin m n anschefin V antscheuein W \textbf{10} brâhte] brehete W  $\cdot$ si gap] die gab V vnd gab W  $\cdot$ liehten] lihten V \textbf{11} \textit{Versfolge 808.11, 18 (¹n), 19-20, 13-17, 18 (²n), 21} n   $\cdot$ Vnd furten sie in den pallas wit \textit{(Fortsetzung von 807.30)} V' \textbf{12} \textit{Vers 808.12 fehlt} n   $\cdot$ \textit{Die Verse 808.12-816.05 fehlen} V'  \textbf{13} âlôe] olee m  $\cdot$ smac] geschmag W \textbf{14} mê] \textit{om.} W  $\cdot$ dâ] do m n V W \textbf{15} dô] \textit{om.} n  $\cdot$ Parcifal] Parzefal V herr partzifal W \textbf{16} dâ] do m n V W \textbf{17} gesiz] sitz n (W) gesizte V  $\cdot$ vor ûz] vorhin W \textbf{18} \textit{Versdoppelung (²n); Lesarten des vorausgehenden Verses mit ¹n bezeichnet} n   $\cdot$ dâ] Do m n V W  $\cdot$ Ferefiz] Ferefis m ferrefis \textsuperscript{1}\hspace{-1.3mm} n ferrevis \textsuperscript{2}\hspace{-1.3mm} n artus ferefis V ferafis W \textbf{19} solte] soltent V (W) \textbf{20} warp] [war*]: wart V ward W  $\cdot$ zühtigen] zúhten V (W) \textbf{21} wer] Swer V Geordent wer W  $\cdot$ dâ] do m n V W \textbf{23} her] \textit{om.} n V W  $\cdot$ gehôrt] gehert m \textbf{24} Anfortasen] anfortassen n V anfortaßen W \textbf{25} siht man] sihttaman m  $\cdot$ glîche] gluͯge m glúcke n \textbf{26} Gahmuretes] gamurettes m gamiretes n gammurehtez V gamuretes W \textbf{27} vür] von m n  $\cdot$ Tampenteries] Tamppenteres m tampenteires n artusen vnd tampentherez V tampenteirs W \textbf{30} zweinzic] zwentzil n  $\cdot$ zal] hal m \newline
\end{minipage}
\end{table}
\newpage
\begin{table}[ht]
\begin{minipage}[t]{0.5\linewidth}
\small
\begin{center}*G
\end{center}
\begin{tabular}{rl}
 & \begin{large}N\end{large}û \textbf{ditze} was \textbf{êt} ân strît,\\ 
 & daz hôrt ode spr\textit{æ}ch ze \textbf{deheiner} zît\\ 
 & iemen von schœnerem wîbe.\\ 
 & si \textbf{het} ouch an ir lîbe\\ 
5 & \textbf{einen} pfelle, den ein künstec hant\\ 
 & worhte, als in Sarant\\ 
 & mit \textbf{grôzen listen} e\textit{r}dâht ê\\ 
 & in der stat ze Tasme.\\ 
 & Feirafiz Antschevin\\ 
10 & si brâht, \textbf{diu} gap liehten schîn,\\ 
 & enmitten durch den palas.\\ 
 & driu grôziu viwer gemachet was,\\ 
 & lign âlôe des viwers smac.\\ 
 & vierzic tepch unde \textbf{mê} dâ lac\\ 
15 & danne \textbf{zeinen zîten}, dô Parzival\\ 
 & ouch dâ vür \textbf{sach tragen} den Grâl.\\ 
 & ein gesitze vor ûz gehêrt was,\\ 
 & dâ Feirafiz unde Anfortas\\ 
 & bî dem wirte \textbf{solden} sitzen.\\ 
20 & dô warp mit \textbf{zühte} witzen,\\ 
 & swer dâ dienen wolde,\\ 
 & sô der Grâl komen solde.\\ 
 & ir habet gehôrt des genuoc,\\ 
 & wie ma\textit{n} in \textit{vür} Anfortasen truoc.\\ 
25 & dem siht man nû gelîche tuon\\ 
 & vür des werden Gahmuretes sun\\ 
 & unde ouch vür Tampunteires \textit{ki}n\textit{t}.\\ 
 & juncvrouwen nû niht lenger sint.\\ 
 & ordenlîch \textbf{die} kômen überal,\\ 
30 & vünf unde zweinzic an der zal.\\ 
\end{tabular}
\scriptsize
\line(1,0){75} \newline
G I L Z \newline
\line(1,0){75} \newline
\textbf{1} \textit{Initiale} G I Z  \textbf{21} \textit{Initiale} I  \newline
\line(1,0){75} \newline
\textbf{2} spræch] sprach G sehe L \textbf{3} schœnerem] [schon*]: schonrem G \textbf{5} künstec] chundic I \textbf{6} in Sarant] ein Sariant L \textbf{7} erdâht] endaht G \textbf{8} Tasme] Tasine L Thasme Z \textbf{9} Feirafiz] feiraviz G Ferefis L Feirefiz Z  $\cdot$ Antschevin] anschoͮwin G entseuin I Anshevin L (Z) \textbf{10} gap] gabe L  $\cdot$ liehten] lychten L \textbf{12} driu grôziu] Daz grosze L Groz Z \textbf{13} lign âlôe] lingalwe G \textbf{14} unde] oder Z  $\cdot$ dâ lac] gesitzes man da phlac I gesitz da lach L \textbf{15} zeinen] zuͯ eine L  $\cdot$ dô] da Z  $\cdot$ Parzival] parzifal I L parcifal Z \textbf{16} dâ vür sach tragen] vur tragen sach I sach fvͤr tragen da Z \textbf{17} gesitze] Gestuͤle I gesesz L \textbf{18} Feirafiz] feiraviz G ferefiz L feirefiz Z  $\cdot$ Anfortas] Amfortas L \textbf{20} zühte] zuhten I (L) \textbf{21} swer] Wer L \textbf{24} man in vür] mangen wis G man vur I  $\cdot$ Anfortasen] anfortassen I Z Amfortaszen L \textbf{26} Gahmuretes] gahmoͮrets G gamureten Z \textbf{27} Tampunteires] tampvteirs G Tanpuntaires I Tampvnteres L  $\cdot$ kint] svn G \textbf{29} ordenlîch] Edenlichen Z \textbf{30} zal] shar I \newline
\end{minipage}
\hspace{0.5cm}
\begin{minipage}[t]{0.5\linewidth}
\small
\begin{center}*T
\end{center}
\begin{tabular}{rl}
 & \begin{large}N\end{large}û \textbf{diz} was âne strît,\\ 
 & daz hôrte oder spræche zuo \textbf{der} zît\\ 
 & iema\textit{n} von schœnereme wîbe.\\ 
 & si \textbf{hete} ouch an ir lîbe\\ 
5 & \textbf{einen} pfelle, den ein künstigiu hant\\ 
 & worhte, als in Sarant\\ 
 & mit \textbf{grôzem liste} erdâhte ê\\ 
 & in der stat zuo Tasme.\\ 
 & Ferefis Anschevin\\ 
10 & si brâhte, \textbf{diu} gap liehten schîn,\\ 
 & enmitten durch den palas.\\ 
 & driu grôziu viur gemachet was,\\ 
 & lign âlôe des viures smac.\\ 
 & vierzic teppich und \textbf{mê gesitze} dâ lac\\ 
15 & dan \textbf{zuo einen zîten}, dô Parcifal\\ 
 & ouch d\textit{â} vür \textbf{tragen sach} den Grâl.\\ 
 & ein gesitze \textbf{d\textit{â}} vor ûz gehêrt was,\\ 
 & d\textit{â} Ferefis und Anfortas\\ 
 & bî dem wirte \textbf{solten} sitzen.\\ 
20 & dô war\textit{p} mit \textbf{zühten} witzen,\\ 
 & wer d\textit{â} dienen wolte,\\ 
 & sô der Grâl komen solte.\\ 
 & ir hât gehôret des genuoc,\\ 
 & wie man in vür Anfortassen truoc.\\ 
25 & dem siht man nû glîche tuon\\ 
 & vür des werden Gahmuretes suon\\ 
 & und ouch vür Tampunteres kint.\\ 
 & juncvrouwen nû niht langer sint.\\ 
 & ordenlîche \textbf{die} kâmen überal,\\ 
30 & vünfe und zwênzic \textbf{kâmen} an der zal.\\ 
\end{tabular}
\scriptsize
\line(1,0){75} \newline
U Q R \newline
\line(1,0){75} \newline
\textbf{1} \textit{Initiale} U R  \newline
\line(1,0){75} \newline
\textbf{1} was] wasz auch Q was echt R \textbf{3} ieman] Jemam U  $\cdot$ schœnereme wîbe] schonren wiben R \textbf{8} der] einer R  $\cdot$ Tasme] thasme U Thasine R \textbf{9} Ferefis] feirefisz Q Feirefis R  $\cdot$ Anschevin] anschovin U anscheuͯin Q anshevin R \textbf{10} brâhte] brachten Q  $\cdot$ liehten] lichten Q \textbf{12} grôziu] grosze R \textbf{13} lign âlôe] [Lingnal*]: Lingnaloe U \textbf{15} Denne zen zitten do parczifal da was R  $\cdot$ Parcifal] Parzifal U partzifal Q \textbf{16} dâ] do U Q \textit{om.} R  $\cdot$ vür tragen sach] sach tragen Q sach fúrtragen R \textbf{17} Ein gesitz auch do wasz Q  $\cdot$ gesitze] gesitte R  $\cdot$ dâ] do U \textbf{18} dâ] Do U Q R  $\cdot$ Ferefis] feirefisz Q feriefis R \textbf{19} solten] solte Q \textbf{20} warp] wart U (R) \textbf{21} dâ] do U Q das R \textbf{22} sô] Do Q  $\cdot$ komen solte] dienen [wolte]: solte R \textbf{23} gehôret] \textit{om.} R \textbf{24} in] \textit{om.} R  $\cdot$ Anfortassen] Antifotassen R \textbf{26} werden] werde R  $\cdot$ Gahmuretes] Gahmuͦretes U gamúretes Q Gahmurtes R \textbf{27} Tampunteres] tempuͦnteires U tampuͯntirs Q tampuͦnterres R \textbf{30} kâmen] \textit{om.} Q R \newline
\end{minipage}
\end{table}
\end{document}
