\documentclass[8pt,a4paper,notitlepage]{article}
\usepackage{fullpage}
\usepackage{ulem}
\usepackage{xltxtra}
\usepackage{datetime}
\renewcommand{\dateseparator}{.}
\dmyyyydate
\usepackage{fancyhdr}
\usepackage{ifthen}
\pagestyle{fancy}
\fancyhf{}
\renewcommand{\headrulewidth}{0pt}
\fancyfoot[L]{\ifthenelse{\value{page}=1}{\today, \currenttime{} Uhr}{}}
\begin{document}
\begin{table}[ht]
\begin{minipage}[t]{0.5\linewidth}
\small
\begin{center}*D
\end{center}
\begin{tabular}{rl}
\textbf{543} & Vor gote ich bin vervluochet,\\ 
 & mînes prîses \textbf{er nimer ruochet}.\\ 
 & durch Orgelusen minne,\\ 
 & der edelen herzoginne,\\ 
5 & muose mir manec werder man\\ 
 & \textbf{sînen} prîs ze mînen handen lân.\\ 
 & dû maht vil prîses erben,\\ 
 & ob dû mich kanst ersterben."\\ 
 & Dô dâhte \textbf{des} \textbf{künec} Lotes sun:\\ 
10 & "deiswâr, i\textbf{ne} sol alsô niht tuon.\\ 
 & sô verlür ich prîses hulde,\\ 
 & erslüege ich âne schulde\\ 
 & disen küenen helt unverzagt.\\ 
 & \textbf{ich hân} ir minne ûf mich \textbf{gejagt},\\ 
15 & der \textbf{minne} \textbf{mich ouch} twinget\\ 
 & unt mir vil kumbers bringet.\\ 
 & Wan lâze ich in durch si genesen?\\ 
 & ob mîn teil an ir sol wesen,\\ 
 & \textbf{des} \textbf{en}mag er niht erwenden,\\ 
20 & \textbf{sol} mirz gelücke senden.\\ 
 & \textbf{wære unser strît von ir gesehen},\\ 
 & ich wæne, si müese \textbf{ouch} mir des jehen,\\ 
 & daz ich nâch minnen dienen kan."\\ 
 & \textbf{Dô} sprach \textbf{mîn} hêr Gawan:\\ 
25 & "ich wil durch die herzogîn\\ 
 & dich bî \textbf{dem} \textbf{leben} lâzen sîn."\\ 
 & grôzer müede si \textbf{niht} vergâzen.\\ 
 & er liez in ûf; si sâzen\\ 
 & von ein ander verre.\\ 
30 & dô kom des schiffes hêrre\\ 
\end{tabular}
\scriptsize
\line(1,0){75} \newline
D \newline
\line(1,0){75} \newline
\textbf{1} \textit{Majuskel} D  \textbf{9} \textit{Majuskel} D  \textbf{17} \textit{Majuskel} D  \textbf{24} \textit{Majuskel} D  \newline
\line(1,0){75} \newline
\textbf{9} Lotes] Lots D \newline
\end{minipage}
\hspace{0.5cm}
\begin{minipage}[t]{0.5\linewidth}
\small
\begin{center}*m
\end{center}
\begin{tabular}{rl}
 & vor got ic\textit{h b}in vervluochet,\\ 
 & mînes prîses \textbf{ist niht geruochet}.\\ 
 & durch Urgeluse minne,\\ 
 & der edeln herzoginne,\\ 
5 & muos mir manic werder man\\ 
 & \textbf{den} prîs zuo mî\textit{n}en handen lân.\\ 
 & dû maht vil prîses e\textit{r}ben,\\ 
 & ob dû mich k\textit{a}nst ersterben."\\ 
 & \begin{large}D\end{large}ô dâhte \textbf{es} \textbf{künic} Lo\textit{te}s sun:\\ 
10 & "daz ist wâr, ich sol alsô niht tuon.\\ 
 & sô verlürich prîses hulde,\\ 
 & erslüege ich âne schulde\\ 
 & disen küenen helt unverzaget.\\ 
 & \textbf{in het} ir minne ûf mich \textbf{\textit{g}ejaget},\\ 
15 & der \textbf{minne} \textbf{mich ouch} twinget\\ 
 & und mir vil kumbers bringet.\\ 
 & wan lâz ich in durch si genesen?\\ 
 & ob mîn teil an ir sol wesen,\\ 
 & \textbf{des} \textbf{en}mac er niht erwenden,\\ 
20 & \textbf{sol} mirz glücke senden.\\ 
 & \textbf{wær unser strît vor ir geschehen},\\ 
 & ich wæne, si mües \textbf{ouch} mir des jehen,\\ 
 & daz ich nâch minnen dienen kan."\\ 
 & \textbf{hie mit} sprach hêr Gawan:\\ 
25 & "ich wil durch die herzogîn\\ 
 & dich bî \textbf{leben} lâzen sîn."\\ 
 & grôzer müede si \textbf{nie} vergâzen.\\ 
 & er liez in ûf; si sâzen\\ 
 & von ein ander verre.\\ 
30 & dô kam des schiffes hêrre\\ 
\end{tabular}
\scriptsize
\line(1,0){75} \newline
m n o \newline
\line(1,0){75} \newline
\textbf{9} \textit{Initiale} m   $\cdot$ \textit{Capitulumzeichen} n  \newline
\line(1,0){75} \newline
\textbf{1} ich bin] ich bich bin m  $\cdot$ vervluochet] versluͯchet n werslichet o \textbf{3} Urgeluse] vrgluse o \textbf{4} herzoginne] herczoginen o \textbf{5} muos] Muͦsz n \textbf{6} mînen] mẏnnen m  $\cdot$ lân] han o \textbf{7} erben] erheben m \textbf{8} kanst] kunst m \textbf{9} Lotes] lois m n lots o \textbf{10} alsô] es n \textbf{13} unverzaget] [verczaget]: vnverczaget o \textbf{14} in] Er o  $\cdot$ gejaget] beiaget m \textbf{17} si] in o \textbf{19} erwenden] enwenden o \textbf{20} \textit{Versdoppelung (²m); Lesarten der vorausgehenden Verse mit ¹m bezeichnet} m   $\cdot$ senden] erwenden \textsuperscript{1}\hspace{-1.3mm} m \textbf{21} unser] vns o  $\cdot$ geschehen] beschehen o \textbf{22} mües] muͯsse n  $\cdot$ ouch mir] mir ouch n auch mirs o \textbf{24} hêr Gawan] hergawan m \textbf{26} leben] lehen n  $\cdot$ sîn] son o \textbf{27} nie] nit n o \newline
\end{minipage}
\end{table}
\newpage
\begin{table}[ht]
\begin{minipage}[t]{0.5\linewidth}
\small
\begin{center}*G
\end{center}
\begin{tabular}{rl}
 & \textit{\begin{large}V\end{large}}or got ich bin vervluochet,\\ 
 & mînes brîses \textbf{er nimmer enruochet}.\\ 
 & durch Orgelusen minne,\\ 
 & der edeln herzoginne,\\ 
5 & muos mir manic \textit{wert} man\\ 
 & \textbf{sînen} brîs ze mînen handen lân.\\ 
 & dû maht vil brîses erben,\\ 
 & ob dû mich kanst ersterben."\\ 
 & dô dâht \textbf{des} \textbf{künic} Lotes sun:\\ 
10 & "dêswâr, ich\textbf{n} sol alsô niht tuon.\\ 
 & sô verlüre ich brîses hulde,\\ 
 & erslüege ich âne schulde\\ 
 & disen küenen helt unverzaget.\\ 
 & \textbf{in hât} ir minne ûf mich \textbf{gejaget},\\ 
15 & der \textbf{minne} \textbf{ouch mich} twinget\\ 
 & unde mir vil kumber\textit{s} bringet.\\ 
 & wan lâze ich in durch si genesen?\\ 
 & op mîn teil an ir sol wesen,\\ 
 & \textbf{des} mag er niht erwenden,\\ 
20 & \textbf{sol} mirz gelücke \textit{s}enden.\\ 
 & \textbf{wære unser strît von ir gesehen},\\ 
 & ich wæne, si müese \textbf{\textit{ou}ch} mir \textit{des j}ehen,\\ 
 & daz ich nâch minne dienen kan."\\ 
 & \textbf{dô} sprach \textbf{mîn} hêrre Gawan:\\ 
25 & "ich wil durch die herzogîn\\ 
 & dich bî \textbf{dem} \textbf{leben} lâzen sîn."\\ 
 & grôzer müede si \textbf{niht} vergâzen.\\ 
 & er liez in ûf; si sâzen\\ 
 & von ein ander verre.\\ 
30 & dô kom des scheffes hêrre\\ 
\end{tabular}
\scriptsize
\line(1,0){75} \newline
G I L M Z \newline
\line(1,0){75} \newline
\textbf{1} \textit{Initiale} G I L Z  \textbf{21} \textit{Initiale} I  \newline
\line(1,0){75} \newline
\textbf{1} Vor] Dor G \textbf{2} nimmer] niht L  $\cdot$ enruochet] ruͤchet I (M) (Z) \textbf{3} Orgelusen] Orgulusen I Orgelisen L orgilusin M \textbf{5} manic wert] manic G werdic manniger M \textbf{6} sînen brîs] Eines prises L Sines prises M \textbf{7} erben] erwerben L (M) Z \textbf{8} kanst ersterben] kast er kerben Z \textbf{9} dô] Da M  $\cdot$ dâht] gedachte M \textbf{10} dêswâr] Zwar Z  $\cdot$ ichn] ich I L  $\cdot$ alsô] so L \textbf{13} küenen] \textit{om.} I \textbf{14} ir] in L  $\cdot$ gejaget] veriagt I iagt Z \textbf{15} ouch mich] auch mich da I mich ouch L M Z  $\cdot$ twinget] betwinget I \textbf{16} vil] vil vil I  $\cdot$ kumbers] chumbir G \textbf{17} in] \textit{om.} M \textbf{18} teil] hail I \textbf{19} des] Der M  $\cdot$ mag] en mach L (Z)  $\cdot$ niht] mich I  $\cdot$ erwenden] erwinden L \textbf{21} von] vor I L  $\cdot$ gesehen] geshehen I (L) \textbf{22} müese] must I Z  $\cdot$ ouch mir des jehen] doch mir uiriehen G mir ouch dez ýehen L \textbf{23} daz] Dach M  $\cdot$ minne] minnen I (M)  $\cdot$ dienen] \textit{om.} Z \textbf{24} dô] Da M Z  $\cdot$ hêrre Gawan] ergawan M \textbf{25} wil] wil dich I \textbf{26} dich] \textit{om.} I  $\cdot$ dem] dime I \textit{om.} L  $\cdot$ leben] libe I \textbf{28} in] [si]: in G \textbf{30} dô] Da M \newline
\end{minipage}
\hspace{0.5cm}
\begin{minipage}[t]{0.5\linewidth}
\small
\begin{center}*T
\end{center}
\begin{tabular}{rl}
 & vor gote ich bin vervluochet,\\ 
 & mînes prîses \textbf{er niht mêr ruochet}.\\ 
 & durch Orgelusen minne,\\ 
 & der edeln herzoginne,\\ 
5 & muose mir manec wert man\\ 
 & \textbf{sînen} prîs zuo mînen henden lân.\\ 
 & dû maht vil prîses erben,\\ 
 & ob dû mich kanst ersterben."\\ 
 & \textit{\begin{large}D\end{large}}ô dâhte \textbf{des} \textbf{küneges} Lotes suon:\\ 
10 & "deiswâr, i\textbf{n} sol alsô niht tuon.\\ 
 & sô verlürich prîses hulde,\\ 
 & erslüege ich âne schulde\\ 
 & disen küenen helt unverzaget.\\ 
 & \textbf{in hât} ir minne ûf mich \textbf{verjaget},\\ 
15 & der \textbf{mich} \textbf{dâ} twinget\\ 
 & unde mir vil kumbers bringet.\\ 
 & wan lâz ich in durch si genesen?\\ 
 & ob mîn teil an ir sol wesen,\\ 
 & \textbf{daz} \textbf{en}mac er niht erwenden,\\ 
20 & \textbf{wil} mirz gelücke senden.\\ 
 & \textbf{wande het si unsern strît gesehen},\\ 
 & ich wæne, si müe\textit{s}e mir des jehen,\\ 
 & daz ich nâch minnen dienen kan."\\ 
 & \textbf{Dô} sprach \textbf{mîn} hêr Gawan:\\ 
25 & "ich wil durch die herzogîn\\ 
 & dich bî \textbf{dem} \textbf{lîbe} lâzen sîn."\\ 
 & grôzer müede si \textbf{niht} vergâzen.\\ 
 & er liez in ûf; si sâzen\\ 
 & von ein ander verre.\\ 
30 & Dô kom des schiffes hêrre\\ 
\end{tabular}
\scriptsize
\line(1,0){75} \newline
T U V W O Q R Fr40 \newline
\line(1,0){75} \newline
\textbf{9} \textit{Initiale} T U V  \textbf{24} \textit{Majuskel} T  \textbf{30} \textit{Majuskel} T  \newline
\line(1,0){75} \newline
\textbf{1} vor] ÷or O  $\cdot$ ich bin] bin ich R \textbf{2} niht mêr] nv́me V  $\cdot$ ruochet] geruͦchet U W \textbf{3} Orgelusen] orelusen Q orgusolen R \textbf{4} edeln herzoginne] edel kúnginne R \textbf{7} vil prîses] wol pris R  $\cdot$ erben] erwerben W O R Fr40 \textbf{8} kanst] kost Q macht R \textbf{9} Dô] ÷o T  $\cdot$ dâhte] gedacht W  $\cdot$ küneges] kúnig W (O) (R) \textbf{10} deiswâr in sol] Entzwar ich Q deswar ich sol Fr40  $\cdot$ alsô] so O Q \textbf{11} verlürich] verliesz ich Q \textbf{12} ich] och R \textbf{13} küenen] \textit{om.} O \textbf{14} \textit{Versfolge 543.15-14} R   $\cdot$ in] Er U  $\cdot$ verjaget] geiaget U V (W) (O) R (Fr40) beiagt Q \textbf{15} der] Der minne U V W O Q R (Fr40)  $\cdot$ dâ] oͮch V do W \textbf{16} vil] \textit{om.} O  $\cdot$ kumbers] chvmber O (Q) \textbf{17} in] sie Q \textbf{19} daz] Des W  $\cdot$ enmac] mag V Fr40 \textbf{20} mirz] [miz]: mirz T mir Q \textbf{21} \textit{Versfolge 543.22-21} W   $\cdot$ wande] \textit{om.} W Wa R \textbf{22} müese] mveze T (U) muͤß W mvͦse O (Q) muͯsz R  $\cdot$ mir des] oͮch mir dez V (O) (Fr40) mir auch des W mir des auch Q \textbf{23} nâch] \textit{om.} Fr40  $\cdot$ minnen] minne U W O Q R Fr40 \textbf{24} Dô] Da O \textbf{25} die] dein Q \textbf{26} dem] \textit{om.} Q  $\cdot$ lîbe] leben W O Q R Fr40 \textbf{27} grôzer] Groß W \newline
\end{minipage}
\end{table}
\end{document}
