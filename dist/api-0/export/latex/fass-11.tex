\documentclass[8pt,a4paper,notitlepage]{article}
\usepackage{fullpage}
\usepackage{ulem}
\usepackage{xltxtra}
\usepackage{datetime}
\renewcommand{\dateseparator}{.}
\dmyyyydate
\usepackage{fancyhdr}
\usepackage{ifthen}
\pagestyle{fancy}
\fancyhf{}
\renewcommand{\headrulewidth}{0pt}
\fancyfoot[L]{\ifthenelse{\value{page}=1}{\today, \currenttime{} Uhr}{}}
\begin{document}
\begin{table}[ht]
\begin{minipage}[t]{0.5\linewidth}
\small
\begin{center}*D
\end{center}
\begin{tabular}{rl}
\textbf{11} & \textit{\begin{large}D\end{large}}ô sprach der junge Anschevin:\\ 
 & "got trôste iuch, \textbf{vrowe}, des vater mîn.\\ 
 & den suln wir \textbf{beidiu gerne} \textbf{klagen}.\\ 
 & \textbf{iu} \textbf{en}mac \textbf{nieman von mir} gesagen\\ 
5 & \textbf{deheiniu klagelîchiu} leit.\\ 
 & ich var durch mîne werdecheit\\ 
 & \textbf{nâch} ritterschaft in vremdiu lant.\\ 
 & vrowe, \textbf{ez ist sus} \textbf{bewant}."\\ 
 & Dô sprach diu küneginne:\\ 
10 & "sît dû \textbf{nâch} hôher minne\\ 
 & wendest dienest und muot,\\ 
 & lieber sun, lâ dir mîn guot\\ 
 & ûf die vart niht versmâhen.\\ 
 & heiz von mir enpfâhen\\ 
15 & \textbf{dîne} kamerære\\ 
 & vier soumschrîn swære.\\ 
 & dâ \textbf{ligent inne} pfelle breit,\\ 
 & ganze, die man nie \textbf{versneit},\\ 
 & und manec tiwer samît.\\ 
20 & süezer man, lâ mich die zît\\ 
 & \textbf{hœren}, \textbf{wenne} dû wider kumest.\\ 
 & an mînen vröuden dû mir vrumest."\\ 
 & "Vrowe, des enweiz ich niht,\\ 
 & in \textbf{welhem lande} man mich siht.\\ 
25 & \textbf{wan swar} ich von iu kêre,\\ 
 & ir habt nâch ritters êre\\ 
 & iwer werdecheit an mir \textbf{getân}.\\ 
 & ouch hât mich der künic \textbf{lân},\\ 
 & als im mîn dienest danken sol.\\ 
30 & \textbf{ich getrûwe iu des vil} wol,\\ 
\end{tabular}
\scriptsize
\line(1,0){75} \newline
D \newline
\line(1,0){75} \newline
\textbf{1} \textit{Initiale} D  \textbf{9} \textit{Versal} D  \textbf{23} \textit{Versal} D  \newline
\line(1,0){75} \newline
\textbf{1} Dô] ÷o \textit{nachträglich korrigiert zu:} Do D  $\cdot$ Anschevin] Ansoiuin D \newline
\end{minipage}
\hspace{0.5cm}
\begin{minipage}[t]{0.5\linewidth}
\small
\begin{center}*m
\end{center}
\begin{tabular}{rl}
 & dô sprach der junge A\textit{n}schevin:\\ 
 & "got trôste iuch, \textbf{vrowe}, des vater mîn.\\ 
 & de\textit{n} sullen wir \textbf{beidiu gerne} \textbf{klagen}.\\ 
 & \textbf{nû} mac \textbf{von mir niemen} gesagen\\ 
5 & \textbf{keiniu klegelîch\textit{iu}} \textit{l}eit.\\ 
 & ich vare durch mîne wirdicheit\\ 
 & \textbf{nâch} ritterschaft in vrömdiu lant.\\ 
 & vrowe, \textbf{alsô ist ez mir} \textbf{gewant}."\\ 
 & \textit{\begin{large}D\end{large}}ô sprach diu küniginne:\\ 
10 & "sît dû \textbf{von} hôher minne\\ 
 & wendest dienst und muot,\\ 
 & lieber sun, lâ dir mîn guot\\ 
 & ûf die vart niht vers\textit{mâ}hen.\\ 
 & heiz von mir enpfâhen\\ 
15 & \textbf{dînen} kamerære\\ 
 & vier soumschrîn swære.\\ 
 & dâ \textbf{ligent inne} pfeller breit,\\ 
 & ganz, die man nie \textbf{sneit},\\ 
 & und menic tiure samît.\\ 
20 & süezer man, lâ mich die zît\\ 
 & \textbf{gehœren}, \textbf{wenne} dû wider kumest.\\ 
 & an mîne\textit{n} vr\textit{öu}den dû mir vrumest."\\ 
 & "vrowe, des enweiz ich niht,\\ 
 & in \textbf{welichen landen} man mich siht.\\ 
25 & \textbf{wenne zwâr} ich von iu kêre,\\ 
 & ir habt nâch ritters êre\\ 
 & iuwer wirdicheit an mir \textbf{getân}.\\ 
 & ouch het mich der künic \textbf{lân},\\ 
 & als im mîn dienest danken sol.\\ 
30 & \textbf{ich ge\textit{t}rûwe iu des vil} wol,\\ 
\end{tabular}
\scriptsize
\line(1,0){75} \newline
m n o \newline
\line(1,0){75} \newline
\textbf{9} \textit{Illustration mit Überschrift:} Wie gahmuret begabet wart von der konigin m  Also gamiret von der konnigin begobet wart n (o)   $\cdot$ \textit{Großinitiale} n   $\cdot$ \textit{Initiale} m o  \newline
\line(1,0){75} \newline
\textbf{1} Anschevin] ausceuin \textit{nachträglich korrigiert zu:} ansceuin m auscenin n ansceẏm o \textbf{2} \textit{Vers 11.2 fehlt} o   $\cdot$ Jch muͦsz dir min clage duͦn schin n \textbf{3} den] Der \textit{nachträglich korrigiert zu:} Den m Der n o  $\cdot$ sullen wir beidiu] sullent [wider]: wir beide \textit{nachträglich korrigiert zu:} sullent beide bruder m \textbf{5} Keine kleckliche klag und leit m · Kein clegeliches leit n (o) \textbf{9} Dô] DDo m \textbf{13} versmâhen] versch*hen \textit{nachträglich korrigiert zu:} verschmohen m \textbf{17} dâ] Do n o \textbf{18} sneit] versneit n (o) \textbf{19} tiure] thor o \textbf{22} an] An \textit{nachträglich korrigiert zu:} Alle m  $\cdot$ mînen] mÿne m  $\cdot$ vröuden] fr::den \textit{nachträglich korrigiert zu:} freude m frúnden n (o)  $\cdot$ vrumest] fruͯndest o \textbf{24} man mich] [mẏn]: man sich o \textbf{25} kêre] keren o \textbf{26} êre] here: o \textbf{27} getân] gestan o \textbf{28} ouch] Vnd ouch n  $\cdot$ lân] gelan n \textbf{29} dienest] dieste o \textbf{30} getrûwe] geruwe \textit{nachträglich korrigiert zu:} getruwe m  $\cdot$ iu] jme n \newline
\end{minipage}
\end{table}
\newpage
\begin{table}[ht]
\begin{minipage}[t]{0.5\linewidth}
\small
\begin{center}*G
\end{center}
\begin{tabular}{rl}
 & \begin{large}D\end{large}ô sprach der junge Antschevin:\\ 
 & "got trôste iuch, \textbf{vrouwe}, des vater mîn.\\ 
 & den sulen wir \textbf{beidiu gerne} \textbf{klagen}.\\ 
 & \textbf{iu}\textbf{ne} mac \textbf{niemen niht} gesagen\\ 
5 & \textbf{von mir} \textbf{dehein klegelîch} leit.\\ 
 & ich var durch mîne werdicheit\\ 
 & \textbf{durch} rîterschaft in vrömdiu lant.\\ 
 & vrouwe, \textbf{ez ist mir sus} \textbf{gewant}."\\ 
 & dô sprach diu küniginne:\\ 
10 & "sît dû \textbf{nâch} hôher minne\\ 
 & wendest dienst und muot,\\ 
 & lieber sun, lâ dir mîn guot\\ 
 & ûf die vart niht versmâhen.\\ 
 & heiz von mir enpfâhen\\ 
15 & \textbf{dîne} kamerære\\ 
 & vier soumschrîn swære.\\ 
 & dâr \textbf{inne ligent} pfelle breit,\\ 
 & ganz, die man nie \textbf{versneit},\\ 
 & unde manic tiure samît.\\ 
20 & süezer man, lâ mich die zît\\ 
 & \textbf{hœren}, \textbf{wenne} dû wider kumest.\\ 
 & an mînen vröuden dû mir vrumest."\\ 
 & "vrouwe, desne weiz ich niht,\\ 
 & in \textbf{welhem lande} man mich siht.\\ 
25 & \textbf{wan swar} ich von iu kêre,\\ 
 & ir habet nâch rîters êre\\ 
 & iwer werdicheit an mir \textbf{begân}.\\ 
 & ouch hât mich der künic \textbf{lân},\\ 
 & als im mîn dienst danken sol.\\ 
30 & \textbf{ouch wil ich iu getriuwen} wol,\\ 
\end{tabular}
\scriptsize
\line(1,0){75} \newline
G O L M Q R W Z Fr29 Fr32 \newline
\line(1,0){75} \newline
\textbf{1} \textit{Initiale} G L M Q W Z Fr29 Fr32  \textbf{13} \textit{Versal} Fr32  \textbf{20} \textit{Capitulumzeichen} R  \newline
\line(1,0){75} \newline
\textbf{1} Dô] Da Z ÷o Fr29  $\cdot$ Antschevin] anschevin G anshevin O (L) Z (Fr29) Fr32 anschefyn M anshevin \textit{nachträglich korrigiert zu:} anshewin Q antscheuin W \textbf{2} trôste] ergetz W  $\cdot$ vrouwe] \textit{om.} L \textbf{3} beidiu gerne] alle gerne W gerne beidiv Fr32 \textbf{4} iune] Vch M (Fr29) Jch Z  $\cdot$ mac] kan W  $\cdot$ niht] von mîr O (L) (M) (Q) (W) (Z) (Fr29) (Fr32)  $\cdot$ gesagen] sagen Q \textbf{5} von mir] \textit{om.} O L M Q W Z Fr29 Fr32 \textbf{7} durch] Nach O L Q W Z Fr29 (Fr32) Nahe M \textbf{8} sus] alsuß W \textbf{9} dô] Da M Z \textbf{10} sît] Sist M  $\cdot$ dû] div Fr32  $\cdot$ nâch] durch W \textbf{11} dienst und] dyn ost myn M \textbf{13} ûf die] Zuͯ der L \textbf{15} \textit{Versfolge 11.16-15} L W   $\cdot$ dîne] Dinen O (Q) Z Fr29 (Fr32) \textbf{16} soumschrîn] schone schryn M \textbf{17} dâr inne ligent] Dor inne liget Q Do ligen inne W \textbf{18} man nie] nie man Z \textbf{20} mich] [mi*]: mir O \textbf{21} hœren] Gehoren O L (M) (Q) (R) Z Fr29 (Fr32) Wissen W  $\cdot$ dû] div Fr32 \textbf{22} an] Alle Z \textbf{24} welhem lande] wilden landen M welchen lande Q  $\cdot$ man mich siht] mich man sicbt W \textbf{25} swar] swanne O (Fr32) war L Q R \textit{om.} W  $\cdot$ von iu] nun hinnan W \textbf{27} begân] getan O (L) (M) (Q) (R) W Z Fr29 Fr32 \textbf{28} lân] gelan L (M) (R) W Fr32 \textbf{30} Jch getrowe iv des vil wol O (L) (M) (Q) (R) (Z) (Fr29) (Fr32)  $\cdot$ Frauwe ich trauwe euch des wol W \newline
\end{minipage}
\hspace{0.5cm}
\begin{minipage}[t]{0.5\linewidth}
\small
\begin{center}*T
\end{center}
\begin{tabular}{rl}
 & \begin{large}D\end{large}ô sprach der junge Anschevin:\\ 
 & "got t\textit{r}ôste iuch des vater mîn.\\ 
 & den sul wir \textbf{gerne beid\textit{iu}} \textbf{tragen}.\\ 
 & \textbf{iu}\textbf{ne} mac \textbf{nieman von mir} gesagen\\ 
5 & \textbf{dehein\textit{iu} klagelîchen} leit.\\ 
 & ich var durch mîne werdecheit\\ 
 & \textbf{nâch} rîterschefte in vremdiu lant.\\ 
 & vrouwe, \textbf{ez ist mir sus} \textbf{gewant}."\\ 
 & Dô sprach diu küneginne:\\ 
10 & "sît dû \textbf{nâch} hôher minne\\ 
 & wendest dienst und muot,\\ 
 & lieber sun, lâ dir mîn guot\\ 
 & ûf die vart niht versmâhen.\\ 
 & heiz von mir enpfâhen\\ 
15 & \textbf{dînen} kamerære\\ 
 & vier soumschrîne swære.\\ 
 & dâr \textbf{inne ligent} pfelle breit,\\ 
 & ganze, die man nie \textbf{versneit},\\ 
 & und manec tiure samît.\\ 
20 & süezer man, lâ mich die zît\\ 
 & \textbf{hœren}, \textbf{daz} dû wider komest.\\ 
 & an mînen vröuden dû mir vromest."\\ 
 & "Vrouwe, des enweiz ich niht,\\ 
 & in \textbf{welhem lande} man mich siht.\\ 
25 & \textbf{wan swar} ich von iu kêre,\\ 
 & ir habt nâch rîters êre\\ 
 & iuwer werdecheit an mir \textbf{getân}.\\ 
 & ouch hât mich der künec \textbf{gelân},\\ 
 & als im\textbf{s} mîn dienst danken sol.\\ 
30 & \textbf{deswâr, ich getriuwe} wol,\\ 
\end{tabular}
\scriptsize
\line(1,0){75} \newline
T U V \newline
\line(1,0){75} \newline
\textbf{1} \textit{Initiale} T U V  \textbf{9} \textit{Majuskel} T  \textbf{23} \textit{Majuskel} T  \newline
\line(1,0){75} \newline
\textbf{1} Anschevin] anscheuin U (V) \textbf{2} trôste] t:oste T [*]: ergetze V  $\cdot$ iuch] îv T (U) v́ch vrowe V \textbf{3} gerne beidiu tragen] gerne beide tragen T beide gerne clagen V \textbf{4} iune mac] Jz in mac U \textbf{5} klagelîchen] clegelich U (V) \textbf{13} die] dine V \textbf{16} soumschrîne] schone schrine U \textbf{17} breit] bereit U \textbf{21} [*]: Wússen wenne dv her wider komest V \textbf{25} swar] war U \textbf{29} ims] im U \newline
\end{minipage}
\end{table}
\end{document}
