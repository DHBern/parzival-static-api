\documentclass[8pt,a4paper,notitlepage]{article}
\usepackage{fullpage}
\usepackage{ulem}
\usepackage{xltxtra}
\usepackage{datetime}
\renewcommand{\dateseparator}{.}
\dmyyyydate
\usepackage{fancyhdr}
\usepackage{ifthen}
\pagestyle{fancy}
\fancyhf{}
\renewcommand{\headrulewidth}{0pt}
\fancyfoot[L]{\ifthenelse{\value{page}=1}{\today, \currenttime{} Uhr}{}}
\begin{document}
\begin{table}[ht]
\begin{minipage}[t]{0.5\linewidth}
\small
\begin{center}*D
\end{center}
\begin{tabular}{rl}
\textbf{115} & beide ir \textbf{gebære} unt ir site.\\ 
 & swelhem \textbf{wîbe} volget kiusche mite,\\ 
 & der lobes kempfe wil ich sîn.\\ 
 & mir ist von herzen leit ir pîn.\\ 
5 & \textbf{\begin{large}E\end{large}in} lop hinket \textbf{ame} spat,\\ 
 & \textbf{swer} allen \textbf{vrouwen} sprichet mat\\ 
 & durch sîn \textbf{eines} vrouwen.\\ 
 & swelhiu mîn reht wil schouwen,\\ 
 & beidiu sehen und hœren,\\ 
10 & die\textbf{ne} \textbf{sol} ich niht betœren.\\ 
 & schildes ambet ist mîn art.\\ 
 & swâ mîn ellen \textbf{sî} gespart,\\ 
 & swelhiu mich minnet umbe \textbf{sanc},\\ 
 & \textbf{sô} dunket mich ir \textbf{witze} kranc.\\ 
15 & Ob ich \textbf{guotes} wîbes minne ger,\\ 
 & \textbf{mag} ich mit schilde unt \textbf{ouch} mit sper\\ 
 & verdienen niht ir minne solt,\\ 
 & al dar nâch sî si mir holt.\\ 
 & vil hôhes topels er doch spilt,\\ 
20 & \textbf{der an} ritterschaft nâch minnen zilt.\\ 
 & Hetenz wîp niht vür ein \textbf{smeichen},\\ 
 & ich \textbf{solt} \textbf{iu} \textbf{vürbaz} reichen\\ 
 & an \textbf{disem} mære unkundiu wort.\\ 
 & ich \textbf{spreche} \textbf{iu} \textbf{d}âventiure vort.\\ 
25 & Swer des von mir geruoche,\\ 
 & der \textbf{en}zel \textbf{si} ze keine\textit{m} buoche.\\ 
 & i\textbf{ne} kan decheinen buochstap.\\ 
 & dâ nement genuoge ir urhap.\\ 
 & disiu âventiure\\ 
30 & vert âne der buoche stiure.\\ 
\end{tabular}
\scriptsize
\line(1,0){75} \newline
D Fr33 \newline
\line(1,0){75} \newline
\textbf{5} \textit{Initiale} D Fr33  \textbf{15} \textit{Majuskel} D  \textbf{21} \textit{Majuskel} D  \textbf{25} \textit{Majuskel} D  \newline
\line(1,0){75} \newline
\textbf{1} J: geberde vnd site Fr33 \textbf{5} Ein] Sin Fr33 \textbf{6} vrouwen] w::: Fr33 \textbf{11} schildes] Schilte Fr33 \textbf{26} ze keinem] zecheinen D \newline
\end{minipage}
\hspace{0.5cm}
\begin{minipage}[t]{0.5\linewidth}
\small
\begin{center}*m
\end{center}
\begin{tabular}{rl}
 & beide ir \textbf{gebærde} und ir site.\\ 
 & welchem \textbf{lîbe} volge\textit{t} \textit{k}iusche mite,\\ 
 & der lobes kempfe wil ich sîn.\\ 
 & mir ist von herzen leit ir pîn.\\ 
5 & \textbf{\begin{large}S\end{large}în} lop hinket \textbf{an eime} spat,\\ 
 & \textbf{wer} allen \textbf{vrouwen} sprichet mat\\ 
 & durch sîn \textbf{eine} vrouwen.\\ 
 & welichiu mîn reht wil schouwen,\\ 
 & bêdiu sehen und hœren,\\ 
10 & die \textbf{en}\textbf{sol} ich niht betœren.\\ 
 & schiltes ambet i\textit{st} mîn \textit{ar}t.\\ 
 & wâ mîn ellen \textbf{sîn} gespart,\\ 
 & welichiu mich minnet umb \textbf{gesanc},\\ 
 & \textbf{sô} dunket mich ir \textbf{witze} kranc.\\ 
15 & ob ich \textbf{guotes} wîbes minne ger,\\ 
 & \textbf{mac} ich mit schilte und mit sper\\ 
 & verdienen niht ir minne solt,\\ 
 & al dar nâch sî si mir holt.\\ 
 & vil hôhes toppels er doch spilt,\\ 
20 & \textbf{der an} ritterschaft nâch minnen zilt.\\ 
 & \textit{h}etenz wîp niht vür ein \textbf{smêhen},\\ 
 & ich \textbf{solte} \textbf{iu} \textbf{vil baz} \dag reiten\dag \\ 
 & an \textbf{disem} mære unkundiu wort.\\ 
 & ich \textbf{sprich} \textbf{iu} \textbf{die} âventiure vort.\\ 
25 & wer des von mir geruoche,\\ 
 & der \textbf{en}zel \textbf{sich} zuo keinem buoche.\\ 
 & \textit{i}\textbf{\textit{n}e} kan dekeine\textit{n} buochstap.\\ 
 & d\textit{â} \textit{n}ement genuoge ir urhap.\\ 
 & disiu âventiure\\ 
30 & vert ân der buoche stiure.\\ 
\end{tabular}
\scriptsize
\line(1,0){75} \newline
m n o \newline
\line(1,0){75} \newline
\textbf{5} \textit{Initiale} m n  \newline
\line(1,0){75} \newline
\textbf{1} gebærde] geberge o \textbf{2} lîbe] wibe n o  $\cdot$ volget kiusche] volget volget kusche m \textbf{3} der lobes kempfe] Des lobes kempfft o \textbf{9} bêdiu] Beide sie o  $\cdot$ hœren] herren o \textbf{10} betœren] bekoͯren n \textbf{11} ist] ich m  $\cdot$ art] rat m \textbf{12} wâ] Wo er o  $\cdot$ sîn] ist n o \textbf{16} mit schilte] nit schilt o  $\cdot$ und] vnd ouch n \textbf{17} ir minne] der mynnen n (o) \textbf{18} nâch sî si] sie noch o \textbf{19} doch] do n \textbf{20} an] ander o \textbf{21} hetenz] Netentz m \textbf{22} vil baz] fúrbas n (o)  $\cdot$ reiten] frehen n (o) \textbf{23} unkundiu wort] vnkunde wart n \textbf{24} iu] \textit{om.} o  $\cdot$ vort] fart n \textbf{26} enzel] zale n (o) \textbf{27} ine] Me m Jch n o  $\cdot$ dekeinen] de keinem m doch keinen n  $\cdot$ buochstap] buͦch [stage]: stabe m buͦchstaben n \textbf{28} dâ] Do m n o  $\cdot$ nement] niement m  $\cdot$ urhap] vrhaben n nehabe o \textbf{30} stiure] fúr n \newline
\end{minipage}
\end{table}
\newpage
\begin{table}[ht]
\begin{minipage}[t]{0.5\linewidth}
\small
\begin{center}*G
\end{center}
\begin{tabular}{rl}
 & beidiu ir \textbf{gebærde} und ir site.\\ 
 & swelhem \textbf{wîbe} volget kiusche mite,\\ 
 & der lobes kempfe wil ich sîn.\\ 
 & mir ist von herzen leit ir pîn.\\ 
5 & \textbf{sîn} lop hinket \textbf{ame} spat,\\ 
 & \textbf{der} allen \textbf{wîben} sprichet mat\\ 
 & durch sîn \textbf{eines} vrouwen.\\ 
 & swelhiu mîn reht wil schouwen,\\ 
 & beidiu sehen und hœren,\\ 
10 & die \textbf{wil} ich niht betœren.\\ 
 & schiltes ambet ist mîn art.\\ 
 & swâ mîn ellen \textbf{sî} gespart,\\ 
 & swelhiu mich minnet umbe \textbf{sanc},\\ 
 & \textbf{diu} dûchet mich ir \textbf{witze} kranc.\\ 
15 & \begin{large}O\end{large}be ich \textbf{werdes} wîbes minne ger,\\ 
 & \textbf{muge} ich mit schilt und mit sper\\ 
 & verdienen niht ir minnen solt,\\ 
 & al dar nâch sî \textit{si} mir holt.\\ 
 & vil hôhes topeles er doch spilt,\\ 
20 & \textbf{der mit} rîterschaft nâch minnen zilt.\\ 
 & hetenz \textbf{diu} wîp niht vür ein \textbf{smeichen},\\ 
 & ich \textbf{wolt} \textbf{vürbaz} reichen\\ 
 & an \textbf{disem} mære unkundiu wort,\\ 
 & ich \textbf{spræche} \textbf{die} âventiure vort.\\ 
25 & swer \textit{d}es von mir geruoche,\\ 
 & der zel \textbf{si} ze deheinem buoche,\\ 
 & \textbf{wan} ich kan deheinen buochstap.\\ 
 & dâ nement genuoge ir urhap.\\ 
 & disiu âventiure\\ 
30 & vert âne der buoche stiure.\\ 
\end{tabular}
\scriptsize
\line(1,0){75} \newline
G I O L M Q R Z \newline
\line(1,0){75} \newline
\textbf{1} \textit{Initiale} O  \textbf{5} \textit{Initiale} L Q R Z  \textbf{13} \textit{Initiale} I  \textbf{15} \textit{Initiale} G  \newline
\line(1,0){75} \newline
\textbf{1} beidiu] ÷eidiv O \textit{om.} M  $\cdot$ gebærde] berde Z \textbf{2} swelhem] Welchem L (M) Q (R)  $\cdot$ volget] folge Q  $\cdot$ kiusche] [chussche]: chiusche G chusse I kuscheit M \textbf{3} der] Des Q  $\cdot$ lobes] lobe L  $\cdot$ kempfe] komppe Q  $\cdot$ ich] \textit{om.} O \textbf{4} von herzen leit] von herzen I leid von herczen R  $\cdot$ ir] \textit{om.} R \textbf{5} ame] ane O Z an im R \textbf{6} der] swer I (O) (Z) Wer L Q R  $\cdot$ allen] alden I  $\cdot$ wîben] frawen O (L) (M) (Q) (R) (Z) \textbf{7} \textit{Vers 115.7 fehlt} R  \textbf{8} swelhiu] Welche L (M) (Q) (R)  $\cdot$ mîn] mit O Z nun Q \textbf{9} sehen] wil sehen O schawen Q \textbf{10} die] din I (M) (R) (Z) Den en L Den Q  $\cdot$ ich] \textit{om.} O  $\cdot$ betœren] toͯrren R \textbf{12} swâ] Wie L Wa M (Q) R  $\cdot$ ellen] eren Q  $\cdot$ sî] sin Z \textbf{13} swelhiu] Welche L (Q) (R)  $\cdot$ minnet] nympt Q  $\cdot$ umbe] durch O (L) (Q) R  $\cdot$ sanc] gesanch O (L) (Q) \textbf{14} Der witze dvncket mich vil krang L  $\cdot$ diu] So Z  $\cdot$ dûchet] dukt R  $\cdot$ ir witze] an wizzen I an wize O (R) \textbf{15} wîbes] \textit{om.} R \textbf{16} muge] Manege M  $\cdot$ mit schilt] mich schilde Q \textbf{17} verdienen] Verdienet Z  $\cdot$ niht] \textit{om.} M  $\cdot$ minnen] minne I (L) (M) (Q) (R) \textbf{18} sî si] si G si sei O \textbf{19} er doch] erda R \textbf{20} der mit] swer mit I (O) (Z) [Mit]: Wer mit  L Wer M Q (R)  $\cdot$ rîterschaft] riter O  $\cdot$ minnen] minne I O (L) Q (R) Z \textbf{21} niht] \textit{om.} M  $\cdot$ ein] \textit{om.} L R  $\cdot$ smeichen] smechin M schmachen R \textbf{22} wolt] wolde iv O (M) (R) wol in Q wol ev Z \textbf{23} disem] diser L  $\cdot$ unkundiu] vnkvndie L (M) (R) \textbf{24} ich] e I Ja R  $\cdot$ spræche] sprach iv O sprechen uͯch L (M) (Q) sprich R spriche ev Z  $\cdot$ die] der R  $\cdot$ vort] vorht I \textbf{25} swer] Wer L M Q R  $\cdot$ des] es G \textit{om.} R  $\cdot$ von] vor I an M  $\cdot$ geruoche] des geruͦche R \textbf{26} der] dern I (Q) (Z)  $\cdot$ zel] zceil M ze R  $\cdot$ si] ez L  $\cdot$ ze] \textit{om.} Q \textbf{27} ich] ichn I (L) (M) (Q)  $\cdot$ buochstap] [bubtab]: bustab Q buͦchstaben R \textbf{28} dâ] Do Q  $\cdot$ nement] nemit M (Q) (R)  $\cdot$ urhap] vrhaben R \textbf{30} vert âne] Vort vnd M  $\cdot$ buoche] buͦchen R \newline
\end{minipage}
\hspace{0.5cm}
\begin{minipage}[t]{0.5\linewidth}
\small
\begin{center}*T (U)
\end{center}
\begin{tabular}{rl}
 & beidir \textbf{gebærde} und ir site.\\ 
 & welchem \textbf{wîbe} volget kiusche mite,\\ 
 & der lobe\textit{s} kempfe wil ich sîn.\\ 
 & mir ist von herzen leit ir pîn.\\ 
5 & \textbf{\begin{large}S\end{large}în} lop hinket \textbf{anm\textit{e}} spat,\\ 
 & \textbf{wer} allen \textbf{vrouwen} sprichet mat\\ 
 & durch sîn \textbf{einege} vrouwen.\\ 
 & welchiu mîn reht wil schouwen,\\ 
 & beidiu sehen und hœren,\\ 
10 & die\textbf{n} \textbf{wil} ich niht \textit{be}tœren.\\ 
 & schiltes ambet ist mîn art.\\ 
 & wâ mîn ellen \textbf{sî} gespart,\\ 
 & welhiu mich minnet umb \textbf{sanc},\\ 
 & \textbf{diu} dunket mich ir \textbf{minne} kranc.\\ 
15 & ob ich \textbf{werdes} wîbes minne ger,\\ 
 & \textbf{muge} ich mit schilte und mit sper\\ 
 & verdienen niht ir minnen solt,\\ 
 & al dar nâch \textbf{sô} sî si mir holt.\\ 
 & vil hôhes topels er doch spilt,\\ 
20 & \textbf{wer mit} ritterschaft nâch minnen zilt.\\ 
 & hetenz \textbf{diu} wîp niht vür ein \textbf{smeichen},\\ 
 & ich \textbf{wolte} \textbf{vürbaz} reichen\\ 
 & an \textbf{dise} mære unkundiu wort.\\ 
 & ich \textbf{spreche} \textbf{iu} \textbf{dise} âventiure vort.\\ 
25 & wer des von mir geruoche,\\ 
 & der\textbf{n} zel\textbf{s} zuo dekeime buoche,\\ 
 & \textbf{wan} ich \textbf{en}kan dekeinen buochstap.\\ 
 & dâ nement genuoge ir urhap.\\ 
 & disiu âventiure\\ 
30 & v\textit{e}rt âne der büecher stiure.\\ 
\end{tabular}
\scriptsize
\line(1,0){75} \newline
U V W T \newline
\line(1,0){75} \newline
\textbf{2} \textit{Majuskel} T  \textbf{5} \textit{Initiale} U V W T  \textbf{21} \textit{Majuskel} T  \textbf{25} \textit{Majuskel} T  \newline
\line(1,0){75} \newline
\textbf{2} welchem] Swelhem T \textbf{3} der] des T  $\cdot$ lobes] lobe U  $\cdot$ kempfe] kampfe W \textbf{5} hinket] doch hinket V  $\cdot$ anme] an mir U im an einem W \textbf{6} wer] swer V T  $\cdot$ vrouwen] weiben W \textbf{7} einege] eines V W T \textbf{8} Welche mit rechte woͤlle schawen W  $\cdot$ welchiu] sweliche V swelhiv T \textbf{10} dien] Die W  $\cdot$ betœren] lietoren U \textbf{12} wâ] swa V T \textbf{13} welhiu] Swele V Swelhiv T \textbf{14} diu] [di*]: do V  $\cdot$ ir minne] ir [wit*]: witze V an witzen W ir witze T \textbf{15} wîbes] \textit{om.} W \textbf{18} sô] \textit{om.} W T \textbf{19} topels] toppellens W \textbf{20} wer] swer V T  $\cdot$ mit] \textit{om.} V \textbf{21} hetenz] Hatens V  $\cdot$ diu] die T  $\cdot$ ein] \textit{om.} W \textbf{22} vürbaz] eúch fúrbas W (T) \textbf{23} dise] disem V W T \textbf{24} spreche] spriche V  $\cdot$ dise] [*]: dise V der W div T \textbf{25} wer] Swer V T \textbf{26} dern zels] der zels V Der zele es W der en zal si T \textbf{27} enkan] kan W \textbf{28} dâ] Do W  $\cdot$ nement] nemt T  $\cdot$ ir] irn V  $\cdot$ urhap] hap T \textbf{29} disiu] dise T \textbf{30} vert] Virt U \newline
\end{minipage}
\end{table}
\end{document}
