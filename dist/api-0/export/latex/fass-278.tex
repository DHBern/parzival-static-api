\documentclass[8pt,a4paper,notitlepage]{article}
\usepackage{fullpage}
\usepackage{ulem}
\usepackage{xltxtra}
\usepackage{datetime}
\renewcommand{\dateseparator}{.}
\dmyyyydate
\usepackage{fancyhdr}
\usepackage{ifthen}
\pagestyle{fancy}
\fancyhf{}
\renewcommand{\headrulewidth}{0pt}
\fancyfoot[L]{\ifthenelse{\value{page}=1}{\today, \currenttime{} Uhr}{}}
\begin{document}
\begin{table}[ht]
\begin{minipage}[t]{0.5\linewidth}
\small
\begin{center}*D
\end{center}
\begin{tabular}{rl}
\textbf{278} & \textbf{geschæhe}, i\textbf{ne} gunde iu trûrens niht,\\ 
 & \textbf{noch} en\textbf{getuon}, \textbf{swâ ez} geschiht.\\ 
 & mir ist liep, daz ir die hulde hât\\ 
 & unt daz ir \textbf{vrowelîche} wât\\ 
5 & tragt nâch iwer grôzen nôt."\\ 
 & \textbf{si sprach}: "hêrre, daz vergelt iu got;\\ 
 & \textbf{dâr an} ir hœhet iwern prîs."\\ 
 & Jeschuten unt ir \textit{â}mîs\\ 
 & vrou Cunneware de Lalant\\ 
10 & dannen vuorte \textbf{sâ} zehant.\\ 
 & einhalp an des küneges rinc\\ 
 & über eines brunnen ursprinc\\ 
 & stuont ir poulûn ûf dem plân,\\ 
 & als \textbf{ez obene} ein trache in sînen klân\\ 
15 & hete, des \textbf{ganzes} apfels halben teil.\\ 
 & den trachen zugen vier wintseil,\\ 
 & reht als er \textbf{lebendec} dâ vlüge\\ 
 & untz poulûn gein den lüften züge.\\ 
 & dâ bî erkandz Orilus,\\ 
20 & wan sîniu wâpen wâren sus.\\ 
 & Er wart entwâpent drunde.\\ 
 & sîn süeziu swester kunde\\ 
 & im bieten êre unt gemach.\\ 
 & über al diu messenîe sprach,\\ 
25 & des rôten rîters ellen\\ 
 & \textbf{næme den prîs zeime gesellen}.\\ 
 & \begin{large}D\end{large}es \textbf{jâhen si} âne \textbf{rûnen}.\\ 
 & Keie bat Kingrunen\\ 
 & Orilus dienen an sîner stat.\\ 
30 & er kund\textbf{z} wol, den ers dâ bat,\\ 
\end{tabular}
\scriptsize
\line(1,0){75} \newline
D \newline
\line(1,0){75} \newline
\textbf{21} \textit{Majuskel} D  \textbf{27} \textit{Initiale} D  \newline
\line(1,0){75} \newline
\textbf{8} Jeschuten] Jescvten D  $\cdot$ âmîs] armis D \newline
\end{minipage}
\hspace{0.5cm}
\begin{minipage}[t]{0.5\linewidth}
\small
\begin{center}*m
\end{center}
\begin{tabular}{rl}
 & \textbf{geschæhe}, i\textbf{ne} gunde iu trûrens niht,\\ 
 & \textbf{noch} en\textbf{tuon} \textbf{ouch niht}, \textbf{waz} geschiht.\\ 
 & mir ist liep, daz ir die hulde hât\\ 
 & und daz \textit{i}r \textbf{vrowelîche} wât\\ 
5 & trage\textit{t} nâch iuwerre grôzen nôt."\\ 
 & \textbf{si sprach}: "hêrre, daz vergelte iu got;\\ 
 & \textbf{dâr an} ir hœhet iuwern prîs."\\ 
 & Jeschuten und ir âmîs\\ 
 & vrouwe C\textit{unn}ew\textit{a}re de Lalant\\ 
10 & dannen vuorte \textbf{sâ} zehant.\\ 
 & einhalp an des küniges rinc\\ 
 & über eines brunnen ursprinc\\ 
 & stuont ir pavelûn ûf dem p\textit{lâ}n,\\ 
 & alsô ein trache in sînen klân\\ 
15 & \multicolumn{1}{l}{ - - - }\\ 
 & \multicolumn{1}{l}{ - - - }\\ 
 & \multicolumn{1}{l}{ - - - }\\ 
 & \dag und\dag  die pavelûn gegen den lüfte\textit{n} z\textit{ü}g\textit{e}.\\ 
 & dâ bî erkant ez Orilus,\\ 
20 & wan sîniu wâpen wâr\textit{en} sus.\\ 
 & er wart entwâpent drunde.\\ 
 & sîn süeziu swester kunde\\ 
 & ime bieten êre und gemach.\\ 
 & über al diu massenîe sprach,\\ 
25 & des rôten ritters ellen\\ 
 & \textbf{næme den prîs zeinem gesellen}.\\ 
 & des \textbf{jâhen si} âne \textbf{riuwen}.\\ 
 & Keie bat Kingrunen\\ 
 & Orilus dienen an sîner stat.\\ 
30 & er kunde \textbf{daz} wol, den ers d\textit{â} bat,\\ 
\end{tabular}
\scriptsize
\line(1,0){75} \newline
m n o \newline
\line(1,0){75} \newline
\newline
\line(1,0){75} \newline
\textbf{1} trûrens] truwens n \textbf{2} niht] nẏemer n (o) \textbf{4} ir] uwer m \textbf{5} traget] tragen m  $\cdot$ iuwerre] ire m \textbf{7} hœhet] hoͯhen n \textbf{8} Jeschuten] Jescutten m Jescuten n Juscuͯten o \textbf{9} Cunneware] komewere m kuneware n koͯne ware o  $\cdot$ de] die o \textbf{12} eines] eyn o \textbf{13} \textit{Versdoppelung 278.13 nach 278.14} o   $\cdot$ plân] paln m \textbf{14} alsô] Also ob n (o) \textbf{18} \textit{Vers 278.18 fehlt} n   $\cdot$ lüften] liuͯffte m  $\cdot$ züge] zugen m \textbf{19} ez] er o  $\cdot$ Orilus] vrelosz o \textbf{20} sîniu wâpen] sin woppent n (o)  $\cdot$ wâren] wor m \textbf{23} ime] Kuͯnde ẏme o \textbf{24} al] alle n o \textbf{25} des] Das o \textbf{27} des] So n Suͯs o  $\cdot$ riuwen] ruͯmen n o \textbf{28} Keie] Keẏe n Kein o  $\cdot$ Kingrunen] [kunig]: kingrunen m kún gruͯnen n kin gruͯnen o \textbf{29} Orilus] Oriluͯs o \textbf{30} dâ] do m n o \newline
\end{minipage}
\end{table}
\newpage
\begin{table}[ht]
\begin{minipage}[t]{0.5\linewidth}
\small
\begin{center}*G
\end{center}
\begin{tabular}{rl}
 & \textbf{geschach}, ich\textbf{ne} gunde iu trûrens niht\\ 
 & \textbf{unde} en\textbf{tuon} \textbf{ouch noch}, \textbf{swaz mir} geschiht.\\ 
 & mir ist liep, daz ir die hulde hât\\ 
 & unt daz ir \textbf{vrœlîche} wât\\ 
5 & traget nâch iuwer grôzen nôt."\\ 
 & "hêrre, daz vergelt iu got;\\ 
 & \textbf{dâr an} ir hœhet iuweren brîs."\\ 
 & \textbf{vrou} Jeschute unde ir âmîs\\ 
 & vrou Kuneware de \textit{La}lant\\ 
10 & dannen vuorte \textbf{al} zehant.\\ 
 & \begin{large}E\end{large}inhalp ans küniges rinc\\ 
 & über eines brunnen ursprinc\\ 
 & stuont ir poulûn ûf dem plân,\\ 
 & als \textbf{ez} ein trache in sînen klân\\ 
15 & hete, des apfels halben teil.\\ 
 & den trachen zugen vier wintseil,\\ 
 & reht alser \textbf{lebende} dâ \textit{v}lüge\\ 
 & unt daz pavelûn gein den lüften züge.\\ 
 & dâ bî erkandez Orillus,\\ 
20 & wan sîniu wâpen wâren sus.\\ 
 & er wart entwâpent drunde.\\ 
 & sîn süeziu swester kunde\\ 
 & im bieten êre unde gemach.\\ 
 & über al diu messenîe sprach,\\ 
25 & \textit{d}es rôten rîters ellen\\ 
 & \textbf{möhte sich niht gezellen}.\\ 
 & des \textbf{jâhen si} âne \textbf{rûnen}.\\ 
 & Kay bat Kingrunen\\ 
 & Orillus dienen an sîner stat.\\ 
30 & er kunde\textbf{z} wol, den ers dâ bat,\\ 
\end{tabular}
\scriptsize
\line(1,0){75} \newline
G I O L M Q R Z Fr30 \newline
\line(1,0){75} \newline
\textbf{3} \textit{Initiale} L  \textbf{7} \textit{Initiale} I  \textbf{11} \textit{Initiale} G  \textbf{25} \textit{Initiale} I  \newline
\line(1,0){75} \newline
\textbf{1} geschach] Geschehe O L (M) Q (R) Z (Fr30)  $\cdot$ ichne gunde] ich gunde I (L) (M) (Q) (R) (Z) ich engan Fr30  $\cdot$ iu] ev doch I  $\cdot$ trûrens] lasters Q (R) \textbf{2} unde] Noch O L M Q R Z  $\cdot$ entuon] getuͯn L (M) (R) (Fr30)  $\cdot$ ouch noch] [*]: stvch noch I halt nimmer O (M) Z \textit{om.} L halt ny nimer Q nyemer R immer Fr30  $\cdot$ swaz] waz L (Q) (R) Z  $\cdot$ mir] \textit{om.} O M Q R Z \textbf{3} die] \textit{om.} L \textbf{4} unt] [Mit]: Vnt G  $\cdot$ vrœlîche] fraweliche O (L) Q (Z) fromiclichen M freuenliche R \textbf{5} grôzen] grozer I (L) \textbf{6} hêrre] Sie sprach herre Q (R)  $\cdot$ daz] des Q \textbf{7} ir hœhet] erhoet ir Q hochet R  $\cdot$ iuweren] úwer R immer Fr30 \textbf{8} vrou] Frowen O (Q) Z  $\cdot$ Jeschute] ieschute G (Fr30) ieskute I Jescv̂ten O jescuͯten L iescuten M (Z) Jescuten Q Jscuten R \textbf{9} Kuneware] chunuwar I kvnware O Cvneware L kunwarin M Conware Q [Cuͦne]: Cuͦnware R kvnneware Z kvrneware Fr30  $\cdot$ de Lalant] delant G der lalant O von labant R \textbf{10} vuorte] vuͤrt I (O) (Q) \textbf{11} ans] an den O \textbf{12} über] V́ben R  $\cdot$ brunnen] bornes M prvnne Fr30 \textbf{13} ir] eyn M  $\cdot$ poulûn] geczelt R  $\cdot$ dem] den R \textbf{14} ez] ez oben O L (M) (Q) (R) Z (Fr30)  $\cdot$ sînen] den Q R \textbf{15} hete] Hat Fr30  $\cdot$ apfels halben] ganzen halben apfels O (M) (Fr30) gantzen aphels L halben appels Q (R) gantzen apfels halben Z \textbf{16} den] der Fr30  $\cdot$ wintseil] wintsel R wine seil Fr30 \textbf{17} alser] alz ez L alsam er Fr30  $\cdot$ lebende] lebendich I O (L) (Q) (R) (Z)  $\cdot$ dâ] \textit{om.} Q  $\cdot$ vlüge] sluge G flugen R \textbf{18} Vnd das geczelt gen dem himel zugen R \textbf{19} dâ bî] Do er bey Q  $\cdot$ erkandez] kant es Q  $\cdot$ Orillus] orrilus G Orilus I (O) M (Q) R (Z) oryllvs Fr30 \textbf{20} sîniu] sin I sie Q \textbf{21} entwâpent drunde] en phangin dar mite M \textbf{23} bieten] erbieten I  $\cdot$ êre] eren Q \textbf{25} des] zoͮ des G \textbf{26} möhte sich niht] get vuͤr all vnser I Nempt den pris O (M) Nem den pris L (Q) (R) (Z) nimt in pris Fr30  $\cdot$ gezellen] gesellen I (M) Fr30 zesellen O zuͯ gesellen L (Q) ze gesellend R zv eim gesellen Z \textbf{27} des] Do R  $\cdot$ jâhen si] iahen O sprachin sie M  $\cdot$ rûnen] ruͤmen I \textbf{28} Kay] kaẏ G kain I Key O M R Z Kaý L keẏ Fr30  $\cdot$ Kingrunen] kingrumen I kyngrvnen O Fr30 Kýngrvnen L kyngruͯnen M kyngrúnen Q \textbf{29} Orillus] Orilus I (O) Q R Z Oriluse M Oryllen Fr30 \textbf{30} kundez] erchundez I kvndes ez L  $\cdot$ den] deme M  $\cdot$ ers] er Q R er sin Fr30  $\cdot$ dâ] do Q \textit{om.} Fr30 \newline
\end{minipage}
\hspace{0.5cm}
\begin{minipage}[t]{0.5\linewidth}
\small
\begin{center}*T
\end{center}
\begin{tabular}{rl}
 & \textbf{geschæhe}, ich gunde iu trûrens niht,\\ 
 & \textbf{noch}n \textbf{getuo} \textbf{niemer}, \textbf{swaz mir} geschiht.\\ 
 & mirst liep, daz ir die hulde hât\\ 
 & unde daz ir \textbf{vrœlîche} wât\\ 
5 & traget nâch iuwer grôzen nôt."\\ 
 & "Hêrre, daz vergeltiu got,\\ 
 & \textbf{daz} ir hœhet iuwern prîs."\\ 
 & \textbf{vroun} Jeschuten unde ir âmîs\\ 
 & vrou Cunneware de Lalant\\ 
10 & dannen vuorte \textbf{al}zehant.\\ 
 & ein\textit{h}alp an des küneges rinc\\ 
 & über eines brunnen ursprin\textit{c}\\ 
 & stuont ir pavelûn ûf dem plân,\\ 
 & als \textbf{ez oben} ein trache in sînen klân\\ 
15 & hete, des \textbf{ganzen} apfels halben teil.\\ 
 & den trachen zugen vier wintseil,\\ 
 & rehte alser \textbf{lebendic} dâ vlüge\\ 
 & unde daz pavelûn gegen den lüften züge.\\ 
 & dâ bî erkandez Orilus,\\ 
20 & wan sîniu wâpen wâren sus.\\ 
 & er wart entwâpent drunde.\\ 
 & sîn süeziu swester kunde\\ 
 & im bieten êre unde gemach.\\ 
 & über al die massenîe \textbf{man} sprach\\ 
25 & des rôten rîters ellen\\ 
 & \textbf{unde gâben im prîs ze gesellen}.\\ 
 & \begin{large}D\end{large}es \textbf{âbendes} âne \textbf{rûnen}\\ 
 & Key bat Kyngrunen\\ 
 & Oriluse dienen an sîner stat.\\ 
30 & er kunde\textbf{z} wol, den er\textit{s} dâ bat,\\ 
\end{tabular}
\scriptsize
\line(1,0){75} \newline
T U V W \newline
\line(1,0){75} \newline
\textbf{3} \textit{Initiale} W  \textbf{6} \textit{Majuskel} T  \textbf{27} \textit{Initiale} T U V  \newline
\line(1,0){75} \newline
\textbf{1} gunde] kuͦnde U \textbf{2} nochn] Noch U V W  $\cdot$ swaz] waz U (W) \textbf{3} die] \textit{om.} W \textbf{4} daz ir] auch W  $\cdot$ vrœlîche] [froͤ*]: froͤwenliche V \textbf{5} iuwer] núwer W  $\cdot$ grôzen] grosse W \textbf{6} Hêrre] [*]: Sv́ sprach herre V  $\cdot$ vergeltiu] verhelt vch U \textbf{7} daz] [*]: Dar an V Dar an W  $\cdot$ hœhet] heibet U \textbf{8} vroun] Vreuͦwe U (W)  $\cdot$ Jeschuten] Jescvten T (U) iescuten V iestuten W \textbf{9} Cunneware] kvnneware T Cvmeware U kunnewar W  $\cdot$ de Lalant] delalant U \textbf{10} vuorte] fuͦr W \textbf{11} einhalp] eintalp T Ein halt U \textbf{12} ursprinc] vrsprin T \textbf{13} pavelûn] gezelt V  $\cdot$ dem] den W \textbf{14} als ez oben] Als [*]: ob V Es het W  $\cdot$ trache] trach obnan W \textbf{15} hete] Vnd het W  $\cdot$ halben teil] [hal*en]: halben teil T haben deil U [halp*]: halben teil V ein halbteil W \textbf{17} lebendic] lebende W  $\cdot$ dâ] do V W \textbf{18} daz] des U ers W  $\cdot$ pavelûn] gezelt V  $\cdot$ gegen den] gen W \textbf{21} er] Es W \textbf{22} kunde] [*]: kuͦnde U munder W \textbf{23} im bieten] Kunde im gebieten W \textbf{24} al] alle W  $\cdot$ man] \textit{om.} W \textbf{25} rîters ellen] rites elle U \textbf{26} Nemen wir fúr die wir kunnen zellen W \textbf{27} [De*]: Dez iahen sv́ ane rvnen V  $\cdot$ Des iahen sy ane ruͦmen hie W \textbf{28} Key] Keyn V  $\cdot$ Kyngrunen] kingruͦnen U kingrunen ich sag úch wie W \textbf{29} Oriluse] Oriluͦsen U Orilusen V Orilo W \textbf{30} ers] [*z]: erz T er W  $\cdot$ dâ] do U V W \newline
\end{minipage}
\end{table}
\end{document}
