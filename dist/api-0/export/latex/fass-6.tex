\documentclass[8pt,a4paper,notitlepage]{article}
\usepackage{fullpage}
\usepackage{ulem}
\usepackage{xltxtra}
\usepackage{datetime}
\renewcommand{\dateseparator}{.}
\dmyyyydate
\usepackage{fancyhdr}
\usepackage{ifthen}
\pagestyle{fancy}
\fancyhf{}
\renewcommand{\headrulewidth}{0pt}
\fancyfoot[L]{\ifthenelse{\value{page}=1}{\today, \currenttime{} Uhr}{}}
\begin{document}
\begin{table}[ht]
\begin{minipage}[t]{0.5\linewidth}
\small
\begin{center}*D
\end{center}
\begin{tabular}{rl}
\textbf{6} & \textbf{brâht} er unz an sînen tôt.\\ 
 & sîn elter sun vür sich gebôt\\ 
 & den vürsten ûz \textbf{sînem} rîche.\\ 
 & \textbf{die} kâmen \textbf{ritterlîche},\\ 
5 & wan si ze rehte solden hân\\ 
 & von im grôz lêhen sunder wân.\\ 
 & Dô si ze hove wâren komen\\ 
 & und ir reht \textbf{was} vernomen,\\ 
 & daz si \textbf{ir lêhen alle} enpfiengen,\\ 
10 & \textbf{nû} hœret, wie si\textbf{z} ane \textbf{geviengen}:\\ 
 & si gerten, als ir triwe riet,\\ 
 & rîch und arme, gar diu diet,\\ 
 & einer \textbf{kranken} ern\textit{st}lîcher bete,\\ 
 & daz der künec an Gahmurete\\ 
15 & bruoderlîche triwe mêrte\\ 
 & und sich selben êrte,\\ 
 & daz er in \textbf{niht} gar verstieze\\ 
 & und im sînes landes lieze\\ 
 & hantgemælde, \textbf{daz} man m\textit{ö}hte sehen,\\ 
20 & dâ von der hêrre \textbf{m\textit{üe}se} jehen\\ 
 & sînes namen und sîner vrîheit.\\ 
 & daz was dem künege niht \textbf{ze} leit.\\ 
 & er sprach: "ir kunnet mâze gern.\\ 
 & ich wil iuch des und vürbaz wern.\\ 
25 & \textbf{wan} \textbf{nennet} \textbf{ir} den bruoder mîn\\ 
 & Gahmuret Anschevin?\\ 
 & Anschowe ist mîn lant.\\ 
 & dâ wesen beide von genant."\\ 
 & \textbf{\begin{large}D\end{large}ô} sprach der künec hêre:\\ 
30 & "\textbf{mîn bruoder, der mac sich} mêre\\ 
\end{tabular}
\scriptsize
\line(1,0){75} \newline
D \newline
\line(1,0){75} \newline
\textbf{7} \textit{Versal} D  \textbf{29} \textit{Initiale} D  \newline
\line(1,0){75} \newline
\textbf{13} ernstlîcher] erntslicher D \textbf{14} Gahmurete] gahmvrete D \textbf{19} möhte] mohte D \textbf{20} müese] mvͦse D \textbf{26} Gahmuret] Gahmvret D  $\cdot$ Anschevin] Anscivin D \textbf{27} Anschouwe] Anscowe D \newline
\end{minipage}
\hspace{0.5cm}
\begin{minipage}[t]{0.5\linewidth}
\small
\begin{center}*m
\end{center}
\begin{tabular}{rl}
 & \textbf{brâhte}  unz an sînen tôt.\\ 
 & sîn alter sun vür sich gebôt\\ 
 & den vürsten ûz \textbf{sînem} rîche.\\ 
 & \textbf{si} kâmen \textbf{ritterlîche},\\ 
5 & wenne si zuo rehte solten hân\\ 
 & von ime grôz lêhen s\textit{und}e\textit{r} \textit{w}ân.\\ 
 & \begin{large}D\end{large}ô si zuo hove \textit{w}âren kome\textit{n}\\ 
 & und \textit{i}r reht \textbf{baz} vernome\textit{n},\\ 
 & daz si \textbf{ir lêhen alle} enpfiengen,\\ 
10 & \textbf{nû} hœret, wie si an \textbf{geviengen}:\\ 
 & si gerten, als ir triuwe riet,\\ 
 & rîch und arme, gar diu diet,\\ 
 & einer \textbf{kranken} ern\textit{st}lîchen be\textit{t},\\ 
 & \dag die\dag  der künic an G\textit{a}hmuret\\ 
15 & bruoderlîche triuwe mêrte\\ 
 & und sich selber êrte,\\ 
 & daz er in \textbf{niht} gar verstieze\\ 
 & und ime sînes landes lieze\\ 
 & hantgemælde; \textbf{d\textit{â}} man m\textit{ö}hte s\textit{e}hen,\\ 
20 & dâ von \textit{der} hêrre \textbf{müese} jehen\\ 
 & sînes namen und sîner vrîheit.\\ 
 & daz was dem künige niht leit.\\ 
 & er s\textit{p}rach: "ir k\textit{unn}et mâze gern.\\ 
 & ich wil iuch des und vürbaz wern.\\ 
25 & \textbf{wenne} \textbf{nennet} \textbf{ir} den bruoder mîn\\ 
 & G\textit{a}hmuret A\textit{n}schevin?\\ 
 & Anschouwe ist mîn lant.\\ 
 & dâ wesen \textbf{wir} beide von genant."\\ 
 & \textbf{ouch} sprach der künîc hêre:\\ 
30 & "\textbf{mîn bruoder mac sich} mêr\textit{e}\\ 
\end{tabular}
\scriptsize
\line(1,0){75} \newline
m n o W \newline
\line(1,0){75} \newline
\textbf{7} \textit{Initiale} m   $\cdot$ \textit{Capitulumzeichen} n  \textbf{9} \textit{Initiale} W  \newline
\line(1,0){75} \newline
\textbf{3} den] [Dine]: Den m Die n W Din o \textbf{4} kâmen] koment n (o) W \textbf{5} zuo rehte] recht n W rehter o \textbf{6} von] Won o  $\cdot$ grôz] grosse o  $\cdot$ sunder wân] solten han \textit{nachträglich korrigiert zu:} entpfahen m \textbf{7} wâren komen] farn koment m \textbf{8} ir] rer m  $\cdot$ vernomen] vernoment m \textbf{9} daz] DO W \textbf{10} an geviengen] anfingen n (o) W \textbf{11} riet] gert o \textbf{12} und] \textit{om.} W  $\cdot$ gar diu] vnd alle die n alle die o vnd alle W \textbf{13} ernstlîchen] erntshlichen m enstlichen o  $\cdot$ bet] beret \textit{nachträglich korrigiert zu:} bet m beriet n beret o \textbf{14} künic] konige m  $\cdot$ Gahmuret] gemuret m gamiret n o gamuret W \textbf{15} mêrte] merten o \textbf{17} verstieze] versties o \textbf{18} lieze] lies o \textbf{19} hantgemælde] Do hant gemelde n Auch ein teil W  $\cdot$ dâ] do m n o das W  $\cdot$ möhte] mochte m  $\cdot$ sehen] schen m \textbf{20} der] \textit{om.} m  $\cdot$ müese] moͯhte o \textbf{22} niht] mit o  $\cdot$ leit] zuͦ leit n (o) (W) \textbf{23} er sprach] Er sfrach m Fr spracb W  $\cdot$ ir] ie o  $\cdot$ kunnet] koment m (o) \textbf{25} wenne] Jran o  $\cdot$ ir] \textit{om.} o \textbf{26} Gahmuret] Gomuͯret \textit{nachträglich korrigiert zu:} Gamuͯret m Gamiret n o Gamuret W  $\cdot$ Anschevin] ausceuin \textit{nachträglich korrigiert zu:} ansceuin m auscuin n antiscim o antscheuin W \textbf{27} Anschouwe] An schouwe \textit{nachträglich korrigiert zu:} An schande m Anschonwe n An schowe o Antschowe W \textbf{28} wesen wir] wesen >wir< m wir seint W  $\cdot$ von] an vnd von n \textbf{29} hêre] herren n (o) (W) \textbf{30} mac] der mag n  $\cdot$ mêre] meren \textit{nachträglich korrigiert zu:} mere m meren n o W \newline
\end{minipage}
\end{table}
\newpage
\begin{table}[ht]
\begin{minipage}[t]{0.5\linewidth}
\small
\begin{center}*G
\end{center}
\begin{tabular}{rl}
 & \textbf{behielt} er unze an sînen tôt.\\ 
 & sîn alter sun vür sich gebôt\\ 
 & \begin{large}D\end{large}en vürsten ûz \textbf{dem} rîche.\\ 
 & \textbf{die} kômen \textbf{al gelîche},\\ 
5 & wan si ze rehte solten hân\\ 
 & von im grôz lêhen sunder wân.\\ 
 & dô si ze hove wâren komen\\ 
 & unde ir reht \textbf{wart} vernomen,\\ 
 & daz si \textbf{alle ir lêhen} enpfiengen,\\ 
10 & \textbf{nû} hœret, wie si\textbf{z} ane \textbf{viengen}:\\ 
 & si gerten, als ir triwe riet,\\ 
 & rîch und arme, gar diu diet,\\ 
 & einer \textbf{kargen} ernstlîchen bet,\\ 
 & daz der künic an Gahmuret\\ 
15 & bruoderlîche triwe mêrte\\ 
 & unde sich selben êrte,\\ 
 & daz er in \textbf{iht} gar verstieze\\ 
 & unde im sînes landes lieze\\ 
 & hantgemahel, \textbf{daz} man m\textit{ö}ht sehen,\\ 
20 & dâ von der hêrre \textbf{m\textit{üe}se} jehen\\ 
 & sînes namen und sîner vrîheit.\\ 
 & daz was dem künige niht \textbf{ze} leit.\\ 
 & er sprach: "ir kunnet mâze geren.\\ 
 & ich wil iuch des und vürbaz wern.\\ 
25 & \textbf{wan} \textbf{n\textit{en}net} \textbf{ir} den bruoder mîn\\ 
 & Gahmuret Antschevin?\\ 
 & Anschouwe ist mîn lant.\\ 
 & dâ wesen bêde von genant."\\ 
 & \textbf{sus} sprach der künic hêre:\\ 
30 & "\textbf{sich sol m\textit{în} bruoder} mêre\\ 
\end{tabular}
\scriptsize
\line(1,0){75} \newline
G O L M Q Z Fr32 Fr58 \newline
\line(1,0){75} \newline
\textbf{1} \textit{Initiale} O  \textbf{3} \textit{Initiale} G  \textbf{7} \textit{Versal} Fr32  \textbf{29} \textit{Initiale} Q Z Fr32  \newline
\line(1,0){75} \newline
\textbf{1} behielt] ÷raht O Brachte L M (Q) (Z) (Fr32) (Fr58)  $\cdot$ er unze] vns M er wisz Q ouch er vnz Fr32 \textbf{2} alter] elczter Q eltesten Fr32  $\cdot$ vür] er vur Fr32  $\cdot$ gebôt] bôt Fr32 \textbf{3} Den] vnde die Fr32  $\cdot$ ûz] von Fr32  $\cdot$ dem] sinem \sout{:::ndem} O sinem L (M) (Q) Z \textbf{4} al gelîche] riterliche O (L) (M) (Q) (Z) (Fr32) \textbf{6} von] Wann Q  $\cdot$ grôz] grosse Q  $\cdot$ lêhen] [lebin]: leben M \textbf{7} dô] Da O M Z Die \textit{nachträglich korrigiert zu:} Do Q \textbf{8} wart] was O (L) M Q Z (Fr32) \textbf{9} alle] \textit{om.} Fr32 \textbf{10} nû] \textit{om.} Fr32  $\cdot$ viengen] geviengen O \textbf{11} gerten] sprachen Z  $\cdot$ riet] geriet Q \textbf{13} kargen] chranchen O (L) (M) (Q) (Z) (Fr32)  $\cdot$ ernstlîchen bet] ernst lîcher pet O (Q) (Z) (Fr32) ernslycherbet L cristlicher bet M \textbf{14} an] \textit{om.} Z  $\cdot$ Gahmuret] gamvret O Fr32 Gahmuͯret L gachmuͯret M gaműert Q Gamuret Z \textbf{16} selben] [selbn]: selbirn M selber Q \textbf{17} iht] niht O L (M) (Q) Z Fr32  $\cdot$ gar] \textit{om.} O \textbf{18} lieze] hîeze Fr32 \textbf{19} man] \textit{om.} Z Fr32  $\cdot$ möht] moht G (L) (Q) Z (Fr32) muͦhte O nicht M \textbf{20} müese] moͮse G (O) (M) (Z) (Fr32) [myse]: mvse  L \textbf{21} sîner vrîheit] sine warheit M seiner wirdikeit Q \textbf{22} daz] Dar ane M Dor zu Q (Z)  $\cdot$ was] enwaz Fr32  $\cdot$ dem künige] den konnigen M  $\cdot$ ze] \textit{om.} Q \textbf{23} kunnet] kuͯmit M  $\cdot$ mâze] mezlichen O (L) merlichen M  $\cdot$ geren] [piten]: gern O gerne M ge geren Z \textbf{24} iuch] iv Fr32  $\cdot$ und] \textit{om.} Q  $\cdot$ wern] gewern O (Q) (Z) \textbf{25} nennet] nnet G meynet M  $\cdot$ bruoder] brvden Z \textbf{26} Gahmuret] Gahmvret G L Gamvret O Gachmuret M Gaműert Q Gamuret Z gamv̂ret Fr32  $\cdot$ Antschevin] ansevin O anshevin L Z auͯschefyn M anszheim Q anshevîn Fr32 \textbf{27} Anschouwe] antschoͮwe G Anschawe O Anschowe L (Q) Z anshôuwe Fr32 \textbf{28} wesen] sie wir Z \textbf{30} sol] sullen M  $\cdot$ mîn] m::: G \newline
\end{minipage}
\hspace{0.5cm}
\begin{minipage}[t]{0.5\linewidth}
\small
\begin{center}*T
\end{center}
\begin{tabular}{rl}
 & \textbf{brâht}er unz an sînen tôt.\\ 
 & Sîn elter sun vür sich gebôt\\ 
 & den vürsten ûz \textbf{sînem} rîche.\\ 
 & \textbf{die} kômen \textbf{rîterlîche},\\ 
5 & wan si ze rehte solten hân\\ 
 & von im grôz lêhen sunder wân.\\ 
 & Dô si ze hove wâren komen\\ 
 & und ir reht \textbf{was} vernomen,\\ 
 & \textbf{unde} daz s\textbf{ir lêhen} enpfienge\textit{n},\\ 
10 & hœret, wie si\textbf{z} an \textbf{viengen}:\\ 
 & Si gerten, als ir triuwe riet,\\ 
 & rîche und arme, gar diu diet,\\ 
 & einer \textbf{kranken} ernestlîchen bete,\\ 
 & daz der künec an Gahmurete\\ 
15 & brüederlîche triuwe mêrte\\ 
 & und sich selben êrte,\\ 
 & daz er in \textbf{niht} gar verstieze\\ 
 & und im sînes landes lieze\\ 
 & hantgemahel, \textbf{daz} man m\textit{ö}hte sehen,\\ 
20 & dâ von der hêrre \textbf{m\textit{ö}hte} jehen\\ 
 & sînes namen und sîner vrîheit.\\ 
 & Daz was dem künege niht \textbf{ze} leit.\\ 
 & er sprach: "ir kunnet \textbf{ze} mâze gern.\\ 
 & ich wil iuch des und vürbaz wern.\\ 
25 & \textbf{man} \textbf{nenne} den bruoder mîn\\ 
 & Gahmuret Anschevin.\\ 
 & Anschouwe, \textbf{daz} ist mîn lant.\\ 
 & dâ wesen beide von genant."\\ 
 & \textbf{\begin{large}A\end{large}ber} sprach der künec hêre:\\ 
30 & "\textbf{sich sol mîn bruoder} mêre\\ 
\end{tabular}
\scriptsize
\line(1,0){75} \newline
T U V \newline
\line(1,0){75} \newline
\textbf{2} \textit{Majuskel} T  \textbf{7} \textit{Majuskel} T  \textbf{11} \textit{Majuskel} T  \textbf{22} \textit{Majuskel} T  \textbf{29} \textit{Initiale} T U V  \newline
\line(1,0){75} \newline
\textbf{3} ûz] in V \textbf{4} kâmen] koment V \textbf{6} grôz] groze U \textbf{7} hove] hore U \textbf{9} unde] \textit{om.} V  $\cdot$ daz sir] do sie ir U Daz sv́ alle ire V  $\cdot$ enpfiengen] enpfienge: T \textbf{10} hœret] Nv hoͤrent V \textbf{13} kranken ernestlîchen] cranglicher steter V \textbf{14} Gahmurete] Gahmvrete T gahmuͦrete U Gamurette V \textbf{16} selben] selber U \textbf{19} möhte] mohte T U \textbf{20} möhte] mohte T (U) [m*]: moͤhte  V \textbf{23} ze] \textit{om.} U \textbf{24} iuch] îv T \textbf{25} man nenne] Wan nennent ir V  $\cdot$ den] dem U \textbf{26} Gahmuret] Gahmvret T Gahmvre U Gamuret V  $\cdot$ Anschevin] anscevin T anscheuin U Anschefin V \textbf{27} Anschouwe] Anschoͮwe T An schowin U Anschowe V  $\cdot$ daz] \textit{om.} U V \textbf{28} wesen] sint wir V \textbf{29} Aber] Do U Svs V \textbf{30} Min bruͦder sol sich mere V \newline
\end{minipage}
\end{table}
\end{document}
