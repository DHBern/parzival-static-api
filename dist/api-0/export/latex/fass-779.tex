\documentclass[8pt,a4paper,notitlepage]{article}
\usepackage{fullpage}
\usepackage{ulem}
\usepackage{xltxtra}
\usepackage{datetime}
\renewcommand{\dateseparator}{.}
\dmyyyydate
\usepackage{fancyhdr}
\usepackage{ifthen}
\pagestyle{fancy}
\fancyhf{}
\renewcommand{\headrulewidth}{0pt}
\fancyfoot[L]{\ifthenelse{\value{page}=1}{\today, \currenttime{} Uhr}{}}
\begin{document}
\begin{table}[ht]
\begin{minipage}[t]{0.5\linewidth}
\small
\begin{center}*D
\end{center}
\begin{tabular}{rl}
\textbf{779} & \begin{large}S\end{large}enfteclîche \textbf{unt doch} in \textbf{vollem} zelt\\ 
 & kom si \textbf{rîtende} über velt.\\ 
 & ir zoum, ir satel, ir runzît\\ 
 & was rîche unt tiure ân allen strît.\\ 
5 & Man lie\textit{z} \textit{si} an den zîten\\ 
 & \textbf{in} den rinc rîten.\\ 
 & diu wîse, niht diu tumbe,\\ 
 & reit den rinc alumbe.\\ 
 & man zeigete ir, \textbf{wâ} Artus saz,\\ 
10 & gein dem si \textbf{grüezen} niht vergaz.\\ 
 & \textbf{En} franzoys was ir sprâche.\\ 
 & si \textbf{warp}, daz ein râche\\ 
 & ûf si verkorn wære\\ 
 & unt daz man hôrt ir mære.\\ 
15 & Den künec unt die künegin\\ 
 & bat si helfe unt an ir rede sîn.\\ 
 & si kêrte von in al zehant,\\ 
 & dâ si Parzivalen \textbf{sitzen} vant\\ 
 & bî Artuse nâhen.\\ 
20 & si begunde ir \textbf{sprunges} gâhen\\ 
 & von dem pferde ûfez gras.\\ 
 & si viel mit \textbf{zühten}, diu an ir was,\\ 
 & Parzivale an sînen vuoz.\\ 
 & \textbf{si} warb al weinende \textbf{umbe} \textbf{sînen} gruoz,\\ 
25 & \textbf{sô} daz er zorn gein ir verlür\\ 
 & unt \textbf{âne kus} \textbf{ûf} si verkür.\\ 
 & Artus unt Feirefiz\\ 
 & an den gewerp leiten vlîz.\\ 
 & Parzival truog \textbf{ûf si} haz;\\ 
30 & durch \textbf{vriwende} bet er des vergaz\\ 
\end{tabular}
\scriptsize
\line(1,0){75} \newline
D \newline
\line(1,0){75} \newline
\textbf{1} \textit{Initiale} D  \textbf{5} \textit{Majuskel} D  \textbf{11} \textit{Majuskel} D  \textbf{15} \textit{Majuskel} D  \newline
\line(1,0){75} \newline
\textbf{5} liez si] lieze D \textbf{18} Parzivalen] Parcivalen D \textbf{23} Parzivale] Parcifale D \textbf{29} Parzival] Parcifal D \newline
\end{minipage}
\hspace{0.5cm}
\begin{minipage}[t]{0.5\linewidth}
\small
\begin{center}*m
\end{center}
\begin{tabular}{rl}
 & senfteclîch \textbf{und doch} \textit{i}n zelt\\ 
 & kam si \textbf{\textit{ge}riten} über velt.\\ 
 & ir zoum, ir satel, ir runzît\\ 
 & was rîch und tiur âne allen strît.\\ 
5 & man liez si an den zîten\\ 
 & \textbf{in} den rinc rîten.\\ 
 & diu wîse, niht diu tumbe,\\ 
 & reit den rinc \textit{al}umbe.\\ 
 & man z\textit{öu}gte ir, \textbf{wâ} Artus saz,\\ 
10 & gegen dem si \textbf{gruozes} niht vergaz.\\ 
 & \textbf{in} franzois was ir sprâche.\\ 
 & si \textbf{warp}, daz ein râche\\ 
 & ûf si verkorn wære\\ 
 & und daz man hôrte ir mære.\\ 
15 & den künic und die künigîn\\ 
 & bat si \textbf{an ir} helf und an ir rede sîn.\\ 
 & si kêrte von in al zuohant,\\ 
 & d\textit{â} si Parcifaln \textbf{sitzen} vant\\ 
 & bî Artuse nâhen.\\ 
20 & si begunde ir \textbf{springens} gâhen\\ 
 & von dem pferde ûf daz gras.\\ 
 & s\textit{i} viel mit \textbf{zuht}, d\textit{iu} an ir was,\\ 
 & Parcifal an sînen vuoz\\ 
 & \textbf{und} warp alweinende \textbf{umb} \textbf{den} gr\textit{uoz},\\ 
25 & \textbf{sô} daz er zorn \textit{geg}e\textit{n ir} verlür\\ 
 & und \textbf{âne kus} \textbf{ûf} si verkür.\\ 
 & Artus und Ferefiz\\ 
 & an den gewerp leiten vlîz.\\ 
 & Parcifal truoc \textbf{ûf si} \textbf{den} haz;\\ 
30 & durch \textbf{vriundes} bete er de\textit{s v}ergaz\\ 
\end{tabular}
\scriptsize
\line(1,0){75} \newline
m n o V V' W \newline
\line(1,0){75} \newline
\newline
\line(1,0){75} \newline
\textbf{1} in] ein m \textbf{2} geriten] ritten m (n) (o) (W)  $\cdot$ über] vber daz V' \textbf{4} rîch und tiur] túre vnde rich V (V') \textbf{7} niht] vnd nicht W \textbf{8} alumbe] vmb m \textbf{9} zöugte] zogtte m (n) zeyget W  $\cdot$ ir] im W \textbf{11} franzois] francois m frantzoisch n franczois o franzoẏs V frantzoys W \textbf{12} si] Die n \textbf{13} verkorn] verloren W \textbf{14} ir mære] [*]: ir mere V \textbf{15} den] Der W \textbf{16} Bat [si*]: sie ir helfe an ir rede sin V' \textbf{17} in] im W  $\cdot$ al] \textit{om.} W \textbf{18} dâ] Do m n o V V' W  $\cdot$ Parcifaln] parcifalen n o parzefaln V palzifaln V' herr partzifal W \textbf{19} Artuse] artúse o artus W \textbf{20} begunde] [geh]: begonde V'  $\cdot$ springens] sprunges n o V' \textbf{21} ûf daz] uffens V \textbf{22} si] So m n o W  $\cdot$ zuht] zuchten V'  $\cdot$ diu] do m n o W \textbf{23} Parcifal] Parzefal V Parzifal V' Partzifal W \textbf{24} und] Sv́ V (V')  $\cdot$ alweinende] al geweinde V'  $\cdot$ den] sinen V (V')  $\cdot$ gruoz] gral m \textbf{25} \textit{Die Verse 779.25-28 fehlen} V'   $\cdot$ gegen ir] uͯber sẏ m \textbf{26} âne kus] one kusse n one :os o allen has W \textbf{27} Artus] Vnd artus o  $\cdot$ Ferefiz] ferefis m o ferrefis n Fereuis V ferafiß W \textbf{28} den gewerp] dem gewerbe W  $\cdot$ leiten] leites o leitent V \textbf{29} Parcifal] Parzefal V Parzifal V' Partzifal W  $\cdot$ truoc ûf si] auff sy truͦg W  $\cdot$ den] \textit{om.} n o V V' W \textbf{30} vriundes] frúnde V (V') W  $\cdot$ des vergaz] des nit vergas m \newline
\end{minipage}
\end{table}
\newpage
\begin{table}[ht]
\begin{minipage}[t]{0.5\linewidth}
\small
\begin{center}*G
\end{center}
\begin{tabular}{rl}
 & \begin{large}S\end{large}enfteclîch \textbf{unde doch} in \textbf{vollen} zelt\\ 
 & kom si \textbf{geriten} über velt.\\ 
 & ir zoum, ir satel, ir runzît\\ 
 & was rîch unde tiure ân allen strît.\\ 
5 & man lie si an den zîten\\ 
 & \textbf{an} den rinc rîten.\\ 
 & diu wîse, niht diu tumbe,\\ 
 & reit den rinc alumbe.\\ 
 & man zeigte ir, \textbf{dâ} Artus saz,\\ 
10 & gein dem si \textbf{gruozes} niht vergaz.\\ 
 & \textbf{in} franzois was ir sprâche.\\ 
 & si \textbf{warte}, daz ein râche\\ 
 & ûf si verkoren wære\\ 
 & unde daz man hôrte ir mære.\\ 
15 & den künic unde die künigîn\\ 
 & bat si helfe unde an ir rede sîn.\\ 
 & si kêrte von in al zehant,\\ 
 & dâ si Parzivalen vant\\ 
 & bî Artuse nâhen.\\ 
20 & si begunde ir \textbf{sprunges} gâhen\\ 
 & von \textit{dem} pferde ûffez gras.\\ 
 & si viel mit \textbf{zuht}, diu an ir was,\\ 
 & Parzival an sînen vuoz.\\ 
 & \textbf{si} warp alweinde \textbf{sînen} gruoz,\\ 
25 & daz er zorn gein ir verlür\\ 
 & unde \textbf{alle wîs} \textbf{ûf} si verkür.\\ 
 & Artus unde Feirafiz,\\ 
 & an den gewerp legten \textbf{si ir} vlîz.\\ 
 & Parzival truoc \textbf{gein ir} haz;\\ 
30 & durch \textbf{vriunde} bet er des vergaz\\ 
\end{tabular}
\scriptsize
\line(1,0){75} \newline
G I L M Z \newline
\line(1,0){75} \newline
\textbf{1} \textit{Initiale} G L M Z  \textbf{15} \textit{Initiale} I  \newline
\line(1,0){75} \newline
\textbf{1} in vollen] envollen G wol I in vollem Z  $\cdot$ zelt] enzelt I \textbf{2} si] \textit{om.} I \textbf{4} rîch unde tiure] tevr vnd rich Z  $\cdot$ allen] \textit{om.} L allem Z \textbf{9} zeigte] sagt I zeiget L (Z) \textbf{10} gruozes] gruͤzens I (L) (Z) \textbf{11} franzois] franzoẏs G fronzoys I franzoys L frantzois Z \textbf{12} warte] warp L Z werte M \textbf{13} verkoren] irkorn M \textbf{17} kêrte] chert I (L) \textbf{18} Parzivalen] parcifalen G parzifaln I M parzifalen L parcifaln Z  $\cdot$ vant] sitzen vant M (Z) \textbf{19} Artuse] artus I Z \textbf{20} sprunges] springens L \textbf{21} dem] ir G \textbf{22} zuht] zcuchten M \textbf{23} Parzival] parcifal G (Z) parzifal I Parzifale L M \textbf{24} sînen] vmme synen M (Z) \textbf{26} alle wîs] in allen wis I an kusse M ane kvss Z  $\cdot$ ûf si] Gein ir I \textbf{27} Feirafiz] feirefiz G Z ferefiz L feirefisz M \textbf{28} an den Gewerp leten si ir vliz I · Legten an den gewerb ir fliz L · Anden giwerp legeten irn (\textit{om.} Z ) fliesz M (Z) \textbf{29} Parzival] parcifal G (Z) parzifal I (L) Parczifal M \textbf{30} des] das M \newline
\end{minipage}
\hspace{0.5cm}
\begin{minipage}[t]{0.5\linewidth}
\small
\begin{center}*T
\end{center}
\begin{tabular}{rl}
 & \begin{large}S\end{large}enfteclîche, in \textbf{vollen} zelt\\ 
 & kam si \textbf{geriten} über velt.\\ 
 & ir zoum, ir satel, ir runzît\\ 
 & was rîche und tiure âne allen strît.\\ 
5 & man liez si an den zîten\\ 
 & \textbf{in} den rinc rîten.\\ 
 & diu wîse, niht diu tumbe,\\ 
 & reit den rinc al umbe.\\ 
 & man zeiget ir, \textbf{dâ} Artus saz,\\ 
10 & gein dem si \textbf{gruozes} niht vergaz.\\ 
 & \textbf{in} franzoys was ir sprâche.\\ 
 & si \textbf{warp}, daz ein râche\\ 
 & ûffe si verkorn wære\\ 
 & und daz man hôrte ir mære.\\ 
15 & den künec und die künegin\\ 
 & bat si helfe und an ir rede sîn.\\ 
 & si kêrte von i\textit{n} al zehant,\\ 
 & dâ si Parcifalen \textbf{sitzen} vant\\ 
 & bî Artuse nâhen.\\ 
20 & si begunde ir \textbf{sprunges} gâhen\\ 
 & von dem pferde ûf daz gras.\\ 
 & si viel mit \textbf{zuht}, diu an ir was,\\ 
 & Parcifale an sînen vuoz.\\ 
 & \textbf{si} war\textit{p} a\textit{l} weinende \textbf{umb} \textbf{sînen} gruoz,\\ 
25 & \textbf{sô} daz er zorn gein ir verlür\\ 
 & und \textbf{âne kus} \textbf{an} si verkür.\\ 
 & Artus und Ferefis\\ 
 & an den gewerp leget\textit{en} vlîz.\\ 
 & Parcifal truoc \textbf{gein ir} haz;\\ 
30 & durch \textbf{vriundinne} bete er des vergaz\\ 
\end{tabular}
\scriptsize
\line(1,0){75} \newline
U Q R \newline
\line(1,0){75} \newline
\textbf{1} \textit{Initiale} U R  \newline
\line(1,0){75} \newline
\textbf{1} Senfftiglich vnd doch en vollen zelt Q · Senfftenklich vnd doch in follem zelt R \textbf{9} zeiget] zeigte Q R  $\cdot$ dâ] do Q \textbf{13} si] die Q  $\cdot$ verkorn] erkorn R \textbf{14} daz] \textit{om.} Q  $\cdot$ ir] ye Q \textbf{15} den] Der R \textbf{16} und] \textit{om.} Q  $\cdot$ sîn] sy R \textbf{17} in] im U \textbf{18} dâ] Do Q R  $\cdot$ Parcifalen] partzifalen Q parczifallen R \textbf{19} Artuse] Artusen R \textbf{22} zuht] zᵫchtten R \textbf{23} Parcifale] Parzifale U Partzifalen Q Parczifaln R \textbf{24} warp] wart U  $\cdot$ al] alle U alleine R \textbf{25} \textit{statt 779.25-26 (variierende Versdoppelungen):} So das er zorn gen ir verkus / Vnd one has vff sy verlus / Och sin zorn an ir verkúr / Vnd einen kus vff sy erkur R  \textbf{26} an] vff Q \textbf{27} Ferefis] feirefisz Q feirefis R \textbf{28} legeten] leget U \textbf{29} Parcifal] Parzifal U Partzifal Q Parczifal R \textbf{30} vriundinne] frewnde Q (R) \newline
\end{minipage}
\end{table}
\end{document}
