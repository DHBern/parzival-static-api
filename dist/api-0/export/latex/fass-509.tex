\documentclass[8pt,a4paper,notitlepage]{article}
\usepackage{fullpage}
\usepackage{ulem}
\usepackage{xltxtra}
\usepackage{datetime}
\renewcommand{\dateseparator}{.}
\dmyyyydate
\usepackage{fancyhdr}
\usepackage{ifthen}
\pagestyle{fancy}
\fancyhf{}
\renewcommand{\headrulewidth}{0pt}
\fancyfoot[L]{\ifthenelse{\value{page}=1}{\today, \currenttime{} Uhr}{}}
\begin{document}
\begin{table}[ht]
\begin{minipage}[t]{0.5\linewidth}
\small
\begin{center}*D
\end{center}
\begin{tabular}{rl}
\textbf{509} & \begin{large}G\end{large}awan bôt ir sînen gruoz.\\ 
 & er sprach: "ob ich erbeizen muoz\\ 
 & mit iweren hulden, vrouwe,\\ 
 & ob ich iuch des willen schouwe,\\ 
5 & daz ir mich gerne bî iu hât,\\ 
 & grôz \textbf{rîwe} mich bî vreuden lât:\\ 
 & sô\textbf{ne} wart nie rîter \textbf{mêr sô} vrô.\\ 
 & \textbf{mîn lîp} \textbf{muoz ersterben} \textbf{sô},\\ 
 & daz mir nimmer wîp gevellet baz."\\ 
10 & "deist êt \textbf{wol}, nû weiz \textbf{ich ouch} daz."\\ 
 & sölch was ir rede, dô si an in sach.\\ 
 & ir süezer munt mêr \textbf{dannoch} sprach:\\ 
 & "nû \textbf{en}lobt mich niht ze sêre;\\ 
 & ir enpfâht \textbf{es} lîhte unêre.\\ 
15 & ich\textbf{n} wil niht, daz ieslîch munt\\ 
 & \textbf{gein mir tuo sîn prüeven} kunt.\\ 
 & wære mîn lop gemeine\\ 
 & - daz hieze \textbf{ein} wirde kleine -\\ 
 & dem wîsen unt dem tumben,\\ 
20 & dem slehten unt dem krumben,\\ 
 & wâ riht ez sich danne vür\\ 
 & nâch der werdecheit kür?\\ 
 & ich sol mîn lop behalten,\\ 
 & daz es die wîsen walten.\\ 
25 & ich\textbf{n} weiz niht, hêrre, wer ir sît;\\ 
 & iwers rîtens wære von mir zît.\\ 
 & mîn p\textit{r}üeven lât iuch doch niht vrî:\\ 
 & ir sît mînem herzen bî -\\ 
 & verre ûzerhalp, niht drinne!\\ 
30 & gert ir mîner minne,\\ 
\end{tabular}
\scriptsize
\line(1,0){75} \newline
D \newline
\line(1,0){75} \newline
\textbf{1} \textit{Initiale} D  \newline
\line(1,0){75} \newline
\textbf{27} prüeven] pvͤuen D \newline
\end{minipage}
\hspace{0.5cm}
\begin{minipage}[t]{0.5\linewidth}
\small
\begin{center}*m
\end{center}
\begin{tabular}{rl}
 & \begin{large}G\end{large}awan bôt ir sînen gruoz.\\ 
 & er sprach: "ob ich erbeizen muoz\\ 
 & mit iuwern hulden, vrouwe,\\ 
 & ob ich iuch des willen schouwe,\\ 
5 & daz \textit{i}r mich gern bî iu hât,\\ 
 & grôz \textbf{riuwen} mich bî vröuden lât:\\ 
 & sô wart nie ritter \textbf{alsô} vrô.\\ 
 & \textbf{mîn lîp} \textbf{ersterben muoz} \textbf{alsô},\\ 
 & daz mir nimmer wîp gevellet baz."\\ 
10 & "daz ist eht, nû weiz \textbf{ich} daz."\\ 
 & solich was ir rede, dô si an in sach.\\ 
 & ir süezer munt \textit{m}ê \textbf{dar nâch} sprach:\\ 
 & "nû lobet mich niht zuo sêre;\\ 
 & ir enpfâhet \textbf{sîn} lîht unêre.\\ 
15 & ich wil niht, daz ieglîch munt\\ 
 & \textbf{gegen mir tuo sîn prüeven} kunt.\\ 
 & wær mîn lop gemeine\\ 
 & - daz hiez \textbf{ein} wirde kleine -\\ 
 & dem wîsen und dem tumben,\\ 
20 & dem slehte\textit{n} und dem krumben,\\ 
 & wâ rihte ez sich dan vür\\ 
 & nâch der wirdicheit kür?\\ 
 & ich sol mîn lop behalten,\\ 
 & daz es die wîsen walten.\\ 
25 & ich weiz niht, hêrre, wer ir sît;\\ 
 & iuwers rîtens wær von mir zît.\\ 
 & mîn brüefen lât iuch doch niht vrî:\\ 
 & ir sît mînem herzen bî -\\ 
 & v\textit{e}rre ûzerhalp, niht drinne!\\ 
30 & gert ir mîner minne,\\ 
\end{tabular}
\scriptsize
\line(1,0){75} \newline
m n o \newline
\line(1,0){75} \newline
\textbf{1} \textit{Initiale} m   $\cdot$ \textit{Capitulumzeichen} n  \newline
\line(1,0){75} \newline
\textbf{5} daz] [Dar]: Das o  $\cdot$ ir] er m \textbf{6} grôz riuwen] Grosse ruͯwe n (o) \textbf{7} alsô] me so n o \textbf{11} was] \sout{vasnaht} was m  $\cdot$ dô] so n  $\cdot$ in] jme n \textbf{12} mê] nie m \textbf{14} lîht] vil lichte n \textbf{16} mir] \textit{om.} o \textbf{18} \textit{Die Versenden 509.18-22 fehlen (abgerissen)} o   $\cdot$ hiez] \textit{om.} o \textbf{20} slehten] slehtte m slehte: o \textbf{27} mîn] Nuͯ n  $\cdot$ vrî] [wit]: fri o \textbf{29} verre] Forre m o \newline
\end{minipage}
\end{table}
\newpage
\begin{table}[ht]
\begin{minipage}[t]{0.5\linewidth}
\small
\begin{center}*G
\end{center}
\begin{tabular}{rl}
 & \begin{large}G\end{large}awan bôt ir sînen gruoz.\\ 
 & er sprach: "ob ich erbeizen muoz\\ 
 & mit iuwern hulden, vrouwe,\\ 
 & ob ich iuch des willen schouwe,\\ 
5 & daz ir mich gern bî iu hât,\\ 
 & grôz \textbf{riuwe} mich bî vröuden lât:\\ 
 & sô \textbf{ne}wart nie rîter \textbf{mêr sô} vrô.\\ 
 & \textbf{ich} \textbf{muoz sterben} \textbf{lîhte} \textbf{alsô},\\ 
 & daz mir nimmer wîp gevallet baz."\\ 
10 & "deist êt \textbf{wol}, nû weiz \textbf{ich ouch} daz."\\ 
 & solch was ir rede, dô si an in sach.\\ 
 & ir süezer munt mêr \textbf{dannoch} sprach:\\ 
 & "nû\textbf{ne} lobet mich niht ze sêre;\\ 
 & ir enpfâht \textbf{es} lîhte unêre.\\ 
15 & ich\textbf{ne} wil niht, daz ieslîch munt\\ 
 & \textit{\textbf{tuo sîn brüeven gein mir}} \textit{kunt.}\\ 
 & wære mîn lop gemeine\\ 
 & - daz hiez \textbf{ein} wirde kleine -\\ 
 & dem wîsen unde dem tumben,\\ 
20 & dem slehten unde dem krumben,\\ 
 & wâ rihte ez sich danne vür\\ 
 & nâch der werdecheite kür?\\ 
 & ich sol mîn lop behalten,\\ 
 & daz es die wîsen walten.\\ 
25 & ich\textbf{ne} weiz niht, hêrre, wer ir sît;\\ 
 & iuwers rîtens wære von mir zît.\\ 
 & mîn prüeven lât iuch doch niht vrî:\\ 
 & ir sît mînem herzen bî -\\ 
 & verre ûzerhalp, niht drinne!\\ 
30 & gert ir mîner minne,\\ 
\end{tabular}
\scriptsize
\line(1,0){75} \newline
G I L M Z Fr22 \newline
\line(1,0){75} \newline
\textbf{1} \textit{Initiale} G I L Z  \textbf{17} \textit{Initiale} I  \newline
\line(1,0){75} \newline
\textbf{5} hât] lat Z \textbf{6} grôz] Vroudin M  $\cdot$ vröuden] vrode L sorgen M rewen Z \textbf{7} newart] wart L M \textbf{8} ich] Min lip L M Z  $\cdot$ sterben lîhte alsô] ersterben so L (M) Z \textbf{9} mir] \textit{om.} Z \textbf{10} êt] \textit{om.} Z  $\cdot$ ouch] \textit{om.} I M \textbf{11} \textit{Die Verse 509.11-12 fehlen} I   $\cdot$ solch] Suͯsz L  $\cdot$ dô] da M Z  $\cdot$ an in] in an L  $\cdot$ sach] gesach Z \textbf{13} nûne] Nu M  $\cdot$ lobet] lop I  $\cdot$ ze] so I L \textbf{14} ir enpfâht es] ir enphahet sin I Jrn pfahet ez L \textbf{15} ichne] Jch L  $\cdot$ ieslîch] eyn iclich M \textbf{16} \textit{Vers 509.16 fehlt} G   $\cdot$ Gein mir tuͯ sin pruͯfen kuͯnt L (M) (Z) (Fr22) \textbf{18} ein] ich I \textbf{19} dem wîsen] den wisen I (L) (M)  $\cdot$ dem tumben] den tunben I (L) (M) (Fr22) \textbf{20} dem slehten] den slehten I (L) (M)  $\cdot$ dem krumben] den crunben I (L) (M) \textbf{21} rihte] riet I rýht L (Z) richtet M \textbf{22} werdecheite] wertlichin M \textbf{25} ichne weiz] ich waiz I (L) (M)  $\cdot$ niht] \textit{om.} L \textbf{26} iuwers rîtens] Vwir varn M \textbf{27} iuch doch] evch I ouch uͯch L doch ivch Fr22 \newline
\end{minipage}
\hspace{0.5cm}
\begin{minipage}[t]{0.5\linewidth}
\small
\begin{center}*T
\end{center}
\begin{tabular}{rl}
 & \begin{large}G\end{large}awan bôt ir sînen gruoz.\\ 
 & er sprach: "ob ich erbeizen muoz\\ 
 & Mit iuwern hulden, vrouwe,\\ 
 & ob ich iuch des willen schouwe,\\ 
5 & daz ir mich gerne bî iu hât,\\ 
 & Grôz \textbf{riuwe} mich bî vreuden lât:\\ 
 & Sô\textbf{n} wart nie rîter \textbf{mê sô} vrô.\\ 
 & \textbf{Mîn lîp} \textbf{muoz ersterben} \textbf{sô},\\ 
 & daz mir niemer wîp gevellet baz."\\ 
10 & "daz ist ê\textit{t} \textbf{wol}, nû weiz \textbf{ouch ich} daz."\\ 
 & Sölch was ir rede, dô si an i\textit{n} sach.\\ 
 & ir süezer munt mê \textbf{dannoch} sprach:\\ 
 & "Nû \textbf{en}lobt mich niht ze sêre;\\ 
 & ir enpfâht \textbf{es} lîhte unêre.\\ 
15 & ich \textbf{en}wil niht, daz \textbf{ein} ieslîch munt\\ 
 & \textbf{Gein mir tuo sîn prüeven} kunt.\\ 
 & wære mîn lop gemeine\\ 
 & - daz hieze wi\textit{rde} kleine -\\ 
 & dem wîsen unde dem tumben,\\ 
20 & dem slehten unde dem krumben,\\ 
 & wâ riht ez sich danne vür\\ 
 & Nâch der werdecheite kür?\\ 
 & ich \textit{sol} mîn lop behalten,\\ 
 & daz e\textit{s} die wîsen walten.\\ 
25 & ich\textbf{n} weiz niht, \textit{hêrr}e, wer ir sît;\\ 
 & iuwers rîtens wære von mir zît.\\ 
 & Mîn prüeven lât iuch doch niht vrî:\\ 
 & ir sît mînem herzen bî -\\ 
 & verre ûzerhalp, niht drinne!\\ 
30 & gert ir mîner minne,\\ 
\end{tabular}
\scriptsize
\line(1,0){75} \newline
T U V W O Q R Fr40 \newline
\line(1,0){75} \newline
\textbf{1} \textit{Initiale} T U V W O R Fr40  \textbf{3} \textit{Majuskel} T  \textbf{6} \textit{Majuskel} T  \textbf{7} \textit{Majuskel} T  \textbf{8} \textit{Majuskel} T  \textbf{11} \textit{Majuskel} T  \textbf{13} \textit{Majuskel} T  \textbf{16} \textit{Majuskel} T  \textbf{22} \textit{Majuskel} T  \textbf{27} \textit{Majuskel} T  \newline
\line(1,0){75} \newline
\textbf{1} Gawan] [Gal]: Gawan Q Gawain R  $\cdot$ bôt ir] bat in Q  $\cdot$ sînen] súszen R \textbf{3} hulden] willen U  $\cdot$ vrouwe] frawen Q \textbf{4} iuch] iv T \textit{om.} Q  $\cdot$ willen schouwe] wille schawen Q willes schowe R \textbf{6} riuwe] treúwe W  $\cdot$ mich] mich gerne Fr40  $\cdot$ lât] labt W \textbf{7} Sôn wart] So wart O Q Do enwart R  $\cdot$ mê] \textit{om.} Q R \textbf{8} sô] also Q \textbf{9} niemer] dehein O \textbf{10} êt] er T Fr40 \textit{om.} U auch Q  $\cdot$ ouch ich] ich auch U (V) (O) (Q) (R) (Fr40) ich W \textbf{11} Sölch] Sælich O (Q)  $\cdot$ an in] an ich T in an U V O in R an [i*]: in Fr40 \textbf{12} mê dannoch] dannoch mer O me darnach R \textbf{13} enlobt] inlobete U lobt O  $\cdot$ mich] ich Q  $\cdot$ ze] so Q \textbf{14} ir] [er]: jr T Es Q  $\cdot$ es] [er]: es T \textit{om.} W  $\cdot$ unêre] vor vnere U \textbf{15} enwil] wil W O Q R Fr40 \textbf{16} tuo] tűn Q  $\cdot$ sîn] [sine]: sin V \textit{om.} R  $\cdot$ prüeven] lobes R \textbf{17} lop] lob niht O lib R \textbf{18} hieze] heise R  $\cdot$ wirde] wider T ich werde U (W) (O) (Q) [*]: eine wurde  V ich wurde Fr40 \textbf{20} Den schlechtten vnd den krumben R  $\cdot$ krumben] [tvmben]: krvmben T \textbf{21} riht] richte W \textbf{23} sol] \textit{om.} T \textbf{24} es] \textit{om.} R \textbf{25} ichn] Jch O (Q) (R) (Fr40)  $\cdot$ hêrre] frowe T \textit{om.} V  $\cdot$ ir] ir herre V [is]: ir R \textbf{26} iuwers rîtens] Jwer varn O \textbf{27} iuch] iv T \textbf{28} mînem] meinen Fr40 \textbf{29} verre] Were Q  $\cdot$ niht] vnd nit R \textbf{30} gert] Gern V \newline
\end{minipage}
\end{table}
\end{document}
