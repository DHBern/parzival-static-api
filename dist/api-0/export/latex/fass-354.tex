\documentclass[8pt,a4paper,notitlepage]{article}
\usepackage{fullpage}
\usepackage{ulem}
\usepackage{xltxtra}
\usepackage{datetime}
\renewcommand{\dateseparator}{.}
\dmyyyydate
\usepackage{fancyhdr}
\usepackage{ifthen}
\pagestyle{fancy}
\fancyhf{}
\renewcommand{\headrulewidth}{0pt}
\fancyfoot[L]{\ifthenelse{\value{page}=1}{\today, \currenttime{} Uhr}{}}
\begin{document}
\begin{table}[ht]
\begin{minipage}[t]{0.5\linewidth}
\small
\begin{center}*D
\end{center}
\begin{tabular}{rl}
\textbf{354} & \begin{large}G\end{large}ar dirre worte hôre\\ 
 & \textbf{kom} Gawane in sîn ôre.\\ 
 & \textbf{die} rede lât sîn, als si nû stê,\\ 
 & \textbf{nû} hœret, wie ez der stat ergê:\\ 
5 & ein \textbf{schifreht} wazzer vür si vlôz\\ 
 & durch eine \textbf{brücke steinîn} grôz,\\ 
 & niht gein der vîende want.\\ 
 & anderhalp was unverhert daz lant.\\ 
 & \textbf{ein} marschalc kom geriten sân.\\ 
10 & vür die \textbf{brücken} ûf den plân\\ 
 & nam er herberge wît.\\ 
 & sîn hêrre kom an \textbf{rehter} zît\\ 
 & unt ander, die dâ solden komen.\\ 
 & ich sagez iu, hât irs niht vernomen,\\ 
15 & wer in des wirtes hilfe reit\\ 
 & unt wer durch in mit triwen streit.\\ 
 & Im kom von Brevigariez\\ 
 & sîn bruoder, \textbf{der herzoge} Marangliez.\\ 
 & durch den kômen zwêne ritter snel:\\ 
20 & der werde künec Schirniel,\\ 
 & der truoc krône ze Lyrivoyn,\\ 
 & als tet sîn bruoder ze Avendroyn.\\ 
 & Dô die burgære \textbf{gesâhen},\\ 
 & daz in helfe wolde nâhen,\\ 
25 & daz ê des was ir aller rât,\\ 
 & daz dûhte si \textbf{dô} \textbf{ein} missetât.\\ 
 & \multicolumn{1}{l}{ - - - }\\ 
 & \multicolumn{1}{l}{ - - - }\\ 
 & der vürste Lyppaut \textbf{dô} sprach:\\ 
 & "owê, daz Bearosche ie \textbf{geschach},\\ 
 & daz ir porten suln vermûrt sîn!\\ 
30 & wan swenne ich gein dem hêrren mîn\\ 
\end{tabular}
\scriptsize
\line(1,0){75} \newline
D \newline
\line(1,0){75} \newline
\textbf{1} \textit{Initiale} D  \textbf{17} \textit{Majuskel} D  \textbf{23} \textit{Majuskel} D  \newline
\line(1,0){75} \newline
\textbf{17} Brevigariez] Brevigarîez D \textbf{18} Marangliez] Maranglîez D \textbf{20} Schirniel] Scirniel D \textbf{27} Lyppaut] Lyppaot D \textbf{28} Bearosche] Bearosce D \newline
\end{minipage}
\hspace{0.5cm}
\begin{minipage}[t]{0.5\linewidth}
\small
\begin{center}*m
\end{center}
\begin{tabular}{rl}
 & gar dirre worte hôre\\ 
 & \textbf{kam} Gawane in sîn ôre.\\ 
 & \textbf{die} rede lât sîn, als si nû stê,\\ 
 & \textbf{und} hœret, wie ez der stat ergê:\\ 
5 & ein \textbf{schifreht} wazzer vür si vlôz\\ 
 & durch eine \textbf{\textit{br}ücke steinîn} grôz,\\ 
 & niht gegen der vîende want.\\ 
 & anderhalp was un\textit{v}erhert daz lant.\\ 
 & \textbf{ein} marschalc kam geriten sân.\\ 
10 & vür die \textbf{brücke} ûf den plân\\ 
 & \textbf{dô} nam er herberge wît.\\ 
 & sîn hêrre kam an \textbf{rehter} zît\\ 
 & und ander, die d\textit{â} solten komen.\\ 
 & ich sag\textit{e} ez iu, habt irs niht venomen,\\ 
15 & wer in des wirtes helfe reit\\ 
 & und wer durch in mit triuwen streit.\\ 
 & im kam von Brevigariez\\ 
 & sîn bruoder, \textbf{duc} Marangliez.\\ 
 & durch den kômen zwêne rîter snel:\\ 
20 & der werde künic Schirniel,\\ 
 & der truoc krône ze Lirw\textit{oi}n,\\ 
 & als tet sîn bruoder ze Avendro\textit{in}.\\ 
 & \begin{large}D\end{large}ô die burgære \textbf{sâhen},\\ 
 & daz in helfe wolte nâhen,\\ 
25 & daz ê d\textit{e}s was ir aller rât,\\ 
 & daz dûhte si \textbf{d\textit{ô}} \textbf{ein} missetât,\\ 
 & ich meine, daz si dâ vor\\ 
 & vermûret heten ir tor.\\ 
 & der vürste Lipp\textit{ou}t \textbf{dô} \textbf{selbe} sprach:\\ 
 & "ouwê, daz Bearosche ie \textbf{geschach},\\ 
 & daz ir porte sullen vermûret sîn!\\ 
30 & wand wenne ich gegen dem hêrren mîn\\ 
\end{tabular}
\scriptsize
\line(1,0){75} \newline
m n o \newline
\line(1,0){75} \newline
\textbf{3} \textit{Capitulumzeichen} n  \textbf{23} \textit{Initiale} m n  \newline
\line(1,0){75} \newline
\textbf{1} worte] worte worte o  $\cdot$ hôre] here n \textbf{2} Gawane] gawan n gewan o \textbf{3} nû stê] muste o \textbf{5} vlôz] floch o \textbf{6} brücke] stucke m bruͯcke sie o \textbf{8} unverhert] vnner hert m vnuerseret o \textbf{10} brücke] burg n o \textbf{13} dâ] do m n o \textbf{14} sage] sagette m \textbf{15} helfe] helffes o \textbf{16} streit] [reit]: streit o \textbf{17} im] Jn n  $\cdot$ Brevigariez] brevigaries n brebigaries o \textbf{18} duc Marangliez] dugmarangleis n dúg maranglies o \textbf{19} den] den \sout{kunig} m \textbf{20} Schirniel] scirniel m n o \textbf{21} Lirwoin] lirwum m liriwoum n Lirwaimm o \textbf{22} tet] der n o  $\cdot$ Avendroin] auendrom m auendroum n anendromm o \textbf{25} ê des] e das m vor hin n (o) \textbf{26} dô] da m \textbf{27} Lippout] lipouat m lipaot n o  $\cdot$ dô] \textit{om.} n o \textbf{28} Bearosche] bearosce m bearosc n brahosc o \textbf{29} porte] porten n \newline
\end{minipage}
\end{table}
\newpage
\begin{table}[ht]
\begin{minipage}[t]{0.5\linewidth}
\small
\begin{center}*G
\end{center}
\begin{tabular}{rl}
 & gar dirre worte hôre\\ 
 & \textbf{kom} Gawan in sîn ôre.\\ 
 & \textbf{die} rede lât sîn, als si nû stê,\\ 
 & \textbf{unde} hœrt \textbf{ouch}, wiez der stat ergê:\\ 
5 & ein \textbf{schifreche} wazzer vür si vlôz\\ 
 & durch eine \textbf{brücke steinîn} grôz,\\ 
 & niht gein der vînde want.\\ 
 & anderhalp was unverhert daz lant.\\ 
 & \textbf{ein} marschalc kom geriten sân.\\ 
10 & vür die \textbf{burc} ûf den plân\\ 
 & nam er herberge wît.\\ 
 & sîn hêrre kom an \textbf{rehter} zît\\ 
 & unde ander, die dâ solten komen.\\ 
 & ich sagez iu, habt irs niht vernomen,\\ 
15 & \begin{large}W\end{large}er ins wirtes helfe reit\\ 
 & unde wer durch in mit triwen streit.\\ 
 & im kom von Brevegariez\\ 
 & sîn bruoder, \textbf{duc} Marangliez.\\ 
 & durch den kômen zwêne rîter snel:\\ 
20 & der werde künic Tschirnel,\\ 
 & der truoc krône ze Liravoyn,\\ 
 & als tet sîn bruoder ze Avendroyn.\\ 
 & dô die burgære \textbf{sâhen},\\ 
 & daz in helfe wolte nâhen,\\ 
25 & daz ê des was ir aller rât,\\ 
 & daz dûhte si \textbf{dô} missetât.\\ 
 & \multicolumn{1}{l}{ - - - }\\ 
 & \multicolumn{1}{l}{ - - - }\\ 
 & der vürste Libaut \textit{\textbf{selbe}} sprach:\\ 
 & "owê, daz Bearotsche ie \textbf{geschach},\\ 
 & daz ir porte\textit{n} sulen vermûret sîn!\\ 
30 & wan swenne ich gein dem hêrren mîn\\ 
\end{tabular}
\scriptsize
\line(1,0){75} \newline
G I O L M Q R Z Fr39 \newline
\line(1,0){75} \newline
\textbf{1} \textit{Initiale} I O L Z Fr39   $\cdot$ \textit{Capitulumzeichen} R  \textbf{15} \textit{Initiale} G I  \newline
\line(1,0){75} \newline
\textbf{1} gar] ÷ar O \textbf{2} Gawan] [g]: Gawan R [Gawam]: Gawan Z  $\cdot$ sîn] \textit{om.} Q \textbf{3} \textit{Versfolge 354.4-3} M   $\cdot$ stê] stet I sten Q \textbf{4} unde] Nun Q  $\cdot$ ouch] \textit{om.} I O Q R  $\cdot$ ergê] erget I \textbf{5} schifreche] shifrehez I schifrætich O \textbf{6} brücke steinîn] steinin brukke I steinbruͯgge L (Fr39) brucken steyn M bruge steine R \textbf{7} \textit{Verfolge 354.8-7} R   $\cdot$ der] \textit{om.} Q  $\cdot$ vînde] wende L Fr39 vigenden R  $\cdot$ want] hand R \textbf{8} unverhert] vngeuert her I \textbf{10} burc] brukke I (O) (M) (Q) (R) (Z)  $\cdot$ den] dem Fr39 \textbf{11} er herberge] irherberge M \textbf{12} hêrre kom] chomen was O  $\cdot$ an rehter] ander rehten I \textbf{13} dâ] do Q Fr39 \textbf{14} ich] Vnd Q  $\cdot$ iu] \textit{om.} M  $\cdot$ habt] hap Q \textbf{15} Wer] Der I  $\cdot$ reit] riet R \textbf{16} wer] \textit{om.} Z  $\cdot$ streit] strit R \textbf{17} Brevegariez] breuegariez I Brevgariez O Brevegariesz M breuegaries Q (R) Brenegariez Z \textbf{18} sîn] Dein Q  $\cdot$ duc] cluc M dux Q herczog R ::: Fr39  $\cdot$ Marangliez] maragliez G I manangliesz M maranglies Q (R) ::: Fr39 \textbf{19} durch den] Duͯrch in L (Fr39) Da M \textbf{20} Tschirnel] Schirmel I tschirmel O tishirmel L schirinel M schirniel Q (R) Ischirviel Z shirniel Fr39 \textbf{21} krône] \textit{om.} O  $\cdot$ ze Liravoyn] zelirauoin I zelyravoyn O zuͯ lyravoýn L zcu litavoyn M zu lyravoyn Q ze Lyravoin R (Z) :::oyn Fr39 \textbf{22} ze Avendroyn] zeauentroin I zuͯ Avendroýn L zcu auendroyn M (Fr39) zu Auendroin R zavendroin Z \textbf{23} dô] Da M R Z  $\cdot$ sâhen] alle sahen I O L (M) Q Fr39 alle sagen Z \textbf{24} helfe] zuͯ helfe L (Fr39) \textbf{25} des] da M \textbf{26} dô] do ein O L R Z Fr39 da eyn M doch eyn Q \textbf{27} Libaut] Lybavt O Z libauͯt L libayt M libant Q Lybant R  $\cdot$ selbe] do G selben M selber Q \textbf{28} daz] das es Q R  $\cdot$ Bearotsche] Bearotsch O L (M) (Fr39) bearosche Q Bearoshe R  $\cdot$ ie] daz îe O  $\cdot$ geschach] gisach M \textbf{29} daz] Da Q  $\cdot$ porten] borte G phorten M (Q)  $\cdot$ sulen] sullet R \textbf{30} swenne] wenne L (M) (Q) (R)  $\cdot$ gein dem] dein R \newline
\end{minipage}
\hspace{0.5cm}
\begin{minipage}[t]{0.5\linewidth}
\small
\begin{center}*T
\end{center}
\begin{tabular}{rl}
 & Gar dirre worte hôre\\ 
 & \textbf{kômen} Gawan in sîn ôre.\\ 
 & \textbf{dise} rede lât sîn, als si nû stê,\\ 
 & \textbf{nû} hœret, wiez der stat ergê:\\ 
5 & ein \textbf{schifreche} wazzer vür si vlôz\\ 
 & \textit{durch eine \textbf{steinîn brücke} grôz,}\\ 
 & niht gegen der vîende want.\\ 
 & anderhalp was un\textit{v}er\textit{her}t \textit{daz lant}.\\ 
 & \textbf{\begin{large}S\end{large}în} marschalc kom geriten sân.\\ 
10 & vür die \textbf{burc} ûf den plân\\ 
 & nam er herberge wît.\\ 
 & sîn hêrre kom an \textbf{der} zît\\ 
 & unde ander, die dâ solten komen.\\ 
 & ich sagz iu, habt irs niht vernomen,\\ 
15 & wer in des wirtes helfe reit\\ 
 & unde wer durch in mit triuwen streit.\\ 
 & Im kom von Prevegariez\\ 
 & sîn bruoder Marangliez.\\ 
 & durch den kômen zwêne rîter snel:\\ 
20 & der werde künec Tschirniel,\\ 
 & der truoc krône ze Lyrivoyn,\\ 
 & als tet sîn bruoder zEvendroyn.\\ 
 & Dô die burgære \textbf{alle} \textbf{sâhen},\\ 
 & daz in helfe wolte nâhen,\\ 
25 & daz ê des was ir aller rât,\\ 
 & daz dûhte si \textbf{ein} missetât.\\ 
 & \multicolumn{1}{l}{ - - - }\\ 
 & \multicolumn{1}{l}{ - - - }\\ 
 & Der vürste Lybaut \textbf{selbe} sprach:\\ 
 & "ouwê, daz \textbf{ich} Bearosch ie \textbf{gesach},\\ 
 & daz ir porten suln vermûret sîn!\\ 
30 & wan swenich gegen dem hêrren mîn\\ 
\end{tabular}
\scriptsize
\line(1,0){75} \newline
T V W \newline
\line(1,0){75} \newline
\textbf{1} \textit{Majuskel} T  \textbf{9} \textit{Initiale} T V  \textbf{17} \textit{Majuskel} T  \textbf{23} \textit{Majuskel} T  \textbf{27} \textit{Majuskel} T  \newline
\line(1,0){75} \newline
\textbf{2} kômen] Koment V Kam W \textbf{3} dise] Die V W  $\cdot$ als si] als es W \textbf{4} nû] vnde V \textbf{6} \textit{Vers 354.6 fehlt (Zeile ausgespart)} T   $\cdot$ steinîn brücke] brucken von stainen W \textbf{8} ander halp was vnerkant T  $\cdot$ unverhert] verhert W \textbf{9} Sîn] Ein V W \textbf{12} an der] in rehter V (W) \textbf{13} dâ] do V W \textbf{14} sagz iu] sag eúchs W  $\cdot$ irs] ir W \textbf{17} Prevegariez] prevegaries V preuagaries W \textbf{18} bruoder] bruͦder duc V  $\cdot$ Marangliez] maranglies V W \textbf{20} Tschirniel] Schirnŷel T schirniel V schirmel W \textbf{21} Lyrivoyn] lẏrivoẏn V lyurwoyn W \textbf{22} zEvendroyn] Zevendôyn T ze Auendroẏn V zuͦ vendroyn W \textbf{23} alle] also V \textbf{26} \textit{nach 354.26:} Jch meine das sú do for / Vermvret hettent ir tor V   $\cdot$ si] sv́ do V \textbf{27} Lybaut] lybot W  $\cdot$ selbe] do selber V selber W \textbf{28} ich] \textit{om.} V W  $\cdot$ Bearosch] Bearotsc T bearotsche V betrosch W  $\cdot$ gesach] geschach V W \textbf{29} porten suln] porte solte W \textbf{30} swenich] wenn ich W \newline
\end{minipage}
\end{table}
\end{document}
