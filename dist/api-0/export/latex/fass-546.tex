\documentclass[8pt,a4paper,notitlepage]{article}
\usepackage{fullpage}
\usepackage{ulem}
\usepackage{xltxtra}
\usepackage{datetime}
\renewcommand{\dateseparator}{.}
\dmyyyydate
\usepackage{fancyhdr}
\usepackage{ifthen}
\pagestyle{fancy}
\fancyhf{}
\renewcommand{\headrulewidth}{0pt}
\fancyfoot[L]{\ifthenelse{\value{page}=1}{\today, \currenttime{} Uhr}{}}
\begin{document}
\begin{table}[ht]
\begin{minipage}[t]{0.5\linewidth}
\small
\begin{center}*D
\end{center}
\begin{tabular}{rl}
\textbf{546} & \textbf{\begin{large}E\end{large}ine wîle hete mirz} verstoln.\\ 
 & einer mûlinne voln\\ 
 & m\textit{ö}ht ir noch \textbf{ê} gewinnen.\\ 
 & ich kan iuch ander\textit{s} minnen:\\ 
5 & Sît \textbf{er} iuch dunket alsô wert,\\ 
 & vür \textbf{daz} ors, \textbf{des} ir \textbf{hie} gert,\\ 
 & habt iu den man, derz \textbf{gein mir} reit.\\ 
 & ist im daz liep oder leit,\\ 
 & dâ kêre ich mich wênec an."\\ 
10 & \textbf{Dô} vreute sich der schifman;\\ 
 & mit lachendem munde er sprach:\\ 
 & "sô rîche gâbe ich nie gesach,\\ 
 & \textbf{swem} si reht wære\\ 
 & zenpfâhene gebære.\\ 
15 & doch, \textbf{hêrre}, welt irs sîn mîn wer,\\ 
 & übergolten ist \textbf{mir} ger.\\ 
 & \textbf{vür wâr}, sîn \textbf{prîs} was ie sô hel,\\ 
 & vünf hundert ors \textit{starc} unt snel\\ 
 & \textbf{ungern ich} vür in næme,\\ 
20 & wandez mir niht gezæme.\\ 
 & Welt ir mich machen rîche,\\ 
 & sô werbet rîterlîche.\\ 
 & megt ir\textbf{s} sô gewaldec sîn,\\ 
 & antwürten in den kocken mîn,\\ 
25 & \textbf{sô} kunnet ir werdecheit wol tuon."\\ 
 & Dô sprach \textbf{des} \textbf{künec} Lotes sun:\\ 
 & "beidiu drîn unt \textbf{dar} vür,\\ 
 & \textbf{unz} inrehalben \textbf{iwerer} tür,\\ 
 & antwürte \textbf{ich in iu} gevangen."\\ 
30 & "sô wert ir wol enpfangen",\\ 
\end{tabular}
\scriptsize
\line(1,0){75} \newline
D Fr7 \newline
\line(1,0){75} \newline
\textbf{1} \textit{Initiale} D  \textbf{5} \textit{Majuskel} D  \textbf{10} \textit{Majuskel} D  \textbf{21} \textit{Majuskel} D  \textbf{26} \textit{Majuskel} D  \newline
\line(1,0){75} \newline
\textbf{1} mirz] mirs Fr7 \textbf{2} mûlinne] muͤlinnen Fr7 \textbf{3} möht ir] moht ir D (Fr7) \textbf{4} anders] ander D \textbf{5} er] ir Fr7 \textbf{18} starc] \textit{om.} D \textbf{26} Lotes] Lots D \newline
\end{minipage}
\hspace{0.5cm}
\begin{minipage}[t]{0.5\linewidth}
\small
\begin{center}*m
\end{center}
\begin{tabular}{rl}
 & \textbf{het mir ez ein wîle} verstoln.\\ 
 & einer m\textit{û}l\textit{în} voln\\ 
 & m\textit{ö}ht ir noch \textbf{ê} gewinnen.\\ 
 & ich kan iuch anders minnen:\\ 
5 & sît \textbf{er} iuch dunket alsô wert,\\ 
 & vür \textbf{daz} ros, \textbf{daz} ir \textbf{hie} gert,\\ 
 & habet iu den man, der ez \textbf{vür} reit.\\ 
 & ist im daz liep oder leit,\\ 
 & dâ kêre ich mich \textbf{vil} wênic an."\\ 
10 & \textbf{dô} vröwe\textit{t} \textit{s}ich der schifman;\\ 
 & mit lachendem munde er sprach:\\ 
 & "sô rîche gâb ich nie gesach,\\ 
 & \textbf{we\textit{m}} si reht wære\\ 
 & \textbf{und} zuo enpfâhen ge\textit{bæ}r\textit{e}.\\ 
15 & doch, \textbf{hêrre}, wellet irs sîn mîn wer,\\ 
 & übergolten ist \dag vil maniger\dag .\\ 
 & \textbf{vür wâr}, sîn \textbf{prîs} was ie sô \textit{h}el,\\ 
 & vünf hundert ros starc und snel\\ 
 & \textbf{ungerne ich} vür in næme,\\ 
20 & wan ez mir niht gezæme.\\ 
 & welt ir mich machen rîch,\\ 
 & sô werbet ritterlîch.\\ 
 & moget ir \textbf{sîn} sô gewaltic \textit{sîn},\\ 
 & \textbf{sô} antwürt i\textit{n} \textbf{mir} in den kocke\textit{n} mîn\\ 
25 & \textbf{und} ku\textit{nn}et ir wirdicheit wol tuon."\\ 
 & dô sprach \textbf{der} \textbf{künic} Lo\textit{t}es sun:\\ 
 & "beidiu drîn und vür,\\ 
 & \textbf{unz} innerhalp \textbf{der} tür,\\ 
 & antwürt \textbf{in ich} gevangen."\\ 
30 & "sô werdet ir wol enpfangen",\\ 
\end{tabular}
\scriptsize
\line(1,0){75} \newline
m n o \newline
\line(1,0){75} \newline
\newline
\line(1,0){75} \newline
\textbf{2} mûlîn] mẏle m muͯlen o \textbf{3} möht] Moht m (o) Mochten n \textbf{10} vröwet sich] froͯwet mich vnd sich m \textbf{13} wem] Wer m \textbf{14} gebære] ger m \textbf{15} wellet] wellen o \textbf{16} übergolten] V́wer gelten n \textbf{17} hel] snel m \textbf{20} ez] es es o \textbf{22} werbet] wirbet o \textbf{23} gewaltic sîn] gewalttig m \textbf{24} in mir] ir mir m  $\cdot$ den] der o  $\cdot$ kocken] kocker m \textbf{25} kunnet] kuͯmment m kundent o \textbf{26} der künic] des kv́nniges n (o)  $\cdot$ Lotes] lohs m n o \textbf{28} innerhalp] jnnenhalp n (o) \textbf{29} in ich] ich in úch n \newline
\end{minipage}
\end{table}
\newpage
\begin{table}[ht]
\begin{minipage}[t]{0.5\linewidth}
\small
\begin{center}*G
\end{center}
\begin{tabular}{rl}
 & \textbf{\begin{large}E\end{large}ine wîle het mirz} verstoln.\\ 
 & einer mûlinne voln\\ 
 & m\textit{ö}ht ir noch \textbf{ê} gewinnen.\\ 
 & ich kan \textit{iuch} anders minnen:\\ 
5 & sît \textbf{ir} iuch dunket als wert,\\ 
 & vür \textbf{diz} ors, \textbf{des} ir \textbf{hie} gert,\\ 
 & habet iu den man, derz \textbf{gein mir} reit.\\ 
 & ist im daz liep ode leit,\\ 
 & dâ kêre ich mich \textbf{vil} wênic an."\\ 
10 & \textbf{dô} vröute sich der schifman;\\ 
 & mit lachendem munde er sprach:\\ 
 & "sô rîche gâbe ich nie gesach,\\ 
 & \textbf{swem} si rehte wære\\ 
 & zempfâhen gebære.\\ 
15 & doch, \textbf{hêrre}, welt irs sîn mîn wer,\\ 
 & übergolten ist \textbf{mîn} ger.\\ 
 & \textbf{vür wâr}, sîn \textbf{brîs} was ie sô hel,\\ 
 & vünf hundert ors starc unde snel\\ 
 & \textbf{ungerne ich} vür in næme,\\ 
20 & wand ez mir niht gezæme.\\ 
 & welt ir mich machen rîche,\\ 
 & sô werbet rîterlîche.\\ 
 & muget ir \textbf{s\textit{în}} sô gewaltic sîn,\\ 
 & antwürte in in \textit{den} k\textit{o}cke\textit{n} mîn,\\ 
25 & \textbf{sô} kunnet ir werdecheit wol tuon."\\ 
 & dô sprach \textbf{der} \textbf{künic} Lotes sun:\\ 
 & "beidiu drîn unde \textbf{dar} vür,\\ 
 & \textbf{unze} innerhalp \textbf{iuwer} tür,\\ 
 & antwürte \textbf{ich iun} gevangen."\\ 
30 & "sô werdet ir wol enpfangen",\\ 
\end{tabular}
\scriptsize
\line(1,0){75} \newline
G I L M Z \newline
\line(1,0){75} \newline
\textbf{1} \textit{Initiale} G M Z  \textbf{11} \textit{Initiale} I  \newline
\line(1,0){75} \newline
\textbf{2} einer] Eyne M \textbf{3} möht] Moht G L (M)  $\cdot$ noch] doch Z \textbf{4} iuch] \textit{om.} G \textbf{5} ir] er Z \textbf{6} diz] daz L (M) Z  $\cdot$ ir] \textit{om.} Z \textbf{7} man] \textit{om.} L M  $\cdot$ mir] mir da Z \textbf{9} vil] \textit{om.} L M Z \textbf{10} dô] Da M Z  $\cdot$ vröute] frouwt L \textbf{11} lachendem] lachende M \textbf{13} swem] Wem L (M) \textbf{14} \textit{Versfolge 546.15-14} Z   $\cdot$ zempfâhen] Zcuhen phaen M \textbf{15} hêrre welt irs] herre welt ir L (Z) wolt irsz herre M  $\cdot$ wer] gewer M \textbf{17} ie sô hel] ê so snel I \textbf{19} ich] ich diu I \textbf{21} mich machen] michen M \textbf{23} ir sîn] irs G ir M  $\cdot$ sô] \textit{om.} L \textbf{24} antwürte] Antwuͯrtet L Antwurten Z  $\cdot$ in in] in L  $\cdot$ den kocken] chuche G die chuchel I \textbf{25} werdecheit] werdeclichen L  $\cdot$ wol] \textit{om.} L M \textbf{26} dô] Da M  $\cdot$ der künic] des konniges M  $\cdot$ Lotes] lotis G \textbf{29} iun] in ev I (L) Z uch M \newline
\end{minipage}
\hspace{0.5cm}
\begin{minipage}[t]{0.5\linewidth}
\small
\begin{center}*T
\end{center}
\begin{tabular}{rl}
 & \textbf{hete mirz eine wîle} verstoln.\\ 
 & einer mûlînen voln\\ 
 & m\textit{ö}ht ir noch \textbf{baz} gewinnen.\\ 
 & ich kan iuch anders minnen:\\ 
5 & sît \textbf{ir} iuch dunket alse wert,\\ 
 & vür \textbf{diz} ors, \textbf{des} ir gert,\\ 
 & habet iu den man, derz \textbf{gegen mir} reit.\\ 
 & ist im daz liep oder leit,\\ 
 & dâ kêre ich mich wênic an."\\ 
10 & \textbf{Des} vröute sich der schifman;\\ 
 & mit lachendem munde er sprach:\\ 
 & "sô rîche gâbe ich nie gesach,\\ 
 & \textbf{dem} si rehte wære\\ 
 & zenpfâhenne gebære.\\ 
15 & doch welt irs sîn mîn wer,\\ 
 & \textbf{gar} übergolten ist \textbf{mîn} ger.\\ 
 & sîn \textbf{tât gegen prîse} was ie sô hel,\\ 
 & vünf hundert ors starc unde snel\\ 
 & \textbf{ich ungerne} vür in næme,\\ 
20 & wandez mir niht gezæme.\\ 
 & welt ir mich machen rîche,\\ 
 & sô werbet rîterlîche.\\ 
 & muget ir \textbf{sîn} sô gewaltic sîn,\\ 
 & antwürt \textbf{mir}n in den kocken mîn,\\ 
25 & \textbf{sô} kunnet ir werdecheit wol tuon."\\ 
 & Dô sprach \textbf{des} \textbf{küneges} Lotes suon:\\ 
 & "beidiu drîn unde \textbf{dar} vür\\ 
 & \textbf{unde} innerhalben \textbf{iuwerre} tür\\ 
 & antwürt\textbf{ich in iu} gevangen."\\ 
30 & "Sô werdet ir wol enpfangen",\\ 
\end{tabular}
\scriptsize
\line(1,0){75} \newline
T U V W O Q R \newline
\line(1,0){75} \newline
\textbf{7} \textit{Initiale} O Q  \textbf{10} \textit{Majuskel} T  \textbf{26} \textit{Initiale} R   $\cdot$ \textit{Majuskel} T  \textbf{30} \textit{Majuskel} T  \newline
\line(1,0){75} \newline
\textbf{1} Eine wile mir het verstoln O \textbf{2} mûlînen] muͦlinne U (V) (W) (O) (Q) \textbf{3} möht ir] mohtir T (U) (O) (Q) Moch er R  $\cdot$ baz] e V \textbf{4} iuch] îv T \textit{om.} U  $\cdot$ anders] baz O \textbf{5} iuch] îv T \textbf{6} diz] das Q (R)  $\cdot$ des] das W (R)  $\cdot$ gert] hie gert V \textbf{7} habet] ÷abt O \textbf{9} Daz sich ich harte chleine an O \textbf{10} vröute] frevt O (R) \textbf{14} zenpfâhenne] [Zenpa*]: Zenpahe U Zenpfahens O \textbf{15} doch] Oͮch V  $\cdot$ irs] ir Q  $\cdot$ mîn] mir W  $\cdot$ wer] gewer W O Q R \textbf{17} gegen prîse] ist gein prise U [gepreise]: gen preise Q  $\cdot$ was ie sô] ie was W was so O \textbf{20} wandez] Wan daz U Wann des W Wan ez O Wann er Q  $\cdot$ mir] mit Q \textbf{21} rîche] Richen R \textbf{23} muget] Mecht Q  $\cdot$ ir sîn] ir iz U ir W (R) irs Q \textbf{24} \textit{statt 546.24:} Das was so lediglichen mein / Dannoch hewt morgen fru / Wolt ir gemaches greiffen zu Q   $\cdot$ mirn] in U V W O R  $\cdot$ den kocken] daz schiffe R \textbf{26} des küneges] des kúnig W der chvnich O  $\cdot$ Lotes] lottes W \textbf{28} unde] Mit U Vnz V (W) O (Q) (R)  $\cdot$ innerhalben] niderhalp Q Ienerthalb R  $\cdot$ iuwerre] der ewren Q úwers Husses R \textbf{29} in iu] in Q úch in R \textbf{30} wol] schon W \newline
\end{minipage}
\end{table}
\end{document}
