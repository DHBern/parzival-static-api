\documentclass[8pt,a4paper,notitlepage]{article}
\usepackage{fullpage}
\usepackage{ulem}
\usepackage{xltxtra}
\usepackage{datetime}
\renewcommand{\dateseparator}{.}
\dmyyyydate
\usepackage{fancyhdr}
\usepackage{ifthen}
\pagestyle{fancy}
\fancyhf{}
\renewcommand{\headrulewidth}{0pt}
\fancyfoot[L]{\ifthenelse{\value{page}=1}{\today, \currenttime{} Uhr}{}}
\begin{document}
\begin{table}[ht]
\begin{minipage}[t]{0.5\linewidth}
\small
\begin{center}*D
\end{center}
\begin{tabular}{rl}
\textbf{679} & \textbf{\textit{\begin{large}O\end{large}}b} von dem werden Gawan\\ 
 & \textbf{werlîche ein} tjost \textbf{dâ} \textbf{wirt} getân,\\ 
 & sô\textbf{ne} \textbf{gevorht} ich sîner êre\\ 
 & an \textbf{prîse} nie sô sêre.\\ 
5 & ich solt ouch sandern angest hân.\\ 
 & daz wil ich ûz den sorgen lân:\\ 
 & \textbf{der} was in strîte eines mannes her.\\ 
 & ûz \textbf{heidenschaft} verre über mer\\ 
 & \textbf{was} brâht diu \textbf{zimierde} sîn.\\ 
10 & noch rœter denn \textbf{ein} rubîn\\ 
 & was sîn kursît unt \textbf{sîn} orses kleit.\\ 
 & der helt nâch \textbf{âventiuren} reit;\\ 
 & sîn schilt was gar durchstochen.\\ 
 & er het ouch gebrochen\\ 
15 & von dem boume, des Gramoflanz\\ 
 & huote, einen \textbf{sô} \textbf{liehten} kranz,\\ 
 & daz Gawan daz rîs erkande.\\ 
 & dô vorht er die schande,\\ 
 & ob sîn der künec dâ het erbiten.\\ 
20 & wære der durch strît gein im geriten,\\ 
 & sô müese \textbf{ouch} \textbf{strîten dâ} \textbf{geschehen},\\ 
 & \textbf{und} solt ez nimmer vrouwe \textbf{ersehen}.\\ 
 & Von Munsalvæsche wâren sie,\\ 
 & beidiu ors, diu alsus hie\\ 
25 & \textbf{liezen nâher} strîchen\\ 
 & ûf den poinder \textbf{hurteclîchen};\\ 
 & mit sporn si wurden des ermant.\\ 
 & \textbf{al} grüene klê, niht stoubic sant\\ 
 & stuont touwec, dâ diu tjost \textbf{geschach}.\\ 
30 & mich müet ir bêder ungemach.\\ 
\end{tabular}
\scriptsize
\line(1,0){75} \newline
D Fr10 \newline
\line(1,0){75} \newline
\textbf{1} \textit{Großinitiale} D  \textbf{23} \textit{Majuskel} D  \newline
\line(1,0){75} \newline
\textbf{1} Ob] BB D \textbf{7} in] im Fr10 \textbf{10} rubîn] Rvbbin D Ru::: Fr10 \textbf{11} unt sîn] vnd des Fr10 \textbf{12} âventiuren] Auentuyer Fr10 \textbf{15} Gramoflanz] Gramo f::: Fr10 \textbf{16} huote] Ouf dem huͦt Fr10  $\cdot$ sô] \textit{om.} Fr10 \textbf{21} müese] muͦs Fr10 \textbf{22} vnd soltenz ouch nimmer vro::: Fr10 \textbf{23} Munsalvæsche] Mvnsalvæsce D montsalvalt Fr10 \textbf{24} beidiu] Paiden Fr10 \newline
\end{minipage}
\hspace{0.5cm}
\begin{minipage}[t]{0.5\linewidth}
\small
\begin{center}*m
\end{center}
\begin{tabular}{rl}
 & \textbf{ob} von dem werden Gawan\\ 
 & \textbf{werlîch ein} juste \textbf{wart} getân,\\ 
 & sô \textbf{gevorht} ich sîner êre\\ 
 & an \textbf{strîte} nie sô sêre.\\ 
5 & ich solt ouch des andern angest hân.\\ 
 & daz wil ich ûz den sorgen lân:\\ 
 & \textbf{der} was \textit{i}n strîte \textit{eines} m\textit{a}nnes her.\\ 
 & û\textit{z} \textbf{heidenscher} verre über mer\\ 
 & \textbf{was} brâht diu \textbf{zimierde} sîn.\\ 
10 & noch rœter danne rubîn\\ 
 & was sîn kursît und \textbf{sînes} rosses kleit.\\ 
 & der helt nâch \textbf{âventiure} reit;\\ 
 & sîn schilt was gar durchstochen.\\ 
 & er hete ouch gebrochen\\ 
15 & von dem boum, des Gramolanz\\ 
 & \dag hete\dag , einen \textbf{werden} kranz,\\ 
 & daz Gawan daz rîs erkande.\\ 
 & dô vorht er die schande,\\ 
 & ob sîn der künic d\textit{â} het erbiten.\\ 
20 & wær der durch strît gegen im geriten,\\ 
 & sô müeste \textbf{ouch} \textbf{strîten} \textbf{beschehen},\\ 
 & \textbf{und} solte e\textit{z n}immer vrowe \textbf{sehen}.\\ 
 & von Muntsalvasche wâren sie,\\ 
 & beidiu ros, diu alsus hie\\ 
25 & \textbf{liezen nâhe\textit{r}} \textit{s}trîchen\\ 
 & ûf den ponder \textbf{herteclîchen};\\ 
 & mit sporn si wurden des ermant.\\ 
 & \textbf{al} g\textit{rüe}ner klê, niht stoubic sant\\ 
 & stuont touwic, d\textit{â} diu just \textbf{beschach}.\\ 
30 & mich müejet ir beider ungemach.\\ 
\end{tabular}
\scriptsize
\line(1,0){75} \newline
m n o \newline
\line(1,0){75} \newline
\newline
\line(1,0){75} \newline
\textbf{2} wart] do wart n \textbf{5} ouch] \textit{om.} n  $\cdot$ angest] anst m \textbf{6} daz] Des n \textbf{7} in] ein m  $\cdot$ eines mannes] so mẏnnes m \textbf{8} ûz] Vff m  $\cdot$ heidenscher] heidenschafft n o \textbf{9} brâht] brachte n \textbf{10} rubîn] ein robin n eyn rúbin o \textbf{11} kleit] ::: o \textbf{12} âventiure] afentare o \textbf{15} boum] boum [*]: so n  $\cdot$ Gramolanz] gramolantz m n gramolancz o \textbf{16} werden] also liechten n o \textbf{19} dâ] do m n o \textbf{20} der] >der< o \textbf{21} sô] Do n  $\cdot$ beschehen] do beschehen n o \textbf{22} ez nimmer] es es nẏmmer m  $\cdot$ sehen] gesehen n geschehen o \textbf{23} Muntsalvasche] muntsaluasce m o montsaluasce n \textbf{25} liezen] Liesse n  $\cdot$ nâher strîchen] naher stritten vnd strichen m noher [stritten]: strichen o \textbf{28} al grüener] Al gemeiner m Alle gruͯner n  $\cdot$ sant] fant n \textbf{29} dâ] do m n o  $\cdot$ beschach] geschach n o \newline
\end{minipage}
\end{table}
\newpage
\begin{table}[ht]
\begin{minipage}[t]{0.5\linewidth}
\small
\begin{center}*G
\end{center}
\begin{tabular}{rl}
 & \textbf{\begin{large}O\end{large}be} von dem werden Gawan\\ 
 & \textbf{werlîchiu} tjost \textbf{dâ} \textbf{wart} getân,\\ 
 & sô\textbf{ne} \textbf{gevorhte} ich sîner êre\\ 
 & an \textbf{strîte} nie sô sêre.\\ 
5 & ich solde ouch des andern angest hân.\\ 
 & daz wil ich ûz den sorgen lân:\\ 
 & \textbf{er} was in strîte \textit{e}in\textit{es} mannes her.\\ 
 & ûz \textbf{heidenschefte} verre über mer\\ 
 & \textbf{wart} brâht diu \textbf{zimier} sîn.\\ 
10 & noch rœter danne \textbf{ein} rubîn\\ 
 & was sîn kursît unde \textbf{sînes} orses kleit.\\ 
 & der helt nâch \textbf{âventiuren} reit;\\ 
 & sîn schilt was gar durchstochen.\\ 
 & er het ouch gebrochen\\ 
15 & von dem boume, des Gramoflanz\\ 
 & huote, einen \textbf{als} \textbf{liehten} kranz,\\ 
 & daz Gawan daz rîs erkande.\\ 
 & dô vorht er die schande,\\ 
 & op sîn der künic dâ hiet erbiten.\\ 
20 & wære der durch strît gein im geriten,\\ 
 & sô m\textit{üe}se \textbf{strît dâ} \textbf{geschehen},\\ 
 & solde ez nimmer vrouwe \textbf{ersehen}.\\ 
 & von Muntsalvatsche  wâren sie,\\ 
 & beidiu ors, diu alsô hie\\ 
25 & \textbf{nâher liezen} strîchen\\ 
 & ûf den poynder \textbf{hurticlîchen};\\ 
 & mit sporen si wurden des ermant.\\ 
 & grüene klê, niht stoubec sant\\ 
 & stuont touwec, dâ diu tjost \textbf{geschach}.\\ 
30 & mich müet ir bêder ungemach.\\ 
\end{tabular}
\scriptsize
\line(1,0){75} \newline
G I L M Z Fr18 Fr22 Fr24 Fr52 Fr61 \newline
\line(1,0){75} \newline
\textbf{1} \textit{Überschrift:} Hie kvmt ditz buch wider an parcifaln vnd ist er vnd her gawan an ein ander komen wie sie sich nu scheiden daz lese man fvrbaz Z   $\cdot$ \textit{Großinitiale} Z  · Initiale G L Fr18 Fr22 Fr24  \textbf{11} \textit{Initiale} I  \textbf{23} \textit{Initiale} I  \newline
\line(1,0){75} \newline
\textbf{2} da werlichev tioste wart getan I  $\cdot$ werlîchiu] wertliche M  $\cdot$ tjost] ein tiost Z  $\cdot$ dâ] \textit{om.} M Z do Fr61  $\cdot$ wart] wirt L M Z Fr18 Fr22 \textbf{3} sône] So M Z Fr18 Fr22 Fr24  $\cdot$ gevorhte] vorcht L (M) (Z) (Fr18) (Fr22) (Fr24) (Fr61)  $\cdot$ ich] \textit{om.} Fr61 \textbf{4} an] doch an I \textbf{5} ouch] \textit{om.} Fr61 \textbf{6} den] der Fr22 \textbf{7} in] ie Fr61  $\cdot$ eines mannes] in mannes G ein rehtez Z \textbf{9} wart] Was Z  $\cdot$ zimier] gezcimire M \textbf{10} rubîn] rv̂bin G Rvbŷn Fr24 \textbf{11} sîn] \textit{om.} L  $\cdot$ sînes] des L M Z Fr18 Fr24  $\cdot$ orses] [ôrs*]: ôrs G \textbf{12} nach auenture er reit I  $\cdot$ âventiuren] aventuͯre L (M) (Z) (Fr18) (Fr24) \textbf{14} gebrochen] da gebrochen I zcu brochin M gebrozen Fr18 \textbf{15} des] \textit{om.} I  $\cdot$ Gramoflanz] Gramorflanz M Gramoflantz Fr18 [Gramot]: Gramoflanz Fr24 \textbf{16} huote] hiut I Hete L  $\cdot$ einen als] einen so L (Fr18) also eynen M  $\cdot$ liehten] lichten L (M) Fr52  $\cdot$ kranz] [kranch]: krancz G \textbf{17} daz Gawan] Gawan I  $\cdot$ daz rîs] diz riz Fr52 \textbf{18} dô] Da M Z \textbf{19} dâ] \textit{om.} Fr24 do Fr52 \textbf{20} wære der] wer er I vnd were Fr52  $\cdot$ durch] gein Fr24  $\cdot$ gein im] da Gein I an in Fr24 \textbf{21} müese] moͮse G (I) (M) (Fr24) (Fr52) muͤst ouch Z  $\cdot$ strît] striten L Z Fr18 Fr24 [stirten]: striten M \textbf{22} solde ez] sol ditz ioch I  $\cdot$ ersehen] gesehen I L (M) (Fr18) Fr24 sehen Fr52 \textbf{23} Muntsalvatsche] mvntsalfasche G muntshaluasche I montsalvatsch Z Mvnsaluatsche Fr24 mvntsalvasce Fr52 \textbf{27} des] \textit{om.} Fr52 \textbf{28} grüene] niwer I Als grvne Z \textbf{29} touwec dâ] der Fr52 \textbf{30} Mich] Mit Z  $\cdot$ müet] mvte Z (Fr24) \newline
\end{minipage}
\hspace{0.5cm}
\begin{minipage}[t]{0.5\linewidth}
\small
\begin{center}*T
\end{center}
\begin{tabular}{rl}
 & \textbf{\begin{large}A\end{large}ber} von dem werden Gawan\\ 
 & \textbf{werlîch} jost \textbf{d\textit{â}} \textbf{wart} getân;\\ 
 & sô \textbf{vorht} ich sîner êre\\ 
 & an \textbf{strîte} nie sô sêre.\\ 
5 & ich solte ouch des andern angest hân.\\ 
 & daz wil ich ûz den sorgen lân:\\ 
 & \textbf{er} was in strîte eines mannes her.\\ 
 & ûz \textbf{heidenschaft} verre über mer\\ 
 & \textbf{wart} brâht diu \textbf{gezierde} sîn.\\ 
10 & noch rœter dan \textbf{ein} rubîn\\ 
 & was sîn kursît und \textbf{des} orses kleit.\\ 
 & der helt nâch \textbf{âventiure} reit;\\ 
 & sîn schilt was gar durchstochen.\\ 
 & er hete ouch gebrochen\\ 
15 & von dem boume, des Gramoflanz\\ 
 & huote, einen \textbf{alsô} \textbf{liehten} kranz,\\ 
 & daz Gawan daz rîs erkande.\\ 
 & dô vorht er die schande,\\ 
 & ob sîn der künec d\textit{â} hete erbiten.\\ 
20 & wære der durch strît gein im geriten,\\ 
 & sô müese \textbf{strîten d\textit{â}} \textbf{geschehen},\\ 
 & soltez niemer vrouwe \textbf{gesehen}.\\ 
 & von Munsalvasche wâren sie,\\ 
 & beidiu ors, diu alsô hie\\ 
25 & \textbf{nâher liezen} strîchen\\ 
 & ûf den poynder \textbf{herteclîchen};\\ 
 & mit sporn si wurden des ermant.\\ 
 & grüene klê, niht stoubic sant\\ 
 & stuont touwic, d\textit{â} diu jost \textbf{geschach}.\\ 
30 & mich müet ir beider ungemach.\\ 
\end{tabular}
\scriptsize
\line(1,0){75} \newline
U V W Q R \newline
\line(1,0){75} \newline
\textbf{1} \textit{Großinitiale} U   $\cdot$ \textit{Initiale} V W Q   $\cdot$ \textit{Capitulumzeichen} R  \newline
\line(1,0){75} \newline
\textbf{1} Aber] [Ober]: Obe V Ob Q (R)  $\cdot$ werden] \textit{om.} V \textbf{2} Werliche [*]: ein ivst do wurt getan V  $\cdot$ dâ] do U \textit{om.} W daz R  $\cdot$ wart] wirt Q \textbf{3} vorht] gevorhte V (W) (Q)  $\cdot$ ich] [*]: ich V \textbf{5} angest] angt R \textbf{7} eines] ein W R \textbf{9} gezierde] zimierde V (W) (Q) R \textbf{10} rubîn] ruͦbin U rubein W Q \textbf{13} durchstochen] stochen R \textbf{15} Gramoflanz] gramaflanz V gramoflantz W gramoflansz Q Gramoflancz R \textbf{16} huote] [*]: Hvͦte V Heúte W (Q) (R)  $\cdot$ liehten] lichten Q \textbf{17} Gawan] Gawin R  $\cdot$ daz] dis W  $\cdot$ rîs] reich Q \textbf{18} die] der V \textbf{19} dâ] do U V Q \textit{om.} W \textbf{21} müese] muͦß W  $\cdot$ strîten] streite Q (R)  $\cdot$ dâ] do U V W Q \textbf{22} gesehen] ersehen W (Q) (R) \textbf{23} Munsalvasche] Muͦntsalvatsche U mvntsalvasche V montsaluatschs W muntsaluasche Q Munsaluashe R \textbf{24} alsô] [al*]: alsvs V \textbf{26} den] dem R  $\cdot$ poynder] poyder V  $\cdot$ herteclîchen] hv́rteclichen V (W) (Q) \textbf{27} si wurden] wurden sy R \textbf{28} niht] noch Q \textbf{29} touwic] towe R  $\cdot$ dâ] do U V W Q  $\cdot$ diu] \textit{om.} W \newline
\end{minipage}
\end{table}
\end{document}
