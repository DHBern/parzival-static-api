\documentclass[8pt,a4paper,notitlepage]{article}
\usepackage{fullpage}
\usepackage{ulem}
\usepackage{xltxtra}
\usepackage{datetime}
\renewcommand{\dateseparator}{.}
\dmyyyydate
\usepackage{fancyhdr}
\usepackage{ifthen}
\pagestyle{fancy}
\fancyhf{}
\renewcommand{\headrulewidth}{0pt}
\fancyfoot[L]{\ifthenelse{\value{page}=1}{\today, \currenttime{} Uhr}{}}
\begin{document}
\begin{table}[ht]
\begin{minipage}[t]{0.5\linewidth}
\small
\begin{center}*D
\end{center}
\begin{tabular}{rl}
\textbf{94} & \textbf{als} der benediz wart getân,\\ 
 & \textbf{dô kom vrou Herzeloyde} sân.\\ 
 & an Gahmuretes lîp si sprach.\\ 
 & \textbf{si} gerte, als ir diu volge jach.\\ 
5 & dô sprach er: "vrouwe, ich hân ein wîp.\\ 
 & diu \textbf{ist mir} lieber danne der lîp.\\ 
 & ob ich der âne wære,\\ 
 & dennoch \textbf{wess} ich ein mære,\\ 
 & dâ mit ich iu enbræste gar,\\ 
10 & næme iemen mînes rehtes war."\\ 
 & \textbf{\begin{large}S\end{large}i sprach}: "ir sult die mœrinne\\ 
 & lân durch mîne minne.\\ 
 & des toufes segen hât \textbf{bezzer} kraft.\\ 
 & nû ânet iuch der heidenschaft\\ 
15 & unt minnet mich nâch \textbf{unserer} ê,\\ 
 & \textbf{wan} mir ist nâch \textbf{iuwerer minne} wê.\\ 
 & oder sol mir gein iu schade sîn\\ 
 & der Franzoyser künegîn,\\ 
 & \textbf{der} boten sprâchen süeziu wort?\\ 
20 & \textbf{si} spilten ir mære unz \textbf{an den} ort."\\ 
 & "\textbf{jâ}, diu ist mîn \textbf{wâriu} vrouwe.\\ 
 & ich brâhte in Anschouwe\\ 
 & ir rât \textbf{und} mîner zühte site.\\ 
 & mir wont noch hiute ir helfe mite\\ 
25 & dâ von, daz mich mîn vrouwe zôch,\\ 
 & die wîbes missewende \textbf{ie} vlôch.\\ 
 & wir wâren \textbf{kinder beidiu} dô\\ 
 & unt doch ze sehen ein ander vrô.\\ 
 & Diu küneginne Ampflise\\ 
30 & \textbf{wont} \textbf{an wîplîchem} prîse.\\ 
\end{tabular}
\scriptsize
\line(1,0){75} \newline
D \newline
\line(1,0){75} \newline
\textbf{11} \textit{Initiale} D  \textbf{29} \textit{Majuskel} D  \newline
\line(1,0){75} \newline
\textbf{3} Gahmuretes] Gahmvretes D \textbf{22} Anschouwe] Anscoͮwe D \textbf{29} Ampflise] Amphîse D \newline
\end{minipage}
\hspace{0.5cm}
\begin{minipage}[t]{0.5\linewidth}
\small
\begin{center}*m
\end{center}
\begin{tabular}{rl}
 & \textbf{und} \textbf{als} der benediz \dag daz\dag  wart getân,\\ 
 & \textbf{dô kam vrouwe Herczeloid\textit{e}} sân.\\ 
 & an Gahmuretes lîp si sprach.\\ 
 & \textbf{si} gerte, als ir diu volge jach.\\ 
5 & dô sprach er: "vrowe, ich hân ein wîp.\\ 
 & diu \textbf{ist mir} lieber danne der lîp.\\ 
 & ob ich der âne wære,\\ 
 & dannoc\textit{h} \textbf{wus\textit{t}e} ich ein mære,\\ 
 & dâ mite ich iu enbræste gar,\\ 
10 & næme ieman mînes rehtes war."\\ 
 & \textbf{\begin{large}S\end{large}i sprach}: "ir sullet die mœrinne\\ 
 & lân durch mîne minne.\\ 
 & des toufes segen hât \textbf{bezzer} kraft.\\ 
 & nû ânet iuch der heidenschaft\\ 
15 & und minnet mich nâch \textbf{mîner} ê,\\ 
 & \textbf{wanne} mir ist nâch \textbf{i\textit{uw}er minne} wê.\\ 
 & oder sol mir gegen iu schade sîn\\ 
 & der Franzoiser künigîn,\\ 
 & \textbf{der} boten sprâchen süeziu wort?\\ 
20 & \textbf{si} spilten ir mære unz \textbf{an den} \textit{or}t."\\ 
 & "\textbf{jâ}, diu ist mîniu \textbf{wâre} vrouwe.\\ 
 & ich brâhte in Anschouwe\\ 
 & ir rât \textbf{und} mîner zühte site.\\ 
 & mir wonet noch hiute ir helfe mite\\ 
25 & dâ von, daz mich mîn vrouwe zôch,\\ 
 & die wîbes missewende \textbf{ie} vlôch.\\ 
 & wir wâren \textbf{beidiu kinder} dô\\ 
 & und \textit{d}och ze \textit{s}ehene ein ander vrô.\\ 
 & diu künigîn Ampflise\\ 
30 & \textbf{wont} \textbf{an wîplîchem} prîse.\\ 
\end{tabular}
\scriptsize
\line(1,0){75} \newline
m n o \newline
\line(1,0){75} \newline
\textbf{11} \textit{Initiale} m   $\cdot$ \textit{Capitulumzeichen} n  \newline
\line(1,0){75} \newline
\textbf{1} der] er n o \textbf{2} Herczeloide] herczeloiden m hertzeloide n herczeleides o \textbf{3} Gahmuretes] gahmurettes m gamiretes n gamúret o \textbf{5} sprach er] sprach er sprach o \textbf{7} der] dar n \textbf{8} dannoch wuste] Dannach were wusche m  $\cdot$ ein] eine o \textbf{9} enbræste] empriste n (o) \textbf{10} rehtes] rechte n rehten o \textbf{15} minnet] nẏmet o  $\cdot$ mîner] vnser n o \textbf{16} iuwer] irer m \textbf{18} Franzoiser] frantzoser m n franczoser o \textbf{20} an] in n o  $\cdot$ ort] rot m rat n o \textbf{22} Anschouwe] an schouͯwe m aneschouwe n an schowe o \textbf{27} beidiu] beider o \textbf{28} doch ze sehene] zoh zehene m \textbf{29} künigîn] konige o  $\cdot$ Ampflise] ampfolise n anpfolise o \textbf{30} wîplîchem] wipliche o \newline
\end{minipage}
\end{table}
\newpage
\begin{table}[ht]
\begin{minipage}[t]{0.5\linewidth}
\small
\begin{center}*G
\end{center}
\begin{tabular}{rl}
 & \textbf{dô} der benediz wart getân,\\ 
 & \textbf{vrô Herzeloide kom dâ} sân.\\ 
 & an Gahmuretes lîp si sprach\\ 
 & \textbf{unde} gerte, als ir diu volge jach.\\ 
5 & dô sprach er: "vrouwe, \textit{ich} hân ein wîp,\\ 
 & diu \textbf{mir ist} lieber dane der lîp.\\ 
 & obe ich der âne wære,\\ 
 & danoch \textbf{wesse} ich ein mære,\\ 
 & dâ mit ich iu enbræste gar,\\ 
10 & næme iemen mînes rehtes war."\\ 
 & "ir sult die mœrinne\\ 
 & lân durch mîne minne.\\ 
 & des toufes segen hât \textbf{grœzer} kraft.\\ 
 & \textit{nû} ânet iuch der heidenschaft\\ 
15 & \textit{unde} minnet mich nâch \textbf{unser} ê.\\ 
 & mir ist nâch \textbf{iwere\textit{r} minn\textit{e}} wê.\\ 
 & oder sol mir gein iu schade sîn\\ 
 & der Franzoiser künigîn?\\ 
 & \textbf{der} boten sprâchen süeziu wort\\ 
20 & \textbf{unde} sp\textit{i}lden ir mære unze \textbf{an den} ort."\\ 
 & \textbf{er sprach}: "diu ist mîn vrouwe.\\ 
 & ich brâht in Antschouwe\\ 
 & ir rât \textbf{an} mîner zühte site.\\ 
 & mir wonet noch hiute ir helfe mite\\ 
25 & dâ von, daz mich mîn vrouwe zôch,\\ 
 & die wîbes missewende vlôch.\\ 
 & wir wâren \textbf{kinder beidiu} dô\\ 
 & unt doch ze sehene ein ander vrô.\\ 
 & diu künigîn Anphlise\\ 
30 & \textbf{wonet} \textbf{in wîbes} prîse.\\ 
\end{tabular}
\scriptsize
\line(1,0){75} \newline
G I O L M Q R Z Fr21 Fr56 \newline
\line(1,0){75} \newline
\textbf{1} \textit{Initiale} I O  \textbf{5} \textit{Capitulumzeichen} L  \textbf{11} \textit{Initiale} L Q R Z  \textbf{19} \textit{Initiale} I  \textbf{29} \textit{Initiale} M  \newline
\line(1,0){75} \newline
\textbf{1} dô] ÷o I O Da Z  $\cdot$ benediz] [benz]: benditz G benediz do I bundicite M segen R benediene Fr21  $\cdot$ getân] geben R \textbf{2} vrô] \textit{om.} R  $\cdot$ Herzeloide] herzenlaude I herzelavde O Hertzelauͯde L herczeleide M herzeloúde Q Herczylaude R herzelovde Z herzelevde Fr21  $\cdot$ dâ] do I L Fr21 dv O so Q  $\cdot$ sân] gegen R \textbf{3} Gahmuretes] gahmurets G Gamvretes O (Q) (Z) Gahmuͯretes L gamuretis M Gahmoretes Fr21 \textbf{4} gerte] gert I O (M) Q Fr21 \textbf{5} dô] Da M Z  $\cdot$ ich] \textit{om.} G \textbf{6} mir ist] ist mir O L M Q R Z Fr21  $\cdot$ der] min I \textbf{7} der] dar R \textbf{8} wesse] wis I \textbf{9} gar] wol gar O \textbf{10} iemen mînes] min yemen R \textbf{11} mœrinne] moͯrin lan R \textbf{12} Durch mine minne san R \textbf{13} des toufes] Der tavffe O (M)  $\cdot$ grœzer] groze I bezer O (L) (M) (Q) (R) (Z) \textbf{14} nû] \textit{om.} G  $\cdot$ ânet] enczichent R \textbf{15} unde] \textit{om.} G  $\cdot$ minnet] meynet M  $\cdot$ mich nâch] mich [a*a]: nach R nach Z \textbf{16} mir] Wan mir O L (M) (Q) R Z (Fr56)  $\cdot$ iwerer minne] iweren minnen G iwer mineie Fr56 \textbf{17} sol] solde M  $\cdot$ schade] schaden Q  $\cdot$ sîn] sẏ R schin Z \textbf{18} Franzoiser] fronzloiser I franzeiser O franzoyser L franzioser M franczoiser Q franczoyser R frantzoiser Z :::zoys er Fr56 \textbf{19} der] Jr Z  $\cdot$ süeziu] sie zuͯ L suͯsze R \textbf{20} spilden] spielden G spaldin M  $\cdot$ unze] biz I  $\cdot$ an den] in den O (M) Q Z in daz L \textbf{21} er sprach] Seht O L (M) Ja Q R Z  $\cdot$ vrouwe] wariv frowe O (L) (M) (Q) (R) (Z) Fr56 \textbf{22} Antschouwe] anschoͮwe G antschoͮe I anschawe O Anschowe L (M) (Q) (R) (Fr56) antschowe Z \textbf{23} ir] er I  $\cdot$ an] vnde O (L) (M) (Q) (R) (Z) Fr56  $\cdot$ mîner] ir Q R mine Fr56 \textbf{26} \textit{Vers 94.26 fehlt} R   $\cdot$ wîbes missewende] [wibessewende]: wibemissewende I  $\cdot$ vlôch] ẏe vloch L \textbf{27} kinder beidiu] beidiv chinder O (Q)  $\cdot$ dô] da M R \textbf{28} ein ander] an ander I ein andren R  $\cdot$ vrô] do froͤ I \textbf{29} Anphlise] anflise G anphise I amphilise O Amfulise L anfilise M anfliesz Q Amflyze R amflise Z Fr56 \textbf{30} in wîbes] an wiplichem O (L) (M) (Q) Z an weltlichem R an wiplicher Fr56 \newline
\end{minipage}
\hspace{0.5cm}
\begin{minipage}[t]{0.5\linewidth}
\small
\begin{center}*T (U)
\end{center}
\begin{tabular}{rl}
 & \textbf{dô} der benediz wart getân,\\ 
 & \textbf{vrô Herzeloyde kam dô} sân.\\ 
 & an Gahmuretes lîp si sprach\\ 
 & \textbf{und} gerte\textbf{s}, als ir diu volge jach.\\ 
5 & dô sprach er: "vrouwe, ich hân ein wîp.\\ 
 & diu \textbf{ist mir} lieber dan der lîp.\\ 
 & ob ich d\textit{e}r âne wære,\\ 
 & dannoch \textbf{weiz} ich eine mære,\\ 
 & dâ mit ich iu enbræst\textit{e} gar,\\ 
10 & næme ieman mînes rehtes war."\\ 
 & "ir solt die mœrinne\\ 
 & lân durch mîne minne.\\ 
 & des \textit{toufes} s\textit{e}gen hât \textbf{bezzer} kraft.\\ 
 & nû ânet iuch der heidenschaft\\ 
15 & und minnet mich nâch \textbf{unserre} ê,\\ 
 & \textbf{wan} mir ist nâch \textbf{iu} wê.\\ 
 & oder sol mir gein iu schade sîn\\ 
 & der Franzoyser künegîn?\\ 
 & \textbf{die} boten sprâchen süeziu wort\\ 
20 & \textbf{und} spilten ir mære unz \textbf{in daz} ort."\\ 
 & "\textbf{seht}, di\textit{u}st mîn \textbf{wâriu} vrouw\textit{e}.\\ 
 & ich brâhte in Anschouwe\\ 
 & ir rât \textbf{und} mîner zühte site.\\ 
 & mir wont noch hiute ir helfe mite\\ 
25 & dâ von, \textit{daz} mich mîn vrouwe zôch,\\ 
 & die wîbes missewende \textbf{ie} vlôch.\\ 
 & wir wâren \textbf{beidiu kinder} dô\\ 
 & und doch zuo sehene ein ander vrô.\\ 
 & diu künegîn Anflise\\ 
30 & \textbf{gewonte} \textbf{an wîplîchem} prîse.\\ 
\end{tabular}
\scriptsize
\line(1,0){75} \newline
U V W T \newline
\line(1,0){75} \newline
\textbf{1} \textit{Majuskel} T  \textbf{5} \textit{Initiale} T  \textbf{11} \textit{Initiale} W   $\cdot$ \textit{Majuskel} T  \textbf{21} \textit{Majuskel} T  \textbf{22} \textit{Majuskel} T  \newline
\line(1,0){75} \newline
\textbf{2} Herzeloyde] herzeleide U hertzelaude V hertzeloyd W herzelôyde T  $\cdot$ dô sân] gegan W \textbf{3} Gahmuretes] Gahmuͦretes U Gamuretes V gamurettes W \textbf{4} gertes] gerte T  $\cdot$ ir] irs V \textit{om.} W \textbf{6} der] min T \textbf{7} der] dar U \textbf{8} weiz] wust V (W) (T)  $\cdot$ eine] ein W T \textbf{9} mit] von W  $\cdot$ enbræste] en bresten U \textbf{10} rehtes] rechten W \textbf{11} die] div T \textbf{13} toufes segen] sagen U \textbf{14} nû ânet iuch] [n*]: nv entenent v́ch V wan tvͦnt îv abe T \textbf{15} unserre] meiner W \textbf{16} wan] \textit{om.} T  $\cdot$ iu] uwerre minne V (W) (T) \textbf{17} gein iu schade] scade scade T \textbf{18} Franzoyser] Franzoẏser V frantzoser W \textbf{19} die] Der V W (T)  $\cdot$ sprâchen] \textit{om.} W \textbf{20} spilten] pflihten T  $\cdot$ ir mære] eúch W  $\cdot$ unz in daz] bit in daz U vntz an das W vnz in den T \textbf{21} diust] dinst U  $\cdot$ wâriu vrouwe] wariu vreuͦwen U warhe T \textbf{22} Anschouwe] Anschowe U V antschauwe W Anscôuwe T \textbf{23} und mîner] vnde [*]: ir V mit meiner W \textbf{25} daz] \textit{om.} U \textbf{26} die wîbes] die V do ich wibes T  $\cdot$ ie] \textit{om.} W T \textbf{28} zuo sehene] \textit{om.} W  $\cdot$ vrô] harte fro W \textbf{29} diu] Zuͦ sehene die W  $\cdot$ Anflise] Anflize U anfolyse W \textbf{30} gewonte] [*]: wonet V Gekroͤnet W wont T \newline
\end{minipage}
\end{table}
\end{document}
