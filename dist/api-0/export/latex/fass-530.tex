\documentclass[8pt,a4paper,notitlepage]{article}
\usepackage{fullpage}
\usepackage{ulem}
\usepackage{xltxtra}
\usepackage{datetime}
\renewcommand{\dateseparator}{.}
\dmyyyydate
\usepackage{fancyhdr}
\usepackage{ifthen}
\pagestyle{fancy}
\fancyhf{}
\renewcommand{\headrulewidth}{0pt}
\fancyfoot[L]{\ifthenelse{\value{page}=1}{\today, \currenttime{} Uhr}{}}
\begin{document}
\begin{table}[ht]
\begin{minipage}[t]{0.5\linewidth}
\small
\begin{center}*D
\end{center}
\begin{tabular}{rl}
\textbf{530} & vür sîn ors ze behalten.\\ 
 & des geltes muoser walten.\\ 
 & \begin{large}S\end{large}i sprach \textbf{hin} zim, ich wæne durch haz:\\ 
 & "saget an, welt ir \textbf{iht} vürbaz?"\\ 
5 & \textbf{dô sprach mîn} hêr Gawan:\\ 
 & "mîn vart von hinnen wirt getân\\ 
 & al nâch iwerm râte."\\ 
 & Si sprach: "der kumt \textbf{iu} spâte."\\ 
 & "nû diene ich iu \textbf{doch} drumbe."\\ 
10 & "des dunket ir mich der tumbe.\\ 
 & welt ir daz niht \textbf{vermîden},\\ 
 & sô müezet ir \textbf{von den blîden}\\ 
 & kêren gein der riwe.\\ 
 & iwer kumber wirt al niwe."\\ 
15 & Dô sprach der minnen gernde:\\ 
 & "ich bin iuch dienstes wernde,\\ 
 & ich enpfâhe\textbf{s} vreude oder nôt,\\ 
 & sît iwer minne mir gebôt,\\ 
 & daz ich \textbf{muoz} ziwerm \textbf{gebote} stên,\\ 
20 & ich mege rîten oder gên."\\ 
 & Al stênde bî der vrouwen\\ 
 & daz marc begunder schouwen.\\ 
 & daz was ze dræter tjoste\\ 
 & ein harte \textbf{krankiu} koste,\\ 
25 & \textbf{diu} stîcleder von baste.\\ 
 & dem edelem, werdem gaste\\ 
 & was etswenne gesatelt baz.\\ 
 & ûf sitzen \textbf{meit} er umbe daz:\\ 
 & er vorhte, daz er zertræte\\ 
30 & des satels \textbf{gewæte}.\\ 
\end{tabular}
\scriptsize
\line(1,0){75} \newline
D Fr7 Fr11 Fr31 \newline
\line(1,0){75} \newline
\textbf{3} \textit{Initiale} D Fr7 Fr11  \textbf{8} \textit{Majuskel} D  \textbf{15} \textit{Majuskel} D  \textbf{21} \textit{Majuskel} D  \newline
\line(1,0){75} \newline
\textbf{2} muoser] muͤst er Fr7 muͯst e: Fr11 \textbf{3} hin] \textit{om.} Fr11 \textbf{5} mîn] \textit{om.} Fr11  $\cdot$ Gawan] Gaw::: Fr11 \textbf{6} von hinnen wirt] wirt von hinnen Fr7 wirt v::: Fr11 \textbf{7} al] Als Fr11  $\cdot$ iwerm] mit iwerm Fr7 \textbf{8} spâte] ze spate Fr7 \textbf{15} minnen] minne Fr11 \textbf{16} dienstes] diens D diͯnst Fr11 \textbf{17} ich enpfâhes] :::ich sein Fr11 \textbf{18} mir] dar Fr11 \textbf{19} daz ich muoz] :::st Fr11 \textbf{26} edelem werdem] edelen werdē Fr7 ::: werden Fr11 ede::: werden Fr31 \textbf{30} gewæte] geræte Fr31 \newline
\end{minipage}
\hspace{0.5cm}
\begin{minipage}[t]{0.5\linewidth}
\small
\begin{center}*m
\end{center}
\begin{tabular}{rl}
 & vür sîn ros zuo behalten.\\ 
 & des gelte\textit{s} muos er walten.\\ 
 & si sprach zuo ime, ich wæne durch haz:\\ 
 & "sagt an, welt ir \textbf{iht} vürbaz?"\\ 
5 & \textbf{dô sprach mîn} hêr Gawan:\\ 
 & "mîn vart vo\textit{n} \textit{h}innen wirt getân\\ 
 & al nâch iuwerm râte."\\ 
 & si sprach: "d\textit{e}r kumt sp\textit{â}te."\\ 
 & "nû diene ich iu dar umb."\\ 
10 & "de\textit{s} dunket ir mich der tumb.\\ 
 & welt ir daz niht \textbf{vermîden},\\ 
 & sô müezet ir \textbf{von den blî\textit{d}en}\\ 
 & kêren gegen der riuwe.\\ 
 & iuwer kumber wirt al niuwe."\\ 
15 & \begin{large}D\end{large}ô sprach der minne\textit{n} gernde:\\ 
 & "ich bin iuch dienstes wernde,\\ 
 & ich enpfâhe \textbf{dô sîn} vröude oder nôt,\\ 
 & sît iuwer minne mir gebôt,\\ 
 & daz ich \textbf{muoz} zuo iuwerm \textbf{bote} stên,\\ 
20 & ich müge rîten oder gên."\\ 
 & alstênde bî der vrouwen\\ 
 & daz marc begunde er schouwen.\\ 
 & daz was zuo dræter juste\\ 
 & ein harte \textbf{kranker} kuste,\\ 
25 & \textbf{diu} stîcleder von bast.\\ 
 & dem edeln, werden gast\\ 
 & was etwen gesatelt baz.\\ 
 & ûf sitzen \textbf{meinde} er umb daz:\\ 
 & er vorhte, daz er zertræte\\ 
30 & des satels \textbf{geræte}.\\ 
\end{tabular}
\scriptsize
\line(1,0){75} \newline
m n o \newline
\line(1,0){75} \newline
\textbf{15} \textit{Initiale} m   $\cdot$ \textit{Capitulumzeichen} n  \newline
\line(1,0){75} \newline
\textbf{2} geltes] geltten m  $\cdot$ muos] muͯsz n o \textbf{4} welt] wilt o \textbf{5} hêr] herre her n \textbf{6} von hinnen] von ẏme hinnen m  $\cdot$ wirt] wart o \textbf{8} der] dar m n  $\cdot$ spâte] spette m úch spote n (o) \textbf{9} diene] diende o \textbf{10} des] De m \textbf{12} müezet] must m muͯst n o  $\cdot$ blîden] bliben m o \textbf{14} wirt al] wurt vil n [vil]: wirt vl o \textbf{15} minnen] minnendbere m  $\cdot$ gernde] gerne o \textbf{16} wernde] werde o \textbf{17} dô] \textit{om.} n o \textbf{19} iuwerm] uwern o  $\cdot$ bote] gebotte n o \textbf{22} Darnach das manig beguͯnde schowen o \textbf{23} dræter] drete n driter o \textbf{24} ein] Er o  $\cdot$ kranker] krancke n o \textbf{26} dem] Wenne dem n \textbf{29} vorhte] forcht o \newline
\end{minipage}
\end{table}
\newpage
\begin{table}[ht]
\begin{minipage}[t]{0.5\linewidth}
\small
\begin{center}*G
\end{center}
\begin{tabular}{rl}
 & \begin{large}V\end{large}ür sîn ors ze behalten.\\ 
 & des geltes muos er walten.\\ 
 & si sprach \textbf{hin} ze im, ich wæne durch haz:\\ 
 & "saget an, welt ir \textbf{iht} vürbaz?"\\ 
5 & \textbf{dô sprach mîn} hêrre Gawan:\\ 
 & "mîn vart von hinnen wirt getân\\ 
 & al nâch iuwerm râte."\\ 
 & si sprach: "der kumet \textbf{iu} spâte."\\ 
 & "nû diene ich \textit{iu} \textbf{doch} drumbe."\\ 
10 & "des dunket ir mich der tumbe.\\ 
 & welt ir daz niht \textbf{vermîden},\\ 
 & sô müezet ir \textbf{von de\textit{n} blîden}\\ 
 & kêren \textit{gei}n der riuwe.\\ 
 & iuwer kumber wirt al niuwe."\\ 
15 & dô sprach der minne gernde:\\ 
 & "ich bin iuch dienstes wernde,\\ 
 & ic\textit{h} enpfâhe\textbf{s} vröude oder nôt,\\ 
 & sît iuwer minne mir gebôt,\\ 
 & daz ich \textbf{muo\textit{z}} ze iuwerem \textbf{\textit{ge}bote} stên,\\ 
20 & ich muge rîten oder gên."\\ 
 & al stênde bî der vrouwen\\ 
 & daz marc begunde er schouwen.\\ 
 & daz was ze dræter tjoste\\ 
 & ein harte \textbf{kleiniu} koste,\\ 
25 & stîcleder von baste.\\ 
 & dem edeln, werden gaste\\ 
 & was eteswenne gesatel\textit{t} baz.\\ 
 & ûf sitzen \textbf{meit} er umb daz:\\ 
 & er vorhte, daz er zertræte\\ 
30 & des satels \textbf{gewæte}.\\ 
\end{tabular}
\scriptsize
\line(1,0){75} \newline
G I L M Z \newline
\line(1,0){75} \newline
\textbf{1} \textit{Initiale} G L Z  \textbf{15} \textit{Initiale} I  \newline
\line(1,0){75} \newline
\textbf{2} muos] must I Z \textbf{3} hin ze im] zuͤ zim I \textit{om.} M \textbf{5} dô] Da M  $\cdot$ hêrre Gawan] herre g* \textit{nachträglich korrigiert zu:} gawan G ergawan M \textbf{8} spâte] ze spate I \textbf{9} iu] \textit{om.} G \textbf{10} der] \textit{om.} Z \textbf{11} daz] des Z \textbf{12} müezet] muͤz I  $\cdot$ den] dem G  $\cdot$ blîden] bliben M \textbf{13} gein] uon G \textbf{15} dô] [S*]: S I Da M  $\cdot$ minne] minnen I Z \textbf{17} ich] Ihne G  $\cdot$ enpfâhes] enphahe sin I (Z)  $\cdot$ oder] olde G \textbf{19} muoz] muͦze G  $\cdot$ iuwerem] ewer einer I  $\cdot$ gebote] bote G \textbf{22} marc] phert M \textbf{23} Das was zcu bose zcu der schuste M \textbf{24} Von harte kranker custe M  $\cdot$ kleiniu] krancke L (Z) \textbf{25} stîcleder] Sin Schelledir M \textbf{26} edeln werden] edelem werdem I \textbf{27} gesatelt] gesatel G \textbf{28} sitzen] siczens M \textbf{30} des] Sines M  $\cdot$ gewæte] gerete M \newline
\end{minipage}
\hspace{0.5cm}
\begin{minipage}[t]{0.5\linewidth}
\small
\begin{center}*T
\end{center}
\begin{tabular}{rl}
 & vür sîn ors ze behalten.\\ 
 & des geltes muoser walten.\\ 
 & Si sprach \textbf{hin} zim, ich wæne durch haz:\\ 
 & "saget an, welt ir vürbaz?"\\ 
5 & "\textbf{Jâ, vrouwe", sprach} hêr Gawan,\\ 
 & "mîn vart von hinnen wirt getân\\ 
 & al nâch iuwerme râte."\\ 
 & Si sprach: "der kumt \textbf{iu} spâte."\\ 
 & "Nû diene ich iu \textbf{doch} drumbe."\\ 
10 & "\begin{large}D\end{large}es dunket ir mich der tumbe.\\ 
 & welt ir daz niht \textbf{mîden},\\ 
 & sô müezet ir \textbf{kumber lîden}\\ 
 & \textbf{unde} kêren gegen der riuwe.\\ 
 & iuwer kumber wirt alniuwe."\\ 
15 & Dô sprach der minne gernde:\\ 
 & "ich bin iuch \textbf{doch} dienstes wernde,\\ 
 & ich enpfâhe\textbf{s} vröude oder nôt,\\ 
 & sît iuwer minne mir gebôt,\\ 
 & daz ich ziuwerm \textbf{gebote} stên,\\ 
20 & ich muge rîten oder gên."\\ 
 & \textit{\begin{large}A\end{large}}lstênde bî der vrouwen\\ 
 & daz marc begunder schouwen.\\ 
 & daz was ze dræter tjoste\\ 
 & ein harte \textbf{krank\textit{iu}} koste,\\ 
25 & \textbf{di\textit{u}} stîcleder von baste.\\ 
 & dem edeln, werden gaste\\ 
 & was etswenne gesatelt baz.\\ 
 & ûf sitzen \textbf{meit} er umbe daz:\\ 
 & er vorhte, daz er zertræte\\ 
30 & des satels \textbf{gewæte}.\\ 
\end{tabular}
\scriptsize
\line(1,0){75} \newline
T U V W O Q R Fr40 \newline
\line(1,0){75} \newline
\textbf{3} \textit{Majuskel} T  \textbf{5} \textit{Initiale} W   $\cdot$ \textit{Majuskel} T  \textbf{7} \textit{Initiale} Fr40  \textbf{8} \textit{Majuskel} T  \textbf{9} \textit{Majuskel} T  \textbf{10} \textit{Majuskel} T  \textbf{15} \textit{Majuskel} T  \textbf{21} \textit{Initiale} T U V  \newline
\line(1,0){75} \newline
\textbf{2} geltes] wechsels R  $\cdot$ muoser] mveser T mvͤst er V \textbf{3} hin] \textit{om.} W Q R  $\cdot$ ich wæne] wen ich R \textbf{4} ir] ir icht W (O) Q (Fr40) \textbf{5} vrouwe] \textit{om.} R  $\cdot$ hêr] \textit{om.} O Q  $\cdot$ Gawan] gewan R \textbf{6} vart] reise O  $\cdot$ von hinnen wirt] wirt hinnan R \textbf{8} kumt] konick Q  $\cdot$ spâte] ze spate V R \textbf{9} \textit{Versfolge 530.10-9} V   $\cdot$ iu doch] [*]: v́ch doch V doch R \textbf{10} der] gar R \textbf{11} [*]: Wellent daz niht vermiden V  $\cdot$ mîden] vermeiden W (O) Q (R) \textbf{12} müezet] muͦzet U (O) [*]: mussent  V must Q (R)  $\cdot$ kumber lîden] von den liben U [*]: von dem bliden V von dem leiden W von den bliden O Q R \textbf{13} unde kêren] [*]: Keren V Keren W (O) (Q) (R)  $\cdot$ der] dir U \textbf{16} iuch doch] iv doch T vch U (W) (O) (Q) (R) [*]: vch  V \textbf{17} ich enpfâhes] Jchn pfhaels Q \textbf{19} ich ziuwerm] [*]: ich zuͦ uwerme V \textbf{21} Alstênde] ÷lstende T  $\cdot$ vrouwen] frowen min R \textbf{22} [*]: Daz marg begunde er schowen V  $\cdot$ schouwen] schowen sin R \textbf{23} ze dræter] zuͦ der driter U zder O zer triste tetten R  $\cdot$ tjoste] hoste R \textbf{24} krankiu] cranke T (R) \textbf{25} diu] die T \textbf{26} werden] werde Q \textbf{27} etswenne gesatelt] gesatelt et swenne O \textbf{29} zertræte] inzwei drete U \textbf{30} des] Das W  $\cdot$ gewæte] gerete V \newline
\end{minipage}
\end{table}
\end{document}
