\documentclass[8pt,a4paper,notitlepage]{article}
\usepackage{fullpage}
\usepackage{ulem}
\usepackage{xltxtra}
\usepackage{datetime}
\renewcommand{\dateseparator}{.}
\dmyyyydate
\usepackage{fancyhdr}
\usepackage{ifthen}
\pagestyle{fancy}
\fancyhf{}
\renewcommand{\headrulewidth}{0pt}
\fancyfoot[L]{\ifthenelse{\value{page}=1}{\today, \currenttime{} Uhr}{}}
\begin{document}
\begin{table}[ht]
\begin{minipage}[t]{0.5\linewidth}
\small
\begin{center}*D
\end{center}
\begin{tabular}{rl}
\textbf{624} & \textbf{\textit{\begin{large}D\end{large}}en herzogen} von Gowerzin,\\ 
 & unt ouch den andern vürsten mîn,\\ 
 & Floranden von Itolac,\\ 
 & der nahtes mîner \textbf{wahte} pflac.\\ 
5 & \textbf{er} was mîn Turkote \textbf{alsô},\\ 
 & sînes trûrens wirde ich nimmer vrô."\\ 
 & Gawan sprach ze\textbf{r} vrouwen:\\ 
 & "ir muget si \textbf{bêde} schouwen\\ 
 & ledec, ê \textbf{daz} uns kom diu naht."\\ 
10 & Dô heten si sich des bedâht\\ 
 & unt vuoren über anz lant.\\ 
 & die herzoginne lieht erkant\\ 
 & huop Gawan \textbf{aber} ûf \textbf{ir} pfert.\\ 
 & manec edel rîter wert\\ 
15 & \textbf{enpfiengen} in unt die herzogîn.\\ 
 & si kêrten gein der bürge hin.\\ 
 & Dâ wart mit vreuden geriten,\\ 
 & \textbf{von in} diu kunst niht vermiten,\\ 
 & daz \textbf{es} der bûhurt het êre.\\ 
20 & waz mag ich sprechen mêre,\\ 
 & wan daz der werde Gawan\\ 
 & unt diu herzoginne \textit{w}olgetân\\ 
 & \textbf{von vrouwen} \textbf{wart enpfangen} \textbf{sô},\\ 
 & si mohten\textbf{s} bêdiu wesen vrô,\\ 
25 & ûf Schastel Marveile?\\ 
 & ir mugt\textbf{s} \textbf{im} jehen ze heile,\\ 
 & daz im diu sælde ie geschach.\\ 
 & dô \textbf{vuort} in an sîn gemach\\ 
 & Arnive unt die daz kunden,\\ 
30 & \textbf{die} bewarten sîne wunden.\\ 
\end{tabular}
\scriptsize
\line(1,0){75} \newline
D Z \newline
\line(1,0){75} \newline
\textbf{1} \textit{Initiale} D Z  \textbf{10} \textit{Majuskel} D  \textbf{17} \textit{Majuskel} D  \newline
\line(1,0){75} \newline
\textbf{1} Den] ÷en D \textbf{3} Itolac] Jtolach D Jcolac Z \textbf{4} der] Des Z \textbf{5} er] Der Z  $\cdot$ Turkote] Turkoite Z  $\cdot$ alsô] so Z \textbf{8} schouwen] wol schowen Z \textbf{10} Dô] Da Z \textbf{13} ir] daz Z \textbf{19} es der] sin Z \textbf{22} wolgetân] volgetan D \textbf{25} ûf Scastelmarveîle D  $\cdot$ Vf tschahtel Marveile Z \textbf{26} mugts] muͤgetz Z \textbf{28} dô vuort] Da furten Z \textbf{29} Arnive] Arnîve D \textbf{30} sîne] im sine Z \newline
\end{minipage}
\hspace{0.5cm}
\begin{minipage}[t]{0.5\linewidth}
\small
\begin{center}*m
\end{center}
\begin{tabular}{rl}
 & \textbf{Li\textit{sch}oise\textit{n}} von Gowertzin,\\ 
 & und ouch den andern vürsten mîn,\\ 
 & Floranden von I\textit{t}ol\textit{a}c,\\ 
 & der nahtes mîner \textbf{wahte} pflac.\\ 
5 & \textbf{er} was mîn Turkoite \textbf{alsô},\\ 
 & sînes t\textit{r}ûr\textit{en}s wirde ich nimer vrô."\\ 
 & Gawan sprach ze\textbf{r} vrouwen:\\ 
 & "ir moget si \textbf{beide} schouwen\\ 
 & ledic, ê uns kome diu naht."\\ 
10 & dô heten si sich des bedâht\\ 
 & und vuoren über an daz lant.\\ 
 & die herzogîn lieht erkant\\ 
 & huop Gawan \textbf{aber} ûf \textbf{ir} pfert.\\ 
 & manic edel \textit{ritter w}ert\\ 
15 & \textbf{enpfienc} in und die herzogîn.\\ 
 & si kêrten gegen der bürge hin.\\ 
 & dô wart mit vröuden geriten\\ 
 & \textbf{und} diu kunst niht vermiten,\\ 
 & daz \textbf{es} der bûhurt het êre.\\ 
20 & waz ma\textit{c i}ch sprechen mêre,\\ 
 & wan daz der werde Gawan\\ 
 & un\textit{d d}iu herzogîn wol getân\\ 
 & \textbf{von vrowen} \textbf{enpf\textit{a}ng\textit{en wart}} \textbf{alsô},\\ 
 & si mohten\textit{\textbf{s}} beidiu wesen vrô,\\ 
25 & ûf Schahte\textit{l} Mar\textit{ve}ile?\\ 
 & ir muget \textbf{es} \textbf{im} jehen zuo heile,\\ 
 & daz im diu sælde ie geschach.\\ 
 & dô \textbf{vuort\textit{e}} in an sîn \textit{g}emach\\ 
 & Ar\textit{niv}e und die daz kun\textit{d}en,\\ 
30 & \textbf{die} bewarten sîne wunden.\\ 
\end{tabular}
\scriptsize
\line(1,0){75} \newline
m n o \newline
\line(1,0){75} \newline
\newline
\line(1,0){75} \newline
\textbf{1} Lischoisen] Liscoise m Liscoisen n o  $\cdot$ Gowertzin] gowerczin m gowortzin n gewerczin o \textbf{3} Itolac] icholog m icholag n o \textbf{4} der] Des o \textbf{5} Turkoite] turkoitte m turkoiten o \textbf{6} trûrens] turnes m truwens o  $\cdot$ wirde] wurde m n o \textbf{14} ritter wert] werck vnd wert m \textbf{17} geriten] mit geritten n \textbf{20} mac ich] mag ist ich m \textbf{22} und diu] Vnd hob die m \textbf{23} enpfangen wart] enpfing m \textbf{24} mohtens] mohttencz m moͯchtens n (o)  $\cdot$ wesen] wesens n \textbf{25} Vff schahtten marnaile m  $\cdot$ Vff schahel narueile n  $\cdot$ Vff schattel morneẏle o \textbf{27} ie] \textit{om.} n \textbf{28} vuorte] fuͯrten m  $\cdot$ gemach] vngemach m gesach o \textbf{29} Arnive] Arune m Arnuwe n o  $\cdot$ kunden] kunnen m \newline
\end{minipage}
\end{table}
\newpage
\begin{table}[ht]
\begin{minipage}[t]{0.5\linewidth}
\small
\begin{center}*G
\end{center}
\begin{tabular}{rl}
 & \textbf{den herzogen} von Gowerzin,\\ 
 & unde ouch den andern vürsten mîn,\\ 
 & Floranden von Itolac,\\ 
 & der nahtes mîner \textbf{wahtære} pflac.\\ 
5 & \textbf{der} was mîn Turkoite \textbf{sô},\\ 
 & sînes trû\textit{re}ns wirde ich nimmer vrô."\\ 
 & Gawan sprach ze \textbf{der} vrouwen:\\ 
 & "ir muget  \textbf{gerne} schouwen\\ 
 & ledec, ê \textbf{daz} uns kome diu naht."\\ 
10 & dô heten si sich des bedâht\\ 
 & unde vuoren über an daz lant.\\ 
 & die herzoginne lieht erkant\\ 
 & huop Gawan ûf\textbf{ez} pfert.\\ 
 & manic edel rîter wert\\ 
15 & \textbf{enpfiengen} in unde die herzogîn.\\ 
 & si kêrten gein der bürge hin.\\ 
 & dô wart mit vröuden geriten\\ 
 & \textbf{unt} diu kunst niht vermiten,\\ 
 & daz \textit{\textbf{sîn}} der bûhurt het êre.\\ 
20 & waz mag ich sprechen mêre,\\ 
 & wan daz der werde Gawan\\ 
 & unde diu herzoginne wolgetân\\ 
 & \textbf{mit vröuden} \textbf{wart enpfangen} \textbf{sô},\\ 
 & si mohten\textbf{s} beidiu wesen vrô,\\ 
25 & \begin{large}Û\end{large}f Tschastel Marveile?\\ 
 & ir muget\textbf{s} \textbf{in} jehen ze heile,\\ 
 & daz im diu sælde ie geschach.\\ 
 & dô \textbf{vuorten si} in an sîn gemach.\\ 
 & Arnive unde die daz kunden,\\ 
30 & \textbf{si} bewarten \textbf{im} sîne wunden.\\ 
\end{tabular}
\scriptsize
\line(1,0){75} \newline
G I L M Z \newline
\line(1,0){75} \newline
\textbf{1} \textit{Initiale} Z  \textbf{3} \textit{Initiale} L  \textbf{11} \textit{Initiale} I  \textbf{25} \textit{Initiale} G I  \newline
\line(1,0){75} \newline
\textbf{1} den] [der]: den G der I  $\cdot$ herzogen] herzoge I (L)  $\cdot$ Gowerzin] geverzen G Goriunzin I gowerczin M \textbf{2} den andern vürsten] der ander furste I \textbf{3} Floranden] Florianden G (I)  $\cdot$ von] vn I  $\cdot$ Itolac] Jtolac G italac I jtolach L ytholac M Jcolac Z \textbf{4} der] Des Z  $\cdot$ wahtære] wahte I (Z) \textbf{5} Turkoite] turchoyte G turkoyde I tvrkoýte L turkoyte M \textbf{6} trûrens] truns G [str*]: sturmes I \textbf{8} muget] ] mvͯgt sie L (M) (Z)  $\cdot$ gerne] bede wol Z \textbf{9} ledec ê daz] idoch ê I \textbf{10} dô] Da M Z \textbf{12} lieht] wert L licht M \textbf{13} ûfez] [a*]: aber vf daz L abir uff das M (Z) \textbf{15} enpfiengen] enphienc I (L) \textbf{16} kêrten] riten I \textbf{17} dô] da I (L) (M) (Z) \textbf{18} unt] Von in die Z \textbf{19} sîn der] er der G \textit{om.} L isz der M sin Z \textbf{20} mag ich] mach L \textbf{23} mit] Von L M Z  $\cdot$ vröuden] vrowen L Z \textbf{25} Vf thahtesel marueile G  $\cdot$ Vf shatelmorueile I  $\cdot$ Vf kastel Marveile L  $\cdot$ Ve schastil marveile M  $\cdot$ Vf tschahtel Marveile Z \textbf{26} mugets in] mugt sin wol I mvgent im L mogit isz yme M (Z) \textbf{27} sælde] seldin M \textbf{28} dô] Da M Z  $\cdot$ si] \textit{om.} Z \textbf{29} Arnive] Anive G Arniue I Armýve L \textbf{30} si] \textit{om.} M \newline
\end{minipage}
\hspace{0.5cm}
\begin{minipage}[t]{0.5\linewidth}
\small
\begin{center}*T
\end{center}
\begin{tabular}{rl}
 & \textbf{den herzogen} von Gowerzin,\\ 
 & und ouch den andern vürsten mîn,\\ 
 & Floranden von Itolac,\\ 
 & der nahtes mîner \textbf{wahter} pflac.\\ 
5 & \textbf{der} was mîn Turkoyte \textbf{sô},\\ 
 & sînes trûrens w\textit{i}rde ich niemer vrô."\\ 
 & \begin{large}G\end{large}awan sprach zuo \textbf{den} vrouwen:\\ 
 & "ir moget si \textbf{gerne} schouwen\\ 
 & ledic, ê \textbf{daz} uns kome diu naht."\\ 
10 & dô heten si sich des bedâh\textit{t}\\ 
 & und vuoren über an daz lant.\\ 
 & die herzoginne lieht erkant\\ 
 & huop Gawan \textbf{aber} ûf \textbf{daz} pfert.\\ 
 & manec edel rîter wert\\ 
15 & \textbf{entviengen} in und die herzogîn.\\ 
 & si kêrten gein der bürge hin.\\ 
 & dô wart mit vreuden geriten\\ 
 & \textbf{und} diu kunst niht vermiten,\\ 
 & daz \textbf{e\textit{s}} der bûhurt het êre.\\ 
20 & waz ma\textit{g} \textit{i}ch sprechen mêre,\\ 
 & wan daz der werde Gawan\\ 
 & und diu herzoginne wol getân\\ 
 & \textbf{von vrouwen} \textbf{wurden entvangen} \textbf{sô},\\ 
 & si mohten beidiu wesen vrô,\\ 
25 & ûf Tschahtel Marveile?\\ 
 & ir moget \textbf{ez} \textbf{im} jehen zuo heile,\\ 
 & daz im diu sælde ie geschach.\\ 
 & dô \textbf{vuort} in an sîn gemach\\ 
 & Arnyve und die daz kunde\textit{n},\\ 
30 & \textbf{si} bewarten \textbf{im} sîne wunden.\\ 
\end{tabular}
\scriptsize
\line(1,0){75} \newline
U V W Q R Fr39 \newline
\line(1,0){75} \newline
\textbf{1} \textit{Initiale} Q Fr39  \textbf{7} \textit{Initiale} U V W  \newline
\line(1,0){75} \newline
\textbf{1} \textit{Versfolge 624.2-1} R   $\cdot$ den] Der Q  $\cdot$ Gowerzin] Gowerzen U gowerzein W kabrizin Q Gowerschin R \textbf{2} und] Dein vnd Q \textbf{3} Itolac] Jtalac U Jtalag V ytolag W ytolac Q Jtolac R \textbf{4} wahter] wahte V wache Q \textbf{5} der] Das R  $\cdot$ Turkoyte] tvrkoite V (Q) (R) (Fr39) \textbf{6} trûrens] trofrens Q  $\cdot$ wirde] worde U (R) (Fr39) wúrd V \textbf{7} Gawan] Gawin R  $\cdot$ zuo den] zer V (W) (Q) R Fr39 \textbf{8} gerne] beide V \textbf{9} diu naht] der tack R \textbf{10} bedâht] bedach U \textbf{11} vuoren] fur Q \textbf{12} die] Dú R (Fr39)  $\cdot$ lieht] li͑ht Q \textbf{13} Gawan] Gewin R \textbf{15} die] div Fr39 \textbf{16} si] Die Q \textbf{19} es] iz U (V)  $\cdot$ der] die W \textbf{20} mag ich] mag mag ich U \textbf{24} mohten] mohtens V (Q) (R) Fr39 \textbf{25} Of Tschatel marveile U  $\cdot$ Vf schatelmarveile V  $\cdot$ Auff kastel marfeile W  $\cdot$ Vff tschachtel marueile Q  $\cdot$ Vff Schahtel Maruiele R  $\cdot$ Vf [tscha*]: tschahtel marveile Fr39 \textbf{26} moget ez im] múgt ims W (R) \textbf{28} vuort] fvͦrten V (W) (Q) (Fr39) fuͦrtencz R \textbf{29} Arnyve] Arnyne U Arniue V Arnyue W R Fr39 Arniúe Q  $\cdot$ kunden] kunde U wol kunden W \newline
\end{minipage}
\end{table}
\end{document}
