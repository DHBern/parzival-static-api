\documentclass[8pt,a4paper,notitlepage]{article}
\usepackage{fullpage}
\usepackage{ulem}
\usepackage{xltxtra}
\usepackage{datetime}
\renewcommand{\dateseparator}{.}
\dmyyyydate
\usepackage{fancyhdr}
\usepackage{ifthen}
\pagestyle{fancy}
\fancyhf{}
\renewcommand{\headrulewidth}{0pt}
\fancyfoot[L]{\ifthenelse{\value{page}=1}{\today, \currenttime{} Uhr}{}}
\begin{document}
\begin{table}[ht]
\begin{minipage}[t]{0.5\linewidth}
\small
\begin{center}*D
\end{center}
\begin{tabular}{rl}
\textbf{39} & Gahmuret, der wîgant,\\ 
 & sprach: "mir sichert iwer hant.\\ 
 & \textit{\begin{large}D\end{large}}iu was \textbf{bî manlîcher} wer.\\ 
 & nû rîtet gein \textbf{der} Schotten her\\ 
5 & unt bittet \textbf{si}, daz \textbf{si uns} verbern\\ 
 & mit strîte, ob si des wellen gern,\\ 
 & unt \textbf{komt} \textbf{nâch mir} in die stat."\\ 
 & swaz er gebôt \textbf{oder} bat,\\ 
 & \textbf{endehaft ez wart} getân.\\ 
10 & die Schotten muosen strîten lân.\\ 
 & Dô kom gevaren Kaylet.\\ 
 & von dem kêrte Gahmuret,\\ 
 & \textbf{wand er} was sîner muomen sun.\\ 
 & waz \textbf{solt} er im dô leides tuon?\\ 
15 & der Spanol rief im nâch genuoc.\\ 
 & einen strûz er ûf dem helme truoc.\\ 
 & gezimieret was der man,\\ 
 & \textbf{als} ich dâ von ze sagenne hân,\\ 
 & \textbf{mit} pfelle wît und lanc.\\ 
20 & daz gevilde nâch dem helde klanc.\\ 
 & sîne schellen gâben gedœne.\\ 
 & er bluome \textbf{an} mannes schœne,\\ 
 & sîn varwe \textbf{an schœne hielt} den strît\\ 
 & \textbf{unz an} zwêne, die \textbf{nâch im wuohsen} sît:\\ 
25 & Beacurs, Lotes kint,\\ 
 & unt Parzival, die dâ niht sint.\\ 
 & \textbf{die} wâren dennoch ungeborn\\ 
 & unt wurden sît vür schœne erkorn.\\ 
 & Gaschier in \textbf{mit} dem zoume nam.\\ 
30 & "iwer wilde wirt vil zam\\ 
\end{tabular}
\scriptsize
\line(1,0){75} \newline
D \newline
\line(1,0){75} \newline
\textbf{3} \textit{Initiale} D  \textbf{11} \textit{Majuskel} D  \newline
\line(1,0){75} \newline
\textbf{1} Gahmuret] Gahmvret D \textbf{3} Diu] ÷iv D \textbf{4} Schotten] Scoten D \textbf{10} Schotten] Scoten D \textbf{12} Gahmuret] Gahmvret D \textbf{15} Spanol] Spanôl D \textbf{25} Beacurs] Beachvrs D \textbf{26} Parzival] Parzifal D \newline
\end{minipage}
\hspace{0.5cm}
\begin{minipage}[t]{0.5\linewidth}
\small
\begin{center}*m
\end{center}
\begin{tabular}{rl}
 & Gahmuret, der wîgant,\\ 
 & sprach: "mir sichert iuwer hant.\\ 
 & diu was \textbf{bî manlîcher} wer.\\ 
 & nû rîtet gegen \textbf{der} Schotten her\\ 
5 & und bittet \textbf{si}, daz \textbf{si uns} verbern\\ 
 & mit strîte, ob si des wellen gern,\\ 
 & und \textbf{komet} \textbf{mir nâch} in die stat."\\ 
 & waz er gebôt \textbf{oder} bat,\\ 
 & \textbf{daz wart} \dag angehaft\dag  getân.\\ 
10 & die Schotten m\textit{uo}sen strîten lân.\\ 
 & dô kam gevarn Kailet.\\ 
 & von dem kêrte Gahmuret.\\ 
 & \textbf{\begin{large}D\end{large}er ander} was sîner muomen sun.\\ 
 & waz \textit{\textbf{solt}} er ime dô leides tuon?\\ 
15 & der Spanol rief \textit{im}e nâch genuoc.\\ 
 & einen strûz er ûf dem helme truoc.\\ 
 & gezimieret was der man,\\ 
 & \textbf{alsô} ich dâ von ze sagende hân,\\ 
 & \textbf{mit} pfelle wît und lanc.\\ 
20 & daz gevilde nâch dem helde klanc.\\ 
 & sîne schellen g\textit{â}ben gedœne.\\ 
 & er bluome \textbf{an} mannes schœne,\\ 
 & sîn varwe \textbf{an schœne hielt} den strît\\ 
 & \textbf{unz an} zwêne, die \textbf{nâch ime wuohsen} sît:\\ 
25 & B\textit{eacur}s, Lotes kint,\\ 
 & und Parcifal, die dâ niht sint.\\ 
 & \textbf{si} wâren dannoch ungeborn\\ 
 & und wurden sît vür schœne erkorn.\\ 
 & Gaschier i\textit{n} \textbf{\textit{m}it} dem zoume nam.\\ 
30 & "iuwer wilde wirt vil zam\\ 
\end{tabular}
\scriptsize
\line(1,0){75} \newline
m n o W \newline
\line(1,0){75} \newline
\textbf{11} \textit{Initiale} W  \textbf{13} \textit{Initiale} m   $\cdot$ \textit{Capitulumzeichen} n  \newline
\line(1,0){75} \newline
\textbf{1} Gahmuret] Gamiret n Gamuret o W \textbf{3} diu] Des o  $\cdot$ manlîcher] [manlichern]: manlicher o \textbf{4} rîtet] ritten W \textbf{5} bittet] [verbittent]: bittent n bitten W  $\cdot$ si daz] das n o W \textbf{6} des] das W \textbf{7} in die stat] nider stat n \textit{om.} o \textbf{8} bat] hat o \textbf{9} angehaft] an der stat n o W \textbf{10} muosen] muͯssen m \textbf{11} Kailet] kaẏlet n gaylet W \textbf{12} Gahmuret] gamiret n gamuͯret o gamuret W \textbf{13} Der ander was] Den kant er er waz W \textbf{14} solt] \textit{om.} m  $\cdot$ dô leides] leides do n \textbf{15} Spanol] [spanol]: spanyol m spaniol n spaniel o spanioͤl W  $\cdot$ ime] vmbe \textit{nachträglich korrigiert zu:} ÿmbe m \textbf{17} gezimieret] Geziniert n Gezmet o \textbf{20} helde] hilde n golde W \textbf{21} gâben] geben m gebent W \textbf{22} er] Jr n o Er waz ein W \textbf{23} an] in o \textbf{24} nâch ime] im nach W \textbf{25} Beacurs] Berrahkins m Berachors n o Beakors W  $\cdot$ Lotes] lothes m laters n lathers o lottes W \textbf{26} Parcifal] parceval m parzifal W  $\cdot$ dâ] do n o W \textbf{28} vür] von o \textbf{29} Gaschier] Gascier m n Gasciret o Gatschier W  $\cdot$ in mit] in in mit m  $\cdot$ zoume] zorn o \textbf{30} vil] so W  $\cdot$ zam] schvr o \newline
\end{minipage}
\end{table}
\newpage
\begin{table}[ht]
\begin{minipage}[t]{0.5\linewidth}
\small
\begin{center}*G
\end{center}
\begin{tabular}{rl}
 & Gahmuret, der wîgant,\\ 
 & sprach: "mir sicheret iwer hant.\\ 
 & diu was \textbf{mit ellenthafter} wer.\\ 
 & nû rîtet gein \textbf{der} Schotten her\\ 
5 & unde bitet, daz \textbf{si uns} verberen\\ 
 & mit strîte, op si des wellen geren,\\ 
 & unde \textbf{kêrt} \textbf{nâch mir} in die stat."\\ 
 & swaz er gebôt \textbf{und} bat,\\ 
 & \textbf{endehaft daz wart} getân.\\ 
10 & die Schotten muosen strîten lân.\\ 
 & dô kom gevaren Kailet.\\ 
 & von dem kêrte Gahmuret,\\ 
 & \textbf{wan er} was sîner muomen sun.\\ 
 & waz \textbf{moht}er im dô leides tuon?\\ 
15 & der Spangol rief im nâch genuoc.\\ 
 & einen strûz er ûf dem helme truoc.\\ 
 & gezimiert was der man,\\ 
 & \textbf{daz} ich dâr von ze sagene hân,\\ 
 & \textbf{in} pfelle wît und lanc.\\ 
20 & daz gevilde nâch dem helde klanc.\\ 
 & sîne schellen gâben gedœne.\\ 
 & er bluome \textbf{an} mannes schœne,\\ 
 & sîn varwe \textbf{an schœne hielt} den strît\\ 
 & \textbf{âne} zwêne, die \textbf{nâch im wuohsen} sît:\\ 
25 & Beakurs, Lotes kint,\\ 
 & unde Parcival, die dâ niht sint.\\ 
 & \textbf{die} wâren dannoch ungeboren\\ 
 & unde wurden sît vür schœne erkoren.\\ 
 & \begin{large}G\end{large}atschier in \textbf{bî} dem zoume nam.\\ 
30 & "iwer wilde wirt vil zam\\ 
\end{tabular}
\scriptsize
\line(1,0){75} \newline
G O L M Q R Z Fr21 \newline
\line(1,0){75} \newline
\textbf{1} \textit{Initiale} O L M  \textbf{11} \textit{Initiale} L Q R Z Fr21  \textbf{29} \textit{Initiale} G  \newline
\line(1,0){75} \newline
\textbf{1} Gahmuret] Gahmvret G ÷amvret O Gahmuͯret L GAmurat M Gamuͯert Q Gamuret Z Gahmoret Fr21 \textbf{2} sicheret] sicherheit L \textbf{4} Schotten] schoten G O (Q) \textbf{5} bitet] betet sie M (R) bitte Fr21  $\cdot$ si] \textit{om.} Fr21  $\cdot$ verberen] verben Q \textbf{6} si] \textit{om.} Fr21  $\cdot$ des] das M  $\cdot$ wellen] willen R \textbf{7} kêrt] chompt O (L) (M) (Q) (Z) (Fr21) \textbf{8} swaz] Waz L (M) (Q) (R)  $\cdot$ er] ir M  $\cdot$ und] oder O L (M) (Q) (R) Z  $\cdot$ bat] gebat O L M Q R Z \textbf{9} endehaft] \textit{om.} Q  $\cdot$ daz wart] daz was O (M) (Q) Z Fr21 wart es L \textbf{10} Schotten] schoten G  $\cdot$ muosen] musten orin M \textbf{11} dô] Nu M Da Z  $\cdot$ kom gevaren] sprach R  $\cdot$ Kailet] kaylet O Q Fr21 kaýlet L kalet M kaylett R \textbf{12} von] Mit R  $\cdot$ kêrte] kert Fr21  $\cdot$ Gahmuret] Gamvret O Gahmvret L gamuret M Z gaműret Q [Gahmaret]: Gahmoret Fr21 \textbf{13} wan er] Wan der Z Wad er Fr21 \textbf{14} mohter] solde er O (L) (M) (Q) (R) (Z) wolt er Fr21  $\cdot$ im] d* \textit{nachträglich korrigiert zu:} dem Q  $\cdot$ dô] da M Z \textit{om.} Q \textbf{15} Spangol] spnaniol O spaniol L Q Fr21 spanigolt M spaniel R spanol Z  $\cdot$ rief] ruͦff R  $\cdot$ nâch] do Q \textbf{16} strûz er] storg Q  $\cdot$ helme] helmen M hopt R \textbf{17} gezimiert] Gahmvret L Geczynnerert M \textbf{18} daz] Als Z  $\cdot$ dâr von] von ým L uch da vone M \textbf{19} in pfelle] Jn spollen L Jm velde Q Mit pfelle Z \textbf{20} gevilde] gewilde R  $\cdot$ helde] \textit{om.} O \textbf{21} schellen] sellen M  $\cdot$ gedœne] zu done Q \textbf{22} er] Ein L (M) Q  $\cdot$ an] \textit{om.} Fr21 \textbf{23} varwe] frowe R  $\cdot$ hielt] behilde M (R) \textbf{24} âne] Vnz an O (L) (R) (Z) Fr21 Bisz an M  $\cdot$ nâch im] \textit{om.} L nach R  $\cdot$ wuohsen] gewachsen M [*]: vͦchsent R [*sen]: wahsen Fr21 \textbf{25} Beakurs] beacurs G (O) (Fr21) Beachuͯrs L By agkirs M Beakuͯrs Q Beachvrs Z  $\cdot$ Lotes] lotis M \textbf{26} Parcival] parzival G Barcifal O Parcifal L (Z) (Fr21) parschefal M parciual Q parczifal R  $\cdot$ dâ] [du]: da O noch M do Q \textbf{29} Gatschier] Garschier R  $\cdot$ bî] mit M Z \textbf{30} wilde] [wide]: wilde O \newline
\end{minipage}
\hspace{0.5cm}
\begin{minipage}[t]{0.5\linewidth}
\small
\begin{center}*T (U)
\end{center}
\begin{tabular}{rl}
 & \begin{large}G\end{large}ahmuret, der wîgant,\\ 
 & sprach: "mir sicher\textit{t} iuwer hant.\\ 
 & diu was \textbf{mit ellen\textit{t}hafter} wer.\\ 
 & nû \textit{r}ît\textit{et} gein \textbf{dem} Schotten her\\ 
5 & und b\textit{i}tet, daz \textbf{uns si} verbern\\ 
 & mit strîte, ob si des wellen gern,\\ 
 & und \textbf{komet} \textbf{nâch mir} in die stat."\\ 
 & waz er gebôt \textbf{und} bat,\\ 
 & \textbf{endehaft ez wart} getân.\\ 
10 & die Schotten muosen \textbf{ir} strîten lân.\\ 
 & dô kam gevarn Kaylet.\\ 
 & von dem kêrte Gahmuret,\\ 
 & \textbf{wan er} was sîner muomen suon.\\ 
 & waz \textbf{solt} er im dô leides tuon?\\ 
15 & der Span\textit{io}l rief im nâch genuoc.\\ 
 & einen strûz er ûf dem helme truoc.\\ 
 & gezimiert was der man,\\ 
 & \textbf{daz} ich \textit{dâ von} ze sagen hân,\\ 
 & \textbf{in} pfelle wît und lanc.\\ 
20 & daz gevilde nâch dem helde klanc.\\ 
 & sîn schellen gâben gedœne.\\ 
 & er bluom\textit{e} \textbf{über} mannes schœne,\\ 
 & sîn varwe \textbf{hielt an schœne} den strît\\ 
 & \textbf{unz an} zwêne, die \textbf{wuohsen nâch im} sît:\\ 
25 & Beakurs, Lotes kint,\\ 
 & und Parcifal, die d\textit{â} niht sint.\\ 
 & \textbf{die} wâren dannoch ungeborn\\ 
 & und wurden sît vür schœne erkorn.\\ 
 & Gatschier in \textbf{bî} dem zoume nam.\\ 
30 & "iuwer wilde w\textit{irt vi}l zam\\ 
\end{tabular}
\scriptsize
\line(1,0){75} \newline
U V T \newline
\line(1,0){75} \newline
\textbf{1} \textit{Initiale} U V  \textbf{4} \textit{Majuskel} T  \textbf{8} \textit{Majuskel} T  \textbf{11} \textit{Initiale} T  \textbf{15} \textit{Majuskel} T  \newline
\line(1,0){75} \newline
\textbf{1} Gahmuret] Gahmuͦret U Gamuret V \textbf{2} sichert] sicherheit U \textbf{3} ellenthafter] ellenschafter U \textbf{4} rîtet] reit U  $\cdot$ dem] der V T \textbf{5} bitet] bietent U  $\cdot$ uns si] si vns V T \textbf{8} waz] swas V (T)  $\cdot$ und] oder T \textbf{9} endehaft ez wart] daz was endehaft T \textbf{10} Schotten] schoten T  $\cdot$ muosen ir] [*]: muͤstent ir V mvesen T \textbf{11} Kaylet] kaẏlet V \textbf{12} Gahmuret] Gahmuͦret U Gamuret V \textbf{14} solt] sol V \textbf{15} Spaniol] spanoyl U \textbf{17} was] waz wol V \textbf{18} daz] swaz T  $\cdot$ dâ von] von dir U \textbf{20} helde] helme V \textbf{22} bluome] bluͦmet U  $\cdot$ über] ob V an T \textbf{23} hielt an schœne] an scône hielt T \textbf{24} die] \textit{om.} V  $\cdot$ nâch im sît] nach \textit{(in neuer Zeile:)} Im seit U sît T \textbf{25} Beakurs] Beacurs U  $\cdot$ Lotes] [Lote*]: Lotez V \textbf{26} Parcifal] parzifal T  $\cdot$ dâ] do U \textbf{29} Gatschier] Gathscier U Gatscier T  $\cdot$ bî] mit T \textbf{30} wirt vil] wol U wirt noch [*]: vil V \newline
\end{minipage}
\end{table}
\end{document}
