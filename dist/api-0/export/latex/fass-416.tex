\documentclass[8pt,a4paper,notitlepage]{article}
\usepackage{fullpage}
\usepackage{ulem}
\usepackage{xltxtra}
\usepackage{datetime}
\renewcommand{\dateseparator}{.}
\dmyyyydate
\usepackage{fancyhdr}
\usepackage{ifthen}
\pagestyle{fancy}
\fancyhf{}
\renewcommand{\headrulewidth}{0pt}
\fancyfoot[L]{\ifthenelse{\value{page}=1}{\today, \currenttime{} Uhr}{}}
\begin{document}
\begin{table}[ht]
\begin{minipage}[t]{0.5\linewidth}
\small
\begin{center}*D
\end{center}
\begin{tabular}{rl}
\textbf{416} & ouch sol mîn prîs erwerben,\\ 
 & daz ichs âne muoze \textbf{ersterben}.\\ 
 & des \textbf{ich vil} wol getrûwe gote.\\ 
 & des sî mîn sælde gein im bote.\\ 
5 & \textbf{Ouch} swâ diz mære wirt vernomen,\\ 
 & Artuses swestersun sî komen\\ 
 & in mîme geleite ûf Schanpfanzun,\\ 
 & Franzoys oder Bertun,\\ 
 & Provenzale oder Burgunjoys,\\ 
10 & Galiciane \textbf{unt die von} Punturtoys,\\ 
 & \textbf{erhœrent} die Gawans nôt,\\ 
 & hân ich prîs, der ist denne tôt.\\ 
 & mir vrümt sîn angestlîcher strît\\ 
 & vil engez lop, mîn laster wît.\\ 
15 & daz sol mir vreude \textbf{swenden}\\ 
 & und mich ûf êren pfenden."\\ 
 & Dô disiu rede was getân,\\ 
 & dô stuont \textbf{dâ} einer des küneges man,\\ 
 & der was geheizen Liddamus.\\ 
20 & Kyot \textbf{in selbe nennet} sus.\\ 
 & Kyot laschantiure hiez,\\ 
 & den sîn kunst \textbf{des} niht erliez,\\ 
 & er \textbf{en}sunge vnd spræche sô,\\ 
 & des noch genuoge werdent vrô.\\ 
25 & Kyot ist ein Provenzal,\\ 
 & der dise \textbf{âventiure} von Parzifal\\ 
 & \textbf{heidensch} geschriben sach.\\ 
 & swaz er \textbf{en} franzoys \textbf{dâ von} gesprach,\\ 
 & bin ich niht der witze laz,\\ 
30 & daz sage ich tiuschen vürbaz.\\ 
\end{tabular}
\scriptsize
\line(1,0){75} \newline
D \newline
\line(1,0){75} \newline
\newline
\line(1,0){75} \newline
\textbf{6} Artuses] Artvs D \textbf{7} Schanpfanzun] Scanpfanzvn D \textbf{9} Burgunjoys] Bvrgvniôys D \textbf{10} Punturtoys] Evntvrtôys D \textbf{21} laschantiure] lascantivre D \textbf{30} tiuschen] tîvscen D \newline
\end{minipage}
\hspace{0.5cm}
\begin{minipage}[t]{0.5\linewidth}
\small
\begin{center}*m
\end{center}
\begin{tabular}{rl}
 & ouch sol mîn prîs erwerben,\\ 
 & daz ich es âne muoze \textbf{sterben}.\\ 
 & des \textbf{wil ich} wol getrûwen gote.\\ 
 & des sî mîn sælde gegen ime bote.\\ 
5 & \textbf{ouch} wâ diz mære wirt vernomen,\\ 
 & Artuses swestersun sî komen\\ 
 & in mînem geleite ûf Scanfanzun,\\ 
 & Franczos oder Brittum,\\ 
 & Prouenzal oder Burgunois,\\ 
10 & Galiciane \textbf{oder} Punturtois,\\ 
 & \textbf{vernement} die Gawanes nôt,\\ 
 & hân ich prîs, der ist denne tôt.\\ 
 & mir vrü\textit{m}t sîn angestlîcher strît\\ 
 & vil engez lop, mîn laster wît.\\ 
15 & daz sol mir vröude \textbf{wenden}\\ 
 & und mich ûf êren pfenden."\\ 
 & \begin{large}D\end{large}ô disiu rede was get\textit{â}n,\\ 
 & dô stuont \textbf{d\textit{â}} einer des küniges man,\\ 
 & der was geheizen Liddamus.\\ 
20 & Ki\textit{o}t \textbf{in selb nennet} sus.\\ 
 & Kiot lascanter hiez,\\ 
 & den sîn kunst \textbf{des} niht erliez,\\ 
 & er sunge und spræche sô,\\ 
 & des noch genuoge werdent vrô.\\ 
25 & Kiot ist ein Prouenzal,\\ 
 & der dise \textbf{rede} von Parcifal\\ 
 & \textbf{heidnisch} geschriben sach.\\ 
 & waz er \textbf{in} franzos \textbf{dâr von} gesprach,\\ 
 & bin ich niht der witze laz,\\ 
30 & daz sage ich tiuschen vürbaz.\\ 
\end{tabular}
\scriptsize
\line(1,0){75} \newline
m n o \newline
\line(1,0){75} \newline
\textbf{17} \textit{Capitulumzeichen} n  \newline
\line(1,0){75} \newline
\textbf{2} sterben] ersterben n o \textbf{4} sî] sú n sie o \textbf{5} diz] dise n das o  $\cdot$ wirt] wert o \textbf{6} swestersun] swester sin o \textbf{7} Scanfanzun] scanfanzuͯn m scanfancẏm o \textbf{8} Franczos] Frantzois n Franczois o  $\cdot$ Brittum] britun n britẏm o \textbf{10} Galiciane] Galician n o  $\cdot$ Punturtois] ponturtois n pontois o \textbf{12} hân] Habe n (o)  $\cdot$ denne] dem o \textbf{13} vrümt] fruͯnt m  $\cdot$ angestlîcher] anglúcher o \textbf{14} engez] engel o \textbf{16} mich] auͯch o \textbf{17} disiu] dise m n o  $\cdot$ getân] getvͯn m \textbf{18} dâ] do m n \textit{om.} o \textbf{19} Liddamus] lidamus n o \textbf{20} Kiot] Kiet m Kẏot o  $\cdot$ selb] selbe o \textbf{21} \textit{Versfolge 416.22-21} n   $\cdot$ Kiot] Kẏot n Kiost o  $\cdot$ lascanter] lascantúre n \textbf{23} sô] also n o \textbf{24} genuoge] genúg n (o)  $\cdot$ vrô] \sout{sol} fro o \textbf{25} Kiot] Kyot o  $\cdot$ Prouenzal] profenczal o \textbf{28} in] do o  $\cdot$ franzos] frantzois n franczoses o  $\cdot$ dâr] do n o  $\cdot$ gesprach] sprach n o \textbf{30} tiuschen] tuczschen m tútschen n tuschen o \newline
\end{minipage}
\end{table}
\newpage
\begin{table}[ht]
\begin{minipage}[t]{0.5\linewidth}
\small
\begin{center}*G
\end{center}
\begin{tabular}{rl}
 & ouch sol mîn brîs erwerben,\\ 
 & daz ich es âne muoze \textbf{sterben}.\\ 
 & des \textbf{ich vil} wol getrûwe gote.\\ 
 & des sî mîn sælde gein im bote.\\ 
5 & \textbf{doch} swâ diz mære wirt vernomen,\\ 
 & Artuses swestersun sî komen\\ 
 & in mînem geleite ûf Tschanfenzun,\\ 
 & Franzois o\textit{der} Britûn,\\ 
 & Provenzale oder Burgonois,\\ 
10 & Galiciane \textbf{ode von} Ponturteis,\\ 
 & \textbf{erhœrent} die Gawans nôt,\\ 
 & hân ich prîs, derst dane tôt.\\ 
 & mir vrumet sîn angestlîcher strît\\ 
 & vil engez lop, mîn laster wît.\\ 
15 & daz sol mir vröude \textbf{wenden}\\ 
 & unde mich ûf êren pfenden."\\ 
 & dô disiu rede was getân,\\ 
 & dô stuont einer des küneges man,\\ 
 & der was geheizen Lidamus.\\ 
20 & Kiot \textbf{nennet in selbe} sus.\\ 
 & Kiot latschanture hiez,\\ 
 & den sîn kunst niht erliez,\\ 
 & er \textbf{en}sünge und spræche sô,\\ 
 & des noch genuoge werdent vrô.\\ 
25 & Kiot ist ein Provenzal,\\ 
 & der dise \textbf{âventiure} von Parcival\\ 
 & \textbf{heidensch} geschriben sach.\\ 
 & swaz er \textbf{in} franzois gesprach,\\ 
 & bin ich niht der witze laz,\\ 
30 & daz sage ich tiuschen vürbaz.\\ 
\end{tabular}
\scriptsize
\line(1,0){75} \newline
G I O L M Q R Z \newline
\line(1,0){75} \newline
\textbf{1} \textit{Initiale} I O L M Z   $\cdot$ \textit{Capitulumzeichen} R  \textbf{17} \textit{Initiale} I  \newline
\line(1,0){75} \newline
\textbf{1} ouch] ÷vch O Nun Q \textbf{2} ich es] ez I ich O M Q R  $\cdot$ muoze] \textit{om.} I muͯsze L (Q) (R) mvͤzze Z  $\cdot$ sterben] ersterben I (M) \textbf{3} des] Dasz M  $\cdot$ ich vil] wil ich M ich Q R \textbf{4} sî] sien L sich R  $\cdot$ bote] min bot I (L) \textbf{5} doch swâ] Avch swa O (Z) Wa ouch L Ouch wa M (Q) Ob wa R  $\cdot$ diz] das Q \textbf{6} Artuses] daz artuses I Artuͯses L Artusesz M Artus Q R Z  $\cdot$ sî] her si I \textbf{7} Tschanfenzun] shapanzuͤn I Schampfazvn O schanphenzcun M tschanpfenzuͯn Q schanfenzun R tschanfanzvn Z \textbf{8} Franzois] ez si franzoys I Franzoys O Q Frantzois L Francziosch M  $\cdot$ oder] olde G  $\cdot$ Britûn] pritun G I Brittvn L brittuͯn Q Brton R \textbf{9} Provenzale] bruvenzal I Prouenzale M Q Povenzale R  $\cdot$ Burgonois] burgoniͦs G burgonoys I pvrgvnoys O Bvrgomoẏs L burgomoisz M burgomoys Q Brugoniois R (Z) \textbf{10} \textit{Vers 416.10 fehlt} R   $\cdot$ Galiciane] Galician I Calciane O Galiziane M Galliciane Q  $\cdot$ ode] oder die O L (M) Q ) Z  $\cdot$ Ponturteis] ponturtois G portunoys I pvntvrtoys O pvntuͯrtois L punturtoisz M púnturtoys Q Pvnturtois Z \textbf{11} erhœrent] erhoret I (R) Er hortet M  $\cdot$ Gawans] Gawanes O L (Z) gawansz M Gewains R \textbf{12} hân ich] Jch hon Q  $\cdot$ derst] der O dy ist M \textbf{13} mir] mit I  $\cdot$ sîn] si I min R \textbf{14} vil] \textit{om.} I  $\cdot$ mîn] vnd I \textbf{15} daz] Da O  $\cdot$ mir] mich O \textit{om.} R  $\cdot$ vröude] laster O vrouden M (Q)  $\cdot$ wenden] swenden O M Q (R) Z fremden L \textbf{16} ûf] an O  $\cdot$ êren] ere I erden Q \textbf{17} dô] Da M \textbf{18} dô] Da M Z  $\cdot$ stuont] stvͦnde O  $\cdot$ einer] ein I (M) \textbf{19} Lidamus] Lyddamvs O Liddamvs L (Q) (Z) liddamu͑sz M \sout{d} [d]: lidamus R \textbf{20} Kiot] Kyot O M Z Kẏot L R Koyt Q  $\cdot$ nennet in selbe] nent in selb I inselbe nennet O (L) (M) (Q) in selbe nennent R \textbf{21} \textit{Versfolge 416.22-21} I   $\cdot$ Kiot] Kyot O Z Kẏot L Kyott R Koyt Q  $\cdot$ latschanture hiez] lat scutuhiez I  $\cdot$ latschanture] latschantvr O (R) Lẏntschantuͯr L lantschantur M litschantur Q lantsantvr Z \textbf{22} den] Do Q  $\cdot$ niht] des niht O (L) (M) (Q) (R) Z \textbf{23} ensünge] svnge O (R)  $\cdot$ spræche] sprechen R \textbf{24} noch] \textit{om.} I R  $\cdot$ genuoge] gnuck Q (R) (Z)  $\cdot$ werdent] wern I wurdent R \textbf{25} Kiot] hiot I Kyot O Z Kẏot L R Koyt Q  $\cdot$ Provenzal] [b*]: bruuenzal I prufenzal M prouenzal Q prouenczal R \textbf{26} Parcival] parzival G parzifal I L M Barcifal O partzifal Q parczifal R parcifal Z \textbf{27} heidensch] Heidensz M \textbf{28} swaz] Waz L (M) (Q) (R)  $\cdot$ in] \textit{om.} O  $\cdot$ franzois] fronzoys I franzoys O fantzoẏs L franzoisz M frantzoys Q  $\cdot$ gesprach] der von gesprach O (L) (R) (Z) dar von sprach M do von gesprach Q \textbf{30} sage] sagt O  $\cdot$ ich] \textit{om.} L  $\cdot$ tiuschen] tuschen G tushen I tevschen O tvtschen L dutsch M deutzen Q tútsche R tevtschen Z \newline
\end{minipage}
\hspace{0.5cm}
\begin{minipage}[t]{0.5\linewidth}
\small
\begin{center}*T
\end{center}
\begin{tabular}{rl}
 & ouch sol mîn prîs erwerben,\\ 
 & daz ichs âne müeze \textbf{ersterben}.\\ 
 & des \textbf{ich} wol getriuwe gote.\\ 
 & des sî mîn sælde gegen im \textbf{ein} bote.\\ 
5 & \textbf{ouch} swâ diz mære wirt vernomen,\\ 
 & Artuses swestersun sî komen\\ 
 & in mînem geleite ûf Tschampfenzun,\\ 
 & Franzois oder Britun,\\ 
 & Provenzal oder Burgenoys,\\ 
10 & Galyciane \textbf{oder die von} Puntertoys,\\ 
 & \textbf{erhœrent} die Gawanes nôt,\\ 
 & hân ich prîs, derst danne tôt.\\ 
 & mir vrümt sîn angestlîcher strît\\ 
 & vil engez lop, mîn laster wît.\\ 
15 & daz sol mir vröude \textbf{swenden}\\ 
 & und mich ûf êren pfenden."\\ 
 & \begin{large}D\end{large}ô disiu rede was getân,\\ 
 & dô stuont einer des küneges man,\\ 
 & der was geh\textit{eizen} Lyddamus.\\ 
20 & Kyot \textbf{in selbe nennet} sus.\\ 
 & Kyot latschanture \textbf{er} hiez,\\ 
 & den sîn kunst \textbf{des} niht erliez,\\ 
 & er\textbf{n} sunge und spræche sô,\\ 
 & des noch genuoge werdent vrô.\\ 
25 & Kyot ist ein Provenzal,\\ 
 & der dise \textbf{âventiure} von Parcifal\\ 
 & \textbf{in heidenschaft} geschriben sach.\\ 
 & swaz er \textbf{in} franzoys \textbf{der von} gesprach,\\ 
 & bin \textit{i}ch niht der witze laz,\\ 
30 & daz sag ich \textbf{iu in} tiuschen vürbaz.\\ 
\end{tabular}
\scriptsize
\line(1,0){75} \newline
T U V W \newline
\line(1,0){75} \newline
\textbf{17} \textit{Initiale} U W  \newline
\line(1,0){75} \newline
\textbf{2} ersterben] sterben W \textbf{5} swâ] wo U W  $\cdot$ diz] dise U \textbf{7} Tschampfenzun] Tscampfenzvn T Tschamfenzvn U schanpfanzvn V \textbf{8} Franzois] Franzoys T U Frantzoys W  $\cdot$ Britun] Brîtvn T Brituͦn U brittvn V \textbf{9} Provenzal] Prouenzale W  $\cdot$ Burgenoys] Bvrgenoŷs T buͦrgonois U burgunois V \textbf{10} Galyciane] galiciane T Galitiane U Galliciane W  $\cdot$ die von] \textit{om.} V  $\cdot$ Puntertoys] Pvntertoŷs T Pontertois U punturtoẏs V punturtoys W \textbf{14} engez lop] enges lobes U \textbf{17} disiu] dise T V W \textbf{19} geheizen] gehie T  $\cdot$ Lyddamus] Liddamvs T Lyddannis U littamus V lidamus W \textbf{20} Kyot] Kẏot V  $\cdot$ selbe] selbes W \textbf{21} latschanture] Latscantvre T laschantúre V laschanteúre W  $\cdot$ er] \textit{om.} W \textbf{25} Provenzal] prouenzal W \textbf{26} Parcifal] parzifal T V partzifal W \textbf{27} in heidenschaft] Haidensch W \textbf{28} swaz] Waz U (W)  $\cdot$ franzoys] franzoẏs V frantzoys W  $\cdot$ gesprach] sprach U \textbf{29} ich] uch T  $\cdot$ der witze] in witzen W \textbf{30} iu] \textit{om.} U V W  $\cdot$ tiuschen] tvscen T tuͦschen U túsch V teútschem W  $\cdot$ vürbaz] fúrfas V \newline
\end{minipage}
\end{table}
\end{document}
