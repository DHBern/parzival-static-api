\documentclass[8pt,a4paper,notitlepage]{article}
\usepackage{fullpage}
\usepackage{ulem}
\usepackage{xltxtra}
\usepackage{datetime}
\renewcommand{\dateseparator}{.}
\dmyyyydate
\usepackage{fancyhdr}
\usepackage{ifthen}
\pagestyle{fancy}
\fancyhf{}
\renewcommand{\headrulewidth}{0pt}
\fancyfoot[L]{\ifthenelse{\value{page}=1}{\today, \currenttime{} Uhr}{}}
\begin{document}
\begin{table}[ht]
\begin{minipage}[t]{0.5\linewidth}
\small
\begin{center}*D
\end{center}
\begin{tabular}{rl}
\textbf{626} & \textbf{\textit{\begin{large}D\end{large}}ô} enbôt \textbf{ouch} hêr Gawan,\\ 
 & ez wære \textbf{vrouwe} oder man,\\ 
 & al der massenîe gar,\\ 
 & daz si ir triwe næmen war\\ 
5 & \textbf{unt} \textbf{daz si}me künege rieten kumen;\\ 
 & daz \textbf{m\textit{ö}hte} an werdecheit \textbf{in} \textbf{gevrumen}.\\ 
 & al den werden er enbôt\\ 
 & sîn dienst unt \textbf{sîne} kampfes nôt.\\ 
 & Der brief niht insigels truoc;\\ 
10 & er schreib in sus \textbf{erkant} genuoc\\ 
 & mit wârzeichen ungelogen.\\ 
 & "Nû \textbf{en}soltûz niht langer zogen",\\ 
 & sprach Gawan \textbf{zem knappen sîn}.\\ 
 & "der künec unt diu künegîn\\ 
15 & sint ze \textbf{Bems} bî der \textbf{Korca}.\\ 
 & die küneginne soltû dâ\\ 
 & \textbf{sprechen} eines morgens vruo.\\ 
 & swaz si dir râte, daz tuo,\\ 
 & unt lâze dir eine witze bî:\\ 
20 & verswîc, daz ich hie hêrre sî.\\ 
 & daz dû hie massenîe sîs,\\ 
 & daz \textbf{en}sage \textbf{in} \textbf{niht} \textbf{decheinen gewîs}."\\ 
 & Dem knappen \textbf{was} dannen gâch.\\ 
 & Arnive sleich im \textbf{sanfte} nâch;\\ 
25 & diu vrâgete in, war er wolde\\ 
 & unt waz er werben solde.\\ 
 & Dô sprach er: "vrouwe, i\textbf{ne} sage\textbf{s} iu niht,\\ 
 & ob mir mîn eit \textbf{rehte} giht.\\ 
 & \textbf{got hüete iwer, ich wil hinnen} varn."\\ 
30 & er reit nâch werdeclîchen scharn.\\ 
\end{tabular}
\scriptsize
\line(1,0){75} \newline
D Z \newline
\line(1,0){75} \newline
\textbf{1} \textit{Initiale} D Z  \textbf{9} \textit{Majuskel} D  \textbf{12} \textit{Majuskel} D  \textbf{23} \textit{Majuskel} D  \textbf{27} \textit{Majuskel} D  \newline
\line(1,0){75} \newline
\textbf{1} Dô] ÷o D Da Z \textbf{2} vrouwe] \sout{were} wip Z \textbf{4} triwe] trewen Z \textbf{5} daz sime] dem Z \textbf{6} möhte] mohte D Z  $\cdot$ in gevrumen] sie fromen Z \textbf{8} sîne] sines Z \textbf{10} erkant] bekant Z \textbf{15} Bems] benis Z  $\cdot$ Korca] Chorcha D chorta Z \textbf{17} sprechen] Besprechen Z \textbf{22} Daz ensage niht [wis]: keinen wis Z \textbf{23} was] wart Z \textbf{25} vrâgete] fragt Z  $\cdot$ wolde] solde Z \textbf{26} solde] wolde Z \textbf{27} ine sages] ich ensagez Z \textbf{29} hüete] \textit{om.} Z \newline
\end{minipage}
\hspace{0.5cm}
\begin{minipage}[t]{0.5\linewidth}
\small
\begin{center}*m
\end{center}
\begin{tabular}{rl}
 & \textbf{dar nâch} enbôt hêr Gawan,\\ 
 & ez wær \textbf{vrowe} oder man,\\ 
 & alder massenîe gar,\\ 
 & daz si \textbf{des} ir triuwe næmen war,\\ 
5 & \textbf{sô} \textbf{daz si} dem künige \dag rittende\dag  komen;\\ 
 & daz \textbf{solt} an werdicheit \textit{\textbf{in}} \textbf{vromen}.\\ 
 & alden werden er enbôt\\ 
 & \textit{s}î\textit{n}en dienst und \textbf{sînes} kampfes nôt.\\ 
 & der brief niht ingesigels truoc;\\ 
10 & er schreip in sus \textbf{erkant} genuoc\\ 
 & mit wârzeichen ungelogen.\\ 
 & "nû solt \dag daz\dag  niht langer zogen",\\ 
 & sprach Gawan, "\textbf{durch den willen mîn}.\\ 
 & der künic und diu künigîn\\ 
15 & sint zuo \textbf{Be\textit{m}s} bî der \textbf{\textit{K}orca}.\\ 
 & die künig\textit{în} soltû dâ\\ 
 & \textbf{sprechen} ein\textit{es} morgens vruo.\\ 
 & waz si dir râte, daz tuo,\\ 
 & und lâz dir ein witze bî:\\ 
20 & verswîc, daz ich hie hêrre sî.\\ 
 & daz dû hie massenîe sîs,\\ 
 & daz sage \textbf{niht} \textbf{dekein wîs}."\\ 
 & dem knappen \textbf{was} dannen gâch.\\ 
 & Ar\textit{niv}e  sleich im \textbf{senftlîch} nâch;\\ 
25 & diu vrâgte in, war er wolte\\ 
 & und waz er werben solte.\\ 
 & dô sprach er: "vrowe, ich sage iu niht,\\ 
 & ob mir mîn eit \textbf{des rehten} giht.\\ 
 & \textbf{gebietet mir, wan ich muoz} varn."\\ 
30 & er reit nâch wirdeclîchen scharn.\\ 
\end{tabular}
\scriptsize
\line(1,0){75} \newline
m n o Fr16 \newline
\line(1,0){75} \newline
\newline
\line(1,0){75} \newline
\textbf{1} hêr Gawan] hergawan o \textbf{4} triuwe] truwen n \textbf{6} in] mich m  $\cdot$ vromen] frowen o \textbf{8} sînen] Gingen m \textbf{12} zogen] zoigen o \textbf{15} sint] Sin o  $\cdot$ Bems] beeins m n bemes o  $\cdot$ Korca] thorka m n torca o \textbf{16} künigîn] kunnig m  $\cdot$ soltû] solt: o \textbf{17} eines] ein m \textbf{18} râte] rotet n \textbf{20} hie hêrre] hierre n \textbf{21} hie] die o \textbf{22} dekein] do keine n \textbf{23} dannen] dannoch o \textbf{24} Arnive] Aruͯne m Arniwe n Aruwe o  $\cdot$ senftlîch] sanffte n o \textbf{27} sage] \textit{om.} n \textbf{28} rehten] rechens o \textbf{30} wirdeclîchen] werdeclicher Fr16 \newline
\end{minipage}
\end{table}
\newpage
\begin{table}[ht]
\begin{minipage}[t]{0.5\linewidth}
\small
\begin{center}*G
\end{center}
\begin{tabular}{rl}
 & \textbf{dô} enbôt hêrre Gawan,\\ 
 & ez wær \textbf{wîp} oder man,\\ 
 & al d\textit{er} massenîe gar,\\ 
 & daz si ir triuwen næmen war\\ 
5 & \textbf{unde} dem künige rieten komen;\\ 
 & daz \textbf{m\textit{ö}hte} an werdecheit \textbf{gevromen}.\\ 
 & al de\textit{n} werden er enbôt\\ 
 & sîn dienst unde kampfes nôt.\\ 
 & der brief niht insigels truoc;\\ 
10 & er schreip in sus \textbf{bekant} genuoc\\ 
 & mit wârzeichen ungelogen.\\ 
 & "nû\textbf{ne} solt dûz niht langer zogen",\\ 
 & \begin{large}S\end{large}prach Gawan \textbf{ze dem knappen sîn}.\\ 
 & "der künic unde diu künegîn\\ 
15 & sint ze \textbf{Sabins} bî der \textbf{Chronica}.\\ 
 & die küneginne soltû dâ\\ 
 & \textbf{gesprechen} eines morgens vruo.\\ 
 & swaz si dir rât, daz tuo,\\ 
 & unt lâ dir ein witze bî:\\ 
20 & verswîc, daz ich hie hêrre sî.\\ 
 & daz dû hie massenîe sîs,\\ 
 & daz \textbf{en}sag \textbf{ouch} \textbf{dehein wîs}."\\ 
 & dem knappen \textbf{wart} dannen gâch.\\ 
 & Arnive sleich im \textbf{sanfte} nâch;\\ 
25 & diu vrâgete in, war er wolde\\ 
 & unde waz er werben solde.\\ 
 & dô sprach er: "vrouwe, i\textbf{n} sag\textbf{s} iu niht,\\ 
 & ob mir mîn eit \textbf{rehte} giht.\\ 
 & \textbf{got hüet iuwer, ich wil hinnen} varn."\\ 
30 & er reit nâch werdeclîchen scharn.\\ 
\end{tabular}
\scriptsize
\line(1,0){75} \newline
G I L M Z Fr51 \newline
\line(1,0){75} \newline
\textbf{1} \textit{Initiale} L Z Fr51  \textbf{5} \textit{Initiale} I  \textbf{13} \textit{Initiale} G  \textbf{21} \textit{Initiale} I  \newline
\line(1,0){75} \newline
\textbf{1} dô] Da M Z  $\cdot$ enbôt] enbot och L (M) (Z) (Fr51) \textbf{3} der] die G  $\cdot$ massenîe] Massenien I \textbf{4} triuwen] nuwe Fr51 \textbf{5} unde dem] Vnd daz sie dem L (M) Das se den Fr51  $\cdot$ rieten] rite M \textbf{6} möhte] mohte G (I) (L) (M) Z (Fr51)  $\cdot$ gevromen] sy gefromen M sie fromen Z im vromen Fr51 \textbf{7} den] der G \textbf{8} sîn] sinen I (Fr51)  $\cdot$ kampfes] sines kampfes L Z (Fr51) syns kaphses M \textbf{9} insigels] ingesigelz L (M) (Fr51) \textbf{10} schreip in sus] beschreib yn sus M was dus Fr51 \textbf{11} wârzeichen] [worzei*]: worzeichen G \textbf{12} nûne] Nu M (Fr51)  $\cdot$ dûz] du I \textbf{13} Gawan] [g*]: gawan G  $\cdot$ ze dem] zon Fr51 \textbf{15} ze Sabins] zesabins G I zuͯ Sabins L zcu sabins M zv benis Z zo sabins Fr51  $\cdot$ Chronica] chonicA G cronica I chercha L korcha M chorta Z korche Fr51 \textbf{16} küneginne] kvninginnen Fr51 \textbf{17} gesprechen] Besprechen Z  $\cdot$ vruo] fry M \textbf{18} swaz] Waz L (M)  $\cdot$ daz tuo] dar zcu M \textbf{19} ein witze bî] daz verboten sin I \textbf{20} daz ditz lant si worden min I \textbf{21} daz] Vnde daz I \textbf{22} ouch] auch nih I niht L (M) o\textit{m. } Fr51  $\cdot$ dehein wîs] [wis]: keinen wis Z \textbf{23} wart] was Fr51 \textbf{24} Arnive] arniue I  $\cdot$ im sanfte] im shoͤne I sanifte L im alles Fr51 \textbf{25} diu] Vnd Fr51  $\cdot$ vrâgete] vragt I (L) (Z)  $\cdot$ wolde] solde Z \textbf{26} solde] wolde Z \textbf{27} dô] Da M \textit{om.} Fr51  $\cdot$ in sags] ich ensage I  $\cdot$ iu] \textit{om.} Fr51 \textbf{28} mir] \textit{om.} L  $\cdot$ rehte] rechtes M \textbf{29} hüet iuwer] ewer Z hotuch Fr51  $\cdot$ hinnen] von hinnen I \newline
\end{minipage}
\hspace{0.5cm}
\begin{minipage}[t]{0.5\linewidth}
\small
\begin{center}*T
\end{center}
\begin{tabular}{rl}
 & \textbf{\begin{large}D\end{large}ô} enbôt \textbf{ouch} hêr Gawan,\\ 
 & ez wære \textbf{wîp} oder man,\\ 
 & al der massenîe gar,\\ 
 & daz si ir triuwen næmen war\\ 
5 & \textbf{und} dem künege rieten komen;\\ 
 & daz \textbf{m\textit{ö}hte} an wirdecheit \textbf{si} \textbf{vromen}.\\ 
 & a\textit{l} den werden er enbôt\\ 
 & sîn dienst und \textbf{sînes} kampfes nôt.\\ 
 & der brief niht ingesigels truoc;\\ 
10 & er schreip in sus \textbf{bekant} genuoc\\ 
 & mit wârzeichen ungelogen.\\ 
 & "nû \textbf{en}solt dû ez niht langer zogen",\\ 
 & sprach Gawan \textbf{zuo dem knappen sîn}.\\ 
 & "der künec und diu künegîn\\ 
15 & sint zuo \textbf{Benis} bî der \textbf{Koicha}.\\ 
 & die küneginne solt dû dâ\\ 
 & \textbf{gesprechen} eines morgens vruo.\\ 
 & waz si dir râte, daz tuo,\\ 
 & und lâz dir eine witze bî:\\ 
20 & verswîc, daz ich hie hêrre sî.\\ 
 & daz dû hie massenîe sîs,\\ 
 & daz \textbf{en}sage \textbf{niht} \textbf{dekeine wîs}."\\ 
 & dem knappen \textbf{wart} dannen gâch.\\ 
 & Arnyve sleich im \textbf{sanfte} nâch;\\ 
25 & diu vrâget in, war er wolte\\ 
 & und waz er werben solte.\\ 
 & dô sprach er: "vrouwe, ich \textbf{en}sage\textbf{z} iu niht,\\ 
 & ob mir mîn eit \textbf{rehtes} giht.\\ 
 & \textbf{got hüete iuwer, ich \textit{wil} hinnen} varn."\\ 
30 & er reit nâch wirdeclîchen scharn.\\ 
\end{tabular}
\scriptsize
\line(1,0){75} \newline
U V W Q R Fr39 \newline
\line(1,0){75} \newline
\textbf{1} \textit{Initiale} U V W Q Fr39  \newline
\line(1,0){75} \newline
\textbf{1} D[*]: ar nach enbot her gawan V \textbf{4} si] \textit{om.} W  $\cdot$ triuwen] treúwe W (R) (Q)  $\cdot$ næmen] nement Fr39 \textbf{6} möhte] mochte U (V) (W) (Q) (Fr39) macht R  $\cdot$ si] [*]: sv́ V sy wol R \sout{si} Fr39  $\cdot$ vromen] gefromen W \textbf{7} al den] Alle den U Allen W R \textbf{8} sîn] Sinen V  $\cdot$ sînes] si͑nst Q \textbf{9} ingesigels] insigels W (R) \textbf{10} sus] als Q \textbf{12} ensolt] solcz R  $\cdot$ dû ez] [d*]: dvs V \textit{om.} R \textbf{13} Gawan] [gaw*]: gawan V Gawin R \textbf{15} Benis] Sems R  $\cdot$ Koicha] borcha U korka V korcha W Q kroka R \textbf{17} gesprechen] Besprechen W (Q)  $\cdot$ eines] \textit{om.} R  $\cdot$ morgens] morgen V R \textbf{18} waz] Swaz V  $\cdot$ si] \textit{om.} Q  $\cdot$ daz tuo] darzuͦ W \textbf{19} und] \textit{om.} R \textbf{22} daz] Dar zu Q  $\cdot$ ensage] sage Q (R) \textbf{24} Arnyve] Arnyue U W Arniue V Q R \textbf{25} diu] Vnd Q R  $\cdot$ vrâget] fragte W \textbf{27} dô sprach er] Er sprach W  $\cdot$ ensagez iu] sag v́ch V (R) ensag úchs W sagsz euch Q \textbf{28} rehtes] rates Q \textbf{29} hüete iuwer] behuͯt úch R  $\cdot$ wil] \textit{om.} U \newline
\end{minipage}
\end{table}
\end{document}
